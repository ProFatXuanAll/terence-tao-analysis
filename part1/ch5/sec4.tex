\section{Ordering the reals}\label{i:sec:5.4}

\begin{defn}\label{i:5.4.1}
  Let \((a_n)_{n = m}^{\infty}\) be a sequence of rationals.
  We say that this sequence is \emph{positively bounded away from zero} iff we have a positive rational \(c \in \Q^+\) such that \(a_n \geq c\) for all \(n \in \Z_{\geq m}\) (in particular, the sequence is entirely positive).
  The sequence is \emph{negatively bounded away from zero} iff we have a negative rational \(c \in \Q^-\) such that \(a_n \leq c\) for all \(n \in \Z_{\geq m}\) (in particular, the sequence is entirely negative).
\end{defn}

\begin{note}
  It is clear that any sequence which is positively or negatively bounded away from zero, is bounded away from zero.
  Also, a sequence cannot be both positively bounded away from zero and negatively bounded away from zero at the same time.
\end{note}

\setcounter{thm}{2}
\begin{defn}\label{i:5.4.3}
  A real number \(x\) is said to be \emph{positive} iff it can be written as \(x = \LIM_{n \to \infty} a_n\) for some Cauchy sequence \((a_n)_{n = 1}^{\infty}\) which is positively bounded away from zero.
  \(x\) is said to be \emph{negative} iff it can be written as \(x = \LIM_{n \to \infty} a_n\) for some sequence \((a_n)_{n = 1}^{\infty}\) which is negatively bounded away from zero.
\end{defn}

\begin{prop}[Basic properties of positive reals]\label{i:5.4.4}
  For every real number \(x\), exactly one of the following three statements is true:
  \begin{enumerate}
    \item \(x\) is zero;
    \item \(x\) is positive;
    \item \(x\) is negative.
  \end{enumerate}
  A real number \(x\) is negative iff \(-x\) is positive.
  If \(x\) and \(y\) are positive, then so are \(x + y\) and \(xy\).
\end{prop}

\begin{proof}[\pf{i:5.4.4}]
  We first show that at least one of the three statements is true.
  Let \(x \in \R\).
  By \cref{i:ac:5.3.1} we know that \(0\) is the formal limit of \((0)_{n = 1}^\infty\), thus by \cref{i:5.3.1,i:5.3.3} we can ask whether \(x = 0\) or \(x \neq 0\).
  If \(x = 0\), then we are done.
  Otherwise by \cref{i:5.3.14} we know that \(x\) is the formal limit of a rational Cauchy sequence \((a_n)_{n = 1}^\infty\) which is bounded away from \(0\).
  In particular, by \cref{i:5.3.12} we know that there exists a \(c \in \Q^+\) such that \(\abs{a_n} \geq c\) for all \(n \in \Z^+\).
  Fix such \(c\).
  Since \((a_n)_{n = 1}^{\infty}\) is a Cauchy sequence and \(c \in \Q^+\), by \cref{i:5.1.8} we have
  \[
    \exists N \in \Z^+ : \forall j, k \in \Z_{\geq N}, \abs{a_j - a_k} \leq c.
  \]
  Fix such \(N\).
  Then we have
  \begin{align*}
             & \forall j \in \Z_{\geq N}, \abs{a_j - a_N} \leq c          &  & (N \in \Z_{\geq N}) \\
    \implies & \forall j \in \Z_{\geq N}, -c \leq a_j - a_N \leq c        &  & \by{i:4.3.3}[c]     \\
    \implies & \forall j \in \Z_{\geq N}, -c + a_N \leq a_j \leq c + a_N. &  & \by{i:4.2.9}[c,d]
  \end{align*}
  Since \(\abs{a_N} \geq c \in \Q^+\), by \cref{i:4.3.1} we know that we have either \(a_N \in \Q^+\) or \(a_N \in \Q^-\).
  So we split into two cases:
  \begin{itemize}
    \item If \(a_N \in \Q^+\), then by \cref{i:4.3.1} we have \(a_N \geq c\).
          Thus
          \begin{align*}
                     & \forall j \in \Z_{\geq N}, 0 \leq -c + a_N \leq a_j \leq c + a_N &  & \by{i:4.2.9}[c,d] \\
            \implies & \forall j \in \Z_{\geq N}, 0 \leq a_j.                           &  & \by{i:4.2.9}[c]
          \end{align*}
          This means \(a_j \in \Q^+\) for all \(j \in \Z_{\geq N}\).
          If we now define \((b_n)_{n = 1}^\infty\) to be the sequence where \(b_n = c\) for all \(n \in \Z^+ \cap \Z_{< N}\) and \(b_n = a_n\) for all \(n \in \Z_{\geq N}\), then we see that \((a_n)_{n = 1}^\infty\) and \((b_n)_{n = 1}^\infty\) are equivalent rational Cauchy sequences and \((b_n)_{n = 1}^\infty\) is positively bounded away from zero.
          Thus, by \cref{i:5.4.3} \(x\) is positive.
    \item If \(a_N \in \Q^-\), then by \cref{i:4.3.1} we have \(-a_N \geq c\).
          By \cref{i:ex:4.2.6} we have \(a_N \leq -c\).
          Thus
          \begin{align*}
                     & \forall j \in \Z_{\geq N}, -c + a_N \leq a_j \leq c + a_N \leq 0 &  & \by{i:4.2.9}[c,d] \\
            \implies & \forall j \in \Z_{\geq N}, a_j \leq 0.                           &  & \by{i:4.2.9}[c]
          \end{align*}
          This means \(a_j \in \Q^-\) for all \(j \in \Z_{\geq N}\).
          If we now define \((b_n)_{n = 1}^\infty\) to be the sequence where \(b_n = -c\) for all \(n \in \Z^+ \cap \Z_{< N}\) and \(b_n = a_n\) for all \(n \in \Z_{\geq N}\), then we see that \((a_n)_{n = 1}^\infty\) and \((b_n)_{n = 1}^\infty\) are equivalent rational Cauchy sequences and \((b_n)_{n = 1}^\infty\) is negatively bounded away from zero.
          Thus, by \cref{i:5.4.3} \(x\) is negative.
  \end{itemize}
  From all cases above, we see that \(x\) is either positive or negative.
  Thus, we conclude that at least one of the three statements is true.

  Next we show that at most one of the three statements is true.
  Let \(x \in \R\).
  Suppose for the sake of contradiction that one of the following three cases is true:
  \begin{itemize}
    \item We have both \(x = 0\) and \(x\) is positive.
          Then from the first part of the proof we know that \(x\) is the formal limit of a rational Cauchy sequence \((a_n)_{n = 1}^\infty\) which is positively bounded away from \(0\).
          In particular, there exists a \(c \in \Q^+\) such that \(a_n \geq c\) for all \(n \in \Z^+\).
          Fix such \(c\).
          Since \(x = 0\), by \cref{i:5.2.6} we know that \((a_n)_{n = 1}^\infty\) and \((0)_{n = 1}^\infty\) are eventually \(\varepsilon\)-close for all \(\varepsilon \in \Q^+\).
          Since \(c \in \Q^+\), we know that \(c / 2 \in \Q^+\).
          Thus, \((a_n)_{n = 1}^\infty\) and \((0)_{n = 1}^\infty\) must be eventually \(c / 2\)-close, i.e., there exists an \(N \in \Z^+\) such that \(a_n = \abs{a_n - 0} \leq c / 2 < c\) for all \(n \in \Z_{\geq N}\).
          But this means for any \(n \in \Z_{\geq N}\), we have both \(a_n < c\) and \(a_n \geq c\), which contradict to \cref{i:4.2.9}(a).
    \item We have both \(x = 0\) and \(x\) is negative.
          Then from the first part of the proof we know that \(x\) is the formal limit of a rational Cauchy sequence \((a_n)_{n = 1}^\infty\) which is negatively bounded away from \(0\).
          In particular, there exists a \(c \in \Q^-\) such that \(a_n \leq c\) for all \(n \in \Z^+\).
          Fix such \(c\).
          Since \(x = 0\), by \cref{i:5.2.6} we know that \((a_n)_{n = 1}^\infty\) and \((0)_{n = 1}^\infty\) are eventually \(\varepsilon\)-close for all \(\varepsilon \in \Q^+\).
          Since \(c \in \Q^-\), we know that \(-c / 2 \in \Q^+\).
          Thus, \((a_n)_{n = 1}^\infty\) and \((0)_{n = 1}^\infty\) must be eventually \(-c / 2\)-close, i.e., there exists an \(N \in \Z^+\) such that \(-a_n = \abs{a_n - 0} \leq -c / 2 < -c\) for all \(n \in \Z_{\geq N}\).
          But this means for any \(n \in \Z_{\geq N}\), we have both \(a_n > c\) and \(a_n \leq c\), which contradict to \cref{i:4.2.9}(a).
    \item We have \(x\) is both positive and negative.
          From the first part of the proof we know that \(x\) is the formal limit of a rational Cauchy sequence \((a_n)_{n = 1}^\infty\) which is positively bounded away from \(0\).
          Similarly, we know that \(x\) is the formal limit of a rational Cauchy sequence \((b_n)_{n = 1}^\infty\) which is negatively bounded away from \(0\).
          By \cref{i:5.4.1} we know that there exist some \(c \in \Q^+\) and \(d \in \Q^-\) such that \(a_n \geq c\) and \(b_n \leq d\) for all \(n \in \Z^+\).
          Fix such \(c, d\) and observe that \(c - d \in \Q^+\).
          Since \(x = \LIM_{n \to \infty} a_n = \LIM_{n \to \infty} b_n\), by \cref{i:5.3.3} we know that \((a_n)_{n = 1}^\infty\) and \((b_n)_{n = 1}^\infty\) are equivalent rational Cauchy sequences.
          Thus, by \cref{i:5.2.6} we have
          \[
            \exists N \in \Z^+ : \forall n \in \Z_{\geq n}, \abs{a_n - b_n} \leq \dfrac{c - d}{2} < c - d.
          \]
          Fix such \(N\).
          But then we have
          \begin{align*}
                     & \forall n \in \Z_{\geq N}, (a_n \geq c) \land (b_n \leq d)                                \\
            \implies & \forall n \in \Z_{\geq N}, (a_n \geq c) \land (-b_n \geq -d)       &  & \by{i:ex:4.2.6}   \\
            \implies & \forall n \in \Z_{\geq N}, a_n - b_n = \abs{a_n - b_n} \geq c - d, &  & \by{i:4.2.9}[c,d]
          \end{align*}
          which contradict to \cref{i:4.2.9}(a).
  \end{itemize}
  From all cases above, we derived contradictions.
  Thus, we conclude that at most one of the three statements is true.

  Next we show that \(x\) is negative iff \(-x\) is positive.
  Suppose that \(x\) is negative.
  From the first part of the proof we know that \(x\) is the formal limit of a rational Cauchy sequence \((a_n)_{n = 1}^\infty\) which is negatively bounded away from \(0\).
  This means \((-a_n)_{n = 1}^\infty\) is positively bounded away from \(0\).
  But by \cref{i:ac:5.3.2} we know that \(-x = \LIM_{n \to \infty} -a_n\), thus by \cref{i:5.4.1} we know that \(-x\) is positive.
  The converse argument holds by reversing the previous reasoning.

  Finally we show that \(x, y\) are positive implies \(x + y\) and \(xy\) are positive.
  From the first part of the proof we know that \(x\) and \(y\) are the formal limits of rational Cauchy sequences \((a_n)_{n = 1}^\infty\) and \((b_n)_{n = 1}^\infty\), respectively, which are positively bounded away from \(0\).
  Clearly, \((a_n + b_n)_{n = 1}^\infty\) and \((a_n b_n)_{n = 1}^\infty\) are positively bounded away from \(0\).
  But by \cref{i:5.3.4,i:5.3.9} we know that \(x + y = \LIM_{n \to \infty} a_n + b_n\) and \(xy = \LIM_{n \to \infty} a_n b_n\), thus by \cref{i:5.4.1} we know that \(x + y\) and \(xy\) are positive.
\end{proof}

\begin{note}
  If \(q\) is a positive rational number, then the Cauchy sequence \(q, q, q, \dots\) is positively bounded away from zero, and hence \(\LIM_{n \to \infty} q = q\) is a positive real number.
  Thus, the notion of positivity for rationals is consistent with that for reals.
  Similarly, the notion of negativity for rationals is consistent with that for reals.
\end{note}

\begin{defn}[Absolute value]\label{i:5.4.5}
  Let \(x\) be a real number.
  We define the \emph{absolute value} \(\abs{x}\) of \(x\) to equal \(x\) if \(x\) is positive, \(-x\) when \(x\) is negative, and \(0\) when \(x\) is zero.
\end{defn}

\begin{defn}[Ordering of the real numbers]\label{i:5.4.6}
  Let \(x\) and \(y\) be real numbers.
  We say that \(x\) is \emph{greater than} \(y\), and write \(x > y\), iff \(x - y\) is a positive real number, and \(x < y\) iff \(x - y\) is a negative real number.
  We define \(x \geq y\) iff \(x > y\) or \(x = y\), and similarly define \(x \leq y\).
\end{defn}

\begin{note}
  Comparing \cref{i:5.4.6} with the definition of order on the rationals from \cref{i:4.2.8} we see that order on the reals is consistent with order on the rationals, i.e., if two rational numbers \(q, q'\) are such that \(q\) is less than \(q'\) in the rational number system, then \(q\) is still less than \(q'\) in the real number system, and similarly for ``greater than.''
  In the same way we see that the definition of absolute value given in \cref{i:5.4.5} is consistent with that in \cref{i:4.3.1}.
\end{note}

\begin{prop}\label{i:5.4.7}
  All the claims in \cref{i:4.2.9} which held for rationals, continue to hold for real numbers.
\end{prop}

\begin{proof}[\pf{i:5.4.7}(a)]
  By \cref{i:5.4.4} \(x - y\) satisfy exactly one of the following three statements:
  \begin{itemize}
    \item \(x - y = 0\).
          Then by \cref{i:5.3.11} we have \(x = y\).
    \item \(x - y\) is a positive rational number.
          Then by \cref{i:5.4.6} we have \(x > y\).
    \item \(x - y\) is a negative rational number.
          Then by \cref{i:5.4.6} we have \(x < y\).
  \end{itemize}
\end{proof}

\begin{proof}[\pf{i:5.4.7}(b)]
  We have
  \begin{align*}
         & x < y                                           \\
    \iff & x - y \text{ is negative}    &  & \by{i:5.4.6}  \\
    \iff & -(x - y) \text{ is positive} &  & \by{i:5.4.4}  \\
    \iff & y - x \text{ is positive}    &  & \by{i:5.3.11} \\
    \iff & y > x.                       &  & \by{i:5.4.6}
  \end{align*}
\end{proof}

\begin{proof}[\pf{i:5.4.7}(c)]
  We have
  \begin{align*}
             & (x < y) \land (y < z)                                                                  \\
    \implies & (x - y \text{ is negative}) \land (y - z \text{ is negative})       &  & \by{i:5.4.6}  \\
    \implies & (-(x - y) \text{ is positive}) \land (-(y - z) \text{ is positive}) &  & \by{i:5.4.4}  \\
    \implies & (y - x \text{ is positive}) \land (z - y \text{ is positive})       &  & \by{i:5.3.11} \\
    \implies & y - x + z - y \text{ is positive}                                   &  & \by{i:5.4.4}  \\
    \implies & -(x - z) \text{ is positive}                                        &  & \by{i:5.3.11} \\
    \implies & x - z \text{ is negative}                                           &  & \by{i:5.4.4}  \\
    \implies & x < z.                                                              &  & \by{i:5.4.6}
  \end{align*}
\end{proof}

\begin{proof}[\pf{i:5.4.7}(d)]
  We have
  \begin{align*}
             & x < y                                                    \\
    \implies & x - y \text{ is negative}             &  & \by{i:5.4.6}  \\
    \implies & x + z - z - y \text{ is negative}     &  & \by{i:5.3.11} \\
    \implies & (x + z) - (y + z) \text{ is negative} &  & \by{i:5.3.11} \\
    \implies & x + z < y + z.                        &  & \by{i:5.4.6}
  \end{align*}
\end{proof}

\begin{proof}[\pf{i:5.4.7}(e)]
  We have
  \begin{align*}
             & x < y                                             \\
    \implies & y > x                        &  & \by{i:5.4.7}[b] \\
    \implies & y - x \text{ is positive}    &  & \by{i:5.4.6}    \\
    \implies & (y - x)z \text{ is positive} &  & \by{i:5.4.4}    \\
    \implies & yz - xz \text{ is positive}  &  & \by{i:5.3.11}   \\
    \implies & yz > xz                      &  & \by{i:5.4.6}    \\
    \implies & xz < yz.                     &  & \by{i:5.4.7}[b]
  \end{align*}
\end{proof}

\begin{prop}\label{i:5.4.8}
  Let \(x\) be a positive real number.
  Then \(x^{-1}\) is also positive.
  Also, if \(y\) is another positive number and \(x > y\), then \(x^{-1} < y^{-1}\).
\end{prop}

\begin{proof}[\pf{i:5.4.8}]
  Let \(x\) be positive.
  Since \(xx^{-1} = 1\), the real number \(x^{-1}\) cannot be zero (since \(x0 = 0 \neq 1\)).
  Also, from \cref{i:5.4.4} it is easy to see that a positive number times a negative number is negative;
  this shows that \(x^{-1}\) cannot be negative, since this would imply that \(xx^{-1} = 1\) is negative, a contradiction.
  Thus, by \cref{i:5.4.4}, the only possibility left is that \(x^{-1}\) is positive.

  Now let \(y\) be positive as well, so \(x^{-1}\) and \(y^{-1}\) are also positive.
  Suppose that \(x > y\).
  If \(x^{-1} \geq y^{-1}\), then by \cref{i:5.4.7} we have \(xx^{-1} > yx^{-1} \geq yy^{-1}\), thus \(1 > 1\), which is a contradiction.
  Thus, we must have \(x^{-1} < y^{-1}\).
\end{proof}

\begin{prop}[The non-negative reals are closed]\label{i:5.4.9}
  Let \((a_n)_{n = 1}^\infty\) be a Cauchy sequence of non-negative rational numbers.
  Then \(\LIM_{n \to \infty} a_n\) is a non-negative real number.
\end{prop}

\begin{proof}[\pf{i:5.4.9}]
  We argue by contradiction, and suppose that the real number \(x \coloneqq \LIM_{n \to \infty} a_n\) is a negative number.
  Then by definition of negative real number, we have \(x = \LIM_{n \to \infty} b_n\) for some sequence \((b_n)_{n = 1}^\infty\) which is negatively bounded away from \(0\), i.e., there is a negative rational \(-c \in \Q^-\) such that \(b_n \leq -c\) for all \(n \in \Z^+\).
  On the other hand, we have \(a_n \in \Q_{\geq 0}\) for all \(n \in \Z^+\), by hypothesis.
  Thus, the numbers \(a_n\) and \(b_n\) are never \(c / 2\)-close, since \(c / 2 < c\).
  Thus, the sequences \((a_n)_{n = 1}^{\infty}\) and \((b_n)_{n = 1}^{\infty}\) are not eventually \(c / 2\)-close.
  Since \(c / 2 \in \Q^+\), this implies that \((a_n)_{n = 1}^{\infty}\) and \((b_n)_{n = 1}^{\infty}\) are not equivalent.
  But this contradicts the fact that both these sequences have \(x\) as their formal limit.
\end{proof}

\begin{note}
  Eventually, we will see a better explanation of \cref{i:5.4.9}:
  the set of non-negative reals is \emph{closed}, whereas the set of positive reals is \emph{open}.
  See \cref{i:sec:11.4}.
\end{note}

\begin{cor}\label{i:5.4.10}
  Let \((a_n)_{n = 1}^{\infty}\) and \((b_n)_{n = 1}^{\infty}\) be Cauchy sequences of rationals such that \(a_n \geq b_n\) for all \(n \in \Z^+\).
  Then \(\LIM_{n \to \infty} a_n \geq \LIM_{n \to \infty} b_n\).
\end{cor}

\begin{proof}[\pf{i:5.4.10}]
  Apply \cref{i:5.4.9} to the sequence \((a_n - b_n)_{n = 1}^\infty\).
\end{proof}

\begin{rmk}\label{i:5.4.11}
  Note that \cref{i:5.4.10} does not work if the \(\geq\) signs are replaced by \(>\):
  for instance if \(a_n \coloneqq 1 + 1 / n\) and \(b_n \coloneqq 1 - 1 / n\), then \(a_n\) is always strictly greater than \(b_n\), but the formal limit of \(a_n\) is not greater than the formal limit of \(b_n\), instead they are equal.
\end{rmk}

\begin{ac}\label{i:ac:5.4.1}
  We now define distance \(d(x, y) \coloneqq \abs{x - y}\) just as we did for the rationals.
  In fact, \cref{i:4.3.3,i:4.3.7} hold not only for the rationals, but for the reals;
  the proof is identical, since the real numbers obey all the laws of algebra and order that the rationals do.
\end{ac}

\begin{prop}[Bounding of reals by rationals]\label{i:5.4.12}
  Let \(x\) be a positive real number.
  Then there exists a positive rational number \(q\) such that \(q \leq x\), and there exists a positive integer \(N\) such that \(x \leq N\).
\end{prop}

\begin{proof}[\pf{i:5.4.12}]
  Since \(x\) is a positive real, it is the formal limit of a rational Cauchy sequence \((a_n)_{n = 1}^{\infty}\) which is positively bounded away from \(0\).
  Also, by \cref{i:5.1.15}, this sequence is bounded.
  Thus, we have rationals \(q, r \in \Q^+\) such that \(q \leq a_n \leq r\) for all \(n \in \Z^+\).
  But by \cref{i:4.4.1} we know that there is some integer \(N\) such that \(r \leq N\);
  since \(q\) is positive and \(q \leq r \leq N\), we see that \(N\) is positive.
  Thus, \(q \leq a_n \leq N\) for all \(n \in \Z^+\).
  Applying \cref{i:5.4.10} we obtain that \(q \leq x \leq N\), as desired.
\end{proof}

\begin{cor}[Archimedean property]\label{i:5.4.13}
  Let \(x\) and \(\varepsilon\) be any positive real numbers.
  Then there exists a positive integer \(M\) such that \(M\varepsilon > x\).
\end{cor}

\begin{proof}[\pf{i:5.4.13}]
  The number \(x / \varepsilon\) is positive, and hence by \cref{i:5.4.12} there exists a positive integer \(N\) such that \(x / \varepsilon \leq N\).
  If we set \(M \coloneqq N + 1\), then \(x / \varepsilon < M\).
  Now multiply by \(\varepsilon\).
\end{proof}

\begin{note}
  This property (\cref{i:5.4.13}) is quite important;
  it says that no matter how large \(x\) is and how small \(\varepsilon\) is, if one keeps adding \(\varepsilon\) to itself, one will eventually overtake \(x\).
\end{note}

\begin{prop}\label{i:5.4.14}
  Given any two real numbers \(x < y\), we can find a rational number \(q\) such that \(x < q < y\).
\end{prop}

\begin{proof}[\pf{i:5.4.14}]
  First, observe that
  \begin{align*}
             & x < y                                                    \\
    \implies & y > x                               &  & \by{i:5.4.7}    \\
    \implies & y - x \text{ is positive}           &  & \by{i:5.4.6}    \\
    \implies & \exists N \in \Z^+ : y - x > 1 / N  &  & \by{i:ex:5.4.4} \\
    \implies & \exists N \in \Z^+ : y > x + 1 / N. &  & \by{i:5.4.7}
  \end{align*}
  Fix one such \(N\).
  Since \(x\) is a real number, by \cref{i:5.3.10} we know that \(Nx\) is also a real number.
  By \cref{i:ex:5.4.3}, there exists an \(M \in \Z\) such that \(M \leq Nx < M + 1\).
  So we have
  \begin{align*}
             & M \leq Nx < M + 1                                                                                \\
    \implies & \dfrac{M}{N} \leq x < \dfrac{M + 1}{N}                                      &  & \by{i:5.4.7}[e] \\
    \implies & \pa{\dfrac{M + 1}{N} \leq x + \dfrac{1}{N}} \land \pa{x < \dfrac{M + 1}{N}} &  & \by{i:5.4.7}[d] \\
    \implies & x < \dfrac{M + 1}{N} \leq x + \dfrac{1}{N} < y.                             &  & \by{i:5.4.7}[c]
  \end{align*}
  Clearly, \((M + 1) / N \in \Q\).
  So by setting \(q = (M + 1) / N\) we are done.
\end{proof}

\begin{rmk}\label{i:5.4.15}
  Up until now, we have not addressed the fact that real numbers can be expressed using the decimal system.
  For instance, the formal limit of
  \[
    1.4, 1.41, 1.414, 1.4142, 1.41421, \dots
  \]
  is more conventionally represented as the decimal \(1.41421\dots\).
  There are some subtleties in the decimal system, for instance \(0.9999\dots\) and \(1.000\dots\) are in fact the same real number.
\end{rmk}

\begin{ac}\label{i:ac:5.4.2}
  Let \(X\) be an non-empty finite subset of \(\R\).
  Then \(X\) has exactly one maximum \(\max(X) \in X\) satisfying
  \[
    \forall x \in X, x \leq \max(X).
  \]
  Similarly, \(X\) has exactly one minimum \(\min(X) \in X\) satisfying
  \[
    \forall x \in X, x \geq \min(X).
  \]
\end{ac}

\begin{proof}[\pf{i:ac:5.4.2}]
  Let \(n = \#(X)\).
  We induct on \(n\) to show that \(\max(X) \in X\) and \(\min(X) \in X\), and we start with \(n = 1\).
  For \(n = 1\), we have \(X = \set{x}\) for some \(x \in \R\).
  Then we have
  \begin{align*}
             & \forall y \in X, y = x                                        &  & \by{i:3.3}   \\
    \implies & (\forall y \in X, y \leq x) \land (\forall y \in X, y \geq x) &  & \by{i:5.4.6} \\
    \implies & \max(X) = \min(X) = x.
  \end{align*}
  Thus, the base case holds.
  Suppose inductively that for some \(n \in \Z^+\) we have \(\max(X) \in X\) and \(\min(X) \in X\).
  Now let \(X\) be a finite subset of real number with \(\#(X) = n + 1\).
  Let \(x \in X\) and let \(X' = X \setminus \set{x}\).
  By \cref{i:3.6.9} we know that \(\#(X') = n\).
  Since \(n \in \Z^+\), we know that \(X'\) is non-empty.
  Then we have
  \begin{align*}
             & (\max(X') \in X') \land (\min(X') \in X')                         &  & \byIH            \\
    \implies & (\max(X') \in X) \land (\min(X') \in X)                           &  & (X' \subseteq X) \\
    \implies & \begin{dcases}
                 \forall y \in X', y \leq \max(X') \leq \max(\max(X'), x) \in X \\
                 \forall y \in X', y \geq \min(X') \geq \min(\min(X'), x) \in X \\
                 x \leq \max(\max(X'), x)                                       \\
                 x \geq \min(\min(X'), x)
               \end{dcases} &  & \by{i:5.4.7}[a]                          \\
    \implies & \begin{dcases}
                 \forall y \in X, y \leq \max(\max(X'), x) \\
                 \forall y \in X, y \geq \min(\min(X'), x)
               \end{dcases}                                               \\
    \implies & \begin{dcases}
                 \max(X) = \max(\max(X'), x) \\
                 \min(X) = \min(\min(X'), x)
               \end{dcases}.
  \end{align*}
  This closes the induction.

  Now we show that both \(\max(X), \min(X)\) are unique.
  Suppose there are two \(x, x' \in X\) such that \(y \leq x\) and \(y \leq x'\) for all \(y \in X\).
  But since \(x, x' \in X\), we have \(x \leq x'\) and \(x' \leq x\).
  Thus, by \cref{i:5.4.7}(a) we must have \(x = x'\) and \(\max(X)\) is unique.
  Using similarly arguments, we can show that \(\min(X)\) is unique.
\end{proof}

\begin{ac}\label{i:ac:5.4.3}
  Let \(x \in \R\).
  Then \(x\) is positive iff \(x > 0\);
  \(x\) is negative iff \(x < 0\).

  Let \(y \in \R\).
  We define the following eight subsets of \(\R\):
  \begin{align*}
    \R_{\leq x}   & \coloneqq \set{r \in \R : r \leq x};        & \R_{< x}   & \coloneqq \set{r \in \R : r < x};     & \R^+ & \coloneqq \R_{> 0}; \\
    \R_{\geq x}   & \coloneqq \set{r \in \R : r \geq x};        & \R_{> x}   & \coloneqq \set{r \in \R : r > x};     & \R^- & \coloneqq \R_{< 0}; \\
    \R_{x \leq y} & \coloneqq \set{r \in \R : x \leq r \leq y}; & \R_{x < y} & \coloneqq \set{r \in \R : x < r < y}. &      &
  \end{align*}
\end{ac}

\begin{proof}[\pf{i:ac:5.4.3}]
  By \cref{i:5.3.11} we have \(x = x - 0\).
  Thus, \(x\) is a positive rational number iff \(x - 0\) is a positive rational number, iff \(x > 0\) (\cref{i:5.4.6}).
  Similarly, \(x\) is a negative rational number iff \(x - 0\) is a negative rational number, iff \(x < 0\) (\cref{i:5.4.6}).
\end{proof}

\exercisesection

\begin{ex}\label{i:ex:5.4.1}
  Prove \cref{i:5.4.4}.
\end{ex}

\begin{proof}[\pf{i:ex:5.4.1}]
  See \cref{i:5.4.4}.
\end{proof}

\begin{ex}\label{i:ex:5.4.2}
  Prove the remaining claims in \cref{i:5.4.7}.
\end{ex}

\begin{proof}[\pf{i:ex:5.4.2}]
  See \cref{i:5.4.7}.
\end{proof}

\begin{ex}\label{i:ex:5.4.3}
  Show that for every real number \(x\) there is exactly one integer \(N\) such that \(N \leq x < N + 1\).
  (This integer \(N\) is called the \emph{integer part} of \(x\), and is sometimes denoted \(N = \floor{x}\).)
\end{ex}

\begin{proof}[\pf{i:ex:5.4.3}]
  We first prove the existence of the integer \(N\).
  By \cref{i:5.4.4}, exactly one of the following three statements is true:
  \begin{itemize}
    \item \(x = 0\).
          Then we choose \(N = 0\) so that \(0 \leq 0 < 1\).
    \item \(x\) is positive.
          Then by \cref{i:5.4.12} there exist some \(q \in \Q^+\) and \(N_1' \in \Z^+\) such that \(q \leq x \leq N_1'\).
          Let \(N_1 = N_1' + 1\).
          Then we have \(x < N_1\).
          By \cref{i:4.4.1} we know that there exists an \(N_2 \in \Z\) such that \(N_2 \leq q\), thus by \cref{i:5.4.7}(c) we have \(N_2 \leq x < N_1\).
          Let \(X\) be the set
          \[
            X = \set{n \in \Z : N_2 \leq n \leq x < N_1}.
          \]
          We know that \(X\) is finite since \(X \subseteq \Z_{N_2 \leq N_1}\) and \(\Z_{N_2 \leq N_1}\) is finite (\cref{i:3.6.14}(c)).
          We also know that \(X\) is non-empty since \(N_2 \in X\).
          By \cref{i:ac:5.4.2} we know that there exists an unique \(\max(X) \in X\).
          Let \(N = \max(X)\).
          By the definition of \(X\) we know that \(N \leq x < N_1\).
          We must have \(x < N + 1\), otherwise if \(N + 1 \leq x\) then we would have \(N + 1 \in X\), which means \(N + 1 \leq \max(X) = N\), a contradiction.
          Thus, we have \(N \leq x < N + 1\).
    \item \(x\) is negative.
          Then by \cref{i:5.4.4} we know that \(-x\) is positive.
          By \cref{i:5.4.13} we know that there exists an \(M \in \Z^+\) such that \(-x < 1M = M\).
          Fix such \(M\).
          By \cref{i:5.4.7}(d) we know that \(0 < x + M\).
          From the above case we know that there exists an \(K \in \Z\) such that \(K \leq x + M < K + 1\).
          So by setting \(N = K - M\) we see that \(N \leq x < N + 1\).
  \end{itemize}
  From all cases above, we conclude that for all \(x \in \R\), there exists an \(N \in \Z\) such that \(N \leq x < N + 1\).

  Now we prove the uniqueness of the integer \(N\).
  Let \(x \in \R\).
  Suppose that there exist some \(N_1, N_2 \in \Z\) such that \(N_1 \leq x < N_1 + 1\) and \(N_2 \leq x < N_2 + 1\).
  Then we have
  \begin{align*}
             & (N_1 \leq x < N_1 + 1) \land (N_2 \leq x < N_2 + 1)                      \\
    \implies & (N_1 < N_2 + 1) \land (N_2 < N_1 + 1)               &  & \by{i:5.4.7}[c] \\
    \implies & (N_1 + 1 \leq N_2 + 1) \land (N_2 + 1 \leq N_1 + 1) &  & \by{i:4.1.10}   \\
    \implies & (N_1 \leq N_2) \land (N_2 \leq N_1)                 &  & \by{i:4.1.10}   \\
    \implies & N_1 = N_2.                                          &  & \by{i:4.1.11}
  \end{align*}
  Thus, for all \(x \in \R\), \(N\) is unique and \(\floor{x}\) is well-defined.
\end{proof}

\begin{ex}\label{i:ex:5.4.4}
  Show that for any positive real number \(x > 0\) there exists a positive integer \(N\) such that \(x > 1 / N > 0\).
\end{ex}

\begin{proof}[\pf{i:ex:5.4.4}]
  We have
  \begin{align*}
             & x > 0                                                   \\
    \implies & x^{-1} > 0                           &  & \by{i:5.4.8}  \\
    \implies & \exists N \in \Z^+ : N1 = N > x^{-1} &  & \by{i:5.4.13} \\
    \implies & N^{-1} = \dfrac{1}{N} < x.           &  & \by{i:5.4.8}
  \end{align*}
\end{proof}

\begin{ex}\label{i:ex:5.4.5}
  Prove \cref{i:5.4.14}.
\end{ex}

\begin{proof}[\pf{i:ex:5.4.5}]
  See \cref{i:5.4.14}.
\end{proof}

\begin{ex}\label{i:ex:5.4.6}
  Let \(x, y \in \R\) and let \(\varepsilon \in \R^+\).
  Show that \(\abs{x - y} < \varepsilon\) iff \(y - \varepsilon < x < y + \varepsilon\), and that \(\abs{x - y} \leq \varepsilon\) iff \(y - \varepsilon \leq x \leq y + \varepsilon\).
\end{ex}

\begin{proof}[\pf{i:ex:5.4.6}]
  We first show that \(\abs{x - y} < \varepsilon \iff y - \varepsilon < x < y + \varepsilon\).
  \begin{align*}
         & \abs{x - y} < \varepsilon                                                                               \\
    \iff & \pa{-(x - y) \leq x - y < \varepsilon} \lor \pa{x - y \leq -(x - y) < \varepsilon} &  & \by{i:5.4.5}    \\
    \iff & (x - y < \varepsilon) \land (-(x - y) < \varepsilon)                               &  & \by{i:5.4.7}[a] \\
    \iff & (x - y < \varepsilon) \land (y - x < \varepsilon)                                  &  & \by{i:5.3.11}   \\
    \iff & (x < y + \varepsilon) \land (y - \varepsilon < x)                                  &  & \by{i:5.4.7}[d] \\
    \iff & y - \varepsilon < x < y + \varepsilon.                                             &  & \by{i:5.4.7}[c]
  \end{align*}

  Now we show that \(\abs{x - y} \leq \varepsilon \iff y - \varepsilon \leq x \leq y + \varepsilon\).
  By replacing \(<\) with \(\leq\) in the above arguments, we are done.
\end{proof}

\begin{ex}\label{i:ex:5.4.7}
  Let \(x, y \in \R\).
  Show that \(x \leq y + \varepsilon\) for all \(\varepsilon \in \R^+\) iff \(x \leq y\).
  Show that \(\abs{x - y} \leq \varepsilon\) for all \(\varepsilon \in \R^+\) iff \(x = y\).
\end{ex}

\begin{proof}[\pf{i:ex:5.4.7}]
  We first show that \(x \leq y + \varepsilon\) for all \(\varepsilon \in \R^+\) iff \(x \leq y\).
  \begin{align*}
         & \forall \varepsilon \in \R^+, x \leq y + \varepsilon                      \\
    \iff & \forall \varepsilon \in \R^+, x - y \leq \varepsilon &  & \by{i:5.4.7}[d] \\
    \iff & \lnot (x - y > 0)                                                         \\
    \iff & x - y \leq 0                                         &  & \by{i:5.4.7}[a] \\
    \iff & x \leq y.                                            &  & \by{i:5.4.7}[d]
  \end{align*}

  Now we show that \(\abs{x - y} \leq \varepsilon\) for all \(\varepsilon \in \R^+\) iff \(x = y\).
  \begin{align*}
         & \forall \varepsilon \in \R^+, \abs{x - y} \leq \varepsilon                                                   \\
    \iff & \forall \varepsilon \in \R^+, y - \varepsilon \leq x \leq y + \varepsilon &  & \by{i:ex:5.4.6}               \\
    \iff & (x \leq y) \land (y \leq x)                                               &  & \text{(from the proof above)} \\
    \iff & x = y.                                                                    &  & \by{i:5.4.7}[a]
  \end{align*}
\end{proof}

\begin{ex}\label{i:ex:5.4.8}
  Let \((a_n)_{n = 1}^{\infty}\) be a Cauchy sequence of rationals, and let \(x\) be a real number.
  Show that if \(a_n \leq x\) for all \(n \in \Z^+\), then \(\LIM_{n \to \infty} a_n \leq x\).
  Similarly, show that if \(a_n \geq x\) for all \(n \in \Z^+\), then \(\LIM_{n \to \infty} a_n \geq x\).
\end{ex}

\begin{proof}[\pf{i:ex:5.4.8}]
  We first show that if \(a_n \leq x\) for all \(n \in \Z^+\), then \(\LIM_{n \to \infty} a_n \leq x\).
  Let \(a = \LIM_{n \to \infty} a_n\).
  Suppose for the sake of contradiction that \(a > x\).
  Then by \cref{i:5.4.14}, there exists a \(q \in \Q\) such that \(a > q > x\).
  Since \(q > x\), we have \(a_n \leq x < q\) for all \(n \in \Z^+\).
  But by \cref{i:5.4.10} we have \(a = \LIM_{n \to \infty} a_n \leq \LIM_{n \to \infty} q = q\), which contradict to \(a > q\).
  Thus, we must have \(a \leq x\).

  Now we show that if \(a_n \geq x\) for all \(n \in \Z^+\), then \(\LIM_{n \to \infty} a_n \geq x\).
  This is true since
  \begin{align*}
             & a_n \geq x                                                          \\
    \implies & -a_n \leq -x                     &  & \by{i:ex:4.2.6}               \\
    \implies & \LIM_{n \to \infty} -a_n \leq -x &  & \text{(from the proof above)} \\
    \implies & -\LIM_{n \to \infty} a_n \leq -x &  & \by{i:5.3.10}                 \\
    \implies & \LIM_{n \to \infty} a_n \geq x.  &  & \by{i:ex:4.2.6}
  \end{align*}
\end{proof}
