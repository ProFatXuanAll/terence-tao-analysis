\chapter{The real numbers}\label{i:ch:5}

\begin{note}
  To review our progress to date, we have rigorously constructed three fundamental number systems:
  the natural number system \(\N\), the integers \(\Z\), and the rationals \(\Q\).
  We defined the natural numbers using the five Peano axioms (\crefrange{i:2.1}{i:2.5}), and postulated that such a number system existed;
  this is plausible, since the natural numbers correspond to the very intuitive and fundamental notion of \emph{sequential counting}.
  Using that number system one could then recursively define addition and multiplication, and verify that they obeyed the usual laws of algebra.
  We then constructed the integers by taking formal differences of the natural numbers, \(a \mm b\).
  We then constructed the rationals by taking formal quotients of the integers, \(a // b\), although we need to exclude division by zero in order to keep the laws of algebra reasonable.
  (You are of course free to design your own number system, possibly including one where division by zero is permitted;
  but you will have to give up one or more of the field axioms from \cref{i:4.2.4}, among other things, and you will probably get a less useful number system in which to do any real-world problems.)
\end{note}

\begin{note}
  The symbols \(\N\), \(\Q\), and \(\R\) stand for ``natural,'' ``quotient,'' and ``real'' respectively.
  \(\Z\) stands for ``Zahlen,'' the German word for numbers.
  There is also the \emph{complex numbers} \(\C\), which obviously stands for ``complex.''
\end{note}

\begin{note}
  \emph{Formal} means ``having the form of'';
  at the beginning of our construction the expression \(a \mm b\) did not actually \emph{mean} the difference \(a - b\), since the symbol \(\mm\) was meaningless.
  It only had the \emph{form} of a difference.
  Later on we defined subtraction and verified that the formal difference was equal to the actual difference, so this eventually became a non-issue, and our symbol for formal differencing was discarded.
  Somewhat confusingly, this use of the term ``formal'' is unrelated to the notions of a formal argument and an informal argument.
\end{note}

\begin{note}
  There is a fundamental area of mathematics where the rational number system does not suffice - that of \emph{geometry}
  (the study of lengths, areas, etc.).
  For instance, a right-angled triangle with both sides equal to \(1\) gives a hypotenuse of \(\sqrt{2}\), which is an \emph{irrational} number, i.e., not a rational number;
  see \cref{i:4.4.4}.
  Things get even worse when one starts to deal with the sub-field of geometry known as \emph{trigonometry}, when one sees numbers such as \(\pi\) or \(\cos(1)\), which turn out to be in some sense ``even more'' irrational than \(\sqrt{2}\).
  (These numbers are known as \emph{transcendental numbers}, but to discuss this further would be far beyond the scope of this text.)
  Thus, in order to have a number system which can adequately describe geometry
  - or even something as simple as measuring lengths on a line
  - one needs to replace the rational number system with the real number system.
  Since differential and integral calculus is also intimately tied up with geometry
  - think of slopes of tangents, or areas under a curve
  - calculus also requires the real number system in order to function properly.
\end{note}

\begin{note}
  In the constructions of integers and rationals, the task was to introduce one more \emph{algebraic} operation to the number system
  - e.g., one can get integers from naturals by introducing subtraction, and get the rationals from the integers by introducing division.
  But to get the reals from the rationals is to pass from a ``discrete'' system to a ``continuous'' one, and requires the introduction of a somewhat different notion
  - that of a \emph{limit}.
\end{note}

\begin{note}
  The limit is a concept which on one level is quite intuitive, but to pin down rigorously turns out to be quite difficult.
  (Even such great mathematicians as Euler and Newton had difficulty with this concept.
  It was only in the nineteenth century that mathematicians such as Cauchy and Dedekind figured out how to deal with limits rigorously.)
\end{note}

\begin{note}
  The real number system will end up being a lot like the rational numbers, but will have some new operations
  - notably that of \emph{supremum}, which can then be used to define limits and thence to everything else that calculus needs.
\end{note}

\begin{note}
  The procedure we give here of obtaining the real numbers as the limit of sequences of rational numbers may seem rather complicated.
  However, it is in fact an instance of a very general and useful procedure, that of \emph{completing} one metric space to form another.
  see \cref{ii:ex:1.4.8}.
\end{note}

\subimport{./}{sec1.tex}
\subimport{./}{sec2.tex}
\subimport{./}{sec3.tex}
\subimport{./}{sec4.tex}
\subimport{./}{sec5.tex}
\subimport{./}{sec6.tex}
