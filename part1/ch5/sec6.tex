\section{Real exponentiation, part I}\label{i:sec:5.6}

\begin{defn}[Exponentiating a real by a natural number]\label{i:5.6.1}
  Let \(x \in \R\).
  To raise \(x\) to the power \(0\), we define \(x^0 \coloneqq 1\).
  Now suppose recursively that \(x^n\) has been defined for some natural number \(n\), then we define \(x^{n + 1} \coloneqq x^n \times x\).
\end{defn}

\begin{defn}[Exponentiating a real by an integer]\label{i:5.6.2}
  Let \(x\) be a non-zero real number.
  Then for any negative integer \(-n\), we define \(x^{-n} \coloneqq 1 / x^n\).
\end{defn}

\begin{prop}\label{i:5.6.3}
  All the properties in \cref{i:4.3.10,i:4.3.12} remain valid if \(x\) and \(y\) are assumed to be real numbers instead of rational numbers.
\end{prop}

\begin{meta-proof}[\pf{i:5.6.3}]
If one inspects the proof of \cref{i:4.3.10,i:4.3.12} we see that they rely on the laws of algebra and the laws of order for the rationals (\cref{i:4.2.4,i:4.2.9}).
But by \cref{i:5.3.11,i:5.4.7}, and the identity \(xx^{-1} = x^{-1} x = 1\) we know that all these laws of algebra and order continue to hold for real numbers as well as rationals.
Thus, we can modify the proof of \cref{i:4.3.10,i:4.3.12} to hold in the case when \(x\) and \(y\) are real.
\end{meta-proof}

\begin{note}
  Instead of giving an actual proof of \cref{i:5.6.3}, we shall give a meta-proof
  (an argument appealing to the nature of proofs, rather than the nature of real and rational numbers).
\end{note}

\begin{defn}\label{i:5.6.4}
  Let \(x \in \R_{\geq 0}\), and let \(n \in \Z^+\).
  We define \(x^{1 / n}\), also known as the \emph{\(n^{\opTh}\) root of \(x\)}, by the formula
  \[
    x^{1 / n} \coloneqq \sup\set{y \in \R : y \geq 0 \text{ and } y^n \leq x}.
  \]
  We often write \(\sqrt{x}\) for \(x^{1 / 2}\).
\end{defn}

\begin{note}
  We do not define the \(n^{\opTh}\) roots of a negative number.
  In fact, we will leave the \(n^{\opTh}\) roots of negative numbers undefined for the rest of the text
  (one can define these \(n^{\opTh}\) roots once one defines the complex numbers, but we shall refrain from doing so).
\end{note}

\begin{lem}[Existence of \(n^{\opTh}\) roots]\label{i:5.6.5}
  Let \(x \in \R_{\geq 0}\), and let \(n \in \Z^+\).
  Then the set \(E \coloneqq \set{y \in \R : y \geq 0 \text{ and } y^n \leq x}\) is non-empty and is also bounded above.
  In particular, \(x^{1 / n}\) is a real number.
\end{lem}

\begin{proof}[\pf{i:5.6.5}]
  The set \(E\) contains \(0\), so it is certainly not empty.
  Now we show it has an upper bound.
  We divide into two cases: \(x \leq 1\) and \(x > 1\).
  First, suppose that we are in the case where \(x \leq 1\).
  Then we claim that the set \(E\) is bounded above by \(1\).
  To see this, suppose for the sake of contradiction that there was an element \(y \in E\) for which \(y > 1\).
  But then \(y^n > 1\), and hence \(y^n > x\), a contradiction.
  Thus, \(E\) has an upper bound.
  Now suppose that we are in the case where \(x > 1\).
  Then we claim that the set \(E\) is bounded above by \(x\).
  To see this, suppose for contradiction that there was an element \(y \in E\) for which \(y > x\).
  Since \(x > 1\), we thus have \(y > 1\).
  Since \(y > x\) and \(y > 1\), we have \(y^n > x\), a contradiction.
  Thus, in both cases \(E\) has an upper bound, and so \(x^{1 / n}\) is finite.
\end{proof}

\begin{lem}\label{i:5.6.6}
  Let \(x, y \in \R_{\geq 0}\), and let \(n, m \in \Z^+\).
  \begin{enumerate}
    \item If \(y = x^{1 / n}\), then \(y^n = x\).
    \item Conversely, if \(y^n = x\), then \(y = x^{1 / n}\).
    \item \(x^{1 / n}\) is a non-negative real number, and is positive iff \(x\) is positive.
    \item We have \(x > y\) iff \(x^{1 / n} > y^{1 / n}\).
    \item Let \(k, l \in \Z^+\).
          If \(x > 1\), then \(x^{1 / k}\) is a decreasing (i.e., \(x^{1 / k} > x^{1 / l}\) whenever \(k < l\)) function of \(k\).
          If \(0 < x < 1\), then \(x^{1 / k}\) is an increasing (i.e., \(x^{1 / k} < x^{1 / l}\) whenever \(k < l\)) function of \(k\).
          If \(x = 1\), then \(x^{1 / k} = 1\) for all \(k\).
    \item We have \((xy)^{1 / n} = x^{1 / n} y^{1 / n}\).
    \item We have \((x^{1 / n})^{1 / m} = x^{1 / nm}\).
  \end{enumerate}
\end{lem}

\begin{proof}[\pf{i:5.6.6}(a)]
  Let \(E = \set{z \in \R_{\geq 0} : z^n \leq x}\).
  By \cref{i:5.6.5} we have \(y = x^{1 / n} = \sup(E)\).
  Suppose for the sake of contradiction that \(y^n \neq x\).
  Then by \cref{i:5.4.7}(a) we have either \(y^n < x\) or \(y^n > x\).
  \begin{itemize}
    \item If \(y^n < x\), then we can find a small number \(\varepsilon \in \Q_{0 < 1}\) such that \((y + \varepsilon)^n < x\)
          (this can be proved by induction, and using the binomial formula, see \cref{i:ex:7.1.4}).
          But by the definition of \(E\) we see that \(y + \varepsilon \in E\), so we must have \(y + \varepsilon \leq y\), a contradiction.
    \item If \(y^n > x\), then we can find a small number \(\varepsilon \in \Q_{0 < 1}\) such that \((y - \varepsilon)^n > x\).
          Then by the definition of \(E\) we see that \((y - \varepsilon)^n > x\) implies \(y - \varepsilon\) is an upper bound of \(E\).
          But by \cref{i:5.5.5} this means \(y - \varepsilon \geq y\), a contradiction.
  \end{itemize}
  From all cases above, we derived contradictions.
  So we must have \(y^n = x\).
\end{proof}

\begin{proof}[\pf{i:5.6.6}(b)]
  Let \(E = \set{z \in \R_{\geq 0} : z^n \leq x}\).
  By \cref{i:5.6.5} we have \(x^{1 / n} = \sup(E)\).
  Since \(y^n = x\), by the definition of \(E\) we know that \(y \in E\).
  Suppose for the sake of contradiction that \(y \neq x^{1 / n}\).
  Then by \cref{i:5.4.7}(a) exactly one of the following statements is true:
  \begin{itemize}
    \item \(y < x^{1 / n}\).
          But then we have
          \begin{align*}
            x & = y^n                                \\
              & < (x^{1 / n})^n &  & \by{i:5.6.3}    \\
              & = x,            &  & \by{i:5.6.6}[a]
          \end{align*}
          a contradiction.
    \item \(y > x^{1 / n}\).
          But then we have
          \begin{align*}
            x & = y^n                                \\
              & > (x^{1 / n})^n &  & \by{i:5.6.3}    \\
              & = x,            &  & \by{i:5.6.6}[a]
          \end{align*}
          a contradiction.
  \end{itemize}
  From all cases above, we derived contradictions.
  So we must have \(y = x^{1 / n}\).
\end{proof}

\begin{proof}[\pf{i:5.6.6}(c)]
  Let \(E = \set{z \in \R_{\geq 0} : z^n \leq x}\).
  By \cref{i:5.6.5} we have \(x^{1 / n} = \sup(E)\).
  Since \(0 \in E\), by \cref{i:5.5.5} we know that \(0 \leq x^{1 / n}\), thus \(x^{1 / n}\) is a non-negative real number.

  Next suppose that \(x^{1 / n} \in \R^+\).
  Then we have
  \begin{align*}
             & x^{1 / n} \in \R^+                                       \\
    \implies & x = (x^{1 / n})^n \in \R^+. &  & \by{i:5.6.3,i:5.6.6}[a]
  \end{align*}

  Finally, suppose that \(x \in \R^+\).
  Suppose for the sake of contradiction that \(x^{1 / n} \notin \R^+\).
  Then from the proof above we know that \(x^{1 / n} = 0\).
  But by \cref{i:5.6.6}(a) we have
  \[
    x = (x^{1 / n})^n = 0^n = 0,
  \]
  a contradiction.
  Thus, we must have \(x^{1 / n} \in \R^+\).
  We conclude that \(x^{1 / n} \in \R^+\) iff \(x \in \R^+\).
\end{proof}

\begin{proof}[\pf{i:5.6.6}(d)]
  We first show that \(x^{1 / n} > y^{1 / n} \implies x > y\).
  \begin{align*}
             & x^{1 / n} > y^{1 / n}                              \\
    \implies & (x^{1 / n})^n > (y^{1 / n})^n &  & \by{i:5.6.3}    \\
    \implies & x > y.                        &  & \by{i:5.6.6}[a]
  \end{align*}

  Now we show that \(x > y \implies x^{1 / n} > y^{1 / n}\).
  Suppose for the sake of contradiction that \(x^{1 / n} \leq y^{1 / n}\).
  But then we have
  \begin{align*}
             & x^{1 / n} \leq y^{1 / n}                              \\
    \implies & (x^{1 / n})^n \leq (y^{1 / n})^n &  & \by{i:5.6.3}    \\
    \implies & x \leq y,                        &  & \by{i:5.6.6}[a]
  \end{align*}
  a contradiction.
  Thus, we must have \(x^{1 / n} > y^{1 / n}\).
  We conclude that \(x > y \iff x^{1 / n} > y^{1 / n}\).
\end{proof}

\begin{proof}[\pf{i:5.6.6}(e)]
  If \(x = 0\), then by \cref{i:5.6.6}(c) we know that \(x^{1 / k} = 0\) for every \(k \in \Z^+\).
  Thus, we only consider the case \(x \in \R^+\).

  We first show that if \(x > 1\), then \(x^{1 / k}\) is a decreasing function of \(k \in \Z^+\).
  Suppose for the sake of contradiction that there exists a \(k \in \Z^+\) such that \(x^{1 / k} \leq x^{1 / (k + 1)}\).
  But then we have
  \begin{align*}
             & x^{1 / k} \leq x^{1 / (k + 1)}                                                                                                      \\
    \implies & (x^{1 / k})^k \leq (x^{1 / (k + 1)})^k                                                                       &  & \by{i:5.6.3}      \\
    \implies & x \leq (x^{1 / (k + 1)})^k                                                                                   &  & \by{i:5.6.6}[a]   \\
    \implies & (x^{1 / (k + 1)})^{k + 1} \leq (x^{1 / (k + 1)})^k                                                           &  & \by{i:5.6.6}[a,b] \\
    \implies & (x^{1 / (k + 1)})^{k + 1} \cdot (x^{1 / (k + 1)})^{-1} \leq (x^{1 / (k + 1)})^k \cdot (x^{1 / (k + 1)})^{-1} &  & \by{i:5.6.6}[c]   \\
    \implies & (x^{1 / (k + 1)})^{k + 1} \cdot (x^{1 / (k + 1)})^{-k} \leq (x^{1 / (k + 1)})^k \cdot (x^{1 / (k + 1)})^{-k} &  & \by{i:5.6.3}      \\
    \implies & x^{1 / (k + 1)} \leq 1                                                                                       &  & \by{i:5.6.3}      \\
    \implies & (x^{1 / (k + 1)})^{k + 1} \leq 1^{k + 1} = 1                                                                 &  & \by{i:5.6.3}      \\
    \implies & x \leq 1,                                                                                                    &  & \by{i:5.6.6}[a]
  \end{align*}
  a contradiction.
  Thus, such \(k\) does not exist, and \(x^{1 / k}\) is a decreasing function of \(k\) when \(x > 1\).

  Next we show that if \(x < 1\), then \(x^{1 / k}\) is an increasing function of \(k \in \Z^+\).
  Suppose for the sake of contradiction that there exists a \(k \in \Z^+\) such that \(x^{1 / k} \geq x^{1 / (k + 1)}\).
  But then we have
  \begin{align*}
             & x^{1 / k} \geq x^{1 / (k + 1)}                                                                                                      \\
    \implies & (x^{1 / k})^k \geq (x^{1 / (k + 1)})^k                                                                       &  & \by{i:5.6.3}      \\
    \implies & x \geq (x^{1 / (k + 1)})^k                                                                                   &  & \by{i:5.6.6}[a]   \\
    \implies & (x^{1 / (k + 1)})^{k + 1} \geq (x^{1 / (k + 1)})^k                                                           &  & \by{i:5.6.6}[a,b] \\
    \implies & (x^{1 / (k + 1)})^{k + 1} \cdot (x^{1 / (k + 1)})^{-1} \geq (x^{1 / (k + 1)})^k \cdot (x^{1 / (k + 1)})^{-1} &  & \by{i:5.6.6}[c]   \\
    \implies & (x^{1 / (k + 1)})^{k + 1} \cdot (x^{1 / (k + 1)})^{-k} \geq (x^{1 / (k + 1)})^k \cdot (x^{1 / (k + 1)})^{-k} &  & \by{i:5.6.3}      \\
    \implies & x^{1 / (k + 1)} \geq 1                                                                                       &  & \by{i:5.6.3}      \\
    \implies & (x^{1 / (k + 1)})^{k + 1} \geq 1^{k + 1} = 1                                                                 &  & \by{i:5.6.3}      \\
    \implies & x \geq 1,                                                                                                    &  & \by{i:5.6.6}[a]
  \end{align*}
  a contradiction.
  Thus, such \(k\) does not exist, and \(x^{1 / k}\) is a increasing function of \(k\) when \(x < 1\).

  Finally we show that if \(x = 1\), then \(x^{1 / k} = 1\) for every \(k \in \Z^+\).
  Suppose for the sake of contradiction that there exists a \(k \in \Z^+\) such that \(x^{1 / k} \neq 1\).
  Then by \cref{i:5.4.7} exactly one of the following two statements is true:
  \begin{itemize}
    \item \(x^{1 / k} > 1\).
          But then we have
          \begin{align*}
                     & (x^{1 / k})^k > 1^k = 1 &  & \by{i:5.6.3}    \\
            \implies & x > 1,                  &  & \by{i:5.6.6}[a]
          \end{align*}
          a contradiction.
    \item \(x^{1 / k} < 1\).
          But then we have
          \begin{align*}
                     & (x^{1 / k})^k < 1^k = 1 &  & \by{i:5.6.3}    \\
            \implies & x < 1,                  &  & \by{i:5.6.6}[a]
          \end{align*}
          a contradiction.
  \end{itemize}
  From all cases above, we derived contradictions.
  Thus, we must have \(x^{1 / k} = 1\) for all \(k \in \Z^+\).
\end{proof}

\begin{proof}[\pf{i:5.6.6}(f)]
  We have
  \begin{align*}
    ((xy)^{1 / n})^n & = xy                          &  & \by{i:5.6.6}[a]   \\
                     & = (x^{1 / n})^n (y^{1 / n})^n &  & \by{i:5.6.6}[a,b] \\
                     & = (x^{1 / n} y^{1 / n})^n.    &  & \by{i:5.6.3}
  \end{align*}
  Thus, by \cref{i:5.6.6}(b) we have \((xy)^{1 / n} = x^{1 / n} y^{1 / n}\).
\end{proof}

\begin{proof}[\pf{i:5.6.6}(g)]
  We have
  \begin{align*}
    (x^{1 / nm})^{nm} & = x                                 &  & \by{i:5.6.6}[a]   \\
                      & = (x^{1 / n})^n                     &  & \by{i:5.6.6}[a,b] \\
                      & = \pa{\pa{(x^{1 / n})^{1 / m}}^m}^n &  & \by{i:5.6.6}[a,b] \\
                      & = \pa{(x^{1 / n})^{1 / m}}^{nm}.    &  & \by{i:5.6.3}
  \end{align*}
  Thus, by \cref{i:5.6.6}(b) we have \(x^{1 / nm} = (x^{1 / n})^{1 / m}\).
\end{proof}

\begin{note}
  The observant reader may note that this definition of \(x^{1 / n}\) might possibly be inconsistent with our previous notion of \(x^n\) when \(n = 1\), but it is easy to check that \(x^{1 / 1} = x = x^1\) by using \cref{i:5.6.6}(e), so there is no inconsistency.
\end{note}

\begin{note}
  One consequence of \cref{i:5.6.6}(b) is another proof of the cancellation law from \cref{i:4.3.12}(c) and \cref{i:5.6.3}:
  if \(y\) and \(z\) are positive and \(y^n = z^n\), then \(y = z\).
  This only works when \(y\) and \(z\) are positive;
  for instance, \((-3)^2 = 3^2\), but we cannot conclude from this that \(-3 = 3\).
\end{note}

\begin{defn}\label{i:5.6.7}
  Let \(x \in \R^+\), and let \(q \in \Q\).
  To define \(x^q\), we write \(q = a / b\) for some \(a \in \Z\) and \(b \in \Z^+\), and define
  \[
    x^q \coloneqq (x^{1 / b})^a.
  \]
\end{defn}

\begin{note}
  Every rational \(q\), whether positive, negative, or zero, can be written in the form \(a / b\) where \(a\) is an integer and \(b\) is positive.
  However, the rational number \(q\) can be expressed in the form \(a / b\) in more than one way, for instance \(1 / 2\) can also be expressed as \(2 / 4\) or \(3 / 6\).
  So to ensure that \cref{i:5.6.7} is well-defined, we need to check that different expressions \(a / b\) give the same formula for \(x^q\).
\end{note}

\begin{lem}\label{i:5.6.8}
  Let \(a, a'\) be integers and \(b, b'\) be positive integers such that \(a / b = a' / b'\), and let \(x\) be a positive real number.
  Then we have \((x^{1 / b'})^{a'} = (x^{1 / b})^a\).
\end{lem}

\begin{proof}[\pf{i:5.6.8}]
  There are three cases: \(a = 0, a > 0, a < 0\).
  If \(a = 0\), then we must have \(a' = 0\) and so both \((x^{1 / b'})^{a'}\) and \((x^{1 / b})^a\) are equal to 1, so we are done.

  Now suppose that \(a > 0\).
  Then \(a' > 0\), and \(ab' = ba'\).
  Write \(y \coloneqq x^{1 / (ab')} = x^{1 / (ba')}\).
  By \cref{i:5.6.6}(g) we have \(y = (x^{1 / b'})^{1 / a}\) and \(y = (x^{1 / b})^{1 / a'}\);
  by \cref{i:5.6.6}(a) we thus have \(y^{a'} = x^{1 / b}\) and \(y^a = x^{1 / b'}\).
  Thus, we have
  \[
    (x^{1 / b'})^{a'} = (y^a)^{a'} = y^{aa'} = (y^{a'})^a = (x^{1 / b})^a
  \]
  as desired.

  Finally, suppose that \(a < 0\).
  Then we have \((-a) / b = (-a') / b'\).
  But \(-a\) is positive, so the previous case applies and we have \((x^{1 / b'})^{-a'} = (x^{1 / b})^{-a}\).
  Taking the reciprocal of both sides we obtain the result.
\end{proof}

\begin{note}
  Thus, \(x^q\) is well-defined for every rational \(q\).
  \cref{i:5.6.7} is consistent with our old definition for \(x^{1 / n}\) (since \(x^{1 / n} = (x^{1 / n})^1\)) and is also consistent with our old definition for \(x^n\) (since \(x^n = (x^{1 / 1})^n\)).
\end{note}

\begin{lem}\label{i:5.6.9}
  Let \(x, y \in \R^+\), and let \(q, r \in \Q\).
  \begin{enumerate}
    \item \(x^q \in \R^+\).
    \item \(x^{q + r} = x^q x^r\) and \((x^q)^r = x^{qr}\).
    \item \(x^{-q} = 1 / x^q\).
    \item If \(q > 0\), then \(x > y\) iff \(x^q > y^q\).
    \item If \(x > 1\), then \(x^q > x^r\) iff \(q > r\).
          If \(x < 1\), then \(x^q > x^r\) iff \(q < r\).
    \item \((xy)^q = x^q y^q\).
  \end{enumerate}
\end{lem}

\begin{proof}[\pf{i:5.6.9}(a)]
  Let \(q = a / b\) where \(a \in \Z\) and \(b \in \Z^+\).
  Then we have
  \begin{align*}
             & x \in \R^+                                  \\
    \implies & x^{1 / b} \in \R^+     &  & \by{i:5.6.6}[c] \\
    \implies & (x^{1 / b})^a \in \R^+ &  & \by{i:5.6.3}    \\
    \implies & x^q \in \R^+.          &  & \by{i:5.6.7}
  \end{align*}
\end{proof}

\begin{proof}[\pf{i:5.6.9}(b)]
  Let \(q = a / b\) and \(r = c / d\) where \(a, c \in \Z\) and \(b, d \in \Z^+\).
  Then we have
  \begin{align*}
    x^{q + r} & = x^{(ad + bc) / bd}                  &  & \by{i:4.2.2}                \\
              & = (x^{1 / bd})^{(ad + bc)}            &  & \by{i:5.6.7}                \\
              & = (x^{1 / bd})^{ad} (x^{1 / bd})^{bc} &  & \by{i:5.6.3}                \\
              & = x^{ad / bd} x^{bc / bd}             &  & \text{(by  \cref{i:5.6.7})} \\
              & = x^{a / b} x^{c / d}                 &  & \by{i:5.6.8}                \\
              & = x^q x^r
  \end{align*}
  and
  \begin{align*}
    x^{qr} & = x^{ac / bd}                                       &  & \by{i:4.2.2}      \\
           & = (x^{1 / bd})^{ac}                                 &  & \by{i:5.6.7}      \\
           & = \pa{(x^{1 / b})^{1 / d}}^{ac}                     &  & \by{i:5.6.6}[g]   \\
           & = \pa{\pa{\pa{(x^{1 / b})^a}^{1 / a}}^{1 / d}}^{ac} &  & \by{i:5.6.6}[a,b] \\
           & = \pa{\pa{(x^{a / b})^{1 / a}}^{1 / d}}^{ac}        &  & \by{i:5.6.7}      \\
           & = \pa{(x^{a / b})^{1 / ad}}^{ac}                    &  & \by{i:5.6.6}[g]   \\
           & = (x^{a / b})^{ac / ad}                             &  & \by{i:5.6.7}      \\
           & = (x^{a / b})^{c / d}                               &  & \by{i:5.6.8}      \\
           & = (x^q)^r.
  \end{align*}
\end{proof}

\begin{proof}[\pf{i:5.6.9}(c)]
  Let \(q = a / b\) where \(a \in \Z\) and \(b \in \Z^+\).
  Then we have
  \begin{align*}
    x^{-q} & = x^{-a / b}                          \\
           & = (x^{1 / b})^{-a}  &  & \by{i:5.6.7} \\
           & = 1 / (x^{1 / b})^a &  & \by{i:5.6.3} \\
           & = 1 / x^q.          &  & \by{i:5.6.7}
  \end{align*}
\end{proof}

\begin{proof}[\pf{i:5.6.9}(d)]
  Let \(q = a / b\) where \(a \in \Z\) and \(b \in \Z^+\).
  Then we have
  \begin{align*}
             & x > y                                              \\
    \implies & x^{1 / b} > y^{1 / b}         &  & \by{i:5.6.6}[d] \\
    \implies & (x^{1 / b})^a > (y^{1 / b})^a &  & \by{i:5.6.3}    \\
    \implies & x^q > y^q                     &  & \by{i:5.6.7}
  \end{align*}
  and
  \begin{align*}
             & x^q > y^q                                                                      \\
    \implies & (x^{1 / b})^a > (y^{1 / b})^a                           &  & \by{i:5.6.7}      \\
    \implies & \pa{(x^{1 / b})^a}^{1 / a} > \pa{(y^{1 / b})^a}^{1 / a} &  & \by{i:5.6.6}[d]   \\
    \implies & x^{1 / b} > y^{1 / b}                                   &  & \by{i:5.6.6}[a,b] \\
    \implies & x > y.                                                  &  & \by{i:5.6.6}[d]
  \end{align*}
  Thus, we conclude that \(x > y \iff x^q > y^q\) when \(x, y \in \R^+\) and \(q \in \Q^+\).
\end{proof}

\begin{proof}[\pf{i:5.6.9}(e)]
  Let \(q = a / b\) and \(r = c / d\) where \(a, c \in \Z\) and \(b, d \in \Z^+\).
  First, suppose that \(x > 1\) and \(x^q > x^r\).
  Then we have
  \begin{align*}
             & x^q > x^r                                                       \\
    \implies & x^q x^{-r} > x^r x^{-r}                   &  & \by{i:5.6.9}[a]  \\
    \implies & x^{q - r} > x^{r - r} = x^0 = 1           &  & \by{i:5.6.9}[b]  \\
    \implies & x^{(ad - bc) / bd} > 1                    &  & \by{i:4.2.2}     \\
    \implies & \pa{x^{(ad - bc) / bd}}^{bd} > 1^{bd} = 1 &  & \by{i:5.6.3}     \\
    \implies & x^{(ad - bc)} > 1 = 1^{(ad - bc)}         &  & \by{i:5.6.9}[b]  \\
    \implies & ad - bc > 0                               &  & \by{i:4.3.12}[b] \\
    \implies & ad > bc                                   &  & \by{i:4.1.11}[b] \\
    \implies & a / b > c / d                             &  & \by{i:4.2.9}[e]  \\
    \implies & q > r.
  \end{align*}

  Now suppose that \(x > 1\) and \(q > r\).
  Then we have
  \begin{align*}
             & q > r                                                   \\
    \implies & q - r > 0                          &  & \by{i:4.2.9}    \\
    \implies & x^{q - r} > 1^{q - r}              &  & \by{i:5.6.9}[d] \\
    \implies & x^{q - r} > 1^{(ad - bc) / bd}     &  & \by{i:4.2.2}    \\
    \implies & x^{q - r} > (1^{1 / bd})^{ad - bc} &  & \by{i:5.6.7}    \\
    \implies & x^{q - r} > 1^{ad - bc} = 1        &  & \by{i:5.6.6}[e] \\
    \implies & x^{q - r} x^r > x^r                &  & \by{i:5.6.9}[a] \\
    \implies & x^q > x^r.                         &  & \by{i:5.6.9}[b]
  \end{align*}
  Thus, we conclude that if \(x > 1\), then \(x^q > x^r \iff q > r\).

  Next suppose that \(x < 1\) and \(x^q > x^r\).
  Then we have
  \begin{align*}
             & x^q > x^r                                                       \\
    \implies & x^q x^{-r} > x^r x^{-r}                   &  & \by{i:5.6.9}[a]  \\
    \implies & x^{q - r} > x^{r - r} = x^0 = 1           &  & \by{i:5.6.9}[b]  \\
    \implies & x^{(ad - bc) / bd} > 1                    &  & \by{i:4.2.2}     \\
    \implies & \pa{x^{(ad - bc) / bd}}^{bd} > 1^{bd} = 1 &  & \by{i:5.6.3}     \\
    \implies & x^{(ad - bc)} > 1 = 1^{(ad - bc)}         &  & \by{i:5.6.9}[b]  \\
    \implies & ad - bc < 0                               &  & \by{i:4.3.12}[b] \\
    \implies & ad < bc                                   &  & \by{i:4.1.11}[b] \\
    \implies & a / b < c / d                             &  & \by{i:4.2.9}[e]  \\
    \implies & q < r.
  \end{align*}

  Finally suppose that \(x < 1\) and \(q < r\).
  Then we have
  \begin{align*}
             & q < r                                                   \\
    \implies & r - q > 0                          &  & \by{i:4.2.9}    \\
    \implies & x^{r - q} < 1^{r - q}              &  & \by{i:5.6.9}[d] \\
    \implies & x^{r - q} < 1^{(bc - ad) / bd}     &  & \by{i:4.2.2}    \\
    \implies & x^{r - q} < (1^{1 / bd})^{bc - ad} &  & \by{i:5.6.7}    \\
    \implies & x^{r - q} < 1^{bc - ad} = 1        &  & \by{i:5.6.6}[e] \\
    \implies & x^{r - q} x^q < x^q                &  & \by{i:5.6.9}[a] \\
    \implies & x^r < x^q.                         &  & \by{i:5.6.9}[b]
  \end{align*}
  Thus, we conclude that if \(x < 1\), then \(x^q > x^r \iff q < r\).
\end{proof}

\begin{proof}[\pf{i:5.6.9}(f)]
  Let \(q = a / b\) where \(a \in \Z\) and \(b \in \Z^+\).
  Then we have
  \begin{align*}
    (xy)^q & = \pa{(xy)^{1 / b}}^a         &  & \by{i:5.6.7}    \\
           & = (x^{1 / b} y^{1 / b})^a     &  & \by{i:5.6.6}[f] \\
           & = (x^{1 / b})^a (y^{1 / b})^a &  & \by{i:5.6.3}    \\
           & = x^q y^q.                    &  & \by{i:5.6.7}
  \end{align*}
\end{proof}

\exercisesection

\begin{ex}\label{i:ex:5.6.1}
  Prove \cref{i:5.6.6}.
\end{ex}

\begin{proof}[\pf{i:ex:5.6.1}]
  See \cref{i:5.6.6}.
\end{proof}

\begin{ex}\label{i:ex:5.6.2}
  Prove \cref{i:5.6.9}.
\end{ex}

\begin{proof}[\pf{i:ex:5.6.2}]
  See \cref{i:5.6.9}.
\end{proof}

\begin{ex}\label{i:ex:5.6.3}
  If \(x\) is a real number, show that \(\abs{x} = (x^2)^{1 / 2}\).
\end{ex}

\begin{proof}[\pf{i:ex:5.6.3}]
  By \cref{i:5.4.7}(a) exactly one of the following three statements is true:
  \begin{itemize}
    \item \(x > 0\).
          Then by \cref{i:5.4.5} we have \(\abs{x} = x\) and by \cref{i:5.6.6}(a)(b) we have \(\abs{x} = x = (x^2)^{1 / 2}\).
    \item \(x = 0\).
          Then by \cref{i:5.4.5} we have \(\abs{0} = 0\) and by \cref{i:5.6.6}(c) we have \(\abs{0} = 0 = (0^2)^{1 / 2}\).
    \item \(x < 0\).
          Then by \cref{i:5.4.5} we have \(\abs{x} = -x > 0\).
          By \cref{i:5.6.6}(b)(c) we have \(-x = ((-x)^2)^{1 / 2}\).
          But by \cref{i:5.3.11} we know that \((-x)^2 = (-x)(-x) = x^2\).
          Thus, we have \(\abs{x} = -x = (x^2)^{1 / 2}\).
  \end{itemize}
  From all cases above, we conclude that \(\abs{x} = (x^2)^{1 / 2}\).
\end{proof}
