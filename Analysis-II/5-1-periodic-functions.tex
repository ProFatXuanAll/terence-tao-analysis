\section{Periodic functions}\label{sec 5.1}

\begin{definition}\label{5.1.1}
    Let \(L > 0\) be a real number.
    A function \(f : \mathbf{R} \to \mathbf{C}\) is periodic with period \(L\), or \(L\)-periodic, if we have \(f(x + L) = f(x)\) for every real number \(x\).
\end{definition}

\begin{example}\label{5.1.2}
    The real-valued functions \(f(x) = \sin(x)\) and \(f(x) = \cos(x)\) are \(2\pi\)-periodic, as is the complex-valued function \(f(x) = e^{i x}\).
    These functions are also \(4\pi\)-periodic, \(6\pi\)-periodic, etc.
    The function \(f(x) = x\), however, is not periodic.
    The constant function \(f(x) = 1\) is \(L\)-periodic for every \(L\).
\end{example}

\begin{remark}\label{5.1.3}
    If a function \(f\) is \(L\)-periodic, then we have \(f(x + kL) = f(x)\) for every integer \(k\)
    (why? Use induction for the positive \(k\), and then use a substitution to convert the positive \(k\) result to a negative \(k\) result.
    The \(k = 0\) case is of course trivial).
    In particular, if a function \(f\) is \(1\)-periodic, then we have \(f(x + k) = f(x)\) for every \(k \in \mathbf{Z}\).
    Because of this, \(1\)-periodic functions are sometimes also called \(\mathbf{Z}\)-periodic
    (and \(L\)-periodic functions called \(L \mathbf{Z}\)-periodic).
\end{remark}

\begin{example}\label{5.1.4}
    For any integer \(n\), the functions \(x \mapsto \cos(2 \pi n x)\), \(x \mapsto \sin(2 \pi n x)\), and \(x \mapsto e^{2 \pi i n x}\) are all \(\mathbf{Z}\)-periodic.
    Another example of a \(\mathbf{Z}\)-periodic function is the function \(f : \mathbf{R} \to \mathbf{C}\) defined by \(f(x) \coloneqq 1\) when \(x \in [n, n + \frac{1}{2})\) for some integer \(n\), and \(f(x) \coloneqq 0\) when \(x \in [n + \frac{1}{2}, n + 1)\) for some integer \(n\).
    This function is an example of a \emph{square wave}.
\end{example}

\begin{note}
    In order to completely specify a \(\mathbf{Z}\)-periodic function \(f : \mathbf{R} \to \mathbf{C}\), one only needs to specify its values on the interval \([0, 1)\), since this will determine the values of \(f\) everywhere else.
    This is because every real number \(x\) can be written in the form \(x = k + y\) where \(k\) is an integer (called the \emph{integer part} of \(x\), and sometimes denoted \([x]\)) and \(y \in [0, 1)\) (this is called the \emph{fractional part} of \(x\), and sometimes denoted \(\{x\}\)).
    Because of this, sometimes when we wish to describe a \(\mathbf{Z}\)-periodic function \(f\) we just describe what it does on the interval \([0, 1)\), and then say that it is \emph{extended periodically} to all of \(\mathbf{R}\).
    This means that we define \(f(x)\) for any real number \(x\) by setting \(f(x) \coloneqq f(y)\), where we have decomposed \(x = k + y\) as discussed above.
    (One can in fact replace the interval \([0, 1)\) by any other half-open interval of length \(1\), but we will not do so here.)
\end{note}

\begin{note}
    The space of complex-valued continuous \(\mathbf{Z}\)-periodic functions is denoted
    \[
        C(\mathbf{R} / \mathbf{Z} ; \mathbf{C}).
    \]
    (The notation \(\mathbf{R} / \mathbf{Z}\) comes from algebra, and denotes the quotient group of the additive group \(\mathbf{R}\) by the additive group \(\mathbf{Z}\);
    more information in this can be found in any algebra text.)
    By ``continuous'' we mean continuous at all points on \(\mathbf{R}\);
    merely being continuous on an interval such as \([0, 1]\) will not suffice, as there may be a discontinuity between the left and right limits at \(1\) (or at any other integer).
    Thus for instance, the functions \(x \mapsto \sin(2 \pi n x)\), \(x \mapsto \cos(2 \pi n x)\), and \(x \mapsto e^{2 \pi i n x}\) are all elements of \(C(\mathbf{R} / \mathbf{Z} ; \mathbf{C})\), as are the constant functions, however the square wave function in Example \ref{5.1.4} is not in \(C(\mathbf{R} / \mathbf{Z} ; \mathbf{C})\) because it is not continuous at every integer.
    Also the function \(\sin(x)\) would also not qualify to be in \(C(\mathbf{R} / \mathbf{Z} ; \mathbf{C})\) since it is not \(\mathbf{Z}\)-periodic.
\end{note}

\begin{lemma}[Basic properties of \(C(\mathbf{R} / \mathbf{Z} ; \mathbf{C})\)]\label{5.1.5}
    \quad
    \begin{enumerate}
        \item (Boundedness)
              If \(f \in C(\mathbf{R} / \mathbf{Z} ; \mathbf{C})\), then \(f\) is bounded
              (i.e., there exists a real number \(M > 0\) such that \(\abs*{f(x)} \leq M\) for all \(x \in \mathbf{R}\)).
        \item (Vector space and algebra properties)
              If \(f, g \in C(\mathbf{R} / \mathbf{Z} ; \mathbf{C})\), then the functions \(f + g\), \(f - g\), and \(f g\) are also in \(C(\mathbf{R} / \mathbf{Z} ; \mathbf{C})\).
              Also, if \(c\) is any complex number, then the function \(cf\) is also in \(C(\mathbf{R} / \mathbf{Z} ; \mathbf{C})\).
        \item (Closure under uniform limits)
              If \((f_n)_{n = 1}^\infty\) is a sequence of functions in \(C(\mathbf{R} / \mathbf{Z} ; \mathbf{C})\) which converges uniformly to another function \(f : \mathbf{R} \to \mathbf{C}\), then \(f\) is also in \(C(\mathbf{R} / \mathbf{Z} ; \mathbf{C})\).
    \end{enumerate}
\end{lemma}

\begin{proof}{(a)}
    Since \(f \in C(\mathbf{R} / \mathbf{Z} ; \mathbf{C})\), by Definition \ref{5.1.1} we have
    \[
        \big\{f(x) : x \in \mathbf{R}\big\} = \big\{f(x) : x \in [0, 1)\big\} = \big\{f(x) : x \in [0, 1]\big\}.
    \]
    So it suffices to show that \(\{f(x) : x \in [0, 1]\}\) is bounded.
    Let \(d_{\mathbf{R}} = d_{l^1}|_{\mathbf{R} \times \mathbf{R}}\) and let \(d_{\mathbf{C}}\) be the metric in Definition \ref{4.6.10}.
    Since \([0, 1]\) is closed and bounded in \((\mathbf{R}, d_{\mathbf{R}})\), by Heine-Borel theorem (Theorem \ref{1.5.7}) we know that \(([0, 1], d_{\mathbf{R}}|_{[0, 1] \times [0, 1]})\) is compact.
    Since \(f\) is continuous on \([0, 1]\), by Theorem \ref{2.3.1} we know that \(\big(f([0, 1]), d_{\mathbf{C}}|_{f([0, 1]) \times f([0, 1])}\big)\) is also compact.
    By Corollary \ref{1.5.6} we know that compactness implies boundness, thus we have
    \begin{align*}
                 & \forall\ z \in \mathbf{C}, \exists\ r \in \mathbf{R}^+ : f([0, 1]) \subseteq B_{(\mathbf{C}, d_{\mathbf{C}})}(z, r) & \text{(by Definition \ref{1.5.3})} \\
        \implies & \exists\ r \in \mathbf{R}^+ : f([0, 1]) \subseteq B_{(\mathbf{C}, d_{\mathbf{C}})}(1, r)                                                                 \\
        \implies & \exists\ r \in \mathbf{R}^+ : \forall\ y \in f([0, 1]), \abs*{y - 1} < r                                            & \text{(by Definition \ref{1.2.1})} \\
        \implies & \exists\ r \in \mathbf{R}^+ : \forall\ y \in f([0, 1]),                                                                                                  \\
                 & \abs*{y} = \abs*{y - 1 + 1} \leq \abs*{y - 1} + 1 < r + 1.                                                          & \text{(by Lemma \ref{4.6.11})}
    \end{align*}
    By setting \(M = r + 1\) we are done.
\end{proof}

\begin{proof}{(b)}
    We have
    \begin{align*}
        (f + g)(x + 1) & = f(x + 1) + g(x + 1)                                      \\
                       & = f(x) + g(x)         & \text{(by Definition \ref{5.1.1})} \\
                       & = (f + g)(x)                                               \\
        (f - g)(x + 1) & = f(x + 1) - g(x + 1)                                      \\
                       & = f(x) - g(x)         & \text{(by Definition \ref{5.1.1})} \\
                       & = (f - g)(x)                                               \\
        (f g)(x + 1)   & = f(x + 1) g(x + 1)                                        \\
                       & = f(x) g(x)           & \text{(by Definition \ref{5.1.1})} \\
                       & = (f g)(x)
    \end{align*}
    and
    \begin{align*}
        \forall\ c \in \mathbf{C}, (c f)(x + 1) & = c f(x + 1)                                      \\
                                                & = c f(x)     & \text{(by Definition \ref{5.1.1})} \\
                                                & = (c f)(x).
    \end{align*}
\end{proof}

\begin{proof}{(c)}
    Let \(d_{\mathbf{R}} = d_{l^1}|_{\mathbf{R} \times \mathbf{R}}\) and let \(d_{\mathbf{C}}\) be the metric in Definition \ref{4.6.10}.
    Since \((f_n)_{n = 0}^\infty\) converges uniformly to \(f\) on \(\mathbf{C}\) with respect to \(d_{\mathbf{C}}\), by Corollary \ref{3.3.2} we know that \(f\) is continuous from \((\mathbf{R}, d_{\mathbf{R}})\) to \((\mathbf{C}, d_{\mathbf{C}})\).
    Suppose for sake of contradiction that \(f \notin C(\mathbf{R} / \mathbf{Z} ; \mathbf{C})\).
    By Definition \ref{5.1.1} this means
    \[
        \exists\ x \in \mathbf{R} : f(x + 1) \neq f(x).
    \]
    By Exercise \ref{ex 3.2.2} we know that \((f_n)_{n = 0}^\infty\) converges pointwise to \(f\) on \(\mathbf{C}\) with respect to \(d_{\mathbf{C}}\), thus by Definition \ref{3.2.1} we have
    \begin{align*}
                 & \begin{cases}
            d - \lim_{n \to \infty} f_n(x) = f(x) \\
            d - \lim_{n \to \infty} f_n(x + 1) = f(x + 1)
        \end{cases}                                                                                                                            \\
        \implies & \forall\ \varepsilon \in \mathbf{R}, \exists\ N \in \mathbf{Z}^+ : \forall\ n \geq N,                                                                 \\
                 & \begin{cases}
            \abs*{f_n(x) - f(x)} < \frac{\varepsilon}{2} \\
            \abs*{f_n(x + 1) - f(x + 1)} < \frac{\varepsilon}{2}
        \end{cases}                                                                                                                            \\
        \implies & \forall\ \varepsilon \in \mathbf{R}, \exists\ N \in \mathbf{Z}^+ : \forall\ n \geq N, \begin{cases}
            \abs*{f_n(x) - f(x)} < \frac{\varepsilon}{2} \\
            \abs*{f_n(x) - f(x + 1)} < \frac{\varepsilon}{2}
        \end{cases} & \text{(by Definition \ref{5.1.1})} \\
        \implies & \forall\ \varepsilon \in \mathbf{R}, \exists\ N \in \mathbf{Z}^+ : \forall\ n \geq N,                                                                 \\
                 & \abs*{f(x) - f(x + 1)} \leq \abs*{f(x) - f_n(x)} + \abs*{f_n(x) - f(x + 1)}                                                                           \\
                 & < \frac{\varepsilon}{2} + \frac{\varepsilon}{2} = \varepsilon                                                                                         \\
        \implies & \forall\ \varepsilon \in \mathbf{R}^+, \abs*{f(x) - f(x + 1)} < \varepsilon                                                                           \\
        \implies & f(x) = f(x + 1).
    \end{align*}
    But this contradict to the definition of \(x\).
    Thus \(f \in C(\mathbf{R} / \mathbf{Z} ; \mathbf{C})\).
\end{proof}

\begin{note}
    One can make \(C(\mathbf{R} / \mathbf{Z} ; \mathbf{C})\) into a metric space by re-introducing the now familiar sup-norm metric
    \[
        d_\infty(f, g) = \sup_{x \in \mathbf{R}} \abs*{f(x) - g(x)} = \sup_{x \in [0, 1)} \abs*{f(x) - g(x)}
    \]
    of uniform convergence.
\end{note}

\begin{additional corollary}[modular operation]\label{ac 5.1.1}
Let \(n \in \mathbf{Z}^+\).
Define \(\text{mod}_n : \mathbf{R} \to \mathbf{R}\) as follow:
\[
    \forall\ x \in \mathbf{R}, \text{mod}_n(x) = x - \bigg[\frac{x}{n}\bigg] n.
\]
Then \(\text{mod}(\mathbf{R}) \subseteq [0, n)\) and \(\text{mod}\) is \(n\)-periodic.
We often use \(x \mod n\) instead of \(\text{mod}_n(x)\).
\end{additional corollary}

\begin{proof}
    Since
    \begin{align*}
                 & \forall\ x \in \mathbf{R}, \bigg[\frac{x}{n}\bigg] \leq \frac{x}{n} < \bigg[\frac{x}{n}\bigg] + 1 & \text{(by Exercise \ref{5.1.1})} \\
        \implies & \bigg[\frac{x}{n}\bigg] n \leq x < \bigg[\frac{x}{n}\bigg] n + n                                  & (n \in \mathbf{Z}^+)             \\
        \implies & 0 \leq x - \bigg[\frac{x}{n}\bigg] n < n,
    \end{align*}
    we know that \(\text{mod}_n(x) \subseteq [0, n)\).
    Since
    \begin{align*}
        \forall\ x \in \mathbf{R}, \text{mod}_n(x + n) & = x + n - \bigg[\frac{x + n}{n}\bigg] n                                                   \\
                                                       & = x + n - \Bigg(\bigg[\frac{x}{n}\bigg] + 1\Bigg) n & \text{(by Exercise \ref{ex 5.1.1})} \\
                                                       & = x + n - \bigg[\frac{x}{n}\bigg] n - n                                                   \\
                                                       & = x - \bigg[\frac{x}{n}\bigg] n                                                           \\
                                                       & = \text{mod}_n(x),
    \end{align*}
    by Definition \ref{5.1.1} we know that \(\text{mod}_n\) is \(n\)-periodic.
\end{proof}

\exercisesection

\begin{exercise}\label{ex 5.1.1}
    Show that every real number \(x\) can be written in exactly one way in the form \(x = k + y\), where \(k\) is an integer and \(y \in [0, 1)\).
\end{exercise}

\begin{proof}
    By Exercise 5.4.3 we know that
    \[
        \forall\ x \in \mathbf{R}, \exists!\ k \in \mathbf{Z} : k \leq x < k + 1.
    \]
    Thus by setting \(y = x - k\) we have \(x = y + k\) and \(y \in [0, 1)\).
\end{proof}

\begin{exercise}\label{ex 5.1.2}
    Prove Lemma \ref{5.1.5}.
\end{exercise}

\begin{proof}
    See Lemma \ref{5.1.5}.
\end{proof}

\begin{exercise}\label{ex 5.1.3}
    Show that \(C(\mathbf{R} / \mathbf{Z} ; \mathbf{C})\) with the sup-norm metric \(d_\infty\) is a metric space.
    Furthermore, show that this metric space is complete.
\end{exercise}