\section{Series of functions; the Weierstrass \emph{M}-test}\label{ii:sec:3.5}

\begin{note}
  Functions whose codomain is \(\R\) are sometimes called \emph{real-valued} functions.
\end{note}

\begin{note}
  given any finite collection \(f^{(1)}, \dots, f^{(N)}\) of functions from \(X\) to \(\R\), we can define the finite sum \(\sum_{i = 1}^N f^{(i)} : X \to \R\) by
  \[
    \bigg(\sum_{i = 1}^N f^{(i)}\bigg)(x) \coloneqq \sum_{i = 1}^N f^{(i)}(x).
  \]
\end{note}

\setcounter{thm}{1}
\begin{defn}[Infinite series]\label{ii:3.5.2}
  Let \((X, d_X)\) be a metric space.
  Let \((f^{(n)})_{n = 1}^\infty\) be a sequence of functions from \(X\) to \(\R\), and let \(f\) be another function from \(X\) to \(\R\).
  If the partial sums \(\sum_{n = 1}^N f^{(n)}\) converges pointwise to \(f\) on \(X\) as \(N \to \infty\), we say that the infinite series \(\sum_{n = 1}^\infty f^{(n)}\) \emph{converges pointwise} to \(f\), and write \(f = \sum_{n = 1}^\infty f^{(n)}\).
  If the partial sums \(\sum_{n = 1}^N f^{(n)}\) converge uniformly to \(f\) on \(X\) as \(N \to \infty\), we say that the infinite series \(\sum_{n = 1}^\infty f^{(n)}\) \emph{converges uniformly} to \(f\), and again write \(f = \sum_{n = 1}^\infty f^{(n)}\).
  (Thus when one sees an expression such as \(f = \sum_{n = 1}^\infty f^{(n)}\), one should look at the context to see in what sense this infinite series converges.)
\end{defn}

\begin{rmk}\label{ii:3.5.3}
  A series \(\sum_{n = 1}^\infty f^{(n)}\) converges pointwise to \(f\) on \(X\) iff \(\sum_{n = 1}^\infty f^{(n)}(x)\) converges to \(f(x)\) for \emph{every} \(x \in X\).
  (Thus if \(\sum_{n = 1}^\infty f^{(n)}\) does not converge pointwise to \(f\), this does not mean that it diverges pointwise;
  it may just be that it converges for some points \(x\) but diverges at other points \(x\).)
\end{rmk}

\begin{note}
  If a series \(\sum_{n = 1}^\infty f^{(n)}\) converges uniformly to \(f\), then it also converges pointwise to \(f\);
  but not vice versa.
\end{note}

\setcounter{thm}{4}
\begin{defn}[Sup norm]\label{ii:3.5.5}
  If \(f : X \to \R\) is a bounded real-valued function, we define the \emph{sup norm} \(\norm*{f}_\infty\) of \(f\) to be the number
  \[
    \norm*{f}_\infty \coloneqq \sup\set{\abs{f(x)} : x \in X}.
  \]
  In other words, \(\norm*{f}_\infty = d_\infty(f, 0)\), where \(0 : X \to \R\) is the zero function \(0(x) \coloneqq 0\), and \(d_\infty\) was defined in \cref{ii:3.4.2}.
  We restrict the definition of \(\norm*{f}_\infty\) to the case when \(X \neq \emptyset\).
  If \(X = \emptyset\), then we instead define \(\norm*{f}_\infty = 0\).
\end{defn}

\begin{note}
  When \(f\) is bounded then \(\norm*{f}_\infty\) will always be a non-negative real number.
\end{note}

\setcounter{thm}{6}
\begin{thm}[Weierstrass \(M\)-test]\label{ii:3.5.7}
  Let \((X, d)\) be a metric space, and let \((f^{(n)})_{n = 1}^\infty\) be a sequence of bounded real-valued continuous functions on \(X\) such that the series \(\sum_{n = 1}^\infty \norm*{f^{(n)}}_\infty\) is convergent.
  (Note that this is a series of plain old real numbers, not of functions.)
  Then the series \(\sum_{n = 1}^\infty f^{(n)}\) converges uniformly to some function \(f\) on \(X\), and that function \(f\) is also continuous.
\end{thm}

\begin{proof}
  Let \(N_1, N_2 \in \Z^+\).
  Let \(d_{C(X \to \R)} = d_{B(X \to \R)}|_{C(X \to \R) \times C(X \to \R)}\).
  We have
  \begin{align*}
             & \sum_{n = 1}^\infty \norm*{f^{(n)}}_\infty = \lim_{N \to \infty} \sum_{n = 1}^N \norm*{f^{(n)}}_\infty                                                                                     \\
    \implies & \forall \varepsilon \in \R^+, \exists M \in \Z^+ : \forall N \geq M,                                                                                                                       \\
             & \abs{\sum_{n = 1}^\infty \norm*{f^{(n)}}_\infty - \sum_{n = 1}^N \norm*{f^{(n)}}_\infty} < \varepsilon &                                 & \by{ii:1.1.14}                                  \\
    \implies & \forall \varepsilon \in \R^+, \exists M \in \Z^+ : \forall N \geq M,                                                                                                                       \\
             & \abs{\sum_{n = N + 1}^\infty \norm*{f^{(n)}}_\infty} < \varepsilon                                     &                                 & \text{(by Proposition 7.2.14(c) in Analysis I)} \\
    \implies & \forall \varepsilon \in \R^+, \exists M \in \Z^+ : \forall N \geq M,                                                                                                                       \\
             & \sum_{n = N + 1}^\infty \norm*{f^{(n)}}_\infty < \varepsilon                                           & (\norm*{f^{(n)}}_\infty \geq 0)                                                   \\
    \implies & \forall \varepsilon \in \R^+, \exists M \in \Z^+ : \forall N \geq M,                                                                                                                       \\
             & \sum_{n = N + 1}^\infty \sup_{x \in X} \abs{f^{(n)}(x)} < \varepsilon.                                 &                                 & \by{ii:3.5.5}
  \end{align*}
  Fix one \(\varepsilon\) and \(M\).
  Since \(f^{(n)} \in C(X \to \R)\), by \cref{ii:ex:3.5.1} we know that \(\sum_{n = 1}^N f^{(n)} \in C(X \to \R)\) for each \(N \in \Z^+\).
  Thus \(d_{C(X \to \R)}\bigg(\sum_{n = 1}^{N_1} f^{(n)}, \sum_{n = 1}^{N_2} f^{(n)}\bigg)\) is well defined for each \(N_1, N_2 \geq M\) and
  \begin{align*}
    \forall N_1, N_2 \geq M, & d_{C(X \to \R)}\bigg(\sum_{n = 1}^{N_1} f^{(n)}, \sum_{n = 1}^{N_2} f^{(n)}\bigg)                                  \\
                             & = \sup_{x \in X} \abs{\sum_{n = 1}^{N_1} f^{(n)}(x) - \sum_{n = 1}^{N_2} f^{(n)}(x)}            &  & \by{ii:3.4.2} \\
                             & = \sup_{x \in X} \abs{\sum_{n = \min(N_1, N_2) + 1}^{\max(N_1, N_2)} f^{(n)}(x)}                                   \\
                             & \leq \sup_{x \in X} \bigg(\sum_{n = \min(N_1, N_2) + 1}^{\max(N_1, N_2)} \abs{f^{(n)}(x)}\bigg)                    \\
                             & \leq \sup_{x \in X} \bigg(\sum_{n = M + 1}^\infty \abs{f^{(n)}(x)}\bigg)                                           \\
                             & \leq \sum_{n = M + 1}^\infty \sup_{x \in X} \abs{f^{(n)}(x)} < \varepsilon.
  \end{align*}
  Since \(\varepsilon\) was arbitrary, we have
  \[
    \forall \varepsilon \in \R^+, \exists M \in \Z^+ : \forall N_1, N_2 \geq M, d_{C(X \to \R)}\bigg(\sum_{n = 1}^{N_1} f^{(n)}, \sum_{n = 1}^{N_2} f^{(n)}\bigg) < \varepsilon.
  \]
  By \cref{ii:1.4.6} \(\bigg(\sum_{n = 1}^N f^{(n)}\bigg)_{N = 1}^\infty\) is a Cauchy sequence in \(\big(C(X \to \R), d_{C(X \to \R)}\big)\).
  Since \((\R, d_{l^1}|_{\R \times \R})\) is complete, by \cref{ii:3.4.5} we know that \(\bigg(\sum_{n = 1}^N f^{(n)}\bigg)_{N = 1}^\infty\) converges uniformly to some \(f \in C(X \to \R)\) on \(X\) with respect to \(d_{l^1}|_{\R \times \R}\).
\end{proof}

\begin{note}
  To put the Weierstrass \(M\)-test succinctly:
  absolute convergence of sup norms implies uniform convergence of functions.
\end{note}

\begin{eg}\label{ii:3.5.8}
  Let \(0 < r < 1\) be a real number, and let \(f^{(n)} : [-r, r] \to \R\) be the series of functions \(f^{(n)}(x) \coloneqq x^n\).
  Then each \(f^{(n)}\) is continuous and bounded, and \(\norm*{f^{(n)}}_\infty = r^n\).
  Since the series \(\sum_{n = 1}^\infty r^n\) is absolutely convergent (e.g., by the root test, Theorem 7.5.1 in Analysis I), we thus see that \(\sum_{n = 1}^\infty f^{(n)}\) converges uniformly in \([-r, r]\) to some continuous function;
  in \cref{ii:ex:3.2.2}(c) we see that this function must in fact be the function \(f : [-r, r] \to \R\) defined by \(f(x) \coloneqq x / (1 - x)\).
  In other words, the series \(\sum_{n = 1}^\infty x^n\) is pointwise convergent, but not uniformly convergent, on \((-1, 1)\), but is uniformly convergent on the smaller interval \([-r, r]\) for any \(0 < r < 1\).
\end{eg}

\begin{note}
  The Weierstrass \(M\)-test is especially useful in relation to power series.
\end{note}

\exercisesection

\begin{ex}\label{ii:ex:3.5.1}
  Let \(f^{(1)}, \dots, f^{(N)}\) be a finite sequence of bounded functions from a metric space \((X, d_X)\) to \(\R\).
  Show that \(\sum_{i = 1}^N f^{(i)}\) is also bounded.
  Prove a similar claim when ``bounded'' is replaced by ``continuous.''
  What if ``continuous'' was replaced by ``uniformly continuous''?
\end{ex}

\begin{proof}
  Let \(d_1 = d_{l^1}|_{\R \times \R}\).
  We first show that \(\sum_{n = 1}^N f^{(n)}\) is bounded on \(X\) with respect to \(d_1\) for each \(N \in \Z^+\).
  Suppose that \(f^{(n)}\) is bounded on \(X\) with respect to \(d_1\) for each \(n \in \Z^+\).
  We induct on \(N\).
  For \(N = 1\), by hypothesis we know that \(\sum_{n = 1}^1 f^{(n)} = f^{(1)}\) is bounded on \(X\).
  Thus, the base case holds.
  Suppose inductively that \(\sum_{n = 1}^N f^{(n)}\) is bounded on \(X\) with respect to \(d_1\) for some \(N \geq 1\).
  By induction hypothesis we have
  \[
    \exists M \in \R^+ : \bigg(\sum_{n = 1}^N f^{(n)}\bigg)(X) \subseteq [-M, M].
  \]
  By hypothesis we know that \(f^{(N + 1)}\) is bounded on \(X\) with respect to \(d_1\), thus we have
  \[
    \exists M' \in \R^+ : f^{(N + 1)}(X) \subseteq [-M', M'].
  \]
  Then we have
  \begin{align*}
    \bigg(\sum_{n = 1}^{N + 1} f^{(n)}\bigg)(X) & = \set{\sum_{n = 1}^{N + 1} f^{(n)}(x) : x \in X}            \\
                                                & = \set{\sum_{n = 1}^N f^{(n)}(x) + f^{(N + 1)}(x) : x \in X} \\
                                                & \subseteq [-(M + M'), M + M'].
  \end{align*}
  This closes the induction.

  Next we show that \(\sum_{n = 1}^N f^{(n)}\) is continuous from \((X, d_X)\) to \((\R, d_1)\) for each \(N \in \Z^+\).
  Suppose that \(f^{(n)}\) is continuous from \((X, d_X)\) to \((\R, d_1)\) for each \(n \in \Z^+\).
  We induct on \(N\).
  For \(N = 1\), by hypothesis we know that \(\sum_{n = 1}^1 f^{(n)} = f^{(1)}\) is continuous from \((X, d_X)\) to \((\R, d_1)\).
  Thus, the base case holds.
  Suppose inductively that \(\sum_{n = 1}^N f^{(n)}\) is continuous from \((X, d_X)\) to \((\R, d_1)\) for some \(N \geq 1\).
  Then by \cref{ii:ac:2.2.1}
  \[
    \sum_{n = 1}^{N + 1} f^{(n)} = \bigg(\sum_{n = 1}^N f^{(n)}\bigg) \oplus f^{(N + 1)}
  \]
  is continuous from \((X, d_X)\) to \((\R, d_1)\).
  This closes the induction.

  Finally we show that \(\sum_{n = 1}^N f^{(n)}\) is uniformly continuous from \((X, d_X)\) to \((\R, d_1)\) for each \(N \in \Z^+\).
  Suppose that \(f^{(n)}\) is uniformly continuous from \((X, d_X)\) to \((\R, d_1)\) for each \(n \in \Z^+\).
  We induct on \(N\).
  For \(N = 1\), by hypothesis we know that \(\sum_{n = 1}^1 f^{(n)} = f^{(1)}\) is uniformly continuous from \((X, d_X)\) to \((\R, d_1)\).
  Thus, the base case holds.
  Suppose inductively that \(\sum_{n = 1}^N f^{(n)}\) is uniformly continuous from \((X, d_X)\) to \((\R, d_1)\) for some \(N \geq 1\).
  Then by \cref{ii:ex:2.3.5}
  \[
    \sum_{n = 1}^{N + 1} f^{(n)} = \bigg(\sum_{n = 1}^N f^{(n)}\bigg) \oplus f^{(N + 1)}
  \]
  is uniformly continuous from \((X, d_X)\) to \((\R, d_1)\).
  This closes the induction.
\end{proof}

\begin{ex}\label{ii:ex:3.5.2}
  Prove \cref{ii:3.5.7}.
\end{ex}

\begin{proof}
  See \cref{ii:3.5.7}.
\end{proof}
