\section{Series of functions; the Weierstrass \emph{M}-test}\label{sec 3.5}

\begin{note}
    Functions whose range is \(\mathbf{R}\) are sometimes called \emph{real-valued} functions.
\end{note}

\begin{note}
    given any finite collection \(f^{(1)}, \dots, f^{(N)}\) of functions from \(X\) to \(\mathbf{R}\), we can define the finite sum \(\sum_{i = 1}^N f^{(i)} : X \to \mathbf{R}\) by
    \[
        \bigg(\sum_{i = 1}^N f^{(i)}\bigg)(x) \coloneqq \sum_{i = 1}^N f^{(i)}(x).
    \]
\end{note}

\setcounter{theorem}{1}
\begin{definition}[Infinite series]\label{3.5.2}
    Let \((X, d_X)\) be a metric space.
    Let \((f^{(n)})_{n = 1}^\infty\) be a sequence of functions from \(X\) to \(\mathbf{R}\), and let \(f\) be another function from \(X\) to \(\mathbf{R}\).
    If the partial sums \(\sum_{n = 1}^N f^{(n)}\) converges pointwise to \(f\) on \(X\) as \(N \to \infty\), we say that the infinite series \(\sum_{n = 1}^\infty f^{(n)}\) \emph{converges pointwise} to \(f\), and write \(f = \sum_{n = 1}^\infty f^{(n)}\).
    If the partial sums \(\sum_{n = 1}^N f^{(n)}\) converge uniformly to \(f\) on \(X\) as \(N \to \infty\), we say that the infinite series \(\sum_{n = 1}^\infty f^{(n)}\) \emph{converges uniformly} to \(f\), and again write \(f = \sum_{n = 1}^\infty f^{(n)}\).
    (Thus when one sees an expression such as \(f = \sum_{n = 1}^\infty f^{(n)}\), one should look at the context to see in what sense this infinite series converges.)
\end{definition}

\begin{remark}\label{3.5.3}
    A series \(\sum_{n = 1}^\infty f^{(n)}\) converges pointwise to \(f\) on \(X\) if and only if \(\sum_{n = 1}^\infty f^{(n)}(x)\) converges to \(f(x)\) for \emph{every} \(x \in X\).
    (Thus if \(\sum_{n = 1}^\infty f^{(n)}\) does not converge pointwise to \(f\), this does not mean that it diverges pointwise;
    it may just be that it converges for some points \(x\) but diverges at other points \(x\).)
\end{remark}

\begin{note}
    If a series \(\sum_{n = 1}^\infty f^{(n)}\) converges uniformly to \(f\), then it also converges pointwise to \(f\);
    but not vice versa.
\end{note}

\setcounter{theorem}{4}
\begin{definition}[Sup norm]\label{3.5.5}
    If \(f : X \to \mathbf{R}\) is a bounded real-valued function, we define the \emph{sup norm} \(\norm*{f}_\infty\) of \(f\) to be the number
    \[
        \norm*{f}_\infty \coloneqq \sup\{\abs*{f(x)} : x \in X\}.
    \]
    In other words, \(\norm*{f}_\infty = d_\infty(f, 0)\), where \(0 : X \to \mathbf{R}\) is the zero function \(0(x) \coloneqq 0\), and \(d_\infty\) was defined in Definition \ref{3.4.2}.
    We restrict the definition of \(\norm*{f}_\infty\) to the case when \(X \neq \emptyset\).
    If \(X = \emptyset\), then we instead define \(\norm*{f}_\infty = 0\).
\end{definition}

\setcounter{theorem}{6}
\begin{theorem}[Weierstrass \(M\)-test]\label{3.5.7}
    Let \((X, d)\) be a metric space, and let \((f^{(n)})_{n = 1}^\infty\) be a sequence of bounded real-valued continuous functions on \(X\) such that the series \(\sum_{n = 1}^\infty \norm*{f^{(n)}}_\infty\) is convergent.
    (Note that this is a series of plain old real numbers, not of functions.)
    Then the series \(\sum_{n = 1}^\infty f^{(n)}\) converges uniformly to some function \(f\) on \(X\), and that function \(f\) is also continuous.
\end{theorem}

\exercisesection

\begin{exercise}\label{ex 3.5.1}
    Let \(f^{(1)}, \dots, f^{(N)}\) be a finite sequence of bounded functions from a metric space \((X, d_X)\) to \(\mathbf{R}\).
    Show that \(\sum_{i = 1}^N f^{(i)}\) is also bounded.
    Prove a similar claim when ``bounded'' is replaced by ``continuous''.
    What if ``continuous'' was replaced by ``uniformly continuous''?
\end{exercise}

\begin{exercise}\label{ex 3.5.2}
    Prove Theorem \ref{3.5.7}.
\end{exercise}

\begin{proof}
    See Theorem \ref{3.5.7}.
\end{proof}