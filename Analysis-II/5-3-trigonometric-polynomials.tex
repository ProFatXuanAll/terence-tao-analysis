\section{Trigonometric polynomials}\label{ii:sec:5.3}

\begin{note}
  We now define the concept of a \emph{trigonometric polynomial}.
  Just as polynomials are combinations of the functions \(x^n\) (sometimes called \emph{monomials}), trigonometric polynomials are combinations of the functions \(e^{2 \pi i n x}\) (sometimes called \emph{characters}).
\end{note}

\begin{defn}[Characters]\label{ii:5.3.1}
  For every integer \(n\), we let \(e_n \in C(\R / \Z ; \C)\) denote the function
  \[
    e_n(x) \coloneqq e^{2 \pi i n x}.
  \]
  This is sometimes referred to as the \emph{character with frequency \(n\)}.
\end{defn}

\begin{defn}[Trigonometric polynomials]\label{ii:5.3.2}
  A function \(f\) in \(C(\R / \Z ; \C)\) is said to be a \emph{trigonometric polynomial} if we can write
  \(f = \sum_{n = -N}^N c_n e_n\) for some integer \(N \geq 0\) and some complex numbers \((c_n)_{n = -N}^N\).
\end{defn}

\setcounter{thm}{3}
\begin{eg}\label{ii:5.3.4}
  For any integer \(n\), the function \(\cos(2 \pi n x)\) is a trigonometric polynomial, since
  \[
    \cos(2 \pi n x) = \dfrac{e^{2 \pi n x} + e^{- 2 \pi n x}}{2} = \dfrac{1}{2} e_{-n} + \dfrac{1}{2} e_n.
  \]
  Similarly the function \(\sin(2 \pi n x) = \dfrac{-1}{2i} e_{-n} + \dfrac{1}{2i} e_n\) is a trigonometric polynomial.
  In fact, any linear combination of sines and cosines is also a trigonometric polynomial.
\end{eg}

\begin{lem}[Characters are an orthonormal system]\label{ii:5.3.5}
  For any integers \(n\) and \(m\), we have \(\inner*{e_n, e_m} = 1\) when \(n = m\) and \(\inner*{e_n, e_m} = 0\) when \(n \neq m\).
  Also, we have \(\norm*{e_n}_2 = 1\).
\end{lem}

\begin{proof}
  Let \(n, m \in \Z\).
  Observe that
  \begin{align*}
    \inner*{e_n, e_m} & = \int_{[0, 1]} e_n(x) \overline{e_m(x)} \; dx                   &  & \by{ii:5.2.1}      \\
                      & = \int_{[0, 1]} e^{2 \pi i n x} \overline{e^{2 \pi i m x}} \; dx &  & \by{ii:5.3.1}      \\
                      & = \int_{[0, 1]} e^{2 \pi i n x} e^{- 2 \pi i m x} \; dx          &  & \by{ii:4.7.2}[c,f] \\
                      & = \int_{[0, 1]} e^{2 \pi i n x - 2 \pi i m x} \; dx              &  & \by{ii:ex:4.6.16}  \\
                      & = \int_{[0, 1]} e^{2 \pi i (n - m) x} \; dx.
  \end{align*}
  If \(n = m\), then we have
  \begin{align*}
    \inner*{e_n, e_n} & = \int_{[0, 1]} e^{2 \pi i (n - n) x} \; dx                       \\
                      & = \int_{[0, 1]} e^0 \; dx                                         \\
                      & = \int_{[0, 1]} 1 \; dx                     &  & \by{ii:4.5.2}[e] \\
                      & = 1
  \end{align*}
  and
  \begin{align*}
    \norm*{e_n}_2 & = \sqrt{\inner*{e_n, e_n}} &  & \by{ii:ac:5.2.1} \\
                  & = \sqrt{1} = 1.
  \end{align*}
  If \(n \neq m\), then we have
  \begin{align*}
     & \inner*{e_n, e_m}                                                                                                     \\
     & = \int_{[0, 1]} e^{2 \pi i (n - m) x} \; dx                                                                           \\
     & = \int_{[0, 1]} \cos\big(2 \pi (n - m) x\big) + i \sin\big(2 \pi (n - m) x\big) \; dx        &  & \by{ii:4.7.2}[f]    \\
     & = \int_{[0, 1]} \cos\big(2 \pi (n - m) x\big) \; dx                                          &  & \by{ii:5.2.2}       \\
     & \quad + i \int_{[0, 1]} \sin\big(2 \pi (n - m) x\big) \; dx                                                           \\
     & = \bigg(\dfrac{\sin\big(2 \pi (n - m) x\big)}{2 \pi (n - m)}|_{x = 0}^{x = 1}\bigg)          &  & \by{ii:4.7.2}[b]    \\
     & \quad + i \bigg(\dfrac{-\cos\big(2 \pi (n - m) x\big)}{2 \pi (n - m)}|_{x = 0}^{x = 1}\bigg)                          \\
     & = 0 - 0                                                                                      &  & \by{ii:ac:4.7.2}[c] \\
     & \quad + i \bigg(\dfrac{- (-1) + (-1)}{2 \pi (n - m)}\bigg)                                   &  & \by{ii:ac:4.7.2}[f] \\
     & = 0.
  \end{align*}
\end{proof}

\begin{cor}\label{ii:5.3.6}
  Let \(f = \sum_{n = -N}^N c_n e_n\) be a trigonometric polynomial.
  Then we have the formula
  \[
    c_n = \inner*{f, e_n}
  \]
  for all integers \(-N \leq n \leq N\).
  Also, we have \(0 = \inner*{f, e_n}\) whenever \(n > N\) or \(n < -N\).
  Also, we have the identity
  \[
    \norm*{f}_2^2 = \sum_{n = -N}^N \abs{c_n}^2.
  \]
\end{cor}

\begin{proof}
  Let \(m \in \N\).
  Then we have
  \begin{align*}
    \inner*{f, e_m} & = \inner*{\sum_{n = -N}^N (c_n e_n), e_m}         &  & \text{(by hypothesis)} \\
                    & = \sum_{n = -N}^N \inner*{c_n e_n, e_m}           &  & \by{ii:5.2.5}[c]       \\
                    & = \sum_{n = -N}^N \big(c_n \inner*{e_n, e_m}\big) &  & \by{ii:5.2.5}[c]       \\
                    & = \begin{dcases}
                          c_m & \text{if } -N \leq m \leq N         \\
                          0   & \text{if } m > N \text{ or } m < -N
                        \end{dcases}      &  & \by{ii:5.3.5}
  \end{align*}
  and
  \begin{align*}
    \norm*{f}_2^2 & = \inner*{f, f}                                            &  & \by{ii:ac:5.2.1}              \\
                  & = \inner*{f, \sum_{n = -N}^N (c_n e_n)}                    &  & \text{(by hypothesis)}        \\
                  & = \sum_{n = -N}^N \inner*{f, c_n e_n}                      &  & \by{ii:5.2.5}[d]              \\
                  & = \sum_{n = -N}^N \big(\overline{c_n} \inner*{f, e_n}\big) &  & \by{ii:5.2.5}[d]              \\
                  & = \sum_{n = -N}^N \big(\overline{c_n} c_n\big)             &  & \text{(from the proof above)} \\
                  & = \sum_{n = -N}^N \abs{c_n}^2.                             &  & \by{ii:4.6.11}
  \end{align*}
\end{proof}

\begin{defn}[Fourier transform]\label{ii:5.3.7}
  For any function \(f \in C(\R / \Z ; \C)\), and any integer \(n \in \Z\), we define the \(n^{\opTh}\) \emph{Fourier coefficient of} \(f\), denoted \(\hat{f}(n)\), by the formula
  \[
    \hat{f}(n) \coloneqq \inner*{f, e_n} = \int_{[0, 1]} f(x) e^{- 2 \pi i n x} \; dx.
  \]
  The function \(\hat{f} : \Z \to \C\) is called the \emph{Fourier transform} of \(f\).
\end{defn}

\begin{ac}\label{ii:ac:5.3.1}
  From \cref{ii:5.3.6}, we see that whenever
  \[
    f = \sum_{n = -N}^N c_n e_n
  \]
  is a trigonometric polynomial, we have
  \[
    f = \sum_{n = -N}^N \inner*{f, e_n} e_n = \sum_{n = -\infty}^\infty \inner*{f, e_n} e_n
  \]
  and in particular we have the \emph{Fourier inversion formula}
  \[
    f = \sum_{n = -\infty}^\infty \hat{f}(n) e_n
  \]
  or in other words
  \[
    f(x) = \sum_{n = -\infty}^\infty \hat{f}(n) e^{2 \pi i n x}.
  \]
  The right-hand side is referred to as the \emph{Fourier series} of \(f\).
  Also, from the second identity of \cref{ii:5.3.6} we have the \emph{Plancherel formula}
  \[
    \norm*{f}_2^2 = \sum_{n = -\infty}^\infty \abs{\hat{f}(n)}^2.
  \]
\end{ac}

\begin{rmk}\label{ii:5.3.8}
  We stress that at present we have only proven the Fourier inversion and Plancherel formulae in the case when \(f\) is a trigonometric polynomial.
  Note that in this case that the Fourier coefficients \(\hat{f}(n)\) are mostly zero (indeed, they can only be non-zero when \(-N \leq n \leq N\)), and so this infinite sum is really just a finite sum in disguise.
  In particular there are no issues about what sense the above series converge in;
  they both converge pointwise, uniformly, and in \(L^2\) metric, since they are just finite sums.
\end{rmk}

\begin{note}
  In the next few sections we will extend the Fourier inversion and Plancherel formulae to general functions in \(C(\R / \Z ; \C)\), not just trigonometric polynomials.
  (It is also possible to extend the formula to discontinuous functions such as the square wave, but we will not do so here.)
  To do this we will need a version of the Weierstrass approximation theorem, this time requiring that a continuous periodic function be approximated uniformly by \emph{trigonometric} polynomials.
  Just as convolutions were used in the proof of the polynomial Weierstrass approximation theorem, we will also need a notion of convolution tailored for periodic functions.
\end{note}

\exercisesection

\begin{ex}\label{ii:ex:5.3.1}
  Show that the sum or product of any two trigonometric polynomials is again a trigonometric polynomial.
\end{ex}

\begin{proof}
  Let \(f, g \in C(\R / \Z ; \C)\) such that
  \begin{align*}
     & \exists N \in \N : \big((c_n)_{n = -N}^N \text{ is in } \C\big) \land \bigg(f = \sum_{n = -N}^N c_n e_n\bigg); \\
     & \exists M \in \N : \big((d_n)_{n = -M}^M \text{ is in } \C\big) \land \bigg(g = \sum_{n = -M}^M d_n e_n\bigg).
  \end{align*}
  Without the loss of generality suppose that \(N \leq M\).
  Then we have
  \begin{align*}
    f + g & = \sum_{n = -N}^N (c_n e_n) + \sum_{n = -M}^M (d_n e_n) \\
          & = \sum_{n = -M}^M (a_n e_n)
  \end{align*}
  where
  \[
    a_n = \begin{dcases}
      c_n + d_n & \text{if } -N \leq n \leq N                     \\
      d_n       & \text{if } (-M \leq n < -N) \lor (N < n \leq M)
    \end{dcases}
  \]
  For \(fg\), we induct on \(M\) to show that \(fg\) is trigonometric polynomial.
  For \(M = 0\), we have
  \begin{align*}
    fg & = \bigg(\sum_{n = -N}^N (c_n e_n)\bigg) (d_0 e^0)                       \\
       & = \bigg(\sum_{n = -N}^N (c_n e_n)\bigg) d_0       &  & \by{ii:4.5.2}[e] \\
       & = \sum_{n = -N}^N (c_n d_0 e_n).
  \end{align*}
  Clearly \(fg\) is trigonometric polynomial and thus the base case holds.
  Suppose inductively that \(fg\) is trigonometric polynomial for some \(M \geq 0\).
  Then for \(M + 1\), we have
  \begin{align*}
    f g & = \bigg(\sum_{n = -N}^N (c_n e_n)\bigg) \bigg(\sum_{m = -(M + 1)}^{M + 1} (d_m e_m)\bigg)                                                        \\
        & = \bigg(\sum_{n = -N}^N (c_n e_n)\bigg) \bigg(\sum_{m = -M}^M (d_m e_m) + d_{-M - 1} e_{-M - 1} + d_{M + 1} e_{M + 1}\bigg)                      \\
        & = \bigg(\sum_{n = -N}^N (c_n e_n)\bigg) \bigg(\sum_{m = -M}^M (d_m e_m)\bigg) + \bigg(\sum_{n = -N}^N (c_n e_n)\bigg) (d_{-M - 1} e_{-M - 1})    \\
        & \quad + \bigg(\sum_{n = -N}^N (c_n e_n)\bigg) (d_{M + 1} e_{M + 1})                                                                              \\
        & = \bigg(\sum_{n = -N}^N (c_n e_n)\bigg) \bigg(\sum_{m = -M}^M (d_m e_m)\bigg) + \sum_{n = -N}^N (c_n d_{-M - 1} e_{n - M - 1})                   \\
        & \quad + \sum_{n = -N}^N (c_n d_{M + 1} e_{n + M + 1})                                                                                            \\
        & = \bigg(\sum_{n = -N}^N (c_n e_n)\bigg) \bigg(\sum_{m = -M}^M (d_m e_m)\bigg) + \sum_{n = -N - M - 1}^{N - M - 1} (c_{n + M + 1} d_{-M - 1} e_n) \\
        & \quad + \sum_{n = -N + M + 1}^{N + M + 1} (c_{n - M - 1} d_{M + 1} e_n).
  \end{align*}
  By setting
  \begin{align*}
     & a_n = \begin{dcases}
               c_{n + M + 1} d_{-M - 1} & \text{if } -N - M - 1 \leq n \leq N - M - 1 \\
               0                        & \text{if } N - M - 1 < n \leq N + M + 1
             \end{dcases} \\
     & b_n = \begin{dcases}
               c_{n - M - 1} d_{M + 1} & \text{if } -N + M + 1 \leq n \leq N + M + 1 \\
               0                       & \text{if } -N - M - 1 \leq n < -N + M + 1
             \end{dcases}
  \end{align*}
  we have
  \[
    fg = \bigg(\sum_{n = -N}^N (c_n e_n)\bigg) \bigg(\sum_{m = -M}^M (d_m e_m)\bigg) + \sum_{n = -N - M - 1}^{N + M + 1} (a_n e_n) + \sum_{n = -N - M - 1}^{N + M + 1} (b_n e_n).
  \]
  By induction hypothesis we know that \(\bigg(\sum_{n = -N}^N (c_n e_n)\bigg) \bigg(\sum_{m = -M}^M (d_m e_m)\bigg)\) is trigonometric polynomial.
  Thus from the proof above we know that \(fg\) is trigonometric polynomial, and this closes the induction.
\end{proof}

\begin{ex}\label{ii:ex:5.3.2}
  Prove \cref{ii:5.3.5}.
\end{ex}

\begin{proof}
  See \cref{ii:5.3.5}.
\end{proof}

\begin{ex}\label{ii:ex:5.3.3}
  Prove \cref{ii:5.3.6}.
\end{ex}

\begin{proof}
  See \cref{ii:5.3.6}.
\end{proof}
