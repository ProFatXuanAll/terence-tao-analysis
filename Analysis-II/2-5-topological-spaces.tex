\section{Topological spaces}\label{sec 2.5}

\begin{note}
    The concept of a metric space can be generalized to that of a \emph{topological space}.
    The idea here is not to view the metric \(d\) as the fundamental object;
    indeed, in a general topological space there is no metric at all.
    Instead, it is the collection of \emph{open sets} which is the fundamental concept.
    Thus, whereas in a metric space one introduces the metric \(d\) first, and then uses the metric to define first the concept of an open ball and then the concept of an open set, in a topological space one starts just with the notion of an open set.
    As it turns out, starting from the open sets, one cannot necessarily reconstruct a usable notion of a ball or metric (thus not all topological spaces will be metric spaces), but remarkably one can still define many of the concepts in the preceding sections.
\end{note}

\begin{definition}[Topological spaces]\label{2.5.1}
    A \emph{topological space} is a pair \((X, \mathcal{F})\), where \(X\) is a set, and \(\mathcal{F} \subseteq 2^X\) is a collection of subsets of \(X\), whose elements are referred to as \emph{open sets}.
    Furthermore, the collection \(\mathcal{F}\) must obey the following properties:
    \begin{itemize}
        \item The empty set \(\emptyset\) and the whole set \(X\) are open;
              in other words, \(\emptyset \in \mathcal{F}\) and \(X \in \mathcal{F}\).
        \item Any finite intersection of open sets is open.
              In other words, if \(V_1 , \dots, V_n\) are elements of \(\mathcal{F}\), then \(V_1 \cap \dots \cap V_n\) is also in \(\mathcal{F}\).
        \item Any arbitrary union of open sets is open (including infinite unions).
              In other words, if \((V_\alpha)_{\alpha \in I}\) is a family of sets in \(\mathcal{F}\), then \(\bigcup_{\alpha \in I} V_\alpha\) is also in \(\mathcal{F}\).
    \end{itemize}
\end{definition}

\begin{note}
    In many cases, the collection \(\mathcal{F}\) of open sets can be deduced from context, and we shall refer to the topological space \((X, \mathcal{F})\) simply as \(X\).
\end{note}

\begin{note}
    From Proposition \ref{1.2.15} we see that every metric space \((X, d)\) is automatically also a topological space
    (if we set \(\mathcal{F}\) equal to the collection of sets which are open in \((X, d)\)).
    However, there do exist topological spaces which do not arise from metric spaces.
\end{note}

\begin{definition}[Neighbourhoods]\label{2.5.2}
    Let \((X, \mathcal{F})\) be a topological space, and let \(x \in X\).
    A \emph{neighbourhood} of \(x\) is defined to be any open set in \(\mathcal{F}\) which contains \(x\).
\end{definition}

\begin{example}\label{2.5.3}
    If \((X, d)\) is a metric space, \(x \in X\), and \(r > 0\), then \(B_{(X, d)}(x, r)\) is a neighbourhood of \(x\) (see Proposition \ref{1.2.15}(c)).
\end{example}

\begin{definition}[Topological convergence]\label{2.5.4}
    Let m be an integer, \((X, \mathcal{F})\) be a topological space and let \((x^{(n)})_{n = m}^\infty\) be a sequence of points in \(X\).
    Let \(x\) be a point in \(X\).
    We say that \((x^{(n)})_{n = m}^\infty\) \emph{converges to} \(x\) if and only if, for every neighbourhood \(V\) of \(x\), there exists an \(N \geq m\) such that \(x^{(n)} \in V\) for all \(n \geq N\).
\end{definition}

\begin{note}
    Definition \ref{2.5.4} is consistent with that of convergence in metric spaces (Definition \ref{1.1.14}).
    One can then ask whether one has the basic property of uniqueness of limits (Proposition \ref{1.1.20}).
    The answer turns out to usually be yes
    - if the topological space has an additional property known as the Hausdorff property
    - but the answer can be no for other topologies.
\end{note}

\begin{definition}[Interior, exterior, boundary]\label{2.5.5}
    Let \((X, \mathcal{F})\) be a topological space, let \(E\) be a subset of \(X\), and let \(x_0\) be a point in \(X\).
    We say that \(x_0\) is an \emph{interior point of} \(E\) if there exists a neighbourhood \(V\) of \(x_0\) such that \(V \subseteq E\).
    We say that \(x_0\) is an \emph{exterior point of} \(E\) if there exists a neighbourhood \(V\) of \(x_0\) such that \(V \cap E = \emptyset\).
    We say that \(x_0\) is a \emph{boundary point of} \(E\) if it is neither an interior point nor an exterior point of \(E\).
\end{definition}

\begin{note}
    Definition \ref{2.5.5} is consistent with the corresponding notion for metric spaces (Definition \ref{1.2.5}).
\end{note}

\begin{definition}[Closure]\label{2.5.6}
    Let \((X, \mathcal{F})\) be a topological space, let \(E\) be a subset of \(X\), and let \(x_0\) be a point in \(X\).
    We say that \(x_0\) is an adherent point of \(E\) if every neighbourhood \(V\) of \(x_0\) has a non-empty intersection with \(E\).
    The set of all adherent points of \(E\) is called the closure of \(E\) and is denoted \(\overline{E}\).
\end{definition}

\begin{note}
    We define a set \(K\) in a topological space \((X, \mathcal{F})\) to be closed iff its complement \(X \setminus K\) is open;
    this is consistent with the metric space definition, thanks to Proposition \ref{1.2.15}(e).
\end{note}

\begin{definition}[Relative topology]\label{2.5.7}
    Let \((X, \mathcal{F})\) be a topological space, and \(Y\) be a subset of \(X\).
    Then we define \(\mathcal{F}_Y \coloneqq \{V \cap Y : V \in F\}\), and refer this as the topology on \(Y\) \emph{induced} by \((X, \mathcal{F})\).
    We call \((Y, \mathcal{F}_Y)\) a \emph{topological subspace} of \((X, \mathcal{F})\).
\end{definition}

\begin{note}
    From Proposition \ref{1.3.4} we see that Definition \ref{2.5.7} is compatible with the one for metric spaces.
\end{note}

\begin{definition}[Continuous functions]\label{2.5.8}
    Let \((X, \mathcal{F}_X)\) and \((Y, \mathcal{F}_Y)\) be topological spaces, and let \(f : X \to Y\) be a function.
    If \(x_0 \in X\), we say that \(f\) is \emph{continuous at} \(x_0\) iff for every neighbourhood \(V\) of \(f(x_0)\), there exists a neighbourhood \(U\) of \(x_0\) such that \(f(U) \subseteq V\).
    We say that \(f\) is \emph{continuous} iff it is continuous at every point \(x \in X\).
\end{definition}

\begin{note}
    Definition \ref{2.5.8} is consistent with that in Definition \ref{2.1.1}.
    In particular, a function is continuous iff the pre-images of every open set is open.
\end{note}

\begin{note}
    There is unfortunately no notion of a Cauchy sequence, a complete space, or a bounded space, for topological spaces.
    However, there is certainly a notion of a compact space.
\end{note}

\begin{definition}[Compact topological spaces]\label{2.5.9}
    Let \((X, \mathcal{F})\) be a topological space.
    We say that this space is \emph{compact} if every open cover of \(X\) has a finite subcover.
    If \(Y\) is a subset of \(X\), we say that \(Y\) is compact if the topological space on \(Y\) induced by \((X, \mathcal{F})\) is compact.
\end{definition}