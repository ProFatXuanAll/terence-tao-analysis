\section{Relative topology}\label{ii:sec:1.3}

\begin{note}
  Consider the plane \(\R^2\) with the Euclidean metric \(d_{l^2}\).
  Inside the plane, we can find the x-axis \(X \coloneqq \set{(x, 0) : x \in \R}\).
  The metric \(d_{l^2}\) can be restricted to \(X\), creating a subspace \((X, d_{l^2}|_{X \times X})\) of \((\R^2, d_{l^2})\).
  This subspace is essentially the same as the real line \((\R, d)\) with the usual metric;
  the precise way of stating this is that \((X, d_{l^2}|_{X \times X})\) is \emph{isometric} to \((\R, d)\).
\end{note}

\setcounter{thm}{2}
\begin{defn}[Relative topology]\label{ii:1.3.3}
  Let \((X, d)\) be a metric space, let \(Y\) be a subset of \(X\), and let \(E\) be a subset of \(Y\).
  We say that \(E\) is \emph{relatively open with respect to \(Y\)} if it is open in the metric subspace \((Y, d|_{Y \times Y})\).
  Similarly, we say that \(E\) is \emph{relatively closed with respect to \(Y\)} if it is closed in the metric space \((Y, d|_{Y \times Y})\).
\end{defn}

\begin{prop}\label{ii:1.3.4}
  Let \((X, d)\) be a metric space, let \(Y\) be a subset of \(X\), and let \(E\) be a subset of \(Y\).
  \begin{enumerate}
    \item \(E\) is relatively open with respect to \(Y\) iff \(E = V \cap Y\) for some set \(V \subseteq X\) which is open in \(X\).
    \item \(E\) is relatively closed with respect to \(Y\) iff \(E = K \cap Y\) for some set \(K \subseteq X\) which is closed in \(X\).
  \end{enumerate}
\end{prop}

\begin{proof}{(a)}
  First suppose that \(E\) is relatively open with respect to \(Y\).
  Then, \(E\) is open in the metric space \((Y, d|_{Y \times Y})\).
  Thus, for every \(x \in E\), there exists a radius \(r > 0\) such that the ball \(B_{(Y, d|_{Y \times Y})}(x, r)\) is contained in \(E\).
  This radius \(r\) depends on \(x\);
  to emphasize this we write \(r_x\) instead of \(r\), thus for every \(x \in E\) the ball \(B_{(Y, d|_{Y \times Y})}(x, r_x)\) is contained in \(E\).
  (Note that we have used the axiom of choice to do this.)

  Now consider the set
  \[
    V \coloneqq \bigcup_{x \in E} B_{(X, d)}(x, r_x).
  \]
  This is a subset of \(X\).
  By \cref{ii:1.2.15}(c) and (g), \(V\) is open in \((X, d)\).
  Now we prove that \(E = V \cap Y\).
  Certainly any point \(x\) in \(E\) lies in \(V \cap Y\), since it lies in \(Y\) and it also lies in \(B_{(X, d)}(x, r_x)\), and hence in \(V\).
  Now suppose that \(y\) is a point in \(V \cap Y\).
  Then \(y \in V\), which implies that there exists an \(x \in E\) such that \(y \in B_{(X, d)}(x, r_x)\).
  But since \(y\) is also in \(Y\) , this implies that \(y \in B_{(Y, d|_{Y \times Y})}(x, r_x)\).
  But by definition of \(r_x\), this means that \(y \in E\), as desired.
  Thus, we have found an open set \(V\) in \((X, d)\) for which \(E = V \cap Y\) as desired.

  Now we do the converse.
  Suppose that \(E = V \cap Y\) for some open set \(V\) in \((X, d)\);
  we have to show that \(E\) is relatively open with respect to \(Y\).
  Let \(x\) be any point in \(E\);
  we have to show that \(x\) is an interior point of \(E\) in the metric space \((Y, d|_{Y \times Y})\).
  Since \(x \in E\), we know \(x \in V\).
  Since \(V\) is open in \((X, d)\), we know that there is a radius \(r > 0\) such that \(B_{(X, d)}(x, r)\) is contained in \(V\).
  Strictly speaking, \(r\) depends on \(x\), and so we could write \(r_x\) instead of \(r\), but for this argument we will only use a single choice of \(x\) (as opposed to the argument in the previous paragraph) and so we will not bother to subscript \(r\) here.
  Since \(E = V \cap Y\), this means that \(B_{(X, d)}(x, r) \cap Y\) is contained in \(E\).
  But \(B_{(X, d)}(x, r) \cap Y\) is exactly the same as \(B_{(Y, d|_{Y \times Y})}(x, r)\), and so \(B_{(Y, d|_{Y \times Y})}(x, r)\) is contained in \(E\).
  Thus, \(x\) is an interior point of \(E\) in the metric space \((Y, d|_{Y \times Y})\), as desired.
\end{proof}

\begin{proof}{(b)}
  First suppose that \(E\) is relatively closed with respect to \(Y\).
  Then by \cref{ii:1.3.3} \(E\) is closed in \((Y, d|_{Y \times Y})\).
  By \cref{ii:1.2.15}(e) \(Y \setminus E\) is open in \((Y, d|_{Y \times Y})\).
  By \cref{ii:1.3.4}(a) we know that \(Y \setminus E = V \cap Y\) for some set \(V \subseteq X\) which is open in \((X, d)\).
  Let \(K = X \setminus V\).
  By \cref{ii:1.2.15}(e) we know that \(K\) is closed in \((X, d)\).
  Then we have
  \begin{align*}
    K \cap Y & = (X \setminus V) \cap Y          \\
             & = (X \cap Y) \setminus (V \cap Y) \\
             & = Y \setminus (V \cap Y)          \\
             & = Y \setminus (Y \setminus E)     \\
             & = E.
  \end{align*}

  Now suppose that \(E = K \cap Y\) for some set \(K \subseteq X\) which is closed in \((X, d)\).
  Then by \cref{ii:1.2.15}(e) \(X \setminus K\) is open in \((X, d)\).
  By \cref{ii:1.3.4}(a) we know that \(F = (X \setminus K) \cap Y\) is relatively open with respect to \(Y\).
  Then by \cref{ii:1.2.15}(e) \(Y \setminus F\) is relatively closed with respect to \(Y\) and
  \begin{align*}
    Y \setminus F & = Y \setminus \big((X \setminus K) \cap Y\big)          \\
                  & = Y \setminus \big((X \cap Y) \setminus (K \cap Y)\big) \\
                  & = Y \setminus \big(Y \setminus (K \cap Y)\big)          \\
                  & = Y \setminus (Y \setminus E)                           \\
                  & = E.
  \end{align*}
\end{proof}

\exercisesection

\begin{ex}\label{ii:ex:1.3.1}
  Prove \cref{ii:1.3.4}(b).
\end{ex}

\begin{proof}
  See \cref{ii:1.3.4}.
\end{proof}
