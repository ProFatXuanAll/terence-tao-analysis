\section{The implicit function theorem}\label{sec:6.8}

\begin{note}
  Any function \(g : \R^n \to \R\) gives rise to a graph \(\Big\{\big(x, g(x)\big) : x \in \R^n\Big\}\) in \(\R^{n + 1}\), which in general looks like some sort of \(n\)-dimensional surface in \(\R^{n + 1}\)
  (the technical term for this is a \emph{hypersurface}).
  Conversely, one may ask which hypersurfaces are actually graphs of some function, and whether that function is continuous or differentiable.
\end{note}

\begin{note}
  If the hypersurface is given geometrically, then one can again invoke the vertical line test to work out whether it is a graph or not.
  But what if the hypersurface is given algebraically, or more generally, the hypersurface is given as some function?
  In this case, it is still possible to say whether the hypersurface is a graph, locally at least, by means of the \emph{implicit function theorem}.
\end{note}

\begin{thm}[Implicit function theorem]\label{6.8.1}
  Let \(E\) be an open subset of \(\R^n\), let \(f : E \to \R\) be continuously differentiable, and let \(y = (y_1, \dots, y_n)\) be a point in \(E\) such that \(f(y) = 0\) and \(\frac{\partial f}{\partial x_n}(y) \neq 0\).
  Then there exists an open subset \(U\) of \(\R^{n - 1}\) containing \((y_1, \dots, y_{n - 1})\), an open subset \(V\) of \(E\) containing \(y\), and a function \(g : U \to \R\) such that \(g(y_1, \dots, y_{n - 1}) = y_n\), and
  \begin{align*}
     & \{(x_1, \dots, x_n) \in V : f(x_1, \dots, x_n) = 0\}                                                     \\
     & = \Big\{\big(x_1, \dots, x_{n - 1}, g(x_1, \dots, x_{n - 1})\big) : (x_1, \dots, x_{n - 1}) \in U\Big\}.
  \end{align*}
  In other words, the set \(\{x \in V : f(x) = 0\}\) is a graph of a function over \(U\).
  Moreover, \(g\) is differentiable at \((y_1, \dots, y_{n - 1})\), and we have
  \[
    \frac{\partial g}{\partial x_j}(y_1, \dots, y_{n - 1}) = -\frac{\partial f}{\partial x_j}(y) / \frac{\partial f}{\partial x_n}(y) \tag{6.1}\label{eq 6.1}
  \]
  for all \(1 \leq j \leq n - 1\).
\end{thm}

\begin{proof}
  This theorem looks somewhat fearsome, but actually it is a fairly quick consequence of the inverse function theorem.
  Let \(F : E \to \R^n\) be the function
  \[
    F(x_1, \dots, x_n) \coloneqq \big(x_1, \dots, x_{n - 1}, f(x_1, \dots, x_n)\big).
  \]
  This function is continuously differentiable.
  Also note that
  \[
    F(y) = (y_1, \dots, y_{n - 1}, 0)
  \]
  and
  \begin{align*}
    D F(y) & = \bigg(\frac{\partial F}{\partial x_1}(y)^\top, \frac{\partial F}{\partial x_2}(y)^\top, \dots, \frac{\partial F}{\partial x_n}(y)^\top\bigg)                                         \\
           & = \begin{pmatrix}
                 1                                  & 0                                  & \dots  & 0                                        & 0                                  \\
                 0                                  & 1                                  & \dots  & 0                                        & 0                                  \\
                 \vdots                             & \vdots                             & \ddots & \vdots                                   & \vdots                             \\
                 0                                  & 0                                  & \dots  & 1                                        & 0                                  \\
                 \frac{\partial f}{\partial x_1}(y) & \frac{\partial f}{\partial x_2}(y) & \dots  & \frac{\partial f}{\partial x_{n - 1}}(y) & \frac{\partial f}{\partial x_n}(y)
               \end{pmatrix}.
  \end{align*}
  Since \(\frac{\partial f}{\partial x_n}(y)\) is assumed by hypothesis to be non-zero, this matrix is invertible;
  this can be seen either by computing the determinant, or using row reduction, or by computing the inverse explicitly, which is
  \[
    D F(y)^{-1} = \begin{pmatrix}
      1                                       & 0                                       & \dots  & 0                                             & 0      \\
      0                                       & 1                                       & \dots  & 0                                             & 0      \\
      \vdots                                  & \vdots                                  & \ddots & \vdots                                        & \vdots \\
      0                                       & 0                                       & \dots  & 1                                             & 0      \\
      -\frac{\partial f}{\partial x_1}(y) / a & -\frac{\partial f}{\partial x_2}(y) / a & \dots  & -\frac{\partial f}{\partial x_{n - 1}}(y) / a & 1 / a
    \end{pmatrix},
  \]
  where we have written \(a = \frac{\partial f}{\partial x_n}(y)\) for short.
  Thus the inverse function theorem (\cref{6.7.2}) applies, and we can find an open set \(V\) in \(E\) containing \(y\), and an open set \(W\) in \(\R^n\) containing \(F(y) = (y_1, \dots, y_{n - 1}, 0)\), such that \(F\) is a bijection from \(V\) to \(W\), and that \(F^{-1}\) is differentiable at \((y_1, \dots, y_{n - 1}, 0)\).

  Let us write \(F^{-1}\) in co-ordinates as
  \[
    F^{-1}(x) = \big(h_1(x), h_2(x), \dots, h_n(x)\big)
  \]
  where \(x \in W\).
  Since \(F\big(F^{-1}(x)\big) = x\), we have \(h_j(x_1, \dots, x_n) = x_j\) for all \(1 \leq j \leq n - 1\) and \(x \in W\), and
  \[
    f\big(x_1, \dots, x_{n - 1}, h_n(x_1, \dots, x_n)\big) = x_n.
  \]
  Also, \(h_n\) is differentiable at \((y_1, \dots, y_{n - 1}, 0)\) since \(F^{-1}\) is.

  Now we set \(U \coloneqq \{(x_1, \dots, x_{n - 1}) \in \R^{n - 1} : (x_1, \dots, x_{n - 1}, 0) \in W\}\).
  Note that \(U\) is open and contains \((y_1, \dots, y_{n - 1})\).
  Now we define \(g : U \to \R\) by \(g(x_1, \dots, x_{n - 1}) \coloneqq h_n(x_1, \dots, x_{n - 1}, 0)\).
  Then \(g\) is differentiable at \((y_1, \dots, y_{n - 1})\).
  Now we prove that
  \begin{align*}
     & \{(x_1, \dots, x_n) \in V : f(x_1, \dots, x_n) = 0\}                                                     \\
     & = \Big\{\big(x_1, \dots, x_{n - 1}, g(x_1, \dots, x_{n - 1})\big) : (x_1, \dots, x_{n - 1}) \in U\Big\}.
  \end{align*}
  First suppose that \((x_1, \dots, x_n) \in V\) and \(f(x_1, \dots, x_n) = 0\).
  Then we have
  \[
    F(x_1, \dots, x_n) = (x_1, \dots, x_{n - 1}, 0),
  \]
  which lies in \(W\).
  Thus \((x_1, \dots, x_{n - 1})\) lies in \(U\).
  Applying \(F^{-1}\), we see that
  \[
    (x_1, \dots, x_n) = F^{-1}(x_1, \dots, x_{n - 1}, 0).
  \]
  In particular \(x_n = h_n(x_1, \dots, x_{n - 1}, 0)\), and hence \(x_n = g(x_1, \dots, x_{n - 1})\).
  Thus every element of the left-hand set lies in the right-hand set.
  The reverse inclusion comes by reversing all the above steps and is left to the reader.

  Finally, we show the formula for the partial derivatives of \(g\).
  From the preceding discussion we have
  \[
    f\big(x_1, \dots, x_{n - 1}, g(x_1, \dots, x_{n - 1})\big) = 0
  \]
  for all \((x_1, \dots, x_{n - 1}) \in U\).
  Since \(g\) is differentiable at \((y_1, \dots, y_{n - 1})\), and \(f\) is differentiable at \(\big(y_1, \dots, y_{n - 1}, g(y_1, \dots, y_{n - 1})\big) = y\), we may use the chain rule, differentiating in \(x_j\), to obtain
  \[
    \frac{\partial f}{\partial x_j}(y) + \frac{\partial f}{\partial x_n}(y) \frac{\partial g}{\partial x_j}(y_1, \dots, y_{n - 1}) = 0
  \]
  and the claim follows by simple algebra.
\end{proof}

\begin{rmk}\label{6.8.2}
  \cref{eq 6.1} is sometimes derived using \emph{implicit differentiation}.
  Basically, the point is that if you know that
  \[
    f(x_1, \dots, x_n) = 0
  \]
  then (as long as \(\frac{\partial f}{\partial x_n} \neq 0\)) the variable \(x_n\) is ``implicitly'' defined in terms of the other \(n - 1\) variables, and one can differentiate the above identity in, say, the \(x_j\) direction using the chain rule to obtain
  \[
    \frac{\partial f}{\partial x_j} + \frac{\partial f}{\partial x_n} \frac{\partial x_n}{\partial x_j} = 0
  \]
  which is \cref{eq 6.1} in disguise
  (we are using \(g\) to represent the implicit function defining \(x_n\) in terms of \(x_1, \dots, x_n\)).
  Thus, the implicit function theorem allows one to define a dependence implicitly, by means of a constraint rather than by a direct formula of the form \(x_n = g(x_1, \dots, x_{n - 1})\).
\end{rmk}

\begin{note}
  In the implicit function theorem, if the derivative \(\frac{\partial f}{\partial x_n}\) equals zero at some point, then it is unlikely that the set \(\{x \in \R^n : f(x) = 0\}\) can be written as a graph of the \(x_n\) variable in terms of the other \(n - 1\) variables near that point.
  However, if some other derivative \(\frac{\partial f}{\partial x_j}\) is non-zero, then it would be possible to write the \(x_j\) variable in terms of the other \(n - 1\) variables, by a variant of the implicit function theorem.
  Thus as long as the gradient \(\nabla f\) is not entirely zero, one can write this set \(\{x \in \R^n : f(x) = 0\}\) as a graph of \emph{some} variable \(x_j\) in terms of the other \(n - 1\) variables.
  (The circle \(\{(x, y) \in \R^2 : x^2 + y^2 - 1 = 0\}\) is a good example of this;
  it is not a graph of \(y\) in terms of \(x\), or \(x\) in terms of \(y\), but near every point it is one of the two.
  And this is because the gradient of \(x^2 + y^2 - 1\) is never zero on the circle.)
  However, if \(\nabla f\) does vanish at some point \(x_0\), then we say that \(f\) has a \emph{critical point} at \(x_0\) and the behavior there is much more complicated.
  For instance, the set \(\{(x, y) \in \R^2 : x^2 - y^2 = 0\}\) has a critical point at \((0, 0)\) and there the set does not look like a graph of any sort
  (it is the union of two lines).
\end{note}

\begin{rmk}\label{6.8.4}
  Sets which look like graphs of continuous functions at every point have a name, they are called \emph{manifolds}.
  Thus \(\{x \in \R^n : f(x) = 0\}\) will be a manifold if it contains no critical points of \(f\).
  The theory of manifolds is very important in modern geometry (especially differential geometry and algebraic geometry), but we will not discuss it here as it is a graduate level topic.
\end{rmk}

\exercisesection

\begin{ex}\label{ex:6.8.1}
  Let the notation and hypotheses be as in \cref{6.8.1}.
  Show that, after shrinking the open sets \(U, V\) if necessary, that the function \(g\) becomes continuously differentiable on all of \(U\), and \cref{eq 6.1} holds at all points of \(U\).
\end{ex}