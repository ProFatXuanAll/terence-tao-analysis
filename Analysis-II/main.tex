% We use chapter structure.
\documentclass[11pt,a4paper]{book}

%==============================================================================
% Preamble.
%==============================================================================

% Correctly showing characters outside ASCII.
\usepackage[T1]{fontenc}
% File is written and read with utf8 encoding.
\usepackage[utf8]{inputenc}
% Set paging layout.
\usepackage[paperwidth=160mm,paperheight=240mm,margin=15mm]{geometry}
% Including `amsfonts'.
% Must be loaded before `mathtools'.
\usepackage{amssymb}
% Including `amsmath' and fixing bugs for `amsmath'.
\usepackage{mathtools}
% Must be loaded after `amsmath' and `mathtools'.
\usepackage{amsthm}
% Define page header and footer layout.
\usepackage{fancyhdr}
% LaTex 3 new command tools.
\usepackage{xparse}

% Use `fancyhdr' page style.  All page style settings must be called only after
% this command.  See `fancyhdr' for details.
\pagestyle{fancy}

% Make header height enough to fit chapter / section titles.
\setlength{\headheight}{15pt}
% Change chapter and section marks' formatting.  Note that I use `\markright'
% to make sure header use chapter information when section information is not
% available (this is need for appendix).
\renewcommand{\chaptermark}[1]{\markright{\textsf{\small Chap. \thechapter \quad #1}}}
\renewcommand{\sectionmark}[1]{\markright{\textsf{\small Sec. \thesection \quad #1}}}
% Cleanup page header settings.
\fancyhead{}
% Put page number on the left of page header.
\fancyhead[L]{\textbf{\textsf{\small \thepage}}}
% Put chapter / section information on the right of page header.
\fancyhead[R]{\rightmark}
% Cleanup page footer settings.
\fancyfoot{}

% Reset `plain' page style which is used for the first page of each chapter.
% I set this to make page number style consistent.
\fancypagestyle{plain}{%
  \fancyhf{}% clear all fields
  \fancyfoot[C]{\textbf{\textsf{\small \thepage}}}%
  \renewcommand{\headrulewidth}{0pt}%
}
% Automatically adjust character spacing at margins.
\usepackage{microtype}
% Provide further utilities and fix bugs for `enumerate', `itemize' and
% `description'.
\usepackage{enumitem}
% Provide better quoting environment.
\usepackage{dirtytalk}
% Parsing list inside `\newcommand'.
\usepackage{listofitems}
% Nice looking if-then-else structure with comparison functionality.
\usepackage{ifthen}
% Automatically add hyperlinks to labels/refs.  Must be loaded after all
% packages above and before `cleveref'.  Recommend to use with `natbib' when
% you need bibtex.
\usepackage{hyperref}

\hypersetup{       % This macro come with `hyperref'.
	colorlinks=true, % Color hyperlinks.
	linkcolor=blue,  % Color local hyperlinks with blue.
	urlcolor=cyan,   % Color url links with cyan.
}

% Must be loaded after `hyperref'.  We always capitalize each cross-references'
% type name.  See `cleveref' for details.
\usepackage[capitalize]{cleveref}

% Allow page break in the middle of multi-line equations.
\allowdisplaybreaks

%------------------------------------------------------------------------------
% Define environments.
%------------------------------------------------------------------------------

% Text inside the body of theorem-like environments are set to Roman font.
% Theorem-like environments share their counters, counters follow section and reset in every sections
% (except for axioms, axioms counters are reset in each chapter).
% Exercises has their owned counter.
% Notes do not use counter.
% See `amsthm' for details.
\theoremstyle{definition}
\newtheorem{ax}{Ax.}[chapter]
\newtheorem{ac}{A.Cor.}[section]
\newtheorem{ex}{Ex.}[section]
\newtheorem{thm}{Thm.}[section]
\newtheorem{cor}[thm]{Cor.}
\newtheorem{defn}[thm]{Def.}
\newtheorem{eg}[thm]{E.g.}
\newtheorem{lem}[thm]{Lem.}
\newtheorem{prop}[thm]{Prop.}
\newtheorem{rmk}[thm]{Rmk.}
\newtheorem*{note}{Note}

\theoremstyle{remark}
\newtheorem*{meta-proof}{Meta-proof}

% Define plural form for theorem-like environments. See `cleveref' for details.
\crefname{ax}{Ax.}{Ax.}
\crefname{ac}{A.Cor.}{A.Cor.}
\crefname{chapter}{Ch.}{Ch.}
\crefname{cor}{Cor.}{Cor.}
\crefname{defn}{Def.}{Def.}
\crefname{eg}{E.g.}{E.g.}
\crefname{ex}{Ex.}{Ex.}
\crefname{lem}{Lem.}{Lem.}
\crefname{note}{Note}{Notes}
\crefname{prop}{Prop.}{Prop.}
\crefname{rmk}{Rmk.}{Rmk.}
\crefname{section}{Sec.}{Sec.}
\crefname{thm}{Thm.}{Thm.}


% In `enumerate' enviroments, items' label are alphabets and surrounded by
% parentheses.  See `enumitem' for details.
\renewcommand{\labelenumi}{\textnormal{(}\alph{enumi}\textnormal{)}}

% Formatting exercises section.
\NewDocumentCommand{\exercisesection}{}{
  \begin{center}
    --- Exercises ---
  \end{center}
}

%------------------------------------------------------------------------------
% Define operators and symbols.
%------------------------------------------------------------------------------

% Absolute value.
\DeclarePairedDelimiter{\absTmp}{\lvert}{\rvert}
\NewDocumentCommand{\abs}{m}{\absTmp*{#1}}
% Ceiling.
\DeclarePairedDelimiter{\ceilTmp}{\lceil}{\rceil}
\NewDocumentCommand{\ceil}{m}{\ceilTmp*{#1}}
% Floor.
\DeclarePairedDelimiter{\floorTmp}{\lfloor}{\rfloor}
\NewDocumentCommand{\floor}{m}{\floorTmp*{#1}}
% Evaluate.
\DeclarePairedDelimiter{\evalTmp}{.}{\rvert}
\NewDocumentCommand{\eval}{m}{\evalTmp*{#1}}
% Parenthese.
\DeclarePairedDelimiter{\paTmp}{\lparen}{\rparen}
\NewDocumentCommand{\pa}{m}{\paTmp*{#1}}
% Bracket.
\DeclarePairedDelimiter{\brTmp}{\lbrack}{\rbrack}
\NewDocumentCommand{\br}{m}{\brTmp*{#1}}
% Brace.
\DeclarePairedDelimiter{\BTmp}{\lbrace}{\rbrace}
\NewDocumentCommand{\B}{m}{\BTmp*{#1}}
% Set.
\NewDocumentCommand{\set}{m}{\B{#1}}
% Norm.
\DeclarePairedDelimiter\norm{\lVert}{\rVert}
% Inner product.
\DeclarePairedDelimiter\inner{\langle}{\rangle}

% Define common symbols.
% See `amsmath' section 9.2 for details.

% Fields.
\NewDocumentCommand{\field}{m}{\mathbf{#1}}
% General field.
\NewDocumentCommand{\F}{}{\field{F}}
% Complex number.
\NewDocumentCommand{\C}{}{\mathbb{C}}
% Natural number.
\NewDocumentCommand{\N}{}{\mathbb{N}}
% Rational number.
\NewDocumentCommand{\Q}{}{\mathbb{Q}}
% Real number.
\NewDocumentCommand{\R}{}{\mathbb{R}}
% Integer number.
\NewDocumentCommand{\Z}{}{\mathbb{Z}}

% Proof statements reference text.
\NewDocumentCommand{\byOptionalArgumentProcess}{m}{(#1)}
\NewDocumentCommand{\by}{m >{\SplitList{,}} o}{%
	\IfNoValueTF{#2}{%
		\text{(by \cref{#1})}%
	}{%
		\text{(by \cref{#1}\ProcessList{#2}{\byOptionalArgumentProcess})}%
	}%
}

% Proof statements reference induction hypothesis.
\NewDocumentCommand{\byIH}{}{\text{(by induction hypothesis)}}

%==============================================================================
% Document.
%==============================================================================

\begin{document}

%------------------------------------------------------------------------------
% Front matters.
%------------------------------------------------------------------------------

\frontmatter

% Author informations.
\title{Analysis II}
\author{ProFatXuanAll}
\maketitle

% Table of contents.
\tableofcontents

%------------------------------------------------------------------------------
% Main matters.
%------------------------------------------------------------------------------

\mainmatter

% All chapters are in separated files.
% We include them here.
\chapter{Metric spaces}\label{ii:ch:1}

\section{Definitions and examples}\label{sec:1.1}

\begin{lem}\label{1.1.1}
  Let \((x_n)_{n = m}^\infty\) be a sequence of real numbers, and let \(x\) be another real number.
  Then \((x_n)_{n = m}^\infty\) converges to \(x\) if and only if \(\lim_{n \to \infty} d(x_n, x) = 0\).
\end{lem}

\begin{proof}
  \begin{align*}
         & \lim_{n \to \infty} x_n = x                                                        \\
    \iff & \lim_{n \to \infty} x_n - x = 0                                                    \\
    \iff & \lim_{n \to \infty} \abs{x_n - x} = 0 &  & \text{(by Corollary 6.4.7, Analysis I)} \\
    \iff & \lim_{n \to \infty} d(x_n, x) = 0.
  \end{align*}
\end{proof}

\begin{note}
  One would now like to generalize this notion of convergence, so that one can take limits not just of sequences of real numbers, but also sequences of complex numbers, or sequences of vectors, or sequences of matrices, or sequences of functions, even sequences of sequences.
  One way to do this is to redefine the notion of convergence each time we deal with a new type of object.
  A more efficient way is to work \emph{abstractly}, defining a very general class of spaces - which includes such standard spaces as the real numbers, complex numbers, vectors, etc. - and define the notion of convergence on this entire class of spaces at once.
  (A \emph{space} is just the set of all objects of a certain type.
  Mathematically, there is not much distinction between a space and a set, except that spaces tend to have much more structure than what a random set would have.)
\end{note}

\begin{note}
  It turns out that there are two very useful classes of spaces which do the job.
  The first class is that of \emph{metric spaces}.
  There is a more general class of spaces, called \emph{topological spaces}.
\end{note}

\begin{defn}[Metric spaces]\label{1.1.2}
  A metric space \((X, d)\) is a space \(X\) of objects (called \emph{points}), together with a \emph{distance function} or \emph{metric} \(d : X \times X \to [0, +\infty)\), which associates to each pair \(x, y\) of points in \(X\) a non-negative real number \(d(x, y) \geq 0\).
  Furthermore, the metric must satisfy the following four axioms:
  \begin{enumerate}
    \item For any \(x \in X\), we have \(d(x, x) = 0\).
    \item (Positivity) For any distinct \(x, y \in X\), we have \(d(x, y) > 0\).
    \item (Symmetry) For any \(x, y \in X\), we have \(d(x, y) = d(y, x)\).
    \item (Triangle inequality) For any \(x, y, z \in X\), we have \(d(x, z) \leq d(x, y) + d(y, z)\).
  \end{enumerate}
\end{defn}

\begin{note}
  In many cases it will be clear what the metric \(d\) is, and we shall abbreviate \((X, d)\) as just \(X\).
\end{note}

\begin{rmk}\label{1.1.3}
  The conditions (a) and (b) of \cref{1.1.1} can be rephrased as follows:
  for any \(x, y \in X\) we have \(d(x, y) = 0\) if and only if \(x = y\).
\end{rmk}

\begin{eg}[The real line]\label{1.1.4}
  Let \(\R\) be the real numbers, and let \(d : \R \times \R \to [0, \infty)\) be the metric \(d(x, y) \coloneqq \abs{x - y}\) mentioned earlier.
  Then \((\R, d)\) is a metric space.
  We refer to \(d\) as the \emph{standard metric} on \(\R\), and if we refer to \(\R\) as a metric space, we assume that the metric is given by the standard metric \(d\) unless otherwise specified.
\end{eg}

\begin{eg}[Induced metric spaces]\label{1.1.5}
  Let \((X, d)\) be any metric space, and let \(Y\) be a subset of \(X\).
  Then we can restrict the metric function \(d : X \times X \to [0, +\infty)\) to the subset \(Y \times Y\) of \(X \times X\) to create a restricted metric function \(d|_{Y \times Y} : Y \times Y \to [0, +\infty)\) of \(Y\);
  this is known as the metric on \(Y\) \emph{induced} by the metric \(d\) on \(X\).
  The pair \((Y, d|_{Y \times Y})\) is a metric space and is known the \emph{subspace} of \((X, d)\) induced by \(Y\).
  Thus for instance the metric on the real line in the \cref{1.1.4} induces a metric space structure on any subset of the reals, such as the integers \(Z\), or an interval \([a, b]\), etc.
\end{eg}

\begin{eg}[Euclidean spaces]\label{1.1.6}
  Let \(n \geq 1\) be a natural number, and let \(\R^n\) be the space of \(n\)-tuples of real numbers:
  \[
    \R^n = \{(x_1, x_2, \dots, x_n) : x_1, \dots, x_n \in \R\}.
  \]
  We define the \emph{Euclidean metric} (also called the \emph{\(l^2\) metric}) \(d_{l^2} : \R^n \times\R^n \to \R\) by
  \begin{align*}
    d_{l^2}((x_1, \dots, x_n), (y_1, \dots, y_n)) & \coloneqq \sqrt{(x_1 - y_1)^2 + \dots + (x_n - y_n)^2} \\
                                                  & = \bigg(\sum_{i = 1}^n (x_i - y_i)^2\bigg)^{1 / 2}.
  \end{align*}
\end{eg}

\begin{note}
  Euclidean metric corresponds to the geometric distance between the two points \((x_1, x_2, \dots, x_n)\), \((y_1, y_2, \dots, y_n)\) as given by Pythagoras' theorem.
  While geometry does give some very important examples of metric spaces, it is possible to have metric spaces which have no obvious geometry whatsoever.
  The verification that \((\R^n, d)\) is indeed a metric space can be seen geometrically (for instance, the triangle inequality now asserts that the length of one side of a triangle is always less than or equal to the sum of the lengths of the other two sides), but can also be proven algebraically.
  We refer to \((\R^n , d_{l^2})\) as the \emph{Euclidean space} of \emph{dimension \(n\)}.
  Extending the convention from \cref{1.1.4}, if we refer to \(\R^n\) as a metric space, we assume that the metric is given by the Euclidean metric unless otherwise specified.
\end{note}

\begin{eg}[Taxi-cab metric]\label{1.1.7}
  Again let \(n \geq 1\), and let \(\R^n\) be as before.
  But now we use a different metric \(d_{l^1}\), the so-called \emph{taxicab metric} (or \emph{\(l^1\) metric}), defined by
  \begin{align*}
    d_{l^1}((x_1, \dots, x_n), (y_1, \dots, y_n)) & \coloneqq \abs{x_1 - y_1} + \dots + \abs{x_n - y_n} \\
                                                  & = \sum_{i = 1}^n \abs{x_i - y_i}.
  \end{align*}
\end{eg}

\begin{note}
  This metric is called the taxi-cab metric, because it models the distance a taxi-cab would have to traverse to get from one point to another if the cab was only allowed to move in cardinal directions (north, south, east, west) and not diagonally.
  As such it is always at least as large as the Euclidean metric, which measures distance ``as the crow flies'', as it were.
  We claim that the space \((\R^n, d_{l^1})\) is also a metric space.
  The metrics are not quite the same, but we do have the inequalities
  \[
    d_{l^2}(x, y) \leq d_{l^1}(x, y) \leq \sqrt{n} d_{l^2}(x, y)
  \]
  for all \(x, y\).
\end{note}

\begin{rmk}\label{1.1.8}
  The taxi-cab metric is useful in several places, for instance in the theory of error correcting codes.
  A string of \(n\) binary digits can be thought of as an element of \(\R^n\).
  The taxi-cab distance between two binary strings is then the number of bits in the two strings which do not match.
  The goal of error-correcting codes is to encode each piece of information (e.g., a letter of the alphabet) as a binary string in such a way that all the binary strings are as far away in the taxicab metric from each other as possible;
  this minimizes the chance that any distortion of the bits due to random noise can accidentally change one of the coded binary strings to another, and also maximizes the chance that any such distortion can be detected and correctly repaired.
\end{rmk}

\begin{eg}[Sup norm metric]\label{1.1.9}
  Again let \(n \geq 1\), and let \(\R^n\) be as before.
  But now we use a different metric \(d_{l^\infty}\), the so-called \emph{sup norm metric} (or \emph{\(l^\infty\) metric}), defined by
  \[
    d_{l^\infty} ((x_1, \dots, x_n), (y_1, \dots, y_n)) \coloneqq \sup\{\abs{x_i - y_i} : 1 \leq i \leq n\}.
  \]
\end{eg}

\begin{note}
  The space \((\R^n, d_{l^\infty})\) is also a metric space, and is related to the \(l^2\) metric by the inequalities
  \[
    \dfrac{1}{\sqrt{n}} d_{l^2}(x, y) \leq d_{l^\infty}(x, y) \leq d_{l^2}(x, y)
  \]
  for all \(x, y\).
\end{note}

\begin{rmk}\label{1.1.10}
  The \(l^1\), \(l^2\), and \(l^\infty\) metrics are special cases of the more general \emph{\(l^p\) metrics}, where \(p \in [1, +\infty)\).
\end{rmk}

\begin{eg}[Discrete metric]\label{1.1.11}
  Let \(X\) be an arbitrary set (finite or infinite), and define the \emph{discrete metric} \(d_{\text{disc}}\) by setting \(d_{\text{disc}}(x, y) \coloneqq 0\) when \(x = y\), and \(d_{\text{disc}}(x, y) \coloneqq 1\) when \(x \neq y\).
  Thus, in this metric, all points are equally far apart.
  The space \((X, d_{\text{disc}})\) is a metric space.
  Thus every set \(X\) has at least one metric on it.
\end{eg}

\setcounter{thm}{13}
\begin{defn}[Convergence of sequences in metric spaces]\label{1.1.14}
  Let \(m\) be an integer, \((X, d)\) be a metric space and let \((x^{(n)})_{n = m}^\infty\) be a sequence of points in \(X\)
  (i.e., for every natural number \(n \geq m\), we assume that \(x^{(n)}\) is an element of \(X\)).
  Let \(x\) be a point in \(X\).
  We say that \emph{\((x^{(n)})_{n = m}^\infty\) converges to \(x\) with respect to the metric \(d\)}, if and only if the limit \(\lim_{n \to \infty} d(x^{(n)}, x)\) exists and is equal to \(0\).
  In other words, \((x^{(n)})_{n = m}^\infty\) converges to \(x\) with respect to \(d\) if and only if for every \(\varepsilon > 0\), there exists an \(N \geq m\) such that \(d(x^{(n)}, x) \leq \varepsilon\) for all \(n \geq N\).
\end{defn}

\begin{rmk}\label{1.1.15}
  In view of \cref{1.1.1} we see that this definition generalizes our existing notion of convergence of sequences of real numbers.
  In many cases, it is obvious what the metric d is, and so we shall often just say ``\((x^{(n)})_{n = m}^\infty\) converges to \(x\)'' instead of ``\((x^{(n)})_{n = m}^\infty\) converges to \(x\) with respect to the metric \(d\)'' when there is no chance of confusion.
  We also sometimes write ``\(x^{(n)} \to x\) as \(n \to \infty\) instead.
\end{rmk}

\begin{rmk}\label{1.1.16}
  There is nothing special about the superscript \(n\) in the above definition;
  it is a dummy variable.
  Saying that \((x^{(n)})_{n = m}^\infty\) converges to \(x\) is exactly the same statement as saying that \((x^{(k)})_{k = m}^\infty\) converges to \(x\), for example;
  and sometimes it is convenient to change superscripts, for instance if the variable \(n\) is already being used for some other purpose.
  Similarly, it is not necessary for the sequence \(x^{(n)}\) to be denoted using the superscript \((n)\);
  the above definition is also valid for sequences \(x_n\), or functions \(f(n)\), or indeed of any expression which depends on \(n\) and takes values in \(X\).
  We see that the starting point \(m\) of the sequence is unimportant for the purposes of taking limits;
  if \((x^{(n)})_{n = m}^\infty\) converges to \(x\), then \((x^{(n)})_{n = m'}^\infty\) also converges to \(x\) for any \(m' \geq m\).
\end{rmk}

\begin{note}
  The convergence of a sequence can depend on what metric one uses.
\end{note}

\setcounter{thm}{17}
\begin{prop}[Equivalence of \(l^1\), \(l^2\), \(l^\infty\)]\label{1.1.18}
  Let \(\R^n\) be a Euclidean space, and let \((x^{(k)})_{k = m}^\infty\) be a sequence of points in \(\R^n\).
  We write \(x^{(k)} = (x_1^{(k)}, x_2^{(k)}, \dots, x_n^{(k)})\), i.e., for \(j = 1, 2, \dots, n\), \(x_j \in \R\) is the \(j^{\text{th}}\) coordinate of \(x^{(k)} \in \R^n\).
  Let \(x = (x_1, \dots, x_n)\) be a point in \(\R^n\).
  Then the following four statements are equivalent:
  \begin{enumerate}
    \item \((x^{(k)})_{k = m}^\infty\) converges to \(x\) with respect to the Euclidean metric \(d_{l^2}\).
    \item \((x^{(k)})_{k = m}^\infty\) converges to \(x\) with respect to the taxi-cab metric \(d_{l^1}\).
    \item \((x^{(k)})_{k = m}^\infty\) converges to \(x\) with respect to the sup norm metric \(d_{l^\infty}\).
    \item For every \(1 \leq j \leq n\), the sequence \((x_j^{(k)})_{k = m}^\infty\) converges to \(x_j\).
          (Notice that this is a sequence of real numbers, not of points in \(\R^n\).)
  \end{enumerate}
\end{prop}

\begin{proof}
  We have
  \begin{align*}
         & \lim_{k \to \infty} d_{l^2}(x^{(k)}, x) = 0                                                            &  & \text{(by \cref{1.1.14})} \\
    \iff & \lim_{k \to \infty} \sqrt{\sum_{j = 1}^n (x_j^{(k)} - x_j)^2} = 0                                      &  & \text{(by \cref{1.1.6})}  \\
    \iff & \lim_{k \to \infty} \bigg(\sum_{j = 1}^n (x_j^{(k)} - x_j)^2\bigg) = 0                                                                \\
    \iff & \sum_{j = 1}^n \bigg(\lim_{k \to \infty} (x_j^{(k)} - x_j)^2\bigg) = 0                                                                \\
    \iff & \forall j \in \{i \in \N : 1 \leq i \leq n\}, \lim_{k \to \infty} x_j^{(k)} - x_j = 0                                                 \\
    \iff & \forall j \in \{i \in \N : 1 \leq i \leq n\}, \lim_{k \to \infty} x_j^{(k)} = x_j                                                     \\
    \iff & \forall j \in \{i \in \N : 1 \leq i \leq n\}, \lim_{k \to \infty} \abs{x_j^{(k)} - x_j} = 0            &  & \text{(by \cref{1.1.1})}  \\
    \iff & \sum_{j = 1}^n \bigg(\lim_{k \to \infty} \abs{x_j^{(k)} - x_j}\bigg) = 0                                                              \\
    \iff & \lim_{k \to \infty} \bigg(\sum_{j = 1}^n \abs{x_j^{(k)} - x_j}\bigg) = 0                                                              \\
    \iff & \lim_{k \to \infty} d_{l^1}(x^{(k)}, x) = 0                                                            &  & \text{(by \cref{1.1.7})}  \\
    \iff & \lim_{k \to \infty} \sup\bigg\{\abs{x_j^{(k)} - x_j} : j \in \{i \in \N : 1 \leq i \leq n\}\bigg\} = 0                                \\
    \iff & \lim_{k \to \infty} d_{l^\infty}(x^{(k)}, x) = 0.                                                      &  & \text{(by \cref{1.1.9})}
  \end{align*}
\end{proof}

\begin{note}
  Because of the equivalence of \cref{1.1.18}(a), (b) and (c), we say that the Euclidean, taxicab, and sup norm metrics on \(\R^n\) are \emph{equivalent}.
  (There are infinite-dimensional analogues of the Euclidean, taxicab, and sup norm metrics which are \emph{not} equivalent.)
\end{note}

\begin{prop}[Convergence in the discrete metric]\label{1.1.19}
  Let \(X\) be any set, and let \(d_{\text{disc}}\) be the discrete metric on \(X\).
  Let \((x^{(n)})_{n = m}^\infty\) be a sequence of points in \(X\), and let \(x\) be a point in \(X\).
  Then \((x^{(n)})_{n = m}^\infty\) converges to \(x\) with respect to the discrete metric \(d_{\text{disc}}\) if and only if there exists an \(N \geq m\) such that \(x^{(n)} = x\) for all \(n \geq N\).
\end{prop}

\begin{proof}
  By \cref{1.1.14} we know that \(\lim_{n \to \infty} d_{\text{disc}}(x^{(n)}, x) = 0\) iff \(\forall \varepsilon \in \R^+\), \(\exists\ N \in \N\) and \(N \geq m\) such that \(d(x^{(n)}, x) \leq \varepsilon\) for all \(n \geq N\).
  By \cref{1.1.11} we know that \(d_{\text{disc}}(x^{(n)}, x) \leq \varepsilon\) iff \(x^{(n)} = x\).
\end{proof}

\begin{prop}[Uniqueness of limits]\label{1.1.20}
  Let \((X, d)\) be a metric space, and let \((x^{(n)})_{n = m}^\infty\) be a sequence in \(X\).
  Suppose that there are two points \(x, x' \in X\) such that \((x^{(n)})_{n = m}^\infty\) converges to \(x\) with respect to \(d\), and \((x^{(n)})_{n = m}^\infty\) also converges to \(x'\) with respect to \(d\).
  Then we have \(x = x'\).
\end{prop}

\begin{proof}
  By \cref{1.1.14} we have \(\lim_{n \to \infty} d(x^{(n)}, x) = 0\) and \(\lim_{n \to \infty} d(x^{(n)}, x') = 0\).
  So
  \begin{align*}
             & \lim_{n \to \infty} d(x^{(n)}, x) + \lim_{n \to \infty} d(x^{(n)}, x') = 0                                     \\
    \implies & \lim_{n \to \infty} \bigg(d(x^{(n)}, x) + d(x^{(n)}, x')\bigg) = 0                                             \\
    \implies & d(x, x') = \lim_{n \to \infty} d(x, x') \leq 0                             &  & \text{(by \cref{1.1.2}(d))}    \\
    \implies & d(x, x') = 0                                                               &  & \text{(by \cref{1.1.2}(a)(b))} \\
    \implies & x = x'.                                                                    &  & \text{(by \cref{1.1.2}(a))}
  \end{align*}
\end{proof}

\begin{note}
  Because of \cref{1.1.20}, it is safe to introduce the following notation:
  if \((x^{(n)})_{n = m}^\infty\) converges to \(x\) in the metric \(d\), then we write \(d - \lim_{n \to \infty} x^{(n)} = x\), or simply \(\lim_{n \to \infty} x^{(n)} = x\) when there is no confusion as to what \(d\) is.
  The meaning of \(d - \lim_{n \to \infty} x^{(n)}\) can depend on what \(d\) is;
  however \cref{1.1.20} assures us that once \(d\) is fixed, there can be at most one value of \(d - \lim_{n \to \infty} x^{(n)}\).
  (Of course, it is still possible that this limit does not exist;
  some sequences are not convergent.)
\end{note}

\begin{rmk}\label{1.1.21}
  It is possible for a sequence to converge to one point using one metric, and another point using a different metric, although such examples are usually quite artificial.
  Thus changing the metric on a space can greatly affect the nature of convergence (also called the \emph{topology}) on that space.
\end{rmk}

\begin{ac}\label{ac:1.1.1}
  Let \(\vec{u} = (u_1, u_2), \vec{v} = (v_1, v_2)\) be two vectors in \(\R^2\) such that \(\abs{\vec{v}} \neq 0\).
  Let \(\alpha \in \R\) and let \(\alpha \vec{v}\) be the vector \(\vec{u}\) project onto \(\vec{v}\).
  Then
  \[
    \alpha = \dfrac{\vec{u} \cdot \vec{v}}{\abs{\vec{v}}^2}.
  \]
\end{ac}

\begin{proof}
  Let \(\vec{z} = (v_2, -v_1)\).
  We know that \(\vec{v} \bot \vec{z}\) since
  \[
    \vec{v} \cdot \vec{z} = v_1 v_2 - v_1 v_2 = 0.
  \]
  Since \(\alpha \vec{v}\) is the vector \(\vec{u}\) project onto \(\vec{v}\), we know that \(\exists\ \beta \in \R\) such that \(\alpha \vec{v} + \beta \vec{z} = \vec{u}\).
  Then we have
  \begin{align*}
             & \alpha \vec{v} + \beta \vec{z} = \vec{u}                                                            \\
    \implies & \begin{dcases}
                 \alpha v_1 + \beta v_2 = u_1 \\
                 \alpha v_2 - \beta v_1 = u_2
               \end{dcases}                                                                        \\
    \implies & \begin{dcases}
                 \alpha v_1^2 + \beta v_1 v_2 = u_1 v_1 \\
                 \alpha v_2^2 - \beta v_1 v_2 = u_2 v_2
               \end{dcases}                                                              \\
    \implies & \alpha (v_1^2 + v_2^2) = u_1 v_1 + u_2 v_2                                                          \\
    \implies & \alpha = \dfrac{u_1 v_1 + u_2 v_2}{v_1^2 + v_2^2} = \dfrac{\vec{u} \cdot \vec{v}}{\abs{\vec{v}}^2}.
  \end{align*}
\end{proof}

\begin{ac}\label{ac:1.1.2}
  Let \(n \in \Z^+\) and let \(\vec{u}, \vec{v}\) be two vectors in \(\R^n\) such that \(\abs{\vec{v}} \neq 0\).
  Let \(\alpha \in \R\) and let \(\alpha \vec{v}\) be the vector \(\vec{u}\) project onto \(\vec{v}\).
  Then
  \[
    \alpha = \dfrac{\vec{u} \cdot \vec{v}}{\abs{\vec{v}}^2}.
  \]
\end{ac}

\begin{proof}
  We can use \(\vec{u}\) and \(\vec{v}\) to form a linear combination \(\vec{z} \in \R^n\) such that \(\vec{z} \bot \vec{v}\).
  Since \(\alpha \vec{v}\) is the vector \(\vec{u}\) project onto \(\vec{v}\), we know that \(\exists\ \beta \in \R\) such that \(\alpha \vec{v} + \beta \vec{z} = \vec{u}\).
  Then we have \(\beta \vec{z} \cdot \vec{v} = 0\) and
  \begin{align*}
             & \alpha \vec{v} + \beta \vec{z} = \vec{u}                                                                        \\
    \implies & \alpha \vec{v} \cdot \vec{v} + \beta \vec{z} \cdot \vec{v} = \vec{u} \cdot \vec{v}                              \\
    \implies & \alpha = \dfrac{\vec{u} \cdot \vec{v}}{\vec{v} \cdot \vec{v}} = \dfrac{\vec{u} \cdot \vec{v}}{\abs{\vec{v}}^2}.
  \end{align*}
\end{proof}

\exercisesection

\begin{ex}\label{ex:1.1.1}
  Prove \cref{1.1.1}.
\end{ex}

\begin{proof}
  See \cref{1.1.1}.
\end{proof}

\begin{ex}\label{ex:1.1.2}
  Show that the real line with the metric \(d(x, y) \coloneqq \abs{x - y}\) is indeed a metric space.
\end{ex}

\begin{proof}
  Let \(x, y, z \in \R\).
  For identity:
  We have \(d(x, x) = \abs{x - x} = 0\).
  For positivity:
  If \(x \neq y\), then \(d(x, y) = \abs{x - y} > 0\).
  For symmetry:
  We have \(d(x, y) = \abs{x - y} = \abs{y - x} = d(y, x)\).
  For triangle inequality:
  We have \(d(x, z) = \abs{x - z} = \abs{x - y + y - z} \leq \abs{x - y} + \abs{y - z} = d(x, y) + d(y, z)\).
  Thus by \cref{1.1.2} \((\R, d)\) is a metric space.
\end{proof}

\begin{ex}\label{ex:1.1.3}
  Let \(X\) be a set, and let \(d : X \times X \to [0, \infty)\) be a function.
  \begin{enumerate}
    \item Give an example of a pair \((X, d)\) which obeys axioms (bcd) of \cref{1.1.2}, but not (a).
    \item Give an example of a pair \((X, d)\) which obeys axioms (acd) of \cref{1.1.2}, but not (b).
    \item Give an example of a pair \((X, d)\) which obeys axioms (abd) of \cref{1.1.2}, but not (c).
    \item Give an example of a pair \((X, d)\) which obeys axioms (abc) of \cref{1.1.2}, but not (d).
  \end{enumerate}
\end{ex}

\begin{proof}{(a)}
  Let \(X = \R\), and let \(d(x, y) = 1\) for all \(x, y \in \R\).
  Then \((X, d)\) does not satisfy \cref{1.1.2}(a) but (bcd).
\end{proof}

\begin{proof}{(b)}
  Let \(X = \R\) and let \(d(x, y) = 0\) for all \(x, y \in \R\).
  Then \((X, d)\) does not satisfy \cref{1.1.2}(b) but (acd).
\end{proof}

\begin{proof}{(c)}
  Let \(X = \{1, 2\}\), let \(x, y \in X\) and let \(d(x, y) = x^y\) if \(x \neq y\) and \(d(x, y) = 0\) if \(x = y\).
  Then \((X, d)\) does not satisfy \cref{1.1.2}(c) but (abd).
\end{proof}

\begin{proof}{(d)}
  Let \(X = \R^+\), let \(x, y \in \R^+\) and let \(d(x, y) = \max(x, y)\) if \(x \neq y\) and \(d(x, y) = 0\) if \(x = y\).
  Then \((X, d)\) does not satisfy \cref{1.1.2}(d) but (abc).
\end{proof}

\begin{ex}\label{ex:1.1.4}
  Show that the pair \((Y, d|_{Y \times Y})\) defined in \cref{1.1.5} is indeed a metric space.
\end{ex}

\begin{proof}
  Let \(x, y, z \in X\).
  Since \(Y \subseteq X\), we know that \(x, y, z \in X\).
  For identity:
  We have \(d|_{Y \times Y}(x, x) = d(x, x) = 0\).
  For positivity:
  If \(x \neq y\), then \(d|_{Y \times Y}(x, y) = d(x, y) > 0\).
  For symmetry:
  We have \(d|_{Y \times Y}(x, y) = d(x, y) = d(y, x) = d|_{Y \times Y}(y, x)\).
  For triangle inequality:
  We have \(d|_{Y \times Y}(x, z) = d(x, z) \leq d(x, y) + d(y, z) = d|_{Y \times Y}(x, y) + d|_{Y \times Y}(y, z)\).
  Thus by \cref{1.1.2} \((Y, d|_{Y \times Y})\) is a metric space.
\end{proof}

\begin{ex}\label{ex:1.1.5}
  Let \(n \geq 1\), and let \(a_1, a_2, \dots, a_n\) and \(b_1, b_2, \dots, b_n\) be real numbers.
  Verify the identity
  \[
    \bigg(\sum_{i = 1}^n a_i b_i\bigg)^2 + \dfrac{1}{2} \sum_{i = 1}^n \sum_{j = 1}^n (a_i b_j - a_j b_i)^2 = \bigg(\sum_{i = 1}^n a_i^2\bigg) \bigg(\sum_{j = 1}^n b_j^2\bigg)
  \]
  and conclude the \emph{Cauchy-Schwarz inequality}
  \[
    \abs{\sum_{i = 1}^n a_i b_i} \leq \bigg(\sum_{i = 1}^n a_i^2\bigg)^{1 / 2} \bigg(\sum_{j = 1}^n b_j^2\bigg)^{1 / 2}.
  \]
  Then use the Cauchy-Schwarz inequality to prove the \emph{triangle inequality}
  \[
    \bigg(\sum_{i = 1}^n (a_i + b_i)^2\bigg)^{1 / 2} \leq \bigg(\sum_{i = 1}^n a_i^2\bigg)^{1 / 2} + \bigg(\sum_{j = 1}^n b_j^2\bigg)^{1 / 2}.
  \]
\end{ex}

\begin{proof}
  We first show the identity is true by induction on \(n\).
  For \(n = 0\), we have
  \[
    \bigg(\sum_{i = 1}^0 a_i b_i\bigg)^2 + \dfrac{1}{2} \sum_{i = 1}^0 \sum_{j = 1}^0 (a_i b_j - a_j b_i)^2 = 0
  \]
  and
  \[
    \bigg(\sum_{i = 1}^0 a_i^2\bigg) \bigg(\sum_{j = 1}^0 b_j^2\bigg) = 0
  \]
  so the base case holds.
  Suppose inductively that the identity is true for some \(n \geq 0\).
  Then for \(n + 1\), we have
  \begin{align*}
      & \bigg(\sum_{i = 1}^{n + 1} a_i b_i\bigg)^2 + \dfrac{1}{2} \sum_{i = 1}^{n + 1} \sum_{j = 1}^{n + 1} (a_i b_j - a_j b_i)^2                                                                                  \\
    = & \bigg(\sum_{i = 1}^n a_i b_i + a_{n + 1} b_{n + 1}\bigg)^2 + \dfrac{1}{2} \sum_{i = 1}^{n + 1} \sum_{j = 1}^{n + 1} (a_i b_j - a_j b_i)^2                                                                  \\
    = & \bigg(\sum_{i = 1}^n a_i b_i\bigg)^2 + 2 \bigg(\sum_{i = 1}^n a_i b_i\bigg) (a_{n + 1} b_{n + 1}) + (a_{n + 1} b_{n + 1})^2 + \dfrac{1}{2} \sum_{i = 1}^{n + 1} \sum_{j = 1}^{n + 1} (a_i b_j - a_j b_i)^2 \\
    = & \bigg(\sum_{i = 1}^n a_i b_i\bigg)^2 + 2 \bigg(\sum_{i = 1}^n a_i b_i\bigg) (a_{n + 1} b_{n + 1}) + (a_{n + 1} b_{n + 1})^2                                                                                \\
      & + \dfrac{1}{2} \sum_{i = 1}^{n + 1} \bigg(\sum_{j = 1}^n (a_i b_j - a_j b_i)^2 + (a_i b_{n + 1} - a_{n + 1} b_i)^2\bigg)                                                                                   \\
    = & \bigg(\sum_{i = 1}^n a_i b_i\bigg)^2 + 2 \bigg(\sum_{i = 1}^n a_i b_i\bigg) (a_{n + 1} b_{n + 1}) + (a_{n + 1} b_{n + 1})^2                                                                                \\
      & + \dfrac{1}{2} \sum_{i = 1}^{n + 1} \sum_{j = 1}^n (a_i b_j - a_j b_i)^2 + \dfrac{1}{2} \sum_{i = 1}^{n + 1} (a_i b_{n + 1} - a_{n + 1} b_i)^2                                                             \\
    = & \bigg(\sum_{i = 1}^n a_i b_i\bigg)^2 + 2 \bigg(\sum_{i = 1}^n a_i b_i\bigg) (a_{n + 1} b_{n + 1}) + (a_{n + 1} b_{n + 1})^2                                                                                \\
      & + \dfrac{1}{2} \sum_{i = 1}^n \sum_{j = 1}^n (a_i b_j - a_j b_i)^2 + \dfrac{1}{2} \sum_{j = 1}^n (a_{n + 1} b_j - a_j b_{n + 1})^2 + \dfrac{1}{2} \sum_{i = 1}^{n + 1} (a_i b_{n + 1} - a_{n + 1} b_i)^2   \\
    = & \bigg(\sum_{i = 1}^n a_i b_i\bigg)^2 + 2 \bigg(\sum_{i = 1}^n a_i b_i\bigg) (a_{n + 1} b_{n + 1}) + (a_{n + 1} b_{n + 1})^2                                                                                \\
      & + \dfrac{1}{2} \sum_{i = 1}^n \sum_{j = 1}^n (a_i b_j - a_j b_i)^2 + \dfrac{1}{2} \sum_{j = 1}^n (a_{n + 1} b_j - a_j b_{n + 1})^2 + \dfrac{1}{2} \sum_{i = 1}^n (a_i b_{n + 1} - a_{n + 1} b_i)^2         \\
    = & \bigg(\sum_{i = 1}^n a_i b_i\bigg)^2 + 2 \bigg(\sum_{i = 1}^n a_i b_i\bigg) (a_{n + 1} b_{n + 1}) + (a_{n + 1} b_{n + 1})^2                                                                                \\
      & + \dfrac{1}{2} \sum_{i = 1}^n \sum_{j = 1}^n (a_i b_j - a_j b_i)^2 + \sum_{i = 1}^n (a_{n + 1} b_i - a_i b_{n + 1})^2
  \end{align*}
  and
  \begin{align*}
      & \bigg(\sum_{i = 1}^{n + 1} a_i^2\bigg) \bigg(\sum_{j = 1}^{n + 1} b_j^2\bigg)                                                                                                             \\
    = & \bigg(\sum_{i = 1}^n a_i^2 + a_{n + 1}^2\bigg) \bigg(\sum_{j = 1}^n b_j^2 + b_{n + 1}^2\bigg)                                                                                             \\
    = & \bigg(\sum_{i = 1}^n a_i^2\bigg) \bigg(\sum_{j = 1}^n b_j^2\bigg) + a_{n + 1}^2 \bigg(\sum_{j = 1}^n b_j^2\bigg) + b_{n + 1}^2 \bigg(\sum_{i = 1}^n a_i^2\bigg) + (a_{n + 1} b_{n + 1})^2 \\
    = & \bigg(\sum_{i = 1}^n a_i^2\bigg) \bigg(\sum_{j = 1}^n b_j^2\bigg) + \bigg(\sum_{j = 1}^n a_{n + 1}^2 b_j^2\bigg) + \bigg(\sum_{i = 1}^n a_i^2 b_{n + 1}^2\bigg) + (a_{n + 1} b_{n + 1})^2 \\
    = & \bigg(\sum_{i = 1}^n a_i^2\bigg) \bigg(\sum_{j = 1}^n b_j^2\bigg) + \bigg(\sum_{j = 1}^n a_{n + 1}^2 b_j^2 - 2 a_{n + 1} b_j a_j b_{n + 1} + a_j^2 b_{n + 1}^2\bigg)                      \\
      & + \sum_{j = 1}^n 2 a_{n + 1} b_j a_j b_{n + 1} - \sum_{j = 1}^n a_j^2 b_{n + 1}^2 + \bigg(\sum_{i = 1}^n a_i^2 b_{n + 1}^2\bigg) + (a_{n + 1} b_{n + 1})^2                                \\
    = & \bigg(\sum_{i = 1}^n a_i^2\bigg) \bigg(\sum_{j = 1}^n b_j^2\bigg) + \bigg(\sum_{j = 1}^n (a_{n + 1} b_j - a_j b_{n + 1})^2\bigg)                                                          \\
      & + 2 \sum_{j = 1}^n \bigg(b_j a_j\bigg)(a_{n + 1} b_{n + 1}) + (a_{n + 1} b_{n + 1})^2.
  \end{align*}
  By induction hypothesis we thus have
  \[
    \bigg(\sum_{i = 1}^{n + 1} a_i b_i\bigg)^2 + \dfrac{1}{2} \sum_{i = 1}^{n + 1} \sum_{j = 1}^{n + 1} (a_i b_j - a_j b_i)^2 = \bigg(\sum_{i = 1}^{n + 1} a_i^2\bigg) \bigg(\sum_{j = 1}^{n + 1} b_j^2\bigg)
  \]
  and this closes the induction.

  Next we show that Cauchy-Schwarz inequality is true.
  We have
  \begin{align*}
             & \bigg(\sum_{i = 1}^n a_i b_i\bigg)^2 + \dfrac{1}{2} \sum_{i = 1}^n \sum_{j = 1}^n (a_i b_j - a_j b_i)^2 = \bigg(\sum_{i = 1}^n a_i^2\bigg) \bigg(\sum_{j = 1}^n b_j^2\bigg) \\
    \implies & \bigg(\sum_{i = 1}^n a_i b_i\bigg)^2 \leq \bigg(\sum_{i = 1}^n a_i^2\bigg) \bigg(\sum_{j = 1}^n b_j^2\bigg)                                                                 \\
    \implies & \abs{\sum_{i = 1}^n a_i b_i} \leq \bigg(\sum_{i = 1}^n a_i^2\bigg)^{1 / 2} \bigg(\sum_{j = 1}^n b_j^2\bigg)^{1 / 2}.
  \end{align*}

  Finally we show that \(d_{l^2}\) satisfy triangle inequality.
  We have
  \begin{align*}
    \sum_{i = 1}^n (a_i + b_i)^2 & = \sum_{i = 1}^n (a_i^2 + 2 a_i b_i + b_i^2)                                                                                           \\
                                 & = \sum_{i = 1}^n a_i^2 + 2 \sum_{i = 1}^n a_i b_i + \sum_{i = 1}^n b_i^2                                                               \\
                                 & \leq \sum_{i = 1}^n a_i^2 + 2 \abs{\sum_{i = 1}^n a_i b_i} + \sum_{i = 1}^n b_i^2                                                      \\
                                 & \leq \sum_{i = 1}^n a_i^2 + 2 \bigg(\sum_{i = 1}^n a_i^2\bigg)^{1 / 2} \bigg(\sum_{j = 1}^n b_j^2\bigg)^{1 / 2} + \sum_{i = 1}^n b_i^2 \\
                                 & = \bigg(\bigg(\sum_{i = 1}^n a_i^2\bigg)^{1 / 2} + \bigg(\sum_{j = 1}^n b_j^2\bigg)^{1 / 2}\bigg)^2
  \end{align*}
  and thus
  \[
    \bigg(\sum_{i = 1}^n (a_i + b_i)^2\bigg)^{1 / 2} \leq \bigg(\sum_{i = 1}^n a_i^2\bigg)^{1 / 2} + \bigg(\sum_{j = 1}^n b_j^2\bigg)^{1 / 2}.
  \]
\end{proof}

\begin{ex}\label{ex:1.1.6}
  Show that \((\R^n, d_{l^2})\) in \cref{1.1.6} is indeed a metric space.
\end{ex}

\begin{proof}
  Let \(n \in \N\) and let \(x, y, z \in \R^n\).
  For each \(n \in \N\), we define \(I_n = \{i \in \N : 1 \leq i \leq n\}\).
  For each \(i \in I_n\), we define \(x_i, y_i, z_i\) to be the \(i^{\text{th}}\) coordinate of \(x, y, z\), respectively.
  For identity:
  We have
  \[
    d_{l^2}(x, x) = \bigg(\sum_{i = 1}^n (x_i - x_i)^2\bigg)^{1 / 2} = 0.
  \]
  For positivity:
  If \(x \neq y\), then \(\exists\ j \in I_n\) such that \(x_j \neq y_j\).
  So
  \[
    d_{l^2}(x, y) = \bigg(\sum_{i = 1}^n (x_i - y_i)^2\bigg)^{1 / 2} \geq \big((x_j - y_j)^2\big)^{1 / 2} > 0.
  \]
  For symmetry:
  We have
  \[
    d_{l^2}(x, y) = \bigg(\sum_{i = 1}^n (x_i - y_i)^2\bigg)^{1 / 2} = \bigg(\sum_{i = 1}^n (y_i - x_i)^2\bigg)^{1 / 2} = d_{l^2}(y, x).
  \]
  For triangle inequality:
  We define \(a_i = x_i - y_i\) and \(b_i = y_i - z_i\) for each \(i \in I_n\).
  Then we have
  \begin{align*}
    d_{l^2}(x, z) & = \bigg(\sum_{i = 1}^n (x_i - z_i)^2\bigg)^{1 / 2}                                                    \\
                  & = \bigg(\sum_{i = 1}^n (x_i - y_i + y_i - z_i)^2\bigg)^{1 / 2}                                        \\
                  & = \bigg(\sum_{i = 1}^n (a_i + b_i)^2\bigg)^{1 / 2}                                                    \\
                  & \leq \bigg(\sum_{i = 1}^n a_i^2\bigg)^{1 / 2} + \bigg(\sum_{i = 1}^n b_i^2\bigg)^{1 / 2}              \\
                  & = \bigg(\sum_{i = 1}^n (x_i - y_i)^2\bigg)^{1 / 2} + \bigg(\sum_{i = 1}^n (y_i - z_i)^2\bigg)^{1 / 2} \\
                  & = d_{l^2}(x, y) + d_{l^2}(y, z).
  \end{align*}
  Thus by \cref{1.1.2} \((\R^n, d_{l^2})\) is a metric space.
\end{proof}

\begin{ex}\label{ex:1.1.7}
  Show that \((\R^n, d_{l^1})\) in \cref{1.1.7} is indeed a metric space.
\end{ex}

\begin{proof}
  Let \(n \in \N\) and let \(x, y, z \in \R^n\).
  For each \(n \in \N\), we define \(I_n = \{i \in \N : 1 \leq i \leq n\}\).
  For each \(i \in I_n\), we define \(x_i, y_i, z_i\) to be the \(i^{\text{th}}\) coordinate of \(x, y, z\), respectively.
  For identity:
  We have
  \[
    d_{l^1}(x, x) = \sum_{i = 1}^n \abs{x_i - x_i} = 0.
  \]
  For positivity:
  If \(x \neq y\), then \(\exists\ j \in I_n\) such that \(x_j \neq y_j\).
  So
  \[
    d_{l^1}(x, y) = \sum_{i = 1}^n \abs{x_i - y_i} \geq \abs{x_j - y_j} > 0.
  \]
  For symmetry:
  We have
  \[
    d_{l^1}(x, y) = \sum_{i = 1}^n \abs{x_i - y_i} = \sum_{i = 1}^n \abs{y_i - x_i} = d_{l^1}(y, x).
  \]
  For triangle inequality:
  We have
  \begin{align*}
    d_{l^1}(x, z) & = \sum_{i = 1}^n \abs{x_i - z_i}                        \\
                  & = \sum_{i = 1}^n \abs{x_i - y_i + y_i - z_i}            \\
                  & \leq \sum_{i = 1}^n (\abs{x_i - y_i} + \abs{y_i - z_i}) \\
                  & = d_{l^1}(x, y) + d_{l^1}(y, z).
  \end{align*}
  Thus by \cref{1.1.2} \((\R^n, d_{l^1})\) is a metric space.
\end{proof}

\begin{ex}\label{ex:1.1.8}
  Prove the two inequalities
  \[
    d_{l^2}(x, y) \leq d_{l^1}(x, y) \leq \sqrt{n} d_{l^2}(x, y)
  \]
  for all \(x, y \in \R^n\).
\end{ex}

\begin{proof}
  Let \(n \in \N\) and let \(x, y \in \R^n\).
  For each \(n \in \N\), we define \(I_n = \{i \in \N : 1 \leq i \leq n\}\).
  For each \(i \in I_n\), we define \(x_i, y_i\) to be the \(i^{\text{th}}\) coordinate of \(x, y\), respectively.
  We have
  \begin{align*}
    \big(d_{l^2}(x, y)\big)^2 & = \sum_{i = 1}^n (x_i - y_i)^2                                                                                                                                           \\
                              & = \sum_{i = 1}^n \abs{x_i - y_i}^2                                                                                                                                       \\
                              & \leq \sum_{i = 1}^n \abs{x_i - y_i}^2 + \sum_{i = 1}^n \Bigg(\abs{x_i - y_i} \bigg(\sum_{j = 1}^{i - 1} \abs{x_j - y_j} + \sum_{j = i + 1}^n \abs{x_j - y_j}\bigg)\Bigg) \\
                              & = \sum_{i = 1}^n \Bigg(\abs{x_i - y_i} \bigg(\sum_{j = 1}^{i - 1} \abs{x_j - y_j} + \abs{x_i - y_i} + \sum_{j = i + 1}^n \abs{x_j - y_j}\bigg)\Bigg)                     \\
                              & = \bigg(\sum_{i = 1}^n \abs{x_i - y_i}\bigg) \bigg(\sum_{j = 1}^n \abs{x_j - y_j}\bigg)                                                                                  \\
                              & = \bigg(\sum_{i = 1}^n \abs{x_i - y_i}\bigg)^2                                                                                                                           \\
                              & = \big(d_{l^1}(x, y)\big)^2
  \end{align*}
  and thus \(d_{l^2}(x, y) \leq d_{l^1}(x, y)\).
  Now let \(a_i = \abs{x_i - y_i}\) and \(b_i = 1\) for each \(i \in I_n\).
  Then we have
  \begin{align*}
    d_{l^1}(x, y) & = \sum_{i = 1}^n \abs{x_i - y_i}                                                                                             \\
                  & = \abs{\sum_{i = 1}^n \abs{x_i - y_i}}                                                                                       \\
                  & = \abs{\sum_{i = 1}^n a_i b_i}                                                                                               \\
                  & \leq \bigg(\sum_{i = 1}^n a_i^2\bigg)^{1 / 2} \bigg(\sum_{i = 1}^n b_i^2\bigg)^{1 / 2}      &  & \text{(by \cref{ex:1.1.5})} \\
                  & = \bigg(\sum_{i = 1}^n \abs{x_i - y_i}^2\bigg)^{1 / 2} \bigg(\sum_{i = 1}^n 1\bigg)^{1 / 2}                                  \\
                  & = \sqrt{n} d_{l^2}(x, y).
  \end{align*}
  Combining the results we have
  \[
    d_{l^2}(x, y) \leq d_{l^1}(x, y) \leq \sqrt{n} d_{l^2}(x, y).
  \]
\end{proof}

\begin{ex}\label{ex:1.1.9}
  Show that \((\R^n, d_{l^\infty})\) in \cref{1.1.9} is indeed a metric space.
\end{ex}

\begin{proof}
  Let \(n \in \N\) and let \(x, y, z \in \R^n\).
  For each \(n \in \N\), we define \(I_n = \{i \in \N : 1 \leq i \leq n\}\).
  For each \(i \in I_n\), we define \(x_i, y_i, z_i\) to be the \(i^{\text{th}}\) coordinate of \(x, y, z\), respectively.
  For identity:
  We have
  \[
    d_{l^\infty}(x, x) = \sup \{\abs{x_i - x_i} : i \in I_n\} = 0.
  \]
  For positivity:
  If \(x \neq y\), then \(\exists\ j \in \N\) and \(1 \leq j \leq n\) such that \(x_j \neq y_j\).
  So
  \[
    d_{l^\infty}(x, y) = \sup \{\abs{x_i - y_i} : i \in I_n\} \geq \abs{x_j - y_j} > 0.
  \]
  For symmetry:
  We have
  \[
    d_{l^\infty}(x, y) = \sup \{\abs{x_i - y_i} : i \in I_n\} = \sup \{\abs{y_i - x_i} : i \in I_n\} = d_{l^\infty}(y, x).
  \]
  For triangle inequality:
  We have
  \begin{align*}
    d_{l^\infty}(x, z) & = \sup \{\abs{x_i - z_i} : i \in I_n\}                                           \\
                       & = \sup \{\abs{x_i - y_i + y_i - z_i} : i \in I_n\}                               \\
                       & \leq \sup \{\abs{x_i - y_i} + \abs{y_i - z_i} : i \in I_n\}                      \\
                       & \leq \sup \{\abs{x_i - y_i} : i \in I_n\} + \sup \{\abs{y_i - z_i} : i \in I_n\} \\
                       & = d_{l^\infty}(x, y) + d_{l^\infty}(y, z).
  \end{align*}
  Thus by \cref{1.1.2} \((\R^n, d_{l^\infty})\) is a metric space.
\end{proof}

\begin{ex}\label{ex:1.1.10}
  Prove the two inequalities
  \[
    \dfrac{1}{\sqrt{n}} d_{l^2}(x, y) \leq d_{l^\infty}(x, y) \leq d_{l^2}(x, y)
  \]
  for all \(x, y \in \R^n\).
\end{ex}

\begin{proof}
  Let \(n \in \N\) and let \(x, y \in \R^n\).
  For each \(n \in \N\), we define \(I_n = \{i \in \N : 1 \leq i \leq n\}\).
  For each \(i \in I_n\), we define \(x_i, y_i\) to be the \(i^{\text{th}}\) coordinate of \(x, y\), respectively.
  Since
  \begin{align*}
    \dfrac{1}{\sqrt{n}} d_{l^2}(x, y) & = \dfrac{1}{\sqrt{n}} \bigg(\sum_{i = 1}^n (x_i - y_i)^2\bigg)^{1 / 2}                                       \\
                                      & \leq \dfrac{1}{\sqrt{n}} \bigg(\sum_{i = 1}^n \big(\sup \{\abs{x_i - y_i} : i \in I_n\}\big)^2\bigg)^{1 / 2} \\
                                      & = \dfrac{1}{\sqrt{n}} \Big(n \big(\sup \{\abs{x_i - y_i} : i \in I_n\}\big)^2\Big)^{1 / 2}                   \\
                                      & = \sup \{\abs{x_i - y_i} : i \in I_n\}                                                                       \\
                                      & = d_{l^\infty}(x, y)                                                                                         \\
                                      & = \Big(\big(\sup \{\abs{x_i - y_i} : i \in I_n\}\big)^2\Big)^{1 / 2}                                         \\
                                      & \leq \bigg(\sum_{i = 1}^n (x_i - y_i)^2\bigg)^{1 / 2}                                                        \\
                                      & = d_{l^2}(x, y),
  \end{align*}
  we have
  \[
    \dfrac{1}{\sqrt{n}} d_{l^2}(x, y) \leq d_{l^\infty}(x, y) \leq d_{l^2}(x, y).
  \]
\end{proof}

\begin{ex}\label{ex:1.1.11}
  Show that \((X, d_{\text{disc}})\) in \cref{1.1.11} is indeed a metric space.
\end{ex}

\begin{proof}
  Let \(x, y, z \in X\).
  For identity:
  We have
  \[
    d_{\text{disc}}(x, x) = 0.
  \]
  For positivity:
  If \(x \neq y\), then we have
  \[
    d_{\text{disc}}(x, y) = 1 > 0.
  \]
  For symmetry:
  We have
  \[
    x = y \iff d_{\text{disc}}(x, y) = 0 = d_{\text{disc}}(y, x)
  \]
  and
  \[
    x \neq y \iff d_{\text{disc}}(x, y) = 1 = d_{\text{disc}}(y, x).
  \]
  For triangle inequality:
  We have
  \[
    x = z \iff d_{\text{disc}}(x, z) = 0 \leq d_{\text{disc}}(x, y) + d_{\text{disc}}(y, z)
  \]
  and
  \[
    x \neq z \iff d_{\text{disc}}(x, z) = 1 \leq d_{\text{disc}}(x, y) + d_{\text{disc}}(y, z).
  \]
  Thus by \cref{1.1.2} \((X, d_{\text{disc}})\) is a metric space.
\end{proof}

\begin{ex}\label{ex:1.1.12}
  Prove \cref{1.1.18}.
\end{ex}

\begin{proof}
  See \cref{1.1.18}.
\end{proof}

\begin{ex}\label{ex:1.1.13}
  Prove \cref{1.1.19}.
\end{ex}

\begin{proof}
  See \cref{1.1.19}.
\end{proof}

\begin{ex}\label{ex:1.1.14}
  Prove \cref{1.1.20}.
\end{ex}

\begin{proof}
  See \cref{1.1.20}.
\end{proof}

\begin{ex}\label{ex:1.1.15}
  Let
  \[
    X \coloneqq \bigg\{(a_n)_{n = 0}^\infty : \sum_{n = 0}^\infty \abs{a_n} < \infty\bigg\}
  \]
  be the space of absolutely convergent sequences. Define the \(l^1\) and \(l^\infty\) metrics
  on this space by
  \begin{align*}
    d_{l^1}\big((a_n)_{n = 0}^\infty, (b_n)_{n = 0}^\infty\big)      & \coloneqq \sum_{n = 0}^\infty \abs{a_n - b_n}; \\
    d_{l^\infty}\big((a_n)_{n = 0}^\infty, (b_n)_{n = 0}^\infty\big) & \coloneqq \sup_{n \in \N} \abs{a_n - b_n}.
  \end{align*}
  Show that these are both metrics on \(X\), but show that there exist sequences \(x^{(1)}, x^{(2)}, \dots\) of elements of \(X\) (i.e., sequences of sequences) which are convergent with respect to the \(d_{l^\infty}\) metric but not with respect to the \(d_{l^1}\) metric.
  Conversely, show that any sequence which converges in the \(d_{l^1}\) metric automatically converges in the \(d_{l^\infty}\) metric.
\end{ex}

\begin{proof}
  Let \((a_n)_{n = 0}^\infty, (b_n)_{n = 0}^\infty, (c_n)_{n = 0}^\infty \in X\).
  Since \(\sum_{n = 0}^\infty a_n\) and \(\sum_{n = 0}^\infty b_n\) converge absolutely, we know that
  \begin{align*}
    \sum_{n = 0}^\infty \abs{a_n} + \sum_{n = 0}^\infty \abs{b_n} & = \lim_{N \to \infty} \sum_{n = 0}^N \abs{a_n} + \lim_{N \to \infty} \sum_{n = 0}^N \abs{b_n} \\
                                                                  & = \lim_{N \to \infty} \sum_{n = 0}^N \big(\abs{a_n} + \abs{b_n}\big)                          \\
                                                                  & \geq \lim_{N \to \infty} \sum_{n = 0}^N \abs{a_n - b_n}                                       \\
                                                                  & = \sum_{n = 0}^\infty \abs{a_n - b_n}                                                         \\
                                                                  & \geq \sup_{n \in N} \abs{a_n - b_n}.
  \end{align*}
  Thus both \(\sum_{n = 0}^\infty \abs{a_n - b_n}\) and \(\sup_{n \in N} \abs{a_n - b_n}\) are well-defined and finite.

  We first show that \((X, d_{l^1})\) and \((X, d_{l^\infty})\) are metric spaces.
  For identity:
  We have
  \[
    d_{l^1}\big((a_n)_{n = 0}^\infty, (a_n)_{n = 0}^\infty\big) = \sum_{n = 0}^\infty \abs{a_n - a_n} = 0
  \]
  and
  \[
    d_{l^1}\big((a_n)_{n = 0}^\infty, (a_n)_{n = 0}^\infty\big) = \sup_{n \in \N} \abs{a_n - a_n} = 0.
  \]
  For positivity:
  If \((a_n)_{n = 0}^\infty \neq (b_n)_{n = 0}^\infty\), then \(\exists\ N \in \N\) such that \(a_N \neq b_N\).
  So we have
  \[
    d_{l^1}\big((a_n)_{n = 0}^\infty, (b_n)_{n = 0}^\infty\big) = \sum_{n = 0}^\infty \abs{a_n - b_n} \geq \abs{a_N - b_N} > 0
  \]
  and
  \[
    d_{l^\infty}\big((a_n)_{n = 0}^\infty, (b_n)_{n = 0}^\infty\big) = \sup_{n \in \N} \abs{a_n - b_n} \geq \abs{a_N - b_N} > 0.
  \]
  For symmetry:
  We have
  \[
    d_{l^1}\big((a_n)_{n = 0}^\infty, (b_n)_{n = 0}^\infty\big) = \sum_{n = 0}^\infty \abs{a_n - b_n} = \sum_{n = 0}^\infty \abs{b_n - a_n} = d_{l^1}\big((b_n)_{n = 0}^\infty, (a_n)_{n = 0}^\infty\big)
  \]
  and
  \[
    d_{l^\infty}\big((a_n)_{n = 0}^\infty, (b_n)_{n = 0}^\infty\big) = \sup_{n \in \N} \abs{a_n - b_n} = \sup_{n \in \N} \abs{b_n - a_n} = d_{l^\infty}\big((b_n)_{n = 0}^\infty, (a_n)_{n = 0}^\infty\big).
  \]
  For triangle inequality:
  We have
  \begin{align*}
    d_{l^1}\big((a_n)_{n = 0}^\infty, (c_n)_{n = 0}^\infty\big) & = \sum_{n = 0}^\infty \abs{a_n - c_n}                                                                                       \\
                                                                & = \sum_{n = 0}^\infty \abs{a_n - b_n + b_n - c_n}                                                                           \\
                                                                & \leq \sum_{n = 0}^\infty \abs{a_n - b_n} + \sum_{n = 0}^\infty \abs{b_n - c_n}                                              \\
                                                                & = d_{l^1}\big((a_n)_{n = 0}^\infty, (b_n)_{n = 0}^\infty\big) + d_{l^1}\big((b_n)_{n = 0}^\infty, (c_n)_{n = 0}^\infty\big)
  \end{align*}
  and
  \begin{align*}
    d_{l^\infty}\big((a_n)_{n = 0}^\infty, (c_n)_{n = 0}^\infty\big) & = \sup_{n \in \N} \abs{a_n - c_n}                                                                                                      \\
                                                                     & = \sup_{n \in \N} \abs{a_n - b_n + b_n - c_n}                                                                                          \\
                                                                     & \leq \sup_{n \in \N} (\abs{a_n - b_n} + \abs{b_n - c_n})                                                                               \\
                                                                     & \leq \sup_{n \in \N} \abs{a_n - b_n} + \sup_{n \in \N} \abs{b_n - c_n}                                                                 \\
                                                                     & = d_{l^\infty}\big((a_n)_{n = 0}^\infty, (b_n)_{n = 0}^\infty\big) + d_{l^\infty}\big((b_n)_{n = 0}^\infty, (c_n)_{n = 0}^\infty\big).
  \end{align*}
  Thus by \cref{1.1.2} \((X, d_{l^1})\) and \((X, d_{l^\infty})\) are metric spaces.

  Next we show that there exist sequences of elements of \(X\) which are convergent with respect to the \(d_{l^\infty}\) metric but not with respect to the \(d_{l^1}\) metric.
  Let \((x^{(k)})_{k = 1}^\infty\) be the sequence of sequence \((x_n^{(k)})_{n = 0}^\infty\) where
  \[
    x_n^{(k)} = \begin{dcases}
      0                             & \text{if } n = 0,    \\
      \dfrac{1}{n^2} + \dfrac{1}{k} & \text{if } n \leq k, \\
      \dfrac{1}{n^2}                & \text{if } n > k.
    \end{dcases}
  \]
  Then we know that \(\sum_{n = 0}^\infty \abs{x_n^{(k)}}\) is absolutely convergent for all \(k \in \N\) and \(k \geq 1\).
  Let \((y_n)_{n = 0}^\infty\) be a sequence where
  \[
    y_n = \begin{dcases}
      0              & \text{if } n = 0, \\
      \dfrac{1}{n^2} & \text{if } n > 0.
    \end{dcases}
  \]
  Then \(\sum_{n = 0}^\infty \abs{y_n}\) is also absolutely convergent.
  Now we show that \((x^{(k)})_{k = 1}^\infty\) converges to \((y_n)_{n = 0}^\infty\) with respect to \(d_{l^\infty}\).
  By \cref{1.1.14} we need to show that
  \[
    \lim_{k \to \infty} d_{l^\infty}\big((x_n^{(k)})_{n = 0}^\infty, (y_n)_{n = 0}^\infty\big) = 0.
  \]
  We have
  \begin{align*}
    d_{l^\infty}\big((x_n^{(k)})_{n = 0}^\infty, (y_n)_{n = 0}^\infty\big) & = \sup_{n \in \N} \abs{x_n^{(k)} - y_n} \\
                                                                           & = \sup \bigg\{0, \dfrac{1}{k}\bigg\}    \\
                                                                           & = \dfrac{1}{k}
  \end{align*}
  and thus
  \[
    \lim_{k \to \infty} d_{l^\infty}\big((x_n^{(k)})_{n = 0}^\infty, (y_n)_{n = 0}^\infty\big) = \lim_{k \to \infty} \dfrac{1}{k} = 0.
  \]
  But we also have
  \begin{align*}
    d_{l^1}\big((x_n^{(k)})_{n = 0}^\infty, (y_n)_{n = 0}^\infty\big) & = \sum_{n = 0}^\infty \abs{x_n^{(k)} - y_n} \\
                                                                      & = \sum_{n = 1}^k \dfrac{1}{k}               \\
                                                                      & = 1
  \end{align*}
  and thus
  \[
    \lim_{k \to \infty} d_{l^1}\big((x_n^{(k)})_{n = 0}^\infty, (y_n)_{n = 0}^\infty\big) = \lim_{k \to \infty} 1 = 1 \neq 0.
  \]
  We conclude that \((x^{(k)})_{k = 1}^\infty\) converges to \((y_n)_{n = 0}^\infty\) with respect to \(d_{l^\infty}\) but not \(d_{l^1}\).

  Finally we show that any sequence which converges in the \(d_{l^1}\) metric automatically converges in the \(d_{l^\infty}\) metric.
  Suppose that \((x^{(k)})_{k = 1}^\infty\) converges to \((y_n)_{n = 0}^\infty\) with respect to \(d_{l^1}\).
  Since
  \begin{align*}
    d_{l^1}\big((x_n^{(k)})_{n = 0}^\infty, (y_n)_{n = 0}^\infty\big) & = \sum_{n = 0}^\infty \abs{x_n^{(k)} - y_n}                               \\
                                                                      & \geq \sup_{n \in N} \abs{x_n^{(k)} - y_n}                                 \\
                                                                      & = d_{l^\infty}\big((x_n^{(k)})_{n = 0}^\infty, (y_n)_{n = 0}^\infty\big),
  \end{align*}
  by squeeze test we have
  \begin{align*}
             & 0 \leq d_{l^\infty}\big((x_n^{(k)})_{n = 0}^\infty, (y_n)_{n = 0}^\infty\big) \leq d_{l^1}\big((x_n^{(k)})_{n = 0}^\infty, (y_n)_{n = 0}^\infty\big)                                                                     \\
    \implies & 0 = \lim_{k \to \infty} 0 \leq \lim_{k \to \infty} d_{l^\infty}\big((x_n^{(k)})_{n = 0}^\infty, (y_n)_{n = 0}^\infty\big) \leq \lim_{k \to \infty} d_{l^1}\big((x_n^{(k)})_{n = 0}^\infty, (y_n)_{n = 0}^\infty\big) = 0 \\
    \implies & \lim_{k \to \infty} d_{l^\infty}\big((x_n^{(k)})_{n = 0}^\infty, (y_n)_{n = 0}^\infty\big) = 0.
  \end{align*}
  Thus any sequence which converges in the \(d_{l^1}\) metric automatically converges in the \(d_{l^\infty}\) metric.
\end{proof}

\begin{ex}\label{ex:1.1.16}
  Let \((x_n)_{n = 1}^\infty\) and \((y_n)_{n = 1}^\infty\) be two sequences in a metric space \((X, d)\).
  Suppose that \((x_n)_{n = 1}^\infty\) converges to a point \(x \in X\), and \((y_n)_{n = 1}^\infty\) converges to a point \(y \in X\).
  Show that \(\lim_{n \to \infty} d(x_n, y_n) = d(x, y)\).
\end{ex}

\begin{proof}
  We have
  \begin{align*}
     & \lim_{n \to \infty} d(x_n, x) = 0;       &  & \text{(by \cref{1.1.14})} \\
     & \lim_{n \to \infty} d(y_n, y) = 0;       &  & \text{(by \cref{1.1.14})} \\
     & \lim_{n \to \infty} d(x, y)   = d(x, y).
  \end{align*}
  Since
  \begin{align*}
    d(x_n, y_n) & \leq d(x_n, x) + d(x, y_n)           &  & \text{(by \cref{1.1.2}(d))} \\
                & \leq d(x_n, x) + d(x, y) + d(y, y_n) &  & \text{(by \cref{1.1.2}(d))} \\
                & = d(x_n, x) + d(x, y) + d(y_n, y)    &  & \text{(by \cref{1.1.2}(c))}
  \end{align*}
  and
  \begin{align*}
    d(x, y) & \leq d(x, x_n) + d(x_n, y)               &  & \text{(by \cref{1.1.2}(d))} \\
            & \leq d(x, x_n) + d(x_n, y_n) + d(y_n, y) &  & \text{(by \cref{1.1.2}(d))} \\
            & = d(x_n, x) + d(x_n, y_n) + d(y_n, y),   &  & \text{(by \cref{1.1.2}(c))}
  \end{align*}
  we have
  \begin{align*}
             & \bigg(d(x_n, y_n) - d(x, y) \leq d(x_n, x) + d(y_n, y)\bigg)                                                                        \\
             & \land \bigg(d(x, y) - d(x_n, y_n) \leq d(x_n, x) + d(y_n, y)\bigg)                                                                  \\
    \implies & 0 \leq \abs{d(x_n, y_n) - d(x, y)} \leq d(x_n, x) + d(y_n, y)                                                                       \\
    \implies & \lim_{n \to \infty} 0 \leq \lim_{n \to \infty} \abs{d(x_n, y_n) - d(x, y)} \leq \lim_{n \to \infty} \big(d(x_n, x) + d(y_n, y)\big) \\
    \implies & 0 \leq \lim_{n \to \infty} \abs{d(x_n, y_n) - d(x, y)} \leq \lim_{n \to \infty} d(x_n, x) + \lim_{n \to \infty} d(y_n, y) = 0       \\
    \implies & \lim_{n \to \infty} \abs{d(x_n, y_n) - d(x, y)} = 0                                                                                 \\
    \implies & \lim_{n \to \infty} \big(d(x_n, y_n) - d(x, y)\big) = 0                                                                             \\
    \implies & \lim_{n \to \infty} d(x_n, y_n) - \lim_{n \to \infty} d(x, y) = 0                                                                   \\
    \implies & \lim_{n \to \infty} d(x_n, y_n) = \lim_{n \to \infty} d(x, y) = d(x, y).
  \end{align*}
\end{proof}
\section{Some point-set topology of metric spaces}\label{sec:1.2}

\begin{note}
  Having defined the operation of convergence on metric spaces, we now define a couple other related notions, including that of open set, closed set, interior, exterior, boundary, and adherent point.
  The study of such notions is known as \emph{point-set topology}.
\end{note}

\begin{defn}[Balls]\label{1.2.1}
  Let \((X, d)\) be a metric space, let \(x_0\) be a point in \(X\), and let \(r > 0\).
  We define the \emph{ball} \(B_{(X, d)}(x_0, r)\) in \(X\), centered at \(x_0\), and with radius \(r\), in the metric \(d\), to be the set
  \[
    B_{(X, d)}(x_0, r) \coloneqq \{x \in X : d(x, x_0) < r\}.
  \]
  When it is clear what the metric space \((X, d)\) is, we shall abbreviate \(B_{(X, d)}(x_0, r)\) as just \(B(x_0, r)\).
\end{defn}

\setcounter{thm}{3}
\begin{rmk}\label{1.2.4}
  Note that the smaller the radius \(r\), the smaller the ball \(B(x_0 , r)\).
  However, \(B(x_0 , r)\) always contains at least one point, namely the center \(x_0\), as long as \(r\) stays positive, thanks to \cref{1.1.2}(a).
  (We don't consider balls of zero radius or negative radius since they are rather boring, being just the empty set.)
\end{rmk}

\begin{defn}[Interior, exterior, boundary]\label{1.2.5}
  Let \((X, d)\) be a metric space, let \(E\) be a subset of \(X\), and let \(x_0\) be a point in \(X\).
  We say that \(x_0\) is an \emph{interior point of} \(E\) if there exists a radius \(r > 0\) such that \(B(x_0, r) \subseteq E\).
  We say that \(x_0\) is an \emph{exterior point of} \(E\) if there exists a radius \(r > 0\) such that \(B(x_0, r) \cap E = \emptyset\).
  We say that \(x_0\) is a \emph{boundary point of} \(E\) if it is neither an interior point nor an exterior point of \(E\).
\end{defn}

\begin{note}
  The set of all interior points of \(E\) is called the interior of \(E\) and is sometimes denoted \(\text{int}(E)\).
  The set of exterior points of \(E\) is called the exterior of \(E\) and is sometimes denoted \(\text{ext}(E)\).
  The set of boundary points of \(E\) is called the boundary of \(E\) and is sometimes denoted \(\partial E\).
\end{note}

\begin{note}
  We use the same notation of metric balls for the set of interior points \(\text{int}_{(X, d)}(E)\), the set of exterior points \(\text{ext}_{(X, d)}(E)\) and the set of boundary points \(\partial_{(X, d)}(E)\).
\end{note}

\begin{rmk}\label{1.2.6}
  If \(x_0\) is an interior point of \(E\), then \(x_0\) must actually be an element of \(E\), since balls \(B(x_0, r)\) always contain their center \(x_0\).
  Conversely, if \(x_0\) is an exterior point of \(E\), then \(x_0\) cannot be an element of \(E\).
  In particular it is not possible for \(x_0\) to simultaneously be an interior and an exterior point of \(E\).
  If \(x_0\) is a boundary point of \(E\), then it could be an element of \(E\), but it could also not lie in \(E\).
\end{rmk}

\begin{note}
  By \cref{1.2.6} we thus have
  \begin{align*}
     & \text{int}_{(X, d)}(E) \subseteq E             \\
     & \text{ext}_{(X, d)}(E) \subseteq X \setminus E
  \end{align*}
  for any metric space \((X, d)\) and any subset \(E\) of \(X\).
\end{note}

\setcounter{thm}{7}
\begin{eg}\label{1.2.8}
  When we give a set \(X\) the discrete metric \(d_{\text{disc}}\), and \(E\) is any subset of \(X\), then every element of \(E\) is an interior point of \(E\), every point not contained in \(E\) is an exterior point of \(E\), and there are no boundary points.
\end{eg}

\begin{proof}
  We have
  \begin{align*}
             & \forall x_0 \in E, d_{\text{disc}}(x_0, x_0) = 0                          &  & \by{1.1.11} \\
    \implies & \forall x_0 \in E, B_{(X, d_{\text{disc}})}(x_0, 1) = \{x_0\} \subseteq E &  & \by{1.2.1}  \\
    \implies & \forall x_0 \in E, x_0 \in \text{int}_{(X, d_{\text{disc}})}(E)           &  & \by{1.2.5}  \\
    \implies & E \subseteq \text{int}_{(X, d_{\text{disc}})}(E)                                           \\
    \implies & E = \text{int}_{(X, d_{\text{disc}})}(E)                                  &  & \by{1.2.6}
  \end{align*}
  and
  \begin{align*}
             & E = \text{int}_{(X, d_{\text{disc}})}(E)                                                                               \\
    \implies & \forall x_0 \in X \setminus E, x_0 \notin \text{int}_{(X, d_{\text{disc}})}(E)                                         \\
    \implies & \forall x_0 \in X \setminus E, \forall r \in \R^+, B_{(X, d_{\text{disc}})}(x_0, r) \not\subseteq E    &  & \by{1.2.5} \\
    \implies & \forall x_0 \in X \setminus E, \forall r \in \R^+, B_{(X, d_{\text{disc}})}(x_0, r) \cap E = \emptyset                 \\
    \implies & \forall x_0 \in X \setminus E, x_0 \in \text{ext}_{(X, d_{\text{disc}})}(E)                                            \\
    \implies & X \setminus E \subseteq \text{ext}_{(X, d_{\text{disc}})}(E)                                                           \\
    \implies & X \setminus E = \text{ext}_{(X, d_{\text{disc}})}(E).                                                  &  & \by{1.2.6}
  \end{align*}
  Since \(X = (X \setminus E) \cup E = \text{ext}(E) \cup \text{int}(E)\), every point in \((X, d_{\text{disc}})\) is either a exterior point or interior point, thus by \cref{1.2.5} there are no boundary points in \((X, d_{\text{disc}})\).
\end{proof}

\begin{defn}[Closure]\label{1.2.9}
  Let \((X, d)\) be a metric space, let \(E\) be a subset of \(X\), and let \(x_0\) be a point in \(X\).
  We say that \(x_0\) is an \emph{adherent point} of \(E\) if for every radius \(r > 0\), the ball \(B(x_0, r)\) has a non-empty intersection with \(E\).
  The set of all adherent points of \(E\) is called the \emph{closure} of \(E\) and is denoted \(\overline{E}\).
\end{defn}

\begin{note}
  Notions in \cref{1.2.9} are consistent with the corresponding notions on the real line.
\end{note}

\begin{note}
  Since the closure of a set \(E\) depends on metric \((X, d)\), we denote the closure of \(E\) with \(\overline{E}_{(X, d)}\).
\end{note}

\begin{prop}\label{1.2.10}
  Let \((X, d)\) be a metric space, let \(E\) be a subset of \(X\), and let \(x_0\) be a point in \(X\).
  Then the following statements are logically equivalent.
  \begin{enumerate}
    \item \(x_0\) is an adherent point of \(E\).
    \item \(x_0\) is either an interior point or a boundary point of \(E\).
    \item There exists a sequence \((x_n)_{n = 1}^\infty\) in \(E\) which converges to \(x_0\) with respect to the metric \(d\).
  \end{enumerate}
\end{prop}

\begin{proof}
  We first show that statement (a) implies statement (b).
  \begin{align*}
             & \forall x_0 \in \overline{E}_{(X, d)}, \forall r \in \R^+, B_{(X, d)}(x_0, r) \cap E \neq \emptyset       &  & \by{1.2.9} \\
    \implies & \forall x_0 \in \overline{E}_{(X, d)}, x_0 \notin \text{ext}_{(X, d)}(E)                                  &  & \by{1.2.5} \\
    \implies & \forall x_0 \in \overline{E}_{(X, d)}, x_0 \in \big(\text{int}_{(X, d)}(E) \cup \partial_{(X, d)}(E)\big) &  & \by{1.2.5} \\
    \implies & \overline{E}_{(X, d)} \subseteq \big(\text{int}_{(X, d)}(E) \cup \partial_{(X, d)}(E)\big)
  \end{align*}

  Next we show that statement (b) implies statement (c).
  \begin{itemize}
    \item Suppose that \(x_0 \in \text{int}_{(X, d)}(E)\).
          Then by setting \(x_n = x_0\) for all \(n \in \Z^+\) we have \(\lim_{n \to \infty} d(x_n, x_0) = \lim_{n \to \infty} 0 = 0\).
    \item Suppose that \(x_0 \in \partial_{(X, d)}(E)\).
          By \cref{1.2.5} for all \(r \in \R^+\), we have \(B_{(X, d)}(x_0, r) \cap E \neq \emptyset\).
          In particular, we have \(B_{(X, d)}(x_0, \dfrac{1}{n}) \cap E \neq \emptyset\) for all \(n \in \Z^+\).
          Let \(X_n = B_{(X, d)}(x_0, \dfrac{1}{n}) \cap E\).
          Since \(X_n \neq \emptyset\), by axiom of choice we can choose \((x_n)_{n = 1}^\infty \in \prod_{n \in \Z^+} X_n\).
          Thus \(d(x_n, x_0) < \dfrac{1}{n}\) for all \(n \in \Z^+\) and by squeeze test we have \(\lim_{n \to \infty} d(x_n, x_0) = 0\).
  \end{itemize}
  From all cases above we conclude that there exists a sequence \((x_n)_{n = 1}^\infty\) in \(E\) which converges to \(x_0\) with respect to \(d\).

  Finally we show that statement (c) implies statement (a).
  Suppose that there exists a sequence \((x_n)_{n = 1}^\infty\) in \(E\) which converges to \(x_0\) with respect to \(d\).
  By \cref{1.1.14} we have
  \[
    \forall r \in \R^+, \exists N \in \Z^+ : \forall n \in \N, n \geq N \implies d(x_n, x_0) \leq r.
  \]
  Since \(x_N \in E\) and \(E \subseteq X\), we know that the set \(B_{(X, d)}(x_0, r) \cap E \neq \emptyset\).
  Since \(r\) is arbitrary, by \cref{1.2.9} we know that \(x_0\) is an adherent point of \(E\).
\end{proof}

\begin{cor}\label{1.2.11}
  Let \((X, d)\) be a metric space, and let \(E\) be a subset of \(X\).
  Then \(\overline{E} = \text{int}(E) \cup \partial E = X \setminus \text{ext}(E)\).
\end{cor}

\begin{proof}
  By \cref{1.2.10}(a)(b) we are done.
\end{proof}

\begin{defn}[Open and closed sets]\label{1.2.12}
  Let \((X, d)\) be a metric space, and let \(E\) be a subset of \(X\).
  We say that \(E\) is \emph{closed} if it contains all of its boundary points, i.e., \(\partial E \subseteq E\).
  We say that \(E\) is \emph{open} if it contains none of its boundary points, i.e., \(\partial E \cap E = \emptyset\).
  If \(E\) contains some of its boundary points but not others, then it is neither open nor closed.
\end{defn}

\setcounter{thm}{13}
\begin{rmk}\label{1.2.14}
  It is possible for a set to be simultaneously open and closed, if it has no boundary.
  For instance, in a metric space \((X, d)\), the whole space \(X\) has no boundary (every point in \(X\) is an interior point), and so \(X\) is both open and closed.
  The empty set \(\emptyset\) also has no boundary (every point in \(X\) is an exterior point), and so is both open and closed.
  In many cases these are the only sets that are simultaneously open and closed, but there are exceptions.
  For instance, using the discrete metric \(d_{\text{disc}}\), \emph{every} set is both open and closed! (See \cref{1.2.8})
\end{rmk}

\begin{note}
  From \cref{1.2.14} we see that the notions of being open and being closed are \emph{not} negations of each other;
  there are sets that are both open and closed, and there are sets which are neither open nor closed.
  Thus, if one knew for instance that \(E\) was not an open set, it would be erroneous to conclude from this that \(E\) was a closed set, and similarly with the roles of open and closed reversed.
\end{note}

\begin{prop}[Basic properties of open and closed sets]\label{1.2.15}
  Let \((X, d)\) be a metric space.
  \begin{enumerate}
    \item Let \(E\) be a subset of \(X\).
          Then \(E\) is open iff \(E = \text{int}(E)\).
          In other words, \(E\) is open iff for every \(x \in E\), there exists an \(r > 0\) such that \(B(x, r) \subseteq E\).
    \item Let \(E\) be a subset of \(X\).
          Then \(E\) is closed iff \(E\) contains all its adherent points.
          In other words, \(E\) is closed iff for every convergent sequence \((x_n)_{n = m}^\infty\) in \(E\), the limit \(\lim_{n \to \infty} x_n\) of that sequence also lies in \(E\).
    \item For any \(x_0 \in X\) and \(r > 0\), then the ball \(B(x_0, r)\) is an open set.
          The set \(\{x \in X : d(x, x_0) \leq r\}\) is a closed set.
          (This set is sometimes called the \emph{closed ball} of radius \(r\) centered at \(x_0\).)
    \item Any singleton set \(\{x_0\}\), where \(x_0 \in X\), is automatically closed.
    \item If \(E\) is a subset of \(X\), then \(E\) is open iff the complement \(X \setminus E \coloneqq \{x \in X : x \notin E\}\) is closed.
    \item If \(E_1, \dots, E_n\) are a finite collection of open sets in \(X\), then \(E_1 \cap E_2 \cap \dots \cap E_n\) is also open.
          If \(F_1, \dots, F_n\) is a finite collection of closed sets in \(X\), then \(F_1 \cup F_2 \cup \dots \cup F_n\) is also closed.
    \item If \(\{E_\alpha\}_{\alpha \in I}\) is a collection of open sets in \(X\) (where the index set \(I\) could be finite, countable, or uncountable), then the union \(\bigcup_{\alpha \in I} E_\alpha \coloneqq \{x \in X : x \in E_\alpha \text{ for some } \alpha \in I\}\) is also open.
          If \(\{F_\alpha\}_{\alpha \in I}\) is a collection of closed sets in \(X\), then the intersection \(\bigcap_{\alpha \in I} F_\alpha \coloneqq \{x \in X : x \in F_\alpha \text{ for all } \alpha \in I\}\) is also closed.
    \item If \(E\) is any subset of \(X\), then \(\text{int}(E)\) is the largest open set which is contained in \(E\);
          in other words, \(\text{int}(E)\) is open, and given any other open set \(V \subseteq E\), we have \(V \subseteq \text{int}(E)\).
          Similarly \(\overline{E}\) is the smallest closed set which contains \(E\);
          in other words, \(\overline{E}\) is closed, and given any other closed set \(K \supseteq E\), \(K \supseteq \overline{E}\).
  \end{enumerate}
\end{prop}

\begin{proof}{(a)}
  \begin{align*}
         & E \text{ is open in } (X, d)                                                                                              \\
    \iff & \partial_{(X, d)}(E) \cap E = \emptyset                                                                  &  & \by{1.2.12} \\
    \iff & \forall x \in E, \big(x \notin \partial_{(X, d)}(E)\big) \land \big(x \notin \text{ext}_{(X, d)}(E)\big) &  & \by{1.2.6}  \\
    \iff & \forall x \in E, x \in \text{int}_{(X, d)}(E)                                                            &  & \by{1.2.5}  \\
    \iff & E \subseteq \text{int}_{(X, d)}(E)                                                                                        \\
    \iff & \text{int}(E) = E.                                                                                       &  & \by{1.2.6}
  \end{align*}
\end{proof}

\begin{proof}{(b)}
  \begin{align*}
         & E \text{ is closed in } (X, d)                                                            \\
    \iff & \partial_{(X, d)}(E) \subseteq E                     &  & \by{1.2.12}                     \\
    \iff & E = \text{int}_{(X, d)}(E) \cup \partial_{(X, d)}(E) &  & \by{1.2.6}                      \\
    \iff & E = \overline{E}_{(X, d)}.                           &  & \text{(by \cref{1.2.10}(a)(b))}
  \end{align*}
\end{proof}

\begin{proof}{(c)}
  We first show that \(B_{(X, d)}(x_0, r)\) is open in \((X, d)\).
  Since \(x_0 \in B_{(X, d)}(x_0, r)\), we know that \(B_{(X, d)}(x_0, r) \neq \emptyset\).
  Let \(x \in B_{(X, d)}(x_0, r)\) and let \(r' = r - d(x, x_0)\).
  By \cref{1.2.1} we have \(d(x, x_0) < r\), so \(r' > 0\).
  Then we have
  \begin{align*}
             & \forall y \in B_{(X, d)}(x, r')                                         \\
    \implies & d(y, x) < r'                           &  & \by{1.2.1}                  \\
    \implies & d(y, x) < r - d(x, x_0)                                                 \\
    \implies & d(y, x) + d(x, x_0) < r                                                 \\
    \implies & d(y, x_0) \leq d(y, x) + d(x, x_0) < r &  & \text{(by \cref{1.1.2}(d))} \\
    \implies & y \in B_{(X, d)}(x_0, r)               &  & \by{1.2.1}
  \end{align*}
  and thus \(B_{(X, d)}(x, r') \subseteq B_{(X, d)}(x_0, r)\).
  Since \(x\) is arbitrary, we have
  \begin{align*}
             & B_{(X, d)}(x_0, r) \subseteq \text{int}_{(X, d)}\big(B_{(X, d)}(x_0, r)\big) &  & \by{1.2.5}                   \\
    \implies & B_{(X, d)}(x_0, r) = \text{int}_{(X, d)}\big(B_{(X, d)}(x_0, r)\big)         &  & \by{1.2.6}                   \\
    \implies & B_{(X, d)}(x_0, r) \text{ is open in } (X, d).                               &  & \text{(by \cref{1.2.15}(a))}
  \end{align*}

  Let \(E = \{x \in X : d(x, x_0) \leq r\}\).
  Now we show that \(E\) is closed in \((X, d)\).
  By \cref{1.2.15}(b) we know that \(E\) is closed in \((X, d)\) iff \(E = \overline{E}_{(X, d)}\).
  By \cref{1.2.10}(c) we know that \(E \subseteq \overline{E}_{(X, d)}\).
  So we only need to show that \(\overline{E}_{(X, d)} \subseteq E\), or equivalently \(\overline{E}_{(X, d)} \setminus E = \emptyset\).
  Suppose for sake of contradiction that \(\overline{E}_{(X, d)} \setminus E \neq \emptyset\).
  Let \(y \in \overline{E}_{(X, d)} \setminus E\).
  By \cref{1.2.10}(c), \(\exists (y_n)_{n = 1}^\infty\) such that \(y_n \in E\) for all \(n \in \Z^+\) and \(\lim_{n \to \infty} d(y_n, y) = 0\).
  Since \(y \notin E\), we have \(d(y, x_0) > r\).
  Then \(d(y, x_0) - r > 0\) and we have
  \begin{align*}
             & \exists N \in \Z^+ : \forall n \geq N,                                                 \\
             & d(y_n, y) \leq \dfrac{d(y, x_0) - r}{2} < d(y, x_0) - r               &  & \by{1.1.14} \\
    \implies & \exists N \in \Z^+ : \forall n \geq N,                                                 \\
             & r < d(y, x_0) - d(y_n, y) \leq d(y, x_0) + d(y_n, y) \leq d(y_n, x_0) &  & \by{1.1.2}  \\
    \implies & \exists N \in \Z^+ : \forall n \geq N, r < d(y_n, x_0).
  \end{align*}
  But \(d(y_n, x_0) > r\) means \(y_n \notin E\), a contradiction.
  Thus we must have \(\overline{E}_{(X, d)} \setminus E = \emptyset\), as desired.
\end{proof}

\begin{proof}{(d)}
  By \cref{1.2.15}(b) we know that \(\{x_0\}\) is closed in \((X, d)\) iff \(\{x_0\} = \overline{\{x_0\}}_{(X, d)}\).
  By \cref{1.2.10}(c) we know that \(\{x_0\} \subseteq \overline{\{x_0\}}_{(X, d)}\).
  So we only need to show that \(\overline{\{x_0\}}_{(X, d)} \subseteq \{x_0\}\).
  Let \(y \in \overline{\{x_0\}}_{(X, d)}\).
  By \cref{1.2.10}(c) we know that \(\exists (y_n)_{n = 1}^\infty\) such that \(y_n \in \{x_0\}\) for all \(n \in \Z^+\) and \(\lim_{n \to \infty} d(y_n, y) = 0\).
  But \(y_n \in \{x_0\}\) implies \(y_n = x_0\) for all \(n \in \Z^+\), thus
  \[
    \lim_{n \to \infty} d(y_n, y) = \lim_{n \to \infty} d(x_0, y) = d(x_0, y) = 0.
  \]
  By \cref{1.1.2}(a) we have \(x_0 = y\).
  This means \(\overline{\{x_0\}}_{(X, d)} \subseteq \{x_0\}\), as desired.
\end{proof}

\begin{proof}{(e)}
  Since
  \begin{align*}
         & x_0 \in \text{int}_{(X, d)}(E)                                                           \\
    \iff & \exists r \in \R^+ : B_{(X, d)}(x_0, r) \subseteq E                      &  & \by{1.2.5} \\
    \iff & \exists r \in \R^+ : B_{(X, d)}(x_0, r) \cap (X \setminus E) = \emptyset                 \\
    \iff & x_0 \in \text{ext}_{(X, d)}(X \setminus E),                              &  & \by{1.2.5}
  \end{align*}
  we know that \(\text{int}_{(X, d)}(E) = \text{ext}_{(X, d)}(X \setminus E)\) for any subset \(E\) of \(X\).
  Then we have
  \begin{align*}
    \partial_{(X, d)}(E) & = X \setminus \big(\text{int}_{(X, d)}(E) \cup \text{ext}_{(X, d)}(E)\big)                         &  & \by{1.2.5} \\
                         & = X \setminus \big(\text{ext}_{(X, d)}(X \setminus E) \cup \text{int}_{(X, d)}(X \setminus E)\big)                 \\
                         & = \partial_{(X, d)}(X \setminus E)                                                                 &  & \by{1.2.5}
  \end{align*}
  and
  \begin{align*}
         & E \text{ is open in } (X, d)                                                \\
    \iff & \partial_{(X, d)}(E) \cap E = \emptyset                    &  & \by{1.2.12} \\
    \iff & \partial_{(X, d)}(E) \subseteq (X \setminus E)             &  & \by{1.2.5}  \\
    \iff & \partial_{(X, d)}(X \setminus E) \subseteq (X \setminus E)                  \\
    \iff & X \setminus E \text{ is closed in } (X, d).                &  & \by{1.2.12}
  \end{align*}
\end{proof}

\begin{proof}{(f)}
  Let \(I_n = \{i \in \N : 1 \leq i \leq n\}\).
  First suppose that \(E_i\) is open in \((X, d)\) for every \(i \in I_n\).
  Let \(x_0 \in \bigcap_{i \in I_n} E_i\).
  Then we have
  \begin{align*}
             & x_0 \in \bigcap_{i \in I_n} E_i                                                                                    \\
    \implies & \forall i \in I_n, x_0 \in E_i                                                                                     \\
    \implies & \forall i \in I_n, x_0 \in \text{int}_{(X, d)}(E_i)                              &  & \text{(by \cref{1.2.15}(a))} \\
    \implies & \forall i \in I_n, \exists r_i \in \R^+ : B_{(X, d)}(x_0, r_i) \subseteq E_i     &  & \by{1.2.5}                   \\
    \implies & \forall i \in I_n, B_{(X, d)}\big(x_0, \min_{j \in I_n}(r_j)\big) \subseteq E_i  &  & \text{(\(I_n\) is finite)}   \\
    \implies & B_{(X, d)}\big(x_0, \min_{j \in I_n}(r_j)\big) \subseteq \bigcap_{i \in I_n} E_i                                   \\
    \implies & x_0 \in \text{int}_{(X, d)}\bigg(\bigcap_{i \in I_n} E_i\bigg).                  &  & \by{1.2.5}
  \end{align*}
  Since \(x_0\) is arbitrary, we have
  \begin{align*}
             & \bigcap_{i \in I_n} E_i \subseteq \text{int}_{(X, d)}\bigg(\bigcap_{i \in I_n} E_i\bigg)                                   \\
    \implies & \bigcap_{i \in I_n} E_i = \text{int}_{(X, d)}\bigg(\bigcap_{i \in I_n} E_i\bigg)         &  & \by{1.2.6}                   \\
    \implies & \bigcap_{i \in I_n} E_i \text{ is open in } (X, d).                                      &  & \text{(by \cref{1.2.15}(a))}
  \end{align*}

  Now suppose that \(F_i\) is closed in \((X, d)\) for every \(i \in I_n\).
  Then we have
  \begin{align*}
             & \forall i \in I_n, F_i \text{ is closed in } (X, d)                                                                          \\
    \implies & \forall i \in I_n, X \setminus F_i \text{ is open in } (X, d)                              &  & \text{(by \cref{1.2.15}(e))} \\
    \implies & \bigcap_{i \in I_n} (X \setminus F_i) \text{ is open in } (X, d)                           &  & \text{(from prove above)}    \\
    \implies & X \setminus \bigg(\bigcap_{i \in I_n} (X \setminus F_i)\bigg) \text{ is closed in } (X, d) &  & \text{(by \cref{1.2.15}(e))} \\
    \implies & \bigcup_{i \in I_n} F_i \text{ is closed in } (X, d).
  \end{align*}
\end{proof}

\begin{proof}{(g)}
  First that \(E_\alpha\) is open in \((X, d)\) for every \(\alpha \in I\).
  Let \(x_0 \in \bigcup_{\alpha \in I} E_\alpha\).
  Then we have
  \begin{align*}
             & x_0 \in \bigcup_{\alpha \in I} E_\alpha                                                                             \\
    \implies & \exists \beta \in I : x_0 \in E_\beta                                                                               \\
    \implies & \exists \beta \in I : x_0 \in \text{int}_{(X, d)}(E_\beta)                        &  & \text{(by \cref{1.2.15}(a))} \\
    \implies & \exists \beta \in I : \exists r \in \R^+ : B_{(X, d)}(x_0, r) \subseteq E_\beta   &  & \by{1.2.5}                   \\
    \implies & \exists r \in \R^+ : B_{(X, d)}(x_0, r) \subseteq \bigcup_{\alpha \in I} E_\alpha                                   \\
    \implies & x_0 \in \text{int}_{(X, d)}\bigg(\bigcup_{\alpha \in I} E_\alpha\bigg).           &  & \by{1.2.5}
  \end{align*}
  Since \(x_0\) is arbitrary, we have
  \begin{align*}
             & \bigcup_{\alpha \in I} E_\alpha \subseteq \text{int}_{(X, d)}\bigg(\bigcup_{\alpha \in I} E_\alpha\bigg)                                   \\
    \implies & \bigcup_{\alpha \in I} E_\alpha = \text{int}_{(X, d)}\bigg(\bigcup_{\alpha \in I} E_\alpha\bigg)         &  & \by{1.2.6}                   \\
    \implies & \bigcup_{\alpha \in I} E_\alpha \text{ is open in } (X, d).                                              &  & \text{(by \cref{1.2.15}(a))}
  \end{align*}

  Now suppose that \(F_\alpha\) is closed in \((X, d)\) for every \(\alpha \in I\).
  Then we have
  \begin{align*}
             & \forall \alpha \in I, F_\alpha \text{ is closed in } (X, d)                                                                          \\
    \implies & \forall \alpha \in I, X \setminus F_\alpha \text{ is open in } (X, d)                              &  & \text{(by \cref{1.2.15}(e))} \\
    \implies & \bigcup_{\alpha \in I} (X \setminus F_\alpha) \text{ is open in } (X, d)                                                             \\
    \implies & X \setminus \bigg(\bigcup_{\alpha \in I} (X \setminus F_\alpha)\bigg) \text{ is closed in } (X, d) &  & \text{(by \cref{1.2.15}(e))} \\
    \implies & \bigcap_{\alpha \in I} F_\alpha \text{ is closed in } (X, d).
  \end{align*}
\end{proof}

\begin{proof}{(h)}
  We first show that \(\text{int}_{(X, d)}(E)\) is open in \((X, d)\).
  Let \(x_0 \in \text{int}_{(X, d)}(E)\).
  Then we have
  \begin{align*}
             & x_0 \in \text{int}_{(X, d)}(E)                                                                                        \\
    \implies & \exists r \in \R^+ : B_{(X, d)}(x_0, r) \subseteq E                                 &  & \by{1.2.5}                   \\
    \implies & \exists r \in \R^+ : \forall y \in B_{(X, d)}(x_0, r), \exists r' \in \R^+ :                                          \\
             & B_{(X, d)}(y, r') \subseteq B_{(X, d)}(x_0, r) \subseteq E                          &  & \text{(by \cref{1.2.15}(c))} \\
    \implies & \exists r \in \R^+ : \forall y \in B_{(X, d)}(x_0, r), y \in \text{int}_{(X, d)}(E) &  & \by{1.2.5}                   \\
    \implies & \exists r \in \R^+ : B_{(X, d)}(x_0, r) \subseteq \text{int}_{(X, d)}(E)                                              \\
    \implies & x_0 \in \text{int}_{(X, d)}\big(\text{int}_{(X, d)}(E)\big).                        &  & \by{1.2.5}
  \end{align*}
  Since \(x_0\) is arbitrary, we have
  \begin{align*}
             & \text{int}_{(X, d)}(E) \subseteq \text{int}_{(X, d)}\big(\text{int}_{(X, d)}(E)\big)                                   \\
    \implies & \text{int}_{(X, d)}(E) = \text{int}_{(X, d)}\big(\text{int}_{(X, d)}(E)\big)         &  & \by{1.2.6}                   \\
    \implies & \text{int}_{(X, d)}(E) \text{ is open in } (X, d).                                   &  & \text{(by \cref{1.2.15}(a))}
  \end{align*}

  Next we show that if \(V \subseteq E\) and \((V, d)\) is open in \(X\), then \(V \subseteq \text{int}_{(X, d)}(E)\).
  \begin{align*}
             & (V \subseteq E) \land \big(V \text{ is open in } (X, d)\big)                                   \\
    \implies & V = \text{int}_{(X, d)}(V) \subseteq E                       &  & \text{(by \cref{1.2.15}(a))} \\
    \implies & \forall x_0 \in V, \exists r \in \R^+ :                                                        \\
             & B_{(X, d)}(x_0, r) \subseteq V \subseteq E                   &  & \by{1.2.5}                   \\
    \implies & \forall x_0 \in V, x_0 \in \text{int}_{(X, d)}(E)            &  & \by{1.2.5}                   \\
    \implies & V \subseteq \text{int}_{(X, d)}(E).
  \end{align*}

  Next we show that \(\overline{E}_{(X, d)}\) is closed in \((X, d)\).
  Let \(x_0 \in X \setminus \overline{E}_{(X, d)}\).
  Then we have
  \begin{align*}
             & x_0 \in X \setminus \overline{E}_{(X, d)}                                                                      \\
    \implies & \exists r \in \R^+ : B_{(X, d)}(x_0, r) \cap E = \emptyset                                     &  & \by{1.2.9} \\
    \implies & \exists r \in \R^+ : \forall y \in B_{(X, d)}(x_0, r), \exists r' \in \R^+ :                                   \\
             & \begin{dcases}
                 B_{(X, d)}(y, r') \subseteq B_{(X, d)}(x_0, r) \\
                 B_{(X, d)}(y, r') \cap E = \emptyset
               \end{dcases}                                              &  & \text{(by \cref{1.2.15}(c))}                    \\
    \implies & \exists r \in \R^+ : \forall y \in B_{(X, d)}(x_0, r), y \in X \setminus \overline{E}_{(X, d)} &  & \by{1.2.9} \\
    \implies & \exists r \in \R^+ : B_{(X, d)}(x_0, r) \subseteq X \setminus \overline{E}_{(X, d)}                            \\
    \implies & x_0 \in \text{int}_{(X, d)}(X \setminus \overline{E}_{(X, d)}).                                &  & \by{1.2.5}
  \end{align*}
  Since \(x_0\) is arbitrary, we have
  \begin{align*}
             & X \setminus \overline{E}_{(X, d)} \subseteq \text{int}_{(X, d)}(X \setminus \overline{E}_{(X, d)})                                   \\
    \implies & X \setminus \overline{E}_{(X, d)} = \text{int}_{(X, d)}(X \setminus \overline{E}_{(X, d)})         &  & \by{1.2.6}                   \\
    \implies & X \setminus \overline{E}_{(X, d)} \text{ is open in } (X, d)                                       &  & \text{(by \cref{1.2.15}(a))} \\
    \implies & \overline{E}_{(X, d)} \text{ is closed in } (X, d).                                                &  & \text{(by \cref{1.2.15}(e))}
  \end{align*}

  Finally we show that if \(E \subseteq K \subseteq X\) and \(K\) is closed in \((X, d)\), then \(\overline{E}_{(X, d)} \subseteq K\).
  \begin{align*}
             & (E \subseteq K) \land \big(K \text{ is closed in } (X, d)\big)                                                                                       \\
    \implies & \forall x_0 \in \overline{E}_{(X, d)}, \forall r \in \R^+, B_{(X, d)}(x_0, r) \cap E \neq \emptyset &                 & \by{1.2.9}                   \\
    \implies & \forall x_0 \in \overline{E}_{(X, d)}, \forall r \in \R^+, B_{(X, d)}(x_0, r) \cap K \neq \emptyset & (E \subseteq K)                                \\
    \implies & \forall x_0 \in \overline{E}_{(X, d)}, x_0 \in \overline{K}_{(X, d)}                                &                 & \by{1.2.9}                   \\
    \implies & \overline{E}_{(X, d)} \subseteq \overline{K}_{(X, d)}                                                                                                \\
    \implies & \overline{E}_{(X, d)} \subseteq K.                                                                  &                 & \text{(by \cref{1.2.15}(b))}
  \end{align*}
\end{proof}

\exercisesection

\begin{ex}\label{ex:1.2.1}
  Verify the claims in \cref{1.2.8}.
\end{ex}

\begin{proof}
  See \cref{1.2.8}.
\end{proof}

\begin{ex}\label{ex:1.2.2}
  Prove \cref{1.2.10}.
\end{ex}

\begin{proof}
  See \cref{1.2.10}.
\end{proof}

\begin{ex}\label{ex:1.2.3}
  Prove \cref{1.2.15}.
\end{ex}

\begin{proof}
  See \cref{1.2.15}.
\end{proof}

\begin{ex}\label{ex:1.2.4}
  Let \((X, d)\) be a metric space, \(x_0\) be a point in \(X\), and \(r > 0\).
  Let \(B\) be the open ball \(B \coloneqq B(x_0, r) = \{x \in X : d(x, x_0) < r\}\), and let \(C\) be the closed ball \(C \coloneqq \{x \in X : d(x, x_0) \leq r\}\).
  \begin{enumerate}
    \item Show that \(\overline{B} \subseteq C\).
    \item Give an example of a metric space \((X, d)\), a point \(x_0\), and a radius \(r > 0\) such that \(\overline{B}\) is \emph{not} equal to \(C\).
  \end{enumerate}
\end{ex}

\begin{proof}{(a)}
  Since \(B \subseteq C\) and \(C\) is closed in \((X, d)\), by \cref{1.2.15}(h) we have \(\overline{B}_{(X, d)} \subseteq C\).
\end{proof}

\begin{proof}{(b)}
  Let \(X = \R\) and let \(d = d_{\text{disc}}\) be the metric function.
  By \cref{ex:1.1.11} we know that \((\R, d_{\text{disc}})\) is a metric space.
  Let \(B = B_{(\R, d_{\text{disc}})}(0, 1)\) and let \(C = \{x \in \R : d_{\text{disc}}(x, 0) \leq 1\}\).
  Then we know that \(B = \{0\}\) and \(C = \R\).
  But by \cref{1.2.8} we have \(\overline{B} = B \neq C\).
\end{proof}
\section{Relative topology}\label{sec:1.3}

\begin{note}
  Consider the plane \(\R^2\) with the Euclidean metric \(d_{l^2}\).
  Inside the plane, we can find the x-axis \(X \coloneqq \{(x, 0) : x \in \R\}\).
  The metric \(d_{l^2}\) can be restricted to \(X\), creating a subspace \((X, d_{l^2}|_{X \times X})\) of \((\R^2, d_{l^2})\).
  This subspace is essentially the same as the real line \((\R, d)\) with the usual metric;
  the precise way of stating this is that \((X, d_{l^2}|_{X \times X})\) is \emph{isometric} to \((\R, d)\).
\end{note}

\setcounter{thm}{2}
\begin{defn}[Relative topology]\label{1.3.3}
  Let \((X, d)\) be a metric space, let \(Y\) be a subset of \(X\), and let \(E\) be a subset of \(Y\).
  We say that \(E\) is \emph{relatively open with respect to \(Y\)} if it is open in the metric subspace \((Y, d|_{Y \times Y})\).
  Similarly, we say that \(E\) is \emph{relatively closed with respect to \(Y\)} if it is closed in the metric space \((Y, d|_{Y \times Y})\).
\end{defn}

\begin{prop}\label{1.3.4}
  Let \((X, d)\) be a metric space, let \(Y\) be a subset of \(X\), and let \(E\) be a subset of \(Y\).
  \begin{enumerate}
    \item \(E\) is relatively open with respect to \(Y\) iff \(E = V \cap Y\) for some set \(V \subseteq X\) which is open in \(X\).
    \item \(E\) is relatively closed with respect to \(Y\) iff \(E = K \cap Y\) for some set \(K \subseteq X\) which is closed in \(X\).
  \end{enumerate}
\end{prop}

\begin{proof}{(a)}
  First suppose that \(E\) is relatively open with respect to \(Y\).
  Then, \(E\) is open in the metric space \((Y, d|_{Y \times Y})\).
  Thus, for every \(x \in E\), there exists a radius \(r > 0\) such that the ball \(B_{(Y, d|_{Y \times Y})}(x, r)\) is contained in \(E\).
  This radius \(r\) depends on \(x\);
  to emphasize this we write \(r_x\) instead of \(r\), thus for every \(x \in E\) the ball \(B_{(Y, d|_{Y \times Y})}(x, r_x)\) is contained in \(E\).
  (Note that we have used the axiom of choice to do this.)

  Now consider the set
  \[
    V \coloneqq \bigcup_{x \in E} B_{(X, d)}(x, r_x).
  \]
  This is a subset of \(X\).
  By \cref{1.2.15}(c) and (g), \(V\) is open in \((X, d)\).
  Now we prove that \(E = V \cap Y\).
  Certainly any point \(x\) in \(E\) lies in \(V \cap Y\), since it lies in \(Y\) and it also lies in \(B_{(X, d)}(x, r_x)\), and hence in \(V\).
  Now suppose that \(y\) is a point in \(V \cap Y\).
  Then \(y \in V\), which implies that there exists an \(x \in E\) such that \(y \in B_{(X, d)}(x, r_x)\).
  But since \(y\) is also in \(Y\) , this implies that \(y \in B_{(Y, d|_{Y \times Y})}(x, r_x)\).
  But by definition of \(r_x\), this means that \(y \in E\), as desired.
  Thus we have found an open set \(V\) in \((X, d)\) for which \(E = V \cap Y\) as desired.

  Now we do the converse.
  Suppose that \(E = V \cap Y\) for some open set \(V\) in \((X, d)\);
  we have to show that \(E\) is relatively open with respect to \(Y\).
  Let \(x\) be any point in \(E\);
  we have to show that \(x\) is an interior point of \(E\) in the metric space \((Y, d|_{Y \times Y})\).
  Since \(x \in E\), we know \(x \in V\).
  Since \(V\) is open in \((X, d)\), we know that there is a radius \(r > 0\) such that \(B_{(X, d)}(x, r)\) is contained in \(V\).
  Strictly speaking, \(r\) depends on \(x\), and so we could write \(r_x\) instead of \(r\), but for this argument we will only use a single choice of \(x\) (as opposed to the argument in the previous paragraph) and so we will not bother to subscript \(r\) here.
  Since \(E = V \cap Y\), this means that \(B_{(X, d)}(x, r) \cap Y\) is contained in \(E\).
  But \(B_{(X, d)}(x, r) \cap Y\) is exactly the same as \(B_{(Y, d|_{Y \times Y})}(x, r)\), and so \(B_{(Y, d|_{Y \times Y})}(x, r)\) is contained in \(E\).
  Thus \(x\) is an interior point of \(E\) in the metric space \((Y, d|_{Y \times Y})\), as desired.
\end{proof}

\begin{proof}{(b)}
  First suppose that \(E\) is relatively closed with respect to \(Y\).
  Then by \cref{1.3.3} \(E\) is closed in \((Y, d|_{Y \times Y})\).
  By \cref{1.2.15}(e) \(Y \setminus E\) is open in \((Y, d|_{Y \times Y})\).
  By \cref{1.3.4}(a) we know that \(Y \setminus E = V \cap Y\) for some set \(V \subseteq X\) which is open in \((X, d)\).
  Let \(K = X \setminus V\).
  By \cref{1.2.15}(e) we know that \(K\) is closed in \((X, d)\).
  Then we have
  \begin{align*}
    K \cap Y & = (X \setminus V) \cap Y          \\
             & = (X \cap Y) \setminus (V \cap Y) \\
             & = Y \setminus (V \cap Y)          \\
             & = Y \setminus (Y \setminus E)     \\
             & = E.
  \end{align*}

  Now suppose that \(E = K \cap Y\) for some set \(K \subseteq X\) which is closed in \((X, d)\).
  Then by \cref{1.2.15}(e) \(X \setminus K\) is open in \((X, d)\).
  By \cref{1.3.4}(a) we know that \(F = (X \setminus K) \cap Y\) is relatively open with respect to \(Y\).
  Then by \cref{1.2.15}(e) \(Y \setminus F\) is relatively closed with respect to \(Y\) and
  \begin{align*}
    Y \setminus F & = Y \setminus \big((X \setminus K) \cap Y\big)          \\
                  & = Y \setminus \big((X \cap Y) \setminus (K \cap Y)\big) \\
                  & = Y \setminus \big(Y \setminus (K \cap Y)\big)          \\
                  & = Y \setminus (Y \setminus E)                           \\
                  & = E.
  \end{align*}
\end{proof}

\exercisesection

\begin{ex}\label{ex:1.3.1}
  Prove \cref{1.3.4}(b).
\end{ex}

\begin{proof}
  See \cref{1.3.4}.
\end{proof}
\section{Cauchy sequences and complete metric spaces}\label{ii:sec:1.4}

\begin{defn}[Subsequences]\label{ii:1.4.1}
  Suppose that \((x^{(n)})_{n = m}^\infty\) is a sequence of points in a metric space \((X, d)\).
  Suppose that \(n_1, n_2, n_3, \dots\) is an increasing sequence of integers which are at least as large as \(m\), thus
  \[
    m \leq n_1 < n_2 < n_3 < \dots.
  \]
  Then we call the sequence \((x^{(n_j)})_{j = 1}^\infty\) a \emph{subsequence} of the original sequence \((x^{(n)})_{n = m}^\infty\).
\end{defn}

\setcounter{thm}{2}
\begin{lem}\label{ii:1.4.3}
  Let \((x^{(n)})_{n = m}^\infty\) be a sequence in \((X, d)\) which converges to some limit \(x_0\).
  Then every subsequence \((x^{(n_j)})_{j = 1}^\infty\) of that sequence also converges to \(x_0\).
\end{lem}

\begin{proof}
  \begin{align*}
             & \lim_{n \to \infty} d(x^{(n)}, x_0) = 0                                                                                                                \\
    \implies & \forall \varepsilon \in \R^+, \exists N \in \N : \forall n \geq N, d(x^{(n)}, x_0) \leq \varepsilon                                &  & \by{ii:1.1.14} \\
    \implies & \forall \varepsilon \in \R^+, \exists j \in \N : \forall j \geq N, (n_j \geq j) \land \big(d(x^{(n_j)}, x_0) \leq \varepsilon\big)                     \\
    \implies & \lim_{j \to \infty} d(x^{(n_j)}, x_0) = 0.                                                                                         &  & \by{ii:1.1.14}
  \end{align*}
\end{proof}

\begin{defn}[Limit points]\label{ii:1.4.4}
  Suppose that \((x^{(n)})_{n = m}^\infty\) is a sequence of points in a metric space \((X, d)\), and let \(L \in X\).
  We say that \(L\) is a \emph{limit point} of \((x^{(n)})_{n = m}^\infty\) iff for every \(N \geq m\) and \(\varepsilon > 0\) there exists an \(n \geq N\) such that \(d(x^{(n)}, L) \leq \varepsilon\).
\end{defn}

\begin{prop}\label{ii:1.4.5}
  Let \((x^{(n)})_{n = m}^\infty\) be a sequence of points in a metric space \((X, d)\), and let \(L \in X\).
  Then the following are equivalent:
  \begin{itemize}
    \item \(L\) is a limit point of \((x^{(n)})_{n = m}^\infty\).
    \item There exists a subsequence \((x^{(n_j)})_{j = 1}^\infty\) of the original sequence \((x^{(n)})_{n = m}^\infty\) which converges to \(L\).
  \end{itemize}
\end{prop}

\begin{proof}
  We first show that if \(L\) is a limit point of \((x^{(n)})_{n = m}^\infty\) in \((X, d)\), then there exists a subsequence of \((x^{(n)})_{n = m}^\infty\) which converges to \(L\) with respect to \(d\).
  Let \(N \in \N\).
  Since \(L\) is a limit point of \((x^{(n)})_{n = m}^\infty\) in \((X, d)\), by \cref{ii:1.4.4} we know that
  \[
    \forall \varepsilon \in \R^+, \forall N \geq m, \exists n \geq N : d(x^{(n)}, L) \leq \varepsilon.
  \]
  In particular, for every \(j \in \Z^+\), we have
  \[
    \forall N \geq m, \exists n \geq N : d(x^{(n)}, L) \leq \dfrac{1}{j}.
  \]
  Now we define \(n_j\) as follow:
  \[
    n_j = \begin{dcases}
      \min\set{n \in \Z^+ : d(x^{(n)}, L) \leq 1}                                            & \text{if } j = 1    \\
      \min\set{n \in \Z^+ : (n > n_{j - 1}) \land \big(d(x^{(n)}, L) \leq \dfrac{1}{j}\big)} & \text{if } j \neq 1
    \end{dcases}
  \]
  By well-ordering theorem \(n_j\) is well-defined for every \(j \in \Z^+\).
  By applying squeeze test to the subsequence \((x^{(n_j)})_{j = 1}^\infty\) we have
  \begin{align*}
             & 0 \leq d(x^{(n_j)}, L) \leq \dfrac{1}{j}                                                                     \\
    \implies & 0 = \lim_{j \to \infty} 0 \leq \lim_{j \to \infty} d(x^{(n_j)}, L) \leq \lim_{j \to \infty} \dfrac{1}{j} = 0 \\
    \implies & \lim_{j \to \infty} d(x^{(n_j)}, L) = 0
  \end{align*}
  and thus by \cref{ii:1.1.14} the sequence \((x^{(n_j)})_{j = 1}^\infty\) converges to \(L\) with respect to \(d\).

  Now we show that if a subsequence of \((x^{(n)})_{n = m}^\infty\) converges to \(L\) with respect to \(d\), then \(L\) is a limit point of \((x^{(n)})_{n = m}^\infty\) in \((X, d)\).
  Let \(N \in \N\) and let \((x^{(n_j)})_{j = 1}^\infty\) be a subsequence of \((x^{(n)})_{n = 0}^\infty\) where \(\lim_{j \to \infty} d(x^{(n_j)}, L) = 0\).
  Then we have
  \begin{align*}
             & \lim_{n \to \infty} d(x^{(n_j)}, L) = 0                                                                                         \\
    \implies & \forall \varepsilon \in \R^+, \exists j \geq 1 : d(x^{(n_j)}, L) \leq \varepsilon                           &  & \by{ii:1.1.14} \\
    \implies & \forall \varepsilon \in \R^+,  \forall N \geq m, \exists n \geq n_j \geq N : d(x^{(n)}, L) \leq \varepsilon                     \\
    \implies & L \text{ is a limit point of } (x^{(n)})_{n = m}^\infty \text{ in } (X, d).                                 &  & \by{ii:1.4.4}
  \end{align*}
\end{proof}

\begin{defn}[Cauchy sequences]\label{ii:1.4.6}
  Let \((x^{(n)})_{n = m}^\infty\) be a sequence of points in a metric space \((X, d)\).
  We say that this sequence is a \emph{Cauchy sequence} iff for every \(\varepsilon > 0\), there exists an \(N \geq m\) such that \(d(x^{(j)}, x^{(k)}) \leq \varepsilon\) for all \(j, k \geq N\).
\end{defn}

\begin{note}
  Here the book use \(<\) instead of \(\leq\), but the two inequalities are more or less the same.
  I use \(\leq\) to ensure consistency with Definition 5.1.8 in Analysis I.
\end{note}

\begin{lem}[Convergent sequences are Cauchy sequences]\label{ii:1.4.7}
  Let \((x^{(n)})_{n = m}^\infty\) be a sequence in \((X, d)\) which converges to some limit \(x_0\).
  Then \((x^{(n)})_{n = m}^\infty\) is also a Cauchy sequence.
\end{lem}

\begin{proof}
  Let \(N \in \N\).
  Since \(\lim_{n \to \infty} d(x^{(n)}, x_0) = 0\), by \cref{ii:1.1.14} we know that
  \[
    \forall \varepsilon \in \R^+, \exists N \geq m : \forall n \geq N, d(x^{(n)}, x_0) \leq \dfrac{\varepsilon}{2}.
  \]
  Let \(k \in \N\) and \(k \geq N\).
  Then we have
  \begin{align*}
    d(x^{(n)}, x^{(k)}) & \leq d(x^{(n)}, x_0) + d(x_0, x^{(k)})               &  & \by{ii:1.1.2}[d] \\
                        & = d(x^{(n)}, x_0) + d(x^{(k)}, x_0)                  &  & \by{ii:1.1.2}[c] \\
                        & \leq \dfrac{\varepsilon}{2} + \dfrac{\varepsilon}{2}                       \\
                        & = \varepsilon
  \end{align*}
  and by \cref{ii:1.4.6} \((x^{(n)})_{n = m}^\infty\) is a Cauchy sequence in \((X, d)\).
\end{proof}

\setcounter{thm}{8}
\begin{lem}\label{ii:1.4.9}
  Let \((x^{(n)})_{n = m}^\infty\) be a Cauchy sequence in \((X, d)\).
  Suppose that there is some subsequence \((x^{(n_j)})_{j = 1}^\infty\) of this sequence which converges to a limit \(x_0\) in \(X\).
  Then the original sequence \((x^{(n)})_{n = m}^\infty\) also converges to \(x_0\).
\end{lem}

\begin{proof}
  Let \(N_1, N_2, i, k \in \N\).
  Since \(\lim_{j \to \infty} d(x^{(n_j)}, x_0) = 0\), by \cref{ii:1.1.14} we know that
  \[
    \forall \varepsilon \in \R^+, \exists N_1 \geq 1 : \forall j \geq N_1, d(x^{(n_j)}, x_0) \leq \dfrac{\varepsilon}{2}.
  \]
  Now fix such \(\varepsilon\).
  Since \((x^{(n)})_{n = m}^\infty\) is a Cauchy sequence in \((X, d)\), by \cref{ii:1.4.6} we know that
  \[
    \exists N_2 \geq m : \forall i, k \geq N_2, d(x^{(i)}, x^{(k)}) \leq \dfrac{\varepsilon}{2}.
  \]
  Let \(N = \max(N_1, N_2)\).
  Then \(\forall i, j \geq N\), we have
  \begin{align*}
    d(x^{(i)}, x_0) & \leq d(x^{(i)}, x^{(n_j)}) + d(x^{(n_j)}, x_0)       &                     & \by{ii:1.1.2}[d] \\
                    & \leq \dfrac{\varepsilon}{2} + \dfrac{\varepsilon}{2} & (n_j \geq j \geq N)                    \\
                    & = \varepsilon.
  \end{align*}
  Since \(\varepsilon\) was arbitrary, by \cref{ii:1.1.14} \(\lim_{n \to \infty} d(x^{(n)}, x_0) = 0\).
\end{proof}

\begin{defn}[Complete metric spaces]\label{ii:1.4.10}
  A metric space \((X, d)\) is said to be \emph{complete} iff every Cauchy sequence in \((X, d)\) is in fact convergent in \((X, d)\).
\end{defn}

\setcounter{thm}{11}
\begin{prop}\label{ii:1.4.12}
  \quad
  \begin{enumerate}
    \item Let \((X, d)\) be a metric space, and let \((Y, d|_{Y \times Y})\) be a subspace of \((X, d)\).
          If \((Y, d|_{Y \times Y})\) is complete, then \(Y\) must be closed in \(X\).
    \item Conversely, suppose that \((X, d)\) is a complete metric space, and \(Y\) is a closed subset of \(X\).
          Then the subspace \((Y, d|_{Y \times Y})\) is also complete.
  \end{enumerate}
\end{prop}

\begin{proof}
  We first show that the statement (a) is true.
  Suppose that \((X, d)\) is a metric space and \((Y, d|_{Y \times Y})\) is a complete subspace of \((X, d)\).
  Let \(x_0 \in \partial_{(X, d)}(Y)\).
  Then we have
  \begin{align*}
             & \begin{dcases}
                 (Y, d|_{Y \times Y}) \text{ is complete} \\
                 x_0 \in \partial_{(X, d)}(Y)
               \end{dcases}                                                                                            \\
    \implies & \begin{dcases}
                 (Y, d|_{Y \times Y}) \text{ is complete} \\
                 \exists (x^{(n)})_{n = m}^\infty \text{ in } Y : \lim_{n \to \infty} d(x^{(n)}, x_0) = 0
               \end{dcases} &  & \by{ii:1.2.10}                      \\
    \implies & x_0 \in Y.                                                                                                      &  & \by{ii:1.4.10}
  \end{align*}
  Since \(x_0\) was arbitrary, we have \(\partial_{(X, d)}(Y) \subseteq Y\) and by \cref{ii:1.2.12} \(Y\) is closed in \((X, d)\).

  Now we show that the statement (b) is true.
  Suppose that \((X, d)\) is a complete metric space, \(Y \subseteq X\) and \(Y\) is closed in \((X, d)\).
  Let \((x^{(n)})_{n = m}^\infty\) be a Cauchy sequence in \((Y, d|_{Y \times Y})\).
  Since \(Y \subseteq X\), we know that \((x^{(n)})_{n = m}^\infty\) is also a Cauchy sequence in \((X, d)\).
  Since \((X, d)\) is complete, by \cref{ii:1.4.10} we know that \(\lim_{n \to \infty} d(x^{(n)}, x_0) = 0\) for some \(x_0 \in X\).
  Since \((x^{(n)})_{n = m}^\infty\) is in \(Y\) and \(\lim_{n \to \infty} d(x^{(n)}, x_0) = 0\), by \cref{ii:1.2.10}(c) we know that \(x_0 \in \overline{Y}_{(X, d)}\).
  But \(Y\) is closed in \((X, d)\), thus by \cref{ii:1.2.15}(b) we know that \(x_0 \in Y\).
  Since \((x^{(n)})_{n = m}^\infty\) was arbitrary Cauchy sequence in \((Y, d|_{Y \times Y})\), by \cref{ii:1.4.10} \((Y, d|_{Y \times Y})\) is complete.
\end{proof}

\exercisesection

\begin{ex}\label{ii:ex:1.4.1}
  Prove \cref{ii:1.4.3}.
\end{ex}

\begin{proof}
  See \cref{ii:1.4.3}.
\end{proof}

\begin{ex}\label{ii:ex:1.4.2}
  Prove Proposition 1.4.5.
\end{ex}

\begin{proof}
  See \cref{ii:1.4.5}.
\end{proof}

\begin{ex}\label{ii:ex:1.4.3}
  Prove \cref{ii:1.4.7}.
\end{ex}

\begin{proof}
  See \cref{ii:1.4.7}.
\end{proof}

\begin{ex}\label{ii:ex:1.4.4}
  Prove \cref{ii:1.4.9}.
\end{ex}

\begin{proof}
  See \cref{ii:1.4.9}.
\end{proof}

\begin{ex}\label{ii:ex:1.4.5}
  Let \((x^{(n)})_{n = m}^\infty\) be a sequence of points in a metric space \((X, d)\), and let \(L \in X\).
  Show that if \(L\) is a limit point of the sequence \((x^{(n)})_{n = m}^\infty\), then \(L\) is an adherent point of the set \(\set{x^{(n)} : n \geq m}\).
  Is the converse true?
\end{ex}

\begin{proof}
  Let \(E = \set{x^{(n)} : n \geq m}\).
  We first show that if \(L\) is a limit point of \((x^{(n)})_{n = m}^\infty\) in \((X, d)\), then \(L \in \overline{E}_{(X, d)}\).
  Suppose that \(L\) is a limit point of \((x^{(n)})_{n = m}^\infty\) in \((X, d)\).
  By \cref{ii:1.4.5} we know that \(\exists (x^{(n_j)})_{j = 1}^\infty\) in \(E\) such that \(\lim_{j \to \infty} d(x^{(n_j)}, L) = 0\).
  Thus, by \cref{ii:1.2.10}(c) \(L \in \overline{E}_{(X, d)}\).

  Now we show that if \(L \in \overline{E}_{(X, d)}\), then \(L\) may not be a limit point of \((x^{(n)})_{n = m}^\infty\) in \((X, d)\).
  Let \((X, d) = (\R, d_{l^1}|_{\R \times \R})\) and let \(x^{(n)} = 1 / n\).
  Then by \cref{ii:1.2.9} we know that \(1 \in \overline{E}_{(X, d)}\).
  But by \cref{ii:1.4.4} we know that \(1\) is not an limit point of \((x^{(n)})_{n = m}^\infty\) in \((\R, d_{l^1})\) since every subsequence of \((x^{(n)})_{n = 1}^\infty\) converges to \(0\) with respect to \(d_{l^1}|_{\R \times \R}\).
  Thus, if \(L\) is an adherent point of \(E\) in \((X, d)\), then \(L\) may not be a limit point of \((x^{(n)})_{n = m}^\infty\) in \((X, d)\).
\end{proof}

\begin{ex}\label{ii:ex:1.4.6}
  Show that every Cauchy sequence can have at most one limit point.
\end{ex}

\begin{proof}
  Suppose for the sake of contradiction that there exists a Cauchy sequence \((x^{(n)})_{n = m}^\infty\) in some metric space \((X, d)\) which has two limit points \(L\) and \(L'\).
  Then by \cref{ii:1.4.5} \(\exists (x^{(n_i)})_{i = 1}^\infty, (x^{(n_j)})_{j = 1}^\infty,\) which converges to \(L\) and \(L'\) respectively.
  Since \((x^{(n)})_{n = m}^\infty\) is a Cauchy sequence in \((X, d)\), by \cref{ii:1.4.9} we know that \((x^{(n)})_{n = m}^\infty\) converges to \(L\) and \(L'\) with respect to \(d\), which contradict to \cref{ii:1.1.20}.
  Thus, every Cauchy sequence can have at most one limit point.
\end{proof}

\begin{ex}\label{ii:ex:1.4.7}
  Prove \cref{ii:1.4.12}.
\end{ex}

\begin{proof}
  See \cref{ii:1.4.12}.
\end{proof}

\begin{ex}\label{ii:ex:1.4.8}
  The following construction generalizes the construction of the reals from the rationals in Chapter 5, allowing one to view any metric space as a subspace of a complete metric space.
  In what follows we let \((X, d)\) be a metric space.
  \begin{enumerate}
    \item Given any Cauchy sequence \((x^{(n)})_{n = m}^\infty\) in \(X\), we introduce the \emph{formal limit} \\
          \(\LIM_{n \to \infty} x_n\).
          We say that two formal limits \(\LIM_{n \to \infty} x_n\) and \(\LIM_{n \to \infty} y_n\) are equal if \(\LIM_{n \to \infty} d(x_n, y_n)\) is equal to zero.
          Show that this equality relation obeys the reflexive, symmetry, and transitive axioms.
    \item Let \(\overline{X}\) be the space of all formal limits of Cauchy sequences in \(X\), with the above equality relation.
          Define a metric \(d_{\overline{X}} : \overline{X} \times \overline{X} \to [0, \infty)\) by setting
          \[
            d_{\overline{X}}(\LIM_{n \to \infty} x_n, \LIM_{n \to \infty} y_n) \coloneqq \lim_{n \to \infty} d(x_n, y_n).
          \]
          Show that this function is well-defined (this means not only that the limit \\
          \(\lim_{n \to \infty} d(x_n, y_n)\) exists, but also that the axiom of substitution is obeyed;
          cf. Lemma 5.3.7), and gives \(\overline{X}\) the structure of a metric space.
    \item Show that the metric space \((\overline{X}, d_{\overline{X}})\) is complete.
    \item We identify an element \(x \in X\) with the corresponding formal limit \(\LIM_{n \to \infty} x\) in \(X\);
          show that this is legitimate by verifying that \(x = y \iff \LIM_{n \to \infty} x = \LIM_{n \to \infty} y\).
          With this identification, show that \(d(x, y) = d_{\overline{X}}(x, y)\), and thus \((X, d)\) can now be thought of as a subspace of \((\overline{X}, d_{\overline{X}})\).
    \item Show that the closure of \(X\) in \(\overline{X}\) is \(\overline{X}\) (which explains the choice of notation \(\overline{X}\)).
    \item Show that the formal limit agrees with the actual limit, thus if \((x_n)_{n = 1}^\infty\) is any Cauchy sequence in \(X\), then we have \(\lim_{n \to \infty} x_n = \LIM_{n \to \infty} x_n\) in \(\overline{X}\).
  \end{enumerate}
\end{ex}

\begin{proof}{(a)}
  Let \((x^{(n)})_{n = m}^\infty\), \((y^{(n)})_{n = m}^\infty\), \((z^{(n)})_{n = m}^\infty\) be Cauchy sequences in \((X, d)\).

  First suppose that \(\LIM_{n \to \infty} x^{(n)}\) is well-defined.
  Then we have
  \begin{align*}
             & \lim_{n \to \infty} d(x^{(n)}, x^{(n)}) = \lim_{n \to \infty} 0 = 0 &  & \by{ii:1.1.2}[a]       \\
    \implies & \LIM_{n \to \infty} x^{(n)} = \LIM_{n \to \infty} x^{(n)}           &  & \text{(by definition)}
  \end{align*}
  and thus the equality relation of \cref{ii:ex:1.4.8}(a) is reflexive.

  Next suppose that \(\LIM_{n \to \infty} x^{(n)}, \LIM_{n \to \infty} y^{(n)}\) are well-defined and \(\LIM_{n \to \infty} x^{(n)} = \LIM_{n \to \infty} y^{(n)}\).
  Then we have
  \begin{align*}
         & \LIM_{n \to \infty} x^{(n)} = \LIM_{n \to \infty} y^{(n)}                             \\
    \iff & \lim_{n \to \infty} d(x^{(n)}, y^{(n)}) = 0               &  & \text{(by definition)} \\
    \iff & \lim_{n \to \infty} d(y^{(n)}, x^{(n)}) = 0               &  & \by{ii:1.1.2}[c]       \\
    \iff & \LIM_{n \to \infty} y^{(n)} = \LIM_{n \to \infty} x^{(n)} &  & \text{(by definition)}
  \end{align*}
  and thus the equality relation of \cref{ii:ex:1.4.8}(a) is symmetry.

  Finally suppose that \(\LIM_{n \to \infty} x^{(n)}, \LIM_{n \to \infty} y^{(n)}, \LIM_{n \to \infty} z^{(n)}\) are well-defined.
  Suppose also that \(\LIM_{n \to \infty} x^{(n)} = \LIM_{n \to \infty} y^{(n)}\) and \(\LIM_{n \to \infty} y^{(n)} = \LIM_{n \to \infty} z^{(n)}\).
  Then we have
  \begin{align*}
             & \begin{dcases}
                 \LIM_{n \to \infty} x^{(n)} = \LIM_{n \to \infty} y^{(n)} \\
                 \LIM_{n \to \infty} y^{(n)} = \LIM_{n \to \infty} z^{(n)}
               \end{dcases}                                                            \\
    \implies & \begin{dcases}
                 \lim_{n \to \infty} d(x^{(n)}, y^{(n)}) = 0 \\
                 \lim_{n \to \infty} d(y^{(n)}, z^{(n)}) = 0
               \end{dcases}                                         &  & \text{(by definition)}                                     \\
    \implies & \lim_{n \to \infty} \big(d(x^{(n)}, y^{(n)}) + d(y^{(n)}, z^{(n)})\big) = 0                                          \\
    \implies & 0 \leq \lim_{n \to \infty} d(x^{(n)}, z^{(n)})                                                                       \\
             & \quad \leq \lim_{n \to \infty} \big(d(x^{(n)}, y^{(n)}) + d(y^{(n)}, z^{(n)})\big) = 0 &  & \by{ii:1.1.2}            \\
    \implies & \lim_{n \to \infty} d(x^{(n)}, z^{(n)}) = 0                                            &  & \text{(by squeeze test)} \\
    \implies & \LIM_{n \to \infty} x^{(n)} = \LIM_{n \to \infty} z^{(n)}                              &  & \text{(by definition)}
  \end{align*}
  and thus the equality relation of \cref{ii:ex:1.4.8}(a) is transitive.
\end{proof}

\begin{proof}{(b)}
  Let \((x^{(n)})_{n = m}^\infty\), \((y^{(n)})_{n = m}^\infty\), \((z^{(n)})_{n = m}^\infty\) be Cauchy sequences in \((X, d)\) with formal limits in \(\overline{X}\).
  We first show that the limit
  \[
    d_{\overline{X}}(\LIM_{n \to \infty} x^{(n)}, \LIM_{n \to \infty} y^{(n)}) \coloneqq \lim_{n \to \infty} d(x^{(n)}, y^{(n)})
  \]
  exists.
  Let \((a^{(n)})_{n = m}^\infty\) be the sequence \(a^{(n)} = d(x^{(n)}, y^{(n)})\).
  To show that the above limit exists, it will suffice to show that \((a^{(n)})_{n = m}^\infty\) converges in \(\R\) with respect to \(d_{l^1}|_{\R \times \R}\).
  Let \(N_1, N_2, j, k \in \N\).
  Since \((x^{(n)})_{n = m}^\infty\) is a Cauchy sequence in \((X, d)\), by \cref{ii:1.4.6} we know that
  \[
    \forall \varepsilon \in \R^+, \exists N_1 \geq m : \forall j, k \geq N_1, d(x^{(j)}, x^{(k)}) \leq \dfrac{\varepsilon}{2}.
  \]
  Similarly,
  \[
    \forall \varepsilon \in \R^+, \exists N_2 \geq m : \forall j, k \geq N_2, d(y^{(j)}, y^{(k)}) \leq \dfrac{\varepsilon}{2}.
  \]
  Let \(N = \max(N_1, N_2)\).
  Then by \cref{ii:1.1.2} we have
  \begin{align*}
     & \forall j, k \geq N, \abs{a_j - a_k}                                                             \\
     & = \abs{d(x^{(j)}, y^{(j)}) - d(x^{(k)}, y^{(k)})}                                                \\
     & = \abs{d(x^{(j)}, y^{(j)}) + d(y^{(j)}, x^{(k)}) - d(y^{(j)}, x^{(k)}) - d(x^{(k)}, y^{(k)})}    \\
     & \leq \abs{d(x^{(j)}, y^{(j)}) + d(y^{(j)}, x^{(k)}) + d(y^{(j)}, x^{(k)}) + d(x^{(k)}, y^{(k)})} \\
     & = d(x^{(j)}, y^{(j)}) + d(y^{(j)}, x^{(k)}) + d(y^{(j)}, x^{(k)}) + d(x^{(k)}, y^{(k)})          \\
     & \leq d(x^{(j)}, x^{(k)}) + d(y^{(j)}, y^{(k)})                                                   \\
     & \leq \varepsilon / 2 + \varepsilon / 2                                                           \\
     & = \varepsilon
  \end{align*}
  and thus \((a^{(n)})_{n = m}^\infty\) is a Cauchy sequence in \((\R, d_{l^1}|_{\R \times \R})\).
  Since \((\R, d_{l^1}|_{\R \times \R})\) is complete (see Theorem 6.4.18, Analysis I), we know that \((a^{(n)})_{n = m}^\infty\) converges in \(\R\) with respect to \(d_{l^1}|_{\R \times \R}\).

  Next we show that \(d_{\overline{X}}\) obeys the axiom of substitution.
  Suppose that \(\LIM_{n \to \infty} x^{(n)} = \LIM_{n \to \infty} z^{(n)}\) and \(\lim_{n \to \infty} d(x^{(n)}, y^{(n)})\) exists.
  Then we have
  \begin{align*}
             & \begin{dcases}
                 d(x^{(n)}, y^{(n)}) \leq d(x^{(n)}, z^{(n)}) + d(z^{(n)}, y^{(n)}) \\
                 d(z^{(n)}, y^{(n)}) \leq d(x^{(n)}, y^{(n)}) + d(x^{(n)}, z^{(n)})
               \end{dcases}                              &  & \by{ii:1.1.2}[c,d]                                                                                                  \\
    \implies & \begin{dcases}
                 d(x^{(n)}, y^{(n)}) - d(z^{(n)}, y^{(n)}) \leq d(x^{(n)}, z^{(n)}) \\
                 d(z^{(n)}, y^{(n)}) - d(x^{(n)}, y^{(n)}) \leq d(x^{(n)}, z^{(n)})
               \end{dcases}                                                                                                  \\
    \implies & 0 \leq \abs{d(x^{(n)}, y^{(n)}) - d(z^{(n)}, y^{(n)})} \leq d(x^{(n)}, z^{(n)})                                                                                     \\
    \implies & 0 = \lim_{n \to \infty} 0 \leq \lim_{n \to \infty} \abs{d(x^{(n)}, y^{(n)}) - d(z^{(n)}, y^{(n)})}                                                                  \\
             & \quad \leq \lim_{n \to \infty} d(x^{(n)}, z^{(n)}) = 0                                             &  & \by{ii:ex:1.4.8}[a]                                         \\
    \implies & \lim_{n \to \infty} \abs{d(x^{(n)}, y^{(n)}) - d(z^{(n)}, y^{(n)})} = 0                            &  & \text{(by squeeze test)}                                    \\
    \implies & \lim_{n \to \infty} \big(d(x^{(n)}, y^{(n)}) - d(z^{(n)}, y^{(n)})\big) = 0                                                                                         \\
    \implies & \lim_{n \to \infty} d(x^{(n)}, y^{(n)}) = \lim_{n \to \infty} d(z^{(n)}, y^{(n)})                  &  & \text{(\(\lim_{n \to \infty} d(x^{(n)}, y^{(n)})\) exists)}
  \end{align*}
  and thus \(d_{\overline{X}}(\LIM_{n \to \infty} x^{(n)}, \LIM_{n \to \infty} y^{(n)}) = d_{\overline{X}}(\LIM_{n \to \infty} z^{(n)}, \LIM_{n \to \infty} y^{(n)})\).

  Now we show that \((\overline{X}, d_{\overline{X}})\) is a metric space.
  For identify:
  By \cref{ii:1.1.2}(a) we have
  \[
    d_{\overline{X}}(\LIM_{n \to \infty} x^{(n)}, \LIM_{n \to \infty} x^{(n)}) = \lim_{n \to \infty} d(x^{(n)}, x^{(n)}) = 0.
  \]
  For positivity:
  If \(\LIM_{n \to \infty} x^{(n)} \neq \LIM_{n \to \infty} y^{(n)}\), then by \cref{ii:ex:1.4.8}(a) we have
  \[
    d_{\overline{X}}(\LIM_{n \to \infty} x^{(n)}, \LIM_{n \to \infty} y^{(n)}) = \lim_{n \to \infty} d(x^{(n)}, y^{(n)}) \neq 0.
  \]
  For symmetry:
  We have
  \begin{align*}
     & d_{\overline{X}}(\LIM_{n \to \infty} x^{(n)}, \LIM_{n \to \infty} y^{(n)})                                \\
     & = \lim_{n \to \infty} d(x^{(n)}, y^{(n)})                                     &  & \text{(by definition)} \\
     & = \lim_{n \to \infty} d(y^{(n)}, x^{(n)})                                     &  & \by{ii:1.1.2}[c]       \\
     & = d_{\overline{X}}(\LIM_{n \to \infty} y^{(n)}, \LIM_{n \to \infty} x^{(n)}). &  & \text{(by definition)}
  \end{align*}
  For transitive:
  We have
  \begin{align*}
     & d_{\overline{X}}(\LIM_{n \to \infty} x^{(n)}, \LIM_{n \to \infty} z^{(n)})                                                                                 \\
     & = \lim_{n \to \infty} d(x^{(n)}, z^{(n)})                                                                                                                  \\
     & \leq \lim_{n \to \infty} \big(d(x^{(n)}, y^{(n)}) + d(y^{(n)}, z^{(n)})\big)                                                                               \\
     & = \lim_{n \to \infty} d(x^{(n)}, y^{(n)}) + \lim_{n \to \infty} d(y^{(n)}, z^{(n)})                                                                        \\
     & = d_{\overline{X}}(\LIM_{n \to \infty} x^{(n)}, \LIM_{n \to \infty} y^{(n)}) + d_{\overline{X}}(\LIM_{n \to \infty} y^{(n)}, \LIM_{n \to \infty} z^{(n)}).
  \end{align*}
  Thus, by \cref{ii:1.1.2} \((\overline{X}, d_{\overline{X}})\) is a metric space.
\end{proof}

\begin{proof}{(c)}
  Let \(I, J, K, N, k_1, k_2, n_1, n_2 \in \Z^+\).
  Let \((a^{(n)})_{n = m}^\infty\) be arbitrary Cauchy sequence in \((\overline{X}, d_{\overline{X}})\).
  Since \((a^{(n)})_{n = m}^\infty\) is a Cauchy sequence in \((\overline{X}, d_{\overline{X}})\), by \cref{ii:1.4.6} we know that
  \begin{align*}
             & \forall \varepsilon \in \R^+, \exists N \geq m : \forall n_1, n_2 \geq N,                                                               \\
             & d_{\overline{X}}(a^{(n_1)}, a^{(n_2)}) \leq \dfrac{\varepsilon}{4}                                                                      \\
    \implies & \forall \varepsilon \in \R^+, \exists N \geq m : \forall n_1, n_2 \geq N,                                                               \\
             & d_{\overline{X}}(\LIM_{k \to \infty} a_k^{(n_1)}, \LIM_{k \to \infty} a_k^{(n_2)}) \leq \dfrac{\varepsilon}{4} &  & \by{ii:ex:1.4.8}[a] \\
    \implies & \forall \varepsilon \in \R^+, \exists N \geq m : \forall n_1, n_2 \geq N,                                                               \\
             & \lim_{k \to \infty} d(a_k^{(n_1)}, a_k^{(n_2)}) \leq \dfrac{\varepsilon}{4}.                                   &  & \by{ii:ex:1.4.8}[b]
  \end{align*}
  Since the choice of \(N\) depends on \(\varepsilon\), we denote such \(N\) as \(N_\varepsilon\).
  We can use axiom of choice to fix \(N_\varepsilon\) for each \(\varepsilon \in \R^+\), and we rewrite the above statement as
  \[
    \forall \varepsilon \in \R^+, \forall n_1, n_2 \geq N_\varepsilon, \lim_{k \to \infty} d(a_k^{(n_1)}, a_k^{(n_2)}) \leq \dfrac{\varepsilon}{4}.
  \]
  Let \(L = \lim_{k \to \infty} d(a_k^{(n_1)}, a_k^{(n_2)})\).
  Since \(\Big(d(a_k^{(n_1)}, a_k^{(n_2)})\Big)_{k = 1}^\infty\) is a sequence in \(\R\) and converges to \(L\) with respect to \(d_{l^1}|_{\R \times \R}\), we have
  \begin{align*}
             & \exists I \geq 1 : \forall k \geq I, \abs{d(a_k^{(n_1)}, a_k^{(n_2)}) - L} \leq \dfrac{\varepsilon}{4}                                                                                              \\
    \implies & \exists I \geq 1 : \forall k \geq I, -\dfrac{\varepsilon}{4} \leq d(a_k^{(n_1)}, a_k^{(n_2)}) - L \leq \dfrac{\varepsilon}{4}                                                                       \\
    \implies & \exists I \geq 1 : \forall k \geq I, 0 \leq d(a_k^{(n_1)}, a_k^{(n_2)}) - L + \dfrac{\varepsilon}{4} \leq \dfrac{\varepsilon}{2}                                                                    \\
    \implies & \exists I \geq 1 : \forall k \geq I, 0 \leq d(a_k^{(n_1)}, a_k^{(n_2)}) \leq d(a_k^{(n_1)}, a_k^{(n_2)}) - L + \dfrac{\varepsilon}{4} \leq \dfrac{\varepsilon}{2} & (L \leq \dfrac{\varepsilon}{4})
  \end{align*}
  Since such \(I\) depends on the choice of \(\varepsilon\), we denote such \(I\) as \(I_\varepsilon\).
  Again we can use axiom of choice to fix \(I_\varepsilon\) for each \(\varepsilon \in \R^+\), and we rewrite the above statement as
  \[
    \forall \varepsilon \in \R^+, \forall n_1, n_2 \geq N_\varepsilon, \forall k \geq I_\varepsilon, d(a_k^{(n_1)}, a_k^{(n_2)}) \leq \dfrac{\varepsilon}{2}.
  \]
  If we let \(M_\varepsilon = \max(N_\varepsilon, I_\varepsilon)\), then we can further reduce the statement as
  \[
    \forall \varepsilon \in \R^+, \forall n_1, n_2, k \geq M_\varepsilon, d(a_k^{(n_1)}, a_k^{(n_2)}) \leq \dfrac{\varepsilon}{2}.
  \]

  Since \(a^{(n)} \in \overline{X}\) for each \(n \geq m\), by \cref{ii:ex:1.4.8}(b) we know that there exists a Cauchy sequence \((a_k^{(n)})_{k = 1}^\infty\) in \((X, d)\) such that \(\LIM_{k \to \infty} a_k^{(n)} = a^{(n)}\) for each \(n \geq m\).
  By \cref{ii:1.4.6} we know that
  \[
    \forall \varepsilon \in \R^+, \exists J \geq 1 : \forall k_1, k_2 \geq J, d(a_{k_1}^{(n)}, a_{k_2}^{(n)}) \leq \dfrac{\varepsilon}{2}.
  \]
  Since such \(J\) depends on the choice of \(n\) and \(\varepsilon\), we denote such \(J\) as \(J_\varepsilon^{(n)}\).
  We can use axiom of choice to fix \(J_\varepsilon^{(n)}\) for each \(\varepsilon \in \R^+\) and for each \(n \geq m\), and we rewrite the above statement as
  \[
    \forall \varepsilon \in \R^+, \forall k_1, k_2 \geq J_\varepsilon^{(n)}, d(a_{k_1}^{(n)}, a_{k_2}^{(n)}) \leq \dfrac{\varepsilon}{2}.
  \]

  We now define a sequence \((b_k)_{k = 1}^\infty\) in \((X, d)\) by setting \(b_k = a_k^{(m + k - 1)}\) for all \(k \geq 1\).
  Informally, \((b_k)_{k = 1}^\infty\) is consist of diagonal elements in \(\big((a_k^{(n)})_{k = 1}^\infty\big)_{n = m}^\infty\).
  We claim that \((b_k)_{k = 1}^\infty\) is a Cauchy sequence in \((X, d)\).
  By \cref{ii:1.4.6} it suffices to show that
  \[
    \forall \varepsilon \in \R^+, \exists K \geq 1 : \forall k_1, k_2 \geq K, d(b_{k_1}, b_{k_2}) \leq \varepsilon.
  \]
  For each \(\varepsilon \in \R^+\), we have
  \begin{align*}
     & \forall k_1, k_2 \geq M_\varepsilon, d(b_{k_1}, b_{k_2})                                                                                                                            \\
     & = d\Big(a_{k_1}^{(m + k_1 - 1)}, a_{k_2}^{(m + k_2 - 1)}\Big)                                                                                                                       \\
     & \leq d\Big(a_{k_1}^{(m + k_1 - 1)}, a_{k_2}^{(m + k_1 - 1)}\Big) + d\Big(a_{k_2}^{(m + k_1 - 1)}, a_{k_2}^{(m + k_2 - 1)}\Big) &  & \by{ii:1.1.2}[d]                                \\
     & \leq d\Big(a_{k_1}^{(m + k_1 - 1)}, a_{k_2}^{(m + k_1 - 1)}\Big) + \dfrac{\varepsilon}{2}.                                     &  & \text{(by the definition of \(M_\varepsilon\))}
  \end{align*}
  By choosing \(K = \max(M_\varepsilon, J_\varepsilon^{(M_\varepsilon)})\) we have
  \begin{align*}
     & \forall k_1, k_2 \geq K, d(b_{k_1}, b_{k_2})                                                                                                                     \\
     & \leq d\Big(a_{k_1}^{(m + k_1 - 1)}, a_{k_2}^{(m + k_1 - 1)}\Big) + \dfrac{\varepsilon}{2}                                                                        \\
     & \leq \dfrac{\varepsilon}{2} + \dfrac{\varepsilon}{2}                                      &  & \text{(by the definition of \(J_\varepsilon^{(M_\varepsilon)}\))} \\
     & = \varepsilon.
  \end{align*}
  Since \(\varepsilon\) was arbitrary, we have showed that
  \[
    \forall \varepsilon \in \R^+, \exists K \geq 1 : \forall k_1, k_2 \geq K, d(b_{k_1}, b_{k_2}) \leq \varepsilon.
  \]
  Thus, by \cref{ii:1.4.6} \((b_k)_{k = 1}^\infty\) is a Cauchy sequence in \((X, d)\).

  Now we show that \((a^{(n)})_{n = m}^\infty\) converges in \((\overline{X}, d_{\overline{X}})\).
  From the proof above we know that \((b_k)_{k = 1}^\infty\) is a Cauchy sequence in \((X, d)\), so \(\LIM_{k \to \infty} b_k \in \overline{X}\).
  We claim that \((a^{(n)})_{n = m}^\infty\) converges to \(\LIM_{k \to \infty} b_k\) with respect to \(d_{\overline{X}}\).
  It suffices to show that
  \begin{align*}
         & \lim_{n \to \infty} d_{\overline{X}}\Big(a^{(n)}, \LIM_{k \to \infty} b_k\Big) = 0                       &  & \by{ii:1.1.14}      \\
    \iff & \lim_{n \to \infty} d_{\overline{X}}\Big(\LIM_{k \to \infty} a_k^{(n)}, \LIM_{k \to \infty} b_k\Big) = 0 &  & \by{ii:ex:1.4.8}[a] \\
    \iff & \lim_{n \to \infty} \big(\lim_{k \to \infty} d(a_k^{(n)}, b_k)\big) = 0                                  &  & \by{ii:ex:1.4.8}[b] \\
    \iff & \forall \varepsilon \in \R^+, \exists N \geq m : \forall n \geq m,                                                                \\
         & \abs{\lim_{k \to \infty} d(a_k^{(n)}, b_k) - 0} \leq \varepsilon.
  \end{align*}
  Since
  \begin{align*}
     & \forall \varepsilon \in \R^+, \exists M_\varepsilon \geq 1 : \forall k \geq M_\varepsilon, d(a_k^{(M_\varepsilon)}, b_k)                                                      \\
     & = d\Big(a_k^{(M_\varepsilon)}, a_k^{(m + k - 1)}\Big)                                                                                                                         \\
     & \leq \dfrac{\varepsilon}{2}                                                                                              &  & \text{(by the definition of \(M_\varepsilon\))} \\
     & < \varepsilon,
  \end{align*}
  we know that \(\lim_{k \to \infty} d(a_k^{(M_\varepsilon)}, b_k)\) exists.
  Since
  \begin{align*}
             & \forall n, k \geq M_\varepsilon, 0 \leq d(a_k^{(n)}, b_k) \leq \dfrac{\varepsilon}{2} < \varepsilon &  & \text{(by the definition of \(M_\varepsilon\))} \\
    \implies & \forall n \geq M_\varepsilon, 0 \leq \lim_{k \to \infty} d(a_k^{(n)}, b_k) \leq \varepsilon         &  & \text{(by comparison test)}                     \\
    \implies & \forall n \geq M_\varepsilon, \abs{\lim_{k \to \infty} d(a_k^{(n)}, b_k) - 0} \leq \varepsilon.
  \end{align*}
  by setting \(N = M_\varepsilon\) we are done.

  Since for arbitrary Cauchy sequence \((a^{(n)})_{n = m}^\infty\) in \((\overline{X}, d_{\overline{X}})\), \((a^{(n)})_{n = m}^\infty\) converges in \(\overline{X}\) with respect to \(d_{\overline{X}}\), by \cref{ii:1.4.10} we know that \((\overline{X}, d_{\overline{X}})\) is complete.
\end{proof}

\begin{proof}{(d)}
  Since for any \(x, y \in X\), we have
  \begin{align*}
         & x = y                                                                                       \\
    \iff & d(x, y) = 0                                                        &  & \by{ii:1.1.2}[a]    \\
    \iff & \lim_{n \to \infty} d(x, y) = 0                                                             \\
    \iff & d_{\overline{X}}(\LIM_{n \to \infty} x, \LIM_{n \to \infty} y) = 0 &  & \by{ii:ex:1.4.8}[b] \\
    \iff & \LIM_{n \to \infty} x = \LIM_{n \to \infty} y.                     &  & \by{ii:1.1.2}[a]
  \end{align*}
  Thus
  \begin{align*}
    d_{\overline{X}}(x, y) & = d_{\overline{X}}(\LIM_{n \to \infty} x, \LIM_{n \to \infty} y)                          \\
                           & = \lim_{n \to \infty} d(x, y)                                    &  & \by{ii:ex:1.4.8}[b] \\
                           & = d(x, y).
  \end{align*}
\end{proof}

\begin{proof}{(e)}
  From \cref{ii:ex:1.4.8}(d) have \(d = d_{\overline{X}}|_{X \times X}\).
  Let \(Y\) be the closure of \(X\) in \((\overline{X}, d_{\overline{X}})\).
  We want to show that \(Y = \overline{X}\).
  By \cref{ii:1.2.9} we know that \(Y \subseteq \overline{X}\).
  Thus, we only need to show that \(\overline{X} \subseteq Y\).

  Let \(x_0 \in \overline{X}\).
  By \cref{ii:ex:1.4.8}(b) there exists a Cauchy sequence \((a_n)_{n = 1}^\infty\) in \((X, d_{\overline{X}})\) such that \(\LIM_{n \to \infty} a_n = x_0\).
  Since \((\overline{X}, d_{\overline{X}})\) is complete, by \cref{ii:1.4.10} we know that \((a_n)_{n = 1}^\infty\) converges in \(\overline{X}\) with respect to \(d_{\overline{X}}\).
  But we know that \((a_n)_{n = 1}^\infty\) is a Cauchy sequence in \((X, d_{\overline{X}})\), thus by \cref{ii:1.2.10}(c) \(x_0\) is an adherent point of \(X\) in \((\overline{X}, d_{\overline{X}})\) and \(x_0 \in Y\).
  Since \(x_0\) was arbitrary, we thus have \(\overline{X} \subseteq Y\).
\end{proof}

\begin{proof}{(f)}
  By \cref{ii:ex:1.4.8}(d) we know that if \((x_n)_{n = 1}^\infty\) is a Cauchy sequence in \((X, d)\), then \((x_n)_{n = 1}^\infty\) is also a Cauchy sequence in \((\overline{X}, d_{\overline{X}})\).
  Thus, by \cref{ii:ex:1.4.8}(c)(e) we have \(\lim_{n \to \infty} x_n = \LIM_{n \to \infty} x_n\).
\end{proof}

\section{Compact metric spaces}\label{sec:1.5}

\begin{defn}[Compactness]\label{1.5.1}
  A metric space \((X, d)\) is said to be \emph{compact} iff every sequence in \((X, d)\) has at least one convergent subsequence.
  A subset \(Y\) of a metric space \(X\) is said to be \emph{compact} if the subspace \((Y, d|_{Y \times Y})\) is compact.
\end{defn}

\begin{rmk}\label{1.5.2}
  The notion of a set \(Y\) being compact is \emph{intrinsic}, in the sense that it only depends on the metric function \(d|_{Y \times Y}\) restricted to \(Y\), and not on the choice of the ambient space \(X\).
  The notions of completeness in \cref{1.4.10}, and of boundedness in \cref{1.5.3}, are also intrinsic, but the notions of open and closed are not
  (see the discussion in \cref{sec:1.3}).
\end{rmk}

\begin{note}
  The notion of a set \(Y\) being compact only depends on the metric function \(d|_{Y \times Y}\) but not ambient space \(X\) since the elements of a sequence in \(Y\) stays the same no matter which spaces \(Y\) is subset to.
  But the notion of a set being open or closed depends on the definition of a metric ball, which may be different when given different ambient spaces.
\end{note}

\begin{note}
  Heine-Borel theorem shows that in the real line \(\R\) with the usual metric, every closed and bounded set is compact, and conversely every compact set is closed and bounded.
\end{note}

\begin{defn}[Bounded sets]\label{1.5.3}
  Let \((X, d)\) be a metric space, and let \(Y\) be a subset of \(X\).
  We say that \(Y\) is \emph{bounded} iff for every \(x \in X\) there exists a ball \(B(x, r)\) in \(X\) which contains \(Y\).
  We call \((X, d)\) bounded if \(X\) is bounded.
\end{defn}

\begin{rmk}\label{1.5.4}
  \cref{1.5.3} is compatible with the definition of a bounded set in \((\R, d_{l^1}|_{\R \times \R})\).
\end{rmk}

\begin{prop}\label{1.5.5}
  Let \((X, d)\) be a compact metric space.
  Then \((X, d)\) is both complete and bounded.
\end{prop}

\begin{proof}
  We first show that \((X, d)\) is complete.
  Let \((a_n)_{n = 1}^\infty\) be a Cauchy sequence in \((X, d)\).
  Since \((X, d)\) is compact, by \cref{1.5.1} we know that there exists a subsequence of \((a_n)_{n = 1}^\infty\) which converges to some \(x_0 \in X\) with respect to \(d\).
  Since \((a_n)_{n = 1}^\infty\) is a Cauchy sequence in \((X, d)\), by \cref{1.4.9} we know that \((a_n)_{n = 1}^\infty\) must converge to \(x_0\) with respect to \(d\).
  Since \((a_n)_{n = 1}^\infty\) is arbitrary, by \cref{1.4.10} we know that \((X, d)\) is complete.

  Now we show that \((X, d)\) is bounded by contradiction.
  Suppose for sake of contradiction that \((X, d)\) is not bounded.
  Then by \cref{1.5.3} we have
  \begin{align*}
             & \lnot\big(\forall x_0 \in X, \exists r \in \R^+ : X \subseteq B_{(X, d)}(x_0, r)\big)            \\
    \implies & \exists x_0 \in X : \forall r \in \R^+, X \not\subseteq B_{(X, d)}(x_0, r)                       \\
    \implies & \exists x_0 \in X : \forall r \in \R^+, \big(X \setminus B_{(X, d)}(x_0, r)\big) \neq \emptyset  \\
    \implies & \exists x_0 \in X : \forall n \in \Z^+, \big(X \setminus B_{(X, d)}(x_0, n)\big) \neq \emptyset.
  \end{align*}
  If \(X = \emptyset\), then \((\emptyset, d)\) is bound since \(x_0 \notin \emptyset\).
  So we only considered the cases \(X \neq \emptyset\).
  Let \((a_n)_{n = 1}^\infty\) be the sequence where \(a_n \in X \setminus B_{(X, d)}(x_0, n)\) for all \(n \in \Z^+\).
  Note that such sequence is well-defined by axiom of choice.
  Since \(a_n \in X \setminus B_{(X, d)}(x_0, n)\), by \cref{1.2.1} we know that \(d(a_n, x_0) \geq n\) for all \(n \in \Z^+\).
  Since \((X, d)\) is compact, by \cref{1.5.1} there exists a subsequence \((a_{n_j})_{j = 1}^\infty\) of \((a_n)_{n = 1}^\infty\) such that \((a_{n_j})_{j = 1}^\infty\) converges in \(X\).
  Let \(\lim_{j \to \infty} d(a_{n_j}, L) = 0\) for some \(L \in X\).
  By \cref{1.1.14} we know that
  \[
    \forall \varepsilon \in \R^+, \exists J \in \Z^+ : \forall j \geq J, d(a_{n_j}, L) \leq \varepsilon.
  \]
  In particular,
  \[
    \exists J \in \Z^+ : \forall j \geq J, d(a_{n_j}, L) \leq 1.
  \]
  Now we fix such \(J\) and let \(i = \max\big(J + 1, \ceil{d(L, x_0)} + 2\big)\).
  Then we have
  \begin{align*}
    d(a_{n_i}, x_0) & \leq d(a_{n_i}, L) + d(L, x_0) &         & \text{(by \cref{1.1.2}(d))} \\
                    & \leq 1 + d(L, x_0)             & (i > J)
  \end{align*}
  and
  \begin{align*}
    d(a_{n_i}, x_0) & \geq n_i            &  & \text{(by the definition of \(a_{n_i}\))} \\
                    & \geq i                                                             \\
                    & \geq d(L, x_0) + 2. &  & \text{(by the definition of \(i\))}
  \end{align*}
  But this means \(d(L, x_0) + 2 \leq d(L, x_0) + 1\), a contradiction.
  Thus \((X, d)\) is bounded.
\end{proof}

\begin{cor}[Compact sets are closed and bounded]\label{1.5.6}
  Let \((X, d)\) be a metric space, and let \(Y\) be a compact subset of \(X\).
  Then \(Y\) is closed and bounded.
\end{cor}

\begin{proof}
  Since \((Y, d|_{Y \times Y})\) is compact, by \cref{1.5.5} we know that \((Y, d)\) is complete and bounded.
  Thus by \cref{1.4.12}(a) we know that \(Y\) is closed in \((X, d)\).
\end{proof}

\begin{thm}[Heine-Borel theorem]\label{1.5.7}
  Let \((\R^n, d)\) be a Euclidean space with either the Euclidean metric, the taxicab metric, or the supnorm metric.
  Let \(E\) be a subset of \(\R^n\).
  Then \(E\) is compact iff it is closed and bounded.
\end{thm}

\begin{proof}
  We first show that for any \(E \subseteq \R^n\), \(E\) is closed and bounded in \((\R^n, d_{l^1}|_{\R^n \times \R^n})\) iff \((E, d_{l^1}|_{E \times E})\) is compact.
  By \cref{ex:1.1.7} we know that \((\R^n, d_{l^1}|_{\R^n \times \R^n})\) is a metric space, and by \cref{1.5.6} we know that if \((E, d_{l^1}|_{E \times E})\) is compact, then \(E\) is closed and bounded in \((\R^n, d_{l^1}|_{\R^n \times \R^n})\).
  So we only need to show that if \(E\) is closed and bounded in \((\R^n, d_{l^1}|_{\R^n \times \R^n})\), then \((E, d_{l^1}|_{E \times E})\) is compact.
  Suppose that \(E \subseteq \R^n\), \(E\) is closed and bounded in \((\R^n, d_{l^1}|_{\R^n \times \R^n})\).
  Since \(E \subseteq \R^n\), we know that for every \(x \in E\), \(x\) is in the form \(x = (x_1, \dots, x_n) = (x_i)_{i = 1}^n \in \R^n\).
  Let \(I_n = \{i \in \N : 1 \leq i \leq n\}\).
  For each \(i \in I_n\), let \(E_i\) be the set
  \[
    E_i = \{y \in \R | \exists x \in E : x_i = y\},
  \]
  i.e., \(E_i\) is the collection of \(i^{\text{th}}\) coordinate of all element \(x \in E\).
  We claim that for every \(i \in I_n\), \(E_i\) is a subset of some closed interval and thus \(E_i\) is bounded in \((\R, d_{l^1}|_{\R \times \R})\).
  This is true since
  \begin{align*}
             & E \text{ is bounded in } (\R^n, d_{l^1}|_{\R^n \times \R^n})                                                                  \\
    \implies & \forall y \in \R^n, \exists r \in \R^+ : E \subseteq B_{(\R^n, d_{l^1}|_{\R^n \times \R^n})}(y, r)            &  & \by{1.5.3} \\
    \implies & \forall y \in \R^n, \exists r \in \R^+ : \forall x \in E, x \in B_{(\R^n, d_{l^1}|_{\R^n \times \R^n})}(y, r)                 \\
    \implies & \forall y \in \R^n, \exists r \in \R^+ : \forall x \in E,                                                                     \\
             & d_{l^1}|_{\R^n \times \R^n}(x, y) = \sum_{i = 1}^n \abs{x_i - y_i} < r                                        &  & \by{1.2.1} \\
    \implies & \forall y \in \R^n, \exists r \in \R^+ : \forall x \in E, \forall i \in I_n,                                                  \\
             & d_{l^1}|_{\R \times \R}(x_i, y_i) = \abs{x_i - y_i} < r                                                                       \\
    \implies & \forall y \in \R^n, \exists r \in \R^+ : \forall x \in E, \forall i \in I_n,                                                  \\
             & x_i \in (y_i - r, y_i + r)                                                                                                    \\
    \implies & \forall y \in \R^n, \exists r \in \R^+ : \forall i \in I_n,                                                                   \\
             & E_i \subseteq (y_i - r, y_i + r) \subseteq [y_i - r, y_i + r]                                                                 \\
    \implies & \forall y \in \R^n, \exists r \in \R^+ : \forall i \in I_n,                                                                   \\
             & E_i \subseteq B_{(\R, d_{l^1}|_{\R \times \R})}(y_i, r)                                                       &  & \by{1.2.1} \\
    \implies & \forall i \in I_n, E_i \text{ is bounded in } (\R, d_{l^1}|_{\R \times \R}).                                  &  & \by{1.5.3}
  \end{align*}

  Let \(P(n)\) be the statement ``If \(F \subseteq \R^n\) such that for every \(i \in I_n\), \(F_i\) is bounded in \((\R, d_{l^1}|_{\R \times \R})\) and \(F_i \subseteq C_i\) for some closed interval \(C_i \subseteq \R\), then for any sequence in \(F\) there exists a subsequence which converges in \(\R^n\) with respect to \(d_{l^1}|_{\R^n \times \R^n}\)''.
  We use induction on \(n\) to show that \(P(n)\) is true for all \(n \in \Z^+\).

  For \(n = 1\), by hypothesis we have \(F = F_1\) is bounded in \((\R, d_{l^1}|_{\R \times \R})\) and \(F_1 \subseteq C_1\) for some closed interval \(C_1 \subseteq \R\).
  By Heine-Borel theorem on real line (Theorem 9.1.24 in Analysis I) we know that for every sequence \((a^{(k)})_{k = 1}^\infty\) in \(F\), there exists a subsequence \((a^{(k_j)})_{j = 1}^\infty\) which converges in \(C_1 \subseteq \R\) with respect to \(d_{l^1}|_{\R \times \R}\).
  Thus the base case holds.

  Suppose inductively that \(P(n)\) is true for some \(n \geq 1\).
  Then we need to show that \(P(n + 1)\) is true.
  Let \(F \subseteq \R^{n + 1}\) such that for every \(i \in I_{n + 1}\), \(F_i\) is bounded in \((\R, d_{l^1}|_{\R \times \R})\) and \(F_i \subseteq C_i\) for some closed interval \(C_i \in \R\).
  Let \((a^{(k)})_{k = 1}^\infty\) be arbitrary sequence in \(F\).
  We define \((b^{(k)})_{k = 1}^\infty\) by setting \(b^{(k)} = (a_1^{(k)}, \dots, a_n^{(k)})\) for each \(k \geq 1\), i.e., \(b^{(k)}\) is the first \(n\) coordinates of \(a^{(k)}\).
  Since for all \(k \geq 1\), \(b^{(k)} \in \R^n\) and \(b_i^{(k)} \in F_i\) for all \(i \in I_n\), by induction hypothesis there exists a subsequence \((b^{(k_j)})_{j = 1}^\infty\) which converges in \(\R^n\) with respect to \(d_{l^1}|_{\R^n \times \R^n}\).
  Since \((a_{n + 1}^{(k_j)})_{j = 1}^\infty\) is in \(F_{n + 1}\) and \(F_{n + 1} \subseteq C_{n + 1}\) for some closed interval \(C_{n + 1} \subseteq \R\), by Heine-Borel theorem on real line (Theorem 9.1.24 in Analysis I) we know that there exists a subsequence \((a^{(k_{j_p})})_{p = 1}^\infty\) which converges in \(C_{n + 1}\) with respect to \(d_{l^1}|_{\R \times \R}\).
  But by \cref{1.4.3} we know that every subsequence of \((b^{(k_j)})_{j = 1}^\infty\) also converges in \(\R^n\) with respect to \(d_{l^1}|_{\R^n \times \R^n}\).
  In particular, \((b^{k_{j_p}})_{p = 1}^\infty\) converges in \(\R^n\) with respect to \(d_{l^1}|_{\R^n \times \R^n}\).
  Thus by \cref{1.1.18}(b)(d) we know that \((a^{(k_{j_p})})_{p = 1}^\infty\) converges in \(\R^{n + 1}\) with respect to \(d_{l^1}|_{\R^{n + 1} \times \R^{n + 1}}\), and this closes the induction.

  From the proof above we know that if \(E \subseteq \R^n\) such that \(E\) is closed and bounded in \((\R^n, d_{l^1}|_{\R^n \times \R^n})\), then for every \(i \in I_n\), \(E_i\) is bounded in \((\R, d_{l^1}|_{\R \times \R})\) and \(E_i \subseteq C_i\) for some closed interval \(C_i \subseteq \R\).
  Thus we know that for every sequence in \(E\) there exists a subsequence which converges in \(\R^n\) with respect to \(d_{l^1}|_{\R^n \times \R^n}\).
  Since \(E\) is closed in \((\R^n, d_{l^1}|_{\R^n \times \R^n})\), by \cref{1.2.15}(b) we know that such subsequence must converges in \(E\) with respect to \(d_{l^1}|_{\R^n \times \R^n}\).
  Thus by \cref{1.5.1} \((E, d_{l^1}|_{E \times E})\) is compact.

  Since every sequence in \(E\) has a subsequence which converges in \(E\) with respect to \(d_{l^1}|_{E \times E}\), by \cref{1.1.18} we know that such subsequence also converges with respect to \(d_{l^2}|_{E \times E}\) and \(d_{l^\infty}|_{E \times E}\).
  Thus \((E, d_{l^2}|_{E \times E})\) and \((E, d_{l^\infty}|_{E \times E})\) are compact iff \((E, d_{l^1}|_{\R^n \times \R^n})\) is compact.
\end{proof}

\begin{note}
  The Heine-Borel theorem is not true for more general metric spaces.
  However, a version of the Heine-Borel theorem is available if one is willing to replace closedness with the stronger notion of completeness, and boundedness with the stronger notion of \emph{total boundedness}.
\end{note}

\begin{note}
  One can characterize compactness topologically via \cref{1.5.8}:
  every open cover of a compact set has a finite subcover.
\end{note}

\begin{thm}\label{1.5.8}
  Let \((X, d)\) be a metric space, and let \(Y\) be a compact subset of \(X\).
  Let \((V_{\alpha})_{\alpha \in A}\) be a collection of open sets in \(X\), and suppose that
  \[
    Y \subseteq \bigcup_{\alpha \in A} V_{\alpha}.
  \]
  (i.e., the collection \((V_{\alpha})_{\alpha \in A}\) \emph{covers} \(Y\)).
  Then there exists a \emph{finite} subset \(F\) of \(A\) such that
  \[
    Y \subseteq \bigcup_{\alpha \in F} V_{\alpha}.
  \]
\end{thm}

\begin{proof}
  We assume for sake of contradiction that there does not exist any finite subset \(F\) of \(A\) for which \(Y \subseteq \bigcup_{\alpha \in F} V_{\alpha}\).

  Let \(y\) be any element of \(Y\).
  Then \(y\) must lie in at least one of the sets \(V_{\alpha}\).
  Since each \(V_{\alpha}\) is open in \((X, d)\), by \cref{1.2.15}(a) there must therefore be an \(r > 0\) such that \(B_{(X, d)}(y, r) \subseteq V_{\alpha}\).
  Now let \(r(y)\) denote the quantity
  \[
    r(y) \coloneqq \sup\big\{r \in (0, \infty) : B_{(X, d)}(y, r) \subseteq V_{\alpha} \text{ for some } \alpha \in A\big\}.
  \]
  By the above discussion, we know that \(r(y) > 0\) for all \(y \in Y\).
  Now, let \(r_0\) denote the quantity
  \[
    r_0 \coloneqq \inf\big\{r(y) : y \in Y\big\}.
  \]
  Since \(r(y) > 0\) for all \(y \in Y\), we have \(r_0 \geq 0\).
  There are three cases: \(r_0 = 0\), \(0 < r_0 < \infty\) and \(r_0 = \infty\).
  \begin{itemize}
    \item Case 1:
          \(r_0 = 0\).
          Then for every integer \(n \geq 1\), there is at least one point \(y\) in \(Y\) such that \(r(y) < 1 / n\) (otherwise the infimum cannot be \(0\)).
          We thus choose, for each \(n \geq 1\), a point \(y^{(n)}\) in \(Y\) such that \(r(y^{(n)}) < 1 / n\)
          (we can do this because of the axiom of choice, see Proposition 8.4.7 in Analysis I).
          In particular we have \(\lim_{n \to \infty} r(y^{(n)}) = 0\), by the squeeze test.
          The sequence \((y^{(n)})_{n = 1}^\infty\) is a sequence in \(Y\);
          since \((Y, d|_{Y \times Y})\) is compact, we can thus find a subsequence \((y^{(n_j)})_{j = 1}^\infty\) which converges to a point \(y_0 \in Y\) with respect to \(d|_{Y \times Y}\).

          As before, we know that there exists some \(\alpha \in A\) such that \(y_0 \in V_{\alpha}\), and hence (since \(V_{\alpha}\) is open in \((X, d)\)) there exists some \(\varepsilon > 0\) such that \(B_{(X, d)}(y_0, \varepsilon) \subseteq V_{\alpha}\).
          Since \((y^{(n_j)})_{j = 1}^\infty\) converges to \(y_0\), there must exist an \(N \geq 1\) such that \(y^{(n_j)} \in B_{(X, d)}(y_0, \varepsilon / 2)\) for all \(n_j \geq N\).
          In particular, by the triangle inequality we have \(B_{(X, d)}(y^{(n_j)}, \varepsilon / 2) \subseteq B_{(X, d)}(y_0, \varepsilon)\), and thus \(B_{(X, d)}(y^{(n_j)}, \varepsilon / 2) \subseteq V_{\alpha}\).
          By definition of \(r(y^{(n_j)})\), this implies that \(r(y^{(n_j)}) \geq \varepsilon / 2\) for all \(n_j \geq N\).
          But this contradicts the fact that \(\lim_{n \to \infty} r(y^{(n)}) = 0\).
    \item Case 2:
          \(0 < r_0 < \infty\).
          In this case we now have \(r(y) > r_0 / 2\) for all \(y \in Y\).
          This implies that for every \(y \in Y\) there exists an \(\alpha \in A\) such that \(B_{(X, d)}(y, r_0 / 2) \subseteq V_{\alpha}\) (by the definition of \(r(y)\)).

          We now construct a sequence \(y^{(1)}, y^{(2)}, \dots\) by the following recursive procedure.
          We let \(y^{(1)}\) be any point in \(Y\).
          The ball \(B_{(X, d)}(y^{(1)}, r_0 / 2)\) is contained in one of the \(V_{\alpha}\) and thus cannot cover all of \(Y\), since we would then obtain a finite cover, a contradiction.
          Thus there exists a point \(y^{(2)}\) which does not lie in \(B_{(X, d)}(y^{(1)}, r_0 / 2)\), so in particular \(d(y^{(2)}, y^{(1)}) \geq r_0 / 2\).
          Choose such a point \(y^{(2)}\).
          The set \(B_{(X, d)}(y^{(1)}, r_0 / 2) \cup B_{(X, d)}(y^{(2)}, r_0 / 2)\) cannot cover all of \(Y\), since we would then obtain two sets \(V_{\alpha_1}\) and \(V_{\alpha_2}\) which covered \(Y\), a contradiction again.
          So we can choose a point \(y^{(3)}\) which does not lie in \(B_{(X, d)}(y^{(1)}, r_0 / 2) \cup B_{(X, d)}(y^{(2)}, r_0 / 2)\), so in particular \(d(y^{(3)}, y^{(1)}) \geq r_0 / 2\) and \(d(y^{(3)}, y^{(2)}) \geq r_0 / 2\).
          Continuing in this fashion we obtain a sequence \((y^{(n)})_{n = 1}^\infty\) in \(Y\) with the property that \(d(y^{(k)}, y^{(j)}) \geq r_0 / 2\) for all \(k > j\).
          In particular the sequence \((y^{(n)})_{n = 1}^\infty\) is not a Cauchy sequence, and in fact no subsequence of \((y^{(n)})_{n = 1}^\infty\) can be a Cauchy sequence either.
          But this contradicts the assumption that \((Y, d|_{Y \times Y})\) is compact (by \cref{1.4.7}).
    \item Case 3:
          \(r_0 = \infty\).
          For this case we argue as in Case 2, but replacing the role of \(r_0 / 2\) by (say) \(1\).
  \end{itemize}
\end{proof}

\begin{note}
  It turns out that \cref{1.5.8} has a converse:
  if \(Y\) has the property that every open cover has a finite sub-cover, then it is compact.
  In fact, this property is often considered the more fundamental notion of compactness than the sequence-based one.
  (For metric spaces, the two notions, that of compactness and sequential compactness, are equivalent, but for more general \emph{topological spaces}, the two notions are slightly different.)
\end{note}

\begin{cor}\label{1.5.9}
  Let \((X, d)\) be a metric space, and let \(K_1, K_2, K_3, \dots\) be a sequence of non-empty compact subsets of \(X\) such that
  \[
    K_1 \supseteq K_2 \supseteq K_3 \supseteq \dots.
  \]
  Then the intersection \(\bigcap_{n = 1}^\infty K_n\) is non-empty.
\end{cor}

\begin{proof}
  Since \((K_1, d|_{K_1 \times K_1})\) is compact and \(K_1 \subseteq X\), by \cref{1.5.6} we know that \(K_1\) is closed in \((X, d)\).
  Since \(K_1\) is closed in \((X, d)\) and \(K_1 \cap K_n = K_n\) for every \(n \geq 1\), by \cref{1.3.4}(b) we know that \(K_n\) is relatively closed in \((K_1, d|_{K_1 \times K_1})\).
  Let \(V_n = K_1 \setminus K_n\) for every \(n \geq 1\).
  Then for every \(n \geq 1\), we have \(V_n \subseteq K_1\) and by \cref{1.2.15}(e) \(V_n\) is open in \((K_1, d|_{K_1 \times K_1})\).

  Suppose for sake of contradiction that \(\bigcap_{n = 1}^\infty K_n = \emptyset\).
  Since
  \[
    \bigcup_{n = 1}^\infty V_n = \bigcup_{n = 1}^\infty (K_1 \setminus K_n) = K_1 \setminus \bigg(\bigcap_{n = 1}^\infty K_n\bigg) = K_1
  \]
  and \((K_1, d)\) is compact, by \cref{1.5.8} we know that there exists a finite set \(F \subseteq \Z^+\) such that
  \[
    K_1 \subseteq \bigcup_{i \in F} V_i.
  \]
  Since \(F\) is finite subset of \(\Z^+\), we know that \(\min(F)\) is well-defined.
  Then we have
  \begin{align*}
             & K_1 \subseteq \bigcup_{i \in F} V_i \subseteq \bigcup_{n = 1}^\infty V_i = K_1                                                             \\
    \implies & K_1 = \bigcup_{i \in F} V_i                                                                                                                \\
    \implies & K_1 = \bigcup_{i \in F} (K_1 \setminus K_i)                                                                                                \\
    \implies & K_1 = K_1 \setminus \bigg(\bigcap_{i \in F} K_i\bigg)                                                                                      \\
    \implies & \bigcap_{i \in F} K_i = \emptyset                                              &  & \text{(since \(\bigcap_{i \in F} K_i \subseteq K_1)\)} \\
    \implies & K_{\min(F)} = \emptyset.                                                       &  & \text{(since \(K_{\min(F)} = \bigcap_{i \in F} K_i)\)}
  \end{align*}
  But by hypothesis we know that \(K_{\min(F)} \neq \emptyset\), a contradiction.
  Thus \(\bigcap_{n = 1}^\infty K_n \neq \emptyset\).
\end{proof}

\begin{thm}\label{1.5.10}
  Let \((X, d)\) be a metric space.
  \begin{enumerate}
    \item If \(Y\) is a compact subset of \(X\), and \(Z \subseteq Y\), then \(Z\) is compact iff \(Z\) is closed.
    \item If \(Y_1, \dots, Y_n\) are a finite collection of compact subsets of \(X\), then their union \(Y_1 \cup \dots \cup Y_n\) is also compact.
    \item Every finite subset of \(X\) (including the empty set) is compact.
  \end{enumerate}
\end{thm}

\begin{proof}{(a)}
  By \cref{1.5.6} we know that if \((Z, d|_{Z \times Z})\) is compact then \(Z\) is closed in \((Y, d|_{Y \times Y})\).
  Now we show that if \(Z\) is closed in \((Y, d|_{Y \times Y})\) then \((Z, d|_{Z \times Z})\) is compact.
  Since \((Y, d|_{Y \times Y})\) is compact and \(Z \subseteq Y\), by \cref{1.5.1} we know that every sequence \((z^{(n)})_{n = 1}^\infty\) in \(Z\) has a convergent subsequence \((z^{(n_j)})_{j = 1}^\infty\) which converges in \(Y\) with respect to \(d|_{Y \times Y}\).
  Since \(Z\) is closed in \((Y, d|_{Y \times Y})\), by \cref{1.2.15}(b) we know that \((z^{(n_j)})_{j = 1}^\infty\) converges in \(Z\) with respect to \(d|_{Y \times Y}\).
  Since \((z^{(n)})_{n = 1}^\infty\) is arbitrary, by \cref{1.5.1} we know that \((Z, d|_{Z \times Z})\) is compact.
\end{proof}

\begin{proof}{(b)}
  We use induction on \(n\) to show that \((\bigcup_{i = 1}^n Y_i, d)\) is compact for every \(n \in \Z^+\).
  For \(n = 1\), we know that \(\bigcup_{i = 1}^1 Y_i = Y_1\) and by hypothesis \((Y_1, d)\) is compact, thus the base case holds.
  Suppose inductively that \((\bigcup_{i = 1}^n Y_i, d)\) is compact for some \(n \geq 1\).
  Then for \(n + 1\), we need to show that \((\bigcup_{i = 1}^{n + 1} Y_i, d)\) is compact.
  Let \((y^{(k)})_{k = 1}^\infty\) be a sequence in \(\bigcup_{i = 1}^{n + 1} Y_i\).
  We now split into two cases:
  \begin{itemize}
    \item If there exists a subsequence \((y^{(k_j)})_{j = 1}^\infty\) whose elements are in \(\bigcup_{i = 1}^n Y_i\), then by induction hypothesis we know that \((y^{(k_j)})_{j = 1}^\infty\) converges in \(\bigcup_{i = 1}^n Y_i\) with respect to \(d\).
          This means \((y^{(k_j)})_{j = 1}^\infty\) also converges in \(\bigcup_{i = 1}^{n + 1} Y_i\) with respect to \(d\) since \(\bigcup_{i = 1}^n Y_i \subseteq \bigcup_{i = 1}^{n + 1} Y_i\).
    \item If there does not exist a subsequence \((y^{(k_j)})_{j = 1}^\infty\) whose elements are in \(\bigcup_{i = 1}^n Y_i\), then there is only finitely many elements in \((y^{(k)})_{k = 1}^\infty\) which are in \(\bigcup_{i = 1}^n Y_i\).
          This means there exists a subsequence \((y^{(k_j)})_{j = 1}^\infty\) whose elements are in \(Y_{n + 1}\).
          By hypothesis we know that \((Y_{n + 1}, d)\) is compact, thus by \cref{1.5.1} there exists a subsequence \((y^{(k_{j_p})})_{p = 1}^\infty\) of \((y^{(k_j)})_{j = 1}^\infty\) converges in \(Y_{n + 1}\) with respect to \(d\).
          Since \(Y_{n + 1} \subseteq \bigcup_{i = 1}^{n + 1} Y_i\), we know that \((y^{(k_{j_p})})_{p = 1}^\infty\) also converges in \(\bigcup_{i = 1}^{n + 1} Y_i\) with respect to \(d\).
  \end{itemize}
  From all cases above we conclude that there exists a subsequence of \((y^{(k)})_{k = 1}^\infty\) which converges in \(\bigcup_{i = 1}^{n + 1} Y_i\) with respect to \(d\).
  Since \((y^{(k)})_{k = 1}^\infty\) is arbitrary, by \cref{1.5.1} \((\bigcup_{i = 1}^{n + 1} Y_i, d)\) is compact.
  This closes the induction.
\end{proof}

\begin{proof}{(c)}
  Let \(Y \subseteq X\) and \(\#(Y) = n\).
  Let \(P(n)\) be the statement ``\(\#(Y) = n\) and for every sequence \((y^{(k)})_{k = 1}^\infty\) in \(Y\), there exists a subsequence of \((y^{(k)})_{k = 1}^\infty\) which converges in \(Y\) with respect to \(d\)''.
  We use induction on \(n\) to show that \(P(n)\) is true for all \(n \in \N\).
  For \(n = 0\), we have \(Y = \emptyset\) and the statement \(P(0)\) is trivially true.
  Thus by \cref{1.5.1} \((\emptyset, d)\) is compact and the base case holds.
  Suppose inductively that \(P(n)\) is true for some \(n \geq 0\).
  Then we need to show that \(P(n + 1)\) is true.
  Let \(Y \subseteq X\) such that \(\#(Y) = n + 1\) and let \(x_0 \in Y\).
  Let \((y^{(k)})_{k = 1}^\infty\) be arbitrary sequence in \(Y\).
  Now we split into two cases:
  \begin{itemize}
    \item If the set \(\{k \in \N : y^{(k)} = x_0\}\) is finite, then we can have a subsequence \((y^{(k_j)})_{j = 1}^\infty\) whose elements are in \(Y \setminus \{x_0\}\).
          Since \((y^{(k_j)})_{j = 1}^\infty\) is in \(Y \setminus \{x_0\}\) and \(\#(Y \setminus \{x_0\}) = n\), by induction hypothesis we know that there exists a subsequence of \((y^{(k_j)})_{j = 1}^\infty\) which converges in \(Y \setminus \{x_0\}\) with respect to \(d\).
          But \((y^{(k_j)})_{j = 1}^\infty\) is also in \(Y\), thus we know that there exists a subsequence of \((y^{(k_j)})_{j = 1}^\infty\) which converges in \(Y\) with respect to \(d\).
    \item If the set \(\{k \in \N : y^{(k)} = x_0\}\) is infinite, then we can have a subsequence \((y^{(k_j)})_{j = 1}^\infty\) whose elements are all \(x_0\) and obviously \((y^{(k_j)})_{j = 1}^\infty\) converges to \(x_0\) with respect to \(d\).
          Since \(x_0 \in Y\), we know that \((y^{(k_j)})_{j = 1}^\infty\) converges in \(Y\) with respect to \(d\).
  \end{itemize}
  From all cases above we conclude that there exists a subsequence of \((y^{(k)})_{k = 1}^\infty\) which converges in \(Y\) with respect to \(d\).
  Since \((y^{(k)})_{k = 1}^\infty\) is arbitrary, we conclude that \(P(n + 1)\) is true and this closes the induction.
  Since \(P(n)\) is true for every \(n \in \N\), by \cref{1.5.1} we know that if \(Y\) is a finite subset of \(X\), then \((Y, d|_{Y \times Y})\) is compact.
\end{proof}

\exercisesection

\begin{ex}\label{ex:1.5.1}
  Show that Definitions 9.1.22 in Analysis I and \cref{1.5.3} match when talking about subsets of the real line with the standard metric.
\end{ex}

\begin{proof}
  Let \(X \subseteq \R\) and let \(d = d_{l^1}|_{\R \times \R}\).
  By \cref{ex:1.1.2} we know that \((\R, d)\) is a metric space.
  Then we have
  \begin{align*}
         & X \text{ is bounded in the sense of Definition 9.1.22 in Analysis I}               \\
    \iff & \exists M \in \R^+ : X \subseteq [-M, M] \subseteq (-M - 1, M + 1)                 \\
    \iff & \exists M \in \R^+ : \forall x \in X, \abs{x - 0} < M + 1                          \\
    \iff & \forall y \in \R, \exists M \in \R^+ : \forall x \in X,                            \\
         & \abs{x - y} \leq \abs{x - 0} + \abs{y - 0} < M + 1 + \abs{y}                       \\
    \iff & \forall y \in \R, \exists M \in \R^+ : X \subseteq B_{(\R, d)}(y, M + 1 + \abs{y}) \\
    \iff & X \text{ is bounded in the sense of \cref{1.5.3}}.
  \end{align*}
\end{proof}

\begin{ex}\label{ex:1.5.2}
  Prove \cref{1.5.5}.
\end{ex}

\begin{proof}
  See \cref{1.5.5}.
\end{proof}

\begin{ex}\label{ex:1.5.3}
  Prove \cref{1.5.7}.
\end{ex}

\begin{proof}
  See \cref{1.5.7}.
\end{proof}

\begin{ex}\label{ex:1.5.4}
  Let \((\R, d)\) be the real line with the standard metric.
  Give an example of a continuous function \(f : \R \to \R\), and an open set \(V \subseteq \R\), such that the image \(f(V) \coloneqq \{f(x) : x \in V\}\) of \(V\) is \emph{not} open.
\end{ex}

\begin{proof}
  Let \(f(x) = 1\) for all \(x \in \R\).
  Since \(f\) is a constant function, we know that \(f\) is continuous.
  By \cref{1.2.1} and \cref{1.2.15}(c) we know that
  \[
    B_{(\R, d_{l^1}|_{\R \times \R})}(1, 1) = (0, 2)
  \]
  is open in \((\R, d_{l^1}|_{\R \times \R})\) and by \cref{1.2.15}(d)
  \[
    f\big((0, 2)\big) = \{1\}
  \]
  is closed in \((\R, d_{l^1}|_{\R \times \R})\).
  By \cref{1.2.15}(a) we know that \(\{1\}\) is not open in \((\R, d_{l^1}|_{\R \times \R})\) since we cannot find an \(r \in \R^+\) such that \(B_{(\R, d_{l^1})}(1, r) \subseteq \{1\}\).
  Thus \(f\) satisfys the requirements.
\end{proof}

\begin{ex}\label{ex:1.5.5}
  Let \((\R, d)\) be the real line with the standard metric.
  Give an example of a continuous function \(f : \R \to \R\), and a closed set \(F \subseteq \R\), such that \(f(F)\) is \emph{not} closed.
\end{ex}

\begin{proof}
  Let \(f(x) = 2^x\) for all \(x \in \R\).
  We know that \(2^x\) is continuous on \((-\infty, \infty)\).
  Let \(F = (-\infty, 0]\).
  If \(x_0 \in \partial_{(\R, d)}(F)\), then by \cref{1.2.5} we must have \(B(x_0, r) \not\subseteq F\) and \(B(x_0, r) \cap F \neq \emptyset\) for every \(r \in \R^+\).
  Thus we must have \(\partial_{(\R, d)}(F) = \{0\}\).
  Since \(0 \in F\), by \cref{1.2.12} we know that \(F\) is closed in \((\R, d)\).
  Since
  \[
    f(F) = f\big((-\infty, 0]\big) = (0, 1]
  \]
  and \(0 \notin (0, 1]\), by \cref{1.2.12} we know that \(f(F)\) is not closed in \((\R, d)\).
  Thus the function \(f\) and the closed set \(F\) satisfy the requirements.
\end{proof}

\begin{ex}\label{ex:1.5.6}
  Prove \cref{1.5.9}.
\end{ex}

\begin{proof}
  See \cref{1.5.9}.
\end{proof}

\begin{ex}\label{ex:1.5.7}
  Prove \cref{1.5.10}.
\end{ex}

\begin{proof}
  See \cref{1.5.10}.
\end{proof}

\begin{ex}\label{ex:1.5.8}
  Let \((X, d_{l^1})\) be the metric space from \cref{ex:1.1.15}.
  For each natural number \(n\), let \(e^{(n)} = (e_j^{(n)})_{j = 0}^\infty\) be the sequence in \(X\) such that \(e_j^{(n)} \coloneqq 1\) when \(n = j\) and \(e_j^{(n)} \coloneqq 0\) when \(n \neq j\).
  Show that the set \(\{e^{(n)} : n \in \N\}\) is a closed and bounded subset of \(X\), but is not compact.
  (This is despite the fact that \((X, d_{l^1})\) is even a complete metric space
  - a fact which we will not prove here.
  The problem is that not that \(X\) is incomplete, but rather that it is ``infinite-dimensional'', in a sense that we will not discuss here.)
\end{ex}

\begin{proof}
  Let \(E = \{e^{(n)} : n \in \N\}\).
  We first show that \(E\) is bounded in \((X, d_{l^1})\).
  Since
  \[
    \sum_{j = 0}^\infty \abs{e_j^{(n)}} = 1
  \]
  for every \(n \in \N\), we know that \(e^{(n)}\) is absolutely convergent and by \cref{ex:1.1.15} \(e^{(n)} \in X\).
  By \cref{ex:1.1.15} we know that
  \[
    \forall (a_j)_{j = 0}^\infty, (b_j)_{j = 0}^\infty \in X, d_{l^1}\big((a_j)_{j = 0}^\infty, (b_j)_{j = 0}^\infty\big)
  \]
  is well-defined, thus \(d_{l^1}\big((a_j)_{j = 0}^\infty, (e_j^{(n)})_{j = 0}^\infty\big)\) is well-defined for every \(n \in \N\).
  Since
  \begin{align*}
     & \forall (a_j)_{j = 0}^\infty \in X, d_{l^1}\big((a_j)_{j = 0}^\infty, (e_j^{(n)})_{j = 0}^\infty\big)                                                                 \\
     & = \sum_{j = 0}^\infty \abs{a_j - e_j^{(n)}}                                                                                                                           \\
     & = \sum_{j = 0 : j \neq n}^\infty \abs{a_j} + \abs{a_n - 1}                                                                                                            \\
     & \leq \sum_{j = 0}^\infty \abs{a_j} + \abs{a_n - 1}                                                    &  & \text{(well-defined since \((a_j)_{j = 0}^\infty \in X\))} \\
     & \leq \sum_{j = 0}^\infty \abs{a_j} + \sup_{n \in \N}\abs{a_n - 1}                                     &  & \text{(well-defined since \((a_j)_{j = 0}^\infty \in X\))} \\
     & < \sum_{j = 0}^\infty \abs{a_j} + \sup_{n \in \N}\abs{a_n - 1} + 1,
  \end{align*}
  by \cref{1.2.1} we know that the ball
  \[
    B_{(X, d_{l^1})}\bigg((a_j)_{j = 0}^\infty, \sum_{j = 0}^\infty \abs{a_j} + \sup_{n \in \N}\abs{a_n - 1} + 1\bigg)
  \]
  contains the set \(E\) for every \((a_j)_{j = 0}^\infty \in X\).
  Thus by \cref{1.5.3} we know that \(E\) is bounded in \((X, d_{l^1})\).

  Next we show that \(E\) is closed in \((X, d_{l^1})\).
  Let \(\overline{E}_{(X, d_{l^1})}\) be the closure of \(E\) and let \(x \in \overline{E}_{(X, d_{l^1})}\).
  By \cref{1.2.10}(c) we know that there exists a sequence \((a^{(k)})_{k = 0}^\infty\) in \(E\) such that \(\lim_{k \to \infty} d_{l^1}\big((a_j^{(k)})_{j = 0}^\infty, x\big) = 0\).
  By \cref{1.4.7} we know that \((a^{(k)})_{k = 0}^\infty\) is a Cauchy sequence in \((X, d_{l^1})\).
  Let \(k, k' \in \N\).
  By \cref{1.4.6} we have
  \[
    \forall \varepsilon \in \R^+, \exists N \in \N : \forall k, k' \geq N, d_{l^1}(a^{(k)}, a^{(k')}) \leq \varepsilon.
  \]
  In particular,
  \[
    \exists N \in \N : \forall k, k' \geq N, d_{l^1}(a^{(k)}, a^{(k')}) \leq \dfrac{1}{2} < 1.
  \]
  Since \(a^{(k)}, a^{(k')} \in E\), we know that \(a^{(k)} = e^{(n)}\) and \(a^{(k')} = e^{(n')}\) for some \(n, n' \in \N\) and
  \[
    d_{l^1}(a^{(k)}, a^{(k')}) = d_{l^1}(e^{(n)}, e^{(n')}) = \begin{dcases}
      0 & \text{if } n = n';    \\
      2 & \text{if } n \neq n'.
    \end{dcases}
  \]
  This means
  \begin{align*}
             & \exists N \in \N : \forall k, k' \geq N, d_{l^1}(a^{(k)}, a^{(k')}) \leq \dfrac{1}{2} < 1                  \\
    \implies & \exists N \in \N : \forall k, k' \geq N, d_{l^1}(a^{(k)}, a^{(k')}) = 0                                    \\
    \implies & \exists N \in \N : \forall k \geq N, d_{l^1}(a^{(k)}, a^{(N)}) = 0                                         \\
    \implies & \exists N \in \N : \lim_{k \to \infty} d_{l^1}(a^{(k)}, a^{(N)}) = 0                      &  & \by{1.1.14} \\
    \implies & \exists N \in \N : a^{(N)} = x                                                            &  & \by{1.1.20} \\
    \implies & x \in E.
  \end{align*}
  Since \(x\) is arbitrary adherent point of \(E\) in \((X, d_{l^1})\), we have
  \begin{align*}
             & \overline{E}_{(X, d_{l^1})} \subseteq E                                   \\
    \implies & \overline{E}_{(X, d_{l^1})} = E         &  & \text{(by \cref{1.2.10}(c))} \\
    \implies & E \text{ is closed in } (X, d_{l^1}).   &  & \text{(by \cref{1.2.15}(b))}
  \end{align*}

  Finally we show that \((E, d_{l^1}|_{E \times E})\) is not compact.
  Let \((e^{(n)})_{n = 0}^\infty\) be a sequence and let \(n, n' \in \N\).
  Since \((e^{(n)})_{n = 0}^\infty\) is a sequence in \(E\), we know that
  \[
    \forall N \in \N, \forall n, n' \geq N, d_{l^1}(e^{(n)}, e^{(n')}) = 2.
  \]
  Thus by \cref{1.4.6} \((e^{(n)})_{n = 0}^\infty\) is not a Cauchy sequence in \((E, d_{l^1}|_{E \times E})\).
  Similarly any subsequence of \((e^{(n)})_{n = 0}^\infty\) is not a Cauchy sequence in \((E, d_{l^1}|_{E \times E})\).
  By \cref{1.4.7} this means no subsequence is convergent in \((E, d_{l^1}|_{E \times E})\), and by \cref{1.5.1} \((E, d_{l^1}|_{E \times E})\) is not compact.
\end{proof}

\begin{ex}\label{ex:1.5.9}
  Show that a metric space \((X, d)\) is compact iff every sequence in \(X\) has at least one limit point.
\end{ex}

\begin{proof}
  \begin{align*}
         & (X, d) \text{ is compact}                                                         \\
    \iff & \text{every sequence in } X \text{ has a convergent subsequence}                  \\
         & \text{which converges in } X                                      &  & \by{1.5.1} \\
    \iff & \text{every sequence in } X \text{ has at least one limit point}. &  & \by{1.4.5}
  \end{align*}
\end{proof}

\begin{ex}\label{ex:1.5.10}
  A metric space \((X, d)\) is called \emph{totally bounded} if for every \(\varepsilon > 0\), there exists a natural number \(n\) and a finite number of balls \(B(x^{(1)}, \varepsilon), \dots, B(x^{(n)}, \varepsilon)\) which cover \(X\) (i.e., \(X = \bigcup_{i = 1}^n B(x^{(i)}, \varepsilon)\)).
  (Note that \(x^{(1)}, \dots, x^{(n)} \in X\))
  \begin{enumerate}
    \item Show that every totally bounded space is bounded.
    \item Show the following stronger version of \cref{1.5.5}:
          if \((X, d)\) is compact, then complete and totally bounded.
    \item Conversely, show that if \(X\) is complete and totally bounded, then \(X\) is compact.
  \end{enumerate}
\end{ex}

\begin{proof}{(a)}
  Suppose that \(X\) is totally bounded in \((X, d)\).
  Let \(i, j, n \in \N\).
  Then by definition we have
  \[
    \forall \varepsilon \in \R^+, \exists n \in \N : X = \bigcup_{i = 1}^n B_{(X, d)}(x^{(i)}, \varepsilon).
  \]
  In particular, we have
  \[
    \exists n \in \N : X = \bigcup_{i = 1}^n B_{(X, d)}(x^{(i)}, 1).
  \]
  If \(n = 0\), then we have \(X = \emptyset\) and by \cref{1.5.3} \(\emptyset\) is bounded in \((\emptyset, d)\).
  So suppose that \(n > 0\).
  Now we use axiom of choice to choose one finite collection of \(x^{(1)}, \dots, x^{(n)} \in X\).
  Let \(I_n = \{i \in \N : 1 \leq i \leq n\}\) and let \(r = \max\{d(x^{(i)}, x^{(1)}) : i \in I_n\}\).
  We know that \(r\) is well-defined since \(I_n\) is finite.
  Then have
  \begin{align*}
             & \forall i \in I_n, \forall y \in B_{(X, d)}(x^{(i)}, 1), d(y, x^{(i)}) < 1    &                                                & \by{1.2.1}                  \\
    \implies & \forall i \in I_n, \forall y \in B_{(X, d)}(x^{(i)}, 1),                                                                                                     \\
             & d(y, x^{(1)}) \leq d(y, x^{(i)}) + d(x^{(i)}, x^{(1)}) < 1 + r                &                                                & \text{(by \cref{1.1.2}(d))} \\
    \implies & \forall y \in \bigcup_{i = 1}^n B_{(X, d)}(x^{(i)}, 1), d(y, x^{(1)}) < 1 + r                                                                                \\
    \implies & \forall y \in X, d(y, x^{(1)}) < 1 + r                                        & (X = \bigcup_{i = 1}^n B_{(X, d)}(x^{(i)}, 1))                               \\
    \implies & \forall y \in X, y \in B_{(X, d)}(x^{(1)}, 1 + r)                             &                                                & \by{1.2.1}                  \\
    \implies & X \subseteq B_{(X, d)}(x^{(1)}, 1 + r)                                                                                                                       \\
    \implies & \forall y \in X, X \subseteq B_{(X, d)}\big(y, d(y, x^{(1)}) + 1 + r\big)     &                                                & \text{(by \cref{1.1.2}(d))} \\
    \implies & X \text{ is bounded in } (X, d).                                              &                                                & \by{1.5.3}
  \end{align*}
\end{proof}

\begin{proof}{(b)}
  By \cref{1.5.5} we know that if \((X, d)\) is compact then \((X, d)\) is complete.
  Thus we only need to show that \(X\) is totally bound in \((X, d)\).
  Let \(\varepsilon \in \R^+\).
  Then we have
  \begin{align*}
             & \forall y \in X, y \in B_{(X, d)}(y, \varepsilon)                                                                         &  & \by{1.2.1}                      \\
    \implies & X \subseteq \bigcup_{y \in X} B_{(X, d)}(y, \varepsilon)                                                                                                       \\
    \implies & \bigcup_{y \in X} B_{(X, d)}(y, \varepsilon) \text{ is an open cover of } X \text{ in } (X, d)                            &  & \text{(by \cref{1.2.15}(c)(g))} \\
    \implies & \exists F \subseteq X : (F \text{ is finite}) \land \bigg(X \subseteq \bigcup_{y \in F} B_{(X, d)}(y, \varepsilon)\bigg). &  & \by{1.5.8}
  \end{align*}
  Since \(\varepsilon\) is arbitrary, by definition we know that \(X\) is totally bounded in \((X, d)\).
\end{proof}

\begin{proof}{(c)}
  Suppose that \((X, d)\) is complete and \(X\) is totally bounded in \((X, d)\).
  If \(X = \emptyset\), then we know that \((\emptyset, d)\) is compact is trivially true.
  So suppose that \(X \neq \emptyset\).
  To show that \((X, d)\) is compact, by \cref{1.5.1} we need to show that every sequence in \(X\) has a convergent subsequence which converges in \(X\) with respect to \(d\).
  So let \((x^{(n)})_{n = 1}^\infty\) be a sequence in \(X\).
  If we have
  \[
    \exists N \in \Z^+ : \big\{n \in \Z^+ : x^{(n)} = x^{(N)}\big\} \text{ is infinite},
  \]
  then there must exist a subsequence of \((x^{(n)})_{n = 1}^\infty\) which converges to \(x^{(N)}\) with respect to \(d\) for some \(N \in \Z^+\).
  So suppose that
  \[
    \forall N \in \Z^+, \big\{n \in \Z^+ : x^{(n)} = x^{(N)}\big\} \text{ is finite}.
  \]
  Let \(E = \{x^{(n)} : n \in \Z^+\}\).
  For each \(\varepsilon \in \R^+\), we define \(F_\varepsilon\) to be the set
  \[
    F_\varepsilon = \Bigg\{F \subseteq X : (F \text{ is finite}) \land \bigg(X \subseteq \bigcup_{y \in F} B_{(X, d)}(y, \varepsilon)\bigg)\Bigg\}.
  \]
  Since \(X\) is totally bounded in \((X, d)\), by definition we know that \(F_\varepsilon \neq \emptyset\) for every \(\varepsilon \in \R^+\).
  For arbitrary \(\varepsilon \in \R^+\) and arbitrary \(F \in F_\varepsilon\), we claim that
  \[
    \exists y \in F : E \cap B_{(X, d)}(y, \varepsilon) \text{ is infinite}.
  \]
  Suppose for sake of contradiction that the claim is false.
  Then we have
  \begin{align*}
             & \forall y \in F, E \cap B_{(X, d)}(y, \varepsilon) \text{ is finite}                                                       \\
    \implies & \bigcup_{y \in F} \bigg(E \cap B_{(X, d)}(y, \varepsilon)\bigg) \text{ is finite}      &  & \text{(since \(F\) is finite)} \\
    \implies & E \cap \bigg(\bigcup_{y \in F} B_{(X, d)}(y, \varepsilon)\bigg) \text{ is finite}                                          \\
    \implies & E = E \cap X \subseteq E \cap \bigg(\bigcup_{y \in F} B_{(X, d)}(y, \varepsilon)\bigg)
  \end{align*}
  and thus \(E\) is finite.
  But by the definition of \(E\) we know that \(E\) is infinite, a contradiction.
  Thus the claim is true.

  Using the claim above we can now define the set \(A_\varepsilon\) for each \(\varepsilon \in \R^+\).
  \[
    A_\varepsilon = \big\{y \in F : (F \in F_\varepsilon) \land (E \cap B_{(X, d)}(y, \varepsilon) \text{ is infinite})\big\}
  \]
  From the claim above we know that \(A_\varepsilon \neq \emptyset\) for every \(\varepsilon \in \R^+\).
  Now we claim that
  \[
    \forall \delta, \varepsilon \in \R^+, \delta < \varepsilon \implies A_\delta \subseteq A_\varepsilon.
  \]
  Let \(\delta, \varepsilon \in \R^+\) and \(\delta < \varepsilon\).
  Then we have
  \begin{align*}
             & \forall y \in A_\delta, E \cap B_{(X, d)}(y, \delta) \text{ is infinite}                                                               \\
    \implies & \forall y \in A_\delta, B_{(X, d)}(y, \delta) \text{ is infinite}                                &  & \text{(since \(E\) is infinite)} \\
    \implies & \forall y \in A_\delta, B_{(X, d)}(y, \delta) \subseteq B_{(X, d)}(y, \varepsilon)               &  & \by{1.2.1}                       \\
    \implies & \forall y \in A_\delta, E \cap B_{(X, d)}(y, \delta) \subseteq E \cap B_{(X, d)}(y, \varepsilon)                                       \\
    \implies & \forall y \in A_\delta, E \cap B_{(X, d)}(y, \varepsilon) \text{ is infinite}                                                          \\
    \implies & \forall y \in A_\delta, y \in A_\varepsilon                                                                                            \\
    \implies & A_\delta \subseteq A_\varepsilon
  \end{align*}
  and thus the claim is true.

  Now we construct a subsequence of \(x^{(n)}\).
  For each \(j \in \Z^+\) we define \(N_j\) as follow:
  \[
    N_j = \bigg\{n \in \Z^+ : x^{(n)} \in E \cap B_{(X, d)}(y, \dfrac{1}{j}) \text{ for some } y \in A_{\dfrac{1}{j}}\bigg\}.
  \]
  From the claim above we know that \(A_{\dfrac{1}{j}}\) is infinite for every \(j \in \Z^+\), thus \(N_j\) is infinite and we have
  \[
    \forall i, j \in \Z^+, i < j \implies \dfrac{1}{j} < \dfrac{1}{i} \implies A_{\dfrac{1}{j}} \subseteq A_{\dfrac{1}{i}} \implies N_j \subseteq N_i.
  \]
  Now we recursively define \(n_j\) for each \(j \in \Z^+\) as follow:
  \[
    n_j = \begin{dcases}
      \min N_1                                & \text{if } j = 1 \\
      \min\{n \in N_{j - 1} : n > n_{j - 1}\} & \text{if } j > 1
    \end{dcases}
  \]
  Since \(N_j\) is infinite for each \(j \in \Z^+\), we know that the set \(\{n \in N_{j - 1} : n > n_{j - 1}\}\) is also infinite for every \(j \geq 2\).
  If not, then the maximum element of \(N_{j - 1}\) would be \(n_{j - 1}\) and \(N_{j - 1}\) is finite, a contradiction.
  Since \(\{n \in N_j : n > n_{j - 1}\} \subseteq \Z^+\) for each \(j \in \Z^+\) and \(j \geq 2\), by well-ordering principle we know that \(\min\{n \in N_j : n > n_{j - 1}\}\) is well-defined.
  Thus \(n_j\) is well-defined for each \(j \in \Z^+\) and \((x^{(n_j)})_{j = 1}^\infty\) is a subsequence of \((x^{(n)})_{n = 1}^\infty\).

  Now we claim that the subsequence \((x^{(n_j)})_{j = 1}^\infty\) converges in \(X\) with respect to \(d\).
  We have
  \begin{align*}
             & \forall \varepsilon \in \R^+, \exists J \in \Z^+ : \dfrac{1}{J} < \varepsilon                                                      &  & \text{(by Archimedean property)}                   \\
    \implies & \forall \varepsilon \in \R^+, \exists J \in \Z^+ :                                                                                                                                         \\
             & (\dfrac{1}{J} < \varepsilon) \land (\forall j \geq J, N_j \subseteq N_J)                                                           &  & \text{(from the claim above)}                      \\
    \implies & \forall \varepsilon \in \R^+, \exists J \in \Z^+ :                                                                                                                                         \\
             & (\dfrac{1}{J} < \varepsilon)                                                                                                                                                               \\
             & \land \big(\exists y \in A_{\dfrac{1}{j}} : \forall j \geq J, d(x^{(n_j)}, y) < \dfrac{1}{j} \leq \dfrac{1}{J} < \varepsilon\big). &  & \text{(by the definition of \(A_{\dfrac{1}{j}}\))}
  \end{align*}
  If we fix \(J\) for each \(\varepsilon\), then we have
  \begin{align*}
             & \forall i, j \geq 2J, \exists y \in A_{\dfrac{1}{2J}} :                                                                            \\
             & d(x^{(n_i)}, y) + d(x^{(n_j)}, y) < \dfrac{1}{2J} + \dfrac{1}{2J} = \dfrac{1}{J} < \varepsilon                                     \\
    \implies & \forall i, j \geq 2J, \exists y \in A_{\dfrac{1}{2J}} :                                                                            \\
             & d(x^{(n_i)}, x^{(n_j)}) \leq d(x^{(n_i)}, y) + d(x^{(n_j)}, y) < \varepsilon                   &  & \text{(by \cref{1.1.2}(c)(d))} \\
    \implies & \forall i, j \geq 2J, d(x^{(n_i)}, x^{(n_j)}) < \varepsilon.
  \end{align*}
  Thus we conclude that
  \[
    \forall \varepsilon \in \R^+, \exists J \in \Z^+ : \forall i, j \geq J, d(x^{(n_i)}, d^{(n_j)}) < \varepsilon
  \]
  and by \cref{1.4.6} \((x^{(n_j)})_{j = 1}^\infty\) is a Cauchy sequence in \((X, d)\).
  Since \((X, d)\) is compatible, by \cref{1.4.10} we know that \((x^{(n_j)})_{j = 1}^\infty\) converges in \(X\) with respect to \(d\).
  Since \((x^{(n)})_{n = 1}^\infty\) is arbitrary, we know that any sequence in \(X\) has a subsequence converges in \(X\) with respect to \(d\), and by \cref{1.5.1} \((X, d)\) is compact.
\end{proof}

\begin{ex}\label{ex:1.5.11}
  Let \((X, d)\) have the property that every open cover of \(X\) has a finite subcover.
  Show that \(X\) is compact.
\end{ex}

\begin{proof}
  Suppose that \((X, d)\) is a metric space such that every open cover of \(X\) has a finite subcover.
  We want to show that \((X, d)\) is compact.
  Suppose for sake of contradiction that \((X, d)\) is not compact.
  Then by \cref{1.5.1} we know that there exists a sequence \((x^{(n)})_{n = 1}^\infty\) in \(X\) which has no convergent subsequence which converges in \(X\) with respect to \(d\).
  We know that
  \[
    \forall N \in \Z^+, \big\{n \in \Z^+ : x^{(n)} = x^{(N)}\big\} \text{ is finite}.
  \]
  If not, then we would have a subsequence which converges to \(x^{(N)}\) with respect to \(d\) for some \(N \in \Z^+\), a contradiction.
  Thus we know that \(\big\{x^{(n)} : n \in \Z^+\big\}\) is infinite.
  By \cref{ex:1.5.9} we know that \((x^{(n)})_{n = 1}^\infty\) cannot have a limit point in \(X\) with respect to \(d\).
  By \cref{1.4.4} this means
  \begin{align*}
             & \lnot\big(\exists y \in X : \forall \varepsilon \in \R^+, \forall N \in \Z^+, \exists n \geq N : d(x^{(n)}, y) \leq \varepsilon\big) \\
    \implies & \forall y \in X, \exists \varepsilon \in \R^+ : \exists N \in \Z^+ : \forall n \geq N, d(x^{(n)}, y) > \varepsilon                   \\
    \implies & \forall y \in X, \exists \varepsilon \in \R^+ : \big\{x^{(n)} : n \in \Z^+\big\} \cap B_{(X, d)}(y, \varepsilon) \text{ is finite}.
  \end{align*}
  Now we fix such \(\varepsilon\).
  Since
  \begin{align*}
             & \forall y \in X, B_{(X, d)}(y, \varepsilon) \text{ is open in } (X, d) &  & \text{(by \cref{1.2.15}(c))} \\
    \implies & X = \bigcup_{y \in X} B_{(X, d)}(y, \varepsilon)                       &  & \by{1.2.1}
  \end{align*}
  and \(\bigcup_{y \in X} B_{(X, d)}(y, \varepsilon)\) is an open cover of \(X\) in \((X, d)\), we know that
  \[
    \exists F \subseteq X : (F \text{ is finite}) \land \bigg(X = \bigcup_{y \in F} B_{(X, d)}(y, \varepsilon)\bigg).
  \]
  Now we fix such \(F\).
  Then we have
  \begin{align*}
    \{x^{(n)} : n \in \Z^+\} & = \{x^{(n)} : n \in \Z^+\} \cap X                                                        \\
                             & = \{x^{(n)} : n \in \Z^+\} \cap \bigg(\bigcup_{y \in F} B_{(X, d)}(y, \varepsilon)\bigg) \\
                             & = \bigcup_{y \in F} \big(\{x^{(n)} : n \in \Z^+\} \cap B_{(X, d)}(y, \varepsilon)\big).
  \end{align*}
  But we know that \(\bigcup_{y \in F} \big(\{x^{(n)} : n \in \Z^+\} \cap B_{(X, d)}(y, \varepsilon)\big)\) is finite since \(F\) is finite and \(\{x^{(n)} : n \in \Z^+\} \cap B_{(X, d)}(y, \varepsilon)\) is finite for every \(y \in F\).
  This means \(\{x^{(n)} : n \in \Z^+\}\) is finite, a contradiction.
  Thus \((X, d)\) is compact.
\end{proof}

\begin{ex}\label{ex:1.5.12}
  Let \((X, d_{\text{disc}})\) be a metric space with the discrete metric \(d_{\text{disc}}\).
  \begin{enumerate}
    \item Show that \(X\) is always complete.
    \item When is \(X\) compact, and when is \(X\) not compact?
          Prove your claim.
  \end{enumerate}
\end{ex}

\begin{proof}{(a)}
  Let \((x^{(n)})_{n = 1}^\infty\) be a Cauchy sequence in \((X, d_{\text{disc}})\).
  Let \(i, j \in \Z^+\).
  By \cref{1.4.6} we know that
  \[
    \forall \varepsilon \in \R^+, \exists N \in \Z^+ : \forall i, j \geq N, d_{\text{disc}}(x^{(i)}, x^{(j)}) \leq \varepsilon.
  \]
  In particular, we have
  \[
    \exists N \in \Z^+ : \forall i, j \geq N, d_{\text{disc}}(x^{(i)}, x^{(j)}) \leq \dfrac{1}{2}.
  \]
  But by \cref{1.1.11} we know that
  \[
    d_{\text{disc}}(x^{(i)}, x^{(j)}) \leq \dfrac{1}{2} \iff x^{(i)} = x^{(j)}.
  \]
  Thus we have
  \[
    \forall \varepsilon \in \R^+, \exists N \in \Z^+ : \forall i \geq N, x^{(i)} = x^{(N)}.
  \]
  and by \cref{1.1.14} we have \(\lim_{n \to \infty} d_{\text{disc}}(x^{(n)}, x^{(N)}) = 0\).
  This means \((x^{(n)})_{n = 1}^\infty\) converges to some \(x^{(N)} \in X\) with respect to \(d_{\text{disc}}\).
  Since \((x^{(n)})_{n = 1}^\infty\) is arbitrary, by \cref{1.4.10} we know that \((X, d_{\text{disc}})\) is complete.
\end{proof}

\begin{proof}{(b)}
  We claim that \((X, d_{\text{disc}})\) is compact iff \(X\) is finite.
  By \cref{1.5.10}(c) we know that if \(X\) is finite then \((X, d_{\text{disc}})\) is compact.
  So we only need to show that if \((X, d_{\text{disc}})\) is compact then \(X\) is finite.
  Suppose that \((X, d_{\text{disc}})\) is compact.
  By \cref{1.5.8} we know that every open cover of \(X\) in \((X, d_{\text{disc}})\) has a finite subcover.
  In particular, we know that
  \[
    \exists F \subseteq X : (F \text{ is finite}) \land \bigg(X = \bigcup_{y \in F} B_{(X, d_{\text{disc}})}(y, \dfrac{1}{2})\bigg).
  \]
  Now we fix such \(F\).
  We know that
  \begin{align*}
             & \forall y \in F, B_{(X, d)}(y, \dfrac{1}{2}) = \{z \in X : d(z, y) < \dfrac{1}{2}\} &  & \by{1.2.1}  \\
    \implies & \forall y \in F, B_{(X, d)}(y, \dfrac{1}{2}) = \{y\}                                &  & \by{1.1.11} \\
    \implies & X = \bigcup_{y \in F} B_{(X, d)}(y, \dfrac{1}{2}) = F
  \end{align*}
  Thus \(X\) is finite.
\end{proof}

\begin{ex}\label{ex:1.5.13}
  Let \(E\) and \(F\) be two compact subsets of \(\R\) (with the standard metric \(d(x, y) = \abs{x - y}\)).
  Show that the Cartesian product \(E \times F \coloneqq \{(x, y) : x \in E, y \in F\}\) is a compact subset of \(\R^2\) (with the Euclidean metric \(d_{l^2}\)).
\end{ex}

\begin{proof}
  Since \(E \times F \subseteq \R^2\), we know that every element \(x \in E \times F\) is in the form \(x = (x_1, x_2)\) where \(x_1 \in E\) and \(x_2 \in F\).
  Let \((x^{(n)})_{n = 1}^\infty\) be a sequence in \(E \times F\).
  Then we know that \((x_1^{(n)})_{n = 1}^\infty\) is a sequence in \(E\) and \((x_2^{(n)})_{n = 1}^\infty\) is a sequence in \(F\).
  Since \((E, d_{l^1}|_{\R \times \R})\) is compact, by \cref{1.5.1} we know that there exists a subsequence \((x_1^{(n_j)})_{j = 1}^\infty\) which converges to some \(L_E \in E\) with respect to \(d_{l^1}|_{\R \times \R}\).
  Since \((x_2^{(n_j)})_{j = 1}^\infty\) is a subsequence of \((x_2^{(n)})_{n = 1}^\infty\) and \((F, d_{l^1}|_{\R \times \R})\) is compact, by \cref{1.5.1} we know that there exists a subsequence \((x_2^{(n_{j_p})})_{p = 1}^\infty\) which converges to some \(L_F \in F\) with respect to \(d_{l^1}|_{\R \times \R}\).
  Since \(\lim_{j \to \infty} d_{l^1}|_{\R \times \R}(x_1^{(n_j)}, L_E) = 0\), by \cref{1.4.9} we know that \(\lim_{p \to \infty} d_{l^1}|_{\R \times \R}(x_1^{(n_{j_p})}, L_E) = 0\).
  Thus by \cref{1.1.18}(a)(b)(d) we have
  \[
    \lim_{p \to \infty} d_{l^1}|_{\R^2 \times \R^2}\big(x^{(n_{j_p})}, (L_E, L_F)\big) = \lim_{p \to \infty} d_{l^2}|_{\R^2 \times \R^2}\big(x^{(n_{j_p})}, (L_E, L_F)\big) = 0.
  \]
  Since \((x^{(n)})_{n = 1}^\infty\) is arbitrary, by \cref{1.5.1} we know that \((E \times F, d_{l^2}|_{\R^2 \times \R^2})\) is compact.
\end{proof}

\begin{ex}\label{ex:1.5.14}
  Let \((X, d)\) be a metric space, let \(E\) be a non-empty compact subset of \(X\), and let \(x_0\) be a point in \(X\).
  Show that there exists a point \(x \in E\) such that
  \[
    d(x_0, x) = \inf\{d(x_0, y) : y \in E\},
  \]
  i.e., \(x\) is the closest point in \(E\) to \(x_0\).
\end{ex}

\begin{proof}
  Let \(R = \inf\{d(x_0, y) : y \in E\}\).
  Since \(d(x_0, y) \geq 0\) for every \(y \in E\), we know that the set \(\{d(x_0, y) : y \in E\}\) is bounded below and thus \(R\) is well-defined.
  Since \(R\) is well-defined, we can construct a sequence \((x^{(n)})_{n = 1}^\infty\) in \(E\) such that \(d(x_0, x^{(n)}) \leq R + \dfrac{1}{n}\) for every \(n \in \Z^+\).
  If not, then we would have
  \[
    \exists n \in \Z^+ : \forall y \in E, d(x_0, y) > R + \dfrac{1}{n}
  \]
  and thus \(\inf\{d(x_0, y) : y \in E\} \geq R + \dfrac{1}{n} > R\), a contradiction.
  Since \(\big(d(x_0, x^{(n)})\big)_{n = 1}^\infty\) is a sequence in \(\R\), we have
  \begin{align*}
             & \forall n \in \Z^+, R \leq d(x_0, x^{(n)}) \leq R + \dfrac{1}{n}                               \\
    \implies & \lim_{n \to \infty} d(x_0, x^{(n)}) = R.                         &  & \text{(by squeeze test)}
  \end{align*}
  Since \((E, d)\) is compact, by \cref{1.5.1} we know that there exists a subsequence \((x^{(n_j)})_{j = 1}^\infty\) such that
  \[
    \lim_{j \to \infty} d(x^{(n_j)}, x) = 0
  \]
  for some \(x \in E\).
  Then we have
  \begin{align*}
             & \lim_{n \to \infty} d(x_0, x^{(n)}) = R                                                                                                   \\
    \implies & \lim_{j \to \infty} d(x_0, x^{(n_j)}) = R                                                                &  & \by{1.4.3}                  \\
    \implies & \lim_{j \to \infty} d(x_0, x^{(n_j)}) + \lim_{j \to \infty} d(x^{(n_j)}, x) = R + 0 = R                                                   \\
    \implies & \lim_{j \to \infty} \big(d(x_0, x^{(n_j)}) + d(x^{(n_j)}, x)\big) = R                                                                     \\
    \implies & \lim_{j \to \infty} d(x_0, x) \leq \lim_{j \to \infty} \big(d(x_0, x^{(n_j)}) + d(x^{(n_j)}, x)\big) = R &  & \text{(by \cref{1.1.2}(d))} \\
    \implies & R \leq d(x_0, x) \leq R                                                                                                                   \\
    \implies & d(x_0, x) = R
  \end{align*}
  and thus \(x\) is the closest point in \(E\) to \(x_0\).
\end{proof}

\begin{ex}\label{ex:1.5.15}
  Let \((X, d)\) be a compact metric space.
  Suppose that \((K_{\alpha})_{\alpha \in I}\) is a collection of closed sets in \(X\) with the property that any finite subcollection of these sets necessarily has non-empty intersection, thus \(\bigcap_{\alpha \in F} K_{\alpha} \neq \emptyset\) for all finite \(F \subseteq I\).
  (This property is known as the \emph{finite intersection property}.)
  Show that the \emph{entire} collection has non-empty intersection, thus \(\bigcap_{\alpha \in I} K_{\alpha} \neq \emptyset\).
  Show by counterexample that this statement fails if \(X\) is not compact.
\end{ex}

\begin{proof}
  Suppose for sake of contradiction that \(\bigcap_{\alpha \in I} K_\alpha = \emptyset\).
  Since
  \[
    X = X \setminus \emptyset = X \setminus \bigcap_{\alpha \in I} K_\alpha = \bigcup_{\alpha \in I} (X \setminus K_\alpha),
  \]
  we know that
  \begin{align*}
             & \forall \alpha \in I, K_\alpha \text{ is closed in } (X, d)                                                                                         \\
    \implies & \forall \alpha \in I, X \setminus K_\alpha \text{ is open in } (X, d)                                             &  & \text{(by \cref{1.2.15}(e))} \\
    \implies & X = \bigcup_{\alpha \in I} (X \setminus K_\alpha) \text{ is an open cover of } X \text{ in } (X, d)                                                 \\
    \implies & \exists F \subseteq I : (F \text{ is finite}) \land \bigg(X = \bigcup_{\alpha \in F} (X \setminus K_\alpha)\bigg) &  & \by{1.5.8}                   \\
    \implies & \exists F \subseteq I : (F \text{ is finite}) \land \bigg(X = X \setminus \bigcap_{\alpha \in F} K_\alpha\bigg)                                     \\
    \implies & \exists F \subseteq I : (F \text{ is finite}) \land \bigg(\bigcap_{\alpha \in F} K_\alpha = \emptyset\bigg).
  \end{align*}
  But by hypothesis we know such \(F\) does not exist, a contradiction.
  Thus we must have \(\bigcap_{\alpha \in I} K_\alpha \neq \emptyset\).

  Now we show an counterexample when \(X, d\) is not compact.
  Let \(X = I = \R^+\) and let \(d = d_{l^1}|_{\R \times \R}\).
  We know that the sequence \((n)_{n = 1}^\infty\) in \(\R^+\) has no convergent subsequence with respect to \(d_{l^1}|_{\R \times \R}\), thus by \cref{1.5.1} \((\R^+, d_{l^1}|_{\R \times \R})\) is not compact.
  For each \(\varepsilon \in \R^+\), we define \(K_\varepsilon = [\varepsilon, \infty)\) (which is an interval in \(\R\)).
  Then we have
  \begin{align*}
             & \forall \varepsilon \in \R^+, \overline{K_\varepsilon}_{(\R^+, d_{l^1}|_{\R \times \R})} = \{\varepsilon\} &  & \by{1.2.5}                   \\
    \implies & \forall \varepsilon \in \R^+, K_\varepsilon \text{ is closed in } (\R^+, d_{l^1}|_{\R \times \R})          &  & \text{(by \cref{1.2.15}(b))}
  \end{align*}
  Let \(F \subseteq \R^+\) such that \(F \neq \emptyset\) and \(F\) is finite.
  Since \(F\) is finite, we know that \(\max(F)\) is well-defined.
  Then we have
  \begin{align*}
             & \forall i \in F, i \leq \max(F)                                                              \\
    \implies & \forall i \in F, [\max(F), \infty) \subseteq [i, \infty)                                     \\
    \implies & \forall i \in F, K_i \subseteq K_{\max(F)}                                                   \\
    \implies & \bigcap_{i \in F} K_i = K_{\max(F)}.                     &  & \text{(since \(F\) is finite)}
  \end{align*}
  Since \(F\) is arbitrary, we know that every finite subcollection of \((K_\varepsilon)_{\varepsilon \in \R^+}\) has non-empty intersection.
  Suppose for sake of contradiction that \(\bigcap_{\varepsilon \in \R^+} K_\varepsilon \neq \emptyset\).
  Let \(r \in \bigcap_{\varepsilon \in \R^+} K_\varepsilon\).
  Then we have
  \begin{align*}
             & \forall \varepsilon \in \R^+, r \in K_\varepsilon \\
    \implies & r \in K_{2r}                                      \\
    \implies & r \in [2r, \infty)                                \\
    \implies & r = 0.
  \end{align*}
  But we know that \(0 \notin K_\varepsilon\) for every \(\varepsilon \in \R^+\).
  Thus \(0 \notin \bigcap_{\varepsilon \in \R^+} K_\varepsilon\), a contradiction.
  So we must have \(\bigcap_{\varepsilon \in \R^+} K_\varepsilon = \emptyset\).
\end{proof}

\chapter{Continuous functions on metric spaces}\label{ch:2}

\section{Continuous functions}\label{sec:2.1}

\begin{defn}[Continuous functions]\label{2.1.1}
  Let \((X, d_X)\) be a metric space, and let \((Y, d_Y)\) be another metric space, and let \(f : X \to Y\) be a function.
  If \(x_0 \in X\), we say that \(f\) is \emph{continuous at \(x_0\)} iff for every \(\varepsilon > 0\), there exists a \(\delta > 0\) such that \(d_Y(f(x), f(x_0 )) < \varepsilon\) whenever \(d_X(x, x_0) < \delta\).
  We say that \(f\) is \emph{continuous} iff it is continuous at every point \(x \in X\).
\end{defn}

\begin{rmk}\label{2.1.2}
  Continuous functions are also sometimes called \emph{continuous maps}.
  Mathematically, there is no distinction between the two terminologies.
\end{rmk}

\begin{rmk}\label{2.1.3}
  If \(f : X \to Y\) is continuous, and \(K\) is any subset of \(X\), then the restriction \(f|_K : K \to Y\) of \(f\) to \(K\) is also continuous.
\end{rmk}

\begin{proof}
  Let \(x_0 \in K\).
  Suppose that \(f : X \to Y\) is continuous at \(x_0\) from \((X, d_X)\) to \((Y, d_Y)\).
  Then we have
  \begin{align*}
             & f : X \to Y \text{ is continuous at } x_0                                                                                       \\
             & \text{from } (X, d_X) \text{ to } (Y, d_Y)                                                                                      \\
    \implies & \forall \varepsilon \in \R^+, \exists \delta \in \R^+ :                                                                         \\
             & \Big(\forall x \in X, d_X(x, x_0) < \delta \implies d_Y\big(f(x), f(x_0) < \varepsilon\big)\Big) &                 & \by{2.1.1} \\
    \implies & \forall \varepsilon \in \R^+, \exists \delta \in \R^+ :                                                                         \\
             & \Big(\forall x \in K, d_X(x, x_0) < \delta \implies d_Y\big(f(x), f(x_0) < \varepsilon\big)\Big) & (K \subseteq X)              \\
    \implies & f|_K : K \to Y \text{ is continuous at } x_0                                                                                    \\
             & \text{from } (K, d_X|_{K \times K}) \text{ to } (Y, d_Y).                                        &                 & \by{2.1.1}
  \end{align*}

  Now suppose that \(f : X \to Y\) is continuous from \((X, d_X)\) to \((Y, d_Y)\).
  Then we have
  \begin{align*}
             & f : X \to Y \text{ is continuous from } (X, d_X) \text{ to } (Y, d_Y)                                                         \\
    \implies & \forall x_0 \in X, f \text{ is continuous at } x_0 \text{ from } (X, d_X) \text{ to } (Y, d_Y) &                 & \by{2.1.1} \\
    \implies & \forall x_0 \in K, f \text{ is continuous at } x_0                                                                            \\
             & \text{from } (K, d_X|_{K \times K}) \text{ to } (Y, d_Y)                                       & (K \subseteq X)              \\
    \implies & f|_K : K \to Y \text{ is continuous from } (K, d_X|_{K \times K}) \text{ to } (Y, d_Y).        &                 & \by{2.1.1}
  \end{align*}
\end{proof}

\begin{thm}[Continuity preserves convergence]\label{2.1.4}
  Suppose that \((X, d_X)\) and \((Y, d_Y)\) are metric spaces.
  Let \(f : X \to Y\) be a function, and let \(x_0 \in X\) be a point in \(X\).
  Then the following three statements are logically equivalent:
  \begin{enumerate}
    \item \(f\) is continuous at \(x_0\).
    \item Whenever \((x^{(n)})_{n = 1}^\infty\) is a sequence in \(X\) which converges to \(x_0\) with respect to the metric \(d_X\), the sequence \(\big(f(x^{(n)})\big)_{n = 1}^\infty\) converges to \(f(x_0)\) with respect to the metric \(d_Y\).
    \item For every open set \(V \subseteq Y\) that contains \(f(x_0)\), there exists an open set \(U \subseteq X\) containing \(x_0\) such that \(f(U) \subseteq V\).
  \end{enumerate}
\end{thm}

\begin{proof}
  We first show that statement (a) implies statement (b).
  Suppose that \(f : X \to Y\) is continuous at \(x_0\) from \((X, d_X)\) to \((Y, d_Y)\).
  Then by \cref{2.1.1} we have
  \[
    \forall \varepsilon \in \R^+, \exists d \in \R^+ : \Big(\forall x \in X, d_X(x, x_0) < \delta \implies d_Y\big(f(x), f(x_0)\big) < \varepsilon\Big).
  \]
  Now we choose \(\delta\) for each \(\varepsilon \in \R^+\) and denoted it as \(\delta_\varepsilon\).
  Let \((x^{(n)})_{n = 1}^\infty\) be a sequence in \(X\) such that \(\lim_{n \to \infty} d_X(x^{(n)}, x_0) = 0\).
  Then we have
  \begin{align*}
             & \lim_{n \to \infty} d_X(x^{(n)}, x_0) = 0                                                                                                                \\
    \implies & \forall \delta \in \R^+, \exists N \in \Z^+ : \forall n \geq N, d_X(x^{(n)}, x_0) \leq \delta                                &  & \by{1.1.14}            \\
    \implies & \forall \varepsilon \in \R^+, \exists \delta_\varepsilon \in \R^+ :                                                                                      \\
             & \bigg(\exists N \in \Z^+ : \forall n \geq N, d_X(x^{(n)}, x_0) \leq \dfrac{\delta_\varepsilon}{2} < \delta_\varepsilon\bigg)                             \\
    \implies & \forall \varepsilon \in \R^+, \exists \delta_\varepsilon \in \R^+ :                                                                                      \\
             & \bigg(\exists N \in \Z^+ : \forall n \geq N, d_Y\big(f(x^{(n)}), f(x_0)\big) < \varepsilon\bigg)                             &  & \text{(by hypothesis)} \\
    \implies & \lim_{n \to \infty} d_Y\big(f(x^{(n)}), f(x_0)\big) = 0.                                                                     &  & \by{1.1.14}
  \end{align*}
  Since \((x^{(n)})_{n = 1}^\infty\) is arbitrary, we know that statement (a) implies statement (b).

  Next we show that statement (b) implies statement (c).
  Suppose that
  \[
    \forall (x^{(n)})_{n = 1}^\infty \text{ in } X, \lim_{n \to \infty} d_X(x^{(n)}, x_0) = 0 \implies \lim_{n \to \infty} d_Y\big(f(x^{(n)}), f(x_0)\big) = 0.
  \]
  Let \(V\) be an open set in \((Y, d_Y)\) such that \(f(x_0) \in V\).
  Then we have
  \begin{align*}
             & V \text{ is open in } (Y, d_Y)                                                                                          \\
    \implies & V = \text{int}_{(Y, d_Y)}(V)                                                          &  & \text{(by \cref{1.2.15}(a))} \\
    \implies & \exists \varepsilon \in \R^+ : B_{(Y, d_Y)}\big(f(x_0), \varepsilon\big) \subseteq V. &  & \by{1.2.5}
  \end{align*}
  Now we choose one \(\varepsilon\) and define \(V_\varepsilon = B_{(Y, d_Y)}\big(f(x_0), \varepsilon\big)\).
  By \cref{1.2.4} we know that \(f(x_0) \in V_\varepsilon\), thus we have \(x_0 \in f^{-1}(V_\varepsilon)\) and \(f^{-1}(V_\varepsilon) \neq \emptyset\).
  Now we claim that
  \[
    \exists \delta \in \R^+ : B_{(X, d_X)}(x_0, \delta) \subseteq f^{-1}(V_\varepsilon).
  \]
  Suppose the claim is false.
  Then we have
  \begin{align*}
             & \forall \delta \in \R^+, B_{(X, d_X)}(x_0, \delta) \not\subseteq f^{-1}(V_\varepsilon)                            \\
    \implies & \forall \delta \in \R^+, B_{(X, d_X)}(x_0, \delta) \setminus f^{-1}(V_\varepsilon) \neq \emptyset                 \\
    \implies & \forall \delta \in \R^+, \exists x \in X :                                                                        \\
             & \big(d_X(x, x_0) < \delta\big) \land \Big(d_Y\big(f(x), f(x_0)\big) \geq \varepsilon\Big)         &  & \by{1.2.1} \\
    \implies & \forall n \in \Z^+, \exists x \in X :                                                                             \\
             & \big(d_X(x, x_0) < \dfrac{1}{n}\big) \land \Big(d_Y\big(f(x), f(x_0)\big) \geq \varepsilon\Big).
  \end{align*}
  For each \(n \in \Z^+\), we define \(X_n = B_{(X, d_X)}(x_0, \dfrac{1}{n}) \setminus f^{-1}(V_\varepsilon)\).
  We choose one sequence \((x^{(n)})_{n = 1}^\infty \in \prod_{n \in \Z^+} X_n\).
  Then we have
  \begin{align*}
             & \forall n \in \Z^+, d_X(x^{(n)}, x_0) < \dfrac{1}{n}                                                                            \\
    \implies & \lim_{n \to \infty} d_X(x^{(n)}, x_0) = 0                                                                                       \\
    \implies & \lim_{n \to \infty} d_Y\big(f(x^{(n)}), f(x_0)\big) = 0                                             &  & \text{(by hypothesis)} \\
    \implies & \exists N \in \Z^+ : \forall n \geq N, d_Y\big(f(x^{(n)}), f(x_0)\big) \leq \dfrac{\varepsilon}{2}. &  & \by{1.1.14}
  \end{align*}
  But by the definition of \((x^{(n)})_{n = 1}^\infty\) we know that
  \[
    \forall n \in \Z^+, d_Y\big(f(x^{(n)}), f(x_0)\big) \geq \varepsilon,
  \]
  a contradiction.
  Thus the claim is true.
  Using the claim we choose one \(\delta\) and define \(U = B_{(X, d_X)}(x_0, \delta)\).
  By \cref{1.2.15}(c) we know that \(U\) is open in \((X, d_X)\).
  By \cref{1.2.4} we know that \(x_0 \in U\).
  Since \(U \subseteq f^{-1}(V_\varepsilon) \subseteq X\), we know that \(f(U) \subseteq V\).

  Finally we show that statement (c) implies statement (a).
  Suppose that
  \begin{align*}
             & \forall V \subseteq Y, \big(V \text{ is open in } (Y, d_Y)\big) \land \big(f(x_0) \in V\big)                         \\
    \implies & \exists U \subseteq X : \big(U \text{ is open in } (X, d_X)\big) \land (x_0 \in U) \land \big(f(U) \subseteq V\big).
  \end{align*}
  Let \(\varepsilon \in \R^+\).
  By \cref{1.2.15}(c) we know that \(B_{(Y, d_Y)}\big(f(x_0), \varepsilon\big)\) is open in \((Y, d_Y)\).
  By hypothesis we know that
  \[
    \exists U \subseteq X : \big(U \text{ is open in } (X, d_X)\big) \land (x_0 \in U) \land \Big(f(U) \subseteq B_{(Y, d_Y)}\big(f(x_0), \varepsilon\big)\Big).
  \]
  Now we choose one such \(U\).
  Since \(U\) is open in \((X, d_X)\) and \(x_0 \in U\), we have
  \begin{align*}
             & x_0 \in \text{int}_{(X, d_X)}(U)                                                                                   &  & \text{(by \cref{1.2.15}(a))} \\
    \implies & \exists \delta \in \R^+ : B_{(X, d_X)}(x_0, \delta) \subseteq U                                                    &  & \by{1.2.5}                   \\
    \implies & \exists \delta \in \R^+ : f\big(B_{(X, d_X)}(x_0, \delta)\big) \subseteq f(U)                                                                        \\
    \implies & \exists \delta \in \R^+ : f\big(B_{(X, d_X)}(x_0, \delta)\big) \subseteq B_{(Y, d_Y)}\big(f(x_0), \varepsilon\big)                                   \\
    \implies & \exists \delta \in \R^+ :                                                                                                                            \\
             & \Big(\forall x \in X, d_X(x, x_0) < \delta \implies d_Y\big(f(x), f(x_0)\big) < \varepsilon\Big).                  &  & \by{1.2.1}
  \end{align*}
  Since \(\varepsilon\) is arbitrary, we have
  \[
    \forall \varepsilon \in \R^+, \exists \delta \in \R^+ : \Big(\forall x \in X, d_X(x, x_0) < \delta \implies d_Y\big(f(x), f(x_0)\big) < \varepsilon\Big).
  \]
  Thus by \cref{2.1.1} \(f\) is continuous at \(x_0\) from \((X, d_X)\) to \((Y, d_Y)\).
  We conclude that statements (a)(b)(c) are equivalent.
\end{proof}

\begin{thm}\label{2.1.5}
  Let \((X, d_X)\) be a metric space, and let \((Y, d_Y)\) be another metric space.
  Let \(f : X \to Y\) be a function.
  Then the following four statements are equivalent:
  \begin{enumerate}
    \item \(f\) is continuous.
    \item Whenever \((x^{(n)})_{n = 1}^\infty\) is a sequence in \(X\) which converges to some point \(x_0 \in X\) with respect to the metric \(d_X\), the sequence \(\big(f(x^{(n)})\big)_{n = 1}^\infty\) converges to \(f(x_0)\) with respect to the metric \(d_Y\).
    \item Whenever \(V\) is an open set in \(Y\), the set \(f^{-1}(V) \coloneqq \set{x \in X : f(x) \in V}\) is an open set in \(X\).
    \item Whenever \(F\) is a closed set in \(Y\), the set \(f^{-1}(F) \coloneqq \set{x \in X : f(x) \in F}\) is a closed set in \(X\).
  \end{enumerate}
\end{thm}

\begin{proof}
  We first show that statements (a)(b) are equivalent.
  \begin{align*}
         & f \text{ is continuous from } (X, d_X) \text{ to } (Y, d_Y)                                                                              \\
    \iff & \forall x_0 \in X, f \text{ is continuous at } x_0                                                                                       \\
         & \text{from } (X, d_X) \text{ to } (Y, d_Y)                                                           &  & \by{2.1.1}                     \\
    \iff & \forall x_0 \in X, \text{ every sequence } (x^{(n)})_{n = 1}^\infty \text{ in } X \text{ satisfies }                                     \\
         & \lim_{n \to \infty} d_X(x^{(n)}, x_0) = 0 \text{ implies}                                                                                \\
         & \lim_{n \to \infty} d_Y\big(f(x^{(n)}), f(x_0)\big) = 0.                                             &  & \text{(by \cref{2.1.4}(a)(b))}
  \end{align*}

  Next we show that statements (a) implies statement (c).
  Suppose that \(f\) is continuous from \((X, d_X)\) to \((Y, d_Y)\).
  Let \(V\) be an open set in \((Y, d_Y)\) and let \(E = f^{-1}(V)\).
  Then we have
  \begin{align*}
             & f \text{ is continuous from } (X, d_X) \text{ to } (Y, d_Y)                                                                          \\
    \implies & f|_{E} \text{ is continuous from } \big(E, d_X|_{E \times E}\big) \text{ to } (Y, d_Y)      &  & \by{2.1.3}                          \\
    \implies & \forall x_0 \in E, f \text{ is continuous at } x_0                                                                                   \\
             & \text{from } \big(E, d_X|_{E \times E}\big) \text{ to } (Y, d_Y)                            &  & \by{2.1.1}                          \\
    \implies & \forall x_0 \in E, \exists U \subseteq X :                                                                                           \\
             & \big(U \text{ is open in } (X, d_X)\big) \land (x_0 \in U) \land \big(f(U) \subseteq V\big) &  & \text{(by \cref{2.1.4}(a)(c))}      \\
    \implies & \forall x_0 \in E, \exists U \subseteq X :                                                                                           \\
             & \big(U \text{ is open in } (X, d_X)\big) \land (x_0 \in U) \land \big(U \subseteq E\big)    &  & \text{(by the definition of \(E\))} \\
    \implies & \forall x_0 \in E, \exists U \subseteq X :                                                                                           \\
             & \big(\exists r \in \R^+ : B_{(X, d_X)}(x_0, r) \subseteq U \subseteq E\big)                 &  & \text{(by \cref{1.2.15}(a))}        \\
    \implies & E \text{ is open in } (X, d_X).                                                             &  & \text{(by \cref{1.2.15}(a))}
  \end{align*}
  Since \(V\) is arbitrary, we know that statement (a) implies statement (c).

  Next we show that statements (c) implies statement (a).
  Suppose that
  \[
    \forall V \subseteq Y, V \text{ is open in } (Y, d_Y) \implies f^{-1}(V) \text{ is open in } (X, d_X).
  \]
  Let \(x_0 \in X\).
  Then we have
  \begin{align*}
             & \forall V \subseteq Y, \big(V \text{ is open in } (Y, d_Y)\big) \land \big(f(x_0) \in V\big)                             \\
    \implies & \big(f^{-1}(V) \text{ is open in } (X, d_X)\big) \land \big(x_0 \in f^{-1}(V)\big)           &  & \text{(by hypothesis)}
  \end{align*}
  and by \cref{2.1.4}(a)(c) we know that \(f\) is continuous at \(x_0\) from \((X, d_X)\) to \((Y, d_Y)\).
  Since \(x_0\) is arbitrary, we know that \(f\) is continuous from \((X, d_X)\) to \((Y, d_Y)\).
  Thus statements (c) implies statement (a) and from the proof above we conclude that statements (a)(c) are equivalent.

  Next we show that statements (c) implies statement (d).
  Suppose that
  \[
    \forall V \subseteq Y, V \text{ is open in } (Y, d_Y) \implies f^{-1}(V) \text{ is open in } (X, d_X).
  \]
  Let \(F\) be an closed set in \((Y, d_Y)\).
  Then we have
  \begin{align*}
             & F \text{ is closed in } (Y, d_Y)                                                                                    \\
    \implies & Y \setminus F \text{ is open in } (Y, d_Y)                                        &  & \text{(by \cref{1.2.15}(e))} \\
    \implies & f^{-1}(Y \setminus F) \text{ is open in } (X, d_X)                                &  & \text{(by hypothesis)}       \\
    \implies & X \setminus f^{-1}(Y \setminus F) \text{ is closed in } (X, d_X)                  &  & \text{(by \cref{1.2.15}(e))} \\
    \implies & X \setminus \set{x \in X : f(x) \in Y \setminus F} \text{ is closed in } (X, d_X)                                   \\
    \implies & \set{x \in X : f(x) \in F} \text{ is closed in } (X, d_X)                                                           \\
    \implies & f^{-1}(F) \text{ is closed in } (X, d_X).
  \end{align*}
  Since \(F\) is arbitrary, we know that statement (c) implies statement (d).

  Finally we show that statements (d) implies statement (c).
  Suppose that
  \[
    \forall F \subseteq Y, F \text{ is closed in } (Y, d_Y) \implies f^{-1}(F) \text{ is closed in } (X, d_X).
  \]
  Let \(V\) be an open set in \((Y, d_Y)\).
  Then we have
  \begin{align*}
             & V \text{ is open in } (Y, d_Y)                                                                                    \\
    \implies & Y \setminus V \text{ is closed in } (Y, d_Y)                                    &  & \text{(by \cref{1.2.15}(e))} \\
    \implies & f^{-1}(Y \setminus V) \text{ is closed in } (X, d_X)                            &  & \text{(by hypothesis)}       \\
    \implies & X \setminus f^{-1}(Y \setminus V) \text{ is open in } (X, d_X)                  &  & \text{(by \cref{1.2.15}(e))} \\
    \implies & X \setminus \set{x \in X : f(x) \in Y \setminus V} \text{ is open in } (X, d_X)                                   \\
    \implies & \set{x \in X : f(x) \in V} \text{ is open in } (X, d_X)                                                           \\
    \implies & f^{-1}(V) \text{ is open in } (X, d_X).
  \end{align*}
  Since \(V\) is arbitrary, we know that statement (d) implies statement (c).
  We conclude that statements (a)(b)(c)(d) are all equivalent.
\end{proof}

\begin{rmk}\label{2.1.6}
  It may seem strange that continuity ensures that the \emph{inverse} image of an open set is open.
  One may guess instead that the reverse should be true, that the \emph{forward} image of an open set is open;
  but this is not true;
  see \cref{ex:1.5.4,ex:1.5.5}.
\end{rmk}

\begin{cor}[Continuity preserved by composition]\label{2.1.7}
  Let \((X, d_X)\), \((Y, d_Y)\), and \((Z, d_Z)\) be metric spaces.
  \begin{enumerate}
    \item If \(f : X \to Y\) is continuous at a point \(x_0 \in X\), and \(g : Y \to Z\) is continuous at \(f(x_0)\), then the composition \(g \circ f : X \to Z\), defined by \(g \circ f(x) \coloneqq g(f(x))\), is continuous at \(x_0\).
    \item If \(f : X \to Y\) is continuous, and \(g : Y \to Z\) is continuous, then \(g \circ f : X \to Z\) is also continuous.
  \end{enumerate}
\end{cor}

\begin{proof}{(a)}
  Since \(f\) is continuous at \(x_0\) from \((X, d_X)\) to \((Y, d_Y)\), by \cref{2.1.4}(a)(c) we know that
  \begin{align*}
             & \forall V \subseteq Y, \big(V \text{ is open in } (Y, d_Y)\big) \land \big(f(x_0) \in V\big)                        \\
    \implies & \exists U \subseteq X : \big(U \text{ is open in } (X, d_X)\big) \land (x_0 \in U) \land \big(f(U) \subseteq V\big)
  \end{align*}
  Now we choose such \(U\) for each open set \(V\) in \((Y, d_Y)\) and denote it as \(U_V\).
  Since \(g\) is continuous at \(f(x_0)\) from \((Y, d_Y)\) to \((Z, d_Z)\), by \cref{2.1.4}(a)(c) we know that
  \begin{align*}
             & \forall W \subseteq Z, \big(W \text{ is open in } (Z, d_Z)\big) \land \Big(g\big(f(x_0)\big) \in W\Big)                                                \\
    \implies & \exists V \subseteq Y : \big(V \text{ is open in } (Y, d_Y)\big) \land (y_0 \in V) \land \big(g(V) \subseteq W\big)                                    \\
    \implies & \exists U_V \subseteq X : \big(U_V \text{ is open in } (X, d_X)\big) \land (x_0 \in U_V) \land \big(f(U_V) \subseteq V\big)                            \\
    \implies & \exists U_V \subseteq X : \big(U_V \text{ is open in } (X, d_X)\big) \land (x_0 \in U_V) \land \Big(g\big(f(U_V)\big) \subseteq g(V) \subseteq W\Big).
  \end{align*}
  Thus by \cref{2.1.4}(a)(c) we know that \(g \circ f\) is continuous at \(x_0\) from \((X, d_X)\) to \((Z, d_Z)\).
\end{proof}

\begin{proof}{(b)}
  Let \(x_0 \in X\).
  Then we have
  \begin{align*}
             & f \text{ is continuous from } (X, d_X) \text{ to } (Y, d_Y)                                  \\
    \implies & f \text{ is continuous at } x_0 \text{ from } (X, d_X) \text{ to } (Y, d_Y). &  & \by{2.1.1}
  \end{align*}
  Since \(f(x_0) \in Y\), we have
  \begin{align*}
             & g \text{ is continuous from } (Y, d_Y) \text{ to } (Z, d_Z)                                                           \\
    \implies & g \text{ is continuous at } f(x_0) \text{ from } (Y, d_Y) \text{ to } (Z, d_Z)       &  & \by{2.1.1}                  \\
    \implies & g \circ f \text{ is continuous at } x_0 \text{ from } (X, d_X) \text{ to } (Z, d_Z). &  & \text{(by \cref{2.1.7}(a))}
  \end{align*}
  Since \(x_0\) is arbitrary, by \cref{2.1.1} we know that \(g \circ f\) is continuous from \((X, d_X)\) to \((Z, d_Z)\).
\end{proof}

\exercisesection

\begin{ex}\label{ex:2.1.1}
  Prove \cref{2.1.4}.
\end{ex}

\begin{proof}
  See \cref{2.1.4}.
\end{proof}

\begin{ex}\label{ex:2.1.2}
  Prove \cref{2.1.5}.
\end{ex}

\begin{proof}
  See \cref{2.1.5}.
\end{proof}

\begin{ex}\label{ex:2.1.3}
  Use \cref{2.1.4} and \cref{2.1.5} to prove \cref{2.1.7}.
\end{ex}

\begin{proof}
  See \cref{2.1.7}.
\end{proof}

\begin{ex}\label{ex:2.1.4}
  Give an example of functions \(f : \R \to \R\) and \(g : \R \to \R\) such that
  \begin{enumerate}
    \item \(f\) is not continuous, but \(g\) and \(g \circ f\) are continuous;
    \item \(g\) is not continuous, but \(f\) and \(g \circ f\) are continuous;
    \item \(f\) and \(g\) are not continuous, but \(g \circ f\) is continuous.
  \end{enumerate}
  Explain briefly why these examples do not contradict \cref{2.1.7}.
\end{ex}

\begin{proof}{(a)}
  Let \(f : \R \to \R\) be the function
  \[
    \forall x \in \R, f(x) = \begin{dcases}
      1 & \text{if } x = 0    \\
      0 & \text{if } x \neq 0
    \end{dcases}
  \]
  and let \(g : \R \to \R\) be the function \(g(x) = 0\) for all \(x \in \R^+\).
  Then we know that \(f\) is not continuous at \(0\) from \((\R, d_{l^1}|_{\R \times \R})\) to \((\R, d_{l^1}|_{\R \times \R})\) and thus \(f\) is not continuous from \((\R, d_{l^1}|_{\R \times \R})\) to \((\R, d_{l^1}|_{\R \times \R})\).
  Since \(g\) is constant function, we know that \(g\) is continuous from \((\R, d_{l^1}|_{\R \times \R})\) to \((\R, d_{l^1}|_{\R \times \R})\).
  Since \(g \circ f\) is also a constant function, we know that \(g \circ f\) is continuous from \((\R, d_{l^1}|_{\R \times \R})\) to \((\R, d_{l^1}|_{\R \times \R})\).
  This does not contradict to \cref{2.1.7} since \(f\) is not continuous from \((\R, d_{l^1}|_{\R \times \R})\) to \((\R, d_{l^1}|_{\R \times \R})\).
\end{proof}

\begin{proof}{(b)}
  Let \(f : \R \to \R\) be the function \(f(x) = 0\) for all \(x \in \R^+\).
  Let \(g : \R \to \R\) be the function
  \[
    \forall x \in \R, g(x) = \begin{dcases}
      1 & \text{if } x = 0    \\
      0 & \text{if } x \neq 0
    \end{dcases}
  \]
  Since \(f\) is constant function, we know that \(f\) is continuous from \((\R, d_{l^1}|_{\R \times \R})\) to \((\R, d_{l^1}|_{\R \times \R})\).
  Since \(g\) is not continuous at \(0\) from \((\R, d_{l^1}|_{\R \times \R})\) to \((\R, d_{l^1}|_{\R \times \R})\), we know that \(g\) is not continuous from \((\R, d_{l^1}|_{\R \times \R})\) to \((\R, d_{l^1}|_{\R \times \R})\).
  Since \(g \circ f\) is a constant function, we know that \(g \circ f\) is continuous from \((\R, d_{l^1}|_{\R \times \R})\) to \((\R, d_{l^1}|_{\R \times \R})\).
  This does not contradict to \cref{2.1.7} since \(g\) is not continuous from \((\R, d_{l^1}|_{\R \times \R})\) to \((\R, d_{l^1}|_{\R \times \R})\).
\end{proof}

\begin{proof}{(c)}
  Let \(f : \R \to \R\) be the function
  \[
    \forall x \in \R, f(x) = \begin{dcases}
      1 & \text{if } x = 0    \\
      0 & \text{if } x \neq 0
    \end{dcases}
  \]
  and let \(g : \R \to \R\) be the function
  \[
    \forall x \in \R, g(x) = \begin{dcases}
      1 & \text{if } x = 2    \\
      0 & \text{if } x \neq 2
    \end{dcases}
  \]
  Since \(f\) is not continuous at \(0\) from \((\R, d_{l^1}|_{\R \times \R})\) to \((\R, d_{l^1}|_{\R \times \R})\), we know that \(f\) is not continuous from \((\R, d_{l^1}|_{\R \times \R})\) to \((\R, d_{l^1}|_{\R \times \R})\).
  Similarly \(g\) is not continuous from \((\R, d_{l^1}|_{\R \times \R})\) to \((\R, d_{l^1}|_{\R \times \R})\).
  Since \(g \circ f\) is a constant function, we know that \(g \circ f\) is continuous from \((\R, d_{l^1}|_{\R \times \R})\) to \((\R, d_{l^1}|_{\R \times \R})\).
  This does not contradict to \cref{2.1.7} since \(f, g\) are not continuous from \((\R, d_{l^1}|_{\R \times \R})\) to \((\R, d_{l^1}|_{\R \times \R})\).
\end{proof}

\begin{ex}\label{ex:2.1.5}
  Let \((X, d)\) be a metric space, and let \((E, d|_{E \times E})\) be a subspace of \((X, d)\).
  Let \(\iota_{E \to X} : E \to X\) be the inclusion map, defined by setting \(\iota_{E \to X}(x) \coloneqq x\) for all \(x \in E\).
  Show that \(\iota_{E \to X}\) is continuous.
\end{ex}

\begin{proof}
  Let \(x_0 \in E\).
  Since
  \begin{align*}
             & \forall \varepsilon \in \R^+, \forall x \in E, d|_{E \times E}(x, x_0) < \varepsilon                                                                              \\
    \implies & d|_{E \times E}\big(\iota_{E \to X}(x), \iota_{E \to X}(x_0)\big) = d|_{E \times E}(x, x_0) < \varepsilon, &  & \text{(by the definition of \(\iota_{E \to X}\))}
  \end{align*}
  by setting \(\delta = \varepsilon\) we have
  \begin{align*}
     & \forall \varepsilon \in \R^+, \exists \delta \in \R^+ :                                                                                              \\
     & \Big(\forall x \in E, d|_{E \times E}(x, x_0) < \delta \implies d|_{E \times E}\big(\iota_{E \to X}(x), \iota_{E \to X}(x_0)\big) < \varepsilon\Big)
  \end{align*}
  and thus by \cref{2.1.1} \(\iota_{E \to X}\) is continuous at \(x_0\) from \((E, d|_{E \times E})\) to \((X, d)\).
  Since \(x_0\) is arbitrary, by \cref{2.1.1} \(\iota_{E \to X}\) is continuous from \((E, d|_{E \times E})\) to \((X, d)\).
\end{proof}

\begin{ex}\label{ex:2.1.6}
  Let \(f : X \to Y\) be a function from one metric space \((X, d_X)\) to another \((Y, d_Y)\).
  Let \(E\) be a subset of \(X\) (which we give the induced metric \(d_X|_{E \times E}\)), and let \(f|_E : E \to Y\) be the restriction of \(f\) to \(E\), thus \(f|_E(x) \coloneqq f(x)\) when \(x \in E\).
  If \(x_0 \in E\) and \(f\) is continuous at \(x_0\), show that \(f|_E\) is also continuous at \(x_0\).
  (Is the converse of this statement true? Explain.)
  Conclude that if \(f\) is continuous, then \(f|_E\) is continuous.
  Thus restriction of the domain of a function does not destroy continuity.
\end{ex}

\begin{proof}
  See \cref{2.1.3}.
  The converse is not true since the statement
  \[
    \forall \varepsilon \in \R^+, \exists \delta \in \R^+ : \Big(\forall x \in E, d_X|_{E \times E}(x, x_0) < \delta \implies d_Y\big(f(x), f(x_0)\big) < \varepsilon\Big)
  \]
  does not imply the statement
  \[
    \forall \varepsilon \in \R^+, \exists \delta \in \R^+ : \Big(\forall x \in X, d_X(x, x_0) < \delta \implies d_Y\big(f(x), f(x_0)\big) < \varepsilon\Big).
  \]
\end{proof}

\begin{ex}\label{ex:2.1.7}
  Let \(f : X \to Y\) be a function from one metric space \((X, d_X)\) to another \((Y, d_Y)\).
  Suppose that the image \(f(X)\) of \(X\) is contained in some subset \(E \subseteq Y\) of \(Y\).
  Let \(g : X \to E\) be the function which is the same as \(f\) but with the range restricted from \(Y\) to \(E\), thus \(g(x) = f(x)\) for all \(x \in X\).
  We give \(E\) the metric \(d_Y|_{E \times E}\) induced from \(Y\).
  Show that for any \(x_0 \in X\), that \(f\) is continuous at \(x_0\) iff \(g\) is continuous at \(x_0\).
  Conclude that \(f\) is continuous iff \(g\) is continuous.
  (Thus the notion of continuity is not affected if one restricts the range of the function.)
\end{ex}

\begin{proof}
  Since \(f(X) \subseteq E \subseteq Y\), we have
  \[
    \forall x \in X, d_Y(x, x_0) = d_Y|_{E \times E}(x, x_0).
  \]
  Thus by \cref{2.1.1} we have
  \begin{align*}
         & \forall \varepsilon \in \R^+, \exists \delta \in \R^+ : \Big(\forall x \in X, d_X(x, x_0) < \delta \implies d_Y\big(f(x), f(x_0)\big) < \varepsilon\Big)                \\
    \iff & \forall \varepsilon \in \R^+, \exists \delta \in \R^+ : \Big(\forall x \in X, d_X(x, x_0) < \delta \implies d_Y|_{E \times E}\big(f(x), f(x_0)\big) < \varepsilon\Big).
  \end{align*}
  This means \(f\) is continuous at \(x_0\) from \((X, d_X)\) to \((Y, d_Y)\) iff \(g\) is continuous at \(x_0\) from \((X, d_X)\) to \((E, d_Y|_{E \times E})\).
  By \cref{2.1.1} we conclude that \(f\) is continuous from \((X, d_X)\) to \((Y, d_Y)\) iff \(g\) is continuous from \((X, d_X)\) to \((E, d_Y|_{E \times E})\).
\end{proof}
\section{Continuity and product spaces}\label{ii:sec:2.2}

\begin{note}
  Given two functions \(f : X \to Y\) and \(g : X \to Z\), one can define their \emph{direct sum} \(f \oplus g : X \to Y \times Z\) defined by \(f \oplus g(x) \coloneqq \big(f(x), g(x)\big)\), i.e., this is the function taking values in the Cartesian product \(Y \times Z\) whose first co-ordinate is \(f(x)\) and whose second co-ordinate is \(g(x)\)
  (cf. Exercise 3.5.7 in Analysis I).
\end{note}

\begin{lem}\label{ii:2.2.1}
  Let \(f : X \to \R\) and \(g : X \to \R\) be functions, and let \(f \oplus g : X \to \R^2\) be their direct sum.
  We give \(\R^2\) the Euclidean metric.
  \begin{enumerate}
    \item If \(x_0 \in X\), then \(f\) and \(g\) are both continuous at \(x_0\) iff \(f \oplus g\) is continuous at \(x_0\).
    \item \(f\) and \(g\) are both continuous iff \(f \oplus g\) is continuous.
  \end{enumerate}
\end{lem}

\begin{proof}{(a)}
  Let \((X, d)\) be a metric space.
  Then we have
  \begin{align*}
         & f, g \text{ are continuous at } x_0                                                                                                             \\
         & \text{from } (X, d) \text{ to } (\R, d_{l^2}|_{\R \times \R})                                                                                   \\
    \iff & \text{every sequence } (x^{(n)})_{n = 1}^\infty \text{ in } X \text{ satisfies the following:}                                                  \\
         & \lim_{n \to \infty} d\big(x^{(n)}, x_0\big) = 0 \text{ implies }                                                                                \\
         & \begin{dcases}
             \lim_{n \to \infty} d_{l^2}|_{\R \times \R}\big(f(x^{(n)}), f(x_0)\big) = 0 \\
             \lim_{n \to \infty} d_{l^2}|_{\R \times \R}\big(g(x^{(n)}), g(x_0)\big) = 0
           \end{dcases}                                          &  & \by{ii:2.1.4}[a,b]                                                                   \\
    \iff & \text{every sequence } (x^{(n)})_{n = 1}^\infty \text{ in } X \text{ satisfies the following:}                                                  \\
         & \lim_{n \to \infty} d\big(x^{(n)}, x_0\big) = 0 \text{ implies }                                                                                \\
         & \lim_{n \to \infty} d_{l^2}|_{\R^2 \times \R^2}\Big(\big(f(x^{(n)}), g(x^{(n)})\big), \big(f(x_0), g(x_0)\big)\Big) = 0 &  & \by{ii:1.1.18}     \\
    \iff & f \oplus g \text{ is continuous at } x_0                                                                                                        \\
         & \text{from } (X, d) \text{ to } (\R^2, d_{l^2}|_{\R^2 \times \R^2}).                                                    &  & \by{ii:2.1.4}[a,b]
  \end{align*}
\end{proof}

\begin{proof}{(b)}
  Let \((X, d)\) be a metric space.
  Then we have
  \begin{align*}
         & f, g \text{ are continuous from } (X, d) \text{ to } (\R, d_{l^2}|_{\R \times \R})                                   \\
    \iff & \forall x_0 \in X, f, g \text{ are continuous at } x_0                                                               \\
         & \text{from } (X, d) \text{ to } (\R, d_{l^2}|_{\R \times \R})                                  &  & \by{ii:2.1.1}    \\
    \iff & \forall x_0 \in X, f \oplus g \text{ is continuous at } x_0                                    &  & \by{ii:2.2.1}[a] \\
         & \text{from } (X, d) \text{ to } (\R^2, d_{l^2}|_{\R^2 \times \R^2})                                                  \\
    \iff & f \oplus g \text{ is continuous from } (X, d) \text{ to } (\R^2, d_{l^2}|_{\R^2 \times \R^2}). &  & \by{ii:2.1.1}
  \end{align*}
\end{proof}

\begin{ac}\label{ii:ac:2.2.1}
  Let \((X, d)\) be a metric space.
  Let \((\R, d_{\R})\) be a metric space where \(d_{\R}\) can be \(d_{l^1}|_{\R \times \R}\), \(d_{l^2}|_{\R \times \R}\) or \(d_{l^\infty}|_{\R \times \R}\).
  Let \((\R^2, d_{\R^2})\) be a metric space where \(d_{\R^2}\) can be \(d_{l^1}|_{\R^2 \times \R^2}\), \(d_{l^2}|_{\R^2 \times \R^2}\) or \(d_{l^\infty}|_{\R^2 \times \R^2}\).
  Let \(f : X \to \R\) and \(g : X \to \R\) be functions, and let \(f \oplus g : X \to \R^2\) be their direct sum.
  \begin{enumerate}
    \item If \(x_0 \in X\), then \(f\) and \(g\) are both continuous at \(x_0\) from \((X, d)\) to \((\R, d_{\R})\) iff \(f \oplus g\) is continuous at \(x_0\) from \((X, d)\) to \((\R^2, d_{\R^2})\).
    \item \(f\) and \(g\) are both continuous from \((X, d)\) to \((\R, d_{\R})\) iff \(f \oplus g\) is continuous from \((X, d)\) to \((\R^2, d_{\R^2})\).
  \end{enumerate}
\end{ac}

\begin{proof}
  By \cref{ii:1.1.18} and \cref{ii:2.2.1} we are done.
\end{proof}

\begin{lem}\label{ii:2.2.2}
  The addition function \((x, y) \mapsto x + y\), the subtraction function \((x, y) \mapsto x - y\), the multiplication function \((x, y) \mapsto xy\), the maximum function \((x, y) \mapsto \max(x, y)\), and the minimum function \((x, y) \mapsto \min(x, y)\), are all continuous functions from \(\R^2\) to \(\R\).
  The division function \((x, y) \mapsto x / y\) is a continuous function from \(\R \times (\R \setminus \set{0}) = \set{(x, y) \in \R^2 : y \neq 0}\) to \(\R\).
  For any real number \(c\), the function \(x \mapsto cx\) is a continuous function from \(\R\) to \(\R\).
\end{lem}

\begin{proof}
  First we show that the addition, subtraction, multiplication, maximum and minimum functions from \(\R^2\) to \(\R\) are continuous from \((\R^2, d_{l^1}|_{\R^2 \times \R^2})\) to \((\R, d_{l^1}|_{\R \times \R})\).
  Let \((x, y) \in \R^2\) and let \((x^{(n)}, y^{(n)})_{n = 1}^\infty\) be a sequence in \(\R^2\) such that
  \[
    \lim_{n \to \infty} d_{l^1}|_{\R^2 \times \R^2}\big((x^{(n)}, y^{(n)}), (x, y)\big) = 0
  \]
  By limit laws we know that
  \begin{align*}
     & \lim_{n \to \infty} d_{l^1}|_{\R \times \R}(x^{(n)} + y^{(n)}, x + y) = 0                   \\
     & \lim_{n \to \infty} d_{l^1}|_{\R \times \R}(x^{(n)} - y^{(n)}, x - y) = 0                   \\
     & \lim_{n \to \infty} d_{l^1}|_{\R \times \R}(x^{(n)} y^{(n)}, xy) = 0                        \\
     & \lim_{n \to \infty} d_{l^1}|_{\R \times \R}\big(\max(x^{(n)}, y^{(n)}), \max(x, y)\big) = 0 \\
     & \lim_{n \to \infty} d_{l^1}|_{\R \times \R}\big(\min(x^{(n)}, y^{(n)}), \min(x, y)\big) = 0 \\
  \end{align*}
  Since \((x^{(n)}, y^{(n)})_{n = 1}^\infty\) was arbitrary, by \cref{ii:2.1.4}(a)(b) we know that the addition, subtraction, multiplication, maximum and minimum functions from \(\R^2\) to \(\R\) are continuous at \((x, y)\) from \((\R^2, d_{l^1}|_{\R^2 \times \R^2})\) to \((\R, d_{l^1}|_{\R \times \R})\).
  Since \((x, y)\) was arbitrary, by \cref{ii:2.1.5}(a)(b) we know that the addition, subtraction, multiplication, maximum and minimum functions from \(\R^2\) to \(\R\) are continuous from \((\R^2, d_{l^1}|_{\R^2 \times \R^2})\) to \((\R, d_{l^1}|_{\R \times \R})\).

  Next we show that the division function from \(E = \R \times (\R \setminus \set{0})\) to \(\R\) is continuous from \((E, d_{l^1}|_{E \times E})\) to \((\R, d_{l^1}|_{\R \times \R})\).
  Let \((x, y) \in E\) and let \((x^{(n)}, y^{(n)})_{n = 1}^\infty\) be a sequence in \(E\) such that
  \[
    \lim_{n \to \infty} d_{l^1}|_{E \times E}\big((x^{(n)}, y^{(n)}), (x, y)\big) = 0
  \]
  By limit laws we know that
  \[
    \lim_{n \to \infty} d_{l^1}|_{\R \times \R}(x^{(n)} / y^{(n)}, x / y) = 0.
  \]
  Thus, using similar arguments as above, we know that the division function from \(E\) to \(\R\) is continuous from \((E, d_{l^1}|_{E \times E})\) to \((\R, d_{l^1}|_{\R \times \R})\).

  Finally we show that the constant multiplication function from \(\R\) to \(\R\) is continuous from \((\R, d_{l^1}|_{\R \times \R})\) to \((\R, d_{l^1}|_{\R \times \R})\).
  Let \(c, x \in \R\) and let \((x^{(n)})_{n = 1}^\infty\) be a sequence in \(\R\) such that
  \[
    \lim_{n \to \infty} d_{l^1}|_{\R \times \R}(x^{(n)}, x) = 0.
  \]
  By limit laws we know that
  \[
    \lim_{n \to \infty} d_{l^1}|_{\R \times \R}(cx^{(n)}, cx) = 0.
  \]
  Thus, using similar arguments as above, we know that the constant function from \(\R\) to \(\R\) is continuous from \((\R, d_{l^1}|_{\R \times \R})\) to \((\R, d_{l^1}|_{\R \times \R})\).
\end{proof}

\begin{cor}\label{ii:2.2.3}
  Let \((X, d)\) be a metric space, let \(f : X \to \R\) and \(g : X \to \R\) be functions.
  Let \(c\) be a real number.
  \begin{enumerate}
    \item If \(x_0 \in X\) and \(f\) and \(g\) are continuous at \(x_0\), then the functions \(f + g : X \to \R\), \(f - g : X \to \R\), \(fg : X \to \R\), \(\max(f, g) : X \to \R\), \(\min(f, g) : X \to \R\), and \(cf : X \to \R\) (see Definition 9.2.1 in Analysis I for definitions) are also continuous at \(x_0\).
          If \(g(x) \neq 0\) for all \(x \in X\), then \(f / g : X \to \R\) is also continuous at \(x_0\).
    \item If \(f\) and \(g\) are continuous, then the functions \(f + g : X \to \R\), \(f - g : X \to \R\), \(fg : X \to \R\), \(\max(f, g) : X \to \R\), \(\min(f, g) : X \to \R\), and \(cf : X \to \R\) are also continuous.
          If \(g(x) \neq 0\) for all \(x \in X\), then \(f / g : X \to \R\) is also continuous.
  \end{enumerate}
\end{cor}

\begin{proof}
  We first prove (a). Since \(f, g\) are continuous at \(x_0\) from \((X, d)\) to \((\R, d_{l^1}|_{\R \times \R})\), then by \cref{ii:ac:2.2.1}(a) \(f \oplus g : X \to \R^2\) is also continuous at \(x_0\) from \((X, d)\) to \((\R^2, d_{l^1}|_{\R^2 \times \R^2})\).
  On the other hand, from \cref{ii:2.2.2} the function \((x, y) \mapsto x + y\) is continuous from \((\R^2, d_{l^1}|_{\R^2 \times \R^2})\) to \((\R, d_{l^1}|_{\R \times \R})\), and in particular is continuous at \(f \oplus g(x_0)\) from \((\R^2, d_{l^1}|_{\R^2 \times \R^2})\) to \((\R, d_{l^1}|_{\R \times \R})\).
  If we then compose these two functions using \cref{ii:2.1.7} we conclude that \(f + g : X \to \R\) is continuous from \((X, d)\) to \((\R, d_{l^1}|_{\R \times \R})\).
  A similar argument gives the continuity of \(f - g\), \(fg\), \(\max(f, g)\), \(\min(f, g)\) and \(cf\).
  To prove the claim for \(f / g\), we first use \cref{ii:ex:2.1.7} to restrict the codomain of \(g\) from \(\R\) to \(\R \setminus \set{0}\), and then one can argue as before.
  The claim (b) follows immediately from (a).
\end{proof}

\exercisesection

\begin{ex}\label{ii:ex:2.2.1}
  Prove \cref{ii:2.2.1}.
\end{ex}

\begin{proof}
  See \cref{ii:2.2.1}.
\end{proof}

\begin{ex}\label{ii:ex:2.2.2}
  Prove \cref{ii:2.2.2}.
\end{ex}

\begin{proof}
  See \cref{ii:2.2.2}.
\end{proof}

\begin{ex}\label{ii:ex:2.2.3}
  Show that if \(f : X \to \R\) is a continuous function, so is the function \(\abs{f} : X \to \R\) defined by \(\abs{f}(x) \coloneqq \abs{f(x)}\).
\end{ex}

\begin{proof}
  Let \((X, d_X)\) be a metric space and let \(f : X \to \R\) be a function which is continuous from \((X, d_X)\) to \(\).
  Since
  \begin{align*}
             & \forall x_0 \in X, \abs{f}(x) = \abs{f(x)} = \max\big(-f(x), f(x)\big) \\
    \implies & \abs{f} = \max(-f, f),
  \end{align*}
  we have
  \begin{align*}
             & f \text{ is continuous from } (X, d_X) \text{ to } (\R, d_{l^1}|_{\R \times \R})                                 \\
    \implies & -f \text{ is continuous from } (X, d_X) \text{ to } (\R, d_{l^1}|_{\R \times \R})          &  & \by{ii:2.2.3}[b] \\
    \implies & \max(f, -f) \text{ is continuous from } (X, d_X) \text{ to } (\R, d_{l^1}|_{\R \times \R}) &  & \by{ii:2.2.3}[b] \\
    \implies & \abs{f} \text{ is continuous from } (X, d_X) \text{ to } (\R, d_{l^1}|_{\R \times \R}).
  \end{align*}
\end{proof}

\begin{ex}\label{ii:ex:2.2.4}
  Let \(\pi_1 : \R^2 \to \R\) and \(\pi_2 : \R^2 \to \R\) be the functions \(\pi_1(x, y) \coloneqq x\) and \(\pi_2(x, y) \coloneqq y\) (these two functions are sometimes called the \emph{co-ordinate functions} on \(\R^2\)).
  Show that \(\pi_1\) and \(\pi_2\) are continuous.
  Conclude that if \(f : \R \to X\) is any continuous function into a metric space \((X, d)\), then the functions \(g_1 : \R^2 \to X\) and \(g_2 : \R^2 \to X\) defined by \(g_1(x, y) \coloneqq f(x)\) and \(g_2(x, y) \coloneqq f(y)\) are also continuous.
\end{ex}

\begin{proof}
  Let \((x, y) \in \R^2\).
  We know that
  \begin{align*}
             & \forall \varepsilon \in \R^+, \forall (x', y') \in \R^2,                                                                \\
             & d_{l^1}|_{\R^2 \times \R^2}\big((x, y), (x', y')\big) < \varepsilon                                                     \\
    \implies & \abs{x - x'} + \abs{y - y'} < \varepsilon                                  &  & \by{ii:1.1.7}                           \\
    \implies & \abs{x - x'} < \varepsilon                                                 &  & \by{ii:1.1.2}[a,b]                      \\
    \implies & d_{l^1}|_{\R \times \R}(x, x') < \varepsilon                               &  & \by{ii:1.1.7}                           \\
    \implies & d_{l^1}|_{\R \times \R}\big(\pi_1(x, y), \pi_1(x', y')\big) < \varepsilon. &  & \text{(by the definition of \(\pi_1\))}
  \end{align*}
  Thus by setting \(\delta = \varepsilon\) we have
  \begin{align*}
     & \forall \varepsilon \in \R^+, \exists \delta \in \R^+ :                                                                                                                                 \\
     & \Big(\forall (x', y') \in \R^2, d_{l^1}|_{\R^2 \times \R^2}\big((x, y), (x', y')\big) < \delta \implies d_{l^1}|_{\R \times \R}\big(\pi_1(x, y), \pi_1(x', y')\big) < \varepsilon\Big).
  \end{align*}
  Since \((x, y)\) was arbitrary, by \cref{ii:2.1.1} \(\pi_1\) is continuous from \((\R^2, d_{l^1}|_{\R^2 \times \R^2})\) to \((\R, d_{l^1}|_{\R \times \R})\).
  Using similar arguments, we can show that \(\pi_2\) is continuous from \((\R^2, d_{l^1}|_{\R^2 \times \R^2})\) to \((\R, d_{l^1}|_{\R \times \R})\).

  Let \(f : \R \to X\) be a function which is continuous from \((\R, d_{l^1}|_{\R \times \R})\) to \((X, d)\).
  Let \(g_1 : \R^2 \to X\) and \(g_2 : \R^2 \to X\) be functions where
  \[
    \forall (x, y) \in \R^2, \begin{dcases}
      g_1(x, y) = f(x) \\
      g_2(x, y) = f(y)
    \end{dcases}
  \]
  Since
  \begin{align*}
     & \forall (x, y) \in \R^2,                                         \\
     & f \circ \pi_1(x, y) = f\big(\pi_1(x, y)\big) = f(x) = g_1(x, y); \\
     & f \circ \pi_2(x, y) = f\big(\pi_2(x, y)\big) = f(y) = g_2(x, y),
  \end{align*}
  we know that \(g_1 = f \circ \pi_1\) and \(g_2 = f \circ \pi_2\).
  Thus by \cref{ii:2.1.7}(b) \(g_1, g_2\) are continuous from \((\R^2, d_{l^1}|_{\R^2 \times \R^2})\) to \((X, d)\).
\end{proof}

\begin{ex}\label{ii:ex:2.2.5}
  Let \(n, m \geq 0\) be integers.
  Suppose that for every \(0 \leq i \leq n\) and \(0 \leq j \leq m\) we have a real number \(c_{ij}\).
  Form the function \(P : \R^2 \to \R\) defined by
  \[
    P(x, y) \coloneqq \sum_{i = 0}^n \sum_{j = 0}^m c_{ij} x^i y^j.
  \]
  (Such a function is known as a \emph{polynomial of two variables})
  Show that \(P\) is continuous.
  Conclude that if \(f : X \to \R\) and \(g : X \to \R\) are continuous functions, then the function \(P(f, g) : X \to \R\) defined by \(P(f, g)(x) \coloneqq P\big(f(x), g(x)\big)\) is also continuous.
\end{ex}

\begin{proof}
  First we show that \(P\) is continuous from \((\R^2, d_{l^1}|_{\R^2 \times \R^2})\) to \((\R, d_{l^1}|_{\R \times \R})\).
  Let \((x, y) \in \R^2\).
  Let \(\pi_1, \pi_2\) be the functions defined in \cref{ii:ex:2.2.4}.
  Since \(\pi_1\) is continuous from \((\R^2, d_{l^1}|_{\R^2 \times \R^2})\) to \((\R, d_{l^1}|_{\R \times \R})\), by \cref{ii:2.2.3}(b) we know that
  \[
    x^i = \prod_{k = 1}^i x = \prod_{k = 1}^i \pi_1(x, y)
  \]
  is continuous at \((x, y)\) from \((\R^2, d_{l^1}|_{\R^2 \times \R^2})\) to \((\R, d_{l^1}|_{\R \times \R})\) for every \(0 \leq i \leq n\).
  Similarly
  \[
    y^j = \prod_{k = 1}^j y = \prod_{k = 1}^j \pi_2(x, y)
  \]
  is continuous at \((x, y)\) from \((\R^2, d_{l^1}|_{\R^2 \times \R^2})\) to \((\R, d_{l^1}|_{\R \times \R})\) for every \(0 \leq j \leq m\).
  Thus by \cref{ii:2.2.3}(b) we know that \(c_{ij} x^i y^j\) is continuous at \((x, y)\) from \((\R^2, d_{l^1}|_{\R^2 \times \R^2})\) to \((\R, d_{l^1}|_{\R \times \R})\) for every \(0 \leq i \leq n\) and \(0 \leq j \leq m\), and
  \[
    \sum_{i = 0}^n \sum_{j = 0}^m c_{ij} x^i y^j = P(x, y)
  \]
  is continuous at \((x, y)\) from \((\R^2, d_{l^1}|_{\R^2 \times \R^2})\) to \((\R, d_{l^1}|_{\R \times \R})\).
  Since \((x, y)\) was arbitrary, by \cref{ii:2.1.1} we know that \(P\) is continuous from \((\R^2, d_{l^1}|_{\R^2 \times \R^2})\) to \((\R, d_{l^1}|_{\R \times \R})\).

  Now suppose that \(f : X \to \R\) and \(g : X \to \R\) are to continuous functions from \((X, d)\) to \((\R, d_{l^1}|_{\R \times \R})\).
  Then we have
  \begin{align*}
             & f \oplus g \text{ is continuous }                                                                            \\
             & \text{from } (X, d) \text{ to } (\R^2, d_{l^1}|_{\R^2 \times \R^2}) &  & \by{ii:ac:2.2.1}[b]                 \\
    \implies & P \circ (f \oplus g) \text{ is continuous }                                                                  \\
             & \text{from } (X, d) \text{ to } (\R, d_{l^1}|_{\R \times \R})       &  & \by{ii:2.1.7}[b]                    \\
    \implies & P(f, g) \text{ is continuous }                                                                               \\
             & \text{from } (X, d) \text{ to } (\R, d_{l^1}|_{\R \times \R}).      &  & \text{(by the definition of \(P\))}
  \end{align*}
\end{proof}

\begin{ex}\label{ii:ex:2.2.6}
  Let \(\R^m\) and \(\R^n\) be Euclidean spaces.
  If \(f : X \to \R^m\) and \(g : X \to \R^n\) are continuous functions, show that \(f \oplus g : X \to \R^{m + n}\) is also continuous, where we have identified \(\R^m \times \R^n\) with \(\R^{m + n}\) in the obvious manner.
  Is the converse statement true?
\end{ex}

\begin{proof}
  Let \((X, d)\) be a metric space.
  For each \(k \in \Z^+\), let \(d_k\) be one of the metric functions \(d_{l^1}|_{\R^k \times \R^k}\), \(d_{l^2}|_{\R^k \times \R^k}\) or \(d_{l^\infty}|_{\R^k \times \R^k}\).
  Let \(f : X \to \R^m\) be a continuous function from \((X, d)\) to \((\R^m, d_m)\), and let \(g : X \to \R^n\) be a continuous function from \((X, d)\) to \((\R^n, d_n)\).
  Let \(x_0 \in X\).
  Then we have
  \begin{align*}
         & \begin{dcases}
             f \text{ is continuous from } (X, d) \text{ to } (\R^m, d_m) \\
             g \text{ is continuous from } (X, d) \text{ to } (\R^n, d_n)
           \end{dcases}                                                                              \\
    \iff & \text{every sequence } (x^{(k)})_{k = 1}^\infty \text{ in } X \text{ satisfies the following:}                                            \\
         & \lim_{k \to \infty} d\big(x^{(k)}, x_0\big) = 0 \text{ implies }                                                                          \\
         & \begin{dcases}
             \lim_{k \to \infty} d_m\big(f(x^{(k)}), f(x_0)\big) = 0 \\
             \lim_{k \to \infty} d_n\big(g(x^{(k)}), g(x_0)\big) = 0
           \end{dcases}                                                        &  & \by{ii:2.1.4}[a,b]                                               \\
    \iff & \text{every sequence } (x^{(k)})_{k = 1}^\infty \text{ in } X \text{ satisfies the following:}                                            \\
         & \lim_{k \to \infty} d\big(x^{(k)}, x_0\big) = 0 \text{ implies }                                                                          \\
         & \lim_{k \to \infty} d_{m + n}\Big(\big(f(x^{(k)}) \oplus g(x^{(k)})\big), \big(f(x_0) \oplus g(x_0)\big)\Big) = 0 &  & \by{ii:1.1.18}     \\
    \iff & f \oplus g \text{ is continuous at } x_0                                                                                                  \\
         & \text{from } (X, d) \text{ to } (\R^{m + n}, d_{m + n}).                                                          &  & \by{ii:2.1.4}[a,b]
  \end{align*}
  Thus the statment is true and the converse is also true.
\end{proof}

\begin{ex}\label{ii:ex:2.2.7}
  Let \(k \geq 1\), let \(I\) be a finite subset of \(\N^k\), and let \(c : I \to \R\) be a function.
  Form the function \(P : \R^k \to \R\) defined by
  \[
    P(x_1, \dots, x_k) \coloneqq \sum_{(i_1, \dots, i_k) \in I} c(i_1, \dots, i_k) x_1^{i_1} \dots x_k^{i_k}.
  \]
  (Such a function is known as a \emph{polynomial of \(k\) variables};
  Show that \(P\) is continuous.
\end{ex}

\begin{proof}
  For each \(k \in \Z^+\), let \(I_k\) be the finite subset of \(\N^k\), let \(d_k = d_{l^1}|_{\R^k \times \R^k}\), and let \(P_k\) be a polynomial of \(k\) variables, i.e.,
  \[
    P_k(x_1, \dots, x_k) = \sum_{(i_1, \dots, i_k) \in I_k} \big(c(i_1, \dots, i_k) x_1^{i_1} \cdots x_k^{i_k}\big).
  \]
  We induct on \(k\) to show that \(P_k\) is continuous from \((\R^k, d_k)\) to \((\R, d_1)\) for every \(k \in \Z^+\).
  We start with \(k = 1\).
  For \(k = 1\), we have
  \begin{align*}
             & \forall i_1 \in I_1, c(i_1) x_1^{i_1} \text{ is continuous from } (\R, d_1) \text{ to } (\R, d_1) &  & \by{ii:2.2.2}[b]                     \\
    \implies & P_1(x_1) = \sum_{i_1 \in I_1} \big(c(i_1) x_1^{i_1}\big) \text{ is continuous}                    &  & \text{(note that \(I_1\) is finite)} \\
             & \text{from } (\R, d_1) \text{ to } (\R, d_1)                                                      &  & \by{ii:2.2.2}[b]
  \end{align*}
  and Thus, the base case holds.
  Suppose inductively that for some \(k \geq 1\), \(P_k\) is continuous from \((\R^k, d_k)\) to \((\R, d_1)\).
  Then for \(k + 1\), we need to show that \(P_{k + 1}\) is continuous from \((\R^{k + 1}, d_{k + 1})\) to \((\R, d_1)\).
  Let \(F : \R^k \to 2^\R\) be the function
  \[
    \forall (x_1, \dots, x_k, x_{k + 1}) \in \R^{k + 1}, F(x_1, \dots, x_k) = \set{x_{k + 1} \in \R : (x_1, \dots, x_k, x_{k + 1}) \in I_{k + 1}},
  \]
  and let \(A = \set{(i_1, \dots, i_k) \in \R^k : (i_1, \dots, i_k, i_{k + 1}) \in I_{k + 1}}\).
  Then we have
  \begin{align*}
     & P_{k + 1}(x_1, \dots, x_k, x_{k + 1})                                                                                                                                                                        \\
     & = \sum_{(i_1 \dots, i_k, i_{k + 1}) \in I_{k + 1}} c(i_1, \dots, i_k, i_{k + 1}) x_1^{i_1} \cdots x_k^{i_k} x_{k + 1}^{i_{k + 1}}                                                                            \\
     & = \sum_{(i_1 \dots, i_k) \in A} \bigg(\sum_{i_{k + 1} \in \R : (i_1, \dots, i_k, i_{k + 1}) \in I_{k + 1}} c(i_1, \dots, i_k, i_{k + 1}) x_1^{i_1} \cdots x_k^{i_k} x_{k + 1}^{i_{k + 1}}\bigg)              \\
     & = \sum_{(i_1 \dots, i_k) \in A} \Bigg(x_1^{i_1} \cdots x_k^{i_k} \bigg(\sum_{i_{k + 1} \in \R : (i_1, \dots, i_k, i_{k + 1}) \in I_{k + 1}} c(i_1, \dots, i_k, i_{k + 1}) x_{k + 1}^{i_{k + 1}}\bigg)\Bigg).
  \end{align*}
  By the induction hypothesis we know that
  \[
    \sum_{i_{k + 1} \in \R : (i_1, \dots, i_k, i_{k + 1}) \in I_{k + 1}} c(i_1, \dots, i_k, i_{k + 1}) x_{k + 1}^{i_{k + 1}}
  \]
  is continuous from \((\R, d_1)\) to \((\R, d_1)\) and
  \[
    x_1^{i_1} \cdots x_k^{i_k}
  \]
  is continuous from \((\R^k, d_k)\) to \((\R, d_1)\).
  Thus by \cref{ii:2.2.2}(b) we know that
  \begin{align*}
             & \forall (i_1, \dots, i_k, i_{k + 1}) \in I_{k + 1},                                                                                                                                                       \\
             & (x_1^{i_1} \cdots x_k^{i_k}) \bigg(\sum_{i_{k + 1} \in \R : (i_1, \dots, i_k, i_{k + 1}) \in I_{k + 1}} c(i_1, \dots, i_k, i_{k + 1}) x_{k + 1}^{i_{k + 1}}\bigg)                                         \\
             & \text{is continuous from } (\R^{k + 1}, d_{k + 1}) \text{ to } (\R, d_1)                                                                                                                                  \\
    \implies & \sum_{(i_1 \dots, i_k) \in A} \Bigg(x_1^{i_1} \cdots x_k^{i_k} \bigg(\sum_{i_{k + 1} \in \R : (i_1, \dots, i_k, i_{k + 1}) \in I_{k + 1}} c(i_1, \dots, i_k, i_{k + 1}) x_{k + 1}^{i_{k + 1}}\bigg)\Bigg) \\
             & \text{is continuous from } (\R^{k + 1}, d_{k + 1}) \text{ to } (\R, d_1)                                                                                                                                  \\
    \implies & P_{k + 1} \text{ is continuous from } (\R^{k + 1}, d_{k + 1}) \text{ to } (\R, d_1)
  \end{align*}
  and this closes the induction.
\end{proof}

\begin{ex}\label{ii:ex:2.2.8}
  Let \((X, d_X)\) and \((Y, d_Y)\) be metric spaces.
  Define the metric \(d_{X \times Y} : (X \times Y) \times (X \times Y) \to [0, \infty)\) by the formula
  \[
    d_{X \times Y}\big((x, y), (x', y')\big) \coloneqq d_X(x, x') + d_Y(y, y').
  \]
  Show that \((X \times Y, d_{X \times Y})\) is a metric space, and deduce an analogue of \cref{ii:1.1.18} and \cref{ii:2.2.1}.
\end{ex}

\begin{proof}
  We first show that \((X \times Y, d_{X \times Y})\) is a metric space.
  For any \((x, y) \in X \times Y\), we have
  \begin{align*}
    d_{X \times Y}\big((x, y), (x, y)\big) & = d_X(x, x) + d_Y(y, y)                       \\
                                           & = 0 + 0 = 0             &  & \by{ii:1.1.2}[a]
  \end{align*}
  and thus \((X \times Y, d_{X \times Y})\) satisfies \cref{ii:1.1.2}(a).
  For any \((x_1, y_1), (x_2, y_2) \in X \times Y\), we have
  \begin{align*}
             & (x_1, y_1) \neq (x_2, y_2)                                                                                  \\
    \implies & (x_1 \neq x_2) \lor (y_1 \neq y_2)                                                                          \\
    \implies & d_{X \times Y}\big((x_1, y_1), (x_2, y_2)\big) = d_X(x_1, x_2) + d_Y(y_1, y_2) \neq 0 &  & \by{ii:1.1.2}[b]
  \end{align*}
  and thus \((X \times Y, d_{X \times Y})\) satisfies \cref{ii:1.1.2}(b).
  For any \((x_1, y_1), (x_2, y_2) \in X \times Y\), we have
  \begin{align*}
     & d_{X \times Y}\big((x_1, y_1), (x_2, y_2)\big)                         \\
     & = d_X(x_1, x_2) + d_Y(y_1, y_2)                                        \\
     & = d_X(x_2, x_1) + d_Y(y_2, y_1)                  &  & \by{ii:1.1.2}[c] \\
     & = d_{X \times Y}\big((x_2, y_2), (x_1, y_1)\big)
  \end{align*}
  and thus \((X \times Y, d_{X \times Y})\) satisfies \cref{ii:1.1.2}(c).
  For any \((x_1, y_1), (x_2, y_2), (x_3, y_3) \in X \times Y\), we have
  \begin{align*}
     & d_{X \times Y}\big((x_1, y_1), (x_2, y_2)\big) + d_{X \times Y}\big((x_2, y_2), (x_3, y_3)\big)                       \\
     & = d_X(x_1, x_2) + d_Y(y_1, y_2) + d_X(x_2, x_3) + d_Y(y_2, y_3)                                                       \\
     & \geq d_X(x_1, x_3) + d_Y(y_1, y_3)                                                              &  & \by{ii:1.1.2}[d] \\
     & = d_{X \times Y}\big((x_1, y_1), (x_3, y_3)\big)
  \end{align*}
  and thus \((X \times Y, d_{X \times Y})\) satisfies \cref{ii:1.1.2}(d).
  By \cref{ii:1.1.2} we conclude that \((X \times Y, d_{X \times Y})\) is a metric space.

  Next we propose an analogue of \cref{ii:1.1.18} and proof it.
  Let \((X, d_X)\), \((Y, d_Y)\) be two metric spaces, let \((x, y) \in X \times Y\) and let \((x^{(n)}, y^{(n)})_{n = 1}^\infty\) be a sequence in \(X \times Y\).
  We claim that the follow two statements are equivalent:
  \begin{itemize}
    \item \(\lim_{n \to \infty} d_{X \times Y}\big((x^{(n)}, y^{(n)}), (x, y)\big) = 0\).
    \item \(\lim_{n \to \infty} d_X(x^{(n)}, x) = 0\) and \(\lim_{n \to \infty} d_Y(y^{(n)}, y) = 0\).
  \end{itemize}
  The claim is true since
  \begin{align*}
         & \lim_{n \to \infty} d_{X \times Y}\big((x^{(n)}, y^{(n)}), (x, y)\big) = 0                                             \\
    \iff & \lim_{n \to \infty} \big(d_X(x^{(n)}, x) + d_Y(y^{(n)}, y)\big) = 0                                                    \\
    \iff & \begin{dcases}
             0 \leq \lim_{n \to \infty} d_X(x^{(n)}, x) \leq \lim_{n \to \infty} \big(d_X(x^{(n)}, x) + d_Y(y^{(n)}, y)\big) \leq 0 \\
             0 \leq \lim_{n \to \infty} d_Y(y^{(n)}, y) \leq \lim_{n \to \infty} \big(d_X(x^{(n)}, x) + d_Y(y^{(n)}, y)\big) \leq 0
           \end{dcases} \\
    \iff & \begin{dcases}
             \lim_{n \to \infty} d_X(x^{(n)}, x) = 0 \\
             \lim_{n \to \infty} d_Y(y^{(n)}, y) = 0
           \end{dcases}
  \end{align*}

  Finally we propose an analogue of \cref{ii:2.2.1} and proof it.
  Let \((X, d_X)\), \((Y_1, d_{Y_1})\), \((Y_2, d_{Y_2})\) be metric spaces, let \(f_1 : X \to Y_1\) and \(f_2 : X \to Y_2\) be two functions, let \(f_1 \oplus f_2 : X \to (Y_1 \times Y_2)\) and let \(x_0 \in X\).
  We claim that the follow two statements are equivalent:
  \begin{itemize}
    \item \(f_1\) is continuous at \(x_0\) from \((X, d_X)\) to \((Y_1, d_{Y_1})\) and \(f_2\) is continuous at \(x_0\) from \((X, d_X)\) to \((Y_2, d_{Y_2})\).
    \item \(f_1 \oplus f_2\) is continuous at \(x_0\) from \((X, d_X)\) to \((Y_1 \times Y_2, d_{Y_1 \times Y_2})\).
  \end{itemize}
  The claim is true since by \cref{ii:2.1.1} we have
  \begin{align*}
         & \begin{dcases}
             f_1 \text{ is continuous at } x_0 \text{ from } (X, d_X) \text{ to } (Y_1, d_{Y_1}) \\
             f_2 \text{ is continuous at } x_0 \text{ from } (X, d_X) \text{ to } (Y_2, d_{Y_2})
           \end{dcases}                                                                   \\
    \iff & \forall \varepsilon \in \R^+, \exists \delta_1, \delta_2 \in \R^+ :                                                                                   \\
         & \begin{dcases}
             \big(\forall x \in X, d_X(x, x_0) < \delta_1 \implies d_{Y_1}\big(f_1(x), f_1(x_0)\big) < \dfrac{\varepsilon}{2}) \\
             \big(\forall x \in X, d_X(x, x_0) < \delta_2 \implies d_{Y_2}\big(f_2(x), f_2(x_0)\big) < \dfrac{\varepsilon}{2})
           \end{dcases}                                     \\
    \iff & \forall \varepsilon \in \R^+, \exists \delta = \min(\delta_1, \delta_2) \in \R^+ :                                                                    \\
         & \big(\forall x \in X, d_X(x, x_0) < \delta \implies d_{Y_1 \times Y_2}\Big(\big(f_1(x), f_2(x)\big), \big(f_1(x_0), f_2(x_0)\big)\Big) < \varepsilon) \\
    \iff & f_1 \oplus f_2 \text{ is continuous at } x_0 \text{ from } (X, d_X) \text{ to } (Y_1 \times Y_2, d_{Y_1 \times Y_2}).
  \end{align*}
\end{proof}

\begin{ex}\label{ii:ex:2.2.9}
  Let \(f : \R^2 \to \R\) be a function from \(\R^2\) to \(\R\).
  Let \((x_0, y_0)\) be a point in \(\R^2\).
  If \(f\) is continuous at \((x_0, y_0)\), show that
  \[
    \lim_{x \to x_0} \limsup_{y \to y_0} f(x, y) = \lim_{y \to y_0} \limsup_{x \to x_0} f(x, y) = f(x_0, y_0)
  \]
  and
  \[
    \lim_{x \to x_0} \liminf_{y \to y_0} f(x, y) = \lim_{y \to y_0} \liminf_{x \to x_0} f(x, y) = f(x_0, y_0).
  \]
  Recall that
  \begin{align*}
     & \limsup_{x \to x_0} f(x) \coloneqq \inf_{r > 0} \sup_{\abs{x - x_0} < r} f(x) \\
     & \liminf_{x \to x_0} f(x) \coloneqq \sup_{r > 0} \inf_{\abs{x - x_0} < r} f(x)
  \end{align*}
  In particular, we have
  \[
    \lim_{x \to x_0} \lim_{y \to y_0} f(x, y) = \lim_{y \to y_0} \lim_{x \to x_0} f(x, y)
  \]
  whenever the limits on both sides exist.
  (Note that the limits do not necessarily exist in general.)
  Discuss the comparison between this result and Example 1.2.7.
\end{ex}

\begin{proof}
  Let \(d_1 = d_{l^1}|_{\R \times \R}\) and let \(d_2 = d_{l^1}|_{\R^2 \times \R^2}\).
  Since \(f\) is continuous at \((x_0, y_0)\) from \((\R^2, d_2)\) to \((\R, d_1)\), by \cref{ii:2.1.1} we have
  \begin{align*}
             & \forall \varepsilon \in \R^+, \exists \delta \in \R^+ : \forall (x, y) \in \R^2,                                                                                     \\
             & d_2\big((x, y), (x_0, y_0)\big) < \delta \text{ implies } d_1\big(f(x, y), f(x_0, y_0)\big) < \varepsilon                                                            \\
    \implies & \forall \varepsilon \in \R^+, \exists \delta \in \R^+ : \forall (x, y) \in \R^2,                                                                                     \\
             & \abs{x - x_0} + \abs{y - y_0} < \delta \text{ implies } \abs{f(x, y) - f(x_0, y_0)} < \varepsilon                                                                    \\
    \implies & \forall \varepsilon \in \R^+, \exists \delta \in \R^+ : \forall (x, y) \in \R^2,                                                                                     \\
             & \abs{x - x_0} + \abs{y - y_0} < \delta \text{ implies}                                                                                                               \\
             & f(x_0, y_0) - \varepsilon < f(x, y) < f(x_0, y_0) + \varepsilon                                                                                                      \\
    \implies & \forall \varepsilon \in \R^+, \exists \delta \in \R^+ : \forall x \in \R, \abs{x - x_0} < \dfrac{\delta}{2} \text{ implies}                                          \\
             & f(x_0, y_0) - \varepsilon \leq \inf_{\abs{y - y_0} < \dfrac{\delta}{2}} f(x, y) \leq \sup_{\abs{y - y_0} < \dfrac{\delta}{2}} f(x, y) \leq f(x_0, y_0) + \varepsilon \\
    \implies & \forall \varepsilon \in \R^+, \exists \delta \in \R^+ : \forall x \in \R, \abs{x - x_0} < \dfrac{\delta}{2} \text{ implies}                                          \\
             & \begin{dcases}
                 \inf\set{\sup_{\abs{y - y_0} < r} f(x, y) : r \in \R^+} \leq \sup_{\abs{y - y_0} < \dfrac{\delta}{2}} f(x, y) \leq f(x_0, y_0) + \varepsilon \\
                 f(x_0, y_0) - \varepsilon \leq \inf_{\abs{y - y_0} < \dfrac{\delta}{2}} f(x, y) \leq \sup\set{\inf_{\abs{y - y_0} < r} f(x, y) : r \in \R^+}
               \end{dcases}          \\
    \implies & \forall \varepsilon \in \R^+, \exists \delta \in \R^+ : \forall x \in \R, \abs{x - x_0} < \dfrac{\delta}{2} \text{ implies}                                          \\
             & \begin{dcases}
                 \limsup_{y \to y_0} f(x, y) \leq f(x_0, y_0) + \varepsilon \\
                 f(x_0, y_0) - \varepsilon \leq \liminf_{y \to y_0} f(x, y)
               \end{dcases}                                                                                                           \\
    \implies & \forall \varepsilon \in \R^+, \exists \delta \in \R^+ : \forall x \in \R, \abs{x - x_0} < \dfrac{\delta}{2} \text{ implies}                                          \\
             & \begin{dcases}
                 \abs{\limsup_{y \to y_0} f(x, y) - f(x_0, y_0)} = \limsup_{y \to y_0} f(x, y) - f(x_0, y_0) \leq \varepsilon \\
                 \abs{\liminf_{y \to y_0} f(x, y) - f(x_0, y_0)} = f(x_0, y_0) - \liminf_{y \to y_0} f(x, y) \leq \varepsilon
               \end{dcases}                                                     \\
    \implies & \begin{dcases}
                 \lim_{x \to x_0} \big(\limsup_{y \to y_0} f(x, y)\big) = f(x_0, y_0) \\
                 \lim_{x \to x_0} \big(\liminf_{y \to y_0} f(x, y)\big) = f(x_0, y_0)
               \end{dcases}
  \end{align*}
  Then we have
  \begin{align*}
             & \forall x \in X, \liminf_{y \to y_0} f(x, y) \leq \lim_{y \to y_0} f(x, y) \leq \limsup_{y \to y_0} f(x, y)                              \\
    \implies & f(x_0, y_0) = \lim_{x \to x_0} \big(\liminf_{y \to y_0} f(x, y)\big)                                                                     \\
             & \quad \leq \lim_{x \to x_0} \big(\lim_{y \to y_0} f(x, y)\big) \leq \lim_{x \to x_0} \big(\limsup_{y \to y_0} f(x, y)\big) = f(x_0, y_0) \\
    \implies & \lim_{x \to x_0} \big(\lim_{y \to y_0} f(x, y)\big) = f(x_0, y_0).
  \end{align*}
  Using similar arguments, we can show that
  \begin{align*}
     & \lim_{y \to y_0} \big(\limsup_{x \to x_0} f(x, y)\big) = f(x_0, y_0) \\
     & \lim_{y \to y_0} \big(\liminf_{x \to x_0} f(x, y)\big) = f(x_0, y_0) \\
     & \lim_{y \to y_0} \big(\lim_{x \to x_0} f(x, y)\big) = f(x_0, y_0)
  \end{align*}
  and we conclude that
  \[
    \lim_{x \to x_0} \big(\lim_{y \to y_0} f(x, y)\big) = \lim_{y \to y_0} \big(\lim_{x \to x_0} f(x, y)\big) = f(x_0, y_0).
  \]
\end{proof}

\begin{ex}\label{ii:ex:2.2.10}
  Let \(f : \R^2 \to \R\) be a continuous function.
  Show that for each \(x \in \R\), the function \(y \mapsto f(x, y)\) is continuous on \(\R\), and for each \(y \in \R\), the function \(x \mapsto f(x, y)\) is continuous on \(\R\).
  Thus a function \(f(x, y)\) which is jointly continuous in \((x, y)\) is also continuous in each variable \(x, y\) separately.
\end{ex}

\begin{proof}
  Let \(d_1 = d_{l^1}|_{\R \times \R}\) and let \(d_2 = d_{l^1}|_{\R^2 \times \R^2}\).
  Let \(x_0, y_0 \in \R\).
  Since \(f\) is continuous from \((\R^2, d_2)\) to \((\R, d_1)\), by \cref{ii:2.1.1} we know that \(f\) is continuous at \((x_0, y_0)\) from \((\R^2, d_2)\) to \((\R, d_1)\) and we have
  \begin{align*}
             & \forall \varepsilon \in \R^+, \exists \delta \in \R^+ : \forall (x, y) \in \R^2,                          \\
             & d_2\big((x, y), (x_0, y_0)\big) < \delta \text{ implies } d_1\big(f(x, y), f(x_0, y_0)\big) < \varepsilon \\
    \implies & \forall \varepsilon \in \R^+, \exists \delta \in \R^+ : \forall (x, y) \in \R^2,                          \\
             & \abs{x - x_0} + \abs{y - y_0} < \delta \text{ implies } \abs{f(x, y) - f(x_0, y_0)} < \varepsilon         \\
    \implies & \forall \varepsilon \in \R^+, \exists \delta \in \R^+ :                                                   \\
             & \begin{dcases}
                 \forall x \in \R, \abs{x - x_0} < \delta \implies \abs{f(x, y_0) - f(x_0, y_0)} < \varepsilon \\
                 \forall y \in \R, \abs{y - y_0} < \delta \implies \abs{f(x_0, y) - f(x_0, y_0)} < \varepsilon
               \end{dcases}             \\
    \implies & \begin{dcases}
                 x \mapsto f(x, y_0) \text{ is continuous at } x_0 \text{ from } (\R, d_1) \text{ to } (\R, d_1) \\
                 y \mapsto f(x_0, y) \text{ is continuous at } y_0 \text{ from } (\R, d_1) \text{ to } (\R, d_1)
               \end{dcases}
  \end{align*}
  Since \(x_0, y_0\) were arbitrary, we conclude thatt \(x \mapsto f(x, y)\) is continuous from \((\R, d_1)\) to \((\R, d_1)\) and \(y \mapsto f(x, y)\) is continuous from \((\R, d_1)\) to \((\R, d_1)\).
\end{proof}

\begin{ex}\label{ii:ex:2.2.11}
  Let \(f : \R^2 \to \R\) be the function defined by \(f(x, y) = \dfrac{xy}{x^2 + y^2}\) when \((x, y) \neq (0, 0)\), and \(f(x, y) = 0\) otherwise.
  Show that for each fixed \(x \in \R\), the function \(y \mapsto f(x, y)\) is continuous on \(\R\), and that for each fixed \(y \in \R\), the function \(x \mapsto f(x, y)\) is continuous on \(\R\), but that the function \(f : \R^2 \to \R\) is not continuous on \(\R^2\).
  This shows that the converse to \cref{ii:ex:2.2.10} fails;
  it is possible to be continuous in each variable separately without being jointly continuous.
\end{ex}

\begin{proof}
  We have
  \[
    \forall y \in \R, f(0, y) = 0
  \]
  and thus by \cref{ii:2.2.2} \(y \mapsto f(0, y)\) is continuous from \((\R, d_{l^1}|_{\R \times \R})\) to \((\R, d_{l^1}|_{\R \times \R})\).
  By \cref{ii:2.2.2} and \cref{ii:ex:2.2.10} we also have
  \begin{align*}
             & \begin{dcases}
                 (x, y) \mapsto x y \text{ is continuous from } (\R^2, d_{l^1}|_{\R^2 \times \R^2}) \text{ to } (\R, d_{l^1}|_{\R \times \R}); \\
                 (x, y) \mapsto x^2 + y^2 \text{ is continuous}                                                                                \\
                 \text{from } \big((\R \setminus \set{0} \times \R), d_{l^1}|_{\big((\R \setminus \set{0}) \times \R\big) \times \big((\R \setminus \set{0}) \times \R\big)}\big) \text{ to } (\R, d_{l^1}|_{\R \times \R})
               \end{dcases}  \\
    \implies & (x, y) \mapsto \dfrac{xy}{x^2 + y^2} \text{ is continuous}                                                                                                                                                 \\
             & \text{from } \big((\R \setminus \set{0} \times \R), d_{l^1}|_{\big((\R \setminus \set{0}) \times \R\big) \times \big((\R \setminus \set{0}) \times \R\big)}\big) \text{ to } (\R, d_{l^1}|_{\R \times \R}) \\
    \implies & \forall x \in \R \setminus \set{0}, y \mapsto \dfrac{xy}{x^2 + y^2} \text{ is continuous}                                                                                                                  \\
             & \text{from } (\R, d_{l^1}|_{\R \times \R}) \text{ to } (\R, d_{l^1}|_{\R \times \R}).
  \end{align*}
  Thus we conclude that \(\forall x \in \R\), \(y \mapsto f(x, y)\) is continuous from \((\R, d_{l^1}|_{\R \times \R})\) to \((\R, d_{l^1}|_{\R \times \R})\).
  Using similar arguments as above, we also have \(\forall y \in \R\), \(x \mapsto f(x, y)\) is continuous from \((\R, d_{l^1}|_{\R \times \R})\) to \((\R, d_{l^1}|_{\R \times \R})\).

  Now we show that \(f\) is not continuous from \((\R^2, d_{l^1}|_{\R^2 \times \R^2})\) to \((\R, d_{l^1}|_{\R \times \R})\).
  In particular, we do this by showing \(f\) is not continuous at \((0, 0)\) from \((\R^2, d_{l^1}|_{\R^2 \times \R^2})\) to \((\R, d_{l^1}|_{\R \times \R})\).
  Consider the sequence \((\dfrac{1}{n}, \dfrac{1}{n})_{n = 1}^\infty\) in \(\R^2\).
  We have
  \begin{align*}
             & \lim_{n \to \infty} d_{l^1}|_{\R, \R}(\dfrac{1}{n}, 0) = 0                                                           \\
    \implies & \lim_{n \to \infty} d_{l^1}|_{\R^2, \R^2}\big((\dfrac{1}{n}, \dfrac{1}{n}), (0, 0)\big) = 0 &  & \by{ii:1.1.18}[b,d]
  \end{align*}
  and
  \begin{align*}
             & \forall n \in \Z^+, f(\dfrac{1}{n}, \dfrac{1}{n}) = \dfrac{\dfrac{1}{n^2}}{\dfrac{1}{n^2} + \dfrac{1}{n^2}} = \dfrac{1}{2} \\
    \implies & \lim_{n \to \infty} f(\dfrac{1}{n}, \dfrac{1}{n}) = \dfrac{1}{2} \neq 0 = f(0, 0).
  \end{align*}
  Thus by \cref{ii:2.1.4}(a)(b) \(f\) is not continuous at \((0, 0)\) from \((\R^2, d_{l^1}|_{\R^2 \times \R^2})\) to \((\R, d_{l^1}|_{\R \times \R})\).
\end{proof}

\begin{ex}\label{ii:ex:2.2.12}
  Let \(f: \R^2 \to \R\) be the function defined by \(f(x, y) \coloneqq x^2 / y\) when \(y \neq 0\), and \(f(x, y) \coloneqq 0\) when \(y = 0\).
  Show that \(\lim_{t \to 0} f(tx, ty) = f(0, 0)\) for every \((x, y) \in \R^2\), but that \(f\) is not continuous at the origin.
  Thus being continuous on every line through the origin is not enough to guarantee continuity at the origin!
\end{ex}

\begin{proof}
  Let \((x_0, y_0) \in \R^2\).
  We split into two cases:
  \begin{itemize}
    \item If \(y_0 = 0\), then we have
          \[
            \forall t \in \R, f(t x_0, t0) = f(t x_0, 0) = 0 = f(0, 0)
          \]
          and thus \(t \mapsto f(t x_0, t0)\) is constant function and is continuous from \((\R, d_{l^1}|_{\R \times \R})\) to \((\R, d_{l^1}|_{\R \times \R})\).
    \item If \(x_0 = 0\), then we have
          \[
            \forall t \in \R, f(t0, t y_0) = f(0, t y_0) = \dfrac{0}{t y_0} = 0 = f(0, 0)
          \]
          and thus \(t \mapsto f(t0, t y_0)\) is constant function and is continuous from \((\R, d_{l^1}|_{\R \times \R})\) to \((\R, d_{l^1}|_{\R \times \R})\).
    \item If \(x_0 \neq 0\) and \(y_0 \neq 0\), then we have
          \begin{align*}
                     & \forall t \in \R, f(t x_0, t y_0) = \begin{dcases}
                                                             \dfrac{t^2 x_0^2}{t y_0} = \dfrac{t x_0^2}{y_0} & \text{if } t \neq 0 \\
                                                             0                                               & \text{if } t = 0
                                                           \end{dcases}                                                            \\
            \implies & \forall \varepsilon \in \R^+, \bigg(\forall t \in \R, \abs{t - 0} < \varepsilon \abs{\dfrac{y_0}{x_0^2}} \implies \abs{\dfrac{t x_0^2}{y_0} - 0} < \varepsilon\bigg) \\
            \implies & \forall \varepsilon \in \R^+, \exists \delta \in \R^+ : \big(\forall t \in \R, \abs{t - 0} < \delta \implies \abs{f(t x_0, t y_0) - f(0, 0)} < \varepsilon\big)      \\
            \implies & \lim_{t \to \infty} f(t x_0, t y_0) = 0 = f(0, 0).
          \end{align*}
          Thus \(t \mapsto f(t x_0, t y_0)\) is continuous from \((\R, d_{l^1}|_{\R \times \R})\) to \((\R, d_{l^1}|_{\R \times \R})\).
  \end{itemize}
  From all cases above we conclude that \(t \mapsto f(t x_0, t y_0)\) is  continuous from \((\R, d_{l^1}|_{\R \times \R})\) to \((\R, d_{l^1}|_{\R \times \R})\).
  Since \((x_0, y_0)\) was arbitrary, we conclude that for any \((x, y) \in \R^2\), \(t \mapsto f(tx, ty)\) is continuous from \((\R, d_{l^1}|_{\R \times \R})\) to \((\R, d_{l^1}|_{\R \times \R})\).

  Now we show that \(f\) is not continuous at \((0, 0)\) from \((\R^2, d_{l^1}|_{\R^2 \times \R^2})\) to \((\R, d_{l^1}|_{\R \times \R})\).
  Consider the sequence \((\dfrac{1}{n}, \dfrac{1}{n^2})_{n = 1}^\infty\) in \(\R^2\).
  Since
  \begin{align*}
             & \lim_{n \to \infty} \dfrac{1}{n} = \lim_{n \to \infty} \dfrac{1}{n^2} = 0                                                    \\
    \implies & \lim_{n \to \infty} d_{l^1}|_{\R^2 \times \R^2}\big((\dfrac{1}{n}, \dfrac{1}{n^2}), (0, 0)\big) = 0 &  & \by{ii:1.1.18}[b,d]
  \end{align*}
  and
  \begin{align*}
             & \forall n \in \Z^+, f(\dfrac{1}{n}, \dfrac{1}{n^2}) = \dfrac{\dfrac{1}{n^2}}{\dfrac{1}{n^2}} = 1 \\
    \implies & \lim_{n \to \infty} f(\dfrac{1}{n}, \dfrac{1}{n^2}) = 1 \neq 0 = f(0, 0),
  \end{align*}
  by \cref{ii:2.1.4}(a)(b) we know that \(f\) is not continuous at \((0, 0)\) from \((\R^2, d_{l^1}|_{\R^2 \times \R^2})\) to \((\R, d_{l^1}|_{\R \times \R})\).
\end{proof}

\section{Continuity and compactness}\label{sec:2.3}

\begin{thm}[Continuous maps preserve compactness]\label{2.3.1}
  Let \(f : X \to Y\) be a continuous map from one metric space \((X, d_X)\) to another \((Y, d_Y)\).
  Let \(K \subseteq X\) be any compact subset of \(X\).
  Then the image \(f(K) \coloneqq \{f(x) : x \in K\}\) of \(K\) is also compact.
\end{thm}

\begin{proof}
  Let \(\bigcup_{\alpha \in I} V_\alpha\) be an open cover of \(f(K)\) in \((Y, d_Y)\), i.e., \(I \subseteq Y\) and for each \(\alpha \in I\), \(V_{\alpha}\) is an open set in \((Y, d_Y)\) such that \(f(K) \subseteq \bigcup_{\alpha \in I} V_\alpha\).
  Since
  \begin{align*}
             & f \text{ is continuous from } (X, d_X) \text{ to } (Y, d_Y)                                                                                                       \\
    \implies & f|_K \text{ is continuous from } (K, d_X|_{K \times K}) \text{ to } (Y, d_Y)                                                  &  & \text{(by \cref{2.1.3})}       \\
    \implies & \forall \alpha \in I, f|_K^{-1}(V_\alpha) \text{ is open in } (K, d_X|_{K \times K})                                          &  & \text{(by \cref{2.1.5}(a)(d))} \\
    \implies & K = \bigcup_{\alpha \in I} f|_K^{-1}(V_\alpha) \text{ is open in } (K, d_X|_{K \times K})                                     &  & \text{(by \cref{1.2.15}(g))}   \\
    \implies & \exists\ F \subseteq I : (F \text{ is finite}) \land \bigg(K = \bigcup_{\alpha \in F} f|_K^{-1}(V_\alpha)\bigg)               &  & \text{(by \cref{1.5.8})}       \\
    \implies & \exists\ F \subseteq I : (F \text{ is finite}) \land \bigg(f(K) = f\big(\bigcup_{\alpha \in F} f|_K^{-1}(V_\alpha)\big)\bigg)                                     \\
    \implies & \exists\ F \subseteq I : (F \text{ is finite}) \land \bigg(f(K) \subseteq \bigcup_{\alpha \in F} V_\alpha)\bigg),
  \end{align*}
  we know that there exists an finite subcover of \(f(K)\) with respect to \(\bigcup_{\alpha \in I} V_\alpha\) in \((Y, d_Y)\).
  Since \(I\) is arbitrary open cover of \(f(K)\) in \((Y, d_Y)\), by \cref{ex:1.5.11} we know that \(\big(f(K), d_Y|_{Y \times Y}\big)\) is compact.
\end{proof}

\begin{prop}[Maximum principle]\label{2.3.2}
  Let \((X, d)\) be a compact metric space, and let \(f : X \to \R\) be a continuous function.
  Then \(f\) is bounded.
  Furthermore, if \(X\) is non-empty, \(f\) attains its maximum at some point \(x_{\max} \in X\), and also attains its minimum at some point \(x_{\min} \in X\).
\end{prop}

\begin{proof}
  We have
  \begin{align*}
             & \begin{dcases}
                 (X, d) \text{ is compact} \\
                 f \text{ is continuous from } (X, d) \text{ to } (\R, d_{l^1}|_{\R \times \R})
               \end{dcases}                                \\
    \implies & \big(f(X), d_{l^1}|_{f(X) \times f(X)}\big) \text{ is compact}                 &  & \text{(by \cref{2.3.1})} \\
    \implies & \begin{dcases}
                 f(X) \text{ is closed in } (\R, d_{l^1}|_{\R \times \R}) \\
                 \big(f(X), d_{l^1}|_{f(X) \times f(X)}\big) \text{ is bounded}
               \end{dcases}                 &  & \text{(by \cref{1.5.6})}                                                \\
    \implies & \begin{dcases}
                 f(X) \text{ is closed in } (\R, d_{l^1}|_{\R \times \R}) \\
                 f(X) \text{ is bounded subset of } \R
               \end{dcases}                    &  & \text{(by \cref{ex:1.5.1})}
  \end{align*}

  Now we show that if \(X \neq \emptyset\), then
  \[
    \exists\ x_{\min}, x_{\max} \in X : \forall x \in X, f(x_{\min}) \leq f(x) \leq f(x_{\max}).
  \]
  Let \(U = \sup\big(f(X)\big)\) and let \(L = \inf\big(f(X)\big)\).
  Since \(f\) is bounded subset of \(\R\), we know that \(U, L \in \R\).
  By the definition of \(U\) and \(L\) we know that
  \[
    \forall n \in \Z^+, \exists\ u_n, l_n \in f(X) : \begin{dcases}
      U - u_n < \dfrac{1}{n} \\
      l_n - L < \dfrac{1}{n}
    \end{dcases}
  \]
  Thus we have
  \begin{align*}
             & \begin{dcases}
                 0 = \lim_{n \to \infty} U - u_n = \lim_{n \to \infty} \dfrac{1}{n} = 0 \\
                 0 = \lim_{n \to \infty} l_n - L = \lim_{n \to \infty} \dfrac{1}{n} = 0
               \end{dcases} \\
    \implies & \begin{dcases}
                 \lim_{n \to \infty} u_n = U \\
                 \lim_{n \to \infty} l_n = L
               \end{dcases}
  \end{align*}
  Since \(f(X)\) is closed in \((\R, d_{l^1}|_{\R \times \R})\) and \((u_n)_{n = 1}^\infty\), \((l_n)_{n = 1}^\infty\) are convergent sequences in \(\R\) with respect to \(d_{l^1}|_{\R \times \R}\), by \cref{1.2.15}(b) we know that \(U, L \in f(X)\).
  Since \(U, L \in f(X)\), we know that
  \[
    \exists\ x_{\min}, x_{\max} \in X : \big(f(x_{\min}) = L\big) \land \big(f(x_{\max}) = U\big).
  \]
\end{proof}

\begin{rmk}\label{2.3.3}
  As was already noted in Exercise 9.6.1 in Analysis I, this principle can fail if \(X\) is not compact.
  \cref{2.3.2} should be compared with Lemma 9.6.3 in Analysis I and Proposition 9.6.7 in Analysis I.
\end{rmk}

\begin{defn}[Uniform continuity]\label{2.3.4}
  Let \(f : X \to Y\) be a map from one metric space \((X, d_X)\) to another \((Y, d_Y)\).
  We say that \(f\) is \emph{uniformly continuous} if, for every \(\varepsilon > 0\), there exists a \(\delta > 0\) such that \(d_Y\big(f(x), f(x')\big) < \varepsilon\) whenever \(x, x' \in X\) are such that \(d_X(x, x') < \delta\).
\end{defn}

\begin{note}
  Every uniformly continuous function is continuous, but not conversely.
  But if the domain \(X\) is compact, then the two notions are equivalent.
\end{note}

\begin{thm}\label{2.3.5}
  Let \((X, d_X)\) and \((Y, d_Y)\) be metric spaces, and suppose that \((X, d_X)\) is compact.
  If \(f : X \to Y\) is function, then \(f\) is continuous if and only if it is uniformly continuous.
\end{thm}

\begin{proof}
  If \(f\) is uniformly continuous then it is also continuous by \cref{ex:2.3.3}.
  Now suppose that \(f\) is continuous.
  Fix \(\varepsilon > 0\).
  For every \(x_0 \in X\), the function \(f\) is continuous at \(x_0\) from \((X, d_X)\) to \((Y, d_Y)\).
  Thus there exists a \(\delta(x_0) > 0\), depending on \(x_0\), such that \(d_Y\big(f(x), f(x_0)\big) < \varepsilon / 2\) whenever \(d_X(x, x_0) < \delta(x_0)\).
  In particular, by the triangle inequality this implies that \(d_Y(f(x), f(x')) < \varepsilon\) whenever \(x \in B_{(X, d_X)}\big(x_0, \delta(x_0) / 2\big)\) and \(d_X(x', x) < \delta(x_0) / 2\).

  Now consider the (possibly infinite) collection of balls
  \[
    \Big\{B_{(X, d_X)}\big(x_0, \delta(x_0) / 2\big) : x_0 \in X\Big\}.
  \]
  Each ball in this collection is of course open in \((X, d_X)\), and the union of all these balls covers \(X\), since each point \(x_0\) in \(X\) is contained in its own ball \(B_{(X, d_X)}\big(x_0, \delta(x_0) / 2\big)\).
  Hence, by \cref{1.5.8}, there exist a finite number of points \(x_1, \dots, x_n\) such that the balls \(B_{(X, d_X)}\big(x_j, \delta(x_j) / 2\big)\) for \(j = 1, \dots, n\) cover \(X\):
  \[
    X \subseteq \bigcup_{j = 1}^n B_{(X, d_X)}\big(x_j, \delta(x_j) / 2\big).
  \]
  Now let \(\delta \coloneqq \min_{j = 1}^n \delta(x_j) / 2\).
  Since each of the \(\delta(x_j)\) are positive, and there are only a finite number of \(j\), we see that \(\delta > 0\).
  Now let \(x, x'\) be any two points in \(X\) such that \(d_X(x, x') < \delta\).
  Since the balls \(B_{(X, d_X)}\big(x_j, \delta(x_j) / 2\big)\) cover \(X\), we see that there must exist \(1 \leq j \leq n\) such that \(x \in B_{(X, d_X)}\big(x_j, \delta(x_j) / 2\big)\).
  Since \(d_X(x, x') < \delta\), we have \(d_X(x, x') < \delta(x_j) / 2\), and so by the previous discussion we have \(d_Y\big(f(x), f(x')\big) < \varepsilon\).
  We have thus found a \(\delta\) such that \(d_Y\big(f(x), f(x')\big) < \varepsilon\) whenever \(d(x, x') < \delta\), and this proves uniform continuity as desired.
\end{proof}

\exercisesection

\begin{ex}\label{ex:2.3.1}
  Prove \cref{2.3.1}.
\end{ex}

\begin{proof}
  See \cref{2.3.1}.
\end{proof}

\begin{ex}\label{ex:2.3.2}
  Prove \cref{2.3.2}.
\end{ex}

\begin{proof}
  See \cref{2.3.2}.
\end{proof}

\begin{ex}\label{ex:2.3.3}
  Show that every uniformly continuous function is continuous, but give an example that shows that not every continuous function is uniformly continuous.
\end{ex}

\begin{proof}
  Let \((X, d_X)\), \((Y, d_Y)\) be two metric spaces and let \(f : X \to Y\) be a function which is uniformly continuous from \((X, d_X)\) to \((Y, d_Y)\).
  Then we have
  \begin{align*}
             & \forall \varepsilon \in \R^+, \exists\ \delta \in \R^+ :                                                                            \\
             & \Big(\forall x, x_0 \in X, d_X(x, x_0) < \delta \implies d_Y\big(f(x), f(x_0)\big) < \varepsilon\Big) &  & \text{(by \cref{2.3.4})} \\
    \implies & \forall x_0 \in X, f \text{ is continuous at } x_0 \text{ from } (X, d_X) \text{ to } (Y, d_Y)        &  & \text{(by \cref{2.1.1})} \\
    \implies & f \text{ is continuous from } (X, d_X) \text{ to } (Y, d_Y).                                          &  & \text{(by \cref{2.1.1})}
  \end{align*}

  For the converse example, see Example 9.9.11 in Analysis I.
\end{proof}

\begin{ex}\label{ex:2.3.4}
  Let \((X, d_X)\), \((Y, d_Y)\), \((Z, d_Z)\) be metric spaces, and let \(f : X \to Y\) and \(g : Y \to Z\) be two uniformly continuous functions.
  Show that \(g \circ f : X \to Z\) is also uniformly continuous.
\end{ex}

\begin{proof}
  Since \(f\) is uniformly continuous from \((X, d_X)\) to \((Y, d_Y)\) and \(g\) is uniformly continuous from \((Y, d_Y)\) to \((Z, d_Z)\), by \cref{2.3.4} we have
  \begin{align*}
             & \begin{dcases}
                 \forall \delta' \in \R^+, \exists\ \delta, \in \R^+ :                                                    \\
                 \Big(\forall x_1, x_2 \in X, d_X(x_1, x_2) < \delta \implies d_Y\big(f(x_1), f(x_2)\big) < \delta'\Big); \\
                 \forall \varepsilon \in \R^+, \exists\ \delta' \in \R^+ :                                                \\
                 \Big(\forall y_1, y_2 \in Y, d_Y(y_1, y_2) < \delta' \implies d_Z\big(g(y_1), g(y_2)\big) < \varepsilon\Big);
               \end{dcases}                         \\
    \implies & \forall \varepsilon \in \R^+, \exists\ \delta \in \R^+ :                                                                             \\
             & \bigg(\forall x_1, x_2 \in X, d_X(x_1, x_2) < \delta \implies d_Z\Big(g\big(f(x_1)\big), g\big(f(x_2)\big)\Big) < \varepsilon\bigg).
  \end{align*}
  Thus by \cref{2.3.4} \(g \circ f\) is uniformly continuous from \((X, d_X)\) to \((Z, d_Z)\).
\end{proof}

\begin{ex}\label{ex:2.3.5}
  Let \((X, d_X)\) be a metric space, and let \(f : X \to \R\) and \(g : X \to \R\) be uniformly continuous functions.
  Show that the direct sum \(f \oplus g : X \to \R^2\) defined by \(f \oplus g(x) \coloneqq \big(f(x), g(x)\big)\) is uniformly continuous.
\end{ex}

\begin{proof}
  Let \(d_1 = d_{l^1}|_{\R \times \R}\) and let \(d_2 = d_{l^1}|_{\R^2 \times \R^2}\).
  Since \(f, g\) are uniformly continuous from \((X, d_X)\) to \((\R, d_1)\), by \cref{2.3.4} we have
  \begin{align*}
             & \forall \varepsilon \in \R^+, \exists\ \delta \in \R^+ :                                                                                           \\
             & \begin{dcases}
                 \forall x_1, x_2 \in X, d_X(x_1, x_2) < \delta \implies d_1\big(f(x_1), f(x_2)\big) < \dfrac{\varepsilon}{2} \\
                 \forall x_1, x_2 \in X, d_X(x_1, x_2) < \delta \implies d_1\big(g(x_1), g(x_2)\big) < \dfrac{\varepsilon}{2}
               \end{dcases}                                       \\
    \implies & \forall \varepsilon \in \R^+, \exists\ \delta \in \R^+ :                                                                                           \\
             & \Big(\forall x_1, x_2 \in X, d_X(x_1, x_2) < \delta \implies d_1\big(f(x_1), f(x_2)\big) + d_1\big(g(x_1), g(x_2)\big) < \varepsilon\Big)          \\
    \implies & \forall \varepsilon \in \R^+, \exists\ \delta \in \R^+ :                                                                                           \\
             & \bigg(\forall x_1, x_2 \in X, d_X(x_1, x_2) < \delta \implies d_2\Big(\big(f(x_1), g(x_1)\big), \big(f(x_2), g(x_2)\big)\Big) < \varepsilon\bigg).
  \end{align*}
  Thus by \cref{2.3.4} \(f \oplus g\) is uniformly continuous from \((X, d_X)\) to \((\R^2, d_2)\).
\end{proof}

\begin{ex}\label{ex:2.3.6}
  Show that the addition function \((x, y) \mapsto x + y\) and the subtraction function \((x, y) \mapsto x - y\) are uniformly continuous from \(\R^2\) to \(\R\), but the multiplication function \((x, y) \mapsto xy\) is not.
  Conclude that if \(f : X \to \R\) and \(g : X \to \R\) are uniformly continuous functions on a metric space \((X, d)\), then \(f + g : X \to \R\) and \(f - g : X \to \R\) are also uniformly continuous.
  Give an example to show that \(fg : X \to \R\) need not be uniformly continuous.
  What is the situation for \(\max(f, g)\), \(\min(f, g)\), \(f / g\), and \(cf\) for a real number \(c\)?
\end{ex}

\begin{proof}
  Let \(d_1 = d_{l^1}|_{\R \times \R}\) and let \(d_2 = d_{l^1}|_{\R^2 \times \R^2}\).
  We first show that \((x, y) \mapsto x + y\) and \((x, y) \mapsto x - y\) are uniformly continuous from \((\R^2, d_2)\) to \((\R, d_1)\).
  Since
  \begin{align*}
             & \forall \varepsilon \in \R^+, \forall (x_1, y_1), (x_2, y_2) \in \R^2,                                      \\
             & \big(\abs{x_1 - x_2} < \dfrac{\varepsilon}{2}\big) \land \big(\abs{y_1 - y_2} < \dfrac{\varepsilon}{2}\big) \\
    \implies & \begin{dcases}
                 \abs{x_1 - x_2 + y_1 - y_2} \leq \abs{x_1 - x_2} + \abs{y_1 - y_2} < \varepsilon \\
                 \abs{x_1 - x_2 - y_1 + y_2} \leq \abs{x_1 - x_2} + \abs{y_2 - y_1} < \varepsilon \\
               \end{dcases}                            \\
    \implies & \begin{dcases}
                 d_1(x_1 + y_1, x_2 + y_2) \leq d_2\big((x_1, y_1), (x_2, y_2)\big) < \varepsilon \\
                 d_1(x_1 - y_1, x_2 - y_2) \leq d_2\big((x_1, y_1), (x_2, y_2)\big) < \varepsilon
               \end{dcases}
  \end{align*}
  by choosing \(\delta = \varepsilon\) we have
  \begin{align*}
     & \forall \varepsilon \in \R^+, \exists\ \delta \in \R^+ : \forall (x_1, y_1), (x_2, y_2) \in \R^2, \\
     & d_2\big((x_1, y_1), (x_2, y_2)\big) < \delta \implies \begin{dcases}
                                                               d_1(x_1 + y_1, x_2 + y_2) < \varepsilon \\
                                                               d_1(x_1 - y_1, x_2 - y_2) < \varepsilon
                                                             \end{dcases}
  \end{align*}
  and thus by \cref{2.3.4} \((x, y) \mapsto x + y\) and \((x, y) \mapsto x - y\) are uniformly continuous from \((\R^2, d_2)\) to \((\R, d_1)\).

  Next we show that \((x, y) \mapsto xy\) is not uniformly continuous from \((\R^2, d_2)\) to \((\R, d_1)\).
  Since
  \[
    \forall n \in \Z^+, \begin{dcases}
      d_2\big((n, n), (n + \dfrac{1}{n}, n)\big) = \dfrac{1}{n} \\
      d_1\big(n \times n, (n + \dfrac{1}{n}) \times n\big) = 1
    \end{dcases}
  \]
  no matter which \(\delta \in \R^+\) we choose, we cannot make \(d_1(n^2, n^2 + 1) < \dfrac{1}{2}\) for every \(n \in \Z^+\).
  Thus by \cref{2.3.4} \((x, y) \mapsto xy\) is not uniformly continuous from \((\R^2, d_2)\) to \((\R, d_1)\).

  Next we show that \((x, y) \mapsto \max(x, y)\) and \((x, y) \mapsto \min(x, y)\) are uniformly continuous from \((X, d)\) to \((\R, d_1)\).
  Since
  \begin{align*}
             & \forall \varepsilon \in \R^+, \forall (x_1, y_1), (x_2, y_2) \in \R^2,                \\
             & \big(\abs{x_1 - x_2} < \varepsilon\big) \land \big(\abs{y_2 - y_1} < \varepsilon\big) \\
    \implies & \begin{dcases}
                 x_2 - \varepsilon < x_1 < x_2 + \varepsilon \\
                 y_2 - \varepsilon < y_1 < y_2 + \varepsilon \\
               \end{dcases}                                           \\
    \implies & \begin{dcases}
                 \max(x_2, y_2) - \varepsilon < \max(x_1, y_1) < \max(x_2, y_2) + \varepsilon \\
                 \min(x_2, y_2) - \varepsilon < \min(x_1, y_1) < \min(x_2, y_2) + \varepsilon
               \end{dcases}          \\
    \implies & \begin{dcases}
                 \abs{\max(x_1, y_1) - \max(x_2, y_2)} < \varepsilon \\
                 \abs{\min(x_1, y_1) - \min(x_2, y_2)} < \varepsilon
               \end{dcases}
  \end{align*}
  by choosing \(\delta = \varepsilon\) we have
  \begin{align*}
     & \forall \varepsilon \in \R^+, \exists\ \delta \in \R^+ :                              \\
     & \forall (x_1, y_1), (x_2, y_2) \in \R^2, d_2\big((x_1, y_1), (x_2, y_2)\big) < \delta \\
     & \implies \begin{dcases}
                  d_1\big(\max(x_1, y_1), \max(x_2, y_2)\big) < \varepsilon \\
                  d_1\big(\min(x_1, y_1), \min(x_2, y_2)\big) < \varepsilon
                \end{dcases}
  \end{align*}
  and thus by \cref{2.3.4} \((x, y) \mapsto \max(x, y)\) and \((x, y) \mapsto \min(x, y)\) are uniformly continuous from \((\R^2, d_2)\) to \((\R, d_1)\).

  Next we show that \(f + g\), \(f - g\), \(\max(f, g)\), \(\min(f, g)\) are uniformly continuous from \((X, d)\) to \((\R, d_1)\) given \(f, g\) are uniformly continuous from \((X, d)\) to \((\R, d_1)\).
  \begin{align*}
             & f, g \text{ are uniformly continuous from } (X, d) \text{ to } (\R, d_1)                                           \\
    \implies & f \oplus g \text{ is uniformly continuous from } (X, d) \text{ to } (\R^2, d_2)   &  & \text{(by \cref{ex:2.3.5})} \\
    \implies & \begin{dcases}
                 f \oplus g(x) \mapsto f(x) + g(x) \text{ is uniformly continuous}              \\
                 \text{from } (X, d) \text{ to } (\R, d_1);                                     \\
                 f \oplus g(x) \mapsto f(x) - g(x) \text{ is uniformly continuous}              \\
                 \text{from } (X, d) \text{ to } (\R, d_1);                                     \\
                 f \oplus g(x) \mapsto \max\big(f(x), g(x)\big) \text{ is uniformly continuous} \\
                 \text{from } (X, d) \text{ to } (\R, d_1);                                     \\
                 f \oplus g(x) \mapsto \min\big(f(x), g(x)\big) \text{ is uniformly continuous} \\
                 \text{from } (X, d) \text{ to } (\R, d_1).
               \end{dcases} &  & \text{(by \cref{ex:2.3.4})}                                     \\
    \implies & f + g, f - g, \max(f, g), \min(f, g) \text{ are uniformly continuous}                                              \\
             & \text{from } (X, d) \text{ to } (\R, d_1).
  \end{align*}

  Next we give a example where \(fg\) is not continuous from \((X, d)\) to \((\R, d_1)\) given \(f, g\) are uniformly continuous from \((X, d)\) to \((\R, d_1)\).
  Let \(f(x) = g(x) = x\).
  Then \(fg(x) = x^2\) and by Example 9.9.11 in Analysis I we know that \(x^2\) is not uniformly continuous from \((\R, d_1)\) to \((\R, d_1)\).

  Next we give a example where \(f / g\) is not continuous from \((X, d)\) to \((\R, d_1)\) given \(f, g\) are uniformly continuous from \((X, d)\) to \((\R, d_1)\).
  Let \(f(x) = 1\) and let \(g(x) = x\).
  Then \(f / g(x) = 1 / x\) and by Example 9.9.10 in Analysis I we know that \(1 / x\) is not uniformly continuous from \((\R, d_1)\) to \((\R, d_1)\).

  Finally we show that for each \(c \in \R\), \(cf\) is uniformly continuous from \((X, d)\) to \((\R, d_1)\) given \(f\) is uniformly continuous from \((X, d)\) to \((\R, d_1)\).
  If \(c = 0\), then \(f\) is a constant function.
  Thus
  \[
    \forall \varepsilon \in \R^+, \forall x \in X, d(x_1, x_2) < \varepsilon \implies d_1\big(f(x_1), f(x_2)\big) = 0 < \varepsilon
  \]
  and by \cref{2.3.4} \(f\) is uniformly continuous from \((X, d)\) to \((\R, d_1)\).
  Suppose that \(c \neq 0\).
  Since \(f\) is uniformly continuous from \((X, d)\) to \((\R, d_1)\), by \cref{2.3.4} we have
  \begin{align*}
             & \forall \varepsilon \in \R^+, \exists\ \delta \in \R^+:                                                                    \\
             & \Big(\forall x_1, x_2 \in X, d(x_1, x_2) < \delta \implies d_1\big(f(x_1), f(x_2)\big) < \dfrac{\varepsilon}{\abs{c}}\Big) \\
    \implies & \forall \varepsilon \in \R^+, \exists\ \delta \in \R^+:                                                                    \\
             & \Big(\forall x_1, x_2 \in X, d(x_1, x_2) < \delta \implies d_1\big(cf(x_1), cf(x_2)\big) < \varepsilon\Big)
  \end{align*}
  and thus by \cref{2.3.4} \(cf\) is uniformly continuous from \((X, d)\) to \((\R, d_1)\).
\end{proof}
\section{Continuity and connectedness}\label{ii:sec:2.4}

\begin{defn}[Connected spaces]\label{ii:2.4.1}
  Let \((X, d)\) be a metric space.
  We say that \(X\) is \emph{disconnected} iff there exist disjoint non-empty open sets \(V\) and \(W\) in \(X\) such that \(V \cup W = X\).
  (Equivalently, \(X\) is disconnected iff \(X\) contains a non-empty proper subset which is simultaneously closed and open, see \cref{ii:1.2.15}(e).)
  We say that \(X\) is \emph{connected} iff it is non-empty and not disconnected.
\end{defn}

\begin{note}
  We declare the empty set \(\emptyset\) as being special
  - it is neither connected nor disconnected;
  one could think of the empty set as ``unconnected.''
\end{note}

\setcounter{thm}{2}
\begin{defn}[Connected sets]\label{ii:2.4.3}
  Let \((X, d)\) be a metric space, and let \(Y\) be a subset of \(X\).
  We say that \(Y\) is \emph{connected} iff the metric space \((Y, d|_{Y \times Y})\) is connected, and we say that \(Y\) is \emph{disconnected} iff the metric space \((Y, d|_{Y \times Y})\) is disconnected.
\end{defn}

\begin{rmk}\label{ii:2.4.4}
  This definition is intrinsic;
  whether a set \(Y\) is connected or not depends only on what the metric is doing on \(Y\), but not on what ambient space \(X\) one placing \(Y\) in.
\end{rmk}

\begin{thm}\label{ii:2.4.5}
  Let \(X\) be a non-empty subset of the real line \(\R\).
  Then the following statements are equivalent.
  \begin{enumerate}
    \item \(X\) is connected.
    \item Whenever \(x, y \in X\) and \(x < y\), the interval \([x, y]\) is also contained in \(X\).
    \item \(X\) is an interval (in the sense of Definition 9.1.1 in Analysis I).
  \end{enumerate}
\end{thm}

\begin{proof}
  First we show that (a) implies (b).
  Let \(d = d_{l^1}|_{\R \times \R}\).
  Suppose that \((X, d)\) is connected, and suppose for the sake of contradiction that we could find points \(x < y\) in \(X\) such that \([x, y]\) is not contained in \(X\).
  Then there exists a real number \(x < z < y\) such that \(z \notin X\).
  Thus, the sets \((-\infty, z) \cap X\) and \((z, \infty) \cap X\) will cover \(X\).
  But these sets are non-empty (because they contain \(x\) and \(y\) respectively) and are open relative to \((X, d)\), and so \(X\) is disconnected, a contradiction.

  Now we show that (b) implies (a).
  Let \(X\) be a set obeying the property (b).
  Suppose for the sake of contradiction that \(X\) is disconnected.
  Then there exist disjoint non-empty sets \(V , W\) which are open relative to \(X\), such that \(V \cup W = X\).
  Since \(V\) and \(W\) are non-empty, we may choose an \(x \in V\) and \(y \in W\).
  Since \(V\) and \(W\) are disjoint, we have \(x \neq y\);
  without loss of generality we may assume \(x < y\).
  By property (b), we know that the entire interval \([x, y]\) is contained in \(X\).

  Now consider the set \([x, y] \cap V\).
  This set is both bounded and non-empty (because it contains \(x\)).
  Thus, it has a supremum
  \[
    z \coloneqq \sup([x, y] \cap V).
  \]
  Clearly, \(z \in [x, y]\), and hence \(z \in X\).
  Thus, either \(z \in V\) or \(z \in W\).
  Suppose first that \(z \in V\).
  Then \(z \neq y\) (since \(y \in W\) and \(V\) is disjoint from \(W\)).
  But \(V\) is open relative to \(X\), which contains \([x, y]\), so there is some ball \(B_{\big([x,y], d\big)}(z, r) = (z - r, z + r)\) which is contained in \(V\).
  But this contradicts the fact that \(z\) is the supremum of \([x, y] \cap V\).
  Now suppose that \(z \in W\).
  Then \(z \neq x\) (since \(x \in V\) and \(V\) is disjoint from \(W\)).
  But \(W\) is open relative to \(X\), which contains \([x, y]\), so there is some ball \(B_{\big([x,y], d\big)}(z, r) = (z - r, z + r)\) which is contained in \(W\).
  But this again contradicts the fact that \(z\) is the supremum of \([x, y] \cap V\).
  Thus, in either case we obtain a contradiction, which means that \(X\) cannot be disconnected, and must therefore be connected.

  Next we show that (b) implies (c).
  Suppose that
  \[
    \forall x, y \in X, x < y \implies [x, y] \subseteq X.
  \]
  Suppose for the sake of contradiction that \(X\) is not an interval.
  Then we would have
  \[
    \exists x, y \in X : x < y \implies \exists z \in \R \setminus X : x < z < y.
  \]
  Clearly, this contradicts to hypothesis, thus \(X\) is an interval.

  Finally we show that (c) implies (b).
  Suppose that \(X\) is an interval.
  Then we have
  \begin{align*}
             & \forall x, y \in X, x < y                                                        \\
    \implies & \inf(X) \leq x < y \leq \sup(X)                                                  \\
    \implies & \big(\forall z \in \R, x \leq z \leq y \implies \inf(X) \leq z \leq \sup(X)\big) \\
    \implies & \big(\forall z \in \R, x \leq z \leq y \implies z \in X\big)                     \\
    \implies & [x, y] \subseteq X.
  \end{align*}
\end{proof}

\begin{thm}tinuity preserves connectedness]\label{ii:2.4.6}
  Let f : \(X \to Y\) be a continuous map from one metric space \((X, d_X)\) to another \((Y, d_Y)\).
  Let \(E\) be any connected subset of \(X\).
  Then \(f(E)\) is also connected.
\end{thm}

\begin{proof}
  Suppose for the sake of contradiction that \(\big(f(E), d_Y|_{f(E) \times f(E)}\big)\) is disconnected.
  Then there exists two open set \(V_1, V_2\) in \(\big(f(E), d_Y|_{f(E) \times f(E)}\big)\) such that \(V_1 \cap V_2 = \emptyset\) and \(V_1 \cup V_2 = f(E)\).
  But then we have
  \begin{align*}
             & f \text{ is continuous from } (X, d_X) \text{ to } (Y, d_Y)                                                                 \\
    \implies & f^{-1}(V_1), f^{-1}(V_2) \text{ are open in } (E, d_X|_{E \times E})                                &  & \by{ii:2.1.5}[a,c] \\
    \implies & \big(f^{-1}(V_1) \cap f^{-1}(V_2) = \emptyset\big) \land \big(f^{-1}(V_1) \cup f^{-1}(V_2) = E\big)                         \\
    \implies & (E, d_X|_{E \times E}) \text{ is disconnected},                                                     &  & \by{ii:2.4.1}
  \end{align*}
  a contradiction.
  Thus, \(\big(f(E), d_Y|_{f(E) \times f(E)}\big)\) is connected.
\end{proof}

\begin{cor}[Intermediate value theorem]\label{ii:2.4.7}
  Let \(f : X \to \R\) be a continuous map from one metric space \((X, d_X)\) to the real line.
  Let \(E\) be any connected subset of \(X\), and let \(a, b\) be any two elements of \(E\).
  Let \(y\) be a real number between \(f(a)\) and \(f(b)\), i.e., either \(f(a) \leq y \leq f(b)\) or \(f(a) \geq y \geq f(b)\).
  Then there exists \(c \in E\) such that \(f(c) = y\).
\end{cor}

\begin{proof}
  Since \(f\) is continuous from \((X, d_X)\) to \((\R, d_{l^1}|_{\R \times \R})\) and \((E, d_X|_{E \times E})\) is connected, by \cref{ii:2.4.6} we know that \(\big(f(E), d_{l^1}|_{f(E) \times f(E)}\big)\) is connected.
  By \cref{ii:2.4.5}(a)(c) we know that \(f(E)\) is an interval.
  Thus, we have
  \begin{align*}
             & \forall a, b \in E, \forall y \in \Big[\min\big(f(a), f(b)\big), \max\big(f(a), f(b)\big)\Big] \\
    \implies & y \in f(E)                                                                                     \\
    \implies & \exists c \in X : f(c) = y.
  \end{align*}
\end{proof}

\exercisesection

\begin{ex}\label{ii:ex:2.4.1}
  Let \((X, d_{\text{disc}})\) be a metric space with the discrete metric.
  Let \(E\) be a subset of \(X\) which contains at least two elements.
  Show that \(E\) is disconnected.
\end{ex}

\begin{proof}
  Let \(x, y \in E\) such that \(x \neq y\), let \(V = \set{x}\) and let \(W = E \setminus V\).
  Since \(x \in V\) and \(y \in W\), we know that \(V \neq \emptyset \neq W\).
  By \cref{ii:1.2.8} we know that \(V, W\) are open in \((E, d_{\text{disc}}|_{E \times E})\).
  Since \(V \neq \emptyset \neq W\), \(V \cup W = E\) and \(V \cap W = \emptyset\), by \cref{ii:2.4.3} we know that \((E, d_{\text{disc}}|_{E \times E})\) is disconnected.
\end{proof}

\begin{ex}\label{ii:ex:2.4.2}
  Let \(f : X \to Y\) be a function from a connected metric space \((X, d)\) to a metric space \((Y, d_{\text{disc}})\) with the discrete metric.
  Show that \(f\) is continuous iff it is constant.
\end{ex}

\begin{proof}
  We first show that if \(f\) is continuous from \((X, d)\) to \((Y, d_{\text{disc}})\), then \(f\) is a constant function.
  Suppose for the sake of contradiction that \(f\) is not a constant function.
  Then we have
  \[
    \exists x_1, x_2 \in X : (x_1 \neq x_2) \land \big(f(x_1) \neq f(x_2)\big)
  \]
  and by \cref{ii:ex:2.4.1} we know that \(\big(f(E), d_{\text{disc}}|_{E \times E}\big)\) is disconnected.
  But \((X, d)\) is connected, thus by \cref{ii:2.4.6} we know that \(\big(f(E), d_{\text{disc}}|_{E \times E}\big)\) is connected, a contradiction.
  Thus, \(f\) is a constant function.

  Now we show that if \(f\) is a constant function, then \(f\) is continuous from \((X, d)\) to \((Y, d_{\text{disc}})\).
  We have
  \begin{align*}
             & \forall x_1, x_2 \in X, f(x_1) = f(x_2)                                                                                               \\
    \implies & \forall \varepsilon \in \R^+, \forall x_1, x_2 \in X, d_{\text{disc}}\big(f(x_1), f(x_2)\big) = 0 < \varepsilon                       \\
    \implies & f \text{ is uniformly continuous from } (X, d) \text{ to } (Y, d_{\text{disc}})                                 &  & \by{ii:2.3.4}    \\
    \implies & f \text{ is continuous from } (X, d) \text{ to } (Y, d_{\text{disc}}).                                          &  & \by{ii:ex:2.3.3}
  \end{align*}
\end{proof}

\begin{ex}\label{ii:ex:2.4.3}
  Prove the equivalence of statements (b) and (c) in \cref{ii:2.4.5}.
\end{ex}

\begin{proof}
  See \cref{ii:2.4.5}.
\end{proof}

\begin{ex}\label{ii:ex:2.4.4}
  Prove \cref{ii:2.4.6}.
\end{ex}

\begin{proof}
  See \cref{ii:2.4.6}.
\end{proof}

\begin{ex}\label{ii:ex:2.4.5}
  Use \cref{ii:2.4.6} to prove \cref{ii:2.4.7}.
\end{ex}

\begin{proof}
  See \cref{ii:2.4.7}.
\end{proof}

\begin{ex}\label{ii:ex:2.4.6}
  Let \((X, d)\) be a metric space, and let \((E_\alpha)_{\alpha \in I}\) be a collection of connected sets in \(X\).
  Suppose also that \(\bigcap_{\alpha \in I} E_\alpha\) is non-empty.
  Show that \(\bigcup_{\alpha \in I} E_\alpha\) is connected.
\end{ex}

\begin{proof}
  Let
  \[
    d_E = d|_{(\bigcup_{\alpha \in I} V_\alpha) \times (\bigcup_{\alpha \in I} V_\alpha)}.
  \]
  Suppose for the sake of contradiction that \((\bigcup_{\alpha \in I} E_\alpha, d_E)\) is disconnected.
  By \cref{ii:2.4.1} we know that
  \[
    \exists V, W \subseteq \bigcup_{\alpha \in I} E_\alpha : \begin{dcases}
      V, W \text{ are open in } \bigg(\bigcup_{\alpha \in I} E_\alpha, d_E\bigg) \\
      V \neq \emptyset \neq W                                                    \\
      V \cap W = \emptyset                                                       \\
      V \cup W = \bigcup_{\alpha \in I} E_\alpha
    \end{dcases}
  \]
  Since \(\bigcap_{\alpha \in I} E_\alpha \neq \emptyset\), we know that \(V_\alpha \neq \emptyset\) for each \(\alpha \in I\).
  We claim that there exists some \(\beta \in I\) such that \(V \cap E_\beta \neq \emptyset\) and \(W \cap E_\beta \neq \emptyset\).
  If not, then we would have
  \begin{align*}
             & (\forall \alpha \in I, V \cap E_\alpha = \emptyset) \lor (\forall \alpha \in I, W \cap E_\alpha = \emptyset) \\
    \implies & (V = \emptyset) \lor (W = \emptyset),
  \end{align*}
  which contradicts to \(V \neq \emptyset \neq W\).
  Now let \(\beta \in I\) such that \(V \cap E_\beta \neq \emptyset\) and \(W \cap E_\beta \neq \emptyset\).
  But then we have
  \begin{align*}
             & E_\beta \subseteq \bigcup_{\alpha \in I} E_\alpha = V \cup W                                      \\
    \implies & \begin{dcases}
                 V \cap E_\beta \text{ is open in } (E_\beta, d_E|_{E_\beta \times E_\beta}) \\
                 W \cap E_\beta \text{ is open in } (E_\beta, d_E|_{E_\beta \times E_\beta}) \\
                 V \cap E_\beta \neq \emptyset \neq W \cap E_\beta                           \\
                 (V \cap E_\beta) \cap (W \cap E_\beta) = \emptyset                          \\
                 (V \cap E_\beta) \cup (W \cap E_\beta) = E_\beta
               \end{dcases} &  & \by{ii:1.3.4}[a]                       \\
    \implies & (E_\beta, d_E|_{E_\beta \times E_\beta}) \text{ is disconnected},              &  & \by{ii:2.4.1}
  \end{align*}
  which contradict to the hypothesis that \((E_\beta, d_E|_{E_\beta \times E_\beta})\) is connected.
  Thus, we know that \((\bigcup_{\alpha \in I} E_\alpha, d_E)\) is connected.
\end{proof}

\begin{ex}\label{ii:ex:2.4.7}
  Let \((X, d)\) be a metric space, and let \(E\) be a subset of \(X\).
  We say that \(E\) is \emph{path-connected} iff, for every \(x, y \in E\), there exists a continuous function \(\gamma : [0, 1] \to E\) from the unit interval \([0, 1]\) to \(E\) such that \(\gamma(0) = x\) and \(\gamma(1) = y\).
  Show that every non-empty path-connected set is connected.
  (The converse is false, but is a bit tricky to show and will not be detailed here.)
\end{ex}

\begin{proof}
  Suppose for the sake of contradiction that \((E, d|_{E \times E})\) is disconnected.
  Then by \cref{ii:2.4.1} we have
  \[
    \exists V, W \subseteq E : \begin{dcases}
      V, W \text{ are open in } (E, d|_{E \times E}) \\
      V \neq \emptyset \neq W                        \\
      V \cap W = \emptyset                           \\
      V \cup W = E
    \end{dcases}
  \]
  Let \(x \in V\) and let \(y \in W\).
  Since \((E, d|_{E \times E})\) is path-connected and \(x, y \in E\), by definition we know that there exists a function \(\gamma : [0, 1] \to E\) which is continuous from \((\R, d_{l^1}|_{\R \times \R})\) to \((E, d|_{E \times E})\) and
  \[
    \begin{dcases}
      \gamma(0) = x;               \\
      \gamma(1) = y;               \\
      x \in \gamma([0, 1]) \cap V; \\
      y \in \gamma([0, 1]) \cap W.
    \end{dcases}
  \]
  But then we have
  \begin{align*}
             & ([0, 1], d_{l^1}|_{\R \times \R}) \text{ is connected}                                   &  & \by{ii:2.4.5}[a,c] \\
    \implies & \Big(\gamma([0, 1]), d|_{\gamma([0, 1]) \times \gamma([0, 1])}\Big) \text{ is connected} &  & \by{ii:2.4.6}
  \end{align*}
  and
  \begin{align*}
             & \begin{dcases}
                 \gamma([0, 1]) \cap V \text{ is open in } \big(r([0, 1]), d|_{\gamma([0, 1]) \times \gamma([0, 1])}\big) \\
                 \gamma([0, 1]) \cap W \text{ is open in } \big(r([0, 1]), d|_{\gamma([0, 1]) \times \gamma([0, 1])}\big) \\
                 \gamma([0, 1]) \cap V \neq \emptyset \neq \gamma([0, 1]) \cap W                                          \\
                 \big(\gamma([0, 1]) \cap V\big) \cap \big(\gamma([0, 1]) \cap W\big) = \emptyset                         \\
                 \big(\gamma([0, 1]) \cap V\big) \cup \big(\gamma([0, 1]) \cap W\big) = \gamma([0, 1])
               \end{dcases} &  & \by{ii:1.3.4}[a]                       \\
    \implies & \big(r([0, 1]), d|_{\gamma([0, 1]) \times \gamma([0, 1])}\big) \text{ is disconnected},                                     &  & \by{ii:2.4.1}
  \end{align*}
  a contradiction.
  Thus, we know that \((E, d|_{E \times E})\) is connected.
\end{proof}

\begin{ex}\label{ii:ex:2.4.8}
  Let \((X, d)\) be a metric space, and let \(E\) be a subset of \(X\).
  Show that if \(E\) is connected, then the closure \(\overline{E}\) of \(E\) is also connected.
  Is the converse true?
\end{ex}

\begin{proof}
  Let \(\overline{E}\) be the closure of \(E\) in \((X, d)\).
  Suppose for the sake of contradiction that \((\overline{E}, d|_{\overline{E} \times \overline{E}})\) is disconnected.
  Then by \cref{ii:2.4.1} we have
  \[
    \exists V, W \subseteq \overline{E} : \begin{dcases}
      V, W \text{ are open in } (\overline{E}, d|_{\overline{E} \times \overline{E}}) \\
      V \neq \emptyset \neq W                                                         \\
      V \cap W = \emptyset                                                            \\
      V \cup W = \overline{E}
    \end{dcases}
  \]
  By \cref{ii:1.2.10}(a)(b) we know that \(E \subseteq \overline{E}\), thus we know that at least one of \(E \cap V\) and \(E \cap W\) is not empty.
  Now we split into two cases:
  \begin{itemize}
    \item \(E \cap V = \emptyset\).
          But then we have \(E \subseteq W\) and
          \begin{align*}
                     & \forall x \in V, x \in \overline{E} \setminus E                                                                                          \\
            \implies & \forall x \in V, \exists r \in \R^+ : B_{(\overline{E}, d|_{\overline{E} \times \overline{E}})}(x, r) \subseteq V &  & \by{ii:1.2.15}[a] \\
            \implies & \forall y \in E, \forall x \in V, \exists r \in \R^+ : d|_{\overline{E} \times \overline{E}}(x, y) \geq r
          \end{align*}
          which contradict to \cref{ii:1.2.10}(c).
          Thus, we must have \(E \cap V \neq \emptyset\).
    \item \(E \cap W = \emptyset\).
          But using similar arguments as above, we know that we must have \(E \cap W \neq \emptyset\).
  \end{itemize}
  From all cases above we conclude that \(E \cap V \neq \emptyset \neq E \cap W\).
  But then we have
  \begin{align*}
             & E \subseteq \overline{E}                                     &  & \by{ii:1.2.10}[a,b] \\
    \implies & \begin{dcases}
                 E \cap V \text{ is open in } \big(E, d|_{E \times E}\big) \\
                 E \cap W \text{ is open in } \big(E, d|_{E \times E}\big) \\
                 E \cap V \neq \emptyset \neq E \cap W                     \\
                 \big(E \cap V\big) \cap \big(E \cap W\big) = \emptyset    \\
                 \big(E \cap V\big) \cup \big(E \cap W\big) = E
               \end{dcases} &  & \by{ii:1.3.4}[a]                             \\
    \implies & (E, d|_{E \times E}) \text{ is disconnected},                &  & \by{ii:2.4.1}
  \end{align*}
  a contradiction.
  Thus, we know that \((\overline{E}, d|_{\overline{E} \times \overline{E}})\) is connected.

  Now we give an example to show that the converse is not true.
  Let \(E = (0, 1) \cup (1, 2)\).
  Then we have \(\overline{E}_{(\R, d_{l^1}|_{\R \times \R})} = [0, 2]\).
  But by \cref{ii:2.4.5}(a)(c) we know that \((\overline{E}_{(\R, d_{l^1}|_{\R \times \R})}, d_{l^1}|_{\R \times \R})\) is connected but \((E, d_{l^1}|_{\R \times \R})\) is not.
\end{proof}

\begin{ex}\label{ii:ex:2.4.9}
  Let \((X, d)\) be a metric space. Let us define a relation \(x \sim y\) on \(X\) by declaring \(x \sim y\) iff there exists a connected subset of \(X\) which contains both \(x\) and \(y\).
  Show that this is an equivalence relation (i.e., it obeys the reflexive, symmetric, and transitive axioms).
  Also, show that the equivalence classes of this relation (i.e., the sets of the form \(\set{y \in  X : y \sim x}\) for some \(x \in X\)) are all closed and connected.
  These sets are known as the \emph{connected components} of \(X\).
\end{ex}

\begin{proof}
  Note that the relation \(\sim\) depends on the metric function \(d\) and we denoted it as \(\sim_d\).
  Since
  \begin{align*}
             & \forall x \in X, \set{x} \subseteq X                                                            \\
    \implies & \big(\set{x}, d|_{\set{x} \times \set{x}}\big) \text{ is connected} &  & \by{ii:2.4.1}          \\
    \implies & x \sim_d x,                                                         &  & \text{(by definition)}
  \end{align*}
  we know that \((X, \sim_d)\) is reflexive.
  Since
  \begin{align*}
             & \forall x, y \in X, x \sim_d y                                        \\
    \implies & \exists S \subseteq X : \begin{dcases}
                                         x, y \in S \\
                                         (S, d|_{S \times S}) \text{ is connected}
                                       \end{dcases} &  & \text{(by definition)}      \\
    \implies & y \sim_d x,                               &  & \text{(by definition)}
  \end{align*}
  we know that \((X, \sim_d)\) is symmetric.
  Since
  \begin{align*}
             & \forall x, y, z \in X, \begin{dcases}
                                        x \sim_d y \\
                                        y \sim_d z
                                      \end{dcases}                                                                                                     \\
    \implies & \exists S_1, S_2 \subseteq X : \begin{dcases}
                                                x, y \in S_1                                    \\
                                                (S_1, d|_{S_1 \times S_1}) \text{ is connected} \\
                                                y, z \in S_2                                    \\
                                                (S_2, d|_{S_2 \times S_2}) \text{ is connected}
                                              \end{dcases}                                       &                      & \text{(by definition)}        \\
    \implies & S_1 \cap S_2 \neq \emptyset                                                              & (y \in S_1 \cap S_2)                          \\
    \implies & \big((S_1 \cup S_2), d|_{(S_1 \cup S_2) \times (S_1 \cup S_2)}\big) \text{ is connected} &                      & \by{ii:ex:2.4.6}       \\
    \implies & x \sim_d z,                                                                              &                      & \text{(by definition)}
  \end{align*}
  we know that \((X, \sim_d)\) is transitive.
  Since \((X, \sim_d)\) is reflexive, symmetric and transitive, we know that \((X, \sim_d)\) is an equivalence relation.

  Next we show that every equivalence class of \((X, \sim_d)\) is connected.
  Let \(x \in X\) and let \(E_x = \set{y \in X : x \sim_d y}\).
  By definition we know that
  \[
    \forall y \in E_x, \exists S_y \subseteq X : \begin{dcases}
      (S_y, d|_{S_y \times S_y}) \text{ is connected} \\
      x, y \in S
    \end{dcases}
  \]
  Since
  \begin{align*}
             & \forall y \in E_x, x \in S_y                                                                                                           \\
    \implies & x \in \bigcap_{y \in E_x} S_y                                                                                                          \\
    \implies & \bigcap_{y \in E_x} S_y \neq \emptyset                                                                                                 \\
    \implies & (\bigcup_{y \in E_x} S_y, d|_{(\bigcup_{y \in E_x} S_y) \times (\bigcup_{y \in E_x} S_y)}) \text{ is connected}, &  & \by{ii:ex:2.4.6}
  \end{align*}
  by definition we know that \(E_x = \bigcup_{y \in E_x} S_y\).
  Thus, \((E_x, d|_{E_x \times E_x})\) is connected.
  Since \(x\) was arbitrary, we know that every equivalent class of \((X, \sim_d)\) is connected.

  Finally we show that every equivalence class of \((X, \sim_d)\) is closed.
  Let \(x \in X\), let \(E_x = \set{y \in X : x \sim_d y}\) and let \(\overline{E}_x\) be the closure of \(E_x\) in \((X, d)\).
  To show that \(E_x\) is closed in \((X, d)\), by \cref{ii:1.2.15}(b) we need to show that \(E_x = \overline{E}_x\).
  Since \((E_x, d|_{E_x \times E_x})\) is connected, by \cref{ii:ex:2.4.8} we know that \((\overline{E}_x, d|_{\overline{E}_x \times \overline{E}_x})\) is connected.
  Thus, we have
  \begin{align*}
             & E_x \subseteq \overline{E}_x                                            &  & \by{ii:1.2.10}[a,b]    \\
    \implies & (x \in \overline{E}_x) \land (\forall y \in \overline{E}_x, y \sim_d x) &  & \text{(by definition)} \\
    \implies & \forall y \in \overline{E}_x, y \in E_x                                                             \\
    \implies & E_x = \overline{E}_x.
  \end{align*}
\end{proof}

\begin{ex}\label{ii:ex:2.4.10}
  Combine \cref{ii:2.3.2} and \cref{ii:2.4.7} to deduce a theorem for continuous functions on a compact connected domain which generalizes Corollary 9.7.4 in Analysis I.
\end{ex}

\begin{proof}
  First we deduce a theorem was asked.
  Let \((X, d)\) be a metric space.
  Let \(f : X \to \R\) be continuous map from \((X, d_X)\) to \((\R, d_{l^1}|_{\R \times \R})\).
  Let \(E \subseteq X\) such that \(E \neq \emptyset\) and \((E, d|_{E \times E})\) is compact and connected.
  Then we have
  \[
    \exists x_{\min}, x_{\max} \in E : f(E) = \big[f(x_{\min}), f(x_{\max})\big].
  \]

  Now we prove the theorem.
  Since \((E, d|_{E \times E})\) is compact, by \cref{ii:2.3.2} we know that
  \[
    \exists x_{\min}, x_{\max} \in E : \forall x \in E, f(x_{\min}) \leq f(x) \leq f(x_{\max}).
  \]
  Then by \cref{ii:2.4.7} we have
  \[
    \forall y \in \big[f(x_{\min}), f(x_{\max})\big], \exists x \in E : f(x) = y.
  \]
\end{proof}

\section{Topological spaces}\label{ii:sec:2.5}

\begin{note}
  The concept of a metric space can be generalized to that of a \emph{topological space}.
  The idea here is not to view the metric \(d\) as the fundamental object;
  indeed, in a general topological space there is no metric at all.
  Instead, it is the collection of \emph{open sets} which is the fundamental concept.
  Thus, whereas in a metric space one introduces the metric \(d\) first, and then uses the metric to define first the concept of an open ball and then the concept of an open set, in a topological space one starts just with the notion of an open set.
  As it turns out, starting from the open sets, one cannot necessarily reconstruct a usable notion of a ball or metric (thus not all topological spaces will be metric spaces), but remarkably one can still define many of the concepts in the preceding sections.
\end{note}

\begin{defn}[Topological spaces]\label{ii:2.5.1}
  A \emph{topological space} is a pair \((X, \mathcal{F})\), where \(X\) is a set, and \(\mathcal{F} \subseteq 2^X\) is a collection of subsets of \(X\), whose elements are referred to as \emph{open sets}.
  Furthermore, the collection \(\mathcal{F}\) must obey the following properties:
  \begin{itemize}
    \item The empty set \(\emptyset\) and the whole set \(X\) are open;
          in other words, \(\emptyset \in \mathcal{F}\) and \(X \in \mathcal{F}\).
    \item Any finite intersection of open sets is open.
          In other words, if \(V_1 , \dots, V_n\) are elements of \(\mathcal{F}\), then \(V_1 \cap \dots \cap V_n\) is also in \(\mathcal{F}\).
    \item Any arbitrary union of open sets is open (including infinite unions).
          In other words, if \((V_\alpha)_{\alpha \in I}\) is a family of sets in \(\mathcal{F}\), then \(\bigcup_{\alpha \in I} V_\alpha\) is also in \(\mathcal{F}\).
  \end{itemize}
\end{defn}

\begin{note}
  In many cases, the collection \(\mathcal{F}\) of open sets can be deduced from context, and we shall refer to the topological space \((X, \mathcal{F})\) simply as \(X\).
\end{note}

\begin{note}
  From \cref{ii:1.2.15} we see that every metric space \((X, d)\) is automatically also a topological space
  (if we set \(\mathcal{F}\) equal to the collection of sets which are open in \((X, d)\)).
  However, there do exist topological spaces which do not arise from metric spaces.
\end{note}

\begin{defn}[Neighbourhoods]\label{ii:2.5.2}
  Let \((X, \mathcal{F})\) be a topological space, and let \(x \in X\).
  A \emph{neighbourhood} of \(x\) is defined to be any open set in \(\mathcal{F}\) which contains \(x\).
\end{defn}

\begin{eg}\label{ii:2.5.3}
  If \((X, d)\) is a metric space, \(x \in X\), and \(r > 0\), then \(B_{(X, d)}(x, r)\) is a neighbourhood of \(x\) (see \cref{ii:1.2.15}(c)).
\end{eg}

\begin{defn}[Topological convergence]\label{ii:2.5.4}
  Let m be an integer, \((X, \mathcal{F})\) be a topological space and let \((x^{(n)})_{n = m}^\infty\) be a sequence of points in \(X\).
  Let \(x\) be a point in \(X\).
  We say that \((x^{(n)})_{n = m}^\infty\) \emph{converges to} \(x\) iff, for every neighbourhood \(V\) of \(x\), there exists an \(N \geq m\) such that \(x^{(n)} \in V\) for all \(n \geq N\).
\end{defn}

\begin{note}
  \cref{ii:2.5.4} is consistent with that of convergence in metric spaces (\cref{ii:1.1.14}).
  One can then ask whether one has the basic property of uniqueness of limits (\cref{ii:1.1.20}).
  The answer turns out to usually be yes
  - if the topological space has an additional property known as the Hausdorff property
  - but the answer can be no for other topologies.
\end{note}

\begin{defn}[Interior, exterior, boundary]\label{ii:2.5.5}
  Let \((X, \mathcal{F})\) be a topological space, let \(E\) be a subset of \(X\), and let \(x_0\) be a point in \(X\).
  We say that \(x_0\) is an \emph{interior point of} \(E\) if there exists a neighbourhood \(V\) of \(x_0\) such that \(V \subseteq E\).
  We say that \(x_0\) is an \emph{exterior point of} \(E\) if there exists a neighbourhood \(V\) of \(x_0\) such that \(V \cap E = \emptyset\).
  We say that \(x_0\) is a \emph{boundary point of} \(E\) if it is neither an interior point nor an exterior point of \(E\).
\end{defn}

\begin{note}
  \cref{ii:2.5.5} is consistent with the corresponding notion for metric spaces (\cref{ii:1.2.5}).
\end{note}

\begin{defn}[Closure]\label{ii:2.5.6}
  Let \((X, \mathcal{F})\) be a topological space, let \(E\) be a subset of \(X\), and let \(x_0\) be a point in \(X\).
  We say that \(x_0\) is an adherent point of \(E\) if every neighbourhood \(V\) of \(x_0\) has a non-empty intersection with \(E\).
  The set of all adherent points of \(E\) is called the closure of \(E\) and is denoted \(\overline{E}\).
\end{defn}

\begin{note}
  We define a set \(K\) in a topological space \((X, \mathcal{F})\) to be closed iff its complement \(X \setminus K\) is open;
  this is consistent with the metric space definition, thanks to \cref{ii:1.2.15}(e).
\end{note}

\begin{defn}[Relative topology]\label{ii:2.5.7}
  Let \((X, \mathcal{F})\) be a topological space, and \(Y\) be a subset of \(X\).
  Then we define \(\mathcal{F}_Y \coloneqq \set{V \cap Y : V \in F}\), and refer this as the topology on \(Y\) \emph{induced} by \((X, \mathcal{F})\).
  We call \((Y, \mathcal{F}_Y)\) a \emph{topological subspace} of \((X, \mathcal{F})\).
\end{defn}

\begin{note}
  From \cref{ii:1.3.4} we see that \cref{ii:2.5.7} is compatible with the one for metric spaces.
\end{note}

\begin{defn}[Continuous functions]\label{ii:2.5.8}
  Let \((X, \mathcal{F}_X)\) and \((Y, \mathcal{F}_Y)\) be topological spaces, and let \(f : X \to Y\) be a function.
  If \(x_0 \in X\), we say that \(f\) is \emph{continuous at} \(x_0\) iff for every neighbourhood \(V\) of \(f(x_0)\), there exists a neighbourhood \(U\) of \(x_0\) such that \(f(U) \subseteq V\).
  We say that \(f\) is \emph{continuous} iff it is continuous at every point \(x \in X\).
\end{defn}

\begin{note}
  \cref{ii:2.5.8} is consistent with that in \cref{ii:2.1.1}.
  In particular, a function is continuous iff the pre-images of every open set is open.
\end{note}

\begin{note}
  There is unfortunately no notion of a Cauchy sequence, a complete space, or a bounded space, for topological spaces.
  However, there is certainly a notion of a compact space.
\end{note}

\begin{defn}[Compact topological spaces]\label{ii:2.5.9}
  Let \((X, \mathcal{F})\) be a topological space.
  We say that this space is \emph{compact} if every open cover of \(X\) has a finite subcover.
  If \(Y\) is a subset of \(X\), we say that \(Y\) is compact if the topological space on \(Y\) induced by \((X, \mathcal{F})\) is compact.
\end{defn}

\begin{note}
  Many basic facts about compact metric spaces continue to hold true for compact topological spaces, notably \cref{ii:2.3.1} and \cref{ii:2.3.2}.
  However, there is no notion of uniform continuity, and so there is no analogue of \cref{ii:2.3.5}.
\end{note}

\begin{note}
  We can also define the notion of connectedness by repeating \cref{ii:2.4.1} verbatim, and also repeating \cref{ii:2.4.3} (but with \cref{ii:2.5.7} instead of \cref{ii:1.3.3}).
  Many of the results and exercises in \cref{ii:sec:2.4} continue to hold for topological spaces
  (with almost no changes to any of the proofs!).
\end{note}

\exercisesection

\begin{ex}\label{ii:ex:2.5.1}
  Let \(X\) be an arbitrary set, and let \(\mathcal{F} \coloneqq \set{\emptyset, X}\).
  Show that \((X, \mathcal{F})\) is a topology
  (called the \emph{trivial topology} on \(X\)).
  If \(X\) contains more than one element, show that the trivial topology cannot be obtained from by placing a metric \(d\) on \(X\).
  Show that this topological space is both compact and connected.
\end{ex}

\begin{proof}
  Let \(X\) be a set and let \(\mathcal{F} = \set{\emptyset, X}\).
  First we show that \((X, \mathcal{F})\) is a topology.
  Let \(n \in \N\), let \(S_1, \dots, S_n \in \mathcal{F}\) and let \(i, j \in \Z^+\).
  If there exists some \(1 \leq j \leq n\) such that \(S_j = \emptyset\), then we know that \(\bigcap_{i = 1}^n S_i = \emptyset \in \mathcal{F}\).
  If such \(j\) does not exist, then we have \(S_i = X\) for every \(1 \leq i \leq n\) and \(\bigcap_{i = 1}^n S_i = X \in \mathcal{F}\).
  Since \(n\) was arbitrary, we conclude that for arbitrary finite collection of element in \(\mathcal{F}\) there intersection is still in \(\mathcal{F}\).

  Let \(S \subseteq 2^{\mathcal{F}}\).
  Then we have
  \[
    \forall s \in S, (s = \emptyset) \lor (s = X) \implies \bigcup S \in F
  \]
  and we conclude that any union of open sets is open.

  Since \(\emptyset, X \in \mathcal{F}\) and the claims above, by \cref{ii:2.5.1} we know that \((X, \mathcal{F})\) is a topology.

  Next we show that if \(X\) contains more than one element, then \((X, \mathcal{F})\) cannot be obtained from by placing a metric \(d\) on \(X\).
  Let \((X, \mathcal{F})\) be a trivial topology and let \(x, y \in X\) such that \(x \neq y\).
  Given arbitrary metric \(d\), by \cref{ii:1.1.2}(b) we know that \(d(x, y) > \R^+\).
  But by \cref{ii:1.2.15}(c) we know that \(B_{(X, d)}\big(x, \dfrac{d(x, y)}{2}\big)\) is open in \((X, d)\), thus by \cref{ii:2.5.1} we must have \(B_{(X, d)}\big(x, \dfrac{d(x, y)}{2}\big) \in \mathcal{F}\), which means \((X, \mathcal{F})\) is a not trivial topology.

  Finally we show that if \(X\) contains more than one element, then \((X, \mathcal{F})\) is compact and connected.
  Since \(\emptyset, X\) are the only two open sets in \(\mathcal{F}\), we know that an open cover of \(X\) is either \(\set{X}\) or \(\set{\emptyset, X}\), and both are finite.
  Thus by \cref{ii:2.5.9} \((X, \mathcal{F})\) is compact.
  Since \(X\) is the only non-empty open set in \(\mathcal{F}\), by \cref{ii:2.4.3} we know that \((X, \mathcal{F})\) is connected.
\end{proof}

\begin{ex}\label{ii:ex:2.5.2}
  Let \((X, d)\) be a metric space
  (and hence a topological space).
  Show that the two notions of convergence of sequences in \cref{ii:1.1.14} and \cref{ii:2.5.4} coincide.
\end{ex}

\begin{proof}
  Let \(\mathcal{F}\) be the set of all open sets in \((X, d)\) and let \(N \in \N\).
  First suppose that \((x^{(n)})_{n = m}^\infty\) converges to \(x\) in the sense of \cref{ii:1.1.14}.
  Let \(V \in \mathcal{F}\) be a neighbourhood of \(x\).
  By \cref{ii:2.5.2} we know that \(V\) is open in \((X, d)\).
  Since \(x \in V\), by \cref{ii:1.2.15}(a) we know that
  \[
    \exists \varepsilon \in \R^+ : B_{(X, d)}(x, \varepsilon) \subseteq V.
  \]
  Now we fix such \(\varepsilon\).
  Then we have
  \begin{align*}
             & \lim_{n \to \infty} d(x^{(n)}, x) = 0                                                                                                   \\
    \implies & \exists N \geq m : \forall n \geq N, d(x^{(n)}, x) < \varepsilon            &                                          & \by{ii:1.1.14} \\
    \implies & \exists N \geq m : \forall n \geq N, x^{(n)} \in B_{(X, d)}(x, \varepsilon) &                                          & \by{ii:1.2.1}  \\
    \implies & \exists N \geq m : \forall n \geq N, x^{(n)} \in V.                         & (B_{(X, d)}(x, \varepsilon) \subseteq V)
  \end{align*}
  Since \(V\) was arbitrary, we know that \((x^{(n)})_{n = m}^\infty\) converges to \(x\) in the sense of \cref{ii:2.5.4}.

  Now suppose that \((x^{(n)})_{n = m}^\infty\) converges to \(x\) in the sense of \cref{ii:2.5.4}.
  Then we have
  \begin{align*}
             & \forall \varepsilon \in \R^+, B_{(X, d)}(x, \varepsilon) \in \mathcal{F}                                  &  & \by{ii:1.2.15}[c] \\
    \implies & \forall \varepsilon \in \R^+, \exists N \geq m : \forall n \geq N, x^{(n)} \in B_{(X, d)}(x, \varepsilon) &  & \by{ii:2.5.4}     \\
    \implies & \forall \varepsilon \in \R^+, \exists N \geq m : \forall n \geq N, d(x^{(n)}, x) < \varepsilon            &  & \by{ii:1.2.1}     \\
    \implies & \lim_{n \to \infty} d(x^{(n)}, x) = 0.                                                                    &  & \by{ii:1.1.14}
  \end{align*}
  Thus \cref{ii:1.1.14} and \cref{ii:2.5.4} coincide.
\end{proof}

\begin{ex}\label{ii:ex:2.5.3}
  Let \((X, d)\) be a metric space (and hence a topological space).
  Show that the two notions of interior, exterior, and boundary in \cref{ii:1.2.5,ii:2.5.5} coincide.
\end{ex}

\begin{proof}
  Let \(\mathcal{F}\) be the set of all open sets in \((X, d)\), let \(E \subseteq X\) and let \(x_0 \in X\).
  First suppose that \(x_0\) is an interior point of \(E\) in the sense of \cref{ii:1.2.5}.
  Then we have
  \begin{align*}
             & x_0 \in \text{int}_{(X, d)}(E)                                         \\
    \implies & \exists r \in \R^+ : B_{(X, d)}(x_0, r) \subseteq E &  & \by{ii:1.2.5} \\
    \implies & \exists r \in \R^+ : \begin{dcases}
                                      B_{(X, d)}(x_0, r) \in \mathcal{F} \\
                                      B_{(X, d)}(x_0, r) \subseteq E
                                    \end{dcases}               &  & \by{ii:1.2.15}[c] \\
    \implies & x_0 \in \text{int}_{(X, \mathcal{F})}(E).           &  & \by{ii:2.5.5}
  \end{align*}

  Next suppose that \(x_0\) is an interior point of \(E\) in the sense of \cref{ii:2.5.5}.
  Then we have
  \begin{align*}
             & x_0 \in \text{int}_{(X, \mathcal{F})}(E)                                               \\
    \implies & \exists V \in \mathcal{F} : (x_0 \in V) \land (V \subseteq E)   &  & \by{ii:2.5.5}     \\
    \implies & \exists r \in \R^+ : B_{(X, d)}(x_0, r) \subseteq V \subseteq E &  & \by{ii:1.2.15}[a] \\
    \implies & x_0 \in \text{int}_{(X, d)}(E).                                 &  & \by{ii:1.2.5}
  \end{align*}

  Next suppose that \(x_0\) is an exterior point of \(E\) in the sense of \cref{ii:1.2.5}.
  Then we have
  \begin{align*}
             & x_0 \in \text{ext}_{(X, d)}(E)                                                \\
    \implies & \exists r \in \R^+ : B_{(X, d)}(x_0, r) \cap E = \emptyset &  & \by{ii:1.2.5} \\
    \implies & \exists r \in \R^+ : \begin{dcases}
                                      B_{(X, d)}(x_0, r) \in \mathcal{F} \\
                                      B_{(X, d)}(x_0, r) \cap E = \emptyset
                                    \end{dcases}                      &  & \by{ii:1.2.15}[c] \\
    \implies & x_0 \in \text{ext}_{(X, \mathcal{F})}(E).                  &  & \by{ii:2.5.5}
  \end{align*}

  Next suppose that \(x_0\) is an exterior point of \(E\) in the sense of \cref{ii:2.5.5}.
  Then we have
  \begin{align*}
             & x_0 \in \text{ext}_{(X, \mathcal{F})}(E)                                                \\
    \implies & \exists V \in \mathcal{F} : (x_0 \in V) \land (V \cap E = \emptyset) &  & \by{ii:2.5.5} \\
    \implies & \exists r \in \R^+ : \begin{dcases}
                                      B_{(X, d)}(x_0, r) \subseteq V \\
                                      B_{(X, d)}(x_0, r) \cap E = \emptyset
                                    \end{dcases}                                &  & \by{ii:1.2.15}[a] \\
    \implies & x_0 \in \text{ext}_{(X, d)}(E).                                      &  & \by{ii:1.2.5}
  \end{align*}

  Next suppose that \(x_0\) is an boundary point of \(E\) in the sense of \cref{ii:1.2.5}.
  Then we have
  \begin{align*}
             & x_0 \in \partial_{(X, d)}(E)                                              \\
    \implies & \forall r \in \R^+, \begin{dcases}
                                     B_{(X, d)}(x_0, r) \not\subseteq E \\
                                     B_{(X, d)}(x_0, r) \cap E \neq \emptyset
                                   \end{dcases}               &  & \by{ii:1.2.5}         \\
    \implies & \forall V \in \mathcal{F}, x_0 \in V \text{ implies}                      \\
             & \begin{dcases}
                 \exists r \in \R^+ : B_{(X, d)}(x_0, r) \subseteq V \\
                 V \not\subseteq E                                   \\
                 V \cap E \neq \emptyset
               \end{dcases} &  & \by{ii:1.2.15}[a]                       \\
    \implies & x_0 \in \partial_{(X, \mathcal{F})}(E).                &  & \by{ii:2.5.5}
  \end{align*}

  Finally suppose that \(x_0\) is an boundary point of \(E\) in the sense of \cref{ii:2.5.5}.
  Then we have
  \begin{align*}
             & x_0 \in \partial_{(X, \mathcal{F})}(E)                                                 \\
    \implies & \forall V \in \mathcal{F}, x_0 \in V \text{ implies} \begin{dcases}
                                                                      V \not\subseteq E \\
                                                                      V \cap E \neq \emptyset
                                                                    \end{dcases} &  & \by{ii:2.5.5}   \\
    \implies & \forall r \in \R^+, \begin{dcases}
                                     B_{(X, d)}(x_0, r) \in \mathcal{F} \\
                                     B_{(X, d)}(x_0, r) \not\subseteq E \\
                                     B_{(X, d)}(x_0, r) \cap E \neq \emptyset
                                   \end{dcases}                            &  & \by{ii:1.2.15}[c]     \\
    \implies & x_0 \in \partial_{(X, d)}(E).                                       &  & \by{ii:1.2.5}
  \end{align*}
  Thus \cref{ii:1.2.5} and \cref{ii:2.5.5} coincide.
\end{proof}

\begin{ex}\label{ii:ex:2.5.4}
  A topological space \((X, \mathcal{F})\) is said to be \emph{Hausdorff} if given any two distinct points \(x, y \in X\), there exists a neighbourhood \(V\) of \(x\) and a neighbourhood \(W\) of \(y\) such that \(V \cap W = \emptyset\).
  Show that any topological space coming from a metric space is Hausdorff, and show that the trivial topology is not Hausdorff if the space contains at least two elements.
  Show that the analogue of \cref{ii:1.1.20} holds for Hausdorff topological spaces, but give an example of a non-Hausdorff topological space in which \cref{ii:1.1.20} fails.
  (In practice, most topological spaces one works with are Hausdorff;
  non-Hausdorff topological spaces tend to be so pathological that it is not very profitable to work with them.)
\end{ex}

\begin{proof}
  We first show that every topological space coming from a metric space is Hausdorff.
  Let \((X, d)\) be a metric space and let \(\mathcal{F}\) be the set of all open sets in \((X, d)\).
  Let \(x, y \in X\).
  Then we have
  \begin{align*}
             & x \neq y                                                                                                                                                 \\
    \implies & d(x, y) \in \R^+                                                                                                             &  & \by{ii:1.1.2}[b]       \\
    \implies & \begin{dcases}
                 B_{(X, d)}\big(x, \dfrac{d(x, y)}{2}\big) \in \mathcal{F} \\
                 B_{(X, d)}\big(y, \dfrac{d(x, y)}{2}\big) \in \mathcal{F}
               \end{dcases}                                                            &  & \by{ii:1.2.15}[c]                                                           \\
    \implies & \bigg(B_{(X, d)}\big(x, \dfrac{d(x, y)}{2}\big)\bigg) \cap \bigg(B_{(X, d)}\big(y, \dfrac{d(x, y)}{2}\big)\bigg) = \emptyset &  & \by{ii:1.1.2}[d]       \\
    \implies & (X, \mathcal{F}) \text{ is a Hausdorff space}.                                                                               &  & \text{(by definition)}
  \end{align*}

  Next we show that if a trivial topological space contains at least two elements, then it is not Hausdorff.
  Let \(X\) be a set such that \(x, y \in X\) and \(x \neq y\).
  Let \(\mathcal{F} = \set{\emptyset, X}\).
  By \cref{ii:2.5.2} the only neighbourhood of \(x\) in \(\mathcal{F}\) is \(X\), similarly the only neighbourhood of \(y\) in \(\mathcal{F}\) is \(X\).
  But \(X \cap X \neq \emptyset\) implies \((X, \mathcal{F})\) is not Hausdorff.

  Next we show that every convergent sequence in a Hausdorff space has only one limit.
  Let \((X, \mathcal{F})\) be a Hausdorff space, let \((x^{(n)})_{n = 1}^\infty\) be a sequence in \(X\) and let \(x, x' \in X\) such that \((x^{(n)})_{n = 1}^\infty\) converges to \(x, x'\), respectively.
  Suppose for sake of contradiction that \(x \neq x'\).
  Since \(x \neq x'\) and \((X, \mathcal{F})\) is Hausdorff, by definition we know that
  \[
    \exists V, V' \in \mathcal{F} : \begin{dcases}
      x \in V   \\
      x' \in V' \\
      V \cap V' = \emptyset
    \end{dcases}
  \]
  But then we have
  \begin{align*}
             & \begin{dcases}
                 (x^{(n)})_{n = 1}^\infty \text{ converges to } x \\
                 (x^{(n)})_{n = 1}^\infty \text{ converges to } x'
               \end{dcases}                           \\
    \implies & \begin{dcases}
                 \exists N \in \Z^+ : \forall n \geq N, x^{(n)} \in V \\
                 \exists N' \in \Z^+ : \forall n \geq N', x^{(n)} \in V'
               \end{dcases}                    &  & \by{ii:2.5.4}                     \\
    \implies & \exists N, N' \in \Z^+ : \forall n \geq \max(N, N'), x^{(n)} \in V \cap V' \\
    \implies & V \cap V' \neq \emptyset,
  \end{align*}
  a contradiction.
  Thus we must have \(x = x'\).

  Finally we give an non-Hausdorff topology space in which \cref{ii:1.1.20} fails.
  Let \(X = \set{0, 1}\) and let \(\mathcal{F} = \set{\emptyset, X}\).
  From the proof above we know that \((X, \mathcal{F})\) is not Hausdorff.
  Let \((x^{(n)})_{n = 1}^\infty\) be a sequence in \(X\).
  By \cref{ii:2.5.2} we know that the only neighbourhood of \(0\) in \(\mathcal{F}\) is \(X\).
  Similarly the only neighbourhood of \(1\) in \(\mathcal{F}\) is \(X\).
  Thus we have
  \begin{align*}
             & \forall n \in \Z^+, x^{(n)} \in X                \\
    \implies & \begin{dcases}
                 (x^{(n)})_{n = 1}^\infty \text{ converges to } 0 \\
                 (x^{(n)})_{n = 1}^\infty \text{ converges to } 1
               \end{dcases} &  & \by{ii:2.5.4}
  \end{align*}
  but \(0 \neq 1\).
\end{proof}

\begin{ex}\label{ii:ex:2.5.5}
  Given any totally ordered set \(X\) with order relation \(\leq\), declare a set \(V \subseteq X\) to be \emph{open} if for every \(x \in V\) there exists a set \(I\) which is an interval \(\set{y \in X : a < y < b}\) for some \(a, b \in X\), a ray \(\set{y \in X : a < y}\) for some \(a \in X\), the ray \(\set{y \in X : y < b}\) for some \(b \in X\), or the whole space \(X\), which contains \(x\) and is contained in \(V\).
  Let \(\mathcal{F}\) be the set of all open subsets of \(X\).
  Show that \((X, \mathcal{F})\) is a topology (this is the \emph{order topology} on the totally ordered set \((X, \leq)\)) which is Hausdorff in the sense of \cref{ii:ex:2.5.4}.
  Show that on the real line \(\R\) (with the standard ordering \(\leq\)), the order topology matches the standard topology (i.e., the topology arising from the standard metric).
  If instead one applies this to the extended real line \(\R^*\), show that \(\R\) is an open set with boundary \(\set{-\infty, +\infty}\).
  If \((x_n)_{n = 1}^\infty\) is a sequence of numbers in \(\R\) (and hence in \(\R^*\)), show that \(x_n\) converges to \(+\infty\) iff \(\liminf_{n \to \infty} x_n = +\infty\), and \(x_n\) converges to \(-\infty\) iff \(\limsup_{n \to \infty} x_n = -\infty\).
\end{ex}

\begin{proof}
  We first show that \((X, \mathcal{F})\) is a topology space.
  By definition we know that \(X\) is open and \(\emptyset\) is open trivially, thus \(X, \emptyset \in \mathcal{F}\).
  Let \(n \in \N\) and let \(S_n \subseteq \mathcal{F}\) such that \(\#(S_n) = n\).
  We use induction on \(n\) to show that \(\bigcap S_n \in \mathcal{F}\) for every \(n \in \N\).
  For \(n = 0\), we have \(S_0 = \emptyset\) and \(\bigcap S_0 = \emptyset\).
  From the proof above we know that \(\emptyset \in \mathcal{F}\), thus the base case holds.
  Suppose inductively that \(\bigcap S_n \in \mathcal{F}\) for some \(n \geq 0\).
  Let \(S_{n + 1} \subseteq \mathcal{F}\) such that \(\#(S_{n + 1}) = n + 1\).
  Then we have \(S_{n + 1} = \set{V_1, \dots, V_{n + 1} : \forall i \in \Z^+, V_i \in \mathcal{F}}\) and \(\bigcap S_{n + 1} = \bigcap_{i = 1}^{n + 1} V_i\).
  If \(\bigcap S_{n + 1} = \emptyset\), then from the proof above we know that \(\emptyset \in \mathcal{F}\).
  So suppose that \(\bigcap S_{n + 1} \neq \emptyset\).
  Let \(x \in \bigcap S_{n + 1}\).
  Since \(x \in \bigcap_{i = 1}^n V_i\) and \(\#(\set{V_1, \dots, V_n}) = n\), by induction hypothesis we know that there exists a set \(I\) in one of the following forms
  \[
    I = \begin{dcases}
      \set{y \in X : a < y < b} \text{ for some } a, b \in X \\
      \set{y \in X : a < y} \text{ for some } a \in X        \\
      \set{y \in X : y < b} \text{ for some } b \in X        \\
      X
    \end{dcases}
  \]
  such that \(x \in I\) and \(I \subseteq \bigcap_{i = 1}^n V_i\).
  Since \(x \in V_{n + 1}\) and \(V_{n + 1} \in \mathcal{F}\), we know that there exists a set \(I'\) in one of the following forms
  \[
    I' = \begin{dcases}
      \set{y \in X : a' < y < b'} \text{ for some } a', b' \in X \\
      \set{y \in X : a' < y} \text{ for some } a' \in X          \\
      \set{y \in X : y < b'} \text{ for some } b' \in X          \\
      X
    \end{dcases}
  \]
  such that \(x \in I'\) and \(I' \subseteq V_{n + 1}\).
  Then we have \(x \in I \cap I'\) and \(I \cap I' \subseteq \bigcap_{i = 1}^{n + 1} V_i\).
  Since \((X, \leq)\) is totally ordered, we know that \(I \cap I'\) is in one of the following forms
  \[
    I \cap I' = \begin{dcases}
      \set{y \in X : \max_{(X, \leq)}(a, a') < y < \min_{(X, \leq)}(b, b')} \\
      \set{y \in X : \max_{(X, \leq)}(a, a') < y < b}                       \\
      \set{y \in X : a < y < \min_{(X, \leq)}(b, b')}                       \\
      \set{y \in X : a < y < b}                                             \\
      \set{y \in X : \max_{(X, \leq)}(a, a') < y < b'}                      \\
      \set{y \in X : \max_{(X, \leq)}(a, a') < y}                           \\
      \set{y \in X : a < y < b'}                                            \\
      \set{y \in X : a < y}                                                 \\
      \set{y \in X : a' < y < \min_{(X, \leq)}(b, b')}                      \\
      \set{y \in X : a' < y < b}                                            \\
      \set{y \in X : y < \min_{(X, \leq)}(b, b')}                           \\
      \set{y \in X : y < b}                                                 \\
      \set{y \in X : a' < y < b'}                                           \\
      \set{y \in X : a' < y}                                                \\
      \set{y \in X : y < b'}                                                \\
      X
    \end{dcases}
  \]
  Thus by definition \(I \cap I'\) is an interval.
  Since \(x\) was arbitrary, we know that \(\bigcap S_{n + 1}\) is open in \((X, \mathcal{F})\), and this closes the induction.
  We conclude that for any finite collection of open sets, their intersection is again open in \((X, \mathcal{F})\).

  Let \(S \subseteq \mathcal{F}\).
  If \(\bigcup S = \emptyset\), then from proof above we know that \(\emptyset \in \mathcal{F}\).
  So suppose that \(\bigcup S \neq \emptyset\).
  Let \(x \in \bigcup S\).
  We know that there exists an \(V \in S\) such that \(x \in V\).
  Since \(V \in S\), we know that \(V\) is open in \((X, \mathcal{F})\) and by definition there exists an interval \(I\) such that \(x \in I\) and \(I \subseteq V\).
  Then we have \(I \subseteq V \subseteq \bigcup S\).
  Since \(x\) was arbitrary, we know that \(\bigcup S\) is open in \((X, \mathcal{F})\).
  Combine all the results above we know that \((X, \mathcal{F})\) is a topological space by \cref{ii:2.5.1}.

  Next we show that \((X, \mathcal{F})\) is Hausdorff.
  Let \(x_1, x_2 \in X\) such that \(x_1 \neq x_2\).
  Since \((X, \leq)\) is totally ordered, we have either \(x_1 < x_2\) or \(x_2 < x_1\).
  Without the loss of generality suppose that \(x_1 < x_2\).
  Let \(I_1 = \set{y \in X : y < x_2}\) and let \(I_2 = \set{y \in X : x_1 < y}\).
  Then we have \(x_1 \in I_1\) and \(x_2 \in I_2\).
  By definition we know that \(I_1, I_2 \in \mathcal{F}\).
  If \(I_1 \cap I_2 = \emptyset\), then we are done.
  So suppose that \(I_1 \cap I_2 \neq \emptyset\).
  Let \(x \in I_1 \cap I_2\), let \(J_1 = \set{y \in X : y < x}\) and let \(J_2 = \set{y \in X : x < y}\).
  Since \(I_1 \cap I_2 = \set{y \in X : x_1 < y < x_2}\), we know that \(x \neq x_1\) and \(x \neq x_2\).
  Since \(x_1 < x\), we have \(x_1 \in J_1\).
  Similarly we have \(x_2 \in J_2\).
  By definition we know that \(J_1, J_2 \in \mathcal{F}\).
  Since \((X, \leq)\) is totally ordered, we know that \(J_1 \cap J_2 = \emptyset\).
  Since \(x_1, x_2\) were arbitrary, by \cref{ii:ex:2.5.4} we know that \((X, \mathcal{F})\) is Hausdorff.

  Next we show that the order topology in \(\R\) with order relation \(\leq\) matches standard topology.
  Let \(\mathcal{F}_o\) be the order topology in \(\R\) and let \(\mathcal{F}_s\) be the standard topology in \(\R\).
  We want to show that \(\mathcal{F}_o = \mathcal{F}_s\).

  Let \(V \in \mathcal{F}_o\) and let \(x \in V\).
  Then we have
  \[
    \exists I \subseteq \R : \begin{dcases}
      I \text{ is an open interval in } \R \\
      x \in I                              \\
      I \subseteq V
    \end{dcases}
  \]
  Now we split into four cases:
  \begin{itemize}
    \item If \(I = (a, b)\) for some \(a, b \in \R\), then we have
          \begin{align*}
                     & x \in (a, b)                                                                            \\
            \implies & r = \min(\abs{x - a}, \abs{x - b}) = \min(x - a, b - x) > 0                             \\
            \implies & (x - r, x + r) \subseteq (a, b) \subseteq V                                             \\
            \implies & B_{(\R, d_{l^1}|_{\R \times \R})}(x, r) \subseteq (a, b) \subseteq V &  & \by{ii:1.2.1} \\
            \implies & x \in \text{int}_{(\R, d_{l^1}|_{\R \times \R})}(V).                 &  & \by{ii:1.2.5}
          \end{align*}
    \item If \(I = (a, \infty)\) for some \(a \in \R\), then we have
          \begin{align*}
                     & x \in (a, \infty)                                                                            \\
            \implies & r = \abs{x - a} = x - a > 0                                                                  \\
            \implies & (x - r, x + r) \subseteq (a, \infty) \subseteq V                                             \\
            \implies & B_{(\R, d_{l^1}|_{\R \times \R})}(x, r) \subseteq (a, \infty) \subseteq V &  & \by{ii:1.2.1} \\
            \implies & x \in \text{int}_{(\R, d_{l^1}|_{\R \times \R})}(V).                      &  & \by{ii:1.2.5}
          \end{align*}
    \item If \(I = (-\infty, b)\) for some \(b \in \R\), then we have
          \begin{align*}
                     & x \in (-\infty, b)                                                                            \\
            \implies & r = \abs{x - b} = b - x > 0                                                                   \\
            \implies & (x - r, x + r) \subseteq (-\infty, b) \subseteq V                                             \\
            \implies & B_{(\R, d_{l^1}|_{\R \times \R})}(x, r) \subseteq (-\infty, b) \subseteq V &  & \by{ii:1.2.1} \\
            \implies & x \in \text{int}_{(\R, d_{l^1}|_{\R \times \R})}(V).                       &  & \by{ii:1.2.5}
          \end{align*}
    \item If \(I = \R\), then we have \(V = \R\) and \(x \in \text{int}_{(\R, d_{l^1}|_{\R \times \R})}(V) = \R\).
  \end{itemize}
  From all cases above we conclude that \(x \in \text{int}_{(\R, d_{l^1}|_{\R \times \R})}(V)\).
  Since \(x\) was arbitrary, by \cref{ii:1.2.15}(a) we know that \(V\) is open in \((X, d_{l^1}|_{\R \times \R})\) and thus \(V \in \mathcal{F}_s\).
  Since \(V\) was arbitrary, we have \(\mathcal{F}_o \subseteq \mathcal{F}_s\).

  Let \(W \in \mathcal{F}_s\).
  Since \(W\) is open in \((\R, d_{l^1}|_{\R \times \R})\), we have
  \begin{align*}
             & \forall x \in W, \exists r \in \R^+ : B_{(\R, d_{l^1}|_{\R \times \R})}(x, r) \subseteq W &  & \by{ii:1.2.15}[a]      \\
    \implies & \forall x \in W, \exists r \in \R^+ : (x - r, x + r) \subseteq W                          &  & \by{ii:1.2.1}          \\
    \implies & W \in \mathcal{F}_o.                                                                      &  & \text{(by definition)}
  \end{align*}
  Since \(W\) was arbitrary, we have \(\mathcal{F}_s \subseteq \mathcal{F}_o\).
  From the proof above we thus have \(\mathcal{F}_o = \mathcal{F}_s\).

  Next we show that if \((\R^*, \mathcal{F})\) is an order topology with order relation \(\leq\), then \(\R\) is open in \((\R^*, \mathcal{F})\).
  Since \((\R^*, \leq)\) is totally ordered, we know that \((\R^*, \mathcal{F})\) is an order topology.
  Since
  \[
    \forall x \in \R, (x - 1, x + 1) \subseteq \R,
  \]
  by definition we know that \(\R\) is open in \((\R^*, \mathcal{F})\).

  Next we show that if \((\R^*, \mathcal{F})\) is an order topology with order relation \(\leq\), then the boundary points of \(\R\) in \((\R^*, \mathcal{F})\) are \(-\infty\) and \(\infty\).
  Since \(\infty \notin \R\) and \(\R\) is open in \((\R^*, \mathcal{F})\), by \cref{ii:2.5.5} we know that \(\infty\) is either an exterior point or a boundary point of \(\R\) in \((\R^*, \mathcal{F})\).
  Suppose for sake of contradiction that \(\infty\) is an exterior point of \(\R\) in \((\R^*, \mathcal{F})\).
  Then by \cref{ii:2.5.5} we have
  \[
    \exists V \in \mathcal{F} : (\infty \in V) \land (\R \cap V = \emptyset).
  \]
  Since \(V \cap \R = \emptyset\), we know that the only possible choices of \(V\) are \(\set{\infty}\) or \(\set{-\infty, \infty}\).
  But in either cases we cannot find an open interval \(I \subseteq \R^*\) such that \(I \subseteq V\), a contradiction.
  Thus \(\infty\) is a boundary point of \(\R\) in \((\R^*, \mathcal{F})\).
  Using similar arguments as above we can show that \(-\infty\) is a boundary point of \(\R\) in \((\R^*, \mathcal{F})\).

  Finally we show that if  \((\R^*, \mathcal{F})\) is an order topology with order relation \(\leq\) and \((x^{(n)})_{n = 1}^\infty\), \((y^{(n)})_{n = 1}^\infty\) are sequence in \(\R\), then we have
  \[
    \begin{dcases}
      x_n \text{ converges to } \infty \text{ in } (\R^*, \mathcal{F})  \\
      y_n \text{ converges to } -\infty \text{ in } (\R^*, \mathcal{F}) \\
    \end{dcases} \iff \begin{dcases}
      \liminf_{n \to \infty} x_n = \infty \\
      \limsup_{n \to \infty} y_n = -\infty
    \end{dcases}
  \]
  This is true since
  \begin{align*}
         & \begin{dcases}
             \liminf_{n \to \infty} x^{(n)} = \infty \\
             \limsup_{n \to \infty} y^{(n)} = -\infty
           \end{dcases}                                                                                      \\
    \iff & \begin{dcases}
             \sup\set{\inf\set{x^{(n)} : n \geq N} : N \geq 1} = \infty \\
             \inf\set{\sup\set{y^{(n)} : n \geq N} : N \geq 1} = -\infty
           \end{dcases}                                                                    \\
    \iff & \begin{dcases}
             \sup\set{x^{(n)} : n \geq 1} = \infty \\
             \inf\set{y^{(n)} : n \geq 1} = -\infty
           \end{dcases}                                                                 \\
    \iff & \begin{dcases}
             \forall \varepsilon \in \R^+, \exists N \geq 1 : \forall n \geq N, x^{(n)} > \varepsilon \\
             \forall \varepsilon \in \R^+, \exists N \geq 1 : \forall n \geq N, y^{(n)} < -\varepsilon
           \end{dcases}                                     \\
    \iff & \begin{dcases}
             \forall \varepsilon \in \R^+, \exists N \geq 1 : \forall n \geq N, x^{(n)} \in (\varepsilon, \infty] \\
             \forall \varepsilon \in \R^+, \exists N \geq 1 : \forall n \geq N, y^{(n)} \in [-\infty, -\varepsilon)
           \end{dcases} \\
    \iff & \begin{dcases}
             \forall V \in \mathcal{F}, \infty \in V \implies \exists \varepsilon \in \R^+ : \begin{dcases}
                                                                                        (\varepsilon, \infty] \subseteq V \\
                                                                                        \exists N \geq 1 : \forall n \geq N, x^{(n)} \in V
                                                                                      \end{dcases} \\
             \forall V \in \mathcal{F}, -\infty \in V \implies \exists \varepsilon \in \R^+ : \begin{dcases}
                                                                                         [-\infty, -\varepsilon) \subseteq V \\
                                                                                         \exists N \geq 1 : \forall n \geq N, y^{(n)} \in V
                                                                                       \end{dcases}
           \end{dcases}                   \\
    \iff & \begin{dcases}
             x_n \text{ converges to } \infty \text{ in } (\R^*, \mathcal{F})  \\
             y_n \text{ converges to } -\infty \text{ in } (\R^*, \mathcal{F}) \\
           \end{dcases}
  \end{align*}
\end{proof}

\begin{ex}\label{ii:ex:2.5.6}
  Let \(X\) be an uncountable set, and let \(\mathcal{F}\) be the collection of all subsets \(E\) in \(X\) which are either empty or co-finite (which means that \(X \setminus E\) is finite).
  Show that \((X, \mathcal{F})\) is a topology (this is called the \emph{co-finite topology} on \(X\)) which is not Hausdorff in the sense of \cref{ii:ex:2.5.4}, and is compact and connected.
  Also, show that if \(x \in X\) and \((V_n)_{n = 1}^\infty\) is any countable collection of open sets containing \(x\), then \(\bigcap_{n = 1}^\infty V_n \neq \set{x}\).
  Use this to show that the co-finite topology cannot be obtained by placing a metric \(d\) on \(X\).
\end{ex}

\begin{proof}
  We first show that \((X, \mathcal{F})\) is a topological space.
  By definition we have \(\emptyset \in \mathcal{F}\).
  Since \(X \setminus X = \emptyset\) is finite, we know that \(X\) is co-finite and \(X \in \mathcal{F}\).
  Let \(n \in \N\), let \(I_n = \set{i \in \N : 1 \leq i \leq n}\) and let \(A = \set{V_i \in \mathcal{F} : i \in I_n}\) be a finite collection of open sets in \((X, \mathcal{F})\).
  If \(\bigcap A = \emptyset\), then from the proof above we know that \(\emptyset \in \mathcal{F}\).
  So suppose that \(A \neq \emptyset\).
  Then we have
  \begin{align*}
             & \forall i \in I_n, V_i \text{ is co-finite}                     \\
    \implies & \forall i \in I_n, X \setminus V_i \text{ is finite}            \\
    \implies & \bigcup_{i = 1}^n (X \setminus V_i) \text{ is finite}           \\
    \implies & X \setminus \bigg(\bigcap_{i = 1}^n V_i\bigg) \text{ is finite} \\
    \implies & X \setminus \bigg(\bigcap A\bigg) \text{ is finite}             \\
    \implies & \bigcap A \text{ is co-finite}
  \end{align*}
  and thus \(\bigcap A \in \mathcal{F}\).
  Since \(n\) was arbitrary, we conclude that the intersection of any finite collection of open sets in \((X, \mathcal{F})\) is open in \((X, \mathcal{F})\).

  Let \(S \subseteq \mathcal{F}\).
  Then we have
  \begin{align*}
             & \forall V \in S, V \text{ is co-finite}             \\
    \implies & \forall V \in S, X \setminus V \text{ is finite}    \\
    \implies & \bigcap_{V \in S} (X \setminus V) \text{ is finite} \\
    \implies & X \setminus \bigg(\bigcup S\bigg) \text{ is finite} \\
    \implies & \bigcup S \text{ is co-finite}
  \end{align*}
  and thus \(\bigcup S \in \mathcal{F}\).
  Since \(S\) was arbitrary, we conclude that the union of arbitrary open sets in \((X, \mathcal{F})\) is open in \((X, \mathcal{F})\).
  Combine all the proofs above we conclude that \((X, \mathcal{F})\) is a topological space by \cref{ii:2.5.1}.

  Next we show that \((X, \mathcal{F})\) is not Hausdorff.
  Suppose for sake of contradiction that \((X, \mathcal{F})\) is Hausdorff.
  Let \(x_1, x_2 \in X\) such that \(x_1 \neq x_2\).
  By \cref{ii:ex:2.5.4} we know that
  \[
    \exists V_1, V_2 \in \mathcal{F} : \begin{dcases}
      x_1 \in V_1 \\
      x_2 \in V_2 \\
      V_1 \cap V_2 = \emptyset
    \end{dcases}
  \]
  But then we have
  \begin{align*}
             & V_1, V_2 \text{ are co-finite}                                                          \\
    \implies & X \setminus V_1, X \setminus V_2 \text{ are finite}                                     \\
    \implies & (X \setminus V_1) \cup (X \setminus V_2) \text{ is finite}                              \\
    \implies & X \setminus (V_1 \cap V_2) \text{ is finite}                                            \\
    \implies & X \text{ is finite},                                       & (V_1 \cap V_2 = \emptyset)
  \end{align*}
  a contradiction.
  Thus \((X, \mathcal{F})\) is not Hausdorff.

  Next we show that \((X, \mathcal{F})\) is compact.
  Let \(S\) be an open cover of \(X\) in \((X, \mathcal{F})\).
  Let \(V_0 \in S\).
  Since \(V_0\) is co-finite, we know that \(X \setminus V_0\) is finite.
  Let \(n = \#(X \setminus V_0)\), let \(I_n = \set{i \in \N : 1 \leq i \leq n}\) and let \(X \setminus V_0 = \set{x_i \in X : i \in I_n}\).
  Then we have
  \begin{align*}
             & \forall i \in I_n, x_i \in X                       \\
    \implies & \forall i \in I_n, \exists V_i \in S : x_i \in V_i \\
    \implies & X = V_0 \cup \bigg(\bigcup_{i \in I_n} V_i\bigg).
  \end{align*}
  Since \(S\) was arbitrary, by \cref{ii:2.5.9} we know that \((X, \mathcal{F})\) is compact.

  Next we show that \((X, \mathcal{F})\) is connected.
  Suppose for sake of contradiction that \((X, \mathcal{F})\) is disconnected.
  Then by \cref{ii:2.4.1} we have
  \[
    \exists V, W \in \mathcal{F} : \begin{dcases}
      V \neq \emptyset \neq W \\
      V \cup W = X            \\
      V \cap W = \emptyset
    \end{dcases}
  \]
  But then we have
  \begin{align*}
             & V, W \text{ are co-finite}                                                      \\
    \implies & X \setminus V, X \setminus W \text{ are finite}                                 \\
    \implies & (X \setminus V) \cup (X \setminus W) \text{ is finite}                          \\
    \implies & X \setminus (V \cap W) \text{ is finite}                                        \\
    \implies & X \text{ is finite},                                   & (V \cap W = \emptyset)
  \end{align*}
  a contradiction.
  Thus \((X, \mathcal{F})\) is connected.

  Next we show that if \(x \in X\) and \((V_n)_{n = 1}^\infty\) is any countable collection of open sets in \((X, \mathcal{F})\) such that \(x \in V_n\) for all \(n \in \Z^+\), then \(\bigcap_{n = 1}^\infty V_n \neq \set{x}\).
  This is true since
  \begin{align*}
             & \forall n \in \Z^+, V_n \text{ is co-finite}                                                                                                    \\
    \implies & \forall n \in \Z^+, X \setminus V_n \text{ is finite}                                                                                           \\
    \implies & \bigcup_{n = 1}^\infty (X \setminus V_n) \text{ is at most countable}                             &  & \text{(by Exercise 8.1.9 in Analysis I)} \\
    \implies & X \setminus \bigg(\bigcap_{n = 1}^\infty V_n\bigg) \text{ is at most countable}                                                                 \\
    \implies & X \setminus \Bigg(X \setminus \bigg(\bigcap_{n = 1}^\infty V_n\bigg)\Bigg) \text{ is uncountable} &  & \text{(\(X\) is uncountable)}            \\
    \implies & \bigcap_{n = 1}^\infty V_n \text{ is uncountable}                                                                                               \\
    \implies & \bigcap_{n = 1}^\infty V_n \neq \set{x}.
  \end{align*}

  Finally we show that \((X, \mathcal{F})\) cannot be obtained by placing a metric \(d\) on \(X\).
  Suppose for sake of contradiction that there exists some metric \(d\) such that \(\mathcal{F} = \set{V \subseteq X : V \text{ is open in } (X, d)}\).
  Let \(x \in X\) and for each \(n \in \Z^+\) let \(V_n = B_{(X, d)}(x, \dfrac{1}{n})\).
  By \cref{ii:1.2.15}(c) we know that \(V_n\) is open in \((X, d)\) for each \(n \in \Z^+\).
  From the proof above we must have \(\bigcap_{n = 1}^\infty V_n \neq \set{x}\).
  So let \(y \in \bigcap_{n = 1}^\infty V_n\) such that \(y \neq x\).
  By \cref{ii:1.1.2}(b) we know that \(d(y, x) \in \R^+\).
  But then we have
  \begin{align*}
             & \exists n \in \Z^+ : d(y, x) > \dfrac{1}{n}               &  & \text{(by Archimedean property)} \\
    \implies & \exists n \in \Z^+ : y \notin B_{(X, d)}(x, \dfrac{1}{n}) &  & \by{ii:1.2.1}                    \\
    \implies & \exists n \in \Z^+ : y \notin V_n                                                               \\
    \implies & y \notin \bigcap_{n = 1}^\infty V_n,
  \end{align*}
  a contradiction.
  Thus \((X, \mathcal{F})\) cannot be obtained by placing a metric \(d\) on \(X\).
\end{proof}

\begin{ex}\label{ii:ex:2.5.7}
  Let \(X\) be an uncountable set, and let \(\mathcal{F}\) be the collection of all subsets \(E\) in \(X\) which are either empty or co-countable
  (which means that \(X \setminus E\) is at most countable).
  Show that \((X, \mathcal{F})\) is a topology (this is called the \emph{co-countable topology} on \(X\)) which is not Hausdorff in the sense of \cref{ii:ex:2.5.4}, and connected, but cannot arise from a metric space and is not compact.
\end{ex}

\begin{proof}
  We first show that \((X, \mathcal{F})\) is a topological space.
  By definition we have \(\emptyset \in \mathcal{F}\).
  Since \(X \setminus X = \emptyset\) is finite, we know that \(X\) is co-countable and \(X \in \mathcal{F}\).
  Let \(n \in \N\), let \(I_n = \set{i \in \N : 1 \leq i \leq n}\) and let \(A = \set{V_i \in \mathcal{F} : i \in I_n}\) be a finite collection of open sets in \((X, \mathcal{F})\).
  If \(\bigcap A = \emptyset\), then from the proof above we know that \(\emptyset \in \mathcal{F}\).
  So suppose that \(A \neq \emptyset\).
  Then we have
  \begin{align*}
             & \forall i \in I_n, V_i \text{ is co-countable}                                                                           \\
    \implies & \forall i \in I_n, X \setminus V_i \text{ is at most countable}                                                          \\
    \implies & \bigcup_{i = 1}^n (X \setminus V_i) \text{ is at most countable}           &  & \text{(by Exercise 8.1.9 in Analysis I)} \\
    \implies & X \setminus \bigg(\bigcap_{i = 1}^n V_i\bigg) \text{ is at most countable}                                               \\
    \implies & X \setminus \bigg(\bigcap A\bigg) \text{ is at most countable}                                                           \\
    \implies & \bigcap A \text{ is co-countable}
  \end{align*}
  and thus \(\bigcap A \in \mathcal{F}\).
  Since \(n\) was arbitrary, we conclude that the intersection of any finite collection of open sets in \((X, \mathcal{F})\) is open in \((X, \mathcal{F})\).

  Let \(S \subseteq \mathcal{F}\).
  Then we have
  \begin{align*}
             & \forall V \in S, V \text{ is co-countable}                     \\
    \implies & \forall V \in S, X \setminus V \text{ is at most countable}    \\
    \implies & \bigcap_{V \in S} (X \setminus V) \text{ is at most countable} \\
    \implies & X \setminus \bigg(\bigcup S\bigg) \text{ is at most countable} \\
    \implies & \bigcup S \text{ is co-countable}
  \end{align*}
  and thus \(\bigcup S \in \mathcal{F}\).
  Since \(S\) was arbitrary, we conclude that the union of arbitrary open sets in \((X, \mathcal{F})\) is open in \((X, \mathcal{F})\).
  Combine all the proofs above we conclude that \((X, \mathcal{F})\) is a topological space by \cref{ii:2.5.1}.

  Next we show that \((X, \mathcal{F})\) is not Hausdorff.
  Suppose for sake of contradiction that \((X, \mathcal{F})\) is Hausdorff.
  Let \(x_1, x_2 \in X\) such that \(x_1 \neq x_2\).
  By \cref{ii:ex:2.5.4} we know that
  \[
    \exists V_1, V_2 \in \mathcal{F} : \begin{dcases}
      x_1 \in V_1 \\
      x_2 \in V_2 \\
      V_1 \cap V_2 = \emptyset
    \end{dcases}
  \]
  But then we have
  \begin{align*}
             & V_1, V_2 \text{ are co-countable}                                                                  \\
    \implies & X \setminus V_1, X \setminus V_2 \text{ are at most countable}                                     \\
    \implies & (X \setminus V_1) \cup (X \setminus V_2) \text{ is at most countable}                              \\
    \implies & X \setminus (V_1 \cap V_2) \text{ is at most countable}                                            \\
    \implies & X \text{ is at most countable},                                       & (V_1 \cap V_2 = \emptyset)
  \end{align*}
  a contradiction.
  Thus \((X, \mathcal{F})\) is not Hausdorff.

  Next we show that \((X, \mathcal{F})\) is connected.
  Suppose for sake of contradiction that \((X, \mathcal{F})\) is disconnected.
  Then by \cref{ii:2.4.1} we have
  \[
    \exists V, W \in \mathcal{F} : \begin{dcases}
      V \neq \emptyset \neq W \\
      V \cup W = X            \\
      V \cap W = \emptyset
    \end{dcases}
  \]
  But then we have
  \begin{align*}
             & V, W \text{ are co-countable}                                                              \\
    \implies & X \setminus V, X \setminus W \text{ are at most countable}                                 \\
    \implies & (X \setminus V) \cup (X \setminus W) \text{ is at most countable}                          \\
    \implies & X \setminus (V \cap W) \text{ is at most countable}                                        \\
    \implies & X \text{ is at most countable},                                   & (V \cap W = \emptyset)
  \end{align*}
  a contradiction.
  Thus \((X, \mathcal{F})\) is connected.

  Next we show that if \(x \in X\) and \((V_n)_{n = 1}^\infty\) is any countable collection of open sets in \((X, \mathcal{F})\) such that \(x \in V_n\) for all \(n \in \Z^+\), then \(\bigcap_{n = 1}^\infty V_n \neq \set{x}\).
  This is true since
  \begin{align*}
             & \forall n \in \Z^+, V_n \text{ is co-countable}                                                                                                 \\
    \implies & \forall n \in \Z^+, X \setminus V_n \text{ is at most countable}                                                                                \\
    \implies & \bigcup_{n = 1}^\infty (X \setminus V_n) \text{ is at most countable}                             &  & \text{(by Exercise 8.1.9 in Analysis I)} \\
    \implies & X \setminus \bigg(\bigcap_{n = 1}^\infty V_n\bigg) \text{ is at most countable}                                                                 \\
    \implies & X \setminus \Bigg(X \setminus \bigg(\bigcap_{n = 1}^\infty V_n\bigg)\Bigg) \text{ is uncountable} &  & \text{(\(X\) is uncountable)}            \\
    \implies & \bigcap_{n = 1}^\infty V_n \text{ is uncountable}                                                                                               \\
    \implies & \bigcap_{n = 1}^\infty V_n \neq \set{x}.
  \end{align*}

  Next we show that \((X, \mathcal{F})\) cannot be obtained by placing a metric \(d\) on \(X\).
  Suppose for sake of contradiction that there exists some metric \(d\) such that \(\mathcal{F} = \set{V \subseteq X : V \text{ is open in } (X, d)}\).
  Let \(x \in X\) and for each \(n \in \Z^+\) let \(V_n = B_{(X, d)}(x, \dfrac{1}{n})\).
  By \cref{ii:1.2.15}(c) we know that \(V_n\) is open in \((X, d)\) for each \(n \in \Z^+\).
  From the proof above we must have \(\bigcap_{n = 1}^\infty V_n \neq \set{x}\).
  So let \(y \in \bigcap_{n = 1}^\infty V_n\) such that \(y \neq x\).
  By \cref{ii:1.1.2}(b) we know that \(d(y, x) \in \R^+\).
  But then we have
  \begin{align*}
             & \exists n \in \Z^+ : d(y, x) > \dfrac{1}{n}               &  & \text{(by Archimedean property)} \\
    \implies & \exists n \in \Z^+ : y \notin B_{(X, d)}(x, \dfrac{1}{n}) &  & \by{ii:1.2.1}                    \\
    \implies & \exists n \in \Z^+ : y \notin V_n                                                               \\
    \implies & y \notin \bigcap_{n = 1}^\infty V_n,
  \end{align*}
  a contradiction.
  Thus \((X, \mathcal{F})\) cannot be obtained by placing a metric \(d\) on \(X\).

  Finally we show that \((X, \mathcal{F})\) is not compact.
  Let \((x^{(n)})_{n = 1}^\infty\) be a countable collection of elements in \(X\).
  For each \(n \in \Z^+\), we define \(E_n = (X \setminus \set{x^{(i)} : i \in \Z^+}) \cup \set{x^{(n)}}\).
  Since
  \begin{align*}
             & \forall n \in \Z^+, X \setminus E_n = \set{x^{(i)} : i \in \N} \setminus \set{x^{(n)}} \text{ is countable} \\
    \implies & \forall n \in \Z^+, E_n \in \mathcal{F}
  \end{align*}
  and
  \begin{align*}
    \bigcup_{n = 1}^\infty E_n & = \bigcup_{n = 1}^\infty \Big(\big(X \setminus \set{x^{(i)} : i \in \Z^+}\big) \cup \set{x^{(n)}}\Big)                                        \\
                               & = \bigg(\bigcup_{n = 1}^\infty \big(X \setminus \set{x^{(i)} : i \in \Z^+}\big)\bigg) \cup \bigg(\bigcup_{n = 1}^\infty \set{x^{(n)}}\bigg)   \\
                               & = \Bigg(X \setminus \bigg(\bigcap_{n = 1}^\infty \set{x^{(i)} : i \in \Z^+}\bigg)\Bigg) \cup \bigg(\bigcup_{n = 1}^\infty \set{x^{(n)}}\bigg) \\
                               & = \big(X \setminus \set{x^{(i)} : i \in \Z^+}\big) \cup \bigg(\bigcup_{n = 1}^\infty \set{x^{(n)}}\bigg)                                      \\
                               & = X,
  \end{align*}
  we know that \(\bigcup_{n = 1}^\infty E_n\) is an open cover of \(X\) in \((X, \mathcal{F})\).
  Let \((E_{n_i})_{i = 1}^k\) be a finite subset of \((E_n)_{n = 1}^\infty\).
  Then we have
  \begin{align*}
             & \forall 1 \leq i \leq k, x_{n_i} \in \bigcup_{j = 1}^k E_{n_j}                             \\
    \implies & \forall 1 \leq i \leq k, x_{n_i} \notin X \setminus \bigg(\bigcup_{i = 1}^k E_{n_j}\bigg).
  \end{align*}
  Since \((E_{n_i})_{i = 1}^k\) was arbitrary, we know that every finite subset of \((E_n)_{n = 1}^\infty\) cannot cover \(X\) in \((X, \mathcal{F})\).
  Thus by \cref{ii:2.5.9} \((X, \mathcal{F})\) is not compact.
\end{proof}

\setcounter{ex}{8}
\begin{ex}\label{ii:ex:2.5.9}
  Let \((X, \mathcal{F})\) be a compact topological space.
  Assume that this space is \emph{first countable}, which means that for every \(x \in X\) there exists a countable collection \(V_1 , V_2 , \dots\) of neighbourhoods of \(x\), such that every neighbourhood of \(x\) contains one of the \(V_n\).
  Show that every sequence in \(X\) has a convergent subsequence, by modifying \cref{ii:ex:1.5.11}.
\end{ex}

\begin{proof}
  If \(X = \emptyset\), then the statement is trivial.
  So suppose that \(X \neq \emptyset\).
  Let \((x^{(n)})_{n = 1}^\infty\) be a sequence in \(X\).
  If the set \(E = \set{x^{(n)} : n \in \Z^+}\) is finite, then there exists a subsequence \((x^{(n_j)})_{j = 1}^\infty\) such that \(x^{(n_j)} = x^{(n_1)}\) for every \(j \in \Z^+\).
  By \cref{ii:2.5.4} we know that \((x^{(n_j)})_{j = 1}^\infty\) converges to \(x\).

  Now suppose that \(E\) is infinite.
  We claim that
  \[
    \exists y \in X : \forall W \in \mathcal{F}, y \in W \implies W \cap E \text{ is infinite}.
  \]
  Suppose for sake of contradiction that the claim is false.
  Then we have
  \[
    \forall y \in X, \exists W \in \mathcal{F} : \begin{dcases}
      y \in W; \\
      W \cap E \text{ is finite}.
    \end{dcases}
  \]
  We choose one such \(W\) for each \(y \in X\) and denote it as \(W_y\).
  But then we have
  \begin{align*}
             & X = \bigcup_{y \in X} W_y                                                   \\
    \implies & \exists Y \subseteq X : \begin{dcases}
                                         Y \text{ is finite} \\
                                         X = \bigcup_{y \in Y} W_y
                                       \end{dcases}               &  & \by{ii:2.5.9}       \\
    \implies & \exists Y \subseteq X : \begin{dcases}
                                         Y \text{ is finite} \\
                                         E = \bigcup_{y \in Y} (W_y \cap E) \text{ is finite}
                                       \end{dcases}
  \end{align*}
  which contradict to the fact that \(E\) is infinite.
  Thus the claim is true and we can choose one \(y \in X\) such that every neighbourhood of \(y\) contains infinitely many elements of \((x^{(n)})_{n = 1}^\infty\).
  Since \((X, \mathcal{F})\) is first countable, we know that there exists a countable collection \((V_j)_{j = 1}^\infty\) of neighbourhoods of \(y\) such that
  \[
    \forall W \in \mathcal{F}, y \in W \implies \exists j \in \Z^+ : V_j \subseteq W.
  \]
  We fix such \((V_j)_{j = 1}^\infty\) and define \(U_j = \bigcap_{i = 1}^j V_j\) for each \(j \in \Z^+\).
  Since \(y \in V_j\) for every \(j \in \Z^+\), we know that \(y \in \bigcap_{j = 1}^\infty V_j\).
  Thus we have \(y \in U_j\) and \(U_j \neq \emptyset\) for every \(j \in \Z^+\).
  Observe that
  \[
    \forall p_1, p_2 \in \Z^+, p_1 < p_2 \implies \bigcap_{i = 1}^{p_2} V_i \subseteq \bigcap_{i = 1}^{p_1} V_i \implies U_{p_2} \subseteq U_{p_1}.
  \]
  Now we construct a subsequence \((x^{(n_j)})_{j = 1}^\infty\) which converges to \(y\).
  Let
  \[
    A_1 = \set{n \in \Z^+ : x^{(n)} \in U_1}.
  \]
  Since \(U_1 = V_1\) is a neighbourhood of \(y\), by the definition of \(y\) we know that \(A_1\) is infinite.
  Since \(A_1 \subseteq \Z^+\), by well-ordering principle we know that \(\min(A_1)\) is well-defined.
  Let \(n_1 = \min(A_1)\).
  Suppose that \(n_j\) is already defined for some \(j \geq 1\).
  Then we define \(n_{j + 1}\) as follow:
  \begin{align*}
     & A_{j + 1} = \set{n \in \Z^+ : (n > n_j) \land (x^{(n)} \in U_{j + 1})} \\
     & n_{j + 1} = \min(A_{j + 1})
  \end{align*}
  Since \(U_{j + 1} = \bigcap_{i = 1}^{j + 1} V_i\), by \cref{ii:2.5.1} we know that \(U_{j + 1}\) is a neighbourhood of \(y\).
  Thus by the definition of \(y\) we know that \(A_{j + 1}\) is infinite.
  Since \(A_{j + 1} \subseteq \Z^+\), by well-ordering principle we know that \(n_{j + 1}\) is well-defined.
  Thus we have construct a subsequence \((x^{(n_j)})_{j = 1}^\infty\).
  Let \(W\) be a neighbourhood of \(y\).
  Since \((X, \mathcal{F})\) is first countable, we know that
  \begin{align*}
             & \exists N \in \Z^+ : V_N \subseteq W                                                                                  \\
    \implies & \exists N \in \Z^+ : U_N \subseteq V_N \subseteq W                   &  & \text{(by the definition of \(U_N\))}       \\
    \implies & \exists N \in \Z^+ : \forall j \geq N, U_j \subseteq U_N \subseteq W &  & \text{(by the definition of \(U_N\))}       \\
    \implies & \exists N \in \Z^+ : \forall j \geq N, x^{(n_j)} \in W.              &  & \text{(by the definition of \(x^{(n_j)}\))}
  \end{align*}
  Since \(W\) was arbitrary, by \cref{ii:2.5.4} we know that \((x^{(n_j)})_{j = 1}^\infty\) converges to \(y\) in \((X, \mathcal{F})\).
  Thus we have found a subsequence of \((x^{(n)})_{n = 1}^\infty\) which converges in \((X, \mathcal{F})\).
  Since \((x^{(n)})_{n = 1}^\infty\) was arbitrary, we conclude that \((X, \mathcal{F})\) is sequentially compact, i.e., every sequence in \(X\) has a convergent subsequence.
\end{proof}

\begin{ex}\label{ii:ex:2.5.10}
  Prove the following partial analogue of \cref{ii:1.2.10} for topological spaces:
  (c) implies both (a) and (b), which are equivalent to each other.
  Show that in the co-countable topology in \cref{ii:ex:2.5.7}, it is possible for (a) and (b) to hold without (c) holding.
\end{ex}

\begin{proof}
  Let \((X, \mathcal{F})\) be a topological space, let \(E \subseteq X\), let \(x_0 \in X\).
  We first show that if there exists a sequence \((x^{(n)})_{n = 1}^\infty\) in \(E\) which converges to \(x_0\) in \((X, \mathcal{F})\), then \(x_0\) is an adherent point of \(E\) in \((X, \mathcal{F})\).
  This is true since
  \begin{align*}
             & (x^{(n)})_{n = 1}^\infty \text{ converges to } x_0 \text{ in } (X, \mathcal{F})                                                 \\
    \implies & \big(\forall V \in \mathcal{F}, x_0 \in V \implies \exists N \in \Z^+ : \forall n \geq N, x^{(n)} \in V\big) &  & \by{ii:2.5.4} \\
    \implies & \big(\forall V \in \mathcal{F}, x_0 \in V \implies \exists N \in \Z^+ : x^{(N)} \in V \cap E\big)                               \\
    \implies & \big(\forall V \in \mathcal{F}, x_0 \in V \implies V \cap E \neq \emptyset\big)                                                 \\
    \implies & x_0 \in \overline{E}_{(X, \mathcal{F})}.                                                                     &  & \by{ii:2.5.6}
  \end{align*}

  Next we show that \(x_0\) is an adherent point of \(E\) in \((X, \mathcal{F})\) iff \(x_0\) is an interior point or a boundary point of \(E\) in \((X, \mathcal{F})\).
  This is true since
  \begin{align*}
         & x_0 \in \overline{E}_{(X, \mathcal{F})}                                                                                      \\
    \iff & \forall V \in \mathcal{F}, x_0 \in V \implies V \cap E \neq \emptyset                                     &  & \by{ii:2.5.6} \\
    \iff & x_0 \notin \text{ext}_{(X, \mathcal{F})}(E)                                                               &  & \by{ii:2.5.5} \\
    \iff & \big(x_0 \in \text{int}_{(X, \mathcal{F})}(E)\big) \lor \big(x_0 \in \partial_{(X, \mathcal{F})}(E)\big). &  & \by{ii:2.5.5}
  \end{align*}

  Finally we show that if \(X\) is uncountable, \((X, \mathcal{F})\) is a co-countable topology and \(x_0 \in E\) for some \(E \in \mathcal{F}\), then there may not exist a sequence in \(E\) which coverges to \(x_0\) in \((X, \mathcal{F})\).
  Let \(C \subseteq X\) such that \(C\) is countable and let \(E = X \setminus C\).
  Then by \cref{ii:ex:2.5.7} we know that \(E \in \mathcal{F}\) and \(E \neq \emptyset\).

  Let \(x_0 \in C\).
  Since \(x_0 \in C\), we know that \(x_0 \notin E\) and by \cref{ii:2.5.5} we know that \(x_0 \notin \text{int}_{(X, \mathcal{F})}(E)\).
  This means \(x_0 \in \text{ext}_{(X, \mathcal{F})}(E)\) or \(x_0 \in \partial_{(X, \mathcal{F})}(E)\).
  Now we claim that \(x_0 \in \partial_{(X, \mathcal{F})}(E)\).
  Suppose for sake of contradiction that the claim is false.
  Then we have \(x_0 \in \text{ext}_{(X, \mathcal{F})}(E)\), i.e.,
  \[
    \exists V \in \mathcal{F} : x_0 \in V \implies V \cap E = \emptyset.
  \]
  Fix this \(V\).
  Since \((X, \mathcal{F})\) is a co-countable, by \cref{ii:ex:2.5.7} we know that \(X \setminus V\) is at most countable.
  But then we have
  \begin{align*}
             & (X \setminus V) \cup C \text{ is countable}                                                   \\
    \implies & X \setminus \big((X \setminus V) \cup C\big) \text{ is co-countable}                          \\
    \implies & V \cap (X \setminus C) \text{ is co-countable}                                                \\
    \implies & V \cap E \text{ is co-countable}                                                              \\
    \implies & X \setminus (V \cap E) \text{ is at most countable}                                           \\
    \implies & X \text{ is at most countable},                                      & (V \cap E = \emptyset)
  \end{align*}
  which contradict to the hypothesis that \(X\) is uncountable.
  Thus the claim is true.
  From the proof above we know that \(x_0 \in \partial_{(X, \mathcal{F})}(E)\) implies \(x_0 \in \overline{E}_{(X, \mathcal{F})}\).

  Suppose for sake of contradiction that there exists a sequence \((x^{(n)})_{n = 1}^\infty\) in \(E\) which converges to \(x_0\) in \((X, \mathcal{F})\).
  Then we have
  \begin{align*}
             & \set{x^{(n)} : n \in \Z^+} \text{ is countable}                \\
    \implies & X \setminus \set{x^{(n)} : n \in \Z^+} \text{ is co-countable} \\
    \implies & X \setminus \set{x^{(n)} : n \in \Z^+} \in \mathcal{F}.
  \end{align*}
  Since \(x_0 \notin E\), we have
  \begin{align*}
             & x_0 \notin \set{x^{(n)} : n \in \Z^+}           \\
    \implies & x_0 \in X \setminus \set{x^{(n)} : n \in \Z^+}.
  \end{align*}
  But \(X \setminus \set{x^{(n)} : n \in \Z^+} \in \mathcal{F}\) means we have found one neighbourhood of \(x_0\) which does not contain any elements of \((x^{(n)})_{n = 1}^\infty\).
  By \cref{ii:2.5.4} this means \((x^{(n)})_{n = 1}^\infty\) does not coverges to \(x_0\) in \((X, \mathcal{F})\), a contradiction.
  Thus we conclude that there does not exist a sequence in \(E\) which converges to \(x_0\) when \(x_0\) is an adherent point of \(E\) in \((X, \mathcal{F})\).
\end{proof}

\begin{ex}\label{ii:ex:2.5.11}
  Let \(E\) be a subset of a topological space \((X, \mathcal{F})\).
  Show that \(E\) is open iff every element of \(E\) is an interior point, and show that \(E\) is closed iff \(E\) contains all of its adherent points.
  Prove analogues of \cref{ii:1.2.15}(e)-(h) (some of these are automatic by definition).
  If we assume in addition that \(X\) is Hausdorff, prove an analogue of \cref{ii:1.2.15}(d) also, but give an example to show that (d) can fail when \(X\) is not Hausdorff.
\end{ex}

\begin{proof}
  We first show that \(E\) is open in \((X, \mathcal{F})\) iff \(E = \text{int}_{(X, \mathcal{F})}(E)\).
  \begin{align*}
         & E \text{ is open in } (X, \mathcal{F})                                                                      \\
    \iff & E \in \mathcal{F}                                             &                             & \by{ii:2.5.1} \\
    \iff & \forall x \in E, \exists V_x \in \mathcal{F} : \begin{dcases}
                                                            x \in V_x \\
                                                            V_x \subseteq E
                                                          \end{dcases} & (\bigcup_{x \in E} V_x = E)                   \\
    \iff & \forall x \in E, x \in \text{int}_{(X, \mathcal{F})}          &                             & \by{ii:2.5.5} \\
    \iff & E = \text{int}_{(X, \mathcal{F})}.                            &                             & \by{ii:2.5.5}
  \end{align*}

  Next we show that \(E\) is closed in \((X, \mathcal{F})\) iff \(E = \overline{E}_{(X, \mathcal{F})}\).
  \begin{align*}
         & E \text{ is closed in } (X, \mathcal{F})                                                                                        \\
    \iff & X \setminus E \text{ is open in } (X, \mathcal{F})                                                                              \\
    \iff & X \setminus E = \text{int}_{(X, \mathcal{F})}(X \setminus E)                                 &  & \text{(from the proof above)} \\
    \iff & \forall x \in X \setminus E, \exists V \in \mathcal{F} : \begin{dcases}
                                                                      x \in V \\
                                                                      V \subseteq X \setminus E
                                                                    \end{dcases}                      &  & \by{ii:2.5.5}                   \\
    \iff & \forall x \in X \setminus E, \exists V \in \mathcal{F} : \begin{dcases}
                                                                      x \in V \\
                                                                      V \cap E = \emptyset
                                                                    \end{dcases}                                                    \\
    \iff & \forall x \in X \setminus E, x \in \text{ext}_{(X, \mathcal{F})}(E)                          &  & \by{ii:2.5.5}                 \\
    \iff & X \setminus E = \text{ext}_{(X, \mathcal{F})}(E)                                             &  & \by{ii:2.5.5}                 \\
    \iff & E = \big(\text{int}_{(X, \mathcal{F})}(E)\big) \cup \big(\partial_{(X, \mathcal{F})}(E)\big) &  & \by{ii:2.5.5}                 \\
    \iff & E = \overline{E}_{(X, \mathcal{F})}.                                                         &  & \by{ii:ex:2.5.10}
  \end{align*}

  Next we show that \(E\) is open in \((X, \mathcal{F})\) iff \(X \setminus E\) is closed in \((X, \mathcal{F})\).
  This is true by definition.

  Next we show that if \(\set{E_1, \dots, E_n}\) is a finite collection of open sets in \((X, \mathcal{F})\), then \(\bigcap_{i = 1}^n E_i\) is open in \((X, \mathcal{F})\).
  This is true by \cref{ii:2.5.1}.

  Next we show that if \(\set{E_1, \dots, E_n}\) is a finite collection of closed sets in \((X, \mathcal{F})\), then \(\bigcup_{i = 1}^n E_i\) is closed in \((X, \mathcal{F})\).
  \begin{align*}
             & \forall 1 \leq i \leq n, E_i \text{ is closed in } (X, \mathcal{F})                                   \\
    \implies & \forall 1 \leq i \leq n, X \setminus E_i \text{ is open in } (X, \mathcal{F})                         \\
    \implies & \bigcap_{i = 1}^n (X \setminus E_i) \text{ is open in } (X, \mathcal{F})           &  & \by{ii:2.5.1} \\
    \implies & X \setminus \bigg(\bigcup_{i = 1}^n E_i\bigg) \text{ is open in } (X, \mathcal{F})                    \\
    \implies & \bigcup_{i = 1}^n E_i \text{ is closed in } (X, \mathcal{F}).
  \end{align*}

  Next we show that if \((E_\alpha)_{\alpha \in I}\) is a collection of open sets in \((X, \mathcal{F})\) where \(I\) is some index set, then \(\bigcup_{\alpha \in I} E_\alpha\) is open in \((X, \mathcal{F})\).
  This is true by \cref{ii:2.5.1}.

  Next we show that if \((E_\alpha)_{\alpha \in I}\) is a collection of closed sets in \((X, \mathcal{F})\) where \(I\) is some index set, then \(\bigcap_{\alpha \in I} E_\alpha\) is closed in \((X, \mathcal{F})\).
  \begin{align*}
             & \forall \alpha \in I, E_\alpha \text{ is closed in } (X, \mathcal{F})                                           \\
    \implies & \forall \alpha \in I, X \setminus E_\alpha \text{ is open in } (X, \mathcal{F})                                 \\
    \implies & \bigcup_{\alpha \in I} (X \setminus E_\alpha) \text{ is open in } (X, \mathcal{F})           &  & \by{ii:2.5.1} \\
    \implies & X \setminus \bigg(\bigcap_{\alpha \in I} E_\alpha\bigg) \text{ is open in } (X, \mathcal{F})                    \\
    \implies & \bigcap_{\alpha \in I} E_\alpha \text{ is closed in } (X, \mathcal{F}).
  \end{align*}

  Next we show that if \(E \subseteq X\), then \(\text{int}_{(X, \mathcal{F})}(E)\) is open in \((X, \mathcal{F})\).
  Suppose for sake of contradiction that \(\text{int}_{(X, \mathcal{F})}(E) \notin \mathcal{F}\).
  Then by \cref{ii:2.5.5} we know that
  \[
    \exists x \in \text{int}_{(X, \mathcal{F})}(E) : \forall V \in \mathcal{F}, x \in V \implies V \not\subseteq \text{int}_{(X, \mathcal{F})}(E).
  \]
  Fix this \(x\).
  By \cref{ii:2.5.5} we know that \(\text{int}_{(X, \mathcal{F})}(E) \subseteq E\).
  But then we have
  \[
    \forall V \in \mathcal{F}, x \in V \implies V \not\subseteq \text{int}_{(X, \mathcal{F})}(E) \subseteq E,
  \]
  which means \(x \notin \text{int}_{(X, \mathcal{F})}(E)\) by \cref{ii:2.5.5}, a contradiction.
  Thus we must have
  \[
    \forall x \in \text{int}_{(X, \mathcal{F})}(E), \exists V \in \mathcal{F} : \begin{dcases}
      x \in V \\
      V \subseteq \text{int}_{(X, \mathcal{F})}(E)
    \end{dcases}
  \]
  and by \cref{ii:2.5.5} we have \(\text{int}_{(X, \mathcal{F})}\big(\text{int}_{(X, \mathcal{F})}(E)\big) = \text{int}_{(X, \mathcal{F})}(E)\).
  From the proof above we conclude that \(\text{int}_{(X, \mathcal{F})}(E)\) is open in \((X, \mathcal{F})\).

  Next we show that if \(E \subseteq X\), then \(\text{int}_{(X, \mathcal{F})}(E)\) is the largest open set in \((X, \mathcal{F})\) which is contained in \(E\).
  Let \(V \in \mathcal{F}\) such that \(V \subseteq E\).
  Then we have
  \begin{align*}
             & \forall x \in V, x \in E                                                   \\
    \implies & \forall x \in V, x \in \text{int}_{(X, \mathcal{F})}(E) &  & \by{ii:2.5.5} \\
    \implies & V \subseteq \text{int}_{(X, \mathcal{F})}(E).
  \end{align*}
  Since \(V\) was arbitrary, we conclude that \(\text{int}_{(X, \mathcal{F})}(E)\) is the largest open set in \((X, \mathcal{F})\) which is contained in \(E\).

  Next we show that if \(E \subseteq X\), then \(\partial_{(X, \mathcal{F})}(E) = \partial_{(X, \mathcal{F})}(X \setminus E)\).
  We have
  \begin{align*}
         & x \in \partial_{(X, \mathcal{F})}(E)                                              \\
    \iff & \forall V \in \mathcal{F}, x \in V \implies \begin{dcases}
                                                         V \cap E \neq \emptyset \\
                                                         V \not\subseteq E
                                                       \end{dcases} &  & \by{ii:2.5.5}       \\
    \iff & \forall V \in \mathcal{F}, x \in V \implies \begin{dcases}
                                                         V \cap (X \setminus E) \neq \emptyset \\
                                                         V \not\subseteq (X \setminus E)
                                                       \end{dcases} \\
    \iff & x \in \partial_{(X, \mathcal{F})}(X \setminus E).          &  & \by{ii:2.5.5}
  \end{align*}
  Thus \(\partial_{(X, \mathcal{F})}(E) = \partial_{(X, \mathcal{F})}(X \setminus E)\).

  Next we show that if \(E \subseteq X\), then \(\text{int}_{(X, \mathcal{F})}(E) = \text{ext}_{(X, \mathcal{F})}(X \setminus E)\).
  We have
  \begin{align*}
         & x \in \text{int}_{(X, \mathcal{F})}(E)                                 \\
    \iff & \exists V \in \mathcal{F} : \begin{dcases}
                                         x \in V \\
                                         V \subseteq E
                                       \end{dcases}          &  & \by{ii:2.5.5}   \\
    \iff & \exists V \in \mathcal{F} : \begin{dcases}
                                         x \in V \\
                                         V \cap (X \setminus E) = \emptyset
                                       \end{dcases}          \\
    \iff & x \in \text{ext}_{(X, \mathcal{F})}(X \setminus E). &  & \by{ii:2.5.5}
  \end{align*}
  Thus \(\text{int}_{(X, \mathcal{F})}(E) = \text{ext}_{(X, \mathcal{F})}(X \setminus E)\).

  Next we show that if \(E \subseteq X\), then \(\overline{E}_{(X, \mathcal{F})}\) is closed in \((X, \mathcal{F})\).
  \begin{align*}
         & \text{int}_{(X, \mathcal{F})}(X \setminus E) \text{ is open in } (X, \mathcal{F})                                                                       &  & \text{(from the proof above)} \\
    \iff & X \setminus \text{int}_{(X, \mathcal{F})}(X \setminus E) \text{ is closed in } (X, \mathcal{F})                                                                                            \\
    \iff & \big(\text{ext}_{(X, \mathcal{F})}(X \setminus E)\big) \cup \big(\partial_{(X, \mathcal{F})}(X \setminus E)\big) \text{ is closed in } (X, \mathcal{F}) &  & \by{ii:2.5.5}                 \\
    \iff & \big(\text{int}_{(X, \mathcal{F})}(E)\big) \cup \big(\partial_{(X, \mathcal{F})}(E)\big) \text{ is closed in } (X, \mathcal{F})                         &  & \text{(from the proof above)} \\
    \iff & \overline{E}_{(X, \mathcal{F})}(E) \text{ is closed in } (X, \mathcal{F}).                                                                              &  & \by{ii:ex:2.5.10}
  \end{align*}

  Next we show that if \(E \subseteq X\), then \(\overline{E}_{(X, \mathcal{F})}\) is the smallest closed set in \((X, \mathcal{F})\) which contains \(E\).
  Let \(V \in \mathcal{F}\) such that \(E \subseteq X \setminus V\).
  Then we have
  \begin{align*}
             & E \subseteq (X \setminus V)                                                   \\
    \implies & E \cap V = \emptyset                                                          \\
    \implies & \overline{E}_{(X, \mathcal{F})} \cap V = \emptyset         &  & \by{ii:2.5.6} \\
    \implies & \overline{E}_{(X, \mathcal{F})} \subseteq (X \setminus V).
  \end{align*}
  Since \(X \setminus V\) was arbitrary, we know that \(\overline{E}_{(X, \mathcal{F})}\) is the smallest closed set in \((X, \mathcal{F})\) which contains \(E\).

  Next we show that if \((X, \mathcal{F})\) is a Hausdorff space, then \(\set{x_0}\) is closed in \((X, \mathcal{F})\) for any \(x_0 \in X\).
  Let \(x_0 \in X\).
  We have
  \begin{align*}
             & \forall y \in X \setminus \set{x_0}, y \neq x_0                                                                                    \\
    \implies & \forall y \in X \setminus \set{x_0}, \exists V, W \in \mathcal{F} : \begin{dcases}
                                                                                     V \neq \emptyset \neq W \\
                                                                                     x_0 \in V               \\
                                                                                     y \in W                 \\
                                                                                     V \cap W = \emptyset
                                                                                   \end{dcases}              &  & \by{ii:ex:2.5.4}                \\
    \implies & \forall y \in X \setminus \set{x_0}, \exists W \in \mathcal{F} : \begin{dcases}
                                                                                  y \in W \\
                                                                                  W \subseteq (X \setminus \set{x_0})
                                                                                \end{dcases}                                \\
    \implies & \forall y \in X \setminus \set{x_0}, y \in \text{int}_{(X, \mathcal{F})}(X \setminus \set{x_0}) &  & \by{ii:2.5.5}                 \\
    \implies & X \setminus \set{x_0} = \text{int}_{(X, \mathcal{F})}(X \setminus \set{x_0})                    &  & \by{ii:2.5.5}                 \\
    \implies & X \setminus \set{x_0} \text{ is open in } (X, \mathcal{F})                                      &  & \text{(from the proof above)} \\
    \implies & \set{x_0} \text{ is closed in } (X, \mathcal{F}).
  \end{align*}
  Since \(x_0\) was arbitrary, we conclude that \(\set{x_0}\) is closed in \((X, \mathcal{F})\) for any \(x_0 \in X\).

  Finally we give an counterexample of \cref{ii:1.2.15}(d) when \((X, \mathcal{F})\) is not Hausdorff.
  Let \(X \neq \emptyset\) and let \((X, \mathcal{F})\) be a trivial topology.
  Then by \cref{ii:ex:2.5.1} we know that \(\mathcal{F} = \set{\emptyset, X}\) and by \cref{ii:ex:2.5.4} \((X, \mathcal{F})\) is not Hausdorff.
  For any \(x_0 \in X\), we have \(X \setminus \set{x_0} \notin \mathcal{F}\), thus \(X \setminus \set{x_0}\) is not open in \((X, \mathcal{F})\) and \(\set{x_0}\) is not closed in \((X, \mathcal{F})\).
\end{proof}

\begin{ex}\label{ii:ex:2.5.12}
  Show that the pair \((Y, \mathcal{F}_Y)\) defined in \cref{ii:2.5.7} is indeed a topological space.
\end{ex}

\begin{proof}
  We have
  \begin{align*}
             & \begin{dcases}
                 X \in \mathcal{F} \\
                 \emptyset \in \mathcal{F}
               \end{dcases}                      &  & \by{ii:2.5.1} \\
    \implies & \begin{dcases}
                 Y \cap X = Y \in \mathcal{F}_Y \\
                 Y \cap \emptyset = \emptyset \in \mathcal{F}_Y
               \end{dcases}
  \end{align*}
  Let \(n \in \N\) and let \((V_Y^{(i)})_{i = 1}^n\) be a finite collection of open sets in \(\mathcal{F}_Y\).
  Then we have
  \begin{align*}
             & \forall 1 \leq i \leq n, \exists V_X^{(i)} \in \mathcal{F} : V_X^{(i)} \cap Y = V_Y^{(i)} &  & \by{ii:2.5.7} \\
    \implies & \bigcap_{i = 1}^n V_X^{(i)} \in \mathcal{F}                                               &  & \by{ii:2.5.1} \\
    \implies & Y \cap \bigg(\bigcap_{i = 1}^n V_X^{(i)}\bigg) \in \mathcal{F}_Y                          &  & \by{ii:2.5.7} \\
    \implies & \bigcap_{i = 1}^n (V_X^{(i)} \cap Y) = \bigcap_{i = 1}^n V_Y^{(i)} \in \mathcal{F}_Y.
  \end{align*}
  Since \(n\) was arbitrary, we conclude that the intersection of any finite collection of open sets in \((Y, \mathcal{F}_Y)\) is open in \((Y, \mathcal{F}_Y)\).
  Let \(S \subseteq \mathcal{F}_Y\).
  Then we have
  \begin{align*}
             & \forall V_Y \in S, \exists V_X \in \mathcal{F} : V_X \cap Y = V_Y                                          &  & \by{ii:2.5.7} \\
    \implies & \bigcup_{V_X \in \mathcal{F} : V_X \cap Y \in S} V_X \in \mathcal{F}                                       &  & \by{ii:2.5.1} \\
    \implies & Y \cap \bigg(\bigcup_{V_X \in \mathcal{F} : V_X \cap Y \in S} V_X\bigg) \in \mathcal{F}_Y                  &  & \by{ii:2.5.7} \\
    \implies & \bigcup_{V_X \in \mathcal{F} : V_X \cap Y \in S} (V_X \cap Y) = \bigcup_{V_Y \in S} V_Y \in \mathcal{F}_Y.
  \end{align*}
  Since \(S\) was arbitrary, we conclude that the union of arbitrary many open sets in \((Y, \mathcal{F}_Y)\) is open in \((Y, \mathcal{F}_Y)\).
  Combine all the proofs above we conclude by \cref{ii:2.5.1} that \((Y, \mathcal{F}_Y)\) is a topological space.
\end{proof}

\begin{ex}\label{ii:ex:2.5.13}
  Generalize \cref{ii:1.5.9} to compact sets in a Hausdorff topological space.
\end{ex}

\begin{proof}
  Let \((X, \mathcal{F})\) be a Hausdorff space, and let \((K_n)_{n = 1}^\infty\) be a countable collection of non-empty compact topological subspaces of \(X\) such that
  \[
    K_1 \supseteq K_2 \supseteq K_3 \supseteq \dots.
  \]
  We want to show that the intersection \(\bigcap_{n = 1}^\infty K_n\) is non-empty.

  Since \(K_n \subseteq K_1\) for each \(n \in \Z^+\), by \cref{ii:2.5.7} we know that
  \begin{align*}
    \forall n \in \Z^+, \mathcal{F}_{K_n} & = \set{K_n \cap V : V \in \mathcal{F}}                                  \\
                                          & = \set{(K_1 \cap K_n) \cap V : V \in \mathcal{F}} & (K_n \subseteq K_1) \\
                                          & = \set{K_n \cap (K_1 \cap V) : V \in \mathcal{F}}                       \\
                                          & = \set{K_n \cap V : V \in \mathcal{F}_{K_1}}.
  \end{align*}
  Thus by \cref{ii:2.5.7} we know that for every \(n \in \Z^+\), \((K_n, \mathcal{F}_{K_n})\) induced by \((X, \mathcal{F})\) can also induced by \((K_1, \mathcal{F}_{K_1})\).
  By \cref{ii:2.5.9} we have
  \begin{align*}
             & \forall n \in \Z^+, (K_n, \mathcal{F}_{K_n}) \text{ is compact topological subspace of } (X, \mathcal{F})                                                               \\
    \implies & \forall n \in \Z^+, \exists S_n \subseteq \mathcal{F}_{K_n} : \begin{dcases}
                                                                               S_n \text{ is finite} \\
                                                                               K_n = \bigcup \mathcal{F}_{K_n} = \bigcup S_n
                                                                             \end{dcases}                                                              \\
    \implies & \forall n \in \Z^+, \exists S_n \subseteq \mathcal{F}_{K_n} : \begin{dcases}
                                                                               S_n' = \set{V \cap W : (V, W) \in S_1 \times S_n} \text{ is finite} \\
                                                                               S_n' \subseteq \mathcal{F}_{K_n} \text{ by \cref{ii:2.5.7}}         \\
                                                                               \forall S \subseteq \mathcal{F}, K_n \subseteq K_1 \subseteq \bigcup S \implies K_n \subseteq \bigcup S_n'
                                                                             \end{dcases} \\
    \implies & \forall n \in \Z^+, (K_n, \mathcal{F}_{K_n}) \text{ is compact topological subspace of } (K_1, \mathcal{F}_{K_1}).
  \end{align*}

  Now we fix one \(n \in \Z^+\).
  Since \((X, \mathcal{F})\) is Hausdorff, we know that
  \begin{align*}
             & \forall x, y \in K_n, x \neq y                                                                    \\
    \implies & \exists V, W \in \mathcal{F} : \begin{dcases}
                                                V \neq \emptyset \neq W \\
                                                x \in V                 \\
                                                y \in W                 \\
                                                V \cap W = \emptyset
                                              \end{dcases}                   & (K_n \subseteq X)                 \\
    \implies & \exists V_{K_n}, W_{K_n} \in \mathcal{F}_{K_n} : \begin{dcases}
                                                                  V_{K_n} = V \cap K_n                \\
                                                                  W_{K_n} = W \cap K_n                \\
                                                                  x \in V_{K_n}                       \\
                                                                  y \in W_{K_n}                       \\
                                                                  V_{K_n} \neq \emptyset \neq W_{K_n} \\
                                                                  V_{K_n} \cap W_{K_n} = \emptyset
                                                                \end{dcases} &                   & \by{ii:2.5.7}
  \end{align*}
  Thus \((K_n, \mathcal{F}_{K_n})\) is Hausdorff.
  Since \(n\) was arbitrary, we know that for each \(n \in \Z^+\), \((K_n, \mathcal{F}_{K_n})\) is Hausdorff.

  We claim that for each \(n \in \Z^+\), \((K_n, \mathcal{F}_{K_n})\) is closed in \((K_1, \mathcal{F}_{K_1})\).
  Suppose for sake of contradiction that there exists some \(n \in \Z^+\) such that \((K_n, \mathcal{F}_{K_n})\) is not closed in \((K_1, \mathcal{F}_{K_1})\).
  Then by \cref{ii:ex:2.5.11} we know that \(\overline{K_n}_{(K_1, \mathcal{F}_{K_1})} \setminus K_n = \emptyset\).
  Let \(y \in \overline{K_n}_{(K_1, \mathcal{F}_{K_1})} \setminus K_n\).
  Since \((K_1, \mathcal{F}_{K_1})\) is Hausdorff, we know that
  \[
    \forall x \in K_n, \exists V_x, W_x \in \mathcal{F}_{K_1} : \begin{dcases}
      V_x \neq \emptyset \neq W_x \\
      x \in V_x                   \\
      y \in W_x                   \\
      V_x \cap W_x = \emptyset
    \end{dcases}
  \]
  Since \((K_n, \mathcal{F}_{K_n})\) is a compact topological subspace of \((K_1, \mathcal{F}_{K_1})\), by \cref{ii:2.5.9} we have
  \[
    K_n \subseteq \bigcup_{x \in K_n} V_x \implies \exists S \subseteq K_n : \begin{dcases}
      S \text{ is finite} \\
      K_n \subseteq \bigcup_{x \in S} V_x
    \end{dcases}
  \]
  By \cref{ii:2.5.1} we have
  \begin{align*}
             & \begin{dcases}
                 \forall x \in S, V_x \in \mathcal{F}_{K_1} \\
                 y \in \bigcap_{x \in S} W_x \in \mathcal{F}_{K_1}
               \end{dcases}                                         \\
    \implies & \forall x \in S, V_x \cap \bigg(\bigcap_{x' \in S} W_{x'}\bigg) = \emptyset              \\
    \implies & \bigcup_{x \in S} \Bigg(V_x \cap \bigg(\bigcap_{x' \in S} W_{x'}\bigg)\Bigg) = \emptyset \\
    \implies & \bigg(\bigcup_{x \in S} V_x\bigg) \cap \bigg(\bigcap_{x' \in S} W_{x'}\bigg) = \emptyset \\
    \implies & K_n \cap \bigg(\bigcap_{x' \in S} W_{x'}\bigg) = \emptyset
  \end{align*}
  But this contradict to the fact that \(y \in \overline{K_n}_{(K_1, \mathcal{F}_{K_1})}\).
  Thus \((K_n, \mathcal{F}_{K_n})\) is closed in \((K_1, \mathcal{F}_{K_1})\) for each \(n \in \Z^+\).

  Let \(V_n = K_1 \setminus K_n\) for every \(n \geq 1\).
  Then for every \(n \geq 1\), we have \(V_n \subseteq K_1\) and \(V_n\) is open in \((K_1, \mathcal{F}_{K_1})\).
  Suppose for sake of contradiction that \(\bigcap_{n = 1}^\infty K_n = \emptyset\).
  Since
  \[
    \bigcup_{n = 1}^\infty V_n = \bigcup_{n = 1}^\infty (K_1 \setminus K_n) = K_1 \setminus \bigg(\bigcap_{n = 1}^\infty K_n\bigg) = K_1
  \]
  and \((K_1, \mathcal{F}_{K_1})\) is compact, by \cref{ii:1.5.8} we know that there exists a finite set \(F \subseteq \Z^+\) such that
  \[
    K_1 \subseteq \bigcup_{i \in F} V_i.
  \]
  Since \(F\) is finite subset of \(\Z^+\), we know that \(\min(F)\) is well-defined.
  Then we have
  \begin{align*}
             & K_1 \subseteq \bigcup_{i \in F} V_i \subseteq \bigcup_{n = 1}^\infty V_i = K_1                                                             \\
    \implies & K_1 = \bigcup_{i \in F} V_i                                                                                                                \\
    \implies & K_1 = \bigcup_{i \in F} (K_1 \setminus K_i)                                                                                                \\
    \implies & K_1 = K_1 \setminus \bigg(\bigcap_{i \in F} K_i\bigg)                                                                                      \\
    \implies & \bigcap_{i \in F} K_i = \emptyset                                              &  & \text{(since \(\bigcap_{i \in F} K_i \subseteq K_1)\)} \\
    \implies & K_{\min(F)} = \emptyset.                                                       &  & \text{(since \(K_{\min(F)} = \bigcap_{i \in F} K_i)\)}
  \end{align*}
  But by hypothesis we know that \(K_{\min(F)} \neq \emptyset\), a contradiction.
  Thus \(\bigcap_{n = 1}^\infty K_n \neq \emptyset\).
\end{proof}

\begin{ex}\label{ii:ex:2.5.14}
  Generalize \cref{ii:1.5.10} to compact sets in a Hausdorff topological space.
\end{ex}

\begin{proof}
  Let \((X, \mathcal{F})\) be a Hausdorff topological space.
  We first show that if \(Z \subseteq Y \subseteq X\) such that \((Y, \mathcal{F}_Y)\) is compact, then \((Z, \mathcal{F}_Z)\) is compact iff \(Z\) is closed in \((Y, \mathcal{F}_Y)\).
  By \cref{ii:ex:2.5.13} we know that if \((Z, \mathcal{F}_Z)\) is compact then \(Z\) is closed in \((Y, \mathcal{F}_Y)\).
  So we only need to show that if \(Z\) is closed in \((Y, \mathcal{F}_Y)\) then \((Z, \mathcal{F}_Z)\) is compact.
  Let \(S_Z \subseteq \mathcal{F}_Z\) be an open cover of \(Z\).
  Then we have
  \begin{align*}
             & Z \text{ is closed in } (Y, \mathcal{F}_Y)           \\
    \implies & Y \setminus Z \text{ is open in } (Y, \mathcal{F}_Y)
  \end{align*}
  and
  \begin{align*}
             & Z = \bigcup S_Z                                                                                                   \\
    \implies & Z \subseteq \bigcup \set{V \in \mathcal{F}_Y : V \cap Z \in S_Z}                               &  & \by{ii:2.5.7} \\
    \implies & Y = \bigg(\bigcup \set{V \in \mathcal{F}_Y : V \cap Z \in S_Z}\bigg) \cup (Y \setminus Z)                         \\
    \implies & \exists S_Y \subseteq \mathcal{F}_Y : \begin{dcases}
                                                       S_Y \text{ is finite}                                      \\
                                                       Y = \bigg(\bigcup \set{V \in S_Y : V \cap Z \in S_Z}\bigg) \\
                                                       \quad \cup (Y \setminus Z)
                                                     \end{dcases}                                  &  & \by{ii:2.5.9}            \\
    \implies & \exists S_Y \subseteq \mathcal{F}_Y : \begin{dcases}
                                                       S_Y \text{ is finite} \\
                                                       Z = \bigcup \set{V \cap Z : V \in S_Y}
                                                     \end{dcases}                                       \\
    \implies & \set{V \cap Z : V \in S_Y} \text{ is an finite subcover of } Z \text{ in } (Z, \mathcal{F}_Z).
  \end{align*}
  Since \(S_Z\) was arbitrary, by \cref{ii:2.5.9} we know that \((Z, \mathcal{F}_Z)\) is compact.

  Next we show that if \((Y_i)_{i = 1}^n\) is a finite collection of compact topological subspaces of \((X, \mathcal{F})\), then \(\big(\bigcup_{i = 1}^n Y_i, \mathcal{F}_{\bigcup_{i = 1}^n Y_i}\big)\) is also a compact topological subspace of \((X, \mathcal{F})\).
  By \cref{ii:2.5.7} we have
  \[
    \mathcal{F}_{\bigcup_{i = 1}^n Y_i} = \set{V \cap \bigg(\bigcup_{i = 1}^n Y_i\bigg) : V \in \mathcal{F}}.
  \]
  Observe that
  \begin{align*}
             & \begin{dcases}
                 X \in \mathcal{F} \\
                 \emptyset \in \mathcal{F}
               \end{dcases}                                                                                 \\
    \implies & \begin{dcases}
                 X \cap \bigg(\bigcup_{i = 1}^n Y_i\bigg) = \bigcup_{i = 1}^n Y_i \in \mathcal{F}_{\bigcup_{i = 1}^n Y_i} \\
                 \emptyset \cap \bigg(\bigcup_{i = 1}^n Y_i\bigg) = \emptyset \in \mathcal{F}_{\bigcup_{i = 1}^n Y_i}
               \end{dcases}
  \end{align*}
  Let \((V^{(j)})_{j = 1}^m\) be a finite collection of elements in \(\mathcal{F}_{\bigcup_{i = 1}^n Y_i}\).
  Then we have
  \begin{align*}
             & \forall 1 \leq j \leq m, \exists V_X^{(j)} \in \mathcal{F} : V_X^{(j)} \cap \bigg(\bigcup_{i = 1}^n Y_i\bigg) = V^{(j)}                             &  & \by{ii:2.5.7} \\
    \implies & \bigcap_{j = 1}^m V_X^{(j)} \in \mathcal{F}                                                                                                         &  & \by{ii:2.5.1} \\
    \implies & \bigg(\bigcap_{j = 1}^m V_X^{(j)}\bigg) \cap \bigg(\bigcup_{i = 1}^n Y_i\bigg) \in \mathcal{F}_{\bigcup_{i = 1}^n Y_i}                              &  & \by{ii:2.5.7} \\
    \implies & \bigcap_{j = 1}^m \Bigg(V_X^{(j)} \cap \bigg(\bigcup_{i = 1}^n Y_i\bigg)\Bigg) = \bigcap_{j = 1}^m V^{(j)} \in \mathcal{F}_{\bigcup_{i = 1}^n Y_i}.
  \end{align*}
  Since \((V^{(j)})_{j = 1}^m\) was arbitrary, we conclude that the intersection of any finite collection of open sets in \(\big(\bigcup_{i = 1}^n Y_i, \mathcal{F}_{\bigcup_{i = 1}^n Y_i}\big)\) is open in \(\big(\bigcup_{i = 1}^n Y_i, \mathcal{F}_{\bigcup_{i = 1}^n Y_i}\big)\).
  Let \(S \subseteq \mathcal{F}_{\bigcup_{i = 1}^n Y_i}\).
  Then we have
  \begin{align*}
             & \forall V \in S, \exists V_X \in \mathcal{F} : V_X \cap \bigg(\bigcup_{i = 1}^n Y_i\bigg) = V                                                                                                           &  & \by{ii:2.5.7} \\
    \implies & \bigcup_{V_x \in \mathcal{F} : V_X \cap (\bigcup_{i = 1}^n Y_i) \in \mathcal{F}_{\bigcup_{i = 1}^n Y_i}} V_x \in \mathcal{F}                                                                            &  & \by{ii:2.5.1} \\
    \implies & \bigg(\bigcup_{V_x \in \mathcal{F} : V_X \cap (\bigcup_{i = 1}^n Y_i) \in \mathcal{F}_{\bigcup_{i = 1}^n Y_i}} V_x\bigg) \cap \bigg(\bigcup_{i = 1}^n Y_i\bigg) \in \mathcal{F}_{\bigcup_{i = 1}^n Y_i} &  & \by{ii:2.5.7} \\
    \implies & \bigcup_{V \in S} V \in \mathcal{F}_{\bigcup_{i = 1}^n Y_i}.
  \end{align*}
  Since \(S\) was arbitrary, we conclude that the union of arbitrary many open sets in \(\big(\bigcup_{i = 1}^n Y_i, \mathcal{F}_{\bigcup_{i = 1}^n Y_i}\big)\) is open in \(\big(\bigcup_{i = 1}^n Y_i, \mathcal{F}_{\bigcup_{i = 1}^n Y_i}\big)\).
  By \cref{ii:2.5.1} and all the proofs above we conclude that \(\big(\bigcup_{i = 1}^n Y_i, \mathcal{F}_{\bigcup_{i = 1}^n Y_i}\big)\) is a topological subspace of \((X, \mathcal{F})\).

  Let \(S \subseteq \mathcal{F}_{\bigcup_{i = 1}^n Y_i}\) be an open cover of \(\bigcup_{i = 1}^n Y_i\) in \(\big(\bigcup_{i = 1}^n Y_i, \mathcal{F}_{\bigcup_{i = 1}^n Y_i}\big)\).
  Then we have
  \begin{align*}
             & \forall 1 \leq i \leq n, Y_i \subseteq \bigcup_{j = 1}^n Y_i = \bigcup S       \\
    \implies & \forall 1 \leq i \leq n, \exists S_i \in \mathcal{F}_{\bigcup_{j = 1}^n Y_j} : \\
             & \begin{dcases}
                 S_i \text{ is finite} \\
                 Y_i \subseteq \bigcup S_i \subseteq \bigcup_{i = 1}^n Y_i = \bigcup S
               \end{dcases}          &  & \by{ii:2.5.9}           \\
    \implies & \bigcup_{i = 1}^n Y_i = \bigcup \bigg(\bigcup_{i = 1}^n S_i\bigg).
  \end{align*}
  Since \(S_i\) is finite for each \(1 \leq i \leq n\), we know that \(\bigcup_{i = 1}^n S_i\) is a finite subcover of \(\bigcup_{i = 1}^n Y_i\) in \(\big(\bigcup_{i = 1}^n Y_i, \mathcal{F}_{\bigcup_{i = 1}^n Y_i}\big)\).
  Since \(S\) was arbitrary, by \cref{ii:2.5.9} we know that \(\big(\bigcup_{i = 1}^n Y_i, \mathcal{F}_{\bigcup_{i = 1}^n Y_i}\big)\) is a compact topological subspace of \((X, \mathcal{F})\).

  Finally we show that every finite subset of \(X\) is compact.
  Let \(x_0 \in X\) and let \(\mathcal{F}_{\set{x_0}} = \set{V \cap \set{x_0} : V \in \mathcal{F}}\).
  We have
  \begin{align*}
             & \begin{dcases}
                 X \in \mathcal{F} \\
                 \emptyset \in \mathcal{F}
               \end{dcases}                                        &  & \by{ii:2.5.1} \\
    \implies & \begin{dcases}
                 X \cap \set{x_0} = \set{x_0} \in \mathcal{F}_{\set{x_0}} \\
                 \emptyset \cap \set{x_0} = \emptyset \in \mathcal{F}_{\set{x_0}}
               \end{dcases}
  \end{align*}
  Since \(\mathcal{F}_{\set{x_0}} = \set{\emptyset, \set{x_0}}\), by \cref{ii:ex:2.5.1} we know that \((\set{x_0}, \mathcal{F}_{\set{x_0}})\) is a topological space.
  Let \(S \subseteq \mathcal{F}_{\set{x_0}}\) such that \(\set{x_0} \subseteq \bigcup S\).
  Then we have
  \begin{align*}
             & \set{x_0} \subseteq \bigcup S \\
    \implies & \exists V \in S : x_0 \in V   \\
    \implies & \set{x_0} \subseteq V.
  \end{align*}
  Since \(S\) was arbitrary, we conclude that every open cover of \(\set{x_0}\) has a finite subcover in \((\set{x_0}, \mathcal{F}_{\set{x_0}})\), and by \cref{ii:2.5.9} we know that \((\set{x_0}, \mathcal{F}_{\set{x_0}})\) is compact.
  Since \(x_0\) was arbitrary, we conclude that every singleton subset of \(X\) is compact.
  And from the proof above we conclude that every finite subset of \(X\) is compact.
\end{proof}

\begin{ex}\label{ii:ex:2.5.15}
  Let \((X, d_X)\) and \((Y, d_Y)\) be metric spaces (and hence a topological space).
  Show that the two notions continuity (both at a point, and on the whole domain) of a function \(f : X \to Y\) in \cref{ii:2.1.1} and \cref{ii:2.5.8} coincide.
\end{ex}

\begin{proof}
  Let
  \begin{align*}
     & \mathcal{F}_X = \set{V \subseteq X : V \text{ is open in } (X, d_X)}; \\
     & \mathcal{F}_Y = \set{V \subseteq Y : V \text{ is open in } (Y, d_Y)}.
  \end{align*}
  Since \((X, d_X)\) and \((Y, d_Y)\) are metric spaces, we know that \((X, \mathcal{F}_X)\) and \((Y, \mathcal{F}_Y)\) are topological spaces.

  Let \(x_0 \in X\).
  First suppose that \(f\) is continuous at \(x_0\) in the sense of \cref{ii:2.1.1}.
  Then we have
  \begin{align*}
             & \forall \varepsilon \in \R^+, \exists \delta \in \R^+ :                                                                \\
             & \Big(\forall x \in X, d_X(x, x_0) < \delta \implies d_Y\big(f(x), f(x_0)\big) < \varepsilon\Big)                       \\
    \implies & \forall \varepsilon \in \R^+, \exists \delta \in \R^+ :                                                                \\
             & \Big(\forall x \in X, x \in B_{(X, d_X)}(x_0, \delta) \implies f(x) \in B_{(Y, d_Y)}\big(f(x_0), \varepsilon\big)\Big) \\
    \implies & \forall \varepsilon \in \R^+, \exists \delta \in \R^+ :                                                                \\
             & f\big(B_{(X, d_X)}(x_0, \delta)\big) \subseteq B_{(Y, d_Y)}\big(f(x_0), \varepsilon\big)
  \end{align*}
  Let \(V \in \mathcal{F}_Y\) such that \(f(x_0) \in V\).
  Then we have
  \begin{align*}
             & f(x_0) \in V                                                                                                                                           \\
    \implies & \exists \varepsilon \in \R^+ : B_{(Y, d_Y)}\big(f(x_0), \varepsilon\big) \subseteq V                                            &  & \by{ii:1.2.15}[a] \\
    \implies & \exists \delta \in \R^+ : f\big(B_{(X, d_X)}(x_0, \delta)\big) \subseteq B_{(Y, d_Y)}\big(f(x_0), \varepsilon\big) \subseteq V.
  \end{align*}
  By \cref{ii:1.2.15}(c) we know that \(B_{(X, d_X)}(x_0, \delta) \in \mathcal{F}_X\).
  Since \(V\) was arbitrary, by \cref{ii:2.5.8} we know that \(f\) is continuous at \(x_0\) from \((X, \mathcal{F}_X)\) to \((Y, \mathcal{F}_Y)\).

  Now suppose that \(f\) is continuous at \(x_0\) in the sense of \cref{ii:2.5.8}.
  Then we have
  \[
    \forall V \in \mathcal{F}_Y, f(x_0) \in V \implies \exists U \in \mathcal{F}_X : \begin{dcases}
      x_0 \in U \\
      f(U) \subseteq V
    \end{dcases}
  \]
  Let \(\varepsilon \in \R^+\).
  Then we have
  \begin{align*}
             & B_{(Y, d_Y)}\big(f(x_0), \varepsilon\big) \in \mathcal{F}_Y                                       &  & \by{ii:1.2.15}[c] \\
    \implies & \exists U \in \mathcal{F}_X : \begin{dcases}
                                               x_0 \in U \\
                                               f(U) \subseteq B_{(Y, d_Y)}\big(f(x_0), \varepsilon\big)
                                             \end{dcases}                                    \\
    \implies & \exists \delta \in \R^+ : \begin{dcases}
                                           B_{(X, d_X)}(x_0, \delta) \subseteq U \\
                                           f(U) \subseteq B_{(Y, d_Y)}\big(f(x_0), \varepsilon\big)
                                         \end{dcases}                                          &  & \by{ii:1.2.15}[a]                   \\
    \implies & \exists \delta \in \R^+ :                                                                                                \\
             & f\big(B_{(X, d_X)}(x_0, \delta)\big) \subseteq B_{(Y, d_Y)}\big(f(x_0), \varepsilon\big)                                 \\
    \implies & \exists \delta \in \R^+ :                                                                                                \\
             & \Big(\forall x \in X, d_X(x, x_0) < \delta \implies d_Y\big(f(x), f(x_0)\big) < \varepsilon\Big).
  \end{align*}
  Since \(\varepsilon\) was arbitrary, by \cref{ii:2.1.1} we know that \(f\) is continuous at \(x_0\) from \((X, d_X)\) to \((Y, d_Y)\).

  Since \(x_0\) was arbitrary, we conclude that \(f\) is continuous on \(X\) from \((X, d_X)\) to \((Y, d_Y)\) iff \(f\) is continuous on \(X\) from \((X, \mathcal{F}_X)\) to \((Y, \mathcal{F}_Y)\).
\end{proof}

\begin{ex}\label{ii:ex:2.5.16}
  Show that when \cref{ii:2.1.4} is extended to topological spaces, that (a) implies (b).
  (The converse is false, but constructing an example is diffcult.)
  Show that when \cref{ii:2.1.5} is extended to topological spaces, that (a), (c), (d) are all equivalent to each other, and imply (b).
  (Again, the converse implications are false, but diffcult to prove.)
\end{ex}

\begin{proof}
  Let \((X, \mathcal{F}_X)\) and \((Y, \mathcal{F}_Y)\) be two topological spaces.
  Let \(f : X \to Y\) be a function.
  Let \(x_0 \in X\).
  We first show that if \(f\) is continuous at \(x_0\) from \((X, \mathcal{F}_X)\) to \((Y, \mathcal{F}_Y)\) and \((x^{(n)})_{n = 1}^\infty\) is a sequence in \(X\) which converges to \(x_0\) in \((X, \mathcal{F}_X)\), then \(\big(f(x)\big)_{n = 1}^\infty\) converges to \(f(x_0)\) in \((Y, \mathcal{F}_Y)\).
  Since \(f\) is continuous at \(x_0\) from \((X, \mathcal{F}_X)\) to \((Y, \mathcal{F}_Y)\), by \cref{ii:2.5.8} we know that
  \[
    \forall V \in \mathcal{F}_Y, f(x_0) \in V \implies \exists U \in \mathcal{F}_X : \begin{dcases}
      x_0 \in U \\
      f(U) \subseteq V
    \end{dcases}
  \]
  Since \((x^{(n)})_{n = 1}^\infty\) converges to \(x_0\) in \((X, \mathcal{F}_X)\), by \cref{ii:2.5.4} we have
  \[
    \forall U \in \mathcal{F}_X, x_0 \in U \implies \exists N \in \Z^+ : \forall n \geq N, x^{(n)} \in U
  \]
  This means
  \[
    \forall V \in \mathcal{F}_Y, f(x_0) \in V \implies \exists U \in \mathcal{F}_X : \begin{dcases}
      x_0 \in U        \\
      f(U) \subseteq V \\
      \exists N \in \Z^+ : \forall n \geq N, f(x^{(n)}) \in f(U) \subseteq V
    \end{dcases}
  \]
  and we have
  \[
    \forall V \in \mathcal{F}_Y, f(x_0) \in V \implies \exists N \in \Z^+ : \forall n \geq N, f(x^{(n)}) \in V.
  \]
  By \cref{ii:2.5.4} we know that \(\big(f(x^{(n)})\big)_{n = 1}^\infty\) converges to \(f(x_0)\) in \((Y, \mathcal{F}_Y)\).
  Since \(x_0\) was arbitrary, we conclude that if \(f\) is continuous on \(X\) from \((X, \mathcal{F}_X)\) to \((Y, \mathcal{F}_Y)\), then whenever \((x^{(n)})_{n = 1}^\infty\) is a sequence in \(X\) which converges to some \(x_0 \in X\) in \((X, \mathcal{F})\), the sequence \(\big(f(x^{(n)})\big)_{n = 1}^\infty\) converges to \(f(x_0)\) in \((Y, \mathcal{F}_Y)\).

  Next we show that if \(f\) is continuous on \(X\) from \((X, \mathcal{F}_X)\) to \((Y, \mathcal{F}_Y)\), then whenever \(V \in \mathcal{F}_Y\), the set \(f^{-1}(V) \in \mathcal{F}_X\).
  Let \(V \in \mathcal{F}_Y\) and let \(x_0 \in f^{-1}(V)\).
  Since \(f\) is continuous at \(x_0\) from \((X, \mathcal{F}_X)\) to \((Y, \mathcal{F}_Y)\), we know that
  \[
    \exists U \in \mathcal{F}_X : \begin{dcases}
      x_0 \in U \\
      f(U) \subseteq V
    \end{dcases}
  \]
  Since \(f(U) \subseteq V\), we know that \(U \subseteq f^{-1}(V)\).
  Since \(U \in \mathcal{F}_X\), by \cref{ii:2.5.5} we know that \(x_0 \in \text{int}_{(X, \mathcal{F}_X)}\big(f^{-1}(V)\big)\).
  Since \(x_0\) was arbitrary, by \cref{ii:ex:2.5.11} we know that \(f^{-1}(V) \in \mathcal{F}_X\).

  Next we show that if \(V \in \mathcal{F}_Y\) implies \(f^{-1}(V) \in \mathcal{F}_X\), then \(U\) is closed in \((Y, \mathcal{F}_Y)\) implies \(f^{-1}(U)\) is closed in \((X, \mathcal{F}_X)\).
  Let \(U \subseteq Y\) such that \(U\) is closed in \((Y, \mathcal{F}_Y)\).
  Then we have
  \begin{align*}
             & U \text{ is closed in } (Y, \mathcal{F}_Y)                                                                    \\
    \implies & Y \setminus U \text{ is open in } (Y, \mathcal{F}_Y)                                                          \\
    \implies & f^{-1}(Y \setminus U) \text{ is open in } (X, \mathcal{F}_X)               &  & \text{(from the proof above)} \\
    \implies & X \setminus f^{-1}(Y \setminus U) \text{ is closed in } (X, \mathcal{F}_X)                                    \\
    \implies & f^{-1}(U) \text{ is closed in } (X, \mathcal{F}_X).
  \end{align*}

  Next we show that if \(U\) is closed in \((Y, \mathcal{F}_Y)\) implies \(f^{-1}(U)\) is closed in \((X, \mathcal{F}_X)\), then \(f\) is continuous on \(X\) from \((X, \mathcal{F}_X)\) to \((Y, \mathcal{F}_Y)\).
  Let \(x_0 \in X\) and let \(V \in \mathcal{F}_Y\) such that \(f(x_0) \in V\).
  We have
  \begin{align*}
             & V \in \mathcal{F}_Y                                                                                  \\
    \implies & V \text{ is open in } (Y, \mathcal{F}_Y)                                                             \\
    \implies & Y \setminus V \text{ is closed in } (Y, \mathcal{F}_Y)                                               \\
    \implies & f^{-1}(Y \setminus V) \text{ is closed in } (Y, \mathcal{F}_Y)           &  & \text{(by hypothesis)} \\
    \implies & X \setminus f^{-1}(Y \setminus V) \text{ is open in } (Y, \mathcal{F}_Y)                             \\
    \implies & f^{-1}(V) \text{ is open in } (Y, \mathcal{F}_Y)                                                     \\
    \implies & f^{-1}(V) \in \mathcal{F}_X.
  \end{align*}
  Since \(V\) was arbitrary, we know that
  \[
    \forall V \in \mathcal{F}_Y, f(x_0) \in V \implies \exists U \in \mathcal{F}_X : \begin{dcases}
      x_0 \in U \\
      f(U) \subseteq V
    \end{dcases}
  \]
  and by \cref{ii:2.5.8} \(f\) is continuous at \(x_0\) from \((X, \mathcal{F}_X)\) to \((Y, \mathcal{F}_Y)\).
  Since \(x_0\) was arbitrary, we conclude that \(f\) is continuous on \(X\) from \((X, \mathcal{F}_X)\) to \((Y, \mathcal{F}_Y)\).

  From all proofs above we conclude that \cref{ii:2.1.5}(a)(c)(d) are equivalent in the topological version.
\end{proof}

\begin{ex}\label{ii:ex:2.5.17}
  Generalize both \cref{ii:2.3.1} and \cref{ii:2.3.2} to compact sets in a topological space.
\end{ex}

\begin{proof}
  Let \((X, \mathcal{F}_X)\) and \((Y, \mathcal{F}_Y)\) be two topological spaces.
  Let \(f : X \to Y\) be continuous function from \((X, \mathcal{F}_X)\) to \((Y, \mathcal{F}_Y)\), and let \(K \subseteq X\) such that \((K, \mathcal{F}_K)\) is compact.
  We want to show that \(\big(f(K), \mathcal{F}_{f(K)}\big)\) is also compact.
  Let \(S \subseteq \mathcal{F}_{f(K)}\) be an open cover of \(f(K)\), i.e., \(f(K) = \bigcup_{V_K \in S} V_K\).
  By \cref{ii:2.5.7} we know that the set \(S_Y = \set{V \in \mathcal{F}_Y : V \cap f(K) \in S}\) is non-empty.
  Then we have
  \begin{align*}
             & \forall V \in S_Y, f^{-1}(V) \in \mathcal{F}_X                                                     &  & \by{ii:ex:2.5.16} \\
    \implies & \forall V \in S_Y, f^{-1}(V) \cap K \in \mathcal{F}_K                                              &  & \by{ii:2.5.7}     \\
    \implies & \forall V \in S_Y, f\big(f^{-1}(V) \cap K\big) = V \cap f(K)                                                              \\
    \implies & f(K) = \bigcup_{V \in S_Y} f\big(f^{-1}(V) \cap K\big) = \bigcup_{V \in S_Y} \big(V \cap f(K)\big)                        \\
    \implies & K \subseteq \bigcup_{V \in S_Y} \big(f^{-1}(V) \cap K\big)                                                                \\
    \implies & \exists S_Y' \subseteq S_Y : \begin{dcases}
                                              S_Y' \text{ is finite}                                      \\
                                              K \subseteq \bigcup_{V \in S_Y'} \big(f^{-1}(V) \cap K\big) \\
                                              f(K) = \bigcup_{V \in S_Y'} \big(V \cap f(K)\big)
                                            \end{dcases}                                     &  & \by{ii:2.5.9}                          \\
  \end{align*}
  Since \(S\) was arbitrary, by \cref{ii:2.5.9} we know that \(\big(f(K), \mathcal{F}_{f(K)}\big)\) is compact.

  Now let \((X, \mathcal{F})\) be a compact topological space.
  Define
  \[
    \mathcal{F}_{\R} = \set{V \subseteq \R : V \text{ is open in } (\R, d_{l^1}|_{\R \times \R})}.
  \]
  Let \(f : X \to \R\) be a continuous function from \((X, \mathcal{F})\) to \((\R, \mathcal{F}_{\R})\).
  We want to show that \(f\) is bounded.
  Furthermore, if \(X \neq \emptyset\), then there exists some \(x_{\min}, x_{\max} \in X\) such that \(f(x_{\min}) \leq f(x) \leq f(x_{\max})\) for each \(x \in X\).
  We have
  \begin{align*}
             & \begin{dcases}
                 (X, \mathcal{F}) \text{ is compact} \\
                 f \text{ is continuous from } (X, \mathcal{F}) \text{ to } (\R, \mathcal{F}_{\R})
               \end{dcases}                                     \\
    \implies & \big(f(X), \mathcal{F}_{f(X)}\big) \text{ is compact}                             &  & \text{(from the proof above)} \\
    \implies & \begin{dcases}
                 f(X) \text{ is closed in } (\R, \mathcal{F}_{\R}) \\
                 \big(f(X), d_{l^1}|_{f(X) \times f(X)}\big) \text{ is bounded}
               \end{dcases}                    &  & \by{ii:1.5.7}                                                        \\
    \implies & \begin{dcases}
                 f(X) \text{ is closed in } (\R, \mathcal{F}_{\R}) \\
                 f(X) \text{ is bounded subset of } \R
               \end{dcases}                              &  & \by{ii:ex:1.5.1}
  \end{align*}
  The rest follows as in \cref{ii:2.3.2}.
\end{proof}

\chapter{Uniform convergence}\label{ch:3}

\begin{note}
  It turns out that there are several different concepts of convergence of functions;
  here we describe the two most important ones, \emph{pointwise convergence} and \emph{uniform convergence}.
  (There are other types of convergence for functions, such as \(L^1\) convergence, \(L^2\) convergence, convergence in measure, almost everywhere convergence, and so forth, but these are beyond the scope of this text.)
  The two notions are related, but not identical;
  the relationship between the two is somewhat analogous to the relationship between continuity and uniform continuity.
\end{note}

\section{Limiting values of functions}\label{sec:3.1}

\begin{defn}[Limiting value of a function]\label{3.1.1}
  Let \((X, d_X)\) and \((Y, d_Y)\) be metric spaces, let \(E\) be a subset of \(X\), and let \(f : E \to Y\) be a function.
  If \(x_0 \in X\) is an adherent point of \(E\), and \(L \in Y\), we say that \emph{\(f(x)\) converges to \(L\) in \(Y\) as \(x\) converges to \(x_0\) in \(E\)}, or write \(\lim_{x \to x_0 ; x \in E} f(x) = L\), if for every \(\varepsilon > 0\) there exists a \(\delta > 0\) such that \(d_Y\big(f(x), L\big) < \varepsilon\) for all \(x \in E\) such that \(d_X(x, x_0) < \delta\).
\end{defn}

\begin{rmk}\label{3.1.2}
  Some authors exclude the case \(x = x_0\) from the above definition, thus requiring \(0 < d_X(x, x_0) < \delta\).
  In our current notation, this would correspond to removing \(x_0\) from \(E\), thus one would consider
  \[
    \lim_{x \to x_0 ; x \in E \setminus \{x_0\}} f(x)
  \]
  instead of
  \[
    \lim_{x \to x_0 ; x \in E} f(x).
  \]
\end{rmk}

\begin{note}
  Comparing this with \cref{2.1.1}, we see that \(f\) is continuous at \(x_0\) if and only if
  \[
    \lim_{x \to x_0 ; x \in X} f(x) = f(x_0).
  \]
  Thus \(f\) is continuous on \(X\) iff we have
  \[
    \lim_{x \to x_0 ; x \in X} f(x) = f(x_0) \text{ for all } x_0 \in X.
  \]
\end{note}

\setcounter{thm}{3}
\begin{rmk}\label{3.1.4}
  Often we shall omit the condition \(x \in X\), and abbreviate
  \[
    \lim_{x \to x_0 ; x \in X} f(x)
  \]
  as simply
  \[
    \lim_{x \to x_0} f(x)
  \]
  when it is clear what space \(x\) will range in.
\end{rmk}

\begin{prop}\label{3.1.5}
  Let \((X, d_X)\) and \((Y, d_Y)\) be metric spaces, let \(E\) be a subset of \(X\), and let \(f : E \to Y\) be a function.
  Let \(x_0 \in X\) be an adherent point of \(E\) and \(L \in Y\).
  Then the following four statements are logically equivalent:
  \begin{enumerate}
    \item \(\lim_{x \to x_0 ; x \in E} f(x) = L\).
    \item For every sequence \((x^{(n)})_{n = 1}^\infty\) in \(E\) which converges to \(x_0\) with respect to the metric \(d_X\), the sequence \(\big(f(x^{(n)})\big)_{n = 1}^\infty\) converges to \(L\) with respect to the metric \(d_Y\).
    \item For every open set \(V \subseteq Y\) which contains \(L\), there exists an open set \(U \subseteq X\) containing \(x_0\) such that \(f(U \cap E) \subseteq V\).
    \item If one defines the function \(g : E \cup \{x_0\} \to Y\) by defining \(g(x_0) \coloneqq L\), and \(g(x) \coloneqq f(x)\) for \(x \in E \setminus \{x_0\}\), then \(g\) is continuous at \(x_0\).
          Furthermore, if \(x_0 \in E\), then \(f(x_0) = L\).
  \end{enumerate}
\end{prop}

\begin{proof}
  We first show that statement (a) implies statement (b).
  Suppose that
  \[
    d_Y - \lim_{x \to x_0 ; x \in E} f(x) = L.
  \]
  By \cref{3.1.1} we have
  \[
    \forall \varepsilon \in \R^+, \exists\ \delta \in \R^+ : \Big(\forall x \in E, d_X(x, x_0) < \delta \implies d_Y\big(f(x), L\big) < \varepsilon\Big).
  \]
  Let \((x^{(n)})_{n = 1}^\infty\) be a sequence in \(E\) such that \(\lim_{n \to \infty} d_X(x^{(n)}, x_0) = 0\).
  By \cref{1.1.14} we have
  \[
    \forall \delta \in \R^+, \exists\ N \in \Z^+ : \forall n \geq N, d_X(x^{(n)}, x_0) < \delta.
  \]
  Since \((x^{(n)})_{n = 1}^\infty\) is in \(E\), we have
  \[
    \forall \varepsilon \in \R^+, \exists\ \delta \in \R^+ : \begin{dcases}
      \exists\ N \in \Z^+ : \forall n \geq N, d_X(x^{(n)}, x_0) < \delta \\
      d_X(x^{(n)}, x_0) < \delta \implies d_Y\big(f(x^{(n)}), L\big) < \varepsilon
    \end{dcases}
  \]
  and
  \[
    \forall \varepsilon \in \R^+, \exists\ N \in \Z^+ : \forall n \geq N, d_Y\big(f(x^{(n)}, L)\big) < \varepsilon.
  \]
  By \cref{1.1.14} we have \(\lim_{n \to \infty} d_Y\big(f(x^{(n)}), L\big) = 0\).
  Since \((x^{(n)})_{n = 1}^\infty\) is arbitrary, we conclude that (a) implies (b).

  Next we show that statement (b) implies statement (a).
  Suppose that if \((x^{(n)})_{n = 1}^\infty\) is a sequence in \(X\) such that \(\lim_{n \to \infty} d_X(x^{(n)}, x_0) = 0\), then \(\lim_{n \to \infty} d_Y\big(f(x), L\big) = 0\).
  Suppose for sake of contradiction that
  \[
    d_Y - \lim_{x \to x_0 ; x \in X} f(x) \neq L.
  \]
  Then by \cref{3.1.1} we have
  \[
    \exists\ \varepsilon \in \R^+ : \forall \delta \in \R^+, \exists\ x \in X : \begin{dcases}
      d_X(x, x_0) < \delta \\
      d_Y\big(f(x), L\big) \geq \varepsilon
    \end{dcases}
  \]
  Thus we can choose one sequence \((x^{(n)})_{n = 1}^\infty\) which satsifies
  \[
    \forall n \in \Z^+, \begin{dcases}
      d_X(x^{(n)}, x_0) < \dfrac{1}{n} \\
      d_Y\big(f(x^{(n)}), L\big) \geq \varepsilon
    \end{dcases}
  \]
  By squeeze test we have \(\lim_{n \to \infty} d_X(x^{(n)}, x_0) = 0\).
  But by hypothesis we know that \(\lim_{n \to \infty} d_Y\big(f(x^{(n)}), L\big) = 0\), which means
  \[
    \exists\ N \in \Z^+ : \forall n \geq N, d_Y\big(f(x^{(n)}), L\big) < \varepsilon,
  \]
  a contradiction.
  Thus we have
  \[
    d_Y - \lim_{x \to x_0 ; x \in X} f(x) = L
  \]
  and we conclude that statements (a)(b) are equivalent.

  Next we show that statement (a) implies statement (c).
  Suppose that
  \[
    d_Y - \lim_{x \to x_0 ; x \in E} f(x) = L.
  \]
  By \cref{3.1.1} we have
  \begin{align*}
             & \forall \varepsilon \in \R^+, \exists\ \delta \in \R^+ : \Big(\forall x \in E, d_X(x, x_0) < \delta \implies d_Y\big(f(x), L\big) < \varepsilon\Big)   \\
    \implies & \forall \varepsilon \in \R^+, \exists\ \delta \in \R^+ : \Big(x \in B_{(X, d_X)}(x_0, \delta) \cap E \implies d_Y\big(f(x), L\big) < \varepsilon\Big).
  \end{align*}
  Let \(V\) be an open set in \((Y, d_Y)\) such that \(L \in V\).
  Then we have
  \begin{align*}
             & V = \text{int}_{(Y, d_Y)}(V)                                                                           &  & \text{(by \cref{1.2.15}(a))} \\
    \implies & \exists\ \varepsilon \in \R^+ : B_{(Y, d_Y)}(L, \varepsilon) \subseteq V                               &  & \by{1.2.5}                   \\
    \implies & \exists\ \delta \in \R^+ :                                                                                                               \\
             & \begin{dcases}
                 x \in B_{(X, d_X)}(x_0, \delta) \cap E \implies d_Y\big(f(x), L\big) < \varepsilon \\
                 f\big(B_{(X, d_X)}(x_0, \delta) \cap E\big) \subseteq B_{(Y, d_Y)}\big(L, \varepsilon\big) \subseteq V
               \end{dcases}
  \end{align*}
  and by \cref{1.2.15}(c) we know that \(B_{(X, d_X)}(x_0, \delta)\) is open in \((X, d_X)\).
  Since \(V\) is arbitrary, we conclude that statement (a) implies statement (c).

  Next we show that statement (c) implies statement (a).
  Suppose that
  \[
    \forall V \subseteq Y, \begin{dcases}
      L \in V \\
      V \text{ is open in } (Y, d_Y)
    \end{dcases} \implies \exists\ U \subseteq X : \begin{dcases}
      x_0 \in U                      \\
      U \text{ is open in } (X, d_X) \\
      f(U \cap E) \subseteq V
    \end{dcases}
  \]
  Let \(\varepsilon \in \R^+\).
  By \cref{1.2.15}(c) we know that \(B_{(Y, d_Y)}(L, \varepsilon)\) is open in \((Y, d_Y)\).
  By hypothesis we know that there exists some \(U \subseteq X\) such that
  \[
    \begin{dcases}
      x_0 \in U                      \\
      U \text{ is open in } (X, d_X) \\
      f(U \cap E) \subseteq B_{(Y, d_Y)}(L, \varepsilon)
    \end{dcases}
  \]
  Then we have
  \begin{align*}
             & \begin{dcases}
                 x_0 \in U \\
                 U = \text{int}_{(X, d_X)}(U)
               \end{dcases}                                                                             &  & \text{(by \cref{1.2.15}(a))} \\
    \implies & \exists\ \delta \in \R^+ : B_{(X, d_X)}(x_0, \delta) \subseteq U                                         &  & \by{1.2.5}   \\
    \implies & \exists\ \delta \in \R^+ : B_{(X, d_X)}(x_0, \delta) \cap E \subseteq U \cap E                                             \\
    \implies & \exists\ \delta \in \R^+ :                                                                                                 \\
             & f\big(B_{(X, d_X)}(x_0, \delta) \cap E\big) \subseteq f(U \cap E) \subseteq B_{(Y, d_Y)}(L, \varepsilon)                   \\
    \implies & \exists\ \delta \in \R^+ :                                                                                                 \\
             & \Big(\forall x \in E, d_X(x, x_0) < \delta \implies d_Y\big(f(x), L\big) < \varepsilon\Big).
  \end{align*}
  Since \(\varepsilon\) is arbitrary, by \cref{3.1.1} we have
  \[
    d_Y - \lim_{x \to x_0 ; x \in E} f(x) = L
  \]
  and we conclude that statements (a)(c) are equivalent.

  Next we show that statement (a) implies statement (d).
  Suppose that
  \[
    d_Y - \lim_{x \to x_0 ; x \in E} f(x) = L.
  \]
  Then by \cref{3.1.1} we have
  \[
    \forall \varepsilon \in \R^+, \exists\ \delta \in \R^+ : \Big(\forall x \in E, d_X(x, x_0) < \delta \implies d_Y\big(f(x), L\big) < \varepsilon\Big).
  \]
  Let \(g : E \cup \{x_0\} \to Y\) be a function where
  \[
    \forall x \in E \cup \{x_0\}, g(x) = \begin{dcases}
      L    & \text{if } x = x_0    \\
      f(x) & \text{if } x \neq x_0
    \end{dcases}
  \]
  Then we have
  \begin{align*}
             & \forall \varepsilon \in \R^+, \exists\ \delta \in \R^+ :                                                       \\
             & \Big(\forall x \in E, d_X(x, x_0) < \delta \implies d_Y\big(f(x), L\big) < \varepsilon\Big)                    \\
    \implies & \forall \varepsilon \in \R^+, \exists\ \delta \in \R^+ :                                                       \\
             & \Big(\forall x \in E \cup \{x_0\}, d_X(x, x_0) < \delta \implies d_Y\big(g(x), g(x_0)\big) < \varepsilon\Big).
  \end{align*}
  Thus by \cref{2.1.1} \(g\) is continuous at \(x_0\) from \((E \cup \{x_0\}, d_X|_{(E \cup \{x_0\}) \times (E \cup \{x_0\})})\) to \((Y, d_Y)\).

  Now suppose that \(x_0 \in E\).
  We claim that \(f(x_0) = L\).
  Suppose for sake of contradiction that \(f(x_0) \neq L\).
  Then by \cref{1.1.2}(b) we have \(d_Y\big(f(x_0), L\big) > 0\).
  Let \(r = d_Y\big(f(x_0), L\big)\).
  By \cref{3.1.1} we have
  \[
    \exists\ \delta \in \R^+ : \forall x \in E, d_X(x, x_0) < \delta \implies d_Y\big(f(x), L\big) < r.
  \]
  Since \(x_0 \in E\), we have \(d_X(x_0, x_0) = 0 < \delta\).
  But then we have \(d_Y\big(f(x_0), L\big) < r = d_Y\big(f(x_0), L\big)\), a contradiction.
  Thus we have \(f(x_0) = L\).

  Finally we show that statement (d) implies statement (a).
  Suppose that \(g : E \cup \{x_0\} \to Y\) is a function where
  \[
    \forall x \in E \cup \{x_0\}, g(x) = \begin{dcases}
      L    & \text{if } x = x_0    \\
      f(x) & \text{if } x \neq x_0
    \end{dcases}
  \]
  and \(g\) is continuous from \((E \cup \{x_0\}, d_X|_{(E \cup \{x_0\}) \times (E \cup \{x_0\})})\) to \((Y, d_Y)\).
  Suppose also that if \(x_0 \in E\), then \(f(x_0) = L\).
  Then by \cref{2.1.1} we have
  \begin{align*}
             & \forall \varepsilon \in \R^+, \exists\ \delta \in \R^+ :                                                       \\
             & \Big(\forall x \in E \cup \{x_0\}, d_X(x, x_0) < \delta \implies d_Y\big(g(x), g(x_0)\big) < \varepsilon\Big)  \\
    \implies & \forall \varepsilon \in \R^+, \exists\ \delta \in \R^+ :                                                       \\
             & \begin{dcases}
                 \forall x \in E \setminus \{x_0\}, d_X(x, x_0) < \delta \implies d_Y\big(f(x), L\big) < \varepsilon \\
                 x_0 \in E \implies d_X(x_0, x_0) = 0 < \delta \implies f(x_0) = L \implies d_Y\big(f(x_0), L\big) < \varepsilon
               \end{dcases} \\
    \implies & \forall \varepsilon \in \R^+, \exists\ \delta \in \R^+ :                                                       \\
             & \Big(\forall x \in E, d_X(x, x_0) < \delta \implies d_Y\big(f(x), L\big) < \varepsilon\Big).
  \end{align*}
  By \cref{3.1.1} this means
  \[
    d_Y - \lim_{x \to x_0 ; x \in E} f(x) = L.
  \]
  We conclude that statements (a)(b)(c)(d) are all equivalent.
\end{proof}

\begin{rmk}\label{3.1.6}
  Observe from \cref{3.1.5}(b) and \cref{1.1.20} that a function \(f(x)\) can converge to at most one limit \(L\) as \(x\) converges to \(x_0\).
  In other words, if the limit
  \[
    \lim_{x \to x_0 ; x \in E} f(x)
  \]
  exists at all, then it can only take at most one value.
\end{rmk}

\begin{rmk}\label{3.1.7}
  The requirement that \(x_0\) be an adherent point of \(E\) is necessary for the concept of limiting value to be useful, otherwise \(x_0\) will lie in the exterior of \(E\), the notion that \(f(x)\) converges to \(L\) as \(x\) converges to \(x_0\) in \(E\) is vacuous
  (for \(\delta\) sufficiently small, there are no points \(x \in E\) so that \(d(x, x_0) < \delta\)).
\end{rmk}

\begin{rmk}\label{3.1.8}
  Strictly speaking, we should write
  \[
    d_Y - \lim_{x \to x_0 ; x \in E} f(x) \text{ instead of } \lim_{x \to x_0 ; x \in E} f(x),
  \]
  since the convergence depends on the metric \(d_Y\).
  However in practice it will be obvious what the metric \(d_Y\) is and so we will omit the \(d_Y -\) prefix from the notation.
\end{rmk}

\exercisesection

\begin{ex}\label{ex:3.1.1}
  Let \((X, d_X)\) and \((Y, d_Y)\) be metric spaces, let \(E\) be a subset of \(X\), let \(f : E \to Y\) be a function, and let \(x_0\) be an element of \(E\).
  Assume that \(x_0\) is an adherent point of \(E \setminus \{x_0\}\)
  (or equivalently, that \(x_0\) is not an \emph{isolated point} of \(E\)).
  Show that the limit \(\lim_{x \to x_0 ; x \in E} f(x)\) exists if and only if the limit \(\lim_{x \to x_0 ; x \in E \setminus \{x_0\}} f(x)\) exists and is equal to \(f(x_0)\).
  Also, show that if the limit \(\lim_{x \to x_0 ; x \in E} f(x)\) exists at all, then it must equal \(f(x_0)\).
\end{ex}

\begin{proof}
  Let \(L \in Y\).
  By \cref{1.1.2}(a) we know that
  \[
    \forall \varepsilon \in \R^+, d_Y\big(f(x_0), L\big) < \varepsilon \iff L = f(x_0).
  \]
  Thus we have
  \begin{align*}
         & d_Y - \lim_{x \to x_0 ; x \in E \setminus \{x_0\}} f(x) = f(x_0)                                                                                                    \\
    \iff & \forall \varepsilon \in \R^+, \exists\ \delta \in \R^+ :                                                                                                            \\
         & \Big(\forall x \in E \setminus \{x_0\}, d_X(x, x_0) < \delta \implies d_Y\big(f(x), f(x_0)\big) < \varepsilon\Big) &                                   & \by{3.1.1} \\
    \iff & \forall \varepsilon \in \R^+, \exists\ \delta \in \R^+ :                                                                                                            \\
         & \Big(\forall x \in E, d_X(x, x_0) < \delta \implies d_Y\big(f(x), f(x_0)\big) < \varepsilon\Big)                   & (E \setminus \{x_0\} \subseteq E)              \\
    \iff & d_Y - \lim_{x \to x_0 ; x \in E} f(x) = f(x_0).                                                                    &                                   & \by{3.1.1}
  \end{align*}
\end{proof}

\begin{ex}\label{ex:3.1.2}
  Prove \cref{3.1.5}.
\end{ex}

\begin{proof}
  See \cref{3.1.5}.
\end{proof}

\begin{ex}\label{ex:3.1.3}
  Use \cref{3.1.5}(c) to define a notion of a limiting value of a function \(f : E \to Y\) from one topological space \((X, \mathcal{F}_X)\) to another \((Y, \mathcal{F}_Y)\) where \(E \subseteq X\).
  If \(X\) is a topological space and \(Y\) is a Hausdorff topological space (see \cref{ex:2.5.4}), prove the equivalence of \cref{3.1.5}(c)(d) in this setting, as well as an analogue of \cref{3.1.6}.
  What happens to these statements of \(Y\) is not Hausdorff?
\end{ex}

\begin{proof}
  Let \((X, \mathcal{F}_X)\), \((Y, \mathcal{F}_Y)\) be topological spaces, let \(E \subseteq X\), let \(f : E \to Y\) be a function, let \(x_0 \in \overline{E}_{(X, \mathcal{F}_X)}\), and let \(L \in Y\).
  We say that \(f(x)\) converges to \(L\) in \(Y\) as \(x\) converges to \(x_0\) in \(E\) iff
  \[
    \forall V \in \mathcal{F}_Y, L \in V \implies \exists\ U \in \mathcal{F}_X : \begin{dcases}
      x_0 \in U \\
      f(U \cap E) \subseteq V
    \end{dcases}
  \]
  We want to show that if \((Y, \mathcal{F}_Y)\) is Hausdorff, then the definition above is equivalent to the follow:
  If \(g : E \cup \{x_0\} \to Y\) is a function such that
  \[
    \forall x \in E \cup \{x_0\}, g(x) = \begin{dcases}
      L    & \text{if } x = x_0    \\
      f(x) & \text{if } x \neq x_0
    \end{dcases}
  \]
  and \((E \cup \{x_0\}, \mathcal{F}_{E \cup \{x_0\}})\) is a topological subspace induced by \((X, \mathcal{F}_X)\), then \(g\) is continuous at \(x_0\) from \((E \cup \{x_0\}, \mathcal{F}_{E \cup \{x_0\}})\) to \((Y, \mathcal{F}_Y)\).

  First suppose that \(f(x)\) converges to \(L\) in \(Y\) as \(x\) converges to \(x_0\) in \(E\).
  Let \(g\) be the function in the definition and let \(V \in \mathcal{F}_Y\) such that \(L \in V\).
  By hypothesis we know that
  \[
    \exists\ U \in \mathcal{F}_X : \begin{dcases}
      x_0 \in U \\
      f(U \cap E) \subseteq V
    \end{dcases}
  \]
  Then by \cref{2.5.7} we have \(U \cap (E \cup \{x_0\}) \in \mathcal{F}_{E \cup \{x_0\}}\) and
  \begin{align*}
    g\big(U \cap (E \cup \{x_0\})\big) & = g\big((U \cap E) \cup \{x_0\}\big)                      \\
                                       & = g\big((U \cap E) \setminus \{x_0\}\big) \cup g(\{x_0\}) \\
                                       & = f\big((U \cap E) \setminus \{x_0\}\big) \cup \{L\}      \\
                                       & \subseteq f(U \cap E) \cup \{L\}                          \\
                                       & \subseteq V.
  \end{align*}
  Since \(V\) is arbitrary, by \cref{2.5.8} we know that \(g\) is continuous at \(x_0\) from \((E \cup \{x_0\}, \mathcal{F}_{E \cup \{x_0\}})\) to \((Y, \mathcal{F}_Y)\).

  Next suppose that \(x_0 \in E\) and \(f(x)\) converges to \(L\) in \(Y\) as \(x\) converges to \(x_0\) in \(E\).
  We want to show that \(f(x_0) = L\).
  Suppose for sake of contradiction that \(f(x_0) \neq L\).
  Since \((Y, \mathcal{F}_Y)\) is Hausdorff, by \cref{ex:2.5.4} we know that
  \[
    \exists\ V, W \in \mathcal{F}_Y : \begin{dcases}
      L \in V      \\
      f(x_0) \in W \\
      V \cap W = \emptyset
    \end{dcases}
  \]
  By definition we have
  \[
    \exists\ U_V, U_W \in \mathcal{F}_X : \begin{dcases}
      x_0 \in U_V               \\
      x_0 \in U_W               \\
      f(U_V \cap E) \subseteq V \\
      f(U_W \cap E) \subseteq W
    \end{dcases}
  \]
  By \cref{2.5.1} we know that \(U_V \cap U_W \in \mathcal{F}_X\).
  But then we have
  \[
    \begin{dcases}
      x_0 \in U_V \cap U_W                                       \\
      f(U_V \cap U_W \cap E) \subseteq f(U_V \cap E) \subseteq V \\
      f(U_V \cap U_W \cap E) \subseteq f(U_W \cap E) \subseteq W
    \end{dcases}
  \]
  which means \(V \cap W \neq \emptyset\), a contradiction.
  Thus we have \(f(x_0) = L\).

  Now suppose that \(g\) is the function in the definition such that \(g\) is continuous at \(x_0\) from \((E \cup \{x_0\}, \mathcal{F}_{E \cup \{x_0\}})\) to \((Y, \mathcal{F}_Y)\).
  Also suppose that if \(x_0 \in E\), then \(f(x_0) = L\).
  Let \(V \in \mathcal{F}_Y\) such that \(g(x_0) = L \in V\).
  By \cref{2.5.8} we know that
  \[
    \exists\ U \in \mathcal{F}_{E \cup \{x_0\}} : \begin{dcases}
      x_0 \in U \\
      g(U) \subseteq V
    \end{dcases}
  \]
  By \cref{2.5.7} we know that
  \[
    \exists\ U_X \in \mathcal{F}_X : U_X \cap (E \cup \{x_0\}) = U.
  \]
  Since \(x_0 \in U\), we know that \(x_0 \in U_X\).
  Thus we have
  \begin{align*}
    f(U_X \cap E) & = f\big((U_X \cap E) \setminus \{x_0\}\big) \cup f(E \cap \{x_0\})  & (f(E \cap \{x_0\}) = \emptyset \iff x_0 \notin E) \\
                  & \subseteq g\big((U_X \cap E) \setminus \{x_0\}\big) \cup g(\{x_0\}) & (x_0 \in E \iff f(x_0) = L = g(x_0))              \\
                  & = g\big(U_X \cap (E \cup \{x_0\})\big)                                                                                  \\
                  & = g(U)                                                                                                                  \\
                  & \subseteq V.
  \end{align*}
  Since \(V\) is arbitrary, we conclude that \(f(x)\) converges to \(L\) in \(Y\) as \(x\) converges to \(x_0\) in \(E\).

  If \((Y, \mathcal{F}_Y)\) is not Hausdorff, then \(x_0 \in E\) may not implies \(f(x_0) = L\).
\end{proof}

\begin{ex}\label{ex:3.1.4}
  Recall from \cref{ex:2.5.5} that the extended real line \(\R^*\) comes with a standard topology (the order topology).
  We view the natural numbers \(\N\) as a subspace of this topological space, and \(+\infty\) as an adherent point of \(\N\) in \(\R^*\).
  Let \((a_n)_{n = 0}^\infty\) be a sequence taking values in a topological space \((Y, \mathcal{F}_Y)\), and let \(L \in Y\).
  Show that \(\lim_{n \to +\infty ; n \in \N} a_n = L\) (in the sense of \cref{ex:3.1.3}) if and only if \(\lim_{n \to \infty} a_n = L\) (in the sense of \cref{2.5.4}).
  This shows that the notions of limiting values of a sequence, and limiting values of a function, are compatible.
\end{ex}

\begin{proof}
  Let \((\R^*, \mathcal{F}_{\R^*})\) be the order topology in \cref{ex:2.5.5}.
  Let \(f : \N \to Y\) be the function where \(f(n) = a_n\) for each \(n \in \N\).
  First suppose that
  \[
    \lim_{n \to +\infty ; n \in \N} a_n = \lim_{n \to +\infty ; n \in \N} f(n) = L.
  \]
  By \cref{ex:3.1.3} we have
  \[
    \forall V \in \mathcal{F}_Y, L \in V \implies \exists\ U \in \mathcal{F}_{\R^*} : \begin{dcases}
      +\infty \in U \\
      f(U \cap \N) \subseteq V
    \end{dcases}
  \]
  Since \(U \in \mathcal{F}_{\R^*}\), by \cref{ex:2.5.5} we know that there exists an interval \(I \subseteq \R^*\) such that \(I \subseteq U\) and \(+\infty \in I\).
  We know that \(I\) must in the form \((a, +\infty]\) for some \(a \in \R\).
  By Archimedean property we know that there exists some \(N \in \N\) such that \(N > a\).
  Then we have
  \begin{align*}
             & \begin{dcases}
                 I \subseteq U \subseteq \R^*            \\
                 I = (a, +\infty] \in \mathcal{F}_{\R^*} \\
                 N > a
               \end{dcases}     \\
    \implies & \forall n \geq N + 1, n \in I               \\
    \implies & \forall n \geq N + 1, n \in U \cap \N       \\
    \implies & \forall n \geq N + 1, f(n) \in f(U \cap \N) \\
    \implies & \forall n \geq N + 1, f(n) \in V.
  \end{align*}
  Since this is true for arbitrary \(V\), we have
  \[
    \lim_{n \to \infty} a_n = \lim_{n \to \infty} f(n) = L
  \]
  in the sense of \cref{2.5.4}.

  Now suppose that
  \[
    \lim_{n \to \infty} a_n = \lim_{n \to \infty} f(n) = L
  \]
  in the sense of \cref{2.5.4}.
  Then we have
  \[
    \forall V \in \mathcal{F}_Y, L \in V \implies \exists\ N \in \N : \forall n \geq N, f(n) \in V.
  \]
  Let \(I = (N, +\infty]\).
  Then we know that \(I\) is an interval of \(\R^*\) and by \cref{ex:2.5.5} we have \(I \in \mathcal{F}_{\R^*}\).
  Observe that
  \begin{align*}
             & I \cap \N = \{m \in \N : m \geq N + 1\} \\
    \implies & \forall n \in I \cap \N, f(n) \in V     \\
    \implies & f(I \cap \N) \subseteq V.
  \end{align*}
  Since this is true for arbitrary \(V \in \mathcal{F}_Y\), by \cref{ex:3.1.3} we have
  \[
    \lim_{n \to +\infty ; n \in \N} a_n = \lim_{n \to +\infty ; n \in \N} f(n) = L.
  \]
\end{proof}

\begin{ex}\label{ex:3.1.5}
  Let \((X, d_X)\), \((Y, d_Y)\), \((Z, d Z)\) be metric spaces, and let \(x_0 \in X\), \(y_0 \in Y\), \(z_0 \in Z\).
  let \(E \subseteq X\) and let \(f : E \to Y\) and \(g : Y \to Z\) be functions.
  If we have \(\lim_{x \to x_0 ; x \in E} f(x) = y_0\) and \(\lim_{y \to y_0 ; y \in f(E)} g(y) = z_0\), conclude that \(\lim_{x \to x_0 ; x \in E} g \circ f(x) = z_0\).
\end{ex}

\begin{proof}
  By \cref{3.1.1} we have
  \begin{align*}
             & d_Z - \lim_{y \to y_0 ; y \in f(E)} g(y) = z_0                                                    \\
    \implies & \forall \varepsilon \in \R^+, \exists\ \delta' \in \R^+ :                                         \\
             & \Big(\forall y \in f(E), d_Y(y, y_0) < \delta' \implies d_Z\big(g(y), z_0\big) < \varepsilon\Big)
  \end{align*}
  and
  \begin{align*}
             & d_Y - \lim_{x \to x_0 ; x \in E} f(x) = y_0                                                \\
    \implies & \forall \delta' \in \R^+, \exists\ \delta \in \R^+ :                                       \\
             & \Big(\forall x \in E, d_X(x, x_0) < \delta \implies d_Y\big(f(x), y_0\big) < \delta'\Big).
  \end{align*}
  Thus we have
  \begin{align*}
     & \forall \varepsilon \in \R^+, \exists\ \delta \in \R^+ :                                                                                             \\
     & \bigg(\forall x \in E, d_X(x, x_0) < \delta \implies d_Y\big(f(x), y_0\big) < \delta' \implies d_Z\Big(g\big(f(x)\big), z_0\Big) < \varepsilon\bigg)
  \end{align*}
  and by \cref{3.1.1} \(d_Z - \lim_{x \to x_0 ; x \in E} g \circ f(x) = z_0\).
\end{proof}

\begin{ex}\label{ex:3.1.6}
  State and prove an analogue of the limit laws in Proposition 9.3.14 in Analysis I when \(X\) is now a metric space rather than a subset of \(\R\).
\end{ex}

\begin{proof}
  Let \((X, d)\) be a metric space, let \(d_1 = d_{l^1}|_{\R \times \R}\), let \(E \subseteq X\), let \(x_0 \in \overline{E}_{(X, d)}\), let \(f : X \to \R\) and \(g : X \to \R\) be functions, and let \(c \in \R\).
  Suppose that
  \begin{align*}
     & d_1 - \lim_{x \to x_0 ; x \in E} f(x) = L \\
     & d_1 - \lim_{x \to x_0 ; x \in E} g(x) = M
  \end{align*}
  We want to show that
  \begin{align*}
     & d_1 - \lim_{x \to x_0 ; x \in E} (f + g)(x) = L + M                                          \\
     & d_1 - \lim_{x \to x_0 ; x \in E} (f - g)(x) = L - M                                          \\
     & d_1 - \lim_{x \to x_0 ; x \in E} (fg)(x) = LM                                                \\
     & d_1 - \lim_{x \to x_0 ; x \in E} \min(f, g)(x) = \min(L, M)                                  \\
     & d_1 - \lim_{x \to x_0 ; x \in E} \max(f, g)(x) = \max(L, M)                                  \\
     & d_1 - \lim_{x \to x_0 ; x \in E} (cf)(x) = cL                                                \\
     & d_1 - \lim_{x \to x_0 ; x \in E} (f / g)(x) = L / M \text{ if } \begin{dcases}
                                                                         \forall x \in E, g(x) \neq 0 \\
                                                                         M \neq 0
                                                                       \end{dcases}
  \end{align*}
  Let \(f^* : E \cup \{x_0\} \to \R\) be the function
  \[
    \forall x \in E, f^*(x) = \begin{dcases}
      L    & \text{if } x = x_0    \\
      f(x) & \text{if } x \neq x_0
    \end{dcases}
  \]
  and let \(g^* : E \cup \{x_0\} \to \R\) be the function
  \[
    \forall x \in E, g^*(x) = \begin{dcases}
      M    & \text{if } x = x_0    \\
      g(x) & \text{if } x \neq x_0
    \end{dcases}
  \]
  By \cref{3.1.5}(c) we know that \(f^*\) and \(g^*\) are continuous at \(x_0\) from \((X, d)\) to \((\R, d_1)\).
  Thus by \cref{2.2.3} we have
  \begin{align*}
     & f^* + g^* \text{ is continuous at } x_0 \text{ from } (X, d) \text{ to } (\R, d_1)                                          \\
     & f^* - g^* \text{ is continuous at } x_0 \text{ from } (X, d) \text{ to } (\R, d_1)                                          \\
     & f^* g^* \text{ is continuous at } x_0 \text{ from } (X, d) \text{ to } (\R, d_1)                                            \\
     & \min(f^*, g^*) \text{ is continuous at } x_0 \text{ from } (X, d) \text{ to } (\R, d_1)                                     \\
     & \max(f^*, g^*) \text{ is continuous at } x_0 \text{ from } (X, d) \text{ to } (\R, d_1)                                     \\
     & c f^* \text{ is continuous at } x_0 \text{ from } (X, d) \text{ to } (\R, d_1)                                              \\
     & f^* / g^* \text{ is continuous at } x_0 \text{ from } (X, d) \text{ to } (\R, d_1) \text{ if } \begin{dcases}
                                                                                                        \forall x \in E, g(x) \neq 0 \\
                                                                                                        M \neq 0
                                                                                                      \end{dcases}
  \end{align*}
  Since
  \begin{align*}
    \forall x \in E \cup \{x_0\}, & (f^* + g^*)(x) = \begin{dcases}
                                                       L + M       & \text{if } x = x_0    \\
                                                       f(x) + g(x) & \text{if } x \neq x_0
                                                     \end{dcases}                     \\
                                  & (f^* - g^*)(x) = \begin{dcases}
                                                       L - M       & \text{if } x = x_0    \\
                                                       f(x) - g(x) & \text{if } x \neq x_0
                                                     \end{dcases}                     \\
                                  & (f^* g^*)(x) = \begin{dcases}
                                                     LM        & \text{if } x = x_0    \\
                                                     f(x) g(x) & \text{if } x \neq x_0
                                                   \end{dcases}                         \\
                                  & \min(f^*, g^*)(x) = \begin{dcases}
                                                          \min(L, M)               & \text{if } x = x_0    \\
                                                          \min\big(f(x), g(x)\big) & \text{if } x \neq x_0
                                                        \end{dcases}     \\
                                  & \max(f^*, g^*)(x) = \begin{dcases}
                                                          \max(L, M)               & \text{if } x = x_0    \\
                                                          \max\big(f(x), g(x)\big) & \text{if } x \neq x_0
                                                        \end{dcases}     \\
                                  & (c f^*)(x) = \begin{dcases}
                                                   cL     & \text{if } x = x_0    \\
                                                   c f(x) & \text{if } x \neq x_0
                                                 \end{dcases}                              \\
                                  & (f^* / g^*)(x) = \begin{dcases}
                                                       L / M       & \text{if } x = x_0    \\
                                                       f(x) / g(x) & \text{if } x \neq x_0
                                                     \end{dcases} \text{ when } \begin{dcases}
                                                                                  \forall x \in E, g(x) \neq 0 \\
                                                                                  M \neq 0
                                                                                \end{dcases}
  \end{align*}
  by \cref{3.1.5}(a)(d) we know that
  \begin{align*}
     & d_1 - \lim_{x \to x_0 ; x \in E} (f + g)(x) = L + M                                          \\
     & d_1 - \lim_{x \to x_0 ; x \in E} (f - g)(x) = L - M                                          \\
     & d_1 - \lim_{x \to x_0 ; x \in E} (fg)(x) = LM                                                \\
     & d_1 - \lim_{x \to x_0 ; x \in E} \min(f, g)(x) = \min(L, M)                                  \\
     & d_1 - \lim_{x \to x_0 ; x \in E} \max(f, g)(x) = \max(L, M)                                  \\
     & d_1 - \lim_{x \to x_0 ; x \in E} (cf)(x) = cL                                                \\
     & d_1 - \lim_{x \to x_0 ; x \in E} (f / g)(x) = L / M \text{ if } \begin{dcases}
                                                                         \forall x \in E, g(x) \neq 0 \\
                                                                         M \neq 0
                                                                       \end{dcases}
  \end{align*}
\end{proof}
\section{Pointwise and uniform convergence}\label{ii:sec:3.2}

\begin{defn}[Pointwise convergence]\label{ii:3.2.1}
  Let \((f^{(n)})_{n = 1}^\infty\) be a sequence of functions from one metric space \((X, d_X)\) to another \((Y, d_Y)\), and let \(f : X \to Y\) be another function.
  We say that \emph{\((f^{(n)})_{n = 1}^\infty\) converges pointwise to \(f\) on \(X\)} if we have
  \[
    \lim_{n \to \infty} f^{(n)}(x) = f(x)
  \]
  for all \(x \in X\), i.e.,
  \[
    \lim_{n \to \infty} d_Y\big(f^{(n)}(x), f(x)\big) = 0.
  \]
  Or in other words, for every \(x\) and every \(\varepsilon > 0\) there exists \(N > 0\) such that \(d_Y\big(f^{(n)}(x), f(x)\big) < \varepsilon\) for every \(n > N\).
  We call the function \(f\) the \emph{pointwise limit} of the functions \(f^{(n)}\).
\end{defn}

\begin{rmk}\label{ii:3.2.2}
  Note that \(f^{(n)}(x)\) and \(f(x)\) are points in \(Y\), rather than functions, so we are using our prior notion of convergence in metric spaces to determine convergence of functions.
  Also note that we are not really using the fact that \((X, d_X)\) is a metric space
  (i.e., we are not using the metric \(d_X\));
  for this definition it would suffice for \(X\) to just be a plain old set with no metric structure.
  However, later on we shall want to restrict our attention to \emph{continuous} functions from \(X\) to \(Y\), and in order to do so we need a metric on \(X\) (and on \(Y\)), or at least a topological structure.
  Also when we introduce the concept of \emph{uniform convergence}, then we will definitely need a metric structure on \(X\) and \(Y\);
  there is no comparable notion for topological spaces.
\end{rmk}

\begin{note}
  From \cref{ii:1.1.20} we see that a sequence \((f^{(n)})_{n = 1}^\infty\) of functions from one metric space \((X, d_X)\) to another \((Y, d_Y)\) can have at most one pointwise limit \(f\)
  (this explains why we can refer to \(f\) as \emph{the} pointwise limit).
  However, it is of course possible for a sequence of functions to have no pointwise limit, just as a sequence of points in a metric space do not necessarily have a limit.
\end{note}

\begin{note}
  Pointwise convergence is a very natural concept, but it has a number of disadvantages:
  it does not preserve continuity, derivatives, limits, or integrals.
  The problem is that while \(f^{(n)}(x)\) converges to \(f(x)\) for each \(x\), the \emph{rate} of that convergence varies substantially with \(x\).
  To put things another way, the convergence of \(f^{(n)}\) to \(f\) is not \emph{uniform} in \(x\)
  - the \(N\) that one needs to get \(f^{(n)}(x)\) within \(\varepsilon\) of \(f\) depends on \(x\) as well as on \(\varepsilon\).
  This motivates a stronger notion of convergence.
\end{note}

\setcounter{thm}{6}
\begin{defn}[Uniform convergence]\label{ii:3.2.7}
  Let \((f^{(n)})_{n = 1}^\infty\) be a sequence of functions from one metric space \((X, d_X)\) to another \((Y, d_Y)\), and let \(f : X \to Y\) be another function.
  We say that \emph{\((f^{(n)})_{n = 1}^\infty\) converges uniformly to \(f\) on \(X\)} if for every \(\varepsilon > 0\) there exists \(N > 0\) such that \(d_Y\big(f^{(n)}(x), f(x)\big) < \varepsilon\) for every \(n > N\) and \(x \in X\).
  We call the function \(f\) the \emph{uniform limit} of the functions \(f^{(n)}\).
\end{defn}

\begin{rmk}\label{ii:3.2.8}
  Note that \cref{ii:3.2.7} is subtly different from the definition for pointwise convergence in \cref{ii:3.2.1}.
  In the definition of pointwise convergence, \(N\) was allowed to depend on \(x\);
  now it is not.
  The reader should compare this distinction to the distinction between continuity and uniform continuity
  (i.e., between \cref{ii:2.1.1} and \cref{ii:2.3.4}).
\end{rmk}

\begin{note}
  If \(f^{(n)}\) converges uniformly to \(f\) on \(X\), then it also converges pointwise to the same function \(f\).
  Thus, when the uniform limit and pointwise limit both exist, then they have to be equal.
  However, the converse is not true.
\end{note}

\begin{note}
  If a sequence \(f^{(n)} : X \to Y\) of functions converges pointwise (or uniformly) to a function \(f : X \to Y\), then the restrictions \(f^{(n)}|_E : E \to Y\) of \(f^{(n)}\) to some subset \(E\) of \(X\) will also converge pointwise (or uniformly) to \(f|_E\).
\end{note}

\exercisesection

\begin{ex}\label{ii:ex:3.2.1}
  The purpose of this exercise is to demonstrate a concrete relationship between continuity and pointwise convergence, and between uniform continuity and uniform convergence.
  Let \(f : \R \to \R\) be a function.
  For any \(a \in \R\), let \(f_a : \R \to \R\) be the shifted function \(f_a(x) \coloneqq f(x - a)\).
  \begin{enumerate}
    \item Show that \(f\) is continuous iff, whenever \((a_n)_{n = 0}^\infty\) is a sequence of real numbers which converges to zero, the shifted functions \(f_{a_n}\) converge pointwise to \(f\).
    \item Show that \(f\) is uniformly continuous iff, whenever \((a_n)_{n = 0}^\infty\) is a sequence of real numbers which converges to zero, the shifted functions \(f_{a_n}\) converge uniformly to \(f\).
  \end{enumerate}
\end{ex}

\begin{proof}{(a)}
  Suppose that \(f\) is continuous on \(\R\).
  Let \((a_n)_{n = 0}^\infty\) be a sequence in \(\R\) such that \(\lim_{n \to \infty} a_n = 0\).
  Let \(x_0 \in \R\).
  Then we have
  \begin{align*}
             & \lim_{n \to \infty} a_n = 0                                                                                                          \\
    \implies & \lim_{n \to \infty} (x_0 - a_n) = x_0                                                                                                \\
    \implies & \lim_{n \to \infty} f_{a_n}(x_0) = \lim_{n \to \infty} f(x_0 - a_n) = f(x_0). &  & \text{(\(f\) is continuous at \(x_0\) on \(\R\))}
  \end{align*}
  Since \(x_0\) was arbitrary, by \cref{ii:3.2.1} we know that \((f_{a_n})_{n = 0}^\infty\) converges pointwise to \(f\) on \(\R\) with respect to \(d_{l^1}|_{\R \times \R}\).
  Since \((a_n)_{n = 0}^\infty\) was arbitrary, we conclude that if \((a_n)_{n = 0}^\infty\) is a sequence in \(\R\) such that \(\lim_{n \to \infty} a_n = 0\), then \((f_{a_n})_{n = 0}^\infty\) converges pointwise to \(f\) on \(\R\) with respect to \(d_{l^1}|_{\R \times \R}\).

  Now suppose that if \((a_n)_{n = 0}^\infty\) is a sequence in \(\R\) such that \(\lim_{n \to \infty} a_n = 0\), then \((f_{a_n})_{n = 0}^\infty\) converges pointwise to \(f\) on \(\R\) with respect to \(d_{l^1}|_{\R \times \R}\).
  Suppose for sake of contradiction that there exists some \(x_0 \in \R\) such that \(\lim_{x \to x_0 ; x \in \R} f(x) \neq f(x_0)\).
  Then we have
  \[
    \exists \varepsilon \in \R^+ : \forall \delta \in \R^+, \exists x \in \R : \begin{dcases}
      \abs{x - x_0} < \delta \\
      \abs{f(x) - f(x_0)} > \varepsilon
    \end{dcases}
  \]
  Fix such \(\varepsilon\).
  We choose a sequence \((a_n)_{n = 0}^\infty\) in \(\R\) such that \(\abs{a_n - x_0} < \dfrac{1}{n + 1}\) for all \(n \in \N\).
  Then we have \(\lim_{n \to \infty} \abs{a_n - x_0} = \lim_{n \to \infty} (a_n - x_0) = 0\).
  By hypothesis we have
  \[
    \forall x \in \R, f(x) = \lim_{n \to \infty} f_{a_n - x_0}(x) = \lim_{n \to \infty} f(x - a_n + x_0).
  \]
  But this means
  \[
    \begin{dcases}
      \abs{x - a_n + x_0 - x} = \abs{-a_n + x_0} < \dfrac{1}{n} \\
      \abs{f(x - a_n + x_0) - f(x)} < \varepsilon
    \end{dcases}
  \]
  a contradiction.
  Thus, we have \(\lim_{x \to x_0 ; x \in \R} f(x) = f(x_0)\) for every \(x_0 \in \R\) and \(f\) is continuous on \(\R\).
\end{proof}

\begin{proof}{(b)}
  Suppose that \(f\) is uniformly continuous on \(\R\).
  Let \((a_n)_{n = 0}^\infty\) be a sequence in \(\R\) such that \(\lim_{n \to \infty} a_n = 0\).
  Since \(f\) is uniformly continuous on \(\R\), we have
  \[
    \forall \varepsilon \in \R^+, \exists \delta \in \R^+ : \forall x_1, x_2 \in \R, \abs{x_1 - x_2} < \delta \implies \abs{f(x_1) - f(x_2)} < \varepsilon.
  \]
  Fix one pair of \(\varepsilon\) and \(\delta\).
  Then we have
  \begin{align*}
             & \exists N \in \N : \forall n \geq N, \abs{a_n} < \delta                                       \\
    \implies & \exists N \in \N : \forall n \geq N, \abs{-a_n} < \delta                                      \\
    \implies & \exists N \in \N : \forall n \geq N, \forall x \in \R, \abs{x - a_n - x} < \delta             \\
    \implies & \exists N \in \N : \forall n \geq N, \forall x \in \R, \abs{f(x - a_n) - f(x)} < \varepsilon  \\
    \implies & \exists N \in \N : \forall n \geq N, \forall x \in \R, \abs{f_{a_n}(x) - f(x)} < \varepsilon.
  \end{align*}
  Since this is true for arbitrary \(\varepsilon\), by \cref{ii:3.2.7} we know that \((f_{a_n})_{n = 0}^\infty\) converges uniformly to \(f\) on \(\R\) with respect to \(d_{l^1}|_{\R \times \R}\).
  Since \((a_n)_{n = 0}^\infty\) was arbitrary, we conclude that if \((a_n)_{n = 0}^\infty\) is a sequence in \(\R\) such that \(\lim_{n \to \infty} a_n = 0\), then \((f_{a_n})_{n = 0}^\infty\) uniformly converges to \(f\) on \(\R\) with respect to \(d_{l^1}|_{\R \times \R}\).

  Now suppose that if \((a_n)_{n = 0}^\infty\) is a sequence in \(\R\) such that \(\lim_{n \to \infty} a_n = 0\), then \((f_{a_n})_{n = 0}^\infty\) uniformly converges to \(f\) on \(\R\) with respect to \(d_{l^1}|_{\R \times \R}\).
  Suppose for sake of contradiction that \(f\) is not uniformly continuous on \(\R\).
  Then we have
  \[
    \exists \varepsilon \in \R^+ : \forall \delta \in \R^+, \exists x_1, x_2 \in \R : \begin{dcases}
      \abs{x_1 - x_2} < \delta \\
      \abs{f(x_1) - f(x_2)} > \varepsilon
    \end{dcases}
  \]
  Let \((a_n)_{n = 0}^\infty\) be a sequence in \(\R\) such that \(\lim_{n \to \infty} \abs{a_n} = 0\).
  Then we have
  \begin{align*}
             & \lim_{n \to \infty} \abs{a_n} = 0 = \lim_{n \to \infty} a_n = \lim_{n \to \infty} -a_n                                 \\
    \implies & \forall \delta \in \R^+, \exists N \in \N : \forall n \geq N, \abs{-a_n} < \delta                                      \\
    \implies & \forall \delta \in \R^+, \exists N \in \N : \forall n \geq N, \forall x \in \R, \abs{x - a_n - x} < \delta             \\
    \implies & \forall \delta \in \R^+, \exists N \in \N : \forall n \geq N, \forall x \in \R, \abs{f(x - a_n) - f(x)} > \varepsilon  \\
    \implies & \forall \delta \in \R^+, \exists N \in \N : \forall n \geq N, \forall x \in \R, \abs{f_{a_n}(x) - f(x)} > \varepsilon.
  \end{align*}
  But by hypothesis we know that \((f_{a_n})_{n = 0}^\infty\) uniformly converges to \(f\) on \(\R\), which by \cref{ii:3.2.7} means
  \[
    \exists N' \in \N : \forall n \geq N', \forall x \in \R, \abs{f_{a_n}(x) - f(x)} < \varepsilon,
  \]
  a contradiction.
  Thus, \(f\) is uniformly continuous on \(\R\).
\end{proof}

\begin{ex}\label{ii:ex:3.2.2}
  \quad
  \begin{enumerate}
    \item Let \((f^{(n)})_{n = 1}^\infty\) be a sequence of functions from one metric space \((X, d_X)\) to another \((Y, d_Y)\), and let \(f : X \to Y\) be another function from \(X\) to \(Y\).
          Show that if \(f^{(n)}\) converges uniformly to \(f\), then \(f^{(n)}\) also converges pointwise to \(f\).
    \item For each integer \(n \geq 1\), let \(f^{(n)} : (-1, 1) \to \R\) be the function \(f^{(n)}(x) \coloneqq x^n\).
          Prove that \(f^{(n)}\) converges pointwise to the zero function \(0\), but does not converge uniformly to any function \(f : (-1, 1) \to \R\).
    \item Let \(g : (-1, 1) \to \R\) be the function \(g(x) \coloneqq x / (1 - x)\).
          With the notation as in (b), show that the partial sums \(\sum_{n = 1}^N f^{(n)}\) converges pointwise as \(N \to \infty\) to \(g\), but does not converge uniformly to \(g\), on the open interval \((-1, 1)\).
          What would happen if we replaced the open interval \((-1, 1)\) with the closed interval \([-1, 1]\)?
  \end{enumerate}
\end{ex}

\begin{proof}{(a)}
  Let \(x_0 \in X\).
  Since
  \begin{align*}
             & (f^{(n)})_{n = 1}^\infty \text{ converges uniformly to } f \text{ on } X                           \\
             & \text{ with respect to } d_Y                                                                       \\
    \implies & \forall \varepsilon \in \R^+, \exists N \in \Z^+ :                                                 \\
             & \forall n \geq N, \forall x \in X, d_Y\big(f^{(n)}(x), f(x)\big) < \varepsilon &  & \by{ii:3.2.7}  \\
    \implies & \forall \varepsilon \in \R^+, \exists N \in \Z^+ :                                                 \\
             & \forall n \geq N, d_Y\big(f^{(n)}(x_0), f(x_0)\big) < \varepsilon                                  \\
    \implies & \lim_{n \to \infty} d_Y\big(f^{(n)}(x_0), f(x_0)\big) = 0                      &  & \by{ii:1.1.14}
  \end{align*}
  and \(x_0\) was arbitrary, by \cref{ii:3.2.1} we know that \((f^{(n)})_{n = 1}^\infty\) converges pointwise to \(f\) on \(X\) with respect to \(d_Y\).
\end{proof}

\begin{proof}{(b)}
  Let \(z : (-1, 1) \to \R\) be the zero function, i.e., \(z(x) = 0\) for each \(x \in (-1, 1)\).
  Then we have
  \begin{align*}
             & \forall x \in (-1, 1), \lim_{n \to \infty} x^n = 0                                                \\
    \implies & \forall x \in (-1, 1), \lim_{n \to \infty} f^{(n)}(x) = 0 = z(x)                                  \\
    \implies & (f^{(n)})_{n = 1}^\infty \text{ converges pointwise to } z \text{ on } (-1, 1)                    \\
             & \text{with respect to } d_{l^1}|_{\R \times \R}.                               &  & \by{ii:3.2.1}
  \end{align*}
  Suppose for sake of contradiction that there exists a \(f : (-1, 1) \to \R\) such that \((f^{(n)})_{n = 1}^\infty\) converges uniformly to \(f\) on \((-1, 1)\) with respect to \(d_{l^1}|_{\R \times \R}\).
  By \cref{ii:ex:3.2.2}(a) and \cref{ii:1.1.20} we know that \(f = z\).
  Then we have
  \begin{align*}
             & \forall \varepsilon \in \R^+, \exists N \in \Z^+ :                                                \\
             & \forall n \geq N, \forall x \in (-1, 1), \abs{f^{(n)}(x) - z(x)} < \varepsilon &  & \by{ii:3.2.7} \\
    \implies & \forall \varepsilon \in \R^+, \exists N \in \Z^+ :                                                \\
             & \forall n \geq N, \forall x \in (-1, 1), \abs{x^n} < \varepsilon.
  \end{align*}
  Now consider \(\varepsilon = \dfrac{1}{2}\).
  Then we have
  \begin{align*}
             & \exists N \in \Z^+ : \forall n \geq N, \forall x \in (-1, 1), \abs{x^n} < \dfrac{1}{2}                                               \\
    \implies & \exists N \in \Z^+ : \begin{dcases}
                                      (\dfrac{1}{2})^{\dfrac{1}{n}} \in (-1, 1) \\
                                      \abs{\big((\dfrac{1}{2})^{\dfrac{1}{n}}\big)^n} = \dfrac{1}{2} < \dfrac{1}{2}
                                    \end{dcases}
  \end{align*}
  a contradiction.
  Thus, \((f^{(n)})_{n = 1}^\infty\) does not converge uniformly to any \(f\) on \((-1, 1)\) with respect to \(d_{l^1}|_{\R \times \R}\).
\end{proof}

\begin{proof}{(c)}
  By Lemma 7.3.3 in Analysis I we have
  \begin{align*}
    \forall x \in (-1, 1), \lim_{N \to \infty} \sum_{n = 1}^N f^{(n)}(x) & = \lim_{N \to \infty} \sum_{n = 1}^N x^n                 \\
                                                                         & = \lim_{N \to \infty} \bigg(\sum_{n = 0}^N x^n - 1\bigg) \\
                                                                         & = \bigg(\lim_{N \to \infty} \sum_{n = 0}^N x^n\bigg) - 1 \\
                                                                         & = \dfrac{1}{1 - x} - 1                                   \\
                                                                         & = \dfrac{x}{1 - x}                                       \\
                                                                         & = g(x).
  \end{align*}
  Thus, by \cref{ii:3.2.1} we know that \((\sum_{n = 1}^N f^{(n)})_{N = 1}^\infty\) converges pointwise to \(g\) on \((-1, 1)\) with respect to \(d_{l^1}|_{\R \times \R}\).
  Suppose for sake of contradiction that \((\sum_{n = 1}^N f^{(n)})_{N = 1}^\infty\) converges uniformly to \(g\) on \((-1, 1)\) with respect to \(d_{l^1}|_{\R \times \R}\).
  Then by \cref{ii:3.2.7} we know that
  \begin{align*}
             & \forall \varepsilon \in \R^+, \exists M \in \Z^+ : \forall N \geq M, \forall x \in (-1, 1), \abs{\sum_{n = 1}^N f^{(n)}(x) - g(x)} < \varepsilon                            \\
    \implies & \forall \varepsilon \in \R^+, \exists M \in \Z^+ : \forall N \geq M, \forall x \in (-1, 1), \abs{\dfrac{x(1 - x^N)}{1 - x} - \dfrac{x}{1 - x}} < \varepsilon                \\
    \implies & \forall \varepsilon \in \R^+, \exists M \in \Z^+ : \forall N \geq M, \forall x \in (-1, 1), \abs{\dfrac{-x^{N + 1}}{1 - x}} = \abs{\dfrac{x^{N + 1}}{1 - x}} < \varepsilon.
  \end{align*}
  Now consider \(\varepsilon = \dfrac{1}{2}\).
  Then we have
  \begin{align*}
             & \exists N \geq M : \forall n \geq N, \forall x \in (-1, 1), \abs{\dfrac{x^{N + 1}}{1 - x}} < \dfrac{1}{2}                                                               \\
    \implies & \exists N \geq M : \begin{dcases}
                                    (\dfrac{1}{2})^{\dfrac{1}{N + 1}} \in (-1, 1) \\
                                    \abs{\dfrac{\big((\dfrac{1}{2})^{\dfrac{1}{N + 1}}\big)^{N + 1}}{1 - (\dfrac{1}{2})^{\dfrac{1}{N + 1}}}} = \dfrac{\dfrac{1}{2}}{1 - (\dfrac{1}{2})^{\dfrac{1}{N + 1}}} < \dfrac{1}{2}
                                  \end{dcases}
  \end{align*}
  But we know that
  \begin{align*}
             & (\dfrac{1}{2})^{\dfrac{1}{N + 1}} > \dfrac{1}{2}                        \\
    \implies & 1 - (\dfrac{1}{2})^{\dfrac{1}{N + 1}} < 1 - \dfrac{1}{2} = \dfrac{1}{2} \\
    \implies & \dfrac{\dfrac{1}{2}}{1 - (\dfrac{1}{2})^{\dfrac{1}{N + 1}}} > 1,
  \end{align*}
  a contradiction.
  Thus, \((\sum_{n = 1}^N f^{(n)})_{N = 1}^\infty\) does not converge uniformly to \(g\) on \((-1, 1)\) with respect to \(d_{l^1}|_{\R \times \R}\).

  If we replace \((-1, 1)\) with \([-1, 1]\), then by Lemma 7.3.3 in Analysis I \(\sum_{n = 1}^\infty 1\) and \(\sum_{n = 1}^\infty -1\) does not converge, thus by \cref{ii:3.2.1} \((\sum_{n = 1}^N f^{(n)})_{N = 1}^\infty\) does not converge pointwise to \(g\) on \([-1, 1]\) with respect to \(d_{l^1}|_{\R \times \R}\).
\end{proof}

\begin{ex}\label{ii:ex:3.2.3}
  Let \((X, d_X)\) a metric space, and for every integer \(n \geq 1\), let \(f_n : X \to \R\) be a real-valued function.
  Suppose that \(f_n\) converges pointwise to another function \(f : X \to \R\) on \(X\)
  (in this question we give \(\R\) the standard metric \(d(x, y) = \abs{x - y}\)).
  Let \(h : \R \to \R\) be a continuous function.
  Show that the functions \(h \circ f_n\) converge pointwise to \(h \circ f\) on \(X\), where \(h \circ f_n : X \to \R\) is the function \(h \circ f_n(x) \coloneqq h\big(f_n(x)\big)\), and similarly for \(h \circ f\).
\end{ex}

\begin{proof}
  Let \(x_0 \in X\).
  We have
  \begin{align*}
             & h \text{ is continuous on } \R \text{ with respect to } d                                      \\
    \implies & h \text{ is continuous at } x_0 \text{ with respect to } d                                     \\
    \implies & \forall \varepsilon \in \R^+, \exists \delta \in \R^+ :                                        \\
             & \big(\forall x \in \R, \abs{x - x_0} < \delta \implies \abs{h(x) - h(x_0)} < \varepsilon\big).
  \end{align*}
  Fix one pair of \(\varepsilon\) and \(\delta\).
  Then we have
  \begin{align*}
             & (f_n)_{n = 1}^\infty \text{ converges pointwise to } f \text{ on } X \text{ with respect to } d                        \\
    \implies & \lim_{n \to \infty} \abs{f_n(x_0) - f(x_0)} = 0                                                     &  & \by{ii:3.2.1} \\
    \implies & \exists N \in \Z^+ : \forall n \geq N, \abs{f_n(x_0) - f(x_0)} < \delta                                                \\
    \implies & \exists N \in \Z^+ : \forall n \geq N, \abs{h\big(f_n(x_0)\big) - h\big(f(x_0)\big)} < \varepsilon.
  \end{align*}
  Since \(\varepsilon\) was arbitrary, by \cref{ii:1.1.14} we know that
  \[
    \lim_{n \to \infty} \abs{h\big(f_n(x_0)\big) - h\big(f(x_0)\big)} = \lim_{n \to \infty} \abs{(h \circ f_n)(x_0) - (h \circ f)(x_0)} = 0.
  \]
  Since \(x_0\) was arbitrary, by \cref{ii:3.2.1} we know \((h \circ f_n)_{n = 1}^\infty\) converges pointwise to \(h \circ f\) on \(X\) with respect to \(d_{l^1}|_{\R \times \R}\).
\end{proof}

\begin{ex}\label{ii:ex:3.2.4}
  Let \(f_n : X \to Y\) be a sequence of bounded functions from one metric space \((X, d_X)\) to another metric space \((Y, d_Y)\).
  Suppose that \(f_n\) converges uniformly to another function \(f : X \to Y\).
  Suppose that \(f\) is a bounded function;
  i.e., there exists a ball \(B_{(Y, d_Y)}(y_0, R)\) in \(Y\) such that \(f(x) \in B_{(Y, d_Y)}(y_0, R)\) for all \(x \in X\).
  Show that the sequence \(f_n\) is \emph{uniformly bounded};
  i.e., there exists a ball \(B_{(Y, d_Y)}(y_0, R)\) in \(Y\) such that \(f_n(x) \in B_{(Y, d_Y)}(y_0, R)\) for all \(x \in X\) and all positive integers \(n\).
\end{ex}

\begin{proof}
  Since \(f\) is bounded, by \cref{ii:1.5.3} we have
  \[
    \forall y \in Y, \exists \varepsilon \in \R^+ : f(X) \subseteq B_{(Y, d_Y)}(y, \varepsilon).
  \]
  We choose one pair of \(y\) and \(\varepsilon\).
  Since \((f_n)_{n = 1}^\infty\) converges uniformly to \(f\) on \(X\) with respect to \(d_Y\), we have
  \begin{align*}
             & \exists N \in \Z^+ : \forall n \geq N, \forall x \in X, d_Y\big(f_n(x), f(x)\big) < \varepsilon          &  & \by{ii:3.2.7}    \\
    \implies & \exists N \in \Z^+ : \forall n \geq N, \forall x \in X,                                                                        \\
             & d_Y\big(f_n(x), y\big) \leq d_Y\big(f_n(x), f(x)\big) + d_Y\big(f(x), y\big) < \varepsilon + \varepsilon &  & \by{ii:1.1.2}[d] \\
    \implies & \exists N \in \Z^+ : \forall n \geq N, \forall x \in X, f_n(x) \in B_{(Y, d_Y)}\big(y, 2\varepsilon\big)                       \\
    \implies & \exists N \in \Z^+ : \forall n \geq N, f_n(X) \subseteq B_{(Y, d_Y)}\big(y, 2\varepsilon\big)
  \end{align*}
  Now fix \(N\).
  Let \(S = \set{n \in \Z^+ : n < N}\).
  Then \(S\) is finite.
  If \(S = \emptyset\), then we have \(N = 1\) and thus by definition \((f_n)_{n = 1}^\infty\) is uniformly bounded.
  So suppose that \(S \neq \emptyset\).
  By hypothesis we know that \(f_n\) is bounded for each \(n \in S\), thus by \cref{ii:1.5.3} we have
  \[
    \forall n \in S, \exists \delta_n \in \R^+ : f_n(X) \subseteq B_{(Y, d_Y)}(y, \delta_n).
  \]
  We choose one \(\delta_n\) for each \(n \in S\).
  Since \(S\) is finite, we know that \(\delta_{\max} = \max\set{\delta_n : n \in S}\) is well-defined.
  Then we have
  \begin{align*}
             & \begin{dcases}
                 \forall n \in S, f_n(X) \subseteq B_{(Y, d_Y)}(y, \delta_{\max}) \\
                 \forall n \geq N, f_n(X) \subseteq B_{(Y, d_Y)}(y, 2\varepsilon)
               \end{dcases}                    \\
    \implies & \forall n \in \Z^+, f_n(X) \subseteq B_{(Y, d_Y)}(y, 2\varepsilon + \delta_{\max}).
  \end{align*}
  Since \(y\) was arbitrary, by \cref{ii:1.5.3} we know that \(f_n\) is bounded for each \(n \in \Z^+\), i.e.,
  \[
    \forall y \in Y, \exists r \in \R^+ : \forall n \in \Z^+, f_n(X) \subseteq B_{(Y, d_Y)}(y, r).
  \]
  And by definition \((f_n)_{n = 1}^\infty\) is uniformly bounded.
\end{proof}

\section{Uniform convergence and continuity}\label{sec:3.3}

\begin{thm}[Uniform limits preserve continuity I]\label{3.3.1}
  Suppose \((f^{(n)})_{n = 1}^\infty\) is a sequence of functions from one metric space \((X, d_X)\) to another \((Y, d_Y)\), and suppose that this sequence converges uniformly to another function \(f : X \to Y\).
  Let \(x_0\) be a point in \(X\).
  If the functions \(f^{(n)}\) are continuous at \(x_0\) for each \(n\), then the limiting function \(f\) is also continuous at \(x_0\).
\end{thm}

\begin{proof}
  We have
  \begin{align*}
             & (f^{(n)})_{n = 1}^\infty \text{ converges uniformly to } f \text{ on } X                                   \\
             & \text{with respect to } d_Y                                                                                \\
    \implies & \forall \varepsilon \in \R^+, \exists N \in \Z^+ :                                                         \\
             & \forall n \geq N, \forall x \in X, d_Y\big(f^{(n)}(x), f(x)\big) < \dfrac{\varepsilon}{3}. &  & \by{3.2.7}
  \end{align*}
  We choose one pair of \(\varepsilon\) and \(N\).
  For each \(n \in \Z^+\), since \(f^{(n)}\) is continuous at \(x_0\) from \((X, d_X)\) to \((Y, d_Y)\), by \cref{2.1.1} we have
  \begin{align*}
             & \forall n \geq N, f^{(n)} \text{ is continuous at } x_0 \text{ from } (X, d_X) \text{ to } (Y, d_Y)                                                                                            \\
    \implies & \forall n \geq N, \exists \delta \in \R^+ :                                                                                                                                                    \\
             & \Big(\forall x \in X, d_X(x, x_0) < \delta \implies d_Y\big(f^{(n)}(x), f^{(n)}(x_0)\big) < \dfrac{\varepsilon}{3}\Big)                                                                        \\
    \implies & \forall n \geq N, \exists \delta \in \R^+ :                                                                                                                                                    \\
             & \Big(\forall x \in X, d_X(x, x_0) < \delta \implies d_Y\big(f(x), f(x_0)\big)                                                                                                                  \\
             & \leq d_Y\big(f(x), f^{(n)}(x)\big) + d_Y\big(f^{(n)}(x), f^{(n)}(x_0)\big) + d_Y\big(f^{(n)}(x_0), f(x_0)\big) < \dfrac{\varepsilon}{3} + \dfrac{\varepsilon}{3} + \dfrac{\varepsilon}{3}\Big) \\
    \implies & \forall n \geq N, \exists \delta \in \R^+ :                                                                                                                                                    \\
             & \Big(\forall x \in X, d_X(x, x_0) < \delta \implies d_Y\big(f(x), f(x_0)\big) < \varepsilon.
  \end{align*}
  Since \(\varepsilon\) is arbitrary, by \cref{2.1.1} we know that \(f\) is continuous at \(x_0\) from \((X, d_X)\) to \((Y, d_Y)\).
\end{proof}

\begin{cor}[Uniform limits preserve continuity II]\label{3.3.2}
  Let \((f^{(n)})_{n = 1}^\infty\) be a sequence of functions from one metric space \((X, d_X)\) to another \((Y, d_Y)\), and suppose that this sequence converges uniformly to another function \(f : X \to Y\).
  If the functions \(f^{(n)}\) are continuous on \(X\) for each \(n\), then the limiting function \(f\) is also continuous on \(X\).
\end{cor}

\begin{proof}
  By applying \cref{3.3.1} to each \(x \in X\) we conclude that \(f\) is continuous on \(X\) from \((X, d_X)\) to \((Y, d_Y)\).
\end{proof}

\begin{prop}[Interchange of limits and uniform limits]\label{3.3.3}
  Let
  \((X, d_X)\) and \((Y, d_Y)\) be metric spaces, with \(Y\) complete, and let \(E\) be a subset of \(X\).
  Let \((f^{(n)})_{n = 1}^\infty\) be a sequence of functions from \(E\) to \(Y\), and suppose that this sequence converges uniformly in \(E\) to some function \(f : E \to Y\).
  Let \(x_0 \in X\) be an adherent point of \(E\), and suppose that for each \(n\) the limit \(\lim_{x \to x_0 ; x \in E} f^{(n)}(x)\) exists.
  Then the limit \(\lim_{x \to x_0 ; x \in E} f(x)\) also exists, and is equal to the limit of the sequence \(\big(\lim_{x \to x_0 ; x \in E} f^{(n)}(x)\big)_{n = 1}^\infty\);
  in other words we have the interchange of limits
  \[
    \lim_{n \to \infty} \lim_{x \to x_0 ; x \in E} f^{(n)}(x) = \lim_{x \to x_0 ; x \in E} \lim_{n \to \infty} f^{(n)}(x).
  \]
\end{prop}

\begin{proof}
  For each \(n \in \Z^+\), we define \(d_Y - \lim_{x \to x_0 ; x \in E} f^{(n)}(x) = L^{(n)}\).
  We claim that the sequence \((L^{(n)})_{n = 1}^\infty\) converges in \(Y\) with respect to \(d_Y\).
  Since \((Y, d_Y)\) is complete, by \cref{1.4.10} it suffices to show that \((L^{(n)})_{n = 1}^\infty\) is a Cauchy sequence in \((Y, d_Y)\).
  Let \(n_1, n_2 \in \Z^+\).
  Then by \cref{3.2.7} we have
  \begin{align*}
             & (f^{(n)})_{n = 1}^\infty \text{ converges uniformly to } f \text{ on } X \text{ with respect to } d_Y                                        \\
    \implies & \forall \varepsilon \in \R^+, \exists N \in \Z^+ : \forall n \geq N, \forall x \in X, d_Y\big(f^{(n)}(x), f(x)\big) < \dfrac{\varepsilon}{4}
  \end{align*}
  Now fix one pair of \(\varepsilon\) and \(N\).
  Since \(L^{(n)}\) exists for each \(n \in \N\), by \cref{3.1.1} we have
  \begin{align*}
             & \forall n \geq N, d_Y - \lim_{x \to x_0 ; x \in E} f^{(n)}(x) = L^{(n)}                                                                                        \\
    \implies & \forall n \geq N, \exists \delta \in \R^+ : \Big(\forall x \in X, d_X(x, x_0) < \delta \implies d_Y\big(f^{(n)}(x), L^{(n)}\big) < \dfrac{\varepsilon}{4}\Big) \\
    \implies & \forall n_1, n_2 \geq N, \exists \delta \in \R^+ :                                                                                                             \\
             & \Big(\forall x \in X, d_X(x, x_0) < \delta \implies d_Y\big(L^{(n_1)}, L^{(n_2)}\big)                                                                          \\
             & \leq d_Y\big(L^{(n_1)}, f^{(n_1)}(x)\big) + d_Y\big(f^{(n_1)}(x), f(x)\big)                                                                                    \\
             & \quad + d_Y\big(f(x), f^{(n_2)}(x)\big) + d_Y\big(f^{(n_2)}(x), L^{(n_2)}\big)                                                                                 \\
             & < \dfrac{\varepsilon}{4} + \dfrac{\varepsilon}{4} + \dfrac{\varepsilon}{4} + \dfrac{\varepsilon}{4}\Big)                                                       \\
    \implies & \forall n_1, n_2 \geq N, d_Y\big(L^{(n_1)}, L^{(n_2)}\big) < \varepsilon
  \end{align*}
  Since \(\varepsilon\) is arbitrary, we have
  \[
    \forall \varepsilon \in \R^+, \exists N \in \Z^+ : \forall n_1, n_2 \geq N, d_Y\big(L^{(n_1)}, L^{(n_2)}\big) < \varepsilon
  \]
  and by \cref{1.4.6} \((L^{(n)})_{n = 1}^\infty\) is a Cauchy sequence in \((Y, d_Y)\).

  Let \(L \in Y\) such that \(d_Y - \lim_{n \to \infty} L^{(n)} = L\).
  Again by \cref{3.2.7} we have
  \begin{align*}
             & (f^{(n)})_{n = 1}^\infty \text{ converges uniformly to } f \text{ on } X \text{with respect to } d_Y                                              \\
    \implies & \forall \varepsilon \in \R^+, \exists N_1 \in \Z^+ : \forall n \geq N_1, \forall x \in X, d_Y\big(f^{(n)}(x), f(x)\big) < \dfrac{\varepsilon}{3}.
  \end{align*}
  Again we choose one pair of \(\varepsilon\) and \(N_1\).
  Since \(L\) exists, by \cref{3.1.1} we have
  \begin{align*}
             & \lim_{n \to \infty} d_Y\big(L^{(n)}, L\big) = 0                                                                                                                                 \\
    \implies & \exists N_2 \in \Z^+ : \forall n \geq N_2, d_Y(L^{(n)}, L) < \dfrac{\varepsilon}{3}                                                                                             \\
    \implies & \exists N = \max(N_1, N_2) : \forall n \geq N,                                                                                                                                  \\
             & \begin{dcases}
                 \exists \delta \in \R^+ : \forall x \in X, d_X(x, x_0) < \delta \implies d_Y\big(f^{(n)}(x), L^{(n)}\big) < \dfrac{\varepsilon}{3} \\
                 d_Y(L^{(n)}, L) < \dfrac{\varepsilon}{3}                                                                                           \\
                 \forall x \in X, d_Y\big(f^{(n)}(x), f(x)\big) < \dfrac{\varepsilon}{3}
               \end{dcases}                                              \\
    \implies & \exists N = \max(N_1, N_2) : \forall n \geq N, \exists \delta \in \R^+ :                                                                                                        \\
             & \Big(\forall x \in X, d_X(x, x_0) < \delta \implies d_Y\big(f(x), L\big)                                                                                                        \\
             & \leq d_Y\big(f(x), f^{(n)}(x)\big) + d_Y\big(f^{(n)}(x), L^{(n)}\big) + d_Y\big(L^{(n)}, L\big) < \dfrac{\varepsilon}{3} + \dfrac{\varepsilon}{3} + \dfrac{\varepsilon}{3}\Big) \\
    \implies & \exists \delta \in \R^+ : \Big(\forall x \in X, d_X(x, x_0) < \delta \implies d_Y\big(f(x), L\big) < \varepsilon\Big).
  \end{align*}
  Since \(\varepsilon\) is arbitrary, by \cref{3.1.1} we know that \(d_Y - \lim_{x \to x_0 ; x \in E} f(x) = L\).
\end{proof}

\begin{prop}\label{3.3.4}
  Let \((f^{(n)})_{n = 1}^\infty\) be a sequence of continuous functions from one metric space \((X, d_X)\) to another \((Y, d_Y)\), and suppose that this sequence converges uniformly to another function \(f : X \to Y\).
  Let \(x^{(n)}\) be a sequence of points in \(X\) which converge to some limit \(x\).
  Then \(f^{(n)}(x^{(n)})\) converges (in \(Y\)) to \(f(x)\).
\end{prop}

\begin{proof}
  Let \(x_0 \in X\).
  Suppose that \((x^{(n)})_{n = 1}^\infty\) is a sequence in \(X\) such that
  \[
    \lim_{n \to \infty} d_X(x^{(n)}, x_0) = 0.
  \]
  By \cref{3.3.1} we know that \(f\) is continuous at \(x_0\) from \((X, d_X)\) to \((Y, d_Y)\).
  Thus by \cref{2.1.4}(a)(b) we have
  \[
    \lim_{n \to \infty} d_Y\big(f(x^{(n)}), f(x_0)\big) = 0
  \]
  and by \cref{1.1.14} we have
  \[
    \forall \varepsilon \in \R^+, \exists N_1 \in \Z^+ : \forall n \geq N_1, d_Y\big(f(x^{(n)}), f(x_0)\big) < \dfrac{\varepsilon}{2}.
  \]
  Now we choose one pair of \(\varepsilon\) and \(N_1\).
  Since \((f^{(n)})_{n = 1}^\infty\) converges uniformly to \(f\) on \(X\) with respect to \(d_Y\), by \cref{3.2.7} we have
  \begin{align*}
             & \exists N_2 \in \Z^+ : \forall n \geq N_2, \forall x \in X, d_Y\big(f^{(n)}(x), f(x)\big) < \dfrac{\varepsilon}{2}                                                       \\
    \implies & \exists N_2 \in \Z^+ : \forall n \geq N_2, d_Y\big(f^{(n)}(x^{(n)}), f(x^{(n)})\big) < \dfrac{\varepsilon}{2}                                                            \\
    \implies & \exists N = \max(N_1, N_2) : \forall n \geq N,                                                                                                                           \\
             & d_Y\big(f^{(n)}(x^{(n)}), f(x_0)\big) \leq d_Y\big(f^{(n)}(x^{(n)}), f(x^{(n)})\big) + d_Y\big(f(x^{(n)}), f(x_0)\big) < \dfrac{\varepsilon}{2} + \dfrac{\varepsilon}{2} \\
    \implies & \exists N = \max(N_1, N_2) : \forall n \geq N, d_Y\big(f^{(n)}(x^{(n)}), f(x_0)\big) < \varepsilon.
  \end{align*}
  Since \(\varepsilon\) is arbitrary, by \cref{1.1.14} we know that
  \[
    \lim_{n \to \infty} d_Y\big(f^{(n)}(x^{(n)}), f(x_0)\big) = 0.
  \]
  Since \(x_0\) is arbitrary, we conclude that for any \(x_0 \in X\), if \((x^{(n)})_{n = 1}^\infty\) is a sequence in \(X\) such that
  \[
    \lim_{n \to \infty} d_X(x^{(n)}, x_0) = 0,
  \]
  then we have
  \[
    \lim_{n \to \infty} d_Y\big(f^{(n)}(x^{(n)}), f(x_0)\big) = 0.
  \]
\end{proof}

\begin{defn}[Bounded functions]\label{3.3.5}
  A function \(f : X \to Y\) from one metric space \((X, d_X)\) to another \((Y, d_Y)\) is \emph{bounded} if \(f(X)\) is a bounded set, i.e., there exists a ball \(B_{(Y, d_Y)}(y_0, R)\) in \(Y\) such that \(f(x) \in B_{(Y, d_Y)}(y_0, R)\) for all \(x \in X\).
\end{defn}

\begin{prop}[Uniform limits preserve boundedness]\label{3.3.6}
  Let \((f^{(n)})_{n = 1}^\infty\) be a sequence of functions from one metric space \((X, d_X)\) to another \((Y, d_Y)\), and suppose that this sequence converges uniformly to another function \(f : X \to Y\).
  If the functions \(f^{(n)}\) are bounded on \(X\) for each \(n\), then the limiting function \(f\) is also bounded on \(X\).
\end{prop}

\begin{proof}
  Since \(f^{(n)}\) is bounded in \((Y, d_Y)\) for each \(n \in \Z^+\), by \cref{3.3.5} and \cref{1.5.3} we have
  \begin{align*}
             & \forall n \in \Z^+, \forall y \in Y, \exists \varepsilon \in \R^+ : f^{(n)}(X) \subseteq B_{(Y, d_Y)}(y, \varepsilon)          \\
    \implies & \forall n \in \Z^+, \forall y \in Y, \exists \varepsilon \in \R^+ : \forall x \in X, d_Y\big(f^{(n)}(x), y\big) < \varepsilon.
  \end{align*}
  Now we choose \(y\) and \(\varepsilon\) for each \(n \in \Z^+\) and we denote them as \(y^{(n)}\) and \(\varepsilon^{(n)}\).
  Since \((f^{(n)})_{n = 1}^\infty\) converges uniformly to \(f\) on \(X\) with respect to \(d_Y\), by \cref{3.2.7} we have
  \begin{align*}
             & \exists N \in \Z^+ : \forall n \geq N, \forall x \in X, d_Y\big(f^{(n)}(x), f(x)\big) < 1                                \\
             & \exists N \in \Z^+ : \forall x \in X, d_Y\big(f^{(N)}(x), f(x)\big) < 1                                                  \\
    \implies & \exists N \in \Z^+ : \forall x \in X,                                                                                    \\
             & d_Y\big(f(x), y^{(N)}\big) \leq d_Y\big(f(x), f^{(N)}(x)\big) + d_Y\big(f^{(N)}(x), y^{(N)}\big) < \varepsilon^{(N)} + 1 \\
    \implies & \exists N \in \Z^+ : \forall x \in X, d_Y\big(f(x), y^{(N)}\big) < \varepsilon^{(N)} + 1                                 \\
    \implies & \exists N \in \Z^+ : f(X) \subseteq B_{(Y, d_Y)}(y^{(N)}, \varepsilon^{(N)} + 1).
  \end{align*}
  Since \(y^{(N)}\) is arbitrary, we have
  \[
    \forall y \in Y, \exists \varepsilon \in \R^+ : f(X) \subseteq B_{(Y, d_Y)}(y, \varepsilon)
  \]
  and by \cref{1.5.3} and \cref{3.3.5} \(f\) is bounded in \((Y, d_Y)\).
\end{proof}

\begin{rmk}\label{3.3.7}
  The above propositions sound very reasonable, but one should caution that it only works if one assumes uniform convergence;
  pointwise convergence is not enough.
\end{rmk}

\exercisesection

\begin{ex}\label{ex:3.3.1}
  Prove \cref{3.3.1}.
  Explain briefly why your proof requires uniform convergence, and why pointwise convergence would not suffice.
\end{ex}

\begin{proof}
  See \cref{3.3.1}.
  Without uniform convergence we cannot make \(f^{(n)}(x)\) and \(f(x)\) arbitrary close.
\end{proof}

\begin{ex}\label{ex:3.3.2}
  Prove \cref{3.3.3}.
\end{ex}

\begin{proof}
  See \cref{3.3.3}.
\end{proof}

\begin{ex}\label{ex:3.3.3}
  Compare \cref{3.3.3} with Example 1.2.8 in Analysis I.
  Can you now explain why the interchange of limits in Example 1.2.8 in Analysis I led to a false statement, whereas the interchange of limits in \cref{3.3.3} is justified?
\end{ex}

\begin{proof}
  By \cref{ex:3.2.2}(b) we know that \(f^{(n)}(x) = x^{(n)}\) does not converge uniformly to any function \(f : (-1, 1) \to \R\), thus the interchange of limits in Example 1.2.8 in Analysis I failed.
\end{proof}

\begin{ex}\label{ex:3.3.4}
  Prove \cref{3.3.4}.
\end{ex}

\begin{proof}
  See \cref{3.3.4}.
\end{proof}

\begin{ex}\label{ex:3.3.5}
  Give an example to show that \cref{3.3.4} fails if the phrase ``converges uniformly'' is replaced by ``converges pointwise''.
\end{ex}

\begin{proof}
  For each \(n \in \Z^+\), let \(f^{(n)} : [0, 1] \to \R\) be the function where \(f^{(n)}(x) = x^n\) for each \(x \in [0, 1]\).
  Let \(f : [0, 1] \to \R\) be the function where
  \[
    \forall x \in [0, 1], f(x) = \begin{dcases}
      1 & \text{if } x = 1        \\
      0 & \text{if } x \in [0, 1)
    \end{dcases}
  \]
  By Example 3.2.4 in Analysis II we know that \((f^{(n)})_{n = 1}^\infty\) converges pointwise to \(f\) on \(X\) with respect to \(d_{l^1}|_{\R \times \R}\).
  By \cref{ex:3.2.2}(b) we know that \(f^{(n)}(x) = x^{(n)}\) does not converge uniformly to \(f\) on \(X\) with respect to \(d_{l^1}|_{\R \times \R}\).
  Let \((x^{(n)})_{n = 1}^\infty\) be the sequence where \(x^{(n)} = (\dfrac{1}{2})^{\dfrac{1}{n}}\) for each \(n \in \Z^+\).
  Then we have
  \[
    \lim_{n \to \infty} x^{(n)} = 1 = f(1).
  \]
  But
  \[
    \lim_{n \to \infty} f^{(n)}(x^{(n)}) = \lim_{n \to \infty} \big((\dfrac{1}{2})^{\dfrac{1}{n}}\big)^n = \lim_{n \to \infty} \dfrac{1}{2} = \dfrac{1}{2} \neq 1.
  \]
  Thus \cref{3.3.4} fails when the phrase ``converges uniformly'' is replaced by ``converges pointwise''.
\end{proof}

\begin{ex}\label{ex:3.3.6}
  Prove \cref{3.3.6}.
\end{ex}

\begin{proof}
  See \cref{3.3.6}.
\end{proof}

\begin{ex}\label{ex:3.3.7}
  Give an example to show that \cref{3.3.6} fails if the phrase ``converges uniformly'' is replaced by ``converges pointwise''.
\end{ex}

\begin{proof}
  By \cref{ex:3.2.2}(c) we know that \(g\) is unbounded since
  \[
    \lim_{x \to -1 ; x \in (-1, 1)} g(x) = \lim_{x \to -1 ; x \in (-1, 1)} \dfrac{x}{1 - x} = \lim_{x \to -1 ; x \in (-1, 1)} \dfrac{1}{1 - x} - 1 = +\infty.
  \]
\end{proof}

\begin{ex}\label{ex:3.3.8}
  Let \((X, d)\) be a metric space, and for every positive integer \(n\), let \(f_n : X \to \R\) and \(g_n : X \to \R\) be functions.
  Suppose that \((f_n)_{n = 1}^\infty\) converges uniformly to another function \(f : X \to \R\), and that \((g_n)_{n = 1}^\infty\) converges uniformly to another function \(g : X \to \R\).
  Suppose also that the functions \((f_n)_{n = 1}^\infty\) and \((g_n)_{n = 1}^\infty\) are uniformly bounded, i.e., there exists an \(M > 0\) such that \(\abs{f_n(x)} \leq M\) and \(\abs{g_n(x)} \leq M\) for all \(n \geq 1\) and \(x \in X\).
  Prove that the functions \(f_n g_n : X \to \R\) converge uniformly to \(fg : X \to \R\).
\end{ex}

\begin{proof}
  Let \(d_1 = d_{l^1}|_{\R \times \R}\).
  Since \((f_n)_{n = 1}^\infty\) and \((g_n)_{n = 1}^\infty\) are uniformly bounded, by \cref{3.3.5} we know that \(f_n\) and \(g_n\) are bounded in \((\R, d_1)\) for each \(n \in \Z^+\).
  By \cref{3.3.6} we know that \(f\) and \(g\) are bounded in \((\R, d_1)\), i.e.,
  \[
    \exists U \in \R^+ : \forall x \in X, \begin{dcases}
      \abs{f(x)} < U \\
      \abs{g(x)} < U
    \end{dcases}
  \]
  Since \((f_n)_{n = 1}^\infty\) and \((g_n)_{n = 1}^\infty\) converge uniformly to \(f\) on \(X\) with respect to \(d_1\), by \cref{3.2.7} we have
  \begin{align*}
     & \forall \varepsilon \in \R^+, \exists N_1 \in \Z^+ : \forall n \geq N_1, \forall x \in X, \abs{f_n(x) - f(x)} < \dfrac{\varepsilon}{2M}; \\
     & \forall \varepsilon \in \R^+, \exists N_2 \in \Z^+ : \forall n \geq N_2, \forall x \in X, \abs{g_n(x) - g(x)} < \dfrac{\varepsilon}{2U}.
  \end{align*}
  Now we fix one \(\varepsilon\) and its corresponding \(N_1, N_2\).
  Let \(N = \max(N_1, N_2)\).
  Then we have
  \begin{align*}
    \forall n \geq N, \forall x \in X, & \abs{f_n(x) g_n(x) - f(x) g(x)}                                        \\
                                       & = \abs{f_n(x) g_n(x) - f(x) g_n(x) + f(x) g_n(x) - f(x) g(x)}          \\
                                       & \leq \abs{f_n(x) g_n(x) - f(x) g_n(x)} + \abs{f(x) g_n(x) - f(x) g(x)} \\
                                       & = \abs{f_n(x) - f(x)} \abs{g_n(x)} + \abs{f(x)} \abs{g_n(x) - g(x)}    \\
                                       & < \dfrac{\varepsilon}{2M} M + U \dfrac{\varepsilon}{2U}                \\
                                       & = \varepsilon.
  \end{align*}
  Since \(\varepsilon\) is arbitrary, we have
  \[
    \forall \varepsilon \in \R^+, \exists N \in \Z^+ : \forall n \geq N, \forall x \in X, \abs{f_n(x) g_n(x) - f(x) g(x)} < \varepsilon
  \]
  and by \cref{3.2.7} \((f_n g_n)_{n = 1}^\infty\) converges uniformly to \(fg\) on \(X\) with respect to \(d_1\).
\end{proof}
\section{The metric of uniform convergence}\label{ii:sec:3.4}

\begin{note}
  We have now developed at least four, apparently separate, notions of limit in this text:
  \begin{enumerate}
    \item limits \(\lim_{n \to \infty} x^{(n)}\) of sequences of points in a metric space
          (\cref{ii:1.1.14};
          see also \cref{ii:2.5.4});
    \item limiting values \(\lim_{x \to x_0 ; x \in E} f(x)\) of functions at a point
          (\cref{ii:3.1.1});
    \item pointwise limits \(f\) of functions \(f^{(n)}\)
          (\cref{ii:3.2.1});
          and
    \item uniform limits \(f\) of functions \(f^{(n)}\)
          (\cref{ii:3.2.7}).
  \end{enumerate}

  This proliferation of limits may seem rather complicated.
  However, we can reduce the complexity slightly by observing that (d) can be viewed as a special case of (a), though in doing so it should be cautioned that because we are now dealing with functions instead of points, the convergence is not in \(X\) or in \(Y\), but rather in a new space, the space of functions from \(X\) to \(Y\).
\end{note}

\begin{rmk}\label{ii:3.4.1}
  If one is willing to work in topological spaces instead of metric spaces, we can also view (a) as a special case of (b), see \cref{ii:ex:3.1.4}, and (c) is also a special case of (a), see \cref{ii:ex:3.4.4}.
  Thus, the notion of convergence in a topological space can be used to unify all the notions of limits we have encountered so far.
\end{rmk}

\begin{defn}[Metric space of bounded functions]\label{ii:3.4.2}
  Suppose \((X, d_X)\) and \((Y, d_Y)\) are metric spaces.
  We let \(B(X \to Y)\) denote the space of bounded functions from \(X\) to \(Y\) :
  \[
    B(X \to Y) \coloneqq \set{f | f : X \to Y \text{ is a bounded function}}.
  \]
  We define a metric \(d_\infty : B(X \to Y) \times B(X \to Y) \to [0, +\infty)\) by defining
  \[
    d_\infty(f, g) \coloneqq \sup_{x \in X} d_Y\big(f(x), g(x)\big) = \sup\set{d_Y\big(f(x), g(x)\big) : x \in X}
  \]
  for all \(f, g \in B(X \to Y)\).
  This metric is sometimes known as the \emph{uniform metric}, or \emph{sup norm metric}, or the \emph{\(L^\infty\) metric}.
  We will also use \(d_{B(X \to Y)}\) as a synonym for \(d_\infty\).
  We restrict the definition of \(d_\infty\) to the case when \(X \neq \emptyset\).
  If \(X = \emptyset\), then we instead define \(d_\infty(f, g) = 0\).
\end{defn}

\begin{note}
  \(B(X \to Y)\) is a set, thanks to the power set axiom (Axiom 3.10 in Analysis I) and the axiom of specification (Axiom 3.5 in Analysis I).
\end{note}

\begin{note}
  The distance \(d_\infty(f, g)\) is always finite because \(f\) and \(g\) are assumed to be bounded on \(X\).
\end{note}

\setcounter{thm}{3}
\begin{prop}\label{ii:3.4.4}
  Let \((X, d_X)\) and \((Y, d_Y)\) be metric spaces.
  Let \((f^{(n)})_{n = 1}^\infty\) be a sequence of functions in \(B(X \to Y)\), and let \(f\) be another function in \(B(X \to Y)\).
  Then \((f^{(n)})_{n = 1}^\infty\) converges to \(f\) in the metric \(d_{B(X \to Y)}\) iff \((f^{(n)})_{n = 1}^\infty\) converges uniformly to \(f\).
\end{prop}

\begin{proof}
  We have
  \begin{align*}
         & d_{B(X \to Y)} - \lim_{n \to \infty} f^{(n)} = f                                                                               \\
    \iff & \forall \varepsilon \in \R^+, \exists N \in \Z^+ :                                                                             \\
         & \forall n \geq N, d_{B(X \to Y)}(f^{(n)}, f) \leq \dfrac{\varepsilon}{2} < \varepsilon                     &  & \by{ii:1.1.14} \\
    \iff & \forall \varepsilon \in \R^+, \exists N \in \Z^+ :                                                                             \\
         & \forall n \geq N, \sup_{x \in X} d_Y\big(f^{(n)}(x), f(x)\big) \leq \dfrac{\varepsilon}{2} < \varepsilon   &  & \by{ii:3.4.2}  \\
    \iff & \forall \varepsilon \in \R^+, \exists N \in \Z^+ :                                                                             \\
         & \forall n \geq N, \forall x \in X, d_Y\big(f^{(n)}(x), f(x)\big) \leq \dfrac{\varepsilon}{2} < \varepsilon                     \\
    \iff & (f^{(n)})_{n = 1}^\infty \text{ converges uniformly to } f \text{ on } X                                                       \\
         & \text{with respect to } d_Y.                                                                               &  & \by{ii:3.2.7}
  \end{align*}
\end{proof}

\begin{note}
  Now let \(C(X \to Y)\) be the space of bounded continuous functions from \(X\) to \(Y\) :
  \[
    C(X \to Y) \coloneqq \set{f \in B(X \to Y) | f \text{ is continuous}}.
  \]
  This set \(C(X \to Y)\) is clearly a subset of \(B(X \to Y)\).
  \cref{ii:3.3.2} asserts that this space \(C(X \to Y)\) is closed in \(\big(B(X \to Y), d_{B(X \to Y)}\big)\).
\end{note}

\begin{thm}[The space of continuous functions is complete]\label{ii:3.4.5}
  Let \((X, d_X)\) be a metric space, and let \((Y, d_Y)\) be a complete metric space.
  The space \(\big(C(X \to Y), d_{B(X \to Y)}|_{C(X \to Y) \times C(X \to Y)}\big)\) is a complete subspace of \(\big(B(X \to Y), d_{B(X \to Y)}\big)\).
  In other words, every Cauchy sequence of functions in \(C(X \to Y)\) converges to a function in \(C(X \to Y)\).
\end{thm}

\begin{proof}
  Let \(d_{C(X \to Y)} = d_{B(X \to Y)}|_{C(X \to Y) \times C(X \to Y)}\) and let \(n_1, n_2 \in \Z^+\).
  Let \((f_n)_{n = 1}^\infty\) be a Cauchy sequence in \(\big(C(X \to Y), d_{C(X \to Y)}\big)\).
  Observe that
  \begin{align*}
             & \forall \varepsilon \in \R^+, \exists N \in \Z^+ : \forall n_1, n_2 \geq N,                                                          \\
             & d_{C(X \to Y)}\big(f^{(n_1)}, f^{(n_2)}\big) < \varepsilon                                                        &  & \by{ii:1.4.6} \\
    \implies & \forall \varepsilon \in \R^+, \exists N \in \Z^+ : \forall n_1, n_2 \geq N,                                                          \\
             & \sup_{x \in X} d_Y\big(f^{(n_1)}(x), f^{(n_2)}(x)\big) < \varepsilon                                              &  & \by{ii:3.4.2} \\
    \implies & \forall x \in X, \forall \varepsilon \in \R^+, \exists N \in \Z^+ : \forall n_1, n_2 \geq N,                                         \\
             & d_Y\big(f^{(n_1)}(x), f^{(n_2)}(x)\big) \leq \sup_{x \in X} d_Y\big(f^{(n_1)}(x), f^{(n_2)}(x)\big) < \varepsilon                    \\
    \implies & \forall x \in X, \big(f_n(x)\big)_{n = 1}^\infty \text{ is a Cauchy sequence in } (Y, d_Y).                       &  & \by{ii:1.4.6}
  \end{align*}
  By hypothesis we know that \((Y, d_Y)\) is complete, thus by \cref{ii:1.4.10} we have
  \[
    \forall x \in X, d_Y - \lim_{n \to \infty} f_n(x) \in Y
  \]
  and we can define a function \(f : X \to Y\) such that
  \[
    \forall x \in X, f(x) = d_Y - \lim_{n \to \infty} f_n(x).
  \]
  By \cref{ii:1.1.14} we have
  \[
    \forall x \in X, \forall \varepsilon \in \R^+, \exists N \in \Z^+ : \forall n \geq N, d_Y\big(f_n(x), f(x)\big) < \dfrac{\varepsilon}{3}.
  \]
  We choose one \(N\) for each pairs of \(x\) and \(\varepsilon\) and denote it as \(N_{x, \varepsilon}\).
  Since \(f_n \in C(X \to Y)\) for all \(n \in \Z^+\), by \cref{ii:2.1.1} we know that
  \[
    \forall x_0 \in X, \forall \varepsilon \in \R^+, \exists \delta \in \R^+ : \forall x \in X, d_X(x, x_0) < \delta \implies d_Y\big(f_n(x), f_n(x_0)\big) < \dfrac{\varepsilon}{3}.
  \]
  If we denote \(M_{x, x_0, \varepsilon} = \max(N_{x, \varepsilon}, N_{x_0, \varepsilon})\), then by \cref{ii:1.1.2}(d) we have
  \begin{align*}
             & \forall x_0 \in X, \forall \varepsilon \in \R^+, \exists \delta \in \R^+ : \forall x \in X, d_X(x, x_0) < \delta         \\
    \implies & \begin{dcases}
                 \forall n \geq M_{x, x_0, \varepsilon}, d_Y\big(f_n(x), f(x)\big) < \dfrac{\varepsilon}{3}     \\
                 \forall n \geq M_{x, x_0, \varepsilon}, d_Y\big(f_n(x_0), f(x_0)\big) < \dfrac{\varepsilon}{3} \\
                 d_Y\big(f_n(x), f_n(x_0)\big) < \dfrac{\varepsilon}{3}
               \end{dcases}                           \\
    \implies & \forall n \geq M_{x, x_0, \varepsilon},                                                                                  \\
             & d_Y\big(f(x), f(x_0)\big) \leq d_Y\big(f_n(x), f(x)\big) + d_Y\big(f_n(x), f_n(x_0)\big) + d_Y\big(f_n(x_0), f(x_0)\big) \\
             & < \dfrac{\varepsilon}{3} + \dfrac{\varepsilon}{3} + \dfrac{\varepsilon}{3} = \varepsilon                                 \\
    \implies & d_Y\big(f(x), f(x_0)\big) < \varepsilon.
  \end{align*}
  By \cref{ii:2.1.1} this means \(f \in C(X \to Y)\).
  Since \((f_n)_{n = 1}^\infty\) was arbitrary, by \cref{ii:1.4.10} \(\big(C(X \to Y), d_{C(X \to Y)}\big)\) is complete.
\end{proof}

\exercisesection

\begin{ex}\label{ii:ex:3.4.1}
  Let \((X, d_X)\) and \((Y, d_Y)\) be metric spaces.
  Show that the space \(B(X \to Y)\) defined in \cref{ii:3.4.2}, with the metric \(d_{B(X \to Y)}\), is indeed a metric space.
\end{ex}

\begin{proof}
  If \(X = \emptyset\), then by \cref{ii:3.4.2} we have
  \begin{itemize}
    \item If \(f \in B(\emptyset \to Y)\), then \(d_{B(X \to Y)}(f, f) = 0\).
    \item If \(f, g \in B(\emptyset \to Y)\), then \(d_{B(X \to Y)}(f, g) = 0 = d_{B(X \to Y)}(g, f)\).
    \item If \(f, g, h \in B(\emptyset \to Y)\), then \(d_{B(X \to Y)}(f, h) = 0 = d_{B(X \to Y)}(f, g) + d_{B(X \to Y)}(g, h)\).
  \end{itemize}
  Thus, by \cref{ii:1.1.2} \(\big(B(\emptyset \to Y), d_{B(X \to Y)}\big)\) is a metric space.
  Now suppose that \(X \neq \emptyset\).
  Since
  \begin{align*}
    \forall f \in B(X \to Y), d_{B(X \to Y)}(f, f) & = \sup_{x \in X} d_Y\big(f(x), f(x)\big) &  & \by{ii:3.4.2}    \\
                                                   & = \sup \set{0}                           &  & \by{ii:1.1.2}[a] \\
                                                   & = 0,
  \end{align*}
  by \cref{ii:1.1.2}(a) we know that \(\big(B(X \to Y), d_{B(X \to Y)}\big)\) is reflexive.
  Since
  \begin{align*}
             & \forall f, g \in B(X \to Y), f \neq g                                                    \\
    \implies & \exists x \in X : f(x) \neq g(x)                                                         \\
    \implies & \exists x \in X : d_Y\big(f(x), g(x)\big) > 0                      &  & \by{ii:1.1.2}[b] \\
    \implies & d_{B(X \to Y)}(f, g) = \sup_{x \in X} d_Y\big(f(x), g(x)\big) > 0, &  & \by{ii:3.4.2}
  \end{align*}
  by \cref{ii:1.1.2}(b) we know that \(\big(B(X \to Y), d_{B(X \to Y)}\big)\) is positive.
  Since
  \begin{align*}
    \forall f, g \in B(X \to Y), d_{B(X \to Y)}(f, g) & = \sup_{x \in X} d_Y\big(f(x), g(x)\big) &  & \by{ii:3.4.2}    \\
                                                      & = \sup_{x \in X} d_Y\big(g(x), f(x)\big) &  & \by{ii:1.1.2}[c] \\
                                                      & = d_{B(X \to Y)}(g, f),                  &  & \by{ii:3.4.2}
  \end{align*}
  by \cref{ii:1.1.2}(c) we know that \(\big(B(X \to Y), d_{B(X \to Y)}\big)\) is symmetry.
  Since
  \begin{align*}
     & \forall f, g, h \in B(X \to Y),                                                                         \\
     & d_{B(X \to Y)}(f, g) + d_{B(X \to Y)}(g, h)                                                             \\
     & = \sup_{x \in X} d_Y\big(f(x), g(x)\big) + \sup_{x \in X} d_Y\big(g(x), h(x)\big) &  & \by{ii:3.4.2}    \\
     & \geq \sup_{x \in X} \Big(d_Y\big(f(x), g(x)\big) + d_Y\big(g(x), h(x)\big)\Big)                         \\
     & \geq \sup_{x \in X} d_Y\big(f(x), h(x)\big)                                       &  & \by{ii:1.1.2}[d] \\
     & = d_{B(X \to Y)}(f, h),                                                           &  & \by{ii:3.4.2}
  \end{align*}
  by \cref{ii:1.1.2}(d) we know that \(\big(B(X \to Y), d_{B(X \to Y)}\big)\) is transitive.
  Combine all the proofs above we conclude by \cref{ii:1.1.2} that \(\big(B(X \to Y), d_{B(X \to Y)}\big)\) is a metric space.
\end{proof}

\begin{ex}\label{ii:ex:3.4.2}
  Prove \cref{ii:3.4.4}.
\end{ex}

\begin{proof}
  See \cref{ii:3.4.4}.
\end{proof}

\begin{ex}\label{ii:ex:3.4.3}
  Prove \cref{ii:3.4.5}.
\end{ex}

\begin{proof}
  See \cref{ii:3.4.5}.
\end{proof}

\begin{ex}\label{ii:ex:3.4.4}
  Let \((X, d_X)\) and \((Y, d_Y)\) be metric spaces, and let \(Y^X \coloneqq \set{f | f : X \to Y }\) be the space of all functions from \(X\) to \(Y\)
  (cf. Axiom 3.10 in Analysis I).
  If \(x_0 \in X\) and \(V\) is an open set in \(Y\), let \(V^{(x_0)} \subseteq Y^X\) be the set
  \[
    V^{(x_0)} \coloneqq \set{f \in Y^X : f(x_0) \in V}.
  \]
  If \(E\) is a subset of \(Y^X\), we say that \(E\) is \emph{open} if for every \(f \in E\), there exists a finite number of points \(x_1, \dots, x_n \in X\) and open sets \(V_1, \dots, V_n \subseteq Y\) such that
  \[
    f \in V_1^{(x_1)} \cap \dots \cap V_n^{(x_n)} \subseteq E.
  \]
  \begin{itemize}
    \item Show that if \(\mathcal{F}\) is the collection of open sets in \(Y^X\), then \((Y^X , \mathcal{F})\) is a topological space.
    \item For each natural number \(n\), let \(f^{(n)} : X \to Y\) be a function from \(X\) to \(Y\), and let \(f : X \to Y\) be another function from \(X\) to \(Y\).
          Show that \(f^{(n)}\) converges to \(f\) in the topology \(\mathcal{F}\) (in the sense of \cref{ii:2.5.4}) iff \(f^{(n)}\) converges to \(f\) pointwise (in the sense of \cref{ii:3.2.1}).
  \end{itemize}
  The topology \(\mathcal{F}\) is known as the \emph{topology of pointwise convergence}, for obvious reasons;
  it is also known as the \emph{product topology}.
  It shows that the concept of pointwise convergence can be viewed as a special case of the more general concept of convergence in a topological space.
\end{ex}

\begin{proof}
  We know that \(\emptyset \in \mathcal{F}\) trivially.
  First we show that \(Y^X \in \mathcal{F}\).
  Let \(f \in Y^X\) and let \(x_0 \in X\).
  By \cref{ii:1.2.15}(c) we know that \(B_{(Y, d_Y)}\big(f(x_0), 1\big)\) is open in \((Y, d_Y)\).
  Then we have
  \[
    f \in \Big(B_{(Y, d_Y)}\big(f(x_0), 1\big)\Big)^{(x_0)} \subseteq Y^X.
  \]
  Since \(f\) was arbitrary, by definition we know that \(Y^X \in \mathcal{F}\).

  Next we show that the intersection of any finite collection of open sets in \((Y^X, \mathcal{F})\) is open in \((Y^X, \mathcal{F})\).
  Let \(n \in \N\) and let \((U_i)_{i = 1}^n\) be a finite collection of open sets in \((Y^X, \mathcal{F})\).
  If \(\bigcap_{i = 1}^n U_i = \emptyset\), then from the proof above we know that \(\emptyset \in \mathcal{F}\).
  So suppose that \(\bigcap_{i = 1}^n U_i \neq \emptyset\).
  Let \(f \in \bigcap_{i = 1}^n U_i\).
  Since
  \begin{align*}
             & \forall 1 \leq i \leq n, f \in U_i                                                                                  \\
    \implies & \forall 1 \leq i \leq n, \exists m_i \in \Z^+ : \begin{dcases}
                                                                 x_{(i, 1)}, \dots, x_{(i, m_i)} \in X                              \\
                                                                 V_{(i, 1)}, \dots, V_{(i, m_i)} \text{ are open sets in } (Y, d_Y) \\
                                                                 f \in \bigcap_{j = 1}^{m_i} V_{(i, j)}^{(x_{(i, j)})} \subseteq U_i
                                                               \end{dcases}  \\
    \implies & f \in \bigcap_{i = 1}^n \bigg(\bigcap_{j = 1}^{m_i} V_{(i, j)}^{(x_{(i, j)})}\bigg) \subseteq \bigcap_{i = 1}^n U_i
  \end{align*}
  and \(f\) was arbitrary, we know that \(\bigcap_{i = 1}^n U_i \in \mathcal{F}\).
  Since \(n\) was arbitrary, we know that the intersection of any finite collection of open sets in \((Y^X, \mathcal{F})\) is open in \((Y^X, \mathcal{F})\).

  Next we show that the union of arbitrary open sets in \((Y^X, \mathcal{F})\) is open in \((Y^X, \mathcal{F})\).
  Let \(S \subseteq \mathcal{F}\) and let \(f \in \bigcup S\).
  Since
  \begin{align*}
             & f \in \bigcup S                                                                      \\
    \implies & \exists U \in S : f \in U                                                            \\
    \implies & \exists U \in S : \begin{dcases}
                                   x_1, \dots, x_n \in X                              \\
                                   V_1, \dots, V_n \text{ are open sets in } (Y, d_Y) \\
                                   f \in \bigcap_{i = 1}^n V_i^{(x_i)} \subseteq U \subseteq \bigcup S
                                 \end{dcases}
  \end{align*}
  and \(f\) was arbitrary, we know that \(\bigcup S \in \mathcal{F}\).
  Since \(S\) was arbitrary, we know that the union of arbitrary open sets in \((Y^X, \mathcal{F})\) is open in \((Y^X, \mathcal{F})\).
  Combine all the proofs above we conclude by \cref{ii:2.5.1} that \((Y^X, \mathcal{F})\) is a topological space.

  Next suppose that \((f^{(n)})_{n = 1}^\infty\) is a sequence in \(Y^X\) and \(f \in Y^X\).
  Suppose also that \((f^{(n)})_{n = 1}^\infty\) converges to \(f\) in \((Y^X, \mathcal{F})\).
  Then by \cref{ii:2.5.4} we have
  \[
    \forall E \in \mathcal{F}, f \in E \implies \exists N \in \Z^+ : \forall n \geq N, f^{(n)} \in E.
  \]
  Let \(x_0 \in X\).
  Then we have
  \begin{align*}
             & \forall \varepsilon \in \R^+, B_{(Y, d_Y)}\big(f(x_0), \varepsilon\big) \text{ is open in } (Y, d_Y)            &  & \by{ii:1.2.15}[c]      \\
    \implies & \forall \varepsilon \in \R^+, f \in \Big(B_{(Y, d_Y)}\big(f(x_0), \varepsilon\big)\Big)^{(x_0)} \in \mathcal{F} &  & \text{(by definition)} \\
    \implies & \forall \varepsilon \in \R^+, \exists N \in \Z^+ : \forall n \geq N,                                                                        \\
             & f^{(n)} \in \Big(B_{(Y, d_Y)}\big(f(x_0), \varepsilon\big)\Big)^{(x_0)}                                         &  & \by{ii:2.5.4}          \\
    \implies & \forall \varepsilon \in \R^+, \exists N \in \Z^+ : \forall n \geq N,                                                                        \\
             & d_Y\big(f^{(n)}(x_0), f(x_0)\big) < \varepsilon                                                                 &  & \text{(by definition)} \\
    \implies & \lim_{n \to \infty} d_Y\big(f^{(n)}(x_0), f(x_0)\big).                                                          &  & \by{ii:1.1.14}
  \end{align*}
  Since \(x_0\) was arbitrary, by \cref{ii:3.2.1} \((f^{(n)})_{n = 1}^\infty\) converges pointwise to \(f\) on \(X\) with respect to \(d_Y\).

  Finally suppose that \((f^{(n)})_{n = 1}^\infty\) is a sequence in \(Y^X\) and \(f \in Y^X\).
  Suppose also that \((f^{(n)})_{n = 1}^\infty\) converges pointwise to \(f\) on \(X\) with respect to \(d_Y\).
  Then we have
  \begin{align*}
             & \forall x \in X, \lim_{n \to \infty} d_Y\big(f^{(n)}(x), f(x)\big)                    &  & \by{ii:3.2.1}  \\
    \implies & \forall x \in X, \forall \varepsilon \in \R^+, \exists N \in \Z^+ : \forall n \geq N,                     \\
             & d_Y\big(f^{(n)}(x), f(x)\big) < \varepsilon.                                          &  & \by{ii:1.1.14}
  \end{align*}
  We choose one \(N\) for each pair of \((x, \varepsilon)\) and denote it as \(N_{(x, \varepsilon)}\).
  Let \(E \in \mathcal{F}\) such that \(f \in E\).
  By definition we know that
  \[
    \exists m \in \Z^+ : \begin{dcases}
      x_1, \dots, x_m \in X                              \\
      V_1, \dots, V_m \text{ are open sets in } (Y, d_Y) \\
      f \in \bigcap_{i = 1}^m V_i^{(x_i)} \subseteq E
    \end{dcases}
  \]
  Then we have
  \begin{align*}
             & \forall 1 \leq i \leq m, f \in V_i^{(x_i)}                                                                                                                    \\
    \implies & \forall 1 \leq i \leq m, \begin{dcases}
                                          f(x_i) \in V_i \\
                                          V_i \text{ is open in } (Y, d_Y)
                                        \end{dcases}                                                                                                      \\
    \implies & \forall 1 \leq i \leq m, \exists \varepsilon_i \in \R^+ : B_{(Y, d_Y)}\big(f(x_i), \varepsilon_i\big) \subseteq V_i                    &  & \by{ii:1.2.15}[a] \\
    \implies & \forall 1 \leq i \leq m, \exists \varepsilon_i \in \R^+ : \exists N_{(x_i, \varepsilon_i)} \in \Z^+ :                                                         \\
             & \forall n \geq N_{(x_i, \varepsilon_i)}, f^{(n)}(x_i) \in B_{(Y, d_Y)}\big(f(x_i), \varepsilon_i\big) \subseteq V_i                                           \\
    \implies & \exists N = \max_{1 \leq i \leq m} N_{(x_i, \varepsilon_i)} : \forall n \geq N, f^{(n)} \in \bigcap_{i = 1}^m V_i^{(x_i)} \subseteq E.
  \end{align*}
  Since \(E\) was arbitrary, by \cref{ii:2.5.4} we know that \((f^{(n)})_{n = 1}^\infty\) converges to \(f\) in \((Y^X, \mathcal{F})\).
\end{proof}

\section{Series of functions; the Weierstrass \emph{M}-test}\label{sec:3.5}

\begin{note}
  Functions whose range is \(\R\) are sometimes called \emph{real-valued} functions.
\end{note}

\begin{note}
  given any finite collection \(f^{(1)}, \dots, f^{(N)}\) of functions from \(X\) to \(\R\), we can define the finite sum \(\sum_{i = 1}^N f^{(i)} : X \to \R\) by
  \[
    \bigg(\sum_{i = 1}^N f^{(i)}\bigg)(x) \coloneqq \sum_{i = 1}^N f^{(i)}(x).
  \]
\end{note}

\setcounter{thm}{1}
\begin{defn}[Infinite series]\label{3.5.2}
  Let \((X, d_X)\) be a metric space.
  Let \((f^{(n)})_{n = 1}^\infty\) be a sequence of functions from \(X\) to \(\R\), and let \(f\) be another function from \(X\) to \(\R\).
  If the partial sums \(\sum_{n = 1}^N f^{(n)}\) converges pointwise to \(f\) on \(X\) as \(N \to \infty\), we say that the infinite series \(\sum_{n = 1}^\infty f^{(n)}\) \emph{converges pointwise} to \(f\), and write \(f = \sum_{n = 1}^\infty f^{(n)}\).
  If the partial sums \(\sum_{n = 1}^N f^{(n)}\) converge uniformly to \(f\) on \(X\) as \(N \to \infty\), we say that the infinite series \(\sum_{n = 1}^\infty f^{(n)}\) \emph{converges uniformly} to \(f\), and again write \(f = \sum_{n = 1}^\infty f^{(n)}\).
  (Thus when one sees an expression such as \(f = \sum_{n = 1}^\infty f^{(n)}\), one should look at the context to see in what sense this infinite series converges.)
\end{defn}

\begin{rmk}\label{3.5.3}
  A series \(\sum_{n = 1}^\infty f^{(n)}\) converges pointwise to \(f\) on \(X\) if and only if \(\sum_{n = 1}^\infty f^{(n)}(x)\) converges to \(f(x)\) for \emph{every} \(x \in X\).
  (Thus if \(\sum_{n = 1}^\infty f^{(n)}\) does not converge pointwise to \(f\), this does not mean that it diverges pointwise;
  it may just be that it converges for some points \(x\) but diverges at other points \(x\).)
\end{rmk}

\begin{note}
  If a series \(\sum_{n = 1}^\infty f^{(n)}\) converges uniformly to \(f\), then it also converges pointwise to \(f\);
  but not vice versa.
\end{note}

\setcounter{thm}{4}
\begin{defn}[Sup norm]\label{3.5.5}
  If \(f : X \to \R\) is a bounded real-valued function, we define the \emph{sup norm} \(\norm*{f}_\infty\) of \(f\) to be the number
  \[
    \norm*{f}_\infty \coloneqq \sup\big\{\abs{f(x)} : x \in X\big\}.
  \]
  In other words, \(\norm*{f}_\infty = d_\infty(f, 0)\), where \(0 : X \to \R\) is the zero function \(0(x) \coloneqq 0\), and \(d_\infty\) was defined in \cref{3.4.2}.
  We restrict the definition of \(\norm*{f}_\infty\) to the case when \(X \neq \emptyset\).
  If \(X = \emptyset\), then we instead define \(\norm*{f}_\infty = 0\).
\end{defn}

\begin{note}
  When \(f\) is bounded then \(\norm*{f}_\infty\) will always be a non-negative real number.
\end{note}

\setcounter{thm}{6}
\begin{thm}[Weierstrass \(M\)-test]\label{3.5.7}
  Let \((X, d)\) be a metric space, and let \((f^{(n)})_{n = 1}^\infty\) be a sequence of bounded real-valued continuous functions on \(X\) such that the series \(\sum_{n = 1}^\infty \norm*{f^{(n)}}_\infty\) is convergent.
  (Note that this is a series of plain old real numbers, not of functions.)
  Then the series \(\sum_{n = 1}^\infty f^{(n)}\) converges uniformly to some function \(f\) on \(X\), and that function \(f\) is also continuous.
\end{thm}

\begin{proof}
  Let \(N_1, N_2 \in \Z^+\).
  Let \(d_{C(X \to \R)} = d_{B(X \to \R)}|_{C(X \to \R) \times C(X \to \R)}\).
  We have
  \begin{align*}
             & \sum_{n = 1}^\infty \norm*{f^{(n)}}_\infty = \lim_{N \to \infty} \sum_{n = 1}^N \norm*{f^{(n)}}_\infty                                                                                     \\
    \implies & \forall \varepsilon \in \R^+, \exists\ M \in \Z^+ : \forall N \geq M,                                                                                                                      \\
             & \abs{\sum_{n = 1}^\infty \norm*{f^{(n)}}_\infty - \sum_{n = 1}^N \norm*{f^{(n)}}_\infty} < \varepsilon &                                 & \text{(by \cref{1.1.14})}                       \\
    \implies & \forall \varepsilon \in \R^+, \exists\ M \in \Z^+ : \forall N \geq M,                                                                                                                      \\
             & \abs{\sum_{n = N + 1}^\infty \norm*{f^{(n)}}_\infty} < \varepsilon                                     &                                 & \text{(by Proposition 7.2.14(c) in Analysis I)} \\
    \implies & \forall \varepsilon \in \R^+, \exists\ M \in \Z^+ : \forall N \geq M,                                                                                                                      \\
             & \sum_{n = N + 1}^\infty \norm*{f^{(n)}}_\infty < \varepsilon                                           & (\norm*{f^{(n)}}_\infty \geq 0)                                                   \\
    \implies & \forall \varepsilon \in \R^+, \exists\ M \in \Z^+ : \forall N \geq M,                                                                                                                      \\
             & \sum_{n = N + 1}^\infty \sup_{x \in X} \abs{f^{(n)}(x)} < \varepsilon.                                 &                                 & \text{(by \cref{3.5.5})}
  \end{align*}
  Fix one \(\varepsilon\) and \(M\).
  Since \(f^{(n)} \in C(X \to \R)\), by \cref{ex:3.5.1} we know that \(\sum_{n = 1}^N f^{(n)} \in C(X \to \R)\) for each \(N \in \Z^+\).
  Thus \(d_{C(X \to \R)}\bigg(\sum_{n = 1}^{N_1} f^{(n)}, \sum_{n = 1}^{N_2} f^{(n)}\bigg)\) is well defined for each \(N_1, N_2 \geq M\) and
  \begin{align*}
    \forall N_1, N_2 \geq M, & d_{C(X \to \R)}\bigg(\sum_{n = 1}^{N_1} f^{(n)}, \sum_{n = 1}^{N_2} f^{(n)}\bigg)                                             \\
                             & = \sup_{x \in X} \abs{\sum_{n = 1}^{N_1} f^{(n)}(x) - \sum_{n = 1}^{N_2} f^{(n)}(x)}            &  & \text{(by \cref{3.4.2})} \\
                             & = \sup_{x \in X} \abs{\sum_{n = \min(N_1, N_2) + 1}^{\max(N_1, N_2)} f^{(n)}(x)}                                              \\
                             & \leq \sup_{x \in X} \bigg(\sum_{n = \min(N_1, N_2) + 1}^{\max(N_1, N_2)} \abs{f^{(n)}(x)}\bigg)                               \\
                             & \leq \sup_{x \in X} \bigg(\sum_{n = M + 1}^\infty \abs{f^{(n)}(x)}\bigg)                                                      \\
                             & \leq \sum_{n = M + 1}^\infty \sup_{x \in X} \abs{f^{(n)}(x)} < \varepsilon.
  \end{align*}
  Since \(\varepsilon\) is arbitrary, we have
  \[
    \forall \varepsilon \in \R^+, \exists\ M \in \Z^+ : \forall N_1, N_2 \geq M, d_{C(X \to \R)}\bigg(\sum_{n = 1}^{N_1} f^{(n)}, \sum_{n = 1}^{N_2} f^{(n)}\bigg) < \varepsilon.
  \]
  By \cref{1.4.6} \(\bigg(\sum_{n = 1}^N f^{(n)}\bigg)_{N = 1}^\infty\) is a Cauchy sequence in \(\big(C(X \to \R), d_{C(X \to \R)}\big)\).
  Since \((\R, d_{l^1}|_{\R \times \R})\) is complete, by \cref{3.4.5} we know that \(\bigg(\sum_{n = 1}^N f^{(n)}\bigg)_{N = 1}^\infty\) converges uniformly to some \(f \in C(X \to \R)\) on \(X\) with respect to \(d_{l^1}|_{\R \times \R}\).
\end{proof}

\begin{note}
  To put the Weierstrass \(M\)-test succinctly:
  absolute convergence of sup norms implies uniform convergence of functions.
\end{note}

\begin{eg}\label{3.5.8}
  Let \(0 < r < 1\) be a real number, and let \(f^{(n)} : [-r, r] \to \R\) be the series of functions \(f^{(n)}(x) \coloneqq x^n\).
  Then each \(f^{(n)}\) is continuous and bounded, and \(\norm*{f^{(n)}}_\infty = r^n\).
  Since the series \(\sum_{n = 1}^\infty r^n\) is absolutely convergent (e.g., by the root test, Theorem 7.5.1 in Analysis I), we thus see that \(\sum_{n = 1}^\infty f^{(n)}\) converges uniformly in \([-r, r]\) to some continuous function;
  in \cref{ex:3.2.2}(c) we see that this function must in fact be the function \(f : [-r, r] \to \R\) defined by \(f(x) \coloneqq x / (1 - x)\).
  In other words, the series \(\sum_{n = 1}^\infty x^n\) is pointwise convergent, but not uniformly convergent, on \((-1, 1)\), but is uniformly convergent on the smaller interval \([-r, r]\) for any \(0 < r < 1\).
\end{eg}

\begin{note}
  The Weierstrass \(M\)-test is especially useful in relation to power series.
\end{note}

\exercisesection

\begin{ex}\label{ex:3.5.1}
  Let \(f^{(1)}, \dots, f^{(N)}\) be a finite sequence of bounded functions from a metric space \((X, d_X)\) to \(\R\).
  Show that \(\sum_{i = 1}^N f^{(i)}\) is also bounded.
  Prove a similar claim when ``bounded'' is replaced by ``continuous''.
  What if ``continuous'' was replaced by ``uniformly continuous''?
\end{ex}

\begin{proof}
  Let \(d_1 = d_{l^1}|_{\R \times \R}\).
  We first show that \(\sum_{n = 1}^N f^{(n)}\) is bounded on \(X\) with respect to \(d_1\) for each \(N \in \Z^+\).
  Suppose that \(f^{(n)}\) is bounded on \(X\) with respect to \(d_1\) for each \(n \in \Z^+\).
  We use induction on \(N\).
  For \(N = 1\), by hypothesis we know that \(\sum_{n = 1}^1 f^{(n)} = f^{(1)}\) is bounded on \(X\).
  Thus the base case holds.
  Suppose inductively that \(\sum_{n = 1}^N f^{(n)}\) is bounded on \(X\) with respect to \(d_1\) for some \(N \geq 1\).
  By induction hypothesis we have
  \[
    \exists\ M \in \R^+ : \bigg(\sum_{n = 1}^N f^{(n)}\bigg)(X) \subseteq [-M, M].
  \]
  By hypothesis we know that \(f^{(N + 1)}\) is bounded on \(X\) with respect to \(d_1\), thus we have
  \[
    \exists\ M' \in \R^+ : f^{(N + 1)}(X) \subseteq [-M', M'].
  \]
  Then we have
  \begin{align*}
    \bigg(\sum_{n = 1}^{N + 1} f^{(n)}\bigg)(X) & = \bigg\{\sum_{n = 1}^{N + 1} f^{(n)}(x) : x \in X\bigg\}            \\
                                                & = \bigg\{\sum_{n = 1}^N f^{(n)}(x) + f^{(N + 1)}(x) : x \in X\bigg\} \\
                                                & \subseteq [-(M + M'), M + M'].
  \end{align*}
  This closes the induction.

  Next we show that \(\sum_{n = 1}^N f^{(n)}\) is continuous from \((X, d_X)\) to \((\R, d_1)\) for each \(N \in \Z^+\).
  Suppose that \(f^{(n)}\) is continuous from \((X, d_X)\) to \((\R, d_1)\) for each \(n \in \Z^+\).
  We use induction on \(N\).
  For \(N = 1\), by hypothesis we know that \(\sum_{n = 1}^1 f^{(n)} = f^{(1)}\) is continuous from \((X, d_X)\) to \((\R, d_1)\).
  Thus the base case holds.
  Suppose inductively that \(\sum_{n = 1}^N f^{(n)}\) is continuous from \((X, d_X)\) to \((\R, d_1)\) for some \(N \geq 1\).
  Then by \cref{ac:2.2.1}
  \[
    \sum_{n = 1}^{N + 1} f^{(n)} = \bigg(\sum_{n = 1}^N f^{(n)}\bigg) \oplus f^{(N + 1)}
  \]
  is continuous from \((X, d_X)\) to \((\R, d_1)\).
  This closes the induction.

  Finally we show that \(\sum_{n = 1}^N f^{(n)}\) is uniformly continuous from \((X, d_X)\) to \((\R, d_1)\) for each \(N \in \Z^+\).
  Suppose that \(f^{(n)}\) is uniformly continuous from \((X, d_X)\) to \((\R, d_1)\) for each \(n \in \Z^+\).
  We use induction on \(N\).
  For \(N = 1\), by hypothesis we know that \(\sum_{n = 1}^1 f^{(n)} = f^{(1)}\) is uniformly continuous from \((X, d_X)\) to \((\R, d_1)\).
  Thus the base case holds.
  Suppose inductively that \(\sum_{n = 1}^N f^{(n)}\) is uniformly continuous from \((X, d_X)\) to \((\R, d_1)\) for some \(N \geq 1\).
  Then by \cref{ex:2.3.5}
  \[
    \sum_{n = 1}^{N + 1} f^{(n)} = \bigg(\sum_{n = 1}^N f^{(n)}\bigg) \oplus f^{(N + 1)}
  \]
  is uniformly continuous from \((X, d_X)\) to \((\R, d_1)\).
  This closes the induction.
\end{proof}

\begin{ex}\label{ex:3.5.2}
  Prove \cref{3.5.7}.
\end{ex}

\begin{proof}
  See \cref{3.5.7}.
\end{proof}
\section{Uniform convergence and integration}\label{ii:sec:3.6}

\begin{thm}\label{ii:3.6.1}
  Let \([a, b]\) be an interval, and for each integer \(n \geq 1\), let \(f^{(n)} : [a, b] \to \R\) be a Riemann-integrable function.
  Suppose \(f^{(n)}\) converges uniformly on \([a, b]\) to a function \(f : [a, b] \to \R\).
  Then \(f\) is also Riemann integrable, and
  \[
    \lim_{n \to \infty} \int_{[a, b]} f^{(n)} = \int_{[a, b]} f.
  \]
\end{thm}

\begin{proof}
  We first show that \(f\) is Riemann integrable on \([a, b]\).
  This is the same as showing that the upper and lower Riemann integrals of \(f\) match:
  \(\underline{\int}_{[a, b]} f = \overline{\int}_{[a, b]} f\).

  Let \(\varepsilon > 0\).
  Since \(f^{(n)}\) converges uniformly to \(f\), we see that there exists an \(N > 0\) such that \(\abs{f^{(n)}(x) - f(x)} < \varepsilon\) for all \(n > N\) and \(x \in [a, b]\).
  In particular we have
  \[
    f^{(n)}(x) - \varepsilon < f(x) < f^{(n)}(x) + \varepsilon
  \]
  for all \(x \in [a, b]\).
  Integrating this on \([a, b]\) we obtain
  \[
    \underline{\int}_{[a, b]} (f^{(n)} - \varepsilon) \leq \underline{\int}_{[a, b]} f \leq \overline{\int}_{[a, b]} f \leq \overline{\int}_{[a, b]} (f^{(n)} + \varepsilon).
  \]
  Since \(f^{(n)}\) is assumed to be Riemann integrable, we thus see
  \[
    \Bigg(\int_{[a, b]} f^{(n)}\Bigg) - \varepsilon (b - a) \leq \underline{\int}_{[a, b]} f \leq \overline{\int}_{[a, b]} f \leq \Bigg(\int_{[a, b]} f^{(n)}\Bigg) + \varepsilon (b - a).
  \]
  In particular, we see that
  \[
    0 \leq \overline{\int}_{[a, b]} f - \underline{\int}_{[a, b]} f \leq 2 \varepsilon (b - a).
  \]
  Since this is true for every \(\varepsilon > 0\), we obtain \(\underline{\int}_{[a, b]} f = \overline{\int}_{[a, b]} f\) as desired.

  The above argument also shows that for every \(\varepsilon > 0\) there exists an \(N > 0\) such that
  \[
    \abs{\int_{[a, b]} f^{(n)} - \int_{[a, b]} f} \leq \varepsilon (b - a)
  \]
  for all \(n \geq N\).
  Since \(\varepsilon\) is arbitrary, we see that \(\int_{[a, b]} f^{(n)}\) converges to \(\int_{[a, b]} f\) as desired.
\end{proof}

\begin{note}
  To rephrase \cref{ii:3.6.1}:
  we can rearrange limits and integrals (on compact intervals \([a, b]\)),
  \[
    \lim_{n \to \infty} \int_{[a, b]} f^{(n)} = \int_{[a, b]} \lim_{n \to \infty} f^{(n)},
  \]
  \emph{provided that} the convergence is uniform.
\end{note}

\begin{cor}\label{ii:3.6.2}
  Let \([a, b]\) be an interval, and let \((f^{(n)})_{n = 1}^\infty\) be a sequence of Riemann integrable functions on \([a, b]\) such that the series \(\sum_{n = 1}^\infty f^{(n)}\) is uniformly convergent.
  Then we have
  \[
    \sum_{n = 1}^\infty \int_{[a, b]} f^{(n)} = \int_{[a, b]} \sum_{n = 1}^\infty f^{(n)}.
  \]
\end{cor}

\begin{proof}
  By Theorem 11.4.1(a) in Analysis I we know that
  \[
    \forall N \in \Z^+, \int_{[a, b]} \sum_{n = 1}^N f^{(n)} = \sum_{n = 1}^N \int_{[a, b]} f^{(n)}.
  \]
  Let \(f : [a, b] \to \R\) be the function such that \(\sum_{n = 1}^\infty f^{(n)}\) converges uniformly to \(f\) on \([a, b]\) with respect to \(d_{l^1}|_{\R \times \R}\).
  By \cref{ii:3.6.1} we have
  \[
    \sum_{n = 1}^\infty \int_{[a, b]} f^{(n)} = \lim_{N \to \infty} \sum_{n = 1}^N \int_{[a, b]} f^{(n)} = \lim_{N \to \infty} \int_{[a, b]} \sum_{n = 1}^N f^{(n)} = \int_{[a, b]} f = \int_{[a, b]} \sum_{n = 1}^\infty f^{(n)}.
  \]
\end{proof}

\begin{note}
  \cref{ii:3.6.2} works particularly well in conjunction with the Weierstrass \(M\)-test
  (\cref{ii:3.5.7}).
\end{note}

\exercisesection

\begin{ex}\label{ii:ex:3.6.1}
  Use \cref{ii:3.6.1} to prove \cref{ii:3.6.2}.
\end{ex}

\begin{proof}
  See \cref{ii:3.6.2}.
\end{proof}

\section{Uniform convergence and derivatives}\label{sec:3.7}

\begin{note}
  In particular we have
  \[
    \dfrac{d}{dx} \lim_{n \to \infty} f_n(x) \neq \lim_{n \to \infty} \dfrac{d}{dx} f_n(x)
  \]
  So, in summary, uniform convergence of the functions \(f_n\) says nothing about the convergence of the derivatives \(f_n'\).
\end{note}

\begin{thm}\label{3.7.1}
  Let \([a, b]\) be an interval, and for every integer \(n \geq 1\), let \(f_n : [a, b] \to \R\) be a differentiable function whose derivative \(f_n' : [a, b] \to \R\) is continuous.
  Suppose that the derivatives \(f_n'\) converge uniformly to a function \(g : [a, b] \to \R\).
  Suppose also that there exists a point \(x_0\) such that the limit \(\lim_{n \to \infty} f_n(x_0)\) exists.
  Then the functions \(f_n\) converge uniformly to a differentiable function \(f\), and the derivative of \(f\) equals \(g\).
\end{thm}

\begin{proof}
  Since \(f_n'\) is continuous, by Corollary 11.5.2 in Analysis I we know that \(f_n'\) is Riemann integrable.
  We see from the fundamental theorem of calculus (Theorem 11.9.4 in Analysis I) that
  \[
    f_n(x) - f_n(x_0) = \int_{[x_0, x]} f_n'
  \]
  when \(x \in [x_0, b]\), and
  \[
    f_n(x) - f_n(x_0) = -\int_{[x, x_0]} f_n'
  \]
  when \(x \in [a, x_0]\).
  Let \(L\) be the limit of \(f_n(x_0)\) as \(n \to \infty\):
  \[
    L \coloneqq \lim_{n \to \infty} f_n(x_0).
  \]
  By hypothesis, \(L\) exists.
  Now, since each \(f_n'\) is continuous on \([a, b]\), and \(f_n'\) converges uniformly to \(g\), we see by \cref{3.3.2} that \(g\) is also continuous.
  By \cref{3.6.1} we have
  \[
    \forall x \in [a, b], \lim_{n \to \infty} \big(f_n(x) - f_n(a)\big) = \lim_{n \to \infty} \int_{[a, x]} f_n' = \int_{[a, x]} \big(\lim_{n \to \infty} f_n'\big) = \int_{[a, x]} g.
  \]
  Now define the function \(f : [a, b] \to \R\) by setting
  \[
    f(x) \coloneqq L - \int_{[a, x_0]} g + \int_{[a, x]} g
  \]
  for all \(x \in [a, b]\).
  To finish the proof, we have to show that \(f_n\) converges uniformly to \(f\), and that \(f\) is differentiable with derivative \(g\).

  We know that \(a \neq b\) since if \(a = b\), then we have \(x_0 = a = b\) and
  \[
    \forall n \in \Z^+, \lim_{x \to x_0; x \in \{x_0\} \setminus \{x_0\}} \dfrac{f_n(x) - f_n(x_0)}{x - x_0} \text{ is undefined}
  \]
  which contradict to the hypothesis that \(f_n\) is differentiable on \([a, b]\).
  Observe that
  \begin{align*}
             & L = \lim_{n \to \infty} f_n(x_0)                                                                                              \\
    \implies & \forall \varepsilon \in \R^+, \exists\ N_1 \in \Z^+ : \forall n \geq N_1, \abs{f_n(x_0) - L} < \dfrac{\varepsilon}{3(b - a)}.
  \end{align*}
  Now we fix one pair of \(\varepsilon\) and \(N_1\).
  Since \((f_n')_{n = 1}^\infty\) converges uniformly to \(g\) on \([a, b]\) with respect to \(d_{l^1}|_{\R \times \R}\), by \cref{3.2.7} we have
  \begin{align*}
             & \exists\ N_2 \in \Z^+ : \forall n \geq N_2, \forall x \in [a, b],                                                                 \\
             & \abs{f_n'(x) - g(x)} < \dfrac{\varepsilon}{3(b - a)}                                                                              \\
    \implies & \exists\ N_2 \in \Z^+ : \forall n \geq N_2, \forall x \in [a, b],                                                                 \\
             & \dfrac{-\varepsilon}{3(b - a)} < f_n'(x) - g(x) < \dfrac{\varepsilon}{3(b - a)}                                                   \\
    \implies & \exists\ N_2 \in \Z^+ : \forall n \geq N_2, \forall x \in [a, b],                                                                 \\
             & \dfrac{-\varepsilon (x - a)}{3(b - a)} \leq \int_{[a, x]} f_n'(x) - \int_{[a, x]} g(x) \leq \dfrac{\varepsilon (x - a)}{3(b - a)} \\
    \implies & \exists\ N_2 \in \Z^+ : \forall n \geq N_2, \forall x \in [a, b],                                                                 \\
             & \dfrac{-\varepsilon (x - a)}{3(b - a)} \leq f_n(x) - f_n(a) - \int_{[a, x]} g(x) \leq \dfrac{\varepsilon (x - a)}{3(b - a)}       \\
    \implies & \exists\ N_2 \in \Z^+ : \forall n \geq N_2, \forall x \in [a, b],                                                                 \\
             & \abs{f_n(x) - f_n(a) - \int_{[a, x]} g(x)} \leq \dfrac{\varepsilon \abs{x - a}}{3(b - a)}.
  \end{align*}
  Let \(N = \max(N_1, N_2)\).
  Then we have
  \begin{align*}
     & \forall n \geq N, \forall x \in [a, b], \abs{f_n(x) - f(x)}                                                               \\
     & = \abs{f_n(x) - f_n(x_0) + f_n(x_0) - f_n(a) + f_n(a) - L + \int_{[a, x_0]} g - \int_{[a, x]} g}                          \\
     & \leq \abs{f_n(x) - f_n(a) - \int_{[a, x]} g} + \abs{f_n(x_0) - L} + \abs{f_n(x_0) - f_n(a) - \int_{[a, x_0]} g}           \\
     & < \dfrac{\varepsilon \abs{x - a}}{3(b - a)} + \dfrac{\varepsilon \abs{x_0 - a}}{3(b - a)} + \dfrac{\varepsilon}{3(b - a)} \\
     & < \dfrac{\varepsilon}{3} + \dfrac{\varepsilon}{3} + \dfrac{\varepsilon}{3} = \varepsilon.
  \end{align*}
  Since \(\varepsilon\) is arbitrary, we have
  \[
    \forall \varepsilon \in \R^+, \exists\ N \in \Z^+ : \forall n \geq N, \forall x \in [a, b], \abs{f_n(x) - f(x)} < \varepsilon
  \]
  and by \cref{3.2.7} \((f_n)_{n = 1}^\infty\) converges uniformly to \(f\) with respect to \(d_{l^1}|_{\R \times \R}\).

  Since \(f_n\) is continuous on \([a, b]\) for each \(n \in \Z^+\), by \cref{3.3.2} we know that \(f\) is also continuous on \([a, b]\).
  Since \(g\) is continuous on \([a, b]\), by fundamental theorem of calculus (Theorem 11.9.1 in Analysis) we know that
  \[
    \forall x \in X, G(x) = \int_{[a, x]} g \text{ is differentiable at } x.
  \]
  Since \(L + \int_{[a, x_0]} g\) is constant, we know that
  \[
    \forall x \in X, f(x) = L + \int_{[a, x_0]} g + G(x) = L + \int_{[a, x_0]} g + \int_{[a, x]} g \text{ is differentiable at } x
  \]
  and by fundamental theorem of calculus (Theorem 11.9.1 in Analysis) we have
  \[
    \forall x \in X, f'(x) = \bigg(\int_{[a, x]} g\bigg)' = g(x).
  \]
\end{proof}

\begin{note}
  Informally, \cref{3.7.1} says that if \(f_n'\) converges uniformly, and \(f_n(x_0)\) converges for some \(x_0\), then \(f_n\) also converges uniformly, and
  \[
    \dfrac{d}{dx} \lim_{n \to \infty} f_n(x) = \lim_{n \to \infty} \dfrac{d}{dx} f_n(x)
  \]
\end{note}

\begin{rmk}\label{3.7.2}
  It turns out that \cref{3.7.1} is still true when the functions \(f_n'\) are not assumed to be continuous, but the proof is more difficult;
  see \cref{ex:3.7.2}.
\end{rmk}

\begin{cor}\label{3.7.3}
  Let \([a, b]\) be an interval, and for every integer \(n \geq 1\), let \(f_n : [a, b] \to \R\) be a differentiable function whose derivative \(f_n' : [a, b] \to \R\) is continuous.
  Suppose that the series \(\sum_{n = 1}^\infty \norm*{f_n'}_\infty\) is absolutely convergent, where
  \[
    \norm*{f_n'}_\infty \coloneqq \sup_{x \in [a, b]} \abs{f_n'(x)}
  \]
  is the sup norm of \(f_n'\), as defined in \cref{3.5.5}.
  Suppose also that the series \(\sum_{n = 1}^\infty f_n(x_0)\) is convergent for some \(x_0 \in [a, b]\).
  Then the series \(\sum_{n = 1}^\infty f_n\) converges uniformly on \([a, b]\) to a differentiable function, and in fact
  \[
    \dfrac{d}{dx} \sum_{n = 1}^\infty f_n(x) = \sum_{n = 1}^\infty \dfrac{d}{dx} f_n(x)
  \]
  for all \(x \in [a, b]\).
\end{cor}

\begin{proof}
  Let \(F_N = \sum_{n = 1}^N f_n\) for each \(N \in \Z^+\).
  Then by Theorem 10.1.13(c) in Analysis I we have
  \[
    \forall N \in \Z^+, F_N' = \bigg(\sum_{n = 1}^N f_n\bigg)' = \sum_{n = 1}^N f_n'.
  \]
  Since \(f_n'\) is continuous on \([a, b]\) for each \(n \in \Z^+\), by Proposition 9.6.7 in Analysis I we know that \(f_n' \in B\big([a, b] \to \R\big)\) and thus \(f_n' \in C\big([a, b] \to \R\big)\).
  By \cref{ex:3.5.1} we know that
  \[
    \forall N \in \Z^+, F_N' = \sum_{n = 1}^N f_n' \in C([a, b] \to \R).
  \]
  Since \(\sum_{n = 1}^\infty \norm*{f_n'}_\infty\) converges and \(f_n' \in C\big([a, b] \to \R\big)\) for each \(n \in \Z^+\), by \cref{3.5.7} we know that there exists some \(G : [a, b] \to \R\) such that \(\big(\sum_{n = 1}^N f_n'\big)_{N = 1}^\infty\) converges uniformly to \(G\) on \([a, b]\) with respect to \(d_{l^1}|_{\R \times \R}\).
  Equivalently, \((F_N')_{N = 1}^\infty\) converges uniformly to \(G\) on \([a, b]\) with respect to \(d_{l^1}|_{\R \times \R}\).
  Since
  \[
    \sum_{n = 1}^\infty f_n(x_0) = \lim_{N \to \infty} \sum_{n = 1}^N f_n(x_0) = \lim_{N \to \infty} F_N(x_0),
  \]
  by \cref{3.7.1} we know that there exists some \(F : [a, b] \to \R\) such that \((F_N)_{N = 1}^\infty\) converges uniformly to \(F\) on \([a, b]\) with respect to \(d_{l^1}|_{\R \times \R}\) and \(F' = G\).
  Then we have
  \begin{align*}
             & \forall x \in [a, b], \begin{dcases}
                                       F(x) = \lim_{N \to \infty} F_N(x) = \lim_{N \to \infty} \sum_{n = 1}^N f_n(x) = \sum_{n = 1}^\infty f_n(x)    \\
                                       G(x) = \lim_{N \to \infty} F_N'(x) = \lim_{N \to \infty} \sum_{n = 1}^N f_n'(x) = \sum_{n = 1}^\infty f_n'(x) \\
                                       F'(x) = G(x)
                                     \end{dcases} \\
    \implies & \forall x \in [a, b], \bigg(\sum_{n = 1}^\infty f_n\bigg)'(x) = \sum_{n = 1}^\infty f_n'(x)                                         \\
    \implies & \bigg(\sum_{n = 1}^\infty f_n\bigg)' = \sum_{n = 1}^\infty f_n'.
  \end{align*}
\end{proof}

\begin{note}
  \cref{3.7.4} was discovered by Weierstrass.
\end{note}

\begin{eg}\label{3.7.4}
  Let \(f : \R \to \R\) be the function
  \[
    f(x) \coloneqq \sum_{n = 1}^\infty 4^{-n} \cos(32^n \pi x).
  \]
  Note that this series is uniformly convergent, thanks to the Weierstrass \(M\)-test, and since each individual function \(4^{-n} \cos(32^n \pi x)\) is continuous, the function \(f\) is also continuous.
  However, it is not differentiable;
  in fact it is a \emph{nowhere differentiable function}, one which is not differentiable at any point, despite being continuous everywhere!
\end{eg}

\exercisesection

\begin{ex}\label{ex:3.7.1}
  Complete the proof of \cref{3.7.1}.
  Compare this theorem with Example 1.2.10 in Analysis I, and explain why this example does not contradict the theorem.
\end{ex}

\begin{proof}
  See \cref{3.7.1}.
  Since \(\lim_{n \to \infty} \dfrac{x^3}{\dfrac{1}{n} + x^2}\) is not continuous at \(x = 0\), is does not contradict to \cref{3.7.1}.
\end{proof}

\begin{ex}\label{ex:3.7.2}
  Prove \cref{3.7.1} without assuming that \(f_n'\) is continuous.
  (This means that you cannot use the fundamental theorem of calculus.
  However, the mean value theorem (Corollary 10.2.9 in Analysis I) is still available.
  Use this to show that if \(d_\infty(f_n', f_m') \leq \varepsilon\), then \(\abs{\big(f_n(x) - f_m(x)\big) - \big(f_n(x_0) - f_m(x_0)\big)} \leq \varepsilon \abs{x - x_0}\) for all \(x \in [a, b]\), and then use this to complete the proof of \cref{3.7.1}.)
\end{ex}

\begin{proof}
  Let \(m \in \Z^+\).
  Let \(d_{C([a, b] \to \R)} = d_{B([a, b] \to \R)}|_{C([a, b] \to \R) \times C([a, b] \to \R)}\).
  Since \(f_n\) is differentiable on \([a, b]\), by Corollary 10.1.12 and Proposition 9.6.7 in Analysis I we know that \(f_n \in C([a, b] \to \R)\) for each \(n \in \Z^+\).
  Since \((f_n')_{n = 1}^\infty\) converges uniformly to \(g\) on \([a, b]\) with respect to \(d_{l^1}|_{\R \times \R}\), by \cref{3.2.7} we have
  \begin{align*}
             & \forall \varepsilon \in \R^+, \exists\ N_1 \in \Z^+ : \forall n \geq N_1, \forall x \in [a, b], \abs{f_n'(x) - g(x)} < \dfrac{\varepsilon}{6(b - a)} \\
    \implies & \forall \varepsilon \in \R^+, \exists\ N_1 \in \Z^+ : \forall n, m \geq N_1, \forall x \in [a, b],                                                   \\
             & \abs{f_n'(x) - f_m'(x)} \leq \abs{f_n'(x) - g(x)} + \abs{f_m'(x) - g(x)} < \dfrac{\varepsilon}{6(b - a)} + \dfrac{\varepsilon}{6(b - a)}             \\
    \implies & \forall \varepsilon \in \R^+, \exists\ N_1 \in \Z^+ : \forall n, m \geq N_1, \forall x \in [a, b],                                                   \\
             & \abs{(f_n' - f_m')(x)} < \dfrac{\varepsilon}{3(b - a)}.
  \end{align*}
  Now we fix one such \(\varepsilon\) and \(N_1\).
  Let \(x \in [a, b]\).
  We split into two cases:
  \begin{itemize}
    \item \(x = x_0\).
          Then we have
          \[
            \forall n, m \geq N_1, \abs{\big(f_n(x_0) - f_m(x_0)\big) - \big(f_n(x_0) - f_m(x_0)\big)} = 0 = \dfrac{\varepsilon \abs{x_0 - x_0}}{3(b - a)}.
          \]
    \item \(x \neq x_0\).
          Suppose that \(x < x_0\).
          Since \([x, x_0] \subseteq [a, b]\), by mean value theorem we know that
          \[
            \forall n, m \geq N_1, \exists\ y \in (x, x_0) : (f_n - f_m)'(y) = \dfrac{(f_n - f_m)(x_0) - (f_n - f_m)(x)}{x_0 - x}.
          \]
          Now suppose that \(x > x_0\).
          Since \([x_0, x] \subseteq [a, b]\), by mean value theorem we know that
          \[
            \forall n, m \geq N_1, \exists\ y \in (x_0, x) : (f_n - f_m)'(y) = \dfrac{(f_n - f_m)(x) - (f_n - f_m)(x_0)}{x - x_0}.
          \]
          In either cases we have
          \begin{align*}
                     & \forall n, m \geq N_1, \exists\ y \in (a, b) :                                                                                           \\
                     & \abs{\dfrac{(f_n - f_m)(x) - (f_n - f_m)(x_0)}{x - x_0}} = \abs{(f_n - f_m)'(y)}                                                         \\
            \implies & \forall n, m \geq N_1, \exists\ y \in (a, b) :                                                                                           \\
                     & \abs{\big(f_n(x) - f_m(x)\big) - \big(f_n(x_0) - f_m(x_0)\big)}                                                                          \\
                     & = \abs{f_n'(y) - f_m'(y)} \abs{x - x_0}                                                                                                  \\
                     & \leq \dfrac{\varepsilon \abs{x - x_0}}{3(b - a)}                                                                                         \\
            \implies & \forall n, m \geq N_1, \abs{\big(f_n(x) - f_m(x)\big) - \big(f_n(x_0) - f_m(x_0)\big)} \leq \dfrac{\varepsilon \abs{x - x_0}}{3(b - a)}.
          \end{align*}
  \end{itemize}
  From all cases above we conclude that
  \[
    \forall n, m \geq N_1, \forall x \in [a, b], \abs{\big(f_n(x) - f_m(x)\big) - \big(f_n(x_0) - f_m(x_0)\big)} \leq \dfrac{\varepsilon \abs{x - x_0}}{3(b - a)}.
  \]

  Let \(L = \lim_{n \to \infty} f_n(x_0)\).
  Then we have
  \begin{align*}
             & \lim_{n \to \infty} f_n(x_0) = L                                                        \\
    \implies & \exists\ N_2 \in \Z^+ : \forall n \geq N_2, \abs{f_n(x_0) - L} < \dfrac{\varepsilon}{3}
  \end{align*}
  Let \(N = \max(N_1, N_2)\).
  Then we have
  \begin{align*}
     & \forall n, m \geq N, \forall x \in [a, b],                                                                     \\
     & \abs{f_n(x) - f_m(x)}                                                                                          \\
     & = \abs{\big(f_n(x) - f_m(x)\big) - \big(f_n(x_0) - f_m(x_0)\big) + \big(f_n(x_0) - f_m(x_0)\big) + L - L}      \\
     & \leq \abs{\big(f_n(x) - f_m(x)\big) - \big(f_n(x_0) - f_m(x_0)\big)} + \abs{f_n(x_0) - L} + \abs{f_m(x_0) - L} \\
     & < \dfrac{\varepsilon \abs{x - x_0}}{3(b - a)} + \dfrac{\varepsilon}{3} + \dfrac{\varepsilon}{3}                \\
     & \leq \dfrac{\varepsilon}{3} + \dfrac{\varepsilon}{3} + \dfrac{\varepsilon}{3} = \varepsilon.
  \end{align*}
  Since \(\varepsilon\) is arbitrary, by \cref{3.4.2} we have
  \begin{align*}
             & \forall \varepsilon \in \R^+, \exists\ N \in \Z^+ : \forall n, m \geq N, \forall x \in [a, b], \abs{f_n(x) - f_m(x)} < \varepsilon \\
    \implies & \forall \varepsilon \in \R^+, \exists\ N \in \Z^+ : \forall n, m \geq N, \sup_{x \in [a, b]}\abs{f_n(x) - f_m(x)} < \varepsilon    \\
    \implies & \forall \varepsilon \in \R^+, \exists\ N \in \Z^+ : \forall n, m \geq N, d_{C([a, b] \to \R)}(f_n, f_m) < \varepsilon.
  \end{align*}
  Thus by \cref{1.4.6} \((f_n)_{n = 1}^\infty\) is a Cauchy sequence in \(\big(C([a, b] \to \R), d_{C([a, b] \to \R)}\big)\).
  Since \((\R, d_{l^1}|_{\R \times \R})\) is complete, by \cref{3.4.5} we know that there exists some \(f : [a, b] \to \R\) such that
  \[
    \begin{dcases}
      f \in C([a, b] \to \R) \\
      d_{C([a, b] \to \R)} - \lim_{n \to \infty} f_n = f
    \end{dcases}
  \]
  By \cref{3.4.4} we know that \((f_n)_{n = 1}^\infty\) convergent uniformly to \(f\) on \([a, b]\) with respect to \(d_{l^1}|_{\R \times \R}\).

  Now we show that \(f\) is differentiable on \([a, b]\) and \(f' = g\).
  Let \(y \in [a, b]\).
  Since \((f_n')_{n = 1}^\infty\) converges uniformly to \(g\) on \([a, b]\) with respect to \(d_{l^1}|_{\R \times \R}\), we have
  \begin{align*}
    g(y) & = \lim_{n \to \infty} f_n'(y)                                                                                   &  & \text{(by \cref{ex:3.2.2}(a))} \\
         & = \lim_{n \to \infty} \bigg(\lim_{x \to y ; x \in [a, b] \setminus \{y\}} \dfrac{f_n(x) - f_n(y)}{x - y}\bigg)                                      \\
         & = \lim_{x \to y ; x \in [a, b] \setminus \{y\}} \bigg(\lim_{n \to \infty} \dfrac{f_n(x) - f_n(y)}{x - y}\bigg). &  & \text{(by \cref{3.3.3})}
  \end{align*}
  Since \((f_n)_{n = 1}^\infty\) converges uniformly to \(f\) on \([a, b]\) with respect to \(d_{l^1}|_{\R \times \R}\), we have
  \begin{align*}
             & \forall x \in [a, b], f(x) = \lim_{n \to \infty} f_n(x)                                                                         &  & \text{(by \cref{ex:3.2.2}(a))} \\
    \implies & \forall x \in [a, b], f(x) - f(y) = \lim_{n \to \infty} f_n(x) - \lim_{n \to \infty} f_n(y)                                                                         \\
    \implies & \forall x \in [a, b], f(x) - f(y) = \lim_{n \to \infty} \big(f_n(x) - f_n(y)\big)                                                                                   \\
    \implies & \forall x \in [a, b] \setminus \{y\}, \dfrac{f(x) - f(y)}{x - y} = \dfrac{\lim_{n \to \infty} \big(f_n(x) - f_n(y)\big)}{x - y}                                     \\
    \implies & \forall x \in [a, b] \setminus \{y\}, \dfrac{f(x) - f(y)}{x - y} = \lim_{n \to \infty} \dfrac{f_n(x) - f_n(y)}{x - y}.
  \end{align*}
  Thus we have
  \[
    g(y) = \lim_{x \to y ; x \in [a, b] \setminus \{y\}} \bigg(\lim_{n \to \infty} \dfrac{f_n(x) - f_n(y)}{x - y}\bigg) = \lim_{x \to y ; x \in [a, b] \setminus \{y\}} \dfrac{f(x) - f(y)}{x - y} = f'(y).
  \]
  Since \(y\) is arbitrary, we conclude that \(f\) is differentiable on \([a, b]\) and \(f' = g\).
\end{proof}

\begin{ex}\label{ex:3.7.3}
  Prove \cref{3.7.3}.
\end{ex}

\begin{proof}
  See \cref{3.7.3}.
\end{proof}
\section{Uniform approximation by polynomials}\label{ii:sec:3.8}

\begin{note}
  As we have just seen, continuous functions can be very badly behaved, for instance they can be nowhere differentiable (\cref{ii:3.7.4}).
  On the other hand, functions such as polynomials are always very well behaved, in particular being always differentiable.
  Fortunately, while most continuous functions are not as well behaved as polynomials, they can always be \emph{uniformly approximated} by polynomials; this important (but difficult) result is known as the \emph{Weierstrass approximation theorem},
\end{note}

\begin{defn}\label{ii:3.8.1}
  Let \([a, b]\) be an interval.
  A \emph{polynomial on \([a, b]\)} is a
  function \(f : [a, b] \to \R\) of the form \(f(x) \coloneqq \sum_{j = 0}^n c_j x^j\), where \(n \geq 0\) is an integer and \(c_0, \dots, c_n\) are real numbers.
  If \(c_n \neq 0\), then \(n\) is called the \emph{degree} of \(f\).
\end{defn}

\setcounter{thm}{2}
\begin{thm}[Weierstrass approximation theorem]\label{ii:3.8.3}
  If \([a, b]\) is an interval, \(f : [a, b] \to \R\) is a continuous function, and \(\varepsilon > 0\), then there exists a polynomial \(P\) on \([a, b]\) such that \(d_\infty(P, f) \leq \varepsilon\)
  (i.e., \(\abs{P(x) - f(x)} \leq \varepsilon\) for all \(x \in [a, b]\)).
\end{thm}

\begin{proof}
  Let \(f : [a, b] \to \R\) be a continuous function on \([a, b]\).
  Let \(g : [0, 1] \to \R\) denote the function
  \[
    g(x) \coloneqq f\big(a + (b - a) x\big) \text{ for all } x \in [0, 1]
  \]
  Observe then that
  \[
    f(y) = g(\dfrac{y - a}{b - a}) \text{ for all } y \in [a, b].
  \]
  The function \(g\) is continuous on \([0, 1]\) since \(y \mapsto \dfrac{y - a}{b - a}\) is bijective on \([a, b]\), and so by \cref{ii:3.8.19} we may find a polynomial \(Q : [0, 1] \to \R\) such that \(\abs{Q(x) - g(x)} \leq \varepsilon\) for all \(x \in [0, 1]\).
  In particular, for any \(y \in [a, b]\), we have
  \[
    \abs{Q(\dfrac{y - a}{b - a}) - g(\dfrac{y - a}{b - a})} \leq \varepsilon.
  \]
  If we thus set \(P(y) \coloneqq Q(\dfrac{y - a}{b - a})\), then we observe that \(P\) is also a polynomial since \(y \mapsto \dfrac{y - a}{b - a}\) is bijective on \([a, b]\), and so we have \(\abs{P(y) - f(y)} \leq \varepsilon\) for all \(y \in [a, b]\), as desired.
\end{proof}

\begin{note}
  Another way of stating \cref{ii:3.8.3} is as follows.
  Recall that \(C([a, b] \to \R)\) was the space of continuous functions from \([a, b]\) to \(\R\), with the uniform metric \(d_\infty\).
  Let \(P([a, b] \to \R)\) be the space of all polynomials on \([a, b]\);
  this is a subspace of \(C([a, b] \to \R)\), since all polynomials are continuous (Exercise 9.4.7 in Analysis I).
  The Weierstrass approximation theorem then asserts that every continuous function is an adherent point of \(P([a, b] \to \R)\);
  or in other words, that the closure of the space of polynomials is the space of continuous functions (see \cref{ii:3.3.2}):
  \[
    \overline{P([a, b] \to \R)}_{\big(C([a, b] \to \R), d_\infty\big)} = C([a, b] \to \R).
  \]
  In particular, every continuous function on \([a, b]\) is the uniform limit of polynomials (see \cref{ii:3.4.4}).
  Another way of saying this is that the space of polynomials is \emph{dense} in the space of continuous functions, in the \emph{uniform topology}.
\end{note}

\begin{defn}[Compactly supported functions]\label{ii:3.8.4}
  Let \([a, b]\) be an interval.
  A function \(f : \R \to \R\) is said to be \emph{supported} on \([a, b]\) iff \(f(x) = 0\) for all \(x \notin [a, b]\).
  We say that \(f\) is \emph{compactly supported} iff it is supported on some interval \([a, b]\).
  If \(f\) is continuous and supported on \([a, b]\), we define the improper integral \(\int_{-\infty}^\infty f\) to be \(\int_{-\infty}^\infty f \coloneqq \int_{[a, b]} f\).
\end{defn}

\begin{note}
  A function can be supported on more than one interval, for instance a function which is supported on \([3, 4]\) is also automatically supported on \([2, 5]\).
\end{note}

\begin{lem}\label{ii:3.8.5}
  If \(f : \R \to \R\) is continuous and supported on an interval \([a, b]\), and is also supported on another interval \([c, d]\), then \(\int_{[a, b]} f = \int_{[c, d]} f\).
\end{lem}

\begin{proof}
  Since
  \begin{align*}
             & \begin{dcases}
                 f \text{ is supported on } [a, b] \\
                 f \text{ is supported on } [c, d]
               \end{dcases}                        \\
    \implies & \begin{dcases}
                 \forall x \notin [a, b], f(x) = 0 \\
                 \forall x \notin [c, d], f(x) = 0
               \end{dcases}                        &  & \by{ii:3.8.4}   \\
    \implies & \begin{dcases}
                 \forall x \in \R, (x < a) \lor (x > b) \implies f(x) = 0 \\
                 \forall x \in \R, (x < c) \lor (x > d) \implies f(x) = 0
               \end{dcases}
  \end{align*}
  we have
  \begin{align*}
    \int_{-\infty}^\infty f & = \int_{[a, b]} f                                                                                                            &  & \by{ii:3.8.4} \\
                            & = \begin{dcases}
                                  \int_{[a, c]} f + \int_{[c, b]} f & \text{if } a \leq c \\
                                  0 + \int_{[a, b]} f               & \text{if } a > c
                                \end{dcases}                     \\
                            & = \begin{dcases}
                                  0 + \int_{[c, b]} f               & \text{if } a \leq c \\
                                  \int_{[c, a]} f + \int_{[a, b]} f & \text{if } a > c
                                \end{dcases}                     \\
                            & = \int_{[c, b]} f                                                                                                                               \\
                            & = \begin{dcases}
                                  \int_{[c, b]} f + 0               & \text{if } b \leq d \\
                                  \int_{[c, d]} f + \int_{[d, b]} f & \text{if } b > d
                                \end{dcases}                     \\
                            & = \begin{dcases}
                                  \int_{[c, b]} f + \int_{[b, d]} f & \text{if } b \leq d \\
                                  \int_{[c, d]} f + 0               & \text{if } b > d
                                \end{dcases}                     \\
                            & = \int_{[c, d]} f.
  \end{align*}
\end{proof}

\begin{defn}[Approximation to the identity]\label{ii:3.8.6}
  Let \(\varepsilon > 0\) and \(0 < \delta < 1\).
  A function \(f : \R \to \R\) is said to be an \emph{\((\varepsilon, \delta)\)-approximation to the identity} if it obeys the following three properties:
  \begin{enumerate}
    \item \(f\) is supported on \([-1, 1]\), and \(f(x) \geq 0\) for all \(-1 \leq x \leq 1\).
    \item \(f\) is continuous, and \(\int_{-\infty}^\infty f = 1\).
    \item \(\abs{f(x)} \leq \varepsilon\) for all \(\delta \leq \abs{x} \leq 1\).
  \end{enumerate}
\end{defn}

\begin{rmk}\label{ii:3.8.7}
  For those of you who are familiar with the Dirac delta function, approximations to the identity are ways to approximate this (very discontinuous) delta function by a continuous function (which is easier to analyze).
\end{rmk}

\begin{lem}[Polynomials can approximate the identity]\label{ii:3.8.8}
  For every \(\varepsilon > 0\) and \(0 < \delta < 1\) there exists an \((\varepsilon, \delta)\)-approximation to the identity which is a polynomial \(P\) on \([-1, 1]\).
\end{lem}

\begin{proof}
  Let \(\varepsilon \in \R^+\) and let \(\delta \in (0, 1)\).
  We have
  \begin{align*}
             & \forall x \in [-1, 1], \delta \leq \abs{x} \leq 1                                                                             \\
    \implies & \delta^2 \leq x^2 \leq 1                                                                                                      \\
    \implies & 0 \leq 1 - x^2 \leq 1 - \delta^2 < 1                                                                                          \\
    \implies & \lim_{n \to \infty} \sqrt{n} (1 - \delta^2)^n = 0                               &  & \text{(by Exercise 7.5.2 in Analysis I)} \\
    \implies & \exists N \in \Z^+ : \forall n \geq N, \sqrt{n} (1 - \delta^2)^n < \varepsilon.
  \end{align*}
  Now we fix such \(N\).
  Define \(g : \R \to \R\) to be the function
  \[
    \forall x \in \R, g(x) = \begin{dcases}
      (1 - x^2)^N & \text{if } x \in [-1, 1]    \\
      0           & \text{if } x \notin [-1, 1]
    \end{dcases}
  \]
  We know that \(g(x) \geq 0\) for all \(x \in \R\).
  By \cref{ii:3.8.4} we know that \(g\) is supported on \([-1, 1]\).
  By Exercise 9.4.7 in Analysis I we know that \(g\) is continuous on \([-1, 1]\), thus by Corollary 11.5.2 in Analysis I \(g\) is Riemann integrable on \([-1, 1]\).
  By \cref{ii:ex:3.8.2}(b) we know that
  \[
    \int_{[-1, 1]} g = \int_{[-1, 1]} (1 - x^2)^N \geq \dfrac{1}{\sqrt{N}} > 0,
  \]
  so we can define \(c = (\int_{[-1, 1]} g)^{-1}\) and we have
  \[
    0 < c = \bigg(\int_{[-1, 1]} g\bigg)^{-1} \leq \sqrt{N}.
  \]
  Now we define \(f : \R \to \R\) to be the function \(f = cg\).
  Again we have \(f\) is continuous and supported on \([-1, 1]\).
  Since \(c > 0\), we know that \(f(x) \geq 0\) for all \(x \in \R\).
  By \cref{ii:3.8.4} we have
  \[
    \int_{-\infty}^\infty f = \int_{[-1, 1]} f = \int_{[-1, 1]} cg = c \int_{[-1, 1]} g = \bigg(\int_{[-1, 1]} g\bigg)^{-1} \bigg(\int_{[-1, 1]} g\bigg) = 1.
  \]
  Since
  \begin{align*}
             & \forall x \in [-1, 1], \delta \leq \abs{x} \leq 1                                                     \\
    \implies & 0 \leq 1 - x^2 \leq 1 - \delta^2 < 1                                                                  \\
    \implies & 0 \leq \sqrt{N} (1 - x^2)^N \leq \sqrt{N} (1 - \delta^2)^N < \varepsilon                              \\
    \implies & 0 \leq \abs{f(x)} = \abs{cg(x)} \leq \abs{\sqrt{N} (1 - x^2)^N} = \sqrt{N} (1 - x^2)^N < \varepsilon,
  \end{align*}
  combine all the proofs above we conclude by \cref{ii:3.8.6} that \(f\) is an \((\varepsilon, \delta)\)-approximation to the identity.
\end{proof}

\begin{defn}[Convolution]\label{ii:3.8.9}
  Let \(f : \R \to \R\) and \(g : \R \to \R\) be continuous, compactly supported functions.
  We define the \emph{convolution} \(f * g : \R \to \R\) of \(f\) and \(g\) to be the function
  \[
    (f * g)(x) \coloneqq \int_{-\infty}^\infty f(y) g(x - y) \; dy.
  \]
\end{defn}

\begin{note}
  If \(f\) and \(g\) are continuous and compactly supported, then for each \(x\) the function \(f(y) g(x - y)\) (thought of as a function of \(y\)) is also continuous and compactly supported, so \cref{ii:3.8.9} makes sense.
\end{note}

\begin{rmk}\label{ii:3.8.10}
  Convolutions play an important role in Fourier analysis and in partial differential equations (PDE), and are also important in physics, engineering, and signal processing.
\end{rmk}

\begin{prop}[Basic properties of convolution]\label{ii:3.8.11}
  Let \(f : \R \to \R\), \(g : \R \to \R\), and \(h : \R \to \R\) be continuous, compactly supported functions.
  Then the following statements are true.
  \begin{enumerate}
    \item The convolution \(f * g\) is also a continuous, compactly supported function.
    \item (Convolution is commutative)
          We have \(f * g = g * f\);
          in other words
          \begin{align*}
            f * g(x) & = \int_{-\infty}^\infty f(y) g(x - y) \; dy \\
                     & = \int_{-\infty}^\infty g(y) f(x - y) \; dy \\
                     & = g * f(x).
          \end{align*}
    \item (Convolution is linear)
          We have \(f * (g + h) = f * g + f * h\).
          Also, for any real number \(c\), we have \(f * (cg) = (cf) * g = c(f * g)\).
  \end{enumerate}
\end{prop}

\begin{proof}{(a)}
  Since \(f, g\) are compactly supported, by \cref{ii:3.8.4} we know that
  \[
    \exists L_f, L_g, U_f, U_g \in \R : \begin{dcases}
      \forall y \in \R \setminus [L_f, U_f], f(y) = 0 \\
      \forall y \in \R \setminus [L_g, U_g], g(y) = 0
    \end{dcases}
  \]
  Note that we can choose \(L_f \neq U_f\).
  Let \(L = \min(L_f, L_g)\), let \(U = \max(U_f, U_g)\) and let \(M = \max(\abs{L}, \abs{U})\).
  Then we have
  \begin{align*}
             & \forall y \in \R \setminus [-M, M], \begin{dcases}
                                                     y < -M \leq L \leq L_f \implies f(y) = 0 \\
                                                     y < -M \leq L \leq L_g \implies g(y) = 0 \\
                                                     y > M \geq U \geq U_f \implies f(y) = 0  \\
                                                     y > M \geq U \geq U_g \implies g(y) = 0
                                                   \end{dcases} \\
    \implies & f(y) = g(y) = 0
  \end{align*}
  and
  \[
    \forall y \in \R \setminus [-2M, 2M], (y < -2M \leq -M) \lor (y > 2M \geq M) \implies f(y) = g(y) = 0.
  \]
  Thus by \cref{ii:3.8.4} \(f, g\) are supported on \([-M, M]\) and \([-2M, 2M]\).
  Observe that
  \begin{align*}
             & \forall x \in (-\infty, -2M), \forall y \in \R, \begin{dcases}
                                                                 x - y < -M \text{ or } \\
                                                                 x - y \geq -M
                                                               \end{dcases}                      \\
    \implies & \forall x \in (-\infty, -2M), \forall y \in \R, \begin{dcases}
                                                                 x - y < -M \text{ or } \\
                                                                 -M > x + M \geq y
                                                               \end{dcases}                      \\
    \implies & \forall x \in (-\infty, -2M), \forall y \in \R, \begin{dcases}
                                                                 g(x - y) = 0 & \text{if } x - y < -M        \\
                                                                 f(y) = 0     & \text{if } -M > x + M \geq y
                                                               \end{dcases} \\
    \implies & \forall x \in (-\infty, -2M), \forall y \in \R, f(y) g(x - y) = 0
  \end{align*}
  and
  \begin{align*}
             & \forall x \in (2M, +\infty), \forall y \in \R, \begin{dcases}
                                                                x - y > M \text{ or } \\
                                                                x - y \leq M
                                                              \end{dcases}                      \\
    \implies & \forall x \in (2M, +\infty), \forall y \in \R, \begin{dcases}
                                                                x - y > M \text{ or } \\
                                                                M < x - M \leq y
                                                              \end{dcases}                      \\
    \implies & \forall x \in (2M, +\infty), \forall y \in \R, \begin{dcases}
                                                                g(x - y) = 0 & \text{if } x - y > M        \\
                                                                f(y) = 0     & \text{if } M < x - M \leq y
                                                              \end{dcases} \\
    \implies & \forall x \in (2M, +\infty), \forall y \in \R, f(y) g(x - y) = 0.
  \end{align*}
  This means
  \[
    \forall x \in \R \setminus [-2M, 2M], \forall y \in \R, f(y) g(x - y) = 0.
  \]
  For each \(x \in \R \setminus [-2M, 2M]\), we define \(z_x : \R \to \R\) by setting \(z_x(y) = f(y) g(x - y)\).
  Since \(z_x\) is continuous on \(\R\), by \cref{ii:3.8.4} and \cref{ii:3.8.9} we have
  \begin{align*}
             & \forall x \in \R \setminus [-2M, 2M], \forall y \in \R, z_x(y) = 0                                         \\
    \implies & \forall x \in \R \setminus [-2M, 2M], \forall y \in \R \setminus [-1, 1], z_x(y) = 0                       \\
    \implies & \forall x \in \R \setminus [-2M, 2M], z_x \text{ is supported on } [-1, 1]                                 \\
    \implies & \forall x \in \R \setminus [-2M, 2M], \int_{-\infty}^\infty z_x = \int_{[-1, 1]} z_x = 0                   \\
    \implies & \forall x \in \R \setminus [-2M, 2M], \int_{-\infty}^\infty z_x(y) \; dy = \int_{[-1, 1]} z_x(y) \; dy = 0 \\
    \implies & \forall x \in \R \setminus [-2M, 2M], (f * g)(x) = \int_{[-1, 1]} f(y) g(x - y) \; dy = 0                  \\
    \implies & f * g \text{ is supported on } [-2M, 2M]                                                                   \\
    \implies & f * g \text{ is compactly supported}.
  \end{align*}
  Since \(f, g\) are compactly supported and continuous on \(\R\), by \cref{ii:ex:3.8.3} we know that
  \[
    \exists N \in \R^+ : \forall x \in \R, \abs{f(x)} \leq N
  \]
  and
  \[
    \forall \varepsilon \in \R^+, \exists \delta \in \R^+ : \forall x_1, x_2 \in \R, \abs{x_1 - x_2} < \delta \implies \abs{g(x_1) - g(x_2)} < \dfrac{\varepsilon}{N (U_f - L_f)}.
  \]
  Fix \(N\) and one pair of \(\varepsilon\) and \(\delta\).
  Let \(x_0 \in \R\).
  Then we have
  \begin{align*}
             & \forall x \in \R, \abs{x - x_0} < \delta                                                                                        \\
    \implies & \abs{(f * g)(x) - (f * g)(x_0)} = \abs{\int_{-\infty}^\infty f(y) g(x - y) \; dy - \int_{-\infty}^\infty f(y) g(x_0 - y) \; dy} \\
             & = \abs{\int_{[L_f, U_f]} f(y) g(x - y) \; dy - \int_{[L_f, U_f]} f(y) g(x_0 - y) \; dy}                                         \\
             & = \abs{\int_{[L_f, U_f]} f(y) \big(g(x - y) - g(x_0 - y)\big) \; dy}                                                            \\
             & \leq \abs{\int_{[L_f, U_f]} N \dfrac{\varepsilon}{N (U_f - L_f)} \; dy} = \varepsilon.
  \end{align*}
  Since \(\varepsilon\) was arbitrary, we know that \(f * g\) is continuous at \(x_0\).
  Since \(x_0\) was arbitrary, we know that \(f * g\) is continuous on \(\R\).
\end{proof}

\begin{proof}{(b)}
  Let \(x_0 \in \R\).
  Since \(f\) is compactly supported, we know that
  \[
    \exists L, U \in \R : \forall y \in \R \setminus [L, U], f(y) = 0.
  \]
  Then we have
  \begin{align*}
             & \forall y \in \R \setminus [L, U], f(y) = 0             \\
    \implies & \forall y \in \R \setminus [L, U], f(y) g(x_0 - y) = 0.
  \end{align*}
  Observe that
  \begin{align*}
             & \forall y \in \R \setminus [L, U], f(y) = 0                         \\
    \implies & \forall y \in \R \setminus [-U, -L], f(-y) = 0                      \\
    \implies & \forall y \in \R \setminus [x_0 - U, x_0 - L], f(x_0 - y) = 0       \\
    \implies & \forall y \in \R \setminus [x_0 - U, x_0 - L], g(y) f(x_0 - y) = 0.
  \end{align*}
  Since \(f, g\) are continuous on \(\R\), we know that
  \begin{align*}
             & \forall y_0 \in \R, \begin{dcases}
                                     f \text{ is continuous at } y_0       \\
                                     g \text{ is continuous at } y_0       \\
                                     f \text{ is continuous at } x_0 - y_0 \\
                                     g \text{ is continuous at } x_0 - y_0 \\
                                     y \mapsto x_0 - y \text{ is continuous at } y_0
                                   \end{dcases}                    \\
    \implies & \forall y_0 \in \R, \begin{dcases}
                                     \lim_{y \to y_0 ; y \in \R} f(y) = f(y_0)             \\
                                     \lim_{y \to y_0 ; y \in \R} g(y) = g(y_0)             \\
                                     \lim_{y \to y_0 ; y \in \R} f(x_0 - y) = f(x_0 - y_0) \\
                                     \lim_{y \to y_0 ; y \in \R} g(x_0 - y) = g(x_0 - y_0)
                                   \end{dcases}             \\
    \implies & \forall y_0 \in \R, \begin{dcases}
                                     \lim_{y \to y_0 ; y \in \R} f(y) g(x_0 - y) = f(y_0) g(x_0 - y_0) \\
                                     \lim_{y \to y_0 ; y \in \R} g(y) f(x_0 - y) = g(y_0) f(x_0 - y_0)
                                   \end{dcases}
  \end{align*}
  This means
  \begin{align*}
    (f * g)(x_0) & = \int_{-\infty}^\infty f(y) g(x_0 - y) \; dy      &  & \by{ii:3.8.9} \\
                 & = \int_{[L, U]} f(y) g(x_0 - y) \; dy;             &  & \by{ii:3.8.4} \\
    (g * f)(x_0) & = \int_{-\infty}^\infty g(y) f(x_0 - y) \; dy      &  & \by{ii:3.8.9} \\
                 & = \int_{[x_0 - U, x_0 - L]} g(y) f(x_0 - y) \; dy; &  & \by{ii:3.8.4}
  \end{align*}
  Let \(\phi : \R \to \R\) be the function \(\phi = y \mapsto x_0 - y\).
  By the formula of changing variable (Exercise 11.10.4 in Analysis I) we have
  \begin{align*}
     & \int_{[L, U]} f(y) g(x_0 - y) \; dy                                                     \\
     & = \int_{[\phi(x_0 - U), \phi(x_0 - L)]} f(y) g(x_0 - y) \; dy                           \\
     & = -\int_{[x_0 - U, x_0 - L]} f\big(\phi(y)\big) g\big(x_0 - \phi(y)\big) \phi'(y) \; dy \\
     & = \int_{[x_0 - U, x_0 - L]} f(x_0 - y) g(y) \; dy                                       \\
     & = \int_{[x_0 - U, x_0 - L]} g(y) f(x_0 - y) \; dy.
  \end{align*}
  Thus \((f * g)(x_0) = (g * f)(x_0)\).
  Since \(x_0\) was arbitrary, we conclude that
  \[
    \forall x \in \R, (f * g)(x) = (g * f)(x).
  \]
\end{proof}

\begin{proof}{(c)}
  Let \(x_0 \in \R\).
  Since \(g, h\) are compactly supported, by \cref{ii:3.8.4} we know that
  \[
    \exists L_g, L_h, U_g, U_h \in \R : \begin{dcases}
      \forall y \in \R \setminus [L_g, U_g], g(y) = 0 \\
      \forall y \in \R \setminus [L_h, U_h], h(y) = 0
    \end{dcases}
  \]
  Let \(L = \min(L_g, L_h)\) and let \(U = \min(U_g, U_h)\).
  Then we have
  \begin{align*}
             & \forall y \in \R \setminus [L, U], \begin{dcases}
                                                    y < L \leq L_g & \implies g(y) = 0 \\
                                                    y > U \geq U_g & \implies g(y) = 0 \\
                                                    y < L \leq L_h & \implies h(y) = 0 \\
                                                    y > U \geq U_h & \implies h(y) = 0
                                                  \end{dcases}                      \\
    \implies & \forall y \in \R \setminus [L, U], g(y) = h(y) = 0                                         \\
    \implies & \forall y \in \R \setminus [-U, -L], g(-y) = h(-y) = 0                                     \\
    \implies & \forall y \in \R \setminus [x_0 - U, x_0 - L], g(x_0 - y) = h(x_0 - y) = 0                 \\
    \implies & \forall y \in \R \setminus [x_0 - U, x_0 - L], f(y) g(x_0 - y) = f(y) h(x_0 - y) = 0       \\
    \implies & \forall y \in \R \setminus [x_0 - U, x_0 - L], f(y) \big(g(x_0 - y) + h(x_0 - y)\big) = 0.
  \end{align*}
  Since \(f, g, h\) are continuous on \(\R\), we know that
  \begin{align*}
             & \forall y_0 \in \R, \begin{dcases}
                                     f \text{ is continuous at } y_0 \\
                                     g \text{ is continuous at } y_0 \\
                                     h \text{ is continuous at } y_0 \\
                                     y \mapsto x_0 - y \text{ is continuous at } y_0
                                   \end{dcases}                                                                     \\
    \implies & \forall y_0 \in \R, \begin{dcases}
                                     \lim_{y \to y_0 ; y \in \R} f(y) = f(y_0)             \\
                                     \lim_{y \to y_0 ; y \in \R} g(x_0 - y) = g(x_0 - y_0) \\
                                     \lim_{y \to y_0 ; y \in \R} h(x_0 - y) = h(x_0 - y_0)
                                   \end{dcases}                                                              \\
    \implies & \forall y_0 \in \R, \begin{dcases}
                                     \lim_{y \to y_0 ; y \in \R} f(y) g(x_0 - y) = f(y_0) g(x_0 - y_0) \\
                                     \lim_{y \to y_0 ; y \in \R} f(y) h(x_0 - y) = f(y_0) h(x_0 - y_0)
                                   \end{dcases}                                                  \\
    \implies & \forall y_0 \in \R, \lim_{y \to y_0 ; y \in \R} f(y) \big(g(x_0 - y) + h(x_0 - y)\big) = f(y_0) \big(g(x_0 - y_0) + h(x_0 - y_0)\big).
  \end{align*}
  Thus we have
  \begin{align*}
     & \big(f * (g + h)\big)(x_0)                                                                                                  \\
     & = \int_{-\infty}^\infty f(y) (g + h)(x_0 - y) \; dy                                                      &  & \by{ii:3.8.9} \\
     & = \int_{-\infty}^\infty f(y) \big(g(x_0 - y) + h(x_0 - y)\big) \; dy                                                        \\
     & = \int_{[x_0 - U, x_0 - L]} f(y) \big(g(x_0 - y) + h(x_0 - y)\big) \; dy                                 &  & \by{ii:3.8.4} \\
     & = \int_{[x_0 - U, x_0 - L]} f(y) g(x_0 - y) \; dy + \int_{[x_0 - U, x_0 - L]} f(y) h(x_0 - y)\big) \; dy                    \\
     & = \int_{-\infty}^\infty f(y) g(x_0 - y) \; dy + \int_{-\infty}^\infty f(y) h(x_0 - y) \; dy              &  & \by{ii:3.8.4} \\
     & = (f * g)(x_0) + (f * h)(x_0).                                                                           &  & \by{ii:3.8.9}
  \end{align*}
  Observe that
  \[
    \forall y \in \R \setminus [x_0 - U, x_0 - L], f(y) g(x_0 - y) = c f(y) g(x_0 - y) = 0.
  \]
  Since \(f\) is continuous on \(\R\), we know that \(cf\) is also continuous on \(\R\) and
  \begin{align*}
             & \forall y_0 \in \R, \begin{dcases}
                                     cf \text{ is continuous at } y_0 \\
                                     g \text{ is continuous at } y_0  \\
                                     y \mapsto x_0 - y \text{ is continuous at } y_0
                                   \end{dcases}                     \\
    \implies & \forall y_0 \in \R, \begin{dcases}
                                     \lim_{y \to y_0 ; y \in \R} cf(y) = cf(y_0) \\
                                     \lim_{y \to y_0 ; y \in \R} g(x_0 - y) = g(x_0 - y_0)
                                   \end{dcases}               \\
    \implies & \forall y_0 \in \R, \begin{dcases}
                                     \lim_{y \to y_0 ; y \in \R} cf(y) g(x_0 - y) = cf(y_0) g(x_0 - y_0)
                                   \end{dcases}
  \end{align*}
  Thus we have
  \begin{align*}
     & \big((cf) * g\big)(x_0)                                                \\
     & = \int_{-\infty}^\infty (cf)(y) g(x_0 - y) \; dy    &  & \by{ii:3.8.9} \\
     & = \int_{-\infty}^\infty c f(y) g(x_0 - y) \; dy                        \\
     & = \int_{[x_0 - U, x_0 - L]} c f(y) g(x_0 - y) \; dy &  & \by{ii:3.8.4} \\
     & = c \int_{[x_0 - U, x_0 - L]} f(y) g(x_0 - y) \; dy                    \\
     & = c \int_{-\infty}^\infty f(y) g(x_0 - y) \; dy     &  & \by{ii:3.8.4} \\
     & = c (f * g)(x_0).                                   &  & \by{ii:3.8.9}
  \end{align*}
  Using similar arguments we can show that \(\big((cg) * f\big)(x_0) = c (g * f)(x_0)\).
  By \cref{ii:3.8.11}(b) we thus have
  \[
    \big(f * (cg)\big)(x_0) = \big((cg) * f\big)(x_0) = c(g * f)(x_0) = c(f * g)(x_0) = \big((cf) * g\big)(x_0).
  \]
  Since \(x_0\) was arbitrary, we conclude that
  \[
    \forall x \in \R, \begin{dcases}
      \big(f * (g + h)\big)(x) = (f * g)(x) + (f * h)(x) \\
      \big(f * (cg)\big)(x) = \big((cf) * g\big)(x) = c(f * g)(x)
    \end{dcases}
  \]
\end{proof}

\begin{rmk}\label{ii:3.8.12}
  There are many other important properties of convolution, for instance it is associative, \((f * g) * h = f * (g * h)\), and it commutes with derivatives, \((f * g)' = f' * g = f * g'\), when \(f\) and \(g\) are differentiable.
  The Dirac delta function \(\delta\) mentioned earlier is an identity for convolution:
  \(f * \delta = \delta * f = f\).
  These results are slightly harder to prove than the ones in \cref{ii:3.8.11}, however, and we will not need them in this text.
\end{rmk}

\begin{lem}\label{ii:3.8.13}
  Let \(f : \R \to \R\) be a continuous function supported on \([0, 1]\), and let \(g : \R \to \R\) be a continuous function supported on \([-1, 1]\) which is a polynomial on \([-1, 1]\).
  Then \(f * g\) is a polynomial on \([0, 1]\).
  (Note however that it may be non-polynomial outside of \([0, 1].\))
\end{lem}

\begin{proof}
  Since \(g\) is polynomial on \([-1, 1]\), we may find an integer \(n \geq 0\) and real numbers \(c_0, c_1, \dots, c_n\) such that
  \[
    g(x) = \sum_{j = 0}^n c_j x^j \text{ for all } x \in [-1, 1].
  \]
  On the other hand, for all \(x \in [0, 1]\), we have
  \[
    f * g(x) = \int_{-\infty}^\infty f(y) g(x - y) \; dy = \int_{[0, 1]} f(y) g(x - y) \; dy
  \]
  since \(f\) is supported on \([0, 1]\).
  Since \(x \in [0, 1]\) and the variable of integration \(y\) is also in \([0, 1]\), we have \(x - y \in [-1, 1]\).
  Thus we may substitute in our formula for \(g\) to obtain
  \[
    f * g(x) = \int_{[0, 1]} f(y) \sum_{j = 0}^n c_j (x - y)^j \; dy.
  \]
  We expand this using the binomial formula (Exercise 7.1.4 in Analysis I) to obtain
  \[
    f * g(x) = \int_{[0, 1]} f(y) \sum_{j = 0}^n c_j \sum_{k = 0}^j \dfrac{j!}{k! (j - k)!} x^k (-y)^{j - k} \; dy.
  \]
  We can interchange the two summations (by Corollary 7.1.14 in Analysis I) to obtain
  \[
    f * g(x) = \int_{[0, 1]} f(y) \sum_{k = 0}^n \sum_{j = k}^n c_j \dfrac{j!}{k! (j - k)!} x^k (-y)^{j - k} \; dy.
  \]
  (why did the limits of summation change? It may help to plot \(j\) and \(k\) on a graph).
  Now we interchange the \(k\) summation with the integral, and observe that \(x^k\) is independent of \(y\), to obtain
  \[
    f * g(x) = \sum_{k = 0}^n x^k \int_{[0, 1]} f(y) \sum_{j = k}^n c_j \dfrac{j!}{k! (j - k)!} (-y)^{j - k} \; dy.
  \]
  If we thus define
  \[
    C_k \coloneqq \int_{[0, 1]} f(y) \sum_{j = k}^n c_j \dfrac{j!}{k! (j - k)!} (-y)^{j - k} \; dy
  \]
  for each \(k = 0, \dots, n\), then \(C_k\) is a number which is independent of \(x\), and we have
  \[
    f * g(x) = \sum_{k = 0}^n C_k x^k
  \]
  for all \(x \in [0, 1]\).
  Thus \(f * g\) is a polynomial on \([0, 1]\).
\end{proof}

\begin{lem}\label{ii:3.8.14}
  Let \(f : \R \to \R\) be a continuous function supported on \([0, 1]\), which is bounded by some \(M > 0\) (i.e., \(\abs{f(x)} \leq M\) for all \(x \in \R\)), and let \(\varepsilon > 0\) and \(0 < \delta < 1\) be such that one has \(\abs{f(x) - f(y)} < \varepsilon\) whenever \(x, y \in \R\) and \(\abs{x - y} < \delta\).
  Let \(g\) be any \((\varepsilon, \delta)\)-approximation to the identity.
  Then we have
  \[
    \abs{f * g(x) - f(x)} \leq (1 + 4M) \varepsilon
  \]
  for all \(x \in [0, 1]\).
\end{lem}

\begin{proof}
  Since \(g\) is an \((\varepsilon, \delta)\)-approximation to the identity, by \cref{ii:3.8.6} we have
  \begin{itemize}
    \item \(g\) is supported on \([-1, 1]\) and \(g(x) \geq 0\) for all \(x \in [-1, 1]\).
    \item \(g\) is continuous on \(\R\) and \(\int_{-\infty}^\infty g = 1\).
    \item \(\abs{g(x)} \leq \varepsilon\) for all \(\delta \leq \abs{x} \leq 1\).
  \end{itemize}
  Since \(f\) is continuous on \(\R\), by \cref{ii:3.8.9} we have
  \begin{align*}
     & \forall x \in [0, 1], (f * g)(x)                                                                                                    \\
     & = \int_{-\infty}^\infty g(y) f(x - y) \; dy                                                                                         \\
     & = \int_{[-1, 1]} g(y) f(x - y) \; dy                                                                                                \\
     & = \int_{[-1, -\delta]} g(y) f(x - y) \; dy + \int_{[-\delta, \delta]} g(y) f(x - y) \; dy + \int_{[\delta, 1]} g(y) f(x - y) \; dy.
  \end{align*}
  By \cref{ii:ex:3.8.6} we have
  \[
    1 - 2 \varepsilon \leq \int_{[-\delta, \delta]} g = \int_{[-\delta, \delta]} g(y) \; dy \leq 1.
  \]
  Since
  \begin{align*}
             & \begin{dcases}
                 \forall x \in \R, \abs{f(x)} \leq M \\
                 \forall y \in \R, \delta \leq \abs{y} \leq 1 \implies \abs{g(y)} < \varepsilon
               \end{dcases}                                                                         \\
    \implies & \forall x \in \R, \forall \delta \leq \abs{y} \leq 1, g(y) f(x - y) \leq M \varepsilon                                                                \\
    \implies & \forall x \in \R, \begin{dcases}
                                   -M \varepsilon (1 - \delta) \leq \int_{[-1, -\delta]} g(y) f(x - y) \; dy \leq M \varepsilon (1 - \delta) \\
                                   -M \varepsilon (1 - \delta) \leq \int_{[\delta, 1]} g(y) f(x - y) \; dy \leq M \varepsilon (1 - \delta)
                                 \end{dcases} \\
    \implies & \forall x \in \R, \begin{dcases}
                                   -M \varepsilon \leq \int_{[-1, -\delta]} g(y) f(x - y) \; dy \leq M \varepsilon \\
                                   -M \varepsilon \leq \int_{[\delta, 1]} g(y) f(x - y) \; dy \leq M \varepsilon
                                 \end{dcases}                           & (\delta < 1)                           \\
    \implies & \forall x \in \R, \abs{\int_{[-1, -\delta]} g(y) f(x - y) \; dy + \int_{[\delta, 1]} g(y) f(x - y) \; dy} \leq 2 M \varepsilon
  \end{align*}
  and
  \begin{align*}
             & \forall x \in [0, 1], \forall y \in [-\delta, \delta], \abs{(x - y) - x} = \abs{y} < \delta                                                                                     \\
    \implies & \forall x \in [0, 1], \forall y \in [-\delta, \delta], \abs{f(x - y) - f(x)} < \varepsilon                                    &                        & \text{(by hypothesis)} \\
    \implies & \forall x \in [0, 1], \forall y \in [-\delta, \delta],                                                                                                                          \\
             & \abs{g(y) f(x - y) - g(y) f(x)} \leq \varepsilon g(y)                                                                         & (g(y) \geq 0)                                   \\
    \implies & \forall x \in [0, 1], \forall y \in [-\delta, \delta],                                                                                                                          \\
             & g(y) f(x) - \varepsilon g(y) \leq g(y) f(x - y) \leq g(y) f(x) + \varepsilon g(y)                                                                                               \\
    \implies & \forall x \in [0, 1],                                                                                                                                                           \\
             & \big(f(x) - \varepsilon\big) \int_{[-\delta, \delta]} g(y) \; dy                                                                                                                \\
             & \leq \int_{[-\delta, \delta]} g(y) f(x - y) \; dy                                                                                                                               \\
             & \leq \big(f(x) + \varepsilon\big) \int_{[-\delta, \delta]} g(y) \; dy                                                                                                           \\
    \implies & \forall x \in [0, 1],                                                                                                                                                           \\
             & \big(f(x) - \varepsilon\big) (1 - 2 \varepsilon) \leq \int_{[-\delta, \delta]} g(y) f(x - y) \; dy \leq f(x) + \varepsilon    &                        & \by{ii:ex:3.8.6}       \\
    \implies & \forall x \in [0, 1],                                                                                                                                                           \\
             & -2 \varepsilon f(x) - \varepsilon + 2 \varepsilon^2 \leq \int_{[-\delta, \delta]} g(y) f(x - y) \; dy - f(x) \leq \varepsilon                                                   \\
    \implies & \forall x \in [0, 1],                                                                                                                                                           \\
             & -2 \varepsilon M - \varepsilon + 2 \varepsilon^2 \leq \int_{[-\delta, \delta]} g(y) f(x - y) \; dy - f(x) \leq \varepsilon    & (f(x) \leq M)                                   \\
    \implies & \forall x \in [0, 1],                                                                                                                                                           \\
             & -2 \varepsilon M - \varepsilon \leq \int_{[-\delta, \delta]} g(y) f(x - y) \; dy - f(x) \leq \varepsilon                      & (\varepsilon \in \R^+)                          \\
    \implies & \forall x \in [0, 1],                                                                                                                                                           \\
             & -\varepsilon (2M + 1) \leq \int_{[-\delta, \delta]} g(y) f(x - y) \; dy - f(x) \leq \varepsilon (2M + 1)                      & (2M + 1 > 1)                                    \\
    \implies & \forall x \in [0, 1], \abs{\int_{[-\delta, \delta]} g(y) f(x - y) \; dy - f(x)} \leq \varepsilon (2M + 1),
  \end{align*}
  we know that
  \begin{align*}
     & \forall x \in [0, 1], \abs{(f * g)(x) - f(x)}                                                                                                            \\
     & = \abs{\int_{[-1, -\delta]} g(y) f(x - y) \; dy + \int_{[-\delta, \delta]} g(y) f(x - y) \; dy - f(x) + \int_{[\delta, 1]} g(y) f(x - y) \; dy}          \\
     & \leq \abs{\int_{[-1, -\delta]} g(y) f(x - y) \; dy + \int_{[\delta, 1]} g(y) f(x - y) \; dy} + \abs{\int_{[-\delta, \delta]} g(y) f(x - y) \; dy - f(x)} \\
     & \leq 2 M \varepsilon + \varepsilon (2M + 1)                                                                                                              \\
     & = (1 + 4M) \varepsilon.
  \end{align*}
\end{proof}

\begin{cor}[Weierstrass approximation theorem I]\label{ii:3.8.15}
  Let \(f : \R \to \R\) be a continuous function supported on \([0, 1]\).
  Then for every \(\varepsilon > 0\), there exists a function \(P : \R \to \R\) which is polynomial on \([0, 1]\) and such that \(\abs{P(x) - f(x)} \leq \varepsilon\) for all \(x \in [0, 1]\).
\end{cor}

\begin{proof}
  Let \(\varepsilon \in \R^+\).
  Since \(f\) is continuous on \(\R\) and supported on \([0, 1]\), by \cref{ii:ex:3.8.3} we know that \(f\) is bounded by some \(M \in \R^+\) and \(f\) is uniformly continuous on \(\R\).
  This means
  \[
    \exists \delta \in \R^+ : \forall x_1, x_2 \in \R, \abs{x_1 - x_2} < \delta \implies \abs{f(x_1) - f(x_2)} < \dfrac{\varepsilon}{1 + 4M}.
  \]
  In particular, we can choose some \(\delta\) such that \(0 < \delta < 1\).
  By \cref{ii:3.8.8} we know that there exists a polynomial \(P\) on \([-1, 1]\) such that \(P\) is an \((\dfrac{\varepsilon}{1 + 4M}, \delta)\)-approximation to the identity.
  By \cref{ii:3.8.6} we know that \(P\) is continuous on \(\R\) and supported on \([-1, 1]\).
  Since \(f\) is continuous on \(\R\) and supported on \([0, 1]\), by \cref{ii:3.8.13} we know that \(f * P\) is a polynomial on \([0, 1]\).
  Then by \cref{ii:3.8.14} we have
  \[
    \forall x \in [0, 1], \abs{(f * P)(x) - f(x)} \leq (1 + 4M) \dfrac{\varepsilon}{1 + 4M} = \varepsilon.
  \]
  Since \(\varepsilon\) was arbitrary, we conclude that
  \[
    \forall \varepsilon \in \R^+, \exists P \in \R^{\R} : \begin{dcases}
      P \text{ is polynomial on } [0, 1] \\
      \forall x \in [0, 1], \abs{P(x) - f(x)} \leq \varepsilon
    \end{dcases}
  \]
\end{proof}

\begin{lem}\label{ii:3.8.16}
  Let \(f : [0, 1] \to \R\) be a continuous function which equals \(0\) on the boundary of \([0, 1]\), i.e., \(f(0) = f(1) = 0\).
  Let \(F : \R \to \R\) be the function defined by setting \(F(x) \coloneqq f(x)\) for \(x \in [0, 1]\) and \(F(x) \coloneqq 0\) for \(x \notin [0, 1]\).
  Then \(F\) is also continuous.
\end{lem}

\begin{proof}
  Since \(f\) is continuous on \([0, 1]\), we know that \(f\) is continuous at \(0\) and \(1\).
  Thus we have
  \begin{align*}
             & \lim_{x \to 0 ; x \in [0, 1]} f(x) = f(0) = 0                                                                                                                             \\
    \implies & \forall \varepsilon \in \R^+, \exists \delta \in \R^+ : \big(\forall x \in [0, 1], \abs{x} < \delta \implies \abs{f(x)} < \varepsilon\big)                                \\
    \implies & \forall \varepsilon \in \R^+, \exists \delta \in \R^+ : \big(\forall x \in [0, \delta], \abs{f(x)} < \varepsilon\big)                                                     \\
    \implies & \forall \varepsilon \in \R^+, \exists \delta \in \R^+ : \big(\forall x \in [0, \delta], \abs{F(x)} < \varepsilon\big)                                                     \\
    \implies & \forall \varepsilon \in \R^+, \exists \delta \in \R^+ : \big(\forall x \in \R, \abs{x} < \delta \implies \abs{F(x)} < \varepsilon\big)     & (F(x) = 0 \text{ if } x < 0) \\
    \implies & \lim_{x \to 0 ; x \in \R} F(x) = F(0) = 0.
  \end{align*}
  Similarly we have \(\lim_{x \to 1 ; x \in \R} F(x) = F(1) = 0\).
  This means \(F\) is continuous at \(0\) and \(1\).

  Since \(f\) is continuous on \([0, 1]\), we know that \(f\) is continuous on \((0, 1)\).
  Let \(x_0 \in (0, 1)\).
  Then we have
  \begin{align*}
             & \lim_{x \to x_0 ; x \in (0, 1)} f(x) = f(x_0)                                                                                                              \\
    \implies & \forall \varepsilon \in \R^+, \exists \delta \in \R^+ : \big(\forall x \in (0, 1), \abs{x - x_0} < \delta \implies \abs{f(x) - f(x_0)} < \varepsilon\big)  \\
    \implies & \forall \varepsilon \in \R^+, \exists \delta \in \R^+ : \big(\forall x \in (0, 1), \abs{x - x_0} < \delta \implies \abs{F(x) - F(x_0)} < \varepsilon\big).
  \end{align*}
  Now we fix one pair of \(\varepsilon\) and \(\delta\).
  Let \(\delta' = \min(\delta, \abs{x_0 - 0}, \abs{1 - x_0})\).
  Then we have \(\delta' \in \R^+\) and
  \begin{align*}
             & \forall x \in \R, \abs{x - x_0} < \delta' \\
    \implies & \begin{dcases}
                 \abs{x - x_0} < \abs{x_0 - 0} \\
                 \abs{x - x_0} < \abs{1 - x_0}
               \end{dcases}             \\
    \implies & \begin{dcases}
                 \abs{x - x_0} < x_0 \\
                 \abs{x - x_0} < 1 - x_0
               \end{dcases}                    \\
    \implies & \begin{dcases}
                 0 < x < 2x_0 \\
                 2x_0 - 1 < x < 1
               \end{dcases}                           \\
    \implies & \max(0, 2x_0 - 1) < x < \min(2x_0, 1)     \\
    \implies & 0 < x < 1.
  \end{align*}
  Thus
  \[
    \forall x \in \R, \abs{x - x_0} < \delta' \implies \abs{F(x) - F(x_0)} < \varepsilon.
  \]
  Since \(\varepsilon\) was arbitrary, we conclude that \(\lim_{x \to x_0 ; x \in \R} F(x) = F(x_0)\).
  Since \(x_0\) was arbitrary, we conclude that \(\lim_{x \to x_0 ; x \in \R} F(x) = F(x_0)\) for each \(x_0 \in (0, 1)\).

  Let \(x_0 \in (-\infty, 0)\) and let \(\delta = \abs{0 - x_0}\).
  Since \(F(x) = 0\) for all \(x \in (-\infty, 0)\), we have \(F(x_0) = 0\) and
  \begin{align*}
             & \forall x \in \R, \abs{x - x_0} < \delta                                \\
    \implies & \abs{x - x_0} < \abs{0 - x_0}                                           \\
    \implies & \abs{x - x_0} < -x_0                                                    \\
    \implies & 2x_0 < x < 0                                                            \\
    \implies & F(x) = 0                                                                \\
    \implies & \forall \varepsilon \in \R^+, \abs{F(x) - F(x_0)} = 0 \leq \varepsilon.
  \end{align*}
  Thus we have
  \[
    \forall \varepsilon \in \R^+, \exists \delta \in \R^+ : \forall x \in \R, \abs{x - x_0} < \delta \implies \abs{F(x) - F(x_0)} < \varepsilon
  \]
  and \(\lim_{x \to x_0 ; x \in \R} F(x) = F(x_0) = 0\).
  Since \(x_0\) was arbitrary, we conclude that
  \[
    \forall x_0 \in (-\infty, 0), \lim_{x \to x_0 ; x \in \R} F(x) = F(x_0) = 0.
  \]
  Using similar arguments we can show that \(\lim_{x \to x_0 ; x \in \R} F(x) = F(x_0) = 0\) for all \(x_0 \in (1, \infty)\).
  Combine all proofs above we have
  \[
    \forall x_0 \in \R, \lim_{x \to x_0 ; x \in \R} F(x) = F(x_0) = 0
  \]
  and thus \(F\) is continuous on \(\R\).
\end{proof}

\begin{rmk}\label{ii:3.8.17}
  The function \(F\) obtained in \cref{ii:3.8.16} is sometimes known as the \emph{extension of \(f\) by zero}.
\end{rmk}

\begin{cor}[Weierstrass approximation theorem II]\label{ii:3.8.18}
  Let \(f : [0, 1] \to \R\) be a continuous function such that \(f(0) = f(1) = 0\).
  Then for every \(\varepsilon > 0\) there exists a polynomial \(P : [0, 1] \to \R\) such that \(\abs{P(x) - f(x)} \leq \varepsilon\) for all \(x \in [0, 1]\).
\end{cor}

\begin{proof}
  Using \cref{ii:3.8.6} we can define an \(F : \R \to \R\) such that
  \[
    \forall x \in \R, F(x) = \begin{dcases}
      0    & \text{if } x \in \R \setminus [0, 1] \\
      f(x) & \text{if } x \in [0, 1]
    \end{dcases}
  \]
  and \(F\) is continuous and supported on \([0, 1]\).
  Then by \cref{ii:3.8.15} we have
  \[
    \forall \varepsilon \in \R^+, \exists P \in \R^{\R} : \begin{dcases}
      P \text{ is a polynomial on } [0, 1] \\
      \forall x \in [0, 1], \abs{P(x) - f(x)} = \abs{P(x) - F(x)} \leq \varepsilon
    \end{dcases}
  \]
\end{proof}

\begin{cor}[Weierstrass approximation theorem III]\label{ii:3.8.19}
  Let \(f : [0, 1] \to \R\) be a continuous function.
  Then for every \(\varepsilon > 0\) there exists a polynomial \(P : [0, 1] \to \R\) such that \(\abs{P(x) - f(x)} \leq \varepsilon\) for all \(x \in [0, 1]\).
\end{cor}

\begin{proof}
  Let \(F : [0, 1] \to \R\) denote the function
  \[
    F(x) \coloneqq f(x) - f(0) - x \big(f(1) - f(0)\big).
  \]
  Observe that \(F\) is also continuous, and that \(F(0) = F(1) = 0\).
  By \cref{ii:3.8.18}, we can thus find a polynomial \(Q : [0, 1] \to \R\) such that \(\abs{Q(x) - F(x)} \leq \varepsilon\) for all \(x \in [0, 1]\).
  But
  \[
    Q(x) - F(x) = Q(x) + f(0) + x \big(f(1) - f(0)\big) - f(x),
  \]
  so the claim follows by setting \(P\) to be the polynomial \(P(x) \coloneqq Q(x) + f(0) + x \big(f(1) - f(0)\big)\).
\end{proof}

\begin{rmk}\label{ii:3.8.20}
  Note that the Weierstrass approximation theorem only works on bounded intervals \([a, b]\);
  continuous functions on \(\R\) cannot be uniformly approximated by polynomials.
  For instance, the exponential function \(f : \R \to \R\) defined by \(f(x) \coloneqq e^x\) (which we shall study rigorously in Section 4.5) cannot be approximated by any polynomial, because exponential functions grow faster than any polynomial (Exercise 4.5.9) and so there is no way one can even make the sup metric between \(f\) and a polynomial finite.
\end{rmk}

\begin{rmk}\label{ii:3.8.21}
  There is a generalization of the Weierstrass approximation theorem to higher dimensions:
  if \(K\) is any compact subset of \(\R^n\) (with the Euclidean metric \(d_{l^2}\)), and \(f : K \to \R\) is a continuous function, then for every \(\varepsilon > 0\) there exists a polynomial \(P : K \to \R\) of \(n\) variables \(x_1, \dots, x_n\) such that \(d_\infty(f, P) < \varepsilon\).
  This general theorem can be proven by a more complicated variant of the arguments here, but we will not do so.
  (There is in fact an even more general version of this theorem applicable to an arbitrary metric space, known as the \emph{Stone-Weierstrass theorem}, but this is beyond the scope of this text.)
\end{rmk}

\exercisesection

\begin{ex}\label{ii:ex:3.8.1}
  Prove \cref{ii:3.8.5}.
\end{ex}

\begin{proof}
  See \cref{ii:3.8.5}.
\end{proof}

\begin{ex}\label{ii:ex:3.8.2}
  \quad
  \begin{enumerate}
    \item Prove that for any real number \(0 \leq y \leq 1\) and any natural number \(n \geq 0\), that \((1 - y)^n \geq 1 - ny\).
    \item Show that \(\int_{-1}^1 (1 - x^2)^n \; dx \geq \dfrac{1}{\sqrt{n}}\).
    \item Prove \cref{ii:3.8.8}.
  \end{enumerate}
\end{ex}

\begin{proof}{(a)}
  For each \(n \in \N\), let \(P(n)\) be the statement ``for each \(y \in \R\), if \(0 \leq y \leq 1\), then \((1 - y)^n \geq 1 - ny\).''
  We induct on \(n\) to show that \(P(n)\) is true for all \(n \in \N\).
  For \(n = 0\), we have
  \[
    \forall y \in \R, 0 \leq y \leq 1 \implies (1 - y)^0 = 1 \geq 1 - 0y = 1.
  \]
  Thus, the base case holds.
  Suppose inductively that \(P(n)\) is true for some \(n \geq 0\).
  Then we want to show that \(P(n + 1)\) is true.
  Let \(y \in \R\) such that \(0 \leq y \leq 1\).
  Then we have
  \begin{align*}
    (1 - y)^{n + 1} & = (1 - y)^n (1 - y)                                \\
                    & \geq (1 - ny) (1 - y)  &                   & \byIH \\
                    & = 1 - (n + 1)y + n y^2                             \\
                    & \geq 1 - (n + 1)y.     & (0 \leq y \leq 1)
  \end{align*}
  Since \(y\) was arbitrary, we know that \(P(n + 1)\) is true and this closes the induction.
\end{proof}

\begin{proof}{(b)}
  Let \(n \in \Z^+\).
  Since \(f(x) = 1 - x^2\) is continuous and bounded on \([-1, 1]\), by Proposition 9.4.9 and 9.6.7 in Analysis I we know that \(f^n(x) = (1 - x^2)^n\) is continuous and bounded on \([-1, 1]\).
  Thus by Corollary 11.5.2 in Analysis I we know that \(f^n\) is Riemann integrable.
  By Corollary 11.10.3 in Analysis I we have
  \[
    \int_{[-1, 1]} f^n = \int_{[-1, 1]} f^n \cdot 1 = \int_{[-1, 1]} f^n \cdot x' = \int_{[-1, 1]} f^n \; dx = \int_{-1}^1 f^n \; dx.
  \]
  Thus
  \[
    \int_{-1}^1 (1 - x^2)^n \; dx = \int_{[-1, 1]} (1 - x^2)^n = \int_{[-1, \dfrac{-1}{\sqrt{n}}]} (1 - x^2)^n + \int_{[\dfrac{-1}{\sqrt{n}}, \dfrac{1}{\sqrt{n}}]} (1 - x^2)^n + \int_{[\dfrac{1}{\sqrt{n}}, 1]} (1 - x^2)^n.
  \]
  Since
  \[
    \forall x \in [-1, 1], 1 \geq \abs{x} \geq \dfrac{1}{\sqrt{n}} \implies 1 \geq x^2 \geq \dfrac{1}{n} \implies 0 \leq 1 - x^2 \leq \dfrac{n - 1}{n},
  \]
  we know that
  \[
    \int_{-1}^1 (1 - x^2)^n \; dx \geq \int_{[\dfrac{-1}{\sqrt{n}}, \dfrac{1}{\sqrt{n}}]} (1 - x^2)^n.
  \]
  By \cref{ii:ex:3.8.2}(a) we have
  \[
    \int_{-1}^1 (1 - x^2)^n \; dx \geq \int_{[\dfrac{-1}{\sqrt{n}}, \dfrac{1}{\sqrt{n}}]} (1 - x^2)^n \geq \int_{[\dfrac{-1}{\sqrt{n}}, \dfrac{1}{\sqrt{n}}]} (1 - n x^2).
  \]
  Since
  \begin{align*}
    \int_{[\dfrac{-1}{\sqrt{n}}, \dfrac{1}{\sqrt{n}}]} (1 - n x^2) & = \int_{[\dfrac{-1}{\sqrt{n}}, \dfrac{1}{\sqrt{n}}]} 1 - n \int_{[\dfrac{-1}{\sqrt{n}}, \dfrac{1}{\sqrt{n}}]} x^2 \\
                                                                   & = \dfrac{2}{\sqrt{n}} - \dfrac{n}{3} \bigg(\dfrac{1}{n \sqrt{n}} - \dfrac{-1}{n \sqrt{n}}\bigg)                   \\
                                                                   & = \dfrac{2}{\sqrt{n}} - \dfrac{2}{3 \sqrt{n}}                                                                     \\
                                                                   & = \dfrac{4}{3 \sqrt{n}} \geq \dfrac{1}{\sqrt{n}},
  \end{align*}
  we have
  \[
    \int_{-1}^1 (1 - x^2)^n \; dx \geq \dfrac{1}{\sqrt{n}}.
  \]
\end{proof}

\begin{proof}{(c)}
  See \cref{ii:3.8.8}.
\end{proof}

\begin{ex}\label{ii:ex:3.8.3}
  Let \(f : \R \to \R\) be a compactly supported, continuous function.
  Show that \(f\) is bounded and uniformly continuous.
\end{ex}

\begin{proof}
  Since \(f\) is compactly supported, by \cref{ii:3.8.4} we know that there exists some \(a, b \in \R\) such that
  \[
    \forall x \notin [a, b], f(x) = 0.
  \]
  Since \([a, b]\) is closed and bounded in \((\R, d_{l^1}|_{\R \times \R})\), by \cref{ii:1.5.7} we know that \(\big([a, b], d_{l^1}|_{\R \times \R}\big)\) is compact.
  Since \(\big([a, b], d_{l^1}|_{\R \times \R}\big)\) is compact and \(f\) is continuous on \([a, b]\), by \cref{ii:2.3.2} we know that \(f\) is bounded.
  Since \(f\) is bounded and continuous on \([a, b]\), by \cref{ii:2.3.5} \(f\) is uniformly continuous on \([a, b]\).

  Since \(f\) is continuous at \(a\), we have
  \begin{align*}
             & \forall \varepsilon \in \R^+, \exists \delta \in \R^+ : \big(\forall x \in \R, \abs{x - a} < \delta \implies \abs{f(x) - f(a)} < \varepsilon\big)               \\
    \implies & \forall \varepsilon \in \R^+, \exists \delta \in \R^+ : \big(\forall x \in \R, x \in (a - \delta, a + \delta) \implies \abs{f(x) - f(a)} < \varepsilon\big)     \\
    \implies & \forall \varepsilon \in \R^+, \exists \delta \in \R^+ : \big(\forall x \in \R, x \in (a - \delta, a) \implies \abs{f(x) - f(a)} = \abs{f(a)} < \varepsilon\big) \\
    \implies & \forall \varepsilon \in \R^+, \abs{f(a)} < \varepsilon                                                                                                          \\
    \implies & f(a) = 0.
  \end{align*}
  Similarly, we have \(f(b) = 0\).
  If \(a = b\), then \(f\) is zero function, and we have
  \[
    \forall \varepsilon \in \R^+, \forall \delta \in \R^+, \forall x_1, x_2 \in \R, \abs{x_1 - x_2} < \delta \implies \abs{f(x_1) - f(x_2)} = 0 < \varepsilon.
  \]
  Thus \(f\) is uniformly continuous on \(\R\).
  Suppose that \(a \neq b\).
  Since \(f\) in uniformly continuous on \([a, b]\), by \cref{ii:2.3.4} we have
  \[
    \forall \varepsilon \in \R^+, \exists \delta_1 \in \R^+ : \forall x_1, x_2 \in [a, b], \abs{x_1 - x_2} < \delta_1 \implies \abs{f(x_1) - f(x_2)} < \varepsilon.
  \]
  Now fix one pair of \(\varepsilon\) and \(\delta_1\).
  Since \(\lim_{x \to a ; x \in \R} f(x) = f(a) = 0\), we have
  \[
    \exists \delta_2 \in \R^+ : \forall x \in \R, \abs{x - a} < \delta_2 < b - a \implies \abs{f(x) - f(a)} = \abs{f(x)} < \varepsilon.
  \]
  Similarly, we have
  \[
    \exists \delta_3 \in \R^+ : \forall x \in \R, \abs{x - b} < \delta_3 < b - a \implies \abs{f(x) - f(b)} = \abs{f(x)} < \varepsilon.
  \]
  Let \(\delta = \min(\delta_1, \delta_2, \delta_3)\).
  Then we have
  \begin{align*}
             & \forall x_1, x_2 \in \R, \abs{x_1 - x_2} < \delta                                                                                                          \\
    \implies & \begin{dcases}
                 \abs{x_1 - x_2} < \delta_1 \implies \abs{f(x_1) - f(x_2)} < \varepsilon & \text{if } \big(x_1, x_2 \in [a, b]\big)                                 \\
                 x_1 - x_2 < \delta_2 \implies a \leq x_1 < x_2 + \delta_2               & \text{if } \big(x_1 \in [a, b]\big) \land \big(x_2 \in (-\infty, a)\big) \\
                 x_2 - x_1 < \delta_3 \implies x_2 - \delta_3 < x_1 \leq b               & \text{if } \big(x_1 \in [a, b]\big) \land \big(x_2 \in (b, \infty)\big)  \\
                 \abs{f(x_1) - f(x_2)} = 0 < \varepsilon                                 & \text{if } \big(x_1, x_2 \notin [a, b]\big)
               \end{dcases} \\
    \implies & \begin{dcases}
                 \abs{f(x_1) - f(x_2)} < \varepsilon     & \text{if } \big(x_1, x_2 \in [a, b]\big)                                 \\
                 x_1 - a < x_2 - a + \delta_2 < \delta_2 & \text{if } \big(x_1 \in [a, b]\big) \land \big(x_2 \in (-\infty, a)\big) \\
                 \delta_3 > b - x_2 + \delta_3 > b - x_1 & \text{if } \big(x_1 \in [a, b]\big) \land \big(x_2 \in (b, \infty)\big)  \\
                 \abs{f(x_1) - f(x_2)} < \varepsilon     & \text{if } \big(x_1, x_2 \notin [a, b]\big)
               \end{dcases}                                 \\
    \implies & \begin{dcases}
                 \abs{f(x_1) - f(x_2)} < \varepsilon                          & \text{if } \big(x_1, x_2 \in [a, b]\big)                                 \\
                 \abs{x_1 - a} < \delta_2 \implies \abs{f(x_1)} < \varepsilon & \text{if } \big(x_1 \in [a, b]\big) \land \big(x_2 \in (-\infty, a)\big) \\
                 \abs{x_1 - b} < \delta_3 \implies \abs{f(x_1)} < \varepsilon & \text{if } \big(x_1 \in [a, b]\big) \land \big(x_2 \in (b, \infty)\big)  \\
                 \abs{f(x_1) - f(x_2)} < \varepsilon                          & \text{if } \big(x_1, x_2 \notin [a, b]\big)
               \end{dcases}            \\
    \implies & \abs{f(x_1) - f(x_2)} < \varepsilon.
  \end{align*}
  Since \(\varepsilon\) was arbitrary, we have
  \[
    \forall \varepsilon \in \R^+, \exists \delta \in \R^+ : \forall x_1, x_2 \in \R, \abs{x_1 - x_2} < \delta \implies \abs{f(x_1) - f(x_2)} < \varepsilon
  \]
  and \(f\) is uniformly continuous on \(\R\).
\end{proof}

\begin{ex}\label{ii:ex:3.8.4}
  Prove \cref{ii:3.8.11}.
\end{ex}

\begin{proof}
  See \cref{ii:3.8.11}.
\end{proof}

\begin{ex}\label{ii:ex:3.8.5}
  Let \(f : \R \to \R\) and \(g : \R \to \R\) be continuous, compactly supported functions.
  Suppose that \(f\) is supported on the interval \([0, 1]\), and \(g\) is constant on the interval \([0, 2]\)
  (i.e., there is a real number \(c\) such that \(g(x) = c\) for all \(x \in [0, 2]\)).
  Show that the convolution \(f * g\) is constant on the interval \([1, 2]\).
\end{ex}

\begin{proof}
  We have
  \begin{align*}
    \forall x \in [1, 2], (f * g)(x) & = \int_{-\infty}^\infty f(y) g(x - y) \; dy &                    & \by{ii:3.8.9} \\
                                     & = \int_{[0, 1]} f(y) g(x - y) \; dy         &                    & \by{ii:3.8.4} \\
                                     & = \int_{[0, 1]} c f(y) \; dy                & (x - y \in [0, 2])                 \\
                                     & = c \int_{[0, 1]} f(y) \; dy.
  \end{align*}
  Since \(\int_{[0, 1]} f(y) \; dy\) is independent of \(x\), we know that \(f * g\) is constant on \([1, 2]\).
\end{proof}

\begin{ex}\label{ii:ex:3.8.6}
  \quad
  \begin{enumerate}
    \item Let \(g\) be an \((\varepsilon, \delta)\) approximation to the identity.
          Show that \(1 - 2 \varepsilon \leq \int_{[-\delta, \delta]} g \leq 1\).
    \item Prove \cref{ii:3.8.14}.
  \end{enumerate}
\end{ex}

\begin{proof}{(a)}
  By \cref{ii:3.8.6} we know that
  \begin{itemize}
    \item \(\varepsilon \in \R^+\).
    \item \(\delta \in \R^+\) such that \(0 < \delta < 1\).
    \item \(g\) is supported on \([-1, 1]\) and \(g(x) \geq 0\) for all \(x \in [-1, 1]\).
    \item \(g\) is continuous on \(\R\) and \(\int_{-\infty}^\infty g = 1\).
    \item \(\abs{g(x)} \leq \varepsilon\) for each \(\delta \leq \abs{x} \leq 1\).
  \end{itemize}
  By \cref{ii:3.8.4} we have
  \[
    \int_{-\infty}^\infty g = \int_{[-1, 1]} g = 1.
  \]
  Thus
  \begin{align*}
             & \forall \delta \leq \abs{x} \leq 1, \abs{g(x)} \leq \varepsilon                                                                                \\
    \implies & 1 = \int_{[-1, 1]} g                                                                                                                           \\
             & = \int_{[-1, -\delta]} g + \int_{[-\delta, \delta]} g + \int_{[\delta, 1]} g                                                                   \\
             & \leq (-\delta + 1) \varepsilon + \int_{[-\delta, \delta]} g + (1 - \delta) \varepsilon                                                         \\
             & = 2 \varepsilon (1 - \delta) + \int_{[-\delta, \delta]} g                                                                                      \\
             & \leq 2 \varepsilon + \int_{[-\delta, \delta]} g                                                 & (1 - \delta < 1)                             \\
             & \leq 2 \varepsilon + \int_{[-1, -\delta]} g + \int_{[-\delta, \delta]} g + \int_{[\delta, 1]} g & (g(x) \geq 0 \text{ for all } x \in [-1, 1]) \\
             & = 2 \varepsilon + \int_{[-1, 1]} g                                                                                                             \\
             & = 2 \varepsilon + 1                                                                                                                            \\
    \implies & 1 - 2 \varepsilon \leq \int_{[-\delta, \delta]} g \leq 1.
  \end{align*}
\end{proof}

\begin{proof}{(b)}
  See \cref{ii:3.8.14}.
\end{proof}

\begin{ex}\label{ii:ex:3.8.7}
  Prove \cref{ii:3.8.15}.
\end{ex}

\begin{proof}
  See \cref{ii:3.8.15}.
\end{proof}

\begin{ex}\label{ii:ex:3.8.8}
  Let \(f : [0, 1] \to \R\) be a continuous function, and suppose that \(\int_{[0, 1]} f(x) x^n \; dx = 0\) for all non-negative integers \(n = 0, 1, 2, \dots\).
  Show that \(f\) must be the zero function \(f \equiv 0\).
\end{ex}

\begin{proof}
  Let \(P : \R \to \R\) be a polynomial with degree \(n\).
  Then by \cref{ii:3.8.1} we have
  \[
    \forall x \in \R, P(x) = \sum_{j = 0}^n c_j x^j.
  \]
  By hypothesis we have
  \begin{align*}
    \int_{[0, 1]} f(x) P(x) \; dx & = \int_{[0, 1]} f(x) \sum_{j = 0}^n c_j x^j \; dx \\
                                  & = \sum_{j = 0}^n c_j \int_{[0, 1]} f(x) x^j \; dx \\
                                  & = 0.
  \end{align*}
  Since \([0, 1]\) is closed and bounded in \((\R, d_{l^1}|_{\R \times \R})\), by \cref{ii:1.5.7} we know that \(\big([0, 1], d_{l^1}|_{\R \times \R}\big)\) is compact.
  By \cref{ii:2.3.2} we know that \(f\) is bounded, i.e., there exists a \(M \in \R^+\) such that \(\abs{f(x)} \leq M\) for all \(x \in [0, 1]\).

  Since \(f\) is continuous on \([0, 1]\), by \cref{ii:3.8.3} we know that
  \[
    \forall \varepsilon \in \R^+, \exists P \in \R^{\R} : \begin{dcases}
      P \text{ is a polynomial on } [0, 1] \\
      d_{\infty}(P, f) \leq \dfrac{\varepsilon}{M}
    \end{dcases}
  \]
  Fix one pair of \(\varepsilon\) and \(P\).
  Then we have
  \begin{align*}
             & d_\infty(P, f) \leq \dfrac{\varepsilon}{M}                                                                                                                          \\
    \implies & \sup_{x \in [0, 1]} \abs{P(x) - f(x)} \leq \dfrac{\varepsilon}{M}                                                                       &  & \by{ii:3.4.2}          \\
    \implies & \forall x \in [0, 1], \abs{P(x) - f(x)} \leq \dfrac{\varepsilon}{M}                                                                                                 \\
    \implies & \forall x \in [0, 1], \abs{f(x) P(x) - f(x) f(x)} \leq \dfrac{\varepsilon \abs{f(x)}}{M} \leq \dfrac{\varepsilon M}{M} \leq \varepsilon                             \\
    \implies & \forall x \in [0, 1], f(x) P(x) - \varepsilon \leq \big(f(x)\big)^2 \leq f(x) P(x) + \varepsilon                                                                    \\
    \implies & \forall x \in [0, 1], \int_{[0, 1]} f(x) P(x) - \varepsilon \; dx = -\varepsilon                                                                                    \\
             & \leq \int_{[0, 1]} \big(f(x)\big)^2 \; dx \leq \int_{[0, 1]} f(x) P(x) + \varepsilon \; dx = \varepsilon                                &  & \text{(by hypothesis)} \\
    \implies & \forall x \in [0, 1], -\varepsilon \leq \int_{[0, 1]} \big(f(x)\big)^2 \; dx \leq \varepsilon.
  \end{align*}
  Since \(\varepsilon\) was arbitrary, we know that
  \[
    \forall \varepsilon \in \R^+, \abs{\int_{[0, 1]} \big(f(x)\big)^2 \; dx} \leq \varepsilon \implies \abs{\int_{[0, 1]} \big(f(x)\big)^2 \; dx} = \int_{[0, 1]} \big(f(x)\big)^2 \; dx = 0.
  \]
  Since \(f\) is continuous on \([0, 1]\) and \(\big(f(x)\big)^2 \geq 0\) for all \(x \in [0, 1]\), by Exercise 11.4.2 in Analysis I we know that
  \[
    \forall x \in [0, 1], \big(f(x)\big)^2 = 0.
  \]
  Thus we have \(f(x) = 0\) for all \(x \in [0, 1]\).
\end{proof}

\begin{ex}\label{ii:ex:3.8.9}
  Prove \cref{ii:3.8.16}.
\end{ex}

\begin{proof}
  See \cref{ii:3.8.16}.
\end{proof}

\chapter{Power series}\label{ii:ch:4}

\section{Formal power series}\label{sec:4.1}

\begin{defn}[Formal power series]\label{4.1.1}
  Let \(a\) be a real number.
  A \emph{formal power series centered at \(a\)} is any series of the form
  \[
    \sum_{n = 0}^\infty c_n (x - a)^n
  \]
  where \(c_0, c_1, \dots\) is a sequence of real numbers (not depending on \(x\));
  we refer to \(c_n\) as the \emph{\(n^{\text{th}}\) coefficient} of this series.
  Note that each term \(c_n (x - a)^n\) in this series is a function of a real variable \(x\).
\end{defn}

\begin{note}
  We call these power series \emph{formal} because we do not yet assume that these series converge for any \(x\).
  However, these series are automatically guaranteed to converge when \(x = a\).
  In general, the closer \(x\) gets to \(a\), the easier it is for this series to converge.
\end{note}

\setcounter{thm}{2}
\begin{defn}[Radius of convergence]\label{4.1.3}
  Let \(\sum_{n = 0}^\infty c_n (x - a)^n\) be a formal power series.
  We define the \emph{radius of convergence \(R\)} of this series to be the quantity
  \[
    R \coloneqq \dfrac{1}{\limsup_{n \to \infty} \abs{c_n}^{\dfrac{1}{n}}}
  \]
  where we adopt the convention that \(\dfrac{1}{0} = +\infty\) and \(\dfrac{1}{+\infty} = 0\).
\end{defn}

\begin{rmk}\label{4.1.4}
  Each number \(\abs{c_n}^{1 / n}\) is non-negative, so the limit \(\limsup_{n \to \infty} \abs{c_n}^{1 / n}\) can take on any value from \(0\) to \(+\infty\), inclusive.
  Thus \(R\) can also take on any value between \(0\) and \(+\infty\) inclusive
  (in particular it is not necessarily a real number).
  Note that the radius of convergence always exists, even if the sequence \(\abs{c_n}^{1 / n}\) is not convergent, because the lim sup of any sequence always exists
  (though it might be \(+\infty\) or \(-\infty\)).
\end{rmk}

\setcounter{thm}{5}
\begin{thm}\label{4.1.6}
  Let \(\sum_{n = 0}^\infty c_n (x - a)^n\) be a formal power series, and let \(R\) be its radius of convergence.
  \begin{enumerate}
    \item (Divergence outside of the radius of convergence)
          If \(x \in \R\) is such that \(\abs{x - a} > R\), then the series \(\sum_{n = 0}^\infty c_n (x - a)^n\) is divergent for that value of \(x\).
    \item (Convergence inside the radius of convergence)
          If \(x \in \R\) is such that \(\abs{x - a} < R\), then the series \(\sum_{n = 0}^\infty c_n (x - a)^n\) is absolutely convergent for that value of \(x\).
  \end{enumerate}
  For parts (c)-(e) we assume that \(R > 0\)
  (i.e., the series converges at at least one other point than \(x = a\)).
  Let \(f : (a - R, a + R) \to \R\) be the function \(f(x) \coloneqq \sum_{n = 0}^\infty c_n (x - a)^n\);
  this function is guaranteed to exist by (b).
  \begin{enumerate}
    \setcounter{enumi}{2}
    \item (Uniform convergence on compact sets)
          For any \(0 < r < R\), the series \(\sum_{n = 0}^\infty c_n (x - a)^n\) converges uniformly to \(f\) on the compact interval \([a - r, a + r]\).
          In particular, \(f\) is continuous on \((a - R, a + R)\).
    \item (Differentiation of power series)
          The function \(f\) is differentiable on \((a - R, a + R)\), and for any \(0 < r < R\), the series \(\sum_{n = 1}^\infty n c_n (x - a)^{n - 1}\) converges uniformly to \(f'\) on the interval \([a - r, a + r]\).
    \item (Integration of power series)
          For any closed interval \([y, z]\) contained in \((a - R, a + R)\), we have
          \[
            \int_{[y, z]} f = \sum_{n = 0}^\infty c_n \dfrac{(z - a)^{n + 1} - (y - a)^{n + 1}}{n + 1}.
          \]
  \end{enumerate}
\end{thm}

\begin{proof}{(a)}{(b)}
  We split into three cases:
  \begin{itemize}
    \item \(R = +\infty\).
          Since \(\abs{x - a} \geq 0\), we cannot have \(\abs{x - a} > +\infty\).
          Thus we only consider the case \(\abs{x - a} < +\infty\).
          \begin{align*}
                     & \abs{x - a} < +\infty = \dfrac{1}{0}                                  &  & \by{4.1.3}                              \\
            \implies & \limsup_{n \to \infty} \abs{c_n}^{\dfrac{1}{n}} = 0                   &  & \by{4.1.3}                              \\
            \implies & \abs{x - a} \cdot \limsup_{n \to \infty} \abs{c_n}^{\dfrac{1}{n}} = 0                                              \\
            \implies & \limsup_{n \to \infty} \abs{c_n (x - a)^n}^{\dfrac{1}{n}} = 0 < 1                                                  \\
            \implies & \sum_{n = 0}^\infty c_n (x - a)^n \text{ is absolutely convergent}.   &  & \text{(by Theorem 7.5.1 in Analysis I)}
          \end{align*}
    \item \(R = 0\).
          Since \(\abs{x - a} \geq 0\), we cannot have \(\abs{x - a} < 0\).
          Thus we only consider the case \(\abs{x - a} > 0\).
          \begin{align*}
                     & \abs{x - a} > 0 = \dfrac{1}{+\infty}                                            &  & \by{4.1.3}                              \\
            \implies & \limsup_{n \to \infty} \abs{c_n}^{\dfrac{1}{n}} = +\infty                       &  & \by{4.1.3}                              \\
            \implies & \limsup_{n \to \infty} \big(\abs{c_n}^{\dfrac{1}{n}} \abs{x - a}\big) = +\infty &  & \text{(proof by contradiction)}         \\
            \implies & \limsup_{n \to \infty} \abs{c_n (x - a)^n}^{\dfrac{1}{n}} = +\infty > 1                                                      \\
            \implies & \sum_{n = 0}^\infty c_n (x - a)^n \text{ is divergent}.                         &  & \text{(by Theorem 7.5.1 in Analysis I)}
          \end{align*}
    \item \(R \in \R^+\).
          First suppose that \(\abs{x - a} > R\).
          Then we have
          \begin{align*}
                     & \abs{x - a} > \dfrac{1}{\limsup_{n \to \infty} \abs{c_n}^{\dfrac{1}{n}}} &  & \by{4.1.3}                              \\
            \implies & \abs{x - a} \cdot \limsup_{n \to \infty} \abs{c_n}^{\dfrac{1}{n}} > 1                                                 \\
            \implies & \limsup_{n \to \infty} \abs{c_n (x - a)^n}^{\dfrac{1}{n}} > 1                                                         \\
            \implies & \sum_{n = 0}^\infty c_n (x - a)^n \text{ is divergent}.                  &  & \text{(by Theorem 7.5.1 in Analysis I)}
          \end{align*}
          Now suppose that \(\abs{x - a} < R\).
          Then we have
          \begin{align*}
                     & \abs{x - a} < \dfrac{1}{\limsup_{n \to \infty} \abs{c_n}^{\dfrac{1}{n}}} &  & \by{4.1.3}                              \\
            \implies & \abs{x - a} \cdot \limsup_{n \to \infty} \abs{c_n}^{\dfrac{1}{n}} < 1                                                 \\
            \implies & \limsup_{n \to \infty} \abs{c_n (x - a)^n}^{\dfrac{1}{n}} < 1                                                         \\
            \implies & \sum_{n = 0}^\infty c_n (x - a)^n \text{ is absolutely convergent}.      &  & \text{(by Theorem 7.5.1 in Analysis I)}
          \end{align*}
  \end{itemize}
  From all cases above we conclude that
  \[
    \begin{dcases}
      \abs{x - a} < R \implies \sum_{n = 0}^\infty c_n (x - a)^n \text{ is absolutely convergent} \\
      \abs{x - a} > R \implies \sum_{n = 0}^\infty c_n (x - a)^n \text{ is divergent}
    \end{dcases}
  \]
\end{proof}

\begin{proof}{(c)}
  Let \(r \in (0, R)\).
  Since
  \[
    \forall x \in [a - r, a + r] \implies \abs{x - a} < r < R,
  \]
  by \cref{4.1.6}(b) we know that \(\sum_{n = 0}^\infty c_n (x - a)^n\) is absolutely convergent for all \(x \in [a - r, a + r]\).
  For each \(n \in \N\), we define \(f_n : [a - r, a + r] \to \R\) by setting \(f_n(x) = c_n (x - a)^n\) for all \(x \in [a - r, a + r]\).
  Since
  \begin{align*}
             & r < R                                                                                                                                            \\
    \implies & \dfrac{r}{R} < 1                                                                                                                                 \\
    \implies & r \big(\limsup_{n \to \infty} \abs{c_n}^{\dfrac{1}{n}}\big) = \limsup_{n \to \infty} \abs{c_n r^n}^{\dfrac{1}{n}} < 1 &  & \by{4.1.3}            \\
    \implies & \sum_{n = 0}^\infty c_n r^n \text{ is absolutely convergent}                                                          &  & \text{(by root test)}
  \end{align*}
  and
  \begin{align*}
             & \forall n \in \N, \forall x \in [a - r, a + r], (x - a)^n \leq r^n                         \\
    \implies & \forall n \in \N, \forall x \in [a - r, a + r], c_n (x - a)^n \leq c_n r^n                 \\
    \implies & \forall n \in \N, \norm*{f_n}_\infty \leq c_n r^n                          &  & \by{3.5.5} \\
    \implies & \sum_{n = 0}^\infty \norm*{f_n}_\infty \leq \sum_{n = 0}^\infty c_n r^n,
  \end{align*}
  by \cref{3.5.7} we know that \(\big(\sum_{n = 0}^N f_n\big)_{N = 0}^\infty\) converges uniformly to some function \(g : [a - r, a + r] \to \R\) on \([a - r, a + r]\) with respect to \(d_{l^1}|_{\R \times \R}\) and \(g\) is continuous on \([a - r, a + r]\).
  But by \cref{3.5.2} we know that
  \[
    \forall x \in [a - r, a + r], g(x) = \sum_{n = 0}^\infty f_n(x) = \sum_{n = 0}^\infty c_n (x - a)^n = f(x).
  \]
  Thus \(f\) is continuous on \([a - r, a + r]\).
  Since \(r\) is arbitrary, we conclude that
  \begin{align*}
     & \forall r \in (0, R), \bigg(x \mapsto \sum_{n = 0}^N c_n (x - a)^n\bigg)_{N = 0}^\infty \text{ converges uniformly to }                 \\
     & f = \bigg(x \mapsto \sum_{n = 0}^\infty c_n (x - a)^n\bigg) \text{ on } [a - r, a + r] \text{ with respect to } d_{l^1}|_{\R \times \R} \\
     & \text{ and } f \text{ is continuous on } (a - R, a + R).
  \end{align*}
\end{proof}

\begin{proof}{(d)}
  Let \(r \in (0, R)\).
  For each \(n \in \Z^+\), we define \(f_n : [a - r, a + r] \to \R\) by setting \(f_n(x) = c_n (x - a)^n\) for all \(x \in [a - r, a + r]\).
  Since \(f_n\) is polynomial for all \(n \in \Z^+\), we know that \(f_n'\) is well-defined and
  \[
    \forall n \in \Z^+, \forall x \in [a - r, a + r], f_n'(x) = n c_n (x - a)^{n - 1}.
  \]
  Again, \(f_n'\) is polynomial and thus is continuous on \([a - r, a + r]\) for all \(n \in \Z^+\).
  By limit laws we have
  \begin{align*}
             & \forall N \in \Z^+, \forall x \in [a - r, a + r], \bigg(\sum_{n = 1}^N f_n\bigg)'(x) = \sum_{n = 1}^N f_n'(x) \\
    \implies & \forall N \in \Z^+, \bigg(\sum_{n = 1}^N f_n\bigg)' = \sum_{n = 1}^N f_n'.
  \end{align*}
  Since
  \begin{align*}
             & \limsup_{n \to \infty} \abs{c_n}^{\dfrac{1}{n}} = \dfrac{1}{R}                                                             &         & \by{4.1.3}                                      \\
    \implies & \limsup_{n \to \infty} \abs{c_n}^{\dfrac{1}{n}} \in \R                                                                     & (R > 0)                                                   \\
    \implies & \big(\lim_{n \to \infty} n^{\dfrac{1}{n}}\big) \big(\limsup_{n \to \infty} \abs{c_n}^{\dfrac{1}{n}}\big) = \dfrac{1}{R}    &         & \text{(by Proposition 7.5.4 in Analysis I)}     \\
    \implies & \big(\limsup_{n \to \infty} n^{\dfrac{1}{n}}\big) \big(\limsup_{n \to \infty} \abs{c_n}^{\dfrac{1}{n}}\big) = \dfrac{1}{R} &         & \text{(by Proposition 6.4.12(f) in Analysis I)} \\
    \implies & \limsup_{n \to \infty} \abs{n c_n}^{\dfrac{1}{n}} = \dfrac{1}{R}                                                                                                                       \\
    \implies & \limsup_{n \to \infty} \abs{n c_n (x - a)^n}^{\dfrac{1}{n}} = \dfrac{\abs{x - a}}{R} < 1,                                  &         & \text{(if \(\abs{x - a} < r\))}
  \end{align*}
  by root test we know that \(\sum_{n = 1}^\infty n c_n (x - a)^n\) is absolutely convergent.
  Since
  \[
    \sum_{n = 1}^\infty n c_n (x - a)^n = (x - a) \bigg(\sum_{n = 1}^\infty n c_n (x - a)^{n - 1}\bigg),
  \]
  we know that \(\sum_{n = 1}^\infty n c_n (x - a)^{n - 1}\) is convergent.
  For each \(n \in \Z^+\), we define \(c_{n - 1}' = n c_n\).
  Then by Proposition 7.2.14 in Analysis I we have
  \[
    \forall x \in [a - r, a + r], \sum_{n = 1}^\infty f_n'(x) = \sum_{n = 1}^\infty n c_n (x - a)^{n - 1} = \sum_{n = 1}^\infty c_{n - 1}' (x - a)^{n - 1} = \sum_{n = 0}^\infty c_n' (x - a)^n.
  \]
  For each \(n \in \N\), we define \(g_n : [a - r, a + r] \to \R\) by setting \(g_n(x) = c_n' (x - a)^n\) for all \(x \in [a - r, a + r]\).
  By \cref{4.1.6}(c) we know that
  \begin{align*}
             & \bigg(\sum_{n = 0}^N g_n\bigg)_{N = 0}^\infty \text{ converges uniformly to some } g : [a - r, a + r] \to \R              \\
             & \text{ on } [a - r, a + r] \text{ with respect to } d_{l^1}|_{\R \times \R}                                               \\
    \implies & \Bigg(\bigg(\sum_{n = 1}^N f_n\bigg)'\Bigg)_{N = 1}^\infty \text{ converges uniformly to some } g : [a - r, a + r] \to \R \\
             & \text{ on } [a - r, a + r] \text{ with respect to } d_{l^1}|_{\R \times \R}
  \end{align*}
  By \cref{3.5.2} we know that
  \[
    \forall x \in [a - r, a + r], g(x) = \sum_{n = 0}^\infty c_n' (x - a)^n = \sum_{n = 1}^\infty n c_n (x - a)^{n - 1} = \sum_{n = 1}^\infty f_n'(x).
  \]
  Since
  \[
    \lim_{N \to \infty} \sum_{n = 1}^N f_n(a) = \lim_{N \to \infty} \sum_{n = 1}^N c_n (a - a)^n = \lim_{N \to \infty} 0 = 0,
  \]
  by \cref{3.7.1} we have
  \begin{align*}
     & \bigg(\sum_{n = 1}^N f_n\bigg)_{N = 1}^\infty \text{ converges uniformly to } f \\
     & \text{ on } [a - r, a + r] \text{ with respect to } d_{l^1}|_{\R \times \R}
  \end{align*}
  and \(f' = g\).
  By \cref{3.5.2} this means
  \[
    \forall x \in [a - r, a + r], f'(x) = \bigg(\sum_{n = 1}^\infty f_n\bigg)'(x) = \sum_{n = 1}^\infty n c_n \abs{x - a}^{n - 1}.
  \]
  Since \(r\) is arbitrary, we conclude that \(f\) is differentiable on \((a - R, a + R)\) and
  \begin{align*}
     & \forall r \in (0, R), \bigg(x \mapsto \sum_{n = 1}^N n c_n (x - a)^{n - 1}\bigg)_{N = 1}^\infty \text{ converges uniformly to }                   \\
     & f' = \bigg(x \mapsto \sum_{n = 1}^\infty n c_n (x - a)^{n - 1}\bigg) \text{ on } [a - r, a + r] \text{ with respect to } d_{l^1}|_{\R \times \R}.
  \end{align*}
\end{proof}

\begin{proof}{(e)}
  If \(y = z = a\), then we have
  \[
    \int_{[a, a]} f = 0 = \sum_{n = 0}^\infty c_n \dfrac{(a - a)^{n + 1} - (a - a)^{n + 1}}{n + 1}.
  \]
  So suppose that \((y \neq a) \lor (z \neq a)\).
  Without the loss of generality, suppose that \(y \neq a\).
  Let \(r = \max(\abs{y - a}, \abs{z - a})\).
  Since \([y, z] \subseteq (a - R, a + R)\), we have
  \begin{align*}
             & a - R < y \leq z < a + R              \\
    \implies & -R < y - a \leq z - a < R             \\
    \implies & \begin{dcases}
                 0 < \abs{y - a} < R \\
                 0 \leq \abs{z - a} < R
               \end{dcases}                 \\
    \implies & \begin{dcases}
                 0 < \abs{y - a} \leq r < R \\
                 0 \leq \abs{z - a} \leq r < R
               \end{dcases}          \\
    \implies & -R < -r \leq y - a < z - a \leq r < R \\
    \implies & [y, z] \subseteq [a - r, a + r].
  \end{align*}
  By \cref{4.1.6}(c) we know that
  \begin{align*}
     & \bigg(x \mapsto \sum_{n = 0}^N c_n (x - a)^n\bigg)_{N = 0}^\infty \text{ converges uniformly to }                                       \\
     & f = \bigg(x \mapsto \sum_{n = 0}^\infty c_n (x - a)^n\bigg) \text{ on } [a - r, a + r] \text{ with respect to } d_{l^1}|_{\R \times \R} \\
     & \text{ and } f \text{ is continuous on } [a - r, a + r].
  \end{align*}
  Thus we have
  \begin{align*}
     & \bigg(x \mapsto \sum_{n = 0}^N c_n (x - a)^n\bigg)_{N = 0}^\infty \text{ converges uniformly to }                               \\
     & f = \bigg(x \mapsto \sum_{n = 0}^\infty c_n (x - a)^n\bigg) \text{ on } [y, z] \text{ with respect to } d_{l^1}|_{\R \times \R} \\
     & \text{ and } f \text{ is continuous on } [y, z].
  \end{align*}
  By Corollary 11.5.2 we know that \(\int_{[y, z]} f\) is well-defined.
  Thus we have
  \begin{align*}
    \int_{[y, z]} f & = \int_{[y, z]} f(x) \; dx                                                                  \\
                    & = \int_{[y, z]} \bigg(\sum_{n = 0}^\infty c_n (x - a)^n\bigg) \; dx                         \\
                    & = \sum_{n = 0}^\infty \bigg(\int_{[y, z]} c_n (x - a)^n \; dx\bigg)         &  & \by{3.6.2} \\
                    & = \sum_{n = 0}^\infty c_n \dfrac{(z - a)^{n + 1} - (y - a)^{n + 1}}{n + 1}.
  \end{align*}
\end{proof}

\begin{note}
  \cref{4.1.6} (a) and (b) of the above theorem give another way to find the radius of convergence, by using your favorite convergence test to work out the range of \(x\) for which the power series converges.
\end{note}

\setcounter{thm}{7}
\begin{rmk}\label{4.1.8}
  \cref{4.1.6} is silent on what happens when \(\abs{x - a} = R\), i.e., at the points \(a - R\) and \(a + R\).
  Indeed, one can have either convergence or divergence at those points.
\end{rmk}

\begin{rmk}\label{4.1.9}
  Note that while \cref{4.1.6} assures us that the power series \(\sum_{n = 0}^\infty c_n (x - a)^n\) will converge pointwise on the interval \((a - R, a + R)\), it need not converge uniformly on that interval
  (see \cref{ex:4.1.2}(e)).
  On the other hand, \cref{4.1.6}(c) assures us that the power series will converge uniformly on any smaller interval \([a - r, a + r]\).
  In particular, being uniformly convergent on every closed subinterval of \((a - R, a + R)\) is not enough to guarantee being uniformly convergent on all of \((a - R, a + R)\).
\end{rmk}

\exercisesection

\begin{ex}\label{ex:4.1.1}
  Prove \cref{4.1.6}.
\end{ex}

\begin{proof}
  See \cref{4.1.6}.
\end{proof}

\begin{ex}\label{ex:4.1.2}
  Give examples of a formal power series \(\sum_{n = 0}^\infty c_n x^n\) centered at \(0\) with radius of convergence \(1\), which
  \begin{enumerate}
    \item diverges at both \(x = 1\) and \(x = -1\);
    \item diverges at \(x = 1\) but converges at \(x = -1\);
    \item converges at \(x = 1\) but diverges at \(x = -1\);
    \item converges at both \(x = 1\) and \(x = -1\).
    \item converges pointwise on \((-1, 1)\), but does not converge uniformly on \((-1, 1)\).
  \end{enumerate}
\end{ex}

\begin{proof}{(a)}
  Let \(c_n = 1\) for all \(n \in \N\).
  Then we have
  \[
    \dfrac{1}{\limsup_{n \to \infty} \abs{c_n}} = \dfrac{1}{\limsup_{n \to \infty} 1} = \dfrac{1}{1} = 1.
  \]
  But we know that \(\sum_{n = 0}^\infty 1^n\) and \(\sum_{n = 0}^\infty (-1)^n\) are divergent.
\end{proof}

\begin{proof}{(b)}
  Let \(c_n = \dfrac{1}{n + 1}\) for all \(n \in \N\).
  Then we have
  \[
    \dfrac{1}{\limsup_{n \to \infty} \abs{c_n}} = \dfrac{1}{\limsup_{n \to \infty} \dfrac{1}{n + 1}} = \dfrac{1}{1} = 1.
  \]
  By Corollary 7.3.7 in Analysis I we know that \(\sum_{n = 0}^\infty \dfrac{1}{n + 1}\) diverges.
  By Corollary 7.2.12 in Analysis I we know that \(\sum_{n = 0}^\infty \dfrac{(-1)^n}{n + 1}\) converges.
\end{proof}

\begin{proof}{(c)}
  Let \(c_n = \dfrac{(-1)^n}{n + 1}\) for all \(n \in \N\).
  Then we have
  \[
    \dfrac{1}{\limsup_{n \to \infty} \abs{c_n}} = \dfrac{1}{\limsup_{n \to \infty} \dfrac{1}{n + 1}} = \dfrac{1}{1} = 1.
  \]
  By Corollary 7.2.12 in Analysis I we know that \(\sum_{n = 0}^\infty \dfrac{(-1)^n 1^n}{n + 1}\) converges.
  By Corollary 7.3.7 in Analysis I we know that \(\sum_{n = 0}^\infty \dfrac{(-1)^{2n}}{n + 1}\) diverges.
\end{proof}

\begin{proof}{(d)}
  Let \(c_n = \dfrac{1}{n^2 - 1 / 2}\) for all \(n \in \N\).
  Then we have
  \[
    \dfrac{1}{\limsup_{n \to \infty} \abs{c_n}} = \dfrac{1}{\limsup_{n \to \infty} \dfrac{1}{n^2 - 1 / 2}} = \dfrac{1}{1} = 1.
  \]
  By Corollary 7.3.7 in Analysis I we know that \(\sum_{n = 0}^\infty \dfrac{1}{n^2 - 1 / 2}\) converges.
  By Corollary 7.2.12 in Analysis I we know that \(\sum_{n = 0}^\infty \dfrac{(-1)^n}{n^2 - 1 / 2}\) converges.
\end{proof}

\begin{proof}{(e)}
  Let \(c_n = 1\) for all \(n \in \N\).
  Then we have
  \[
    \dfrac{1}{\limsup_{n \to \infty} \abs{c_n}} = \dfrac{1}{\limsup_{n \to \infty} 1} = \dfrac{1}{1} = 1.
  \]
  By Lemma 7.3.3 in Analysis I we know that \(\sum_{n = 0}^\infty x^n\) converges for all \(x \in (-1, 1)\).
  But by \cref{3.5.8} we know that \(\sum_{n = 0}^\infty x^n\) does not converge uniformly on \((-1, 1)\).
\end{proof}
\section{Real analytic functions}\label{sec:4.2}

\begin{defn}[Real analytic functions]\label{4.2.1}
  Let \(E\) be a subset of \(\R\), and let \(f : E \to \R\) be a function.
  If \(a\) is an interior point of \(E\), we say that \(f\) is \emph{real analytic at \(a\)} if there exists an open interval \((a - r, a + r)\) in \(E\) for some \(r > 0\) such that there exists a power series \(\sum_{n = 0}^\infty c_n (x - a)^n\) centered at \(a\) which has a radius of convergence greater than or equal to \(r\), and which converges to \(f\) on \((a - r, a + r)\).
  If \(E\) is an open set, and \(f\) is real analytic at every point \(a\) of \(E\), we say that \(f\) is \emph{real analytic on \(E\)}.
\end{defn}

\begin{eg}\label{4.2.2}
  Consider the function \(f : \R \setminus \set{1} \to \R\) defined by \(f(x) \coloneqq \dfrac{1}{1 - x}\).
  This function is real analytic at \(0\) because we have a power series \(\sum_{n = 0}^\infty x^n\) centred at \(0\) which converges to \(\dfrac{1}{1 - x} = f(x)\) on the interval \((-1, 1)\).
  This function is also real analytic at \(2\) because we have a power series \(\sum_{n = 0}^\infty (-1)^{n + 1} (x - 2)^n\) which converges to \(\dfrac{-1}{1 - \big(-(x - 2)\big)} = \dfrac{1}{1 - x} = f(x)\) on the interval \((1, 3)\)
  (why? use Lemma 7.3.3 in Analysis I).
  In fact this function is real analytic on all of \(\R \setminus \set{1}\);
  see \cref{ex:4.2.2}.
\end{eg}

\begin{rmk}\label{4.2.3}
  The notion of being real analytic is closely related to another notion, that of being \emph{complex analytic}, but this is a topic for complex analysis, and will not be discussed here.
\end{rmk}

\begin{note}
  From \cref{4.1.6}(c) and (d) we see that if \(f\) is real analytic at a point \(a\), then \(f\) is both continuous and differentiable on \((a - r, a + r)\) for some \(r > 0\).
\end{note}

\begin{defn}[\(k\)-times differentiability]\label{4.2.4}
  Let \(E\) be a subset of \(\R\) with the property that every element of \(E\) is a limit point of \(E\).
  We say a function \(f : E \to \R\) is \emph{once differentiable on \(E\)} iff it is differentiable, in particular \(f': E \to \R\) is also a function on \(E\).
  More generally, for any \(k \geq 2\) we say that \(f : E \to \R\) is \emph{\(k\) times differentiable on \(E\)}, or just \emph{\(k\) times differentiable}, iff \(f\) is differentiable, and \(f'\) is \(k - 1\) times differentiable.
  If \(f\) is \(k\) times differentiable, we define the \(k^{\text{th}}\) derivative \(f^{(k)} : E \to \R\) by the recursive rule \(f^{(1)} \coloneqq f'\), and \(f^{(k)} = (f^{(k - 1)})'\) for all \(k \geq 2\).
  We also define \(f^{(0)} \coloneqq f\) (this is \(f\) differentiated \(0\) times), and we allow every function to be zero times differentiable (since clearly \(f^{(0)}\) exists for every \(f\)).
  A function is said to be \emph{infinitely differentiable} (or \emph{smooth}) iff it is \(k\) times differentiable for every \(k \geq 0\).
\end{defn}

\begin{ac}\label{ac:4.2.1}
  For each \(k \in \N\), we have
  \[
    \lim_{n \to \infty} \bigg(\dfrac{(n + k)!}{n!}\bigg)^{\dfrac{1}{n}} = 1.
  \]
\end{ac}

\begin{proof}
  We use induction on \(k\).
  For \(k = 0\), we have
  \[
    \lim_{n \to \infty} \bigg(\dfrac{(n + 0)!}{n!}\bigg)^{\dfrac{1}{n}} = \lim_{n \to \infty} 1^{\dfrac{1}{n}} = \lim_{n \to \infty} 1 = 1.
  \]
  Thus the base case holds.
  Suppose inductively that
  \[
    \lim_{n \to \infty} \bigg(\dfrac{(n + k)!}{n!}\bigg)^{\dfrac{1}{n}} = 1
  \]
  for some \(k \geq 0\).
  We want to show that \(k + 1\) is also true.
  Observe that
  \begin{align*}
             & \exists N \in \Z^+ : \forall n \geq N, Nn > k + 1                                                           \\
    \implies & \exists N \in \Z^+ : \forall n \geq N, (N + 1)n > n + k + 1                                                 \\
    \implies & \exists N \in \Z^+ : \forall n \geq N, Nn > n + k + 1 > n                                                   \\
    \implies & \exists N \in \Z^+ : \forall n \geq N, (Nn)^{\dfrac{1}{n}} > (n + k + 1)^{\dfrac{1}{n}} > n^{\dfrac{1}{n}}.
  \end{align*}
  Now we fix such \(N\).
  Since
  \begin{align*}
             & \begin{dcases}
                 \lim_{n \to \infty} N^{\dfrac{1}{n}} = 1 \\ %&  & \text{(by Lemma 6.5.3 in Analysis I)} \\
                 \lim_{n \to \infty} n^{\dfrac{1}{n}} = 1 %&  & \text{(by Lemma 7.5.4 in Analysis I)}
               \end{dcases}                                \\
    \implies & \lim_{n \to \infty} (Nn)^{\dfrac{1}{n}} = 1                                                                          \\
    \implies & \lim_{n \to \infty} (n + k + 1)^{\dfrac{1}{n}} = 1,                                    &  & \text{(by squeeze test)}
  \end{align*}
  we have
  \begin{align*}
             & \lim_{n \to \infty} \bigg(\dfrac{(n + k)!}{n!}\bigg)^{\dfrac{1}{n}} = 1                                                                        &  & \byIH \\
    \implies & \Bigg(\lim_{n \to \infty} \bigg(\dfrac{(n + k)!}{n!}\bigg)^{\dfrac{1}{n}}\Bigg) \bigg(\lim_{n \to \infty} (n + k + 1)^{\dfrac{1}{n}}\bigg) = 1            \\
    \implies & \lim_{n \to \infty} \Bigg(\bigg(\dfrac{(n + k)!}{n!}\bigg)^{\dfrac{1}{n}} (n + k + 1)^{\dfrac{1}{n}}\Bigg) = 1                                            \\
    \implies & \lim_{n \to \infty} \bigg(\dfrac{(n + k + 1)!}{n!}\bigg)^{\dfrac{1}{n}} = 1.
  \end{align*}
  This closes the induction.
\end{proof}

\setcounter{thm}{5}
\begin{prop}[Real analytic functions are \(k\)-times differentiable]\label{4.2.6}
  Let \(E\) be a subset of \(\R\), let \(a\) be an interior point of \(E\), and and let \(f\) be a function which is real analytic at \(a\), thus there is an \(r > 0\) for which we have the power series expansion
  \[
    f(x) = \sum_{n = 0}^\infty c_n (x - a)^n
  \]
  for all \(x \in (a - r, a + r)\).
  Then for every \(k \geq 0\), the function \(f\) is \(k\)-times differentiable on \((a - r, a + r)\), and for each \(k \geq 0\) the \(k^{\text{th}}\) derivative is given by
  \begin{align*}
    f^{(k)}(x) & = \sum_{n = 0}^\infty c_{n + k} (n + 1) (n + 2) \dots (n + k) (x - a)^n \\
               & = \sum_{n = 0}^\infty c_{n + k} \dfrac{(n + k)!}{n!} (x - a)^n
  \end{align*}
  for all \(x \in (a - r, a + r)\).
\end{prop}

\begin{proof}
  Let \(R\) be the radius of convergence of \(f\).
  By \cref{4.2.1} we know that \(r \leq R\) and thus
  \[
    \forall x \in (a - r, a + r), \abs{x - a} < r \leq R.
  \]
  We use induction on \(k\).
  For \(k = 0\), by \cref{4.2.4} we have
  \[
    \forall x \in (a - r, a + r), f^{(0)}(x) = f(x) = \sum_{n = 0}^\infty c_n (x - a)^n = \sum_{n = 0}^\infty c_{n + 0} \dfrac{(n + 0)!}{n!} (x - a)^n.
  \]
  Thus the base case holds.
  Suppose inductively that
  \[
    \forall x \in (a - r, a + r), f^{(k)}(x) = \sum_{n = 0}^\infty c_{n + k} \dfrac{(n + k)!}{n!} (x - a)^n
  \]
  for some \(k \geq 0\).
  By \cref{ac:4.2.1} we know that
  \begin{align*}
             & \limsup_{n \to \infty} \abs{c_{n + k}}^{\dfrac{1}{n}} = \limsup_{n \to \infty} \abs{c_n}^{\dfrac{1}{n}} = \dfrac{1}{R}                                              \\
    \implies & \Bigg(\limsup_{n \to \infty} \abs{c_{n + k}}^{\dfrac{1}{n}}\Bigg) \Bigg(\limsup_{n \to \infty} \bigg(\dfrac{(n + k)!}{n!}\bigg)^{\dfrac{1}{n}}\Bigg) = \dfrac{1}{R} \\
    \implies & \limsup_{n \to \infty} \abs{c_{n + k} \dfrac{(n + k)!}{n!}}^{\dfrac{1}{n}} = \dfrac{1}{R}.
  \end{align*}
  Thus by \cref{4.1.6}(b) we know that \(f^{(k)}(x)\) converges for all \(x \in (a - r, a + r)\).
  Now we define
  \[
    \forall n \in \N, b_n = c_{n + k} \dfrac{(n + k)!}{n!}.
  \]
  Then by \cref{4.1.6}(d) we have
  \begin{align*}
             & \forall x \in (a - r, a + r), f^{(k)}(x) = \sum_{n = 0}^\infty c_{n + k} \dfrac{(n + k)!}{n!} (x - a)^n = \sum_{n = 0}^\infty b_n (x - a)^n                     \\
    \implies & \forall x \in (a - r, a + r), f^{(k + 1)}(x) = (f^{(k)})'(x) = \sum_{n = 1}^\infty n b_n (x - a)^{n - 1}                                                        \\
    \implies & \forall x \in (a - r, a + r), f^{(k + 1)}(x) = \sum_{n = 1}^\infty n c_{n + k} \dfrac{(n + k)!}{n!} (x - a)^{n - 1}                                             \\
    \implies & \forall x \in (a - r, a + r), f^{(k + 1)}(x) = \sum_{n = 1}^\infty c_{n + k} \dfrac{(n + k)!}{(n - 1)!} (x - a)^{n - 1}                                         \\
    \implies & \forall x \in (a - r, a + r), f^{(k + 1)}(x) = \sum_{n = 0}^\infty c_{(n + 1) + k} \dfrac{\big((n + 1) + k\big)!}{\big((n + 1) - 1\big)!} (x - a)^{(n + 1) - 1} \\
    \implies & \forall x \in (a - r, a + r), f^{(k + 1)}(x) = \sum_{n = 0}^\infty c_{n + k + 1} \dfrac{(n + k + 1)!}{n!} (x - a)^n
  \end{align*}
  and this closes the induction.
\end{proof}

\begin{cor}[Real analytic functions are infinitely differentiable]\label{4.2.7}
  Let \(E\) be an open subset of \(\R\), and let \(f : E \to \R\) be a real analytic function on \(E\).
  Then \(f\) is infinitely differentiable on \(E\).
  Also, all derivatives of \(f\) are also real analytic on \(E\).
\end{cor}

\begin{proof}
  For every point \(a \in E\) and \(k \geq 0\), we know from \cref{4.2.6} that \(f\) is \(k\)-times differentiable at \(a\)
  (we will have to apply Exercise 10.1.1 in Analysis I \(k\) times here).
  Thus \(f\) is \(k\)-times differentiable on \(E\) for every \(k \geq 0\) and is hence infinitely differentiable.
  Also, from \cref{4.2.6} we see that each derivative \(f^{(k)}\) of \(f\) has a convergent power series expansion at every \(x \in E\) and thus \(f^{(k)}\) is real analytic.
\end{proof}

\setcounter{thm}{8}
\begin{rmk}\label{4.2.9}
  The converse statement to \cref{4.2.7} is not true;
  there are infinitely differentiable functions which are not real analytic.
\end{rmk}

\begin{note}
  \cref{4.2.6} has an important corollary (\cref{4.2.10}), due to Brook Taylor (1685--1731).
\end{note}

\begin{cor}[Taylor's formula]\label{4.2.10}
  Let \(E\) be a subset of \(\R\), let \(a\) be an interior point of \(E\), and let \(f : E \to \R\) be a function which is real analytic at \(a\) and has the power series expansion
  \[
    f(x) = \sum_{n = 0}^\infty c_n (x - a)^n
  \]
  for all \(x \in (a - r, a + r)\) and some \(r > 0\).
  Then for any integer \(k \geq 0\), we have
  \[
    f^{(k)}(a) = k! c_k,
  \]
  where \(k! \coloneqq 1 \times 2 \times \dots \times k\)
  (and we adopt the convention that \(0! = 1\)).
  In particular, we have Taylor's formula
  \[
    f(x) = \sum_{n = 0}^\infty \dfrac{f^{(n)}(a)}{n!} (x - a)^n
  \]
  for all \(x\) in \((a - r, a + r)\).
\end{cor}

\begin{proof}
  We have
  \begin{align*}
    \forall k \in \N, f^{(k)}(a) & = \sum_{n = 0}^\infty c_{n + k} \dfrac{(n + k)!}{n!} (a - a)^n &           & \by{4.2.6} \\
                                 & = c_k k!                                                       & (0^0 = 1)
  \end{align*}
  and thus
  \begin{align*}
    \forall x \in (a - r, a + r), f(x) & = \sum_{n = 0}^\infty c_n (x - a)^n                                                        \\
                                       & = \sum_{n = 0}^\infty \dfrac{c_n n!}{n!} (x - a)^n                                         \\
                                       & = \sum_{n = 0}^\infty \dfrac{f^{(n)}(a)}{n!} (x - a)^n. &  & \text{(from the proof above)}
  \end{align*}
\end{proof}

\begin{note}
  The power series \(\sum_{n = 0}^\infty \dfrac{f^{(n)}(a)}{n!} (x - a)^n\) is sometimes called the \emph{Taylor series} of \(f\) around \(a\).
  Taylor's formula thus asserts that if a function is real analytic, then it is equal to its Taylor series.
\end{note}

\begin{rmk}\label{4.2.11}
  Note that Taylor's formula only works for functions which are real analytic;
  there are examples of functions which are infinitely differentiable but for which Taylor's theorem fails.
\end{rmk}

\begin{cor}[Uniqueness of power series]\label{4.2.12}
  Let \(E\) be a subset of \(\R\), let \(a\) be an interior point of \(E\), and let \(f : E \to \R\) be a function which is real analytic at \(a\).
  Suppose that \(f\) has two power series expansions
  \[
    f(x) = \sum_{n = 0}^\infty c_n (x - a)^n
  \]
  and
  \[
    f(x) = \sum_{n = 0}^\infty d_n (x - a)^n
  \]
  centered at \(a\), each with a non-zero radius of convergence.
  Then \(c_n = d_n\) for all \(n \geq 0\).
\end{cor}

\begin{proof}
  By \cref{4.2.10}, we have \(f^{(k)}(a) = k! c_k\) for all \(k \geq 0\).
  But we also have \(f^{(k)}(a) = k! d_k\), by similar reasoning.
  Since \(k!\) is never zero, we can cancel it and obtain \(c_k = d_k\) for all \(k \geq 0\), as desired.
\end{proof}

\begin{rmk}\label{4.2.13}
  While a real analytic function has a unique power series around any given point, it can certainly have different power series at different points.
  For instance, the function \(f(x) \coloneqq \dfrac{1}{1 - x}\), defined on \(\R \setminus \set{1}\), has the power series
  \[
    f(x) \coloneqq \sum_{n = 0}^\infty x^n
  \]
  around \(0\), on the interval \((-1, 1)\), but also has the power series
  \begin{align*}
    f(x) & = \dfrac{1}{1 - x}                                                  \\
         & = \dfrac{2}{1 - 2(x - \dfrac{1}{2})}                                \\
         & = \sum_{n = 0}^\infty 2 \bigg(2\bigg(x - \dfrac{1}{2}\bigg)\bigg)^n \\
         & = \sum_{n = 0}^\infty 2^{n + 1} \bigg(x - \dfrac{1}{2}\bigg)^n
  \end{align*}
  around \(1 / 2\), on the interval \((0, 1)\)
  (note that the above power series has a radius of convergence of \(1 / 2\), thanks to the root test).
\end{rmk}

\exercisesection

\begin{ex}\label{ex:4.2.1}
  Let \(n \geq 0\) be an integer, let \(c, a\) be real numbers, and let \(f\) be the function \(f(x) \coloneqq c (x - a)^n\).
  Show that \(f\) is infinitely differentiable, and that \(f^{(k)}(x) = c \dfrac{n!}{(n - k)!} (x - a)^{n - k}\) for all integers \(0 \leq k \leq n\).
  What happens when \(k > n\)?
\end{ex}

\begin{proof}
  For each \(n \in \N\), let \(P(n)\) be the statement ``If \(f(x) = c (x - a)^n\), then \(f^{(k)}(x) = c \dfrac{n!}{(n - k)!} (x - a)^{n - k}\) for all \(0 \leq k \leq n\)''.
  We use induction on \(n\) to show that \(P(n)\) is true for all \(n \in \N\).
  For \(n = 0\), we have \(0 \leq k \leq 0 \implies k = 0\).
  By \cref{4.2.4} we have
  \[
    f^{(0)} = f = c (x - a)^0 = c = c \dfrac{0!}{(0 - 0)!} (x - a)^{0 - 0}
  \]
  and thus the base case holds.
  Suppose inductive that \(P(n)\) is true for some \(n \geq 0\).
  Then we want to show that \(P(n + 1)\) is true.
  Let \(f(x) = c (x - a)^{n + 1}\).
  Then we have
  \[
    f'(x) = c (n + 1) (x - a)^n.
  \]
  By induction hypothesis we know that
  \begin{align*}
             & \dfrac{f'(x)}{n + 1} = c (x - a)^n                                                                                       \\
    \implies & \forall 0 \leq k \leq n, \bigg(\dfrac{f'(x)}{n + 1}\bigg)^{(k)} = c \dfrac{n!}{(n - k)!} (x - a)^{n - k}                 \\
    \implies & \forall 0 \leq k \leq n, \big(f'(x)\big)^{(k)} = c \dfrac{(n + 1)!}{(n - k)!} (x - a)^{n - k}                            \\
    \implies & \forall 0 \leq k \leq n, f(x)^{(k + 1)} = c \dfrac{(n + 1)!}{(n - k)!} (x - a)^{n - k}                   &  & \by{4.2.4} \\
    \implies & \forall 1 \leq k \leq n + 1,                                                                                             \\
             & f(x)^{\big((k - 1) + 1\big)} = c \dfrac{(n + 1)!}{\big(n - (k - 1)\big)!} (x - a)^{n - (k - 1)}                          \\
    \implies & \forall 1 \leq k \leq n + 1, f(x)^{(k)} = c \dfrac{(n + 1)!}{(n + 1 - k)!} (x - a)^{n + 1 - k}
  \end{align*}
  and we know that
  \[
    f(x)^{(0)} = f(x) = c (x - a)^{n + 1} = c \dfrac{(n + 1)!}{(n + 1 - 0)!} (x - a)^{n + 1 - 0}.
  \]
  Thus we have
  \[
    \forall 0 \leq k \leq n + 1, f(x)^{(k)} = c \dfrac{(n + 1)!}{(n + 1 - k)!} (x - a)^{n + 1 - k}
  \]
  and this closes the induction.

  Now let \(n \in \N\) and let \(f(x) = c (x - a)^n\).
  From the proof above we know that
  \[
    f^{(n)}(x) = c \dfrac{n!}{(n - n)!} (x - a)^{n - n} = c n!
  \]
  is a constant function.
  Thus we have
  \[
    \forall k > n, f^{(k)}(x) = 0
  \]
  and by \cref{4.2.4} \(f\) is infinitely differentiable.
\end{proof}

\begin{ex}\label{ex:4.2.2}
  Show that the function \(f\) defined in \cref{4.2.2} is real analytic on all of \(\R \setminus \set{1}\).
\end{ex}

\begin{proof}
  Let \(a \in \R \setminus \set{1}\), let \(r = \abs{1 - a}\), let \(x \in (a - r, a + r)\) and let \(c_n = (\dfrac{1}{1 - a})^{n + 1}\) for all \(n \in \N\).
  Then we have
  \begin{align*}
             & a - r < x < a + r              \\
    \implies & \abs{x - a} < r                \\
    \implies & \abs{\dfrac{x - a}{1 - a}} < 1
  \end{align*}
  and by Lemma 7.3.3 in Analysis I we know that
  \begin{align*}
    \sum_{n = 0}^\infty c_n (x - a)^n & = \sum_{n = 0}^\infty \bigg(\dfrac{1}{1 - a}\bigg)^{n + 1} (x - a)^n      \\
                                      & = \dfrac{1}{1 - a} \sum_{n = 0}^\infty \bigg(\dfrac{x - a}{1 - a}\bigg)^n \\
                                      & = \dfrac{1}{1 - a} \dfrac{1}{1 - \dfrac{x - a}{1 - a}}                    \\
                                      & = \dfrac{1}{1 - x}.
  \end{align*}
  Since \(x\) is arbitrary, we know that \(\sum_{n = 0}^\infty c_n (x - a)^n\) converges to \(f\) on \((a - r, a + r)\).
  By \cref{4.2.1} we know that \(f\) is real analytic at \(a\).
  Since \(a\) is arbitrary, by \cref{4.2.1} we know that \(f\) is real analysis at \(a\) for each \(a \in \R \setminus \set{1}\).
\end{proof}

\begin{ex}\label{ex:4.2.3}
  Prove \cref{4.2.6}.
\end{ex}

\begin{proof}
  See \cref{4.2.6}.
\end{proof}

\begin{ex}\label{ex:4.2.4}
  Use \cref{4.2.6} and \cref{ex:4.2.1} to prove \cref{4.2.10}.
\end{ex}

\begin{proof}
  See \cref{4.2.10}.
\end{proof}

\begin{ex}\label{ex:4.2.5}
  Let \(a, b\) be real numbers, and let \(n \geq 0\) be an integer.
  Prove the identity
  \[
    (x - a)^n = \sum_{m = 0}^n \dfrac{n!}{m! (n - m)!} (b - a)^{n - m} (x - b)^m
  \]
  or any real number \(x\).
  Explain why this identity is consistent with Taylor's theorem and \cref{ex:4.2.1}.
  (Note however that Taylor's theorem cannot be rigorously applied until one verifies \cref{ex:4.2.6} below.)
\end{ex}

\begin{proof}
  Let \(f : \R \to \R\) be the function by setting \(f(x) = (x - a)^n\) for all \(x \in \R\).
  By Exercise 7.1.4 in Analysis I we have
  \begin{align*}
    \forall x \in \R, f(x) & = (x - a)^n                                                         \\
                           & = (x - b + b - a)^n                                                 \\
                           & = \sum_{m = 0}^n \dfrac{n!}{m! (n - m)!} (x - b)^m (b - a)^{n - m}.
  \end{align*}
  If we define
  \[
    \forall m \in \N, c_m = \begin{dcases}
      \dfrac{n! (b - a)^{n - m}}{m! (n - m)!} & \text{if } m \leq n \\
      0                                       & \text{if } m > n
    \end{dcases}
  \]
  then we have
  \[
    \forall x \in \R, f(x) = \sum_{m = 0}^\infty c_m (x - b)^m = \sum_{m = 0}^n c_m (x - b)^m.
  \]
  Thus for arbitrary \(r \in \R^+\), \(\sum_{m = 0}^\infty c_m (x - b)^m\) converges to \(f(x)\) for all \(x \in (b - r, b + r)\).
  By \cref{4.2.1} we know that \(f\) is real analytic at \(b\).
  By \cref{4.2.10} we have
  \begin{align*}
    \forall x \in \R, f(x) = \sum_{m = 0}^\infty \dfrac{f^{(m)}(b)}{m!} (x - b)^m.
  \end{align*}
  By \cref{ex:4.2.1} we have
  \[
    \forall m \in \N, \forall x \in \R, f^{(m)}(x) = \begin{dcases}
      \dfrac{n!}{(n - m)!} (x - a)^{n - m} & \text{if } 0 \leq m \leq n \\
      0                                    & \text{if } m > n
    \end{dcases}
  \]
  Thus
  \begin{align*}
    \forall x \in \R, f(x) & = \sum_{m = 0}^\infty \dfrac{f^{(m)}(b)}{m!} (x - b)^m              \\
                           & = \sum_{m = 0}^n \dfrac{f^{(m)}(b)}{m!} (x - b)^m                   \\
                           & = \sum_{m = 0}^n \dfrac{n!}{m! (n - m)!} (x - b)^m (b - a)^{n - m}.
  \end{align*}
\end{proof}

\begin{ex}\label{ex:4.2.6}
  Using \cref{ex:4.2.5}, show that every polynomial \(P(x)\) of one variable is real analytic on \(\R\).
\end{ex}

\begin{proof}
  Let \(P : \R \to \R\) be a polynomial with degree \(n\).
  First we show that \(P\) is real analytic at \(0\).
  By \cref{3.8.1} we know that
  \[
    \forall x \in \R, P(x) = \sum_{i = 0}^n c_i x^i
  \]
  where \(c_0, \dots, c_n \in \R\) and \(c_n \neq 0\).
  If we define \(c_i = 0\) for all \(i > n\), then we have
  \[
    \forall x \in \R, \sum_{i = 0}^\infty c_i (x - 0)^i = \sum_{i = 0}^\infty c_i x^i = \sum_{i = 0}^n c_i x^i = P(x).
  \]
  Thus for arbitrary \(r \in \R^+\), \(\sum_{i = 0}^\infty c_i (x - 0)^i\) converges to \(P(x)\) for all \(x \in (-r, r)\).
  By \cref{4.2.1} \(P\) is real analytic at \(0\).

  Now we show that \(P\) is real analytic at \(a \in \R \setminus \set{0}\).
  Since \(a \neq 0\), we have
  \begin{align*}
    \forall x \in \R, P(x) & = \sum_{i = 0}^n c_i x^i                                                                                                       \\
                           & = \sum_{i = 0}^n c_i \bigg(\sum_{m = 0}^i \dfrac{i!}{m! (i - m)!} a^{i - m} (x - a)^m\bigg)                 &  & \by{ex:4.2.5} \\
                           & = \sum_{i = 0}^n c_i \bigg(\sum_{m = 0}^i \dfrac{i!}{m! (i - m)!} a^{i - m} (x - a)^{m - i}\bigg) (x - a)^i
  \end{align*}
  If we define
  \[
    \forall i \in \N, d_i = \begin{dcases}
      c_i \bigg(\sum_{m = 0}^i \dfrac{i!}{m! (i - m)!} a^{i - m} (x - a)^{m - i}\bigg) & \text{if } 0 \leq i \leq n \\
      0                                                                                & \text{if } i > n
    \end{dcases}
  \]
  Then we have
  \[
    \forall x \in \R, \sum_{i = 0}^\infty d_i (x - a)^i = \sum_{i = 0}^n d_i (x - a)^i = P(x).
  \]
  Thus for arbitrary \(r \in \R^+\), \(\sum_{i = 0}^\infty d_i (x - a)^i\) converges to \(P(x)\) for all \(x \in (-r, r)\).
  By \cref{4.2.1} \(P\) is real analytic at \(a\).
  Combine the proof above we conclude that \(P\) is real analytic on \(\R\).
  Since \(P\) is arbitrary, we conclude that polynomials of one variable are real analytic on \(\R\).
\end{proof}

\begin{ex}\label{ex:4.2.7}
  Let \(m \geq 0\) be a positive integer, and let \(0 < x < r\) be real numbers.
  Use Lemma 7.3.3 in Analysis I to establish the identity
  \[
    \dfrac{r}{r - x} = \sum_{n = 0}^\infty x^n r^{-n}
  \]
  for all \(x \in (-r, r)\).
  Using \cref{4.2.6}, conclude the identity
  \[
    \dfrac{r}{(r - x)^{m + 1}} = \sum_{n = m}^\infty \dfrac{n!}{m! (n - m)!} x^{n - m} r^{-n}
  \]
  for all integers \(m \geq 0\) and \(x \in (-r, r)\).
  Also explain why the series on the right-hand side is absolutely convergent.
\end{ex}

\begin{proof}
  By Lemma 7.3.3 in Analysis I we have
  \begin{align*}
             & 0 < x < r                                                                                                                         \\
    \implies & \dfrac{x}{r} < 1                                                                                                                  \\
    \implies & \sum_{n = 0}^\infty \bigg(\dfrac{x}{r}\bigg)^n = \sum_{n = 0}^\infty x^n r^{-n} = \dfrac{1}{1 - \dfrac{x}{r}} = \dfrac{r}{r - x}.
  \end{align*}
  Since
  \[
    x = 0 \implies \sum_{n = 0}^\infty 0^n r^{-n} = 0^0 r^{0} + \sum_{n = 1}^\infty 0^n r^{-n} = 1 = \dfrac{r}{r - 0},
  \]
  we know that
  \[
    \forall x \in (-r, r), \dfrac{r}{r - x} = \sum_{n = 0}^\infty x^n r^{-n}
  \]
  and by \cref{4.2.1} \(x \mapsto \dfrac{r}{r - x}\) is real analytic at \(0\).

  Next we use induction on \(m\) to show that
  \[
    \forall m \in \N, \forall x \in (-r, r), \bigg(y \mapsto \dfrac{r}{r - y}\bigg)^{(m)}(x) = \dfrac{m! r}{(r - x)^{m + 1}}.
  \]
  For \(m = 0\), we have
  \[
    \forall x \in (-r, r), \bigg(y \mapsto \dfrac{r}{r - y}\bigg)^{(0)}(x) = \dfrac{r}{r - x} = \dfrac{0! r}{(r - x)^{0 + 1}}
  \]
  and thus the base case holds.
  Suppose inductively that
  \[
    \forall x \in (-r, r), \bigg(y \mapsto \dfrac{r}{r - y}\bigg)^{(m)}(x) = \dfrac{m! r}{(r - x)^{m + 1}}
  \]
  for some \(m \geq 0\).
  Then by \cref{4.2.4} we have
  \[
    \forall x \in (-r, r), \bigg(y \mapsto \dfrac{r}{r - y}\bigg)^{(m + 1)}(x) = \bigg(\dfrac{m! r}{(r - x)^{m + 1}}\bigg)'(x) = \dfrac{(m + 1)! r}{(r - x)^{m + 2}}.
  \]
  This closes the induction.

  Now we show that
  \[
    \forall m \in \N, \dfrac{r}{(r - x)^{m + 1}} = \sum_{n = m}^\infty \dfrac{n!}{m! (n - m)!} x^{n - m} r^{-n}.
  \]
  From the proof above we know that
  \[
    \forall m \in \N, \forall x \in (-r, r), \dfrac{r}{(r - x)^{m + 1}} = \dfrac{1}{m!} \bigg(y \mapsto \dfrac{r}{r - y}\bigg)^{(m)}.
  \]
  By \cref{4.2.6} we know that
  \begin{align*}
    \forall m \in \N, \forall x \in (-r, r), & \bigg(y \mapsto \dfrac{r}{r - y}\bigg)^{(m)}                       \\
                                             & = \sum_{n = 0}^\infty r^{-(n + m)} \dfrac{(n + m)!}{n!} x^n        \\
                                             & = \sum_{n = m}^\infty r^{-n} \dfrac{n!}{(n - m)!} x^{n - m}        \\
                                             & = m! \sum_{n = m}^\infty r^{-n} \dfrac{n!}{m! (n - m)!} x^{n - m}.
  \end{align*}
  Thus we have
  \[
    \forall m \in \N, \forall x \in (-r, r), \dfrac{r}{(r - x)^{m + 1}} = \sum_{n = m}^\infty r^{-n} \dfrac{n!}{m! (n - m)!} x^{n - m}.
  \]

  Since
  \begin{align*}
             & \forall m \in \N, \lim_{n \to \infty} \bigg(\dfrac{(n + m)!}{n!}\bigg)^{\dfrac{1}{n}} = 1        &  & \by{ac:4.2.1} \\
    \implies & \forall m \in \N, \lim_{n \to \infty} \bigg(\dfrac{n!}{(n - m)!}\bigg)^{\dfrac{1}{n - m}} = 1                       \\
    \implies & \forall m \in \N, \lim_{n \to \infty} \bigg(\dfrac{n!}{(n - m)!}\bigg)^{\dfrac{1}{n}} = 1                           \\
    \implies & \forall m \in \N, \lim_{n \to \infty} \bigg(\dfrac{n!}{m! (n - m)!}\bigg)^{\dfrac{1}{n}} \leq 1,
  \end{align*}
  we have
  \begin{align*}
             & \forall m \in \N, \limsup_{n \to \infty} \abs{\bigg(\dfrac{n!}{m! (n - m)!}\bigg)^{\dfrac{1}{n}}} \leq 1                          \\
    \implies & \forall m \in \N, \limsup_{n \to \infty} \abs{\bigg(\dfrac{n!}{m! (n - m)!}\bigg)^{\dfrac{1}{n}} \dfrac{x}{r}} < 1                \\
    \implies & \forall m \in \N, \limsup_{n \to \infty} \abs{\bigg(\dfrac{n!}{m! (n - m)!}\bigg)^{\dfrac{1}{n}} (x^n r^{-n})^{\dfrac{1}{n}}} < 1
  \end{align*}
  and by root test
  \[
    \sum_{n = m}^\infty \dfrac{n!}{m! (n - m)!} x^n r^{-n}
  \]
  is absolutely converges, and so does
  \[
    x^{-m} \sum_{n = m}^\infty \dfrac{n!}{m! (n - m)!} x^n r^{-n} = \sum_{n = m}^\infty \dfrac{n!}{m! (n - m)!} x^{n - m} r^{-n}.
  \]
\end{proof}

\begin{ex}\label{ex:4.2.8}
  Let \(E\) be a subset of \(\R\), let \(a\) be an interior point of \(E\), and let \(f : E \to \R\) be a function which is real analytic at \(a\), and has a power series expansion
  \[
    f(x) = \sum_{n = 0}^\infty c_n (x - a)^n
  \]
  at \(a\) which converges on the interval \((a - r, a + r)\).
  Let \((b - s, b + s)\) be any sub-interval of \((a - r, a + r)\) for some \(s > 0\).
  \begin{enumerate}
    \item Prove that \(\abs{a - b} \leq r - s\), so in particular \(\abs{a - b} < r\).
    \item Show that for every \(0 < \varepsilon < r\), there exists a \(C > 0\) such that \(\abs{c_n} \leq C(r - \varepsilon)^{-n}\) for all integers \(n \geq 0\).
    \item Show that the numbers \(d_0, d_1, \dots\) given by the formula
          \[
            d_m \coloneqq \sum_{n = m}^\infty \dfrac{n!}{m! (n - m)!} (b - a)^{n - m} c_n \text{ for all integers } m \geq 0
          \]
          are well-defined, in the sense that the above series is absolutely convergent.
    \item Show that for every \(0 < \varepsilon < s\) there exists a \(C > 0\) such that
          \[
            \abs{d_m} \leq C(s - \varepsilon)^{-m}
          \]
          for all integers \(m \geq 0\).
    \item Show that the power series \(\sum_{m = 0}^\infty d_m (x - b)^m\) is absolutely convergent for \(x \in (b - s, b + s)\) and converges to \(f(x)\).
    \item Conclude that \(f\) is real analytic at every point in \((a - r, a + r)\).
  \end{enumerate}
\end{ex}

\begin{proof}{(a)}
  \begin{align*}
             & a - r \leq b - s \leq b + s \leq a + r            \\
    \implies & a - b - r \leq -s \leq s \leq a - b + r           \\
    \implies & s - r \leq a - b \leq r - s                       \\
    \implies & \abs{a - b} \leq r - s < r.             & (s > 0)
  \end{align*}
\end{proof}

\begin{proof}{(b)}
  Let \(\varepsilon \in (0, r)\).
  Since
  \begin{align*}
             & 0 < \varepsilon < r                                                                                                     \\
    \implies & 0 < r - \varepsilon < r                                                                                                 \\
    \implies & (a - r + \varepsilon, a + r - \varepsilon) \subseteq (a - r, a + r)                                                     \\
    \implies & \forall x \in (a, a + r) \setminus (a, a + r - \varepsilon), r - \varepsilon \leq x - a < r                             \\
    \implies & \forall n \in \N, \forall x \in (a, a + r) \setminus (a, a + r - \varepsilon), (r - \varepsilon)^n \leq (x - a)^n < r^n
  \end{align*}
  and
  \begin{align*}
             & \forall x \in (a, a + r) \setminus (a, a + r - \varepsilon),                                                                               \\
             & r - \varepsilon < x - a < r \leq \dfrac{1}{\limsup_{n \to \infty} \abs{c_n}^{\dfrac{1}{n}}} &  & \by{4.2.1}                                \\
    \implies & \forall x \in (a, a + r) \setminus (a, a + r - \varepsilon),                                                                               \\
             & \sum_{n = 0}^\infty c_n (x - a)^n \text{ is absolutely convergent}                          &  & \text{(by \cref{4.1.6}(b))}               \\
    \implies & \forall x \in (a, a + r) \setminus (a, a + r - \varepsilon),                                                                               \\
             & \lim_{n \to \infty} c_n (x - a)^n = 0                                                       &  & \text{(by Corollary 7.2.6 in Analysis I)} \\
    \implies & \forall x \in (a, a + r) \setminus (a, a + r - \varepsilon),                                                                               \\
             & \lim_{n \to \infty} \abs{c_n} (x - a)^n = 0                                                                                                \\
    \implies & \lim_{n \to \infty} \abs{c_n} (r - \varepsilon)^n = 0,                                      &  & \text{(by squeeze test)}
  \end{align*}
  we know that
  \begin{align*}
             & \lim_{n \to \infty} \abs{c_n} (r - \varepsilon)^n = 0                      \\
    \implies & \exists N \in \Z^+ : \forall n \geq N, \abs{c_n (r - \varepsilon)^n} < 1   \\
    \implies & \exists N \in \Z^+ : \forall n \geq N, \abs{c_n} < (r - \varepsilon)^{-n}.
  \end{align*}
  Now we fix such \(N\).
  If we define
  \[
    C = 1 + \max\bigg(\abs{c_0} (r - \varepsilon)^0, \dots, \abs{c_{N - 1}} (r - \varepsilon)^{N - 1}\bigg),
  \]
  then we have
  \begin{align*}
             & \begin{dcases}
                 \forall n \geq N, \abs{c_n} < (r - \varepsilon)^{-n} \\
                 \forall 0 \leq n \leq N - 1, \abs{c_n} \leq C
               \end{dcases}                    \\
    \implies & \forall n \geq N, \abs{c_n} \leq C (r - \varepsilon)^{-n}. & (C \geq 1)
  \end{align*}
  Since \(\varepsilon\) is arbitrary, we conclude that
  \[
    \forall \varepsilon \in (0, r), \exists C \in \R^+ : \forall n \in \N, \abs{c_n} \leq C (r - \varepsilon)^{-n}.
  \]
\end{proof}

\begin{proof}{(c)}
  By \cref{ex:4.2.8}(a) we know that \(\abs{b - a} = \abs{a - b} < r\), thus
  \[
    \abs{b - a} < r \implies \exists \varepsilon \in \R^+ : \begin{dcases}
      \abs{b - a} + \varepsilon < r \\
      \varepsilon < r
    \end{dcases} \implies \exists \varepsilon \in \R^+ : \begin{dcases}
      \abs{b - a} < r - \varepsilon \\
      0 < r - \varepsilon
    \end{dcases}
  \]
  Fix such \(\varepsilon\).
  By \cref{ex:4.2.8}(b) we know that
  \[
    \exists C \in \R^+ : \forall n \in \N, \abs{c_n} \leq C (r - \varepsilon)^{-n}.
  \]
  Fix such \(C\).
  Since
  \begin{align*}
     & \forall m \in \N, \dfrac{r - \varepsilon}{(r - \varepsilon - \abs{b - a})^{m + 1}}                                                       \\
     & = \sum_{n = m}^\infty \dfrac{n!}{m! (n - m)!} \abs{b - a}^{n - m} (r - \varepsilon)^{-n}                &  & \by{4.2.7}                  \\
     & = \dfrac{1}{C} \sum_{n = m}^\infty \dfrac{n!}{m! (n - m)!} \abs{b - a}^{n - m} C (r - \varepsilon)^{-n}                                  \\
     & \geq \dfrac{1}{C} \sum_{n = m}^\infty \dfrac{n!}{m! (n - m)!} \abs{b - a}^{n - m} \abs{c_n}                                              \\
     & \geq \dfrac{1}{C} \sum_{n = m}^\infty \dfrac{n!}{m! (n - m)!} (b - a)^{n - m} c_n                       &  & \text{(by comparison test)} \\
     & = \dfrac{1}{C} d_m,
  \end{align*}
  we know that \(d_m\) are absolutely convergent for all \(m \in \N\).
\end{proof}

\begin{proof}{(d)}
  Let \(\varepsilon \in (0, s)\).
  By \cref{ex:4.2.8}(a) we have
  \begin{align*}
             & \abs{a - b} \leq r - s                                                    \\
    \implies & \abs{a - b} + s \leq r                                                    \\
    \implies & s \leq r                                                                  \\
    \implies & 0 < \varepsilon < r                                                       \\
    \implies & 0 < s - \varepsilon \leq r - \varepsilon                                  \\
    \implies & \forall n \in \N, 0 < (r - \varepsilon)^{-n} \leq (s - \varepsilon)^{-n}.
  \end{align*}
  By \cref{ex:4.2.8}(b) we know that
  \[
    \exists C \in \R^+ : \forall n \in \N, \abs{c_n} \leq C (r - \varepsilon)^{-n} \leq C (s - \varepsilon)^{-n}.
  \]
  Since
  \begin{align*}
             & \abs{a - b} \leq r - s < r - \varepsilon                                      \\
    \implies & \forall m \in \N, \abs{a - b}^{m} \leq (r - s)^m                              \\
    \implies & \forall m \in \N, \forall n \geq m, \abs{a - b}^{n - m} \leq (r - s)^{n - m},
  \end{align*}
  we know that
  \begin{align*}
    \forall m \in \N, \abs{d_m} & = \abs{\sum_{n = m}^\infty \dfrac{n!}{m! (n - m)!} (b - a)^{n - m} c_n}                                                          \\
                                & \leq \sum_{n = m}^\infty \abs{\dfrac{n!}{m! (n - m)!} (b - a)^{n - m} c_n}                      &  & \text{(by comparison test)} \\
                                & \leq C \sum_{n = m}^\infty \abs{\dfrac{n!}{m! (n - m)!} (b - a)^{n - m} (r - \varepsilon)^{-n}}                                  \\
                                & = C \sum_{n = m}^\infty \dfrac{n!}{m! (n - m)!} \abs{b - a}^{n - m} (r - \varepsilon)^{-n}                                       \\
                                & \leq C \sum_{n = m}^\infty \dfrac{n!}{m! (n - m)!} (r - s)^{n - m} (r - \varepsilon)^{-n}                                        \\
                                & = C \dfrac{r - \varepsilon}{\big((r - \varepsilon) - (r - s)\big)^{m + 1}}                      &  & \by{4.2.7}                  \\
                                & = C \dfrac{r - \varepsilon}{(s - \varepsilon)^{m + 1}}                                                                           \\
                                & = C \dfrac{r - \varepsilon}{s - \varepsilon} (s - \varepsilon)^{-m}.
  \end{align*}
  Thus by setting \(C' = C \dfrac{r - \varepsilon}{s - \varepsilon}\) we have
  \[
    \forall m \in \N, \abs{d_m} \leq C' (s - \varepsilon)^{-m}.
  \]
  Since \(\varepsilon\) is arbitrary, we conclude that
  \[
    \forall \varepsilon \in (0, s), \exists C \in \R^+ : \forall m \in \N, \abs{d_m} \leq C (s - \varepsilon)^{-m}.
  \]
\end{proof}

\begin{proof}{(e)}
  Let \(x \in (b - s, b + s)\).
  We have
  \begin{align*}
             & x \in (b - s, b + s)                                                                                         \\
    \implies & 0 \leq \abs{x - b} < s                                                                                       \\
    \implies & \exists \varepsilon \in \R^+ : 0 \leq \abs{x - b} < s - \varepsilon                                          \\
    \implies & \exists \varepsilon \in \R^+ : \begin{dcases}
                                                0 \leq \abs{\dfrac{x - b}{s - \varepsilon}} < 1 \\
                                                \exists C \in \R^+ : \forall m \in \N, \abs{d_m} \leq C (s - \varepsilon)^{-m}
                                              \end{dcases} &  & \text{(by \cref{ex:4.2.8}(e))}
  \end{align*}
  Fix such \(\varepsilon\) and \(C\).
  Since
  \begin{align*}
             & \abs{\dfrac{x - b}{s - \varepsilon}} < 1                                                                                                \\
    \implies & \sum_{m = 0}^\infty \abs{\dfrac{x - b}{s - \varepsilon}}^m \text{ is absolutely convergent}, &  & \text{(by Lemma 7.3.3 in Analysis I)}
  \end{align*}
  we have
  \begin{align*}
             & \forall m \in \N, \abs{d_m} \leq C (s - \varepsilon)^{-m}                                                                                                                                                     \\
    \implies & \forall m \in \N, \abs{d_m (x - b)^m} \leq C \abs{\dfrac{x - b}{s - \varepsilon}}^{m}                                                                                                                         \\
    \implies & \sum_{m = 0}^\infty \abs{d_m (x - b)^m} \leq \sum_{m = 0}^\infty C \abs{\dfrac{x - b}{s - \varepsilon}}^{m} = C \sum_{m = 0}^\infty \abs{\dfrac{x - b}{s - \varepsilon}}^{m}                                  \\
    \implies & \sum_{m = 0}^\infty d_m (x - b)^m \text{ is absolutely convergent}.                                                                                                          &  & \text{(by comparison test)}
  \end{align*}
  Since \(\N \times \N\) is countable, we know that \(\N \times S\) is also countable for every non-empty subset \(S\) of \(\N\).
  Thus
  \begin{align*}
     & \sum_{m = 0}^\infty d_m (x - b)^m                                                                                                             \\
     & = \sum_{m = 0}^\infty \bigg(\sum_{n = m}^\infty \dfrac{n!}{m! (n - m)!} (b - a)^{n - m} c_n\bigg) (x - b)^m                                   \\
     & = \sum_{m = 0}^\infty \bigg(\sum_{n = m}^\infty \dfrac{c_n n!}{m! (n - m)!} (b - a)^{n - m} (x - b)^m\bigg)                                   \\
     & = \sum_{n = 0}^\infty \bigg(\sum_{m = 0}^n \dfrac{c_n n!}{m! (n - m)!} (b - a)^{n - m} (x - b)^m\bigg)      &  & \text{(by Fubini's theorem)} \\
     & = \sum_{n = 0}^\infty c_n \bigg(\sum_{m = 0}^n \dfrac{n!}{m! (n - m)!} (b - a)^{n - m} (x - b)^m\bigg)                                        \\
     & = \sum_{n = 0}^\infty c_n (x - a)                                                                           &  & \by{ex:4.2.5}                \\
     & = f(x).
  \end{align*}
  Since \(x\) is arbitrary, we conclude that
  \[
    \forall x \in (b - s, b + s), \sum_{m = 0}^\infty d_m (x - b)^m = f(x) \text{ is absolutely convergent}.
  \]
\end{proof}

\begin{proof}{(f)}
  Let \(b \in (a - r, a + r)\).
  Since
  \begin{align*}
             & \abs{b - a} < r                                     \\
    \implies & \exists s \in \R^+ : \abs{b - a} < r - s            \\
    \implies & \exists s \in \R^+ : s - r < b - a < r - s          \\
    \implies & \exists s \in \R^+ : a - r < b - s < b + s < a + r,
  \end{align*}
  by \cref{ex:4.2.8}(e) we know that \(f\) is real analytic at \(b\).
  Since \(b\) is arbitrary, we conclude that \(f\) is real analytic at \(x\) for all \(x \in (a - r, a + r)\).
\end{proof}

\section{Abel's theorem}\label{sec:4.3}

\begin{note}
  Let \(f(x) = \sum_{n = 0}^\infty c_n (x - a)^n\) be a power series centered at \(a\) with a radius of convergence \(0 < R < \infty\) strictly between \(0\) and infinity.
  From \cref{4.1.6} we know that the power series converges absolutely whenever \(\abs{x - a} < R\), and diverges when \(\abs{x - a} > R\).
  However, at the boundary \(\abs{x - a} = R\) the situation is more complicated;
  the series may either converge or diverge (see \cref{ex:4.1.2}).
  However, if the series does converge at the boundary point, then it is reasonably well behaved;
  in particular, it is continuous at that boundary point.
\end{note}

\begin{thm}[Abel's theorem]\label{4.3.1}
  Let \(f(x) = \sum_{n = 0}^\infty c_n (x - a)^n\) be a power series centered at \(a\) with radius of convergence \(0 < R < \infty\).
  If the power series converges at \(a + R\), then \(f\) is continuous at \(a + R\), i.e.
  \[
    \lim_{x \to a + R ; x \in (a - R, a + R)} \sum_{n = 0}^\infty c_n (x - a)^n = \sum_{n = 0}^\infty c_n R^n.
  \]
  Similarly, if the power series converges at \(a - R\), then \(f\) is continuous at \(a - R\), i.e.
  \[
    \lim_{x \to a - R ; x \in (a - R, a + R)} \sum_{n = 0}^\infty c_n (x - a)^n = \sum_{n = 0}^\infty c_n (-R)^n.
  \]
\end{thm}

\begin{proof}
  It will suffice to prove the first claim, i.e., that
  \[
    \lim_{x \to a + R ; x \in (a - R, a + R)} \sum_{n = 0}^\infty c_n (x - a)^n = \sum_{n = 0}^\infty c_n R^n.
  \]
  whenever the sum \(\sum_{n = 0}^\infty c_n R^n\) converges;
  the second claim will then follow by replacing \(c_n\) by \((-1)^n c_n\) in the above claim.
  If we make the substitutions \(d_n \coloneqq c_n R^n\) and \(y \coloneqq \dfrac{x - a}{R}\), then the above claim can be rewritten as
  \[
    \lim_{y \to 1 ; y \in (-1, 1)} \sum_{n = 0}^\infty d_n y^n = \sum_{n = 0}^\infty d_n
  \]
  whenever the sum \(\sum_{n = 0}^\infty d_n\) converges.

  Write \(D \coloneqq \sum_{n = 0}^\infty d_n\), and for every \(N \geq 0\) write
  \[
    S_N \coloneqq \bigg(\sum_{n = 0}^{N - 1} d_n\bigg) - D
  \]
  so in particular \(S_0 = -D\).
  Then observe that \(\lim_{N \to \infty} S_N = 0\), and that \(d_n = S_{n + 1} - S_n\).
  Thus for any \(y \in (-1, 1)\) we have
  \[
    \sum_{n = 0}^\infty d_n y^n = \sum_{n = 0}^\infty (S_{n + 1} - S_n) y^n.
  \]
  Applying the summation by parts formula (\cref{4.3.2}), and noting that \(\lim_{n \to \infty} y^n = 0\), we obtain
  \[
    \sum_{n = 0}^\infty d_n y^n = - S_0 y^0 - \sum_{n = 0}^\infty S_{n + 1} (y^{n + 1} - y^n).
  \]
  Observe that \(- S_0 y^0 = +D\).
  Thus to finish the proof of Abel's theorem,
  it will suffice to show that
  \[
    \lim_{y \to 1 ; y \in (-1, 1)} \sum_{n = 0}^\infty S_{n + 1} (y^{n + 1} - y^n) = 0.
  \]
  Since \(y\) converges to \(1\), we may as well restrict \(y\) to \([0, 1)\) instead of \((-1, 1)\);
  in particular we may take \(y\) to be positive.

  From the triangle inequality for series (Proposition 7.2.9 in Analysis I), we have
  \begin{align*}
    \abs{\sum_{n = 0}^\infty S_{n + 1} (y^{n + 1} - y^n)} & \leq \sum_{n = 0}^\infty \abs{S_{n + 1} (y^{n + 1} - y^n)} \\
                                                          & = \sum_{n = 0}^\infty \abs{S_{n + 1}} (y^n - y^{n + 1}),
  \end{align*}
  so by the squeeze test (Corollary 6.4.14 in Analysis I) it suffices to show that
  \[
    \lim_{y \to 1 ; y \in [0, 1)} \sum_{n = 0}^\infty \abs{S_{n + 1}} (y^n - y^{n + 1}) = 0.
  \]
  The expression \(\sum_{n = 0}^\infty \abs{S_{n + 1}} (y^n - y^{n + 1})\) is clearly non-negative, so it will suffice to show that
  \[
    \limsup_{y \to 1 ; y \in [0, 1)} \sum_{n = 0}^\infty \abs{S_{n + 1}} (y^n - y^{n + 1}) = 0.
  \]
  Let \(\varepsilon > 0\).
  Since \(S_n\) converges to \(0\), there exists an \(N\) such that \(\abs{S_n} \leq \varepsilon\) for all \(n > N\).
  Thus we have
  \[
    \sum_{n = 0}^\infty \abs{S_{n + 1}} (y^n - y^{n + 1}) \leq \sum_{n = 0}^N \abs{S_{n + 1}} (y^n - y^{n + 1}) + \sum_{n = N + 1}^\infty \varepsilon (y^n - y^{n + 1}).
  \]
  The last summation is a telescoping series, which sums to \(\varepsilon y^{N + 1}\) (See Lemma 7.2.15 in Analysis I, recalling from Lemma 6.5.2 in Analysis I that \(y^n \to 0\) as \(n \to \infty\)), and thus
  \[
    \sum_{n = 0}^\infty \abs{S_{n + 1}} (y^n - y^{n + 1}) \leq \sum_{n = 0}^N \abs{S_{n + 1}} (y^n - y^{n + 1}) + \varepsilon y^{N + 1}.
  \]
  Now take limits as \(y \to 1\).
  Observe that \(y^n - y^{n + 1} \to 0\) as \(y \to 1\) for every \(n \in 0, 1, \dots, N\).
  Since we can interchange limits and \emph{finite} sums (Exercise 7.1.5 in Analysis I), we thus have
  \[
    \limsup_{y \to 1 ; y \in [0, 1)} \sum_{n = 0}^\infty \abs{S_{n + 1}} (y^n - y^{n + 1}) \leq \varepsilon.
  \]
  But \(\varepsilon > 0\) was arbitrary, and thus we must have
  \[
    \limsup_{y \to 1 ; y \in [0, 1)} \sum_{n = 0}^\infty \abs{S_{n + 1}} (y^n - y^{n + 1}) = 0
  \]
  since the left-hand side must be non-negative.
  The claim follows.
\end{proof}

\begin{lem}[Summation by parts formula]\label{4.3.2}
  Let \((a_n)_{n = 0}^\infty\) and \((b_n)_{n = 0}^\infty\) be sequences of real numbers which converge to limits \(A\) and \(B\) respectively, i.e., \(\lim_{n \to \infty} a_n = A\) and \(\lim_{n \to \infty} b_n = B\).
  Suppose that the sum \(\sum_{n = 0}^\infty (a_{n + 1} - a_n) b_n\) is convergent.
  Then the sum \(\sum_{n = 0}^\infty a_{n + 1} (b_{n + 1} - b_n)\) is also convergent, and
  \[
    \sum_{n = 0}^\infty (a_{n + 1} - a_n) b_n = AB - a_0 b_0 - \sum_{n = 0}^\infty a_{n + 1} (b_{n + 1} - b_n).
  \]
\end{lem}

\begin{proof}
  Since
  \begin{align*}
    \forall N \in \N, & \sum_{n = 0}^N (a_{n + 1} - a_n) b_n + \sum_{n = 0}^N a_{n + 1} (b_{n + 1} - b_n) \\
                      & = \sum_{n = 0}^N a_{n + 1} (b_n - a_n b_n + a_{n + 1} b_{n + 1} - a_{n + 1} b_n)  \\
                      & = \sum_{n = 0}^N a_{n + 1} (b_{n + 1} - a_n b_n)                                  \\
                      & = a_{N + 1} b_{N + 1} - a_0 b_0
  \end{align*}
  and
  \begin{align*}
             & \begin{dcases}
                 \lim_{n \to \infty} a_n = A \\
                 \lim_{n \to \infty} b_n = B
               \end{dcases}       \\
    \implies & \lim_{n \to \infty} a_n b_n = AB,
  \end{align*}
  we have
  \begin{align*}
     & \lim_{N \to \infty} \bigg(\sum_{n = 0}^N (a_{n + 1} - a_n) b_n + \sum_{n = 0}^N a_{n + 1} (b_{n + 1} - b_n)\bigg) \\
     & = \lim_{N \to \infty} (a_{N + 1} b_{N + 1} - a_0 b_0)                                                             \\
     & = \lim_{N \to \infty} (a_{N + 1} b_{N + 1}) - a_0 b_0                                                             \\
     & = AB - a_0 b_0.
  \end{align*}
  Thus
  \begin{align*}
     & \bigg(\sum_{n = 0}^\infty (a_{n + 1} - a_n) b_n\bigg) - AB + a_0 b_0                                                                                                                       \\
     & = \lim_{N \to \infty} \bigg(\sum_{n = 0}^N (a_{n + 1} - a_n) b_n\bigg) - \lim_{N \to \infty} \bigg(\sum_{n = 0}^N (a_{n + 1} - a_n) b_n + \sum_{n = 0}^N a_{n + 1} (b_{n + 1} - b_n)\bigg) \\
     & = \lim_{N \to \infty} - \bigg(\sum_{n = 0}^N a_{n + 1} (b_{n + 1} - b_n)\bigg)                                                                                                             \\
     & = - \lim_{N \to \infty} \bigg(\sum_{n = 0}^N a_{n + 1} (b_{n + 1} - b_n)\bigg)                                                                                                             \\
     & = - \sum_{n = 0}^\infty a_{n + 1} (b_{n + 1} - b_n)
  \end{align*}
  and
  \[
    \sum_{n = 0}^\infty (a_{n + 1} - a_n) b_n = AB - a_0 b_0 - \sum_{n = 0}^\infty a_{n + 1} (b_{n + 1} - b_n).
  \]
\end{proof}

\begin{rmk}\label{4.3.3}
  One should compare this formula with the more well-known \emph{integration by parts formula}
  \[
    \int_0^\infty f'(x) g(x) \; dx = f(x) g(x) |_0^\infty - \int_0^\infty f(x) g'(x) \; dx,
  \]
  see Proposition 11.10.1 in Analysis I.
\end{rmk}

\exercisesection

\begin{ex}\label{ex:4.3.1}
  Prove \cref{4.3.2}.
\end{ex}

\begin{proof}
  See \cref{4.3.2}.
\end{proof}
\section{Multiplication of power series}\label{ii:sec:4.4}

\begin{thm}\label{ii:4.4.1}
  Let \(f : (a - r, a + r) \to \R\) and \(g : (a - r, a + r) \to \R\) be functions analytic on \((a - r, a + r)\), with power series expansions
  \[
    f(x) = \sum_{n = 0}^\infty c_n (x - a)^n
  \]
  and
  \[
    g(x) = \sum_{n = 0}^\infty d_n (x - a)^n
  \]
  respectively.
  Then \(fg : (a - r, a + r) \to \R\) is also analytic on \((a - r, a + r)\), with power series expansion
  \[
    f(x) g(x) = \sum_{n = 0}^\infty e_n (x - a)^n
  \]
  where \(e_n \coloneqq \sum_{m = 0}^n c_m d_{n - m}\).
\end{thm}

\begin{proof}
  We have to show that the series \(\sum_{n = 0}^\infty e_n (x - a)^n\) converges to \(f(x) g(x)\) for all \(x \in (a - r, a + r)\).
  Now fix \(x\) to be any point in \((a - r, a + r)\).
  By \cref{ii:4.1.6}, we see that both \(f\) and \(g\) have radii of convergence at least \(r\).
  In particular, the series \(\sum_{n = 0}^\infty c_n (x - a)^n\) and \(\sum_{n = 0}^\infty d_n (x - a)^n\) are absolutely convergent.
  Thus, if we define
  \[
    C \coloneqq \sum_{n = 0}^\infty \abs{c_n (x - a)^n}
  \]
  and
  \[
    D \coloneqq \sum_{n = 0}^\infty \abs{d_n (x - a)^n}
  \]
  then \(C\) and \(D\) are both finite.

  For any \(N \geq 0\), consider the partial sum
  \[
    \sum_{n = 0}^N \sum_{m = 0}^\infty \abs{c_m (x - a)^m d_n (x - a)^n}.
  \]
  We can rewrite this as
  \[
    \sum_{n = 0}^N \abs{d_n (x - a)^n} \sum_{m = 0}^\infty \abs{c_m (x - a)^m},
  \]
  which by definition of \(C\) is equal to
  \[
    \sum_{n = 0}^N \abs{d_n (x - a)^n} C,
  \]
  which by definition of \(D\) is less than or equal to \(DC\).
  Thus, the above partial sums are bounded by \(DC\) for every \(N\).
  In particular, the series
  \[
    \sum_{n = 0}^\infty \sum_{m = 0}^\infty \abs{c_m (x - a)^m d_n (x - a)^n}
  \]
  is convergent, which means that the sum
  \[
    \sum_{n = 0}^\infty \sum_{m = 0}^\infty c_m (x - a)^m d_n (x - a)^n
  \]
  is absolutely convergent.

  Let us now compute this sum in two ways.
  First of all, we can pull the \(d_n (x - a)^n\) factor out of the \(\sum_{m = 0}^\infty\) summation, to obtain
  \[
    \sum_{n = 0}^\infty d_n (x - a)^n \sum_{m = 0}^\infty c_m (x - a)^m.
  \]
  By our formula for \(f(x)\), this is equal to
  \[
    \sum_{n = 0}^\infty d_n (x - a)^n f(x);
  \]
  by our formula for \(g(x)\), this is equal to \(f(x) g(x)\).
  Thus
  \[
    f(x) g(x) = \sum_{n = 0}^\infty \sum_{m = 0}^\infty c_m (x - a)^m d_n (x - a)^n.
  \]
  Now we compute this sum in a different way.
  We rewrite it as
  \[
    f(x) g(x) = \sum_{n = 0}^\infty \sum_{m = 0}^\infty c_m d_n (x - a)^{n + m}.
  \]
  By Fubini's theorem for series (Theorem 8.2.2 in Analysis I), because the series was absolutely convergent, we may rewrite it as
  \[
    f(x) g(x) = \sum_{m = 0}^\infty \sum_{n = 0}^\infty c_m d_n (x - a)^{n + m}.
  \]
  Now make the substitution \(n' \coloneqq n + m\), to rewrite this as
  \[
    f(x) g(x) = \sum_{m = 0}^\infty \sum_{n' = m}^\infty c_m d_{n' - m} (x - a)^{n'}.
  \]
  If we adopt the convention that \(d_j = 0\) for all negative \(j\), then this is equal to
  \[
    f(x) g(x) = \sum_{m = 0}^\infty \sum_{n' = 0}^\infty c_m d_{n' - m} (x - a)^{n'}.
  \]
  Applying Fubini's theorem again, we obtain
  \[
    f(x) g(x) = \sum_{n' = 0}^\infty \sum_{m = 0}^\infty c_m d_{n' - m} (x - a)^{n'},
  \]
  which we can rewrite as
  \[
    f(x) g(x) = \sum_{n' = 0}^\infty (x - a)^{n'} \sum_{m = 0}^\infty c_m d_{n' - m}.
  \]
  Since \(d_j\) was \(0\) when \(j\) is negative, we can rewrite this as
  \[
    f(x) g(x) = \sum_{n' = 0}^\infty (x - a)^{n'} \sum_{m = 0}^{n'} c_m d_{n' - m},
  \]
  which by definition of \(e\) is
  \[
    f(x) g(x) = \sum_{n' = 0}^\infty e_{n'} (x - a)^{n'},
  \]
  as desired.
\end{proof}

\begin{rmk}\label{ii:4.4.2}
  The sequence \((e_n)_{n = 0}^\infty\) is sometimes referred to as the \emph{convolution} of the sequences \((c_n)_{n = 0}^\infty\) and \((d_n)_{n = 0}^\infty\);
  it is closely related (though not identical) to the notion of convolution introduced in \cref{ii:3.8.9}.
\end{rmk}

\section{The exponential and logarithm functions}\label{sec:4.5}

\begin{defn}[Exponential function]\label{4.5.1}
  For every real number \(x\), we define the \emph{exponential function} \(\exp(x)\) to be the real number
  \[
    \exp(x) \coloneqq \sum_{n = 0}^\infty \dfrac{x^n}{n!}.
  \]
\end{defn}

\begin{thm}[Basic properties of exponential]\label{4.5.2}
  \quad
  \begin{enumerate}
    \item For every real number \(x\), the series \(\sum_{n = 0}^\infty \dfrac{x^n}{n!}\) is absolutely convergent.
          In particular, \(\exp(x)\) exists and is real for every \(x \in \R\), the power series \(\sum_{n = 0}^\infty \dfrac{x^n}{n!}\) has an infinite radius of convergence, and \(\exp\) is a real analytic function on \((-\infty, \infty)\).
    \item \(\exp\) is differentiable on \(R\), and for every \(x \in \R\), \(\exp'(x) = \exp(x)\).
    \item \(\exp\) is continuous on \(R\), and for every interval \([a, b]\), we have \(\int_{[a, b]} \exp(x) \; dx = \exp(b) - \exp(a)\).
    \item For every \(x, y \in \R\), we have \(\exp(x + y) = \exp(x) \exp(y)\).
    \item We have \(\exp(0) = 1\).
          Also, for every \(x \in \R\), \(\exp(x)\) is positive, and \(\exp(-x) = 1 / \exp(x)\).
    \item \(\exp\) is strictly monotone increasing:
          in other words, if \(x, y\) are real numbers, then we have \(\exp(y) > \exp(x)\) iff \(y > x\).
  \end{enumerate}
\end{thm}

\begin{proof}{(a)}
  If \(x = 0\), then we have
  \[
    1 = 1 + 0 = \dfrac{0^0}{0!} + \sum_{n = 1}^\infty \dfrac{0^n}{n!} = \sum_{n = 0}^\infty \dfrac{0^n}{n!} = \exp(0).
  \]
  So suppose that \(x \in \R \setminus \set{0}\).
  Since
  \begin{align*}
             & \limsup_{n \to \infty} \abs{\dfrac{x^{n + 1}}{(n + 1)!} \dfrac{n!}{x^n}} = \limsup_{n \to \infty} \dfrac{\abs{x}}{n + 1} = 0 < 1                             \\
    \implies & \sum_{n = 0}^\infty \dfrac{x^n}{n!} \text{ is absolutely convergent},                                                            &  & \text{(by ratio test)}
  \end{align*}
  we know that \(\exp(x)\) exists for all \(x \in \R \setminus \set{0}\).
  Combine all proofs above we have
  \[
    \forall x \in \R, \begin{dcases}
      \sum_{n = 0}^\infty \dfrac{x^n}{n!} \text{ is absolutely convergent} \\
      \exp(x) \in \R
    \end{dcases}
  \]
  and by \cref{4.2.1} \(\exp\) is real analytic on \((-\infty, \infty)\).
\end{proof}

\begin{proof}{(b)}
  By \cref{4.5.2}(a) we know that \(\exp\) is real analytic on \(\R\), thus by \cref{4.1.6}(d) we know that \(\exp\) is differentiable on \(\R\) and
  \[
    \forall x \in \R, \exp'(x) = \sum_{n = 1}^\infty n \dfrac{x^{n - 1}}{n!} = \sum_{n = 1}^\infty \dfrac{x^{n - 1}}{(n - 1)!} = \sum_{n = 0}^\infty \dfrac{x^n}{n!} = \exp(x).
  \]
\end{proof}

\begin{proof}{(c)}
  By \cref{4.5.2}(a) we know that \(\exp\) is real analytic on \(\R\), thus by \cref{4.1.6}(c) we know that \(\exp\) is continuous on \(\R\).
  Let \(a, b \in \R\) such that \(a \leq b\).
  Then by \cref{4.5.2}(a) we know that \(\exp(a)\) and \(\exp(b)\) are well-defined.
  Since \([a, b] \subseteq \R\), by \cref{4.1.6}(e) we know that \(\exp\) is Riemann integrable on \([a, b]\) and
  \begin{align*}
    \int_{[a, b]} \exp & = \int_a^b \exp(x) \; dx                                                                                              \\
                       & = \sum_{n = 0}^\infty \dfrac{1}{n!} \dfrac{(b - 0)^{n + 1} - (a - 0)^{n + 1}}{n + 1} &  & \text{(by \cref{4.1.6}(e))} \\
                       & = \sum_{n = 0}^\infty \dfrac{b^{n + 1} - a^{n + 1}}{(n + 1)!}                                                         \\
                       & = \sum_{n = 1}^\infty \dfrac{b^n - a^n}{n!}                                                                           \\
                       & = \dfrac{b^0 - a^0}{0!} + \sum_{n = 1}^\infty \dfrac{b^n - a^n}{n!}                                                   \\
                       & = \sum_{n = 0}^\infty \dfrac{b^n - a^n}{n!}                                                                           \\
                       & = \sum_{n = 0}^\infty \dfrac{b^n}{n!} - \sum_{n = 0}^\infty \dfrac{a^n}{n!}                                           \\
                       & = \exp(b) - \exp(a).                                                                 &  & \by{4.5.1}
  \end{align*}
  Since \(a, b\) is arbitrary, we conclude that
  \[
    \forall [a, b] \subseteq \R, \int_{[a, b]} \exp = \exp(b) - \exp(a).
  \]
\end{proof}

\begin{proof}{(d)}
  If \(x = 0\), then we have
  \[
    \exp(0) = \sum_{n = 0}^\infty \dfrac{0^n}{n!} = \dfrac{0^0}{0!} + \sum_{n = 1}^\infty \dfrac{0^n}{n!} = 1
  \]
  and thus
  \[
    \forall y \in \R, \exp(0 + y) = \exp(y) = \exp(0) \exp(y).
  \]
  Since addition and multiplication of real numbers are commutative, we have
  \[
    \forall x \in \R, \exp(x + 0) = \exp(0 + x) = \exp(0) \exp(x) = \exp(x) \exp(0).
  \]
  So suppose that \(x, y \in \R \setminus \set{0}\).
  By \cref{4.5.2}(a) we know that \(\sum_{n = 0}^\infty \dfrac{x^n}{n!}\) and \(\sum_{n = 0}^\infty \dfrac{y^n}{n!}\) are absolutely convergent.
  Thus we know that
  \begin{align*}
    X & = \sum_{n = 0}^\infty \abs{\dfrac{x^n}{n!}} \\
    Y & = \sum_{n = 0}^\infty \abs{\dfrac{y^n}{n!}}
  \end{align*}
  are well-defined.
  Since
  \begin{align*}
    \forall N \in \N, & \sum_{n = 0}^N \sum_{m = 0}^\infty \abs{\dfrac{x^n y^m}{n! m!}}                              \\
                      & = \sum_{n = 0}^N \bigg(\abs{\dfrac{x^n}{n!}} \sum_{m = 0}^\infty \abs{\dfrac{y^m}{m!}}\bigg) \\
                      & = \sum_{n = 0}^N \bigg(\abs{\dfrac{x^n}{n!}} Y\bigg)                                         \\
                      & = Y \sum_{n = 0}^N \abs{\dfrac{x^n}{n!}}                                                     \\
                      & \leq Y X,
  \end{align*}
  and \((\sum_{n = 0}^N \sum_{m = 0}^\infty \abs{\dfrac{x^n y^m}{n! m!}})_{N = 0}^\infty\) is monotone increasing, we know that \(\sum_{n = 0}^\infty \sum_{m = 0}^\infty \dfrac{x^n y^m}{n! m!}\) is absolutely convergent.
  Now we define
  \[
    \forall n \in \Z, c_n = \begin{dcases}
      \dfrac{1}{n!} & \text{if } n \geq 0; \\
      0             & \text{if } n < 0.
    \end{dcases}
  \]
  Then we have
  \begin{align*}
     & \exp(x) \exp(y)                                                                                                                                                           \\
     & = \sum_{n = 0}^\infty \sum_{m = 0}^\infty \dfrac{x^n y^m}{n! m!}                                                                                                          \\
     & = \sum_{n = 0}^\infty \sum_{m = 0}^\infty c_n c_m x^n y^m                                                                                                                 \\
     & = \sum_{n = 0}^\infty \sum_{m = n}^\infty c_n c_{m - n} x^n y^{m - n}                                                                                                     \\
     & = \sum_{n = 0}^\infty \sum_{m = 0}^\infty c_n c_{m - n} x^n y^{m - n}                                  & (c_{m - n} = 0 \text{ if } m < n)                                \\
     & = \sum_{m = 0}^\infty \sum_{n = 0}^\infty c_n c_{m - n} x^n y^{m - n}                                  &                                   & \text{(by Fubini's theorem)} \\
     & = \sum_{m = 0}^\infty \sum_{n = 0}^m c_n c_{m - n} x^n y^{m - n}                                       & (c_{m - n} = 0 \text{ if } n > m)                                \\
     & = \sum_{m = 0}^\infty \sum_{n = 0}^m \dfrac{x^n y^{m - n}}{n! (m - n)!}                                                                                                   \\
     & = \sum_{m = 0}^\infty \bigg(\dfrac{1}{m!} \sum_{n = 0}^m \dfrac{m!}{n! (m - n)!} (x^n y^{m - n})\bigg)                                                                    \\
     & = \sum_{m = 0}^\infty \dfrac{(x + y)^m}{m!}                                                            &                                   & \by{ex:4.2.5}                \\
     & = \exp(x + y).                                                                                         &                                   & \by{4.5.1}
  \end{align*}
  Combine all proofs above we conclude that
  \[
    \forall x, y \in \R, \exp(x + y) = \exp(x) \exp(y).
  \]
\end{proof}

\begin{proof}{(e)}
  We have
  \[
    \exp(0) = \sum_{n = 0}^\infty \dfrac{0^n}{n!} = \dfrac{0^0}{0!} + \sum_{n = 1}^\infty \dfrac{0^n}{n!} = 1
  \]
  and
  \[
    \forall x \in \R^+, \exp(x) = \sum_{n = 0}^\infty \dfrac{x^n}{n!} = \dfrac{x^0}{0!} + \sum_{n = 1}^\infty \dfrac{x^n}{n!} \geq 1.
  \]
  Since
  \begin{align*}
    \forall x \in \R, \exp(0) & = \exp(x - x)                                       \\
                              & = \exp(x) \exp(-x) &  & \text{(by \cref{4.5.2}(d))} \\
                              & = 1,
  \end{align*}
  we know that
  \[
    \forall x \in \R, \exp(-x) = \dfrac{1}{\exp(x)}.
  \]
  Thus
  \[
    \forall x \in \R^-, \exp(-x) \geq 1 \implies \exp(x) = \dfrac{1}{\exp(-x)} > 0.
  \]
  Combine all proofs above we conclude that
  \[
    \forall x \in \R, \exp(x) > 0.
  \]
\end{proof}

\begin{proof}{(f)}
  By \cref{4.5.2}(b)(e) we know that
  \[
    \forall x \in \R, \exp'(x) = \exp(x) > 0.
  \]
  Thus by Proposition 10.3.3 in Analysis I we know that \(\exp\) is strictly monotone increasing.
\end{proof}

\begin{note}
  One can write the exponential function in a more compact form, introducing famous \emph{Euler's number} \(e = 2.71828183 \dots\), also known as the \emph{base of the natural logarithm}.
\end{note}

\begin{defn}[Euler's number]\label{4.5.3}
  The number \(e\) is defined to be
  \[
    e = \exp(1) = \sum_{n = 0}^\infty \dfrac{1}{n!} = \dfrac{1}{0!} + \dfrac{1}{1!} + \dfrac{1}{2!} + \dfrac{1}{3!} + \dots.
  \]
\end{defn}

\begin{prop}\label{4.5.4}
  For every real number \(x\), we have \(\exp(x) = e^x\).
\end{prop}

\begin{proof}
  First we use induction on \(x\) to show that \(e^x = \exp(x)\) for all \(x \in \N\).
  For \(x = 0\), we have
  \begin{align*}
    e^0 & = \big(\exp(1)\big)^0 &  & \by{4.5.3}                  \\
        & = 1                   &  & \text{(by \cref{4.5.2}(a))} \\
        & = \exp(0)             &  & \text{(by \cref{4.5.2}(e))}
  \end{align*}
  and the base case holds.
  Suppose inductively that \(e^x = \exp(x)\) for some \(x \geq 0\).
  Then for \(x + 1\), we have
  \begin{align*}
    e^{x + 1} & = \exp(1)^{x + 1}   &  & \by{4.5.3}                  \\
              & = \exp(1) \exp(1)^x &  & \text{(by \cref{4.5.2}(a))} \\
              & = \exp(1) e^x       &  & \by{4.5.3}                  \\
              & = \exp(1) \exp(x)   &  & \byIH                       \\
              & = \exp(x + 1)       &  & \text{(by \cref{4.5.2}(d))}
  \end{align*}
  and this closes the induction.

  Next we show that \(e^x = \exp(x)\) for all \(x \in \Z\).
  Let \(x \in \Z^-\).
  Since \(-x \in \N\), we know that
  \begin{align*}
    e^{-x} & = \exp(-x)               &  & \text{(from the proof above)} \\
           & = \dfrac{1}{\exp(x)}     &  & \text{(by \cref{4.5.2}(e))}   \\
           & = \big(\exp(x)\big)^{-1}
  \end{align*}
  and thus \(e^x = \exp(x)\).
  Since \(x\) is arbitrary, combine the proofs above we conclude that \(e^x = \exp(x)\) for all \(x \in \Z\).

  Next we show that \(e^x = \exp(x)\) for all \(x \in \Q\).
  Let \(x \in \Q\).
  Since \(x = \dfrac{a}{b}\) for some \(a \in \Z\) and \(b \in \Z^+\), we know that
  \begin{align*}
    e^a & = \exp(a)                            &  & \text{(from the proof above)} \\
        & = \exp(\dfrac{ab}{b})                                                   \\
        & = \exp(\sum_{i = 1}^b \dfrac{a}{b})                                     \\
        & = \prod_{i = 1}^b \exp(\dfrac{a}{b}) &  & \text{(by \cref{4.5.2}(d))}   \\
        & = \exp(\dfrac{a}{b})^b                                                  \\
        & = \exp(x)^b
  \end{align*}
  and thus
  \begin{align*}
    e^x & = e^{\dfrac{a}{b}}                                                                \\
        & = \big(e^a\big)^{\dfrac{1}{b}}                                                    \\
        & = \Big(\big(\exp(x)\big)^b\Big)^{\dfrac{1}{b}} &  & \text{(from the proof above)} \\
        & = \exp(x).
  \end{align*}
  Since \(x\) is arbitrary, we conclude that \(e^x = \exp(x)\) for all \(x \in \Q\).

  Finally we show that \(e^x = \exp(x)\) for all \(x \in \R\).
  Let \(x \in \R\).
  Then we know that there exists a Cauchy sequence \((q_n)_{n = 1}^\infty\) in \(Q\) such that \(\lim_{n \to \infty} q_n = x\).
  Thus
  \begin{align*}
    e^x & = \lim_{n \to \infty} e^{q_n}                                      \\
        & = \lim_{n \to \infty} \exp(q_n) &  & \text{(from the proof above)} \\
        & = \exp(x).                      &  & \text{(by \cref{4.5.2}(c))}
  \end{align*}
  Since \(x\) is arbitrary, we conclude that \(e^x = \exp(x)\) for all \(x \in \R\).
\end{proof}

\begin{note}
  In light of \cref{4.5.3} we can and will use \(e^x\) and \(\exp(x)\) interchangeably.
\end{note}

\begin{note}
  Since \(e > 1\), we see that \(e^x \to +\infty\) as \(x \to +\infty\), and \(e^x \to 0\) as \(x \to -\infty\).
  From this and the intermediate value theorem (Theorem 9.7.1 in Analysis I) we see that the range of the function \(\exp\) is \((0, \infty)\).
  Since \(\exp\) is increasing, it is injective, and hence \(\exp\) is a bijection from \(\R\) to \((0, \infty)\), and thus has an inverse from \((0, \infty) \to \R\).
\end{note}

\begin{defn}[Logarithm]\label{4.5.5}
  We define the \emph{natural logarithm function}
  \[
    \log : (0, \infty) \to \R
  \]
  (also called \(\ln\)) to be the inverse of the exponential function.
  Thus \(\exp\big(\log(x)\big) = x\) and \(\log\big(\exp(x)\big) = x\).
\end{defn}

\begin{thm}arithm properties]\label{4.5.6}
  \quad
  \begin{enumerate}
    \item For every \(x \in (0, \infty)\), we have \(\ln'(x) = \dfrac{1}{x}\).
          In particular, by the fundamental theorem of calculus, we have \(\int_{[a, b]} \dfrac{1}{x} \; dx = \ln(b) - \ln(a)\) for any interval \([a, b]\) in \((0, \infty)\).
    \item We have \(\ln(xy) = \ln(x) + \ln(y)\) for all \(x, y \in (0, \infty)\).
    \item We have \(\ln(1) = 0\) and \(\ln(1 / x) = -\ln(x)\) for all \(x \in (0, \infty)\).
    \item For any \(x \in (0, \infty)\) and \(y \in R\), we have \(\ln(x^y) = y \ln(x)\).
    \item For any \(x \in (-1, 1)\), we have
          \[
            \ln(1 - x) = - \sum_{n = 1}^\infty \dfrac{x^n}{n}.
          \]
          In particular, \(\ln\) is analytic at \(1\), with the power series expansion
          \[
            \ln(x) = \sum_{n = 1}^\infty \dfrac{(-1)^{n + 1}}{n} (x - 1)^n
          \]
          for \(x \in (0, 2)\), with radius of convergence \(1\).
  \end{enumerate}
\end{thm}

\begin{proof}{(a)}
  Since \(\exp\) is continuous and strictly monotone increasing, we see that \(\ln\) is also continuous and strictly monotone increasing (see Proposition 9.8.3 in Analysis I).
  Since \(\exp\) is also differentiable, and the derivative is never zero, we see from the inverse function theorem (Theorem 10.4.2 in Analysis I) that \(\ln\) is also differentiable.
  Thus we have
  \begin{align*}
             & \forall x \in (0, \infty), \exp\big(\ln(x)\big) = x                   &  & \by{4.5.5}                            \\
    \implies & \forall x \in (0, \infty), \ln'(x) = \dfrac{1}{\exp'\big(\ln(x)\big)} &  & \text{(Theorem 10.4.2 in Analysis I)} \\
    \implies & \forall x \in (0, \infty), \ln'(x) = \dfrac{1}{\exp\big(\ln(x)\big)}  &  & \text{(by \cref{4.5.2}(b))}           \\
    \implies & \forall x \in (0, \infty), \ln'(x) = \dfrac{1}{x}.                    &  & \by{4.5.5}
  \end{align*}
  Since \(\dfrac{1}{x}\) is continuous on arbitrary interval \([a, b] \subseteq (0, \infty)\), by Corollary 11.5.2 in Analysis I we know that \(\dfrac{1}{x}\) is Riemann integrable on \([a, b]\).
  By the fundamental theorem of calculus (Theorem 11.9.4) we thus have
  \[
    \int_a^b \dfrac{1}{x} \; dx = \ln(b) - \ln(a).
  \]
\end{proof}

\begin{proof}{(b)}
  We have
  \begin{align*}
    \forall x, y \in (0, \infty), \ln(xy) & = \ln(e^{\ln(x)} e^{\ln(y)}) &  & \by{4.5.5}                  \\
                                          & = \ln(e^{\ln(x) + \ln(y)})   &  & \text{(by \cref{4.5.2}(d))} \\
                                          & = \ln(x) + \ln(y).           &  & \by{4.5.5}
  \end{align*}
\end{proof}

\begin{proof}{(c)}
  We have
  \begin{align*}
    \ln(1) & = \ln(e^0) &  & \text{(by \cref{4.5.2}(e))} \\
           & = 0        &  & \by{4.5.5}
  \end{align*}
  and
  \begin{align*}
             & \forall x \in (0, \infty), \ln(\dfrac{x}{x}) = 0 \text{(from the proof above)}                                  \\
    \implies & \forall x \in (0, \infty), \ln(x) + \ln(\dfrac{1}{x}) = 0                      &  & \text{(by \cref{4.5.6}(b))} \\
    \implies & \forall x \in (0, \infty), \ln(\dfrac{1}{x}) = -\ln(x).
  \end{align*}
\end{proof}

\begin{proof}{(d)}
  Let \(x \in (0, \infty)\).
  We know that \(x^y \in (0, \infty)\) for all \(y \in \R\), thus by \cref{4.5.5} \(\ln(x^y)\) is well-defined and we have
  \begin{align*}
    y \ln(x) & = \ln(e^{y \ln(x)})             &  & \by{4.5.5} \\
             & = \ln\big((e^{\ln(x)})^{y}\big)                 \\
             & = \ln(x^y).                     &  & \by{4.5.5}
  \end{align*}
  Since \(x\) is arbitrary, we conclude that
  \[
    \forall x \in (0, \infty), \forall y \in \R, y \ln(x) = \ln(x^y).
  \]
\end{proof}

\begin{proof}{(e)}
  Since
  \[
    x \in (-1, 1) \iff 1 - x \in (0, 2),
  \]
  by \cref{4.5.5} we know that \(\ln(1 - x)\) is well-defined.
  Observe that
  \begin{align*}
    \forall x \in (-1, 1), & \ln'(1 - x)                                                                                              \\
                           & = \big(\ln'(y)|_{y = 1 - x}\big) \times \big((y \mapsto 1 - y)'(x)\big) &  & \text{(by chain rule)}      \\
                           & = \dfrac{1}{1 - x} \times (-1)                                          &  & \text{(by \cref{4.5.6}(a))} \\
                           & = \dfrac{-1}{1 - x}
  \end{align*}
  and
  \begin{align*}
             & x \in (-1, 1)                                                                          \\
    \implies & \forall n \in \Z^+, x^n \in (-1, 1)                                                    \\
    \implies & \sum_{n = 0}^\infty x^n = \dfrac{1}{1 - x}. &  & \text{(by Lemma 7.3.3 in Analysis I)}
  \end{align*}

  First suppose that \(x = 0\).
  Then we have
  \begin{align*}
    \ln(1 - 0) & = \ln(1)                                                                 \\
               & = 0                                     &  & \text{(by \cref{4.5.6}(c))} \\
               & = - \sum_{n = 1}^\infty \dfrac{0^n}{n}.
  \end{align*}

  Now suppose that \(x \in (0, 1)\).
  Then we have
  \begin{align*}
             & - \sum_{n = 0}^\infty x^n = \ln'(1 - x)                                                                                                  \\
    \implies & \int_0^x \bigg(- \sum_{n = 0}^\infty y^n\bigg) \; dy = \int_0^x \ln'(1 - y) \; dy       &  & \text{(by \cref{4.5.6}(a))}                 \\
    \implies & - \sum_{n = 0}^\infty \dfrac{x^{n + 1} - 0^{n + 1}}{n + 1} = \int_0^x \ln'(1 - y) \; dy &  & \text{(by \cref{4.1.6}(e))}                 \\
    \implies & - \sum_{n = 0}^\infty \dfrac{x^{n + 1}}{n + 1} = \ln(1 - x) - \ln(1 - 0)                &  & \text{(by fundamental theorem of calculus)} \\
    \implies & - \sum_{n = 0}^\infty \dfrac{x^{n + 1}}{n + 1} = \ln(1 - x)                             &  & \text{(by \cref{4.5.6}(c))}                 \\
    \implies & - \sum_{n = 1}^\infty \dfrac{x^n}{n} = \ln(1 - x).
  \end{align*}

  Now suppose that \(x \in (-1, 0)\).
  Then we have
  \begin{align*}
             & - \sum_{n = 0}^\infty x^n = \ln'(1 - x)                                                                                                  \\
    \implies & \int_x^0 \bigg(- \sum_{n = 0}^\infty y^n\bigg) \; dy = \int_x^0 \ln'(1 - y) \; dy       &  & \text{(by \cref{4.5.6}(a))}                 \\
    \implies & - \sum_{n = 0}^\infty \dfrac{0^{n + 1} - x^{n + 1}}{n + 1} = \int_x^0 \ln'(1 - y) \; dy &  & \text{(by \cref{4.1.6}(e))}                 \\
    \implies & \sum_{n = 0}^\infty \dfrac{x^{n + 1}}{n + 1} = \ln(1 - 0) - \ln(1 - x)                  &  & \text{(by fundamental theorem of calculus)} \\
    \implies & \sum_{n = 0}^\infty \dfrac{x^{n + 1}}{n + 1} = -\ln(1 - x)                              &  & \text{(by \cref{4.5.6}(c))}                 \\
    \implies & - \sum_{n = 1}^\infty \dfrac{x^n}{n} = \ln(1 - x).
  \end{align*}

  Combine all proofs above we conclude that
  \begin{align*}
             & \forall x \in (-1, 1), \ln(1 - x) = - \sum_{n = 1}^\infty \dfrac{x^n}{n}                                                  \\
    \implies & \forall -x \in (-1, 1), \ln\big(1 - (-x)\big) = - \sum_{n = 1}^\infty \dfrac{(-x)^n}{n}                                   \\
    \implies & \forall -(x - 1) \in (-1, 1), \ln\Big(1 - \big(-(x - 1)\big)\Big) = - \sum_{n = 1}^\infty \dfrac{\big(-(x - 1)\big)^n}{n} \\
    \implies & \forall x \in (0, 2), \ln(x) = - \sum_{n = 1}^\infty \dfrac{(-1)^n (x - 1)^n}{n}                                          \\
    \implies & \forall x \in (0, 2), \ln(x) = \sum_{n = 1}^\infty \dfrac{(-1)^{n + 1} (x - 1)^n}{n}.
  \end{align*}
  By \cref{4.2.1} \(\ln\) is real analytic at \(1\) with radius of convergence \(1\).
\end{proof}

\begin{eg}\label{4.5.7}
  We now give a modest application of Abel's theorem (\cref{4.3.1}):
  from the alternating series test we see that \(\sum_{n = 1}^\infty \dfrac{(-1)^{n + 1}}{n}\) is convergent.
  By Abel's theorem we thus see that
  \[
    \sum_{n = 1}^\infty \dfrac{(-1)^{n + 1}}{n} = \lim_{x \to 2 ; x \in (0, 2)} \sum_{n = 1}^\infty \dfrac{(-1)^{n + 1}}{n} (x - 1)^n = \lim_{x \to 2 ; x \in (0, 2)} \ln(x) = \ln(2),
  \]
  thus we have the formula
  \[
    \ln(2) = 1 - \dfrac{1}{2} + \dfrac{1}{3} - \dfrac{1}{4} +\dfrac{1}{5} - \dots.
  \]
\end{eg}

\begin{ac}\label{ac:4.5.1}
  \(e^x > x\) for all \(x \in \R\).
\end{ac}

\begin{proof}
  We have \(e^0 = 1 > 0\).
  Suppose that \(x \in \R^+\).
  Then we have
  \begin{align*}
    e^x & = \sum_{n = 0}^\infty \dfrac{x^n}{n!}                                     &         & \by{4.5.1} \\
        & = \dfrac{x^0}{0!} + \dfrac{x^1}{1!} + \sum_{n = 2}^\infty \dfrac{x^n}{n!}                        \\
        & = 1 + x + \sum_{n = 2}^\infty \dfrac{x^n}{n!}                                                    \\
        & > 1 + x                                                                   & (x > 0)              \\
        & > x.
  \end{align*}
  Now suppose that \(x \in \R^-\).
  Then by \cref{4.5.2}(e) we have \(e^x > 0 > x\).
  Combine all proofs above we conclude that \(e^x > x\) for all \(x \in \R\).
\end{proof}

\exercisesection

\begin{ex}\label{ex:4.5.1}
  Prove \cref{4.5.2}.
\end{ex}

\begin{proof}
  See \cref{4.5.2}.
\end{proof}

\begin{ex}\label{ex:4.5.2}
  Show that for every integer \(n \geq 3\), we have
  \[
    0 < \dfrac{1}{(n + 1)!} + \dfrac{1}{(n + 2)!} + \dots < \dfrac{1}{n!}.
  \]
  Conclude that \(n! e\) is not an integer for every \(n \geq 3\).
  Deduce from this that \(e\) is irrational.
\end{ex}

\begin{proof}
  We first show that \((n + k)! > 2^k n!\) for all \(n \geq 3\) and \(k \in \Z^+\).
  We use induction on \(k\).
  For \(k = 1\), we have
  \[
    \forall n \geq 3, (n + 1)! = (n + 1) (n!) \geq 4 (n!) > 2^1 (n!)
  \]
  and the base case holds.
  Suppose inductively that \((n + k)! > 2^k n!\) for some \(k \geq 1\).
  Then for \(k + 1\), we have
  \begin{align*}
    \forall n \geq 3, (n + k + 1)! & = (n + k + 1) (n + k)!            \\
                                   & \geq (4 + k)(n + k)!              \\
                                   & > 2 (n + k)!                      \\
                                   & > 2 (2^k) (n!)         &  & \byIH \\
                                   & = 2^{k + 1} (n!)
  \end{align*}
  and this closes the induction.

  Next we show that
  \[
    \forall n \geq 3, 0 < \dfrac{1}{(n + 1)!} + \dfrac{1}{(n + 2)!} + \dots < \dfrac{1}{n!}.
  \]
  Since
  \begin{align*}
             & \forall n \geq 3, \forall k \geq 1, \dfrac{1}{(n + k)!} < \dfrac{1}{2^k (n!)}                                                                                                                        &  & \text{(from the proof above)} \\
    \implies & \forall n \geq 3,                                                                                                                                                                                                                       \\
             & \sum_{k = 1}^\infty \dfrac{1}{(n + k)!} \leq \sum_{k = 1}^\infty \dfrac{1}{2^k (n!)} = \dfrac{1}{n!} \sum_{k = 1}^\infty \dfrac{1}{2^k} = \dfrac{1}{2 (n!)} \sum_{k = 1}^\infty \dfrac{1}{2^{k - 1}} &  & \text{(geometric series)}     \\
             & = \dfrac{1}{2 (n!)} \sum_{k = 0}^\infty \dfrac{1}{2^k} = \dfrac{2}{2 (n!)} = \dfrac{1}{n!}                                                                                                                                              \\
    \implies & \forall n \geq 3, \sum_{k = 1}^\infty \dfrac{1}{(n + k)!} \leq \dfrac{1}{n!},
  \end{align*}
  we only need to show that
  \[
    \forall n \geq 3, \sum_{k = 1}^\infty \dfrac{1}{(n + k)!} \neq \dfrac{1}{n!}.
  \]
  So suppose for sake of contradiction that there exists some \(n \geq 3\) such that the identity above holds.
  Then we have
  \begin{align*}
             & \sum_{k = 1}^\infty \dfrac{1}{(n + k)!} = \dfrac{1}{n!} = \sum_{k = 1}^\infty \dfrac{1}{2^k (n!)} \\
    \implies & \sum_{k = 1}^\infty \bigg(\dfrac{1}{2^k (n!)} - \dfrac{1}{(n + k)!}\bigg) = 0.
  \end{align*}
  But we know that
  \[
    \forall k \geq 1, \dfrac{1}{2^k (n!)} - \dfrac{1}{(n + k)!} > 0 \implies \sum_{k = 1}^\infty \bigg(\dfrac{1}{2^k (n!)} - \dfrac{1}{(n + k)!}\bigg) > 0,
  \]
  a contradiction.
  Thus we have
  \[
    \forall n \geq 3, \sum_{k = 1}^\infty \dfrac{1}{(n + k)!} < \dfrac{1}{n!}.
  \]

  Now we show that \(n! e\) is not an integer for all \(n \geq 3\).
  Since
  \begin{align*}
             & \forall n \geq 3, \sum_{m = n + 1}^\infty \dfrac{1}{m!} < \dfrac{1}{n!} \\
    \implies & \forall n \geq 3, \sum_{m = n + 1}^\infty \dfrac{n!}{m!} < 1
  \end{align*}
  and
  \[
    \forall n \geq 3, \sum_{m = 0}^n \dfrac{n!}{m!} = \sum_{m = 0}^n (n - m)! \in \N,
  \]
  we have
  \begin{align*}
    \forall n \geq 3, n! e & = n! \sum_{m = 0}^\infty \dfrac{1}{m!}                                                        &  & \by{4.5.3} \\
                           & = \sum_{m = 0}^\infty \dfrac{n!}{m!}                                                                          \\
                           & = \sum_{m = 0}^n \dfrac{n!}{m!} + \sum_{m = n + 1}^\infty \dfrac{n!}{m!}                                      \\
                           & = \sum_{m = 0}^n (n - m)! + \sum_{m = n + 1}^\infty \dfrac{n!}{m!}                                            \\
                           & \in \Bigg(\bigg(\sum_{m = 0}^n (n - m)!\bigg), \bigg(\sum_{m = 0}^n (n - m)! + 1\bigg)\Bigg).
  \end{align*}
  Thus \(n! e\) is not an integer for all \(n \geq 3\).

  Finally we show that \(e\) is irrational.
  Suppose for sake of contradiction that \(e \in \Q\).
  Then we know that \(e = \dfrac{a}{b}\) for some \(a \in \Z\) and \(b \in \Z^+\).
  But then we have
  \begin{align*}
             & e = \dfrac{a}{b} = \dfrac{3a}{3b}                         \\
    \implies & (3b)! e = \dfrac{(3a) (3b)!}{3b} = (3a) (3b - 1)! \in \N,
  \end{align*}
  a contradiction.
  Thus \(e \in \R \setminus \Q\).
\end{proof}

\begin{ex}\label{ex:4.5.3}
  Prove \cref{4.5.4}.
\end{ex}

\begin{proof}
  See \cref{4.5.4}.
\end{proof}

\begin{ex}\label{ex:4.5.4}
  Let \(f : \R \to \R\) be the function defined by setting \(f(x) \coloneqq \exp(-1 / x)\) when \(x > 0\), and \(f(x) \coloneqq 0\) when \(x \leq 0\).
  Prove that \(f\) is infinitely differentiable, and \(f^{(k)}(0) = 0\) for every integer \(k \geq 0\), but that \(f\) is not real analytic at \(0\).
\end{ex}

\begin{proof}
  First we use induction on \(k\) to show that
  \[
    \forall k \in \N, \forall x \in \R^+, f^{(k)}(x) = P_k(x^{-1}) \exp(-x^{-1}) \text{ where } P_k(x) \text{ is some polynomial}.
  \]
  For \(k = 0\), we have
  \[
    \forall x \in \R^+, f^{(0)}(x) = f(x) = \exp(-x^{-1}) = (x^{-1})^0 \exp(-x^{-1}).
  \]
  Thus the base case holds.
  Suppose inductively that
  \[
    \forall x \in \R^+, f^{(k)}(x) = P_k(x^{-1}) \exp(-x^{-1}) \text{ where } P_k(x) \text{ is some polynomial}.
  \]
  for some \(k \geq 0\).
  Then for \(k + 1\), we have
  \begin{align*}
    \forall x \in \R^+, & f^{(k + 1)}(x)                                                  \\
                        & = (f^{(k)})'(x)                                                 \\
                        & = \bigg(y \mapsto P_k(y^{-1}) \exp(-x^{-1})\bigg)'(x)           \\
                        & = P_k'(x^{-1}) \exp(-x^{-1}) + P_k(x^{-1}) \exp(-x^{-1}) x^{-2} \\
                        & = \big(P_k'(x^{-1}) + P_k(x^{-1}) x^{-2}\big) \exp(-x^{-1})
  \end{align*}
  and this closes the induction.
  Thus \(f\) is infinitely differentiable on \(\R^+\).

  Since \(f(x) = 0\) for all \(x \in \R^-\), we know that \(f\) infinitely differentiable on \(\in \R^-\).
  So we only left to show that \(f\) is infinitely differentiable at \(0\).
  Again, we use induction on \(k\).
  For \(k = 0\), we have \(f^{(0)}(0) = f(0) = 0\).
  Thus the base case holds.
  Suppose inductively that \(f^{(k)}(0) = 0\) for some \(k \geq 0\).
  Then for \(k + 1\), we have
  \begin{align*}
     & \lim_{x \to 0 ; x \in \R^+} \dfrac{f^{(k)}(x) - f^{(k)}(0)}{x - 0}                                      \\
     & = \lim_{x \to 0 ; x \in \R^+} \dfrac{P_k(x^{-1}) \exp(-x^{-1}) - 0}{x} &  & \byIH                       \\
     & = \lim_{x \to 0 ; x \in \R^+} \dfrac{P_k(x^{-1}) x^{-1}}{\exp(x^{-1})} &  & \text{(by \cref{4.5.2}(e))} \\
     & = \lim_{x \to \infty ; x \in \R^+} \dfrac{P_k(x) x}{\exp(x)}                                            \\
     & = 0                                                                    &  & \by{ex:4.5.8}               \\
     & = \lim_{x \to 0 ; x \in \R^-} \dfrac{0 - 0}{x - 0}                                                      \\
     & = \lim_{x \to 0 ; x \in \R^-} \dfrac{f^{(k)}(x) - f^{(k)}(0)}{x - 0}
  \end{align*}
  and thus \(f^{(k + 1)}(0) = 0\).
  This closes the induction.

  Finally we show that \(f\) is not real analytic at \(0\).
  Suppose for sake of contradiction that \(f\) is real analytic at \(0\).
  Then by \cref{4.2.10} there exists an \(r \in \R^*\) such that
  \[
    \forall x \in (-r, r), f(x) = \sum_{n = 0}^\infty \dfrac{f^{(n)}(0)}{n!} x^n.
  \]
  But from the proof above we have
  \begin{align*}
    \sum_{n = 0}^\infty \dfrac{f^{(n)}(0)}{n!} (\dfrac{r}{2})^n = 0 \neq f(\dfrac{r}{2}) = \exp(\dfrac{-r}{2}) > 0.
  \end{align*}
  a contradiction.
  Thus \(f\) is not real analytic at \(0\).
\end{proof}

\begin{ex}\label{ex:4.5.5}
  Prove \cref{4.5.6}.
\end{ex}

\begin{proof}
  See \cref{4.5.6}.
\end{proof}

\begin{ex}\label{ex:4.5.6}
  Prove that the natural logarithm function is real analytic on \((0, +\infty)\).
\end{ex}

\begin{proof}
  Let \(a \in \R^+\).
  By \cref{4.5.6}(e) we know that \(\ln\) is real analytic at \(1\), thus
  \begin{align*}
             & \forall x \in (0, 2a), \dfrac{x}{a} \in (0, 2)                                                                                                            \\
    \implies & \forall x \in (0, 2a), \ln(\dfrac{x}{a}) = \sum_{n = 1}^\infty \dfrac{(-1)^{n + 1}}{n} \bigg(\dfrac{x}{a} - 1\bigg)^n &  & \text{(by \cref{4.5.6}(e))}    \\
    \implies & \forall x \in (0, 2a), \ln(x) - \ln(a) = \sum_{n = 1}^\infty \dfrac{(-1)^{n + 1}}{n} \bigg(\dfrac{x}{a} - 1\bigg)^n   &  & \text{(by \cref{4.5.6}(b)(c))} \\
    \implies & \forall x \in (0, 2a), \ln(x) = \ln(a) + \sum_{n = 1}^\infty \dfrac{(-1)^{n + 1}}{n} \bigg(\dfrac{x}{a} - 1\bigg)^n                                       \\
             & = \ln(a) (x - a)^0 + \sum_{n = 1}^\infty \dfrac{(-1)^{n + 1}}{n a^n} (x - a)^n                                                                            \\
    \implies & \ln \text{ is real analytic at } a \text{ with radius of convergence } a.                                             &  & \by{4.2.1}
  \end{align*}
  Since \(a\) is arbitrary, we conclude that \(\ln\) is real analytic on \(\R^+\).
\end{proof}

\begin{ex}\label{ex:4.5.7}
  Let \(f : \R \to (0, \infty)\) be a positive, real analytic function such that \(f'(x) = f(x)\) for all \(x \in \R\).
  Show that \(f(x) = C e^x\) for some positive constant \(C\);
  justify your reasoning.
\end{ex}

\begin{proof}
  Since \(f\) is real analytic on \(\R\), we have
  \begin{align*}
    \forall x \in \R, f(x) & = \sum_{n = 0}^\infty \dfrac{f^{(n)}(0)}{n!} (x - 0)^n &  & \by{4.2.10}            \\
                           & = \sum_{n = 0}^\infty \dfrac{f(0)}{n!} x^n             &  & \text{(by hypothesis)} \\
                           & = f(0) \bigg(\sum_{n = 0}^\infty \dfrac{x^n}{n!}\bigg)                             \\
                           & = f(0) e^x.                                            &  & \by{4.5.1}
  \end{align*}
\end{proof}

\begin{ex}\label{ex:4.5.8}
  Let \(m > 0\) be an integer.
  Show that
  \[
    \lim_{x \to +\infty} \dfrac{e^x}{x^m} = +\infty.
  \]
\end{ex}

\begin{proof}
  Let \(m \in \Z^+\).
  Observe that
  \begin{align*}
             & \lim_{n \to \infty} \dfrac{1}{n + 1} = 0                                      \\
    \implies & \lim_{n \to \infty} \bigg(1 - \dfrac{1}{n + 1}\bigg) = 1                      \\
    \implies & \lim_{n \to \infty} \dfrac{n}{n + 1} = 1                                      \\
    \implies & \lim_{n \to \infty} \bigg(\dfrac{n}{n + 1}\bigg)^m = 1^m = 1                  \\
    \implies & \lim_{n \to \infty} e \bigg(\dfrac{n}{n + 1}\bigg)^m = e                      \\
    \implies & \lim_{n \to \infty} \dfrac{e^{n + 1}}{e^n} \bigg(\dfrac{n}{n + 1}\bigg)^m = e
  \end{align*}
  and
  \[
    e = \exp(1) = \sum_{n = 0}^\infty \dfrac{1}{n!} = \dfrac{1}{0!} + \dfrac{1}{1!} + \sum_{n = 2}^\infty \dfrac{1}{n!} > 2.
  \]
  We know that
  \begin{align*}
             & \exists N \in \Z^+ : \forall n \geq N, \abs{\dfrac{e^{n + 1}}{e^n} \bigg(\dfrac{n}{n + 1}\bigg)^m - e} < \dfrac{e}{2}         & (e > 2) \\
    \implies & \exists N \in \Z^+ : \forall n \geq N, \abs{\dfrac{e^{n + 1}}{(n + 1)^m} - \dfrac{e^{n + 1}}{n^m}} < \dfrac{e^{n + 1}}{2 n^m}           \\
    \implies & \exists N \in \Z^+ : \forall n \geq N, \dfrac{e^{n + 1}}{2 n^m} < \dfrac{e^{n + 1}}{(n + 1)^m} < \dfrac{3 e^{n + 1}}{2 n^m}             \\
    \implies & \exists N \in \Z^+ : \forall n \geq N, \dfrac{e^n}{n^m} < \dfrac{e^{n + 1}}{2 n^m} < \dfrac{e^{n + 1}}{(n + 1)^m}             & (e > 2) \\
    \implies & \exists N \in \Z^+ : (\dfrac{e^n}{n^m})_{n = N}^\infty \text{ is monotone increasing sequence}.
  \end{align*}
  Fix such \(N\).
  Now we show that \(\sup(\dfrac{e^n}{n^m})_{n = N}^\infty = +\infty\).
  Since
  \begin{align*}
             & \forall n \geq N, \dfrac{e^{n + 1}}{(n + 1)^m} > \dfrac{e^{n + 1}}{2 n^m}                                                                                \\
    \implies & \forall n \geq N, \dfrac{e^{n + 1}}{(n + 1)^m} > \bigg(\dfrac{e}{2} - 1 + 1\bigg) \dfrac{e^n}{n^m}                                                       \\
    \implies & \forall n \geq N, \dfrac{e^{n + 1}}{(n + 1)^m} - \dfrac{e^n}{n^m} > \bigg(\dfrac{e}{2} - 1\bigg) \dfrac{e^n}{n^m}                                        \\
    \implies & \forall n \geq N, \dfrac{e^{n + 1}}{(n + 1)^m} - \dfrac{e^n}{n^m} > \bigg(\dfrac{e}{2} - 1\bigg) \dfrac{e^N}{N^m} > 0, &  & \text{(monotone increasing)}
  \end{align*}
  we know that
  \begin{align*}
    \forall n \geq N, & \dfrac{e^{n + 1}}{(n + 1)^m} - \dfrac{e^N}{N^m}                                                                 \\
                      & = \sum_{j = N}^n \bigg(\dfrac{e^{j + 1}}{(j + 1)^m} - \dfrac{e^j}{j^m}\bigg) &  & \text{(telescope series)}     \\
                      & > \sum_{j = N}^n \Bigg(\bigg(\dfrac{e}{2} - 1\bigg) \dfrac{e^N}{N^m}\Bigg)   &  & \text{(from the proof above)} \\
                      & = (n - N) \Bigg(\bigg(\dfrac{e}{2} - 1\bigg) \dfrac{e^N}{N^m}\Bigg)
  \end{align*}
  and
  \begin{align*}
             & \forall n \geq N, \dfrac{e^{n + 1}}{(n + 1)^m} - \dfrac{e^N}{N^m} > (n - N) \Bigg(\bigg(\dfrac{e}{2} - 1\bigg) \dfrac{e^N}{N^m}\Bigg)                                        \\
    \implies & \forall n \geq N, \dfrac{e^{n + 1}}{(n + 1)^m} > (n - N) \Bigg(\bigg(\dfrac{e}{2} - 1\bigg) \dfrac{e^N}{N^m}\Bigg) + \dfrac{e^N}{N^m}                                        \\
    \implies & \forall n \geq N, \dfrac{e^{n + 1}}{(n + 1)^m} > (n - N) \Bigg(\bigg(\dfrac{e}{2} - 1\bigg) \dfrac{e^N}{N^m}\Bigg) + \bigg(\dfrac{e}{2} - 1\bigg) \dfrac{e^N}{N^m} & (e < 3) \\
    \implies & \forall n \geq N, \dfrac{e^{n + 1}}{(n + 1)^m} > (n + 1 - N) \Bigg(\bigg(\dfrac{e}{2} - 1\bigg) \dfrac{e^N}{N^m}\Bigg)
  \end{align*}
  Since \(\dfrac{e^N}{N^m} > 0\), by Archimedean property we know that
  \begin{align*}
             & \forall \varepsilon \in \R^+, \exists K \in \Z^+ : K \Bigg(\bigg(\dfrac{e}{2} - 1\bigg) \dfrac{e^N}{N^m}\Bigg) > \varepsilon                                   \\
    \implies & \forall \varepsilon \in \R^+, \exists K \in \Z^+ : \dfrac{e^{N + K}}{(N + K)^m} > \varepsilon                                                                  \\
    \implies & \forall \varepsilon \in \R^+, \exists K \in \Z^+ : \dfrac{e^K}{K^m} > \varepsilon                                                                              \\
    \implies & \forall \varepsilon \in \R^+, \exists K \in \Z^+ : \forall n \geq K, \dfrac{e^n}{n^m} > \varepsilon                          &  & \text{(monotone increasing)} \\
    \implies & \lim_{n \to \infty} \dfrac{e^n}{n^m} = +\infty.
  \end{align*}
  Since \(m\) is arbitrary, we conclude that
  \[
    \forall m \in \Z^+, \lim_{n \to \infty} \dfrac{e^n}{n^m} = +\infty.
  \]
\end{proof}

\begin{ex}\label{ex:4.5.9}
  Let \(P(x)\) be a polynomial, and let \(c > 0\).
  Show that there exists a real number \(N > 0\) such that \(e^{cx} > \abs{P(x)}\) for all \(x > N\);
  thus an exponentially growing function, no matter how small the growth rate \(c\), will eventually overtake any given polynomial \(P(x)\), no matter how large.
\end{ex}

\begin{proof}
  By \cref{3.8.1} we know that
  \[
    \forall x \in \R, P(x) = \sum_{i = 0}^m a_i x^i
  \]
  for some \(m \in \N\).
  Fix such \(m\).
  Since \(m\) is finite, \(a = \max_{i = 0}^m \abs{a_i}\) is well-defined.
  Let \(b = \ceil{a} + 1 \in \Z^+\).
  By \cref{ex:4.5.8} we have
  \begin{align*}
             & \forall 0 \leq i \leq m, \exists N_i \in \Z^+ : \forall n \geq N_i, \dfrac{e^n}{n^i} > 1       \\
    \implies & \forall 0 \leq i \leq m, \exists N_i \in \Z^+ : \forall n \geq N_i, \dfrac{e^{bn}}{(bn)^i} > 1 \\
    \implies & \forall 0 \leq i \leq m, \exists N_i \in \Z^+ : \forall n \geq N_i, e^{bn} > (bn)^i.
  \end{align*}
  Let \(K = \max_{i = 0}^m(N_i) + 1\).
  Then we have
  \begin{align*}
             & \forall n \geq K, e^{bn} > (bn)^m                                                                                        \\
    \implies & \forall x \in (K, +\infty), e^{bx} > (bx)^m > (ax)^m                                                                     \\
    \implies & \forall x \in (K, +\infty),                                                                                              \\
             & e^{b (m + 1) x} > \big(b (m + 1) x\big)^m > \big(a (m + 1) x\big)^m                &  & \text{(by Archimedean property)} \\
    \implies & \forall x \in (K, +\infty),                                                                                              \\
             & e^{b (m + 1) x} > \big(a (m + 1) x\big)^m = \sum_{i = 0}^m (ax)^m \geq \abs{P(x)}.
  \end{align*}
  By setting \(N = \dfrac{b K (m + 1)}{c}\) we are done.
\end{proof}

\begin{ex}\label{ex:4.5.10}
  Let \(f : (0, +\infty) \times \R \to \R\) be the exponential function \(f(x, y) \coloneqq x^y\).
  Show that \(f\) is continuous.
\end{ex}

\begin{proof}
  Let \(d = d_{l^1}|_{(\R^+ \times \R) \times (\R^+ \times \R)}\).
  We have
  \begin{align*}
    \forall (x, y) \in \R^+ \times \R, f(x, y) & = x^y                                                     \\
                                               & = \exp\big(\ln(x^y)\big) &  & \by{4.5.5}                  \\
                                               & = \exp(y \ln(x)).        &  & \text{(by \cref{4.5.6}(d))}
  \end{align*}
  Let \((x_n, y_n)_{n = 1}^\infty\) be a sequence in \(\R^+ \times \R\) such that
  \[
    \lim_{n \to \infty} d\big((x_n, y_n), (x, y)\big) = 0.
  \]
  By Proposition 9.8.3 in Analysis I and \cref{4.5.5} we know that \(\ln\) is continuous on \(\R^+\).
  Thus we have
  \begin{align*}
             & \lim_{n \to \infty} \ln(x_n) = \ln(x)                                    &  & \text{(by \cref{2.1.4}(a)(b))} \\
    \implies & \lim_{n \to \infty} y_n \ln(x_n) = y \ln(x)                              &  & \by{2.2.2}                     \\
    \implies & \lim_{n \to \infty} \exp\big(y_n \ln(x_n)\big) = \exp\big(y \ln(x)\big). &  & \text{(by \cref{4.5.2}(c))}
  \end{align*}
  Since \((x_n, y_n)_{n = 1}^\infty\) is arbitrary, by \cref{2.1.4}(a)(b) we conclude that \(f\) is continuous at \((x, y)\) from \((\R^+ \times \R, d)\) to \((\R, d_{l^1}|_{\R \times \R})\).
  Since \(x, y\) are arbitrary, we conclude that \(f\) is continuous on \(\R^+ \times \R\) from \((\R^+ \times \R, d)\) to \((\R, d_{l^1}|_{\R \times \R})\).
\end{proof}
\section{A digression on complex numbers}\label{ii:sec:4.6}

\setcounter{thm}{1}
\begin{defn}[Formal definition of complex numbers]\label{ii:4.6.2}
  A \emph{complex number} is any pair of the form \((a, b)\), where \(a, b\) are real numbers.
  Two complex numbers \((a, b)\), \((c, d)\) are said to be equal iff \(a = c\) and \(b = d\).
  The set of all complex numbers is denoted \(\C\).
\end{defn}

\begin{ac}\label{ii:ac:4.6.1}
  \cref{ii:4.6.2} is reflexive, symmetry and transitive.
\end{ac}

\begin{proof}
  Let \((a_1, b_1), (a_2, b_2), (a_3, b_3)\) be complex numbers.
  By \cref{ii:4.6.2} we know that \(a_1, a_2, a_3, b_1, b_2, b_3 \in \R\).
  Since
  \begin{align*}
             & (a_1 = a_1) \land (b_1 = b_1) & (a_1, b_1 \in \R)                 \\
    \implies & (a_1, b_1) = (a_1, b_1),      &                   & \by{ii:4.6.2}
  \end{align*}
  we know that \cref{ii:4.6.2} is reflexive.
  Since
  \begin{align*}
         & (a_1, b_1) = (a_2, b_2)                                                     \\
    \iff & (a_1 = a_2) \land (b_1 = b_2) &                             & \by{ii:4.6.2} \\
    \iff & (a_2 = a_1) \land (b_2 = b_1) & (a_1, a_2, b_1, b_2 \in \R)                 \\
    \iff & (a_2, b_2) = (a_1, b_1),      &                             & \by{ii:4.6.2}
  \end{align*}
  we know that \cref{ii:4.6.2} is symmetry.
  Since
  \begin{align*}
         & \big((a_1, b_1) = (a_2, b_2)\big) \land \big((a_2, b_2) = (a_3, b_3)\big)                                                         \\
    \iff & (a_1 = a_2) \land (b_1 = b_2) \land (a_2 = a_3) \land (b_2 = b_3)         &                                       & \by{ii:4.6.2} \\
    \iff & (a_1 = a_3) \land (b_1 = b_3)                                             & (a_1, a_2, a_3, b_1, b_2, b_3 \in \R)                 \\
    \iff & (a_1, b_1) = (a_3, b_3),                                                  &                                       & \by{ii:4.6.2}
  \end{align*}
  we know that \cref{ii:4.6.2} is transitive.
\end{proof}

\begin{note}
  At this stage the complex numbers \(\C\) are indistinguishable from the Cartesian product \(\R^2 = \R \times \R\)
  (also known as the \emph{Cartesian plane}).
  However, we will introduce a number of operations on the complex numbers, notably that of \emph{complex multiplication}, which are not normally placed on the Cartesian plane \(\R^2\).
  Thus, one can think of the complex number system \(\C\) as the Cartesian plane \(\R^2\) equipped with a number of additional structures.
\end{note}

\begin{defn}[Complex addition, negation, and zero]\label{ii:4.6.3}
  If \(z = (a, b)\) and \(w = (c, d)\) are two complex numbers, we define their \emph{sum} \(z + w\) to be the complex number \(z + w \coloneqq (a + c, b + d)\).
  We also define the \emph{negation} \(-z\) of \(z\) to be the complex number \(-z \coloneqq (-a, -b)\).
  We also define the \emph{complex zero} \(0_{\C}\) to be the complex number \(0_{\C} = (0, 0)\).
\end{defn}

\begin{ac}\label{ii:ac:4.6.2}
  If \(w, w', z, z' \in \C\) and \(w = w'\) and \(z = z'\), then \(w + z = w' + z'\) and \(-w = -w'\).
\end{ac}

\begin{proof}
  Let \(w = (a, b), w' = (a', b'), z = (c, d), z' = (c', d')\).
  By \cref{ii:4.6.2} we know that \(a, a', b, b', c, c', d, d' \in \R\).
  By \cref{ii:ac:4.6.1} we know that
  \begin{align*}
    a & = a'; \\
    b & = b'; \\
    c & = c'; \\
    d & = d'.
  \end{align*}
  Then we have
  \begin{align*}
    w + z & = (a + c, b + d)     &                                     & \by{ii:4.6.3} \\
          & = (a' + c', b' + d') & (a, a', b, b', c, c', d, d' \in \R)                 \\
          & = w' + z'            &                                     & \by{ii:4.6.3}
  \end{align*}
  and
  \begin{align*}
    -w & = (-a, -b)   &                       & \by{ii:4.6.3} \\
       & = (-a', -b') & (a, a', b, b' \in \R)                 \\
       & = -w'.       &                       & \by{ii:4.6.3}
  \end{align*}
\end{proof}

\begin{lem}[The complex numbers are an additive group]\label{ii:4.6.4}
  If \(z_1, z_2, z_3\) are complex numbers, then we have the commutative property \(z_1 + z_2 = z_2 + z_1\), the associative property \((z_1 + z_2) + z_3 = z_1 + (z_2 + z_3)\), the identity property \(z_1 + 0_{\C} = 0_{\C} + z_1 = z_1\), and the inverse property \(z_1 + (-z_1) = (-z_1) + z_1 = 0_{\C}\).
\end{lem}

\begin{proof}
  Let \(z_1 = (a, b), z_2 = (c, d), z_3 = (e, f)\).
  By \cref{ii:4.6.2} we know that \(a, b, c, d, e, f \in \R\).
  Since
  \begin{align*}
    z_1 + z_2 & = (a + c, b + d) &                     & \by{ii:4.6.3} \\
              & = (c + a, d + b) & (a, b, c, d \in \R)                 \\
              & = z_2 + z_1,     &                     & \by{ii:4.6.3}
  \end{align*}
  we know that the addition operation in \cref{ii:4.6.3} is commutative.
  Since
  \begin{align*}
    (z_1 + z_2) + z_3 & = (a + c, b + d) + z_3               &                           & \by{ii:4.6.3} \\
                      & = \big((a + c) + e, (b + d) + f\big) &                           & \by{ii:4.6.3} \\
                      & = \big(a + (c + e), b + (d + f)\big) & (a, b, c, d, e, f \in \R)                 \\
                      & = z_1 + (c + e, d + f)               &                           & \by{ii:4.6.3} \\
                      & = z_1 + (z_2 + z_3),                 &                           & \by{ii:4.6.3}
  \end{align*}
  we know that the addition operation in \cref{ii:4.6.3} is associative.
  Since
  \begin{align*}
    0_{\C} + z_1 & = z_1 + 0_{\C}   &                  & \text{(from the proof above)} \\
                 & = (a + 0, b + 0) &                  & \by{ii:4.6.3}                 \\
                 & = (a, b)         & (a, b, 0 \in \R)                                 \\
                 & = z_1
  \end{align*}
  and
  \begin{align*}
    (-z_1) + z_1 & = z_1 + (-z_1)                 &               & \text{(from the proof above)} \\
                 & = \big(a + (-a), b + (-b)\big) &               & \by{ii:4.6.3}                 \\
                 & = (0, 0)                       & (a, b \in \R)                                 \\
                 & = 0_{\C},                      &               & \by{ii:4.6.3}
  \end{align*}
  we know that \(0_{\C}\) is the additive identity in \(\C\).
\end{proof}

\begin{defn}[Complex multiplication]\label{ii:4.6.5}
  If \(z = (a, b)\) and \(w = (c, d)\) are complex numbers, then we define their \emph{product} \(zw\) to be the complex number \(zw \coloneqq (ac - bd, ad + bc)\).
  We also introduce the \emph{complex identity} \(1_{\C} \coloneqq (1, 0)\).
\end{defn}

\begin{ac}\label{ii:ac:4.6.3}
  If \(w, w', z, z' \in \C\) and \(w = w'\) and \(z = z'\), then \(wz = w'z'\).
\end{ac}

\begin{proof}
  Let \(w = (a, b), w' = (a', b'), z = (c, d), z' = (c', d')\).
  By \cref{ii:4.6.2} we know that \(a, a', b, b', c, c', d, d' \in \R\).
  By \cref{ii:ac:4.6.1} we know that
  \begin{align*}
    a & = a'; \\
    b & = b'; \\
    c & = c'; \\
    d & = d'.
  \end{align*}
  Then we have
  \begin{align*}
    wz & = (ac - bd, ad + bc)             &                                     & \by{ii:4.6.5} \\
       & = (a' c' - b' d', a' d' + b' c') & (a, a', b, b', c, c', d, d' \in \R)                 \\
       & = w' z'.                         &                                     & \by{ii:4.6.5}
  \end{align*}
\end{proof}

\begin{lem}\label{ii:4.6.6}
  If \(z_1, z_2, z_3\) are complex numbers, then we have the commutative property \(z_1 z_2 = z_2 z_1\), the associative property \((z_1 z_2) z_3 = z_1 (z_2 z_3)\), the identity property \(z_1 1_{\C} = 1_{\C} z_1 = z_1\), and the distributivity properties \(z_1 (z_2 + z_3) = z_1 z_2 + z_1 z_3\) and \((z_2 + z_3) z_1 = z_2 z_1 + z_3 z_1\).
\end{lem}

\begin{proof}
  Let \(z_1 = (a, b), z_2 = (c, d), z_3 = (e, f)\).
  By \cref{ii:4.6.2} we know that \(a, b, c, d, e, f \in \R\).
  Since
  \begin{align*}
    z_1 z_2 & = (ac - bd, ad + bc) &                     & \by{ii:4.6.5} \\
            & = (ca - db, da + cb) & (a, b, c, d \in \R)                 \\
            & = z_2 z_1,           &                     & \by{ii:4.6.5}
  \end{align*}
  we know that the multiplication operation in \cref{ii:4.6.5} is commutative.
  Since
  \begin{align*}
     & (z_1 z_2) z_3                                                                                                \\
     & = (ac - bd, ad + bc) z_3                                         &                           & \by{ii:4.6.5} \\
     & = \big((ac - bd) e - (ad + bc) f, (ac - bd) f + (ad + bc) e\big) &                           & \by{ii:4.6.5} \\
     & = \big(a (ce - df) - b (cf + de), a (cf + de) + b (ce - df)\big) & (a, b, c, d, e, f \in \R)                 \\
     & = z_1 (ce - df, cf + de)                                         &                           & \by{ii:4.6.5} \\
     & = z_1 (z_2 z_3),                                                 &                           & \by{ii:4.6.5}
  \end{align*}
  we know that the multiplication operation in \cref{ii:4.6.5} is associative.
  Since
  \begin{align*}
    1_{\C} z_1 & = z_1 1_{\C}             &                  & \text{(from the proof above)} \\
               & = (a 1 - b 0, a 0 + b 1) &                  & \by{ii:4.6.5}                 \\
               & = (a, b)                 & (a, b, 1 \in \R)                                 \\
               & = z_1,
  \end{align*}
  we know that \(1_{\C}\) is the multiplicative identity in \(\C\).
  Since
  \begin{align*}
    z_1 (z_2 + z_3) & = z_1 (c + e, d + f)                                      &                           & \by{ii:4.6.3} \\
                    & = \big(a (c + e) - b (d + f), a (d + f) + b (c + e)\big)  &                           & \by{ii:4.6.5} \\
                    & = \big((ac - bd) + (ae - bf), (ad + bc) + (af + be) \big) & (a, b, c, d, e, f \in \R)                 \\
                    & = (ac - bd, ad + bc) + (ae - bf, af + be)                 &                           & \by{ii:4.6.3} \\
                    & = z_1 z_2 + z_1 z_3                                       &                           & \by{ii:4.6.3}
  \end{align*}
  and
  \begin{align*}
    (z_2 + z_3) z_1 & = z_1 (z_2 + z_3)    &  & \text{(from the proof above)} \\
                    & = z_1 z_2 + z_1 z_3  &  & \text{(from the proof above)} \\
                    & = z_2 z_1 + z_3 z_1, &  & \text{(from the proof above)}
  \end{align*}
  we know that the multiplication operation in \cref{ii:4.6.5} and the addition operation in \cref{ii:4.6.3} are distributive.
\end{proof}

\begin{note}
  \cref{ii:4.6.6} can also be stated more succinctly, as the assertion that \(\C\) is a commutative ring.
  As is usual, we now write \(z - w\) as shorthand for \(z + (-w)\).
\end{note}

\begin{note}
  We now identify the real numbers \(\R\) with a subset of the complex numbers \(\C\) by identifying any real number \(x\) with the complex number \((x, 0)\), thus \(x \equiv (x, 0)\).
  Note that this identification is consistent with equality (thus \(x = y\) iff \((x, 0) = (y, 0)\)), with addition (\(x_1 + x_2 = x_3\) iff \((x_1, 0) + (x_2, 0) = (x_3, 0)\)), with negation (\(x = -y\) iff \((x, 0) = -(y, 0)\)), and multiplication (\(x_1 x_2 = x_3\) iff \((x_1, 0) (x_2, 0) = (x_3, 0)\)), so we will no longer need to distinguish between ``real addition'' and ``complex addition'', and similarly for equality, negation, and multiplication.
  Note also that \(0 \equiv 0_\C\) and \(1 \equiv 1_\C\), so we can now drop the \(\C\) subscripts from the zero \(0\) and the identity \(1\).
\end{note}

\begin{note}
  We now define \(i\) to be the complex number \(i \coloneqq (0, 1)\).
\end{note}

\begin{lem}\label{ii:4.6.7}
  Every complex number \(z \in \C\) can be written as \(z = a + bi\) for exactly one pair \(a, b\) of real numbers.
  Also, we have \(i^2 = -1\), and \(-z = (-1)z\).
\end{lem}

\begin{proof}
  Let \(z = (a, b)\).
  We have
  \begin{align*}
    a + bi & = (a, 0) + (b, 0) \times (0, 1)                    \\
           & = (a, 0) + (0, b)               &  & \by{ii:4.6.5} \\
           & = (a, b)                        &  & \by{ii:4.6.3} \\
           & = z.
  \end{align*}
  If \(z' = (a', b')\) such that \(z = z'\), then we have
  \begin{align*}
    z' & = (a', b')                                              \\
       & = (a', 0) + (0, b')               &  & \by{ii:4.6.3}    \\
       & = (a', 0) + (b', 0) \times (0, 1) &  & \by{ii:4.6.5}    \\
       & = a' + b' i                                             \\
       & = a + bi.                         &  & \by{ii:ac:4.6.1}
  \end{align*}

  Now we show that \(i \times i = -1\).
  \begin{align*}
    i \times i & = (0, 1) \times (0, 1)                                    \\
               & = (0^2 - 1^2, 0 \times 1 + 1 \times 0) &  & \by{ii:4.6.5} \\
               & = (-1, 0)                                                 \\
               & = -(1, 0)                              &  & \by{ii:4.6.3} \\
               & = -1.
  \end{align*}

  Finally we show that \(-z = (-1) z\).
  \begin{align*}
    -z & = -(a, b)                                  \\
       & = (-a, -b)              &  & \by{ii:4.6.3} \\
       & = (-1, 0) \times (a, b) &  & \by{ii:4.6.5} \\
       & = (-1) z.
  \end{align*}
\end{proof}

\begin{note}
  Because \cref{ii:4.6.7}, we will now refer to complex numbers in the more usual notation \(a + bi\), and discard the formal notation \((a, b)\) henceforth.
\end{note}

\begin{defn}[Real and imaginary parts]\label{ii:4.6.8}
  If \(z\) is a complex number with the representation \(z = a + bi\) for some real numbers \(a, b\), we shall call \(a\) the \emph{real part} of \(z\) and denote \(\Re(z) \coloneqq a\), and call \(b\) the \emph{imaginary part} of \(z\) and denote \(\Im(z) \coloneqq b\).
  In general \(z = \Re(z) + i \Im(z)\).
  Note that \(z\) is real iff \(\Im(z) = 0\).
  We say that \(z\) is \emph{imaginary} iff \(\Re(z) = 0\).
  \(0\) is both real and imaginary.
  We define the \emph{complex conjugate} \(\overline{z}\) of \(z\) to be the complex number \(\overline{z} \coloneqq \Re(z) - i \Im(z)\).
\end{defn}

\begin{ac}\label{ii:ac:4.6.4}
  If \(z, z' \in \C\) such that \(z = z'\), then \(\Re(z) = \Re(z')\), \(\Im(z) = \Im(z')\) and \(\overline{z} = \overline{z'}\).
\end{ac}

\begin{proof}
  Let \(z = a + bi\) and \(z' = a' + b' i\).
  By \cref{ii:4.6.7} we know that \(a = a'\) and \(b = b'\).
  Thus, by \cref{ii:4.6.8} we have
  \begin{align*}
     & \Re(z) = a = a' = \Re(z')                         \\
     & \Im(z) = b = b' = \Im(z')                         \\
     & \overline{z} = a - bi = a' - b' i = \overline{z'}
  \end{align*}
\end{proof}

\begin{lem}[Complex conjugation is an involution]\label{ii:4.6.9}
  Let \(z, w\) be complex numbers, then \(\overline{z + w} = \overline{z} + \overline{w}\), \(\overline{-z} = -\overline{z}\), and \(\overline{zw} = \overline{z} \; \overline{w}\).
  Also \(\overline{\overline{z}} = z\).
  Finally, we have \(\overline{z} = \overline{w}\) iff \(z = w\), and \(\overline{z} = z\) iff \(z\) is real.
\end{lem}

\begin{proof}
  First we show that \(\overline{z + w} = \overline{z} + \overline{w}\).
  \begin{align*}
    \overline{z + w} & = \overline{\Re(z) + i \Im(z) + \Re(w) + i \Im(w)}                     &  & \by{ii:4.6.8} \\
                     & = \overline{\big(\Re(z) + \Re(w)\big) + \big(i \Im(z) + i \Im(w)\big)} &  & \by{ii:4.6.4} \\
                     & = \overline{\big(\Re(z) + \Re(w)\big) + i \big(\Im(z) + \Im(w)\big)}   &  & \by{ii:4.6.6} \\
                     & = \big(\Re(z) + \Re(w)\big) - i \big(\Im(z) + \Im(w)\big)              &  & \by{ii:4.6.8} \\
                     & = \big(\Re(z) + \Re(w)\big) + (-1) i \big(\Im(z) + \Im(w)\big)         &  & \by{ii:4.6.7} \\
                     & = \big(\Re(z) + \Re(w)\big) + \big((-1) i \Im(z) + (-1) i \Im(w)\big)  &  & \by{ii:4.6.6} \\
                     & = \big(\Re(z) + (-1) i \Im(z)\big) + \big(\Re(w) + (-1) i \Im(w)\big)  &  & \by{ii:4.6.4} \\
                     & = \big(\Re(z) - i \Im(z)\big) + \big(\Re(w) - i \Im(w)\big)            &  & \by{ii:4.6.7} \\
                     & = \overline{\Re(z) + i \Im(z)} + \overline{\Re(w) + i \Im(w)}          &  & \by{ii:4.6.8} \\
                     & = \overline{z} + \overline{w}.                                         &  & \by{ii:4.6.8}
  \end{align*}

  Next we show that \(\overline{-z} = -\overline{z}\).
  \begin{align*}
    \overline{-z} & = \overline{-\Re(z) - i \Im(z)}          &  & \by{ii:4.6.8} \\
                  & = \overline{(-1) \Re(z) + (-1) i \Im(z)} &  & \by{ii:4.6.7} \\
                  & = (-1) \Re(z) - (-1) i \Im(z)            &  & \by{ii:4.6.8} \\
                  & = (-1) \big(\Re(z) - i \Im(z)\big)       &  & \by{ii:4.6.6} \\
                  & = (-1) \overline{\Re(z) + i \Im(z)}      &  & \by{ii:4.6.8} \\
                  & = (-1) \overline{z}                      &  & \by{ii:4.6.8} \\
                  & = -\overline{z}.                         &  & \by{ii:4.6.7}
  \end{align*}

  Next we show that \(\overline{zw} = \overline{z} \; \overline{w}\).
  \begin{align*}
    \overline{zw} & = \overline{\big(\Re(z) + i \Im(z)\big) \times \big(\Re(w) + i \Im(w)\big)}            &  & \by{ii:4.6.8} \\
                  & = \overline{\Re(z) \Re(w) - \Im(z) \Im(w) + i \big(\Re(z) \Im(w) + \Im(z) \Re(w)\big)} &  & \by{ii:4.6.5} \\
                  & = \Re(z) \Re(w) - \Im(z) \Im(w) - i \big(\Re(z) \Im(w) + \Im(z) \Re(w)\big)            &  & \by{ii:4.6.8} \\
                  & = \big(\Re(z) - i \Im(z)\big) \times \big(\Re(w) - i \Im(w)\big)                       &  & \by{ii:4.6.5} \\
                  & = \overline{\Re(z) + i \Im(z)} \; \overline{\Re(w) + i \Im(w)}                         &  & \by{ii:4.6.8} \\
                  & = \overline{z} \; \overline{w}.                                                        &  & \by{ii:4.6.5}
  \end{align*}

  Next we show that \(\overline{\overline{z}} = z\).
  \begin{align*}
    \overline{\overline{z}} & = \overline{\overline{\Re(z) + i \Im(z)}} &  & \by{ii:4.6.8} \\
                            & = \overline{\Re(z) - i \Im(z)}            &  & \by{ii:4.6.8} \\
                            & = \overline{\Re(z) + (-1) i \Im(z)}       &  & \by{ii:4.6.7} \\
                            & = \overline{\Re(z) + i (-1) \Im(z)}       &  & \by{ii:4.6.6} \\
                            & = \Re(z) - i (-1) \Im(z)                  &  & \by{ii:4.6.8} \\
                            & = \Re(z) + (-1) i (-1) \Im(z)             &  & \by{ii:4.6.7} \\
                            & = \Re(z) + (-1) (-1) i \Im(z)             &  & \by{ii:4.6.7} \\
                            & = \Re(z) + i \Im(z)                       &  & \by{ii:4.6.5} \\
                            & = z.                                      &  & \by{ii:4.6.8}
  \end{align*}

  Next we show that \(\overline{z} = \overline{w} \iff z = w\).
  \begin{align*}
             & \overline{z} = \overline{w}                                                          \\
    \implies & \overline{\overline{z}} = \overline{\overline{w}} &  & \by{ii:ac:4.6.4}              \\
    \implies & z = w                                             &  & \text{(from the proof above)} \\
    \implies & \overline{z} = \overline{w}.                      &  & \by{ii:ac:4.6.4}
  \end{align*}

  Finally we show that \(\overline{z} = z \iff \Im(z) = 0\).
  \begin{align*}
         & \overline{z} = z                                                                                 \\
    \iff & \overline{\Re(z) + i \Im(z)} = \Re(z) + i \Im(z)               &                 & \by{ii:4.6.8} \\
    \iff & \Re(z) - i \Im(z) = \Re(z) + i \Im(z)                          &                 & \by{ii:4.6.8} \\
    \iff & \Re(z) + (-1) i \Im(z) = \Re(z) + i \Im(z)                     &                 & \by{ii:4.6.7} \\
    \iff & \Re(z) + i (-1) \Im(z) = \Re(z) + i \Im(z)                     &                 & \by{ii:4.6.6} \\
    \iff & \big(\Re(z) = \Re(z)\big) \land \big((-1) \Im(z) = \Im(z)\big) &                 & \by{ii:4.6.2} \\
    \iff & (-1) \Im(z) = \Im(z)                                                                             \\
    \iff & \Im(z) = 0.                                                    & (\Im(z) \in \R)
  \end{align*}
\end{proof}

\begin{note}
  We cannot extend the definition of absolute value directly to the complex numbers, as most complex numbers are neither positive nor negative.
  (For instance, we do not classify \(i\) as either a positive or negative number)
\end{note}

\begin{defn}[Complex absolute value]\label{ii:4.6.10}
  If \(z = a + bi\) is a complex number, we define the \emph{absolute value} \(\abs{z}\) of \(z\) to be the real number \(\abs{z} \coloneqq \sqrt{a^2 + b^2} = (a^2 + b^2)^{1 / 2}\).
\end{defn}

\begin{note}
  From Exercise 5.6.3 in Analysis I we see that \cref{ii:4.6.10} generalizes the notion of real absolute value.
\end{note}

\begin{lem}[Properties of complex absolute value]\label{ii:4.6.11}
  Let \(z, w\) be complex numbers.
  Then \(\abs{z}\) is a non-negative real number, and \(\abs{z} = 0\) iff \(z = 0\).
  Also we have the identity \(z \overline{z} = \abs{z}^2\), and so \(\abs{z} = \sqrt{z \overline{z}}\).
  As a consequence we have \(\abs{zw} = \abs{z} \abs{w}\) and \(\abs{\overline{z}} = \abs{z}\).
  Finally, we have the inequalities
  \[
    -\abs{z} \leq \Re(z) \leq \abs{z}; \quad -\abs{z} \leq \Im(z) \leq \abs{z}; \quad \abs{z} \leq \abs{\Re(z)} + \abs{\Im(z)}
  \]
  as well as the triangle inequality \(\abs{z + w} \leq \abs{z} + \abs{w}\).
\end{lem}

\begin{proof}
  We have
  \begin{align*}
    \abs{z} & = \abs{\Re(z) + i \Im(z)}                        &                         & \by{ii:4.6.8}  \\
            & = \sqrt{\big(\Re(z)\big)^2 + \big(\Im(z)\big)^2} &                         & \by{ii:4.6.10} \\
            & \geq 0                                           & (\Re(z), \Im(z) \in \R)
  \end{align*}
  and
  \begin{align*}
         & \abs{z} = 0                                                                                   \\
    \iff & \abs{\Re(z) + i \Im(z)} = 0                        &                         & \by{ii:4.6.8}  \\
    \iff & \sqrt{\big(\Re(z)\big)^2 + \big(\Im(z)\big)^2} = 0 &                         & \by{ii:4.6.10} \\
    \iff & \big(\Re(z) = 0\big) \land \big(\Im(z) = 0\big)    & (\Re(z), \Im(z) \in \R)                  \\
    \iff & z = 0.                                             &                         & \by{ii:4.6.2}
  \end{align*}
  Since
  \begin{align*}
    z \overline{z} & = \big(\Re(z) + i \Im(z)\big) \times \big(\overline{\Re(z) + i \Im(z)}\big)                     &                                         & \by{ii:4.6.8}  \\
                   & = \big(\Re(z) + i \Im(z)\big) \times \big(\Re(z) - i \Im(z)\big)                                &                                         & \by{ii:4.6.8}  \\
                   & = \big(\Re(z) + i \Im(z)\big) \times \big(\Re(z) + (-1) i \Im(z)\big)                           &                                         & \by{ii:4.6.7}  \\
                   & = \big(\Re(z) + i \Im(z)\big) \times \big(\Re(z) + i (-1) \Im(z)\big)                           &                                         & \by{ii:4.6.6}  \\
                   & = \big(\Re(z)\big)^2 - (-1) \big(\Im(z)\big)^2 + i \big((-1) \Re(z) \Im(z) + \Re(z) \Im(z)\big) &                                         & \by{ii:4.6.5}  \\
                   & = \big(\Re(z)\big)^2 + \big(\Im(z)\big)^2                                                       & (\Re(z), \Im(z) \in \R)                                  \\
                   & = \Big(\sqrt{\big(\Re(z)\big)^2 + \big(\Im(z)\big)^2}\Big)^2                                    & \big(\Re(z)\big)^2 + \big(\Im(z)\big)^2                  \\
                   & = \abs{z}^2,                                                                                    &                                         & \by{ii:4.6.10}
  \end{align*}
  we know that \(\abs{z} = \sqrt{\abs{z}^2} = \sqrt{z \overline{z}}\).
  Thus
  \begin{align*}
    \abs{z} \abs{w} & = \sqrt{z \overline{z}} \sqrt{w \overline{w}} &                                         & \text{(from the proof above)} \\
                    & = \sqrt{z \overline{z} w \overline{w}}        & (z \overline{z}, w \overline{w} \in \R)                                 \\
                    & = \sqrt{zw \overline{z} \; \overline{w}}      &                                         & \by{ii:4.6.6}                 \\
                    & = \sqrt{zw \overline{zw}}                     &                                         & \by{ii:4.6.9}                 \\
                    & = \abs{zw}                                    &                                         & \text{(from the proof above)}
  \end{align*}
  and
  \begin{align*}
    \abs{\overline{z}} & = \sqrt{\overline{z} \; \overline{\overline{z}}} &  & \text{(from the proof above)} \\
                       & = \sqrt{\overline{z} z}                          &  & \by{ii:4.6.9}                 \\
                       & = \sqrt{z \overline{z}}                          &  & \by{ii:4.6.6}                 \\
                       & = \abs{z}.                                       &  & \text{(from the proof above)}
  \end{align*}
  Since
  \begin{align*}
             & \begin{dcases}
                 \abs{\Re(z)} = \sqrt{\abs{\Re(z)}^2} \leq \sqrt{\big(\Re(z)\big)^2 + \big(\Im(z)\big)^2} \\
                 \abs{\Im(z)} = \sqrt{\abs{\Im(z)}^2} \leq \sqrt{\big(\Re(z)\big)^2 + \big(\Im(z)\big)^2} \\
                 \big(\abs{\Re(z)} + \abs{\Im(z)}\big)^2 \geq \big(\Re(z)\big)^2 + \big(\Im(z)\big)^2
               \end{dcases} & (\Re(z), \Im(z) \in \R) \\
    \implies & \begin{dcases}
                 \abs{\Re(z)} \leq \abs{z} \\
                 \abs{\Im(z)} \leq \abs{z} \\
                 \abs{\Re(z)} + \abs{\Im(z)} \geq \sqrt{\big(\Re(z)\big)^2 + \big(\Im(z)\big)^2} = \abs{z}
               \end{dcases}                                                    &                         & \by{ii:4.6.10}                                  \\
    \implies & \begin{dcases}
                 -\abs{z} \leq \Re(z) \leq \abs{z} \\
                 -\abs{z} \leq \Im(z) \leq \abs{z} \\
                 \abs{\Re(z)} + \abs{\Im(z)} \geq \abs{z}
               \end{dcases}
  \end{align*}
  we know that
  \begin{align*}
    \Re(z \overline{w}) & \leq \abs{z \overline{w}}    &  & \text{(from the proof above)} \\
                        & = \abs{z} \abs{\overline{w}} &  & \text{(from the proof above)} \\
                        & = \abs{z} \abs{w}.           &  & \text{(from the proof above)}
  \end{align*}
  Thus
  \begin{align*}
             & \overline{z \overline{w}} = \overline{z} \; \overline{\overline{w}} = \overline{z} w                   &                                        & \by{ii:4.6.9}                 \\
    \implies & \Re(z \overline{w}) = \dfrac{z \overline{w} + \overline{z} w}{2}                                       &                                        & \by{ii:ex:4.6.5}              \\
    \implies & \dfrac{z \overline{w} + \overline{z} w}{2} \leq \abs{z} \abs{w}                                        &                                        & \text{(from the proof above)} \\
    \implies & z \overline{w} + \overline{z} w \leq 2 \abs{z} \abs{w}                                                                                                                          \\
    \implies & \abs{z}^2 + z \overline{w} + \overline{z} w + \abs{w}^2 \leq \abs{z}^2 + 2 \abs{z} \abs{w} + \abs{w}^2                                                                          \\
    \implies & \abs{z}^2 + z \overline{w} + \overline{z} w + \abs{w}^2 \leq (\abs{z} + \abs{w})^2                                                                                              \\
    \implies & z \overline{z} + z \overline{w} + \overline{z} w + w \overline{w} \leq (\abs{z} + \abs{w})^2           &                                        & \text{(from the proof above)} \\
    \implies & (z + w) (\overline{z} + \overline{w}) \leq (\abs{z} + \abs{w})^2                                       &                                        & \by{ii:4.6.6}                 \\
    \implies & (z + w) (\overline{z + w}) \leq (\abs{z} + \abs{w})^2                                                  &                                        & \by{ii:4.6.9}                 \\
    \implies & \abs{z + w}^2 \leq (\abs{z} + \abs{w})^2                                                               &                                        & \text{(from the proof above)} \\
    \implies & \abs{z + w} \leq \abs{z} + \abs{w}.                                                                    & (\abs{z + w}, \abs{z}, \abs{w} \in \R)
  \end{align*}
\end{proof}

\begin{defn}[Complex reciprocal]\label{ii:4.6.12}
  If \(z\) is a non-zero complex number, we define the \emph{reciprocal} \(z^{-1}\) of \(z\) to be the complex number \(z^{-1} \coloneqq \abs{z}^{-2} \overline{z}\)
  (note that \(\abs{z}^{-2}\) is well-defined as a positive real number because \(\abs{z}\) is positive real, thanks to \cref{ii:4.6.11}).
  If \(z\) is zero, \(z = 0\), we leave the reciprocal \(0^{-1}\) undefined.
\end{defn}

\begin{ac}\label{ii:ac:4.6.5}
  If \(z, w \in \C\) such that \(z = w \neq 0\), then \(z^{-1} = w^{-1}\).
\end{ac}

\begin{proof}
  \begin{align*}
             & z = w                                                                                     \\
    \implies & \overline{z} = \overline{w}                           &                & \by{ii:ac:4.6.4} \\
    \implies & z \overline{z} = w \overline{w}                       &                & \by{ii:ac:4.6.3} \\
    \implies & \abs{z} = \abs{w}                                     &                & \by{ii:4.6.11}   \\
    \implies & \abs{z}^{-2} = \abs{w}^{-2}                           & (z = w \neq 0)                    \\
    \implies & \abs{z}^{-2} \overline{z} = \abs{w}^{-2} \overline{w} &                & \by{ii:ac:4.6.3} \\
    \implies & z^{-1} = w^{-1}.                                      &                & \by{ii:4.6.12}
  \end{align*}
\end{proof}

\begin{note}
  From the \cref{ii:4.6.12} and \cref{ii:4.6.11}, we see that
  \[
    z z^{-1} = z^{-1} z = \abs{z}^{-2} \overline{z} z = \abs{z}^{-2} \abs{z}^2 = 1,
  \]
  and so \(z^{-1}\) is indeed the reciprocal of \(z\).
  We can thus define a notion of quotient \(z / w\) for any two complex numbers \(z, w\) with \(w \neq 0\) in the usual manner by the formula \(z / w \coloneqq z w^{-1}\).
\end{note}

\begin{lem}\label{ii:4.6.13}
  If we define \(d(z, w) = \abs{z - w}\), then the complex numbers \(\C\) with the distance \(d\) form a metric space.
  If \((z_n)_{n = 1}^\infty\) is a sequence of complex numbers, and \(z\) is another complex number, then we have \(\lim_{n \to \infty} z_n = z\) in this metric space iff \(\lim_{n \to \infty} \Re(z_n) = \Re(z)\) and \(\lim_{n \to \infty} \Im(z_n) = \Im(z)\).
\end{lem}

\begin{proof}
  First we show that \((\C, d)\) is a metric space.
  Since
  \[
    \forall z \in \C, d(z, z) = \abs{z - z} = \abs{0} = 0,
  \]
  we know that \((\C, d)\) is a reflexive.
  Since
  \begin{align*}
             & \forall w, z \in \C, w \neq z                                                                                \\
    \implies & \big(\Re(w) \neq \Re(z)\big) \lor \big(\Im(w) \neq \Im(z)\big)                           &  & \by{ii:4.6.2}  \\
    \implies & \Big(\big(\Re(w) - \Re(z)\big)^2 > 0\Big) \lor \Big(\big(\Im(w) - \Im(z)\big)^2 > 0\Big)                     \\
    \implies & \sqrt{\big(\Re(w) - \Re(z)\big)^2 + \big(\Im(w) - \Im(z)\big)^2} > 0                                         \\
    \implies & d(w, z) = \abs{w - z} > 0,                                                               &  & \by{ii:4.6.10}
  \end{align*}
  we know that \((\C, d)\) is a positive.
  Since
  \begin{align*}
    \forall w, z \in \C, d(w, z) & = \abs{w - z}                              \\
                                 & = \abs{-1} \abs{w - z}                     \\
                                 & = \abs{(-1) (w - z)}   &  & \by{ii:4.6.11} \\
                                 & = \abs{z - w}          &  & \by{ii:4.6.4}  \\
                                 & = d(z, w),
  \end{align*}
  we know that \((\C, d)\) is a symmetry.
  Since
  \begin{align*}
    \forall x, y, z \in \C, d(x, z) & = \abs{x - z}                                      \\
                                    & = \abs{x - y + y - z}                              \\
                                    & \leq \abs{x - y} + \abs{y - z} &  & \by{ii:4.6.11} \\
                                    & = d(x, y) + d(y, z),
  \end{align*}
  we know that \((\C, d)\) is a transitive.
  Combine all proofs above we conclude by \cref{ii:1.1.2} that \((\C, d)\) is a metric space.

  Now we show that
  \[
    d - \lim_{n \to \infty} z_n = z \iff \begin{dcases}
      \lim_{n \to \infty} \Re(z_n) = \Re(z) \\
      \lim_{n \to \infty} \Im(z_n) = \Im(z)
    \end{dcases}
  \]
  This is true since
  \begin{align*}
             & d - \lim_{n \to \infty} z_n = z                                                                                                                         \\
    \implies & \forall \varepsilon \in \R^+, \exists N \in \Z^+ : \forall n \geq N, d(z_n, z) < \varepsilon                                                            \\
    \implies & \forall \varepsilon \in \R^+, \exists N \in \Z^+ : \forall n \geq N, \abs{z_n - z} < \varepsilon                                                        \\
    \implies & \forall \varepsilon \in \R^+, \exists N \in \Z^+ : \forall n \geq N, \sqrt{\big(\Re(z_n) - \Re(z)\big)^2 + \big(\Im(z_n) - \Im(z)\big)^2} < \varepsilon \\
    \implies & \forall \varepsilon \in \R^+, \exists N \in \Z^+ : \forall n \geq N, \begin{dcases}
                                                                                      \sqrt{\big(\Re(z_n) - \Re(z)\big)^2} < \varepsilon \\
                                                                                      \sqrt{\big(\Im(z_n) - \Im(z)\big)^2} < \varepsilon
                                                                                    \end{dcases}                                 \\
    \implies & \forall \varepsilon \in \R^+, \exists N \in \Z^+ : \forall n \geq N, \begin{dcases}
                                                                                      \abs{\Re(z_n) - \Re(z)} < \varepsilon \\
                                                                                      \abs{\Im(z_n) - \Im(z)} < \varepsilon
                                                                                    \end{dcases}                                              \\
    \implies & \begin{dcases}
                 \lim_{n \to \infty} \Re(z_n) = \Re(z) \\
                 \lim_{n \to \infty} \Im(z_n) = \Im(z)
               \end{dcases}
  \end{align*}
  and
  \begin{align*}
             & \begin{dcases}
                 \lim_{n \to \infty} \Re(z_n) = \Re(z) \\
                 \lim_{n \to \infty} \Im(z_n) = \Im(z)
               \end{dcases}                                                                                                                                                   \\
    \implies & \forall \varepsilon \in \R^+, \exists N \in \Z^+ : \forall n \geq N, \begin{dcases}
                                                                                      \abs{\Re(z_n) - \Re(z)} < \dfrac{\varepsilon}{2} \\
                                                                                      \abs{\Im(z_n) - \Im(z)} < \dfrac{\varepsilon}{2}
                                                                                    \end{dcases}                                                                   \\
    \implies & \forall \varepsilon \in \R^+, \exists N \in \Z^+ : \forall n \geq N, \begin{dcases}
                                                                                      \big(\Re(z_n) - \Re(z)\big)^2 < \dfrac{\varepsilon^2}{4} \\
                                                                                      \big(\Im(z_n) - \Im(z)\big)^2 < \dfrac{\varepsilon^2}{4}
                                                                                    \end{dcases}                           \\
    \implies & \forall \varepsilon \in \R^+, \exists N \in \Z^+ : \forall n \geq N, \big(\Re(z_n) - \Re(z)\big)^2 + \big(\Im(z_n) - \Im(z)\big)^2 < \dfrac{\varepsilon^2}{2}                           \\
    \implies & \forall \varepsilon \in \R^+, \exists N \in \Z^+ : \forall n \geq N, \sqrt{\big(\Re(z_n) - \Re(z)\big)^2 + \big(\Im(z_n) - \Im(z)\big)^2} < \dfrac{\varepsilon}{\sqrt{2}} < \varepsilon \\
    \implies & \forall \varepsilon \in \R^+, \exists N \in \Z^+ : \forall n \geq N, \abs{\Re(z_n) - \Re(z) + i \big(\Im(z_n) - \Im(z)\big)} < \varepsilon                                              \\
    \implies & \forall \varepsilon \in \R^+, \exists N \in \Z^+ : \forall n \geq N, \abs{z_n - z} < \varepsilon                                                                                        \\
    \implies & \forall \varepsilon \in \R^+, \exists N \in \Z^+ : \forall n \geq N, d(z_n, z) < \varepsilon                                                                                            \\
    \implies & d - \lim_{n \to \infty} z_n = z.
  \end{align*}
\end{proof}

\begin{lem}[Complex limit laws]\label{ii:4.6.14}
  Let \((z_n)_{n = 1}^\infty\) and \((w_n)_{n = 1}^\infty\) be convergent sequences of complex numbers, and let \(c\) be a complex number.
  Then the sequences \((z_n + w_n)_{n = 1}^\infty\), \((z_n - w_n)_{n = 1}^\infty\), \((c z_n)_{n = 1}^\infty\), \((z_n w_n)_{n = 1}^\infty\), and \((\overline{z_n})_{n = 1}^\infty\) are also convergent, with
  \begin{align*}
    \lim_{n \to \infty} z_n + w_n      & = \lim_{n \to \infty} z_n + \lim_{n \to \infty} w_n                       \\
    \lim_{n \to \infty} z_n - w_n      & = \lim_{n \to \infty} z_n - \lim_{n \to \infty} w_n                       \\
    \lim_{n \to \infty} c z_n          & = c \lim_{n \to \infty} z_n                                               \\
    \lim_{n \to \infty} z_n w_n        & = \bigg(\lim_{n \to \infty} z_n\bigg) \bigg(\lim_{n \to \infty} w_n\bigg) \\
    \lim_{n \to \infty} \overline{z_n} & = \overline{\lim_{n \to \infty} z_n}
  \end{align*}
  Also, if the \(w_n\) are all non-zero and \(\lim_{n \to \infty} w_n\) is also non-zero, then \((z_n / w_n)_{n = 1}^\infty\) is also a convergent sequence, with
  \[
    \lim_{n \to \infty} z_n / w_n = \bigg(\lim_{n \to \infty} z_n\bigg) / \bigg(\lim_{n \to \infty} w_n\bigg).
  \]
\end{lem}

\begin{proof}
  Let \(d\) be the metric in \cref{ii:4.6.10}.
  Suppose that \(d - \lim_{n \to \infty} w_n = w\) and \(d - \lim_{n \to \infty} z_n = z\) for some \(z, w \in \C\).
  By \cref{ii:4.6.13} this means
  \begin{align*}
     & \lim_{n \to \infty} \Re(w_n) = \Re(w); \\
     & \lim_{n \to \infty} \Im(w_n) = \Im(w); \\
     & \lim_{n \to \infty} \Re(z_n) = \Re(z); \\
     & \lim_{n \to \infty} \Im(z_n) = \Im(z).
  \end{align*}
  Then we have
  \begin{align*}
     & \Re(z + w)                                                                                                           \\
     & = \Re(z) + \Re(w)                                                                        &  & \by{ii:4.6.3}          \\
     & = \lim_{n \to \infty} \big(\Re(z_n) + \Re(w_n)\big)                                      &  & \text{(by limit laws)} \\
     & = \lim_{n \to \infty} \Re\Big(\Re(z_n) + \Re(w_n) + i \big(\Im(z_n) + \Im(w_n)\big)\Big) &  & \by{ii:4.6.8}          \\
     & = \lim_{n \to \infty} \Re(z_n + w_n);                                                    &  & \by{ii:4.6.3}          \\
     & \Im(z + w)                                                                                                           \\
     & = \Im(z) + \Im(w)                                                                        &  & \by{ii:4.6.3}          \\
     & = \lim_{n \to \infty} \big(\Im(z_n) + \Im(w_n)\big)                                      &  & \text{(by limit laws)} \\
     & = \lim_{n \to \infty} \Im\Big(\Re(z_n) + \Re(w_n) + i \big(\Im(z_n) + \Im(w_n)\big)\Big) &  & \by{ii:4.6.8}          \\
     & = \lim_{n \to \infty} \Im(z_n + w_n);                                                    &  & \by{ii:4.6.3}          \\
     & \Re(z - w)                                                                                                           \\
     & = \Re(z) - \Re(w)                                                                        &  & \by{ii:4.6.3}          \\
     & = \lim_{n \to \infty} \big(\Re(z_n) - \Re(w_n)\big)                                      &  & \text{(by limit laws)} \\
     & = \lim_{n \to \infty} \Re\Big(\Re(z_n) - \Re(w_n) + i \big(\Im(z_n) - \Im(w_n)\big)\Big) &  & \by{ii:4.6.8}          \\
     & = \lim_{n \to \infty} \Re(z_n - w_n);                                                    &  & \by{ii:4.6.3}          \\
     & \Im(z - w)                                                                                                           \\
     & = \Im(z) - \Im(w)                                                                        &  & \by{ii:4.6.3}          \\
     & = \lim_{n \to \infty} \big(\Im(z_n) - \Im(w_n)\big)                                      &  & \text{(by limit laws)} \\
     & = \lim_{n \to \infty} \Im\Big(\Re(z_n) - \Re(w_n) + i \big(\Im(z_n) - \Im(w_n)\big)\Big) &  & \by{ii:4.6.8}          \\
     & = \lim_{n \to \infty} \Im(z_n - w_n).                                                    &  & \by{ii:4.6.3}          \\
  \end{align*}
  Thus, by \cref{ii:4.6.13} we have
  \begin{align*}
         & \begin{dcases}
             \Re(z + w) = \lim_{n \to \infty} \Re(z_n + w_n); \\
             \Im(z + w) = \lim_{n \to \infty} \Im(z_n + w_n); \\
             \Re(z - w) = \lim_{n \to \infty} \Re(z_n - w_n); \\
             \Im(z - w) = \lim_{n \to \infty} \Im(z_n - w_n);
           \end{dcases}                                                                             \\
    \iff & \begin{dcases}
             d - \lim_{n \to \infty} (z_n + w_n) = z + w = \big(d - \lim_{n \to \infty} z_n\big) + \big(d - \lim_{n \to \infty} w_n\big); \\
             d - \lim_{n \to \infty} (z_n - w_n) = z - w = \big(d - \lim_{n \to \infty} z_n\big) - \big(d - \lim_{n \to \infty} w_n\big).
           \end{dcases}
  \end{align*}

  By \cref{ii:4.6.5} we know that
  \[
    \forall n \in \Z^+, z_n w_n = \Re(z_n) \Re(w_n) - \Im(z_n) \Im(w_n) + i \big(\Re(z_n) \Im(w_n) + \Im(z_n) \Re(w_n)\big)
  \]
  and
  \[
    z w = \Re(z) \Re(w) - \Im(z) \Im(w) + i \big(\Re(z) \Im(w) + \Im(z) \Re(w)\big).
  \]
  Since
  \begin{align*}
     & \Re(zw)                                                                                           \\
     & = \Re(z) \Re(w) - \Im(z) \Im(w)                                       &  & \by{ii:4.6.8}          \\
     & = \lim_{n \to \infty} \big(\Re(z_n) \Re(w_n) - \Im(z_n) \Im(w_n)\big) &  & \text{(by limit laws)} \\
     & = \lim_{n \to \infty} \Re(z_n w_n);                                   &  & \by{ii:4.6.8}          \\
     & \Im(zw)                                                                                           \\
     & = \Re(z) \Im(w) + \Im(z) \Re(w)                                       &  & \by{ii:4.6.8}          \\
     & = \lim_{n \to \infty} \big(\Re(z_n) \Im(w_n) + \Im(z_n) \Re(w_n)\big) &  & \text{(by limit laws)} \\
     & = \lim_{n \to \infty} \Im(z_n w_n),                                   &  & \by{ii:4.6.8}
  \end{align*}
  by \cref{ii:4.6.13} we have
  \begin{align*}
         & \begin{dcases}
             \Re(zw) = \lim_{n \to \infty} \Re(z_n w_n); \\
             \Im(zw) = \lim_{n \to \infty} \Im(z_n w_n);
           \end{dcases}                                                                           \\
    \iff & d - \lim_{n \to \infty} (z_n w_n) = zw = \big(d - \lim_{n \to \infty} z_n\big) \big(d - \lim_{n \to \infty} w_n\big).
  \end{align*}

  Since
  \begin{align*}
             & \forall c \in \C, \forall \varepsilon \in \R^+, \forall n \in \Z^+, d(c, c) = \abs{c - c} = 0 < \varepsilon \\
    \implies & \forall c \in \C, \lim_{n \to \infty} c = c,
  \end{align*}
  from the proof above we have
  \[
    \forall c \in \C, d - \lim_{n \to \infty} (c z_n) = cz = c \big(d - \lim_{n \to \infty} z_n\big).
  \]

  By \cref{ii:4.6.8} we know that
  \begin{align*}
     & \forall n \in \Z^+, z_n = \Re(z_n) + i \Im(z_n);            \\
     & \forall n \in \Z^+, \overline{z_n} = \Re(z_n) - i \Im(z_n); \\
     & z = \Re(z) + i \Im(z);                                      \\
     & \overline{z} = \Re(z) - i \Im(z).
  \end{align*}
  Since
  \begin{align*}
    \Re(\overline{z}) & = \Re(z)                                   &  & \by{ii:4.6.8}          \\
                      & = \lim_{n \to \infty} \Re(z_n)                                         \\
                      & = \lim_{n \to \infty} \Re(\overline{z_n}); &  & \by{ii:4.6.8}          \\
    \Im(\overline{z}) & = -\Im(z)                                  &  & \by{ii:4.6.8}          \\
                      & = \lim_{n \to \infty} -\Im(z_n)            &  & \text{(by limit laws)} \\
                      & = \lim_{n \to \infty} \Im(\overline{z_n}), &  & \by{ii:4.6.8}
  \end{align*}
  by \cref{ii:4.6.13} we have
  \begin{align*}
         & \begin{dcases}
             \Re(\overline{z}) = \lim_{n \to \infty} \Re(\overline{z_n}); \\
             \Im(\overline{z}) = \lim_{n \to \infty} \Im(\overline{z_n});
           \end{dcases}                                    \\
    \iff & d - \lim_{n \to \infty} \overline{z_n} = \overline{z} = \overline{d - \lim_{n \to \infty} z_n}.
  \end{align*}

  Now suppose that \(w \neq 0\) and \(w_n \neq 0\) for all \(n \in \Z^+\).
  Since
  \begin{align*}
     & \dfrac{z}{w}                                                                                                                                     \\
     & = z w^{-1}                                                                                                                                       \\
     & = z \abs{w}^{-2} \overline{w}                                                                                                &  & \by{ii:4.6.12} \\
     & = \abs{w}^{-2} z \overline{w}                                                                                                &  & \by{ii:4.6.6}  \\
     & = \dfrac{z \overline{w}}{\big(\Re(w)\big)^2 + \big(\Im(w)\big)^2}                                                            &  & \by{ii:4.6.10} \\
     & = \dfrac{\Re(z) \Re(w) + \Im(z) \Im(w) - i \big(\Re(z) \Im(w) - \Im(z) \Re(w)\big)}{\big(\Re(w)\big)^2 + \big(\Im(w)\big)^2} &  & \by{ii:4.6.5}
  \end{align*}
  and
  \[
    \forall n \in \Z^+, \dfrac{z_n}{w_n} = \dfrac{\Re(z_n) \Re(w_n) + \Im(z_n) \Im(w_n)}{\big(\Re(w_n)\big)^2 + \big(\Im(w_n)\big)^2} - i \dfrac{\big(\Re(z_n) \Im(w_n) - \Im(z_n) \Re(w_n)\big)}{\big(\Re(w_n)\big)^2 + \big(\Im(w_n)\big)^2},
  \]
  we know that
  \begin{align*}
    \Re(\dfrac{z}{w}) & = \dfrac{\Re(z) \Re(w) + \Im(z) \Im(w)}{\big(\Re(w)\big)^2 + \big(\Im(w)\big)^2}                                   &  & \by{ii:4.6.8}          \\
                      & = \lim_{n \to \infty} \dfrac{\Re(z_n) \Re(w_n) + \Im(z_n) \Im(w_n)}{\big(\Re(w_n)\big)^2 + \big(\Im(w_n)\big)^2}   &  & \text{(by limit laws)} \\
                      & = \lim_{n \to \infty} \Re(\dfrac{z_n}{w_n});                                                                       &  & \by{ii:4.6.8}          \\
    \Im(\dfrac{z}{w}) & = \dfrac{- \Re(z) \Im(w) + \Im(z) \Re(w)}{\big(\Re(w)\big)^2 + \big(\Im(w)\big)^2}                                 &  & \by{ii:4.6.8}          \\
                      & = \lim_{n \to \infty} \dfrac{- \Re(z_n) \Im(w_n) + \Im(z_n) \Re(w_n)}{\big(\Re(w_n)\big)^2 + \big(\Im(w_n)\big)^2} &  & \text{(by limit laws)} \\
                      & = \lim_{n \to \infty} \Im(\dfrac{z_n}{w_n}).                                                                       &  & \by{ii:4.6.8}
  \end{align*}
  Thus, by \cref{ii:4.6.13} we have
  \begin{align*}
         & \begin{dcases}
             \Re(\dfrac{z}{w}) = \lim_{n \to \infty} \Re(\dfrac{z_n}{w_n}); \\
             \Im(\dfrac{z}{w}) = \lim_{n \to \infty} \Im(\dfrac{z_n}{w_n});
           \end{dcases}                                                              \\
    \iff & d - \lim_{n \to \infty} \dfrac{z_n}{w_n} = \dfrac{z}{w} = \dfrac{d - \lim_{n \to \infty} z_n}{d - \lim_{n \to \infty} w_n}.
  \end{align*}
\end{proof}

\begin{note}
  Observe that the real and complex number systems are in fact quite similar;
  they both obey similar laws of arithmetic, and they have similar structure as metric spaces.
  Indeed many of the results in this textbook that were proven for real-valued functions, are also valid for complex-valued functions, simply by replacing ``real'' with ``complex'' in the proofs but otherwise leaving all the other details of the proof unchanged.
  Alternatively, one can always split a complex-valued function \(f\) into real and imaginary parts \(\Re(f)\), \(\Im(f)\), thus \(f = \Re(f) + i \Im(f)\), and then deduce results for the complex-valued function \(f\) from the corresponding results for the real-valued functions \(\Re(f)\), \(\Im(f)\).
  For instance, the theory of pointwise and uniform convergence from \cref{ii:ch:3}, or the theory of power series from this chapter, extends without any difficulty to complex-valued functions.
  In particular, we can define the complex exponential function in exactly the same manner as for real numbers.
\end{note}

\begin{ac}[Complex series]\label{ii:ac:4.6.6}
  Let \(d\) be the metric in \cref{ii:4.6.10} and let \((a_n)_{n = 0}^\infty\) be a sequence in \(\C\).
  If \(d - \lim_{N \to \infty} \sum_{n = 0}^N a_n \in \C\), we define
  \[
    \sum_{n = 0}^\infty a_n = d - \lim_{N \to \infty} \sum_{n = 0}^N a_n.
  \]
  Then \(\sum_{n = 0}^\infty a_n\) converges iff both \(\sum_{n = 0}^\infty \Re(a_n)\) and \(\sum_{n = 0}^\infty \Im(a_n)\) converges, and we have
  \[
    \sum_{n = 0}^\infty a_n = \sum_{n = 0}^\infty \Re(a_n) + i \bigg(\sum_{n = 0}^\infty \Im(a_n)\bigg).
  \]
\end{ac}

\begin{proof}
  First suppose that \(\sum_{n = 0}^\infty \Re(a_n)\) and \(\sum_{n = 0}^\infty \Im(a_n)\) converges.
  Then we have
  \begin{align*}
     & \sum_{n = 0}^\infty \Re(a_n) + i \bigg(\sum_{n = 0}^\infty \Im(a_n)\bigg)                                                             \\
     & = d - \lim_{N \to \infty} \sum_{n = 0}^N \Re(a_n) + i \bigg(d - \lim_{N \to \infty} \sum_{n = 0}^N \Im(a_n)\bigg)                     \\
     & = d - \lim_{N \to \infty} \Bigg(\sum_{n = 0}^N \Re(a_n) + i \bigg(\sum_{n = 0}^N \Im(a_n)\bigg)\Bigg)             &  & \by{ii:4.6.14} \\
     & = d - \lim_{N \to \infty} \bigg(\sum_{n = 0}^N \Re(a_n) + \sum_{n = 0}^N \big(i \Im(a_n)\big)\bigg)               &  & \by{ii:4.6.6}  \\
     & = d - \lim_{N \to \infty} \sum_{n = 0}^N \big(\Re(a_n) + i \Im(a_n)\big)                                          &  & \by{ii:4.6.4}  \\
     & = d - \lim_{N \to \infty} \sum_{n = 0}^N a_n                                                                      &  & \by{ii:4.6.8}  \\
     & = \sum_{n = 0}^\infty a_n.
  \end{align*}

  Now suppose that \(\sum_{n = 0}^\infty a_n\) converges.
  Then we have
  \begin{align*}
     & \sum_{n = 0}^\infty a_n                                                                                                                                               \\
     & = \Re\bigg(\sum_{n = 0}^\infty a_n\bigg) + i \Im\bigg(\sum_{n = 0}^\infty a_n\bigg)                                                               &  & \by{ii:4.6.8}  \\
     & = \Re\bigg(d - \lim_{N \to \infty} \sum_{n = 0}^N a_n\bigg) + i \Im\bigg(d - \lim_{N \to \infty} \sum_{n = 0}^N a_n\bigg)                                             \\
     & = \Bigg(d - \lim_{N \to \infty} \Re\bigg(\sum_{n = 0}^N a_n\bigg)\Bigg) + i \Bigg(d - \lim_{N \to \infty} \Im\bigg(\sum_{n = 0}^N a_n\bigg)\Bigg) &  & \by{ii:4.6.13} \\
     & = \bigg(d - \lim_{N \to \infty} \sum_{n = 0}^N \Re(a_n)\bigg) + i \bigg(d - \lim_{N \to \infty} \sum_{n = 0}^N \Im(a_n)\bigg)                     &  & \by{ii:4.6.3}  \\
     & = \sum_{n = 0}^\infty \Re(a_n) + i \bigg(\sum_{n = 0}^\infty \Im(a_n)\bigg).
  \end{align*}
\end{proof}

\begin{ac}[Zero test of complex series]\label{ii:ac:4.6.7}
  Let \(d\) be the metric in \cref{ii:4.6.10} and let \((a_n)_{n = 0}^\infty\) be a sequence in \(\C\).
  Suppose that \(\sum_{n = 0}^\infty a_n \in \C\).
  Then \(d - \lim_{n \to \infty} a_n = 0\).
\end{ac}

\begin{proof}
  By \cref{ii:ac:4.6.6} we have
  \[
    \sum_{n = 0}^\infty a_n = \sum_{n = 0}^\infty \Re(a_n) + i \bigg(\sum_{n = 0}^\infty \Im(a_n)\bigg).
  \]
  By \cref{ii:4.6.8} we know that \(\sum_{n = 0}^\infty \Re(a_n) \in \R\) and \(\sum_{n = 0}^\infty \Im(a_n) \in \R\).
  By zero test of real series (Corollary 7.2.6 in Analysis I) this means
  \begin{align*}
     & \lim_{n \to \infty} \Re(a_n) = 0; \\
     & \lim_{n \to \infty} \Im(a_n) = 0.
  \end{align*}
  Then we have
  \begin{align*}
             & \begin{dcases}
                 \lim_{n \to \infty} \Re(a_n) = 0 = \Re(d - \lim_{n \to \infty} a_n) \\
                 \lim_{n \to \infty} \Im(a_n) = 0 = \Im(d - \lim_{n \to \infty} a_n)
               \end{dcases}                                   &  & \by{ii:4.6.13}                                                         \\
    \implies & d - \lim_{n \to \infty} a_n = \Re(d - \lim_{n \to \infty} a_n) + i \Im(d - \lim_{n \to \infty} a_n) = 0. &  & \by{ii:4.6.8}
  \end{align*}
\end{proof}

\begin{ac}[Absolutely convergent of complex series]\label{ii:ac:4.6.8}
  Let \(d\) be the metric in \cref{ii:4.6.10} and let \((a_n)_{n = 0}^\infty\) be a sequence in \(\C\).
  We say that \(\sum_{n = 0}^\infty a_n\) is absolutely convergent if \(\sum_{n = 0}^\infty \abs{a_n} \in \R\).
  Then we have
  \[
    \sum_{n = 0}^\infty \abs{a_n} \in \R \implies \sum_{n = 0}^\infty a_n \in \C.
  \]
\end{ac}

\begin{proof}
  By \cref{ii:4.6.10} we know that \(\abs{a_n} \in \R\) for all \(n \in \N\), thus \(\sum_{n = 0}^\infty \abs{a_n}\) is well-defined.
  Let \(N_1, N_2 \in \N\).
  Since
  \[
    \sum_{n = 0}^\infty \abs{a_n} = \lim_{N \to \infty} \sum_{n = 0}^N \abs{a_n},
  \]
  we know that the sequence \((\sum_{n = 0}^N \abs{a_n})_{N = 0}^\infty\) is a Cauchy sequence in \((\R, d_{l^1}|_{\R \times \R})\).
  Then we have
  \begin{align*}
             & \forall \varepsilon \in \R^+, \exists M \in \N : \forall N_1, N_2 \geq M, \abs{\sum_{n = 0}^{N_1} \abs{a_n} - \sum_{n = 0}^{N_2} \abs{a_n}} < \varepsilon \\
    \implies & \forall \varepsilon \in \R^+, \exists M \in \N : \forall N_1, N_2 \geq M,                                                                                 \\
             & \begin{dcases}
                 \sum_{n = N_1 + 1}^{N_2} \abs{a_n} = \abs{\sum_{n = N_1 + 1}^{N_2} \abs{a_n}} < \varepsilon & \text{if } N_1 \leq N_2 \\
                 \sum_{n = N_2 + 1}^{N_1} \abs{a_n} = \abs{\sum_{n = N_2 + 1}^{N_1} \abs{a_n}} < \varepsilon & \text{if } N_1 > N_2
               \end{dcases}                        \\
    \implies & \forall \varepsilon \in \R^+, \exists M \in \N : \forall N_1, N_2 \geq M,                                                                                 \\
             & \begin{dcases}
                 \abs{\sum_{n = N_1 + 1}^{N_2} a_n} \leq \sum_{n = N_1 + 1}^{N_2} \abs{a_n} < \varepsilon & \text{if } N_1 \leq N_2 \\
                 \abs{\sum_{n = N_2 + 1}^{N_1} a_n} \leq \sum_{n = N_2 + 1}^{N_1} \abs{a_n} < \varepsilon & \text{if } N_1 > N_2
               \end{dcases}                                      &  & \by{ii:4.6.11}                                         \\
    \implies & \forall \varepsilon \in \R^+, \exists M \in \N : \forall N_1, N_2 \geq M, \abs{\sum_{n = 0}^{N_1} a_n - \sum_{n = 0}^{N_2} a_n} < \varepsilon.
  \end{align*}
  This means \((\sum_{n = 0}^N a_n)_{N = 0}^\infty\) is a Cauchy sequence in \((\C, d)\).
  By \cref{ii:ex:4.6.10} we know that \((\C, d)\) is complete, thus \(\sum_{n = 0}^\infty a_n\) converges in \(\C\) with respect to \(d\).
\end{proof}

\begin{ac}[Cauchy Product]\label{ii:ac:4.6.9}
  Let \((a_n)_{n = 0}^\infty\), \((b_n)_{n = 0}^\infty\) be sequences in \(\C\).
  Suppose that \(\sum_{n = 0}^\infty a_n\), \(\sum_{n = 0}^\infty b_n\) are absolutely convergent.
  Then we have
  \[
    \bigg(\sum_{n = 0}^\infty a_n\bigg) \bigg(\sum_{n = 0}^\infty b_n\bigg) = \sum_{n = 0}^\infty \bigg(\sum_{k = 0}^n \big(a_k b_{n - k}\big)\bigg).
  \]
\end{ac}

\begin{proof}
  Let \(d\) be the metric in \cref{ii:4.6.10}.
  Let \(A = \sum_{n = 0}^\infty a_n\) and \(B = \sum_{n = 0}^\infty b_n\).
  By \cref{ii:ac:4.6.8} we know that \(A, B\) are well-defined.
  We define partial sums
  \begin{align*}
     & \forall N \in \N, A_N = \sum_{n = 0}^N a_n                          \\
     & \forall N \in \N, B_N = \sum_{n = 0}^N b_n                          \\
     & \forall N \in \N, C_N = \sum_{n = 0}^N \sum_{k = 0}^n a_k b_{n - k}
  \end{align*}
  By \cref{ii:ac:4.6.6} we know that \(d - \lim_{N \to \infty} (B_N - B) = 0\).
  Thus, if we define
  \[
    \forall N \in \N, \beta_N = B_N - B,
  \]
  then \(d - \lim_{N \to \infty} \beta_N = 0\).
  Observe that
  \begin{align*}
    \forall N \in \N, C_N & = \sum_{n = 0}^N \sum_{k = 0}^n (a_k b_{n - k})                                            \\
                          & = \sum_{(n, k) \in \N^2 : n + k \leq N} (a_n b_k)                                          \\
                          & = \sum_{n = 0}^N \sum_{k = 0}^{N - n} (a_n b_k)                                            \\
                          & = \sum_{n = 0}^N a_n \bigg(\sum_{k = 0}^{N - n} b_k\bigg)               &  & \by{ii:4.6.6} \\
                          & = \sum_{n = 0}^N (a_n B_{N - n})                                                           \\
                          & = \sum_{n = 0}^N \big(a_n (B_{N - n} - B + B)\big)                      &  & \by{ii:4.6.4} \\
                          & = \sum_{n = 0}^N \big(a_n (\beta_{N - n} + B)\big)                      &  & \by{ii:4.6.6} \\
                          & = \sum_{n = 0}^N (a_n \beta_{N - n} + a_n B)                            &  & \by{ii:4.6.6} \\
                          & = \sum_{n = 0}^N (a_n \beta_{N - n}) + \sum_{n = 0}^N (a_n B)           &  & \by{ii:4.6.4} \\
                          & = \sum_{n = 0}^N (a_n \beta_{N - n}) + \bigg(\sum_{n = 0}^N a_n\bigg) B &  & \by{ii:4.6.6} \\
                          & = \sum_{n = 0}^N (a_n \beta_{N - n}) + A_N B.
  \end{align*}
  By \cref{ii:ac:4.6.6} we want to show that
  \[
    AB = \lim_{N \to \infty} C_N.
  \]
  Since
  \[
    AB = (\lim_{N \to \infty} A_N) B = \lim_{N \to \infty} (A_N B),
  \]
  if we define
  \[
    \forall N \in \N, \gamma_N = \sum_{n = 0}^N (a_n \beta_{N - n}),
  \]
  then it suffices to show that
  \[
    \lim_{N \to \infty} \gamma_N = \lim_{N \to \infty} (C_N - A_N B) = 0.
  \]

  Let \(\varepsilon \in \R^+\) and let \(\alpha = \sum_{n = 0}^\infty \abs{a_n}\).
  By hypothesis we know that \(\sum_{n = 0}^\infty a_n\) is absolutely convergent, so \(\alpha\) is well-defined.
  Since \(d - \lim_{N \to \infty} \beta_N = 0\), we know that
  \[
    \exists M_1 \in \N : \forall N \geq M_1, \abs{\beta_N - 0} = \abs{\beta_N} < \dfrac{\varepsilon}{2 \alpha}.
  \]
  Fix such \(M_1\).
  Since \(\sum_{n = 0}^\infty a_n \in \C\), by \cref{ii:ac:4.6.7} we know that \(d - \lim_{N \to \infty} a_N = 0\).
  Thus, we have
  \[
    \exists M_2 \in \N : \forall N \geq M_2, \abs{a_N - 0} = \abs{a_N} < \dfrac{\varepsilon}{2 M_1 (\max_{k = 0}^{M_1} \abs{\beta_k})}.
  \]
  Let \(M = M_1 + M_2\) and choose one \(N \geq M\).
  Then we have
  \begin{align*}
     & \abs{\gamma_N} = \abs{\sum_{n = 0}^N (a_n \beta_{N - n})} = \abs{\sum_{n = 0}^N (a_{N - n} \beta_n)}                             \\
     & \leq \sum_{n = 0}^N \abs{a_{N - n} \beta_n} = \sum_{n = 0}^N (\abs{a_{N - n}} \abs{\beta_n})                 &  & \by{ii:4.6.11} \\
     & = \sum_{n = 0}^{M_1} (\abs{a_{N - n}} \abs{\beta_n}) + \sum_{n = M_1 + 1}^N (\abs{a_{N - n}} \abs{\beta_n}). &  & \by{ii:4.6.4}
  \end{align*}
  Since
  \begin{align*}
         & 0 \leq n \leq M_1                                     \\
    \iff & 0 \geq -n \geq -M_1                                   \\
    \iff & N \geq N - n \geq N - M_1 \geq M_1 + M_2 - M_1 = M_2,
  \end{align*}
  we know that
  \begin{align*}
    \sum_{n = 0}^{M_1} (\abs{a_{N - n}} \abs{\beta_n}) & \leq \sum_{n = 0}^{M_1} \dfrac{\varepsilon \abs{\beta_n}}{2 M_1 (\max_{k = 0}^{M_1} \abs{\beta_k})}                                       \\
                                                       & = \dfrac{\varepsilon}{2 M_1} \bigg(\sum_{n = 0}^{M_1} \dfrac{\abs{\beta_n}}{\max_{k = 0}^{M_1} \abs{\beta_k}}\bigg)                       \\
                                                       & \leq \dfrac{\varepsilon}{2 M_1} \bigg(\sum_{n = 0}^{M_1} \dfrac{\max_{k = 0}^{M_1} \abs{\beta_k}}{\max_{k = 0}^{M_1} \abs{\beta_k}}\bigg) \\
                                                       & = \dfrac{\varepsilon}{2 M_1} M_1                                                                                                          \\
                                                       & = \dfrac{\varepsilon}{2}.
  \end{align*}
  Thus, we have
  \begin{align*}
    \abs{\gamma_N} & \leq \sum_{n = 0}^{M_1} (\abs{a_{N - n}} \abs{\beta_n}) + \sum_{n = M_1 + 1}^N (\abs{a_{N - n}} \abs{\beta_n})                                                                        \\
                   & \leq \dfrac{\varepsilon}{2} + \sum_{n = M_1 + 1}^N (\abs{a_{N - n}} \abs{\beta_n})                                                                                                    \\
                   & \leq \dfrac{\varepsilon}{2} + \sum_{n = M_1 + 1}^N (\abs{a_{N - n}} \dfrac{\varepsilon}{2 \alpha})             &  & \text{(by the definition of \(M_1\))}                             \\
                   & = \dfrac{\varepsilon}{2} + \bigg(\sum_{n = M_1 + 1}^N \abs{a_{N - n}}\bigg) \dfrac{\varepsilon}{2 \alpha}                                                                             \\
                   & \leq \dfrac{\varepsilon}{2} + \alpha \dfrac{\varepsilon}{2 \alpha}                                             &  & \text{(\(\sum_{n = 0}^\infty \abs{a_n}\) is monotone increasing)} \\
                   & = \varepsilon.
  \end{align*}
  Since \(N\) was arbitrary, we have
  \[
    \forall N \geq M, \abs{\gamma_N} \leq \varepsilon.
  \]
  Since \(\varepsilon\) was arbitrary, we have
  \[
    \forall \varepsilon \in \R^+, \exists M \in \N : \forall N \geq M, \abs{\gamma_N} \leq \varepsilon
  \]
  and thus \(d - \lim_{N \to \infty} \gamma_N = 0\), as desired.
\end{proof}

\begin{defn}[Complex exponential]\label{ii:4.6.15}
  If \(z\) is a complex number, we define the function \(\exp(z)\) by the formula
  \[
    \exp(z) \coloneqq \sum_{n = 0}^\infty \dfrac{z^n}{n!}.
  \]
\end{defn}

\begin{note}
  Inspired by \cref{ii:4.5.4}, we shall use \(\exp(z)\) and \(e^z\) interchangeably.
  It is also possible to define \(a^z\) for complex \(z\) and real \(a > 0\), but we will not need to do so in this text.
\end{note}

\exercisesection

\begin{ex}\label{ii:ex:4.6.1}
  Prove \cref{ii:4.6.4}.
\end{ex}

\begin{proof}
  See \cref{ii:4.6.4}.
\end{proof}

\begin{ex}\label{ii:ex:4.6.2}
  Prove \cref{ii:4.6.6}.
\end{ex}

\begin{proof}
  See \cref{ii:4.6.6}.
\end{proof}

\begin{ex}\label{ii:ex:4.6.3}
  Prove \cref{ii:4.6.7}.
\end{ex}

\begin{proof}
  See \cref{ii:4.6.7}.
\end{proof}

\begin{ex}\label{ii:ex:4.6.4}
  Prove \cref{ii:4.6.9}.
\end{ex}

\begin{proof}
  See \cref{ii:4.6.9}.
\end{proof}

\begin{ex}\label{ii:ex:4.6.5}
  If \(z\) is a complex number, show that \(\Re(z) = \dfrac{z + \overline{z}}{2}\) and \(\Im(z) = \dfrac{z - \overline{z}}{2i}\).
\end{ex}

\begin{proof}
  We have
  \begin{align*}
    \dfrac{z + \overline{z}}{2} & = \dfrac{\Re(z) + i \Im(z) + \overline{\Re(z) + i \Im(z)}}{2} &                 & \by{ii:4.6.8} \\
                                & = \dfrac{\Re(z) + i \Im(z) + \Re(z) - i \Im(z)}{2}            &                 & \by{ii:4.6.8} \\
                                & = \dfrac{\Re(z) + \Re(z) + i \Im(z) - i \Im(z)}{2}            &                 & \by{ii:4.6.4} \\
                                & = \dfrac{2 \Re(z)}{2}                                         &                 & \by{ii:4.6.4} \\
                                & = \Re(z)                                                      & (\Re(z) \in \R)
  \end{align*}
  and
  \begin{align*}
    \dfrac{z - \overline{z}}{2i} & = \dfrac{\Re(z) + i \Im(z) - \overline{\Re(z) + i \Im(z)}}{2i}            &  & \by{ii:4.6.8}  \\
                                 & = \dfrac{\Re(z) + i \Im(z) + \overline{-\big(\Re(z) + i \Im(z)\big)}}{2i} &  & \by{ii:4.6.9}  \\
                                 & = \dfrac{\Re(z) + i \Im(z) + \overline{-\Re(z) - i \Im(z)\big)}}{2i}      &  & \by{ii:4.6.3}  \\
                                 & = \dfrac{\Re(z) + i \Im(z) - \Re(z) + i \Im(z)}{2i}                       &  & \by{ii:4.6.8}  \\
                                 & = \dfrac{\Re(z) - \Re(z) + i \Im(z) + i \Im(z)}{2i}                       &  & \by{ii:4.6.4}  \\
                                 & = \dfrac{2i \Im(z)}{2i}                                                   &  & \by{ii:4.6.4}  \\
                                 & = \Im(z).                                                                 &  & \by{ii:4.6.12}
  \end{align*}
\end{proof}

\begin{ex}\label{ii:ex:4.6.6}
  Prove \cref{ii:4.6.11}.
\end{ex}

\begin{proof}
  See \cref{ii:4.6.11}.
\end{proof}

\begin{ex}\label{ii:ex:4.6.7}
  Show that if \(z, w\) are complex numbers with \(w \neq 0\), then \(\abs{z / w} = \abs{z} / \abs{w}\).
\end{ex}

\begin{proof}
  \begin{align*}
    \abs{\dfrac{z}{w}} & = \abs{z w^{-1}}                                                                       \\
                       & = \abs{z \abs{w}^{-2} \overline{w}}       &                           & \by{ii:4.6.12} \\
                       & = \abs{z} \abs{w}^{-2} \abs{\overline{w}} &                           & \by{ii:4.6.11} \\
                       & = \abs{z} \abs{w}^{-2} \abs{w}            &                           & \by{ii:4.6.11} \\
                       & = \abs{z} \abs{w}^{-1}                    & (\abs{z}, \abs{w} \in \R)                  \\
                       & = \dfrac{\abs{z}}{\abs{w}}.
  \end{align*}
\end{proof}

\begin{ex}\label{ii:ex:4.6.8}
  Let \(z, w\) be non-zero complex numbers.
  Show that \(\abs{z + w} = \abs{z} + \abs{w}\) iff there exists a positive real number \(c > 0\) such that \(z = cw\).
\end{ex}

\begin{proof}
  Since
  \begin{align*}
    \abs{z + w} & = \abs{\Re(z) + \Re(w) + i \big(\Im(z) + \Im(w)\big)}              &  & \by{ii:4.6.8}  \\
                & = \sqrt{\big(\Re(z) + \Re(w)\big)^2 + \big(\Im(z) + \Im(w)\big)^2} &  & \by{ii:4.6.10}
  \end{align*}
  and
  \begin{align*}
    \abs{z} + \abs{w} & = \abs{\Re(z) + i \Im(z)} + \abs{\Re(w) + i \Im(w)}                                                &  & \by{ii:4.6.8}  \\
                      & = \sqrt{\big(\Re(z)\big)^2 + \big(\Im(z)\big)^2} + \sqrt{\big(\Re(w)\big)^2 + \big(\Im(w)\big)^2}, &  & \by{ii:4.6.10}
  \end{align*}
  we know that
  \begin{align*}
         & \abs{z + w} = \abs{z} + \abs{w}                                                                                 \\
    \iff & \abs{z + w}^2 = \abs{z}^2 + 2 \abs{z} \abs{w} + \abs{w}^2                                                       \\
    \iff & \big(\Re(z) + \Re(w)\big)^2 + \big(\Im(z) + \Im(w)\big)^2                                                       \\
         & = \big(\Re(z)\big)^2 + \big(\Im(z)\big)^2 + 2 \abs{z} \abs{w} + \big(\Re(w)\big)^2 + \big(\Im(w)\big)^2         \\
    \iff & \Re(z) \Re(w) + \Im(z) \Im(w) = \abs{z} \abs{w}                                                                 \\
    \iff & \big(\Re(z) \Re(w)\big)^2 + 2 \Re(z) \Re(w) \Im(z) \Im(w) + \big(\Im(z) \Im(w)\big)^2                           \\
         & = \big(\Re(z) \Re(w)\big)^2 + \big(\Re(z) \Im(w)\big)^2 + \big(\Im(z) \Re(w)\big)^2 + \big(\Im(z) \Im(w)\big)^2 \\
    \iff & 2 \Re(z) \Re(w) \Im(z) \Im(w) = \big(\Re(z) \Im(w)\big)^2 + \big(\Im(z) \Re(w)\big)^2                           \\
    \iff & \big(\Re(z) \Im(w)\big)^2 - 2 \Re(z) \Re(w) \Im(z) \Im(w) + \big(\Im(z) \Re(w)\big)^2 = 0                       \\
    \iff & \big(\Re(z) \Im(w) - \Im(z) \Re(w)\big)^2 = 0                                                                   \\
    \iff & \Re(z) \Im(w) - \Im(z) \Re(w) = 0                                                                               \\
    \iff & \Re(z) \Im(w) = \Im(z) \Re(w).
  \end{align*}
  By hypothesis we know that \(w \neq 0 \neq z\).
  Thus, we can split into two cases:
  \begin{itemize}
    \item \(\Re(z) \neq 0\) and \(\Re(w) \neq 0\).
          Then we have
          \begin{align*}
                 & \abs{z + w} = \abs{z} + \abs{w}                 \\
            \iff & \Re(z) \Im(w) = \Im(z) \Re(w)                   \\
            \iff & \dfrac{\Im(z)}{\Re(z)} = \dfrac{\Im(w)}{\Re(w)} \\
            \iff & \exists c \in \R^+ : z = cw.
          \end{align*}
    \item \(\Re(z) \neq 0\) and \(\Im(w) \neq 0\).
          Then we have
          \begin{align*}
                 & \abs{z + w} = \abs{z} + \abs{w}                 \\
            \iff & \Re(z) \Im(w) = \Im(z) \Re(w) \neq 0            \\
            \iff & \dfrac{\Im(z)}{\Re(z)} = \dfrac{\Im(w)}{\Re(w)} \\
            \iff & \exists c \in \R^+ : z = cw.
          \end{align*}
    \item \(\Im(z) \neq 0\) and \(\Re(w) \neq 0\).
          Then we have
          \begin{align*}
                 & \abs{z + w} = \abs{z} + \abs{w}                 \\
            \iff & \Re(z) \Im(w) = \Im(z) \Re(w) \neq 0            \\
            \iff & \dfrac{\Im(z)}{\Re(z)} = \dfrac{\Im(w)}{\Re(w)} \\
            \iff & \exists c \in \R^+ : z = cw.
          \end{align*}
    \item \(\Im(z) \neq 0\) and \(\Im(w) \neq 0\).
          Then we have
          \begin{align*}
                 & \abs{z + w} = \abs{z} + \abs{w}                 \\
            \iff & \Re(z) \Im(w) = \Im(z) \Re(w)                   \\
            \iff & \dfrac{\Re(z)}{\Im(z)} = \dfrac{\Re(w)}{\Im(w)} \\
            \iff & \exists c \in \R^+ : z = cw.
          \end{align*}
  \end{itemize}
  From all cases above we conclude that
  \begin{align*}
         & \abs{z + w} = \abs{z} + \abs{w} \\
    \iff & \Re(z) \Im(w) = \Im(z) \Re(w)   \\
    \iff & \exists c \in \R^+ : z = cw.
  \end{align*}
\end{proof}

\begin{ex}\label{ii:ex:4.6.9}
  Prove \cref{ii:4.6.13}.
\end{ex}

\begin{proof}
  See \cref{ii:4.6.13}.
\end{proof}

\begin{ex}\label{ii:ex:4.6.10}
  Show that the complex numbers \(\C\) (with the usual metric \(d\)) form a complete metric space.
\end{ex}

\begin{proof}
  Suppose that \((\C, d)\) is not complete.
  By \cref{ii:1.4.10} we know that there is a Cauchy sequence \((a_n)_{n = 1}^\infty\) in \((\C, d)\) which does not converge.
  Let \(x \in \C\).
  Then we have
  \[
    \exists \varepsilon \in \R^+ : \forall N_1 \in \Z^+, \exists n \geq N_1 : d(a_n, x) \geq \varepsilon.
  \]
  Fix such \(\varepsilon\).
  Let \(i, j \in \Z^+\).
  Since \((a_n)_{n = 1}^\infty\) is a Cauchy sequence in \((\C, d)\), by \cref{ii:1.4.6} we know that
  \[
    \exists N_2 \in \Z^+ : \forall i, j \geq N_2, d(a_i, a_j) < \varepsilon.
  \]
  Let \(N = \max(N_1, N_2)\).
  Then we have
  \begin{align*}
             & \exists n \geq N : \begin{dcases}
                                    d(a_n, x) \geq \varepsilon \\
                                    \forall i, j \geq n, d(a_i, a_j) < \varepsilon
                                  \end{dcases}                                                  \\
    \implies & \exists n \geq N : \forall i, j \geq n, 2\varepsilon \leq d(a_i, x) + d(a_j, x) \leq d(a_i, a_j) \leq \varepsilon \\
    \implies & 2\varepsilon \leq \varepsilon.
  \end{align*}
  But \(\varepsilon \in \R^+\), a contradiction.
  Thus, such Cauchy sequence does not exists, and \((\C, d)\) must be complete.
\end{proof}

\begin{ex}\label{ii:ex:4.6.11}
  Let \(f : \R^2 \to \C\) be the map \(f(a, b) \coloneqq a + bi\).
  Show that \(f\) is a bijection, and that \(f\) and \(f^{-1}\) are both continuous maps.
\end{ex}

\begin{proof}
  First we show that \(f\) is injective.
  Let \(a, b, c, d \in \R\) such that \(f(a, b) = f(c, d)\).
  Then we have
  \begin{align*}
             & f(a, b) = f(c, d)     &  & \text{(by hypothesis)}              \\
    \implies & a + bi = c + di       &  & \text{(by the definition of \(f\))} \\
    \implies & (a = c) \land (b = d) &  & \by{ii:4.6.2}                       \\
    \implies & (a, b) = (c, d).
  \end{align*}
  Thus, \(f\) is injective.

  Next we show that \(f\) is surjective.
  Let \(x \in \C\).
  Then by \cref{ii:4.6.2} we know that there exist some \(a, b \in \R\) such that \(x = a + bi\).
  By the definition of \(f\) we know that \(f(a, b) = x\).
  Since \(x\) was arbitrary, we know that \(f\) is surjective.
  Since \(f\) is both injective and surjective, we know that \(f\) is bijective, thus \(f^{-1}\) is well-defined.

  Next we show that \(f\) is a continuous map.
  Let \((a, b) \in \R^2\).
  Let \(d_2 = d_{l^1}|_{\R^2 \times \R^2}\) and let \(d\) be the metric defined in \cref{ii:4.6.10}.
  We want to show that \(f\) is continuous at \((a, b)\) from \((\R^2, d_2)\) to \((\C, d)\).
  By \cref{ii:2.1.1} we need to show that
  \[
    \forall \varepsilon \in \R^+, \exists \delta \in \R^+ : \forall (x, y) \in \R^2, d_2\big((x, y), (a, b)\big) < \delta \implies d\big(f(x, y), f(a, b)\big) < \varepsilon.
  \]
  Now fix one \(\varepsilon\).
  Then we have
  \begin{align*}
             & \forall (x, y) \in \R^2, d\big(f(x, y), f(a, b)\big) < \varepsilon                                                \\
    \implies & \sqrt{(x - a)^2 + (y - b)^2} < \varepsilon                                                  &  & \by{ii:4.6.10}   \\
    \implies & \sqrt{2} \sqrt{(x - a)^2 + (y - b)^2} < \sqrt{2} \varepsilon                                                      \\
    \implies & \abs{x - a} + \abs{y - b} \leq \sqrt{2} \sqrt{(x - a)^2 + (y - b)^2} < \sqrt{2} \varepsilon &  & \by{ii:ex:1.1.8} \\
    \implies & d_2\big((x, y), (a, b)\big) \leq \sqrt{2} \varepsilon.
  \end{align*}
  By setting \(\delta = \sqrt{2} \varepsilon\) we are done.
  Since \((a, b)\) was arbitrary, we know that \(f\) is continuous on \(\R^2\) from \((\R^2, d_2)\) to \((\C, d)\).

  Finally we show that \(f^{-1}\) is a continuous map.
  Let \(x \in \C\).
  We want to show that \(f^{-1}\) is continuous at \(x\) from \((\C, d)\) to \((\R^2, d_2)\).
  By \cref{ii:2.1.1} we need to show that
  \[
    \forall \varepsilon \in \R^+, \exists \delta \in \R^+ : \forall y \in \C, d\big(x, y\big) < \delta \implies d_2\big(f^{-1}(x), f^{-1}(y)\big) < \varepsilon.
  \]
  Now fix one \(\varepsilon\).
  Then we have
  \begin{align*}
             & \forall y \in \C, d_2\big(f^{-1}(x), f^{-1}(y)\big) < \varepsilon                                    \\
    \implies & d_2\Big(\big(\Re(x), \Im(x)\big), \big(\Re(y), \Im(y)\big)\Big) < \varepsilon  &  & \by{ii:4.6.8}    \\
    \implies & \abs{\Re(x) - \Re(y)} + \abs{\Im(x) - \Im(y)} < \varepsilon                                          \\
    \implies & \sqrt{\big(\Re(x) - \Re(y)\big)^2 + \big(\Im(x) - \Im(y)\big)^2} < \varepsilon &  & \by{ii:ex:1.1.8} \\
    \implies & d(x, y) < \varepsilon.
  \end{align*}
  By setting \(\delta = \varepsilon\) we are done.
  Since \(x\) was arbitrary, we know that \(f^{-1}\) is continuous on \(\C\) from \((\C, d)\) to \((\R^2, d_2)\).
\end{proof}

\begin{ex}\label{ii:ex:4.6.12}
  Show that the complex numbers \(\C\) (with the usual metric \(d\)) form a connected metric space.
\end{ex}

\begin{proof}
  We first show that \((C, d)\) is path-connected.
  By \cref{ii:ex:2.4.7} we want to show that for every \(x, y \in \C\), there exists a continuous function \(\gamma : [0, 1] \to \C\) such that \(\gamma(0) = x\) and \(\gamma(1) = y\).
  So let \(x, y \in \C\).
  We define \(\gamma : [0, 1] \to \C\) as follow:
  \[
    \forall z \in [0, 1], \gamma(z) = (1 - z)x + zy.
  \]
  Obviously \(\gamma(0) = x\) and \(\gamma(1) = y\).
  Let \(d_1 = d_{l^1}|_{[0, 1] \times [0, 1]}\).
  Now we show that \(\gamma\) is continuous on \([0, 1]\) from \(([0, 1], d_1)\) to \((\C, d)\).
  Let \(f : \R^2 \to \C\) be the function in \cref{ii:ex:4.6.11}.
  Then we have
  \[
    \forall z \in [0, 1], \gamma(z) = f\big((1 - z) \Re(x) + z \Re(y), (1 - z) \Im(x) + z \Im(y)\big).
  \]
  By \cref{ii:ex:4.6.11} we know that \(f\) is continuous on \(\R^2\) from \((\R^2, d_{l^1}|_{\R^2 \times \R^2})\) to \((\C, d)\).
  Thus, by \cref{ii:2.1.3} we know that \(\gamma\) is continuous on \([0, 1]\) from \(([0, 1], d_1)\) to \((\C, d)\).
  Since \(x, y\) were arbitrary, by \cref{ii:ex:2.4.7} we know that \((\C, d)\) is path-connected, and we conclude that \((\C, d)\) is connected.
\end{proof}

\begin{ex}\label{ii:ex:4.6.13}
  Let \(E\) be a subset of \(\C\).
  Show that \(E\) is compact iff \(E\) is closed and bounded.
  In particular, show that \(\C\) is not compact.
\end{ex}

\begin{proof}
  Let \(d\) be the metric defined in \cref{ii:4.6.10}.
  By \cref{ii:1.5.6} we know that \((E, d|_{E \times E})\) is compact implies \((E, d|_{E \times E})\) is closed and bounded.
  So we only need to show that \((E, d|_{E \times E})\) is closed and bounded implies \((E, d|_{E \times E})\) is compact.
  Suppose that \((E, d|_{E \times E})\) is closed and bounded.
  Let \(d_2 = d_{l^1}|_{\R^2 \times \R^2}\) and let \(f : \R^2 \to \C\) be the function defined in \cref{ii:ex:4.6.11}.
  Then we know that \(f^{-1}\) is well-defined and both \(f, f^{-1}\) are continuous.
  Since \((E, d|_{E \times E})\) is closed, by \cref{ii:2.1.5}(a)(d) we know that \(\big(f^{-1}(E), d_2|_{f^{-1}(E) \times f^{-1}(E)}\big)\) is closed.
  Since \((E, d|_{E \times E})\) is bounded, we know that
  \begin{align*}
             & \forall z \in \C, \exists r \in \R^+ : E \subseteq B_{(\C, d)}(z, r)                     &  & \by{ii:1.5.3}  \\
    \implies & \exists r \in \R^+ : E \subseteq B_{(\C, d)}(0, r)                                                           \\
    \implies & \exists r \in \R^+ : f^{-1}(E) \subseteq f^{-1}\big(B_{(\C, d)}(0, r)\big)                                   \\
    \implies & \exists r \in \R^+ : \forall x \in E, \sqrt{\big(\Re(x)\big)^2 + \big(\Im(x)\big)^2} < r &  & \by{ii:4.6.10} \\
    \implies & \exists r \in \R^+ : \forall x \in E, \big(\Re(x)\big)^2 + \big(\Im(x)\big)^2 < r^2                          \\
    \implies & \exists r \in \R^+ : \forall x \in E, \begin{dcases}
                                                       \big(\Re(x)\big)^2 < r^2 \\
                                                       \big(\Im(x)\big)^2 < r^2
                                                     \end{dcases}                                               \\
    \implies & \exists r \in \R^+ : \forall x \in E, \begin{dcases}
                                                       \abs{\Re(x)} < r \\
                                                       \abs{\Im(x)} < r
                                                     \end{dcases}                                                       \\
    \implies & \exists r \in \R^+ : f^{-1}(E) \subseteq B_{(\R^2, d_2)}(0, r).
  \end{align*}
  Thus, \(\big(f^{-1}(E), d_2|_{f^{-1}(E) \times f^{-1}(E)}\big)\) is bounded.
  Since \(\big(f^{-1}(E), d_2|_{f^{-1}(E) \times f^{-1}(E)}\big)\) is closed and bounded, by \cref{ii:1.5.7} we know that \(\big(f^{-1}(E), d_2|_{f^{-1}(E) \times f^{-1}(E)}\big)\) is compact.
  Since \(E = f\big(f^{-1}(E)\big)\), by \cref{ii:2.3.1} we know that \((E, d|_{E \times E})\) is compact.
  Thus, we conclude that \((E, d|_{E \times E})\) is compact iff \((E, d|_{E \times E})\) is closed and bounded.

  Since \(\R \subseteq \C\) and \((\R, d|_{\R \times \R})\) is not compact, we conclude that \((\C, d)\) is not compact.
\end{proof}

\begin{ex}\label{ii:ex:4.6.14}
  Prove \cref{ii:4.6.14}.
\end{ex}

\begin{proof}
  See \cref{ii:4.6.14}.
\end{proof}

\begin{ex}\label{ii:ex:4.6.15}
  The purpose of this exercise is to explain why we do not try to organize the complex numbers into positive and negative parts.
  Suppose that there was a notion of a ``positive complex number'' and a ``negative complex number'' which obeyed the following reasonable axioms (cf. Proposition 4.2.9 in Analysis I):
  \begin{itemize}
    \item (Trichotomy)
          For every complex number \(z\), exactly one of the following statements is true:
          \(z\) is positive, \(z\) is negative, \(z\) is zero.
    \item (Negation)
          If \(z\) is a positive complex number, then \(-z\) is negative.
          If \(z\) is a negative complex number, then \(-z\) is positive.
    \item (Additivity)
          If \(z\) and \(w\) are positive complex numbers, then \(z + w\) is also positive.
    \item (Multiplicativity)
          If \(z\) and \(w\) are positive complex numbers, then \(zw\) is also positive.
  \end{itemize}
  Show that these four axioms are inconsistent, i.e., one can use these axioms to deduce a contradiction.
\end{ex}

\begin{proof}
  First we show that \(1\) is positive.
  Obviously \(1 \neq 0\).
  So \(1\) can only be positive or negative.
  Suppose for sake of contradiction that \(1\) is negative.
  Then by axiom above we know that \(-1\) is positive.
  But by axiom we know that \((-1)(-1) = 1\) is positive, a contradiction.
  Thus, \(1\) must be positive.

  Since \(1\) is positive, we know from axiom that \(-1\) is negative.
  Obviously \(i \neq 0\).
  So \(i\) can only be positive or negative.
  Now we split into two cases:
  \begin{itemize}
    \item If \(i\) is positive, then by axiom we know that \(i^2 = -1\) is positive.
          But this contradict to the fact that \(-1\) is negative.
    \item If \(i\) is negative, then by axiom we know that \(-i\) is positive.
          Again by axiom we know that \((-i)(-i) = -1\) is positive.
          But this contradict to the fact that \(-1\) is negative.
  \end{itemize}
  From all cases above we derived contradictions.
  Thus, the axioms are inconsistent.
\end{proof}

\begin{ex}\label{ii:ex:4.6.16}
  Prove the ratio test for complex series, and use it to show that the series used to define the complex exponential is absolutely convergent.
  Then prove that \(\exp(z + w) = \exp(z) \exp(w)\) for all complex numbers \(z, w\).
\end{ex}

\begin{proof}
  Let \(d\) be the metric in \cref{ii:4.6.10}.
  First we proof the ratio test of complex series.
  Let \((a_n)_{n = m}^\infty\) be a sequence in \(\C \setminus \set{0}\) and let \(b_n = \abs{a_n}\) for all \(n \in \set{k \in \N : k \geq m}\).
  Then we know that \((b_n)_{n = 0}^\infty\) is a sequence in \(\R\).
  Observe that
  \[
    \forall n \geq m, \dfrac{\abs{b_{n + 1}}}{\abs{b_n}} = \dfrac{\abs{\abs{a_{n + 1}}}}{\abs{\abs{a_n}}} \dfrac{\abs{b_{n + 1}}}{\abs{b_n}} = \dfrac{\abs{a_{n + 1}}}{\abs{a_n}}.
  \]
  By ratio test of real series we now split into three cases.
  \begin{itemize}
    \item \(\limsup_{n \to \infty} \dfrac{\abs{b_{n + 1}}}{\abs{b_n}} < 1\).
          Then the series \(\sum_{n = 0}^\infty b_n\) is absolutely convergent.
          But we have
          \[
            \limsup_{n \to \infty} \dfrac{\abs{b_{n + 1}}}{\abs{b_n}} < 1 \implies \limsup_{n \to \infty} \dfrac{\abs{a_{n + 1}}}{\abs{a_n}} < 1
          \]
          and
          \[
            \sum_{n = 0}^\infty b_n = \sum_{n = 0}^\infty \abs{a_n}.
          \]
          By \cref{ii:ac:4.6.8} we know that \(\sum_{n = 0}^\infty \abs{a_n} \in \R\) implies \(\sum_{n = 0}^\infty a_n \in \C\).
    \item \(\liminf_{n \to \infty} \dfrac{\abs{b_{n + 1}}}{\abs{b_n}} > 1\).
          Then the series \(\sum_{n = 0}^\infty b_n\) is divergent.
          Since
          \begin{align*}
                     & \liminf_{n \to \infty} \dfrac{\abs{a_{n + 1}}}{\abs{a_n}} > 1 \\
            \implies & \lim_{n \to \infty} \abs{a_n} \neq 0                          \\
            \implies & d - \lim_{n \to \infty} a_n \neq 0,
          \end{align*}
          by \cref{ii:ac:4.6.7} we know that \(\sum_{n = 0}^\infty a_n\) is divergent.
    \item The remaining cases.
          As in real series we cannot assert any conclusion.
  \end{itemize}
  Thus, we conclude that
  \begin{itemize}
    \item If \(\limsup_{n \to \infty} \dfrac{\abs{a_{n + 1}}}{\abs{a_n}} < 1\), then the series \(\sum_{n = 0}^\infty a_n\) is absolutely convergent.
    \item If \(\liminf_{n \to \infty} \dfrac{\abs{a_{n + 1}}}{\abs{a_n}} > 1\), then the series \(\sum_{n = 0}^\infty a_n\) is divergent.
    \item In the remaining cases, we cannot assert any conclusion.
  \end{itemize}

  Next we show that \cref{ii:4.6.15} is absolutely convergent.
  Let \(z \in \C\).
  Suppose that \(z = 0\).
  Then we have
  \begin{align*}
             & \abs{z} = 0                                   &  & \by{ii:4.6.11}   \\
    \implies & \exp(0) = \sum_{n = 0}^\infty \dfrac{0^n}{n!} &  & \by{ii:4.6.15}   \\
    \implies & \exp(0) = 1.                                  &  & \by{ii:4.5.2}[e]
  \end{align*}
  Thus, complex exponential is absolutely convergent at \(0\).
  Now suppose that \(z \neq 0\).
  By \cref{ii:4.6.10} we know that \(\abs{z} \in \R\).
  Since
  \begin{align*}
    \forall n \in \N, \dfrac{\abs{\dfrac{z^{n + 1}}{(n + 1)!}}}{\abs{\dfrac{z^n}{n!}}} & = \dfrac{\abs{z^{n + 1}} \abs{n!}}{\abs{z^n} \abs{(n + 1)!}} &  & \by{ii:ex:4.6.7} \\
                                                                                       & = \dfrac{\abs{z}^{n + 1} \abs{n!}}{\abs{z}^n \abs{(n + 1)!}} &  & \by{ii:4.6.11}   \\
                                                                                       & = \dfrac{\abs{z}}{n + 1},
  \end{align*}
  we know that
  \begin{align*}
    \limsup_{n \to \infty} \dfrac{\abs{\dfrac{z^{n + 1}}{(n + 1)!}}}{\abs{\dfrac{z^n}{n!}}} & = \limsup_{n \to \infty} \dfrac{\abs{z}}{n + 1}   \\
                                                                                            & = \abs{z} \limsup_{n \to \infty} \dfrac{1}{n + 1} \\
                                                                                            & = \abs{z} 0 = 0 < 1.
  \end{align*}
  By ratio test above we know that \(\exp(z) = \sum_{n = 0}^\infty \dfrac{z^n}{n!}\) is absolutely convergent.

  Finally we show that \(\exp(z + w) = \exp(z) \exp(w)\) for any \(z, w \in \C\).
  \begin{align*}
     & \exp(z) \exp(w)                                                                                                                        \\
     & = \bigg(\sum_{n = 0}^\infty \dfrac{z^n}{n!}\bigg) \bigg(\sum_{n = 0}^\infty \dfrac{w^n}{n!}\bigg)                &  & \by{ii:4.6.15}   \\
     & = \sum_{n = 0}^\infty \bigg(\sum_{k = 0}^n (\dfrac{z^k}{k!} \dfrac{w^{n - k}}{(n - k)!})\bigg)                   &  & \by{ii:ac:4.6.9} \\
     & = \sum_{n = 0}^\infty \Bigg(\dfrac{1}{n!} \sum_{k = 0}^n \bigg(\dfrac{n!}{k! (n - k)!} z^k w^{n - k}\bigg)\Bigg)                       \\
     & = \sum_{n = 0}^\infty \dfrac{(z + w)^n}{n!}                                                                      &  & \by{ii:ex:4.2.5} \\
     & = \exp(z + w).                                                                                                   &  & \by{ii:4.6.15}
  \end{align*}
\end{proof}

\section{Trigonometric functions}\label{ii:sec:4.7}

\begin{note}
  There are several other useful special functions in mathematics, such as the hyperbolic trigonometric functions and hypergeometric functions, the gamma and zeta functions, and elliptic functions, but they occur more rarely than trigonometric functions.
\end{note}

\begin{note}
  Trigonometric functions are often defined using geometric concepts, notably those of circles, triangles, and angles.
  However, it is also possible to define them using more analytic concepts, and in particular the (complex) exponential function.
\end{note}

\begin{defn}[Trigonometric functions]\label{ii:4.7.1}
  If \(z\) is a complex number, then we define
  \[
    \cos(z) \coloneqq \dfrac{e^{iz} + e^{-iz}}{2}
  \]
  and
  \[
    \sin(z) \coloneqq \dfrac{e^{iz} - e^{-iz}}{2i}.
  \]
  We refer to \(\cos\) and \(\sin\) as the \emph{cosine} and \emph{sine} functions respectively.
\end{defn}

\begin{note}
  \cref{ii:4.7.1} were discovered by Leonhard Euler (1707--1783) in 1748, who recognized the link between the complex exponential and the trigonometric functions.
  Since we have defined the sine and cosine for complex numbers \(z\), we automatically have defined them also for real numbers \(x\).
  In fact in most applications one is only interested in the trigonometric functions when applied to real numbers.
\end{note}

\begin{ac}\label{ii:ac:4.7.1}
  From \cref{ii:4.6.15}, we have
  \[
    e^{i z} = 1 + i z - \dfrac{z^2}{2!} - \dfrac{i z^3}{3!} + \dfrac{z^4}{4!} + \dots
  \]
  and
  \[
    e^{- i z} = 1 - i z - \dfrac{z^2}{2!} + \dfrac{i z^3}{3!} + \dfrac{z^4}{4!} - \dots
  \]
  and so from the above formulae we have
  \[
    \cos(z) = 1 - \dfrac{z^2}{2!} + \dfrac{z^4}{4!} - \dots = \sum_{n = 0}^\infty \dfrac{(-1)^n z^{2n}}{(2n)!}
  \]
  and
  \[
    \sin(z) = z - \dfrac{z^3}{3!} + \dfrac{z^5}{5!} - \dots = \sum_{n = 0}^\infty \dfrac{(-1)^n z^{2n + 1}}{(2n + 1)!}.
  \]
  In particular, \(\cos(x)\) and \(\sin(x)\) are always real whenever \(x\) is real.
  From the ratio test we see that the two power series \(\sum_{n = 0}^\infty \dfrac{(-1)^n x^{2n}}{(2n)!}\), \(\sum_{n = 0}^\infty \dfrac{(-1)^n x^{2n + 1}}{(2n + 1)!}\) are absolutely convergent for every \(x\), thus \(\sin(x)\) and \(\cos(x)\) are real analytic at \(0\) with an infinite radius of convergence.
  From \cref{ii:ex:4.2.8} we thus see that the sine and cosine functions are real analytic on all of \(\R\).
  (They are also complex analytic on all of \(\C\), but we will not pursue this matter in this text.)
  In particular the sine and cosine functions are continuous and differentiable (see \cref{ii:4.2.6}).
\end{ac}

\begin{thm}[Trigonometric identities]\label{ii:4.7.2}
  Let \(x, y\) be real numbers.
  \begin{enumerate}
    \item We have \(\big(\sin(x)\big)^2 + \big(\cos(x)\big)^2 = 1\).
          In particular, we have \(\sin(x) \in [-1, 1]\) and \(\cos(x) \in [-1, 1]\) for all \(x \in \R\).
    \item We have \(\sin'(x) = \cos(x)\) and \(\cos'(x) = -\sin(x)\).
    \item We have \(\sin(-x) = -\sin(x)\) and \(\cos(-x) = \cos(x)\).
    \item We have \(\cos(x + y) = \cos(x) \cos(y) - \sin(x) \sin(y)\) and \(\sin(x + y) = \sin(x) \cos(y) + \cos(x) \sin(y)\).
    \item We have \(\sin(0) = 0\) and \(\cos(0) = 1\).
    \item We have \(e^{i x} = \cos(x) + i \sin(x)\) and \(e^{- i x} = \cos(x) - i \sin(x)\).
          In particular \(\cos(x) = \Re(e^{i x})\) and \(\sin(x) = \Im(e^{i x})\).
  \end{enumerate}
\end{thm}

\begin{proof}{(a)}
  Let \(x \in \R\).
  Then we have
  \begin{align*}
     & \big(\sin(x)\big)^2 + \big(\cos(x)\big)^2                                                                                                                                 \\
     & = \bigg(\dfrac{e^{i x} - e^{- i x}}{2i}\bigg)^2 + \bigg(\dfrac{e^{i x} + e^{- i x}}{2}\bigg)^2                                                     &  & \by{ii:4.7.1}     \\
     & = \dfrac{e^{i x} e^{i x} - 2 e^{i x} e^{- i x} + e^{- i x} e^{- i x}}{-4} + \dfrac{e^{i x} e^{i x} + 2 e^{i x} e^{- i x} + e^{- i x} e^{- i x}}{4} &  & \by{ii:4.6.5}     \\
     & = \dfrac{4 e^{i x} e^{- i x}}{4}                                                                                                                   &  & \by{ii:4.6.4}     \\
     & = e^{i x} e^{- i x}                                                                                                                                                       \\
     & = e^{i x - i x}                                                                                                                                    &  & \by{ii:ex:4.6.16} \\
     & = e^0                                                                                                                                                                     \\
     & = 1.                                                                                                                                               &  & \by{ii:4.5.2}[d]
  \end{align*}
  By \cref{ii:ac:4.7.1} we know that \(\sin(x), \cos(x) \in \R\) when \(x \in \R\).
  Thus we have
  \begin{align*}
             & \begin{dcases}
                 \big(\sin(x)\big)^2 \leq 1 \\
                 \big(\cos(x)\big)^2 \leq 1
               \end{dcases} \\
    \implies & \begin{dcases}
                 \sin(x) \in [-1, 1] \\
                 \cos(x) \in [-1, 1]
               \end{dcases}
  \end{align*}
\end{proof}

\begin{proof}{(b)}
  Let \(x \in \R\).
  By \cref{ii:ac:4.7.1} we know that \(\sin'\) and \(\cos'\) are well-defined.
  Then we have
  \begin{align*}
     & \sin'(x)                                                                                                                                          \\
     & = \bigg(y \mapsto \dfrac{e^{i y} - e^{- i y}}{2i}\bigg)'(x)                                                                 &  & \by{ii:4.7.1}    \\
     & = \bigg(y \mapsto \dfrac{\sum_{n = 0}^\infty \dfrac{(i y)^n}{n!} - \sum_{n = 0}^\infty \dfrac{(- i y)^n}{n!}}{2i}\bigg)'(x) &  & \by{ii:4.6.15}   \\
     & = \bigg(y \mapsto \sum_{n = 0}^\infty \dfrac{(i y)^n - (- i y)^n}{(2i) (n!)}\bigg)'(x)                                      &  & \by{ii:4.6.14}   \\
     & = \bigg(y \mapsto \sum_{n = 0}^\infty \dfrac{i^n y^n - (-1)^n i^n y^n}{(2i) (n!)}\bigg)'(x)                                 &  & \by{ii:4.6.6}    \\
     & = \Bigg(y \mapsto \sum_{n = 0}^\infty \bigg(\dfrac{\big(1 - (-1)^n\big) i^n}{(2i) (n!)} y^n\bigg)\Bigg)'(x)                 &  & \by{ii:4.6.6}    \\
     & = \Bigg(y \mapsto \sum_{n = 1}^\infty \bigg(\dfrac{n \big(1 - (-1)^n\big) i^n}{(2i) (n!)} y^{n - 1}\bigg)\Bigg)(x)          &  & \by{ii:4.1.6}[d] \\
     & = \Bigg(y \mapsto \sum_{n = 1}^\infty \bigg(\dfrac{\big(1 - (-1)^n\big) i^n}{(2i) (n - 1)!} y^{n - 1}\bigg)\Bigg)(x)                              \\
     & = \bigg(y \mapsto 1 - \dfrac{y^2}{2!} + \dfrac{y^4}{4!} - \dfrac{y^6}{6!} + \dots\bigg)(x)                                                        \\
     & = \bigg(y \mapsto \sum_{n = 0}^\infty \dfrac{(-1)^n y^{2n}}{(2n)!}\bigg)(x)                                                                       \\
     & = \cos(x)                                                                                                                   &  & \by{ii:ac:4.7.1}
  \end{align*}
  and
  \begin{align*}
     & \cos'(x)                                                                                                                                         \\
     & = \bigg(y \mapsto \dfrac{e^{i y} + e^{- i y}}{2}\bigg)'(x)                                                                 &  & \by{ii:4.7.1}    \\
     & = \bigg(y \mapsto \dfrac{\sum_{n = 0}^\infty \dfrac{(i y)^n}{n!} + \sum_{n = 0}^\infty \dfrac{(- i y)^n}{n!}}{2}\bigg)'(x) &  & \by{ii:4.6.15}   \\
     & = \bigg(y \mapsto \sum_{n = 0}^\infty \dfrac{(i y)^n + (- i y)^n}{2 (n!)}\bigg)'(x)                                        &  & \by{ii:4.6.14}   \\
     & = \bigg(y \mapsto \sum_{n = 0}^\infty \dfrac{i^n y^n + (-1)^n i^n y^n}{2 (n!)}\bigg)'(x)                                   &  & \by{ii:4.6.6}    \\
     & = \Bigg(y \mapsto \sum_{n = 0}^\infty \bigg(\dfrac{\big(1 + (-1)^n\big) i^n}{2 (n!)} y^n\bigg)\Bigg)'(x)                   &  & \by{ii:4.6.6}    \\
     & = \Bigg(y \mapsto \sum_{n = 1}^\infty \bigg(\dfrac{n \big(1 + (-1)^n\big) i^n}{2 (n!)} y^{n - 1}\bigg)\Bigg)(x)            &  & \by{ii:4.1.6}[d] \\
     & = \Bigg(y \mapsto \sum_{n = 1}^\infty \bigg(\dfrac{\big(1 + (-1)^n\big) i^n}{2 (n - 1)!} y^{n - 1}\bigg)\Bigg)(x)                                \\
     & = \bigg(y \mapsto -y + \dfrac{y^3}{3!} - \dfrac{y^5}{5!} + \dfrac{y^7}{7!} + \dots\bigg)(x)                                                      \\
     & = \bigg(y \mapsto \sum_{n = 0}^\infty \dfrac{(-1)^{n + 1} y^{2n + 1}}{(2n + 1)!}\bigg)(x)                                                        \\
     & = \Bigg(y \mapsto -\bigg(\sum_{n = 0}^\infty \dfrac{(-1)^n y^{2n + 1}}{(2n + 1)!}\bigg)\Bigg)(x)                           &  & \by{ii:4.6.14}   \\
     & = \big(y \mapsto -\sin(y)\big)(x)                                                                                          &  & \by{ii:ac:4.7.1} \\
     & = -\sin(x).
  \end{align*}
\end{proof}

\begin{proof}{(c)}
  Let \(x \in \R\).
  Then we have
  \begin{align*}
    \sin(-x) & = \dfrac{e^{i (-x)} - e^{- i (-x)}}{2i} &  & \by{ii:4.7.1} \\
             & = \dfrac{e^{- i x} - e^{i x}}{2i}       &  & \by{ii:4.6.6} \\
             & = -\dfrac{e^{i x} - e^{- i x}}{2i}      &  & \by{ii:4.6.6} \\
             & = -\sin(x)                              &  & \by{ii:4.7.1}
  \end{align*}
  and
  \begin{align*}
    \cos(-x) & = \dfrac{e^{i (-x)} + e^{- i (-x)}}{2} &  & \by{ii:4.7.1} \\
             & = \dfrac{e^{- i x} + e^{i x}}{2}       &  & \by{ii:4.6.6} \\
             & = \dfrac{e^{i x} + e^{- i x}}{2}       &  & \by{ii:4.6.4} \\
             & = \cos(x).                             &  & \by{ii:4.7.1}
  \end{align*}
\end{proof}

\begin{proof}{(d)}
  Let \(x, y \in \R\).
  Then we have
  \begin{align*}
     & \sin(x) \cos(y) + \cos(x) \sin(y)                                                                                                                         \\
     & = \dfrac{e^{i x} - e^{- i x}}{2 i} \dfrac{e^{i y} + e^{- i y}}{2} + \dfrac{e^{i x} + e^{- i x}}{2} \dfrac{e^{i y} - e^{- i y}}{2i} &  & \by{ii:4.7.1}     \\
     & = \dfrac{e^{i x} e^{i y} + e^{i x} e^{- i y} - e^{- i x} e^{i y} - e^{- i x} e^{- i y}}{4i}                                        &  & \by{ii:4.6.6}     \\
     & \quad + \dfrac{e^{i x} e^{i y} - e^{i x} e^{- i y} + e^{- i x} e^{i y} - e^{- i x} e^{- i y}}{4i}                                                         \\
     & = \dfrac{2 e^{i x} e^{i y} - 2 e^{- i x} e^{- i y}}{4i}                                                                            &  & \by{ii:4.6.4}     \\
     & = \dfrac{e^{i x} e^{i y} - e^{- i x} e^{- i y}}{2i}                                                                                &  & \by{ii:4.6.12}    \\
     & = \dfrac{e^{i x + i y} - e^{- i x - i y}}{2i}                                                                                      &  & \by{ii:ex:4.6.16} \\
     & = \dfrac{e^{i (x + y)} - e^{- i (x + y)}}{2i}                                                                                      &  & \by{ii:4.6.6}     \\
     & = \sin(x + y)                                                                                                                      &  & \by{ii:4.7.1}
  \end{align*}
  and
  \begin{align*}
     & \cos(x) \cos(y) - \sin(x) \sin(y)                                                                                                                        \\
     & = \dfrac{e^{i x} + e^{- i x}}{2} \dfrac{e^{i y} + e^{- i y}}{2} - \dfrac{e^{i x} - e^{- i x}}{2i} \dfrac{e^{i y} - e^{- i y}}{2i} &  & \by{ii:4.7.1}     \\
     & = \dfrac{e^{i x} e^{i y} + e^{i x} e^{- i y} + e^{- i x} e^{i y} + e^{- i x} e^{- i y}}{4}                                        &  & \by{ii:4.6.6}     \\
     & \quad + \dfrac{e^{i x} e^{i y} - e^{i x} e^{- i y} - e^{- i x} e^{i y} + e^{- i x} e^{- i y}}{4}                                                         \\
     & = \dfrac{2 e^{i x} e^{i y} + 2 e^{- i x} e^{- i y}}{4}                                                                            &  & \by{ii:4.6.4}     \\
     & = \dfrac{e^{i x} e^{i y} + e^{- i x} e^{- i y}}{2}                                                                                &  & \by{ii:4.6.12}    \\
     & = \dfrac{e^{i x + i y} + e^{- i x - i y}}{2}                                                                                      &  & \by{ii:ex:4.6.16} \\
     & = \dfrac{e^{i (x + y)} + e^{- i (x + y)}}{2}                                                                                      &  & \by{ii:4.6.6}     \\
     & = \cos(x + y).                                                                                                                    &  & \by{ii:4.7.1}
  \end{align*}
\end{proof}

\begin{proof}{(e)}
  We have
  \begin{align*}
             & \sin(-0) = -\sin(0) &  & \by{ii:4.7.2}[c] \\
    \implies & 2 \sin(0) = 0                             \\
    \implies & \sin(0) = 0
  \end{align*}
  and
  \begin{align*}
    \cos(0) & = \dfrac{e^{i 0} + e^{- i 0}}{2} &  & \by{ii:4.7.1}    \\
            & = \dfrac{e^{0} + e^{0}}{2}       &  & \by{ii:4.6.5}    \\
            & = e^0 = 1.                       &  & \by{ii:4.5.2}[e]
  \end{align*}
\end{proof}

\begin{proof}{(f)}
  Let \(x \in \R\).
  Then we have
  \begin{align*}
     & \cos(x) + i \sin(x)                                                                      \\
     & = \dfrac{e^{i x} + e^{- i x}}{2} + i \dfrac{e^{i x} - e^{- i x}}{2i} &  & \by{ii:4.7.1}  \\
     & = \dfrac{e^{i x} + e^{- i x}}{2} + \dfrac{e^{i x} - e^{- i x}}{2}    &  & \by{ii:4.6.12} \\
     & = e^{i x}                                                            &  & \by{ii:4.6.4}
  \end{align*}
  and
  \begin{align*}
     & \cos(x) - i \sin(x)                                                                      \\
     & = \dfrac{e^{i x} + e^{- i x}}{2} - i \dfrac{e^{i x} - e^{- i x}}{2i} &  & \by{ii:4.7.1}  \\
     & = \dfrac{e^{i x} + e^{- i x}}{2} - \dfrac{e^{i x} - e^{- i x}}{2}    &  & \by{ii:4.6.12} \\
     & = e^{- i x}.                                                         &  & \by{ii:4.6.4}
  \end{align*}
  By \cref{ii:4.7.2}(a) we know that \(\sin(x), \cos(x) \in \R\).
  Thus we have
  \[
    \Re(e^{i x}) = \Re\big(\cos(x) + i \sin(x)\big) = \cos(x)
  \]
  and
  \[
    \Im(e^{i x}) = \Im\big(\cos(x) + i \sin(x)\big) = \sin(x).
  \]
\end{proof}

\begin{lem}\label{ii:4.7.3}
  There exists a positive number \(x\) such that \(\sin(x)\) is equal to \(0\).
\end{lem}

\begin{proof}
  Suppose for sake of contradiction that \(\sin(x) \neq 0\) for all \(x \in (0, \infty)\).
  Observe that this would also imply that \(\cos(x) \neq 0\) for all \(x \in (0, \infty)\), since if \(\cos(x) = 0\) then \(\sin(2x) = 0\) by \cref{ii:4.7.2}(d).
  Since \(\cos(0) = 1\), this implies by the intermediate value theorem (Theorem 9.7.1 in Analysis I) that \(\cos(x) > 0\) for all \(x > 0\)
  (since by \cref{ii:ac:4.7.1} we know that \(\cos\) is continuous on \(\R\) and by \cref{ii:4.7.2}(a) this means \(\cos(x) \in (0, 1]\)).
  Also, since \(\sin(0) = 0\) and \(\sin'(0) = 1 > 0\), we see that \(\sin\) is increasing near \(0\), hence is positive to the right of \(0\).
  By the intermediate value theorem again we conclude that \(\sin(x) > 0\) for all \(x > 0\)
  (otherwise \(\sin\) would have a zero on \((0, +\infty)\)).

  In particular if we define the cotangent function \(\cot(x) \coloneqq \cos(x) / \sin(x)\), then \(\cot(x)\) would be positive and differentiable on all of \((0, +\infty)\).
  From the quotient rule (Theorem 10.1.13(h) in Analysis I) and \cref{ii:4.7.2} we see that the derivative of \(\cot(x)\) is
  \begin{align*}
    \cot'(x) & = \dfrac{\cos'(x) \sin(x) - \cos(x) \sin'(x)}{\big(\sin(x)\big)^2}                              \\
             & = \dfrac{-\big(\sin(x)\big)^2 - \big(\cos(x)\big)^2}{\big(\sin(x)\big)^2} &  & \by{ii:4.7.2}[b] \\
             & = \dfrac{-1}{\big(\sin(x)\big)^2}.                                        &  & \by{ii:4.7.2}[a]
  \end{align*}
  In particular, we have \(\cot'(x) \leq -1\) for all \(x > 0\).
  By the fundamental theorem of calculus (Theorem 11.9.1 in Analysis I) this implies that
  \begin{align*}
             & \int_x^{x + s} \cot'(t) \; dt \leq \int_x^{x + s} -1 \; dt \\
    \implies & \cot(x + s) - \cot(x) \leq -s                              \\
    \implies & \cot(x + s) \leq \cot(x) - s
  \end{align*}
  for all \(x > 0\) and \(s > 0\).
  Now fix one \(x > 0\) and let \(s = \cot(x)\).
  Since \(s > 0\), we know that \(x + s + 1 > 0\), and thus \(\cot(x + s + 1) > 0\).
  But
  \[
    \cot(x + s + 1) \leq \cot(x) - (s + 1) = \cot(x) - \cot(x) - 1 < 0,
  \]
  a contradiction.
  Thus by letting \(s \to \infty\) we see that this contradicts our assertion that \(\cot\) is positive on \((0, \infty)\).
\end{proof}

\begin{defn}\label{ii:4.7.4}
  We define \(\pi\) to be the number
  \[
    \pi \coloneqq \inf\set{x \in (0, +\infty) : \sin(x) = 0}.
  \]
\end{defn}

\begin{ac}\label{ii:ac:4.7.2}
  The following statements are true.
  \begin{enumerate}
    \item \(\pi\) is well-defined.
    \item \(\pi > 0\).
    \item \(\sin(\pi) = 0\).
    \item \(\sin(x) > 0\) for all \(x \in (0, \pi)\).
    \item \(\cos\) is strictly decreasing on \((0, \pi)\).
    \item \(\cos(\pi) = -1\).
    \item \(e^{\pi i} = \cos(\pi) + i \sin(\pi) = -1\).
  \end{enumerate}
\end{ac}

\begin{proof}
  Let \(E\) be the set \(E \coloneqq \set{x \in (0, +\infty) : \sin(x) = 0}\), i.e., \(E\) is the set of roots of \(\sin\) on \((0, +\infty)\).
  By \cref{ii:4.7.3}, \(E\) is non-empty.
  Since \(\sin'(0) > 0\), there exists a \(c > 0\) such that \(E \subseteq [c, +\infty)\) (see \cref{ii:ex:4.7.2}).
  Also, since \(\sin\) is continuous in \([c, +\infty)\), \(E\) is closed in \([c, +\infty)\)
  (since \(\sin(E) = \set{0} = [0, 0]\) is closed in \((\R, d_{l^1}|_{\R \times \R})\), by \cref{ii:2.1.5}(d) we know that \(E = \sin^{-1}(\set{0})\) is closed in \((\R, d_{l^1}|_{\R \times \R})\)).
  Since \([c, +\infty)\) is closed in \(\R\), we conclude that \(E\) is closed in \(\R\).
  Thus \(E\) contains all its adherent points, and thus contains \(\inf(E) = \pi\) (see Definition 9.1.8 and 9.1.10 in Analysis I).

  We have \(\pi \in E \subseteq [c, +\infty)\) (so in particular \(\pi > 0\)) and \(\sin(\pi) = 0\).
  By \cref{ii:4.7.4}, \(\sin\) cannot have any zeroes in \((0, \pi)\), and so in particular must be positive on \((0, \pi)\)
  (cf. the arguments in \cref{ii:4.7.3} using the intermediate value theorem).
  Since \(\cos'(x) = -\sin(x)\), we thus conclude that \(\cos(x)\) is strictly decreasing on \((0, \pi)\).
  Since \(\cos(0) = 1\), this implies in particular that \(\cos(\pi) < 1\);
  since \(\big(\sin(\pi)\big)^2 + \big(\cos(\pi)\big)^2 = 1\) and \(\sin(\pi) = 0\), we thus conclude that \(\cos(\pi) = -1\).
  Thus we conclude by \cref{ii:4.7.2}(f) that \(e^{i \pi} = \cos(\pi) + i \sin(\pi) = -1\).
\end{proof}

\begin{thm}iodicity of trigonometric functions]\label{ii:4.7.5}
  Let \(x\) be a real number.
  \begin{enumerate}
    \item We have \(\cos(x + \pi) = -\cos(x)\) and \(\sin(x + \pi) = -\sin(x)\).
          In particular we have \(\cos(x + 2\pi) = \cos(x)\) and \(\sin(x + 2\pi) = \sin(x)\), i.e., \(\sin\) and \(\cos\) are periodic with period \(2\pi\).
    \item We have \(\sin(x) = 0\) iff \(x / \pi\) is an integer.
    \item We have \(\cos(x) = 0\) iff \(x / \pi\) is an integer plus \(1 / 2\).
  \end{enumerate}
\end{thm}

\begin{proof}{(a)}
  We have
  \begin{align*}
    \cos(x + \pi) & = \cos(x) \cos(\pi) - \sin(x) \sin(\pi) &  & \by{ii:4.7.2}[d]      \\
                  & = \cos(x) (-1) - \sin(x) 0              &  & \by{ii:ac:4.7.2}[b,f] \\
                  & = -\cos(x)
  \end{align*}
  and
  \begin{align*}
    \sin(x + \pi) & = \sin(x) \cos(\pi) + \cos(x) \sin(\pi) &  & \by{ii:4.7.2}[d]      \\
                  & = \sin(x) (-1) - \cos(x) 0              &  & \by{ii:ac:4.7.2}[b,f] \\
                  & = -\sin(x).
  \end{align*}
  Thus
  \begin{align*}
    \cos(x + 2\pi) & = \cos(x + \pi + \pi)                                    \\
                   & = -\cos(x + \pi)      &  & \text{(from the proof above)} \\
                   & = -\big(-\cos(x)\big) &  & \text{(from the proof above)} \\
                   & = \cos(x)
  \end{align*}
  and
  \begin{align*}
    \sin(x + 2\pi) & = \sin(x + \pi + \pi)                                    \\
                   & = -\sin(x + \pi)      &  & \text{(from the proof above)} \\
                   & = -\big(-\sin(x)\big) &  & \text{(from the proof above)} \\
                   & = \sin(x).
  \end{align*}
\end{proof}

\begin{proof}{(b)}
  First we show that \(\dfrac{x}{\pi} \in \Z\) implies \(\sin(x) = 0\).
  Let \(P(n)\) be the statement ``\(n = \dfrac{x}{\pi}\) implies \(\sin(x) = 0\).''
  We use induction to show that \(P(n)\) is true for all \(n \in \N\).
  For \(n = 0\), we have
  \begin{align*}
             & 0 = \dfrac{x}{\pi}                           \\
    \implies & x = 0                                        \\
    \implies & \sin(x) = \sin(0) = 0. &  & \by{ii:4.7.2}[e]
  \end{align*}
  Thus the base holds.
  Suppose inductively that \(P(n)\) is true for some \(n \geq 0\).
  Then for \(n + 1\), we have
  \begin{align*}
             & n + 1 = \dfrac{x}{\pi}                                           \\
    \implies & x = n \pi + \pi                                                  \\
    \implies & \sin(x) = \sin(n \pi + \pi) = -\sin(n \pi) &  & \by{ii:4.7.4}[a] \\
    \implies & \sin(x) = -\sin(n \pi) = 0.                &  & \byIH
  \end{align*}
  This closes the induction.
  Thus \(P(n)\) is true for all \(n \in \N\).
  Since
  \begin{align*}
             & \forall n \in \Z^-, n = \dfrac{x}{\pi}                                    \\
    \implies & -n = -\dfrac{x}{\pi} \in \Z^+                                             \\
    \implies & \sin(-x) = \sin(- n \pi) = 0           &  & \text{(from the proof above)} \\
    \implies & -\sin(x) = 0                           &  & \by{ii:4.7.2}[c]              \\
    \implies & \sin(x) = 0,
  \end{align*}
  we know that \(P(n)\) is true for all \(n \in \Z\).

  Now we show that \(\sin(x) = 0\) implies \(\dfrac{x}{\pi}\) is an integer.
  Let \(x \in \R\).
  Suppose that \(\sin(x) = 0\).
  Now we split into three cases:
  \begin{itemize}
    \item \(x = 0\).
          By \cref{ii:ac:4.7.2}(b) we know that \(\pi > 0\), thus we know that \(\dfrac{0}{\pi} = 0 \in \N\) and by \cref{ii:4.7.2}(e) we have \(\sin(0) = 0\).
    \item \(x \in \R^+\).
          Since \(\pi > 0\), by Archimedean property we know that
          \begin{align*}
                     & \exists n \in \N : n \pi \leq x < (n + 1) \pi \\
            \implies & \exists n \in \N : 0 \leq x - n \pi < \pi.
          \end{align*}
          Fix such \(n\).
          By \cref{ii:ac:4.7.2}(d) we know that \(\sin(y) > 0\) for all \(y \in (0, \pi)\).
          Thus we must have \(x - n \pi = 0\).
          This means \(\dfrac{x}{\pi} = n \in \N\).
    \item \(x \in \R^-\).
          Then we have
          \begin{align*}
                     & \sin(x) = 0                                                \\
            \implies & -\big(-\sin(x)\big) = 0                                    \\
            \implies & -\sin(-x) = 0           &  & \by{ii:4.7.2}[e]              \\
            \implies & \sin(-x) = 0                                               \\
            \implies & \dfrac{-x}{\pi} \in \N  &  & \text{(from the proof above)} \\
            \implies & \dfrac{x}{\pi} \in \Z.
          \end{align*}
  \end{itemize}
  From all cases above we conclude that
  \[
    \sin(x) = 0 \implies \dfrac{x}{\pi} \in \Z.
  \]
\end{proof}

\begin{proof}{(c)}
  By \cref{ii:ac:4.7.2}(d) we know that \(\sin(\dfrac{\pi}{2}) > 0\).
  Since
  \begin{align*}
    0 & = \sin(\pi)                                                                             &  & \by{ii:ac:4.7.2}[c] \\
      & = \sin(\dfrac{\pi}{2} + \dfrac{\pi}{2})                                                                          \\
      & = \sin(\dfrac{\pi}{2}) \cos(\dfrac{\pi}{2}) + \cos(\dfrac{\pi}{2}) \sin(\dfrac{\pi}{2}) &  & \by{ii:4.7.2}[d]    \\
      & = 2 \sin(\dfrac{\pi}{2}) \cos(\dfrac{\pi}{2}),
  \end{align*}
  we know that \(\cos(\dfrac{\pi}{2}) = 0\).
  Since
  \begin{align*}
             & \big(\sin(\dfrac{\pi}{2})\big)^2 + \big(\cos(\dfrac{\pi}{2})\big)^2 = 1 &  & \by{ii:4.7.2}[a]              \\
    \implies & \big(\sin(\dfrac{\pi}{2})\big)^2 = 1                                    &  & \text{(from the proof above)} \\
    \implies & \sin(\dfrac{\pi}{2}) = 1,                                               &  & \by{ii:ac:4.7.2}[d]
  \end{align*}
  We know that
  \begin{align*}
    \forall x \in \R, \cos(x) & = \cos(x + \dfrac{\pi}{2} - \dfrac{\pi}{2})                                                                                          \\
                              & = \cos(x + \dfrac{\pi}{2}) \cos(-\dfrac{\pi}{2}) - \sin(x + \dfrac{\pi}{2}) \sin(-\dfrac{\pi}{2}) &  & \by{ii:4.7.2}[d]              \\
                              & = \cos(x + \dfrac{\pi}{2}) \cos(\dfrac{\pi}{2}) + \sin(x + \dfrac{\pi}{2}) \sin(\dfrac{\pi}{2})   &  & \by{ii:4.7.2}[c]              \\
                              & = \cos(x + \dfrac{\pi}{2}) \times 0 + \sin(x + \dfrac{\pi}{2}) \times 1                           &  & \text{(from the proof above)} \\
                              & = \sin(x + \dfrac{\pi}{2}).
  \end{align*}
  Thus we have
  \begin{align*}
         & \forall x \in \R, \bigg(\sin(x + \dfrac{\pi}{2}) = 0 \iff \dfrac{x + \dfrac{\pi}{2}}{\pi} \in \Z\bigg)  &  & \by{ii:4.7.5}[b]              \\
    \iff & \forall x \in \R, \bigg(\sin(x + \dfrac{\pi}{2}) = 0 \iff \dfrac{x}{\pi} + \dfrac{1}{2} \in \Z\bigg)                                       \\
    \iff & \forall x \in \R, \bigg(\cos(x) = 0 \iff \dfrac{x}{\pi} + \dfrac{1}{2} \in \Z\bigg)                     &  & \text{(from the proof above)} \\
    \iff & \forall x \in \R, \bigg(\cos(x) = 0 \iff \exists n \in \Z : n = \dfrac{x}{\pi} + \dfrac{1}{2}\bigg)                                        \\
    \iff & \forall x \in \R, \bigg(\cos(x) = 0 \iff \exists n \in \Z : n + 1 = \dfrac{x}{\pi} + \dfrac{1}{2}\bigg)                                    \\
    \iff & \forall x \in \R, \bigg(\cos(x) = 0 \iff \exists n \in \Z : n + \dfrac{1}{2} = \dfrac{x}{\pi}\bigg).
  \end{align*}
\end{proof}

\exercisesection

\begin{ex}\label{ii:ex:4.7.1}
  Prove \cref{ii:4.7.2}.
\end{ex}

\begin{proof}
  See \cref{ii:4.7.2}.
\end{proof}

\begin{ex}\label{ii:ex:4.7.2}
  Let \(f : \R \to \R\) be a function which is differentiable at \(x_0\), with \(f(x_0) = 0\) and \(f'(x_0) \neq 0\).
  Show that there exists a \(c > 0\) such that \(f(y)\) is non-zero whenever \(0 < \abs{x_0 - y} < c\).
  Conclude in particular that there exists a \(c > 0\) such that \(\sin(x) \neq 0\) for all \(0 < x < c\).
\end{ex}

\begin{proof}
  Since
  \begin{align*}
             & f'(x_0) \neq 0                                                                                                                                                                                \\
    \implies & \lim_{y \to x_0 ; y \in \R \setminus \set{x_0}} \dfrac{f(y) - f(x_0)}{y - x_0} = \lim_{y \to x_0 ; y \in \R \setminus \set{x_0}} \dfrac{f(y)}{y - x_0} \neq 0                                 \\
    \implies & \forall \varepsilon \in \R^+, \exists \delta \in \R^+ : \forall y \in \R \setminus \set{x_0}, \bigg(\abs{y - x_0} < \delta \implies \abs{\dfrac{f(y)}{y - x_0} - f'(x_0)} < \varepsilon\bigg) \\
    \implies & \forall \varepsilon \in \R^+, \exists \delta \in \R^+ : \forall y \in \R, \bigg(0 < \abs{y - x_0} < \delta \implies \abs{\dfrac{f(y)}{y - x_0} - f'(x_0)} < \varepsilon\bigg)                 \\
    \implies & \exists \delta \in \R^+ : \forall y \in \R, \bigg(0 < \abs{y - x_0} < \delta \implies \abs{\dfrac{f(y)}{y - x_0} - f'(x_0)} < \dfrac{\abs{f'(x_0)}}{2}\bigg)                                  \\
    \implies & \exists \delta \in \R^+ : \forall y \in \R,                                                                                                                                                   \\
             & \bigg(0 < \abs{y - x_0} < \delta \implies f'(x_0) - \dfrac{\abs{f'(x_0)}}{2} < \dfrac{f(y)}{y - x_0} < f'(x_0) + \dfrac{\abs{f'(x_0)}}{2}\bigg),
  \end{align*}
  by setting \(c = \delta\) we know that
  \begin{align*}
             & 0 < \abs{x - x_0} < c                                                                                               \\
    \implies & \begin{dcases}
                 \dfrac{3 f'(x_0)}{2} < \dfrac{f(y)}{y - x_0} < \dfrac{f'(x_0)}{2} < 0 & \text{if} f'(x_0) < 0 \\
                 0 < \dfrac{f'(x_0)}{2} < \dfrac{f(y)}{y - x_0} < \dfrac{3 f'(x_0)}{2} & \text{if} f'(x_0) > 0
               \end{dcases}                       \\
    \implies & f(y) \neq 0.                                                                                     & (y - x_0 \neq 0)
  \end{align*}

  By \cref{ii:4.7.2}(b)(e) we know that \(\sin'(0) = \cos(0) = 1\), thus we conclude that
  \[
    \exists c \in \R^+ : \forall x \in (0, c), \sin(x) \neq 0.
  \]
\end{proof}

\begin{ex}\label{ii:ex:4.7.3}
  Prove \cref{ii:4.7.5}.
\end{ex}

\begin{proof}
  See \cref{ii:4.7.5}.
\end{proof}

\begin{ex}\label{ii:ex:4.7.4}
  Let \(x, y\) be real numbers such that \(x^2 + y^2 = 1\).
  Show that there is exactly one real number \(\theta \in (-\pi, \pi]\) such that \(x = \sin(\theta)\) and \(y = \cos(\theta)\).
\end{ex}

\begin{proof}
  Observe that
  \begin{align*}
             & x^2 + y^2 = 1     \\
    \implies & x, y \in [-1, 1].
  \end{align*}
  We split into three cases:
  \begin{itemize}
    \item \(x = 0\).
          Then we know that \(y = \pm 1\).
          Since
          \begin{align*}
                     & \theta \in (-\pi, \pi]                                                                   \\
            \implies & \dfrac{\theta}{\pi} \in (-1, 1]                                                          \\
            \implies & \bigg(\sin(\theta) = 0 \iff \dfrac{\theta}{\pi} = \set{0, 1}\bigg) &  & \by{ii:4.7.5}[b] \\
            \implies & \bigg(\sin(\theta) = 0 \iff \theta = \set{0, \pi}\bigg)
          \end{align*}
          and
          \begin{align*}
             & \cos(0) = 1;    &  & \by{ii:4.7.2}[e]    \\
             & \cos(\pi) = -1, &  & \by{ii:ac:4.7.2}[f]
          \end{align*}
          we know that
          \begin{align*}
             & (x, y) = (0, 1) = \big(\sin(0), \cos(0)\big) \iff \theta = 0;        \\
             & (x, y) = (0, -1) = \big(\sin(\pi), \cos(\pi)\big) \iff \theta = \pi.
          \end{align*}
    \item \(x \in (0, 1]\).
          Then we have
          \[
            x^2 + y^2 = 1 \implies y = \pm \sqrt{1 - x^2} \in (-1, 1).
          \]
          Since
          \begin{align*}
                     & z \in (0, \pi)                          \\
            \implies & \sin(z) > 0    &  & \by{ii:ac:4.7.2}[d] \\
            \implies & -\sin(z) < 0                            \\
            \implies & \sin(-z) < 0,  &  & \by{ii:4.7.2}[c]
          \end{align*}
          we know that
          \[
            \forall z \in (-\pi, 0), \sin(z) < 0.
          \]
          Thus
          \begin{align*}
                     & \sin(\theta) = x > 0 \\
            \implies & \theta \in (0, \pi).
          \end{align*}
          Since \(\sin(\dfrac{\pi}{2}) = 1\) (cf. the proof of \cref{ii:4.7.5}(c)) and \(\sin\) is continuous on \(\R\) (by \cref{ii:ac:4.7.1}), by intermediate value theorem we know that
          \begin{align*}
             & \exists \theta_1 \in (0, \dfrac{\pi}{2}] : \sin(\theta_1) = x   & \text{since } \sin\big((0, \dfrac{\pi}{2}]\big) \subseteq (0, 1]   \\
             & \exists \theta_2 \in [\dfrac{\pi}{2}, \pi) : \sin(\theta_2) = x & \text{since } \sin\big([\dfrac{\pi}{2}, \pi)\big) \subseteq (0, 1]
          \end{align*}
          Since \(\cos(0) = 1\) (by \cref{ii:4.7.2}(e)) and \(\cos(\dfrac{\pi}{2}) = 0\) (cf. the proof of \cref{ii:4.7.5}(c)), we know that
          \begin{align*}
                     & \cos \text{ is strictly decreasing on } (0, \pi)    &  & \by{ii:ac:4.7.2}[e] \\
            \implies & \cos\big((0, \dfrac{\pi}{2}]\big) \subseteq [0, 1).
          \end{align*}
          Using similar arguments we can show that \(\cos\big([\dfrac{\pi}{2}, \pi)\big) \subseteq (-1, 0]\).
          Thus we have
          \begin{align*}
             & \big(x \in (0, 1]\big) \land \big(y \in [0, 1)\big) \implies \exists \theta_1 \in (0, \dfrac{\pi}{2}] : \big(\sin(\theta_1) = x\big) \land \big(\cos(\theta_1) = y\big);    \\
             & \big(x \in (0, 1]\big) \land \big(y \in (-1, 0]\big) \implies \exists \theta_2 \in [\dfrac{\pi}{2}, \pi) : \big(\sin(\theta_2) = x\big) \land \big(\cos(\theta_2) = y\big).
          \end{align*}
          But \(\cos\) is strictly decreasing on \((0, \pi)\) implies the choices of \(\theta_1\) and \(\theta_2\) are unique.
          And we conclude that
          \[
            \forall x \in (0, 1], \exists!\ \theta \in (0, \pi) : \begin{dcases}
              \sin(\theta) = x \\
              \cos(\theta) = y \\
              x^2 + y^2 = 1
            \end{dcases}
          \]
    \item \(x \in [-1, 0)\).
          Then we have
          \begin{align*}
                     & -x \in (0, 1]                                                                    \\
            \implies & \exists!\ \theta \in (0, \pi) : \begin{dcases}
                                                         \sin(\theta) = -x \\
                                                         \cos(\theta) = y  \\
                                                         (-x)^2 + y^2 = x^2 + y^2 = 1
                                                       \end{dcases}  &  & \text{(from the proof above)} \\
            \implies & \exists!\ \theta \in (0, \pi) : \begin{dcases}
                                                         \sin(-\theta) = -\sin(\theta) = x \\
                                                         \cos(-\theta) = \cos(\theta) = y  \\
                                                         x^2 + y^2 = 1
                                                       \end{dcases}  &  & \by{ii:4.7.2}[c]              \\
            \implies & \exists!\ \theta \in (-\pi, 0) : \begin{dcases}
                                                          \sin(\theta) = x \\
                                                          \cos(\theta) = y \\
                                                          x^2 + y^2 = 1
                                                        \end{dcases}
          \end{align*}
  \end{itemize}
  From all cases above we conclude that
  \[
    \exists!\ \theta \in (-\pi, \pi] : \begin{dcases}
      \sin(\theta) = x \\
      \cos(\theta) = y \\
      x^2 + y^2 = 1
    \end{dcases}
  \]
\end{proof}

\begin{ex}\label{ii:ex:4.7.5}
  Show that if \(r, s > 0\) are positive real numbers, and \(\theta, \alpha\) are real numbers such that \(r e^{i \theta} = s e^{i \alpha}\), then \(r = s\) and \(\theta = \alpha + 2 \pi k\) for some integer \(k\).
\end{ex}

\begin{proof}
  By \cref{ii:ac:4.7.2}(g) we know that
  \begin{align*}
    r e^{i \theta} & = r \big(\cos(\theta) + i \sin(\theta)\big); \\
    s e^{i \alpha} & = s \big(\cos(\alpha) + i \sin(\alpha)\big).
  \end{align*}
  Since
  \begin{align*}
             & r e^{i \theta} = s e^{i \alpha}                                                       \\
    \implies & r \big(\cos(\theta) + i \sin(\theta)\big) = s \big(\cos(\alpha) + i \sin(\alpha)\big) \\
    \implies & r \cos(\theta) - s \cos(\alpha) + i \big(r \sin(\theta) - s \sin(\alpha)\big) = 0     \\
    \implies & \begin{dcases}
                 r \cos(\theta) - s \cos(\alpha) = 0 \\
                 r \sin(\theta) - s \sin(\alpha) = 0
               \end{dcases}                                                   \\
    \implies & \begin{dcases}
                 r \cos(\theta) = s \cos(\alpha) \\
                 r \sin(\theta) = s \sin(\alpha)
               \end{dcases}
  \end{align*}
  we know that
  \begin{align*}
             & (r \sin(\theta))^2 + (r \cos(\theta))^2 = (s \sin(\alpha))^2 + (s \cos(\alpha))^2                                                                                          \\
    \implies & r^2 \Big(\big(\sin(\theta)\big)^2 + \big(\cos(\theta)\big)^2\Big) = s^2 \Big(\big(\sin(\alpha)\big)^2 + \big(\cos(\alpha)\big)^2\Big)                                      \\
    \implies & r^2 = s^2                                                                                                                             &                 & \by{ii:4.7.2}[a] \\
    \implies & r = s.                                                                                                                                & (r, s \in \R^+)
  \end{align*}
  Thus we have
  \begin{align*}
             & \begin{dcases}
                 r \cos(\theta) = s \cos(\alpha) \\
                 r \sin(\theta) = s \sin(\alpha)
               \end{dcases}     \\
    \implies & \begin{dcases}
                 \cos(\theta) = \cos(\alpha) \\
                 \sin(\theta) = \sin(\alpha)
               \end{dcases}     & (r = s \in \R^+)
  \end{align*}
  and
  \begin{align*}
    \sin(\theta - \alpha) & = \sin(\theta) \cos(-\alpha) + \cos(\theta) \sin(-\alpha) &  & \by{ii:4.7.2}[d]              \\
                          & = \sin(\theta) \cos(\alpha) - \cos(\theta) \sin(\alpha)   &  & \by{ii:4.7.2}[c]              \\
                          & = \sin(\alpha) \cos(\alpha) - \cos(\alpha) \sin(\alpha)   &  & \text{(from the proof above)} \\
                          & = 0.
  \end{align*}
  By \cref{ii:4.7.5}(b) we know that
  \begin{align*}
         & \sin(\theta - \alpha) = 0                           \\
    \iff & \dfrac{\theta - \alpha}{\pi} \in \Z                 \\
    \iff & \exists k \in \Z : k = \dfrac{\theta - \alpha}{\pi} \\
    \iff & \exists k \in \Z : k \pi = \theta - \alpha          \\
    \iff & \exists k \in \Z : \theta = \alpha + k \pi.
  \end{align*}
  By \cref{ii:4.7.5} we know that for any \(\alpha \in \R\), \(\cos(\alpha + k \pi) = \cos(\alpha)\) when \(k\) is even.
  So we only need to show that for any \(\alpha \in \R\), \(\cos(\alpha + k \pi) \neq \cos(\alpha)\) when \(k\) is odd.
  Suppose for sake of contradiction that for any \(\alpha \in \R\), \(\cos(\alpha + k \pi) = \cos(\alpha)\) when \(k\) is odd.
  Let \(k = 2n + 1\) for some \(n \in \Z\).
  Then we have
  \begin{align*}
    \cos(\theta) & = \cos(\alpha + k \pi)                                                   \\
                 & = \cos\big(\alpha + (2n + 1) \pi\big)                                    \\
                 & = \cos(\alpha + 2 n \pi + \pi)                                           \\
                 & = -\cos(\alpha + 2 n \pi)             &  & \by{ii:4.7.5}[a]              \\
                 & = -\cos(\alpha)                       &  & \by{ii:4.7.5}[a]              \\
                 & = -\cos(\theta).                      &  & \text{(from the proof above)}
  \end{align*}
  This means \(\cos(\theta) = 0\).
  But by \cref{ii:4.7.5}(c) we know that
  \begin{align*}
             & \exists m \in \Z : m + \dfrac{1}{2} = \dfrac{\theta}{\pi} = \dfrac{\alpha}{\pi} + k \\
    \implies & \exists m \in \Z : m - k + \dfrac{1}{2} = \dfrac{\alpha}{\pi} \notin \Z.
  \end{align*}
  Thus when \(\alpha = \pi\) we derive contradiction.
  We conclude that
  \[
    \forall k \in \Z, \theta = \alpha + 2 k \pi.
  \]
\end{proof}

\begin{ex}\label{ii:ex:4.7.6}
  Let \(z\) be a non-zero complex number.
  Using \cref{ii:ex:4.7.4}, show that there is exactly one pair of real numbers \(r, \theta\) such that \(r > 0\), \(\theta \in (-\pi, \pi]\), and \(z = r e^{i \theta}\).
  (This is sometimes known as the \emph{standard polar representation} of \(z\).)
\end{ex}

\begin{proof}
  Observe that
  \begin{align*}
             & z \neq 0                                                                                                                                                          \\
    \implies & \abs{z} > 0                                                                                                                                 &  & \by{ii:4.6.11}   \\
    \implies & \dfrac{z}{\abs{z}} \in \C                                                                                                                                         \\
    \implies & \abs{\dfrac{z}{\abs{z}}} = \dfrac{\abs{z}}{\abs{\abs{z}}} = \dfrac{\abs{z}}{\abs{z}} = 1                                                    &  & \by{ii:ex:4.6.7} \\
    \implies & \big(\Re(\dfrac{z}{\abs{z}})\big)^2 + \big(\Im(\dfrac{z}{\abs{z}})\big)^2 = 1                                                               &  & \by{ii:4.6.10}   \\
    \implies & \exists!\ \theta \in (-\pi, \pi] : \begin{dcases}
                                                    \cos(\theta) = \Re(\dfrac{z}{\abs{z}}) \\
                                                    \sin(\theta) = \Im(\dfrac{z}{\abs{z}})
                                                  \end{dcases}                                                                    &  & \by{ii:ex:4.7.4}                          \\
    \implies & \exists!\ \theta \in (-\pi, \pi] : \dfrac{z}{\abs{z}} = \Re(\dfrac{z}{\abs{z}}) + i \Im(\dfrac{z}{\abs{z}}) = \cos(\theta) + i \sin(\theta) &  & \by{ii:4.6.8}    \\
    \implies & \exists!\ \theta \in (-\pi, \pi] : \dfrac{z}{\abs{z}} = e^{i \theta}                                                                        &  & \by{ii:4.7.2}[f] \\
    \implies & \exists!\ \theta \in (-\pi, \pi] : z = \abs{z} e^{i \theta}.                                                                                &  & \by{ii:4.7.2}[f]
  \end{align*}
  By setting \(r = \abs{z}\) we are done.
\end{proof}

\begin{ex}\label{ii:ex:4.7.7}
  For any real number \(\theta\) and integer \(n\), prove the \emph{de Moivre identities}
  \[
    \cos(n \theta) = \Re\Big(\big(\cos(\theta) + i \sin(\theta)\big)^n\Big); \quad \sin(n \theta) = \Im\Big(\big(\cos(\theta) + i \sin(\theta)\big)^n\Big).
  \]
\end{ex}

\begin{proof}
  By \cref{ii:4.7.2}(a) we know that \(\big(\cos(\theta)\big)^2 + \big(\sin(\theta)\big)^2 = 1\), thus we cannot have \(\cos(\theta) = 0\) and \(\sin(\theta) = 0\) at the same time.
  By \cref{ii:4.6.12} this means \(\big(\cos(\theta) + i \sin(\theta)\big)^{-1}\) is well-defined and
  \[
    \big(\cos(\theta) + i \sin(\theta)\big)^0 = \big(\cos(\theta) + i \sin(\theta)\big) \big(\cos(\theta) + i \sin(\theta)\big)^{-1} = 1.
  \]
  First suppose that \(n = 0\).
  Then we have
  \begin{align*}
    \cos(0 \theta) & = \cos(0)                                                                      \\
                   & = 1                                                      &  & \by{ii:4.7.2}[e] \\
                   & = \Re(1)                                                 &  & \by{ii:4.6.8}    \\
                   & = \Re\Big(\big(\cos(\theta) + i \sin(\theta)\big)^0\Big) &  & \by{ii:4.6.12}
  \end{align*}
  and
  \begin{align*}
    \sin(0 \theta) & = \sin(0)                                                                      \\
                   & = 0                                                      &  & \by{ii:4.7.2}[e] \\
                   & = \Im(1)                                                 &  & \by{ii:4.6.8}    \\
                   & = \Im\Big(\big(\cos(\theta) + i \sin(\theta)\big)^0\Big) &  & \by{ii:4.6.12}
  \end{align*}

  Next suppose that \(n \in \Z^+\).
  Since
  \begin{align*}
    \big(\cos(\theta) + i \sin(\theta)\big)^n & = (e^{i \theta})^n &  & \by{ii:4.7.2}[f]  \\
                                              & = e^{n i \theta}   &  & \by{ii:ex:4.6.16} \\
                                              & = e^{i n \theta},  &  & \by{ii:4.6.6}
  \end{align*}
  we know that
  \begin{align*}
    \cos(n \theta) & = \Re(e^{i n \theta})                                    &  & \by{ii:4.7.2}[f]              \\
                   & = \Re\Big(\big(\cos(\theta) + i \sin(\theta)\big)^n\Big) &  & \text{(from the proof above)}
  \end{align*}
  and
  \begin{align*}
    \sin(n \theta) & = \Im(e^{i n \theta})                                     &  & \by{ii:4.7.2}[f]              \\
                   & = \Im\Big(\big(\cos(\theta) + i \sin(\theta)\big)^n\Big). &  & \text{(from the proof above)}
  \end{align*}

  Finally suppose that \(n \in \Z^-\).
  Let \(k \in \Z^+\) such that \(-k = n\).
  Since
  \begin{align*}
     & \big(\cos(\theta) + i \sin(\theta)\big)^{-1}                                                                                             \\
     & = \abs{\cos(\theta) + i \sin(\theta)}^{-2} \big(\overline{\cos(\theta) + i \sin(\theta)}\big)                      &  & \by{ii:4.6.12}   \\
     & = \Big(\big(\cos(\theta)\big)^2 + \big(\sin(\theta)\big)^2\Big) \big(\overline{\cos(\theta) + i \sin(\theta)}\big) &  & \by{ii:4.6.10}   \\
     & = \overline{\cos(\theta) + i \sin(\theta)}                                                                         &  & \by{ii:4.7.2}[a] \\
     & = \cos(\theta) - i \sin(\theta)                                                                                                          \\
     & = e^{- i \theta},                                                                                                  &  & \by{ii:4.7.2}[f]
  \end{align*}
  we know that
  \begin{align*}
     & \big(\cos(\theta) + i \sin(\theta)\big)^n                                                     \\
     & = \big(\cos(\theta) + i \sin(\theta)\big)^{-k}                                                \\
     & = \Big(\big(\cos(\theta) + i \sin(\theta)\big)^{-1}\Big)^k &  & \by{ii:4.6.12}                \\
     & = (e^{- i \theta})^k                                       &  & \text{(from the proof above)} \\
     & = e^{k (- i \theta)}                                       &  & \by{ii:ex:4.6.16}             \\
     & = e^{- i k \theta}                                         &  & \by{ii:4.6.6}                 \\
     & = \cos(k \theta) - i \sin(k \theta)                        &  & \by{ii:4.7.2}[f]              \\
     & = \cos(- k \theta) + i \sin(- k \theta)                    &  & \by{ii:4.7.2}[c]              \\
     & = \cos(n \theta) + i \sin(n \theta).
  \end{align*}
  Thus we have
  \begin{align*}
    \cos(n \theta) & = \Re\big(\cos(n \theta) + i \sin(n \theta)\big)         &  & \by{ii:4.6.8}                 \\
                   & = \Re\Big(\big(\cos(\theta) + i \sin(\theta)\big)^n\Big) &  & \text{(from the proof above)}
  \end{align*}
  and
  \begin{align*}
    \sin(n \theta) & = \Im\big(\cos(n \theta) + i \sin(n \theta)\big)          &  & \by{ii:4.6.8}                 \\
                   & = \Im\Big(\big(\cos(\theta) + i \sin(\theta)\big)^n\Big). &  & \text{(from the proof above)}
  \end{align*}
\end{proof}

\begin{ex}\label{ii:ex:4.7.8}
  Let \(\tan : (- \pi / 2, \pi / 2) \to \R\) be the tangent function \(\tan(x) \coloneqq \sin(x) / \cos(x)\).
  Show that \(\tan\) is differentiable and monotone increasing, with
  \[
    \dfrac{d}{dx} \tan(x) = 1 + \big(\tan(x)\big)^2,
  \]
  and that \(\lim_{x \to \pi / 2} \tan(x) = +\infty\) and \(\lim_{x \to -\pi / 2} \tan(x) = -\infty\).
  Conclude that \(\tan\) is in fact a bijection from \((- \pi / 2, \pi / 2) \to \R\), and thus has an inverse function \(\tan^{-1} : \R \to (- \pi / 2, \pi / 2)\)
  (this function is called the \emph{arctangent function}).
  Show that \(\tan^{-1}\) is differentiable and \(\dfrac{d}{dx} \tan^{-1}(x) = \dfrac{1}{1 + x^2}\).
\end{ex}

\begin{proof}
  By \cref{ii:4.7.5}(c) we know that \(\cos(-\dfrac{\pi}{2}) = \cos(\dfrac{\pi}{2}) = 0\) and \(\cos(x) \neq 0\) for all \(x \in (-\dfrac{\pi}{2}, \dfrac{\pi}{2})\).
  Thus \(\tan(x)\) is well-defined on \((-\dfrac{\pi}{2}, \dfrac{\pi}{2})\).
  By \cref{ii:ac:4.7.1} we know that \(\sin\) and \(\cos\) are differentiable on \(\R\), thus by Theorem 10.1.13(h) in Analysis I we know that \(\tan\) is differentiable on \((-\dfrac{\pi}{2}, \dfrac{\pi}{2})\).
  In particular, for all \(x \in (-\dfrac{\pi}{2}, \dfrac{\pi}{2})\), we have
  \begin{align*}
    \tan'(x) & = \dfrac{\sin'(x) \cos(x) - \sin(x) \cos'(x)}{\big(\cos(x)\big)^2}                             \\
             & = \dfrac{\big(\cos(x)\big)^2 + \big(\sin(x)\big)^2}{\big(\cos(x)\big)^2} &  & \by{ii:4.7.2}[b] \\
             & = 1 + \big(\tan(x)\big)^2.
  \end{align*}
  Since \(\tan'(x) > 0\) for all \(x \in (-\dfrac{\pi}{2}, \dfrac{\pi}{2})\), by Proposition 10.3.3 we know that \(\tan\) is strictly monotone increasing on \((-\dfrac{\pi}{2}, \dfrac{\pi}{2})\).

  Since \(\sin(\dfrac{\pi}{2}) = 1\) (cf. the proof of \cref{ii:4.7.5}(c)), by \cref{ii:4.7.2}(b) we know that \(\sin(-\dfrac{\pi}{2}) = -1\).
  Since \(\sin\) and \(\cos\) are continuous on \(\R\), we know that
  \begin{align*}
     & \lim_{x \to \dfrac{\pi}{2}} \sin(x) = \sin(\dfrac{\pi}{2}) = 1    \\
     & \lim_{x \to -\dfrac{\pi}{2}} \sin(x) = \sin(-\dfrac{\pi}{2}) = -1 \\
     & \lim_{x \to \dfrac{\pi}{2}} \cos(x) = \cos(\dfrac{\pi}{2}) = 0    \\
     & \lim_{x \to -\dfrac{\pi}{2}} \cos(x) = \cos(-\dfrac{\pi}{2}) = 0
  \end{align*}
  Since \(\tan\) is monotone increasing on \((-\dfrac{\pi}{2}, \dfrac{\pi}{2})\) and \(\tan(0) = \dfrac{\sin(0)}{\cos(0)} = 0\), we know that \(\tan(x) > 0\) for all \(x \in (0, \dfrac{\pi}{2})\) and \(\tan(x) < 0\) for all \(x \in (-\dfrac{\pi}{2})\).
  Suppose for sake of contradiction that \(\lim_{x \to \dfrac{\pi}{2}} \tan(x) \in \R\).
  Then we have
  \begin{align*}
             & \begin{dcases}
                 \forall x \in (0, \dfrac{\pi}{2}), \tan(x) > 0 \\
                 \tan \text{ is strictly monotone increasing on } (-\dfrac{\pi}{2}, \dfrac{\pi}{2})
               \end{dcases}                                                                                                                                                                                      \\
    \implies & \lim_{x \to \dfrac{\pi}{2}} \tan(x) \in \R^+                                                                                                                                                                                                                           \\
    \implies & \dfrac{\lim_{x \to \dfrac{\pi}{2}} \tan(x)}{\sin(\dfrac{\pi}{2})} = \dfrac{\lim_{x \to \dfrac{\pi}{2}} \tan(x)}{\lim_{x \to \dfrac{\pi}{2}} \sin(x)} = \lim_{x \to \dfrac{\pi}{2}} \dfrac{\tan(x)}{\sin(x)} = \lim_{x \to \dfrac{\pi}{2}} \dfrac{1}{\cos(x)} \in \R^+.
  \end{align*}
  But this contradict to the fact that \(\lim_{x \to \dfrac{\pi}{2}} \dfrac{1}{\cos(x)} = +\infty\).
  Thus we know that
  \[
    \lim_{x \to \dfrac{\pi}{2}} \tan(x) \in \set{-\infty, \infty}.
  \]
  But strictly monotone increasing implies
  \[
    \lim_{x \to \dfrac{\pi}{2}} \tan(x) = +\infty.
  \]
  Using similar arguments we can show that
  \[
    \lim_{x \to -\dfrac{\pi}{2}} \tan(x) = -\infty.
  \]
  This means \(\tan\) is a bijection from \((-\dfrac{\pi}{2}, \dfrac{\pi}{2})\) to \(\R\).
  Thus \(\tan^{-1} : \R \to (-\dfrac{\pi}{2}, \dfrac{\pi}{2})\) is well-defined.

  Since \(\tan'(x) > 0\) for all \(x \in (-\dfrac{\pi}{2}, \dfrac{\pi}{2})\), by inverse function theorem (Theorem 10.4.2 in Analysis I) we have
  \begin{align*}
             & \forall x \in (-\dfrac{\pi}{2}, \dfrac{\pi}{2}), \tan(x) = y                                     \\
    \implies & (\tan^{-1})'(y) = \dfrac{1}{\tan'(x)} = \dfrac{1}{1 + \big(\tan(x)\big)^2} = \dfrac{1}{1 + y^2}.
  \end{align*}
\end{proof}

\begin{ex}\label{ii:ex:4.7.9}
  Recall the arctangent function \(\tan^{-1}\) from \cref{ii:ex:4.7.8}.
  By modifying the proof of \cref{ii:4.5.6}(e), establish the identity
  \[
    \tan^{-1}(x) = \sum_{n = 0}^\infty \dfrac{(-1)^n x^{2n + 1}}{2n + 1}
  \]
  for all \(x \in (-1, 1)\).
  Using Abel's theorem (\cref{ii:4.3.1}) to extend this identity to the case \(x = 1\), conclude in particular the identity
  \[
    \pi = 4 - \dfrac{4}{3} + \dfrac{4}{5} - \dfrac{4}{7} + \dots = 4 \sum_{n = 0}^\infty \dfrac{(-1)^n}{2n + 1}.
  \]
  (Note that the series converges by the alternating series test, Proposition 7.2.12 in Analysis I.)
  Conclude in particular that \(4 - \dfrac{4}{3} < \pi < 4\).
  (One can of course compute \(\pi = 3.1415926 \dots\) to much higher accuracy, though if one wishes to do so it is advisable to use a different formula than the one above, which converges very slowly.)
\end{ex}

\begin{proof}
  By \cref{ii:ex:4.7.9} we know that \(\tan^{-1}(x)\) is well-defined for all \(x \in \R\), in particular \(\tan(x)\) is well-defined for all \(x \in (-1, 1)\).
  First suppose that \(x = 0\).
  By \cref{ii:ex:4.7.9} we know that \(\tan\) is bijective from \((-\dfrac{\pi}{2}, \dfrac{\pi}{2})\).
  Thus we have
  \begin{align*}
    \tan^{-1}(0) & = \tan^{-1}\bigg(\dfrac{\sin(0)}{\cos(0)}\bigg)          &  & \by{ii:4.7.2}[e] \\
                 & = \tan^{-1}\big(\tan(0)\big)                             &  & \by{ii:ex:4.7.9} \\
                 & = 0                                                                            \\
                 & = \sum_{n = 0}^\infty \dfrac{(-1)^n 0^{2n + 1}}{2n + 1}.
  \end{align*}

  Now suppose that \(x \in (-1, 1)\).
  Observe that
  \[
    x \in (-1, 1) \implies x^2 \in (-1, 1) \implies -x^2 \in (-1, 1).
  \]
  Since
  \begin{align*}
    (\tan^{-1})'(x) & = \dfrac{1}{1 + x^2}                           &                    & \by{ii:ex:4.7.8}          \\
                    & = \dfrac{1}{1 - (-x^2)}                        & (-x^2 \in (-1, 1))                             \\
                    & = \sum_{n = 0}^\infty (-x^2)^n                 &                    & \text{(geometric series)} \\
                    & = \sum_{n = 0}^\infty \big((-1)^n x^{2n}\big),
  \end{align*}
  we know that
  \begin{align*}
     & \tan^{-1}(x)                                                                                                                                  \\
     & = \tan^{-1}(x) - \tan^{-1}(0)                                              & (\tan^{-1}(0) = 0)                                               \\
     & = \int_0^x (\tan^{-1})'(y) \; dy                                           &                    & \text{(by fundamental theorem of calculus)} \\
     & = \int_{0}^x \bigg(\sum_{n = 0}^\infty \big((-1)^n y^{2n}\big)\bigg) \; dy &                    & \by{ii:4.1.6}[c,e]                          \\
     & = \sum_{n = 0}^\infty \bigg(\int_{0}^x \big((-1)^n y^{2n}\big)\bigg) \; dy &                    & \by{ii:3.6.2}                               \\
     & = \sum_{n = 0}^\infty \dfrac{(-1)^n (x^{2n + 1} - 0^{2n + 1})}{2n + 1}                                                                        \\
     & = \sum_{n = 0}^\infty \dfrac{(-1)^n x^{2n + 1}}{2n + 1}.
  \end{align*}

  Since the sequence \((\dfrac{1}{2n + 1})_{n = 0}^\infty\) is monotone decreasing, we know that the following series
  \begin{align*}
    \sum_{n = 0}^\infty \dfrac{(-1)^n}{2n + 1}  & = \sum_{n = 0}^\infty \dfrac{(-1)^n 1^{2n + 1}}{2n + 1};                                                      \\
    -\sum_{n = 0}^\infty \dfrac{(-1)^n}{2n + 1} & = \sum_{n = 0}^\infty \dfrac{(-1)^{n + 1}}{2n + 1} = \sum_{n = 0}^\infty \dfrac{(-1)^n (-1)^{2n + 1}}{2n + 1}
  \end{align*}
  are convergent.
  Thus by Abel's theorem (\cref{ii:4.3.1}) we know that
  \[
    \forall x \in [-1, 1], \tan^{-1}(x) = \sum_{n = 0}^\infty \dfrac{(-1)^n x^{2n + 1}}{2n + 1}.
  \]
  Since
  \begin{align*}
    0 & = \cos(\dfrac{\pi}{2})                                                                  &  & \by{ii:4.7.5}[c] \\
      & = \cos(\dfrac{\pi}{4} + \dfrac{\pi}{4})                                                                       \\
      & = \cos(\dfrac{\pi}{4}) \cos(\dfrac{\pi}{4}) - \sin(\dfrac{\pi}{4}) \sin(\dfrac{\pi}{4}) &  & \by{ii:4.7.2}[d]
  \end{align*}
  we know that
  \begin{align*}
             & \big(\sin(\dfrac{\pi}{4})\big)^2 = \big(\cos(\dfrac{\pi}{4})\big)^2                                                                                                             \\
    \implies & \bigg(\dfrac{\sin(\dfrac{\pi}{4})}{\cos(\dfrac{\pi}{4})}\bigg)^2 = \big(\tan(\dfrac{\pi}{4})\big)^2 = 1 &                                                    & \by{ii:ex:4.7.8} \\
    \implies & \tan(\dfrac{\pi}{4}) = 1.                                                                               & (\tan\big((0, \dfrac{\pi}{2})\big) \subseteq \R^+)
  \end{align*}
  Thus we have
  \begin{align*}
             & \dfrac{\pi}{4} = \tan^{-1}\big(\tan(\dfrac{\pi}{4})\big) = \tan^{-1}(1) = \sum_{n = 0}^\infty \dfrac{(-1)^n}{2n + 1} &  & \text{(from the proof above)} \\
    \implies & \pi = 4 \bigg(\sum_{n = 0}^\infty \dfrac{(-1)^n}{2n + 1}\bigg).
  \end{align*}
  This means
  \begin{align*}
    \pi & = 4 \bigg(\sum_{n = 0}^\infty \dfrac{(-1)^n}{2n + 1}\bigg)                                                                                 \\
        & = 4 + 4 \bigg(\sum_{n = 1}^\infty \dfrac{(-1)^n}{2n + 1}\bigg)                                                                             \\
        & = 4 + 4 \bigg(\sum_{n = 0}^\infty \dfrac{(-1)^{n + 1}}{2(n + 1) + 1}\bigg)                                                                 \\
        & = 4 + 4 \bigg(\sum_{n = 0}^\infty \dfrac{(-1)^{n + 1}}{2n + 3}\bigg)                                                                       \\
        & = 4 + 4 \Bigg(\sum_{n = 0}^\infty \bigg(\dfrac{-1}{2(2n) + 3} + \dfrac{1}{2(2n + 1) + 3}\bigg)\Bigg) &  & \text{(grouping each two terms)} \\
        & = 4 + 4 \Bigg(\sum_{n = 0}^\infty \bigg(\dfrac{-1}{4n + 3} + \dfrac{1}{4n + 5}\bigg)\Bigg)                                                 \\
        & = 4 + 4 \Bigg(\sum_{n = 0}^\infty \bigg(\dfrac{-2}{(4n + 3)(4n + 5)}\bigg)\Bigg)                                                           \\
        & = 4 - 4 \Bigg(\sum_{n = 0}^\infty \bigg(\dfrac{2}{(4n + 3)(4n + 5)}\bigg)\Bigg)                                                            \\
        & < 4
  \end{align*}
  and
  \begin{align*}
    \pi & = 4 \bigg(\sum_{n = 0}^\infty \dfrac{(-1)^n}{2n + 1}\bigg)                                                                            \\
        & = 4 \Bigg(\sum_{n = 0}^\infty \bigg(\dfrac{1}{2(2n) + 1} - \dfrac{1}{2(2n + 1) + 1}\bigg)\Bigg) &  & \text{(grouping each two terms)} \\
        & = 4 \Bigg(\sum_{n = 0}^\infty \bigg(\dfrac{1}{4n + 1} - \dfrac{1}{4n + 3}\bigg)\Bigg)                                                 \\
        & = 4 \bigg(\sum_{n = 0}^\infty \dfrac{2}{(4n + 1)(4n + 3)}\bigg)                                                                       \\
        & = \dfrac{8}{3} + 4 \bigg(\sum_{n = 1}^\infty \dfrac{2}{(4n + 1)(4n + 3)}\bigg)                                                        \\
        & = 4 - \dfrac{4}{3} + 4 \bigg(\sum_{n = 1}^\infty \dfrac{2}{(4n + 1)(4n + 3)}\bigg)                                                    \\
        & > 4 - \dfrac{4}{3}.
  \end{align*}
  Thus we conclude that \(\pi \in (\dfrac{4}{3}, 4)\).
\end{proof}

\begin{ex}\label{ii:ex:4.7.10}
  Let \(f : \R \to \R\) be the function
  \[
    f(x) \coloneqq \sum_{n = 1}^\infty \big(4^{-n} \cos(32^n \pi x)\big).
  \]
  \begin{enumerate}
    \item Show that this series is uniformly convergent, and that \(f\) is continuous.
    \item Show that for every integer \(j\) and every integer \(m \geq 1\), we have
          \[
            \abs{f\bigg(\dfrac{j + 1}{32^m}\bigg) - f\bigg(\dfrac{j}{32^m}\bigg)} \geq 4^{-m}.
          \]
    \item Using (b), show that for every real number \(x_0\), the function \(f\) is not differentiable at \(x_0\).
    \item Explain briefly why the result in (c) does not contradict \cref{ii:3.7.3}.
  \end{enumerate}
\end{ex}

\begin{proof}{(a)}
  Since
  \begin{align*}
     & \sum_{n = 1}^\infty \norm*{4^{-n} \cos(32^n \pi x)}_{\infty}                                               \\
     & = \sum_{n = 1}^\infty \sup\set{\abs{4^{-n} \cos(32^n \pi x)} : x \in \R} &  & \by{ii:3.5.5}                \\
     & = \sum_{n = 1}^\infty 4^{-n}                                                                               \\
     & = \dfrac{1}{1 - \dfrac{1}{4}}                                            &  & \text{(by geometric series)}
  \end{align*}
  By Weierstrass \(M\)-test (\cref{ii:3.5.7}) we know that \(\sum_{n = 1}^\infty \big(4^{-n} \cos(32^n \pi x)\big)\) converges uniformly to \(f\) on \(\R\) with respect to \(d_{l^1}|_{\R \times \R}\).
\end{proof}

\begin{proof}{(b)}
  First we show that
  \[
    \forall x, y \in \R, \abs{\cos(x) - \cos(y)} \leq \abs{x - y}.
  \]
  Fix one pair of \(x, y\) and without the loss of generality suppose that \(y \leq x\).
  By \cref{ii:4.7.2}(a)(b) we know that \(\sin\) is continuous and bounded on \([y, x]\), thus by Corollary 11.5.2 in Analysis I we know that \(\sin\) is Riemann integrable on \([y, x]\).
  Then we have
  \begin{align*}
             & \int_y^x -1 \; dz \leq \int_y^x \sin(z) \; dz \leq \int_y^x 1 \; dz &  & \by{ii:4.7.2}[a] \\
    \implies & -(x - y) \leq -\big(\cos(x) - \cos(y)\big) \leq x - y               &  & \by{ii:4.7.2}[b] \\
    \implies & \abs{\cos(x) - \cos(y)} \leq \abs{x - y}.
  \end{align*}
  The case \(x \leq y\) can be proven similarly.

  Define
  \[
    \forall (j, m, n) \in \Z \times \Z^+ \times \Z^+, a_n^{(j, m)} = 4^{-n} \Bigg(\cos\bigg(\dfrac{32^n \pi (j + 1)}{32^m}\bigg) - \cos\bigg(\dfrac{32^n \pi j}{32^m}\bigg)\Bigg).
  \]
  Then we have
  \begin{align*}
     & \abs{f\bigg(\dfrac{j + 1}{32^m}\bigg) - f\bigg(\dfrac{j}{32^m}\bigg)}                                                                                                           \\
     & = \abs{\sum_{n = 1}^\infty \Bigg(4^{-n} \cos\bigg(\dfrac{32^n \pi (j + 1)}{32^m}\bigg)\Bigg) - \sum_{n = 1}^\infty \Bigg(4^{-n} \cos\bigg(\dfrac{32^n \pi j}{32^m}\bigg)\Bigg)} \\
     & = \abs{\sum_{n = 1}^\infty \Bigg(4^{-n} \cos\bigg(\dfrac{32^n \pi (j + 1)}{32^m}\bigg) - 4^{-n} \cos\bigg(\dfrac{32^n \pi j}{32^m}\bigg)\Bigg)}                                 \\
     & = \abs{\sum_{n = 1}^\infty a_n^{(j, m)}}.
  \end{align*}
  Now we split into three cases:
  \begin{itemize}
    \item If \(n < m\), then we have
          \begin{align*}
            \abs{a_n^{(j, m)}} & = \abs{4^{-n} \Bigg(\cos\bigg(\dfrac{32^n \pi (j + 1)}{32^m}\bigg) - \cos\bigg(\dfrac{32^n \pi j}{32^m}\bigg)\Bigg)}                                           \\
                               & \leq \abs{4^{-n} \bigg(\dfrac{32^n \pi (j + 1)}{32^m} - \dfrac{32^n \pi j}{32^m}\bigg)}                              &         & \text{(from the proof above)} \\
                               & = 4^{-n} \dfrac{32^n \pi}{32^m} = \dfrac{8^n \pi}{32^m} = \dfrac{8^n \pi}{8^m 4^m} = \dfrac{\pi}{8^{m - n} 4^m}                                                \\
                               & \leq \dfrac{4}{8^{m - n} 4^m}                                                                                        &         & \by{ii:ex:4.7.9}              \\
                               & \leq \dfrac{4^{m - n}}{8^{m - n} 4^m}                                                                                & (m > n)                                 \\
                               & = \dfrac{4^{m - n}}{4^{m - n} 2^{m - n} 4^m} = 2^{n - m} 4^{-m}.
          \end{align*}
    \item If \(n = m\), then we have
          \begin{align*}
             & \abs{a_m^{(j, m)}}                                                                        \\
             & = \abs{4^{-m} \Big(\cos\big(\pi (j + 1)\big) - \cos(\pi j)\Big)}                          \\
             & = \abs{4^{-m} \big(-2 \cos(\pi j)\big)}                          &  & \by{ii:4.7.5}[a]    \\
             & = 2 \cdot 4^{-m}.                                                &  & \by{ii:ac:4.7.2}[f]
          \end{align*}
    \item If \(n > m\), then we have
          \begin{align*}
             & a_n^{(j, m)}                                                                                              \\
             & = 4^{-n} \Big(\cos\big(32^{n - m} \pi (j + 1)\big) - \cos(32^{n - m} \pi j)\Big)                          \\
             & = 4^{-n} \Big(\cos(32^{n - m} \pi j + 32^{n - m} \pi) - \cos(32^{n - m} \pi j)\Big)                       \\
             & = 4^{-n} \Big(\cos(32^{n - m} \pi j) - \cos(32^{n - m} \pi j)\Big)                  &  & \by{ii:4.7.5}[a] \\
             & = 0.
          \end{align*}
  \end{itemize}
  From all cases above we conclude that
  \begin{align*}
     & \abs{f\bigg(\dfrac{j + 1}{32^m}\bigg) - f\bigg(\dfrac{j}{32^m}\bigg)}                                                                                                  \\
     & = \abs{\sum_{n = 1}^\infty a_n^{(j, m)}}                                                                                                                               \\
     & = \abs{\sum_{n = 1}^{m - 1} a_n^{(j, m)} + a_m^{(j, m)} + \sum_{n = m + 1}^\infty a_n^{(j, m)}}                                                                        \\
     & = \abs{\sum_{n = 1}^{m - 1} a_n^{(j, m)} + a_m^{(j, m)}}                                        &                                      & \text{(from the proof above)} \\
     & \geq \abs{a_m^{(j, m)}} - \abs{\sum_{n = 1}^{m - 1} a_n^{(j, m)}}                               & (\abs{x + y} \geq \abs{x} - \abs{y})                                 \\
     & \geq \abs{a_m^{(j, m)}} - \sum_{n = 1}^{m - 1} \abs{a_n^{(j, m)}}                                                                                                      \\
     & \geq 2 \cdot 4^{-m} - \sum_{n = 1}^{m - 1} (2^{n - m} 4^{-m})                                   &                                      & \text{(from the proof above)} \\
     & = 2 \cdot 4^{-m} - 4^{-m} \bigg(\sum_{n = 1}^{m - 1} 2^{n - m}\bigg)                                                                                                   \\
     & \geq 2 \cdot 4^{-m} - 4^{-m} \bigg(\sum_{n = 1}^\infty 2^{-n}\bigg)                                                                                                    \\
     & = 2 \cdot 4^{-m} - 4^{-m} \bigg(-1 + \sum_{n = 0}^\infty 2^{-n}\bigg)                                                                                                  \\
     & = 2 \cdot 4^{-m} - 4^{-m} \bigg(-1 + \dfrac{1}{1 - \dfrac{1}{2}}\bigg)                          &                                      & \text{(geometric series)}     \\
     & = 2 \cdot 4^{-m} - 4^{-m} = 4^{-m}.
  \end{align*}
\end{proof}

\begin{proof}{(c)}
  Suppose for sake of contradiction that there exists a \(x_0 \in \R\) such that \(f'(x_0) \in \R\).
  Then by Proposition 10.1.7 in Analysis I we have
  \begin{align*}
     & \forall \varepsilon \in \R^+, \exists \delta \in \R^+ :                                                                          \\
     & \forall x \in \R, \abs{x - x_0} < \delta \implies \abs{\big(f(x) - f(x_0)\big) - f'(x_0) (x - x_0)} < \varepsilon \abs{x - x_0}.
  \end{align*}
  In particular, we have
  \begin{align*}
             & \exists \delta \in \R^+ : \forall x \in \R, \abs{x - x_0} < \delta                                        \\
    \implies & \abs{f(x) - f(x_0) - f'(x_0) (x - x_0)} < \abs{x - x_0}                                                   \\
    \implies & \abs{f(x) - f(x_0)} - \abs{f'(x_0) (x - x_0)} < \abs{x - x_0}      & (\abs{a} - \abs{b} \leq \abs{a - b}) \\
    \implies & \abs{f(x) - f(x_0)} < \abs{f'(x_0) (x - x_0)} + \abs{x - x_0}                                             \\
    \implies & \abs{f(x) - f(x_0)} < \big(\abs{f'(x_0)} + 1\big) \abs{x - x_0}.
  \end{align*}
  Fix such \(\delta\).
  Since \(\lim_{m \to \infty} 32^m = +\infty\), we know that
  \[
    \exists m_1 \in \Z^+ : 32^{m_1} > \dfrac{1}{\delta} \implies \exists m_1 \in \Z^+ : \dfrac{1}{32^{m_1}} < \delta.
  \]
  Similarly, since \(\lim_{m \to \infty} 8^m = +\infty\), we know that
  \[
    \exists m_2 \in \Z^+ : 8^{m_2} > \abs{f'(x_0)} + 1.
  \]
  Let \(m = \max(m_1, m_2)\).
  By Exercise 5.4.3 in Analysis I we know that
  \begin{align*}
             & 32^m x_0 \in \R                                                    \\
    \implies & \exists j \in \Z : j \leq 32^m x_0 < j + 1                         \\
    \implies & \exists j \in \Z : \dfrac{j}{32^m} \leq x_0 < \dfrac{j + 1}{32^m}.
  \end{align*}
  Fix such \(j\).
  Then we have
  \begin{align*}
             & \dfrac{j + 1}{32^m} - \dfrac{j}{32^m} = \dfrac{1}{32^m} < \delta                                                                                \\
    \implies & \begin{dcases}
                 \dfrac{j + 1}{32^m} - x_0 < \delta \\
                 x_0 - \dfrac{j}{32^m} < \delta
               \end{dcases}                                                                                                              \\
    \implies & \begin{dcases}
                 \abs{f\bigg(\dfrac{j + 1}{32^m}\bigg) - f(x_0)} < \big(\abs{f'(x_0)} + 1\big) \bigg(\dfrac{j + 1}{32^m} - x_0\bigg) \\
                 \abs{f\bigg(\dfrac{j}{32^m}\bigg) - f(x_0)} < \big(\abs{f'(x_0)} + 1\big) \bigg(x_0 - \dfrac{j}{32^m}\bigg)
               \end{dcases} &  & \text{(from the proof above)}                           \\
    \implies & 4^{-m} \leq \abs{f\bigg(\dfrac{j + 1}{32^m}\bigg) - f\bigg(\dfrac{j}{32^m}\bigg)}                                        &  & \by{ii:ex:4.7.10} \\
             & \leq \abs{f\bigg(\dfrac{j + 1}{32^m}\bigg) - f(x_0)} + \abs{f\bigg(\dfrac{j}{32^m}\bigg) - f(x_0)}                                              \\
             & < \dfrac{\abs{f'(x_0)} + 1}{32^m}                                                                                                               \\
    \implies & 8^m < \abs{f'(x_0)} + 1.
  \end{align*}
  But this contradict to the fact that \(8^m > \abs{f'(x_0)} + 1\).
  Thus such \(x_0\) does not exist and \(f\) is not differentiable on \(\R\).
\end{proof}

\begin{proof}{(d)}
  Let \(f_n(x) = 4^{-n} \cos(32^n \pi x)\) for all \(n \in \Z^+\).
  By \cref{ii:4.7.2}(b) we know that \(f_n\) is differentiable on \(\R\) and by chain rule we have
  \[
    \forall n \in \Z^+, f_n'(x) = 4^{-n} \big(-\sin(32^n \pi x)\big) (32^n \pi) = - 8^n \pi \sin(32^n \pi x).
  \]
  Since
  \begin{align*}
    \sum_{n = 1}^\infty \norm*{f_n'}_\infty & = \sum_{n = 1}^\infty \sup\set{\abs{f_n'(x)} : x \in \R} &  & \by{ii:3.5.5}          \\
                                            & = \sum_{n = 1}^\infty 8^n \pi                                                        \\
                                            & = +\infty,                                               &  & \text{(by ratio test)}
  \end{align*}
  the condition in \cref{ii:3.7.3} is not satisfied.
  Thus this does not contradict to \cref{ii:3.7.3}.
\end{proof}


\chapter{Fourier series}\label{ii:ch:5}

\begin{note}
  Power series are already immensely useful, especially when dealing with special functions such as the exponential and trigonometric functions discussed earlier.
  However, there are some circumstances where power series are not so useful, because one has to deal with functions (e.g., \(\sqrt{x}\)) which are not real analytic, and so do not have power series.
\end{note}

\begin{note}
  Fortunately, there is another type of series expansion, known as \emph{Fourier series}, which is also a very powerful tool in analysis
  (though used for slightly different purposes).
  Instead of analyzing compactly supported functions, it instead analyzes \emph{periodic functions};
  instead of decomposing into polynomials, it decomposes into \emph{trigonometric polynomials}.
  Roughly speaking, the theory of Fourier series asserts that just about every periodic function can be decomposed as an (infinite) sum of sines and cosines.
\end{note}

\begin{rmk}\label{ii:5.0.1}
  Jean-Baptiste Fourier (1768--1830) was, among other things, an administrator accompanying Napoleon on his invasion of Egypt, and then a Prefect in France during Napoleons reign.
  After the Napoleonic wars, he returned to mathematics.
  He introduced Fourier series in an important 1807 paper in which he used them to solve what is now known as the heat equation.
  At the time, the claim that every periodic function could be expressed as a sum of sines and cosines was extremely controversial, even such leading mathematicians as Euler declared that it was impossible.
  Nevertheless, Fourier managed to show that this was indeed the case, although the proof was not completely rigorous and was not totally accepted for almost another hundred years.
\end{rmk}

\begin{note}
  For instance, the convergence of Fourier series is usually not uniform (i.e., not in the \(L^\infty\) metric), but instead we have convergence in a different metric, the \(L^2\)-metric.
  We will need to use complex numbers heavily in our theory, while they played only a tangential rôle in power series.
\end{note}

\begin{note}
  The theory of Fourier series (and of related topics such as Fourier integrals and the Laplace transform) is vast, and deserves an entire course in itself.
  It has many, many applications, most directly to differential equations, signal processing, electrical engineering, physics, and analysis, but also to algebra and number theory.
\end{note}

\section{Periodic functions}\label{ii:sec:5.1}

\begin{defn}\label{ii:5.1.1}
  Let \(L > 0\) be a real number.
  A function \(f : \R \to \C\) is periodic with period \(L\), or \(L\)-periodic, if we have \(f(x + L) = f(x)\) for every real number \(x\).
\end{defn}

\begin{eg}\label{ii:5.1.2}
  The real-valued functions \(f(x) = \sin(x)\) and \(f(x) = \cos(x)\) are \(2\pi\)-periodic, as is the complex-valued function \(f(x) = e^{i x}\).
  These functions are also \(4\pi\)-periodic, \(6\pi\)-periodic, etc.
  The function \(f(x) = x\), however, is not periodic.
  The constant function \(f(x) = 1\) is \(L\)-periodic for every \(L\).
\end{eg}

\begin{rmk}\label{ii:5.1.3}
  If a function \(f\) is \(L\)-periodic, then we have \(f(x + kL) = f(x)\) for every integer \(k\)
  (why? Use induction for the positive \(k\), and then use a substitution to convert the positive \(k\) result to a negative \(k\) result.
  The \(k = 0\) case is of course trivial).
  In particular, if a function \(f\) is \(1\)-periodic, then we have \(f(x + k) = f(x)\) for every \(k \in \Z\).
  Because of this, \(1\)-periodic functions are sometimes also called \(\Z\)-periodic
  (and \(L\)-periodic functions called \(L \Z\)-periodic).
\end{rmk}

\begin{eg}\label{ii:5.1.4}
  For any integer \(n\), the functions \(x \mapsto \cos(2 \pi n x)\), \(x \mapsto \sin(2 \pi n x)\), and \(x \mapsto e^{2 \pi i n x}\) are all \(\Z\)-periodic.
  Another example of a \(\Z\)-periodic function is the function \(f : \R \to \C\) defined by \(f(x) \coloneqq 1\) when \(x \in [n, n + \dfrac{1}{2})\) for some integer \(n\), and \(f(x) \coloneqq 0\) when \(x \in [n + \dfrac{1}{2}, n + 1)\) for some integer \(n\).
  This function is an example of a \emph{square wave}.
\end{eg}

\begin{note}
  In order to completely specify a \(\Z\)-periodic function \(f : \R \to \C\), one only needs to specify its values on the interval \([0, 1)\), since this will determine the values of \(f\) everywhere else.
  This is because every real number \(x\) can be written in the form \(x = k + y\) where \(k\) is an integer (called the \emph{integer part} of \(x\), and sometimes denoted \([x]\)) and \(y \in [0, 1)\) (this is called the \emph{fractional part} of \(x\), and sometimes denoted \(\set{x}\)).
  Because of this, sometimes when we wish to describe a \(\Z\)-periodic function \(f\) we just describe what it does on the interval \([0, 1)\), and then say that it is \emph{extended periodically} to all of \(\R\).
  This means that we define \(f(x)\) for any real number \(x\) by setting \(f(x) \coloneqq f(y)\), where we have decomposed \(x = k + y\) as discussed above.
  (One can in fact replace the interval \([0, 1)\) by any other half-open interval of length \(1\), but we will not do so here.)
\end{note}

\begin{note}
  The space of complex-valued continuous \(\Z\)-periodic functions is denoted
  \[
    C(\R / \Z ; \C).
  \]
  (The notation \(\R / \Z\) comes from algebra, and denotes the quotient group of the additive group \(\R\) by the additive group \(\Z\);
  more information in this can be found in any algebra text.)
  By ``continuous'' we mean continuous at all points on \(\R\);
  merely being continuous on an interval such as \([0, 1]\) will not suffice, as there may be a discontinuity between the left and right limits at \(1\) (or at any other integer).
  Thus, for instance, the functions \(x \mapsto \sin(2 \pi n x)\), \(x \mapsto \cos(2 \pi n x)\), and \(x \mapsto e^{2 \pi i n x}\) are all elements of \(C(\R / \Z ; \C)\), as are the constant functions, however the square wave function in \cref{ii:5.1.4} is not in \(C(\R / \Z ; \C)\) because it is not continuous at every integer.
  Also the function \(\sin(x)\) would also not qualify to be in \(C(\R / \Z ; \C)\) since it is not \(\Z\)-periodic.
\end{note}

\begin{lem}[Basic properties of \(C(\R / \Z ; \C)\)]\label{ii:5.1.5}
  \quad
  \begin{enumerate}
    \item (Boundedness)
          If \(f \in C(\R / \Z ; \C)\), then \(f\) is bounded
          (i.e., there exists a real number \(M > 0\) such that \(\abs{f(x)} \leq M\) for all \(x \in \R\)).
    \item (Vector space and algebra properties)
          If \(f, g \in C(\R / \Z ; \C)\), then the functions \(f + g\), \(f - g\), and \(f g\) are also in \(C(\R / \Z ; \C)\).
          Also, if \(c\) is any complex number, then the function \(cf\) is also in \(C(\R / \Z ; \C)\).
    \item (Closure under uniform limits)
          If \((f_n)_{n = 1}^\infty\) is a sequence of functions in \(C(\R / \Z ; \C)\) which converges uniformly to another function \(f : \R \to \C\), then \(f\) is also in \(C(\R / \Z ; \C)\).
  \end{enumerate}
\end{lem}

\begin{proof}{(a)}
  Since \(f \in C(\R / \Z ; \C)\), by \cref{ii:5.1.1} we have
  \[
    \set{f(x) : x \in \R} = \set{f(x) : x \in [0, 1)} = \set{f(x) : x \in [0, 1]}.
  \]
  So it suffices to show that \(\set{f(x) : x \in [0, 1]}\) is bounded.
  Let \(d_{\R} = d_{l^1}|_{\R \times \R}\) and let \(d_{\C}\) be the metric in \cref{ii:4.6.10}.
  Since \([0, 1]\) is closed and bounded in \((\R, d_{\R})\), by Heine-Borel theorem (\cref{ii:1.5.7}) we know that \(([0, 1], d_{\R}|_{[0, 1] \times [0, 1]})\) is compact.
  Since \(f\) is continuous on \([0, 1]\), by \cref{ii:2.3.1} we know that \(\big(f([0, 1]), d_{\C}|_{f([0, 1]) \times f([0, 1])}\big)\) is also compact.
  By \cref{ii:1.5.6} we know that compactness implies boundness, thus we have
  \begin{align*}
             & \forall z \in \C, \exists r \in \R^+ : f([0, 1]) \subseteq B_{(\C, d_{\C})}(z, r) &  & \by{ii:1.5.3}  \\
    \implies & \exists r \in \R^+ : f([0, 1]) \subseteq B_{(\C, d_{\C})}(1, r)                                       \\
    \implies & \exists r \in \R^+ : \forall y \in f([0, 1]), \abs{y - 1} < r                     &  & \by{ii:1.2.1}  \\
    \implies & \exists r \in \R^+ : \forall y \in f([0, 1]),                                                         \\
             & \abs{y} = \abs{y - 1 + 1} \leq \abs{y - 1} + 1 < r + 1.                           &  & \by{ii:4.6.11}
  \end{align*}
  By setting \(M = r + 1\) we are done.
\end{proof}

\begin{proof}{(b)}
  We have
  \begin{align*}
    (f + g)(x + 1) & = f(x + 1) + g(x + 1)                    \\
                   & = f(x) + g(x)         &  & \by{ii:5.1.1} \\
                   & = (f + g)(x)                             \\
    (f - g)(x + 1) & = f(x + 1) - g(x + 1)                    \\
                   & = f(x) - g(x)         &  & \by{ii:5.1.1} \\
                   & = (f - g)(x)                             \\
    (f g)(x + 1)   & = f(x + 1) g(x + 1)                      \\
                   & = f(x) g(x)           &  & \by{ii:5.1.1} \\
                   & = (f g)(x)
  \end{align*}
  and
  \begin{align*}
    \forall c \in \C, (c f)(x + 1) & = c f(x + 1)                    \\
                                   & = c f(x)     &  & \by{ii:5.1.1} \\
                                   & = (c f)(x).
  \end{align*}
\end{proof}

\begin{proof}{(c)}
  Let \(d_{\R} = d_{l^1}|_{\R \times \R}\) and let \(d_{\C}\) be the metric in \cref{ii:4.6.10}.
  Since \((f_n)_{n = 0}^\infty\) converges uniformly to \(f\) on \(\C\) with respect to \(d_{\C}\), by \cref{ii:3.3.2} we know that \(f\) is continuous from \((\R, d_{\R})\) to \((\C, d_{\C})\).
  Suppose for the sake of contradiction that \(f \notin C(\R / \Z ; \C)\).
  By \cref{ii:5.1.1} this means
  \[
    \exists x \in \R : f(x + 1) \neq f(x).
  \]
  By \cref{ii:ex:3.2.2} we know that \((f_n)_{n = 0}^\infty\) converges pointwise to \(f\) on \(\C\) with respect to \(d_{\C}\), thus by \cref{ii:3.2.1} we have
  \begin{align*}
             & \begin{dcases}
                 d - \lim_{n \to \infty} f_n(x) = f(x) \\
                 d - \lim_{n \to \infty} f_n(x + 1) = f(x + 1)
               \end{dcases}                                                                       \\
    \implies & \forall \varepsilon \in \R, \exists N \in \Z^+ : \forall n \geq N,                                                 \\
             & \begin{dcases}
                 \abs{f_n(x) - f(x)} < \dfrac{\varepsilon}{2} \\
                 \abs{f_n(x + 1) - f(x + 1)} < \dfrac{\varepsilon}{2}
               \end{dcases}                                                                \\
    \implies & \forall \varepsilon \in \R, \exists N \in \Z^+ : \forall n \geq N, \begin{dcases}
                                                                                    \abs{f_n(x) - f(x)} < \dfrac{\varepsilon}{2} \\
                                                                                    \abs{f_n(x) - f(x + 1)} < \dfrac{\varepsilon}{2}
                                                                                  \end{dcases} &  & \by{ii:5.1.1} \\
    \implies & \forall \varepsilon \in \R, \exists N \in \Z^+ : \forall n \geq N,                                                 \\
             & \abs{f(x) - f(x + 1)} \leq \abs{f(x) - f_n(x)} + \abs{f_n(x) - f(x + 1)}                                           \\
             & < \dfrac{\varepsilon}{2} + \dfrac{\varepsilon}{2} = \varepsilon                                                    \\
    \implies & \forall \varepsilon \in \R^+, \abs{f(x) - f(x + 1)} < \varepsilon                                                  \\
    \implies & f(x) = f(x + 1).
  \end{align*}
  But this contradict to the definition of \(x\).
  Thus, \(f \in C(\R / \Z ; \C)\).
\end{proof}

\begin{note}
  One can make \(C(\R / \Z ; \C)\) into a metric space by re-introducing the now familiar sup-norm metric
  \[
    d_\infty(f, g) = \sup_{x \in \R} \abs{f(x) - g(x)} = \sup_{x \in [0, 1)} \abs{f(x) - g(x)}
  \]
  of uniform convergence.
\end{note}

\begin{ac}[modular operation]\label{ii:ac:5.1.1}
  Let \(n \in \Z^+\).
  Define \(\text{mod}_n : \R \to \R\) as follow:
  \[
    \forall x \in \R, \text{mod}_n(x) = x - \bigg[\dfrac{x}{n}\bigg] n.
  \]
  Then \(\text{mod}(\R) \subseteq [0, n)\) and \(\text{mod}\) is \(n\)-periodic.
  We often use \(x \mod n\) instead of \(\text{mod}_n(x)\).
\end{ac}

\begin{proof}
  Since
  \begin{align*}
             & \forall x \in \R, \bigg[\dfrac{x}{n}\bigg] \leq \dfrac{x}{n} < \bigg[\dfrac{x}{n}\bigg] + 1 &              & \by{ii:5.1.1} \\
    \implies & \bigg[\dfrac{x}{n}\bigg] n \leq x < \bigg[\dfrac{x}{n}\bigg] n + n                          & (n \in \Z^+)                 \\
    \implies & 0 \leq x - \bigg[\dfrac{x}{n}\bigg] n < n,
  \end{align*}
  we know that \(\text{mod}_n(x) \subseteq [0, n)\).
  Since
  \begin{align*}
    \forall x \in \R, \text{mod}_n(x + n) & = x + n - \bigg[\dfrac{x + n}{n}\bigg] n                                   \\
                                          & = x + n - \Bigg(\bigg[\dfrac{x}{n}\bigg] + 1\Bigg) n &  & \by{ii:ex:5.1.1} \\
                                          & = x + n - \bigg[\dfrac{x}{n}\bigg] n - n                                   \\
                                          & = x - \bigg[\dfrac{x}{n}\bigg] n                                           \\
                                          & = \text{mod}_n(x),
  \end{align*}
  by \cref{ii:5.1.1} we know that \(\text{mod}_n\) is \(n\)-periodic.
\end{proof}

\exercisesection

\begin{ex}\label{ii:ex:5.1.1}
  Show that every real number \(x\) can be written in exactly one way in the form \(x = k + y\), where \(k\) is an integer and \(y \in [0, 1)\).
\end{ex}

\begin{proof}
  By Exercise 5.4.3 we know that
  \[
    \forall x \in \R, \exists!\ k \in \Z : k \leq x < k + 1.
  \]
  Thus, by setting \(y = x - k\) we have \(x = y + k\) and \(y \in [0, 1)\).
\end{proof}

\begin{ex}\label{ii:ex:5.1.2}
  Prove \cref{ii:5.1.5}.
\end{ex}

\begin{proof}
  See \cref{ii:5.1.5}.
\end{proof}

\begin{ex}\label{ii:ex:5.1.3}
  Show that \(C(\R / \Z ; \C)\) with the sup-norm metric \(d_\infty\) is a metric space.
  Furthermore, show that this metric space is complete.
\end{ex}

\begin{proof}
  First we show that \(\big(C(\R / \Z ; \C), d_\infty\big)\) is a metric space.
  Since
  \[
    \forall f \in C(\R / \Z ; \C), d_\infty(f, f) = \sup_{x \in [0, 1)} \abs{f(x) - f(x)} = 0,
  \]
  we know that \(d_\infty\) satisfied \cref{ii:1.1.2}(a).
  Since
  \begin{align*}
             & \forall f, g \in C(\R / \Z ; \C), f \neq g                                \\
    \implies & \exists x \in \R : f(x) \neq g(x)                                         \\
    \implies & \exists x \in [0, 1) : f(x) \neq g(x)               &  & \by{ii:5.1.1}    \\
    \implies & 0 < \sup_{x \in [0, 1)} \abs{f(x) - g(x)} < +\infty &  & \by{ii:5.1.5}[a] \\
    \implies & 0 < d_\infty(f, g) < +\infty,
  \end{align*}
  we know that \(d_\infty\) satisfied \cref{ii:1.1.2}(b).
  Since
  \begin{align*}
    \forall f, g \in C(\R / \Z ; \C), d_\infty(f, g) & = \sup_{x \in [0, 1)} \abs{f(x) - g(x)}                     \\
                                                     & = \sup_{x \in [0, 1)} \abs{g(x) - f(x)} &  & \by{ii:4.6.10} \\
                                                     & = d_\infty(g, f),
  \end{align*}
  we know that \(d_\infty\) satisfied \cref{ii:1.1.2}(c).
  Since
  \begin{align*}
     & \forall f, g, h \in C(\R / \Z ; \C), d_\infty(f, h)                                          \\
     & = \sup_{x \in [0, 1)} \abs{f(x) - h(x)}                                                      \\
     & = \sup_{x \in [0, 1)} \abs{f(x) - g(x) + g(x) - h(x)}                                        \\
     & \leq \sup_{x \in [0, 1)} \big(\abs{f(x) - g(x)} + \abs{g(x) - h(x)}\big) &  & \by{ii:4.6.11} \\
     & = d_\infty(f, g) + d_\infty(g, h),
  \end{align*}
  we know that \(d_\infty\) satisfied \cref{ii:1.1.2}(d).
  From all proofs above we conclude by \cref{ii:1.1.2} that \(\big(C(\R / \Z ; \C), d_\infty\big)\) is a metric space.

  Now we show that \(\big(C(\R / \Z ; \C), d_\infty\big)\) is complete.
  Let \(d_{\R} = d_{l^1}|_{\R \times \R}\) and let \(d_{\C}\) be the metric in \cref{ii:4.6.10}.
  Let \((f_n)_{n = 1}^\infty\) be a Cauchy sequence in \(\big(C(\R / \Z ; \C), d_\infty\big)\) and let \(n_1, n_2 \in \Z^+\).
  Then by \cref{ii:1.4.6} we have
  \begin{align*}
             & \forall \varepsilon \in \R^+, \exists N \in \Z^+ : \forall n_1, n_2 \geq N, d_\infty(f_{n_1}, f_{n_2}) < \varepsilon        \\
    \implies & \forall x \in \R, \forall \varepsilon \in \R^+, \exists N \in \Z^+ : \forall n_1, n_2 \geq N,                               \\
             & \abs{f_{n_1}(x) - f_{n_2}(x)} \leq \sup_{y \in \R} \abs{f_{n_1}(y) - f_{n_2}(y)} = d_\infty(f_{n_1}, f_{n_2}) < \varepsilon \\
    \implies & \forall x \in \R, \big(f_n(x)\big)_{n = 1}^\infty \text{ is a Cauchy sequence in } (\C, d_{\C}).
  \end{align*}
  Since \((\C, d_{\C})\) is complete (by \cref{ii:ex:4.6.10}), we know that
  \[
    \forall x \in \C, \lim_{n \to \infty} f_n(x) \in \C.
  \]
  Thus, we can define \(f : \R \to \C\) as follow:
  \[
    \forall x \in \R, f(x) = \lim_{n \to \infty} f_n(x).
  \]
  And we have
  \begin{align*}
             & \forall x \in \R, \forall \varepsilon \in \R^+, \exists N \in \Z^+ : \forall n \geq N, \\
             & \begin{dcases}
                 \abs{f(x) - f_n(x)} < \dfrac{\varepsilon}{2} \\
                 \abs{f(x + 1) - f_n(x + 1)} = \abs{f(x + 1) - f_n(x)} < \dfrac{\varepsilon}{2}
               \end{dcases}         & (f_n \in C(\R / \Z ; \C))          \\
    \implies & \forall x \in \R, \forall \varepsilon \in \R^+, \exists N \in \Z^+ : \forall n \geq N, \\
             & \abs{f(x) - f(x + 1)} \leq \abs{f(x) - f_n(x)} + \abs{f(x + 1) - f_n(x)}               \\
             & < \dfrac{\varepsilon}{2} + \dfrac{\varepsilon}{2} = \varepsilon                        \\
    \implies & \forall x \in \R, \forall \varepsilon \in \R^+, \abs{f(x) - f(x + 1)} < \varepsilon    \\
    \implies & \forall x \in \R, f(x) = f(x + 1)                                                      \\
    \implies & f \in C(\R / \Z ; \C).
  \end{align*}
  Since \((f_n)_{n = 1}^\infty\) was arbitrary, by \cref{ii:1.4.10} we know that \(\big(C(\R / \Z ; \C), d_\infty\big)\) is complete.
\end{proof}

\section{Inner products on periodic functions}\label{sec:5.2}

\begin{defn}[Inner product]\label{5.2.1}
  If \(f, g \in C(\R / \Z ; \C)\), we define the \emph{inner product} \(\inner*{f, g}\) to be the quantity
  \[
    \inner*{f, g} = \int_{[0, 1]} f(x) \overline{g(x)} \; dx.
  \]
\end{defn}

\begin{rmk}\label{5.2.2}
  In order to integrate a complex-valued function over real variables, we use the definition that
  \[
    \int_{[a, b]} f(x) \; dx \coloneqq \int_{[a, b]} \Re\big(f(x)\big) \; dx + i \int_{[a,b]} \Im\big(f(x)\big) \; dx;
  \]
  i.e., we integrate the real and imaginary parts of the function separately.
  It is easy to verify that all the standard rules of calculus (integration by parts, fundamental theorem of calculus, substitution, etc.) still hold when the functions are complex-valued instead of real-valued.
\end{rmk}

\begin{proof}
  Let \(f \in C(\R / \Z ; \C)\), let \(d_{\R} = d_{l^1}|_{\R \times \R}\) and let \(d_{\C}\) be the metric in \cref{4.6.10}.
  Let \(x_0 \in \R\) and let \((a_n)_{n = 1}^\infty\) be a sequence in \(\R\) such that \(\lim_{n \to \infty} a_n = x_0\).
  Since \(f\) is continuous on \(\R\) from \((\R, d_{\R})\) to \((\C, d_{\C})\), we know that
  \begin{align*}
             & \lim_{n \to \infty} f(a_n) = f(x_0)                                           &  & \by{2.1.4} \\
    \implies & \begin{dcases}
                 \lim_{n \to \infty} \Re(f\big(a_n)\big) = \Re\big(f(x_0)\big) \\
                 \lim_{n \to \infty} \Im(f\big(a_n)\big) = \Im\big(f(x_0)\big)
               \end{dcases} &  & \by{4.6.13}
  \end{align*}
  Since \((a_n)_{n = 0}^\infty\) is arbitrary, by \cref{2.1.4} we know that \(\Re \circ f\) is continuous at \(x_0\) from \((\R, d_{\R})\) to \((\R, d_{\R})\).
  Since \(x_0\) is arbitrary, by \cref{2.1.5} we know that \(\Re \circ f\) is continuous on \(\R\) from \((\R, d_{\R})\) to \((\R, d_{\R})\).
  Using similar arguments we can show that \(\Im \circ f\) is continuous on \(\R\) from \((\R, d_{\R})\) to \((\R, d_{\R})\).
  Since
  \begin{align*}
     & \forall x \in \R, \Re(f(x + 1)) = \Re(f(x)) \implies \Re \circ f \in C(\R / \Z ; \C); \\
     & \forall x \in \R, \Im(f(x + 1)) = \Im(f(x)) \implies \Im \circ f \in C(\R / \Z ; \C),
  \end{align*}
  by \cref{5.1.5}(a) we know that both \(\Re \circ f\) and \(\Im \circ f\) are bounded in \((\C, d_{\C})\).
  In particular, by \cref{4.6.8} we know that \((\Re \circ f)(\R) \subseteq \R\) and \((\Im \circ f)(\R) \subseteq \R\).
  Thus both \(\Re \circ f\) and \(\Im \circ f\) are bounded in \((\R, d_{\R})\).
  Since \(\Re \circ f\) and \(\Im \circ f\) are continuous and bounded on \([0, 1]\), by Corollary 11.5.2 in Analysis I we know that \(\Re \circ f\) and \(\Im \circ f\) are Riemann integrable on \([0, 1]\).
  Thus
  \[
    \int_{[0, 1]} f(x) \; dx = \int_{[0, 1]} \Re\big(f(x)\big) \; dx + i \bigg(\int_{[0, 1]} \Im\big(f(x)\big) \; dx\bigg) \in \C
  \]
  is well-defined.
  The same argument holds on arbitrary closed interval \([a, b]\) since \(f \in C(\R / \Z ; \C)\).
\end{proof}

\begin{eg}\label{5.2.3}
  Let \(f\) be the constant function \(f(x) \coloneqq 1\), and let \(g(x)\) be the function \(g(x) \coloneqq e^{2 \pi i x}\).
  Then we have
  \begin{align*}
    \inner*{f, g} & = \int_{[0, 1]} 1 \overline{e^{2 \pi i x}} \; dx       \\
                  & = \int_{[0, 1]} e^{- 2 \pi i x} \; dx                  \\
                  & = \dfrac{e^{- 2 \pi i x}}{- 2 \pi i} |_{x = 0}^{x = 1} \\
                  & = \dfrac{e^{- 2 \pi i } - e^0}{- 2 \pi i}              \\
                  & = \dfrac{1 - 1}{- 2 \pi i}                             \\
                  & = 0.
  \end{align*}
\end{eg}

\begin{rmk}\label{5.2.4}
  In general, the inner product \(\inner*{f, g}\) will be a complex number.
  (Note that \(f(x) \overline{g(x)}\) will be Riemann integrable since both functions are bounded and continuous.)
\end{rmk}

\begin{note}
  Roughly speaking, the inner product \(\inner*{f, g}\) is to the space \(C(\R / \Z ; \C)\) what the dot product \(x \cdot y\) is to Euclidean spaces such as \(\R^n\).
  A more in-depth study of inner products on vector spaces can be found in any linear algebra text but is beyond the scope of this text.
\end{note}

\begin{lem}\label{5.2.5}
  Let \(f, g, h \in C(\R / \Z ; \C)\).
  \begin{enumerate}
    \item (Hermitian property)
          We have \(\inner*{g, f} = \overline{\inner*{f, g}}\).
    \item (Positivity)
          We have \(\inner*{f, f} \geq 0\).
          Furthermore, we have \(\inner*{f, f} = 0\) iff \(f = 0\)
          (i.e., \(f(x) = 0\) for all \(x \in \R\)).
    \item (Linearity in the first variable)
          We have \(\inner*{f +g, h}\) = \(\inner*{f, h} + \inner*{g, h}\).
          For any complex number \(c\), we have \(\inner*{cf, g} = c \inner*{f, g}\).
    \item (Antilinearity in the second variable)
          We have \(\inner*{f, g + h} = \inner*{f, g} + \inner*{f, h}\).
          For any complex number \(c\), we have \(\inner*{f, cg} = c \inner*{f, g}\).
  \end{enumerate}
\end{lem}

\begin{proof}{(a)}
  We have
  \begin{align*}
    \overline{\inner*{f, g}} & = \overline{\int_{[0, 1]} f(x) \overline{g(x)} \; dx}                                                                                               &  & \by{5.2.1} \\
                             & = \overline{\int_{[0, 1]} \Re\big(f(x) \overline{g(x)}\big) \; dx + i \bigg(\int_{[0, 1]} \Im\big(f(x) \overline{g(x)}\big) \; dx\bigg)}            &  & \by{5.2.2} \\
                             & = \int_{[0, 1]} \Re\big(f(x) \overline{g(x)}\big) \; dx - i \bigg(\int_{[0, 1]} \Im\big(f(x) \overline{g(x)}\big) \; dx\bigg)                       &  & \by{4.6.8} \\
                             & = \int_{[0, 1]} \Re\big(f(x) \overline{g(x)}\big) \; dx + i \bigg(-\int_{[0, 1]} \Im\big(f(x) \overline{g(x)}\big) \; dx\bigg)                      &  & \by{4.6.6} \\
                             & = \int_{[0, 1]} \Re\big(f(x) \overline{g(x)}\big) \; dx + i \bigg(\int_{[0, 1]} -\Im\big(f(x) \overline{g(x)}\big) \; dx\bigg)                                      \\
                             & = \int_{[0, 1]} \Re\big(\overline{f(x) \overline{g(x)}}\big) \; dx + i \bigg(\int_{[0, 1]} \Im\big(\overline{f(x) \overline{g(x)}}\big) \; dx\bigg) &  & \by{4.6.8} \\
                             & = \int_{[0, 1]} \Re\big(\overline{f(x)} g(x)\big) \; dx + i \bigg(\int_{[0, 1]} \Im\big(\overline{f(x)} g(x)\big) \; dx\bigg)                       &  & \by{4.6.9} \\
                             & = \int_{[0, 1]} \Re\big(g(x) \overline{f(x)}\big) \; dx + i \bigg(\int_{[0, 1]} \Im\big(g(x) \overline{f(x)}\big) \; dx\bigg)                       &  & \by{4.6.6} \\
                             & = \int_{[0, 1]} g(x) \overline{f(x)} \; dx                                                                                                          &  & \by{5.2.2} \\
                             & = \inner*{g, f}.                                                                                                                                    &  & \by{5.2.1}
  \end{align*}
\end{proof}

\begin{proof}{(b)}
  We have
  \begin{align*}
    \inner*{f, f} & = \int_{[0, 1]} f(x) \overline{f(x)} \; dx &  & \by{5.2.1}                                  \\
                  & = \int_{[0, 1]} \abs{f(x)}^2 \; dx         &  & \by{4.6.11}                                 \\
                  & \geq \int_{[0, 1]} 0 \; dx                 &  & \text{(by Theorem 11.4.1(d) in Analysis I)} \\
                  & = 0
  \end{align*}
  and
  \begin{align*}
         & \int_{[0, 1]} \abs{f(x)}^2 \; dx = 0                                                  \\
    \iff & \forall x \in [0, 1], \abs{f(x)}^2 = 0 &  & \text{(by Exercise 11.4.2 in Analysis I)} \\
    \iff & \forall x \in [0, 1], \abs{f(x)} = 0                                                  \\
    \iff & \forall x \in [0, 1], f(x) = 0.        &  & \by{4.6.11}
  \end{align*}
\end{proof}

\begin{proof}{(c)}
  We have
  \begin{align*}
     & \inner*{f + g, h}                                                                                                                                                                                       \\
     & = \int_{[0, 1]} (f + g)(x) \overline{h(x)} \; dx                                                                                    &                                                      & \by{5.2.1} \\
     & = \int_{[0, 1]} \Re\big((f + g)(x) \overline{h(x)}\big) \; dx                                                                       &                                                      & \by{5.2.2} \\
     & \quad + i \bigg(\int_{[0, 1]} \Im\big((f + g)(x) \overline{h(x)}\big) \; dx\bigg)                                                                                                                       \\
     & = \int_{[0, 1]} \Re\big(f(x) \overline{h(x)} + g(x) \overline{h(x)}\big) \; dx                                                                                                                          \\
     & \quad + i \bigg(\int_{[0, 1]} \Im\big(f(x) \overline{h(x)} + g(x) \overline{h(x)}\big) \; dx\bigg)                                                                                                      \\
     & = \int_{[0, 1]} \Re\big(f(x) \overline{h(x)}) + \Re\big(g(x) \overline{h(x)}\big) \; dx                                             &                                                      & \by{4.6.8} \\
     & \quad + i \bigg(\int_{[0, 1]} \Im\big(f(x) \overline{h(x)}\big) + \Im\big(g(x) \overline{h(x)}\big) \; dx\bigg)                                                                                         \\
     & = \int_{[0, 1]} \Re\big(f(x) \overline{h(x)}) \; dx + \int_{[0, 1]} \Re\big(g(x) \overline{h(x)}\big) \; dx                         & (f \overline{h}, g \overline{h} \in C(\R / \Z ; \C))              \\
     & \quad + i \bigg(\int_{[0, 1]} \Im\big(f(x) \overline{h(x)}\big) \; dx + \int_{[0, 1]} \Im\big(g(x) \overline{h(x)}\big) \; dx\bigg)                                                                     \\
     & = \int_{[0, 1]} f(x) \overline{h(x)} \; dx + \int_{[0, 1]} g(x) \overline{h(x)} \; dx                                               &                                                      & \by{5.2.2} \\
     & = \inner*{f, h} + \inner*{g, h}                                                                                                     &                                                      & \by{5.2.1}
  \end{align*}
  and
  \begin{align*}
     & \inner*{cf, g}                                                                                                                                                                                   \\
     & = \int_{[0, 1]} (cf)(x) \overline{g(x)} \; dx                                                                                                &                                      & \by{5.2.1} \\
     & = \int_{[0, 1]} \Re\big((cf)(x) \overline{g(x)}\big) \; dx + i \bigg(\int_{[0, 1]} \Im\big((cf)(x) \overline{g(x)}\big) \; dx\bigg)          &                                      & \by{5.2.2} \\
     & = \int_{[0, 1]} \Re\big(cf(x) \overline{g(x)}\big) \; dx + i \bigg(\int_{[0, 1]} \Im\big(cf(x) \overline{g(x)}\big) \; dx\bigg)                                                                  \\
     & = \int_{[0, 1]} \Re(c) \Re\big(f(x) \overline{g(x)}\big) - \Im(c) \Im\big(f(x) \overline{g(x)}\big) \; dx                                    &                                      & \by{4.6.5} \\
     & \quad + i \bigg(\int_{[0, 1]} \Re(c) \Im\big(f(x) \overline{g(x)}\big) + \Im(c) \Re\big(f(x) \overline{g(x)}\big) \; dx\bigg)                                                                    \\
     & = \Re(c) \bigg(\int_{[0, 1]} \Re\big(f(x) \overline{g(x)}\big) \; dx\bigg)                                                                   & (f \overline{g} \in C(\R / \Z ; \C))              \\
     & \quad - \Im(c) \bigg(\int_{[0, 1]} \Im\big(f(x) \overline{g(x)}\big) \; dx\bigg)                                                                                                                 \\
     & \quad + i \Re(c) \bigg(\int_{[0, 1]} \Im\big(f(x) \overline{g(x)}\big) \; dx\bigg)                                                                                                               \\
     & \quad + i \Im(c) \bigg(\int_{[0, 1]} \Re\big(f(x) \overline{g(x)}\big) \; dx\bigg)                                                                                                               \\
     & = \Re(c) \Bigg(\int_{[0, 1]} \Re\big(f(x) \overline{g(x)}\big) \; dx + i \int_{[0, 1]} \Im\big(f(x) \overline{g(x)}\big) \; dx\Bigg)         &                                      & \by{4.6.6} \\
     & \quad + i \Im(c) \Bigg(\int_{[0, 1]} \Re\big(f(x) \overline{g(x)}\big) \; dx + i \int_{[0, 1]} \Im\big(f(x) \overline{g(x)}\big) \; dx\Bigg) &                                      & \by{4.6.5} \\
     & = \Re(c) \int_{[0, 1]} f(x) \overline{g(x)} \; dx + i \Im(c) \int_{[0, 1]} f(x) \overline{g(x)} \; dx                                        &                                      & \by{5.2.2} \\
     & = \Re(c) \inner*{f, g} + i \Im(c) \inner*{f, g}                                                                                              &                                      & \by{5.2.1} \\
     & = \big(\Re(c) + i \Im(c)\big) \inner*{f, g}                                                                                                  &                                      & \by{4.6.6} \\
     & = c \inner*{f, g}.                                                                                                                           &                                      & \by{4.6.8}
  \end{align*}
\end{proof}

\begin{proof}{(d)}
  We have
  \begin{align*}
     & \inner*{f, g + h}                                                                                                                                                                                       \\
     & = \int_{[0, 1]} f(x) \overline{(g + h)(x)} \; dx                                                                                    &                                                      & \by{5.2.1} \\
     & = \int_{[0, 1]} \Re\big(f(x) \overline{(g + h)(x)}\big) \; dx                                                                                                                                           \\
     & \quad + i \bigg(\int_{[0, 1]} \Im\big(f(x) \overline{(g + h)(x)}\big) \; dx\bigg)                                                   &                                                      & \by{5.2.2} \\
     & = \int_{[0, 1]} \Re\big(f(x) \overline{g(x)} + f(x) \overline{h(x)}\big) \; dx                                                      &                                                      & \by{4.6.9} \\
     & \quad + i \bigg(\int_{[0, 1]} \Im\big(f(x) \overline{g(x)} + f(x) \overline{h(x)}\big) \; dx\bigg)                                                                                                      \\
     & = \int_{[0, 1]} \Re\big(f(x) \overline{g(x)}) + \Re\big(f(x) \overline{h(x)}\big) \; dx                                             &                                                      & \by{4.6.8} \\
     & \quad + i \bigg(\int_{[0, 1]} \Im\big(f(x) \overline{g(x)}\big) + \Im\big(f(x) \overline{h(x)}\big) \; dx\bigg)                                                                                         \\
     & = \int_{[0, 1]} \Re\big(f(x) \overline{g(x)}) \; dx + \int_{[0, 1]} \Re\big(f(x) \overline{h(x)}\big) \; dx                         & (f \overline{g}, f \overline{h} \in C(\R / \Z ; \C))              \\
     & \quad + i \bigg(\int_{[0, 1]} \Im\big(f(x) \overline{g(x)}\big) \; dx + \int_{[0, 1]} \Im\big(f(x) \overline{h(x)}\big) \; dx\bigg)                                                                     \\
     & = \int_{[0, 1]} f(x) \overline{g(x)} \; dx + \int_{[0, 1]} f(x) \overline{h(x)} \; dx                                               &                                                      & \by{5.2.2} \\
     & = \inner*{f, g} + \inner*{f, h}                                                                                                     &                                                      & \by{5.2.1}
  \end{align*}
  and
  \begin{align*}
     & \inner*{f, cg}                                                                                                                                                                               \\
     & = \int_{[0, 1]} (x) \overline{(cg)(x)} \; dx                                                                                                                &  & \by{5.2.1}                  \\
     & = \int_{[0, 1]} \Re\big(f(x) \overline{(cg)(x)}\big) \; dx + i \bigg(\int_{[0, 1]} \Im\big(f(x) \overline{(cg)(x)}\big) \; dx\bigg)                         &  & \by{5.2.2}                  \\
     & = \int_{[0, 1]} \Re\big(\overline{c} f(x) \overline{g(x)}\big) \; dx + i \bigg(\int_{[0, 1]} \Im\big(\overline{c} f(x) \overline{g(x)}\big) \; dx\bigg)     &  & \by{4.6.9}                  \\
     & = \int_{[0, 1]} \Re\big((\overline{c} f)(x) \overline{g(x)}\big) \; dx + i \bigg(\int_{[0, 1]} \Im\big((\overline{c} f)(x) \overline{g(x)}\big) \; dx\bigg)                                  \\
     & = \int_{[0, 1]} (\overline{c} f)(x) \overline{g(x)} \; dx                                                                                                   &  & \by{5.2.2}                  \\
     & = \inner*{\overline{c} f, g}                                                                                                                                &  & \by{5.2.1}                  \\
     & = \overline{c} \inner*{f, g}.                                                                                                                               &  & \text{(by \cref{5.2.5}(c))}
  \end{align*}
\end{proof}

\begin{ac}\label{ac:5.2.1}
  From the positivity property (\cref{5.2.5}(b)), it makes sense to define the \(L^2\) norm \(\norm*{f}_2\) of a function \(f \in C(\R / \Z ; \C)\) by the formula
  \[
    \norm*{f}_2 \coloneqq \sqrt{\inner*{f, f}} = \bigg(\int_{[0, 1]} f(x) \overline{f(x)} \; dx\bigg)^{1 / 2} = \bigg(\int_{[0, 1]} \abs{f(x)}^2 \; dx\bigg)^{1 / 2}.
  \]
  Thus \(\norm*{f}_2 \geq 0\) for all \(f\).
  The norm \(\norm*{f}_2\) is sometimes called the \emph{root mean square} of \(f\).
\end{ac}

\begin{note}
  This \(L^2\) norm is related to, but is distinct from, the \(L^\infty\) norm
  \[
    \norm*{f}_\infty \coloneqq \sup_{x \in \R} \abs{f(x)}.
  \]
  In general, the best one can say is that \(0 \leq \norm*{f}_2 \leq \norm*{f}_\infty\).
\end{note}

\setcounter{thm}{6}
\begin{lem}\label{5.2.7}
  Let \(f, g \in C(\R / \Z ; \C)\).
  \begin{enumerate}
    \item (Non-degeneracy)
          We have \(\norm*{f}_2 = 0\) iff \(f = 0\).
    \item (Cauchy-Schwarz inequality)
          We have \(\abs{\inner*{f, g}} \leq \norm*{f}_2 \norm*{g}_2\).
    \item (Triangle inequality)
          We have \(\norm*{f + g}_2 \leq \norm*{f}_2 + \norm*{g}_2\).
    \item (Pythagoras' theorem)
          If \(\inner*{f, g} = 0\), then \(\norm*{f + g}_2^2 = \norm*{f}_2^2 + \norm*{g}_2^2\).
    \item (Homogeneity)
          We have \(\norm*{cf}_2 = \abs{c} \norm*{f}_2\) for all \(c \in \C\).
  \end{enumerate}
\end{lem}

\begin{proof}{(a)}
  We have
  \begin{align*}
         & \norm*{f}_2 = 0                                           \\
    \iff & \sqrt{\inner*{f, f}} = 0 &  & \by{ac:5.2.1}               \\
    \iff & \inner*{f, f} = 0                                         \\
    \iff & f = 0.                   &  & \text{(by \cref{5.2.5}(b))}
  \end{align*}
\end{proof}

\begin{proof}{(b)}
  If \(g\) is zero function on \([0, 1]\), then we have
  \begin{align*}
    \abs{\inner*{f, g}} & = \abs{\int_{[0, 1]} f(x) \overline{g(x)} \; dx} &  & \by{5.2.1}                  \\
                        & = \abs{\int_{[0, 1]} f(x) \cdot 0 \; dx}                                          \\
                        & = \abs{\int_{[0, 1]} 0 \; dx}                                                     \\
                        & = 0                                                                               \\
                        & = \norm*{f}_2 \norm*{g}_2.                       &  & \text{(by \cref{5.2.7}(a))}
  \end{align*}
  So suppose that \(g\) is not zero function on \([0, 1]\).
  Observe that
  \begin{align*}
             & \norm*{g}_2 \in \R                                                 &  & \by{ac:5.2.1}               \\
    \implies & \norm*{g}_2^2 \in \R                                                                                \\
    \implies & \norm*{g}_2^2 \cdot f \in C(\R / \Z ; \C)                          &  & \text{(by \cref{5.1.5}(b))} \\
    \implies & \norm*{g}_2^2 \cdot f - \inner*{f, g} \cdot g \in C(\R / \Z ; \C). &  & \text{(by \cref{5.1.5}(b))}
  \end{align*}
  If we let \(h = \norm*{g}_2^2 \cdot f - \inner*{f, g} \cdot g\), then by \cref{ac:5.2.1} we know that \(\inner*{h, h}\) is well-defined.
  Thus we have
  \begin{align*}
     & \inner*{h, h}                                                                                                                                             \\
     & = \inner*{\norm*{g}_2^2 \cdot f - \inner*{f, g} \cdot g, \norm*{g}_2^2 \cdot f - \inner*{f, g} \cdot g}                                                   \\
     & = \inner*{\norm*{g}_2^2 \cdot f, \norm*{g}_2^2 \cdot f - \inner*{f, g} \cdot g}                                          &  & \text{(by \cref{5.2.5}(c))} \\
     & \quad + \inner*{-\inner*{f, g} \cdot g, \norm*{g}_2^2 \cdot f - \inner*{f, g} \cdot g}                                                                    \\
     & = \inner*{\norm*{g}_2^2 \cdot f, \norm*{g}_2^2 \cdot f} + \inner*{\norm*{g}_2^2 \cdot f, -\inner*{f, g} \cdot g}         &  & \text{(by \cref{5.2.5}(d))} \\
     & \quad + \inner*{-\inner*{f, g} \cdot g, \norm*{g}_2^2 \cdot f} + \inner*{-\inner*{f, g} \cdot g, -\inner*{f, g} \cdot g}                                  \\
     & = \norm*{g}_2^2 \inner*{f, \norm*{g}_2^2 \cdot f} + \norm*{g}_2^2 \inner*{f, -\inner*{f, g} \cdot g}                     &  & \text{(by \cref{5.2.5}(c))} \\
     & \quad - \inner*{f, g} \inner*{g, \norm*{g}_2^2 \cdot f} - \inner*{f, g} \inner*{g, -\inner*{f, g} \cdot g}                                                \\
     & = \norm*{g}_2^2 \overline{\norm*{g}_2^2} \inner*{f, f} + \norm*{g}_2^2 \overline{-\inner*{f, g}} \inner*{f, g}           &  & \text{(by \cref{5.2.5}(d))} \\
     & \quad - \inner*{f, g} \overline{\norm*{g}_2^2} \inner*{g, f} - \inner*{f, g} \overline{-\inner*{f, g}} \inner*{g, g}                                      \\
     & = \norm*{g}_2^4 \inner*{f, f} - \norm*{g}_2^2 \overline{\inner*{f, g}} \inner*{f, g}                                     &  & \by{4.6.9}                  \\
     & \quad - \inner*{f, g} \norm*{g}_2^2 \inner*{g, f} + \inner*{f, g} \overline{\inner*{f, g}} \inner*{g, g}                                                  \\
     & = \norm*{g}_2^4 \inner*{f, f} - \norm*{g}_2^2 \overline{\inner*{f, g}} \inner*{f, g}                                                                      \\
     & \quad - \inner*{f, g} \norm*{g}_2^2 \overline{\inner*{f, g}} + \inner*{f, g} \overline{\inner*{f, g}} \inner*{g, g}      &  & \text{(by \cref{5.2.5}(a))} \\
     & = \norm*{g}_2^4 \inner*{f, f} - 2 \norm*{g}_2^2 \abs{\inner*{f, g}}^2 + \abs{\inner*{f, g}}^2 \inner*{g, g}              &  & \by{4.6.10}                 \\
     & = \norm*{g}_2^4 \norm*{f}_2^2 - 2 \norm*{g}_2^2 \abs{\inner*{f, g}}^2 + \abs{\inner*{f, g}}^2 \norm*{g}_2^2              &  & \by{ac:5.2.1}               \\
     & = \norm*{g}_2^4 \norm*{f}_2^2 - \norm*{g}_2^2 \abs{\inner*{f, g}}^2
  \end{align*}
  and
  \begin{align*}
             & \inner*{h, h} \geq 0                                                     &  & \text{(by \cref{5.2.5}(b))} \\
    \implies & \norm*{g}_2^4 \norm*{f}_2^2 - \norm*{g}_2^2 \abs{\inner*{f, g}}^2 \geq 0                                  \\
    \implies & \norm*{g}_2^2 \norm*{f}_2^2 - \abs{\inner*{f, g}}^2 \geq 0               &  & \text{(by \cref{5.2.7}(a))} \\
    \implies & \norm*{g}_2 \norm*{f}_2 \geq \abs{\inner*{f, g}}.                        &  & \text{(by \cref{5.2.5}(b))}
  \end{align*}
\end{proof}

\begin{proof}{(c)}
  We have
  \begin{align*}
    \norm*{f + g}_2^2 & = \inner*{f + g, f + g}                                         &  & \by{ac:5.2.1}                  \\
                      & = \inner*{f, f} + \inner*{f, g} + \inner*{g, f} + \inner*{g, g} &  & \text{(by \cref{5.2.5}(c)(d))} \\
                      & = \norm*{f}_2^2 + \inner*{f, g} + \inner*{g, f} + \norm*{g}_2^2 &  & \by{ac:5.2.1}                  \\
                      & \leq \norm*{f}_2^2 + 2 \norm*{f}_2 \norm*{g}_2 + \norm*{g}_2^2  &  & \text{(by \cref{5.2.7}(b))}    \\
                      & = \big(\norm*{f}_2 + \norm*{g}_2\big)^2.
  \end{align*}
  Thus
  \begin{align*}
             & \norm*{f + g}_2^2 \leq (\norm*{f}_2 + \norm*{g}_2)^2                                  \\
    \implies & \norm*{f + g}_2 \leq \norm*{f}_2 + \norm*{g}_2.      &  & \text{(by \cref{5.2.5}(b))}
  \end{align*}
\end{proof}

\begin{proof}{(d)}
  We have
  \begin{align*}
    \norm*{f + g}_2^2 & = \inner*{f + g, f + g}                                                    &  & \by{ac:5.2.1}                  \\
                      & = \inner*{f, f} + \inner*{f, g} + \inner*{g, f} + \inner*{g, g}            &  & \text{(by \cref{5.2.5}(c)(d))} \\
                      & = \inner*{f, f} + \inner*{f, g} + \overline{\inner*{f, g}} + \inner*{g, g} &  & \text{(by \cref{5.2.5}(a))}    \\
                      & = \inner*{f, f} + \inner*{g, g}                                            &  & \text{(by hypothesis)}         \\
                      & = \norm*{f}_2^2 + \norm*{g}_2^2.                                           &  & \by{ac:5.2.1}
  \end{align*}
\end{proof}

\begin{proof}{(e)}
  We have
  \begin{align*}
    \norm*{cf}_2 & = \sqrt{\inner*{cf, cf}}              &  & \by{ac:5.2.1}                  \\
                 & = \sqrt{c \overline{c} \inner*{f, f}} &  & \text{(by \cref{5.2.5}(c)(d))} \\
                 & = \sqrt{\abs{c}^2 \inner*{f, f}}      &  & \by{4.6.11}                    \\
                 & = \abs{c} \sqrt{\inner*{f, f}}                                            \\
                 & = \abs{c} \norm*{f}_2.                &  & \by{ac:5.2.1}
  \end{align*}
\end{proof}

\begin{note}
  In light of Pythagoras' theorem, we sometimes say that \(f\) and \(g\) are \emph{orthogonal} iff \(\inner*{f, g} = 0\).
\end{note}

\begin{ac}\label{ac:5.2.2}
  We can now define the \(L^2\) metric \(d_{L^2}\) on \(C(\R / \Z ; \C)\) by defining
  \[
    d_{L^2}(f, g) \coloneqq \norm*{f - g}_2 = \bigg(\int_{[0, 1]} \abs{f(x) - g(x)}^2 \; dx\bigg)^{1 / 2}.
  \]
\end{ac}

\begin{rmk}\label{5.2.8}
  One can verify that \(d_{L^2}\) is indeed a metric.
  Indeed, the \(L^2\) metric is very similar to the \(l^2\) metric on Euclidean spaces \(\R^n\), which is why the notation is deliberately chosen to be similar;
  you should compare the two metrics yourself to see the analogy.
\end{rmk}

\begin{note}
  A sequence \(f_n\) of functions in \(C(\R / \Z ; \C)\) will \emph{converge in the \(L^2\) metric} to \(f \in C(\R / \Z ; \C)\) if \(d_{L^2}(f_n, f) \to 0\) as \(n \to \infty\), or in other words that
  \[
    \lim_{n \to \infty} \int_{[0, 1]} \abs{f_n(x) - f(x)}^2 \; dx = 0.
  \]
\end{note}

\begin{rmk}\label{5.2.9}
  The notion of convergence in \(L^2\) metric is different from that of uniform or pointwise convergence.
\end{rmk}

\begin{rmk}\label{5.2.10}
  The \(L^2\) metric is not as well-behaved as the \(L_\infty\) metric.
  For instance, it turns out the space \(C(\R / \Z ; \C)\) is not complete in the \(L^2\) metric, despite being complete in the \(L_\infty\) metric.
\end{rmk}

\exercisesection

\begin{ex}\label{ex:5.2.1}
  Prove \cref{5.2.5}.
\end{ex}

\begin{proof}
  See \cref{5.2.5}.
\end{proof}

\begin{ex}\label{ex:5.2.2}
  Prove \cref{5.2.7}.
\end{ex}

\begin{proof}
  See \cref{5.2.7}.
\end{proof}

\begin{ex}\label{ex:5.2.3}
  If \(f \in C(\R / \Z ; \C)\) is a non-zero function, show that \(0 < \norm*{f}_2 \leq \norm*{f}_\infty\).
  Conversely, if \(0 < A \leq B\) are real numbers, show that there exists a non-zero function \(f \in C(\R / \Z ; \C)\) such that \(\norm*{f}_2 = A\) and \(\norm*{f}_\infty = B\).
\end{ex}

\begin{proof}
  First we show that \(f \in C(\R / \Z ; \C)\) and \(f \neq 0\) implies \(0 < \norm*{f}_2 \leq \norm*{f}_{\infty}\).
  By \cref{5.2.7}(a) we know that \(0 < \norm*{f}_2\).
  Thus we only need to show that \(\norm*{f}_2 \leq \norm*{f}_\infty\).
  By \cref{5.1.5}(a) we know that \(f\) is bounded, thus by \cref{3.5.5}
  \[
    \norm*{f}_{\infty} = \sup_{y \in \R} \abs{f(x)} = \sup_{y \in [0, 1]} \abs{f(x)} \in \R^+ \cup \set{0}.
  \]
  Since
  \begin{align*}
    \norm*{f}_2^2 & = \int_{[0, 1]} \abs{f(x)}^2 \; dx                                  &  & \by{ac:5.2.1} \\
                  & \leq \int_{[0, 1]} \big(\sup_{y \in [0, 1]} \abs{f(y)}\big)^2 \; dx                    \\
                  & = \big(\sup_{y \in [0, 1]} \abs{f(y)}\big)^2                                           \\
                  & = \norm*{f}_{\infty}^2,                                             &  & \by{3.5.5}
  \end{align*}
  we know that
  \[
    \norm*{f}_2^2 \leq \norm*{f}_{\infty}^2 \implies \norm*{f}_2 \leq \norm*{f}_{\infty}.
  \]

  Now we show that for arbitrary \(A, B \in \R\), we have
  \[
    0 < A \leq B \implies \exists f \in C(\R / \Z ; \C) : \begin{dcases}
      f \neq 0        \\
      \norm*{f}_2 = A \\
      \norm*{f}_{\infty} = B
    \end{dcases}
  \]
  So let \(A, B \in \R\) such that \(0 < A \leq B\).
  We want to find some \(f \in C(\R / \Z ; \C)\) such that
  \begin{align*}
    A^2 & = \norm*{f}_2^2 = \int_{[0, 1]} \abs{f(x)}^2 \; dx;                  \\
    B^2 & = \norm*{f}_{\infty}^2 = \big(\sup_{x \in [0, 1]} \abs{f(x)}\big)^2.
  \end{align*}
  In particular, we want our \(f\) to look like
  \[
    \forall x \in [0, 1], f(x) = \sqrt{c + d g(x)},
  \]
  where \(c, d \in \R^+\) and \(g \in C(\R / \Z ; \C)\) such that \(g(\R) \subseteq \R^+ \cup \set{0}\).
  So we are trying to solve the following equations:
  \begin{align*}
    A^2 & = \int_{[0, 1]} \abs{\sqrt{c + dg(x)}}^2 \; dx           \\
        & = \int_{[0, 1]} c + dg(x) \; dx                          \\
        & = c + d \int_{[0, 1]} g(x) \; dx;                        \\
    B^2 & = \big(\sup_{x \in [0, 1]} \abs{\sqrt{c + dg(x)}}\big)^2 \\
        & = \sup_{x \in [0, 1]} \abs{\sqrt{c + dg(x)}}^2           \\
        & = \sup_{x \in [0, 1]} \big(c + dg(x)\big)                \\
        & = c + d \big(\sup_{x \in [0, 1]} g(x)\big).
  \end{align*}
  By setting
  \begin{align*}
     & c = \dfrac{A^2}{2};                                                                                                                                      \\
     & d = \dfrac{1}{2};                                                                                                                                        \\
     & \forall x \in [0, 1], g(x) = \begin{dcases}
                                      \dfrac{(2 B^2 - A^2)^2}{A^2} x                    & \text{if } x \in [0, \dfrac{A^2}{2 B^2 - A^2})                          \\
                                      \dfrac{-(2 B^2 - A^2)^2}{A^2} x + 2 (2 B^2 - A^2) & \text{if } x \in [\dfrac{A^2}{2 B^2 - A^2}, \dfrac{2 A^2}{2 B^2 - A^2}) \\
                                      0                                                 & \text{if } x \in [\dfrac{2 A^2}{2 B^2 - A^2}, 1]
                                    \end{dcases},
  \end{align*}
  we have
  \begin{align*}
     & \int_{[0, 1]} g(x) \; dx                                                                                                                                                                                                     \\
     & = \int_{[0, \dfrac{A^2}{2 B^2 - A^2}]} \dfrac{(2 B^2 - A^2)^2}{A^2} x \; dx + \int_{[\dfrac{A^2}{2 B^2 - A^2}, \dfrac{2 A^2}{2 B^2 - A^2}]} \dfrac{-(2 B^2 - A^2)^2}{A^2} x + 2 (2 B^2 - A^2) \; dx                          \\
     & = \dfrac{(2 B^2 - A^2)^2}{A^2} \bigg(\dfrac{x^2}{2}|_{x = 0}^{x = \dfrac{A^2}{2 B^2 - A^2}}\bigg) - \dfrac{(2 B^2 - A^2)^2}{A^2} \bigg(\dfrac{x^2}{2}|_{x = \dfrac{A^2}{2 B^2 - A^2}}^{x = \dfrac{2 A^2}{2 B^2 - A^2}}\bigg) \\
     & \quad + 2 (2 B^2 - A^2) \bigg(\dfrac{2 A^2}{2 B^2 - A^2} - \dfrac{A^2}{2 B^2 - A^2}\bigg)                                                                                                                                    \\
     & = \dfrac{A^2}{2} - 2 A^2 + \dfrac{A^2}{2} + 2 A^2                                                                                                                                                                            \\
     & = A^2
  \end{align*}
  and
  \begin{align*}
    \sup_{[0, 1]} g(x) & = \dfrac{(2 B^2 - A^2)^2}{A^2} \dfrac{A^2}{2 B^2 - A^2} \\
                       & = 2 B^2 - A^2.
  \end{align*}
  Thus
  \begin{align*}
    c + d \int_{[0, 1]} g(x) \; dx           & = \dfrac{A^2}{2} + \dfrac{A^2}{2}         \\
                                             & = A^2;                                    \\
    c + d \big(\sup_{x \in [0, 1]} g(x)\big) & = \dfrac{A^2}{2} + \dfrac{2 B^2 - A^2}{2} \\
                                             & = B^2.
  \end{align*}
  Note that the idea behind the definition of \(g\) is we try to build a triangle in the interval \([0, 1]\) with height equals to \(2 B^2 - A^2\) (this explains the result of supremum), and we want that triangle's area equals to \(A^2\) (this explains the result of integration).
  One can easily show that by extended \(g\) periodically with period \(1\) we know that \(g \in C(\R / \Z ; \C)\).
\end{proof}

\begin{ex}\label{ex:5.2.4}
  Prove that the \(d_{L^2}\) metric on \(C(\R / \Z ; \C)\) does indeed turn \(C(\R / \Z ; \C)\) into a metric space.
\end{ex}

\begin{proof}
  Let \(f, g, h \in C(\R / \Z ; \C)\).
  Since
  \begin{align*}
    d_{L^2}(f, f) & = \norm*{f - f}_2 &  & \by{ac:5.2.2}               \\
                  & = \norm*{0}_2     &  & \text{(by \cref{5.2.5}(b))} \\
                  & = 0,              &  & \text{(by \cref{5.2.7}(a))}
  \end{align*}
  we know that \(\big(C(\R / \Z ; \C), d_{L^2}\big)\) satisfies \cref{1.1.2}(a).
  Since
  \begin{align*}
             & f \neq g                                             \\
    \implies & f - g \neq 0        &  & \text{(by \cref{5.2.5}(b))} \\
    \implies & \norm*{f - g}_2 > 0 &  & \text{(by \cref{5.2.7}(a))} \\
    \implies & d_{L^2}(f, g) > 0,  &  & \by{ac:5.2.2}
  \end{align*}
  we know that \(\big(C(\R / \Z ; \C), d_{L^2}\big)\) satisfies \cref{1.1.2}(b).
  Since
  \begin{align*}
    d_{L^2}(f, g) & = \norm*{f - g}_2              &  & \by{ac:5.2.2}                  \\
                  & = \sqrt{\inner*{f - g, f - g}} &  & \by{ac:5.2.1}                  \\
                  & = \sqrt{\inner*{g - f, g - f}} &  & \text{(by \cref{5.2.5}(c)(d))} \\
                  & = \norm*{g - f}_2              &  & \by{ac:5.2.1}                  \\
                  & = d_{L^2}(g, f),               &  & \by{ac:5.2.2}
  \end{align*}
  we know that \(\big(C(\R / \Z ; \C), d_{L^2}\big)\) satisfies \cref{1.1.2}(c).
  Since
  \begin{align*}
    d_{L^2}(f, g) + d_{L^2}(g, h) & = \norm*{f - g}_2 + \norm*{g - h}_2 &  & \by{ac:5.2.2}               \\
                                  & \geq \norm{f - g + g - h}_2         &  & \text{(by \cref{5.2.7}(c))} \\
                                  & = \norm{f - h}_2                                                     \\
                                  & = d_{L^2}(f, h),                    &  & \by{ac:5.2.2}
  \end{align*}
  we know that \(\big(C(\R / \Z ; \C), d_{L^2}\big)\) satisfies \cref{1.1.2}(d).
  From all proofs above we conclude by \cref{1.1.2} that \(\big(C(\R / \Z ; \C), d_{L^2}\big)\) is a metric space.
\end{proof}

\begin{ex}\label{ex:5.2.5}
  Find a sequence of continuous periodic functions which converge in \(L^2\) to a discontinuous periodic function.
\end{ex}

\begin{proof}
  By \cref{5.1.4} we can define a \(\Z\)-periodic square wave function \(f : \R \to \C\) as follow:
  \[
    \forall x \in \R, f(x) = \begin{dcases}
      1 & \text{if } x \in [n, n + \dfrac{1}{2}) \text{ for some } n \in \Z     \\
      0 & \text{if } x \in [n + \dfrac{1}{2}, n + 1) \text{ for some } n \in \Z
    \end{dcases}
  \]
  Note that \(f\) is \(1\)-periodic but \(f\) is not continuous on \(\R\).
  Let \(\N_{\geq 10} = \set{n \in \N : n \geq 10}\).
  For each \(k \in \N_{\geq 10}\), we define \(f_k : [0, 1) \to \C\) to be the function:
  \[
    \forall x \in [0, 1), f_k(x) = \begin{dcases}
      kx                 & \text{if } x \in [0, \dfrac{1}{k})                           \\
      1                  & \text{if } x \in [\dfrac{1}{k}, \dfrac{1}{2} - \dfrac{1}{k}) \\
      -kx + \dfrac{k}{2} & \text{if } x \in [\dfrac{1}{2} - \dfrac{1}{k}, \dfrac{1}{2}) \\
      0                  & \text{if } x \in [0 + \dfrac{1}{2}, 1)
    \end{dcases}
  \]
  If we extended \(f_k\) periodically with period \(1\), then \(f_k \in C(\R / \Z ; \C)\) for all \(k \in \N_{\geq 10}\).
  Note that the choice of \(10\) is to make sure \(\dfrac{1}{k} < \dfrac{1}{2} - \dfrac{1}{k} < \dfrac{1}{2}\).
  Now we show that \((f_k)_{k = 10}^\infty\) converges to \(f\) on \([0, 1)\) with respect to \(d_{L^2}\).
  In particular, we want to show that
  \begin{align*}
         & \lim_{k \to \infty} d_{L^2}(f_k, f) = 0                                                                  \\
    \iff & \lim_{k \to \infty} \bigg(\int_{[0, 1]} \abs{f_k(x) - f(x)}^2 \; dx\bigg)^{1 / 2} = 0 &  & \by{ac:5.2.2} \\
    \iff & \lim_{k \to \infty} \int_{[0, 1]} \abs{f_k(x) - f(x)}^2 \; dx = 0.
  \end{align*}
  Since for each \(k \in \N_{\geq 10}\), we have
  \begin{align*}
     & \int_{[0, 1]} \abs{f_k(x) - f(x)}^2 \; dx                                                                                                                                                                                                    \\
     & = \int_{[0, \dfrac{1}{k}]} (1 - kx)^2 \; dx + \int_{[\dfrac{1}{2} - \dfrac{1}{k}, \dfrac{1}{2}]} \bigg(1 - \dfrac{k}{2} + kx\bigg)^2 \; dx                                                                                                   \\
     & = \int_{[0, \dfrac{1}{k}]} 1 - 2kx + k^2 x^2 \; dx + \int_{[\dfrac{1}{2} - \dfrac{1}{k}, \dfrac{1}{2}]} 1 - k + \dfrac{k^2}{4} + 2kx - k^2 x + k^2 x^2 \; dx                                                                                 \\
     & = \dfrac{1}{k} - 2k \bigg(\dfrac{x^2}{2}|_{x = 0}^{x = \dfrac{1}{k}}\bigg) + k^2 \bigg(\dfrac{x^3}{3}|_{x = 0}^{x = \dfrac{1}{k}}\bigg)                                                                                                      \\
     & \quad + \dfrac{1}{k} \bigg(1 - k + \dfrac{k^2}{4}\bigg) + (2k - k^2) \bigg(\dfrac{x^2}{2}|_{x = \dfrac{1}{2} - \dfrac{1}{k}}^{x = \dfrac{1}{2}}\bigg) + k^2 \bigg(\dfrac{x^3}{3}|_{x = \dfrac{1}{2} - \dfrac{1}{k}}^{x = \dfrac{1}{2}}\bigg) \\
     & = \dfrac{1}{k} - \dfrac{1}{k} + \dfrac{1}{3k} + \dfrac{1}{k} - 1 + \dfrac{k}{4} + \dfrac{2k - k^2}{2} \bigg(\dfrac{1}{k} - \dfrac{1}{k^2}\bigg) + \dfrac{k^2}{3} \bigg(\dfrac{3}{4k} - \dfrac{3}{2k^2} + \dfrac{1}{k^3}\bigg)                \\
     & = \dfrac{2}{3k},
  \end{align*}
  we know that
  \[
    \lim_{k \to \infty} \int_{[0, 1]} \abs{f_k(x) - f(x)}^2 \; dx = \lim_{k \to \infty} \dfrac{2}{3k} = 0.
  \]
  Thus \((f_k)_{k = 10}^\infty\) converges to \(f\) on \([0, 1)\) with respect to \(d_{L^2}\).
  Since \(f\) and \(f_k\) are \(1\)-periodic for all \(k \in \N_{\geq 10}\), we know that \((f_k)_{k = 10}^\infty\) converges to \(f\) on \(\R\) with respect to \(d_{L^2}\).
\end{proof}

\begin{ex}\label{ex:5.2.6}
  Let \(f \in C(\R / \Z ; \C)\), and let \((f_n)_{n = 1}^\infty\) be a sequence of functions in \(C(\R / \Z ; \C)\).
  \begin{enumerate}
    \item Show that if \(f_n\) converges uniformly to \(f\), then \(f_n\) also converges to \(f\) in the \(L^2\) metric.
    \item Give an example where \(f_n\) converges to \(f\) in the \(L^2\) metric, but does not converge to \(f\) uniformly.
    \item Give an example where \(f_n\) converges to \(f\) in the \(L^2\) metric, but does not converge to \(f\) pointwise.
    \item Give an example where \(f_n\) converges to \(f\) pointwise, but does not converge to \(f\) in the \(L^2\) metric.
  \end{enumerate}
\end{ex}

\begin{proof}{(a)}
  We have
  \begin{align*}
             & \forall \varepsilon \in \R^+, \exists N \in \Z^+ : \forall n \geq N, \forall x \in \R,                                           &  & \by{3.2.7}    \\
             & \abs{f_n(x) - f(x)} < \dfrac{\varepsilon^{\dfrac{1}{2}}}{2}                                                                                         \\
    \implies & \forall \varepsilon \in \R^+, \exists N \in \Z^+ : \forall n \geq N, \forall x \in [0, 1],                                                          \\
             & \abs{f_n(x) - f(x)} < \dfrac{\varepsilon^{\dfrac{1}{2}}}{2}                                                                                         \\
    \implies & \forall \varepsilon \in \R^+, \exists N \in \Z^+ : \forall n \geq N, \forall x \in [0, 1],                                                          \\
             & \abs{f_n(x) - f(x)}^2 < \dfrac{\varepsilon}{4}                                                                                                      \\
    \implies & \forall \varepsilon \in \R^+, \exists N \in \Z^+ : \forall n \geq N,                                                                                \\
             & \int_{[0, 1]} \abs{f_n(x) - f(x)}^2 \; dx \leq \int_{[0, 1]} \dfrac{\varepsilon}{4} \; dx = \dfrac{\varepsilon}{4} < \varepsilon                    \\
    \implies & \forall \varepsilon \in \R^+, \exists N \in \Z^+ : \forall n \geq N,                                                                                \\
             & d_{L^2}(f_n, f) < \varepsilon                                                                                                    &  & \by{ac:5.2.2} \\
    \implies & d_{L^2} - \lim_{n \to \infty} f_n = f.                                                                                           &  & \by{1.1.14}
  \end{align*}
\end{proof}

\begin{proof}{(b)}
  Let \(f \in C(\R / \Z ; \C)\) such that \(f = 0\) and let \(\N_{\geq 2} = \set{n \in \N : n \geq 2}\).
  For all \(n \in \N_{\geq 2}\), we define \(f_n \in C(\R / \Z ; \C)\) as follow:
  \[
    \forall x \in [0, 1), f_n(x) = \begin{dcases}
      0                           & \text{if } x \in [0, \dfrac{1}{2} - \dfrac{1}{n^3})            \\
      n^4 x + n - \dfrac{n^4}{2}  & \text{if } x \in [\dfrac{1}{2} - \dfrac{1}{n^3}, \dfrac{1}{2}) \\
      -n^4 x + n + \dfrac{n^4}{2} & \text{if } x \in [\dfrac{1}{2}, \dfrac{1}{2} + \dfrac{1}{n^3}) \\
      0                           & \text{if } x \in [\dfrac{1}{2} + \dfrac{1}{n^3}, 1)
    \end{dcases}
  \]
  Since for all \(n \in \N_{\geq 2}\), we have
  \begin{align*}
     & \int_{[0, 1]} \abs{f_n(x) - f(x)}^2 \; dx                                                                                                                                                                                                                                                                      \\
     & = \int_{[\dfrac{1}{2} - \dfrac{1}{n^3}, \dfrac{1}{2}]} (n^4 x + n - \dfrac{n^4}{2})^2 \; dx + \int_{[\dfrac{1}{2}, \dfrac{1}{2} + \dfrac{1}{n^3}]} (-n^4 x + n + \dfrac{n^4}{2})^2 \; dx                                                                                                                       \\
     & = \int_{[\dfrac{1}{2} - \dfrac{1}{n^3}, \dfrac{1}{2}]} n^8 x^2 + (2n^5 - n^8) x + n^2 - n^5 + \dfrac{n^8}{4} \; dx                                                                                                                                                                                             \\
     & \quad + \int_{[\dfrac{1}{2}, \dfrac{1}{2} + \dfrac{1}{n^3}]} n^8 x^2 + (-2n^5 - n^8) x + n^2 + n^5 + \dfrac{n^8}{4} \; dx                                                                                                                                                                                      \\
     & = n^8 \bigg(\dfrac{x^3}{3}|_{x = \dfrac{1}{2} - \dfrac{1}{n^3}}^{x = \dfrac{1}{2} + \dfrac{1}{n^3}}\bigg) + (2n^5 - n^8) \bigg(\dfrac{x^2}{2}|_{x = \dfrac{1}{2} - \dfrac{1}{n^3}}^{x = \dfrac{1}{2}}\bigg) + (-2n^5 - n^8) \bigg(\dfrac{x^2}{2}|_{x = \dfrac{1}{2}}^{x = \dfrac{1}{2} + \dfrac{1}{n^3}}\bigg) \\
     & \quad + \bigg(n^2 - n^5 + \dfrac{n^8}{4}\bigg) \dfrac{1}{n^3} + \bigg(n^2 + n^5 + \dfrac{n^8}{4}\bigg) \dfrac{1}{n^3}                                                                                                                                                                                          \\
     & = \dfrac{n^5}{2} + \dfrac{2}{3n} + n^2 - \dfrac{n^5}{2} - \dfrac{1}{n} + \dfrac{n^2}{2} - n^2 - \dfrac{n^5}{2} - \dfrac{1}{n} - \dfrac{n^2}{2} + \dfrac{2}{n} + \dfrac{n^5}{2}                                                                                                                                 \\
     & = \dfrac{2}{3n},
  \end{align*}
  we know that
  \begin{align*}
    \lim_{n \to \infty} d_{L^2}(f_n, f) & = \lim_{n \to \infty} \int_{[0, 1]} \abs{f_n(x) - f(x)}^2 \; dx &  & \by{ac:5.2.2} \\
                                        & = \lim_{n \to \infty} \dfrac{2}{3n}                                                \\
                                        & = 0.
  \end{align*}
  Thus by \cref{1.1.14} we have
  \[
    d_{L^2} - \lim_{n \to \infty} f_n = f.
  \]
  But for all \(n \in \N_{\geq 2}\), we have
  \begin{align*}
             & f_n(\dfrac{1}{2}) = n                                                                                                \\
    \implies & \abs{f_n(\dfrac{1}{2}) - f(\dfrac{1}{2})} = \abs{f_n(\dfrac{1}{2})} \geq n > 1                                       \\
    \implies & \exists \varepsilon \in \R^+ : \forall n \geq \N_{\geq 2}, \exists x \in [0, 1) : \abs{f_n(x) - f(x)} > \varepsilon.
  \end{align*}
  Thus by \cref{3.2.7} \((f_n)_{n = 2}^\infty\) does not converges uniformly to \(f\) on \(\R\) with respect to \(d_{l^1}|_{\R \times \R}\).
\end{proof}

\begin{proof}{(c)}
  Using the definition of \(f, f_n\) in \cref{ex:5.2.6}(b), we see that \((f_n)_{n = 2}^\infty\) does not converges pointwise to \(f\) on \(\R\) with respect to \(d_{l^1}|_{\R \times \R}\).
\end{proof}

\begin{proof}{(d)}
  Let \(f \in C(\R / \Z ; \C)\) such that \(f = 0\) and let \(\N_{\geq 2} = \set{n \in \N : n \geq 2}\).
  For all \(n \in \N_{\geq 2}\), we define \(f_n \in C(\R / \Z ; \C)\) as follow:
  \[
    \forall x \in [0, 1), f_n(x) = \begin{dcases}
      2 n^2 x       & \text{if } x \in [0, \dfrac{1}{2n})            \\
      -2 n^2 x + 2n & \text{if } x \in [\dfrac{1}{2n}, \dfrac{1}{n}) \\
      0             & \text{if } x \in [\dfrac{1}{n}, 1)
    \end{dcases}
  \]
  Since
  \begin{align*}
             & \lim_{n \to \infty} \dfrac{1}{n} = 0                                                                 \\
    \implies & \forall \varepsilon \in \R^+, \exists N \in \Z^+ : \forall n \geq N, \dfrac{1}{n} < x                \\
    \implies & \forall x \in (0, \dfrac{1}{2}), \exists N \in \Z^+ : \forall n \geq N, \dfrac{1}{n} < x             \\
    \implies & \forall x \in (0, \dfrac{1}{2}), \exists N \in \Z^+ : \forall n \geq N, f(\dfrac{1}{n}) = f_n(x) = 0 \\
    \implies & \forall x \in (0, \dfrac{1}{2}), \lim_{n \to \infty} f_n(x) = 0 = f(x)
  \end{align*}
  and
  \begin{align*}
     & \lim_{n \to \infty} f_n(0) = 0 = f(0)                                  \\
     & \forall x \in [\dfrac{1}{2}, 1), \lim_{n \to \infty} f_n(x) = 0 = f(x)
  \end{align*}
  by \cref{3.2.1} we know that \((f_n)_{n = 2}^\infty\) converges pointwise to \(f\) on \(\R\) with respect to \(d_{l^1}|_{\R \times \R}\).
  But
  \begin{align*}
     & \int_{[0, 1]} \abs{f_n(x) - f(x)}^2 \; dx                                                                                                                    \\
     & = \int_{[0, \dfrac{1}{2n}]} (2 n^2 x)^2 \; dx + \int_{[\dfrac{1}{2n}, \dfrac{1}{n}]} (-2n^2 x + 2n)^2 \; dx                                                  \\
     & = \int_{[0, \dfrac{1}{2n}]} 4 n^4 x^2 \; dx + \int_{[\dfrac{1}{2n}, \dfrac{1}{n}]} 4 n^4 x^2 - 4 n^3 x + 4n^2 \; dx                                          \\
     & = 4n^4 \bigg(\dfrac{x^3}{3}|_{x = 0}^{x = \dfrac{1}{n}}\bigg) - 4n^3 \bigg(\dfrac{x^2}{2}|_{x = \dfrac{1}{2n}}^{x = \dfrac{1}{n}}\bigg) + 4n^2 \dfrac{1}{2n} \\
     & = \dfrac{4n}{3} - \dfrac{3n}{2} + 2n                                                                                                                         \\
     & = \dfrac{11n}{6}
  \end{align*}
  implies
  \begin{align*}
    \lim_{n \to \infty} d_{L^2}(f_n, f) & = \lim_{n \to \infty} \int_{[0, 1]} \abs{f_n(x) - f(x)}^2 \; dx &  & \by{ac:5.2.2} \\
                                        & = \lim_{n \to \infty} \dfrac{11n}{6}                                               \\
                                        & = +\infty.
  \end{align*}
  Thus \((f_n)_{n = 2}^\infty\) does not converges to \(f\) with respect to \(d_{L^2}\).
\end{proof}
\section{Trigonometric polynomials}\label{ii:sec:5.3}

\begin{note}
  We now define the concept of a \emph{trigonometric polynomial}.
  Just as polynomials are combinations of the functions \(x^n\) (sometimes called \emph{monomials}), trigonometric polynomials are combinations of the functions \(e^{2 \pi i n x}\) (sometimes called \emph{characters}).
\end{note}

\begin{defn}[Characters]\label{ii:5.3.1}
  For every integer \(n\), we let \(e_n \in C(\R / \Z ; \C)\) denote the function
  \[
    e_n(x) \coloneqq e^{2 \pi i n x}.
  \]
  This is sometimes referred to as the \emph{character with frequency \(n\)}.
\end{defn}

\begin{defn}[Trigonometric polynomials]\label{ii:5.3.2}
  A function \(f\) in \(C(\R / \Z ; \C)\) is said to be a \emph{trigonometric polynomial} if we can write
  \(f = \sum_{n = -N}^N c_n e_n\) for some integer \(N \geq 0\) and some complex numbers \((c_n)_{n = -N}^N\).
\end{defn}

\setcounter{thm}{3}
\begin{eg}\label{ii:5.3.4}
  For any integer \(n\), the function \(\cos(2 \pi n x)\) is a trigonometric polynomial, since
  \[
    \cos(2 \pi n x) = \dfrac{e^{2 \pi n x} + e^{- 2 \pi n x}}{2} = \dfrac{1}{2} e_{-n} + \dfrac{1}{2} e_n.
  \]
  Similarly the function \(\sin(2 \pi n x) = \dfrac{-1}{2i} e_{-n} + \dfrac{1}{2i} e_n\) is a trigonometric polynomial.
  In fact, any linear combination of sines and cosines is also a trigonometric polynomial.
\end{eg}

\begin{lem}[Characters are an orthonormal system]\label{ii:5.3.5}
  For any integers \(n\) and \(m\), we have \(\inner*{e_n, e_m} = 1\) when \(n = m\) and \(\inner*{e_n, e_m} = 0\) when \(n \neq m\).
  Also, we have \(\norm*{e_n}_2 = 1\).
\end{lem}

\begin{proof}
  Let \(n, m \in \Z\).
  Observe that
  \begin{align*}
    \inner*{e_n, e_m} & = \int_{[0, 1]} e_n(x) \overline{e_m(x)} \; dx                   &  & \by{ii:5.2.1}      \\
                      & = \int_{[0, 1]} e^{2 \pi i n x} \overline{e^{2 \pi i m x}} \; dx &  & \by{ii:5.3.1}      \\
                      & = \int_{[0, 1]} e^{2 \pi i n x} e^{- 2 \pi i m x} \; dx          &  & \by{ii:4.7.2}[c,f] \\
                      & = \int_{[0, 1]} e^{2 \pi i n x - 2 \pi i m x} \; dx              &  & \by{ii:ex:4.6.16}  \\
                      & = \int_{[0, 1]} e^{2 \pi i (n - m) x} \; dx.
  \end{align*}
  If \(n = m\), then we have
  \begin{align*}
    \inner*{e_n, e_n} & = \int_{[0, 1]} e^{2 \pi i (n - n) x} \; dx                       \\
                      & = \int_{[0, 1]} e^0 \; dx                                         \\
                      & = \int_{[0, 1]} 1 \; dx                     &  & \by{ii:4.5.2}[e] \\
                      & = 1
  \end{align*}
  and
  \begin{align*}
    \norm*{e_n}_2 & = \sqrt{\inner*{e_n, e_n}} &  & \by{ii:ac:5.2.1} \\
                  & = \sqrt{1} = 1.
  \end{align*}
  If \(n \neq m\), then we have
  \begin{align*}
     & \inner*{e_n, e_m}                                                                                                     \\
     & = \int_{[0, 1]} e^{2 \pi i (n - m) x} \; dx                                                                           \\
     & = \int_{[0, 1]} \cos\big(2 \pi (n - m) x\big) + i \sin\big(2 \pi (n - m) x\big) \; dx        &  & \by{ii:4.7.2}[f]    \\
     & = \int_{[0, 1]} \cos\big(2 \pi (n - m) x\big) \; dx                                          &  & \by{ii:5.2.2}       \\
     & \quad + i \int_{[0, 1]} \sin\big(2 \pi (n - m) x\big) \; dx                                                           \\
     & = \bigg(\dfrac{\sin\big(2 \pi (n - m) x\big)}{2 \pi (n - m)}|_{x = 0}^{x = 1}\bigg)          &  & \by{ii:4.7.2}[b]    \\
     & \quad + i \bigg(\dfrac{-\cos\big(2 \pi (n - m) x\big)}{2 \pi (n - m)}|_{x = 0}^{x = 1}\bigg)                          \\
     & = 0 - 0                                                                                      &  & \by{ii:ac:4.7.2}[c] \\
     & \quad + i \bigg(\dfrac{- (-1) + (-1)}{2 \pi (n - m)}\bigg)                                   &  & \by{ii:ac:4.7.2}[f] \\
     & = 0.
  \end{align*}
\end{proof}

\begin{cor}\label{ii:5.3.6}
  Let \(f = \sum_{n = -N}^N c_n e_n\) be a trigonometric polynomial.
  Then we have the formula
  \[
    c_n = \inner*{f, e_n}
  \]
  for all integers \(-N \leq n \leq N\).
  Also, we have \(0 = \inner*{f, e_n}\) whenever \(n > N\) or \(n < -N\).
  Also, we have the identity
  \[
    \norm*{f}_2^2 = \sum_{n = -N}^N \abs{c_n}^2.
  \]
\end{cor}

\begin{proof}
  Let \(m \in \N\).
  Then we have
  \begin{align*}
    \inner*{f, e_m} & = \inner*{\sum_{n = -N}^N (c_n e_n), e_m}         &  & \text{(by hypothesis)} \\
                    & = \sum_{n = -N}^N \inner*{c_n e_n, e_m}           &  & \by{ii:5.2.5}[c]       \\
                    & = \sum_{n = -N}^N \big(c_n \inner*{e_n, e_m}\big) &  & \by{ii:5.2.5}[c]       \\
                    & = \begin{dcases}
                          c_m & \text{if } -N \leq m \leq N         \\
                          0   & \text{if } m > N \text{ or } m < -N
                        \end{dcases}      &  & \by{ii:5.3.5}
  \end{align*}
  and
  \begin{align*}
    \norm*{f}_2^2 & = \inner*{f, f}                                            &  & \by{ii:ac:5.2.1}              \\
                  & = \inner*{f, \sum_{n = -N}^N (c_n e_n)}                    &  & \text{(by hypothesis)}        \\
                  & = \sum_{n = -N}^N \inner*{f, c_n e_n}                      &  & \by{ii:5.2.5}[d]              \\
                  & = \sum_{n = -N}^N \big(\overline{c_n} \inner*{f, e_n}\big) &  & \by{ii:5.2.5}[d]              \\
                  & = \sum_{n = -N}^N \big(\overline{c_n} c_n\big)             &  & \text{(from the proof above)} \\
                  & = \sum_{n = -N}^N \abs{c_n}^2.                             &  & \by{ii:4.6.11}
  \end{align*}
\end{proof}

\begin{defn}[Fourier transform]\label{ii:5.3.7}
  For any function \(f \in C(\R / \Z ; \C)\), and any integer \(n \in \Z\), we define the \(n^{\opTh}\) \emph{Fourier coefficient of} \(f\), denoted \(\hat{f}(n)\), by the formula
  \[
    \hat{f}(n) \coloneqq \inner*{f, e_n} = \int_{[0, 1]} f(x) e^{- 2 \pi i n x} \; dx.
  \]
  The function \(\hat{f} : \Z \to \C\) is called the \emph{Fourier transform} of \(f\).
\end{defn}

\begin{ac}\label{ii:ac:5.3.1}
  From \cref{ii:5.3.6}, we see that whenever
  \[
    f = \sum_{n = -N}^N c_n e_n
  \]
  is a trigonometric polynomial, we have
  \[
    f = \sum_{n = -N}^N \inner*{f, e_n} e_n = \sum_{n = -\infty}^\infty \inner*{f, e_n} e_n
  \]
  and in particular, we have the \emph{Fourier inversion formula}
  \[
    f = \sum_{n = -\infty}^\infty \hat{f}(n) e_n
  \]
  or in other words
  \[
    f(x) = \sum_{n = -\infty}^\infty \hat{f}(n) e^{2 \pi i n x}.
  \]
  The right-hand side is referred to as the \emph{Fourier series} of \(f\).
  Also, from the second identity of \cref{ii:5.3.6} we have the \emph{Plancherel formula}
  \[
    \norm*{f}_2^2 = \sum_{n = -\infty}^\infty \abs{\hat{f}(n)}^2.
  \]
\end{ac}

\begin{rmk}\label{ii:5.3.8}
  We stress that at present we have only proven the Fourier inversion and Plancherel formulae in the case when \(f\) is a trigonometric polynomial.
  Note that in this case that the Fourier coefficients \(\hat{f}(n)\) are mostly zero (indeed, they can only be non-zero when \(-N \leq n \leq N\)), and so this infinite sum is really just a finite sum in disguise.
  In particular, there are no issues about what sense the above series converge in;
  they both converge pointwise, uniformly, and in \(L^2\) metric, since they are just finite sums.
\end{rmk}

\begin{note}
  In the next few sections we will extend the Fourier inversion and Plancherel formulae to general functions in \(C(\R / \Z ; \C)\), not just trigonometric polynomials.
  (It is also possible to extend the formula to discontinuous functions such as the square wave, but we will not do so here.)
  To do this we will need a version of the Weierstrass approximation theorem, this time requiring that a continuous periodic function be approximated uniformly by \emph{trigonometric} polynomials.
  Just as convolutions were used in the proof of the polynomial Weierstrass approximation theorem, we will also need a notion of convolution tailored for periodic functions.
\end{note}

\exercisesection

\begin{ex}\label{ii:ex:5.3.1}
  Show that the sum or product of any two trigonometric polynomials is again a trigonometric polynomial.
\end{ex}

\begin{proof}
  Let \(f, g \in C(\R / \Z ; \C)\) such that
  \begin{align*}
     & \exists N \in \N : \big((c_n)_{n = -N}^N \text{ is in } \C\big) \land \bigg(f = \sum_{n = -N}^N c_n e_n\bigg); \\
     & \exists M \in \N : \big((d_n)_{n = -M}^M \text{ is in } \C\big) \land \bigg(g = \sum_{n = -M}^M d_n e_n\bigg).
  \end{align*}
  Without the loss of generality suppose that \(N \leq M\).
  Then we have
  \begin{align*}
    f + g & = \sum_{n = -N}^N (c_n e_n) + \sum_{n = -M}^M (d_n e_n) \\
          & = \sum_{n = -M}^M (a_n e_n)
  \end{align*}
  where
  \[
    a_n = \begin{dcases}
      c_n + d_n & \text{if } -N \leq n \leq N                     \\
      d_n       & \text{if } (-M \leq n < -N) \lor (N < n \leq M)
    \end{dcases}
  \]
  For \(fg\), we induct on \(M\) to show that \(fg\) is trigonometric polynomial.
  For \(M = 0\), we have
  \begin{align*}
    fg & = \bigg(\sum_{n = -N}^N (c_n e_n)\bigg) (d_0 e^0)                       \\
       & = \bigg(\sum_{n = -N}^N (c_n e_n)\bigg) d_0       &  & \by{ii:4.5.2}[e] \\
       & = \sum_{n = -N}^N (c_n d_0 e_n).
  \end{align*}
  Clearly, \(fg\) is trigonometric polynomial and Thus, the base case holds.
  Suppose inductively that \(fg\) is trigonometric polynomial for some \(M \geq 0\).
  Then for \(M + 1\), we have
  \begin{align*}
    f g & = \bigg(\sum_{n = -N}^N (c_n e_n)\bigg) \bigg(\sum_{m = -(M + 1)}^{M + 1} (d_m e_m)\bigg)                                                        \\
        & = \bigg(\sum_{n = -N}^N (c_n e_n)\bigg) \bigg(\sum_{m = -M}^M (d_m e_m) + d_{-M - 1} e_{-M - 1} + d_{M + 1} e_{M + 1}\bigg)                      \\
        & = \bigg(\sum_{n = -N}^N (c_n e_n)\bigg) \bigg(\sum_{m = -M}^M (d_m e_m)\bigg) + \bigg(\sum_{n = -N}^N (c_n e_n)\bigg) (d_{-M - 1} e_{-M - 1})    \\
        & \quad + \bigg(\sum_{n = -N}^N (c_n e_n)\bigg) (d_{M + 1} e_{M + 1})                                                                              \\
        & = \bigg(\sum_{n = -N}^N (c_n e_n)\bigg) \bigg(\sum_{m = -M}^M (d_m e_m)\bigg) + \sum_{n = -N}^N (c_n d_{-M - 1} e_{n - M - 1})                   \\
        & \quad + \sum_{n = -N}^N (c_n d_{M + 1} e_{n + M + 1})                                                                                            \\
        & = \bigg(\sum_{n = -N}^N (c_n e_n)\bigg) \bigg(\sum_{m = -M}^M (d_m e_m)\bigg) + \sum_{n = -N - M - 1}^{N - M - 1} (c_{n + M + 1} d_{-M - 1} e_n) \\
        & \quad + \sum_{n = -N + M + 1}^{N + M + 1} (c_{n - M - 1} d_{M + 1} e_n).
  \end{align*}
  By setting
  \begin{align*}
     & a_n = \begin{dcases}
               c_{n + M + 1} d_{-M - 1} & \text{if } -N - M - 1 \leq n \leq N - M - 1 \\
               0                        & \text{if } N - M - 1 < n \leq N + M + 1
             \end{dcases} \\
     & b_n = \begin{dcases}
               c_{n - M - 1} d_{M + 1} & \text{if } -N + M + 1 \leq n \leq N + M + 1 \\
               0                       & \text{if } -N - M - 1 \leq n < -N + M + 1
             \end{dcases}
  \end{align*}
  we have
  \[
    fg = \bigg(\sum_{n = -N}^N (c_n e_n)\bigg) \bigg(\sum_{m = -M}^M (d_m e_m)\bigg) + \sum_{n = -N - M - 1}^{N + M + 1} (a_n e_n) + \sum_{n = -N - M - 1}^{N + M + 1} (b_n e_n).
  \]
  By the induction hypothesis we know that \(\bigg(\sum_{n = -N}^N (c_n e_n)\bigg) \bigg(\sum_{m = -M}^M (d_m e_m)\bigg)\) is trigonometric polynomial.
  Thus, from the proof above we know that \(fg\) is trigonometric polynomial, and this closes the induction.
\end{proof}

\begin{ex}\label{ii:ex:5.3.2}
  Prove \cref{ii:5.3.5}.
\end{ex}

\begin{proof}
  See \cref{ii:5.3.5}.
\end{proof}

\begin{ex}\label{ii:ex:5.3.3}
  Prove \cref{ii:5.3.6}.
\end{ex}

\begin{proof}
  See \cref{ii:5.3.6}.
\end{proof}

\section{Periodic convolutions}\label{sec:5.4}

\begin{thm}\label{5.4.1}
  Let \(f \in C(\R / \Z ; \C)\), and let \(\varepsilon > 0\).
  Then there exists a trigonometric polynomial \(P\) such that \(\norm*{f - P}_{\infty} \leq \varepsilon\).
\end{thm}

\begin{proof}
  Let \(f\) be any element of \(C(\R / \Z ; \C)\);
  we know that \(f\) is bounded (by \cref{5.1.5}(a)), so that we have some \(M > 0\) such that \(\abs{f(x)} \leq M\) for all \(x \in \R\).

  Let \(\varepsilon > 0\) be arbitrary.
  Since \(f\) is uniformly continuous (by \cref{2.3.5}), there exists a \(\delta > 0\) such that \(\abs{f(x) - f(y)} \leq \varepsilon\) whenever \(\abs{x - y} \leq \delta\).
  Now use \cref{5.4.6} to find a trigonometric polynomial \(P\) which is a \((\varepsilon, \delta)\) approximation to the identity.
  Then \(f * P\) is also a trigonometric polynomial (by \cref{ac:5.4.1}).
  We now estimate \(\norm*{f - f * P}_{\infty}\).

  Let \(x\) be any real number.
  We have
  \begin{align*}
     & \abs{f(x) - f * P(x)}                                                                                         \\
     & = \abs{f(x) - P * f(x)}                                                   &  & \text{(by \cref{5.4.4}(a)(b))} \\
     & = \abs{f(x) - \int_{[0, 1]} f(x - y) P(y) \; dy}                          &  & \by{5.4.2}                     \\
     & = \abs{\int_{[0, 1]} f(x) P(y) \; dy - \int_{[0, 1]} f(x - y) P(y) \; dy} &  & \text{(by \cref{5.4.5}(a))}    \\
     & = \abs{\int_{[0, 1]} \big(f(x) - f(x - y)\big) P(y) \; dy}                                                    \\
     & \leq \int_{[0, 1]} \abs{f(x) - f(x - y)} P(y) \; dy.                      &  & \by{5.2.2}
  \end{align*}
  The right-hand side can be split as
  \begin{align*}
    \int_{[0, \delta]} \abs{f(x) - f(x - y)} P(y) \; dy & + \int_{[\delta, 1 - \delta]} \abs{f(x) - f(x - y)} P(y) \; dy \\
                                                        & + \int_{[1 - \delta, 1]} \abs{f(x) - f(x - y)} P(y) \; dy
  \end{align*}
  which we can bound from above by
  \begin{align*}
     & \leq \int_{[0, \delta]} \varepsilon P(y) \; dy + \int_{[\delta, 1 - \delta]} 2 M \varepsilon \; dy + \int_{[1 - \delta, 1]} \abs{f(x - 1) - f(x - y)} P(y) \; dy \\
     & \leq \int_{[0, \delta]} \varepsilon P(y) \; dy + \int_{[\delta, 1 - \delta]} 2 M \varepsilon \; dy + \int_{[1 - \delta, 1]} \varepsilon P(y) \; dy               \\
     & \leq \varepsilon + 2 M \varepsilon + \varepsilon                                                                                                                 \\
     & = (2M + 2) \varepsilon.
  \end{align*}
  Thus we have \(\norm*{f - f * P}_{\infty} \leq (2M + 2) \varepsilon\).
  Since \(M\) is fixed and \(\varepsilon\) is arbitrary, we can thus make \(f * P\) arbitrarily close to \(f\) in sup norm, which proves the periodic Weierstrass approximation theorem.
\end{proof}

\begin{note}
  \cref{5.4.1} asserts that any continuous periodic function can be uniformly approximated by trigonometric polynomials.
  To put it another way, if we let
  \[
    P(\R / \Z ; \C)
  \]
  denote the space of all trigonometric polynomials, then the closure of \(P(\R / \Z ; \C)\) in the \(L^\infty\) metric is \(C(\R / \Z ; \C)\).
\end{note}

\begin{note}
  It is possible to prove this theorem directly from the Weierstrass approximation theorem for polynomials (\cref{3.8.3}), and both theorems are a special case of a much more general theorem known as the \emph{Stone-Weierstrass theorem}, which we will not discuss here.
  However we shall instead prove this theorem from scratch, in order to introduce a couple of interesting notions, notably that of periodic convolution.
  The proof here, though, should strongly remind you of the arguments used to prove \cref{3.8.3}.
\end{note}

\begin{defn}[Periodic convolution]\label{5.4.2}
  Let \(f, g \in C(\R / \Z ; \C)\).
  Then we define the periodic convolution \(f * g : \R \to \C\) of \(f\) and \(g\) by the formula
  \[
    f * g(x) \coloneqq \int_{[0, 1]} f(y) g(x - y) \; dy
  \]
\end{defn}

\begin{rmk}\label{5.4.3}
  Note that \cref{5.4.2} is slightly different from the convolution for compactly supported functions defined in \cref{3.8.9}, because we are only integrating over \([0, 1]\) and not on all of \(\R\).
  Thus, in principle we have given the symbol \(f * g\) two conflicting meanings.
  However, in practice there will be no confusion, because it is not possible for a non-zero function to both be periodic and compactly supported.
\end{rmk}

\begin{lem}[Basic properties of periodic convolution]\label{5.4.4}
  Let \(f, g, h \in C(\R / \Z ; \C)\).
  \begin{enumerate}
    \item (Closure)
          The convolution \(f * g\) is continuous and \(\Z\)-periodic.
          In other words, \(f * g \in C(\R / \Z ; \C)\).
    \item (Commutativity)
          We have \(f * g = g * f\).
    \item (Bilinearity)
          We have \(f * (g + h) = f * g + f * h\) and \((f + g) * h = f * h + g * h\).
          For any complex number \(c\), we have \(c(f * g) = (cf) * g = f * (cg)\).
  \end{enumerate}
\end{lem}

\begin{proof}{(a)}
  By \cref{5.2.2} we know that \(f * g\) is continuous on \(\R\).
  Since
  \begin{align*}
    \forall x \in \R, f * g(x + 1) & = \int_{[0, 1]} f(y) g(x + 1 - y) \; dy &                         & \by{5.4.2} \\
                                   & = \int_{[0, 1]} f(y) g(x - y) \; dy     & (g \in C(\R / \Z ; \C))              \\
                                   & = f * g(x),                             &                         & \by{5.4.2}
  \end{align*}
  we know that \(f * g \in C(\R / \Z ; \C)\).
\end{proof}

\begin{proof}{(b)}
  Let \(x \in \R\) and let \(\phi : [x - 1, x] \mapsto [0, 1]\) be the function \(\phi(y) = x - y\).
  Then we have
  \begin{align*}
     & f * g(x)                                                                                                                      \\
     & = \int_{[0, 1]} f(y) g(x - y) \; dy                                           &  & \by{5.4.2}                                 \\
     & = \int_{\big[\phi(x), \phi(x - 1)\big]} f(y) g(x - y) \; dy                                                                   \\
     & = -\int_{[x - 1, x]} f\big(\phi(y)\big) g\big(x - \phi(y)\big) \phi'(y) \; dy &  & \text{(by Exercise 11.10.4 in Analysis I)} \\
     & = \int_{[x - 1, x]} f(x - y) g(y) \; dy                                                                                       \\
     & = \int_{[x - 1, x]} g(y) f(x - y) \; dy.                                                                                      \\
  \end{align*}
  Let \([x]\) be the integer defined in \cref{ex:5.1.1}.
  Then we have
  \begin{align*}
             & [x] \leq x < [x] + 1     \\
    \implies & [x] - 1 \leq x - 1 < [x]
  \end{align*}
  and
  \begin{align*}
     & f * g(x)                                                                                                                                  \\
     & = \int_{[x - 1, x]} g(y) f(x - y) \; dy                                                                                                   \\
     & = \int_{\big[x - 1, [x]\big]} g(y) f(x - y) \; dy + \int_{\big[[x], x\big]} g(y) f(x - y) \; dy                                           \\
     & = \int_{\big[[x], x\big]} g(y) f(x - y) \; dy + \int_{\big[x - 1, [x]\big]} g(y) f(x - y) \; dy                                           \\
     & = \int_{\big[[x], x\big]} g(y) f(x - y) \; dy                                                                                             \\
     & \quad + \int_{\big[x - 1 + 1, [x] + 1\big]} g(y - 1) f(x - y - 1) \; dy                                                                   \\
     & = \int_{\big[[x], x\big]} g(y) f(x - y) \; dy + \int_{\big[x, [x] + 1\big]} g(y) f(x - y) \; dy & (f, g \in C(\R / \Z ; \C))              \\
     & = \int_{\big[[x], [x] + 1\big]} g(y) f(x - y) \; dy                                                                                       \\
     & = \int_{\big[[x] - [x], [x] + 1 - [x]\big]} g(y + [x]) f(x - y + [x]) \; dy                                                               \\
     & = \int_{[0, 1]} g(y) f(x - y) \; dy                                                             & (f, g \in C(\R / \Z ; \C))              \\
     & = g * f(x).                                                                                     &                            & \by{5.4.2}
  \end{align*}
  Since \(x\) is arbitrary, we conclude that \(f * g = g * f\).
\end{proof}

\begin{proof}{(c)}
  By \cref{5.1.5}(b) we know that \(f + g, g + h, cf, cg \in C(\R / \Z ; \C)\).
  Thus \(f * (g + h), (f + g) * h, (cf) * g, f * (cg)\) are well-defined.
  Let \(x \in \R\).
  Then we have
  \begin{align*}
     & \big(f * (g + h)\big)(x)                                                                                              \\
     & = \int_{[0, 1]} f(y) \cdot (g + h)(x - y) \; dy                         &  & \by{5.4.2}                               \\
     & = \int_{[0, 1]} f(y) \cdot \big(g(x - y) + h(x - y)\big) \; dy                                                        \\
     & = \int_{[0, 1]} f(y) g(x - y) + f(y) h(x - y) \; dy                                                                   \\
     & = \int_{[0, 1]} f(y) g(x - y) \; dy + \int_{[0, 1]} f(y) h(x - y) \; dy &  & \text{(cf the proof of \cref{5.2.5}(c))} \\
     & = (f * g)(x) + (f * h)(x)                                               &  & \by{5.4.2}                               \\
     & = (f * g + f * h)(x)
  \end{align*}
  and
  \begin{align*}
    \big((cf) * g\big)(x) & = \int_{[0, 1]} (cf)(y) \cdot g(x - y) \; dy &  & \by{5.4.2}                               \\
                          & = \int_{[0, 1]} c f(y) g(x - y) \; dy                                                      \\
                          & = c \int_{[0, 1]} f(y) g(x - y) \; dy        &  & \text{(cf the proof of \cref{5.2.5}(c))} \\
                          & = c (f * g)(x).                              &  & \by{5.4.2}
  \end{align*}
  Since \(x\) is arbitrary, we conclude that \(f * (g + h) = f * g + f * h\) and \((cf) * g = c (f * g)\).
  This implies
  \begin{align*}
    (f + g) * h & = h * (f + g)   &  & \text{(by \cref{5.4.4}(b))}   \\
                & = h * f + h * g &  & \text{(from the proof above)} \\
                & = f * h + g * h &  & \text{(by \cref{5.4.4}(b))}
  \end{align*}
  and
  \begin{align*}
    f * (cg) & = (cg) * f  &  & \text{(by \cref{5.4.4}(b))}   \\
             & = c(g * f)  &  & \text{(from the proof above)} \\
             & = c(f * g). &  & \text{(by \cref{5.4.4}(b))}
  \end{align*}
\end{proof}

\begin{ac}\label{ac:5.4.1}
  Now we observe an interesting identity:
  for any \(f \in C(\R / \Z ; \C)\) and any integer \(n\), we have
  \[
    f * e_n = \hat{f}(n) e_n.
  \]
  To prove this, we compute
  \begin{align*}
    f * e_n(x) & = \int_{[0, 1]} f(y) e_n(x - y) \; dy                        &  & \by{5.4.2}                               \\
               & = \int_{[0, 1]} f(y) e^{2 \pi i n (x - y)} \; dy             &  & \by{5.3.1}                               \\
               & = e^{2 \pi i n x} \int_{[0, 1]} f(y) e^{- 2 \pi i n y} \; dy &  & \text{(cf the proof of \cref{5.2.5}(c))} \\
               & = \inner*{f, e_n} e^{2 \pi i n x}                            &  & \by{5.2.1}                               \\
               & = \hat{f}(n) e^{2 \pi i n x}                                 &  & \by{5.3.7}                               \\
               & = \hat{f}(n) e_n                                             &  & \by{5.3.1}
  \end{align*}
  as desired.
  More generally, we see from \cref{5.4.4}(c) that for any trigonometric polynomial \(P = \sum_{n = -N}^N c_n e_n\), we have
  \[
    f * P = \sum_{n = -N}^N c_n (f * e_n) = \sum_{n = -N}^N \hat{f}(n) c_n e_n.
  \]
  Thus the periodic convolution of any function in \(C(\R / \Z ; \C)\) with a trigonometric polynomial, is again a trigonometric polynomial.
  (Compare with \cref{3.8.13}.)
\end{ac}

\begin{defn}[Periodic approximation to the identity]\label{5.4.5}
  Let \(\varepsilon > 0\) and \(0 < \delta < 1 / 2\).
  A function \(f \in C(\R / \Z ; \C)\) is said to be a \emph{periodic \((\varepsilon, \delta)\) approximation to the identity} if the following properties are true:
  \begin{enumerate}
    \item \(f(x) \geq 0\) for all \(x \in \R\), and \(\int_{[0, 1]} f(x) \; dx = 1\).
    \item We have \(f(x) < \varepsilon\) for all \(\delta \leq \abs{x} \leq 1 - \delta\).
  \end{enumerate}
\end{defn}

\begin{ac}[Fejér kernel]\label{ac:5.4.2}
  Let \(N \geq 1\) be an integer.
  Then we have
  \[
    \sum_{n = -N}^N \bigg(1 - \dfrac{\abs{n}}{N}\bigg) e_n = \dfrac{1}{N} \abs{\sum_{n = 0}^{N - 1} e_n}^2.
  \]
\end{ac}

\begin{proof}
  First we claim that
  \[
    \sum_{n = 0}^{2N - 2} (N - \abs{n - N + 1}) \cdot e_n = \bigg(\sum_{n = 0}^{N - 1} e_n\bigg)^2.
  \]
  We proof the claim by induction on \(N\) and we start with \(N = 1\).
  For \(N = 1\), we have
  \begin{align*}
    \sum_{n = 0}^0 (1 - \abs{n - 1 + 1}) \cdot e_n & = \sum_{n = 0}^0 (1 - \abs{n}) e_n                                   \\
                                                   & = 1 e_0                                                              \\
                                                   & = 1                                 &  & \text{(by \cref{4.5.2}(e))} \\
                                                   & = e_0^2                             &  & \text{(by \cref{4.5.2}(e))} \\
                                                   & = \bigg(\sum_{n = 0}^0 e_n\bigg)^2.
  \end{align*}
  Thus the base case holds.
  Suppose inductively that the claim is true for some \(N \geq 1\).
  Then for \(N + 1\), we want to show that
  \[
    \sum_{n = 0}^{2(N + 1) - 2} \big(N + 1 - \abs{n - (N + 1) + 1}\big) \cdot e_n = \sum_{n = 0}^{2N} (N + 1 - \abs{n - N}) \cdot e_n = \bigg(\sum_{n = 0}^N e_n\bigg)^2.
  \]
  This is true since
  \begin{align*}
     & \bigg(\sum_{n = 0}^N e_n\bigg)^2                                                                                                               \\
     & = \Bigg(\bigg(\sum_{n = 0}^{N - 1} e_n\bigg) + e_N\Bigg)^2                                                                     &  & \by{4.6.4} \\
     & = \bigg(\sum_{n = 0}^{N - 1} e_n\bigg)^2 + 2 e_N \bigg(\sum_{n = 0}^{N - 1} e_n\bigg) + e_N^2                                  &  & \by{4.6.6} \\
     & = \sum_{n = 0}^{2N - 2} (N - \abs{n - N + 1}) \cdot e_n + 2 e_N \bigg(\sum_{n = 0}^{N - 1} e_n\bigg) + e_N^2                   &  & \byIH      \\
     & = \sum_{n = 0}^{N - 1} (N - \abs{n - N + 1}) \cdot e_n + \sum_{n = N}^{2N - 2} (N - \abs{n - N + 1}) \cdot e_n                 &  & \by{4.6.4} \\
     & \quad + 2 e_N \bigg(\sum_{n = 0}^{N - 1} e_n\bigg) + e_N^2                                                                                     \\
     & = \sum_{n = 0}^{N - 1} (n + 1) \cdot e_n + \sum_{n = N}^{2N - 2} (2N - n - 1) \cdot e_n                                                        \\
     & \quad + 2 e_N \bigg(\sum_{n = 0}^{N - 1} e_n\bigg) + e_N^2                                                                                     \\
     & = \sum_{n = 0}^{N - 1} n e_n + \sum_{n = 0}^{N - 1} e_n + \sum_{n = N}^{2N - 2} (2N - n) \cdot e_n - \sum_{n = N}^{2N - 2} e_n &  & \by{4.6.6} \\
     & \quad + 2 \bigg(\sum_{n = 0}^{N - 1} e_N e_n\bigg) + e_N^2                                                                     &  & \by{4.6.6} \\
     & = \sum_{n = 0}^{N - 1} n e_n + \sum_{n = N}^{2N - 2} (2N - n) \cdot e_n                                                        &  & \by{4.6.4} \\
     & \quad + \sum_{n = 0}^{N - 1} e_n - \sum_{n = N}^{2N - 2} e_n + 2 \bigg(\sum_{n = 0}^{N - 1} e_N e_n\bigg) + e_N^2                              \\
     & = \sum_{n = 0}^{N - 1} (N - \abs{n - N}) \cdot e_n + \sum_{n = N}^{2N - 2} (N - \abs{n - N}) \cdot e_n                                         \\
     & \quad + \sum_{n = 0}^{N - 1} e_n - \sum_{n = N}^{2N - 2} e_n + 2 \bigg(\sum_{n = 0}^{N - 1} e_N e_n\bigg) + e_N^2
  \end{align*}
  \begin{align*}
     & = \sum_{n = 0}^{2N - 2} (N - \abs{n - N}) \cdot e_n                                                               &  & \text{(conti. from above)} \\
     & \quad + \sum_{n = 0}^{N - 1} e_n - \sum_{n = N}^{2N - 2} e_n + 2 \bigg(\sum_{n = 0}^{N - 1} e_N e_n\bigg) + e_N^2                                 \\
     & = \sum_{n = 0}^{2N - 2} (N + 1 - \abs{n - N}) \cdot e_n - \sum_{n = 0}^{2N - 2} e_n                               &  & \by{4.6.6}                 \\
     & \quad + \sum_{n = 0}^{N - 1} e_n - \sum_{n = N}^{2N - 2} e_n + 2 \bigg(\sum_{n = 0}^{N - 1} e_N e_n\bigg) + e_N^2                                 \\
     & = \sum_{n = 0}^{2N - 2} (N + 1 - \abs{n - N}) \cdot e_n                                                           &  & \by{4.6.6}                 \\
     & \quad - 2 \bigg(\sum_{n = N}^{2N - 2} e_n\bigg) + 2 \bigg(\sum_{n = 0}^{N - 1} e_N e_n\bigg) + e_N^2                                              \\
     & = \sum_{n = 0}^{2N} (N + 1 - \abs{n - N}) \cdot e_n - 2 e_{2N - 1} - e_{2N}                                       &  & \by{4.6.6}                 \\
     & \quad - 2 \bigg(\sum_{n = N}^{2N - 2} e_n\bigg) + 2 \bigg(\sum_{n = 0}^{N - 1} e_N e_n\bigg) + e_N^2                                              \\
     & = \sum_{n = 0}^{2N} (N + 1 - \abs{n - N}) \cdot e_n - 2 e_{2N - 1} - e_{2N}                                                                       \\
     & \quad - 2 \bigg(\sum_{n = N}^{2N - 2} e_n\bigg) + 2 \bigg(\sum_{n = 0}^{N - 1} e_{n + N}\bigg) + e_{2N}           &  & \by{ex:4.6.16}             \\
     & = \sum_{n = 0}^{2N} (N + 1 - \abs{n - N}) \cdot e_n                                                               &  & \by{4.6.4}                 \\
     & \quad - 2 \bigg(\sum_{n = N}^{2N - 1} e_n\bigg) + 2 \bigg(\sum_{n = 0}^{N - 1} e_{n + N}\bigg)                                                    \\
     & = \sum_{n = 0}^{2N} (N + 1 - \abs{n - N}) \cdot e_n                                                                                               \\
     & \quad - 2 \bigg(\sum_{n = N}^{2N - 1} e_n\bigg) + 2 \bigg(\sum_{n = N}^{2N - 1} e_n\bigg)                                                         \\
     & = \sum_{n = 0}^{2N} (N + 1 - \abs{n - N}) \cdot e_n.
  \end{align*}
  This closes the induction.

  Using the claim above we have
  \begin{align*}
     & \sum_{n = -N}^N \bigg(1 - \dfrac{\abs{n}}{N}\bigg) e_n                                                                                 \\
     & = \dfrac{1}{N} \sum_{n = -N}^N (N - \abs{n}) e_n                                                    &  & \by{4.6.6}                    \\
     & = \dfrac{1}{N} \sum_{n = -(N - 1)}^{N - 1} (N - \abs{n}) \cdot e_n                                                                     \\
     & = \dfrac{1}{N} \sum_{n = 0}^{2N - 2} (N - \abs{n - N + 1}) \cdot e_{n - N + 1}                                                         \\
     & = \dfrac{1}{N} \sum_{n = 0}^{2N - 2} (N - \abs{n - N + 1}) \cdot e_n \cdot e_{-N + 1}               &  & \by{ex:4.6.16}                \\
     & = \dfrac{e_{-N + 1}}{N} \sum_{n = 0}^{2N - 2} (N - \abs{n - N + 1}) \cdot e_n                       &  & \by{4.6.6}                    \\
     & = \dfrac{e_{-N + 1}}{N} \bigg(\sum_{n = 0}^{N - 1} e_n\bigg)^2                                      &  & \text{(from the claim above)} \\
     & = \dfrac{e_{-N + 1}}{N} \bigg(\sum_{n = 0}^{N - 1} e_n\bigg) \bigg(\sum_{n = 0}^{N - 1} e_n\bigg)                                      \\
     & = \dfrac{1}{N} \bigg(\sum_{n = 0}^{N - 1} e_n\bigg) \bigg(\sum_{n = 0}^{N - 1} e_{-N + 1} e_n\bigg) &  & \by{4.6.6}                    \\
     & = \dfrac{1}{N} \bigg(\sum_{n = 0}^{N - 1} e_n\bigg) \bigg(\sum_{n = 0}^{N - 1} e_{n - N + 1}\bigg)  &  & \by{ex:4.6.16}                \\
     & = \dfrac{1}{N} \bigg(\sum_{n = 0}^{N - 1} e_n\bigg) \bigg(\sum_{n = 0}^{N - 1} e_{-n}\bigg)         &  & \by{4.6.4}                    \\
     & = \dfrac{1}{N} \bigg(\sum_{n = 0}^{N - 1} e_n\bigg) \bigg(\sum_{n = 0}^{N - 1} \overline{e_n}\bigg) &  & \by{4.6.15}                   \\
     & = \dfrac{1}{N} \bigg(\sum_{n = 0}^{N - 1} e_n\bigg) \bigg(\overline{\sum_{n = 0}^{N - 1} e_n}\bigg) &  & \by{4.6.9}                    \\
     & = \dfrac{1}{N} \abs{\sum_{n = 0}^{N - 1} e_n}^2.                                                    &  & \by{4.6.11}
  \end{align*}
\end{proof}

\begin{lem}\label{5.4.6}
  For every \(\varepsilon > 0\) and \(0 < \delta < 1 / 2\), there exists a trigonometric polynomial \(P\) which is an \((\varepsilon, \delta)\) approximation to the identity.
\end{lem}

\begin{proof}
  Let \(N \geq 1\) be an integer.
  We define the \emph{Fejér kernel} \(F_N\) to be the function
  \[
    F_N = \sum_{n = -N}^N \bigg(1 - \dfrac{\abs{n}}{N}\bigg) e_n.
  \]
  Clearly \(F_N\) is a trigonometric polynomial.
  We observe the identity
  \[
    F_N = \dfrac{1}{N} \abs{\sum_{n = 0}^{N - 1} e_n}^2
  \]
  by \cref{ac:5.4.2}.
  But from the geometric series formula (Lemma 7.3.3 in Analysis I) we have
  \begin{align*}
    \sum_{n = 0}^{N - 1} e_n(x) & = \sum_{n = 0}^{N - 1} \big(e_1(x)\big)^n                                                                                &  & \by{ex:4.6.16}              \\
                                & = \dfrac{\big(e_1(x)\big)^N - 1}{e_1(x) - 1}                                                                             &  & \text{(geometric series)}   \\
                                & = \dfrac{\big(e_1(x)\big)^N - e_0(x)}{e_1(x) - e_0(x)}                                                                   &  & \text{(by \cref{4.5.2}(e))} \\
                                & = \dfrac{e_N(x) - e_0(x)}{e_1(x) - e_0(x)}                                                                               &  & \by{ex:4.6.16}              \\
                                & = \dfrac{e^{2 \pi i N x} - e^0}{e^{2 \pi i x} - e^0}                                                                     &  & \by{5.3.1}                  \\
                                & = \dfrac{e^{\pi i N x} e^{\pi i N x} - e^{\pi i N x} e^{-\pi i N x}}{e^{\pi i x} e^{\pi i x} - e^{\pi i x} e^{-\pi i x}} &  & \by{ex:4.6.16}              \\
                                & = \dfrac{e^{\pi i N x} (e^{\pi i N x} - e^{-\pi i N x})}{e^{\pi i x} (e^{\pi i x} - e^{-\pi i x})}                       &  & \by{4.6.6}                  \\
                                & = \dfrac{e^{\pi i (N - 1) x} (e^{\pi i N x} - e^{-\pi i N x})}{e^{\pi i x} - e^{-\pi i x}}                               &  & \by{4.6.12}                 \\
                                & = \dfrac{2i e^{\pi i (N - 1) x} \sin(\pi N x)}{2i \sin(\pi x)}                                                           &  & \by{4.7.1}                  \\
                                & = \dfrac{e^{\pi i (N - 1) x} \sin(\pi N x)}{\sin(\pi x)}                                                                 &  & \by{4.6.12}
  \end{align*}
  when \(x\) is not an integer, and hence we have the formula
  \begin{align*}
    F_N(x) & = \dfrac{1}{N} \abs{\sum_{n = 0}^{N - 1} e_n(x)}^2                                                       &            & \by{ac:5.4.2}                 \\
           & = \dfrac{1}{N} \abs{\dfrac{e^{\pi i (N - 1) x} \sin(\pi N x)}{\sin(\pi x)}}^2                            &            & \text{(from the proof above)} \\
           & = \dfrac{\abs{e^{\pi i (N - 1) x}}^2 \abs{\sin(\pi N x)}^2}{N \abs{\sin(\pi x)}^2}                       &            & \by{ex:4.6.7}                 \\
           & = \dfrac{\abs{e^{\pi i (N - 1) x}}^2 \big(\sin(\pi N x)\big)^2}{N \big(\sin(\pi x)\big)^2}               & (x \in \R)                                 \\
           & = \dfrac{e^{\pi i (N - 1) x} e^{- \pi i (N - 1) x} \big(\sin(\pi N x)\big)^2}{N \big(\sin(\pi x)\big)^2} &            & \by{4.6.11}                   \\
           & = \dfrac{e^0 \big(\sin(\pi N x)\big)^2}{N \big(\sin(\pi x)\big)^2}                                       &            & \by{ex:4.6.16}                \\
           & = \dfrac{\big(\sin(\pi N x)\big)^2}{N \big(\sin(\pi x)\big)^2}.                                          &            & \text{(by \cref{4.5.2}(e))}
  \end{align*}
  When \(x\) is an integer, the geometric series formula does not apply, but one has \(F_N(x) = N\) in that case, as one can see by direct computation.
  In either case we see that \(F_N(x) \geq 0\) for any \(x\).
  Also, we have
  \begin{align*}
     & \int_{[0, 1]} F_N(x) \; dx                                                                                                                 \\
     & = \int_{[0, 1]} \sum_{n = -N}^N \bigg(1 - \dfrac{\abs{n}}{N}\bigg) e_n(x) \; dx                                                            \\
     & = \sum_{n = -N}^N \Bigg(\bigg(1 - \dfrac{\abs{n}}{N}\bigg) \int_{[0, 1]} e_n(x) \; dx\Bigg)                            &  & \by{5.2.2}     \\
     & = \sum_{n = -N}^N \Bigg(\bigg(1 - \dfrac{\abs{n}}{N}\bigg) \int_{[0, 1]} e_{n - 1}(x) e_1(x) \; dx\Bigg)               &  & \by{ex:4.6.16} \\
     & = \sum_{n = -N}^N \Bigg(\bigg(1 - \dfrac{\abs{n}}{N}\bigg) \int_{[0, 1]} e_{n - 1}(x) \overline{e_{-1}(x)} \; dx\Bigg) &  & \by{4.6.15}    \\
     & = \sum_{n = -N}^N \Bigg(\bigg(1 - \dfrac{\abs{n}}{N}\bigg) \inner*{e_{n - 1}, e_{-1}}\Bigg)                            &  & \by{5.2.1}     \\
     & = \bigg(1 - \dfrac{\abs{0}}{N}\bigg) 1                                                                                 &  & \by{5.3.5}     \\
     & = 1.
  \end{align*}
  Finally, since \(\sin(\pi N x) \leq 1\), we have
  \[
    F_N(x) \leq \dfrac{1}{N \big(\sin(\pi x)\big)^2} \leq \dfrac{1}{N \big(\sin(\pi \delta)\big)^2}
  \]
  whenever \(\delta < \abs{x} < 1 - \delta\)
  (this is because \(\sin\) is increasing on \([0, \pi / 2]\) and decreasing on \([\pi / 2, \pi]\)).
  Thus by choosing \(N\) large enough, we can make \(F_N (x) \leq \varepsilon\) for all \(\delta < \abs{x} < 1 - \delta\).
  Note that since
  \begin{align*}
    \big(\sin(\pi \abs{x})\big)^2 & = \begin{dcases}
                                        \big(\sin(\pi x)\big)^2  & \text{if } x \geq 0 \\
                                        \big(\sin(\pi -x)\big)^2 & \text{if } x < 0
                                      \end{dcases} \\
                                  & = \begin{dcases}
                                        \big(\sin(\pi x)\big)^2  & \text{if } x \geq 0 \\
                                        \big(-\sin(\pi x)\big)^2 & \text{if } x < 0
                                      \end{dcases} &  & \text{(by \cref{4.7.2}(c))} \\
                                  & = \big(\sin(\pi x)\big)^2
  \end{align*}
  and
  \begin{align*}
             & \begin{dcases}
                 \delta < \abs{x} \leq \dfrac{1}{2}  & \text{if } \abs{x} \leq \dfrac{1}{2} \\
                 \dfrac{1}{2} < \abs{x} < 1 - \delta & \text{if } \abs{x} > \dfrac{1}{2}    \\
               \end{dcases}                               \\
    \implies & \begin{dcases}
                 \pi \delta < \pi \abs{x} \leq \dfrac{\pi}{2}    & \text{if } \abs{x} \leq \dfrac{1}{2} \\
                 \dfrac{\pi}{2} < \pi \abs{x} < \pi - \pi \delta & \text{if } \abs{x} > \dfrac{1}{2}    \\
               \end{dcases}                   \\
    \implies & \begin{dcases}
                 \sin(\pi \delta) < \sin(\pi \abs{x}) \leq \sin(\dfrac{\pi}{2})    & \text{if } \abs{x} \leq \dfrac{1}{2} \\
                 \sin(\dfrac{\pi}{2}) > \sin(\pi \abs{x}) > \sin(\pi - \pi \delta) & \text{if } \abs{x} > \dfrac{1}{2}    \\
               \end{dcases} \\
    \implies & \begin{dcases}
                 \sin(\pi \delta) < \sin(\pi \abs{x}) \leq \sin(\dfrac{\pi}{2}) & \text{if } \abs{x} \leq \dfrac{1}{2} \\
                 \sin(\dfrac{\pi}{2}) > \sin(\pi \abs{x}) > -\sin(-\pi \delta)  & \text{if } \abs{x} > \dfrac{1}{2}    \\
               \end{dcases}    &  & \text{(by \cref{4.7.5}(a))}    \\
    \implies & \begin{dcases}
                 \sin(\pi \delta) < \sin(\pi \abs{x}) \leq \sin(\dfrac{\pi}{2}) & \text{if } \abs{x} \leq \dfrac{1}{2} \\
                 \sin(\dfrac{\pi}{2}) > \sin(\pi \abs{x}) > \sin(\pi \delta)    & \text{if } \abs{x} > \dfrac{1}{2}    \\
               \end{dcases}    &  & \text{(by \cref{4.7.2}(c))}    \\
    \implies & \sin(\pi \delta) < \sin(\pi \abs{x}),
  \end{align*}
  we have
  \begin{align*}
             & 0 < \big(\sin(\pi \delta)\big)^2 < \big(\sin(\pi \abs{x})\big)^2                  &  & \text{(by \cref{ac:4.7.2}(d))} \\
    \implies & 0 < \big(\sin(\pi \delta)\big)^2 < \big(\sin(\pi x)\big)^2                        &  & \text{(from the proof above)}  \\
    \implies & 0 < \dfrac{1}{\big(\sin(\pi x)\big)^2} < \dfrac{1}{\big(\sin(\pi \delta)\big)^2}.
  \end{align*}
\end{proof}

\exercisesection

\begin{ex}\label{ex:5.4.1}
  Show that if \(f : \R \to \C\) is both compactly supported and \(\Z\)-periodic, then it is identically zero.
\end{ex}

\begin{proof}
  Since \(f\) is compactly supported, by \cref{3.8.4} we know that
  \[
    \exists L, U \in \R : \forall x \in \R \setminus [L, U], f(x) = 0.
  \]
  Fix such \(L, U\).
  Let \([U - L]\) be the integer defined in \cref{ex:5.1.1}.
  Then we have
  \begin{align*}
             & \forall x \in [L, U], L \leq x                                                                 \\
    \implies & L + [U - L] \leq x + [U - L]                                                                   \\
    \implies & U = L + U - L < L + [U - L] + 1 \leq x + [U - L] + 1 &                         & \by{ex:5.1.1} \\
    \implies & f(x) = f(x + [U - L] + 1) = 0.                       & (f \in C(\R / \Z ; \C))
  \end{align*}
  Thus \(f = 0\).
\end{proof}

\begin{ex}\label{ex:5.4.2}
  Prove \cref{5.4.4}.
\end{ex}

\begin{proof}
  See \cref{5.4.4}.
\end{proof}

\begin{ex}\label{ex:5.4.3}
  Fill in the gaps marked in \cref{5.4.6}.
\end{ex}

\begin{proof}
  See \cref{5.4.6}.
\end{proof}
\section{The Fourier and Plancherel theorems}\label{ii:sec:5.5}

\begin{thm}[Fourier theorem]\label{ii:5.5.1}
  For any \(f \in C(\R / \Z ; \C)\), the series \(\sum_{n = -\infty}^\infty \hat{f}(n) e_n\) converges in \(L^2\) metric to \(f\).
  In other words, we have
  \[
    \lim_{N \to \infty} \norm*{f - \sum_{n = -N}^N \hat{f}(n) e_n}_2 = 0.
  \]
\end{thm}

\begin{proof}
  Let \(\varepsilon > 0\).
  We have to show that there exists an \(N_0\) such that
  \[
    \norm*{f - \sum_{n = -N}^N \hat{f}(n) e_n}_2 \leq \varepsilon
  \]
  for all \(N \geq N_0\).

  By the Weierstrass approximation theorem (\cref{ii:5.4.1}), we can find a trigonometric polynomial \(P = \sum_{n = -N_0}^{N_0} c_n e_n\) such that \(\norm*{f - P}_{\infty} \leq \varepsilon\), for some \(N_0 > 0\).
  In particular, we have \(\norm*{f - P}_2 \leq \varepsilon\) (\cref{ii:ex:5.2.3}).

  Now let \(N > N_0\), and let \(F_N \coloneqq \sum_{n = -N}^N \hat{f}(n) e_n\).
  We claim that \(\norm*{f - F_N}_2 \leq \varepsilon\).
  First, observe that for any \(\abs{m} \leq N\), we have
  \[
    \inner*{f - F_N, e_m} = \inner*{f, e_m} - \sum_{n = -N}^N \hat{f}(n) \inner*{e_n, e_m} = \hat{f}(m) - \hat{f}(m) = 0,
  \]
  where we have used \cref{ii:5.3.5} and \cref{ii:5.2.5}.
  In particular, we have
  \[
    \inner*{f - F_N, F_N - P} = 0
  \]
  since we can write \(F_N - P\) as a linear combination of the \(e_m\) for which \(\abs{m} \leq N\).
  By Pythagoras' theorem (\cref{ii:5.2.7}(d)) we therefore have
  \[
    \norm*{f - P}_2^2 = \norm*{f - F_N}_2^2 + \norm*{F_N - P}_2^2
  \]
  and in particular
  \[
    \norm*{f - F_N}_2 \leq \norm*{f - P}_2 \leq \varepsilon
  \]
  as desired.
\end{proof}

\begin{rmk}\label{ii:5.5.2}
  Note that we have only obtained convergence of the Fourier series \(\sum_{n = -\infty}^\infty \hat{f}(n) e_n\) to \(f\) in the \(L^2\) metric.
  One may ask whether one has convergence in the uniform or pointwise sense as well, but it turns out (perhaps somewhat surprisingly) that the answer is no to both of those questions.
  However, if one assumes that the function \(f\) is not only continuous, but is also differentiable, then one can recover pointwise convergence;
  if one assumes continuously differentiable, then one gets uniform convergence as well.
  These results are beyond the scope of this text and will not be proven here.
  However, we will prove one theorem about when one can improve the \(L^2\) convergence to uniform convergence.
\end{rmk}

\begin{thm}\label{ii:5.5.3}
  Let \(f \in C(\R / \Z ; \C)\), and suppose that the series \(\sum_{n = -\infty}^\infty \abs{\hat{f}(n)}\) is absolutely convergent.
  Then the series \(\sum_{n = -\infty}^\infty \hat{f}(n) e_n\) converges uniformly to \(f\).
  In other words, we have
  \[
    \lim_{N \to \infty} \norm*{f - \sum_{n = -N}^N \hat{f}(n) e_n}_{\infty} = 0.
  \]
\end{thm}

\begin{proof}
  By the Weierstrass \(M\)-test (\cref{ii:3.5.7}), we see that \(\sum_{n = -\infty}^\infty \hat{f}(n) e_n\) converges to some function \(F\), which by \cref{ii:5.1.5}(c) is also continuous and \(\Z\)-periodic.
  (Strictly speaking, the Weierstrass \(M\)-test was phrased for series from \(n = 1\) to \(n = +\infty\), but also works for series from \(n = -\infty\) to \(n = +\infty\);
  this can be seen by splitting the doubly infinite series into two pieces.)
  Thus
  \[
    \lim_{N \to \infty} \norm*{F - \sum_{n = -N}^N \hat{f}(n) e_n}_{\infty} = 0
  \]
  which implies that
  \[
    \lim_{N \to \infty} \norm*{F - \sum_{n = -N}^N \hat{f}(n) e_n}_2 = 0
  \]
  since the \(L^2\) norm is always less than or equal to the \(L^\infty\) norm (\cref{ii:ex:5.2.3}).
  But the sequence \(\sum_{n = -N}^N \hat{f}(n) e_n\) is already converging in \(L^2\) metric to \(f\) by the Fourier theorem (\cref{ii:5.5.1}), so can only converge in \(L^2\) metric to \(F\) if \(F = f\)
  (cf. \cref{ii:1.1.20}).
  Thus, \(F = f\), and so we have
  \[
    \lim_{N \to \infty} \norm*{f - \sum_{n = -N}^N \hat{f}(n) e_n}_{\infty} = 0
  \]
  as desired.
\end{proof}

\begin{thm}ncherel theorem]\label{ii:5.5.4}
  For any \(f \in C(\R / \Z ; \C)\), the series
  \[
    \sum_{n = -\infty}^\infty \abs{\hat{f}(n)}^2
  \]
  is absolutely convergent, and
  \[
    \norm*{f}_2^2 = \sum_{n = -\infty}^\infty \abs{\hat{f}(n)}^2.
  \]
\end{thm}

\begin{proof}
  Let \(\varepsilon > 0\).
  By the Fourier theorem (\cref{ii:5.5.1}) we know that
  \[
    \norm*{f - \sum_{n = -N}^N \hat{f}(n) e_n}_2 \leq \varepsilon
  \]
  if \(N\) is large enough (depending on \(\varepsilon\)).
  In particular, by the triangle inequality (\cref{ii:5.2.7}(c)(e)) this implies that
  \[
    \norm*{f}_2 - \varepsilon \leq \norm*{\sum_{n = -N}^N \hat{f}(n) e_n}_2 \leq \norm*{f}_2 + \varepsilon.
  \]
  On the other hand, by \cref{ii:5.3.6} we have
  \[
    \norm*{\sum_{n = -N}^N \hat{f}(n) e_n}_2 = \bigg(\sum_{n = -N}^N \abs{\hat{f}(n)}^2\bigg)^{1 / 2}
  \]
  and hence
  \[
    (\norm*{f}_2 - \varepsilon)^2 \leq \sum_{n = -N}^N \abs{\hat{f}(n)}^2 \leq (\norm*{f}_2 + \varepsilon)^2.
  \]
  Taking \(\limsup\), we obtain
  \[
    (\norm*{f}_2 - \varepsilon)^2 \leq \limsup_{N \to \infty} \sum_{n = -N}^N \abs{\hat{f}(n)}^2 \leq (\norm*{f}_2 + \varepsilon)^2.
  \]
  Since \(\varepsilon\) was arbitrary, we thus obtain by the squeeze test that
  \[
    \limsup_{N \to \infty} \sum_{n = -N}^N \abs{\hat{f}(n)}^2 = \norm*{f}_2^2
  \]
  and the claim follows.
\end{proof}

\begin{note}
  \cref{ii:5.5.4} is also known as \emph{Parseval's theorem}.
\end{note}

\exercisesection

\begin{ex}\label{ii:ex:5.5.1}
  Let \(f\) be a function in \(C(\R / \Z ; \C)\), and define the \emph{trigonometric Fourier coefficients} \(a_n, b_n\) for \(n = 0, 1, 2, 3, \dots\) by
  \[
    a_n = 2 \int_{[0, 1]} f(x) \cos(2 \pi n x) \; dx; \quad b_n = 2 \int_{[0, 1]} f(x) \sin(2 \pi n x) \; dx.
  \]
  \begin{enumerate}
    \item Show that the series
          \[
            \dfrac{1}{2} a_0 + \sum_{n = 1}^\infty \big(a_n \cos(2 \pi n x) + b_n \sin(2 \pi n x)\big)
          \]
          converges in \(L_2\) metric to \(f\).
    \item Show that if \(\sum_{n = 1}^\infty a_n\) and \(\sum_{n = 1}^\infty b_n\) are absolutely convergent, then the above series actually converges uniformly to \(f\), and not just in \(L_2\) metric.
  \end{enumerate}
\end{ex}

\begin{proof}{(a)}
  Observe that for all \(n \in \Z\), we have
  \begin{align*}
     & \hat{f}(n)                                                                                                                                \\
     & = \int_{[0, 1]} f(x) e^{- 2 \pi i n x} \; dx                                                                        &  & \by{ii:5.3.7}    \\
     & = \int_{[0, 1]} f(x) \big(\cos(2 \pi n x) - i \sin(2 \pi n x)\big) \; dx                                            &  & \by{ii:4.7.2}[f] \\
     & = \int_{[0, 1]} f(x) \cos(2 \pi n x) \; dx - i \int_{[0, 1]} f(x) \sin(2 \pi n x) \; dx                             &  & \by{ii:5.2.2}    \\
     & = \dfrac{1}{2} \bigg(2 \int_{[0, 1]} f(x) \cos(2 \pi n x) \; dx - 2i \int_{[0, 1]} f(x) \sin(2 \pi n x) \; dx\bigg)                       \\
     & = \dfrac{1}{2} (a_n - i b_n)
  \end{align*}
  and
  \begin{align*}
     & \hat{f}(-n)                                                                                                                                            \\
     & = \dfrac{1}{2} (a_{-n} - i b_{-n})                                                                                  &  & \text{(from the proof above)} \\
     & = \dfrac{1}{2} \bigg(2 \int_{[0, 1]} f(x) \cos(- 2 \pi n x) \; dx                                                                                      \\
     & \quad - 2i \int_{[0, 1]} f(x) \sin(- 2 \pi n x) \; dx\bigg)                                                                                            \\
     & = \dfrac{1}{2} \bigg(2 \int_{[0, 1]} f(x) \cos(2 \pi n x) \; dx                                                                                        \\
     & \quad - 2i \int_{[0, 1]} -f(x) \sin(2 \pi n x) \; dx\bigg)                                                          &  & \by{ii:4.7.2}[c]              \\
     & = \dfrac{1}{2} \bigg(2 \int_{[0, 1]} f(x) \cos(2 \pi n x) \; dx + 2i \int_{[0, 1]} f(x) \sin(2 \pi n x) \; dx\bigg) &  & \by{ii:5.2.2}                 \\
     & = \dfrac{1}{2} (a_n + i b_n).
  \end{align*}
  By Fourier theroem (\cref{ii:5.5.1}) we know that
  \[
    \lim_{N \to \infty} \norm*{f - \sum_{n = -N}^N \hat{f}(n) e_n}_2 = 0.
  \]
  Since for all \(N \in \Z^+\), we have
  \begin{align*}
     & \sum_{n = -N}^N \hat{f}(n) e_n                                                                                                                                   \\
     & = \hat{f}(0) e_0 + \sum_{n = 1}^N \hat{f}(n) e_n + \sum_{n = -N}^{-1} \hat{f}(n) e_n                                                                             \\
     & = \hat{f}(0) e_0 + \sum_{n = 1}^N \hat{f}(n) e_n + \sum_{n = 1}^N \hat{f}(-n) e_{-n}                                                                             \\
     & = \dfrac{(a_0 - i b_0) e_0}{2} + \sum_{n = 1}^N \dfrac{(a_n - i b_n) e_n}{2} + \sum_{n = 1}^N \dfrac{(a_n + i b_n) e_{-n}}{2} &  & \text{(from the proof above)} \\
     & = \dfrac{a_0 e_0}{2} + \sum_{n = 1}^N \dfrac{(a_n - i b_n) e_n}{2} + \sum_{n = 1}^N \dfrac{(a_n + i b_n) e_{-n}}{2}           &  & \by{ii:4.7.2}[e]              \\
     & = \dfrac{a_0}{2} + \sum_{n = 1}^N \dfrac{(a_n - i b_n) e_n}{2} + \sum_{n = 1}^N \dfrac{(a_n + i b_n) e_{-n}}{2}               &  & \by{ii:4.5.2}[e]              \\
     & = \dfrac{a_0}{2} + \sum_{n = 1}^N \dfrac{a_n (e_n + e_{-n}) - i b_n (e_n - e_{-n})}{2}                                        &  & \by{ii:4.6.6}                 \\
     & = \dfrac{a_0}{2} + \sum_{n = 1}^N a_n \cos(2 \pi n x) + b_n \sin(2 \pi n x),                                                  &  & \by{ii:4.7.1}
  \end{align*}
  we know that
  \[
    \lim_{N \to \infty} \norm*{f - \bigg(\dfrac{a_0}{2} + \sum_{n = 1}^N a_n \cos(2 \pi n x) + b_n \sin(2 \pi n x)\bigg)}_2 = 0.
  \]
  Thus, by \cref{ii:1.1.14} we have
  \begin{align*}
     & d_{L^2} - \lim_{N \to \infty} \bigg(\dfrac{a_0}{2} + \sum_{n = 1}^N a_n \cos(2 \pi n x) + b_n \sin(2 \pi n x)\bigg) \\
     & = \dfrac{a_0}{2} + \sum_{n = 1}^\infty a_n \cos(2 \pi n x) + b_n \sin(2 \pi n x)                                    \\
     & = f.
  \end{align*}
\end{proof}

\begin{proof}{(b)}
  Observe that
  \begin{align*}
     & \dfrac{a_0}{2} + \sum_{n = 1}^\infty \abs{a_n} + \sum_{n = 1}^\infty \abs{b_n}                                                               &  & \text{(by hypothesis)} \\
     & = \dfrac{a_0}{2} + \lim_{N \to \infty} \sum_{n = 1}^N \abs{a_n} + \lim_{N \to \infty} \sum_{n = 1}^N \abs{b_n}                               &  & \by{ii:ac:4.6.6}       \\
     & = \dfrac{a_0}{2} + \lim_{N \to \infty} \sum_{n = 1}^N \abs{a_n} + \abs{b_n}                                                                  &  & \by{ii:4.6.14}         \\
     & = \dfrac{a_0}{2} + 2 \bigg(\lim_{N \to \infty} \sum_{n = 1}^N \dfrac{\abs{a_n} + \abs{b_n}}{2}\bigg)                                         &  & \by{ii:4.6.14}         \\
     & = \dfrac{a_0}{2} + 2 \sum_{n = 1}^\infty \dfrac{\abs{a_n} + \abs{b_n}}{2}                                                                    &  & \by{ii:ac:4.6.6}       \\
     & = \dfrac{a_0}{2} + \sum_{n = 1}^\infty \dfrac{\abs{a_n} + \abs{b_n}}{2} + \sum_{n = 1}^\infty \dfrac{\abs{a_n} + \abs{b_n}}{2}                                           \\
     & = \dfrac{a_0}{2} + \sum_{n = 1}^\infty \dfrac{\abs{a_n} + \abs{i b_n}}{2} + \sum_{n = 1}^\infty \dfrac{\abs{a_n} + \abs{-i b_n}}{2}          &  & \by{ii:4.6.11}         \\
     & \geq \dfrac{a_0}{2} + \sum_{n = 1}^\infty \dfrac{\abs{a_n + i b_n}}{2} + \sum_{n = 1}^\infty \dfrac{\abs{a_n - i b_n}}{2}                    &  & \by{ii:4.6.11}         \\
     & = \dfrac{a_0}{2} + \sum_{n = 1}^\infty \dfrac{\abs{a_{-n} - i b_{-n}}}{2} + \sum_{n = 1}^\infty \dfrac{\abs{a_n - i b_n}}{2}                 &  & \by{ii:4.7.2}[c]       \\
     & = \dfrac{a_0}{2} + \sum_{n = 1}^\infty \dfrac{\abs{a_{-n} - i b_{-n}}}{2} + \sum_{n = 1}^\infty \dfrac{\abs{a_n - i b_n}}{2}                 &  & \by{ii:4.7.2}[e]       \\
     & = \lim_{N \to \infty} \dfrac{a_0}{2} + \sum_{n = 1}^N \dfrac{\abs{a_{-n} - i b_{-n}}}{2} + \sum_{n = 1}^N \dfrac{\abs{a_n - i b_n}}{2}       &  & \by{ii:4.6.14}         \\
     & = \lim_{N \to \infty} \dfrac{a_0}{2} + \sum_{n = -N}^{-1} \dfrac{\abs{a_n - i b_n}}{2} + \sum_{n = 1}^N \dfrac{\abs{a_n - i b_n}}{2}                                     \\
     & = \lim_{N \to \infty} \dfrac{a_0 - i b_0}{2} + \sum_{n = -N}^{-1} \dfrac{\abs{a_n - i b_n}}{2} + \sum_{n = 1}^N \dfrac{\abs{a_n - i b_n}}{2} &  & \by{ii:4.7.2}[e]       \\
     & = \lim_{N \to \infty} \sum_{n = -N}^N \dfrac{\abs{a_{-n} - i b_{-n}}}{2}                                                                                                 \\
     & = \sum_{n = -\infty}^\infty \dfrac{\abs{a_{-n} - i b_{-n}}}{2}.
  \end{align*}
  Since
  \[
    \hat{f}(n) = \dfrac{1}{2} (a_n - i b_n)
  \]
  for all \(n \in \Z\) (cf. the proof of \cref{ii:ex:5.5.1}(a)), we know that
  \begin{align*}
    \sum_{n = -\infty}^\infty \abs{\hat{f}(n)} & = \sum_{n = -\infty}^\infty \abs{\dfrac{a_n - i b_n}{2}}                                                                \\
                                               & = \sum_{n = -\infty}^\infty \dfrac{\abs{a_n - i b_n}}{2}                             &  & \by{ii:ex:4.6.7}              \\
                                               & \leq \dfrac{a_0}{2} + \sum_{n = 1}^\infty \abs{a_n} + \sum_{n = 1}^\infty \abs{b_n}. &  & \text{(from the proof above)}
  \end{align*}
  Thus, \(\sum_{n = -\infty}^\infty \abs{\hat{f}(n)}\) is absolutely convergent.
  By \cref{ii:5.5.3} we know that
  \[
    \lim_{N \to \infty} \norm*{f - \sum_{n = -N}^N \hat{f}(n) e_n}_{\infty} = 0.
  \]
  and \(\sum_{n = -\infty}^\infty \hat{f}(n) e_n\) converges uniformly to \(f\) on \(\R\) with respect to \(d_{l^1}|_{\C \times \C}\).
  In particular, we have (by \cref{ii:ex:5.2.3} and squeeze test)
  \[
    \lim_{N \to \infty} \norm*{f - \sum_{n = -N}^N \hat{f}(n) e_n}_2 = 0.
  \]
  By \cref{ii:ex:5.5.1}(a) we know that
  \[
    \lim_{N \to \infty} \norm*{f - \bigg(\dfrac{a_0}{2} + \sum_{n = 1}^N a_n \cos(2 \pi n x) + b_n \sin(2 \pi n x)\bigg)}_2 = 0.
  \]
  Thus, by \cref{ii:1.1.20} we have
  \[
    \lim_{N \to \infty} \norm*{f - \bigg(\dfrac{a_0}{2} + \sum_{n = 1}^N a_n \cos(2 \pi n x) + b_n \sin(2 \pi n x)\bigg)}_{\infty} = 0.
  \]
  and \(\dfrac{a_0}{2} + \sum_{n = 1}^\infty a_n \cos(2 \pi n x) + b_n \sin(2 \pi n x)\) converges uniformly to \(f\) on \(\R\) with respect to \(d_{l^1}|_{\C \times \C}\).
\end{proof}

\begin{ex}\label{ii:ex:5.5.2}
  Let \(f(x)\) be the function defined by \(f(x) = (1 - 2x)^2\) when \(x \in [0, 1)\), and extended to be \(\Z\)-periodic for the rest of the real line.
  \begin{enumerate}
    \item Using \cref{ii:ex:5.5.1}, show that the series
          \[
            \dfrac{1}{3} + \sum_{n = 1}^\infty \dfrac{4}{\pi^2 n^2} \cos(2 \pi n x)
          \]
          converges uniformly to \(f\) .
    \item Conclude that \(\sum_{n = 1}^\infty \dfrac{1}{n^2} = \dfrac{\pi^2}{6}\).
    \item Conclude that \(\sum_{n = 1}^\infty \dfrac{1}{n^4} = \dfrac{\pi^4}{90}\).
  \end{enumerate}
\end{ex}

\begin{proof}{(a)}
  For each \(n \in \N\), we define \(a_n, b_n\) as in \cref{ii:ex:5.5.1}.
  Observe that for all \(n \in \Z^+\), we have
  \begin{align*}
    a_n & = 2 \int_{[0, 1]} (1 - 2x)^2 \cos(2 \pi n x) \; dx                                           &  & \by{ii:ex:5.5.1}                \\
        & = \dfrac{2}{2 \pi n} \int_{[0, 1]} (1 - 2x)^2 \sin'(2 \pi n x) \; dx                         &  & \by{ii:4.7.2}[b]                \\
        & = \dfrac{1}{\pi n} \bigg(\big((1 - 2x)^2 \sin(2 \pi n x)\big)|_{x = 0}^{x = 1}               &  & \text{(by Proposition 11.10.1)} \\
        & \quad - \int_{[0, 1]} -4 (1 - 2x) \sin(2 \pi n x) \; dx\bigg)                                                                     \\
        & = \dfrac{4}{\pi n} \int_{[0, 1]} (1 - 2x) \sin(2 \pi n x) \; dx                              &  & \by{ii:4.7.2}[e]                \\
        & = \dfrac{-4}{2 \pi^2 n^2} \int_{[0, 1]} (1 - 2x) \cos'(2 \pi n x) \; dx                      &  & \by{ii:4.7.2}[b]                \\
        & = \dfrac{-2}{\pi^2 n^2} \bigg(\big((1 - 2x) \cos(2 \pi n x)\big)|_{x = 0}^{x = 1}                                                 \\
        & \quad - \int_{[0, 1]} -2 \cos(2 \pi n x) \; dx\bigg)                                         &  & \text{(by Proposition 11.10.1)} \\
        & = \dfrac{-2}{\pi^2 n^2} \bigg(-2 + 2 \int_{[0, 1]} \cos(2 \pi n x) \; dx\bigg)               &  & \by{ii:4.7.2}[e]                \\
        & = \dfrac{-2}{\pi^2 n^2} \bigg(-2 + \dfrac{2 \sin(2 \pi n x)}{2 \pi n}|_{x = 0}^{x = 1}\bigg) &  & \by{ii:4.7.2}[b]                \\
        & = \dfrac{4}{\pi^2 n^2}                                                                       &  & \by{ii:4.7.2}[e]
  \end{align*}
  and
  \begin{align*}
    b_n & = 2 \int_{[0, 1]} (1 - 2x)^2 \sin(2 \pi n x) \; dx                                &  & \by{ii:ex:5.5.1}                \\
        & = \dfrac{-2}{2 \pi n} \int_{[0, 1]} (1 - 2x)^2 \cos'(2 \pi n x) \; dx             &  & \by{ii:4.7.2}[b]                \\
        & = \dfrac{-1}{\pi n} \bigg(\big((1 - 2x)^2 \cos(2 \pi n x)\big)|_{x = 0}^{x = 1}   &  & \text{(by Proposition 11.10.1)} \\
        & \quad - \int_{[0, 1]} -4 (1 - 2x) \cos(2 \pi n x) \; dx\bigg)                                                          \\
        & = \dfrac{-4}{\pi n} \int_{[0, 1]} (1 - 2x) \cos(2 \pi n x) \; dx                  &  & \by{ii:4.7.2}[e]                \\
        & = \dfrac{-4}{2 \pi^2 n^2} \int_{[0, 1]} (1 - 2x) \sin'(2 \pi n x) \; dx           &  & \by{ii:4.7.2}[b]                \\
        & = \dfrac{-2}{\pi^2 n^2} \bigg(\big((1 - 2x) \sin(2 \pi n x)\big)|_{x = 0}^{x = 1}                                      \\
        & \quad - \int_{[0, 1]} -2 \sin(2 \pi n x) \; dx\bigg)                              &  & \text{(by Proposition 11.10.1)} \\
        & = \dfrac{-4}{\pi^2 n^2} \int_{[0, 1]} \sin(2 \pi n x) \; dx                       &  & \by{ii:4.7.2}[e]                \\
        & = \dfrac{-4}{\pi^2 n^2} \dfrac{-\cos(2 \pi n x)}{2 \pi n}|_{x = 0}^{x = 1}        &  & \by{ii:4.7.2}[b]                \\
        & = 0.                                                                              &  & \by{ii:4.7.2}[e]
  \end{align*}
  Since
  \begin{align*}
    \sum_{n = 1}^\infty \abs{a_n} & = \sum_{n = 1}^\infty \dfrac{4}{\pi^2 n^2} \\
    \sum_{n = 1}^\infty \abs{b_n} & = \sum_{n = 1}^\infty 0
  \end{align*}
  are absolutely convergent (by Corollary 7.3.7 in Analysis I), by \cref{ii:ex:5.5.1}(b) we know that the series
  \[
    \dfrac{a_0}{2} + \sum_{n = 1}^\infty (a_n \cos(2 \pi n x) + b_n \sin(2 \pi n x))
  \]
  converges uniformly to \(f\) on \(\R\) with respect to \(d_{l^1}|_{\C \times \C}\), and
  \begin{align*}
     & \dfrac{a_0}{2} + \sum_{n = 1}^\infty (a_n \cos(2 \pi n x) + b_n \sin(2 \pi n x))                                                                                                 \\
     & = \int_{[0, 1]} (1 - 2x)^2 \cos(0) \; dx + \sum_{n = 1}^\infty \dfrac{4}{\pi^2 n^2} \cos(2 \pi n x)                                           &  & \text{(from the proof above)} \\
     & = \int_{[0, 1]} (1 - 2x)^2 \; dx + \sum_{n = 1}^\infty \dfrac{4}{\pi^2 n^2} \cos(2 \pi n x)                                                   &  & \by{ii:4.7.2}[e]              \\
     & = 1 - \big(2x^2|_{x = 0}^{x = 1}\big) + \big(\dfrac{4x^3}{3}|_{x = 0}^{x = 1}\big) + \sum_{n = 1}^\infty \dfrac{4}{\pi^2 n^2} \cos(2 \pi n x)                                    \\
     & = 1 - 2 + \dfrac{4}{3} + \sum_{n = 1}^\infty \dfrac{4}{\pi^2 n^2} \cos(2 \pi n x)                                                                                                \\
     & = \dfrac{1}{3} + \sum_{n = 1}^\infty \dfrac{4}{\pi^2 n^2} \cos(2 \pi n x).
  \end{align*}
\end{proof}

\begin{proof}{(b)}
  We have
  \begin{align*}
             & (1 - 2 \cdot 0)^2 = \dfrac{1}{3} + \sum_{n = 1}^\infty \dfrac{4}{\pi^2 n^2} \cos(2 \pi n \cdot 0) &  & \by{ii:ex:5.5.2}[a] \\
    \implies & 1 = \dfrac{1}{3} + \sum_{n = 1}^\infty \dfrac{4}{\pi^2 n^2}                                       &  & \by{ii:4.7.2}[e]    \\
    \implies & \dfrac{2}{3} = \sum_{n = 1}^\infty \dfrac{4}{\pi^2 n^2}                                                                    \\
    \implies & \dfrac{\pi^2}{6} = \sum_{n = 1}^\infty \dfrac{1}{n^2}.
  \end{align*}
\end{proof}

\begin{proof}{(c)}
  By \cref{ii:ex:5.5.2}(a) we know that
  \[
    f(x) = (1 - 2x)^2 = \dfrac{1}{3} + \sum_{n = 1}^\infty \dfrac{4}{\pi^2 n^2} \cos(2 \pi n x)
  \]
  and the series on the right hand side converges uniformly to \(f\).
  Observe that for each \(m \in \Z\), we have
  \begin{align*}
     & \hat{f}(m)                                                                                                                                                                                       \\
     & = \int_{[0, 1]} \bigg(\dfrac{1}{3} + \sum_{n = 1}^\infty \dfrac{4}{\pi^2 n^2} \cos(2 \pi n x)\bigg) e^{-2 \pi i m x} \; dx                                                &  & \by{ii:5.3.7}     \\
     & = \int_{[0, 1]} \dfrac{e^{- 2 \pi i m x}}{3} \; dx + \int_{[0, 1]} \sum_{n = 1}^\infty \dfrac{4 e^{- 2 \pi i m x}}{\pi^2 n^2} \cos(2 \pi n x) \; dx                       &  & \by{ii:5.2.2}     \\
     & = \int_{[0, 1]} \dfrac{e^{- 2 \pi i m x}}{3} \; dx + \sum_{n = 1}^\infty \int_{[0, 1]} \dfrac{4 e^{- 2 \pi i m x}}{\pi^2 n^2} \cos(2 \pi n x) \; dx                       &  & \by{ii:3.6.2}     \\
     & = \int_{[0, 1]} \dfrac{e^{- 2 \pi i m x}}{3} \; dx + \sum_{n = 1}^\infty \int_{[0, 1]} \dfrac{2 e^{- 2 \pi i m x} (e^{2 \pi i n x} + e^{- 2 \pi i n x})}{\pi^2 n^2} \; dx &  & \by{ii:4.7.1}     \\
     & = \int_{[0, 1]} \dfrac{e^{- 2 \pi i m x}}{3} \; dx + \sum_{n = 1}^\infty \int_{[0, 1]} \dfrac{2 (e^{2 \pi i (n - m) x} + e^{- 2 \pi i (n + m) x})}{\pi^2 n^2} \; dx.      &  & \by{ii:ex:4.6.16}
  \end{align*}
  Now we split into two cases:
  \begin{itemize}
    \item If \(m = 0\), then we have
          \begin{align*}
            \hat{f}(0) & = \int_{[0, 1]} \dfrac{1}{3} \; dx + \sum_{n = 1}^\infty \int_{[0, 1]} \dfrac{2 (e^{2 \pi i n x} + e^{- 2 \pi i n x})}{\pi^2 n^2} \; dx &  & \by{ii:4.5.2}[e] \\
                       & = \dfrac{1}{3} + \sum_{n = 1}^\infty \bigg(\dfrac{2}{\pi^2 n^2} \int_{[0, 1]} e^{2 \pi i n x} + e^{- 2 \pi i n x} \; dx\bigg)                                 \\
                       & = \dfrac{1}{3} + \sum_{n = 1}^\infty \dfrac{2}{\pi^2 n^2} (0 + 0)                                                                       &  & \by{ii:5.3.5}    \\
                       & = \dfrac{1}{3}.
          \end{align*}
    \item If \(m \neq 0\), then we have
          \begin{align*}
             & \hat{f}(m)                                                                                                                                                \\
             & = 0 + \sum_{n = 1}^\infty \int_{[0, 1]} \dfrac{2 (e^{2 \pi i (n - m) x} + e^{- 2 \pi i (n + m) x})}{\pi^2 n^2} \; dx                   &  & \by{ii:5.3.5} \\
             & = \sum_{n = 1}^\infty \Bigg(\dfrac{2}{\pi^2 n^2} \bigg(\int_{[0, 1]} e^{2 \pi i (n - m) x} + e^{- 2 \pi i (n + m) x} \; dx\bigg)\Bigg)                    \\
             & = \sum_{n = 1}^\infty \bigg(\dfrac{2}{\pi^2 n^2} \big(\inner*{e_n, e_m} + \inner*{e_{-n}, e_m}\big)\bigg)                              &  & \by{ii:5.2.1} \\
             & = \dfrac{2}{\pi^2 m^2}.                                                                                                                &  & \by{ii:5.3.5}
          \end{align*}
  \end{itemize}
  From all cases above, we have
  \begin{align*}
             & \norm*{f}_2^2 = \sum_{n = -\infty}^\infty \abs{\hat{f}(n)}^2                                                                         &  & \by{ii:5.5.4}                 \\
    \implies & \int_{[0, 1]} (1 - 2x)^4 \; dx = \sum_{n = -\infty}^\infty \abs{\hat{f}(n)}^2                                                                                           \\
    \implies & \int_{[0, 1]} 1 - 8x + 24x^2 - 32x^3 + 16x^4 \; dx = \sum_{n = -\infty}^\infty \abs{\hat{f}(n)}^2                                                                       \\
    \implies & 1 - 4 (x^2|_{x = 0}^{x = 1}) + 8 (x^3|_{x = 0}^{x = 1}) - 8(x^4|_{x = 0}^{x = 1}) + \dfrac{16}{5}(x^5|_{x = 0}^{x = 1})                                                 \\
             & = \sum_{n = -\infty}^\infty \abs{\hat{f}(n)}^2                                                                                                                          \\
    \implies & \dfrac{1}{5} = \dfrac{1}{9} + \sum_{n = 1}^\infty \abs{\dfrac{2}{\pi^2 n^2}}^2 + \sum_{n = 1}^\infty \abs{\dfrac{2}{\pi^2 (-n)^2}}^2 &  & \text{(from the proof above)} \\
    \implies & \dfrac{4}{45} = 2 \sum_{n = 1}^\infty \abs{\dfrac{2}{\pi^2 n^2}}^2                                                                                                      \\
    \implies & \dfrac{4}{45} = 8 \sum_{n = 1}^\infty \dfrac{1}{\pi^4 n^4}                                                                                                              \\
    \implies & \dfrac{\pi^4}{90} = \sum_{n = 1}^\infty \dfrac{1}{n^4}.
  \end{align*}
\end{proof}

\begin{ex}\label{ii:ex:5.5.3}
  If \(f \in C(\R / \Z ; \C)\) and \(P\) is a trigonometric polynomial, show that
  \[
    \widehat{f * P}(n) = \hat{f}(n) c_n = \hat{f}(n) \hat{P}(n)
  \]
  for all integers \(n\).
  More generally, if \(f, g \in C(\R / \Z ; \C)\), show that
  \[
    \widehat{f * g}(n) = \hat{f}(n) \hat{g}(n)
  \]
  for all integers \(n\).
  (A fancy way of saying this is that the Fourier transform \emph{intertwines} convolution and multiplication.)
\end{ex}

\begin{proof}
  Let \(P = \sum_{n = -N}^N c_n e_n\) for some \(N \in \Z^+\) and some \((c_n)_{n = -N}^N\) in \(\C\).
  By \cref{ii:ac:5.4.1} we know that
  \[
    f * P = \sum_{n = -N}^N \hat{f}(n) c_n e_n.
  \]
  Thus, we have
  \begin{align*}
    \forall m \in \Z, \widehat{f * P}(m) & = \inner{f * P, e_m}                               &  & \by{ii:5.3.7}    \\
                                         & = \sum_{n = -N}^N \hat{f}(n) c_n \inner{e_n, e_m}  &  & \by{ii:5.2.5}[c] \\
                                         & = \begin{dcases}
                                               \hat{f}(m) c_m & \text{if } n = m    \\
                                               0              & \text{if } n \neq m
                                             \end{dcases}            &  & \by{ii:5.3.5}                             \\
                                         & = \hat{f}(m) \sum_{n = -N}^N c_n \inner*{e_n, e_m} &  & \by{ii:5.3.5}    \\
                                         & = \hat{f}(m) \inner*{\sum_{n = -N}^N c_n e_n, e_m} &  & \by{ii:5.2.5}[c] \\
                                         & = \hat{f}(m) \hat{P}(m).                           &  & \by{ii:5.3.7}
  \end{align*}

  Now we show that \(\widehat{f * g}(n) = \hat{f}(n) \hat{g}(n)\) for all \(n \in \Z\).
  In particular, we want to show that
  \[
    \forall \varepsilon \in \R^+, \abs{\widehat{f * g}(n) - \hat{f}(n) \hat{g}(n)} \leq \varepsilon.
  \]
  Let \(\varepsilon \in \R^+\).
  Since \(g \in C(\R / \Z ; \C)\), by \cref{ii:5.4.1} we know that there exists a trigonometric polynomial \(P\) such that
  \[
    \norm*{g - P}_{\infty} \leq \varepsilon.
  \]
  Fix such \(P\).
  Since \(f \in C(\R / \Z ; \C)\), by \cref{ii:5.5.4} we know that \(\sum_{n = -\infty}^\infty \abs{\hat{f}(n)}^2 \in \R\), thus
  \[
    \exists M \in \R^+ : \forall n \in \Z, \abs{\hat{f}(n)} \leq M.
  \]
  Fix such \(M\).
  Then for all \(n \in \Z\), we have
  \begin{align*}
    \abs{\widehat{f * g}(n) - \widehat{f * P}(n)} & = \abs{\inner*{f * g, e_n} - \inner*{f * P, e_n}} &  & \by{ii:5.3.7}    \\
                                                  & = \abs{\inner*{f * g - f * P, e_n}}               &  & \by{ii:5.2.5}[c] \\
                                                  & = \abs{\inner*{f * (g - P), e_n}}                 &  & \by{ii:5.4.4}[c] \\
                                                  & \leq \norm*{f * (g - P)}_2 \norm*{e_n}_2          &  & \by{ii:5.2.7}[b] \\
                                                  & = \norm*{f * (g - P)}_2.                          &  & \by{ii:5.3.5}
  \end{align*}
  Need some helps.
\end{proof}

\begin{ex}\label{ii:ex:5.5.4}
  Let \(f \in C(\R / \Z ; \C)\) be a function which is differentiable, and whose derivative \(f'\) is also continuous.
  Show that \(f'\) also lies in \(C(\R / \Z ; \C)\), and that \(\hat{f}'(n) = 2 \pi i n \hat{f}(n)\) for all integers \(n\).
  Here the derivative of a complex-valued function is defined in exactly the same fashion as for real-valued functions.
\end{ex}

\begin{ex}\label{ii:ex:5.5.5}
  Let \(f, g \in C(\R / \Z ; \C)\).
  Prove the \emph{Parseval identity}
  \[
    \Re\bigg(\int_0^1 f(x) \overline{g(x)} \; dx\bigg) = \Re\bigg(\sum_{n \in \Z} \hat{f}(n) \overline{\hat(g)(n)}\bigg).
  \]
  Then conclude that the real parts can be removed, thus
  \[
    \int_0^1 f(x) \overline{g(x)} \; dx = \sum_{n \in \Z} \hat{f}(n) \overline{\hat(g)(n)}.
  \]
\end{ex}

\begin{ex}\label{ii:ex:5.5.6}
  In this exercise we shall develop the theory of Fourier series for functions of any fixed period \(L\).

  Let \(L > 0\), and let \(f : \R \to \C\) be a complex-valued function which is continuous and \(L\)-periodic.
  Define the numbers \(c_n\) for every integer \(n\) by
  \[
    c_n \coloneqq \dfrac{1}{L} \int_{[0, L]} f(x) e^{- 2 \pi i n x / L} \; dx.
  \]
  \begin{enumerate}
    \item Show that the series
          \[
            \sum_{n = -\infty}^\infty c_n e^{2 \pi i n x / L}
          \]
          converges in \(L_2\) metric to \(f\).
          More precisely, show that
          \[
            \lim_{N \to \infty} \int_{[0, L]} \abs{f(x) - \sum_{n = -N}^N c_n e^{2 \pi i n x / L}}^2 \; dx = 0.
          \]
    \item If the series \(\sum_{n = -\infty}^\infty \abs{c_n}\) is absolutely convergent, show that
          \[
            \sum_{n = -\infty}^\infty c_n e^{2 \pi i n x / L}
          \]
          converges uniformly to \(f\).
    \item Show that
          \[
            \dfrac{1}{L} \int_{[0, L]} \abs{f(x)}^2 \; dx = \sum_{n = -\infty}^\infty \abs{c_n}^2.
          \]
  \end{enumerate}
\end{ex}


\chapter{Several variable differential calculus}\label{ii:ch:6}

\section{Linear transformations}\label{ii:sec:6.1}

\begin{defn}[Row vector]\label{ii:6.1.1}
  Let \(n \geq 1\) be an integer.
  We refer to elements of \(\R^n\) as \emph{\(n\)-dimensional row vectors}.
  A typical \(n\)-dimensional row vector may take the form \(x = (x_1, x_2, \dots, x_n)\), which we abbreviate as \((x_i)_{1 \leq i \leq n}\);
  the quantities \(x_1, x_2, \dots, x_n\) are of course real numbers.
  If \((x_i)_{1 \leq i \leq n}\) and \((y_i)_{1 \leq i \leq n}\) are \(n\)-dimensional row vectors, we can define their vector sum by
  \[
    (x_i)_{1 \leq i \leq n} + (y_i)_{1 \leq i \leq n} = (x_i + y_i)_{1 \leq i \leq n},
  \]
  and also if \(c \in \R\) is any scalar, we can define the scalar product \(c (x_i)_{1 \leq i \leq n}\) by
  \[
    c (x_i)_{1 \leq i \leq n} \coloneqq (cx_i)_{1 \leq i \leq n}.
  \]
  Of course one has similar operations on \(\R^m\) as well.
  However, if \(n \neq m\), then we do not define any operation of vector addition between vectors in \(\R^n\) and vectors in \(\R^m\).
  We also refer to the vector \((0, \dots, 0)\) in \(\R^n\) as the \emph{zero vector} and also denote it by \(0\).
  (Strictly speaking, we should denote the zero vector of \(\R^n\) by \(0_{\R^n}\), as they are technically distinct from each other and from the number zero, but we shall not take care to make this distinction.)
  We abbreviate \((-1) x\) as \(-x\).
\end{defn}

\begin{lem}[\(\R^n\) is a vector space]\label{ii:6.1.2}
  Let \(x, y, z\) be vectors in \(\R^n\), and let \(c, d\) be real numbers.
  Then we have the commutativity property \(x + y = y + x\), the additive associativity property \((x + y) + z = x + (y + z)\), the additive identity property \(x + 0 = 0 + x = x\), the additive inverse property \(x + (-x) = (-x) + x = 0\), the multiplicative associativity property \((cd)x = c(dx)\), the distributivity properties \(c(x + y) = cx + cy\) and \((c + d)x = cx + dx\), and the multiplicative identity property \(1x = x\).
\end{lem}

\begin{proof}
  First we show the commutative property.
  \begin{align*}
    x + y & = (x_i)_{1 \leq i \leq n} + (y_i)_{1 \leq i \leq n} &                   & \by{ii:6.1.1} \\
          & = (x_i + y_i)_{1 \leq i \leq n}                     &                   & \by{ii:6.1.1} \\
          & = (y_i + x_i)_{1 \leq i \leq n}                     & (x_i, y_i \in \R)                 \\
          & = (y_i)_{1 \leq i \leq n} + (x_i)_{1 \leq i \leq n} &                   & \by{ii:6.1.1} \\
          & = y + x.                                            &                   & \by{ii:6.1.1}
  \end{align*}

  Next we show the additive associativity property.
  \begin{align*}
    (x + y) + z & = \big((x_i)_{1 \leq i \leq n} + (y_i)_{1 \leq i \leq n}\big) + (z_i)_{1 \leq i \leq n} &                        & \by{ii:6.1.1} \\
                & = (x_i + y_i)_{1 \leq i \leq n} + (z_i)_{1 \leq i \leq n}                               &                        & \by{ii:6.1.1} \\
                & = \big((x_i + y_i) + z_i\big)_{1 \leq i \leq n}                                         &                        & \by{ii:6.1.1} \\
                & = \big(x_i + (y_i + z_i)\big)_{1 \leq i \leq n}                                         & (x_i, y_i, z_i \in \R)                 \\
                & = (x_i)_{1 \leq i \leq n} + (y_i + z_i)_{1 \leq i \leq n}                               &                        & \by{ii:6.1.1} \\
                & = (x_i)_{1 \leq i \leq n} + \big((y_i)_{1 \leq i \leq n} + (z_i)_{1 \leq i \leq n}\big) &                        & \by{ii:6.1.1} \\
                & = x + (y + z).                                                                          &                        & \by{ii:6.1.1}
  \end{align*}

  Next we show that \(0_{\R^n}\) is the additive identity.
  \begin{align*}
    x + 0_{\R^n} & = 0_{\R^n} + x                                    &              & \text{(from the proof above)} \\
                 & = (0)_{1 \leq i \leq n} + (x_i)_{1 \leq i \leq n} &              & \by{ii:6.1.1}                 \\
                 & = (0 + x_i)_{1 \leq i \leq n}                     &              & \by{ii:6.1.1}                 \\
                 & = (x_i)_{1 \leq i \leq n}                         & (x_i \in \R)                                 \\
                 & = x.                                              &              & \by{ii:6.1.1}
  \end{align*}

  Next we show that every \(-x\) is the additive inverse of \(x\).
  \begin{align*}
    x + (-x) & = (-x) + x                                                      &              & \text{(from the proof above)} \\
             & = (-1)(x) + x                                                   &              & \by{ii:6.1.1}                 \\
             & = (-1)(x_i)_{1 \leq i \leq n} + (x_i)_{1 \leq i \leq n}         &              & \by{ii:6.1.1}                 \\
             & = \big((-1)x_i\big)_{1 \leq i \leq n} + (x_i)_{1 \leq i \leq n} &              & \by{ii:6.1.1}                 \\
             & = \big((-1)x_i + x_i\big)_{1 \leq i \leq n}                     &              & \by{ii:6.1.1}                 \\
             & = (0)_{1 \leq i \leq n}                                         & (x_i \in \R)                                 \\
             & = 0_{\R^n}.                                                     &              & \by{ii:6.1.1}
  \end{align*}

  Next we show that multiplicative associativity property.
  \begin{align*}
    (cd)x & = (cd)(x_i)_{1 \leq i \leq n}          &                    & \by{ii:6.1.1} \\
          & = \big((cd) x_i\big)_{1 \leq i \leq n} &                    & \by{ii:6.1.1} \\
          & = \big(c(d x_i)\big)_{1 \leq i \leq n} & (c, d, x_i \in \R)                 \\
          & = c(d x_i)_{1 \leq i \leq n}           &                    & \by{ii:6.1.1} \\
          & = c(dx).                               &                    & \by{ii:6.1.1}
  \end{align*}

  Next we show that distributivity property.
  \begin{align*}
    c(x + y) & = c\big((x_i)_{1 \leq i \leq n} + (y_i)_{1 \leq i \leq n}\big) &                      & \by{ii:6.1.1} \\
             & = c(x_i + y_i)_{1 \leq i \leq n}                               &                      & \by{ii:6.1.1} \\
             & = \big(c(x_i + y_i)\big)_{1 \leq i \leq n}                     &                      & \by{ii:6.1.1} \\
             & = (c x_i + c y_i)_{1 \leq i \leq n}                            & (c, x_i, y_i \in \R)                 \\
             & = (c x_i)_{1 \leq i \leq n} + (c y_i)_{1 \leq i \leq n}        &                      & \by{ii:6.1.1} \\
             & = c(x_i)_{1 \leq i \leq n} + c(y_i)_{1 \leq i \leq n}          &                      & \by{ii:6.1.1} \\
             & = cx + cy.                                                     &                      & \by{ii:6.1.1}
  \end{align*}
  \begin{align*}
    (c + d)x & = (c + d)(x_i)_{1 \leq i \leq n}                        &                    & \by{ii:6.1.1} \\
             & = \big((c + d) x_i\big)_{1 \leq i \leq n}               &                    & \by{ii:6.1.1} \\
             & = (c x_i + d x_i)_{1 \leq i \leq n}                     & (c, d, x_i \in \R)                 \\
             & = (c x_i)_{1 \leq i \leq n} + (d x_i)_{1 \leq i \leq n} &                    & \by{ii:6.1.1} \\
             & = c(x_i)_{1 \leq i \leq n} + d(x_i)_{1 \leq i \leq n}   &                    & \by{ii:6.1.1} \\
             & = cx + dx.                                              &                    & \by{ii:6.1.1}
  \end{align*}

  Finally we show that \(1\) is the multiplicative identity.
  \begin{align*}
    1 x & = 1 (x_i)_{1 \leq i \leq n} &              & \by{ii:6.1.1} \\
        & = (1 x_i)_{1 \leq i \leq n} &              & \by{ii:6.1.1} \\
        & = (x_i)_{1 \leq i \leq n}   & (x_i \in \R)                 \\
        & = x.                        &              & \by{ii:6.1.1}
  \end{align*}
\end{proof}

\begin{defn}[Transpose]\label{ii:6.1.3}
  If \((x_i)_{1 \leq i \leq n} = (x_1, x_2, \dots, x_n)\) is an \(n\)-dimensional row vector, we can define its \emph{transpose} \((x_i)_{1 \leq i \leq n}^\top\) by
  \[
    (x_i)_{1 \leq i \leq n}^\top = (x_1, x_2, \dots, x_n)^\top \coloneqq \begin{pmatrix}
      x_1    \\
      x_2    \\
      \vdots \\
      x_n
    \end{pmatrix}.
  \]
  We refer to objects such as \((x_i)_{1 \leq i \leq n}^\top\) as \emph{\(n\)-dimensional column vectors}.
\end{defn}

\begin{rmk}\label{ii:6.1.4}
  There is no functional difference between a row vector and a column vector (e.g., one can add and scalar multiply column vectors just as well as we can row vectors), however we shall (rather annoyingly) need to transpose our row vectors into column vectors in order to be consistent with the conventions of matrix multiplication, which we will see later.
  Note that we view row vectors and column vectors as residing in different spaces;
  thus for instance we will not define the sum of a row vector with a column vector, even when they have the same number of elements.
\end{rmk}

\begin{defn}[Standard basis row vectors]\label{ii:6.1.5}
  We identify \(n\) special vectors in \(\R^n\), the \emph{standard basis row vectors} \(e_1, \dots, e_n\).
  For each \(1 \leq j \leq n\), \(e_j\) is the vector which has \(0\) in all entries except for the \(j^{\opTh}\) entry, which is equal to \(1\).
\end{defn}

\begin{note}
  If \(x = (x_i)_{1 \leq i \leq n}\) is a vector in \(\R^n\), then
  \[
    x = x_1 e_1 + x_2 e_2 + \dots + x_n e_n = \sum_{j = 1}^n x_j e_j,
  \]
  or in other words every vector in \(\R^n\) is a \emph{linear combination} of the standard basis vectors \(e_1, \dots, e_n\).
  (The notation \(\sum_{j = 1}^n x_j e_j\) is unambiguous because the operation of vector addition is both commutative and associative).
  Of course, just as every row vector is a linear combination of standard basis row vectors, every column vector is a linear combination of standard basis column vectors:
  \[
    x^\top = x_1 e_1^\top + x_2 e_2^\top + \dots + x_n e_n^\top = \sum_{j = 1}^n x_j e_j^\top.
  \]
\end{note}

\begin{defn}[Linear transformations]\label{ii:6.1.6}
  A \emph{linear transformation} \(T : \R^n \to \R^m\) is any function from one Euclidean space \(\R^n\) to another \(\R^m\) which obeys the following two axioms:
  \begin{enumerate}
    \item (Additivity)
          For every \(x, x' \in \R^n\), we have \(T(x + x') = T(x) + T(x')\).
    \item (Homogeneity)
          For every \(x \in \R^n\) and every \(c \in \R\), we have \(T(cx) = cT(x)\).
  \end{enumerate}
\end{defn}

\setcounter{thm}{9}
\begin{defn}[Matrices]\label{ii:6.1.10}
  An \(m \times n\) matrix is an object \(A\) of the form
  \[
    A = \begin{pmatrix}
      a_{11} & a_{12} & \dots  & a_{1n} \\
      a_{21} & a_{22} & \dots  & a_{2n} \\
      \vdots & \vdots & \ddots & \vdots \\
      a_{m1} & a_{m2} & \dots  & a_{mn}
    \end{pmatrix};
  \]
  we shall abbreviate this as
  \[
    A = (a_{ij})_{1 \leq i \leq m ; 1 \leq j \leq n}.
  \]
  In particular, \(n\)-dimensional row vectors are \(1 \times n\) matrices, while \(n\)-dimensional column vectors are \(n \times 1\) matrices.
\end{defn}

\begin{defn}[Matrix product]\label{ii:6.1.11}
  Given an \(m \times n\) matrix \(A\) and an \(n \times p\) matrix \(B\), we can define the matrix product \(AB\) to be the \(m \times p\) matrix defined as
  \[
    (a_{ij})_{1 \leq i \leq m ; 1 \leq j \leq n} (b_{jk})_{1 \leq j \leq n ; 1 \leq k \leq p} \coloneqq \bigg(\sum_{j = 1}^n a_{ij} b_{jk}\bigg)_{1 \leq i \leq m ; 1 \leq k \leq p}.
  \]
  In particular, if \(x^\top = (x_i)_{1 \leq i \leq n}^\top\) is an \(n\)-dimensional column vector, and
  \[
    A = (a_{ij})_{1 \leq i \leq m ; 1 \leq j \leq n}
  \]
  is an \(m \times n\) matrix, then \(A x^\top\) is an \(m\)-dimensional column vector:
  \[
    A x^\top = \bigg(\sum_{j = 1}^n a_{ij} x_j\bigg)_{1 \leq i \leq m}^\top.
  \]
\end{defn}

\begin{ac}\label{ii:ac:6.1.1}
  We now relate matrices to linear transformations.
  If \(A\) is an \(m \times n\) matrix, we can define the transformation \(L_A : \R^n \to \R^m\) by the formula
  \[
    \big(L_A(x)\big)^\top = Ax^\top.
  \]
  More generally, if
  \[
    A = \begin{pmatrix}
      a_{11} & a_{12} & \dots  & a_{1n} \\
      a_{21} & a_{22} & \dots  & a_{2n} \\
      \vdots & \vdots & \ddots & \vdots \\
      a_{m1} & a_{m2} & \dots  & a_{mn}
    \end{pmatrix}
  \]
  then we have
  \[
    L_A\big((x_j)_{1 \leq j \leq n}\big) = \bigg(\sum_{j = 1}^n a_{ij} x_j\bigg)_{1 \leq i \leq m}.
  \]
  For any \(m \times n\) matrix \(A\), the transformation \(L_A\) is automatically linear;
  one can easily verify that \(L_A(x + y) = L_A(x) + L_A(y)\) and \(L_A(cx) = c L_A(x)\) for any \(n\)-dimensional row vectors \(x, y\) and any scalar \(c\).
\end{ac}

\begin{proof}
  We have
  \begin{align*}
    L_A(x + y) & = L_A\big((x_j)_{1 \leq j \leq n} + (y_j)_{1 \leq j \leq n}\big)                                                     &  & \by{ii:6.1.1} \\
               & = L_A\big((x_j + y_j)_{1 \leq j \leq n}\big)                                                                         &  & \by{ii:6.1.1} \\
               & = \bigg(\sum_{j = 1}^n a_{ij} (x_j + y_j)\bigg)_{1 \leq i \leq m}                                                                       \\
               & = \bigg(\sum_{j = 1}^n a_{ij} x_j + \sum_{j = 1}^n a_{ij} y_j\bigg)_{1 \leq i \leq m}                                                   \\
               & = \bigg(\sum_{j = 1}^n a_{ij} x_j\bigg)_{1 \leq i \leq m} + \bigg(\sum_{j = 1}^n a_{ij} y_j)\bigg)_{1 \leq i \leq m} &  & \by{ii:6.1.1} \\
               & = L_A\big((x_j)_{1 \leq j \leq n}\big) + L_A\big((y_j)_{1 \leq j \leq n}\big)                                                           \\
               & = L_A(x) + L_A(y)                                                                                                    &  & \by{ii:6.1.1}
  \end{align*}
  and
  \begin{align*}
    L_A(cx) & = L_A\big(c(x_j)_{1 \leq j \leq n}\big)                       &  & \by{ii:6.1.1} \\
            & = L_A\big((cx_j)_{1 \leq j \leq n}\big)                       &  & \by{ii:6.1.1} \\
            & = \bigg(\sum_{j = 1}^n a_{ij} (c x_j)\bigg)_{1 \leq i \leq m}                    \\
            & = \bigg(c \sum_{j = 1}^n a_{ij} x_j\bigg)_{1 \leq i \leq m}                      \\
            & = c \bigg(\sum_{j = 1}^n a_{ij} x_j\bigg)_{1 \leq i \leq m}   &  & \by{ii:6.1.1} \\
            & = c L_A\big((x_j)_{1 \leq j \leq n}\big)                                         \\
            & = c L_A(x).                                                   &  & \by{ii:6.1.1}
  \end{align*}
  Thus, by \cref{ii:6.1.6} \(L_A\) is a linear transformation from \(\R^n\) to \(\R^m\).
\end{proof}

\setcounter{thm}{12}
\begin{lem}\label{ii:6.1.13}
  Let \(T : \R^n \to \R^m\) be a linear transformation.
  Then there exists exactly one \(m \times n\) matrix \(A\) such that \(T = L_A\).
\end{lem}

\begin{proof}
  Suppose \(T : \R^n \to \R^m\) is a linear transformation.
  Let \(e_1, e_2, \dots, e_n\) be the standard basis row vectors of \(\R^n\).
  Then \(T(e_1), T(e_2), \dots, T(e_n)\) are vectors in \(\R^m\).
  For each \(1 \leq j \leq n\), we write \(T(e_j)\) in co-ordinates as
  \[
    T(e_j) = (a_{1j}, a_{2j}, \dots, a_{mj}) = (a_{ij})_{1 \leq i \leq m},
  \]
  i.e., we define \(a_{ij}\) to be the \(i^{\opTh}\) component of \(T(e_j)\).
  Then for any \(n\)- dimensional row vector \(x = (x_1, \dots, x_n)\), we have
  \[
    T(x) = T\bigg(\sum_{j = 1}^n x_j e_j\bigg),
  \]
  which (since \(T\) is linear) is equal to
  \begin{align*}
     & = \sum_{j = 1}^n T(x_j e_j)                                \\
     & = \sum_{j = 1}^n x_j T(e_j)                                \\
     & = \sum_{j = 1}^n x_j (a_{ij})_{1 \leq i \leq m}            \\
     & = \sum_{j = 1}^n (a_{ij} x_j)_{1 \leq i \leq m}            \\
     & = \bigg(\sum_{j = 1}^n a_{ij} x_j\bigg)_{1 \leq i \leq m}.
  \end{align*}
  But if we let \(A\) be the matrix
  \[
    A = \begin{pmatrix}
      a_{11} & a_{12} & \dots  & a_{1n} \\
      a_{21} & a_{22} & \dots  & a_{2n} \\
      \vdots & \vdots & \ddots & \vdots \\
      a_{m1} & a_{m2} & \dots  & a_{mn}
    \end{pmatrix}
  \]
  then the previous vector is precisely \(L_A(x)\).
  Thus, \(T(x) = L_A(x)\) for all \(n\)-dimensional vectors \(x\), and thus \(T = L_A\).

  Now we show that \(A\) is unique, i.e., there does not exist any other matrix
  \[
    B = \begin{pmatrix}
      b_{11} & b_{12} & \dots  & b_{1n} \\
      b_{21} & b_{22} & \dots  & b_{2n} \\
      \vdots & \vdots & \ddots & \vdots \\
      b_{m1} & b_{m2} & \dots  & b_{mn}
    \end{pmatrix}
  \]
  for which \(T\) is equal to \(L_B\).
  Suppose for sake of contradiction that we could find such a matrix \(B\) which was different from \(A\).
  Then we would have \(L_A = L_B\).
  In particular, we have \(L_A(e_j) = L_B(e_j)\) for every \(1 \leq j \leq n\).
  But from the definition of \(L_A\) we see that
  \[
    L_A(e_j) = (a_{ij})_{1 \leq i \leq m}
  \]
  and
  \[
    L_B(e_j) = (b_{ij})_{1 \leq i \leq m}
  \]
  and thus we have \(a_{ij} = b_{ij}\) for every \(1 \leq i \leq m\) and \(1 \leq j \leq n\), thus \(A\) and \(B\) are equal, a contradiction.
\end{proof}

\begin{rmk}\label{ii:6.1.14}
  \cref{ii:6.1.13} establishes a one-to-one correspondence between linear transformations and matrices, and is one of the fundamental reasons why matrices are so important in linear algebra.
  One may ask then why we bother dealing with linear transformations at all, and why we don't just work with matrices all the time.
  The reason is that sometimes one does not want to work with the standard basis \(e_1, \dots, e_n\), but instead wants to use some other basis.
  In that case, the correspondence between linear transformations and matrices changes, and so it is still important to keep the notions of linear transformation and matrix distinct.
  More discussion on this somewhat subtle issue can be found in any linear algebra text.
\end{rmk}

\begin{rmk}\label{ii:6.1.15}
  If \(T = L_A\), then \(A\) is sometimes called the \emph{matrix representation} of \(T\), and is sometimes denoted \(A = [T]\).
  We shall avoid this notation here, however.
\end{rmk}

\begin{note}
  The composition \(T \circ S\) of two linear transformations \(T, S\) is again a linear transformation (\cref{ii:ex:6.1.2}).
  It is customary in linear algebra to abbreviate such compositions \(T \circ S\) of linear transformations by droppinng the \(\circ\) symbol, thus \(T \circ S = TS\).
\end{note}

\begin{lem}\label{ii:6.1.16}
  Let \(A\) be an \(m \times n\) matrix, and let \(B\) be an \(n \times p\) matrix.
  Then \(L_A L_B = L_{AB}\).
\end{lem}

\begin{proof}
  Note that \(L_A L_B = L_A \circ L_B\), and we will work with the notation \(L_A \circ L_B\) instead.
  By \cref{ii:6.1.11} \(AB\) is well-defined.
  Let \(C = AB = (c_{ik})_{1 \leq i \leq m ; 1 \leq k \leq p}\).
  Then by \cref{ii:6.1.11} we have
  \[
    (c_{ik})_{1 \leq i \leq m ; 1 \leq k \leq p} = \bigg(\sum_{j = 1}^n a_{ij} b_{jk}\bigg)_{1 \leq i \leq m ; 1 \leq k \leq p}.
  \]
  Let \(x \in \R^p\).
  Then we have
  \begin{align*}
     & (L_A \circ L_B)(x)                                                                                                \\
     & = L_A\big(L_B(x)\big)                                                                                             \\
     & = L_A\Big(L_B\big((x_k)_{1 \leq k \leq p}\big)\Big)                                         &  & \by{ii:6.1.1}    \\
     & = L_A\Bigg(\bigg(\sum_{k = 1}^p b_{jk} x_k\bigg)_{1 \leq j \leq n}\Bigg)                    &  & \by{ii:ac:6.1.1} \\
     & = \Bigg(\sum_{j = 1}^n a_{ij} \bigg(\sum_{k = 1}^p b_{jk} x_k\bigg)\Bigg)_{1 \leq i \leq m} &  & \by{ii:ac:6.1.1} \\
     & = \Bigg(\sum_{j = 1}^n \bigg(\sum_{k = 1}^p a_{ij} b_{jk} x_k\bigg)\Bigg)_{1 \leq i \leq m}                       \\
     & = \Bigg(\sum_{k = 1}^p \bigg(\sum_{j = 1}^n a_{ij} b_{jk} x_k\bigg)\Bigg)_{1 \leq i \leq m}                       \\
     & = \Bigg(\sum_{k = 1}^p \bigg(\sum_{j = 1}^n a_{ij} b_{jk}\bigg) x_k\Bigg)_{1 \leq i \leq m}                       \\
     & = \bigg(\sum_{k = 1}^p c_{ik} x_k\bigg)_{1 \leq i \leq m}                                   &  & \by{ii:6.1.11}   \\
     & = L_C\big((x_k)_{1 \leq k \leq p}\big)                                                      &  & \by{ii:ac:6.1.1} \\
     & = L_C(x)                                                                                    &  & \by{ii:6.1.1}    \\
     & = L_{AB}(x).
  \end{align*}
  Since \(x\) was arbitrary, we have \(L_A \circ L_B = L_{AB}\).
\end{proof}

\exercisesection

\begin{ex}\label{ii:ex:6.1.1}
  Prove \cref{ii:6.1.2}.
\end{ex}

\begin{proof}
  See \cref{ii:6.1.2}.
\end{proof}

\begin{ex}\label{ii:ex:6.1.2}
  If \(T : \R^n \to \R^m\) is a linear transformation, and \(S : \R^p \to \R^n\) is a linear transformation, show that the composition \(T \circ S : \R^p \to \R^m\) of the two transforms, defined by \(T \circ S(x) \coloneqq T\big(S(x)\big)\), is also a linear transformation.
\end{ex}

\begin{proof}
  Let \(x, y \in \R^p\) and let \(c \in \R\).
  Then we have
  \begin{align*}
    (T \circ S)(x + y) & = T\big(S(x + y)\big)                                  \\
                       & = T\big(S(x) + S(y)\big)            &  & \by{ii:6.1.6} \\
                       & = T\big(S(x)\big) + T\big(S(y)\big) &  & \by{ii:6.1.6} \\
                       & = (T \circ S)(x) + (T \circ S)(y)
  \end{align*}
  and
  \begin{align*}
    (T \circ S)(cx) & = T\big(S(cx)\big)                     \\
                    & = T\big(c S(x)\big) &  & \by{ii:6.1.6} \\
                    & = c T\big(S(x)\big) &  & \by{ii:6.1.6} \\
                    & = c (T \circ S)(x).
  \end{align*}
  Thus, by \cref{ii:6.1.6} we know that \(T \circ S\) is a linear transformation from \(\R^p\) to \(\R^m\).
\end{proof}

\begin{ex}\label{ii:ex:6.1.3}
  Prove \cref{ii:6.1.16}.
\end{ex}

\begin{proof}
  See \cref{ii:6.1.16}.
\end{proof}

\begin{ex}\label{ii:ex:6.1.4}
  Let \(T : \R^n \to \R^m\) be a linear transformation.
  Show that there exists a number \(M > 0\) such that \(\norm*{T(x)} \leq M \norm*{x}\) for all \(x \in \R^n\).
  Conclude in particular that every linear transformation from \(\R^n\) to \(\R^m\) is continuous.
\end{ex}

\begin{proof}
  Since \(T\) is a linear transformations from \(\R^n\) to \(\R^m\), by \cref{ii:6.1.13} we know that there exists an \(m \times n\) matrix such that
  \[
    \forall x \in \R^n, \big(T(x)\big)^\top = Ax^\top.
  \]
  If we write
  \[
    A = (a_{ij})_{1 \leq i \leq m ; 1 \leq j \leq n},
  \]
  then
  \[
    M = \sum_{i = 1}^m \sum_{j = 1}^n \abs{a_{ij}}
  \]
  is well-defined.

  Let \(x \in \R^n\).
  Then we have
  \begin{align*}
    \norm*{T(x)} & = \norm*{\bigg(\sum_{j = 1}^n a_{ij} x_j\bigg)_{1 \leq i \leq m}}                 &  & \by{ii:6.1.11}   \\
                 & \leq \abs{\bigg(\sum_{i = 1}^m \sum_{j = 1}^n a_{ij} x_j\bigg)_{1 \leq i \leq m}} &  & \by{ii:ex:1.1.8} \\
                 & \leq \sum_{i = 1}^m \sum_{j = 1}^n \abs{a_{ij} x_j}                                                     \\
                 & = \sum_{i = 1}^m \sum_{j = 1}^n \abs{a_{ij}} \abs{x_j}                                                  \\
                 & = \sum_{j = 1}^n \sum_{i = 1}^m \abs{a_{ij}} \abs{x_j}                                                  \\
                 & = \sum_{j = 1}^n \bigg(\sum_{i = 1}^m \abs{a_{ij}}\bigg) \abs{x_j}                                      \\
                 & \leq \sum_{j = 1}^n M \abs{x_j}                                                                         \\
                 & = M \sum_{j = 1}^n \abs{x_j}                                                                            \\
                 & \leq M \sqrt{n} \norm*{x}.                                                        &  & \by{ii:ex:1.1.8}
  \end{align*}
  Since \(x\) was arbitrary, we have \(\norm*{T(x)} \leq M \sqrt{n} \norm*{x}\) for all \(x \in \R^n\).

  Now we show that \(T\) is continuous on \(\R^n\) from \((\R^n, d_{l^1}|_{\R^n \times \R^n})\) to \((\R^m, d_{l^1}|_{\R^m \times \R^m})\).
  From the proof above we know that there exists a \(M \in \R^+\) such that \(\norm*{T(x)} \leq M \norm*{x}\) for all \(x \in \R^n\).
  Fix such \(M\).
  Let \(x_0 \in \R^n\).
  Since
  \begin{align*}
             & \forall \varepsilon \in \R^+, \forall x \in \R^n, \norm*{x - x_0} < \dfrac{\varepsilon}{M}                    \\
    \implies & M \norm*{x - x_0} < \varepsilon                                                                               \\
    \implies & \norm*{T(x - x_0)} \leq M \norm*{x - x_0} < \varepsilon                                                       \\
    \implies & \norm*{T(x) - T(x_0)} \leq M \norm*{x - x_0} < \varepsilon,                                &  & \by{ii:6.1.6}
  \end{align*}
  by setting \(\delta = \dfrac{\varepsilon}{M}\) we know that
  \[
    \forall \varepsilon \in \R^+, \exists \delta \in \R^+ : \forall x \in \R^n, \norm*{x - x_0} < \delta \implies \norm*{T(x) - T(x_0)} < \varepsilon.
  \]
  Since \(x_0\) was arbitrary, by \cref{ii:2.1.1} this means \(T\) is continuous on \(\R^n\) from \((\R^n, d_{l^1}|_{\R^n \times \R^n})\) to \((\R^m, d_{l^1}|_{\R^m \times \R^m})\).
\end{proof}

\section{Derivatives in several variable calculus}\label{sec:6.2}

\begin{note}
  In single variable calculus, when one wants to differentiate a function \(f : E \to \R\) at a point \(x_0\), where \(E\) is a subset of \(\R\) and \(x_0\) is a limit point of \(E\), this is given by
  \[
    f'(x_0) \coloneqq \lim_{x \to x_0 ; x \in E \setminus \{x_0\}} \frac{f(x) - f(x_0)}{x - x_0}.
  \]
  One could try to mimic this definition in the several variable case \(f : E \to \R^m\), where \(E\) is now a subset of \(\R^n\), however we encounter a difficulty in this case:
  the quantity \(f(x) - f(x_0)\) will live in \(\R^m\), and \(x - x_0\) lives in \(\R^n\), and we do not know how to divide an \(m\)-dimensional vector by an \(n\)-dimensional vector.

  To get around this problem, we first rewrite the concept of derivative (in one dimension) in a way which does not involve division of vectors.
  Instead, we view differentiability at a point \(x_0\) as an assertion that a function \(f\) is ``approximately linear'' near \(x_0\).
\end{note}

\begin{lem}\label{6.2.1}
  Let \(E\) be a subset of \(\R\), \(f : E \to \R\) be a function, \(L \in \R\), and let \(x_0\) be a limit point of \(E\).
  Then the following two statements are equivalent.
  \begin{enumerate}
    \item \(f\) is differentiable at \(x_0\), and \(f'(x_0) = L\).
    \item We have
          \[
            \lim_{x \to x_0 ; x \in E \setminus \{x_0\}} \frac{\abs{f(x) - \big(f(x_0) + L (x - x_0)\big)}}{\abs{x - x_0}} = 0.
          \]
  \end{enumerate}
\end{lem}

\begin{proof}
  We have
  \begin{align*}
         & \lim_{x \to x_0 ; x \in E \setminus \{x_0\}} \frac{f(x) - f(x_0)}{x - x_0} = L                                                       \\
    \iff & \forall \varepsilon \in \R^+, \exists\ \delta \in \R^+ : \forall x \in E \setminus \{x_0\},                                          \\
         & \bigg(\abs{x - x_0} < \delta \implies \abs{\frac{f(x) - f(x_0)}{x - x_0} - L} < \varepsilon\bigg)                                    \\
    \iff & \forall \varepsilon \in \R^+, \exists\ \delta \in \R^+ : \forall x \in E \setminus \{x_0\},                                          \\
         & \bigg(\abs{x - x_0} < \delta \implies \abs{\frac{f(x) - f(x_0) - L(x - x_0)}{x - x_0}} < \varepsilon\bigg)                           \\
    \iff & \forall \varepsilon \in \R^+, \exists\ \delta \in \R^+ : \forall x \in E \setminus \{x_0\},                                          \\
         & \bigg(\abs{x - x_0} < \delta \implies \abs{\frac{f(x) - \big(f(x_0) + L(x - x_0)\big)}{x - x_0}} < \varepsilon\bigg)                 \\
    \iff & \forall \varepsilon \in \R^+, \exists\ \delta \in \R^+ : \forall x \in E \setminus \{x_0\},                                          \\
         & \bigg(\abs{x - x_0} < \delta \implies \abs{\frac{\abs{f(x) - \big(f(x_0) + L(x - x_0)\big)}}{\abs{x - x_0}}} < \varepsilon\bigg)     \\
    \iff & \forall \varepsilon \in \R^+, \exists\ \delta \in \R^+ : \forall x \in E \setminus \{x_0\},                                          \\
         & \bigg(\abs{x - x_0} < \delta \implies \abs{\frac{\abs{f(x) - \big(f(x_0) + L(x - x_0)\big)}}{\abs{x - x_0}} - 0} < \varepsilon\bigg) \\
    \iff & \lim_{x \to x_0 ; x \in E \setminus \{x_0\}} \frac{\abs{f(x) - \big(f(x_0) + L(x - x_0)\big)}}{\abs{x - x_0}} = 0.
  \end{align*}
\end{proof}

\begin{note}
  In light of \cref{6.2.1}, we see that the derivative \(f'(x_0)\) can be interpreted as the number \(L\) for which \(\abs{f(x) - \big(f(x_0) + L(x - x_0)\big)}\) is small, in the sense that it tends to zero as \(x\) tends to \(x_0\), even if we divide out by the very small number \(\abs{x - x_0}\).
  More informally, the derivative is the quantity \(L\) such that we have the approximation \(f(x) - f(x_0) \approx L(x - x_0)\).

  This does not seem too different from the usual notion of differentiation, but the point is that we are no longer explicitly dividing by \(x - x_0\).
  (We are still dividing by \(\abs{x - x_0}\), but this will turn out to be OK.)
  When we move to the several variable case \(f : E \to \R^m\), where \(E \subseteq \R^n\), we shall still want the derivative to be some quantity \(L\) such that \(f(x) - f(x_0) \approx L(x - x_0)\).
  However, since \(f(x) - f(x_0)\) is now an \(m\)-dimensional vector and \(x - x_0\) is an \(n\)-dimensional vector, we no longer want \(L\) to be a scalar;
  we want it to be a linear transformation.
\end{note}

\begin{defn}[Differentiability]\label{6.2.2}
  Let \(E\) be a subset of \(\R^n\), \(f : E \to \R^m\) be a function, \(x_0 \in E\) be a limit point, and let \(L : \R^n \to \R^m\) be a linear transformation.
  We say that \(f\) is \emph{differentiable at \(x_0\) with derivative \(L\)} if we have
  \[
    \lim_{x \to x_0 ; x \in E \setminus \{x_0\}} \frac{\norm*{f(x) - \big(f(x_0) + L(x - x_0)\big)}}{\norm*{x - x_0}} = 0.
  \]
  Here \(\norm*{x}\) is the length of \(x\) (as measured in the \(l^2\) metric)
  \[
    \norm*{(x_1, x_2, \dots, x_n)} = (x_1^2 + x_2^2 + \dots + x_n^2)^{1 / 2}.
  \]
\end{defn}

\setcounter{thm}{3}
\begin{lem}[Uniqueness of derivatives]\label{6.2.4}
  Let \(E\) be a subset of \(\R^n\), \(f : E \to \R^m\) be a function, \(x_0 \in E\) be an \emph{interior point} of \(E\), and let \(L_1 : \R^n \to \R^m\) and \(L_2 : \R^n \to \R^m\) be linear transformations.
  Suppose that \(f\) is differentiable at \(x_0\) with derivative \(L_1\), and also differentiable at \(x_0\) with derivative \(L_2\).
  Then \(L_1 = L_2\).
\end{lem}

\begin{proof}
  Suppose for sake of contradiction that \(L_1 \neq L_2\).
  Then there exists a \(v \in \R^n\) such that \(L_1(v) \neq L_2(v)\).
  Fix such \(v\).
  We know that \(v \neq 0_{\R^n}\) since if \(v = 0_{\R^n}\), then we would have
  \[
    L_1(0_{\R^n}) = L_1(0 \cdot 0_{\R^n}) = 0 L_1(0_{\R^n}) = 0_{\R^m} = 0 L_2(0_{\R^n}) = L_2(0 \cdot 0_{\R^n}) = L_2(0_{\R^n}).
  \]
  Observe that
  \begin{align*}
    \forall t \in \R, \norm*{x_0 + tv - x_0} & = \sqrt{\sum_{i = 1}^n \big((x_0 + tv - x_0)_i\big)^2} & \text{(by \cref{1.1.6})} \\
                                             & = \sqrt{\sum_{i = 1}^n (t v_i)^2}                      & \text{(by \cref{6.1.1})} \\
                                             & = \abs{t} \sqrt{\sum_{i = 1}^n (v_i)^2}.
  \end{align*}
  Since \(\lim_{t \to 0 ; t \neq 0} \abs{t} = 0\), we have
  \[
    \lim_{t \to 0 ; t \neq 0} \norm*{x_0 + tv - x_0} = 0
  \]
  which means
  \[
    \forall \varepsilon_t \in \R^+, \exists\ \delta \in \R^+ : \forall t \in \R \setminus \{0\}, \abs{t} < \delta \implies \norm*{x_0 + tv - x_0} < \varepsilon_t.
  \]
  Since \(x_0\) is an interior point, by \cref{1.2.5} we know that
  \[
    \exists\ t \in \R \setminus \{0\} : x_0 + tv \in E.
  \]
  Thus we have
  \[
    \forall \varepsilon_t \in \R^+, \exists\ \delta \in \R^+ : \forall t \in \R \setminus \{0\}, \abs{t} < \delta \implies \begin{cases}
      \norm*{x_0 + tv - x_0} < \varepsilon_t; \\
      x_0 + tv \in E.
    \end{cases}
  \]
  Since \(f\) is differentiable at \(x_0\) with derivative \(L_1\), by \cref{6.2.2} we know that
  \begin{align*}
     & \forall \varepsilon \in \R^+, \exists\ \delta_1 \in \R^+ : \forall x \in E \setminus \{x_0\},                             \\
     & \norm*{x - x_0} < \delta_1 \implies \frac{\norm*{f(x) - \big(f(x_0) + L_1(x - x_0)\big)}}{\norm*{x - x_0}} < \varepsilon.
  \end{align*}
  Similarly, we have
  \begin{align*}
     & \forall \varepsilon \in \R^+, \exists\ \delta_2 \in \R^+ : \forall x \in E \setminus \{x_0\},                             \\
     & \norm*{x - x_0} < \delta_2 \implies \frac{\norm*{f(x) - \big(f(x_0) + L_2(x - x_0)\big)}}{\norm*{x - x_0}} < \varepsilon.
  \end{align*}
  Let \(\varepsilon_t = \min(\delta_1, \delta_2)\).
  Then we have
  \begin{align*}
             & \forall \varepsilon \in \R^+, \exists\ \delta \in \R^+ : \forall t \in \R \setminus \{x_0\}, \abs{t} < \delta \\
    \implies & \begin{cases}
                 \norm*{x_0 + tv - x_0} < \varepsilon_t \\
                 x_0 + tv \in E
               \end{cases}                                                                        \\
    \implies & \begin{cases}
                 \frac{\norm*{f(x_0 + tv) - \big(f(x_0) + L_1(x_0 + tv - x_0)\big)}}{\norm*{x_0 + tv - x_0}} < \varepsilon \\
                 \frac{\norm*{f(x_0 + tv) - \big(f(x_0) + L_2(x_0 + tv - x_0)\big)}}{\norm*{x_0 + tv - x_0}} < \varepsilon
               \end{cases}                                           \\
    \implies & \begin{cases}
                 \frac{\norm*{f(x_0 + tv) - \big(f(x_0) + L_1(tv)\big)}}{\norm*{tv}} < \varepsilon \\
                 \frac{\norm*{f(x_0 + tv) - \big(f(x_0) + L_2(tv)\big)}}{\norm*{tv}} < \varepsilon
               \end{cases}
  \end{align*}
  and thus
  \begin{align*}
     & \lim_{t \to 0 ; t \neq 0, x_0 + tv \in E} \frac{\norm*{f(x_0 + tv) - \big(f(x_0) + L_1(tv)\big)}}{\norm*{tv}} = 0; \\
     & \lim_{t \to 0 ; t \neq 0, x_0 + tv \in E} \frac{\norm*{f(x_0 + tv) - \big(f(x_0) + L_2(tv)\big)}}{\norm*{tv}} = 0.
  \end{align*}
  By limit laws we have
  \begin{align*}
     & \lim_{t \to 0 ; t \neq 0, x_0 + tv \in E} \frac{\norm*{f(x_0 + tv) - \big(f(x_0) + L_1(tv)\big)} + \norm*{f(x_0 + tv) - \big(f(x_0) + L_2(tv)\big)}}{\norm*{tv}} \\
     & = 0.
  \end{align*}
  By \cref{1.1.6} and \cref{1.1.2} we know that
  \begin{align*}
     & \norm*{f(x_0 + tv) - \big(f(x_0) + L_1(tv)\big)} + \norm*{f(x_0 + tv) - \big(f(x_0) + L_2(tv)\big)}   \\
     & = \norm*{f(x_0 + tv) - \big(f(x_0) + L_1(tv)\big)} + \norm*{\big(f(x_0) + L_2(tv)\big) - f(x_0 + tv)} \\
     & \geq \norm*{f(x_0 + tv) - \big(f(x_0) + L_1(tv)\big) + \big(f(x_0) + L_2(tv)\big) - f(x_0 + tv)}      \\
     & = \norm*{L_2(tv) - L_1(tv)}                                                                           \\
     & \geq 0.
  \end{align*}
  Thus by squeeze test we have
  \[
    \lim_{t \to 0 ; t \neq 0, x_0 + tv \in E} \frac{\norm*{L_2(tv) - L_1(tv)}}{\norm*{tv}} = 0
  \]
  which implies
  \begin{align*}
     & \lim_{t \to 0 ; t \neq 0, x_0 + tv \in E} \frac{\norm*{L_2(tv) - L_1(tv)}}{\norm*{tv}}                                                       \\
     & = \lim_{t \to 0 ; t \neq 0, x_0 + tv \in E} \frac{\norm*{t L_2(v) - t L_1(v)}}{\norm*{tv}}                        & \text{(by \cref{6.1.6})} \\
     & = \lim_{t \to 0 ; t \neq 0, x_0 + tv \in E} \frac{\abs{t} \cdot \norm*{L_2(v) - L_1(v)}}{\abs{t} \cdot \norm*{v}}                            \\
     & = \lim_{t \to 0 ; t \neq 0, x_0 + tv \in E} \frac{\norm*{L_2(v) - L_1(v)}}{\norm*{v}}                                                        \\
     & = \frac{\norm*{L_2(v) - L_1(v)}}{\norm*{v}}                                                                                                  \\
     & = 0.
  \end{align*}
  But \(v \neq 0\) implies \(\norm*{v} \neq 0\) (by \cref{1.1.2}(b)), so we must have \(L_2(v) = L_1(v)\), a contradiction.
  Thus \(L_1 = L_2\).
\end{proof}

\begin{note}
  Because of \cref{6.2.4}, we can now talk about \emph{the} derivative of \(f\) at interior points \(x_0\), and we will denote this derivative by \(f'(x_0)\).
  Thus \(f'(x_0)\) is the unique linear transformation from \(\R^n\) to \(\R^m\) such that
  \[
    \lim_{x \to x_0 ; x \in E \setminus \{x_0\}} \frac{\norm*{f(x) - \big(f(x_0) + f'(x_0)(x - x_0)\big)}}{\norm*{x - x_0}} = 0.
  \]
  Informally, this means that the derivative \(f'(x_0))\) is the linear transformation such that we have
  \[
    f(x) - f(x_0) \approx f'(x_0)(x - x_0)
  \]
  or equivalently
  \[
    f(x) \approx f(x_0) + f'(x_0)(x - x_0)
  \]
  (this is known as Newton's approximation;
  compare with Proposition 10.1.7 in Analysis I).
\end{note}

\begin{note}
  Another consequence of \cref{6.2.4} is that if you know that \(f(x) = g(x)\) for all \(x \in E\), and \(f, g\) are differentiable at \(x_0\), then you also know that \(f'(x_0) = g'(x_0)\) at every \emph{interior} point of \(E\).
  However, this is not necessarily true if \(x_0\) is a boundary point of \(E\);
  for instance, if \(E\) is just a single point \(E = \{x_0\}\), merely knowing that \(f(x_0) = g(x_0)\) does not imply that \(f'(x_0) = g'(x_0)\).
  We will not deal with these boundary issues here, and only compute derivatives on the interior of the domain.
\end{note}

\begin{note}
  We will sometimes refer to \(f'\) as the \emph{total derivative} of \(f\), to distinguish this concept from that of partial and directional derivatives below.
  The total derivative \(f\) is also closely related to the \emph{derivative matrix} \(Df\),
  which we shall define in the next section.
\end{note}

\exercisesection

\begin{ex}\label{ex:6.2.1}
  Prove \cref{6.2.1}.
\end{ex}

\begin{proof}
  See \cref{6.2.1}.
\end{proof}

\begin{ex}\label{ex:6.2.2}
  Prove \cref{6.2.4}.
\end{ex}

\begin{proof}
  See \cref{6.2.4}.
\end{proof}
\section{Partial and directional derivatives}\label{sec:6.3}

\begin{defn}[Directional derivative]\label{6.3.1}
  Let \(E\) be a subset of \(\R^n\), \(f : E \to \R^m\) be a function, let \(x_0\) be an interior point of \(E\), and let \(v\) be a vector in \(\R^n\).
  If the limit
  \[
    \lim_{t \to 0 ; t > 0, x_0 + tv \in E} \dfrac{f(x_0 + tv) - f(x_0)}{t}
  \]
  exists, we say that \(f\) is \emph{differentiable in the direction \(v\) at \(x_0\)}, and we denote the above limit by \(D_v f(x_0)\):
  \[
    D_v f(x_0) \coloneqq \lim_{t \to 0 ; t > 0, x_0 + tv \in E} \dfrac{f(x_0 + tv) - f(x_0)}{t}.
  \]
\end{defn}

\begin{rmk}\label{6.3.2}
  One should compare \cref{6.3.1} with \cref{6.2.2}.
  Note that we are dividing by a scalar \(t\), rather than a vector, so this definition makes sense, and \(D_v f(x_0)\) will be a vector in \(\R^m\).
  It is sometimes possible to also define directional derivatives on the boundary of \(E\), if the vector \(v\) is pointing in an ``inward'' direction
  (this generalizes the notion of left derivatives and right derivatives from single variable calculus);
  but we will not pursue these matters here.
\end{rmk}

\begin{eg}\label{6.3.3}
  If \(f : \R \to \R\) is a function, then \(D_{+1} f(x)\) is the same as the right derivative of \(f(x)\) (if it exists), and similarly \(D_{-1} f(x)\) is the same as the negative of the left derivative of \(f(x)\) (if it exists).
\end{eg}

\begin{proof}
  We have
  \begin{align*}
    D_{+1} f(x) & = \lim_{t \to 0 ; t > 0} \dfrac{f(x_0 + t) - f(x_0)}{t}                           &  & \text{(by \cref{6.3.1})} \\
                & = \lim_{x \to x_0 ; x > x_0} \dfrac{f\big(x_0 + (x - x_0)\big) - f(x_0)}{x - x_0}                               \\
                & = \lim_{x \to x_0 ; x > x_0} \dfrac{f(x) - f(x_0)}{x - x_0}                                                     \\
                & = f(x_0+)
  \end{align*}
  and
  \begin{align*}
    D_{-1} f(x) & = \lim_{t \to 0 ; t > 0} \dfrac{f(x_0 - t) - f(x_0)}{t}                           &  & \text{(by \cref{6.3.1})} \\
                & = \lim_{x \to x_0 ; x < x_0} \dfrac{f\big(x_0 - (x_0 - x)\big) - f(x_0)}{x_0 - x}                               \\
                & = \lim_{x \to x_0 ; x < x_0} \dfrac{f(x) - f(x_0)}{x_0 - x}                                                     \\
                & = -\lim_{x \to x_0 ; x < x_0} \dfrac{f(x) - f(x_0)}{x - x_0}                                                    \\
                & = -f(x_0-).
  \end{align*}
\end{proof}

\setcounter{thm}{4}
\begin{lem}\label{6.3.5}
  Let \(E\) be a subset of \(\R^n\), \(f : E \to \R^m\) be a function, \(x_0\) be an interior point of \(E\), and let \(v\) be a vector in \(\R^n\).
  If \(f\) is differentiable at \(x_0\), then \(f\) is also differentiable in the direction \(v\) at \(x_0\), and
  \[
    D_v f(x_0) = f'(x_0)(v).
  \]
\end{lem}

\begin{proof}
  Since \(f\) is differentiable at \(x_0\), by \cref{6.2.2} we know that \(f'(x_0) : \R^n \to \R^m\) exists and \(f'(x_0)\) is a linear transformation.
  If \(v = 0_{\R^n}\), then we have
  \begin{align*}
     & f'(x_0)(0_{\R^n})                                                                                                                    \\
     & = 0_{\R^m}                                                                               &  & \text{(cf. the proof of \cref{6.2.4})} \\
     & = \lim_{t \to 0 ; t > 0, x_0 + t 0_{\R^n} \in E} \dfrac{f(x_0 + t 0_{\R^n}) - f(x_0)}{t}                                             \\
     & = D_{0_{\R^n}} f(x_0).                                                                   &  & \text{(by \cref{6.3.1})}
  \end{align*}
  So suppose that \(v \neq 0_{\R^n}\).
  Then we have
  \begin{align*}
             & \lim_{x \to x_0 ; x \in E \setminus \{x_0\}} \dfrac{\norm*{f(x) - f(x_0) - f'(x_0)(x - x_0)}}{\norm*{x - x_0}} = 0                                  \\
    \implies & \lim_{t \to 0 ; t > 0, x_0 + tv \in E \setminus \{x_0\}} \dfrac{\norm*{f(x_0 + tv) - f(x_0) - f'(x_0)(x_0 + tv - x_0)}}{\norm*{x_0 + tv - x_0}} = 0 \\
    \implies & \lim_{t \to 0 ; t > 0, x_0 + tv \in E \setminus \{x_0\}} \dfrac{\norm*{f(x_0 + tv) - f(x_0) - f'(x_0)(tv)}}{\norm*{tv}} = 0                         \\
    \implies & \lim_{t \to 0 ; t > 0, x_0 + tv \in E \setminus \{x_0\}} \dfrac{\norm*{t \dfrac{f(x_0 + tv) - f(x_0)}{t} - f'(x_0)(tv)}}{\norm*{tv}} = 0            \\
    \implies & \lim_{t \to 0 ; t > 0, x_0 + tv \in E \setminus \{x_0\}} \dfrac{\norm*{t \dfrac{f(x_0 + tv) - f(x_0)}{t} - t f'(x_0)(v)}}{\norm*{tv}} = 0           \\
    \implies & \lim_{t \to 0 ; t > 0, x_0 + tv \in E \setminus \{x_0\}} \dfrac{t \norm*{\dfrac{f(x_0 + tv) - f(x_0)}{t} - f'(x_0)(v)}}{t \norm*{v}} = 0            \\
    \implies & \lim_{t \to 0 ; t > 0, x_0 + tv \in E \setminus \{x_0\}} \dfrac{\norm*{\dfrac{f(x_0 + tv) - f(x_0)}{t} - f'(x_0)(v)}}{\norm*{v}} = 0                \\
    \implies & \lim_{t \to 0 ; t > 0, x_0 + tv \in E \setminus \{x_0\}} \norm*{\dfrac{f(x_0 + tv) - f(x_0)}{t} - f'(x_0)(v)} = 0                                   \\
    \implies & \lim_{t \to 0 ; t > 0, x_0 + tv \in E \setminus \{x_0\}} \dfrac{f(x_0 + tv) - f(x_0)}{t} = f'(x_0)(v)                                               \\
    \implies & D_v f(x_0) = f'(x_0)(v).
  \end{align*}
  Thus we conclude that
  \[
    f'(x_0) \text{ exists } \implies \forall v \in \R^n, D_v f(x_0) = f'(x_0)(v).
  \]
\end{proof}

\begin{rmk}\label{6.3.6}
  One consequence of \cref{6.3.5} is that total differentiability implies directional differentiability.
  However, the converse is not true;
  see \cref{ex:6.3.3}.
\end{rmk}

\begin{defn}[Partial derivative]\label{6.3.7}
  Let \(E\) be a subset of \(\R^n\), let \(f : E \to \R^m\) be a function, let \(x_0\) be an interior point of \(E\), and let \(1 \leq j \leq n\).
  Then the \emph{partial derivative of \(f\) with respect to the \(x_j\) variable} at \(x_0\), denoted \(\dfrac{\partial f}{\partial x_j}(x_0)\), is defined by
  \[
    \dfrac{\partial f}{\partial x_j}(x_0) \coloneqq \lim_{t \to 0 ; t \neq 0, x_0 + t e_j \in E} \dfrac{f(x_0 + t e_j) - f(x_0)}{t} = \dfrac{d}{dt} f(x_0 + t e_j)|_{t = 0}
  \]
  provided of course that the limit exists.
  (If the limit does not exist, we leave \(\dfrac{\partial f}{\partial x_j}(x_0)\) undefined).
\end{defn}

\begin{ac}\label{ac:6.3.1}
  Informally, the partial derivative can be obtained by holding all the variables other than \(x_j\) fixed, and then applying the single-variable calculus derivative in the \(x_j\) variable.
  Note that if \(f\) takes values in \(\R^m\), then so will \(\dfrac{\partial f}{\partial x_j}\).
  Indeed, if we write \(f\) in components as \(f = (f_1, \dots, f_m)\), it is easy to see (by \cref{1.1.18}) that
  \[
    \dfrac{\partial f}{\partial x_j}(x_0) = \bigg(\dfrac{\partial f_1}{\partial x_j}(x_0), \dots, \dfrac{\partial f_m}{\partial x_j}(x_0)\bigg)
  \]
  i.e., to differentiate a vector-valued function one just has to differentiate each of the components separately.
\end{ac}

\begin{note}
  We sometimes replace the variables \(x_j\) in \(\dfrac{\partial f}{\partial x_j}\) with other symbols.
  One should caution however that one should only relabel the variables if it is absolutely clear which symbol refers to the first variable, which symbol refers to the second variable, etc.;
  otherwise one may become unintentionally confused.
  The operation of total differentiation \(\dfrac{d}{dx}\) is not the same as that of partial differentiation \(\dfrac{\partial}{\partial x}\).
\end{note}

\begin{ac}\label{ac:6.3.2}
  From \cref{6.3.5} (and Proposition 9.5.3 from Analysis I), we know that if a function is differentiable at a point \(x_0\), then all the partial derivatives \(\dfrac{\partial f}{\partial x_j}\) exists at \(x_0\), and that
  \[
    \dfrac{\partial f}{\partial x_j}(x_0) = D_{e_j} f(x_0) = - D_{-e_j} f(x_0) = f'(x_0)(e_j).
  \]
  Also, if \(v = (v_1, \dots, v_n) = \sum_{j = 1}^n v_j e_j\), then we have
  \[
    D_v f(x_0) = f'(x_0) \bigg(\sum_{j = 1}^n v_j e_j\bigg) = \sum_{j = 1}^n v_j f'(x_0)(e_j)
  \]
  (since \(f'(x_0)\) is linear) and thus
  \[
    D_v f(x_0) = \sum_{j = 1}^n v_j \dfrac{\partial f}{\partial x_j}(x_0).
  \]
  Thus one can write directional derivatives in terms of partial derivatives, \emph{provided that} the function is actually differentiable at that point.
\end{ac}

\begin{note}
  Just because the partial derivatives exist at a point \(x_0\), we cannot conclude that the function is differentiable there (\cref{ex:6.3.3}).
  However, if we know that the partial derivatives not only exist, but are continuous, then we can in fact conclude differentiability, thanks to the \cref{6.3.8}
\end{note}

\begin{thm}\label{6.3.8}
  Let \(E\) be a subset of \(\R^n\), \(f : E \to \R^m\) be a function, \(F\) be a subset of \(E\), and \(x_0\) be an interior point of \(F\).
  If all the partial derivatives \(\dfrac{\partial f}{\partial x_j}\) exist on \(F\) and are continuous at \(x_0\), then \(f\) is differentiable at \(x_0\), and the linear transformation \(f'(x_0) : \R^n \to \R^m\) is defined by
  \[
    f'(x_0)\big((v_j)_{1 \leq j \leq n}\big) = \sum_{j = 1}^n v_j \dfrac{\partial f}{\partial x_j}(x_0).
  \]
\end{thm}

\begin{proof}
  Let \(L : \R^n \to \R^m\) be the linear transformation
  \[
    L\big((v_j)_{1 \leq j \leq n}\big) \coloneqq \sum_{j = 1}^n v_j \dfrac{\partial f}{\partial x_j}(x_0).
  \]
  We have to prove that
  \[
    \lim_{x \to x_0 ; x \in E \setminus \{x_0\}} \dfrac{\norm*{f(x) - \big(f(x_0) + L(x - x_0)\big)}}{\norm*{x - x_0}} = 0.
  \]
  Let \(\varepsilon > 0\).
  It will suffice to find a radius \(\delta > 0\) such that
  \[
    \dfrac{\norm*{f(x) - \big(f(x_0) + L(x - x_0)\big)}}{\norm*{x - x_0}} \leq \varepsilon
  \]
  for all \(x \in B_{(\R^n, d_{l^2})}(x_0, \delta) \setminus \{x_0\}\).
  Equivalently, we wish to show that
  \[
    \norm*{f(x) - f(x_0) - L(x - x_0)} \leq \varepsilon \norm*{x - x_0}
  \]
  for all \(x \in B_{(\R^n, d_{l^2})}(x_0, \delta) \setminus \{x_0\}\).

  Because \(x_0\) is an interior point of \(F\), there exists a ball \(B_{(\R^n, d_{l^2})}(x_0, r)\) which is contained inside \(F\).
  Because each partial derivative \(\dfrac{\partial f}{\partial x_j}\) exists on \(F\) and is continuous at \(x_0\), there thus exists an \(0 < \delta_j < r\) such that \(\norm*{\dfrac{\partial f}{\partial x_j}(x) - \dfrac{\partial f}{\partial x_j}(x_0)} \leq \dfrac{\varepsilon}{nm}\) for every \(x \in B_{(\R^n, d_{l^2})}(x_0, \delta_j)\).
  If we take \(\delta = \min(\delta_1, \dots, \delta_n)\), then we thus have \(\norm*{\dfrac{\partial f}{\partial x_j}(x) - \dfrac{\partial f}{\partial x_j}(x_0)} \leq \dfrac{\varepsilon}{nm}\) for every \(x \in B_{(\R^n, d_{l^2})}(x_0, \delta)\) and every \(1 \leq j \leq n\).

  Let \(x \in B_{(\R^n, d_{l^2})}(x_0, \delta)\).
  We write \(x = x_0 + v_1 e_1 + v_2 e_2 + \dots + v_n e_n\) for some scalars \(v_1, \dots, v_n\).
  Note that
  \[
    \norm*{x - x_0} = \sqrt{v_1^2 + v_2^2 + \dots + v_n^2}
  \]
  and in particular we have \(\abs{v_j} \leq \norm*{x - x_0}\) for all \(1 \leq j \leq n\).
  Our task is to show that
  \[
    \norm*{f(x_0 + v_1 e_1 + \dots + v_n e_n) - f(x_0) - \sum_{j = 1}^n v_j \dfrac{\partial f}{\partial x_j}(x_0)} \leq \varepsilon \norm*{x - x_0}.
  \]
  Write \(f\) in components as \(f = (f_1 , f_2, \dots, f_m)\)
  (so each \(f_i\) is a function from \(E\) to \(\R\)).
  From the mean value theorem in the \(x_1\) variable, we see that
  \[
    f_i(x_0 + v_1 e_1) - f_i(x_0) = \dfrac{\partial f_i}{\partial x_1}(x_0 + t_i e_1) v_1
  \]
  for some \(t_i\) between \(0\) and \(v_1\).
  This is done as follow:
  If \(v_1 = 0\), then by setting \(t_i = 0\) we have
  \[
    f_i(x_0 + 0 e_1) - f_i(x_0) = 0_{\R^m} = \dfrac{\partial f_i}{\partial x_1}(x_0 + 0 e_1) \cdot 0.
  \]
  So suppose that \(0 < v_1\).
  First observe that for any \(y \in F\), we have
  \begin{align*}
             & \dfrac{\partial f}{\partial x_1}(y) \in \R^m                                                                         \\
    \implies & \lim_{t \to 0 ; t \neq 0, y + t e_1 \in F} \dfrac{f(y + t e_1) - f(y)}{t} \in \R^m    &  & \text{(by \cref{6.3.7})}  \\
    \implies & \forall 1 \leq i \leq m,                                                                                             \\
             & \lim_{t \to 0 ; t \neq 0, y + t e_1 \in F} \dfrac{f_i(y + t e_1) - f_i(y)}{t} \in \R. &  & \text{(by \cref{1.1.18})}
  \end{align*}
  Since \(B_{(\R^n, l^2)}(x_0, \delta) \subseteq F\), by the definition of \(v_1\) we know that
  \begin{align*}
             & \forall v \in [0, v_1], x_0 + v e_1 \in B_{(\R^n, l^2)}(x_0, \delta) \subseteq F                                                         \\
    \implies & \forall 1 \leq i \leq m, \lim_{t \to 0 ; t \neq 0, x_0 + (v + t) e_1 \in F} \dfrac{f_i(x_0 + (v + t) e_1) - f_i(x_0 + v e_1)}{t} \in \R.
  \end{align*}
  If we define \(g_i : [0, v_1] \to \R\) for all \(1 \leq i \leq m\) as follow:
  \[
    \forall v \in [0, v_1], g_i(v) = f_i(x_0 + v e_1),
  \]
  then we know that
  \begin{align*}
             & \lim_{t \to 0 ; t \neq 0, x_0 + (v + t) e_1 \in F} \dfrac{f_i(x_0 + (v + t) e_1) - f_i(x_0 + v e_1)}{t} \in \R \\
    \implies & \lim_{t \to 0 ; t \neq 0, v + t \in [0, v_1]} \dfrac{g_i(v + t) - g_i(v)}{t} \in \R                            \\
    \implies & \lim_{w \to v ; w \in [0, v_1] \setminus \{v\}} \dfrac{g_i(w) - g_i(v)}{w - v} \in \R                          \\
    \implies & g_i'(v) \in \R
  \end{align*}
  for every \(1 \leq i \leq m\) and every \(v \in [0, v_1]\).
  Thus by mean value theorem we know that
  \begin{align*}
             & \exists\ t_i \in (0, v_1) : g_i'(t_i) = \dfrac{g_i(v_1) - g_i(0)}{v_1 - 0}                                                                                             \\
    \implies & \exists\ t_i \in (0, v_1) : \lim_{t \to t_i; t \in [0, v_1] \setminus \{t_i\}} \dfrac{g_i(t) - g_i(t_i)}{t - t_i} = \dfrac{g_i(v_1) - g_i(0)}{v_1 - 0}                 \\
    \implies & \exists\ t_i \in (0, v_1) : \lim_{t \to 0; t \neq 0, t + t_i \in [0, v_1] \setminus \{t_i\}} \dfrac{g_i(t + t_i) - g_i(t_i)}{t} = \dfrac{g_i(v_1) - g_i(0)}{v_1 - 0}   \\
    \implies & \exists\ t_i \in (0, v_1) :                                                                                                                                            \\
             & \lim_{t \to 0; t \neq 0, t + t_i \in [0, v_1] \setminus \{t_i\}} \dfrac{f_i(x_0 + (t + t_i) e_1) - f_i(x_0 + t_i e_1)}{t} = \dfrac{f_i(x_0 + v_1 e_1) - f_i(x_0)}{v_1} \\
    \implies & \exists\ t_i \in (0, v_1) : \dfrac{\partial f_i}{\partial x_1}(x_0 + t_i e_1) = \dfrac{f_i(x_0 + v_1 e_1) - f_i(x_0)}{v_1}                                             \\
    \implies & \exists\ t_i \in (0, v_1) : \dfrac{\partial f_i}{\partial x_1}(x_0 + t_i e_1) v_1 = f_i(x_0 + v_1 e_1) - f_i(x_0).
  \end{align*}
  The case \(v_1 < 0\) can be proven similarly.
  But we have
  \[
    \abs{\dfrac{\partial f_i}{\partial x_j}(x_0 + t_i e_1) - \dfrac{\partial f_i}{\partial x_j}(x_0)} \leq \norm*{\dfrac{\partial f}{\partial x_j}(x_0 + t_i e_1) - \dfrac{\partial f}{\partial x_j}(x_0)} \leq \dfrac{\varepsilon}{nm}
  \]
  and hence
  \[
    \abs{f_i(x_0 + v_1 e_1) - f_i(x_0) - \dfrac{\partial f_i}{\partial x_1}(x_0) v_1} \leq \dfrac{\varepsilon \abs{v_1}}{nm}.
  \]
  Summing this over all \(1 \leq i \leq m\) (and noting that \(\norm*{(y_1, \dots, y_m)} \leq \abs{y_1} + \dots + \abs{y_m}\) from the triangle inequality) we obtain
  \[
    \norm*{f(x_0 + v_1 e_1) - f(x_0) - \dfrac{\partial f}{\partial x_1}(x_0) v_1} \leq \dfrac{\varepsilon \abs{v_1}}{n};
  \]
  since \(\abs{v_1} \leq \norm*{x - x_0}\), we thus have
  \[
    \norm*{f(x_0 + v_1 e_1) - f(x_0) - \dfrac{\partial f}{\partial x_1}(x_0) v_1} \leq \dfrac{\varepsilon \norm*{x - x_0}}{n}.
  \]
  A similar argument gives
  \[
    \norm*{f(x_0 + v_1 e_1 + v_2 e_2) - f(x_0 + v_1 e_1) - \dfrac{\partial f}{\partial x_2}(x_0) v_2} \leq \dfrac{\varepsilon \norm*{x - x_0}}{n}
  \]
  and so forth up to
  \begin{align*}
     & \norm*{f(x_0 + v_1 e_1 + \dots + v_n e_n) - f(x_0 + v_1 e_1 + \dots + v_{n - 1} e_{n - 1}) - \dfrac{\partial f}{\partial x_n}(x_0) v_n} \\
     & \leq \dfrac{\varepsilon \norm*{x - x_0}}{n}.
  \end{align*}
  If we sum these \(n\) inequalities and use the triangle inequality \(\norm*{x + y} \leq \norm*{x} + \norm*{y}\), we obtain a telescoping series which simplifies to
  \[
    \norm*{f(x_0 + v_1 e_1 + \dots + v_n e_n) - f(x_0) - \sum_{j = 1}^n \dfrac{\partial f}{\partial x_j}(x_0) v_j} \leq \varepsilon \norm*{x - x_0}
  \]
  as desired.
\end{proof}

\begin{ac}\label{ac:6.3.3}
  From \cref{6.3.8} and \cref{6.3.5} we see that if the partial derivatives of a function \(f : E \to \R^m\) exist and are continuous on some set \(F\), then all the directional derivatives also exist at every interior point \(x_0\) of \(F\), and we have the formula
  \[
    D_{(v_1, \dots, v_n)} f(x_0) = \sum_{j = 1}^n v_j \dfrac{\partial f}{\partial x_j}(x_0).
  \]
  In particular, if \(f : E \to \R\) is a real-valued function, and we define the \emph{gradient} \(\nabla f(x_0)\) of \(f\) at \(x_0\) to be the \(n\)-dimensional row vector
  \[
    \nabla f(x_0) \coloneqq \bigg(\dfrac{\partial f}{\partial x_1}(x_0), \dots, \dfrac{\partial f}{\partial x_n}(x_0)\bigg),
  \]
  then we have the familiar formula
  \[
    D_v f(x_0) = v \cdot \nabla f(x_0)
  \]
  whenever \(x_0\) is in the interior of the region where the gradient exists and is continuous.
\end{ac}

\begin{ac}\label{ac:6.3.4}
  More generally, if \(f : E \to \R^m\) is a function taking values in \(\R^m\), with \(f = (f_1, \dots, f_m)\), and \(x_0\) is in the interior of the region where the partial derivatives of \(f\) exist and are continuous, then we have from \cref{6.3.8} that
  \[
    f'(x_0)\big((v_j)_{1 \leq j \leq n}\big) = \sum_{j = 1}^n v_j \dfrac{\partial f}{\partial x_j}(x_0) = \bigg(\sum_{j = 1}^n v_j \dfrac{\partial f_i}{\partial x_j}(x_0)\bigg)_{1 \leq i \leq m},
  \]
  which we can rewrite as
  \[
    L_{D f(x_0)}\big((v_j)_{1 \leq j \leq n}\big)
  \]
  where \(D f(x_0)\) is the \(m \times n\) matrix
  \begin{align*}
    D f(x_0) & \coloneqq \bigg(\dfrac{\partial f_i}{\partial x_j}(x_0)\bigg)_{1 \leq i \leq m ; 1 \leq j \leq n}                                      \\
             & = \begin{pmatrix}
                   \dfrac{\partial f_1}{\partial x_1}(x_0) & \dfrac{\partial f_1}{\partial x_2}(x_0) & \dots  & \dfrac{\partial f_1}{\partial x_n}(x_0) \\
                   \dfrac{\partial f_2}{\partial x_1}(x_0) & \dfrac{\partial f_2}{\partial x_2}(x_0) & \dots  & \dfrac{\partial f_2}{\partial x_n}(x_0) \\
                   \vdots                                  & \vdots                                  & \ddots & \vdots                                  \\
                   \dfrac{\partial f_m}{\partial x_1}(x_0) & \dfrac{\partial f_m}{\partial x_2}(x_0) & \dots  & \dfrac{\partial f_m}{\partial x_n}(x_0)
                 \end{pmatrix}.
  \end{align*}
  Thus we have
  \[
    \big(D_v f(x_0)\big)^\top = \big(f'(x_0)(v)\big)^\top = D f(x_0) v^\top.
  \]

  The matrix \(D f(x_0)\) is sometimes also called the \emph{derivative matrix} or \emph{differential matrix} of \(f\) at \(x_0\), and is closely related to the total derivative \(f'(x_0)\).
  One can also write \(Df\) as
  \[
    D f(x_0) = \bigg(\dfrac{\partial f}{\partial x_1}(x_0)^\top, \dfrac{\partial f}{\partial x_2}(x_0)^\top, \dots, \dfrac{\partial f}{\partial x_n}(x_0)^\top\bigg),
  \]
  i.e., each of the columns of \(D f(x_0)\) is one of the partial derivatives of \(f\), expressed as a column vector.
  Or one could write
  \[
    D f(x_0) = \begin{pmatrix}
      \nabla f_1(x_0) \\
      \nabla f_2(x_0) \\
      \vdots          \\
      \nabla f_m(x_0) \\
    \end{pmatrix}
  \]
  i.e., the rows of \(D f(x_0)\) are the gradient of various components of \(f\).
  In particular, if \(f\) is scalar-valued (i.e., \(m = 1\)), then \(Df\) is the same as \(\nabla f\).
\end{ac}

\exercisesection

\begin{ex}\label{ex:6.3.1}
  Prove \cref{6.3.5}.
\end{ex}

\begin{proof}
  See \cref{6.3.5}.
\end{proof}

\begin{ex}\label{ex:6.3.2}
  Let \(E\) be a subset of \(\R^n\), let \(f : E \to \R^m\) be a function, let \(x_0\) be an interior point of \(E\), and let \(1 \leq j \leq n\).
  Show that \(\dfrac{\partial f}{\partial x_j}(x_0)\) exists if and only if \(D_{e_j} f(x_0)\) and \(D_{-e_j} f(x_0)\) exist and are negatives of each other
  (thus \(D_{e_j} f(x_0) = -D_{-e_j} f(x_0)\));
  furthermore, one has \(\dfrac{\partial f}{\partial x_j}(x_0) = D_{e_j} f(x_0)\) in this case.
\end{ex}

\begin{proof}
  We have
  \begin{align*}
     & \dfrac{\partial f}{\partial x_j}(x_0)                                                                                  \\
     & = \lim_{t \to 0 ; t \neq 0, x_0 + t e_j \in E} \dfrac{f(x_0 + t e_j) - f(x_0)}{t}        &  & \text{(by \cref{6.3.7})} \\
     & = \begin{dcases}
           \lim_{t \to 0 ; t > 0, x_0 + t e_j \in E} \dfrac{f(x_0 + t e_j) - f(x_0)}{t} \\
           \lim_{t \to 0 ; t < 0, x_0 + t e_j \in E} \dfrac{f(x_0 + t e_j) - f(x_0)}{t}
         \end{dcases}          &  & \text{(by Proposition 9.5.3 in Analysis I)}                                         \\
     & = \begin{dcases}
           \lim_{t \to 0 ; t > 0, x_0 + t e_j \in E} \dfrac{f(x_0 + t e_j) - f(x_0)}{t} \\
           \lim_{t \to 0 ; t > 0, x_0 + t e_j \in E} \dfrac{f(x_0 - t e_j) - f(x_0)}{-t}
         \end{dcases}                                         \\
     & = \begin{dcases}
           \lim_{t \to 0 ; t > 0, x_0 + t e_j \in E} \dfrac{f(x_0 + t e_j) - f(x_0)}{t} \\
           -\lim_{t \to 0 ; t > 0, x_0 + t e_j \in E} \dfrac{f\big(x_0 + t (-e_j)\big) - f(x_0)}{t}
         \end{dcases}                              \\
     & = \begin{dcases}
           D_{e_j} f(x_0) \\
           - D_{-e_j} f(x_0)
         \end{dcases}.                                                                        &  & \text{(by \cref{6.3.1})}
  \end{align*}
\end{proof}

\begin{ex}\label{ex:6.3.3}
  Let \(f : \R^2 \to \R\) be the function defined by \(f(x, y) \coloneqq \dfrac{x^3}{x^2 + y^2}\) when \((x, y) \neq (0, 0)\), and \(f(0, 0) \coloneqq 0\).
  Show that \(f\) is not differentiable at \((0, 0)\), despite being differentiable in every direction \(v \in \R^2\) at \((0, 0)\).
  Explain why this does not contradict \cref{6.3.8}.
\end{ex}

\begin{proof}
  First we show that \(f\) is differentiable in every direction \(v \in \R^2\) at \((0, 0)\).
  Since
  \begin{align*}
    \forall v \in \R \setminus \{(0, 0)\}, & \lim_{t \to 0 ; t > 0, (0, 0) + tv \in \R^2} \dfrac{f\big((0, 0) + tv\big) - f(0, 0)}{t}    \\
                                           & = \lim_{t \to 0 ; t > 0, (0, 0) + tv \in \R^2} \dfrac{f(tv)}{t}                             \\
                                           & = \lim_{t \to 0 ; t > 0, (0, 0) + tv \in \R^2} \dfrac{t^3 v_1^3}{(t^2 v_1^2 + t^2 v_2^2) t} \\
                                           & = \lim_{t \to 0 ; t > 0, (0, 0) + tv \in \R^2} \dfrac{v_1^3}{v_1^2 + v_2^2}                 \\
                                           & = \dfrac{v_1^3}{v_1^2 + v_2^2}
  \end{align*}
  and
  \begin{align*}
     & \lim_{t \to 0 ; t > 0, (0, 0) + t (0, 0) \in \R^2} \dfrac{f\big((0, 0) + t(0, 0)\big) - f(0, 0)}{t} \\
     & = \lim_{t \to 0 ; t > 0, (0, 0) + t (0, 0) \in \R^2} 0                                              \\
     & = 0,
  \end{align*}
  by \cref{6.3.1} we know that \(f\) is differentiable in every direction \(v \in \R^2\) at \((0, 0)\).

  Next we show that \(f\) is not differentiable at \((0, 0)\).
  Suppose for sake of contradiction that \(f\) is differentiable at \((0, 0)\).
  Then by \cref{ac:6.3.2} we know that
  \[
    \forall v \in \R^2, D_v f(0, 0) = f'(0, 0)(v) = \sum_{i = 1}^2 v_i f'(0, 0)(e_i).
  \]
  But
  \[
    f'(0, 0)(1, 1) = \dfrac{1^3}{1^2 + 1^2} = \dfrac{1}{2}
  \]
  is not equal to
  \[
    \sum_{i = 1}^2 1 f'(0, 0)(e_i) = f'(0, 0)\big((1, 0)\big) + f'(0, 0)\big((0, 1)\big) = \dfrac{1^3}{1^2 + 0^2} + \dfrac{0^3}{0^2 + 1^2} = 1,
  \]
  a contradiction.
  Thus \(f\) is not differentiable at \((0, 0)\).

  Now we show that this does not contradict \cref{6.3.8}.
  We claim that \(\dfrac{\partial f}{\partial x}\) is not continuous at \((0, 0)\).
  Since
  \[
    D_{e_1} f(0, 0) = \dfrac{1^3}{1^2 + 0^2} = 1 = -\dfrac{(-1)^3}{(-1)^2 + 0^2} = -D_{-e_1} f(0, 0),
  \]
  by \cref{ex:6.3.2} we know that \(\dfrac{\partial f}{\partial x}(0, 0) = 1\).
  But for each \((x_0, y_0) \in \R^2 \setminus \{(0, 0)\}\), we have
  \begin{align*}
     & = \dfrac{\partial f}{\partial x}(x_0, y_0)                                                                                                                                                                                      \\
     & = \lim_{t \to 0 ; t \neq 0, (x_0, y_0) + t(1, 0) \in \R^2} \dfrac{f\big((x_0, y_0) + t(1, 0)\big) - f(x_0, y_0)}{t}                                                                                                             \\
     & = \lim_{t \to 0 ; t \neq 0, (x_0, y_0) + t(1, 0) \in \R^2} \dfrac{f(x_0 + t, y_0) - f(x_0, y_0)}{t}                                                                                                                             \\
     & = \lim_{t \to 0 ; t \neq 0, (x_0, y_0) + t(1, 0) \in \R^2} \dfrac{\dfrac{(x_0 + t)^3}{(x_0 + t)^2 + y_0^2} - \dfrac{x_0^3}{x_0^2 + y_0^2}}{t}                                                                                   \\
     & = \lim_{t \to 0 ; t \neq 0, (x_0, y_0) + t(1, 0) \in \R^2} \dfrac{(x_0 + t)^3 (x_0^2 + y_0^2) - x_0^3 \big((x_0 + t)^2 + y_0^2\big)}{t \big((x_0 + t)^2 + y_0^2\big) (x_0^2 + y_0^2)}                                           \\
     & = \lim_{t \to 0 ; t \neq 0, (x_0, y_0) + t(1, 0) \in \R^2} \dfrac{(x_0^3 + 3 x_0^2 t + 3 x_0 t^2 + t^3) (x_0^2 + y_0^2) - x_0^3 (x_0^2 + 2 t x_0 + t^2 + y_0^2)}{t (x_0^2 + 2 t x_0 + t^2 + y_0^2) (x_0^2 + y_0^2)}             \\
     & = \lim_{t \to 0 ; t \neq 0, (x_0, y_0) + t(1, 0) \in \R^2} \dfrac{(3 x_0^2 + 3 x_0 t + t^2) (x_0^2 + y_0^2) - x_0^3 (2 x_0 + t)}{(x_0^2 + 2 t x_0 + t^2 + y_0^2) (x_0^2 + y_0^2)}                                               \\
     & = \lim_{t \to 0 ; t \neq 0, (x_0, y_0) + t(1, 0) \in \R^2} \dfrac{x_0^4 + 2 t x_0^3 + t^2 x_0^2 + 3 x_0^2 y_0^2 + 3 t x_0 y_0^2 + t^2 y_0^2}{x_0^4 + 2 t x_0^3 + t^2 x_0^2 + 2 x_0^2 y_0^2 + 2 t x_0 y_0^2 + t^2 y_0^2 + y_0^4} \\
     & = \dfrac{x_0^4 + 3 x_0^2 y_0^2}{x_0^4 + 2 x_0^2 y_0^2 + y_0^4}.
  \end{align*}
  Thus we see that \((x_0, y_0) \to (0, 0)\) implies \(\dfrac{\partial f}{\partial x}(x_0, y_0) \not\to 1\), which means \(\dfrac{\partial f}{\partial x}\) is not continuous at \((0, 0)\).
\end{proof}

\begin{ex}\label{ex:6.3.4}
  Let \(f : \R^n \to \R^m\) be a differentiable function such that \(f'(x) = 0\) for all \(x \in \R^n\).
  Show that \(f\) is constant.
  For a tougher challenge, replace the domain \(\R^n\) by an open connected subset \(\Omega\) of \(\R^n\).
\end{ex}

\begin{proof}
  First we show the case when the domain of \(f\) is \(\R^n\).
  By \cref{ac:6.3.2} we know that
  \begin{align*}
             & \forall x_0 \in \R^n, \forall y \in \R^n, f'(x_0)(y) = \sum_{j = 1}^n y_j \dfrac{\partial f}{\partial x_j}(x_0) = 0_{\R^m} \\
    \implies & \forall x_0 \in \R^n, \forall 1 \leq j \leq n, \dfrac{\partial f}{\partial x_j}(x_0) = 0_{\R^m}.
  \end{align*}
  Let \(y \in \R^n\).
  Since
  \[
    y = \sum_{j = 1}^n y_j e_j,
  \]
  by mean value theorem (cf. the proof of \cref{6.3.8}) we know that
  \begin{align*}
     & \exists\ t_i \in (y_1, 0) \cup (0, y_1) :                                                                       \\
     & f_i(0_{\R^n} + y_1 e_1) - f_i(0_{\R^n}) = \dfrac{\partial f_i}{\partial x_1}(0_{\R^n} + t_i e_1) y_1 = 0_{\R^m}
  \end{align*}
  for all \(1 \leq i \leq m\).
  Similar arguments show that
  \[
    f_i(0_{\R^n} + y_1 e_1 + y_2 e_2) - f_i(0_{\R^n} + y_1 e_1) = 0_{\R^m}
  \]
  and
  \[
    f_i(0_{\R^n} + \sum_{j = 1}^n y_j e_j) - f_i(0_{\R^n} + \sum_{j = 1}^{n - 1} y_j e_j) = 0_{\R^m}.
  \]
  Summing all \(n\) terms above we obtain a telescoping series
  \[
    f_i(0_{\R^n} + \sum_{j = 1}^n y_j e_j) - f_i(0_{\R^n}) = 0_{\R^m}
  \]
  which means
  \[
    f_i(y) - f_i(0_{\R^n}) = 0_{\R^m}.
  \]
  Thus we have \(f(y) = f(0_{\R^n})\).
  Since \(y\) is arbitrary, we conclude that \(f\) is constant on \(\R^n\).

  Now we show the case when \(\Omega\), the domain of \(f\), is an open connected subset of \(\R^n\).
  By \cref{ac:6.3.2} we know that
  \begin{align*}
             & \forall x_0 \in \Omega, \forall y \in \Omega, f'(x_0)(y) = \sum_{j = 1}^n y_j \dfrac{\partial f}{\partial x_j}(x_0) = 0_{\R^m} \\
    \implies & \forall x_0 \in \Omega, \forall 1 \leq j \leq n, \dfrac{\partial f}{\partial x_j}(x_0) = 0.
  \end{align*}
  Let \(x_0 \in \Omega\) and let \(d = d_{l^2}|_{\Omega \times \Omega}\).
  Since \((\Omega, d)\) is open, by \cref{1.2.15}(a) we know that
  \[
    \exists\ \delta \in \R^+ : B_{(\Omega, d)}(x_0, \delta) \subseteq \Omega.
  \]
  We now claim that \(f\) is constant on \(B_{(\Omega, d)}(x_0, \delta)\).
  Let \(y \in B_{(\Omega, d)}(x_0, \delta)\).
  We write \(y = x_0 + v_1 e_1 + \dots + v_n e_n\).
  By mean value theorem (cf. the proof of \cref{6.3.8}) we know that
  \begin{align*}
     & \exists\ t_i \in (v_1, 0) \cup (0, v_1) :                                                        \\
     & f_i(x_0 + v_1 e_1) - f_i(x_0) = \dfrac{\partial f_i}{\partial x_1}(x_0 + t_i e_1) v_1 = 0_{\R^m}
  \end{align*}
  for all \(1 \leq i \leq m\).
  Similar arguments show that
  \[
    f_i(x_0 + v_1 e_1 + v_2 e_2) - f_i(x_0 + v_1 e_1) = 0_{\R^m}
  \]
  and
  \[
    f_i(x_0 + \sum_{j = 1}^n v_j e_j) - f_i(x_0 + \sum_{j = 1}^{n - 1} v_j e_j) = 0_{\R^m}.
  \]
  Summing all \(n\) terms above we obtain a telescoping series
  \[
    f_i(x_0 + \sum_{j = 1}^n v_j e_j) - f_i(x_0) = 0_{\R^m}
  \]
  which means
  \[
    f_i(y) - f_i(x_0) = 0_{\R^m}.
  \]
  Thus we have \(f(y) = f(x_0)\).
  Since \(y\) is arbitrary, we conclude that \(f\) is constant on \(B_{(\Omega, d)}(x_0, \delta)\).
  Since \(x_0\) is arbitrary, we conclude that \(f\) is constant on every open ball of \(\Omega\).
  But by \cref{2.4.1} we know that \((\Omega, d)\) is connected implies
  \begin{align*}
             & \forall x_0, y_0 \in \Omega, \exists\ \delta_1, \delta_2 \in \R^+ : \begin{dcases}
                                                                                     B_{(\Omega, d)}(x_0, \delta_1) \subseteq \Omega \\
                                                                                     B_{(\Omega, d)}(y_0, \delta_2) \subseteq \Omega \\
                                                                                     B_{(\Omega, d)}(x_0, \delta_1) \cap B_{(\Omega, d)}(y_0, \delta_2) \neq \emptyset
                                                                                   \end{dcases} \\
    \implies & \forall z \in B_{(\Omega, d)}(x_0, \delta_1) \cap B_{(\Omega, d)}(y_0, \delta_2), f(x_0) = f(z) = f(y_0).
  \end{align*}
  Thus \(f\) is constant on \(\Omega\).
\end{proof}
\section{The several variable calculus chain rule}\label{sec:6.4}

\begin{thm}[Several variable calculus chain rule]\label{6.4.1}
  Let \(E\) be a subset of \(\R^n\), and let \(F\) be a subset of \(\R^m\).
  Let \(f : E \to F\) be a function, and let \(g : F \to \R^p\) be another function.
  Let \(x_0\) be a point in the interior of \(E\).
  Suppose that \(f\) is differentiable at \(x_0\), and that \(f(x_0)\) is in the interior of \(F\).
  Suppose also that \(g\) is differentiable at \(f(x_0)\).
  Then \(g \circ f : E \to \R^p\) is also differentiable at \(x_0\), and we have the formula
  \[
    (g \circ f)'(x_0) = g'\big(f(x_0)\big) \circ f'(x_0).
  \]
\end{thm}

\begin{proof}
  By \cref{6.2.2} we want to show that
  \[
    \lim_{x \to x_0 ; x \in E \setminus \{x_0\}} \dfrac{\norm*{(g \circ f)(x) - (g \circ f)(x_0) - \Big(g'\big(f(x_0)\big) \circ f'(x_0)\Big)(x - x_0)}}{\norm*{x - x_0}} = 0.
  \]
  Equivalently, we want to show that
  \begin{align*}
             & \forall \varepsilon \in \R^+, \exists\ \delta \in \R^+ : \forall x \in E \setminus \{x_0\}, \norm*{x - x_0} < \delta                   \\
    \implies & \dfrac{\norm*{(g \circ f)(x) - (g \circ f)(x_0) - \Big(g'\big(f(x_0)\big) \circ f'(x_0)\Big)(x - x_0)}}{\norm*{x - x_0}} < \varepsilon \\
    \implies & \norm*{(g \circ f)(x) - (g \circ f)(x_0) - \Big(g'\big(f(x_0)\big) \circ f'(x_0)\Big)(x - x_0)} < \varepsilon \norm*{x - x_0}.
  \end{align*}
  Since \(f'(x_0)\) exists, we know that
  \[
    \lim_{x \to x_0 ; x \in E \setminus \{x_0\}} \dfrac{\norm*{f(x) - f(x_0) - f'(x_0)(x - x_0)}}{\norm*{x - x_0}} = 0.
  \]
  Equivalently, we know that
  \begin{align*}
             & \forall \varepsilon_f \in \R^+, \exists\ \delta \in \R^+ : \forall x \in E \setminus \{x_0\}, \norm*{x - x_0} < \delta \\
    \implies & \dfrac{\norm*{f(x) - f(x_0) - f'(x_0)(x - x_0)}}{\norm*{x - x_0}} < \varepsilon_f                                      \\
    \implies & \norm*{f(x) - f(x_0) - f'(x_0)(x - x_0)} < \varepsilon_f \norm*{x - x_0}                                               \\
    \implies & \norm*{f(x) - f(x_0)} < \varepsilon_f \norm*{x - x_0} + \norm*{f'(x_0)(x - x_0)}.
  \end{align*}
  Since \(g'\big(f(x_0)\big)\) exists, we know that
  \[
    \lim_{y \to f(x_0) ; x \in F \setminus \{f(x_0)\}} \dfrac{\norm*{g(y) - g\big(f(x_0)\big) - g'\big(f(x_0)\big)\big(y - f(x_0)\big)}}{\norm*{y - f(x_0)}} = 0.
  \]
  Equivalently, we know that
  \begin{align*}
             & \forall \varepsilon \in \R^+, \exists\ \delta_g \in \R^+ : \forall y \in F \setminus \{f(x_0)\}, \norm*{y - f(x_0)} < \delta_g \\
    \implies & \dfrac{\norm*{g(y) - g\big(f(x_0)\big) - g'\big(f(x_0)\big)\big(y - f(x_0)\big)}}{\norm*{y - f(x_0)}} < \varepsilon            \\
    \implies & \norm*{g(y) - g\big(f(x_0)\big) - g'\big(f(x_0)\big)\big(y - f(x_0)\big)} < \varepsilon \norm*{y - f(x_0)}.
  \end{align*}
  Fix one pair of \(\varepsilon\) and \(\delta_g\).
  Then we have
  \begin{align*}
             & \exists\ \delta \in \R^+ : \forall x \in E \setminus \{x_0\}, \norm*{x - x_0} < \delta                                       \\
    \implies & \begin{dcases}
                 \norm*{f(x) - f(x_0)} < \varepsilon \norm*{x - x_0} + \norm*{f'(x_0)(x - x_0)}; \\
                 \norm*{f(x) - f(x_0)} < \delta_g;
               \end{dcases}                                              \\
    \implies & \begin{dcases}
                 \norm*{f(x) - f(x_0)} < \varepsilon \norm*{x - x_0} + \norm*{f'(x_0)(x - x_0)}; \\
                 \norm*{g\big(f(x)\big) - g\big(f(x_0)\big) - g'\big(f(x_0)\big) \big(f(x) - f(x_0)\big)} < \varepsilon \norm*{f(x) - f(x_0)};
               \end{dcases} \\
    \implies & \norm*{g\big(f(x)\big) - g\big(f(x_0)\big) - g'\big(f(x_0)\big) \big(f(x) - f(x_0)\big)}                                     \\
             & < \varepsilon^2 \norm*{x - x_0} + \varepsilon \norm*{f'(x_0)(x - x_0)}                                                       \\
    \implies & \norm*{g\big(f(x)\big) - g\big(f(x_0)\big) - \Big(g'\big(f(x_0)\big) \circ f'(x_0)\Big) (x - x_0)}                           \\
             & < \varepsilon^2 \norm*{x - x_0} + \varepsilon \norm*{f'(x_0)(x - x_0)}                                                       \\
             & \quad + \norm*{g'\big(f(x_0)\big) \big(f(x) - f(x_0)\big) - \Big(g'\big(f(x_0)\big) \circ f'(x_0)\Big) (x - x_0)}.
  \end{align*}
  Since \(g'\big(f(x_0)\big)\) is a linear transformation, by \cref{6.1.6} we know that
  \begin{align*}
     & \norm*{g'\big(f(x_0)\big) \big(f(x) - f(x_0)\big) - \Big(g'\big(f(x_0)\big) \circ f'(x_0)\Big) (x - x_0)} \\
     & = \norm*{g'\big(f(x_0)\big) \big(f(x) - f(x_0) - f'(x_0) (x - x_0)\big)}.
  \end{align*}
  By \cref{ex:6.1.4} we know that
  \begin{align*}
    \exists\ M \in \R^+ : & \norm*{g'\big(f(x_0)\big) \big(f(x) - f(x_0) - f'(x_0) (x - x_0)\big)} \\
                          & \leq M \norm*{f(x) - f(x_0) - f'(x_0) (x - x_0)}                       \\
                          & \leq M \varepsilon \norm*{x - x_0}.
  \end{align*}
  Fix such \(M\).
  Then we have
  \begin{align*}
             & \exists\ \delta \in \R^+ : \forall x \in E \setminus \{x_0\}, \norm*{x - x_0} < \delta                  \\
    \implies & \norm*{g\big(f(x)\big) - g\big(f(x_0)\big) - \Big(g'\big(f(x_0)\big) \circ f'(x_0)\Big) (x - x_0)}      \\
             & < \varepsilon^2 \norm*{x - x_0} + \varepsilon \norm*{f'(x_0)(x - x_0)} + M \varepsilon \norm*{x - x_0}.
  \end{align*}
  Since \(\varepsilon\) is arbitrary, we conclude that
  \begin{align*}
             & \forall \varepsilon \in \R^+, \exists\ \delta \in \R^+ : \forall x \in E \setminus \{x_0\}, \norm*{x - x_0} < \delta \\
    \implies & \norm*{g\big(f(x)\big) - g\big(f(x_0)\big) - \Big(g'\big(f(x_0)\big) \circ f'(x_0)\Big) (x - x_0)} < \varepsilon.
  \end{align*}
\end{proof}

\begin{note}
  As a corollary of the chain rule and \cref{6.1.16} (and \cref{6.1.13}), we see that
  \[
    D (g \circ f)(x_0) = D g\big(f(x_0)\big) \cdot D f(x_0);
  \]
  i.e., we can write the chain rule in terms of matrices and matrix multiplication, instead of in terms of linear transformations and composition.
\end{note}

\begin{eg}\label{6.4.2}
  Let \(f : \R^n \to \R\) and \(g : \R^n \to \R\) be differentiable functions.
  We form the combined function \(h : \R^n \to \R^2\) by defining \(h(x) \coloneqq \big(f(x), g(x)\big)\).
  Now let \(k : \R^2 \to \R\) be the multiplication function \(k(a, b) \coloneqq ab\).

  We first show that
  \[
    D h(x_0) = \begin{pmatrix}
      \nabla f(x_0) \\
      \nabla g(x_0)
    \end{pmatrix}.
  \]
  Let \(x_0 \in \R^n\).
  Since
  \begin{align*}
     & \norm*{h(x) - h(x_0) - (x - x_0) D h(x_0)^\top}                                                                                  \\
     & = \norm*{\big(f(x), g(x)\big) - \big(f(x_0), g(x_0)\big) - \big((x - x_0) \nabla f(x_0)^\top, (x - x_0) \nabla g(x_0)^\top\big)} \\
     & = \norm*{\big(f(x) - f(x_0) - (x - x_0) \nabla f(x_0)^\top, g(x) - g(x_0) - (x - x_0) \nabla g(x_0)^\top\big)}                   \\
     & \leq \norm*{f(x) - f(x_0) - (x - x_0) \nabla f(x_0)^\top} + \norm*{g(x) - g(x_0) - (x - x_0) \nabla g(x_0)^\top}
  \end{align*}
  (note that the last line follow by \cref{ex:1.1.8}),
  by squeeze test we know that
  \begin{align*}
             & \begin{dcases}
                 \lim_{x \to x_0 ; x \in \R^n \setminus \{x_0\}} \dfrac{\norm*{f(x) - f(x_0) - f'(x_0)(x - x_0)}}{\norm*{x - x_0}} = 0 \\
                 \lim_{x \to x_0 ; x \in \R^n \setminus \{x_0\}} \dfrac{\norm*{g(x) - g(x_0) - g'(x_0)(x - x_0)}}{\norm*{x - x_0}} = 0
               \end{dcases}                    \\
    \implies & \begin{dcases}
                 \lim_{x \to x_0 ; x \in \R^n \setminus \{x_0\}} \dfrac{\norm*{f(x) - f(x_0) - (x - x_0) \nabla f(x_0)^\top}}{\norm*{x - x_0}} = 0 \\
                 \lim_{x \to x_0 ; x \in \R^n \setminus \{x_0\}} \dfrac{\norm*{g(x) - g(x_0) - (x - x_0) \nabla g(x_0)^\top}}{\norm*{x - x_0}} = 0
               \end{dcases}                    \\
    \implies & \lim_{x \to x_0 ; x \in \R^n \setminus \{x_0\}} \dfrac{\norm*{h(x) - h(x_0) - (x - x_0) D h(x_0)^\top}}{\norm*{x - x_0}} = 0.
  \end{align*}
  Since \(x_0\) is arbitrary, we conclude that the identity is true.

  Now we show that
  \[
    D k(a, b) = (b, a).
  \]
  Let \((a, b) \in \R^2\).
  Observe that for any \((x, y) \in \R^2 \setminus \{(a, b)\}\), we have
  \begin{align*}
     & \dfrac{\norm*{k(x, y) - k(a, b) - \big((x, y) - (a, b)\big) (b, a)^\top}}{\norm*{(x, y) - (a, b)}} \\
     & = \dfrac{\norm*{xy - ab - xb - ay + ab + ba}}{\norm*{(x - a, y - b)}}                              \\
     & = \dfrac{\norm*{(x - a)(y - b)}}{\norm*{(x - a, y - b)}}                                           \\
     & = \sqrt{\dfrac{(x - a)^2 (y - b)^2}{(x - a)^2 + (y - b)^2}}                                        \\
     & \leq \sqrt{\dfrac{2 (x - a)^2 (y - b)^2}{(x - a)^2 + (y - b)^2}}                                   \\
     & \leq \sqrt{\dfrac{(x - a)^4 + 2 (x - a)^2 (y - b)^2 + (y - b)^4}{(x - a)^2 + (y - b)^2}}           \\
     & = \sqrt{\dfrac{\big((x - a)^2 + (y - b)^2\big)^2}{(x - a)^2 + (y - b)^2}}                          \\
     & = \sqrt{(x - a)^2 + (y - b)^2}.
  \end{align*}
  Since
  \begin{align*}
             & \lim_{(x, y) \to (a, b) ; (x, y) \in \R^2 \setminus \{(a, b)\}} (x - a)^2 + (y - b)^2 = 0                                                                              \\
    \implies & \lim_{(x, y) \to (a, b) ; (x, y) \in \R^2 \setminus \{(a, b)\}} \sqrt{(x - a)^2 + (y - b)^2} = 0                                                                       \\
    \implies & \lim_{(x, y) \to (a, b) ; (x, y) \in \R^2 \setminus \{(a, b)\}} \dfrac{\norm*{k(x, y) - k(a, b) - \big((x, y) - (a, b)\big) (b, a)^\top}}{\norm*{(x, y) - (a, b)}} = 0
  \end{align*}
  (note that the last line follow by squeeze test),
  by \cref{6.2.2} we know that the identity is true.

  By the chain rule, we thus see that
  \[
    D (k \circ h)(x_0) = \big(g(x_0), f(x_0)\big) \begin{pmatrix}
      \nabla f(x_0) \\
      \nabla g(x_0)
    \end{pmatrix} = g(x_0) \nabla f(x_0) + f(x_0) \nabla g(x_0).
  \]
  But \(k \circ h = fg\), and \(D (fg) = \nabla (fg)\).
  We have thus proven the \emph{product rule}
  \[
    \nabla (fg) = g \nabla f + f \nabla g.
  \]
  A similar argument gives the sum rule \(\nabla (f + g) = \nabla f + \nabla g\), or the difference rule \(\nabla (f - g) = \nabla f - \nabla g\), as well as the quotient rule (\cref{ex:6.4.4}).
\end{eg}

\begin{note}
  We do record one further useful application of the chain rule.
  Let \(T : \R^n \to \R^m\) be a linear transformation.
  From \cref{ex:6.4.1} we observe that \(T\) is continuously differentiable at every point, and in fact \(T'(x) = T\) for every \(x\).
  (This equation may look a little strange, but perhaps it is easier to swallow if you view it in the form \(\dfrac{d}{dx} (Tx) = T\).)
  Thus, for any differentiable function \(f : E \to \R^n\), we see that \(T f : E \to \R^m\) is also differentiable, and hence by the chain rule
  \[
    (T f)'(x_0) = T\big(f'(x_0)\big).
  \]
  This is a generalization of the single-variable calculus rule \((cf)' = c(f')\) for constant scalars \(c\).
\end{note}

\begin{ac}\label{ac:6.4.1}
  If \(f : \R^n \to \R^m\) is some differentiable function, and \(x_j : \R \to \R\) are differentiable functions for each \(j = 1, \dots n\), then
  \[
    \dfrac{d}{dt} f\big(x_1(t), x_2(t), \dots, x_n(t)\big) = \sum_{j = 1}^n x_j'(t) \dfrac{\partial f}{\partial x_j} \big(x_1(t), x_2(t), \dots, x_n(t)\big).
  \]
\end{ac}

\begin{proof}
  Let \(h : \R \to \R^n\) be the function
  \[
    \forall t \in \R, h(t) = \big(x_1(t), \dots, x_n(t)\big).
  \]
  We claim that \(h\) is differentiable on \(\R\), and
  \[
    \forall t \in \R, D h(t) = \big(x_1'(t), \dots, x_n'(t)\big)^\top.
  \]
  Let \(t_0 \in \R\).
  Since
  \begin{align*}
     & \dfrac{\norm*{h(t) - h(t_0) - (t - t_0) D_h(t_0)^\top}}{\norm*{t - t_0}}                                                                                  \\
     & = \dfrac{\norm*{\big(x_1(t), \dots, x_n(t)\big) - \big(x_1(t_0), \dots, x_n(t_0)\big) - (t - t_0) \big(x_1'(t), \dots, x_n'(t)\big)^\top}}{\abs{t - t_0}} \\
     & = \dfrac{\norm*{\big(x_1(t) - x_1(t_0) - x_1'(t_0)(t - t_0), \dots, x_n(t) - x_n(t_0) - x_n'(t_0)(t - t_0)\big)}}{\abs{t - t_0}}                          \\
     & \leq \dfrac{\sum_{i = 1}^n \abs{\big(x_i(t) - x_i(t_0) - x_i'(t_0)(t - t_0)\big)}}{\abs{t - t_0}}                                                         \\
     & = \sum_{i = 1}^n \dfrac{\abs{\big(x_i(t) - x_i(t_0) - x_i'(t_0)(t - t_0)\big)}}{\abs{t - t_0}},
  \end{align*}
  we know that
  \begin{align*}
             & \forall 1 \leq i \leq n, \lim_{t \to t_0 ; t \in \R \setminus \{t_0\}} \dfrac{\abs{x_i(t) - x_i(t_0) - x_i'(t_0)(x - x_0)}}{\abs{t - t_0}} = 0                               \\
    \implies & \lim_{t \to t_0 ; t \in \R \setminus \{t_0\}} \sum_{i = 1}^n \dfrac{\abs{x_i(t) - x_i(t_0) - x_i'(t_0)(x - x_0)}}{\abs{t - t_0}} = 0           &  & \text{(by limit laws)}   \\
    \implies & \lim_{t \to t_0 ; t \in \R \setminus \{t_0\}} \dfrac{\norm*{h(t) - h(t_0) - (t - t_0) D_h(t_0)^\top}}{\norm*{t - t_0}}.                        &  & \text{(by squeeze test)}
  \end{align*}
  Since \(t_0\) is arbitrary, by \cref{6.2.2} we know that
  \[
    \forall t \in \R, D h(t) = \big(x_1'(t), \dots, x_n'(t)\big)^\top.
  \]
  Using chain rule we have
  \begin{align*}
    \forall t \in \R, & D (f \circ h)(t)                                                                                                           \\
                      & = D f\big(h(t)\big) \cdot D h(t)                                                          &  & \text{(by \cref{6.4.1})}    \\
                      & = D f\big(x_1(t), \dots, x_n(t)\big) \cdot \big(x_1'(t), \dots, x_n'(t)\big)^\top                                          \\
                      & = D_{\big(x_1'(t), \dots, x_n'(t)\big)} f\big(x_1(t), \dots, x_n(t)\big)                  &  & \text{(by \cref{6.3.5})}    \\
                      & = \sum_{j = 1}^n x_j'(t) \dfrac{\partial f}{\partial x_j}\big(x_1(t), \dots, x_n(t)\big). &  & \text{(by \cref{ac:6.3.2})}
  \end{align*}
\end{proof}

\exercisesection

\begin{ex}\label{ex:6.4.1}
  Let \(T : \R^n \to \R^m\) be a linear transformation.
  Show that \(T\) is continuously differentiable at every point, and in fact \(T'(x) = T\) for every \(x\).
  What is \(D T\)?
\end{ex}

\begin{proof}
  Let \(x_0 \in \R^n\).
  Since
  \begin{align*}
     & \lim_{x \to x_0 ; x \in \R^n \setminus \{x_0\}} \dfrac{\norm*{T(x) - T(x_0) - T(x - x_0)}}{\norm*{x - x_0}}                                    \\
     & = \lim_{x \to x_0 ; x \in \R^n \setminus \{x_0\}} \dfrac{\norm*{T(x) - T(x_0) - T(x) - T(x_0)}}{\norm*{x - x_0}} &  & \text{(by \cref{6.1.6})} \\
     & = \lim_{x \to x_0 ; x \in \R^n \setminus \{x_0\}} \dfrac{\norm*{0_{\R^m}}}{\norm*{x - x_0}}                                                    \\
     & = 0
  \end{align*}
  and \(x_0\) is arbitrary, we conclude by \cref{6.2.2} that
  \[
    \forall x \in \R^n, T'(x) = T.
  \]
  By \cref{6.1.13} we know that
  \[
    \forall x \in \R^n, D T(x) = \big(T(e_1)^\top, \dots, T(e_n)^\top\big).
  \]
\end{proof}

\begin{ex}\label{ex:6.4.2}
  Let \(E\) be a subset of \(\R^n\).
  Prove that if a function \(f : E \to \R^m\) is differentiable at an interior point \(x_0\) of \(E\), then it is also continuous at \(x_0\).
\end{ex}

\begin{proof}
  By \cref{6.2.2} we have
  \begin{align*}
             & \forall \varepsilon \in \R^+, \exists\ \delta_1 \in \R^+ : \forall x \in E \setminus \{x_0\},                                              \\
             & \bigg(\norm*{x - x_0} < \delta_1 \implies \dfrac{\norm*{f(x) - f(x_0) - f'(x_0)(x - x_0)}}{\norm*{x - x_0}} < \dfrac{\varepsilon}{2}\bigg) \\
    \implies & \forall \varepsilon \in \R^+, \exists\ \delta_1 \in \R^+ : \forall x \in E \setminus \{x_0\},                                              \\
             & \bigg(\norm*{x - x_0} < \delta_1 \implies \norm*{f(x) - f(x_0) - f'(x_0)(x - x_0)} < \dfrac{\varepsilon}{2} \norm*{x - x_0}\bigg)          \\
    \implies & \forall \varepsilon \in \R^+, \exists\ \delta_1 \in \R^+ : \forall x \in E \setminus \{x_0\},                                              \\
             & \bigg(\norm*{x - x_0} < \delta_1 \implies \norm*{f(x) - f(x_0)} < \dfrac{\varepsilon}{2} \norm*{x - x_0} + \norm*{f'(x_0)(x - x_0)}\bigg)  \\
    \implies & \forall \varepsilon \in \R^+, \exists\ \delta_1 \in \R^+ : \forall x \in E,                                                                \\
             & \bigg(\norm*{x - x_0} < \delta_1 \implies \norm*{f(x) - f(x_0)} < \dfrac{\varepsilon}{2} \norm*{x - x_0} + \norm*{f'(x_0)(x - x_0)}\bigg).
  \end{align*}
  By \cref{ex:6.1.4} we know that every linear transformation is continuous.
  Thus we know that
  \begin{align*}
             & \forall \varepsilon \in \R^+, \exists\ \delta_2 \in \R^+ : \forall x \in E,                                                              \\
             & \bigg(\norm*{x - x_0} < \delta_2 \implies \norm*{f'(x_0)(x) - f'(x_0)(x_0)} < \dfrac{\varepsilon}{2}\bigg)                               \\
    \implies & \forall \varepsilon \in \R^+, \exists\ \delta_2 \in \R^+ : \forall x \in E,                                                              \\
             & \bigg(\norm*{x - x_0} < \delta_2 \implies \norm*{f'(x_0)(x - x_0)} < \dfrac{\varepsilon}{2}\bigg).         &  & \text{(by \cref{6.1.6})}
  \end{align*}
  Let \(\delta = \min(\delta_1, \delta_2, 1)\).
  Then we have
  \begin{align*}
             & \forall \varepsilon \in \R^+, \exists\ \delta \in \R^+ : \forall x \in E, \norm*{x - x_0} < \delta                                                         \\
    \implies & \norm*{f(x) - f(x_0)} < \dfrac{\varepsilon}{2} \norm*{x - x_0} + \norm*{f'(x_0)(x - x_0)} < \dfrac{\varepsilon}{2} + \dfrac{\varepsilon}{2} = \varepsilon.
  \end{align*}
  Thus \(f\) is continuous at \(x_0\) from \((E, d_{l^2}|_{E \times E})\) to \((\R^m, d_{l^2}|_{\R^m \times \R^m})\).
\end{proof}

\begin{ex}\label{ex:6.4.3}
  Prove \cref{6.4.1}.
\end{ex}

\begin{proof}
  See \cref{6.4.1}.
\end{proof}

\begin{ex}\label{ex:6.4.4}
  State and prove some version of the quotient rule for functions of several variables (i.e., functions of the form \(f : E \to \R\) for some subset \(E\) of \(\R^n\)).
  In other words, state a rule which gives a formula for the gradient of \(f / g\);
  compare your answer with Theorem 10.1.13(h) in Analysis I.
  Be sure to make clear what all your assumptions are.
\end{ex}

\begin{proof}
  Let \(E \subseteq \R^n\) and let \(x_0\) be an interior point of \(E\).
  Let \(f : E \to \R\) and \(g : E \to \R\) be functions where \(g(x) \neq 0\) for all \(x \in E\).
  If \(f, g\) are differentiable at \(x_0\), then \(f / g\) is differentiable at \(x_0\), and
  \[
    \nabla \bigg(\dfrac{f}{g}\bigg)(x_0) = \dfrac{g(x_0) \nabla f(x_0) - f(x_0) \nabla g(x_0)}{\big(g(x_0)\big)^2}.
  \]
  If \(f, g\) are differentiable on \(E\), then \(f / g\) is differentiable on \(E\), and
  \[
    \nabla \bigg(\dfrac{f}{g}\bigg) = \dfrac{g \nabla f - f \nabla g}{g^2}.
  \]

  Let \(k : \R \times (\R \setminus \{0\}) \to \R\) be the function
  \[
    \forall (a, b) \in \R \times (\R \setminus \{0\}), k(a, b) = \dfrac{a}{b}.
  \]
  We claim that \(k\) is differentiable on \(\R \times (\R \setminus \{0\})\) and
  \[
    \forall (a, b) \in \R \times (\R \setminus \{0\}), D k(a, b) = \bigg(\dfrac{1}{b}, \dfrac{-a}{b^2}\bigg).
  \]
  Let \((a, b) \in \R \times (\R \setminus \{0\})\).
  Since for each \((x, y) \in \big(\R \times (\R \setminus \{0\})\big) \setminus \{(a, b)\}\), we have
  \begin{align*}
     & \dfrac{\norm*{k(x, y) - k(a, b) - \big((x, y) - (a, b)\big) \bigg(\dfrac{1}{b}, \dfrac{-a}{b^2}\bigg)^\top}}{\norm*{(x, y) - (a, b)}} \\
     & = \dfrac{\norm*{\dfrac{x}{y} - \dfrac{a}{b} - \dfrac{x - a}{b} + \dfrac{a(y - b)}{b^2}}}{\norm*{(x - a, y - b)}}                      \\
     & = \dfrac{\norm*{\dfrac{b^2 x - bxy + ay^2 - aby}{b^2y}}}{\norm*{(x - a, y - b)}}                                                      \\
     & = \dfrac{\norm*{\dfrac{(ay - bx)(y - b)}{b^2y}}}{\norm*{(x - a, y - b)}}                                                              \\
     & = \sqrt{\dfrac{(ay - bx)^2 (y - b)^2}{b^4 y^2 \big((x - a)^2 + (y - b)^2\big)}}                                                       \\
     & \leq \sqrt{\dfrac{(ay - bx)^2 \big((x - a)^2 + (y - b)^2\big)}{b^4 y^2 \big((x - a)^2 + (y - b)^2\big)}}                              \\
     & = \sqrt{\dfrac{(ay - bx)^2}{b^4 y^2}},
  \end{align*}
  we know that
  \begin{align*}
             & \lim_{(x, y) \to (a, b) ; (x, y) \in \big(\R \times (\R \setminus \{0\})\big) \setminus \{(a, b)\}} (ay - bx)^2 = 0                                                                                                                     \\
    \implies & \lim_{(x, y) \to (a, b) ; (x, y) \in \big(\R \times (\R \setminus \{0\})\big) \setminus \{(a, b)\}} \dfrac{(ay - bx)^2}{b^4 y^2} = 0                                                                                                    \\
    \implies & \lim_{(x, y) \to (a, b) ; (x, y) \in \big(\R \times (\R \setminus \{0\})\big) \setminus \{(a, b)\}} \sqrt{\dfrac{(ay - bx)^2}{b^4 y^2}} = 0                                                                                             \\
    \implies & \lim_{(x, y) \to (a, b) ; (x, y) \in \big(\R \times (\R \setminus \{0\})\big) \setminus \{(a, b)\}} \dfrac{\norm*{k(x, y) - k(a, b) - \big((x, y) - (a, b)\big) \big(\dfrac{1}{b}, \dfrac{-a}{b^2}\big)^\top}}{\norm*{(x, y) - (a, b)}} \\
             & = 0.
  \end{align*}
  (note that the last line was done by squeeze test)
  Since \((a, b)\) is arbitrary, by \cref{6.2.2} we conclude that
  \[
    \forall (a, b) \in \R \times (\R \setminus \{0\}), D k(a, b) = \bigg(\dfrac{1}{b}, \dfrac{-a}{b^2}\bigg).
  \]

  Let \(h\) be the function in \cref{6.4.2}.
  Using chain rule we have
  \begin{align*}
    D (k \circ h)(x_0) & = D k\big(h(x_0)\big) \cdot D h(x_0)                                                       &  & \text{(by \cref{6.4.1})} \\
                       & = D k\big(f(x_0), g(x_0)\big) \cdot \begin{pmatrix}
                                                               \nabla f(x_0) \\
                                                               \nabla g(x_0)
                                                             \end{pmatrix}                                        &  & \text{(by \cref{6.4.2})}   \\
                       & = \bigg(\dfrac{1}{g(x_0)}, \dfrac{-f(x_0)}{\big(g(x_0)\big)^2}\bigg) \cdot \begin{pmatrix}
                                                                                                      \nabla f(x_0) \\
                                                                                                      \nabla g(x_0)
                                                                                                    \end{pmatrix}                                \\
                       & = \dfrac{\nabla f(x_0)}{g(x_0)} - \dfrac{f(x_0) \nabla g(x_0)}{\big(g(x_0)\big)^2}                                       \\
                       & = \dfrac{g(x_0) \nabla f(x_0) - f(x_0) \nabla g(x_0)}{\big(g(x_0)\big)^2}.
  \end{align*}
  But \(k \circ h = f / g\).
  Thus we have
  \[
    \nabla \bigg(\dfrac{f}{g}\bigg)(x_0) = \dfrac{g(x_0) \nabla f(x_0) - f(x_0) \nabla g(x_0)}{\big(g(x_0)\big)^2}.
  \]
\end{proof}

\begin{ex}\label{ex:6.4.5}
  Let \(\vec{x} : \R \to \R^3\) be a differentiable function, and let \(r : \R \to \R\) be the function \(r(t) \coloneqq \norm*{\vec{x}(t)}\), where \(\norm*{\vec{x}}\) denotes the length of \(\vec{x}\) as measured in the usual \(l^2\) metric.
  Let \(t_0\) be a real number.
  Show that if \(r(t_0) \neq 0\), then \(r\) is differentiable at \(t_0\), and
  \[
    r'(t_0) = \dfrac{\vec{x}'(t_0) \cdot \vec{x}(t_0)}{r(t_0)}.
  \]
\end{ex}

\begin{proof}
  Let \(f : \R^3 \to \R\) be the function
  \[
    \forall x \in \R^3, f(x) = \norm*{x}.
  \]
  Since
  \[
    \forall y \in \R^3 \setminus \{0_{\R^3}\}, \forall 1 \leq i \leq 3, \dfrac{\partial f}{\partial x_i}(y) = \dfrac{y_i}{\norm*{y}},
  \]
  and \(\dfrac{\partial f}{\partial x_i}\) is continuous on \(\R^3 \setminus \{0_{\R^3}\}\), by \cref{6.3.8} we know that
  \begin{align*}
    \forall y \in \R^3 \setminus \{0_{\R^3}\}, D f(y) & = \nabla f(y)                                                                                                               \\
                                                      & = \bigg(\dfrac{\partial f}{\partial x_1}(y), \dfrac{\partial f}{\partial x_2}(y), \dfrac{\partial f}{\partial x_3}(y)\bigg) \\
                                                      & = \bigg(\dfrac{y_1}{\norm*{y}}, \dfrac{y_2}{\norm*{y}}, \dfrac{y_3}{\norm*{y}}\bigg)                                        \\
                                                      & = \dfrac{y}{\norm*{y}}.
  \end{align*}
  By chain rule (\cref{6.4.1}) we know that
  \begin{align*}
             & \forall t \in \R, \vec{x}(t) \neq 0                                                                                                  \\
    \implies & D (f \circ \vec{x})(t) = D f\big(\vec{x}(t)\big) \cdot D \vec{x}(t) = \dfrac{\vec{x}(t)}{\norm*{\vec{x}(t)}} \cdot \vec{x}'(t)^\top.
  \end{align*}
  Since \(r = f \circ \vec{x}\), we know that
  \begin{align*}
             & \forall t \in \R, \vec{x}(t) \neq 0                       \\
    \implies & \norm*{t} \neq 0                                          \\
    \implies & r(t) \neq 0                                               \\
    \implies & D r(t) = \dfrac{\vec{x}(t) \cdot \vec{x}'(t)^\top}{r(t)}.
  \end{align*}
\end{proof}
\section{Double derivatives and Clairaut's theorem}\label{sec:6.5}

\begin{defn}[Twice continuous differentiability]\label{6.5.1}
  Let \(E\) be an open subset of \(\R^n\), and let \(f : E \to \R^m\) be a function.
  We say that \(f\) is \emph{continuously differentiable} if the partial derivatives \(\frac{\partial f}{\partial x_1}, \dots, \frac{\partial f}{\partial x_n}\) exist and are continuous on \(E\).
  We say that \(f\) is \emph{twice continuously differentiable} if it is continuously differentiable, and the partial derivatives \(\frac{\partial f}{\partial x_1}, \dots, \frac{\partial f}{\partial x_n}\) are themselves continuously differentiable.
\end{defn}

\begin{rmk}\label{6.5.2}
  Continuously differentiable functions are sometimes called \(C^1\) functions;
  twice continuously differentiable functions are sometimes called \(C^2\) functions.
  One can also define \(C^3\), \(C^4\), etc. but we shall not do so here.
\end{rmk}

\setcounter{thm}{3}
\begin{thm}[Clairaut's theorem]\label{6.5.4}
  Let \(E\) be an open subset of \(\R^n\), and let \(f : E \to \R^m\) be a twice continuously differentiable function on \(E\).
  Then we have \(\frac{\partial}{\partial x_j} \frac{\partial f}{\partial x_i}(x_0) = \frac{\partial}{\partial x_i} \frac{\partial f}{\partial x_j}(x_0)\) for all \(1 \leq i, j \leq n\).
\end{thm}

\begin{proof}
  By working with one component of \(f\) at a time we can assume that \(m = 1\).
  The claim is trivial if \(i = j\), so we shall assume that \(i \neq j\).
  We shall prove the theorem for \(x_0 = 0\);
  the general case is similar.
  (Actually, once one proves Clairaut's theorem for \(x_0 = 0\), one can immediately obtain it for general \(x_0\) by applying the theorem with \(f(x)\) replaced by \(f(x + x_0)\).)

  Let \(a\) be the number \(a \coloneqq \frac{\partial}{\partial x_j} \frac{\partial f}{\partial x_i}(0)\), and \(a'\) denote the quantity \(a' \coloneqq \frac{\partial}{\partial x_i} \frac{\partial f}{\partial x_j}(0)\).
  Our task is to show that \(a' = a\).

  Let \(\varepsilon > 0\).
  Because the double derivatives of \(f\) are continuous, we can find a \(\delta > 0\) such that
  \[
    \abs{\frac{\partial}{\partial x_j} \frac{\partial f}{\partial x_i}(x) - a} \leq \varepsilon
  \]
  and
  \[
    \abs{\frac{\partial}{\partial x_i} \frac{\partial f}{\partial x_j}(x) - a'} \leq \varepsilon
  \]
  whenever \(\norm*{x} \leq 2\delta\).

  Now we consider the quantity
  \[
    X \coloneqq f(\delta e_i + \delta e_j) - f(\delta e_i) - f(\delta e_j) + f(0).
  \]
  From the fundamental theorem of calculus in the \(e_i\) variable, we have
  \[
    f(\delta e_i + \delta e_j) - f(\delta e_j) = \int_0^{\delta} \frac{\partial f}{\partial x_i}(x_i e_i + \delta e_j) \; d x_i
  \]
  and
  \[
    f(\delta e_i) - f(0) = \int_0^{\delta} \frac{\partial f}{\partial x_i}(x_i e_i) \; d x_i
  \]
  and hence
  \[
    X = \int_0^{\delta} \bigg(\frac{\partial f}{\partial x_i} (x_i e_i + \delta e_j) - \frac{\partial f}{\partial x_i} (x_i e_i)\bigg) \; d x_i.
  \]
  But by the mean value theorem, for each \(x_i\) we have
  \[
    \frac{\partial f}{\partial x_i} (x_i e_i + \delta e_j) - \frac{\partial f}{\partial x_i} (x_i e_i) = \delta \frac{\partial}{\partial x_j} \frac{\partial f}{\partial x_i} (x_i e_i + x_j e_j)
  \]
  for some \(0 \leq x_j \leq \delta\).
  By our construction of \(\delta\), we thus have
  \[
    \abs{\frac{\partial f}{\partial x_i} (x_i e_i + \delta e_j) - \frac{\partial f}{\partial x_i} (x_i e_i) - \delta a} \leq \varepsilon \delta.
  \]
  Integrating this from \(0\) to \(\delta\), we thus obtain
  \[
    \abs{X - \delta^2 a} \leq \varepsilon \delta^2.
  \]

  We can run the same argument with the rôle of \(i\) and \(j\) reversed (note that \(X\) is symmetric in \(i\) and \(j\)), to obtain
  \[
    \abs{X - \delta^2 a'} \leq \varepsilon \delta^2.
  \]
  From the triangle inequality we thus obtain
  \[
    \abs{\delta^2 a - \delta^2 a'} \leq 2 \varepsilon \delta^2,
  \]
  and thus
  \[
    \abs{a - a'} \leq 2 \varepsilon.
  \]
  But this is true for all \(\varepsilon > 0\), and \(a\) and \(a'\) do not depend on \(\varepsilon\), and so we must have \(a = a'\), as desired.
\end{proof}

\exercisesection

\begin{ex}\label{ex:6.5.1}
  Let \(f : \R^2 \to \R\) be the function defined by \(f(x, y) \coloneqq \frac{x y^3}{x^2 + y^2}\) when \((x, y) \neq (0, 0)\), and \(f(0, 0) \coloneqq 0\).
  Show that \(f\) is continuously differentiable, and the double derivatives \(\frac{\partial}{\partial y} \frac{\partial f}{\partial x}\) and \(\frac{\partial}{\partial x} \frac{\partial f}{\partial y}\) exist, but are not equal to each other at \((0, 0)\).
  Explain why this does not contradict Clairaut's theorem.
\end{ex}

\begin{proof}
  Let \((a, b) \in \R^2 \setminus \{(0, 0)\}\).
  We have
  \begin{align*}
     & \frac{\partial f}{\partial x}(a, b)                                                                                                                              \\
     & = \lim_{t \to 0 ; t \neq 0} \frac{f\big((a, b) + t(1, 0)\big) - f(a, b)}{t}                                                           & \text{(by \cref{6.3.7})} \\
     & = \lim_{t \to 0 ; t \neq 0} \frac{f(a + t, b) - f(a, b)}{t}                                                                                                      \\
     & = \lim_{t \to 0 ; t \neq 0} \frac{\frac{(a + t) b^3}{(a + t)^2 + b^2} - \frac{ab^3}{a^2 + b^2}}{t}                                                               \\
     & = \lim_{t \to 0 ; t \neq 0} \frac{b^3 (a + t) (a^2 + b^2) - a b^3 \big((a + t)^2 + b^2\big)}{t \big((a + t)^2 + b^2\big) (a^2 + b^2)}                            \\
     & = \lim_{t \to 0 ; t \neq 0} \frac{t b^5 - t a^2 b^3 - t^2 a b^3}{t \big((a + t)^2 + b^2\big) (a^2 + b^2)}                                                        \\
     & = \lim_{t \to 0 ; t \neq 0} \frac{b^5 - a^2 b^3 - t a b^3}{\big((a + t)^2 + b^2\big) (a^2 + b^2)}                                                                \\
     & = \frac{b^5 - a^2 b^3}{(a^2 + b^2)^2}
  \end{align*}
  and
  \begin{align*}
     & \frac{\partial f}{\partial y}(a, b)                                                                                                                                \\
     & = \lim_{t \to 0 ; t \neq 0} \frac{f\big((a, b) + t(0, 1)\big) - f(a, b)}{t}                                                             & \text{(by \cref{6.3.7})} \\
     & = \lim_{t \to 0 ; t \neq 0} \frac{f(a, b + t) - f(a, b)}{t}                                                                                                        \\
     & = \lim_{t \to 0 ; t \neq 0} \frac{\frac{a (b + t)^3}{a^2 + (b + t)^2} - \frac{ab^3}{a^2 + b^2}}{t}                                                                 \\
     & = \lim_{t \to 0 ; t \neq 0} \frac{a (b + t)^3 (a^2 + b^2) - a b^3 \big(a^2 + (b + t)^2\big)}{t \big(a^2 + (b + t)^2\big) (a^2 + b^2)}                              \\
     & = \lim_{t \to 0 ; t \neq 0} \frac{(3 b^2 t + 3 b t^2 + t^3) (a^3 + a b^2) - a b^3 (2bt + t^2)}{t \big(a^2 + (b + t)^2\big) (a^2 + b^2)}                            \\
     & = \lim_{t \to 0 ; t \neq 0} \frac{(3 b^2 + 3 b t + t^2) (a^3 + a b^2) - a b^3 (2b + t)}{\big(a^2 + (b + t)^2\big) (a^2 + b^2)}                                     \\
     & = \frac{3 a^3 b^2 + a b^4}{(a^2 + b^2)^2}.
  \end{align*}
  Since \((a, b)\) is arbitrary, by \cref{2.2.2} we know that \(\frac{\partial f}{\partial x}\) and \(\frac{\partial f}{\partial y}\) are continuous on \(\R^2 \setminus \{(0, 0)\}\) from \((\R^2, d_{l^2}|_{\R^2 \times \R^2})\) to \((\R, d_{l^1}|_{\R \times \R})\).
  Observe that
  \begin{align*}
    \abs{\frac{b^5 - a^2 b^3}{(a^2 + b^2)^2}} & = \frac{\abs{b (b^4 - a^2 b^2)}}{(a^2 + b^2)^2}            \\
                                              & = \frac{\abs{b} \abs{b^4 - a^2 b^2}}{(a^2 + b^2)^2}        \\
                                              & \leq \frac{\abs{b} (b^4 + 2 a^2 b^2 + a^4)}{(a^2 + b^2)^2} \\
                                              & = \frac{\abs{b} (a^2 + b^2)^2}{(a^2 + b^2)^2}              \\
                                              & = \abs{b}
  \end{align*}
  and
  \begin{align*}
    \abs{\frac{3 a^3 b^2 + a b^4}{(a^2 + b^2)^2}} & = \frac{\abs{a (2 a^2 b^2 + b^4) + a (a^2 b^2)}}{(a^2 + b^2)^2}                              \\
                                                  & \leq \frac{\abs{a} \abs{2 a^2 b^2 + b^4} + \abs{a} \abs{a^2 b^2}}{(a^2 + b^2)^2}             \\
                                                  & \leq \frac{\abs{a} (a^4 + 2 a^2 b^2 + b^4) + \abs{a} (a^4 + 2 a^2 b^2 + b^4)}{(a^2 + b^2)^2} \\
                                                  & = \frac{\abs{a} (a^2 + b^2)^2 + \abs{a} (a^2 + b^2)^2}{(a^2 + b^2)^2}                        \\
                                                  & = 2 \abs{a}.
  \end{align*}
  Thus we have
  \begin{align*}
             & \lim_{(a, b) \to (0, 0) ; (a, b) \neq (0, 0)} \abs{b} = 0                                                              \\
    \implies & \lim_{(a, b) \to (0, 0) ; (a, b) \neq (0, 0)} \abs{\frac{b^5 - a^2 b^3}{(a^2 + b^2)^2}} = 0 & \text{(by squeeze test)} \\
    \implies & \lim_{(a, b) \to (0, 0) ; (a, b) \neq (0, 0)} \frac{b^5 - a^2 b^3}{(a^2 + b^2)^2} = 0                                  \\
    \implies & \lim_{(a, b) \to (0, 0) ; (a, b) \neq (0, 0)} \frac{\partial f}{\partial x}(a, b) = 0
  \end{align*}
  and
  \begin{align*}
             & \lim_{(a, b) \to (0, 0) ; (a, b) \neq (0, 0)} 2 \abs{a} = 0                                                                \\
    \implies & \lim_{(a, b) \to (0, 0) ; (a, b) \neq (0, 0)} \abs{\frac{3 a^3 b^2 + a b^4}{(a^2 + b^2)^2}} = 0 & \text{(by squeeze test)} \\
    \implies & \lim_{(a, b) \to (0, 0) ; (a, b) \neq (0, 0)} \frac{3 a^3 b^2 + a b^4}{(a^2 + b^2)^2} = 0                                  \\
    \implies & \lim_{(a, b) \to (0, 0) ; (a, b) \neq (0, 0)} \frac{\partial f}{\partial y}(a, b) = 0.
  \end{align*}
  Since
  \begin{align*}
    \frac{\partial f}{\partial x}(0, 0) & = \lim_{t \to 0 ; t \neq 0} \frac{f\big((0, 0) + t(1, 0)\big) - f(0, 0)}{t} & \text{(by \cref{6.3.7})} \\
                                        & = \lim_{t \to 0 ; t \neq 0} \frac{\frac{t 0^3}{t^2 + 0^2} - 0}{t}                                      \\
                                        & = 0
  \end{align*}
  and
  \begin{align*}
    \frac{\partial f}{\partial y}(0, 0) & = \lim_{t \to 0 ; t \neq 0} \frac{f\big((0, 0) + t(0, 1)\big) - f(0, 0)}{t} & \text{(by \cref{6.3.7})} \\
                                        & = \lim_{t \to 0 ; t \neq 0} \frac{\frac{0 t^3}{0^2 + t^2} - 0}{t}                                      \\
                                        & = 0,
  \end{align*}
  by \cref{2.1.1} we know that \(\frac{\partial f}{\partial x}\) and \(\frac{\partial f}{\partial y}\) are continuous at \((0, 0)\) from \((\R^2, d_{l^2}|_{\R^2 \times \R^2})\) to \((\R, d_{l^1}|_{\R \times \R})\).
  Combine the proof above we conclude by \cref{6.5.1} that \(f\) is continuously differentiable on \(\R^2\).

  Let \((a, b) \in \R^2 \setminus \{(0, 0)\}\).
  Observe that
  \begin{align*}
     & \frac{\partial}{\partial y} \frac{\partial f}{\partial x}(a, b)                                                                                                                 \\
     & = \lim_{t \to 0 ; t \neq 0} \frac{\frac{\partial f}{\partial x}\big((a, b) + t(0, 1)\big) - \frac{\partial f}{\partial x}(a, b)}{t}                                             \\
     & = \lim_{t \to 0 ; t \neq 0} \frac{\frac{\partial f}{\partial x}(a, b + t) - \frac{\partial f}{\partial x}(a, b)}{t}                                                             \\
     & = \lim_{t \to 0 ; t \neq 0} \frac{\frac{(b + t)^5 - a^2 (b + t)^3}{\big(a^2 + (b + t)^2\big)^2} - \frac{b^5 - a^2 b^3}{(a^2 + b^2)^2}}{t}                                       \\
     & = \lim_{t \to 0 ; t \neq 0} \frac{\big((b + t)^5 - a^2 (b + t)^3\big) (a^2 + b^2)^2 - (b^5 - a^2 b^3) \big(a^2 + (b + t)^2\big)^2}{t \big(a^2 + (b + t)^2\big)^2 (a^2 + b^2)^2} \\
     & = \frac{-3 a^6 b^2 + 3 a^4 b^4 + 7 a^2 b^6 + b^8}{(a^2 + b^2)^4}
  \end{align*}
  and
  \begin{align*}
     & \frac{\partial}{\partial x} \frac{\partial f}{\partial y}(a, b)                                                                                                                         \\
     & = \lim_{t \to 0 ; t \neq 0} \frac{\frac{\partial f}{\partial y}\big((a, b) + t(1, 0)\big) - \frac{\partial f}{\partial y}(a, b)}{t}                                                     \\
     & = \lim_{t \to 0 ; t \neq 0} \frac{\frac{\partial f}{\partial y}(a + t, b) - \frac{\partial f}{\partial y}(a, b)}{t}                                                                     \\
     & = \lim_{t \to 0 ; t \neq 0} \frac{\frac{3 (a + t)^3 b^2 + (a + t) b^4}{\big((a + t)^2 + b^2\big)^2} - \frac{3 a^3 b^2 + a b^4}{(a^2 + b^2)^2}}{t}                                       \\
     & = \lim_{t \to 0 ; t \neq 0} \frac{\big(3 (a + t)^3 b^2 + (a + t) b^4\big) (a^2 + b^2)^2 - (3 a^3 b^2 + a b^4) \big((a + t)^2 + b^2\big)^2}{t \big((a + t)^2 + b^2\big)^2 (a^2 + b^2)^2} \\
     & = \frac{-3 a^6 b^2 + 3 a^4 b^4 + 7 a^2 b^6 + b^8}{(a^2 + b^2)^4}.
  \end{align*}
  Thus by \cref{6.3.7} \(\frac{\partial}{\partial y} \frac{\partial f}{\partial x}\) and \(\frac{\partial}{\partial x} \frac{\partial f}{\partial y}\) exist for all \((a, b) \in \R^2 \setminus \{(0, 0)\}\).
  Since
  \begin{align*}
     & \frac{\partial}{\partial y} \frac{\partial f}{\partial x}(0, 0)                                                                     \\
     & = \lim_{t \to 0 ; t \neq 0} \frac{\frac{\partial f}{\partial x}\big((0, 0) - t(0, 1)\big) - \frac{\partial f}{\partial x}(0, 0)}{t} \\
     & = \lim_{t \to 0 ; t \neq 0} \frac{\frac{t^5 - 0^2 t^3}{(0^2 + t^2)^2} - 0}{t}                                                       \\
     & = \lim_{t \to 0 ; t \neq 0} \frac{t^5}{t^5}                                                                                         \\
     & = 1
  \end{align*}
  and
  \begin{align*}
     & \frac{\partial}{\partial x} \frac{\partial f}{\partial y}(0, 0)                                                                     \\
     & = \lim_{t \to 0 ; t \neq 0} \frac{\frac{\partial f}{\partial y}\big((0, 0) - t(1, 0)\big) - \frac{\partial f}{\partial y}(0, 0)}{t} \\
     & = \lim_{t \to 0 ; t \neq 0} \frac{\frac{3 t^3 0^2 + t 0^4}{(t^2 + 0^2)^2} - 0}{t}                                                   \\
     & = 0,
  \end{align*}
  we know that both \(\frac{\partial}{\partial y} \frac{\partial f}{\partial x}\) and \(\frac{\partial}{\partial x} \frac{\partial f}{\partial y}\) exist at \((0, 0)\) but are not equal to each other.
  This does not contradict to \cref{6.5.4} since both \(\frac{\partial}{\partial y} \frac{\partial f}{\partial x}\) and \(\frac{\partial}{\partial x} \frac{\partial f}{\partial y}\) are not continuous at \((0, 0)\).
\end{proof}
\section{The contraction mapping theorem}\label{ii:sec:6.6}

\begin{defn}[Contraction]\label{ii:6.6.1}
  Let \((X, d)\) be a metric space, and let \(f : X \to X\) be a map.
  We say that \(f\) is a \emph{contraction} if we have \(d\big(f(x), f(y)\big) \leq d(x, y)\) for all \(x, y \in X\).
  We say that \(f\) is a \emph{strict contraction} if there exists a constant \(0 < c < 1\) such that \(d\big(f(x), f(y)\big) \leq c d(x, y)\) for all \(x, y \in X\);
  we call \(c\) the \emph{contraction constant} of \(f\).
\end{defn}

\begin{eg}\label{ii:6.6.2}
  The map \(f : \R \to \R\) defined by \(f(x) \coloneqq x + 1\) is a contraction but not a strict contraction.
  The map \(f : \R \to \R\) defined by \(f(x) \coloneqq x / 2\) is a strict contraction.
  The map \(f : [0, 1] \to [0, 1]\) defined by \(f(x) \coloneqq x - x^2\) is a contraction but not a strict contraction.
\end{eg}

\begin{proof}
  Since
  \[
    \forall x, y \in \R, \abs{(x + 1) - (y + 1)} = \abs{x - y} \leq \abs{x - y},
  \]
  by \cref{ii:6.6.1} we know that \(x \mapsto x + 1\) is a contraction.
  Suppose for sake of contradiction that \(x \mapsto x + 1\) is a strict contraction.
  Then there exists a \(c \in (0, 1)\) such that
  \[
    \forall x, y \in \R, \abs{(x + 1) - (y + 1)} \leq c \abs{x - y}.
  \]
  But we see that when \(x \neq y\), we have
  \[
    \abs{x - y} \leq c \abs{x - y} \implies 1 \leq c,
  \]
  which contradict to the fact that \(c \in (0, 1)\).
  Thus \(x \mapsto x + 1\) is not a strict contraction.

  Since
  \[
    \forall x, y \in \R, \abs{\dfrac{x}{2} - \dfrac{y}{2}} = \dfrac{1}{2} \abs{x - y} \leq \dfrac{1}{2} \abs{x - y},
  \]
  by \cref{ii:6.6.1} we know that \(x \mapsto \dfrac{x}{2}\) is a strict contraction.

  Since
  \begin{align*}
             & \forall x, y \in [0, 1], \begin{dcases}
                                          -1 \leq -x \leq 0 \\
                                          -1 \leq -y \leq 0
                                        \end{dcases} \\
    \implies & -2 \leq -x - y \leq 0                      \\
    \implies & -1 \leq 1 - x - y \leq 1                   \\
    \implies & 0 \leq \abs{1 - x - y} \leq 1,
  \end{align*}
  we know that
  \begin{align*}
    \forall x, y \in [0, 1], \abs{(x - x^2) - (y - y^2)} & = \abs{x - y - (x^2 - y^2)}    \\
                                                         & = \abs{x - y - (x - y)(x + y)} \\
                                                         & = \abs{(x - y)(1 - x - y)}     \\
                                                         & = \abs{x - y} \abs{1 - x - y}  \\
                                                         & \leq \abs{x - y}.
  \end{align*}
  Thus by \cref{ii:6.6.1} we know that \(x \mapsto x - x^2\) is a contraction on \([0, 1]\).
  Suppose for sake of contradiction that \(x \mapsto x - x^2\) is a strict contraction on \([0, 1]\).
  Then there exists a \(c \in (0, 1)\) such that
  \[
    \forall x, y \in \R, \abs{(x - x^2) - (y - y^2)} \leq c \abs{x - y}.
  \]
  But when \((x, y) = (\dfrac{1 - c}{2}, 0)\) we have
  \begin{align*}
    \abs{(x - x^2) - (y - y^2)} & = \abs{\dfrac{1 - c}{2} - \bigg(\dfrac{1 - c}{2}\bigg)^2 - (0 - 0^2)}                  \\
                                & = \dfrac{1 - c}{2} \abs{1 - \dfrac{1 - c}{2}}                                          \\
                                & = \dfrac{1 - c}{2} \dfrac{1 + c}{2}                                                    \\
                                & > \dfrac{1 - c}{2} \dfrac{c + c}{2}                                   & (c \in (0, 1)) \\
                                & = c \dfrac{1 - c}{2}                                                                   \\
                                & = c \abs{\dfrac{1 - c}{2} - 0}                                                         \\
                                & = c \abs{x - y},
  \end{align*}
  a contradiction.
  Thus \(x \mapsto x - x^2\) is not a strict contraction on \([0, 1]\).
\end{proof}

\begin{defn}[Fixed points]\label{ii:6.6.3}
  Let \(f : X \to X\) be a map, and \(x \in X\).
  We say that \(x\) is a \emph{fixed point} of \(f\) if \(f(x) = x\).
\end{defn}

\begin{note}
  Contractions do not necessarily have any fixed points;
  for instance, the map \(f : \R \to \R\) defined by \(f(x) = x + 1\) does not.
  However, it turns out that \emph{strict} contractions always do, at least when \(X\) is complete.
\end{note}

\begin{thm}[Contraction mapping theorem]\label{ii:6.6.4}
  Let \((X, d)\) be a metric space, and let \(f : X \to X\) be a strict contraction.
  Then \(f\) can have at most one fixed point.
  Moreover, if we also assume that \(X\) is non-empty and complete, then \(f\) has exactly one fixed point.
\end{thm}

\begin{proof}
  We first show that \(f\) can have at most one fixed point.
  Suppose for sake of contradiction that \(f\) has two fixed point \(x_1, x_2 \in X\).
  Since \(f\) is a strict contraction, by \cref{ii:6.6.1} we know that there exists a \(c \in (0, 1)\) such that
  \[
    d\big(f(x_1), f(x_2)\big) \leq c d(x_1, x_2).
  \]
  Since \(x_1, x_2\) are fixed points, by \cref{ii:6.6.3} we know that
  \[
    d(x_1, x_2) \leq c d(x_1, x_2)
  \]
  which implies \(1 \leq c\), a contradiction.
  Thus \(f\) can have at most one fixed point.

  Now we show that when \(X \neq \emptyset\) and \((X, d)\) is complete, \(f\) has exactly one fixed point.
  Let \(x_0 \in X\).
  Since \(f\) is a strict contraction, by \cref{ii:6.6.1} we know that there exists a \(c \in (0, 1)\) such that
  \[
    \forall x, y \in X, d\big(f(x), f(y)\big) \leq c d(x, y).
  \]
  Now we define a sequence \((x_n)_{n = 1}^\infty\) as follow:
  \[
    \forall n \in \Z^+, x_n = f(x_{n - 1}).
  \]
  We claim that
  \[
    \forall n \in \Z^+, d(x_{n + 1}, x_n) \leq c^n d(x_1, x_0).
  \]
  We induct on \(n\) to proof the claim.
  For \(n = 0\), we have
  \[
    d(x_1, d_0) \leq c^0 d(x_1, x_0) = d(x_1, x_0).
  \]
  Thus, the base case holds.
  Suppose inductively that \(d(x_{n + 1}, x_n) \leq c^n d(x_1, x_0)\) for some \(n \geq 0\).
  Then for \(n + 1\), we have
  \begin{align*}
    d(x_{n + 2}, x_{n + 1}) & = d\big(f(x_{n + 1}), f(x_n)\big) &  & \text{(by the definition of \((x_n)_{n = 1}^\infty\))} \\
                            & \leq c d(x_{n + 1}, x_n)          &  & \by{ii:6.6.1}                                          \\
                            & \leq c \cdot c^n d(x_1, x_0)                                                                  \\
                            & = c^{n + 1} d(x_1, x_0).
  \end{align*}
  This closes the induction.

  Next we claim that \((x_n)_{n = 1}^\infty\) is a Cauchy sequence in \((X, d)\).
  Let \(\varepsilon \in \R^+\).
  Observe that
  \begin{align*}
             & c \in (0, 1)                                                                                                                      \\
    \implies & \lim_{n \to \infty} c^n = 0                                                                                                       \\
    \implies & \exists N \in \Z^+ : \forall n \geq N, c^n < \dfrac{\varepsilon (1 - c)}{1 + d(x_1, x_0)}                                         \\
    \implies & \exists N \in \Z^+ : \forall n \geq N, \dfrac{d(x_1, x_0)}{1 - c} \leq \dfrac{1 + d(x_1, x_0)}{1 - c} < \dfrac{\varepsilon}{c^n}.
  \end{align*}
  Fix such \(N\).
  Let \(n, m \geq N\) and without the loss of generality suppose that \(n \leq m\).
  Since
  \begin{align*}
    d(x_n, x_m) & \leq \sum_{i = n}^m d(x_i, x_{i + 1})                    &                & \by{ii:1.1.2}[d]              \\
                & = \sum_{i = n}^m d(x_{i + 1}, x_i)                       &                & \by{ii:1.1.2}[c]              \\
                & \leq \sum_{i = n}^m c^i d(x_1, x_0)                      &                & \text{(from the claim above)} \\
                & = d(x_1, x_0) \bigg(\sum_{i = n}^m c^i\bigg)                                                              \\
                & = d(x_1, x_0) c^n \bigg(\sum_{i = 0}^{m - n} c^i\bigg)                                                    \\
                & \leq d(x_1, x_0) c^n \bigg(\sum_{i = 0}^\infty c^i\bigg) & (c \in (0, 1))                                 \\
                & = \dfrac{d(x_1, x_0) c^n}{1 - c}                         &                & \text{(by geometric series)}  \\
                & < \dfrac{\varepsilon c^n}{c^n}                           & (n \geq N)                                     \\
                & = \varepsilon
  \end{align*}
  and \(\varepsilon\) was arbitrary, we know that
  \[
    \forall \varepsilon \in \R^+, \exists N \in \Z^+ : \forall n, m \geq N, d(x_n, x_m) \leq \varepsilon.
  \]
  By \cref{ii:1.4.6} this means \((x_n)_{n = 1}^\infty\) is a Cauchy sequence in \((X, d)\).
  By hypothesis we know that \((X, d)\) is complete, thus by \cref{ii:1.4.10} there exists a \(y_0 \in X\) such that \(\lim_{n \to \infty} d(y_0, x_n) = 0\).

  Now we claim that \(y_0\) is the fixed point of \(f\).
  Suppose for sake of contradiction that \(y_0\) is not the fixed point of \(f\).
  By \cref{ii:6.6.3} this means \(f(y_0) \neq y_0\), in other words, \(d\big(f(y_0), y_0\big) > 0\).
  Since \((x_n)_{n = 1}^\infty\) converges to \(y_0\) in \(X\) with respect to \(d\), we know that
  \begin{align*}
             & \forall \varepsilon \in \R^+, \exists N \in \Z^+ : \forall n \geq N, d(y_0, x_n) < \varepsilon \\
    \implies & \exists N \in \Z^+ : \forall n \geq N, d(y_0, x_n) < \dfrac{d\big(f(y_0), y_0\big)}{2}.
  \end{align*}
  Fix such \(N\).
  But then we have
  \begin{align*}
             & \begin{dcases}
                 d(y_0, x_N) < \dfrac{d\big(f(y_0), y_0\big)}{2} \\
                 d(y_0, x_{N + 1}) < \dfrac{d\big(f(y_0), y_0\big)}{2}
               \end{dcases}                                                   \\
    \implies & \begin{dcases}
                 c d(y_0, x_N) < c \dfrac{d\big(f(y_0), y_0\big)}{2} \\
                 d(y_0, x_{N + 1}) < \dfrac{d\big(f(y_0), y_0\big)}{2}
               \end{dcases}                                                   \\
    \implies & \begin{dcases}
                 d\big(f(y_0), f(x_N)\big) < c d(y_0, x_N) < c \dfrac{d\big(f(y_0), y_0\big)}{2} \\
                 d(y_0, x_{N + 1}) < \dfrac{d\big(f(y_0), y_0\big)}{2}
               \end{dcases}                        \\
    \implies & \begin{dcases}
                 d\big(f(y_0), x_{N + 1}\big) < c d(y_0, x_N) < c \dfrac{d\big(f(y_0), y_0\big)}{2} \\
                 d(y_0, x_{N + 1}) < \dfrac{d\big(f(y_0), y_0\big)}{2}
               \end{dcases}                     \\
    \implies & d\big(f(y_0), y_0\big) \leq d\big(f(y_0), x_{N + 1}\big) + d(y_0, x_{N + 1})                           \\
             & < (c + 1) \dfrac{d\big(f(y_0), y_0\big)}{2} < d\big(f(y_0), y_0\big),                 & (c \in (0, 1))
  \end{align*}
  a contraction.
  Thus \(y_0\) is the fixed point of \(f\).
\end{proof}

\begin{rmk}\label{ii:6.6.5}
  The contraction mapping theorem is one example of a \emph{fixed point theorem}
  - a theorem which guarantees, assuming certain conditions, that a map will have a fixed point.
  There are a number of other fixed point theorems which are also useful.
  One amusing one is the so-called \emph{hairy ball theorem}, which (among other things) states that any continuous map \(f : S^2 \to S^2\) from the sphere \(S^2 \coloneqq \set{(x, y, z) \in \R^3 : x^2 + y^2 + z^2 = 1}\) to itself, must contain either a fixed point, or an anti-fixed point
  (a point \(x \in S^2\) such that \(f(x) = -x\)).
  A proof of this theorem can be found in any topology text;
  it is beyond the scope of this text.
\end{rmk}

\begin{note}
  Basically, \cref{ii:6.6.6} says that any map \(f\) on a ball which is a ``small'' perturbation (since \(g\) is a strict contraction) of the identity map (since \(f(x) = x + g(x)\)), remains one-to-one and cannot create any internal holes in the ball
  (there is a smaller ball contained in the origin ball such that every element in the smaller ball can be mapped by \(f\)).
\end{note}

\begin{lem}\label{ii:6.6.6}
  Let \(B(0, r)\) be a ball in \(\R^n\) centered at the origin, and let \(g : B(0, r) \to \R^n\) be a map such that \(g(0) = 0\) and
  \[
    \norm*{g(x) - g(y)} \leq \dfrac{1}{2} \norm*{x - y}
  \]
  for all \(x, y \in B(0, r)\)
  (here \(\norm*{x}\) denotes the length of \(x\) in \(\R^n\)).
  Then the function \(f : B(0, r) \to \R^n\) defined by \(f(x) \coloneqq x + g(x)\) is one-to-one, and furthermore the image \(f\big(B(0, r)\big)\) of this map contains the ball \(B(0, r / 2)\).
\end{lem}

\begin{proof}
  We first show that \(f\) is one-to-one.
  Suppose for sake of contradiction that we had two different points \(x, y \in B(0, r)\) such that \(f(x) = f(y)\).
  But then we would have \(x + g(x) = y + g(y)\), and hence
  \[
    \norm*{g(x) - g(y)} = \norm*{x - y}.
  \]
  The only way this can be consistent with our hypothesis \(\norm*{g(x) - g(y)} \leq \dfrac{1}{2} \norm*{x - y}\) is if \(\norm*{x - y} = 0\), i.e., if \(x = y\), a contradiction.
  Thus \(f\) is one-to-one.

  Now we show that \(f\big(B(0, r)\big)\) contains \(B(0, r / 2)\).
  Let \(y\) be any point in \(B(0, r / 2)\);
  our objective is to find a point \(x \in B(0, r)\) such that \(f(x) = y\), or in other words that \(x = y - g(x)\).
  So the problem is now to find a fixed point of the map \(x \mapsto y - g(x)\).

  Let \(F : B(0, r) \to B(0, r)\) denote the function \(F(x) \coloneqq y - g(x)\).
  Observe that if \(x \in B(0, r)\), then
  \[
    \norm*{F(x)} \leq \norm*{y} + \norm*{g(x)} \leq \dfrac{r}{2} + \norm*{g(x) - g(0)} \leq \dfrac{r}{2} + \dfrac{1}{2} \norm*{x - 0} < \dfrac{r}{2} + \dfrac{r}{2} = r,
  \]
  so \(F\) does indeed map \(B(0, r)\) to itself.
  The same argument shows that for a sufficiently small \(\varepsilon > 0\), \(F\) maps the closed ball \(\overline{B(0, r - \varepsilon)}\) to itself.
  Also, for any \(x, x'\) in \(B(0, r)\) we have
  \[
    \norm*{F(x) - F(x')} = \norm*{g(x') - g(x)} \leq \dfrac{1}{2} \norm*{x' - x}
  \]
  so \(F\) is a strict contraction on \(B(0, r)\), and hence on the complete space \(\overline{B(0, r - \varepsilon)}\) (see \cref{ii:1.5.7} and \cref{ii:1.5.5}).
  By the contraction mapping theorem, \(F\) has a fixed point, i.e., there exists an \(x\) such that \(x = y - g(x)\).
  But this means that \(f(x) = y\), as desired.
\end{proof}

\exercisesection

\begin{ex}\label{ii:ex:6.6.1}
  Let \(f : [a, b] \to [a, b]\) be a differentiable function of one variable such that \(\abs{f'(x)} \leq 1\) for all \(x \in [a, b]\).
  Prove that \(f\) is a contraction.
  If in addition \(\abs{f'(x)} < 1\) for all \(x \in [a, b]\) and \(f'\) is continuous, show that \(f\) is a strict contraction.
\end{ex}

\begin{proof}
  First we show that \(\abs{f'(x)} \leq 1\) for all \(x \in [a, b]\) implies \(f\) is a contraction.
  Let \(x, y \in [a, b]\).
  If \(x = y\), then we have
  \[
    \abs{f(x) - f(y)} = 0 \leq 0 = \abs{x - y}.
  \]
  So suppose that \(x \neq y\).
  By mean value theorem (Corollary 10.2.9 in Analysis I) we know that
  \begin{align*}
             & \exists c \in (x, y) \cup (y, x) : \dfrac{f(x) - f(y)}{x - y} = f'(c)                    \\
    \implies & \exists c \in (x, y) \cup (y, x) : \abs{\dfrac{f(x) - f(y)}{x - y}} = \abs{f'(c)} \leq 1 \\
    \implies & \abs{f(x) - f(y)} \leq \abs{x - y}.
  \end{align*}
  Thus by \cref{ii:6.6.1} \(f\) is a contraction.

  Now we show that \(\abs{f'(x)} < 1\) for all \(x \in [a, b]\) and \(f'\) is continuous implies \(f\) is a strict contraction.
  Since \(f'\) is continuous, by Proposition 9.6.7 in Analysis I we know that
  \begin{align*}
             & \exists x_{\min}, x_{\max} \in [a, b] : \forall x \in [a, b], f'(x_{\min}) \leq f'(x) \leq f'(x_{\max})                                  \\
    \implies & \exists x_{\min}, x_{\max} \in [a, b] : \forall x \in [a, b], \abs{f'(x)} \leq \max\big(\abs{f'(x_{\min})}, \abs{f'(x_{\max})}\big) < 1.
  \end{align*}
  Let \(x, y \in [a, b]\).
  If \(x = y\), then we have
  \begin{align*}
    \abs{f(x) - f(y)} & = 0                                                                       \\
                      & \leq 0                                                                    \\
                      & = \max\big(\abs{f'(x_{\min})}, \abs{f'(x_{\max})}\big) \cdot 0            \\
                      & = \max\big(\abs{f'(x_{\min})}, \abs{f'(x_{\max})}\big) \cdot \abs{x - y}.
  \end{align*}
  So suppose that \(x \neq y\).
  By mean value theorem (Corollary 10.2.9) we know that
  \begin{align*}
             & \exists c \in (x, y) \cup (y, x) : \dfrac{f(x) - f(y)}{x - y} = f'(c)                                                                    \\
    \implies & \exists c \in (x, y) \cup (y, x) : \abs{\dfrac{f(x) - f(y)}{x - y}} = \abs{f'(c)} < \max\big(\abs{f'(x_{\min})}, \abs{f'(x_{\max})}\big) \\
    \implies & \abs{f(x) - f(y)} \leq \max\big(\abs{f'(x_{\min})}, \abs{f'(x_{\max})}\big) \cdot \abs{x - y}.
  \end{align*}
  Thus by \cref{ii:6.6.1} \(f\) is a strict contraction.
\end{proof}

\begin{ex}\label{ii:ex:6.6.2}
  Show that if \(f : [a, b] \to \R\) is differentiable and is a contraction, then \(\abs{f'(x)} \leq 1\).
\end{ex}

\begin{proof}
  We have
  \begin{align*}
             & \forall x_0 \in [a, b], \forall x \in [a, b] \setminus \set{x_0}, \abs{f(x) - f(x_0)} \leq \abs{x - x_0}                &  & \by{ii:6.6.1} \\
    \implies & \forall x_0 \in [a, b], \forall x \in [a, b] \setminus \set{x_0}, \abs{\dfrac{f(x) - f(x_0)}{x - x_0}} \leq 1                              \\
    \implies & \forall x_0 \in [a, b], \lim_{x \to x_0 ; x \in [a, b] \setminus \set{x_0}} \abs{\dfrac{f(x) - f(x_0)}{x - x_0}} \leq 1                    \\
    \implies & \forall x_0 \in [a, b], f'(x_0) \leq 1.
  \end{align*}
\end{proof}

\begin{ex}\label{ii:ex:6.6.3}
  Give an example of a function \(f : [a, b] \to \R\) which is continuously differentiable and such that \(\abs{f(x) - f(y)} < \abs{x - y}\) for all distinct \(x, y \in [a, b]\), but such that \(\abs{f'(x)} = 1\) for at least one value of \(x \in [a, b]\).
\end{ex}

\begin{proof}
  Let \(f : [0, 0.5] \to \R\) be the function
  \[
    \forall x \in [0, 0.5], f(x) = x^2.
  \]
  Let \(x, y \in [0, 0.5]\) such that \(x \neq y\).
  Since \(x \neq y\), we know that \(x + y < 1\).
  Then we have
  \begin{align*}
    \abs{f(x) - f(y)} & = \abs{x^2 - y^2}                    \\
                      & = \abs{(x - y)(x + y)}               \\
                      & = \abs{x - y} (x + y)                \\
                      & < \abs{x - y}.         & (x + y < 1)
  \end{align*}
  Since \(f'(x) = 2x\), we know that \(\abs{f'(0.5)} = \abs{1} = 1\).
\end{proof}

\begin{ex}\label{ii:ex:6.6.4}
  Given an example of a function \(f : [a, b] \to \R\) which is a strict contraction but which is not differentiable for at least one point \(x\) in \([a, b]\).
\end{ex}

\begin{proof}
  Let \(f : [0, 1] \to \R\) be the function
  \[
    \forall x \in [0, 1], f(x) = \begin{dcases}
      \dfrac{x}{2} & \text{if } x \in [0.5, 1] \\
      \dfrac{x}{3} & \text{if } x \in [0, 0.5)
    \end{dcases}.
  \]
  Observe that
  \[
    \lim_{x \to 0.5+} f(x) = \dfrac{0.5}{2} \neq \dfrac{0.5}{3} = \lim_{x \to 0.5-} f(x).
  \]
  Thus \(f\) is not continuous at \(0.5\) and by Proposition 10.1.10 in Analysis I \(f\) is not differentiable at \(0.5\).
  Since
  \begin{align*}
             & \forall x, y \in [0, 1], \begin{dcases}
                                          \abs{f(x) - f(y)} = \dfrac{1}{2} \abs{x - y}    & \text{if } x, y \in [0.5, 1]                   \\
                                          \abs{f(x) - f(y)} = \dfrac{1}{3} \abs{x - y}    & \text{if } x, y \in [0, 0.5)                   \\
                                          \abs{f(x) - f(y)} = \dfrac{x}{2} - \dfrac{y}{3} & \text{if } x \in [0.5, 1] \land y \in [0, 0.5) \\
                                          \abs{f(x) - f(y)} = \dfrac{y}{2} - \dfrac{x}{3} & \text{if } x \in [0, 0.5) \land y \in [0.5, 1]
                                        \end{dcases}   \\
    \implies & \begin{dcases}
                 \abs{f(x) - f(y)} \leq \dfrac{1}{2} \abs{x - y}                            & \text{if } x, y \in [0.5, 1]                   \\
                 \abs{f(x) - f(y)} = \dfrac{1}{3} \abs{x - y} \leq \dfrac{1}{2} \abs{x - y} & \text{if } x, y \in [0, 0.5)                   \\
                 \abs{f(x) - f(y)} < \dfrac{1}{3} (x - y) \leq \dfrac{1}{2} \abs{x - y}     & \text{if } x \in [0.5, 1] \land y \in [0, 0.5) \\
                 \abs{f(x) - f(y)} < \dfrac{1}{3} (y - x) \leq \dfrac{1}{2} \abs{x - y}     & \text{if } x \in [0, 0.5) \land y \in [0.5, 1]
               \end{dcases} \\
    \implies & \abs{f(x) - f(y)} \leq \dfrac{1}{2} \abs{x - y},
  \end{align*}
  by \cref{ii:6.6.1} we know that \(f\) is a strict contraction.
\end{proof}

\begin{ex}\label{ii:ex:6.6.5}
  Verify the claims in \cref{ii:6.6.2}.
\end{ex}

\begin{proof}
  See \cref{ii:6.6.2}.
\end{proof}

\begin{ex}\label{ii:ex:6.6.6}
  Show that every contraction on a metric space \(X\) is necessarily continuous.
\end{ex}

\begin{proof}
  Let \((X, d)\) be a metric space, let \(f : X \to X\) be a contraction of \(X\) and let \(x_0 \in X\).
  We have
  \begin{align*}
             & \forall \varepsilon \in \R^+, \forall x \in X, d(x, x_0) < \varepsilon                    \\
    \implies & d\big(f(x), f(x_0)\big) \leq d(x, x_0) < \varepsilon.                  &  & \by{ii:6.6.1}
  \end{align*}
  By setting \(\delta = \varepsilon\) we see that
  \[
    \forall \varepsilon \in \R^+, \exists \delta \in \R^+ : \forall x \in X, d(x, x_0) < \delta \implies d\big(f(x), f(x_0)\big) < \varepsilon.
  \]
  Since \(x_0\) was arbitrary, by \cref{ii:2.1.1} this means \(f\) is continuous on \(X\) from \((X, d)\) to \((X, d)\).
\end{proof}

\begin{ex}\label{ii:ex:6.6.7}
  Prove \cref{ii:6.6.4}.
\end{ex}

\begin{proof}
  See \cref{ii:6.6.4}.
\end{proof}

\begin{ex}\label{ii:ex:6.6.8}
  Let \((X, d)\) be a complete metric space, and let \(f : X \to X\) and \(g : X \to X\) be two strict contractions on \(X\) with contraction coefficients \(c\) and \(c'\) respectively.
  From \cref{ii:6.6.4} we know that \(f\) has some fixed point \(x_0\), and \(g\) has some fixed point \(y_0\).
  Suppose we know that there is an \(\varepsilon > 0\) such that \(d\big(f(x), g(x)\big) \leq \varepsilon\) for all \(x \in X\)
  (i.e., \(f\) and \(g\) are within \(\varepsilon\) of each other in the uniform metric).
  Show that \(d(x_0, y_0) \leq \varepsilon / \big(1 - \min(c, c')\big)\).
  Thus nearby contractions have nearby fixed points.
\end{ex}

\begin{proof}
  We have
  \begin{align*}
    d(x_0, y_0) & = d\big(f(x_0), g(y_0)\big)                                &  & \by{ii:6.6.3}          \\
                & \leq d\big(f(x_0), g(x_0)\big) + d\big(g(x_0), g(y_0)\big) &  & \by{ii:1.1.2}[d]       \\
                & \leq \varepsilon + d\big(g(x_0), g(y_0)\big)               &  & \text{(by hypothesis)} \\
                & \leq \varepsilon + c' d(x_0, y_0)                          &  & \by{ii:6.6.1}
  \end{align*}
  and
  \begin{align*}
    d(x_0, y_0) & = d\big(f(x_0), g(y_0)\big)                                &  & \by{ii:6.6.3}          \\
                & \leq d\big(f(x_0), f(y_0)\big) + d\big(f(y_0), g(y_0)\big) &  & \by{ii:1.1.2}[d]       \\
                & \leq d\big(f(x_0), f(y_0)\big) + \varepsilon               &  & \text{(by hypothesis)} \\
                & \leq c d(x_0, y_0) + \varepsilon.                          &  & \by{ii:6.6.1}
  \end{align*}
  Thus
  \begin{align*}
             & \begin{dcases}
                 d(x_0, y_0) \leq \varepsilon + \max(c, c') \cdot d(x_0, y_0) \\
                 d(x_0, y_0) \leq \varepsilon + \min(c, c') \cdot d(x_0, y_0)
               \end{dcases} \\
    \implies & \begin{dcases}
                 \big(1 - \max(c, c')\big) d(x_0, y_0) \leq \varepsilon \\
                 \big(1 - \min(c, c')\big) d(x_0, y_0) \leq \varepsilon
               \end{dcases}       \\
    \implies & \begin{dcases}
                 d(x_0, y_0) \leq \dfrac{\varepsilon}{1 - \max(c, c')} \\
                 d(x_0, y_0) \leq \dfrac{\varepsilon}{1 - \min(c, c')}
               \end{dcases}.
  \end{align*}
\end{proof}

\section{The inverse function theorem in several variable calculus}\label{sec:6.7}

\begin{note}
  We recall the inverse function theorem in single variable calculus (Theorem 10.4.2 in Analysis I), which asserts that if a function \(f : \R \to \R\) is invertible, differentiable, and \(f'(x_0)\) is non-zero, then \(f^{-1}\) is differentiable at \(f(x_0)\), and
  \[
    (f^{-1})'\big(f(x_0)\big) = \dfrac{1}{f'(x_0)}.
  \]

  In fact, one can say something even when \(f'\) is not invertible, as long as we know that \(f\) is \emph{continuously} differentiable.
  If \(f'(x_0)\) is non-zero, then \(f'(x_0)\) must be either strictly positive or strictly negative, which implies (since we are assuming \(f'\) to be continuous) that \(f'(x)\) is either strictly positive for \(x\) near \(x_0\), or strictly negative for \(x\) near \(x_0\).
  In particular, \(f\) must be either strictly increasing near \(x_0\), or strictly decreasing near \(x_0\).
  In either case, \(f\) will become invertible if we restrict the domain and codomain of \(f\) to be sufficiently close to \(x_0\) and to \(f(x_0)\) respectively.
  (The technical terminology for this is that \(f\) is \emph{locally invertible near \(x_0\)}.)
\end{note}

\begin{lem}\label{6.7.1}
  Let \(T : \R^n \to \R^n\) be a linear transformation which is also invertible.
  Then the inverse transformation \(T^{-1} : \R^n \to \R^n\) is also linear.
\end{lem}

\begin{proof}
  Let \(x, y \in \R^n\) and let \(c \in \R\).
  We have
  \begin{align*}
    T^{-1}(x + y) & = T^{-1}\Big(T\big(T^{-1}(x)\big) + T\big(T^{-1}(y)\big)\Big)                 \\
                  & = T^{-1}\Big(T\big(T^{-1}(x) + T^{-1}(y)\big)\Big)            &  & \by{6.1.6} \\
                  & = T^{-1}(x) + T^{-1}(y)
  \end{align*}
  and
  \begin{align*}
    T^{-1}(cx) & = T^{-1}\Big(c T\big(T^{-1}(x)\big)\Big)                 \\
               & = T^{-1}\Big(T\big(c T^{-1}(x)\big)\Big) &  & \by{6.1.6} \\
               & = c T^{-1}(x).
  \end{align*}
  Thus by \cref{6.1.6} \(T^{-1}\) is a linear transformation.
\end{proof}

\begin{thm}[Inverse function theorem]\label{6.7.2}
  Let \(E\) be an open subset of \(\R^n\), and let \(f : E \to \R^n\) be a function which is continuously differentiable on \(E\).
  Suppose \(x_0 \in E\) is such that the linear transformation \(f'(x_0) : \R^n \to \R^n\) is invertible.
  Then there exists an open set \(U\) in \(E\) containing \(x_0\), and an open set \(V\) in \(\R^n\) containing \(f(x_0)\), such that \(f\) is a bijection from \(U\) to \(V\).
  In particular, there is an inverse map \(f^{-1} : V \to U\).
  Furthermore, this inverse map is differentiable at \(f(x_0)\), and
  \[
    (f^{-1})' \big(f(x_0)\big) = \big(f'(x_0)\big)^{-1}.
  \]
\end{thm}

\begin{proof}
  We first observe that once we know the inverse map \(f^{-1}\) is differentiable, the formula \((f^{-1})' \big(f(x_0)\big) = \big(f'(x_0)\big)^{-1}\) is automatic.
  This comes from starting with the identity
  \[
    I = f^{-1} \circ f
  \]
  on \(U\), where \(I : \R^n \to \R^n\) is the identity map \(I(x) \coloneqq x\), and then differentiating both sides using the chain rule at \(x_0\) to obtain
  \[
    I'(x_0) = (f^{-1})' \big(f(x_0)\big) \circ f'(x_0).
  \]
  Since \(I'(x_0) = I\), we thus have \((f^{-1})' \big(f(x_0)\big) = \big(f'(x_0)\big)^{-1}\) as desired.

  We remark that this argument shows that if \(f'(x_0)\) is \emph{not} invertible, then there is no way that an inverse \(f^{-1}\) can exist and be differentiable at \(f(x_0)\).

  Next, we observe that it suffices to prove the theorem under the additional assumption \(f(x_0) = 0\).
  The general case then follows from the special case by replacing \(f\) by a new function \(\tilde{f}(x) \coloneqq f(x) - f(x_0)\) and then applying the special case to \(\tilde{f}\)
  (note that \(V\) will have to shift by \(f(x_0)\)).
  Note that if \(V_f = \set{y \in \R^n : y - f(x_0) \in V}\), then
  \[
    \begin{dcases}
      \tilde{f} : U \to V \\
      \tilde{f}^{-1} : V \to U
    \end{dcases} \implies \begin{dcases}
      f : U \to V_f \\
      f^{-1} : V_f \to U
    \end{dcases}
  \]
  (one can show that \(f\) is bijective using proof by contradiction)
  and thus
  \begin{align*}
             & \forall x \in U, f(x) = y                                                                                                          \\
    \implies & f^{-1}(y) = x = \tilde{f}^{-1}\big(\tilde{f}(x)\big) = \tilde{f}^{-1}\big(f(x) - f(x_0)\big) = \tilde{f}^{-1}\big(y - f(x_0)\big).
  \end{align*}
  Henceforth we will always assume \(f(x_0) = 0\).

  In a similar manner, one can make the assumption \(x_0 = 0\).
  The general case then follows from this case by replacing \(f\) by a new function
  \(\tilde{f}(x) \coloneqq f(x + x_0)\) and applying the special case to \(\tilde{f}\)
  (note that \(E\) and \(U\) will have to shift by \(x_0\)).
  Note that if \(U_f = \set{x \in E : x - x_0 \in U}\), then
  \[
    \begin{dcases}
      \tilde{f} : U \to V \\
      \tilde{f}^{-1} : V \to U
    \end{dcases} \implies \begin{dcases}
      f : U_f \to V \\
      f^{-1} : V \to U_f
    \end{dcases}
  \]
  (one can show that \(f\) is bijective using proof by contradiction)
  and thus
  \begin{align*}
             & \forall x \in U, \tilde{f}(x) = f(x + x_0) = y                                                                                         \\
    \implies & f^{-1}(y) = x + x_0 = \tilde{f}^{-1}\big(\tilde{f}(x)\big) + x_0 = \tilde{f}^{-1}\big(f(x + x_0)\big) + x_0 = \tilde{f}^{-1}(y) + x_0.
  \end{align*}
  Henceforth we will always assume \(x_0 = 0\).
  Thus we now have that \(f(0) = 0\) and that \(f'(0)\) is invertible.

  Finally, one can assume that \(f'(0) = I\), where \(I : \R^n \to \R^n\) is the identity transformation \(I(x) = x\).
  The general case then follows from this case by replacing \(f\) with a new function \(\tilde{f} : E \to \R^n\) defined by \(\tilde{f}(x) \coloneqq \big(f'(0)\big)^{-1} \big(f(x)\big)\), and applying the special case to this case.
  Note from \cref{6.7.1} that \(\big(f'(0)\big)^{-1}\) is a linear transformation.
  In particular, we note that \(\tilde{f}(0) = 0\) and that
  \begin{align*}
    \tilde{f}'(0) & = \Big(\big(f'(0)\big)^{-1}\Big)'\big(f(0)\big) \circ f'(0) &  & \by{6.4.1}    \\
                  & = \big(f'(0)\big)^{-1} \circ f'(0)                          &  & \by{ex:6.4.1} \\
                  & = I,
  \end{align*}
  so by the special case of the inverse function theorem we know that there exists an open set \(U'\) containing \(0\), and an open set \(V'\) containing \(0\), such that \(\tilde{f}\) is a bijection from \(U'\) to \(V'\), and that \(\tilde{f}^{-1} : V' \to U'\) is differentiable at \(0\) with derivative \(I\).
  But we have
  \begin{align*}
             & \tilde{f}(x) = \big(f'(0)\big)^{-1} \big(f(x)\big)                                 \\
    \implies & f'(0) \big(\tilde{f}(x)\big) = f'(0) \Big(\big(f'(0)\big)^{-1} \big(f(x)\big)\Big) \\
    \implies & f(x) = f'(0) \big(\tilde{f}(x)\big),
  \end{align*}
  and hence \(f\) is a bijection from \(U'\) to \(f'(0)(V')\)
  (note that \(f'(0)\) is also a bijection).
  Since \(f'(0)\) and its inverse are both continuous, \(f'(0)(V')\) is open (see \cref{2.1.5}(a)(c)), and it certainly contains \(0\).
  Now consider the inverse function \(f^{-1} : f'(0)(V') \to U'\).
  Note that
  \begin{align*}
             & f = f'(0) \circ \tilde{f}                                                             \\
    \implies & f^{-1} = \tilde{f}^{-1} \circ \big(f'(0)\big)^{-1}                                    \\
    \implies & \forall y \in f'(0)(V'), f^{-1}(y) = \tilde{f}^{-1}\Big(\big(f'(0)\big)^{-1}(y)\Big).
  \end{align*}
  In particular we see that \(f^{-1}\) is differentiable at \(0\).

  So all we have to do now is prove the inverse function theorem in the special case, when \(x_0 = 0\), \(f(x_0) = 0\), and \(f'(x_0) = I\).
  Let \(g : E \to \R^n\) denote the function \(g(x) = f(x) - x\).
  Then \(g(0) = 0\) and \(g'(0) = 0\).
  In particular
  \[
    \dfrac{\partial g}{\partial x_j}(0) = 0
  \]
  for \(j = 1, \dots, n\).
  Since \(g\) is continuously differentiable, there thus exists a ball \(B(0, r)\) in \(E\) such that
  \[
    \norm*{\dfrac{\partial g}{\partial x_j}(x)} \leq \dfrac{1}{2 n^2}
  \]
  for all \(x \in B(0, r)\).
  (There is nothing particularly special about \(\dfrac{1}{2 n^2}\), we just need a nice small number here.)
  In particular, for any \(x \in B(0, r)\) and \(v = (v_1, \dots, v_n)\) we have
  \begin{align*}
    \norm*{D_v g(x)} & = \norm*{\sum_{j = 1}^n v_j \dfrac{\partial g}{\partial x_j} (x)}         \\
                     & \leq \sum_{j = 1}^n \abs{v_j} \norm*{\dfrac{\partial g}{\partial x_j}(x)} \\
                     & \leq \sum_{j = 1}^n \norm*{v} \dfrac{1}{2 n^2}                            \\
                     & \leq \dfrac{1}{2n} \norm*{v}.
  \end{align*}
  But now for any \(x, y \in B(0, r)\), we have by the fundamental theorem of calculus
  \begin{align*}
    g(y) - g(x) & = g\big(x + t(y - x)\big) \big|_{t = 0}^{t = 1}        \\
                & = \int_0^1 \dfrac{d}{dt} g\big(x + t(y - x)\big) \; dt \\
                & = \int_0^1 D_{y - x} g\big(x + t(y - x)\big) \; dt
  \end{align*}
  where the integral of a vector-valued function is defined by integrating each component separately.
  By the previous remark, the vectors \(D_{y - x} g\big(x + t(y - x)\big)\) have a magnitude of at most \(\dfrac{1}{2n} \norm*{y - x}\).
  Thus every component of these vectors has magnitude at most \(\dfrac{1}{2n} \norm*{y - x}\).
  Thus every component of \(g(y) - g(x)\) has magnitude at most \(\dfrac{1}{2n} \norm*{y - x}\), and hence \(g(y) - g(x)\) itself has magnitude at most \(\dfrac{1}{2} \norm*{y - x}\)
  (actually, it will be substantially less than this, but this bound will be enough for our purposes).
  In other words, \(g\) is a contraction.
  By \cref{6.6.6}, the map \(f = g + I\) is thus one-to-one on \(B(0, r)\), and the image \(f\big(B(0, r)\big)\) contains \(B(0, \dfrac{r}{2})\).
  In particular we have an inverse map \(f^{-1} : B(0, \dfrac{r}{2}) \to B(0, r)\) defined on \(B(0, \dfrac{r}{2})\).

  Applying the contraction bound with \(y = 0\) we obtain in particular that
  \[
    \norm*{g(x)} \leq \dfrac{1}{2} \norm*{x}
  \]
  for all \(x \in B(0, r)\), and so by the triangle inequality
  \[
    \dfrac{1}{2} \norm*{x} \leq \norm*{f(x)} \leq \dfrac{3}{2} \norm*{x}
  \]
  for all \(x \in B(0, r)\).

  Now we set \(V \coloneqq B(0, \dfrac{r}{2})\) and \(U \coloneqq f^{-1}(V) \cap B(0, r)\).
  Then by construction \(f\) is a bijection from \(U\) to \(V\).
  \(V\) is clearly open, and \(U\) is also open since \(f\) is continuous.
  (Notice that if a set is open relative to \(B(0, r)\), then it is open in \(\R^n\) as well.)
  Now we want to show that \(f^{-1} : V \to U\) is differentiable at \(0\) with derivative \(I^{-1} = I\).
  In other words, we wish to show that
  \[
    \lim_{x \to 0 ; x \in V \setminus \set{0}} \dfrac{\norm*{f^{-1}(x) - f^{-1}(0) - I(x - 0)}}{\norm*{x}} = 0.
  \]
  Since \(f(0) = 0\), we have \(f^{-1}(0) = 0\), and the above simplifies to
  \[
    \lim_{x \to 0 ; x \in V \setminus \set{0}} \dfrac{\norm*{f^{-1}(x) - x}}{\norm*{x}} = 0.
  \]
  Let \((x_n)_{n = 1}^\infty\) be any sequence in \(V \setminus \set{0}\) that converges to \(0\).
  By \cref{3.1.5}(b), it suffices to show that
  \[
    \lim_{n \to \infty} \dfrac{\norm*{f^{-1}(x_n) - x_n}}{\norm*{x_n}} = 0.
  \]
  Write \(y_n \coloneqq f^{-1}(x_n)\).
  Then \(y_n \in B(0, r)\) and \(x_n = f(y_n)\).
  In particular we have
  \[
    \dfrac{1}{2} \norm*{y_n} \leq \norm*{x_n} \leq \dfrac{3}{2} \norm*{y_n}
  \]
  and so since \(\norm*{x_n}\) goes to \(0\), \(\norm*{y_n}\) goes to \(0\) also, and their ratio remains bounded.
  It will thus suffice to show that
  \[
    \lim_{n \to \infty} \dfrac{\norm*{y_n - f(y_n)}}{\norm*{y_n}} = 0.
  \]
  But since \(y_n\) is going to \(0\), and \(f\) is differentiable at \(0\), we have
  \[
    \lim_{n \to \infty} \dfrac{\norm*{f(y_n) - f(0) - f'(0)(y_n - 0)}}{\norm*{y_n}} = 0
  \]
  as desired (since \(f(0) = 0\) and \(f'(0) = I\)).
\end{proof}

\begin{note}
  The inverse function theorem gives a useful criterion for when a function is (locally) invertible at a point \(x_0\)
  - all we need is for its derivative \(f'(x_0)\) to be invertible
  (and then we even get further information, for instance we can compute the derivative of \(f^{-1}\) at \(f(x_0)\)).
  Of course, this begs the question of how one can tell whether the linear transformation \(f'(x_0)\) is invertible or not.
  Recall that we have \(f'(x_0) = L_{D f(x_0)}\), so by \cref{6.1.13,6.1.16} we see that the linear transformation \(f'(x_0)\) is invertible iff the matrix \(D f(x_0)\) is.
  There are many ways to check whether a matrix such as \(D f(x_0)\) is invertible;
  for instance, one can use determinants, or alternatively Gaussian elimination methods.
  We will not pursue this matter here, but refer the reader to any linear algebra text.
\end{note}

\begin{note}
  If \(f'(x_0)\) exists but is non-invertible, then the inverse function theorem does not apply.
  In such a situation it is not possible for \(f^{-1}\) to exist and be differentiable at \(f(x_0)\);
  this was remarked in the proof of \cref{6.7.2}.
  But it is still possible for \(f\) to be invertible.
  For instance, the single-variable function \(f : \R \to \R\) defined by \(f(x) = x^3\) is invertible despite \(f'(0)\) not being invertible.
\end{note}

\exercisesection

\begin{ex}\label{ex:6.7.1}
  Let \(f : \R \to \R\) be the function defined by \(f(x) \coloneqq x + x^2 \sin(1 / x^4)\) for \(x \neq 0\) and \(f(0) \coloneqq 0\).
  Show that \(f\) is differentiable and \(f'(0) = 1\), but \(f\) is not increasing on any open set containing \(0\).
\end{ex}

\begin{proof}
  Let \(x \in \R \setminus \set{0}\).
  Then we have
  \begin{align*}
             & (x \mapsto x^{-4})' = (-4) x^{-5}                                               \\
    \implies & \big(x \mapsto \sin(x^{-4})\big)' = (-4) x^{-5} \cos(x^{-4})                    \\
    \implies & \big(x \mapsto x^2 \sin(x^{-4})\big)' = 2x \sin(x^{-4}) - 4 x^{-3} \cos(x^{-4}) \\
    \implies & f'(x) = 1 + 2 x \sin(x^{-4}) - 4 x^{-3} \cos(x^{-4}).
  \end{align*}
  Observe that
  \[
    \forall x \in \R \setminus \set{0}, \abs{x \sin(x^{-4})} = \abs{x} \abs{\sin(x^{-4})} \leq \abs{x} \cdot 1.
  \]
  Thus we have
  \[
    \lim_{x \to 0 ; x \in \R \setminus \set{0}} x = 0 \implies \lim_{x \to 0 ; x \in \R \setminus \set{0}} x \sin(x^{-4}) = 0
  \]
  and by squeeze test
  \begin{align*}
     & \lim_{x \to 0 ; x \in \R \setminus \set{0}} \dfrac{f(x) - f(0)}{x - 0}        \\
     & = \lim_{x \to 0 ; x \in \R \setminus \set{0}} \dfrac{x + x^2 \sin(x^{-4})}{x} \\
     & = \lim_{x \to 0 ; x \in \R \setminus \set{0}} 1 + x \sin(x^{-4})              \\
     & = 1.
  \end{align*}
  We conclude that \(f\) is differentiable on \(\R\) and \(f'(0) = 1\).

  Let \(E\) be an open set in \(\R\) containing \(0\).
  By \cref{1.2.15}(a) we know that
  \[
    \exists r \in \R^+ : B(0, r) \subseteq E \implies (-r, r) \subseteq E.
  \]
  Fix such \(r\).
  Since
  \begin{align*}
             & \lim_{n \to \infty} \dfrac{1}{n} = 0                                     \\
    \implies & \lim_{n \to \infty} \dfrac{1}{2 n \pi} = 0                               \\
    \implies & \lim_{n \to \infty} \sqrt[4]{\dfrac{1}{2 n \pi}} = 0                     \\
    \implies & \exists N \in \Z^+ : \forall n \geq N, \sqrt[4]{\dfrac{1}{2 n \pi}} < r,
  \end{align*}
  by fixing such \(N\) we know that
  \begin{align*}
             & \sqrt[4]{\dfrac{1}{2 N \pi}} \in (-r, r) \subseteq E                                                                                     \\
    \implies & f'\bigg(\sqrt[4]{\dfrac{1}{2 N \pi}}\bigg) = 1 + 2 \sqrt[4]{\dfrac{1}{2 N \pi}} \sin(2 N \pi) - 4 (2 N \pi)^{\dfrac{3}{4}} \cos(2 N \pi) \\
             & = 1 - 4 (2 N \pi)^{\dfrac{3}{4}} \leq 1 - 4 = -3 < 0.
  \end{align*}
  Thus \(f\) is not increasing at \(\sqrt[4]{\dfrac{1}{2 N \pi}}\), and not increasing on \(E\).
\end{proof}

\begin{ex}\label{ex:6.7.2}
  Prove \cref{6.7.1}.
\end{ex}

\begin{proof}
  See \cref{6.7.1}.
\end{proof}

\begin{ex}\label{ex:6.7.3}
  Let \(f : \R^n \to \R^n\) be a continuously differentiable function such that \(f'(x)\) is an invertible linear transformation for every \(x \in \R^n\).
  Show that whenever \(V\) is an open set in \(\R^n\), that \(f(V)\) is also open.
\end{ex}

\begin{proof}
  Let \(d = d_{l^2}|_{\R^n \times \R^n}\), let \(V\) be an open set in \((\R^n, d)\) and let \(y \in f(V)\).
  We know that there exists a \(x \in V\) such that \(f(x) = y\).
  Since \(f\) is continuously differentiable on \(\R^n\) and \(f'(x)\) is invertible, by inverse function theorem (\cref{6.7.2}) we know that
  \[
    \exists U, W \subseteq \R^n : \begin{dcases}
      U, W \text{ are open sets in } (\R^n, d) \\
      x \in U                                  \\
      y \in W                                  \\
      f : U \to W \text{ is a bijection}       \\
      (f^{-1})'(y) = (f^{-1})'\big(f(x)\big) = \big(f'(x)\big)^{-1}
    \end{dcases}.
  \]
  Fix such \(U, W\).
  Since \(V\) is open in \((\R^n, d)\), we know that there exists a \(r \in \R^+\) such that \(B_{(\R^n, d)}(x, r) \subseteq V\).
  Fix such \(r\).
  By \cref{1.2.15}(f) we know that \(U \cap B_{(\R^n, d)}(x, r)\) is open in \((\R^n, d)\).
  Since \(U \cap B_{(\R^n, d)}(x, r) \subseteq U\), we know that \(f\) is a bijection from \(U \cap B_{(\R^n, d)}(x, r)\) to \(f\big(U \cap B_{(\R^n, d)}(x, r)\big)\).
  By hypotheses we know that \(f^{-1}\) is differentiable on \(f\big(U \cap B_{(\R^n, d)}(x, r)\big)\), by \cref{ex:6.4.2} we know that \(f^{-1}\) is continuous \(f\big(U \cap B_{(\R^n, d)}(x, r)\big)\).
  Thus by \cref{2.1.5}(a)(c) we know that \(f\big(U \cap B_{(\R^n, d)}(x, r)\big)\) is open in \((\R, d)\).
  Now we have
  \begin{align*}
             & x \in U \cap B_{(\R^n, d)}(x, r)\big)                                                                                                                                \\
    \implies & y \in f\big(U \cap B_{(\R^n, d)}(x, r)\big)                                                                                                                          \\
    \implies & \exists r' \in \R^+ : B_{(\R^n, d)}(y, r') \subseteq f\big(U \cap B_{(\R^n, d)}(x, r)\big) &                                          & \text{(by \cref{1.2.15}(a))} \\
    \implies & \exists r' \in \R^+ : B_{(\R^n, d)}(y, r') \subseteq f(V).                                 & (U \cap B_{(\R^n, d)}(x, r) \subseteq V)
  \end{align*}
  Since \(y\) is arbitrary, we know that \(f(V)\) is open in \((\R^n, d)\).
  Since \(V\) is arbitrary, we know that if \(V\) is an open set in \((\R^n, d)\), then \(f(V)\) is also an open set in \((\R^n, d)\).
\end{proof}

\begin{ex}\label{ex:6.7.4}
  Let the notation and hypotheses be as in \cref{6.7.2}.
  Show that, after shrinking the open sets \(U, V\) if necessary (while still having \(x_0 \in U\), \(f(x_0) \in V\) of course), the derivative map \(f'(x)\) is invertible for all \(x \in U\), and that the inverse map \(f^{-1}\) is differentiable at every point of \(V\) with \((f^{-1})' \big(f(x)\big) = \big(f'(x)\big)^{-1}\) for all \(x \in U\).
  Finally, show that \(f^{-1}\) is continuously differentiable on \(V\).
\end{ex}

\section{The implicit function theorem}\label{sec:6.8}

\begin{note}
  Any function \(g : \R^n \to \R\) gives rise to a graph \(\Big\{\big(x, g(x)\big) : x \in \R^n\Big\}\) in \(\R^{n + 1}\), which in general looks like some sort of \(n\)-dimensional surface in \(\R^{n + 1}\)
  (the technical term for this is a \emph{hypersurface}).
  Conversely, one may ask which hypersurfaces are actually graphs of some function, and whether that function is continuous or differentiable.
\end{note}

\begin{note}
  If the hypersurface is given geometrically, then one can again invoke the vertical line test to work out whether it is a graph or not.
  But what if the hypersurface is given algebraically, or more generally, the hypersurface is given as some function?
  In this case, it is still possible to say whether the hypersurface is a graph, locally at least, by means of the \emph{implicit function theorem}.
\end{note}

\begin{thm}[Implicit function theorem]\label{6.8.1}
  Let \(E\) be an open subset of \(\R^n\), let \(f : E \to \R\) be continuously differentiable, and let \(y = (y_1, \dots, y_n)\) be a point in \(E\) such that \(f(y) = 0\) and \(\dfrac{\partial f}{\partial x_n}(y) \neq 0\).
  Then there exists an open subset \(U\) of \(\R^{n - 1}\) containing \((y_1, \dots, y_{n - 1})\), an open subset \(V\) of \(E\) containing \(y\), and a function \(g : U \to \R\) such that \(g(y_1, \dots, y_{n - 1}) = y_n\), and
  \begin{align*}
     & \{(x_1, \dots, x_n) \in V : f(x_1, \dots, x_n) = 0\}                                                     \\
     & = \Big\{\big(x_1, \dots, x_{n - 1}, g(x_1, \dots, x_{n - 1})\big) : (x_1, \dots, x_{n - 1}) \in U\Big\}.
  \end{align*}
  In other words, the set \(\{x \in V : f(x) = 0\}\) is a graph of a function over \(U\).
  Moreover, \(g\) is differentiable at \((y_1, \dots, y_{n - 1})\), and we have
  \[
    \dfrac{\partial g}{\partial x_j}(y_1, \dots, y_{n - 1}) = -\dfrac{\partial f}{\partial x_j}(y) / \dfrac{\partial f}{\partial x_n}(y) \tag{6.1}\label{eq 6.1}
  \]
  for all \(1 \leq j \leq n - 1\).
\end{thm}

\begin{proof}
  This theorem looks somewhat fearsome, but actually it is a fairly quick consequence of the inverse function theorem.
  Let \(F : E \to \R^n\) be the function
  \[
    F(x_1, \dots, x_n) \coloneqq \big(x_1, \dots, x_{n - 1}, f(x_1, \dots, x_n)\big).
  \]
  This function is continuously differentiable.
  Also note that
  \[
    F(y) = (y_1, \dots, y_{n - 1}, 0)
  \]
  and
  \begin{align*}
    D F(y) & = \bigg(\dfrac{\partial F}{\partial x_1}(y)^\top, \dfrac{\partial F}{\partial x_2}(y)^\top, \dots, \dfrac{\partial F}{\partial x_n}(y)^\top\bigg)                                          \\
           & = \begin{pmatrix}
                 1                                   & 0                                   & \dots  & 0                                         & 0                                   \\
                 0                                   & 1                                   & \dots  & 0                                         & 0                                   \\
                 \vdots                              & \vdots                              & \ddots & \vdots                                    & \vdots                              \\
                 0                                   & 0                                   & \dots  & 1                                         & 0                                   \\
                 \dfrac{\partial f}{\partial x_1}(y) & \dfrac{\partial f}{\partial x_2}(y) & \dots  & \dfrac{\partial f}{\partial x_{n - 1}}(y) & \dfrac{\partial f}{\partial x_n}(y)
               \end{pmatrix}.
  \end{align*}
  Since \(\dfrac{\partial f}{\partial x_n}(y)\) is assumed by hypothesis to be non-zero, this matrix is invertible;
  this can be seen either by computing the determinant, or using row reduction, or by computing the inverse explicitly, which is
  \[
    D F(y)^{-1} = \begin{pmatrix}
      1                                        & 0                                        & \dots  & 0                                              & 0      \\
      0                                        & 1                                        & \dots  & 0                                              & 0      \\
      \vdots                                   & \vdots                                   & \ddots & \vdots                                         & \vdots \\
      0                                        & 0                                        & \dots  & 1                                              & 0      \\
      -\dfrac{\partial f}{\partial x_1}(y) / a & -\dfrac{\partial f}{\partial x_2}(y) / a & \dots  & -\dfrac{\partial f}{\partial x_{n - 1}}(y) / a & 1 / a
    \end{pmatrix},
  \]
  where we have written \(a = \dfrac{\partial f}{\partial x_n}(y)\) for short.
  Thus the inverse function theorem (\cref{6.7.2}) applies, and we can find an open set \(V\) in \(E\) containing \(y\), and an open set \(W\) in \(\R^n\) containing \(F(y) = (y_1, \dots, y_{n - 1}, 0)\), such that \(F\) is a bijection from \(V\) to \(W\), and that \(F^{-1}\) is differentiable at \((y_1, \dots, y_{n - 1}, 0)\).

  Let us write \(F^{-1}\) in co-ordinates as
  \[
    F^{-1}(x) = \big(h_1(x), h_2(x), \dots, h_n(x)\big)
  \]
  where \(x \in W\).
  Since \(F\big(F^{-1}(x)\big) = x\), we have \(h_j(x_1, \dots, x_n) = x_j\) for all \(1 \leq j \leq n - 1\) and \(x \in W\), and
  \[
    f\big(x_1, \dots, x_{n - 1}, h_n(x_1, \dots, x_n)\big) = x_n.
  \]
  Also, \(h_n\) is differentiable at \((y_1, \dots, y_{n - 1}, 0)\) since \(F^{-1}\) is.

  Now we set \(U \coloneqq \{(x_1, \dots, x_{n - 1}) \in \R^{n - 1} : (x_1, \dots, x_{n - 1}, 0) \in W\}\).
  Note that \(U\) is open and contains \((y_1, \dots, y_{n - 1})\).
  Now we define \(g : U \to \R\) by \(g(x_1, \dots, x_{n - 1}) \coloneqq h_n(x_1, \dots, x_{n - 1}, 0)\).
  Then \(g\) is differentiable at \((y_1, \dots, y_{n - 1})\).
  Now we prove that
  \begin{align*}
     & \{(x_1, \dots, x_n) \in V : f(x_1, \dots, x_n) = 0\}                                                     \\
     & = \Big\{\big(x_1, \dots, x_{n - 1}, g(x_1, \dots, x_{n - 1})\big) : (x_1, \dots, x_{n - 1}) \in U\Big\}.
  \end{align*}
  First suppose that \((x_1, \dots, x_n) \in V\) and \(f(x_1, \dots, x_n) = 0\).
  Then we have
  \[
    F(x_1, \dots, x_n) = (x_1, \dots, x_{n - 1}, 0),
  \]
  which lies in \(W\).
  Thus \((x_1, \dots, x_{n - 1})\) lies in \(U\).
  Applying \(F^{-1}\), we see that
  \[
    (x_1, \dots, x_n) = F^{-1}(x_1, \dots, x_{n - 1}, 0).
  \]
  In particular \(x_n = h_n(x_1, \dots, x_{n - 1}, 0)\), and hence \(x_n = g(x_1, \dots, x_{n - 1})\).
  Thus every element of the left-hand set lies in the right-hand set.
  The reverse inclusion comes by reversing all the above steps and is left to the reader.

  Finally, we show the formula for the partial derivatives of \(g\).
  From the preceding discussion we have
  \[
    f\big(x_1, \dots, x_{n - 1}, g(x_1, \dots, x_{n - 1})\big) = 0
  \]
  for all \((x_1, \dots, x_{n - 1}) \in U\).
  Since \(g\) is differentiable at \((y_1, \dots, y_{n - 1})\), and \(f\) is differentiable at \(\big(y_1, \dots, y_{n - 1}, g(y_1, \dots, y_{n - 1})\big) = y\), we may use the chain rule, differentiating in \(x_j\), to obtain
  \[
    \dfrac{\partial f}{\partial x_j}(y) + \dfrac{\partial f}{\partial x_n}(y) \dfrac{\partial g}{\partial x_j}(y_1, \dots, y_{n - 1}) = 0
  \]
  and the claim follows by simple algebra.
\end{proof}

\begin{rmk}\label{6.8.2}
  \cref{eq 6.1} is sometimes derived using \emph{implicit differentiation}.
  Basically, the point is that if you know that
  \[
    f(x_1, \dots, x_n) = 0
  \]
  then (as long as \(\dfrac{\partial f}{\partial x_n} \neq 0\)) the variable \(x_n\) is ``implicitly'' defined in terms of the other \(n - 1\) variables, and one can differentiate the above identity in, say, the \(x_j\) direction using the chain rule to obtain
  \[
    \dfrac{\partial f}{\partial x_j} + \dfrac{\partial f}{\partial x_n} \dfrac{\partial x_n}{\partial x_j} = 0
  \]
  which is \cref{eq 6.1} in disguise
  (we are using \(g\) to represent the implicit function defining \(x_n\) in terms of \(x_1, \dots, x_n\)).
  Thus, the implicit function theorem allows one to define a dependence implicitly, by means of a constraint rather than by a direct formula of the form \(x_n = g(x_1, \dots, x_{n - 1})\).
\end{rmk}

\begin{note}
  In the implicit function theorem, if the derivative \(\dfrac{\partial f}{\partial x_n}\) equals zero at some point, then it is unlikely that the set \(\{x \in \R^n : f(x) = 0\}\) can be written as a graph of the \(x_n\) variable in terms of the other \(n - 1\) variables near that point.
  However, if some other derivative \(\dfrac{\partial f}{\partial x_j}\) is non-zero, then it would be possible to write the \(x_j\) variable in terms of the other \(n - 1\) variables, by a variant of the implicit function theorem.
  Thus as long as the gradient \(\nabla f\) is not entirely zero, one can write this set \(\{x \in \R^n : f(x) = 0\}\) as a graph of \emph{some} variable \(x_j\) in terms of the other \(n - 1\) variables.
  (The circle \(\{(x, y) \in \R^2 : x^2 + y^2 - 1 = 0\}\) is a good example of this;
  it is not a graph of \(y\) in terms of \(x\), or \(x\) in terms of \(y\), but near every point it is one of the two.
  And this is because the gradient of \(x^2 + y^2 - 1\) is never zero on the circle.)
  However, if \(\nabla f\) does vanish at some point \(x_0\), then we say that \(f\) has a \emph{critical point} at \(x_0\) and the behavior there is much more complicated.
  For instance, the set \(\{(x, y) \in \R^2 : x^2 - y^2 = 0\}\) has a critical point at \((0, 0)\) and there the set does not look like a graph of any sort
  (it is the union of two lines).
\end{note}

\begin{rmk}\label{6.8.4}
  Sets which look like graphs of continuous functions at every point have a name, they are called \emph{manifolds}.
  Thus \(\{x \in \R^n : f(x) = 0\}\) will be a manifold if it contains no critical points of \(f\).
  The theory of manifolds is very important in modern geometry (especially differential geometry and algebraic geometry), but we will not discuss it here as it is a graduate level topic.
\end{rmk}

\exercisesection

\begin{ex}\label{ex:6.8.1}
  Let the notation and hypotheses be as in \cref{6.8.1}.
  Show that, after shrinking the open sets \(U, V\) if necessary, that the function \(g\) becomes continuously differentiable on all of \(U\), and \cref{eq 6.1} holds at all points of \(U\).
\end{ex}


%------------------------------------------------------------------------------
% Back matters.
%------------------------------------------------------------------------------

\backmatter

\end{document}
