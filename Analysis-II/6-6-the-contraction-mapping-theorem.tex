\section{The contraction mapping theorem}\label{ii:sec:6.6}

\begin{defn}[Contraction]\label{ii:6.6.1}
  Let \((X, d)\) be a metric space, and let \(f : X \to X\) be a map.
  We say that \(f\) is a \emph{contraction} if we have \(d\big(f(x), f(y)\big) \leq d(x, y)\) for all \(x, y \in X\).
  We say that \(f\) is a \emph{strict contraction} if there exists a constant \(0 < c < 1\) such that \(d\big(f(x), f(y)\big) \leq c d(x, y)\) for all \(x, y \in X\);
  we call \(c\) the \emph{contraction constant} of \(f\).
\end{defn}

\begin{eg}\label{ii:6.6.2}
  The map \(f : \R \to \R\) defined by \(f(x) \coloneqq x + 1\) is a contraction but not a strict contraction.
  The map \(f : \R \to \R\) defined by \(f(x) \coloneqq x / 2\) is a strict contraction.
  The map \(f : [0, 1] \to [0, 1]\) defined by \(f(x) \coloneqq x - x^2\) is a contraction but not a strict contraction.
\end{eg}

\begin{proof}
  Since
  \[
    \forall x, y \in \R, \abs{(x + 1) - (y + 1)} = \abs{x - y} \leq \abs{x - y},
  \]
  by \cref{ii:6.6.1} we know that \(x \mapsto x + 1\) is a contraction.
  Suppose for sake of contradiction that \(x \mapsto x + 1\) is a strict contraction.
  Then there exists a \(c \in (0, 1)\) such that
  \[
    \forall x, y \in \R, \abs{(x + 1) - (y + 1)} \leq c \abs{x - y}.
  \]
  But we see that when \(x \neq y\), we have
  \[
    \abs{x - y} \leq c \abs{x - y} \implies 1 \leq c,
  \]
  which contradict to the fact that \(c \in (0, 1)\).
  Thus \(x \mapsto x + 1\) is not a strict contraction.

  Since
  \[
    \forall x, y \in \R, \abs{\dfrac{x}{2} - \dfrac{y}{2}} = \dfrac{1}{2} \abs{x - y} \leq \dfrac{1}{2} \abs{x - y},
  \]
  by \cref{ii:6.6.1} we know that \(x \mapsto \dfrac{x}{2}\) is a strict contraction.

  Since
  \begin{align*}
             & \forall x, y \in [0, 1], \begin{dcases}
                                          -1 \leq -x \leq 0 \\
                                          -1 \leq -y \leq 0
                                        \end{dcases} \\
    \implies & -2 \leq -x - y \leq 0                      \\
    \implies & -1 \leq 1 - x - y \leq 1                   \\
    \implies & 0 \leq \abs{1 - x - y} \leq 1,
  \end{align*}
  we know that
  \begin{align*}
    \forall x, y \in [0, 1], \abs{(x - x^2) - (y - y^2)} & = \abs{x - y - (x^2 - y^2)}    \\
                                                         & = \abs{x - y - (x - y)(x + y)} \\
                                                         & = \abs{(x - y)(1 - x - y)}     \\
                                                         & = \abs{x - y} \abs{1 - x - y}  \\
                                                         & \leq \abs{x - y}.
  \end{align*}
  Thus by \cref{ii:6.6.1} we know that \(x \mapsto x - x^2\) is a contraction on \([0, 1]\).
  Suppose for sake of contradiction that \(x \mapsto x - x^2\) is a strict contraction on \([0, 1]\).
  Then there exists a \(c \in (0, 1)\) such that
  \[
    \forall x, y \in \R, \abs{(x - x^2) - (y - y^2)} \leq c \abs{x - y}.
  \]
  But when \((x, y) = (\dfrac{1 - c}{2}, 0)\) we have
  \begin{align*}
    \abs{(x - x^2) - (y - y^2)} & = \abs{\dfrac{1 - c}{2} - \bigg(\dfrac{1 - c}{2}\bigg)^2 - (0 - 0^2)}                  \\
                                & = \dfrac{1 - c}{2} \abs{1 - \dfrac{1 - c}{2}}                                          \\
                                & = \dfrac{1 - c}{2} \dfrac{1 + c}{2}                                                    \\
                                & > \dfrac{1 - c}{2} \dfrac{c + c}{2}                                   & (c \in (0, 1)) \\
                                & = c \dfrac{1 - c}{2}                                                                   \\
                                & = c \abs{\dfrac{1 - c}{2} - 0}                                                         \\
                                & = c \abs{x - y},
  \end{align*}
  a contradiction.
  Thus \(x \mapsto x - x^2\) is not a strict contraction on \([0, 1]\).
\end{proof}

\begin{defn}[Fixed points]\label{ii:6.6.3}
  Let \(f : X \to X\) be a map, and \(x \in X\).
  We say that \(x\) is a \emph{fixed point} of \(f\) if \(f(x) = x\).
\end{defn}

\begin{note}
  Contractions do not necessarily have any fixed points;
  for instance, the map \(f : \R \to \R\) defined by \(f(x) = x + 1\) does not.
  However, it turns out that \emph{strict} contractions always do, at least when \(X\) is complete.
\end{note}

\begin{thm}[Contraction mapping theorem]\label{ii:6.6.4}
  Let \((X, d)\) be a metric space, and let \(f : X \to X\) be a strict contraction.
  Then \(f\) can have at most one fixed point.
  Moreover, if we also assume that \(X\) is non-empty and complete, then \(f\) has exactly one fixed point.
\end{thm}

\begin{proof}
  We first show that \(f\) can have at most one fixed point.
  Suppose for sake of contradiction that \(f\) has two fixed point \(x_1, x_2 \in X\).
  Since \(f\) is a strict contraction, by \cref{ii:6.6.1} we know that there exists a \(c \in (0, 1)\) such that
  \[
    d\big(f(x_1), f(x_2)\big) \leq c d(x_1, x_2).
  \]
  Since \(x_1, x_2\) are fixed points, by \cref{ii:6.6.3} we know that
  \[
    d(x_1, x_2) \leq c d(x_1, x_2)
  \]
  which implies \(1 \leq c\), a contradiction.
  Thus \(f\) can have at most one fixed point.

  Now we show that when \(X \neq \emptyset\) and \((X, d)\) is complete, \(f\) has exactly one fixed point.
  Let \(x_0 \in X\).
  Since \(f\) is a strict contraction, by \cref{ii:6.6.1} we know that there exists a \(c \in (0, 1)\) such that
  \[
    \forall x, y \in X, d\big(f(x), f(y)\big) \leq c d(x, y).
  \]
  Now we define a sequence \((x_n)_{n = 1}^\infty\) as follow:
  \[
    \forall n \in \Z^+, x_n = f(x_{n - 1}).
  \]
  We claim that
  \[
    \forall n \in \Z^+, d(x_{n + 1}, x_n) \leq c^n d(x_1, x_0).
  \]
  We induct on \(n\) to proof the claim.
  For \(n = 0\), we have
  \[
    d(x_1, d_0) \leq c^0 d(x_1, x_0) = d(x_1, x_0).
  \]
  Thus, the base case holds.
  Suppose inductively that \(d(x_{n + 1}, x_n) \leq c^n d(x_1, x_0)\) for some \(n \geq 0\).
  Then for \(n + 1\), we have
  \begin{align*}
    d(x_{n + 2}, x_{n + 1}) & = d\big(f(x_{n + 1}), f(x_n)\big) &  & \text{(by the definition of \((x_n)_{n = 1}^\infty\))} \\
                            & \leq c d(x_{n + 1}, x_n)          &  & \by{ii:6.6.1}                                          \\
                            & \leq c \cdot c^n d(x_1, x_0)                                                                  \\
                            & = c^{n + 1} d(x_1, x_0).
  \end{align*}
  This closes the induction.

  Next we claim that \((x_n)_{n = 1}^\infty\) is a Cauchy sequence in \((X, d)\).
  Let \(\varepsilon \in \R^+\).
  Observe that
  \begin{align*}
             & c \in (0, 1)                                                                                                                      \\
    \implies & \lim_{n \to \infty} c^n = 0                                                                                                       \\
    \implies & \exists N \in \Z^+ : \forall n \geq N, c^n < \dfrac{\varepsilon (1 - c)}{1 + d(x_1, x_0)}                                         \\
    \implies & \exists N \in \Z^+ : \forall n \geq N, \dfrac{d(x_1, x_0)}{1 - c} \leq \dfrac{1 + d(x_1, x_0)}{1 - c} < \dfrac{\varepsilon}{c^n}.
  \end{align*}
  Fix such \(N\).
  Let \(n, m \geq N\) and without the loss of generality suppose that \(n \leq m\).
  Since
  \begin{align*}
    d(x_n, x_m) & \leq \sum_{i = n}^m d(x_i, x_{i + 1})                    &                & \by{ii:1.1.2}[d]              \\
                & = \sum_{i = n}^m d(x_{i + 1}, x_i)                       &                & \by{ii:1.1.2}[c]              \\
                & \leq \sum_{i = n}^m c^i d(x_1, x_0)                      &                & \text{(from the claim above)} \\
                & = d(x_1, x_0) \bigg(\sum_{i = n}^m c^i\bigg)                                                              \\
                & = d(x_1, x_0) c^n \bigg(\sum_{i = 0}^{m - n} c^i\bigg)                                                    \\
                & \leq d(x_1, x_0) c^n \bigg(\sum_{i = 0}^\infty c^i\bigg) & (c \in (0, 1))                                 \\
                & = \dfrac{d(x_1, x_0) c^n}{1 - c}                         &                & \text{(by geometric series)}  \\
                & < \dfrac{\varepsilon c^n}{c^n}                           & (n \geq N)                                     \\
                & = \varepsilon
  \end{align*}
  and \(\varepsilon\) was arbitrary, we know that
  \[
    \forall \varepsilon \in \R^+, \exists N \in \Z^+ : \forall n, m \geq N, d(x_n, x_m) \leq \varepsilon.
  \]
  By \cref{ii:1.4.6} this means \((x_n)_{n = 1}^\infty\) is a Cauchy sequence in \((X, d)\).
  By hypothesis we know that \((X, d)\) is complete, thus by \cref{ii:1.4.10} there exists a \(y_0 \in X\) such that \(\lim_{n \to \infty} d(y_0, x_n) = 0\).

  Now we claim that \(y_0\) is the fixed point of \(f\).
  Suppose for sake of contradiction that \(y_0\) is not the fixed point of \(f\).
  By \cref{ii:6.6.3} this means \(f(y_0) \neq y_0\), in other words, \(d\big(f(y_0), y_0\big) > 0\).
  Since \((x_n)_{n = 1}^\infty\) converges to \(y_0\) in \(X\) with respect to \(d\), we know that
  \begin{align*}
             & \forall \varepsilon \in \R^+, \exists N \in \Z^+ : \forall n \geq N, d(y_0, x_n) < \varepsilon \\
    \implies & \exists N \in \Z^+ : \forall n \geq N, d(y_0, x_n) < \dfrac{d\big(f(y_0), y_0\big)}{2}.
  \end{align*}
  Fix such \(N\).
  But then we have
  \begin{align*}
             & \begin{dcases}
                 d(y_0, x_N) < \dfrac{d\big(f(y_0), y_0\big)}{2} \\
                 d(y_0, x_{N + 1}) < \dfrac{d\big(f(y_0), y_0\big)}{2}
               \end{dcases}                                                   \\
    \implies & \begin{dcases}
                 c d(y_0, x_N) < c \dfrac{d\big(f(y_0), y_0\big)}{2} \\
                 d(y_0, x_{N + 1}) < \dfrac{d\big(f(y_0), y_0\big)}{2}
               \end{dcases}                                                   \\
    \implies & \begin{dcases}
                 d\big(f(y_0), f(x_N)\big) < c d(y_0, x_N) < c \dfrac{d\big(f(y_0), y_0\big)}{2} \\
                 d(y_0, x_{N + 1}) < \dfrac{d\big(f(y_0), y_0\big)}{2}
               \end{dcases}                        \\
    \implies & \begin{dcases}
                 d\big(f(y_0), x_{N + 1}\big) < c d(y_0, x_N) < c \dfrac{d\big(f(y_0), y_0\big)}{2} \\
                 d(y_0, x_{N + 1}) < \dfrac{d\big(f(y_0), y_0\big)}{2}
               \end{dcases}                     \\
    \implies & d\big(f(y_0), y_0\big) \leq d\big(f(y_0), x_{N + 1}\big) + d(y_0, x_{N + 1})                           \\
             & < (c + 1) \dfrac{d\big(f(y_0), y_0\big)}{2} < d\big(f(y_0), y_0\big),                 & (c \in (0, 1))
  \end{align*}
  a contraction.
  Thus \(y_0\) is the fixed point of \(f\).
\end{proof}

\begin{rmk}\label{ii:6.6.5}
  The contraction mapping theorem is one example of a \emph{fixed point theorem}
  - a theorem which guarantees, assuming certain conditions, that a map will have a fixed point.
  There are a number of other fixed point theorems which are also useful.
  One amusing one is the so-called \emph{hairy ball theorem}, which (among other things) states that any continuous map \(f : S^2 \to S^2\) from the sphere \(S^2 \coloneqq \set{(x, y, z) \in \R^3 : x^2 + y^2 + z^2 = 1}\) to itself, must contain either a fixed point, or an anti-fixed point
  (a point \(x \in S^2\) such that \(f(x) = -x\)).
  A proof of this theorem can be found in any topology text;
  it is beyond the scope of this text.
\end{rmk}

\begin{note}
  Basically, \cref{ii:6.6.6} says that any map \(f\) on a ball which is a ``small'' perturbation (since \(g\) is a strict contraction) of the identity map (since \(f(x) = x + g(x)\)), remains one-to-one and cannot create any internal holes in the ball
  (there is a smaller ball contained in the origin ball such that every element in the smaller ball can be mapped by \(f\)).
\end{note}

\begin{lem}\label{ii:6.6.6}
  Let \(B(0, r)\) be a ball in \(\R^n\) centered at the origin, and let \(g : B(0, r) \to \R^n\) be a map such that \(g(0) = 0\) and
  \[
    \norm*{g(x) - g(y)} \leq \dfrac{1}{2} \norm*{x - y}
  \]
  for all \(x, y \in B(0, r)\)
  (here \(\norm*{x}\) denotes the length of \(x\) in \(\R^n\)).
  Then the function \(f : B(0, r) \to \R^n\) defined by \(f(x) \coloneqq x + g(x)\) is one-to-one, and furthermore the image \(f\big(B(0, r)\big)\) of this map contains the ball \(B(0, r / 2)\).
\end{lem}

\begin{proof}
  We first show that \(f\) is one-to-one.
  Suppose for sake of contradiction that we had two different points \(x, y \in B(0, r)\) such that \(f(x) = f(y)\).
  But then we would have \(x + g(x) = y + g(y)\), and hence
  \[
    \norm*{g(x) - g(y)} = \norm*{x - y}.
  \]
  The only way this can be consistent with our hypothesis \(\norm*{g(x) - g(y)} \leq \dfrac{1}{2} \norm*{x - y}\) is if \(\norm*{x - y} = 0\), i.e., if \(x = y\), a contradiction.
  Thus \(f\) is one-to-one.

  Now we show that \(f\big(B(0, r)\big)\) contains \(B(0, r / 2)\).
  Let \(y\) be any point in \(B(0, r / 2)\);
  our objective is to find a point \(x \in B(0, r)\) such that \(f(x) = y\), or in other words that \(x = y - g(x)\).
  So the problem is now to find a fixed point of the map \(x \mapsto y - g(x)\).

  Let \(F : B(0, r) \to B(0, r)\) denote the function \(F(x) \coloneqq y - g(x)\).
  Observe that if \(x \in B(0, r)\), then
  \[
    \norm*{F(x)} \leq \norm*{y} + \norm*{g(x)} \leq \dfrac{r}{2} + \norm*{g(x) - g(0)} \leq \dfrac{r}{2} + \dfrac{1}{2} \norm*{x - 0} < \dfrac{r}{2} + \dfrac{r}{2} = r,
  \]
  so \(F\) does indeed map \(B(0, r)\) to itself.
  The same argument shows that for a sufficiently small \(\varepsilon > 0\), \(F\) maps the closed ball \(\overline{B(0, r - \varepsilon)}\) to itself.
  Also, for any \(x, x'\) in \(B(0, r)\) we have
  \[
    \norm*{F(x) - F(x')} = \norm*{g(x') - g(x)} \leq \dfrac{1}{2} \norm*{x' - x}
  \]
  so \(F\) is a strict contraction on \(B(0, r)\), and hence on the complete space \(\overline{B(0, r - \varepsilon)}\) (see \cref{ii:1.5.7} and \cref{ii:1.5.5}).
  By the contraction mapping theorem, \(F\) has a fixed point, i.e., there exists an \(x\) such that \(x = y - g(x)\).
  But this means that \(f(x) = y\), as desired.
\end{proof}

\exercisesection

\begin{ex}\label{ii:ex:6.6.1}
  Let \(f : [a, b] \to [a, b]\) be a differentiable function of one variable such that \(\abs{f'(x)} \leq 1\) for all \(x \in [a, b]\).
  Prove that \(f\) is a contraction.
  If in addition \(\abs{f'(x)} < 1\) for all \(x \in [a, b]\) and \(f'\) is continuous, show that \(f\) is a strict contraction.
\end{ex}

\begin{proof}
  First we show that \(\abs{f'(x)} \leq 1\) for all \(x \in [a, b]\) implies \(f\) is a contraction.
  Let \(x, y \in [a, b]\).
  If \(x = y\), then we have
  \[
    \abs{f(x) - f(y)} = 0 \leq 0 = \abs{x - y}.
  \]
  So suppose that \(x \neq y\).
  By mean value theorem (Corollary 10.2.9 in Analysis I) we know that
  \begin{align*}
             & \exists c \in (x, y) \cup (y, x) : \dfrac{f(x) - f(y)}{x - y} = f'(c)                    \\
    \implies & \exists c \in (x, y) \cup (y, x) : \abs{\dfrac{f(x) - f(y)}{x - y}} = \abs{f'(c)} \leq 1 \\
    \implies & \abs{f(x) - f(y)} \leq \abs{x - y}.
  \end{align*}
  Thus by \cref{ii:6.6.1} \(f\) is a contraction.

  Now we show that \(\abs{f'(x)} < 1\) for all \(x \in [a, b]\) and \(f'\) is continuous implies \(f\) is a strict contraction.
  Since \(f'\) is continuous, by Proposition 9.6.7 in Analysis I we know that
  \begin{align*}
             & \exists x_{\min}, x_{\max} \in [a, b] : \forall x \in [a, b], f'(x_{\min}) \leq f'(x) \leq f'(x_{\max})                                  \\
    \implies & \exists x_{\min}, x_{\max} \in [a, b] : \forall x \in [a, b], \abs{f'(x)} \leq \max\big(\abs{f'(x_{\min})}, \abs{f'(x_{\max})}\big) < 1.
  \end{align*}
  Let \(x, y \in [a, b]\).
  If \(x = y\), then we have
  \begin{align*}
    \abs{f(x) - f(y)} & = 0                                                                       \\
                      & \leq 0                                                                    \\
                      & = \max\big(\abs{f'(x_{\min})}, \abs{f'(x_{\max})}\big) \cdot 0            \\
                      & = \max\big(\abs{f'(x_{\min})}, \abs{f'(x_{\max})}\big) \cdot \abs{x - y}.
  \end{align*}
  So suppose that \(x \neq y\).
  By mean value theorem (Corollary 10.2.9) we know that
  \begin{align*}
             & \exists c \in (x, y) \cup (y, x) : \dfrac{f(x) - f(y)}{x - y} = f'(c)                                                                    \\
    \implies & \exists c \in (x, y) \cup (y, x) : \abs{\dfrac{f(x) - f(y)}{x - y}} = \abs{f'(c)} < \max\big(\abs{f'(x_{\min})}, \abs{f'(x_{\max})}\big) \\
    \implies & \abs{f(x) - f(y)} \leq \max\big(\abs{f'(x_{\min})}, \abs{f'(x_{\max})}\big) \cdot \abs{x - y}.
  \end{align*}
  Thus by \cref{ii:6.6.1} \(f\) is a strict contraction.
\end{proof}

\begin{ex}\label{ii:ex:6.6.2}
  Show that if \(f : [a, b] \to \R\) is differentiable and is a contraction, then \(\abs{f'(x)} \leq 1\).
\end{ex}

\begin{proof}
  We have
  \begin{align*}
             & \forall x_0 \in [a, b], \forall x \in [a, b] \setminus \set{x_0}, \abs{f(x) - f(x_0)} \leq \abs{x - x_0}                &  & \by{ii:6.6.1} \\
    \implies & \forall x_0 \in [a, b], \forall x \in [a, b] \setminus \set{x_0}, \abs{\dfrac{f(x) - f(x_0)}{x - x_0}} \leq 1                              \\
    \implies & \forall x_0 \in [a, b], \lim_{x \to x_0 ; x \in [a, b] \setminus \set{x_0}} \abs{\dfrac{f(x) - f(x_0)}{x - x_0}} \leq 1                    \\
    \implies & \forall x_0 \in [a, b], f'(x_0) \leq 1.
  \end{align*}
\end{proof}

\begin{ex}\label{ii:ex:6.6.3}
  Give an example of a function \(f : [a, b] \to \R\) which is continuously differentiable and such that \(\abs{f(x) - f(y)} < \abs{x - y}\) for all distinct \(x, y \in [a, b]\), but such that \(\abs{f'(x)} = 1\) for at least one value of \(x \in [a, b]\).
\end{ex}

\begin{proof}
  Let \(f : [0, 0.5] \to \R\) be the function
  \[
    \forall x \in [0, 0.5], f(x) = x^2.
  \]
  Let \(x, y \in [0, 0.5]\) such that \(x \neq y\).
  Since \(x \neq y\), we know that \(x + y < 1\).
  Then we have
  \begin{align*}
    \abs{f(x) - f(y)} & = \abs{x^2 - y^2}                    \\
                      & = \abs{(x - y)(x + y)}               \\
                      & = \abs{x - y} (x + y)                \\
                      & < \abs{x - y}.         & (x + y < 1)
  \end{align*}
  Since \(f'(x) = 2x\), we know that \(\abs{f'(0.5)} = \abs{1} = 1\).
\end{proof}

\begin{ex}\label{ii:ex:6.6.4}
  Given an example of a function \(f : [a, b] \to \R\) which is a strict contraction but which is not differentiable for at least one point \(x\) in \([a, b]\).
\end{ex}

\begin{proof}
  Let \(f : [0, 1] \to \R\) be the function
  \[
    \forall x \in [0, 1], f(x) = \begin{dcases}
      \dfrac{x}{2} & \text{if } x \in [0.5, 1] \\
      \dfrac{x}{3} & \text{if } x \in [0, 0.5)
    \end{dcases}.
  \]
  Observe that
  \[
    \lim_{x \to 0.5+} f(x) = \dfrac{0.5}{2} \neq \dfrac{0.5}{3} = \lim_{x \to 0.5-} f(x).
  \]
  Thus \(f\) is not continuous at \(0.5\) and by Proposition 10.1.10 in Analysis I \(f\) is not differentiable at \(0.5\).
  Since
  \begin{align*}
             & \forall x, y \in [0, 1], \begin{dcases}
                                          \abs{f(x) - f(y)} = \dfrac{1}{2} \abs{x - y}    & \text{if } x, y \in [0.5, 1]                   \\
                                          \abs{f(x) - f(y)} = \dfrac{1}{3} \abs{x - y}    & \text{if } x, y \in [0, 0.5)                   \\
                                          \abs{f(x) - f(y)} = \dfrac{x}{2} - \dfrac{y}{3} & \text{if } x \in [0.5, 1] \land y \in [0, 0.5) \\
                                          \abs{f(x) - f(y)} = \dfrac{y}{2} - \dfrac{x}{3} & \text{if } x \in [0, 0.5) \land y \in [0.5, 1]
                                        \end{dcases}   \\
    \implies & \begin{dcases}
                 \abs{f(x) - f(y)} \leq \dfrac{1}{2} \abs{x - y}                            & \text{if } x, y \in [0.5, 1]                   \\
                 \abs{f(x) - f(y)} = \dfrac{1}{3} \abs{x - y} \leq \dfrac{1}{2} \abs{x - y} & \text{if } x, y \in [0, 0.5)                   \\
                 \abs{f(x) - f(y)} < \dfrac{1}{3} (x - y) \leq \dfrac{1}{2} \abs{x - y}     & \text{if } x \in [0.5, 1] \land y \in [0, 0.5) \\
                 \abs{f(x) - f(y)} < \dfrac{1}{3} (y - x) \leq \dfrac{1}{2} \abs{x - y}     & \text{if } x \in [0, 0.5) \land y \in [0.5, 1]
               \end{dcases} \\
    \implies & \abs{f(x) - f(y)} \leq \dfrac{1}{2} \abs{x - y},
  \end{align*}
  by \cref{ii:6.6.1} we know that \(f\) is a strict contraction.
\end{proof}

\begin{ex}\label{ii:ex:6.6.5}
  Verify the claims in \cref{ii:6.6.2}.
\end{ex}

\begin{proof}
  See \cref{ii:6.6.2}.
\end{proof}

\begin{ex}\label{ii:ex:6.6.6}
  Show that every contraction on a metric space \(X\) is necessarily continuous.
\end{ex}

\begin{proof}
  Let \((X, d)\) be a metric space, let \(f : X \to X\) be a contraction of \(X\) and let \(x_0 \in X\).
  We have
  \begin{align*}
             & \forall \varepsilon \in \R^+, \forall x \in X, d(x, x_0) < \varepsilon                    \\
    \implies & d\big(f(x), f(x_0)\big) \leq d(x, x_0) < \varepsilon.                  &  & \by{ii:6.6.1}
  \end{align*}
  By setting \(\delta = \varepsilon\) we see that
  \[
    \forall \varepsilon \in \R^+, \exists \delta \in \R^+ : \forall x \in X, d(x, x_0) < \delta \implies d\big(f(x), f(x_0)\big) < \varepsilon.
  \]
  Since \(x_0\) was arbitrary, by \cref{ii:2.1.1} this means \(f\) is continuous on \(X\) from \((X, d)\) to \((X, d)\).
\end{proof}

\begin{ex}\label{ii:ex:6.6.7}
  Prove \cref{ii:6.6.4}.
\end{ex}

\begin{proof}
  See \cref{ii:6.6.4}.
\end{proof}

\begin{ex}\label{ii:ex:6.6.8}
  Let \((X, d)\) be a complete metric space, and let \(f : X \to X\) and \(g : X \to X\) be two strict contractions on \(X\) with contraction coefficients \(c\) and \(c'\) respectively.
  From \cref{ii:6.6.4} we know that \(f\) has some fixed point \(x_0\), and \(g\) has some fixed point \(y_0\).
  Suppose we know that there is an \(\varepsilon > 0\) such that \(d\big(f(x), g(x)\big) \leq \varepsilon\) for all \(x \in X\)
  (i.e., \(f\) and \(g\) are within \(\varepsilon\) of each other in the uniform metric).
  Show that \(d(x_0, y_0) \leq \varepsilon / \big(1 - \min(c, c')\big)\).
  Thus nearby contractions have nearby fixed points.
\end{ex}

\begin{proof}
  We have
  \begin{align*}
    d(x_0, y_0) & = d\big(f(x_0), g(y_0)\big)                                &  & \by{ii:6.6.3}          \\
                & \leq d\big(f(x_0), g(x_0)\big) + d\big(g(x_0), g(y_0)\big) &  & \by{ii:1.1.2}[d]       \\
                & \leq \varepsilon + d\big(g(x_0), g(y_0)\big)               &  & \text{(by hypothesis)} \\
                & \leq \varepsilon + c' d(x_0, y_0)                          &  & \by{ii:6.6.1}
  \end{align*}
  and
  \begin{align*}
    d(x_0, y_0) & = d\big(f(x_0), g(y_0)\big)                                &  & \by{ii:6.6.3}          \\
                & \leq d\big(f(x_0), f(y_0)\big) + d\big(f(y_0), g(y_0)\big) &  & \by{ii:1.1.2}[d]       \\
                & \leq d\big(f(x_0), f(y_0)\big) + \varepsilon               &  & \text{(by hypothesis)} \\
                & \leq c d(x_0, y_0) + \varepsilon.                          &  & \by{ii:6.6.1}
  \end{align*}
  Thus
  \begin{align*}
             & \begin{dcases}
                 d(x_0, y_0) \leq \varepsilon + \max(c, c') \cdot d(x_0, y_0) \\
                 d(x_0, y_0) \leq \varepsilon + \min(c, c') \cdot d(x_0, y_0)
               \end{dcases} \\
    \implies & \begin{dcases}
                 \big(1 - \max(c, c')\big) d(x_0, y_0) \leq \varepsilon \\
                 \big(1 - \min(c, c')\big) d(x_0, y_0) \leq \varepsilon
               \end{dcases}       \\
    \implies & \begin{dcases}
                 d(x_0, y_0) \leq \dfrac{\varepsilon}{1 - \max(c, c')} \\
                 d(x_0, y_0) \leq \dfrac{\varepsilon}{1 - \min(c, c')}
               \end{dcases}.
  \end{align*}
\end{proof}
