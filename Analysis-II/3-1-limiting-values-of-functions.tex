\section{Limiting values of functions}\label{sec 3.1}

\begin{definition}[Limiting value of a function]\label{3.1.1}
    Let \((X, d_X)\) and \((Y, d_Y)\) be metric spaces, let \(E\) be a subset of \(X\), and let \(f : E \to Y\) be a function.
    If \(x_0 \in X\) is an adherent point of \(E\), and \(L \in Y\), we say that \emph{\(f(x)\) converges to \(L\) in \(Y\) as \(x\) converges to \(x_0\) in \(E\)}, or write \(\lim_{x \to x_0 ; x \in E} f(x) = L\), if for every \(\varepsilon > 0\) there exists a \(\delta > 0\) such that \(d_Y\big(f(x), L\big) < \varepsilon\) for all \(x \in E\) such that \(d_X(x, x_0) < \delta\).
\end{definition}

\begin{remark}\label{3.1.2}
    Some authors exclude the case \(x = x_0\) from the above definition, thus requiring \(0 < d_X(x, x_0) < \delta\).
    In our current notation, this would correspond to removing \(x_0\) from \(E\), thus one would consider
    \[
        \lim_{x \to x_0 ; x \in E \setminus \{x_0\}} f(x)
    \]
    instead of
    \[
        \lim_{x \to x_0 ; x \in E} f(x).
    \]
\end{remark}

\begin{note}
    Comparing this with Definition \ref{2.1.1}, we see that \(f\) is continuous at \(x_0\) if and only if
    \[
        \lim_{x \to x_0 ; x \in X} f(x) = f(x_0).
    \]
    Thus \(f\) is continuous on \(X\) iff we have
    \[
        \lim_{x \to x_0 ; x \in X} f(x) = f(x_0) \text{ for all } x_0 \in X.
    \]
\end{note}

\setcounter{theorem}{3}
\begin{remark}\label{3.1.4}
    Often we shall omit the condition \(x \in X\), and abbreviate
    \[
        \lim_{x \to x_0 ; x \in X} f(x)
    \]
    as simply
    \[
        \lim_{x \to x_0} f(x)
    \]
    when it is clear what space \(x\) will range in.
\end{remark}

\begin{proposition}\label{3.1.5}
    Let \((X, d_X)\) and \((Y, d_Y)\) be metric spaces, let \(E\) be a subset of \(X\), and let \(f : E \to Y\) be a function.
    Let \(x_0 \in X\) be an adherent point of \(E\) and \(L \in Y\).
    Then the following four statements are logically equivalent:
    \begin{enumerate}
        \item \(\lim_{x \to x_0 ; x \in E} f(x) = L\).
        \item For every sequence \((x^{(n)})_{n = 1}^\infty\) in \(E\) which converges to \(x_0\) with respect to the metric \(d_X\), the sequence \(\big(f(x^{(n)})\big)_{n = 1}^\infty\) converges to \(L\) with respect to the metric \(d_Y\).
        \item For every open set \(V \subseteq Y\) which contains \(L\), there exists an open set \(U \subseteq X\) containing \(x_0\) such that \(f(U \cap E) \subseteq V\).
        \item If one defines the function \(g : E \cup \{x_0\} \to Y\) by defining \(g(x_0) \coloneqq L\), and \(g(x) \coloneqq f(x)\) for \(x \in E \setminus \{x_0\}\), then \(g\) is continuous at \(x_0\).
    \end{enumerate}
\end{proposition}

\begin{proof}
    We first show that statement (a) implies statement (b).
    Suppose that
    \[
        d_Y - \lim_{x \to x_0 ; x \in E} f(x) = L.
    \]
    By Definition \ref{3.1.1} we have
    \[
        \forall\ \varepsilon \in \mathbf{R}^+, \exists\ \delta \in \mathbf{R}^+ : \Big(\forall\ x \in E, d_X(x, x_0) < \delta \implies d_Y\big(f(x), L\big) < \varepsilon\Big).
    \]
    Let \((x^{(n)})_{n = 1}^\infty\) be a sequence in \(E\) such that \(\lim_{n \to \infty} d_X(x^{(n)}, x_0) = 0\).
    By Definition \ref{1.1.14} we have
    \[
        \forall\ \delta \in \mathbf{R}^+, \exists\ N \in \mathbf{Z}^+ : \forall\ n \geq N, d_X(x^{(n)}, x_0) < \delta.
    \]
    Since \((x^{(n)})_{n = 1}^\infty\) is in \(E\), we have
    \[
        \forall\ \varepsilon \in \mathbf{R}^+, \exists\ \delta \in \mathbf{R}^+ : \begin{cases}
            \exists\ N \in \mathbf{Z}^+ : \forall\ n \geq N, d_X(x^{(n)}, x_0) < \delta \\
            d_X(x^{(n)}, x_0) < \delta \implies d_Y\big(f(x^{(n)}), L\big) < \varepsilon
        \end{cases}
    \]
    and
    \[
        \forall\ \varepsilon \in \mathbf{R}^+, \exists\ N \in \mathbf{Z}^+ : \forall\ n \geq N, d_Y\big(f(x^{(n)}, L)\big) < \varepsilon.
    \]
    By Definition \ref{1.1.14} we have \(\lim_{n \to \infty} d_Y\big(f(x^{(n)}), L\big) = 0\).
    Since \((x^{(n)})_{n = 1}^\infty\) is arbitrary, we conclude that (a) implies (b).

    Next we show that statement (b) implies statement (a).
    Suppose that if \((x^{(n)})_{n = 1}^\infty\) is a sequence in \(X\) such that \(\lim_{n \to \infty} d_X(x^{(n)}, x_0) = 0\), then \(\lim_{n \to \infty} d_Y\big(f(x), L\big) = 0\).
    Suppose for sake of contradiction that
    \[
        d_Y - \lim_{x \to x_0 ; x \in X} f(x) \neq L.
    \]
    Then by Definition \ref{3.1.1} we have
    \[
        \exists\ \varepsilon \in \mathbf{R}^+ : \forall\ \delta \in \mathbf{R}^+, \exists\ x \in X : \begin{cases}
            d_X(x, x_0) < \delta \\
            d_Y\big(f(x), L\big) \geq \varepsilon
        \end{cases}
    \]
    Thus we can choose one sequence \((x^{(n)})_{n = 1}^\infty\) which satsifies
    \[
        \forall\ n \in \mathbf{Z}^+, \begin{cases}
            d_X(x^{(n)}, x_0) < \frac{1}{n} \\
            d_Y\big(f(x^{(n)}), L\big) \geq \varepsilon
        \end{cases}
    \]
    By squeeze test we have \(\lim_{n \to \infty} d_X(x^{(n)}, x_0) = 0\).
    But by hypothesis we know that \(\lim_{n \to \infty} d_Y\big(f(x^{(n)}), L\big) = 0\), which means
    \[
        \exists\ N \in \mathbf{Z}^+ : \forall\ n \geq N, d_Y\big(f(x^{(n)}), L\big) < \varepsilon,
    \]
    a contradiction.
    Thus we have
    \[
        d_Y - \lim_{x \to x_0 ; x \in X} f(x) = L
    \]
    and we conclude that statements (a)(b) are equivalent.

    Next we show that statement (a) implies statement (c).
    Suppose that
    \[
        d_Y - \lim_{x \to x_0 ; x \in E} f(x) = L.
    \]
    By Definition \ref{3.1.1} we have
    \begin{align*}
                 & \forall\ \varepsilon \in \mathbf{R}^+, \exists\ \delta \in \mathbf{R}^+ : \Big(\forall\ x \in E, d_X(x, x_0) < \delta \implies d_Y\big(f(x), L\big) < \varepsilon\Big)  \\
        \implies & \forall\ \varepsilon \in \mathbf{R}^+, \exists\ \delta \in \mathbf{R}^+ : \Big(x \in B_{(X, d_X)}(x_0, \delta) \cap E \implies d_Y\big(f(x), L\big) < \varepsilon\Big).
    \end{align*}
    Let \(V\) be an open set in \((Y, d_Y)\) such that \(L \in V\).
    Then we have
    \begin{align*}
                 & V = \text{int}_{(Y, d_Y)}(V)                                                     & \text{(by Proposition \ref{1.2.15}(a))} \\
        \implies & \exists\ \varepsilon \in \mathbf{R}^+ : B_{(Y, d_Y)}(L, \varepsilon) \subseteq V & \text{(by Definition \ref{1.2.5})}      \\
        \implies & \exists\ \delta \in \mathbf{R}^+ :                                                                                         \\
                 & \begin{cases}
            x \in B_{(X, d_X)}(x_0, \delta) \cap E \implies d_Y\big(f(x), L\big) < \varepsilon \\
            f\big(B_{(X, d_X)}(x_0, \delta) \cap E\big) \subseteq B_{(Y, d_Y)}\big(L, \varepsilon\big) \subseteq V
        \end{cases}
    \end{align*}
    and by Proposition \ref{1.2.15}(c) we know that \(B_{(X, d_X)}(x_0, \delta)\) is open in \((X, d_X)\).
    Since \(V\) is arbitrary, we conclude that statement (a) implies statement (c).

    Next we show that statement (c) implies statement (a).
    Suppose that
    \[
        \forall\ V \subseteq Y, \begin{cases}
            L \in V \\
            V \text{ is open in } (Y, d_Y)
        \end{cases} \implies \exists\ U \subseteq X : \begin{cases}
            x_0 \in U                      \\
            U \text{ is open in } (X, d_X) \\
            f(U \cap E) \subseteq V
        \end{cases}
    \]
    Let \(\varepsilon \in \mathbf{R}^+\).
    By Proposition \ref{1.2.15}(c) we know that \(B_{(Y, d_Y)}(L, \varepsilon)\) is open in \((Y, d_Y)\).
    By hypothesis we know that there exists some \(U \subseteq X\) such that
    \[
        \begin{cases}
            x_0 \in U                      \\
            U \text{ is open in } (X, d_X) \\
            f(U \cap E) \subseteq B_{(Y, d_Y)}(L, \varepsilon)
        \end{cases}
    \]
    Then we have
    \begin{align*}
                 & \begin{cases}
            x_0 \in U \\
            U = \text{int}_{(X, d_X)}(U)
        \end{cases}                                                                               & \text{(by Proposition \ref{1.2.15}(a))} \\
        \implies & \exists\ \delta \in \mathbf{R}^+ : B_{(X, d_X)}(x_0, \delta) \subseteq U                                 & \text{(by Definition \ref{1.2.5})}      \\
        \implies & \exists\ \delta \in \mathbf{R}^+ : B_{(X, d_X)}(x_0, \delta) \cap E \subseteq U \cap E                                                             \\
        \implies & \exists\ \delta \in \mathbf{R}^+ :                                                                                                                 \\
                 & f\big(B_{(X, d_X)}(x_0, \delta) \cap E\big) \subseteq f(U \cap E) \subseteq B_{(Y, d_Y)}(L, \varepsilon)                                           \\
        \implies & \exists\ \delta \in \mathbf{R}^+ :                                                                                                                 \\
                 & \Big(\forall\ x \in E, d_X(x, x_0) < \delta \implies d_Y\big(f(x), L\big) < \varepsilon\Big).
    \end{align*}
    Since \(\varepsilon\) is arbitrary, by Definition \ref{3.1.1} we have
    \[
        d_Y - \lim_{x \to x_0 ; x \in E} f(x) = L
    \]
    and we conclude that statements (a)(c) are equivalent.

    Next we show that statement (a) implies statement (d).
    Suppose that
    \[
        d_Y - \lim_{x \to x_0 ; x \in E} f(x) = L.
    \]
    Then by Definition \ref{3.1.1} we have
    \[
        \forall\ \varepsilon \in \mathbf{R}^+, \exists\ \delta \in \mathbf{R}^+ : \Big(\forall\ x \in E, d_X(x, x_0) < \delta \implies d_Y\big(f(x), L\big) < \varepsilon\Big).
    \]
    Let \(g : E \cup \{x_0\} \to Y\) be a function where
    \[
        \forall\ x \in E \cup \{x_0\}, g(x) = \begin{cases}
            L    & \text{if } x = x_0    \\
            f(x) & \text{if } x \neq x_0
        \end{cases}
    \]
    Then we have
    \begin{align*}
                 & \forall\ \varepsilon \in \mathbf{R}^+, \exists\ \delta \in \mathbf{R}^+ :                                       \\
                 & \Big(\forall\ x \in E, d_X(x, x_0) < \delta \implies d_Y\big(f(x), L\big) < \varepsilon\Big)                    \\
        \implies & \forall\ \varepsilon \in \mathbf{R}^+, \exists\ \delta \in \mathbf{R}^+ :                                       \\
                 & \Big(\forall\ x \in E \cup \{x_0\}, d_X(x, x_0) < \delta \implies d_Y\big(g(x), g(x_0)\big) < \varepsilon\Big).
    \end{align*}
    Thus by Definition \ref{2.1.1} \(g\) is continuous at \(x_0\) from \((E \cup \{x_0\}, d_X|_{(E \cup \{x_0\}) \times (E \cup \{x_0\})})\) to \((Y, d_Y)\).

    Finally we show that statement (d) implies statement (a).
    Suppose that \(g : E \cup \{x_0\} \to Y\) is a function where
    \[
        \forall\ x \in E \cup \{x_0\}, g(x) = \begin{cases}
            L    & \text{if } x = x_0    \\
            f(x) & \text{if } x \neq x_0
        \end{cases}
    \]
    and \(g\) is continuous from \((E \cup \{x_0\}, d_X|_{(E \cup \{x_0\}) \times (E \cup \{x_0\})})\) to \((Y, d_Y)\).
    Then by Definition \ref{2.1.1} we have
    \begin{align*}
                 & \forall\ \varepsilon \in \mathbf{R}^+, \exists\ \delta \in \mathbf{R}^+ :                                      \\
                 & \Big(\forall\ x \in E \cup \{x_0\}, d_X(x, x_0) < \delta \implies d_Y\big(g(x), g(x_0)\big) < \varepsilon\Big) \\
        \implies & \forall\ \varepsilon \in \mathbf{R}^+, \exists\ \delta \in \mathbf{R}^+ :                                      \\
                 & \Big(\forall\ x \in E, d_X(x, x_0) < \delta \implies d_Y\big(f(x), L\big) < \varepsilon\Big).
    \end{align*}
    By Definition \ref{3.1.1} this means
    \[
        d_Y - \lim_{x \to x_0 ; x \in E} f(x) = L.
    \]
    We conclude that statements (a)(b)(c)(d) are all equivalent.
\end{proof}

\begin{remark}\label{3.1.6}
    Observe from Proposition \ref{3.1.5}(b) and Proposition \ref{1.1.20} that a function \(f(x)\) can converge to at most one limit \(L\) as \(x\) converges to \(x_0\).
    In other words, if the limit
    \[
        \lim_{x \to x_0 ; x \in E} f(x)
    \]
    exists at all, then it can only take at most one value.
\end{remark}

\begin{remark}\label{3.1.7}
    The requirement that \(x_0\) be an adherent point of \(E\) is necessary for the concept of limiting value to be useful, otherwise \(x_0\) will lie in the exterior of \(E\), the notion that \(f(x)\) converges to \(L\) as \(x\) converges to \(x_0\) in \(E\) is vacuous
    (for \(\delta\) sufficiently small, there are no points \(x \in E\) so that \(d(x, x_0) < \delta\)).
\end{remark}

\begin{remark}\label{3.1.8}
    Strictly speaking, we should write
    \[
        d_Y - \lim_{x \to x_0 ; x \in E} f(x) \text{ instead of } \lim_{x \to x_0 ; x \in E} f(x),
    \]
    since the convergence depends on the metric \(d_Y\).
    However in practice it will be obvious what the metric \(d_Y\) is and so we will omit the \(d_Y -\) prefix from the notation.
\end{remark}

\exercisesection

\begin{exercise}\label{ex 3.1.1}
    Let \((X, d_X)\) and \((Y, d_Y)\) be metric spaces, let \(E\) be a subset of \(X\), let \(f : E \to Y\) be a function, and let \(x_0\) be an element of \(E\).
    Assume that \(x_0\) is an adherent point of \(E \setminus \{x_0\}\)
    (or equivalently, that \(x_0\) is not an \emph{isolated point} of \(E\)).
    Show that the limit \(\lim_{x \to x_0 ; x \in E} f(x)\) exists if and only if the limit \(\lim_{x \to x_0 ; x \in E \setminus \{x_0\}} f(x)\) exists and is equal to \(f(x_0)\).
    Also, show that if the limit \(\lim_{x \to x_0 ; x \in E} f(x)\) exists at all, then it must equal \(f(x_0)\).
\end{exercise}

\begin{proof}
    Let \(L \in Y\).
    By Definition \ref{1.1.2}(a) we know that
    \[
        \forall\ \varepsilon \in \mathbf{R}^+, d_Y\big(f(x_0), L\big) < \varepsilon \iff L = f(x_0).
    \]
    Thus we have
    \begin{align*}
             & d_Y - \lim_{x \to x_0 ; x \in E \setminus \{x_0\}} f(x) = f(x_0)                                                                                         \\
        \iff & \forall\ \varepsilon \in \mathbf{R}^+, \exists\ \delta \in \mathbf{R}^+ :                                                                                \\
             & \Big(\forall\ x \in E \setminus \{x_0\}, d_X(x, x_0) < \delta \implies d_Y\big(f(x), f(x_0)\big) < \varepsilon\Big) & \text{(by Definition \ref{3.1.1})} \\
        \iff & \forall\ \varepsilon \in \mathbf{R}^+, \exists\ \delta \in \mathbf{R}^+ :                                                                                \\
             & \Big(\forall\ x \in E, d_X(x, x_0) < \delta \implies d_Y\big(f(x), f(x_0)\big) < \varepsilon\Big)                   & (E \setminus \{x_0\} \subseteq E)  \\
        \iff & d_Y - \lim_{x \to x_0 ; x \in E} f(x) = f(x_0).                                                                     & \text{(by Definition \ref{3.1.1})}
    \end{align*}
\end{proof}

\begin{exercise}\label{ex 3.1.2}
    Prove Proposition \ref{3.1.5}.
\end{exercise}

\begin{proof}
    See Proposition \ref{3.1.5}.
\end{proof}

\begin{exercise}\label{ex 3.1.3}
    Use Proposition \ref{3.1.5}(c) to define a notion of a limiting value of a function \(f : E \to Y\) from one topological space \((X, \mathcal{F}_X)\) to another \((Y, \mathcal{F}_Y)\) where \(E \subseteq X\).
    If \(X\) is a topological space and \(Y\) is a Hausdorff topological space (see Exercise \ref{ex 2.5.4}), prove the equivalence of Proposition \ref{3.1.5}(c) and \ref{3.1.5}(d) in this setting, as well as an analogue of Remark \ref{3.1.6}.
    What happens to these statements of \(Y\) is not Hausdorff?
\end{exercise}

\begin{exercise}\label{ex 3.1.4}
    Recall from Exercise \ref{ex 2.5.5} that the extended real line \(\mathbf{R}^*\) comes with a standard topology (the order topology).
    We view the natural numbers \(\mathbf{N}\) as a subspace of this topological space, and \(+\infty\) as an adherent point of \(\mathbf{N}\) in \(\mathbf{R}^*\).
    Let \((a_n)_{n = 0}^\infty\) be a sequence taking values in a topological space \((Y, \mathcal{F}_Y)\), and let \(L \in Y\).
    Show that \(\lim_{n \to +\infty ; n \in \mathbf{N}} a_n = L\) (in the sense of Exercise \ref{ex 3.1.3}) if and only if \(\lim_{n \to \infty} a_n = L\) (in the sense of Definition \ref{2.5.4}).
    This shows that the notions of limiting values of a sequence, and limiting values of a function, are compatible.
\end{exercise}

\begin{exercise}\label{ex 3.1.5}
    Let \((X, d_X)\), \((Y, d_Y)\), \((Z, d Z)\) be metric spaces, and let \(x_0 \in X\), \(y_0 \in Y\), \(z_0 \in Z\).
    Let \(f : E \to Y\) and \(g : Y \to Z\) be functions, and let \(E\) be a set.
    If we have \(\lim_{x \to x_0 ; x \in E} f(x) = y_0\) and \(\lim_{y \to y_0 ; y \in f(E)} g(y) = z_0\), conclude that \(\lim_{x \to x_0 ; x \in E} g \circ f(x) = z_0\).
\end{exercise}

\begin{exercise}\label{ex 3.1.6}
    State and prove an analogue of the limit laws in Proposition 9.3.14 in Analysis I when \(X\) is now a metric space rather than a subset of \(\mathbf{R}\).
\end{exercise}