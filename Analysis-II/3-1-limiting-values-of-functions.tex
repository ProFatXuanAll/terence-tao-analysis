\section{Limiting values of functions}\label{ii:sec:3.1}

\begin{defn}[Limiting value of a function]\label{ii:3.1.1}
  Let \((X, d_X)\) and \((Y, d_Y)\) be metric spaces, let \(E\) be a subset of \(X\), and let \(f : E \to Y\) be a function.
  If \(x_0 \in X\) is an adherent point of \(E\), and \(L \in Y\), we say that \emph{\(f(x)\) converges to \(L\) in \(Y\) as \(x\) converges to \(x_0\) in \(E\)}, or write \(\lim_{x \to x_0 ; x \in E} f(x) = L\), if for every \(\varepsilon > 0\) there exists a \(\delta > 0\) such that \(d_Y\big(f(x), L\big) < \varepsilon\) for all \(x \in E\) such that \(d_X(x, x_0) < \delta\).
\end{defn}

\begin{rmk}\label{ii:3.1.2}
  Some authors exclude the case \(x = x_0\) from the above definition, thus requiring \(0 < d_X(x, x_0) < \delta\).
  In our current notation, this would correspond to removing \(x_0\) from \(E\), thus one would consider
  \[
    \lim_{x \to x_0 ; x \in E \setminus \set{x_0}} f(x)
  \]
  instead of
  \[
    \lim_{x \to x_0 ; x \in E} f(x).
  \]
\end{rmk}

\begin{note}
  Comparing this with \cref{ii:2.1.1}, we see that \(f\) is continuous at \(x_0\) iff
  \[
    \lim_{x \to x_0 ; x \in X} f(x) = f(x_0).
  \]
  Thus, \(f\) is continuous on \(X\) iff we have
  \[
    \lim_{x \to x_0 ; x \in X} f(x) = f(x_0) \text{ for all } x_0 \in X.
  \]
\end{note}

\setcounter{thm}{3}
\begin{rmk}\label{ii:3.1.4}
  Often we shall omit the condition \(x \in X\), and abbreviate
  \[
    \lim_{x \to x_0 ; x \in X} f(x)
  \]
  as simply
  \[
    \lim_{x \to x_0} f(x)
  \]
  when it is clear what space \(x\) will range in.
\end{rmk}

\begin{prop}\label{ii:3.1.5}
  Let \((X, d_X)\) and \((Y, d_Y)\) be metric spaces, let \(E\) be a subset of \(X\), and let \(f : E \to Y\) be a function.
  Let \(x_0 \in X\) be an adherent point of \(E\) and \(L \in Y\).
  Then the following four statements are logically equivalent:
  \begin{enumerate}
    \item \(\lim_{x \to x_0 ; x \in E} f(x) = L\).
    \item For every sequence \((x^{(n)})_{n = 1}^\infty\) in \(E\) which converges to \(x_0\) with respect to the metric \(d_X\), the sequence \(\big(f(x^{(n)})\big)_{n = 1}^\infty\) converges to \(L\) with respect to the metric \(d_Y\).
    \item For every open set \(V \subseteq Y\) which contains \(L\), there exists an open set \(U \subseteq X\) containing \(x_0\) such that \(f(U \cap E) \subseteq V\).
    \item If one defines the function \(g : E \cup \set{x_0} \to Y\) by defining \(g(x_0) \coloneqq L\), and \(g(x) \coloneqq f(x)\) for \(x \in E \setminus \set{x_0}\), then \(g\) is continuous at \(x_0\).
          Furthermore, if \(x_0 \in E\), then \(f(x_0) = L\).
  \end{enumerate}
\end{prop}

\begin{proof}
  We first show that statement (a) implies statement (b).
  Suppose that
  \[
    d_Y - \lim_{x \to x_0 ; x \in E} f(x) = L.
  \]
  By \cref{ii:3.1.1} we have
  \[
    \forall \varepsilon \in \R^+, \exists \delta \in \R^+ : \Big(\forall x \in E, d_X(x, x_0) < \delta \implies d_Y\big(f(x), L\big) < \varepsilon\Big).
  \]
  Let \((x^{(n)})_{n = 1}^\infty\) be a sequence in \(E\) such that \(\lim_{n \to \infty} d_X(x^{(n)}, x_0) = 0\).
  By \cref{ii:1.1.14} we have
  \[
    \forall \delta \in \R^+, \exists N \in \Z^+ : \forall n \geq N, d_X(x^{(n)}, x_0) < \delta.
  \]
  Since \((x^{(n)})_{n = 1}^\infty\) is in \(E\), we have
  \[
    \forall \varepsilon \in \R^+, \exists \delta \in \R^+ : \begin{dcases}
      \exists N \in \Z^+ : \forall n \geq N, d_X(x^{(n)}, x_0) < \delta \\
      d_X(x^{(n)}, x_0) < \delta \implies d_Y\big(f(x^{(n)}), L\big) < \varepsilon
    \end{dcases}
  \]
  and
  \[
    \forall \varepsilon \in \R^+, \exists N \in \Z^+ : \forall n \geq N, d_Y\big(f(x^{(n)}, L)\big) < \varepsilon.
  \]
  By \cref{ii:1.1.14} we have \(\lim_{n \to \infty} d_Y\big(f(x^{(n)}), L\big) = 0\).
  Since \((x^{(n)})_{n = 1}^\infty\) was arbitrary, we conclude that (a) implies (b).

  Next we show that statement (b) implies statement (a).
  Suppose that if \((x^{(n)})_{n = 1}^\infty\) is a sequence in \(X\) such that \(\lim_{n \to \infty} d_X(x^{(n)}, x_0) = 0\), then \(\lim_{n \to \infty} d_Y\big(f(x), L\big) = 0\).
  Suppose for the sake of contradiction that
  \[
    d_Y - \lim_{x \to x_0 ; x \in X} f(x) \neq L.
  \]
  Then by \cref{ii:3.1.1} we have
  \[
    \exists \varepsilon \in \R^+ : \forall \delta \in \R^+, \exists x \in X : \begin{dcases}
      d_X(x, x_0) < \delta \\
      d_Y\big(f(x), L\big) \geq \varepsilon
    \end{dcases}
  \]
  Thus, we can choose one sequence \((x^{(n)})_{n = 1}^\infty\) which satsifies
  \[
    \forall n \in \Z^+, \begin{dcases}
      d_X(x^{(n)}, x_0) < \dfrac{1}{n} \\
      d_Y\big(f(x^{(n)}), L\big) \geq \varepsilon
    \end{dcases}
  \]
  By squeeze test we have \(\lim_{n \to \infty} d_X(x^{(n)}, x_0) = 0\).
  But by hypothesis we know that \(\lim_{n \to \infty} d_Y\big(f(x^{(n)}), L\big) = 0\), which means
  \[
    \exists N \in \Z^+ : \forall n \geq N, d_Y\big(f(x^{(n)}), L\big) < \varepsilon,
  \]
  a contradiction.
  Thus, we have
  \[
    d_Y - \lim_{x \to x_0 ; x \in X} f(x) = L
  \]
  and we conclude that statements (a)(b) are equivalent.

  Next we show that statement (a) implies statement (c).
  Suppose that
  \[
    d_Y - \lim_{x \to x_0 ; x \in E} f(x) = L.
  \]
  By \cref{ii:3.1.1} we have
  \begin{align*}
             & \forall \varepsilon \in \R^+, \exists \delta \in \R^+ : \Big(\forall x \in E, d_X(x, x_0) < \delta \implies d_Y\big(f(x), L\big) < \varepsilon\Big)   \\
    \implies & \forall \varepsilon \in \R^+, \exists \delta \in \R^+ : \Big(x \in B_{(X, d_X)}(x_0, \delta) \cap E \implies d_Y\big(f(x), L\big) < \varepsilon\Big).
  \end{align*}
  Let \(V\) be an open set in \((Y, d_Y)\) such that \(L \in V\).
  Then we have
  \begin{align*}
             & V = \text{int}_{(Y, d_Y)}(V)                                                                           &  & \by{ii:1.2.15}[a] \\
    \implies & \exists \varepsilon \in \R^+ : B_{(Y, d_Y)}(L, \varepsilon) \subseteq V                                &  & \by{ii:1.2.5}     \\
    \implies & \exists \delta \in \R^+ :                                                                                                     \\
             & \begin{dcases}
                 x \in B_{(X, d_X)}(x_0, \delta) \cap E \implies d_Y\big(f(x), L\big) < \varepsilon \\
                 f\big(B_{(X, d_X)}(x_0, \delta) \cap E\big) \subseteq B_{(Y, d_Y)}\big(L, \varepsilon\big) \subseteq V
               \end{dcases}
  \end{align*}
  and by \cref{ii:1.2.15}(c) we know that \(B_{(X, d_X)}(x_0, \delta)\) is open in \((X, d_X)\).
  Since \(V\) was arbitrary, we conclude that statement (a) implies statement (c).

  Next we show that statement (c) implies statement (a).
  Suppose that
  \[
    \forall V \subseteq Y, \begin{dcases}
      L \in V \\
      V \text{ is open in } (Y, d_Y)
    \end{dcases} \implies \exists U \subseteq X : \begin{dcases}
      x_0 \in U                      \\
      U \text{ is open in } (X, d_X) \\
      f(U \cap E) \subseteq V
    \end{dcases}
  \]
  Let \(\varepsilon \in \R^+\).
  By \cref{ii:1.2.15}(c) we know that \(B_{(Y, d_Y)}(L, \varepsilon)\) is open in \((Y, d_Y)\).
  By hypothesis we know that there exists some \(U \subseteq X\) such that
  \[
    \begin{dcases}
      x_0 \in U                      \\
      U \text{ is open in } (X, d_X) \\
      f(U \cap E) \subseteq B_{(Y, d_Y)}(L, \varepsilon)
    \end{dcases}
  \]
  Then we have
  \begin{align*}
             & \begin{dcases}
                 x_0 \in U \\
                 U = \text{int}_{(X, d_X)}(U)
               \end{dcases}                                                                             &  & \by{ii:1.2.15}[a]             \\
    \implies & \exists \delta \in \R^+ : B_{(X, d_X)}(x_0, \delta) \subseteq U                                          &  & \by{ii:1.2.5} \\
    \implies & \exists \delta \in \R^+ : B_{(X, d_X)}(x_0, \delta) \cap E \subseteq U \cap E                                               \\
    \implies & \exists \delta \in \R^+ :                                                                                                   \\
             & f\big(B_{(X, d_X)}(x_0, \delta) \cap E\big) \subseteq f(U \cap E) \subseteq B_{(Y, d_Y)}(L, \varepsilon)                    \\
    \implies & \exists \delta \in \R^+ :                                                                                                   \\
             & \Big(\forall x \in E, d_X(x, x_0) < \delta \implies d_Y\big(f(x), L\big) < \varepsilon\Big).
  \end{align*}
  Since \(\varepsilon\) was arbitrary, by \cref{ii:3.1.1} we have
  \[
    d_Y - \lim_{x \to x_0 ; x \in E} f(x) = L
  \]
  and we conclude that statements (a)(c) are equivalent.

  Next we show that statement (a) implies statement (d).
  Suppose that
  \[
    d_Y - \lim_{x \to x_0 ; x \in E} f(x) = L.
  \]
  Then by \cref{ii:3.1.1} we have
  \[
    \forall \varepsilon \in \R^+, \exists \delta \in \R^+ : \Big(\forall x \in E, d_X(x, x_0) < \delta \implies d_Y\big(f(x), L\big) < \varepsilon\Big).
  \]
  Let \(g : E \cup \set{x_0} \to Y\) be a function where
  \[
    \forall x \in E \cup \set{x_0}, g(x) = \begin{dcases}
      L    & \text{if } x = x_0    \\
      f(x) & \text{if } x \neq x_0
    \end{dcases}
  \]
  Then we have
  \begin{align*}
             & \forall \varepsilon \in \R^+, \exists \delta \in \R^+ :                                                          \\
             & \Big(\forall x \in E, d_X(x, x_0) < \delta \implies d_Y\big(f(x), L\big) < \varepsilon\Big)                      \\
    \implies & \forall \varepsilon \in \R^+, \exists \delta \in \R^+ :                                                          \\
             & \Big(\forall x \in E \cup \set{x_0}, d_X(x, x_0) < \delta \implies d_Y\big(g(x), g(x_0)\big) < \varepsilon\Big).
  \end{align*}
  Thus, by \cref{ii:2.1.1} \(g\) is continuous at \(x_0\) from \((E \cup \set{x_0}, d_X|_{(E \cup \set{x_0}) \times (E \cup \set{x_0})})\) to \((Y, d_Y)\).

  Now suppose that \(x_0 \in E\).
  We claim that \(f(x_0) = L\).
  Suppose for the sake of contradiction that \(f(x_0) \neq L\).
  Then by \cref{ii:1.1.2}(b) we have \(d_Y\big(f(x_0), L\big) > 0\).
  Let \(r = d_Y\big(f(x_0), L\big)\).
  By \cref{ii:3.1.1} we have
  \[
    \exists \delta \in \R^+ : \forall x \in E, d_X(x, x_0) < \delta \implies d_Y\big(f(x), L\big) < r.
  \]
  Since \(x_0 \in E\), we have \(d_X(x_0, x_0) = 0 < \delta\).
  But then we have \(d_Y\big(f(x_0), L\big) < r = d_Y\big(f(x_0), L\big)\), a contradiction.
  Thus, we have \(f(x_0) = L\).

  Finally we show that statement (d) implies statement (a).
  Suppose that \(g : E \cup \set{x_0} \to Y\) is a function where
  \[
    \forall x \in E \cup \set{x_0}, g(x) = \begin{dcases}
      L    & \text{if } x = x_0    \\
      f(x) & \text{if } x \neq x_0
    \end{dcases}
  \]
  and \(g\) is continuous from \((E \cup \set{x_0}, d_X|_{(E \cup \set{x_0}) \times (E \cup \set{x_0})})\) to \((Y, d_Y)\).
  Suppose also that if \(x_0 \in E\), then \(f(x_0) = L\).
  Then by \cref{ii:2.1.1} we have
  \begin{align*}
             & \forall \varepsilon \in \R^+, \exists \delta \in \R^+ :                                                         \\
             & \Big(\forall x \in E \cup \set{x_0}, d_X(x, x_0) < \delta \implies d_Y\big(g(x), g(x_0)\big) < \varepsilon\Big) \\
    \implies & \forall \varepsilon \in \R^+, \exists \delta \in \R^+ :                                                         \\
             & \begin{dcases}
                 \forall x \in E \setminus \set{x_0}, d_X(x, x_0) < \delta \implies d_Y\big(f(x), L\big) < \varepsilon \\
                 x_0 \in E \implies d_X(x_0, x_0) = 0 < \delta \implies f(x_0) = L \implies d_Y\big(f(x_0), L\big) < \varepsilon
               \end{dcases}  \\
    \implies & \forall \varepsilon \in \R^+, \exists \delta \in \R^+ :                                                         \\
             & \Big(\forall x \in E, d_X(x, x_0) < \delta \implies d_Y\big(f(x), L\big) < \varepsilon\Big).
  \end{align*}
  By \cref{ii:3.1.1} this means
  \[
    d_Y - \lim_{x \to x_0 ; x \in E} f(x) = L.
  \]
  We conclude that statements (a)(b)(c)(d) are all equivalent.
\end{proof}

\begin{rmk}\label{ii:3.1.6}
  Observe from \cref{ii:3.1.5}(b) and \cref{ii:1.1.20} that a function \(f(x)\) can converge to at most one limit \(L\) as \(x\) converges to \(x_0\).
  In other words, if the limit
  \[
    \lim_{x \to x_0 ; x \in E} f(x)
  \]
  exists at all, then it can only take at most one value.
\end{rmk}

\begin{rmk}\label{ii:3.1.7}
  The requirement that \(x_0\) be an adherent point of \(E\) is necessary for the concept of limiting value to be useful, otherwise \(x_0\) will lie in the exterior of \(E\), the notion that \(f(x)\) converges to \(L\) as \(x\) converges to \(x_0\) in \(E\) is vacuous
  (for \(\delta\) sufficiently small, there are no points \(x \in E\) so that \(d(x, x_0) < \delta\)).
\end{rmk}

\begin{rmk}\label{ii:3.1.8}
  Strictly speaking, we should write
  \[
    d_Y - \lim_{x \to x_0 ; x \in E} f(x) \text{ instead of } \lim_{x \to x_0 ; x \in E} f(x),
  \]
  since the convergence depends on the metric \(d_Y\).
  However in practice it will be obvious what the metric \(d_Y\) is and so we will omit the \(d_Y -\) prefix from the notation.
\end{rmk}

\exercisesection

\begin{ex}\label{ii:ex:3.1.1}
  Let \((X, d_X)\) and \((Y, d_Y)\) be metric spaces, let \(E\) be a subset of \(X\), let \(f : E \to Y\) be a function, and let \(x_0\) be an element of \(E\).
  Assume that \(x_0\) is an adherent point of \(E \setminus \set{x_0}\)
  (or equivalently, that \(x_0\) is not an \emph{isolated point} of \(E\)).
  Show that the limit \(\lim_{x \to x_0 ; x \in E} f(x)\) exists iff the limit \(\lim_{x \to x_0 ; x \in E \setminus \set{x_0}} f(x)\) exists and is equal to \(f(x_0)\).
  Also, show that if the limit \(\lim_{x \to x_0 ; x \in E} f(x)\) exists at all, then it must equal \(f(x_0)\).
\end{ex}

\begin{proof}
  Let \(L \in Y\).
  By \cref{ii:1.1.2}(a) we know that
  \[
    \forall \varepsilon \in \R^+, d_Y\big(f(x_0), L\big) < \varepsilon \iff L = f(x_0).
  \]
  Thus, we have
  \begin{align*}
         & d_Y - \lim_{x \to x_0 ; x \in E \setminus \set{x_0}} f(x) = f(x_0)                                                                                                         \\
    \iff & \forall \varepsilon \in \R^+, \exists \delta \in \R^+ :                                                                                                                    \\
         & \Big(\forall x \in E \setminus \set{x_0}, d_X(x, x_0) < \delta \implies d_Y\big(f(x), f(x_0)\big) < \varepsilon\Big) &                                     & \by{ii:3.1.1} \\
    \iff & \forall \varepsilon \in \R^+, \exists \delta \in \R^+ :                                                                                                                    \\
         & \Big(\forall x \in E, d_X(x, x_0) < \delta \implies d_Y\big(f(x), f(x_0)\big) < \varepsilon\Big)                     & (E \setminus \set{x_0} \subseteq E)                 \\
    \iff & d_Y - \lim_{x \to x_0 ; x \in E} f(x) = f(x_0).                                                                      &                                     & \by{ii:3.1.1}
  \end{align*}
\end{proof}

\begin{ex}\label{ii:ex:3.1.2}
  Prove \cref{ii:3.1.5}.
\end{ex}

\begin{proof}
  See \cref{ii:3.1.5}.
\end{proof}

\begin{ex}\label{ii:ex:3.1.3}
  Use \cref{ii:3.1.5}(c) to define a notion of a limiting value of a function \(f : E \to Y\) from one topological space \((X, \mathcal{F}_X)\) to another \((Y, \mathcal{F}_Y)\) where \(E \subseteq X\).
  If \(X\) is a topological space and \(Y\) is a Hausdorff topological space (see \cref{ii:ex:2.5.4}), prove the equivalence of \cref{ii:3.1.5}(c)(d) in this setting, as well as an analogue of \cref{ii:3.1.6}.
  What happens to these statements of \(Y\) is not Hausdorff?
\end{ex}

\begin{proof}
  Let \((X, \mathcal{F}_X)\), \((Y, \mathcal{F}_Y)\) be topological spaces, let \(E \subseteq X\), let \(f : E \to Y\) be a function, let \(x_0 \in \overline{E}_{(X, \mathcal{F}_X)}\), and let \(L \in Y\).
  We say that \(f(x)\) converges to \(L\) in \(Y\) as \(x\) converges to \(x_0\) in \(E\) iff
  \[
    \forall V \in \mathcal{F}_Y, L \in V \implies \exists U \in \mathcal{F}_X : \begin{dcases}
      x_0 \in U \\
      f(U \cap E) \subseteq V
    \end{dcases}
  \]
  We want to show that if \((Y, \mathcal{F}_Y)\) is Hausdorff, then the definition above is equivalent to the follow:
  If \(g : E \cup \set{x_0} \to Y\) is a function such that
  \[
    \forall x \in E \cup \set{x_0}, g(x) = \begin{dcases}
      L    & \text{if } x = x_0    \\
      f(x) & \text{if } x \neq x_0
    \end{dcases}
  \]
  and \((E \cup \set{x_0}, \mathcal{F}_{E \cup \set{x_0}})\) is a topological subspace induced by \((X, \mathcal{F}_X)\), then \(g\) is continuous at \(x_0\) from \((E \cup \set{x_0}, \mathcal{F}_{E \cup \set{x_0}})\) to \((Y, \mathcal{F}_Y)\).

  First suppose that \(f(x)\) converges to \(L\) in \(Y\) as \(x\) converges to \(x_0\) in \(E\).
  Let \(g\) be the function in the definition and let \(V \in \mathcal{F}_Y\) such that \(L \in V\).
  By hypothesis we know that
  \[
    \exists U \in \mathcal{F}_X : \begin{dcases}
      x_0 \in U \\
      f(U \cap E) \subseteq V
    \end{dcases}
  \]
  Then by \cref{ii:2.5.7} we have \(U \cap (E \cup \set{x_0}) \in \mathcal{F}_{E \cup \set{x_0}}\) and
  \begin{align*}
    g\big(U \cap (E \cup \set{x_0})\big) & = g\big((U \cap E) \cup \set{x_0}\big)                        \\
                                         & = g\big((U \cap E) \setminus \set{x_0}\big) \cup g(\set{x_0}) \\
                                         & = f\big((U \cap E) \setminus \set{x_0}\big) \cup \set{L}      \\
                                         & \subseteq f(U \cap E) \cup \set{L}                            \\
                                         & \subseteq V.
  \end{align*}
  Since \(V\) was arbitrary, by \cref{ii:2.5.8} we know that \(g\) is continuous at \(x_0\) from \((E \cup \set{x_0}, \mathcal{F}_{E \cup \set{x_0}})\) to \((Y, \mathcal{F}_Y)\).

  Next suppose that \(x_0 \in E\) and \(f(x)\) converges to \(L\) in \(Y\) as \(x\) converges to \(x_0\) in \(E\).
  We want to show that \(f(x_0) = L\).
  Suppose for the sake of contradiction that \(f(x_0) \neq L\).
  Since \((Y, \mathcal{F}_Y)\) is Hausdorff, by \cref{ii:ex:2.5.4} we know that
  \[
    \exists V, W \in \mathcal{F}_Y : \begin{dcases}
      L \in V      \\
      f(x_0) \in W \\
      V \cap W = \emptyset
    \end{dcases}
  \]
  By definition we have
  \[
    \exists U_V, U_W \in \mathcal{F}_X : \begin{dcases}
      x_0 \in U_V               \\
      x_0 \in U_W               \\
      f(U_V \cap E) \subseteq V \\
      f(U_W \cap E) \subseteq W
    \end{dcases}
  \]
  By \cref{ii:2.5.1} we know that \(U_V \cap U_W \in \mathcal{F}_X\).
  But then we have
  \[
    \begin{dcases}
      x_0 \in U_V \cap U_W                                       \\
      f(U_V \cap U_W \cap E) \subseteq f(U_V \cap E) \subseteq V \\
      f(U_V \cap U_W \cap E) \subseteq f(U_W \cap E) \subseteq W
    \end{dcases}
  \]
  which means \(V \cap W \neq \emptyset\), a contradiction.
  Thus, we have \(f(x_0) = L\).

  Now suppose that \(g\) is the function in the definition such that \(g\) is continuous at \(x_0\) from \((E \cup \set{x_0}, \mathcal{F}_{E \cup \set{x_0}})\) to \((Y, \mathcal{F}_Y)\).
  Also suppose that if \(x_0 \in E\), then \(f(x_0) = L\).
  Let \(V \in \mathcal{F}_Y\) such that \(g(x_0) = L \in V\).
  By \cref{ii:2.5.8} we know that
  \[
    \exists U \in \mathcal{F}_{E \cup \set{x_0}} : \begin{dcases}
      x_0 \in U \\
      g(U) \subseteq V
    \end{dcases}
  \]
  By \cref{ii:2.5.7} we know that
  \[
    \exists U_X \in \mathcal{F}_X : U_X \cap (E \cup \set{x_0}) = U.
  \]
  Since \(x_0 \in U\), we know that \(x_0 \in U_X\).
  Thus, we have
  \begin{align*}
    f(U_X \cap E) & = f\big((U_X \cap E) \setminus \set{x_0}\big) \cup f(E \cap \set{x_0})  & (f(E \cap \set{x_0}) = \emptyset \iff x_0 \notin E) \\
                  & \subseteq g\big((U_X \cap E) \setminus \set{x_0}\big) \cup g(\set{x_0}) & (x_0 \in E \iff f(x_0) = L = g(x_0))                \\
                  & = g\big(U_X \cap (E \cup \set{x_0})\big)                                                                                      \\
                  & = g(U)                                                                                                                        \\
                  & \subseteq V.
  \end{align*}
  Since \(V\) was arbitrary, we conclude that \(f(x)\) converges to \(L\) in \(Y\) as \(x\) converges to \(x_0\) in \(E\).

  If \((Y, \mathcal{F}_Y)\) is not Hausdorff, then \(x_0 \in E\) may not implies \(f(x_0) = L\).
\end{proof}

\begin{ex}\label{ii:ex:3.1.4}
  Recall from \cref{ii:ex:2.5.5} that the extended real line \(\R^*\) comes with a standard topology (the order topology).
  We view the natural numbers \(\N\) as a subspace of this topological space, and \(+\infty\) as an adherent point of \(\N\) in \(\R^*\).
  Let \((a_n)_{n = 0}^\infty\) be a sequence taking values in a topological space \((Y, \mathcal{F}_Y)\), and let \(L \in Y\).
  Show that \(\lim_{n \to +\infty ; n \in \N} a_n = L\) (in the sense of \cref{ii:ex:3.1.3}) iff \(\lim_{n \to \infty} a_n = L\) (in the sense of \cref{ii:2.5.4}).
  This shows that the notions of limiting values of a sequence, and limiting values of a function, are compatible.
\end{ex}

\begin{proof}
  Let \((\R^*, \mathcal{F}_{\R^*})\) be the order topology in \cref{ii:ex:2.5.5}.
  Let \(f : \N \to Y\) be the function where \(f(n) = a_n\) for each \(n \in \N\).
  First suppose that
  \[
    \lim_{n \to +\infty ; n \in \N} a_n = \lim_{n \to +\infty ; n \in \N} f(n) = L.
  \]
  By \cref{ii:ex:3.1.3} we have
  \[
    \forall V \in \mathcal{F}_Y, L \in V \implies \exists U \in \mathcal{F}_{\R^*} : \begin{dcases}
      +\infty \in U \\
      f(U \cap \N) \subseteq V
    \end{dcases}
  \]
  Since \(U \in \mathcal{F}_{\R^*}\), by \cref{ii:ex:2.5.5} we know that there exists an interval \(I \subseteq \R^*\) such that \(I \subseteq U\) and \(+\infty \in I\).
  We know that \(I\) must in the form \((a, +\infty]\) for some \(a \in \R\).
  By Archimedean property we know that there exists some \(N \in \N\) such that \(N > a\).
  Then we have
  \begin{align*}
             & \begin{dcases}
                 I \subseteq U \subseteq \R^*            \\
                 I = (a, +\infty] \in \mathcal{F}_{\R^*} \\
                 N > a
               \end{dcases}     \\
    \implies & \forall n \geq N + 1, n \in I               \\
    \implies & \forall n \geq N + 1, n \in U \cap \N       \\
    \implies & \forall n \geq N + 1, f(n) \in f(U \cap \N) \\
    \implies & \forall n \geq N + 1, f(n) \in V.
  \end{align*}
  Since this is true for arbitrary \(V\), we have
  \[
    \lim_{n \to \infty} a_n = \lim_{n \to \infty} f(n) = L
  \]
  in the sense of \cref{ii:2.5.4}.

  Now suppose that
  \[
    \lim_{n \to \infty} a_n = \lim_{n \to \infty} f(n) = L
  \]
  in the sense of \cref{ii:2.5.4}.
  Then we have
  \[
    \forall V \in \mathcal{F}_Y, L \in V \implies \exists N \in \N : \forall n \geq N, f(n) \in V.
  \]
  Let \(I = (N, +\infty]\).
  Then we know that \(I\) is an interval of \(\R^*\) and by \cref{ii:ex:2.5.5} we have \(I \in \mathcal{F}_{\R^*}\).
  Observe that
  \begin{align*}
             & I \cap \N = \set{m \in \N : m \geq N + 1} \\
    \implies & \forall n \in I \cap \N, f(n) \in V       \\
    \implies & f(I \cap \N) \subseteq V.
  \end{align*}
  Since this is true for arbitrary \(V \in \mathcal{F}_Y\), by \cref{ii:ex:3.1.3} we have
  \[
    \lim_{n \to +\infty ; n \in \N} a_n = \lim_{n \to +\infty ; n \in \N} f(n) = L.
  \]
\end{proof}

\begin{ex}\label{ii:ex:3.1.5}
  Let \((X, d_X)\), \((Y, d_Y)\), \((Z, d Z)\) be metric spaces, and let \(x_0 \in X\), \(y_0 \in Y\), \(z_0 \in Z\).
  let \(E \subseteq X\) and let \(f : E \to Y\) and \(g : Y \to Z\) be functions.
  If we have \(\lim_{x \to x_0 ; x \in E} f(x) = y_0\) and \(\lim_{y \to y_0 ; y \in f(E)} g(y) = z_0\), conclude that \(\lim_{x \to x_0 ; x \in E} g \circ f(x) = z_0\).
\end{ex}

\begin{proof}
  By \cref{ii:3.1.1} we have
  \begin{align*}
             & d_Z - \lim_{y \to y_0 ; y \in f(E)} g(y) = z_0                                                    \\
    \implies & \forall \varepsilon \in \R^+, \exists \delta' \in \R^+ :                                          \\
             & \Big(\forall y \in f(E), d_Y(y, y_0) < \delta' \implies d_Z\big(g(y), z_0\big) < \varepsilon\Big)
  \end{align*}
  and
  \begin{align*}
             & d_Y - \lim_{x \to x_0 ; x \in E} f(x) = y_0                                                \\
    \implies & \forall \delta' \in \R^+, \exists \delta \in \R^+ :                                        \\
             & \Big(\forall x \in E, d_X(x, x_0) < \delta \implies d_Y\big(f(x), y_0\big) < \delta'\Big).
  \end{align*}
  Thus, we have
  \begin{align*}
     & \forall \varepsilon \in \R^+, \exists \delta \in \R^+ :                                                                                              \\
     & \bigg(\forall x \in E, d_X(x, x_0) < \delta \implies d_Y\big(f(x), y_0\big) < \delta' \implies d_Z\Big(g\big(f(x)\big), z_0\Big) < \varepsilon\bigg)
  \end{align*}
  and by \cref{ii:3.1.1} \(d_Z - \lim_{x \to x_0 ; x \in E} g \circ f(x) = z_0\).
\end{proof}

\begin{ex}\label{ii:ex:3.1.6}
  State and prove an analogue of the limit laws in Proposition 9.3.14 in Analysis I when \(X\) is now a metric space rather than a subset of \(\R\).
\end{ex}

\begin{proof}
  Let \((X, d)\) be a metric space, let \(d_1 = d_{l^1}|_{\R \times \R}\), let \(E \subseteq X\), let \(x_0 \in \overline{E}_{(X, d)}\), let \(f : X \to \R\) and \(g : X \to \R\) be functions, and let \(c \in \R\).
  Suppose that
  \begin{align*}
     & d_1 - \lim_{x \to x_0 ; x \in E} f(x) = L \\
     & d_1 - \lim_{x \to x_0 ; x \in E} g(x) = M
  \end{align*}
  We want to show that
  \begin{align*}
     & d_1 - \lim_{x \to x_0 ; x \in E} (f + g)(x) = L + M                                          \\
     & d_1 - \lim_{x \to x_0 ; x \in E} (f - g)(x) = L - M                                          \\
     & d_1 - \lim_{x \to x_0 ; x \in E} (fg)(x) = LM                                                \\
     & d_1 - \lim_{x \to x_0 ; x \in E} \min(f, g)(x) = \min(L, M)                                  \\
     & d_1 - \lim_{x \to x_0 ; x \in E} \max(f, g)(x) = \max(L, M)                                  \\
     & d_1 - \lim_{x \to x_0 ; x \in E} (cf)(x) = cL                                                \\
     & d_1 - \lim_{x \to x_0 ; x \in E} (f / g)(x) = L / M \text{ if } \begin{dcases}
                                                                         \forall x \in E, g(x) \neq 0 \\
                                                                         M \neq 0
                                                                       \end{dcases}
  \end{align*}
  Let \(f^* : E \cup \set{x_0} \to \R\) be the function
  \[
    \forall x \in E, f^*(x) = \begin{dcases}
      L    & \text{if } x = x_0    \\
      f(x) & \text{if } x \neq x_0
    \end{dcases}
  \]
  and let \(g^* : E \cup \set{x_0} \to \R\) be the function
  \[
    \forall x \in E, g^*(x) = \begin{dcases}
      M    & \text{if } x = x_0    \\
      g(x) & \text{if } x \neq x_0
    \end{dcases}
  \]
  By \cref{ii:3.1.5}(c) we know that \(f^*\) and \(g^*\) are continuous at \(x_0\) from \((X, d)\) to \((\R, d_1)\).
  Thus, by \cref{ii:2.2.3} we have
  \begin{align*}
     & f^* + g^* \text{ is continuous at } x_0 \text{ from } (X, d) \text{ to } (\R, d_1)                                          \\
     & f^* - g^* \text{ is continuous at } x_0 \text{ from } (X, d) \text{ to } (\R, d_1)                                          \\
     & f^* g^* \text{ is continuous at } x_0 \text{ from } (X, d) \text{ to } (\R, d_1)                                            \\
     & \min(f^*, g^*) \text{ is continuous at } x_0 \text{ from } (X, d) \text{ to } (\R, d_1)                                     \\
     & \max(f^*, g^*) \text{ is continuous at } x_0 \text{ from } (X, d) \text{ to } (\R, d_1)                                     \\
     & c f^* \text{ is continuous at } x_0 \text{ from } (X, d) \text{ to } (\R, d_1)                                              \\
     & f^* / g^* \text{ is continuous at } x_0 \text{ from } (X, d) \text{ to } (\R, d_1) \text{ if } \begin{dcases}
                                                                                                        \forall x \in E, g(x) \neq 0 \\
                                                                                                        M \neq 0
                                                                                                      \end{dcases}
  \end{align*}
  Since
  \begin{align*}
    \forall x \in E \cup \set{x_0}, & (f^* + g^*)(x) = \begin{dcases}
                                                         L + M       & \text{if } x = x_0    \\
                                                         f(x) + g(x) & \text{if } x \neq x_0
                                                       \end{dcases}                     \\
                                    & (f^* - g^*)(x) = \begin{dcases}
                                                         L - M       & \text{if } x = x_0    \\
                                                         f(x) - g(x) & \text{if } x \neq x_0
                                                       \end{dcases}                     \\
                                    & (f^* g^*)(x) = \begin{dcases}
                                                       LM        & \text{if } x = x_0    \\
                                                       f(x) g(x) & \text{if } x \neq x_0
                                                     \end{dcases}                         \\
                                    & \min(f^*, g^*)(x) = \begin{dcases}
                                                            \min(L, M)               & \text{if } x = x_0    \\
                                                            \min\big(f(x), g(x)\big) & \text{if } x \neq x_0
                                                          \end{dcases}     \\
                                    & \max(f^*, g^*)(x) = \begin{dcases}
                                                            \max(L, M)               & \text{if } x = x_0    \\
                                                            \max\big(f(x), g(x)\big) & \text{if } x \neq x_0
                                                          \end{dcases}     \\
                                    & (c f^*)(x) = \begin{dcases}
                                                     cL     & \text{if } x = x_0    \\
                                                     c f(x) & \text{if } x \neq x_0
                                                   \end{dcases}                              \\
                                    & (f^* / g^*)(x) = \begin{dcases}
                                                         L / M       & \text{if } x = x_0    \\
                                                         f(x) / g(x) & \text{if } x \neq x_0
                                                       \end{dcases} \text{ when } \begin{dcases}
                                                                                    \forall x \in E, g(x) \neq 0 \\
                                                                                    M \neq 0
                                                                                  \end{dcases}
  \end{align*}
  by \cref{ii:3.1.5}(a)(d) we know that
  \begin{align*}
     & d_1 - \lim_{x \to x_0 ; x \in E} (f + g)(x) = L + M                                          \\
     & d_1 - \lim_{x \to x_0 ; x \in E} (f - g)(x) = L - M                                          \\
     & d_1 - \lim_{x \to x_0 ; x \in E} (fg)(x) = LM                                                \\
     & d_1 - \lim_{x \to x_0 ; x \in E} \min(f, g)(x) = \min(L, M)                                  \\
     & d_1 - \lim_{x \to x_0 ; x \in E} \max(f, g)(x) = \max(L, M)                                  \\
     & d_1 - \lim_{x \to x_0 ; x \in E} (cf)(x) = cL                                                \\
     & d_1 - \lim_{x \to x_0 ; x \in E} (f / g)(x) = L / M \text{ if } \begin{dcases}
                                                                         \forall x \in E, g(x) \neq 0 \\
                                                                         M \neq 0
                                                                       \end{dcases}
  \end{align*}
\end{proof}
