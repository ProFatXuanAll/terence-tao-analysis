\section{Limiting values of functions}\label{sec 3.1}

\begin{definition}[Limiting value of a function]\label{3.1.1}
    Let \((X, d_X)\) and \((Y, d_Y)\) be metric spaces, let \(E\) be a subset of \(X\), and let \(f : E \to Y\) be a function.
    If \(x_0 \in X\) is an adherent point of \(E\), and \(L \in Y\), we say that \emph{\(f(x)\) converges to \(L\) in \(Y\) as \(x\) converges to \(x_0\) in \(E\)}, or write \(\lim_{x \to x_0 ; x \in E} f(x) = L\), if for every \(\varepsilon > 0\) there exists a \(\delta > 0\) such that \(d_Y\big(f(x), L\big) < \varepsilon\) for all \(x \in E\) such that \(d_X(x, x_0) < \delta\).
\end{definition}

\begin{remark}\label{3.1.2}
    Some authors exclude the case \(x = x_0\) from the above definition, thus requiring \(0 < d_X(x, x_0) < \delta\).
    In our current notation, this would correspond to removing \(x_0\) from \(E\), thus one would consider
    \[
        \lim_{x \to x_0 ; x \in E \setminus \{x_0\}} f(x)
    \]
    instead of
    \[
        \lim_{x \to x_0 ; x \in E} f(x).
    \]
\end{remark}

\begin{note}
    Comparing this with Definition \ref{2.1.1}, we see that \(f\) is continuous at \(x_0\) if and only if
    \[
        \lim_{x \to x_0 ; x \in X} f(x) = f(x_0).
    \]
    Thus \(f\) is continuous on \(X\) iff we have
    \[
        \lim_{x \to x_0 ; x \in X} f(x) = f(x_0) \text{ for all } x_0 \in X.
    \]
\end{note}

\setcounter{theorem}{3}
\begin{remark}\label{3.1.4}
    Often we shall omit the condition \(x \in X\), and abbreviate
    \[
        \lim_{x \to x_0 ; x \in X} f(x)
    \]
    as simply
    \[
        \lim_{x \to x_0} f(x)
    \]
    when it is clear what space \(x\) will range in.
\end{remark}

\begin{proposition}\label{3.1.5}
    Let \((X, d_X)\) and \((Y, d_Y)\) be metric spaces, let \(E\) be a subset of \(X\), and let \(f : E \to Y\) be a function.
    Let \(x_0 \in X\) be an adherent point of \(E\) and \(L \in Y\).
    Then the following four statements are logically equivalent:
    \begin{enumerate}
        \item \(\lim_{x \to x_0 ; x \in E} f(x) = L\).
        \item For every sequence \((x^{(n)})_{n = 1}^\infty\) in \(E\) which converges to \(x_0\) with respect to the metric \(d_X\), the sequence \(\big(f(x^{(n)})\big)_{n = 1}^\infty\) converges to \(L\) with respect to the metric \(d_Y\).
        \item For every open set \(V \subseteq Y\) which contains \(L\), there exists an open set \(U \subseteq X\) containing \(x_0\) such that \(f(U \cap E) \subseteq V\).
        \item If one defines the function \(g : E \cup \{x_0\} \to Y\) by defining \(g(x_0) \coloneqq L\), and \(g(x) \coloneqq f(x)\) for \(x \in E \setminus \{x_0\}\), then \(g\) is continuous at \(x_0\).
              Furthermore, if \(x_0 \in E\), then \(f(x_0) = L\).
    \end{enumerate}
\end{proposition}

\begin{remark}\label{3.1.6}
    Observe from Proposition \ref{3.1.5}(b) and Proposition \ref{1.1.20} that a function \(f(x)\) can converge to at most one limit \(L\) as \(x\) converges to \(x_0\).
    In other words, if the limit
    \[
        \lim_{x \to x_0 ; x \in E} f(x)
    \]
    exists at all, then it can only take at most one value.
\end{remark}

\begin{remark}\label{3.1.7}
    The requirement that \(x_0\) be an adherent point of \(E\) is necessary for the concept of limiting value to be useful, otherwise \(x_0\) will lie in the exterior of \(E\), the notion that \(f(x)\) converges to \(L\) as \(x\) converges to \(x_0\) in \(E\) is vacuous
    (for \(\delta\) sufficiently small, there are no points \(x \in E\) so that \(d(x, x_0) < \delta\)).
\end{remark}

\begin{remark}\label{3.1.8}
    Strictly speaking, we should write
    \[
        d_Y - \lim_{x \to x_0 ; x \in E} f(x) \text{ instead of } \lim_{x \to x_0 ; x \in E} f(x),
    \]
    since the convergence depends on the metric \(d_Y\).
    However in practice it will be obvious what the metric \(d_Y\) is and so we will omit the \(d_Y -\) prefix from the notation.
\end{remark}

\exercisesection

\begin{exercise}\label{ex 3.1.1}
    Let \((X, d_X)\) and \((Y, d_Y)\) be metric spaces, let \(E\) be a subset of \(X\), let \(f : E \to Y\) be a function, and let \(x_0\) be an element of \(E\).
    Assume that \(x_0\) is an adherent point of \(E \setminus \{x_0\}\)
    (or equivalently, that \(x_0\) is not an \emph{isolated point} of \(E\)).
    Show that the limit \(\lim_{x \to x_0 ; x \in E} f(x)\) exists if and only if the limit \(\lim_{x \to x_0 ; x \in E \setminus \{x_0\}} f(x)\) exists and is equal to \(f(x_0)\).
    Also, show that if the limit \(\lim_{x \to x_0 ; x \in E} f(x)\) exists at all, then it must equal \(f(x_0)\).
\end{exercise}

\begin{exercise}\label{ex 3.1.2}
    Prove Proposition \ref{3.1.5}.
\end{exercise}

\begin{exercise}\label{ex 3.1.3}
    Use Proposition \ref{3.1.5}(c) to define a notion of a limiting value of a function \(f : E \to Y\) from one topological space \((X, \mathcal{F}_X)\) to another \((Y, \mathcal{F}_Y)\) where \(E \subseteq X\).
    If \(X\) is a topological space and \(Y\) is a Hausdorff topological space (see Exercise \ref{ex 2.5.4}), prove the equivalence of Proposition \ref{3.1.5}(c) and \ref{3.1.5}(d) in this setting, as well as an analogue of Remark \ref{3.1.6}.
    What happens to these statements of \(Y\) is not Hausdorff?
\end{exercise}

\begin{exercise}\label{ex 3.1.4}
    Recall from Exercise \ref{ex 2.5.5} that the extended real line \(\mathbf{R}^*\) comes with a standard topology (the order topology).
    We view the natural numbers \(\mathbf{N}\) as a subspace of this topological space, and \(+\infty\) as an adherent point of \(\mathbf{N}\) in \(\mathbf{R}^*\).
    Let \((a_n)_{n = 0}^\infty\) be a sequence taking values in a topological space \((Y, \mathcal{F}_Y)\), and let \(L \in Y\).
    Show that \(\lim_{n \to +\infty ; n \in \mathbf{N}} a_n = L\) (in the sense of Exercise \ref{ex 3.1.3}) if and only if \(\lim_{n \to \infty} a_n = L\) (in the sense of Definition \ref{2.5.4}).
    This shows that the notions of limiting values of a sequence, and limiting values of a function, are compatible.
\end{exercise}

\begin{exercise}\label{ex 3.1.5}
    Let \((X, d_X)\), \((Y, d_Y)\), \((Z, d Z)\) be metric spaces, and let \(x_0 \in X\), \(y_0 \in Y\), \(z_0 \in Z\).
    Let \(f : E \to Y\) and \(g : Y \to Z\) be functions, and let \(E\) be a set.
    If we have \(\lim_{x \to x_0 ; x \in E} f(x) = y_0\) and \(\lim_{y \to y_0 ; y \in f(E)} g(y) = z_0\), conclude that \(\lim_{x \to x_0 ; x \in E} g \circ f(x) = z_0\).
\end{exercise}

\begin{exercise}\label{ex 3.1.6}
    State and prove an analogue of the limit laws in Proposition 9.3.14 in Analysis I when \(X\) is now a metric space rather than a subset of \(\mathbf{R}\).
\end{exercise}