\section{Limiting values of functions}\label{sec 3.1}

\begin{definition}[Limiting value of a function]\label{3.1.1}
    Let \((X, d_X)\) and \((Y, d_Y)\) be metric spaces, let \(E\) be a subset of \(X\), and let \(f : E \to Y\) be a function.
    If \(x_0 \in X\) is an adherent point of \(E\), and \(L \in Y\), we say that \emph{\(f(x)\) converges to \(L\) in \(Y\) as \(x\) converges to \(x_0\) in \(E\)}, or write \(\lim_{x \to x_0 ; x \in E} f(x) = L\), if for every \(\varepsilon > 0\) there exists a \(\delta > 0\) such that \(d_Y\big(f(x), L\big) < \varepsilon\) for all \(x \in E\) such that \(d_X(x, x_0) < \delta\).
\end{definition}

\begin{remark}\label{3.1.2}
    Some authors exclude the case \(x = x_0\) from the above definition, thus requiring \(0 < d_X(x, x_0) < \delta\).
    In our current notation, this would correspond to removing \(x_0\) from \(E\), thus one would consider
    \[
        \lim_{x \to x_0 ; x \in E \setminus \{x_0\}} f(x)
    \]
    instead of
    \[
        \lim_{x \to x_0 ; x \in E} f(x).
    \]
\end{remark}

\begin{note}
    Comparing this with Definition \ref{2.1.1}, we see that \(f\) is continuous at \(x_0\) if and only if
    \[
        \lim_{x \to x_0 ; x \in X} f(x) = f(x_0).
    \]
    Thus \(f\) is continuous on \(X\) iff we have
    \[
        \lim_{x \to x_0 ; x \in X} f(x) = f(x_0) \text{ for all } x_0 \in X.
    \]
\end{note}

\setcounter{theorem}{3}
\begin{remark}\label{3.1.4}
    Often we shall omit the condition \(x \in X\), and abbreviate
    \[
        \lim_{x \to x_0 ; x \in X} f(x)
    \]
    as simply
    \[
        \lim_{x \to x_0} f(x)
    \]
    when it is clear what space \(x\) will range in.
\end{remark}