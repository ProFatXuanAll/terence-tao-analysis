\section{Multiplication of power series}\label{ii:sec:4.4}

\begin{thm}\label{ii:4.4.1}
  Let \(f : (a - r, a + r) \to \R\) and \(g : (a - r, a + r) \to \R\) be functions analytic on \((a - r, a + r)\), with power series expansions
  \[
    f(x) = \sum_{n = 0}^\infty c_n (x - a)^n
  \]
  and
  \[
    g(x) = \sum_{n = 0}^\infty d_n (x - a)^n
  \]
  respectively.
  Then \(fg : (a - r, a + r) \to \R\) is also analytic on \((a - r, a + r)\), with power series expansion
  \[
    f(x) g(x) = \sum_{n = 0}^\infty e_n (x - a)^n
  \]
  where \(e_n \coloneqq \sum_{m = 0}^n c_m d_{n - m}\).
\end{thm}

\begin{proof}
  We have to show that the series \(\sum_{n = 0}^\infty e_n (x - a)^n\) converges to \(f(x) g(x)\) for all \(x \in (a - r, a + r)\).
  Now fix \(x\) to be any point in \((a - r, a + r)\).
  By \cref{ii:4.1.6}, we see that both \(f\) and \(g\) have radii of convergence at least \(r\).
  In particular, the series \(\sum_{n = 0}^\infty c_n (x - a)^n\) and \(\sum_{n = 0}^\infty d_n (x - a)^n\) are absolutely convergent.
  Thus if we define
  \[
    C \coloneqq \sum_{n = 0}^\infty \abs{c_n (x - a)^n}
  \]
  and
  \[
    D \coloneqq \sum_{n = 0}^\infty \abs{d_n (x - a)^n}
  \]
  then \(C\) and \(D\) are both finite.

  For any \(N \geq 0\), consider the partial sum
  \[
    \sum_{n = 0}^N \sum_{m = 0}^\infty \abs{c_m (x - a)^m d_n (x - a)^n}.
  \]
  We can rewrite this as
  \[
    \sum_{n = 0}^N \abs{d_n (x - a)^n} \sum_{m = 0}^\infty \abs{c_m (x - a)^m},
  \]
  which by definition of \(C\) is equal to
  \[
    \sum_{n = 0}^N \abs{d_n (x - a)^n} C,
  \]
  which by definition of \(D\) is less than or equal to \(DC\).
  Thus the above partial sums are bounded by \(DC\) for every \(N\).
  In particular, the series
  \[
    \sum_{n = 0}^\infty \sum_{m = 0}^\infty \abs{c_m (x - a)^m d_n (x - a)^n}
  \]
  is convergent, which means that the sum
  \[
    \sum_{n = 0}^\infty \sum_{m = 0}^\infty c_m (x - a)^m d_n (x - a)^n
  \]
  is absolutely convergent.

  Let us now compute this sum in two ways.
  First of all, we can pull the \(d_n (x - a)^n\) factor out of the \(\sum_{m = 0}^\infty\) summation, to obtain
  \[
    \sum_{n = 0}^\infty d_n (x - a)^n \sum_{m = 0}^\infty c_m (x - a)^m.
  \]
  By our formula for \(f(x)\), this is equal to
  \[
    \sum_{n = 0}^\infty d_n (x - a)^n f(x);
  \]
  by our formula for \(g(x)\), this is equal to \(f(x) g(x)\).
  Thus
  \[
    f(x) g(x) = \sum_{n = 0}^\infty \sum_{m = 0}^\infty c_m (x - a)^m d_n (x - a)^n.
  \]
  Now we compute this sum in a different way.
  We rewrite it as
  \[
    f(x) g(x) = \sum_{n = 0}^\infty \sum_{m = 0}^\infty c_m d_n (x - a)^{n + m}.
  \]
  By Fubini's theorem for series (Theorem 8.2.2 in Analysis I), because the series was absolutely convergent, we may rewrite it as
  \[
    f(x) g(x) = \sum_{m = 0}^\infty \sum_{n = 0}^\infty c_m d_n (x - a)^{n + m}.
  \]
  Now make the substitution \(n' \coloneqq n + m\), to rewrite this as
  \[
    f(x) g(x) = \sum_{m = 0}^\infty \sum_{n' = m}^\infty c_m d_{n' - m} (x - a)^{n'}.
  \]
  If we adopt the convention that \(d_j = 0\) for all negative \(j\), then this is equal to
  \[
    f(x) g(x) = \sum_{m = 0}^\infty \sum_{n' = 0}^\infty c_m d_{n' - m} (x - a)^{n'}.
  \]
  Applying Fubini's theorem again, we obtain
  \[
    f(x) g(x) = \sum_{n' = 0}^\infty \sum_{m = 0}^\infty c_m d_{n' - m} (x - a)^{n'},
  \]
  which we can rewrite as
  \[
    f(x) g(x) = \sum_{n' = 0}^\infty (x - a)^{n'} \sum_{m = 0}^\infty c_m d_{n' - m}.
  \]
  Since \(d_j\) was \(0\) when \(j\) is negative, we can rewrite this as
  \[
    f(x) g(x) = \sum_{n' = 0}^\infty (x - a)^{n'} \sum_{m = 0}^{n'} c_m d_{n' - m},
  \]
  which by definition of \(e\) is
  \[
    f(x) g(x) = \sum_{n' = 0}^\infty e_{n'} (x - a)^{n'},
  \]
  as desired.
\end{proof}

\begin{rmk}\label{ii:4.4.2}
  The sequence \((e_n)_{n = 0}^\infty\) is sometimes referred to as the \emph{convolution} of the sequences \((c_n)_{n = 0}^\infty\) and \((d_n)_{n = 0}^\infty\);
  it is closely related (though not identical) to the notion of convolution introduced in \cref{ii:3.8.9}.
\end{rmk}
