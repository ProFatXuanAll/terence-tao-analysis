\chapter{Fourier series}\label{ch:5}

\begin{note}
  Power series are already immensely useful, especially when dealing with special functions such as the exponential and trigonometric functions discussed earlier.
  However, there are some circumstances where power series are not so useful, because one has to deal with functions (e.g., \(\sqrt{x}\)) which are not real analytic, and so do not have power series.
\end{note}

\begin{note}
  Fortunately, there is another type of series expansion, known as \emph{Fourier series}, which is also a very powerful tool in analysis
  (though used for slightly different purposes).
  Instead of analyzing compactly supported functions, it instead analyzes \emph{periodic functions};
  instead of decomposing into polynomials, it decomposes into \emph{trigonometric polynomials}.
  Roughly speaking, the theory of Fourier series asserts that just about every periodic function can be decomposed as an (infinite) sum of sines and cosines.
\end{note}

\begin{rmk}\label{5.0.1}
  Jean-Baptiste Fourier (1768--1830) was, among other things, an administrator accompanying Napoleon on his invasion of Egypt, and then a Prefect in France during Napoleons reign.
  After the Napoleonic wars, he returned to mathematics.
  He introduced Fourier series in an important 1807 paper in which he used them to solve what is now known as the heat equation.
  At the time, the claim that every periodic function could be expressed as a sum of sines and cosines was extremely controversial, even such leading mathematicians as Euler declared that it was impossible.
  Nevertheless, Fourier managed to show that this was indeed the case, although the proof was not completely rigorous and was not totally accepted for almost another hundred years.
\end{rmk}

\begin{note}
  For instance, the convergence of Fourier series is usually not uniform (i.e., not in the \(L^\infty\) metric), but instead we have convergence in a different metric, the \(L^2\)-metric.
  We will need to use complex numbers heavily in our theory, while they played only a tangential rôle in power series.
\end{note}

\begin{note}
  The theory of Fourier series (and of related topics such as Fourier integrals and the Laplace transform) is vast, and deserves an entire course in itself.
  It has many, many applications, most directly to differential equations, signal processing, electrical engineering, physics, and analysis, but also to algebra and number theory.
\end{note}

\section{Periodic functions}\label{ii:sec:5.1}

\begin{defn}\label{ii:5.1.1}
  Let \(L > 0\) be a real number.
  A function \(f : \R \to \C\) is periodic with period \(L\), or \(L\)-periodic, if we have \(f(x + L) = f(x)\) for every real number \(x\).
\end{defn}

\begin{eg}\label{ii:5.1.2}
  The real-valued functions \(f(x) = \sin(x)\) and \(f(x) = \cos(x)\) are \(2\pi\)-periodic, as is the complex-valued function \(f(x) = e^{i x}\).
  These functions are also \(4\pi\)-periodic, \(6\pi\)-periodic, etc.
  The function \(f(x) = x\), however, is not periodic.
  The constant function \(f(x) = 1\) is \(L\)-periodic for every \(L\).
\end{eg}

\begin{rmk}\label{ii:5.1.3}
  If a function \(f\) is \(L\)-periodic, then we have \(f(x + kL) = f(x)\) for every integer \(k\)
  (why? Use induction for the positive \(k\), and then use a substitution to convert the positive \(k\) result to a negative \(k\) result.
  The \(k = 0\) case is of course trivial).
  In particular, if a function \(f\) is \(1\)-periodic, then we have \(f(x + k) = f(x)\) for every \(k \in \Z\).
  Because of this, \(1\)-periodic functions are sometimes also called \(\Z\)-periodic
  (and \(L\)-periodic functions called \(L \Z\)-periodic).
\end{rmk}

\begin{eg}\label{ii:5.1.4}
  For any integer \(n\), the functions \(x \mapsto \cos(2 \pi n x)\), \(x \mapsto \sin(2 \pi n x)\), and \(x \mapsto e^{2 \pi i n x}\) are all \(\Z\)-periodic.
  Another example of a \(\Z\)-periodic function is the function \(f : \R \to \C\) defined by \(f(x) \coloneqq 1\) when \(x \in [n, n + \dfrac{1}{2})\) for some integer \(n\), and \(f(x) \coloneqq 0\) when \(x \in [n + \dfrac{1}{2}, n + 1)\) for some integer \(n\).
  This function is an example of a \emph{square wave}.
\end{eg}

\begin{note}
  In order to completely specify a \(\Z\)-periodic function \(f : \R \to \C\), one only needs to specify its values on the interval \([0, 1)\), since this will determine the values of \(f\) everywhere else.
  This is because every real number \(x\) can be written in the form \(x = k + y\) where \(k\) is an integer (called the \emph{integer part} of \(x\), and sometimes denoted \([x]\)) and \(y \in [0, 1)\) (this is called the \emph{fractional part} of \(x\), and sometimes denoted \(\set{x}\)).
  Because of this, sometimes when we wish to describe a \(\Z\)-periodic function \(f\) we just describe what it does on the interval \([0, 1)\), and then say that it is \emph{extended periodically} to all of \(\R\).
  This means that we define \(f(x)\) for any real number \(x\) by setting \(f(x) \coloneqq f(y)\), where we have decomposed \(x = k + y\) as discussed above.
  (One can in fact replace the interval \([0, 1)\) by any other half-open interval of length \(1\), but we will not do so here.)
\end{note}

\begin{note}
  The space of complex-valued continuous \(\Z\)-periodic functions is denoted
  \[
    C(\R / \Z ; \C).
  \]
  (The notation \(\R / \Z\) comes from algebra, and denotes the quotient group of the additive group \(\R\) by the additive group \(\Z\);
  more information in this can be found in any algebra text.)
  By ``continuous'' we mean continuous at all points on \(\R\);
  merely being continuous on an interval such as \([0, 1]\) will not suffice, as there may be a discontinuity between the left and right limits at \(1\) (or at any other integer).
  Thus, for instance, the functions \(x \mapsto \sin(2 \pi n x)\), \(x \mapsto \cos(2 \pi n x)\), and \(x \mapsto e^{2 \pi i n x}\) are all elements of \(C(\R / \Z ; \C)\), as are the constant functions, however the square wave function in \cref{ii:5.1.4} is not in \(C(\R / \Z ; \C)\) because it is not continuous at every integer.
  Also the function \(\sin(x)\) would also not qualify to be in \(C(\R / \Z ; \C)\) since it is not \(\Z\)-periodic.
\end{note}

\begin{lem}[Basic properties of \(C(\R / \Z ; \C)\)]\label{ii:5.1.5}
  \quad
  \begin{enumerate}
    \item (Boundedness)
          If \(f \in C(\R / \Z ; \C)\), then \(f\) is bounded
          (i.e., there exists a real number \(M > 0\) such that \(\abs{f(x)} \leq M\) for all \(x \in \R\)).
    \item (Vector space and algebra properties)
          If \(f, g \in C(\R / \Z ; \C)\), then the functions \(f + g\), \(f - g\), and \(f g\) are also in \(C(\R / \Z ; \C)\).
          Also, if \(c\) is any complex number, then the function \(cf\) is also in \(C(\R / \Z ; \C)\).
    \item (Closure under uniform limits)
          If \((f_n)_{n = 1}^\infty\) is a sequence of functions in \(C(\R / \Z ; \C)\) which converges uniformly to another function \(f : \R \to \C\), then \(f\) is also in \(C(\R / \Z ; \C)\).
  \end{enumerate}
\end{lem}

\begin{proof}{(a)}
  Since \(f \in C(\R / \Z ; \C)\), by \cref{ii:5.1.1} we have
  \[
    \set{f(x) : x \in \R} = \set{f(x) : x \in [0, 1)} = \set{f(x) : x \in [0, 1]}.
  \]
  So it suffices to show that \(\set{f(x) : x \in [0, 1]}\) is bounded.
  Let \(d_{\R} = d_{l^1}|_{\R \times \R}\) and let \(d_{\C}\) be the metric in \cref{ii:4.6.10}.
  Since \([0, 1]\) is closed and bounded in \((\R, d_{\R})\), by Heine-Borel theorem (\cref{ii:1.5.7}) we know that \(([0, 1], d_{\R}|_{[0, 1] \times [0, 1]})\) is compact.
  Since \(f\) is continuous on \([0, 1]\), by \cref{ii:2.3.1} we know that \(\big(f([0, 1]), d_{\C}|_{f([0, 1]) \times f([0, 1])}\big)\) is also compact.
  By \cref{ii:1.5.6} we know that compactness implies boundness, thus we have
  \begin{align*}
             & \forall z \in \C, \exists r \in \R^+ : f([0, 1]) \subseteq B_{(\C, d_{\C})}(z, r) &  & \by{ii:1.5.3}  \\
    \implies & \exists r \in \R^+ : f([0, 1]) \subseteq B_{(\C, d_{\C})}(1, r)                                       \\
    \implies & \exists r \in \R^+ : \forall y \in f([0, 1]), \abs{y - 1} < r                     &  & \by{ii:1.2.1}  \\
    \implies & \exists r \in \R^+ : \forall y \in f([0, 1]),                                                         \\
             & \abs{y} = \abs{y - 1 + 1} \leq \abs{y - 1} + 1 < r + 1.                           &  & \by{ii:4.6.11}
  \end{align*}
  By setting \(M = r + 1\) we are done.
\end{proof}

\begin{proof}{(b)}
  We have
  \begin{align*}
    (f + g)(x + 1) & = f(x + 1) + g(x + 1)                    \\
                   & = f(x) + g(x)         &  & \by{ii:5.1.1} \\
                   & = (f + g)(x)                             \\
    (f - g)(x + 1) & = f(x + 1) - g(x + 1)                    \\
                   & = f(x) - g(x)         &  & \by{ii:5.1.1} \\
                   & = (f - g)(x)                             \\
    (f g)(x + 1)   & = f(x + 1) g(x + 1)                      \\
                   & = f(x) g(x)           &  & \by{ii:5.1.1} \\
                   & = (f g)(x)
  \end{align*}
  and
  \begin{align*}
    \forall c \in \C, (c f)(x + 1) & = c f(x + 1)                    \\
                                   & = c f(x)     &  & \by{ii:5.1.1} \\
                                   & = (c f)(x).
  \end{align*}
\end{proof}

\begin{proof}{(c)}
  Let \(d_{\R} = d_{l^1}|_{\R \times \R}\) and let \(d_{\C}\) be the metric in \cref{ii:4.6.10}.
  Since \((f_n)_{n = 0}^\infty\) converges uniformly to \(f\) on \(\C\) with respect to \(d_{\C}\), by \cref{ii:3.3.2} we know that \(f\) is continuous from \((\R, d_{\R})\) to \((\C, d_{\C})\).
  Suppose for the sake of contradiction that \(f \notin C(\R / \Z ; \C)\).
  By \cref{ii:5.1.1} this means
  \[
    \exists x \in \R : f(x + 1) \neq f(x).
  \]
  By \cref{ii:ex:3.2.2} we know that \((f_n)_{n = 0}^\infty\) converges pointwise to \(f\) on \(\C\) with respect to \(d_{\C}\), thus by \cref{ii:3.2.1} we have
  \begin{align*}
             & \begin{dcases}
                 d - \lim_{n \to \infty} f_n(x) = f(x) \\
                 d - \lim_{n \to \infty} f_n(x + 1) = f(x + 1)
               \end{dcases}                                                                       \\
    \implies & \forall \varepsilon \in \R, \exists N \in \Z^+ : \forall n \geq N,                                                 \\
             & \begin{dcases}
                 \abs{f_n(x) - f(x)} < \dfrac{\varepsilon}{2} \\
                 \abs{f_n(x + 1) - f(x + 1)} < \dfrac{\varepsilon}{2}
               \end{dcases}                                                                \\
    \implies & \forall \varepsilon \in \R, \exists N \in \Z^+ : \forall n \geq N, \begin{dcases}
                                                                                    \abs{f_n(x) - f(x)} < \dfrac{\varepsilon}{2} \\
                                                                                    \abs{f_n(x) - f(x + 1)} < \dfrac{\varepsilon}{2}
                                                                                  \end{dcases} &  & \by{ii:5.1.1} \\
    \implies & \forall \varepsilon \in \R, \exists N \in \Z^+ : \forall n \geq N,                                                 \\
             & \abs{f(x) - f(x + 1)} \leq \abs{f(x) - f_n(x)} + \abs{f_n(x) - f(x + 1)}                                           \\
             & < \dfrac{\varepsilon}{2} + \dfrac{\varepsilon}{2} = \varepsilon                                                    \\
    \implies & \forall \varepsilon \in \R^+, \abs{f(x) - f(x + 1)} < \varepsilon                                                  \\
    \implies & f(x) = f(x + 1).
  \end{align*}
  But this contradict to the definition of \(x\).
  Thus, \(f \in C(\R / \Z ; \C)\).
\end{proof}

\begin{note}
  One can make \(C(\R / \Z ; \C)\) into a metric space by re-introducing the now familiar sup-norm metric
  \[
    d_\infty(f, g) = \sup_{x \in \R} \abs{f(x) - g(x)} = \sup_{x \in [0, 1)} \abs{f(x) - g(x)}
  \]
  of uniform convergence.
\end{note}

\begin{ac}[modular operation]\label{ii:ac:5.1.1}
  Let \(n \in \Z^+\).
  Define \(\text{mod}_n : \R \to \R\) as follow:
  \[
    \forall x \in \R, \text{mod}_n(x) = x - \bigg[\dfrac{x}{n}\bigg] n.
  \]
  Then \(\text{mod}(\R) \subseteq [0, n)\) and \(\text{mod}\) is \(n\)-periodic.
  We often use \(x \mod n\) instead of \(\text{mod}_n(x)\).
\end{ac}

\begin{proof}
  Since
  \begin{align*}
             & \forall x \in \R, \bigg[\dfrac{x}{n}\bigg] \leq \dfrac{x}{n} < \bigg[\dfrac{x}{n}\bigg] + 1 &              & \by{ii:5.1.1} \\
    \implies & \bigg[\dfrac{x}{n}\bigg] n \leq x < \bigg[\dfrac{x}{n}\bigg] n + n                          & (n \in \Z^+)                 \\
    \implies & 0 \leq x - \bigg[\dfrac{x}{n}\bigg] n < n,
  \end{align*}
  we know that \(\text{mod}_n(x) \subseteq [0, n)\).
  Since
  \begin{align*}
    \forall x \in \R, \text{mod}_n(x + n) & = x + n - \bigg[\dfrac{x + n}{n}\bigg] n                                   \\
                                          & = x + n - \Bigg(\bigg[\dfrac{x}{n}\bigg] + 1\Bigg) n &  & \by{ii:ex:5.1.1} \\
                                          & = x + n - \bigg[\dfrac{x}{n}\bigg] n - n                                   \\
                                          & = x - \bigg[\dfrac{x}{n}\bigg] n                                           \\
                                          & = \text{mod}_n(x),
  \end{align*}
  by \cref{ii:5.1.1} we know that \(\text{mod}_n\) is \(n\)-periodic.
\end{proof}

\exercisesection

\begin{ex}\label{ii:ex:5.1.1}
  Show that every real number \(x\) can be written in exactly one way in the form \(x = k + y\), where \(k\) is an integer and \(y \in [0, 1)\).
\end{ex}

\begin{proof}
  By Exercise 5.4.3 we know that
  \[
    \forall x \in \R, \exists!\ k \in \Z : k \leq x < k + 1.
  \]
  Thus, by setting \(y = x - k\) we have \(x = y + k\) and \(y \in [0, 1)\).
\end{proof}

\begin{ex}\label{ii:ex:5.1.2}
  Prove \cref{ii:5.1.5}.
\end{ex}

\begin{proof}
  See \cref{ii:5.1.5}.
\end{proof}

\begin{ex}\label{ii:ex:5.1.3}
  Show that \(C(\R / \Z ; \C)\) with the sup-norm metric \(d_\infty\) is a metric space.
  Furthermore, show that this metric space is complete.
\end{ex}

\begin{proof}
  First we show that \(\big(C(\R / \Z ; \C), d_\infty\big)\) is a metric space.
  Since
  \[
    \forall f \in C(\R / \Z ; \C), d_\infty(f, f) = \sup_{x \in [0, 1)} \abs{f(x) - f(x)} = 0,
  \]
  we know that \(d_\infty\) satisfied \cref{ii:1.1.2}(a).
  Since
  \begin{align*}
             & \forall f, g \in C(\R / \Z ; \C), f \neq g                                \\
    \implies & \exists x \in \R : f(x) \neq g(x)                                         \\
    \implies & \exists x \in [0, 1) : f(x) \neq g(x)               &  & \by{ii:5.1.1}    \\
    \implies & 0 < \sup_{x \in [0, 1)} \abs{f(x) - g(x)} < +\infty &  & \by{ii:5.1.5}[a] \\
    \implies & 0 < d_\infty(f, g) < +\infty,
  \end{align*}
  we know that \(d_\infty\) satisfied \cref{ii:1.1.2}(b).
  Since
  \begin{align*}
    \forall f, g \in C(\R / \Z ; \C), d_\infty(f, g) & = \sup_{x \in [0, 1)} \abs{f(x) - g(x)}                     \\
                                                     & = \sup_{x \in [0, 1)} \abs{g(x) - f(x)} &  & \by{ii:4.6.10} \\
                                                     & = d_\infty(g, f),
  \end{align*}
  we know that \(d_\infty\) satisfied \cref{ii:1.1.2}(c).
  Since
  \begin{align*}
     & \forall f, g, h \in C(\R / \Z ; \C), d_\infty(f, h)                                          \\
     & = \sup_{x \in [0, 1)} \abs{f(x) - h(x)}                                                      \\
     & = \sup_{x \in [0, 1)} \abs{f(x) - g(x) + g(x) - h(x)}                                        \\
     & \leq \sup_{x \in [0, 1)} \big(\abs{f(x) - g(x)} + \abs{g(x) - h(x)}\big) &  & \by{ii:4.6.11} \\
     & = d_\infty(f, g) + d_\infty(g, h),
  \end{align*}
  we know that \(d_\infty\) satisfied \cref{ii:1.1.2}(d).
  From all proofs above we conclude by \cref{ii:1.1.2} that \(\big(C(\R / \Z ; \C), d_\infty\big)\) is a metric space.

  Now we show that \(\big(C(\R / \Z ; \C), d_\infty\big)\) is complete.
  Let \(d_{\R} = d_{l^1}|_{\R \times \R}\) and let \(d_{\C}\) be the metric in \cref{ii:4.6.10}.
  Let \((f_n)_{n = 1}^\infty\) be a Cauchy sequence in \(\big(C(\R / \Z ; \C), d_\infty\big)\) and let \(n_1, n_2 \in \Z^+\).
  Then by \cref{ii:1.4.6} we have
  \begin{align*}
             & \forall \varepsilon \in \R^+, \exists N \in \Z^+ : \forall n_1, n_2 \geq N, d_\infty(f_{n_1}, f_{n_2}) < \varepsilon        \\
    \implies & \forall x \in \R, \forall \varepsilon \in \R^+, \exists N \in \Z^+ : \forall n_1, n_2 \geq N,                               \\
             & \abs{f_{n_1}(x) - f_{n_2}(x)} \leq \sup_{y \in \R} \abs{f_{n_1}(y) - f_{n_2}(y)} = d_\infty(f_{n_1}, f_{n_2}) < \varepsilon \\
    \implies & \forall x \in \R, \big(f_n(x)\big)_{n = 1}^\infty \text{ is a Cauchy sequence in } (\C, d_{\C}).
  \end{align*}
  Since \((\C, d_{\C})\) is complete (by \cref{ii:ex:4.6.10}), we know that
  \[
    \forall x \in \C, \lim_{n \to \infty} f_n(x) \in \C.
  \]
  Thus, we can define \(f : \R \to \C\) as follow:
  \[
    \forall x \in \R, f(x) = \lim_{n \to \infty} f_n(x).
  \]
  And we have
  \begin{align*}
             & \forall x \in \R, \forall \varepsilon \in \R^+, \exists N \in \Z^+ : \forall n \geq N, \\
             & \begin{dcases}
                 \abs{f(x) - f_n(x)} < \dfrac{\varepsilon}{2} \\
                 \abs{f(x + 1) - f_n(x + 1)} = \abs{f(x + 1) - f_n(x)} < \dfrac{\varepsilon}{2}
               \end{dcases}         & (f_n \in C(\R / \Z ; \C))          \\
    \implies & \forall x \in \R, \forall \varepsilon \in \R^+, \exists N \in \Z^+ : \forall n \geq N, \\
             & \abs{f(x) - f(x + 1)} \leq \abs{f(x) - f_n(x)} + \abs{f(x + 1) - f_n(x)}               \\
             & < \dfrac{\varepsilon}{2} + \dfrac{\varepsilon}{2} = \varepsilon                        \\
    \implies & \forall x \in \R, \forall \varepsilon \in \R^+, \abs{f(x) - f(x + 1)} < \varepsilon    \\
    \implies & \forall x \in \R, f(x) = f(x + 1)                                                      \\
    \implies & f \in C(\R / \Z ; \C).
  \end{align*}
  Since \((f_n)_{n = 1}^\infty\) was arbitrary, by \cref{ii:1.4.10} we know that \(\big(C(\R / \Z ; \C), d_\infty\big)\) is complete.
\end{proof}

\section{Inner products on periodic functions}\label{sec:5.2}

\begin{defn}[Inner product]\label{5.2.1}
  If \(f, g \in C(\R / \Z ; \C)\), we define the \emph{inner product} \(\inner*{f, g}\) to be the quantity
  \[
    \inner*{f, g} = \int_{[0, 1]} f(x) \overline{g(x)} \; dx.
  \]
\end{defn}

\begin{rmk}\label{5.2.2}
  In order to integrate a complex-valued function over real variables, we use the definition that
  \[
    \int_{[a, b]} f(x) \; dx \coloneqq \int_{[a, b]} \Re\big(f(x)\big) \; dx + i \int_{[a,b]} \Im\big(f(x)\big) \; dx;
  \]
  i.e., we integrate the real and imaginary parts of the function separately.
  It is easy to verify that all the standard rules of calculus (integration by parts, fundamental theorem of calculus, substitution, etc.) still hold when the functions are complex-valued instead of real-valued.
\end{rmk}

\begin{proof}
  Let \(f \in C(\R / \Z ; \C)\), let \(d_{\R} = d_{l^1}|_{\R \times \R}\) and let \(d_{\C}\) be the metric in \cref{4.6.10}.
  Let \(x_0 \in \R\) and let \((a_n)_{n = 1}^\infty\) be a sequence in \(\R\) such that \(\lim_{n \to \infty} a_n = x_0\).
  Since \(f\) is continuous on \(\R\) from \((\R, d_{\R})\) to \((\C, d_{\C})\), we know that
  \begin{align*}
             & \lim_{n \to \infty} f(a_n) = f(x_0)                                           &  & \by{2.1.4} \\
    \implies & \begin{dcases}
                 \lim_{n \to \infty} \Re(f\big(a_n)\big) = \Re\big(f(x_0)\big) \\
                 \lim_{n \to \infty} \Im(f\big(a_n)\big) = \Im\big(f(x_0)\big)
               \end{dcases} &  & \by{4.6.13}
  \end{align*}
  Since \((a_n)_{n = 0}^\infty\) is arbitrary, by \cref{2.1.4} we know that \(\Re \circ f\) is continuous at \(x_0\) from \((\R, d_{\R})\) to \((\R, d_{\R})\).
  Since \(x_0\) is arbitrary, by \cref{2.1.5} we know that \(\Re \circ f\) is continuous on \(\R\) from \((\R, d_{\R})\) to \((\R, d_{\R})\).
  Using similar arguments we can show that \(\Im \circ f\) is continuous on \(\R\) from \((\R, d_{\R})\) to \((\R, d_{\R})\).
  Since
  \begin{align*}
     & \forall x \in \R, \Re(f(x + 1)) = \Re(f(x)) \implies \Re \circ f \in C(\R / \Z ; \C); \\
     & \forall x \in \R, \Im(f(x + 1)) = \Im(f(x)) \implies \Im \circ f \in C(\R / \Z ; \C),
  \end{align*}
  by \cref{5.1.5}(a) we know that both \(\Re \circ f\) and \(\Im \circ f\) are bounded in \((\C, d_{\C})\).
  In particular, by \cref{4.6.8} we know that \((\Re \circ f)(\R) \subseteq \R\) and \((\Im \circ f)(\R) \subseteq \R\).
  Thus both \(\Re \circ f\) and \(\Im \circ f\) are bounded in \((\R, d_{\R})\).
  Since \(\Re \circ f\) and \(\Im \circ f\) are continuous and bounded on \([0, 1]\), by Corollary 11.5.2 in Analysis I we know that \(\Re \circ f\) and \(\Im \circ f\) are Riemann integrable on \([0, 1]\).
  Thus
  \[
    \int_{[0, 1]} f(x) \; dx = \int_{[0, 1]} \Re\big(f(x)\big) \; dx + i \bigg(\int_{[0, 1]} \Im\big(f(x)\big) \; dx\bigg) \in \C
  \]
  is well-defined.
  The same argument holds on arbitrary closed interval \([a, b]\) since \(f \in C(\R / \Z ; \C)\).
\end{proof}

\begin{eg}\label{5.2.3}
  Let \(f\) be the constant function \(f(x) \coloneqq 1\), and let \(g(x)\) be the function \(g(x) \coloneqq e^{2 \pi i x}\).
  Then we have
  \begin{align*}
    \inner*{f, g} & = \int_{[0, 1]} 1 \overline{e^{2 \pi i x}} \; dx       \\
                  & = \int_{[0, 1]} e^{- 2 \pi i x} \; dx                  \\
                  & = \dfrac{e^{- 2 \pi i x}}{- 2 \pi i} |_{x = 0}^{x = 1} \\
                  & = \dfrac{e^{- 2 \pi i } - e^0}{- 2 \pi i}              \\
                  & = \dfrac{1 - 1}{- 2 \pi i}                             \\
                  & = 0.
  \end{align*}
\end{eg}

\begin{rmk}\label{5.2.4}
  In general, the inner product \(\inner*{f, g}\) will be a complex number.
  (Note that \(f(x) \overline{g(x)}\) will be Riemann integrable since both functions are bounded and continuous.)
\end{rmk}

\begin{note}
  Roughly speaking, the inner product \(\inner*{f, g}\) is to the space \(C(\R / \Z ; \C)\) what the dot product \(x \cdot y\) is to Euclidean spaces such as \(\R^n\).
  A more in-depth study of inner products on vector spaces can be found in any linear algebra text but is beyond the scope of this text.
\end{note}

\begin{lem}\label{5.2.5}
  Let \(f, g, h \in C(\R / \Z ; \C)\).
  \begin{enumerate}
    \item (Hermitian property)
          We have \(\inner*{g, f} = \overline{\inner*{f, g}}\).
    \item (Positivity)
          We have \(\inner*{f, f} \geq 0\).
          Furthermore, we have \(\inner*{f, f} = 0\) iff \(f = 0\)
          (i.e., \(f(x) = 0\) for all \(x \in \R\)).
    \item (Linearity in the first variable)
          We have \(\inner*{f +g, h}\) = \(\inner*{f, h} + \inner*{g, h}\).
          For any complex number \(c\), we have \(\inner*{cf, g} = c \inner*{f, g}\).
    \item (Antilinearity in the second variable)
          We have \(\inner*{f, g + h} = \inner*{f, g} + \inner*{f, h}\).
          For any complex number \(c\), we have \(\inner*{f, cg} = c \inner*{f, g}\).
  \end{enumerate}
\end{lem}

\begin{proof}{(a)}
  We have
  \begin{align*}
    \overline{\inner*{f, g}} & = \overline{\int_{[0, 1]} f(x) \overline{g(x)} \; dx}                                                                                               &  & \by{5.2.1} \\
                             & = \overline{\int_{[0, 1]} \Re\big(f(x) \overline{g(x)}\big) \; dx + i \bigg(\int_{[0, 1]} \Im\big(f(x) \overline{g(x)}\big) \; dx\bigg)}            &  & \by{5.2.2} \\
                             & = \int_{[0, 1]} \Re\big(f(x) \overline{g(x)}\big) \; dx - i \bigg(\int_{[0, 1]} \Im\big(f(x) \overline{g(x)}\big) \; dx\bigg)                       &  & \by{4.6.8} \\
                             & = \int_{[0, 1]} \Re\big(f(x) \overline{g(x)}\big) \; dx + i \bigg(-\int_{[0, 1]} \Im\big(f(x) \overline{g(x)}\big) \; dx\bigg)                      &  & \by{4.6.6} \\
                             & = \int_{[0, 1]} \Re\big(f(x) \overline{g(x)}\big) \; dx + i \bigg(\int_{[0, 1]} -\Im\big(f(x) \overline{g(x)}\big) \; dx\bigg)                                      \\
                             & = \int_{[0, 1]} \Re\big(\overline{f(x) \overline{g(x)}}\big) \; dx + i \bigg(\int_{[0, 1]} \Im\big(\overline{f(x) \overline{g(x)}}\big) \; dx\bigg) &  & \by{4.6.8} \\
                             & = \int_{[0, 1]} \Re\big(\overline{f(x)} g(x)\big) \; dx + i \bigg(\int_{[0, 1]} \Im\big(\overline{f(x)} g(x)\big) \; dx\bigg)                       &  & \by{4.6.9} \\
                             & = \int_{[0, 1]} \Re\big(g(x) \overline{f(x)}\big) \; dx + i \bigg(\int_{[0, 1]} \Im\big(g(x) \overline{f(x)}\big) \; dx\bigg)                       &  & \by{4.6.6} \\
                             & = \int_{[0, 1]} g(x) \overline{f(x)} \; dx                                                                                                          &  & \by{5.2.2} \\
                             & = \inner*{g, f}.                                                                                                                                    &  & \by{5.2.1}
  \end{align*}
\end{proof}

\begin{proof}{(b)}
  We have
  \begin{align*}
    \inner*{f, f} & = \int_{[0, 1]} f(x) \overline{f(x)} \; dx &  & \by{5.2.1}                                  \\
                  & = \int_{[0, 1]} \abs{f(x)}^2 \; dx         &  & \by{4.6.11}                                 \\
                  & \geq \int_{[0, 1]} 0 \; dx                 &  & \text{(by Theorem 11.4.1(d) in Analysis I)} \\
                  & = 0
  \end{align*}
  and
  \begin{align*}
         & \int_{[0, 1]} \abs{f(x)}^2 \; dx = 0                                                  \\
    \iff & \forall x \in [0, 1], \abs{f(x)}^2 = 0 &  & \text{(by Exercise 11.4.2 in Analysis I)} \\
    \iff & \forall x \in [0, 1], \abs{f(x)} = 0                                                  \\
    \iff & \forall x \in [0, 1], f(x) = 0.        &  & \by{4.6.11}
  \end{align*}
\end{proof}

\begin{proof}{(c)}
  We have
  \begin{align*}
     & \inner*{f + g, h}                                                                                                                                                                                       \\
     & = \int_{[0, 1]} (f + g)(x) \overline{h(x)} \; dx                                                                                    &                                                      & \by{5.2.1} \\
     & = \int_{[0, 1]} \Re\big((f + g)(x) \overline{h(x)}\big) \; dx                                                                       &                                                      & \by{5.2.2} \\
     & \quad + i \bigg(\int_{[0, 1]} \Im\big((f + g)(x) \overline{h(x)}\big) \; dx\bigg)                                                                                                                       \\
     & = \int_{[0, 1]} \Re\big(f(x) \overline{h(x)} + g(x) \overline{h(x)}\big) \; dx                                                                                                                          \\
     & \quad + i \bigg(\int_{[0, 1]} \Im\big(f(x) \overline{h(x)} + g(x) \overline{h(x)}\big) \; dx\bigg)                                                                                                      \\
     & = \int_{[0, 1]} \Re\big(f(x) \overline{h(x)}) + \Re\big(g(x) \overline{h(x)}\big) \; dx                                             &                                                      & \by{4.6.8} \\
     & \quad + i \bigg(\int_{[0, 1]} \Im\big(f(x) \overline{h(x)}\big) + \Im\big(g(x) \overline{h(x)}\big) \; dx\bigg)                                                                                         \\
     & = \int_{[0, 1]} \Re\big(f(x) \overline{h(x)}) \; dx + \int_{[0, 1]} \Re\big(g(x) \overline{h(x)}\big) \; dx                         & (f \overline{h}, g \overline{h} \in C(\R / \Z ; \C))              \\
     & \quad + i \bigg(\int_{[0, 1]} \Im\big(f(x) \overline{h(x)}\big) \; dx + \int_{[0, 1]} \Im\big(g(x) \overline{h(x)}\big) \; dx\bigg)                                                                     \\
     & = \int_{[0, 1]} f(x) \overline{h(x)} \; dx + \int_{[0, 1]} g(x) \overline{h(x)} \; dx                                               &                                                      & \by{5.2.2} \\
     & = \inner*{f, h} + \inner*{g, h}                                                                                                     &                                                      & \by{5.2.1}
  \end{align*}
  and
  \begin{align*}
     & \inner*{cf, g}                                                                                                                                                                                   \\
     & = \int_{[0, 1]} (cf)(x) \overline{g(x)} \; dx                                                                                                &                                      & \by{5.2.1} \\
     & = \int_{[0, 1]} \Re\big((cf)(x) \overline{g(x)}\big) \; dx + i \bigg(\int_{[0, 1]} \Im\big((cf)(x) \overline{g(x)}\big) \; dx\bigg)          &                                      & \by{5.2.2} \\
     & = \int_{[0, 1]} \Re\big(cf(x) \overline{g(x)}\big) \; dx + i \bigg(\int_{[0, 1]} \Im\big(cf(x) \overline{g(x)}\big) \; dx\bigg)                                                                  \\
     & = \int_{[0, 1]} \Re(c) \Re\big(f(x) \overline{g(x)}\big) - \Im(c) \Im\big(f(x) \overline{g(x)}\big) \; dx                                    &                                      & \by{4.6.5} \\
     & \quad + i \bigg(\int_{[0, 1]} \Re(c) \Im\big(f(x) \overline{g(x)}\big) + \Im(c) \Re\big(f(x) \overline{g(x)}\big) \; dx\bigg)                                                                    \\
     & = \Re(c) \bigg(\int_{[0, 1]} \Re\big(f(x) \overline{g(x)}\big) \; dx\bigg)                                                                   & (f \overline{g} \in C(\R / \Z ; \C))              \\
     & \quad - \Im(c) \bigg(\int_{[0, 1]} \Im\big(f(x) \overline{g(x)}\big) \; dx\bigg)                                                                                                                 \\
     & \quad + i \Re(c) \bigg(\int_{[0, 1]} \Im\big(f(x) \overline{g(x)}\big) \; dx\bigg)                                                                                                               \\
     & \quad + i \Im(c) \bigg(\int_{[0, 1]} \Re\big(f(x) \overline{g(x)}\big) \; dx\bigg)                                                                                                               \\
     & = \Re(c) \Bigg(\int_{[0, 1]} \Re\big(f(x) \overline{g(x)}\big) \; dx + i \int_{[0, 1]} \Im\big(f(x) \overline{g(x)}\big) \; dx\Bigg)         &                                      & \by{4.6.6} \\
     & \quad + i \Im(c) \Bigg(\int_{[0, 1]} \Re\big(f(x) \overline{g(x)}\big) \; dx + i \int_{[0, 1]} \Im\big(f(x) \overline{g(x)}\big) \; dx\Bigg) &                                      & \by{4.6.5} \\
     & = \Re(c) \int_{[0, 1]} f(x) \overline{g(x)} \; dx + i \Im(c) \int_{[0, 1]} f(x) \overline{g(x)} \; dx                                        &                                      & \by{5.2.2} \\
     & = \Re(c) \inner*{f, g} + i \Im(c) \inner*{f, g}                                                                                              &                                      & \by{5.2.1} \\
     & = \big(\Re(c) + i \Im(c)\big) \inner*{f, g}                                                                                                  &                                      & \by{4.6.6} \\
     & = c \inner*{f, g}.                                                                                                                           &                                      & \by{4.6.8}
  \end{align*}
\end{proof}

\begin{proof}{(d)}
  We have
  \begin{align*}
     & \inner*{f, g + h}                                                                                                                                                                                       \\
     & = \int_{[0, 1]} f(x) \overline{(g + h)(x)} \; dx                                                                                    &                                                      & \by{5.2.1} \\
     & = \int_{[0, 1]} \Re\big(f(x) \overline{(g + h)(x)}\big) \; dx                                                                                                                                           \\
     & \quad + i \bigg(\int_{[0, 1]} \Im\big(f(x) \overline{(g + h)(x)}\big) \; dx\bigg)                                                   &                                                      & \by{5.2.2} \\
     & = \int_{[0, 1]} \Re\big(f(x) \overline{g(x)} + f(x) \overline{h(x)}\big) \; dx                                                      &                                                      & \by{4.6.9} \\
     & \quad + i \bigg(\int_{[0, 1]} \Im\big(f(x) \overline{g(x)} + f(x) \overline{h(x)}\big) \; dx\bigg)                                                                                                      \\
     & = \int_{[0, 1]} \Re\big(f(x) \overline{g(x)}) + \Re\big(f(x) \overline{h(x)}\big) \; dx                                             &                                                      & \by{4.6.8} \\
     & \quad + i \bigg(\int_{[0, 1]} \Im\big(f(x) \overline{g(x)}\big) + \Im\big(f(x) \overline{h(x)}\big) \; dx\bigg)                                                                                         \\
     & = \int_{[0, 1]} \Re\big(f(x) \overline{g(x)}) \; dx + \int_{[0, 1]} \Re\big(f(x) \overline{h(x)}\big) \; dx                         & (f \overline{g}, f \overline{h} \in C(\R / \Z ; \C))              \\
     & \quad + i \bigg(\int_{[0, 1]} \Im\big(f(x) \overline{g(x)}\big) \; dx + \int_{[0, 1]} \Im\big(f(x) \overline{h(x)}\big) \; dx\bigg)                                                                     \\
     & = \int_{[0, 1]} f(x) \overline{g(x)} \; dx + \int_{[0, 1]} f(x) \overline{h(x)} \; dx                                               &                                                      & \by{5.2.2} \\
     & = \inner*{f, g} + \inner*{f, h}                                                                                                     &                                                      & \by{5.2.1}
  \end{align*}
  and
  \begin{align*}
     & \inner*{f, cg}                                                                                                                                                                               \\
     & = \int_{[0, 1]} (x) \overline{(cg)(x)} \; dx                                                                                                                &  & \by{5.2.1}                  \\
     & = \int_{[0, 1]} \Re\big(f(x) \overline{(cg)(x)}\big) \; dx + i \bigg(\int_{[0, 1]} \Im\big(f(x) \overline{(cg)(x)}\big) \; dx\bigg)                         &  & \by{5.2.2}                  \\
     & = \int_{[0, 1]} \Re\big(\overline{c} f(x) \overline{g(x)}\big) \; dx + i \bigg(\int_{[0, 1]} \Im\big(\overline{c} f(x) \overline{g(x)}\big) \; dx\bigg)     &  & \by{4.6.9}                  \\
     & = \int_{[0, 1]} \Re\big((\overline{c} f)(x) \overline{g(x)}\big) \; dx + i \bigg(\int_{[0, 1]} \Im\big((\overline{c} f)(x) \overline{g(x)}\big) \; dx\bigg)                                  \\
     & = \int_{[0, 1]} (\overline{c} f)(x) \overline{g(x)} \; dx                                                                                                   &  & \by{5.2.2}                  \\
     & = \inner*{\overline{c} f, g}                                                                                                                                &  & \by{5.2.1}                  \\
     & = \overline{c} \inner*{f, g}.                                                                                                                               &  & \text{(by \cref{5.2.5}(c))}
  \end{align*}
\end{proof}

\begin{ac}\label{ac:5.2.1}
  From the positivity property (\cref{5.2.5}(b)), it makes sense to define the \(L^2\) norm \(\norm*{f}_2\) of a function \(f \in C(\R / \Z ; \C)\) by the formula
  \[
    \norm*{f}_2 \coloneqq \sqrt{\inner*{f, f}} = \bigg(\int_{[0, 1]} f(x) \overline{f(x)} \; dx\bigg)^{1 / 2} = \bigg(\int_{[0, 1]} \abs{f(x)}^2 \; dx\bigg)^{1 / 2}.
  \]
  Thus \(\norm*{f}_2 \geq 0\) for all \(f\).
  The norm \(\norm*{f}_2\) is sometimes called the \emph{root mean square} of \(f\).
\end{ac}

\begin{note}
  This \(L^2\) norm is related to, but is distinct from, the \(L^\infty\) norm
  \[
    \norm*{f}_\infty \coloneqq \sup_{x \in \R} \abs{f(x)}.
  \]
  In general, the best one can say is that \(0 \leq \norm*{f}_2 \leq \norm*{f}_\infty\).
\end{note}

\setcounter{thm}{6}
\begin{lem}\label{5.2.7}
  Let \(f, g \in C(\R / \Z ; \C)\).
  \begin{enumerate}
    \item (Non-degeneracy)
          We have \(\norm*{f}_2 = 0\) iff \(f = 0\).
    \item (Cauchy-Schwarz inequality)
          We have \(\abs{\inner*{f, g}} \leq \norm*{f}_2 \norm*{g}_2\).
    \item (Triangle inequality)
          We have \(\norm*{f + g}_2 \leq \norm*{f}_2 + \norm*{g}_2\).
    \item (Pythagoras' theorem)
          If \(\inner*{f, g} = 0\), then \(\norm*{f + g}_2^2 = \norm*{f}_2^2 + \norm*{g}_2^2\).
    \item (Homogeneity)
          We have \(\norm*{cf}_2 = \abs{c} \norm*{f}_2\) for all \(c \in \C\).
  \end{enumerate}
\end{lem}

\begin{proof}{(a)}
  We have
  \begin{align*}
         & \norm*{f}_2 = 0                                           \\
    \iff & \sqrt{\inner*{f, f}} = 0 &  & \by{ac:5.2.1}               \\
    \iff & \inner*{f, f} = 0                                         \\
    \iff & f = 0.                   &  & \text{(by \cref{5.2.5}(b))}
  \end{align*}
\end{proof}

\begin{proof}{(b)}
  If \(g\) is zero function on \([0, 1]\), then we have
  \begin{align*}
    \abs{\inner*{f, g}} & = \abs{\int_{[0, 1]} f(x) \overline{g(x)} \; dx} &  & \by{5.2.1}                  \\
                        & = \abs{\int_{[0, 1]} f(x) \cdot 0 \; dx}                                          \\
                        & = \abs{\int_{[0, 1]} 0 \; dx}                                                     \\
                        & = 0                                                                               \\
                        & = \norm*{f}_2 \norm*{g}_2.                       &  & \text{(by \cref{5.2.7}(a))}
  \end{align*}
  So suppose that \(g\) is not zero function on \([0, 1]\).
  Observe that
  \begin{align*}
             & \norm*{g}_2 \in \R                                                 &  & \by{ac:5.2.1}               \\
    \implies & \norm*{g}_2^2 \in \R                                                                                \\
    \implies & \norm*{g}_2^2 \cdot f \in C(\R / \Z ; \C)                          &  & \text{(by \cref{5.1.5}(b))} \\
    \implies & \norm*{g}_2^2 \cdot f - \inner*{f, g} \cdot g \in C(\R / \Z ; \C). &  & \text{(by \cref{5.1.5}(b))}
  \end{align*}
  If we let \(h = \norm*{g}_2^2 \cdot f - \inner*{f, g} \cdot g\), then by \cref{ac:5.2.1} we know that \(\inner*{h, h}\) is well-defined.
  Thus we have
  \begin{align*}
     & \inner*{h, h}                                                                                                                                             \\
     & = \inner*{\norm*{g}_2^2 \cdot f - \inner*{f, g} \cdot g, \norm*{g}_2^2 \cdot f - \inner*{f, g} \cdot g}                                                   \\
     & = \inner*{\norm*{g}_2^2 \cdot f, \norm*{g}_2^2 \cdot f - \inner*{f, g} \cdot g}                                          &  & \text{(by \cref{5.2.5}(c))} \\
     & \quad + \inner*{-\inner*{f, g} \cdot g, \norm*{g}_2^2 \cdot f - \inner*{f, g} \cdot g}                                                                    \\
     & = \inner*{\norm*{g}_2^2 \cdot f, \norm*{g}_2^2 \cdot f} + \inner*{\norm*{g}_2^2 \cdot f, -\inner*{f, g} \cdot g}         &  & \text{(by \cref{5.2.5}(d))} \\
     & \quad + \inner*{-\inner*{f, g} \cdot g, \norm*{g}_2^2 \cdot f} + \inner*{-\inner*{f, g} \cdot g, -\inner*{f, g} \cdot g}                                  \\
     & = \norm*{g}_2^2 \inner*{f, \norm*{g}_2^2 \cdot f} + \norm*{g}_2^2 \inner*{f, -\inner*{f, g} \cdot g}                     &  & \text{(by \cref{5.2.5}(c))} \\
     & \quad - \inner*{f, g} \inner*{g, \norm*{g}_2^2 \cdot f} - \inner*{f, g} \inner*{g, -\inner*{f, g} \cdot g}                                                \\
     & = \norm*{g}_2^2 \overline{\norm*{g}_2^2} \inner*{f, f} + \norm*{g}_2^2 \overline{-\inner*{f, g}} \inner*{f, g}           &  & \text{(by \cref{5.2.5}(d))} \\
     & \quad - \inner*{f, g} \overline{\norm*{g}_2^2} \inner*{g, f} - \inner*{f, g} \overline{-\inner*{f, g}} \inner*{g, g}                                      \\
     & = \norm*{g}_2^4 \inner*{f, f} - \norm*{g}_2^2 \overline{\inner*{f, g}} \inner*{f, g}                                     &  & \by{4.6.9}                  \\
     & \quad - \inner*{f, g} \norm*{g}_2^2 \inner*{g, f} + \inner*{f, g} \overline{\inner*{f, g}} \inner*{g, g}                                                  \\
     & = \norm*{g}_2^4 \inner*{f, f} - \norm*{g}_2^2 \overline{\inner*{f, g}} \inner*{f, g}                                                                      \\
     & \quad - \inner*{f, g} \norm*{g}_2^2 \overline{\inner*{f, g}} + \inner*{f, g} \overline{\inner*{f, g}} \inner*{g, g}      &  & \text{(by \cref{5.2.5}(a))} \\
     & = \norm*{g}_2^4 \inner*{f, f} - 2 \norm*{g}_2^2 \abs{\inner*{f, g}}^2 + \abs{\inner*{f, g}}^2 \inner*{g, g}              &  & \by{4.6.10}                 \\
     & = \norm*{g}_2^4 \norm*{f}_2^2 - 2 \norm*{g}_2^2 \abs{\inner*{f, g}}^2 + \abs{\inner*{f, g}}^2 \norm*{g}_2^2              &  & \by{ac:5.2.1}               \\
     & = \norm*{g}_2^4 \norm*{f}_2^2 - \norm*{g}_2^2 \abs{\inner*{f, g}}^2
  \end{align*}
  and
  \begin{align*}
             & \inner*{h, h} \geq 0                                                     &  & \text{(by \cref{5.2.5}(b))} \\
    \implies & \norm*{g}_2^4 \norm*{f}_2^2 - \norm*{g}_2^2 \abs{\inner*{f, g}}^2 \geq 0                                  \\
    \implies & \norm*{g}_2^2 \norm*{f}_2^2 - \abs{\inner*{f, g}}^2 \geq 0               &  & \text{(by \cref{5.2.7}(a))} \\
    \implies & \norm*{g}_2 \norm*{f}_2 \geq \abs{\inner*{f, g}}.                        &  & \text{(by \cref{5.2.5}(b))}
  \end{align*}
\end{proof}

\begin{proof}{(c)}
  We have
  \begin{align*}
    \norm*{f + g}_2^2 & = \inner*{f + g, f + g}                                         &  & \by{ac:5.2.1}                  \\
                      & = \inner*{f, f} + \inner*{f, g} + \inner*{g, f} + \inner*{g, g} &  & \text{(by \cref{5.2.5}(c)(d))} \\
                      & = \norm*{f}_2^2 + \inner*{f, g} + \inner*{g, f} + \norm*{g}_2^2 &  & \by{ac:5.2.1}                  \\
                      & \leq \norm*{f}_2^2 + 2 \norm*{f}_2 \norm*{g}_2 + \norm*{g}_2^2  &  & \text{(by \cref{5.2.7}(b))}    \\
                      & = \big(\norm*{f}_2 + \norm*{g}_2\big)^2.
  \end{align*}
  Thus
  \begin{align*}
             & \norm*{f + g}_2^2 \leq (\norm*{f}_2 + \norm*{g}_2)^2                                  \\
    \implies & \norm*{f + g}_2 \leq \norm*{f}_2 + \norm*{g}_2.      &  & \text{(by \cref{5.2.5}(b))}
  \end{align*}
\end{proof}

\begin{proof}{(d)}
  We have
  \begin{align*}
    \norm*{f + g}_2^2 & = \inner*{f + g, f + g}                                                    &  & \by{ac:5.2.1}                  \\
                      & = \inner*{f, f} + \inner*{f, g} + \inner*{g, f} + \inner*{g, g}            &  & \text{(by \cref{5.2.5}(c)(d))} \\
                      & = \inner*{f, f} + \inner*{f, g} + \overline{\inner*{f, g}} + \inner*{g, g} &  & \text{(by \cref{5.2.5}(a))}    \\
                      & = \inner*{f, f} + \inner*{g, g}                                            &  & \text{(by hypothesis)}         \\
                      & = \norm*{f}_2^2 + \norm*{g}_2^2.                                           &  & \by{ac:5.2.1}
  \end{align*}
\end{proof}

\begin{proof}{(e)}
  We have
  \begin{align*}
    \norm*{cf}_2 & = \sqrt{\inner*{cf, cf}}              &  & \by{ac:5.2.1}                  \\
                 & = \sqrt{c \overline{c} \inner*{f, f}} &  & \text{(by \cref{5.2.5}(c)(d))} \\
                 & = \sqrt{\abs{c}^2 \inner*{f, f}}      &  & \by{4.6.11}                    \\
                 & = \abs{c} \sqrt{\inner*{f, f}}                                            \\
                 & = \abs{c} \norm*{f}_2.                &  & \by{ac:5.2.1}
  \end{align*}
\end{proof}

\begin{note}
  In light of Pythagoras' theorem, we sometimes say that \(f\) and \(g\) are \emph{orthogonal} iff \(\inner*{f, g} = 0\).
\end{note}

\begin{ac}\label{ac:5.2.2}
  We can now define the \(L^2\) metric \(d_{L^2}\) on \(C(\R / \Z ; \C)\) by defining
  \[
    d_{L^2}(f, g) \coloneqq \norm*{f - g}_2 = \bigg(\int_{[0, 1]} \abs{f(x) - g(x)}^2 \; dx\bigg)^{1 / 2}.
  \]
\end{ac}

\begin{rmk}\label{5.2.8}
  One can verify that \(d_{L^2}\) is indeed a metric.
  Indeed, the \(L^2\) metric is very similar to the \(l^2\) metric on Euclidean spaces \(\R^n\), which is why the notation is deliberately chosen to be similar;
  you should compare the two metrics yourself to see the analogy.
\end{rmk}

\begin{note}
  A sequence \(f_n\) of functions in \(C(\R / \Z ; \C)\) will \emph{converge in the \(L^2\) metric} to \(f \in C(\R / \Z ; \C)\) if \(d_{L^2}(f_n, f) \to 0\) as \(n \to \infty\), or in other words that
  \[
    \lim_{n \to \infty} \int_{[0, 1]} \abs{f_n(x) - f(x)}^2 \; dx = 0.
  \]
\end{note}

\begin{rmk}\label{5.2.9}
  The notion of convergence in \(L^2\) metric is different from that of uniform or pointwise convergence.
\end{rmk}

\begin{rmk}\label{5.2.10}
  The \(L^2\) metric is not as well-behaved as the \(L_\infty\) metric.
  For instance, it turns out the space \(C(\R / \Z ; \C)\) is not complete in the \(L^2\) metric, despite being complete in the \(L_\infty\) metric.
\end{rmk}

\exercisesection

\begin{ex}\label{ex:5.2.1}
  Prove \cref{5.2.5}.
\end{ex}

\begin{proof}
  See \cref{5.2.5}.
\end{proof}

\begin{ex}\label{ex:5.2.2}
  Prove \cref{5.2.7}.
\end{ex}

\begin{proof}
  See \cref{5.2.7}.
\end{proof}

\begin{ex}\label{ex:5.2.3}
  If \(f \in C(\R / \Z ; \C)\) is a non-zero function, show that \(0 < \norm*{f}_2 \leq \norm*{f}_\infty\).
  Conversely, if \(0 < A \leq B\) are real numbers, show that there exists a non-zero function \(f \in C(\R / \Z ; \C)\) such that \(\norm*{f}_2 = A\) and \(\norm*{f}_\infty = B\).
\end{ex}

\begin{proof}
  First we show that \(f \in C(\R / \Z ; \C)\) and \(f \neq 0\) implies \(0 < \norm*{f}_2 \leq \norm*{f}_{\infty}\).
  By \cref{5.2.7}(a) we know that \(0 < \norm*{f}_2\).
  Thus we only need to show that \(\norm*{f}_2 \leq \norm*{f}_\infty\).
  By \cref{5.1.5}(a) we know that \(f\) is bounded, thus by \cref{3.5.5}
  \[
    \norm*{f}_{\infty} = \sup_{y \in \R} \abs{f(x)} = \sup_{y \in [0, 1]} \abs{f(x)} \in \R^+ \cup \set{0}.
  \]
  Since
  \begin{align*}
    \norm*{f}_2^2 & = \int_{[0, 1]} \abs{f(x)}^2 \; dx                                  &  & \by{ac:5.2.1} \\
                  & \leq \int_{[0, 1]} \big(\sup_{y \in [0, 1]} \abs{f(y)}\big)^2 \; dx                    \\
                  & = \big(\sup_{y \in [0, 1]} \abs{f(y)}\big)^2                                           \\
                  & = \norm*{f}_{\infty}^2,                                             &  & \by{3.5.5}
  \end{align*}
  we know that
  \[
    \norm*{f}_2^2 \leq \norm*{f}_{\infty}^2 \implies \norm*{f}_2 \leq \norm*{f}_{\infty}.
  \]

  Now we show that for arbitrary \(A, B \in \R\), we have
  \[
    0 < A \leq B \implies \exists f \in C(\R / \Z ; \C) : \begin{dcases}
      f \neq 0        \\
      \norm*{f}_2 = A \\
      \norm*{f}_{\infty} = B
    \end{dcases}
  \]
  So let \(A, B \in \R\) such that \(0 < A \leq B\).
  We want to find some \(f \in C(\R / \Z ; \C)\) such that
  \begin{align*}
    A^2 & = \norm*{f}_2^2 = \int_{[0, 1]} \abs{f(x)}^2 \; dx;                  \\
    B^2 & = \norm*{f}_{\infty}^2 = \big(\sup_{x \in [0, 1]} \abs{f(x)}\big)^2.
  \end{align*}
  In particular, we want our \(f\) to look like
  \[
    \forall x \in [0, 1], f(x) = \sqrt{c + d g(x)},
  \]
  where \(c, d \in \R^+\) and \(g \in C(\R / \Z ; \C)\) such that \(g(\R) \subseteq \R^+ \cup \set{0}\).
  So we are trying to solve the following equations:
  \begin{align*}
    A^2 & = \int_{[0, 1]} \abs{\sqrt{c + dg(x)}}^2 \; dx           \\
        & = \int_{[0, 1]} c + dg(x) \; dx                          \\
        & = c + d \int_{[0, 1]} g(x) \; dx;                        \\
    B^2 & = \big(\sup_{x \in [0, 1]} \abs{\sqrt{c + dg(x)}}\big)^2 \\
        & = \sup_{x \in [0, 1]} \abs{\sqrt{c + dg(x)}}^2           \\
        & = \sup_{x \in [0, 1]} \big(c + dg(x)\big)                \\
        & = c + d \big(\sup_{x \in [0, 1]} g(x)\big).
  \end{align*}
  By setting
  \begin{align*}
     & c = \dfrac{A^2}{2};                                                                                                                                      \\
     & d = \dfrac{1}{2};                                                                                                                                        \\
     & \forall x \in [0, 1], g(x) = \begin{dcases}
                                      \dfrac{(2 B^2 - A^2)^2}{A^2} x                    & \text{if } x \in [0, \dfrac{A^2}{2 B^2 - A^2})                          \\
                                      \dfrac{-(2 B^2 - A^2)^2}{A^2} x + 2 (2 B^2 - A^2) & \text{if } x \in [\dfrac{A^2}{2 B^2 - A^2}, \dfrac{2 A^2}{2 B^2 - A^2}) \\
                                      0                                                 & \text{if } x \in [\dfrac{2 A^2}{2 B^2 - A^2}, 1]
                                    \end{dcases},
  \end{align*}
  we have
  \begin{align*}
     & \int_{[0, 1]} g(x) \; dx                                                                                                                                                                                                     \\
     & = \int_{[0, \dfrac{A^2}{2 B^2 - A^2}]} \dfrac{(2 B^2 - A^2)^2}{A^2} x \; dx + \int_{[\dfrac{A^2}{2 B^2 - A^2}, \dfrac{2 A^2}{2 B^2 - A^2}]} \dfrac{-(2 B^2 - A^2)^2}{A^2} x + 2 (2 B^2 - A^2) \; dx                          \\
     & = \dfrac{(2 B^2 - A^2)^2}{A^2} \bigg(\dfrac{x^2}{2}|_{x = 0}^{x = \dfrac{A^2}{2 B^2 - A^2}}\bigg) - \dfrac{(2 B^2 - A^2)^2}{A^2} \bigg(\dfrac{x^2}{2}|_{x = \dfrac{A^2}{2 B^2 - A^2}}^{x = \dfrac{2 A^2}{2 B^2 - A^2}}\bigg) \\
     & \quad + 2 (2 B^2 - A^2) \bigg(\dfrac{2 A^2}{2 B^2 - A^2} - \dfrac{A^2}{2 B^2 - A^2}\bigg)                                                                                                                                    \\
     & = \dfrac{A^2}{2} - 2 A^2 + \dfrac{A^2}{2} + 2 A^2                                                                                                                                                                            \\
     & = A^2
  \end{align*}
  and
  \begin{align*}
    \sup_{[0, 1]} g(x) & = \dfrac{(2 B^2 - A^2)^2}{A^2} \dfrac{A^2}{2 B^2 - A^2} \\
                       & = 2 B^2 - A^2.
  \end{align*}
  Thus
  \begin{align*}
    c + d \int_{[0, 1]} g(x) \; dx           & = \dfrac{A^2}{2} + \dfrac{A^2}{2}         \\
                                             & = A^2;                                    \\
    c + d \big(\sup_{x \in [0, 1]} g(x)\big) & = \dfrac{A^2}{2} + \dfrac{2 B^2 - A^2}{2} \\
                                             & = B^2.
  \end{align*}
  Note that the idea behind the definition of \(g\) is we try to build a triangle in the interval \([0, 1]\) with height equals to \(2 B^2 - A^2\) (this explains the result of supremum), and we want that triangle's area equals to \(A^2\) (this explains the result of integration).
  One can easily show that by extended \(g\) periodically with period \(1\) we know that \(g \in C(\R / \Z ; \C)\).
\end{proof}

\begin{ex}\label{ex:5.2.4}
  Prove that the \(d_{L^2}\) metric on \(C(\R / \Z ; \C)\) does indeed turn \(C(\R / \Z ; \C)\) into a metric space.
\end{ex}

\begin{proof}
  Let \(f, g, h \in C(\R / \Z ; \C)\).
  Since
  \begin{align*}
    d_{L^2}(f, f) & = \norm*{f - f}_2 &  & \by{ac:5.2.2}               \\
                  & = \norm*{0}_2     &  & \text{(by \cref{5.2.5}(b))} \\
                  & = 0,              &  & \text{(by \cref{5.2.7}(a))}
  \end{align*}
  we know that \(\big(C(\R / \Z ; \C), d_{L^2}\big)\) satisfies \cref{1.1.2}(a).
  Since
  \begin{align*}
             & f \neq g                                             \\
    \implies & f - g \neq 0        &  & \text{(by \cref{5.2.5}(b))} \\
    \implies & \norm*{f - g}_2 > 0 &  & \text{(by \cref{5.2.7}(a))} \\
    \implies & d_{L^2}(f, g) > 0,  &  & \by{ac:5.2.2}
  \end{align*}
  we know that \(\big(C(\R / \Z ; \C), d_{L^2}\big)\) satisfies \cref{1.1.2}(b).
  Since
  \begin{align*}
    d_{L^2}(f, g) & = \norm*{f - g}_2              &  & \by{ac:5.2.2}                  \\
                  & = \sqrt{\inner*{f - g, f - g}} &  & \by{ac:5.2.1}                  \\
                  & = \sqrt{\inner*{g - f, g - f}} &  & \text{(by \cref{5.2.5}(c)(d))} \\
                  & = \norm*{g - f}_2              &  & \by{ac:5.2.1}                  \\
                  & = d_{L^2}(g, f),               &  & \by{ac:5.2.2}
  \end{align*}
  we know that \(\big(C(\R / \Z ; \C), d_{L^2}\big)\) satisfies \cref{1.1.2}(c).
  Since
  \begin{align*}
    d_{L^2}(f, g) + d_{L^2}(g, h) & = \norm*{f - g}_2 + \norm*{g - h}_2 &  & \by{ac:5.2.2}               \\
                                  & \geq \norm{f - g + g - h}_2         &  & \text{(by \cref{5.2.7}(c))} \\
                                  & = \norm{f - h}_2                                                     \\
                                  & = d_{L^2}(f, h),                    &  & \by{ac:5.2.2}
  \end{align*}
  we know that \(\big(C(\R / \Z ; \C), d_{L^2}\big)\) satisfies \cref{1.1.2}(d).
  From all proofs above we conclude by \cref{1.1.2} that \(\big(C(\R / \Z ; \C), d_{L^2}\big)\) is a metric space.
\end{proof}

\begin{ex}\label{ex:5.2.5}
  Find a sequence of continuous periodic functions which converge in \(L^2\) to a discontinuous periodic function.
\end{ex}

\begin{proof}
  By \cref{5.1.4} we can define a \(\Z\)-periodic square wave function \(f : \R \to \C\) as follow:
  \[
    \forall x \in \R, f(x) = \begin{dcases}
      1 & \text{if } x \in [n, n + \dfrac{1}{2}) \text{ for some } n \in \Z     \\
      0 & \text{if } x \in [n + \dfrac{1}{2}, n + 1) \text{ for some } n \in \Z
    \end{dcases}
  \]
  Note that \(f\) is \(1\)-periodic but \(f\) is not continuous on \(\R\).
  Let \(\N_{\geq 10} = \set{n \in \N : n \geq 10}\).
  For each \(k \in \N_{\geq 10}\), we define \(f_k : [0, 1) \to \C\) to be the function:
  \[
    \forall x \in [0, 1), f_k(x) = \begin{dcases}
      kx                 & \text{if } x \in [0, \dfrac{1}{k})                           \\
      1                  & \text{if } x \in [\dfrac{1}{k}, \dfrac{1}{2} - \dfrac{1}{k}) \\
      -kx + \dfrac{k}{2} & \text{if } x \in [\dfrac{1}{2} - \dfrac{1}{k}, \dfrac{1}{2}) \\
      0                  & \text{if } x \in [0 + \dfrac{1}{2}, 1)
    \end{dcases}
  \]
  If we extended \(f_k\) periodically with period \(1\), then \(f_k \in C(\R / \Z ; \C)\) for all \(k \in \N_{\geq 10}\).
  Note that the choice of \(10\) is to make sure \(\dfrac{1}{k} < \dfrac{1}{2} - \dfrac{1}{k} < \dfrac{1}{2}\).
  Now we show that \((f_k)_{k = 10}^\infty\) converges to \(f\) on \([0, 1)\) with respect to \(d_{L^2}\).
  In particular, we want to show that
  \begin{align*}
         & \lim_{k \to \infty} d_{L^2}(f_k, f) = 0                                                                  \\
    \iff & \lim_{k \to \infty} \bigg(\int_{[0, 1]} \abs{f_k(x) - f(x)}^2 \; dx\bigg)^{1 / 2} = 0 &  & \by{ac:5.2.2} \\
    \iff & \lim_{k \to \infty} \int_{[0, 1]} \abs{f_k(x) - f(x)}^2 \; dx = 0.
  \end{align*}
  Since for each \(k \in \N_{\geq 10}\), we have
  \begin{align*}
     & \int_{[0, 1]} \abs{f_k(x) - f(x)}^2 \; dx                                                                                                                                                                                                    \\
     & = \int_{[0, \dfrac{1}{k}]} (1 - kx)^2 \; dx + \int_{[\dfrac{1}{2} - \dfrac{1}{k}, \dfrac{1}{2}]} \bigg(1 - \dfrac{k}{2} + kx\bigg)^2 \; dx                                                                                                   \\
     & = \int_{[0, \dfrac{1}{k}]} 1 - 2kx + k^2 x^2 \; dx + \int_{[\dfrac{1}{2} - \dfrac{1}{k}, \dfrac{1}{2}]} 1 - k + \dfrac{k^2}{4} + 2kx - k^2 x + k^2 x^2 \; dx                                                                                 \\
     & = \dfrac{1}{k} - 2k \bigg(\dfrac{x^2}{2}|_{x = 0}^{x = \dfrac{1}{k}}\bigg) + k^2 \bigg(\dfrac{x^3}{3}|_{x = 0}^{x = \dfrac{1}{k}}\bigg)                                                                                                      \\
     & \quad + \dfrac{1}{k} \bigg(1 - k + \dfrac{k^2}{4}\bigg) + (2k - k^2) \bigg(\dfrac{x^2}{2}|_{x = \dfrac{1}{2} - \dfrac{1}{k}}^{x = \dfrac{1}{2}}\bigg) + k^2 \bigg(\dfrac{x^3}{3}|_{x = \dfrac{1}{2} - \dfrac{1}{k}}^{x = \dfrac{1}{2}}\bigg) \\
     & = \dfrac{1}{k} - \dfrac{1}{k} + \dfrac{1}{3k} + \dfrac{1}{k} - 1 + \dfrac{k}{4} + \dfrac{2k - k^2}{2} \bigg(\dfrac{1}{k} - \dfrac{1}{k^2}\bigg) + \dfrac{k^2}{3} \bigg(\dfrac{3}{4k} - \dfrac{3}{2k^2} + \dfrac{1}{k^3}\bigg)                \\
     & = \dfrac{2}{3k},
  \end{align*}
  we know that
  \[
    \lim_{k \to \infty} \int_{[0, 1]} \abs{f_k(x) - f(x)}^2 \; dx = \lim_{k \to \infty} \dfrac{2}{3k} = 0.
  \]
  Thus \((f_k)_{k = 10}^\infty\) converges to \(f\) on \([0, 1)\) with respect to \(d_{L^2}\).
  Since \(f\) and \(f_k\) are \(1\)-periodic for all \(k \in \N_{\geq 10}\), we know that \((f_k)_{k = 10}^\infty\) converges to \(f\) on \(\R\) with respect to \(d_{L^2}\).
\end{proof}

\begin{ex}\label{ex:5.2.6}
  Let \(f \in C(\R / \Z ; \C)\), and let \((f_n)_{n = 1}^\infty\) be a sequence of functions in \(C(\R / \Z ; \C)\).
  \begin{enumerate}
    \item Show that if \(f_n\) converges uniformly to \(f\), then \(f_n\) also converges to \(f\) in the \(L^2\) metric.
    \item Give an example where \(f_n\) converges to \(f\) in the \(L^2\) metric, but does not converge to \(f\) uniformly.
    \item Give an example where \(f_n\) converges to \(f\) in the \(L^2\) metric, but does not converge to \(f\) pointwise.
    \item Give an example where \(f_n\) converges to \(f\) pointwise, but does not converge to \(f\) in the \(L^2\) metric.
  \end{enumerate}
\end{ex}

\begin{proof}{(a)}
  We have
  \begin{align*}
             & \forall \varepsilon \in \R^+, \exists N \in \Z^+ : \forall n \geq N, \forall x \in \R,                                           &  & \by{3.2.7}    \\
             & \abs{f_n(x) - f(x)} < \dfrac{\varepsilon^{\dfrac{1}{2}}}{2}                                                                                         \\
    \implies & \forall \varepsilon \in \R^+, \exists N \in \Z^+ : \forall n \geq N, \forall x \in [0, 1],                                                          \\
             & \abs{f_n(x) - f(x)} < \dfrac{\varepsilon^{\dfrac{1}{2}}}{2}                                                                                         \\
    \implies & \forall \varepsilon \in \R^+, \exists N \in \Z^+ : \forall n \geq N, \forall x \in [0, 1],                                                          \\
             & \abs{f_n(x) - f(x)}^2 < \dfrac{\varepsilon}{4}                                                                                                      \\
    \implies & \forall \varepsilon \in \R^+, \exists N \in \Z^+ : \forall n \geq N,                                                                                \\
             & \int_{[0, 1]} \abs{f_n(x) - f(x)}^2 \; dx \leq \int_{[0, 1]} \dfrac{\varepsilon}{4} \; dx = \dfrac{\varepsilon}{4} < \varepsilon                    \\
    \implies & \forall \varepsilon \in \R^+, \exists N \in \Z^+ : \forall n \geq N,                                                                                \\
             & d_{L^2}(f_n, f) < \varepsilon                                                                                                    &  & \by{ac:5.2.2} \\
    \implies & d_{L^2} - \lim_{n \to \infty} f_n = f.                                                                                           &  & \by{1.1.14}
  \end{align*}
\end{proof}

\begin{proof}{(b)}
  Let \(f \in C(\R / \Z ; \C)\) such that \(f = 0\) and let \(\N_{\geq 2} = \set{n \in \N : n \geq 2}\).
  For all \(n \in \N_{\geq 2}\), we define \(f_n \in C(\R / \Z ; \C)\) as follow:
  \[
    \forall x \in [0, 1), f_n(x) = \begin{dcases}
      0                           & \text{if } x \in [0, \dfrac{1}{2} - \dfrac{1}{n^3})            \\
      n^4 x + n - \dfrac{n^4}{2}  & \text{if } x \in [\dfrac{1}{2} - \dfrac{1}{n^3}, \dfrac{1}{2}) \\
      -n^4 x + n + \dfrac{n^4}{2} & \text{if } x \in [\dfrac{1}{2}, \dfrac{1}{2} + \dfrac{1}{n^3}) \\
      0                           & \text{if } x \in [\dfrac{1}{2} + \dfrac{1}{n^3}, 1)
    \end{dcases}
  \]
  Since for all \(n \in \N_{\geq 2}\), we have
  \begin{align*}
     & \int_{[0, 1]} \abs{f_n(x) - f(x)}^2 \; dx                                                                                                                                                                                                                                                                      \\
     & = \int_{[\dfrac{1}{2} - \dfrac{1}{n^3}, \dfrac{1}{2}]} (n^4 x + n - \dfrac{n^4}{2})^2 \; dx + \int_{[\dfrac{1}{2}, \dfrac{1}{2} + \dfrac{1}{n^3}]} (-n^4 x + n + \dfrac{n^4}{2})^2 \; dx                                                                                                                       \\
     & = \int_{[\dfrac{1}{2} - \dfrac{1}{n^3}, \dfrac{1}{2}]} n^8 x^2 + (2n^5 - n^8) x + n^2 - n^5 + \dfrac{n^8}{4} \; dx                                                                                                                                                                                             \\
     & \quad + \int_{[\dfrac{1}{2}, \dfrac{1}{2} + \dfrac{1}{n^3}]} n^8 x^2 + (-2n^5 - n^8) x + n^2 + n^5 + \dfrac{n^8}{4} \; dx                                                                                                                                                                                      \\
     & = n^8 \bigg(\dfrac{x^3}{3}|_{x = \dfrac{1}{2} - \dfrac{1}{n^3}}^{x = \dfrac{1}{2} + \dfrac{1}{n^3}}\bigg) + (2n^5 - n^8) \bigg(\dfrac{x^2}{2}|_{x = \dfrac{1}{2} - \dfrac{1}{n^3}}^{x = \dfrac{1}{2}}\bigg) + (-2n^5 - n^8) \bigg(\dfrac{x^2}{2}|_{x = \dfrac{1}{2}}^{x = \dfrac{1}{2} + \dfrac{1}{n^3}}\bigg) \\
     & \quad + \bigg(n^2 - n^5 + \dfrac{n^8}{4}\bigg) \dfrac{1}{n^3} + \bigg(n^2 + n^5 + \dfrac{n^8}{4}\bigg) \dfrac{1}{n^3}                                                                                                                                                                                          \\
     & = \dfrac{n^5}{2} + \dfrac{2}{3n} + n^2 - \dfrac{n^5}{2} - \dfrac{1}{n} + \dfrac{n^2}{2} - n^2 - \dfrac{n^5}{2} - \dfrac{1}{n} - \dfrac{n^2}{2} + \dfrac{2}{n} + \dfrac{n^5}{2}                                                                                                                                 \\
     & = \dfrac{2}{3n},
  \end{align*}
  we know that
  \begin{align*}
    \lim_{n \to \infty} d_{L^2}(f_n, f) & = \lim_{n \to \infty} \int_{[0, 1]} \abs{f_n(x) - f(x)}^2 \; dx &  & \by{ac:5.2.2} \\
                                        & = \lim_{n \to \infty} \dfrac{2}{3n}                                                \\
                                        & = 0.
  \end{align*}
  Thus by \cref{1.1.14} we have
  \[
    d_{L^2} - \lim_{n \to \infty} f_n = f.
  \]
  But for all \(n \in \N_{\geq 2}\), we have
  \begin{align*}
             & f_n(\dfrac{1}{2}) = n                                                                                                \\
    \implies & \abs{f_n(\dfrac{1}{2}) - f(\dfrac{1}{2})} = \abs{f_n(\dfrac{1}{2})} \geq n > 1                                       \\
    \implies & \exists \varepsilon \in \R^+ : \forall n \geq \N_{\geq 2}, \exists x \in [0, 1) : \abs{f_n(x) - f(x)} > \varepsilon.
  \end{align*}
  Thus by \cref{3.2.7} \((f_n)_{n = 2}^\infty\) does not converges uniformly to \(f\) on \(\R\) with respect to \(d_{l^1}|_{\R \times \R}\).
\end{proof}

\begin{proof}{(c)}
  Using the definition of \(f, f_n\) in \cref{ex:5.2.6}(b), we see that \((f_n)_{n = 2}^\infty\) does not converges pointwise to \(f\) on \(\R\) with respect to \(d_{l^1}|_{\R \times \R}\).
\end{proof}

\begin{proof}{(d)}
  Let \(f \in C(\R / \Z ; \C)\) such that \(f = 0\) and let \(\N_{\geq 2} = \set{n \in \N : n \geq 2}\).
  For all \(n \in \N_{\geq 2}\), we define \(f_n \in C(\R / \Z ; \C)\) as follow:
  \[
    \forall x \in [0, 1), f_n(x) = \begin{dcases}
      2 n^2 x       & \text{if } x \in [0, \dfrac{1}{2n})            \\
      -2 n^2 x + 2n & \text{if } x \in [\dfrac{1}{2n}, \dfrac{1}{n}) \\
      0             & \text{if } x \in [\dfrac{1}{n}, 1)
    \end{dcases}
  \]
  Since
  \begin{align*}
             & \lim_{n \to \infty} \dfrac{1}{n} = 0                                                                 \\
    \implies & \forall \varepsilon \in \R^+, \exists N \in \Z^+ : \forall n \geq N, \dfrac{1}{n} < x                \\
    \implies & \forall x \in (0, \dfrac{1}{2}), \exists N \in \Z^+ : \forall n \geq N, \dfrac{1}{n} < x             \\
    \implies & \forall x \in (0, \dfrac{1}{2}), \exists N \in \Z^+ : \forall n \geq N, f(\dfrac{1}{n}) = f_n(x) = 0 \\
    \implies & \forall x \in (0, \dfrac{1}{2}), \lim_{n \to \infty} f_n(x) = 0 = f(x)
  \end{align*}
  and
  \begin{align*}
     & \lim_{n \to \infty} f_n(0) = 0 = f(0)                                  \\
     & \forall x \in [\dfrac{1}{2}, 1), \lim_{n \to \infty} f_n(x) = 0 = f(x)
  \end{align*}
  by \cref{3.2.1} we know that \((f_n)_{n = 2}^\infty\) converges pointwise to \(f\) on \(\R\) with respect to \(d_{l^1}|_{\R \times \R}\).
  But
  \begin{align*}
     & \int_{[0, 1]} \abs{f_n(x) - f(x)}^2 \; dx                                                                                                                    \\
     & = \int_{[0, \dfrac{1}{2n}]} (2 n^2 x)^2 \; dx + \int_{[\dfrac{1}{2n}, \dfrac{1}{n}]} (-2n^2 x + 2n)^2 \; dx                                                  \\
     & = \int_{[0, \dfrac{1}{2n}]} 4 n^4 x^2 \; dx + \int_{[\dfrac{1}{2n}, \dfrac{1}{n}]} 4 n^4 x^2 - 4 n^3 x + 4n^2 \; dx                                          \\
     & = 4n^4 \bigg(\dfrac{x^3}{3}|_{x = 0}^{x = \dfrac{1}{n}}\bigg) - 4n^3 \bigg(\dfrac{x^2}{2}|_{x = \dfrac{1}{2n}}^{x = \dfrac{1}{n}}\bigg) + 4n^2 \dfrac{1}{2n} \\
     & = \dfrac{4n}{3} - \dfrac{3n}{2} + 2n                                                                                                                         \\
     & = \dfrac{11n}{6}
  \end{align*}
  implies
  \begin{align*}
    \lim_{n \to \infty} d_{L^2}(f_n, f) & = \lim_{n \to \infty} \int_{[0, 1]} \abs{f_n(x) - f(x)}^2 \; dx &  & \by{ac:5.2.2} \\
                                        & = \lim_{n \to \infty} \dfrac{11n}{6}                                               \\
                                        & = +\infty.
  \end{align*}
  Thus \((f_n)_{n = 2}^\infty\) does not converges to \(f\) with respect to \(d_{L^2}\).
\end{proof}
\section{Trigonometric polynomials}\label{ii:sec:5.3}

\begin{note}
  We now define the concept of a \emph{trigonometric polynomial}.
  Just as polynomials are combinations of the functions \(x^n\) (sometimes called \emph{monomials}), trigonometric polynomials are combinations of the functions \(e^{2 \pi i n x}\) (sometimes called \emph{characters}).
\end{note}

\begin{defn}[Characters]\label{ii:5.3.1}
  For every integer \(n\), we let \(e_n \in C(\R / \Z ; \C)\) denote the function
  \[
    e_n(x) \coloneqq e^{2 \pi i n x}.
  \]
  This is sometimes referred to as the \emph{character with frequency \(n\)}.
\end{defn}

\begin{defn}[Trigonometric polynomials]\label{ii:5.3.2}
  A function \(f\) in \(C(\R / \Z ; \C)\) is said to be a \emph{trigonometric polynomial} if we can write
  \(f = \sum_{n = -N}^N c_n e_n\) for some integer \(N \geq 0\) and some complex numbers \((c_n)_{n = -N}^N\).
\end{defn}

\setcounter{thm}{3}
\begin{eg}\label{ii:5.3.4}
  For any integer \(n\), the function \(\cos(2 \pi n x)\) is a trigonometric polynomial, since
  \[
    \cos(2 \pi n x) = \dfrac{e^{2 \pi n x} + e^{- 2 \pi n x}}{2} = \dfrac{1}{2} e_{-n} + \dfrac{1}{2} e_n.
  \]
  Similarly the function \(\sin(2 \pi n x) = \dfrac{-1}{2i} e_{-n} + \dfrac{1}{2i} e_n\) is a trigonometric polynomial.
  In fact, any linear combination of sines and cosines is also a trigonometric polynomial.
\end{eg}

\begin{lem}[Characters are an orthonormal system]\label{ii:5.3.5}
  For any integers \(n\) and \(m\), we have \(\inner*{e_n, e_m} = 1\) when \(n = m\) and \(\inner*{e_n, e_m} = 0\) when \(n \neq m\).
  Also, we have \(\norm*{e_n}_2 = 1\).
\end{lem}

\begin{proof}
  Let \(n, m \in \Z\).
  Observe that
  \begin{align*}
    \inner*{e_n, e_m} & = \int_{[0, 1]} e_n(x) \overline{e_m(x)} \; dx                   &  & \by{ii:5.2.1}      \\
                      & = \int_{[0, 1]} e^{2 \pi i n x} \overline{e^{2 \pi i m x}} \; dx &  & \by{ii:5.3.1}      \\
                      & = \int_{[0, 1]} e^{2 \pi i n x} e^{- 2 \pi i m x} \; dx          &  & \by{ii:4.7.2}[c,f] \\
                      & = \int_{[0, 1]} e^{2 \pi i n x - 2 \pi i m x} \; dx              &  & \by{ii:ex:4.6.16}  \\
                      & = \int_{[0, 1]} e^{2 \pi i (n - m) x} \; dx.
  \end{align*}
  If \(n = m\), then we have
  \begin{align*}
    \inner*{e_n, e_n} & = \int_{[0, 1]} e^{2 \pi i (n - n) x} \; dx                       \\
                      & = \int_{[0, 1]} e^0 \; dx                                         \\
                      & = \int_{[0, 1]} 1 \; dx                     &  & \by{ii:4.5.2}[e] \\
                      & = 1
  \end{align*}
  and
  \begin{align*}
    \norm*{e_n}_2 & = \sqrt{\inner*{e_n, e_n}} &  & \by{ii:ac:5.2.1} \\
                  & = \sqrt{1} = 1.
  \end{align*}
  If \(n \neq m\), then we have
  \begin{align*}
     & \inner*{e_n, e_m}                                                                                                     \\
     & = \int_{[0, 1]} e^{2 \pi i (n - m) x} \; dx                                                                           \\
     & = \int_{[0, 1]} \cos\big(2 \pi (n - m) x\big) + i \sin\big(2 \pi (n - m) x\big) \; dx        &  & \by{ii:4.7.2}[f]    \\
     & = \int_{[0, 1]} \cos\big(2 \pi (n - m) x\big) \; dx                                          &  & \by{ii:5.2.2}       \\
     & \quad + i \int_{[0, 1]} \sin\big(2 \pi (n - m) x\big) \; dx                                                           \\
     & = \bigg(\dfrac{\sin\big(2 \pi (n - m) x\big)}{2 \pi (n - m)}|_{x = 0}^{x = 1}\bigg)          &  & \by{ii:4.7.2}[b]    \\
     & \quad + i \bigg(\dfrac{-\cos\big(2 \pi (n - m) x\big)}{2 \pi (n - m)}|_{x = 0}^{x = 1}\bigg)                          \\
     & = 0 - 0                                                                                      &  & \by{ii:ac:4.7.2}[c] \\
     & \quad + i \bigg(\dfrac{- (-1) + (-1)}{2 \pi (n - m)}\bigg)                                   &  & \by{ii:ac:4.7.2}[f] \\
     & = 0.
  \end{align*}
\end{proof}

\begin{cor}\label{ii:5.3.6}
  Let \(f = \sum_{n = -N}^N c_n e_n\) be a trigonometric polynomial.
  Then we have the formula
  \[
    c_n = \inner*{f, e_n}
  \]
  for all integers \(-N \leq n \leq N\).
  Also, we have \(0 = \inner*{f, e_n}\) whenever \(n > N\) or \(n < -N\).
  Also, we have the identity
  \[
    \norm*{f}_2^2 = \sum_{n = -N}^N \abs{c_n}^2.
  \]
\end{cor}

\begin{proof}
  Let \(m \in \N\).
  Then we have
  \begin{align*}
    \inner*{f, e_m} & = \inner*{\sum_{n = -N}^N (c_n e_n), e_m}         &  & \text{(by hypothesis)} \\
                    & = \sum_{n = -N}^N \inner*{c_n e_n, e_m}           &  & \by{ii:5.2.5}[c]       \\
                    & = \sum_{n = -N}^N \big(c_n \inner*{e_n, e_m}\big) &  & \by{ii:5.2.5}[c]       \\
                    & = \begin{dcases}
                          c_m & \text{if } -N \leq m \leq N         \\
                          0   & \text{if } m > N \text{ or } m < -N
                        \end{dcases}      &  & \by{ii:5.3.5}
  \end{align*}
  and
  \begin{align*}
    \norm*{f}_2^2 & = \inner*{f, f}                                            &  & \by{ii:ac:5.2.1}              \\
                  & = \inner*{f, \sum_{n = -N}^N (c_n e_n)}                    &  & \text{(by hypothesis)}        \\
                  & = \sum_{n = -N}^N \inner*{f, c_n e_n}                      &  & \by{ii:5.2.5}[d]              \\
                  & = \sum_{n = -N}^N \big(\overline{c_n} \inner*{f, e_n}\big) &  & \by{ii:5.2.5}[d]              \\
                  & = \sum_{n = -N}^N \big(\overline{c_n} c_n\big)             &  & \text{(from the proof above)} \\
                  & = \sum_{n = -N}^N \abs{c_n}^2.                             &  & \by{ii:4.6.11}
  \end{align*}
\end{proof}

\begin{defn}[Fourier transform]\label{ii:5.3.7}
  For any function \(f \in C(\R / \Z ; \C)\), and any integer \(n \in \Z\), we define the \(n^{\opTh}\) \emph{Fourier coefficient of} \(f\), denoted \(\hat{f}(n)\), by the formula
  \[
    \hat{f}(n) \coloneqq \inner*{f, e_n} = \int_{[0, 1]} f(x) e^{- 2 \pi i n x} \; dx.
  \]
  The function \(\hat{f} : \Z \to \C\) is called the \emph{Fourier transform} of \(f\).
\end{defn}

\begin{ac}\label{ii:ac:5.3.1}
  From \cref{ii:5.3.6}, we see that whenever
  \[
    f = \sum_{n = -N}^N c_n e_n
  \]
  is a trigonometric polynomial, we have
  \[
    f = \sum_{n = -N}^N \inner*{f, e_n} e_n = \sum_{n = -\infty}^\infty \inner*{f, e_n} e_n
  \]
  and in particular, we have the \emph{Fourier inversion formula}
  \[
    f = \sum_{n = -\infty}^\infty \hat{f}(n) e_n
  \]
  or in other words
  \[
    f(x) = \sum_{n = -\infty}^\infty \hat{f}(n) e^{2 \pi i n x}.
  \]
  The right-hand side is referred to as the \emph{Fourier series} of \(f\).
  Also, from the second identity of \cref{ii:5.3.6} we have the \emph{Plancherel formula}
  \[
    \norm*{f}_2^2 = \sum_{n = -\infty}^\infty \abs{\hat{f}(n)}^2.
  \]
\end{ac}

\begin{rmk}\label{ii:5.3.8}
  We stress that at present we have only proven the Fourier inversion and Plancherel formulae in the case when \(f\) is a trigonometric polynomial.
  Note that in this case that the Fourier coefficients \(\hat{f}(n)\) are mostly zero (indeed, they can only be non-zero when \(-N \leq n \leq N\)), and so this infinite sum is really just a finite sum in disguise.
  In particular, there are no issues about what sense the above series converge in;
  they both converge pointwise, uniformly, and in \(L^2\) metric, since they are just finite sums.
\end{rmk}

\begin{note}
  In the next few sections we will extend the Fourier inversion and Plancherel formulae to general functions in \(C(\R / \Z ; \C)\), not just trigonometric polynomials.
  (It is also possible to extend the formula to discontinuous functions such as the square wave, but we will not do so here.)
  To do this we will need a version of the Weierstrass approximation theorem, this time requiring that a continuous periodic function be approximated uniformly by \emph{trigonometric} polynomials.
  Just as convolutions were used in the proof of the polynomial Weierstrass approximation theorem, we will also need a notion of convolution tailored for periodic functions.
\end{note}

\exercisesection

\begin{ex}\label{ii:ex:5.3.1}
  Show that the sum or product of any two trigonometric polynomials is again a trigonometric polynomial.
\end{ex}

\begin{proof}
  Let \(f, g \in C(\R / \Z ; \C)\) such that
  \begin{align*}
     & \exists N \in \N : \big((c_n)_{n = -N}^N \text{ is in } \C\big) \land \bigg(f = \sum_{n = -N}^N c_n e_n\bigg); \\
     & \exists M \in \N : \big((d_n)_{n = -M}^M \text{ is in } \C\big) \land \bigg(g = \sum_{n = -M}^M d_n e_n\bigg).
  \end{align*}
  Without the loss of generality suppose that \(N \leq M\).
  Then we have
  \begin{align*}
    f + g & = \sum_{n = -N}^N (c_n e_n) + \sum_{n = -M}^M (d_n e_n) \\
          & = \sum_{n = -M}^M (a_n e_n)
  \end{align*}
  where
  \[
    a_n = \begin{dcases}
      c_n + d_n & \text{if } -N \leq n \leq N                     \\
      d_n       & \text{if } (-M \leq n < -N) \lor (N < n \leq M)
    \end{dcases}
  \]
  For \(fg\), we induct on \(M\) to show that \(fg\) is trigonometric polynomial.
  For \(M = 0\), we have
  \begin{align*}
    fg & = \bigg(\sum_{n = -N}^N (c_n e_n)\bigg) (d_0 e^0)                       \\
       & = \bigg(\sum_{n = -N}^N (c_n e_n)\bigg) d_0       &  & \by{ii:4.5.2}[e] \\
       & = \sum_{n = -N}^N (c_n d_0 e_n).
  \end{align*}
  Clearly, \(fg\) is trigonometric polynomial and Thus, the base case holds.
  Suppose inductively that \(fg\) is trigonometric polynomial for some \(M \geq 0\).
  Then for \(M + 1\), we have
  \begin{align*}
    f g & = \bigg(\sum_{n = -N}^N (c_n e_n)\bigg) \bigg(\sum_{m = -(M + 1)}^{M + 1} (d_m e_m)\bigg)                                                        \\
        & = \bigg(\sum_{n = -N}^N (c_n e_n)\bigg) \bigg(\sum_{m = -M}^M (d_m e_m) + d_{-M - 1} e_{-M - 1} + d_{M + 1} e_{M + 1}\bigg)                      \\
        & = \bigg(\sum_{n = -N}^N (c_n e_n)\bigg) \bigg(\sum_{m = -M}^M (d_m e_m)\bigg) + \bigg(\sum_{n = -N}^N (c_n e_n)\bigg) (d_{-M - 1} e_{-M - 1})    \\
        & \quad + \bigg(\sum_{n = -N}^N (c_n e_n)\bigg) (d_{M + 1} e_{M + 1})                                                                              \\
        & = \bigg(\sum_{n = -N}^N (c_n e_n)\bigg) \bigg(\sum_{m = -M}^M (d_m e_m)\bigg) + \sum_{n = -N}^N (c_n d_{-M - 1} e_{n - M - 1})                   \\
        & \quad + \sum_{n = -N}^N (c_n d_{M + 1} e_{n + M + 1})                                                                                            \\
        & = \bigg(\sum_{n = -N}^N (c_n e_n)\bigg) \bigg(\sum_{m = -M}^M (d_m e_m)\bigg) + \sum_{n = -N - M - 1}^{N - M - 1} (c_{n + M + 1} d_{-M - 1} e_n) \\
        & \quad + \sum_{n = -N + M + 1}^{N + M + 1} (c_{n - M - 1} d_{M + 1} e_n).
  \end{align*}
  By setting
  \begin{align*}
     & a_n = \begin{dcases}
               c_{n + M + 1} d_{-M - 1} & \text{if } -N - M - 1 \leq n \leq N - M - 1 \\
               0                        & \text{if } N - M - 1 < n \leq N + M + 1
             \end{dcases} \\
     & b_n = \begin{dcases}
               c_{n - M - 1} d_{M + 1} & \text{if } -N + M + 1 \leq n \leq N + M + 1 \\
               0                       & \text{if } -N - M - 1 \leq n < -N + M + 1
             \end{dcases}
  \end{align*}
  we have
  \[
    fg = \bigg(\sum_{n = -N}^N (c_n e_n)\bigg) \bigg(\sum_{m = -M}^M (d_m e_m)\bigg) + \sum_{n = -N - M - 1}^{N + M + 1} (a_n e_n) + \sum_{n = -N - M - 1}^{N + M + 1} (b_n e_n).
  \]
  By the induction hypothesis we know that \(\bigg(\sum_{n = -N}^N (c_n e_n)\bigg) \bigg(\sum_{m = -M}^M (d_m e_m)\bigg)\) is trigonometric polynomial.
  Thus, from the proof above we know that \(fg\) is trigonometric polynomial, and this closes the induction.
\end{proof}

\begin{ex}\label{ii:ex:5.3.2}
  Prove \cref{ii:5.3.5}.
\end{ex}

\begin{proof}
  See \cref{ii:5.3.5}.
\end{proof}

\begin{ex}\label{ii:ex:5.3.3}
  Prove \cref{ii:5.3.6}.
\end{ex}

\begin{proof}
  See \cref{ii:5.3.6}.
\end{proof}

\section{Periodic convolutions}\label{sec:5.4}

\begin{thm}\label{5.4.1}
  Let \(f \in C(\R / \Z ; \C)\), and let \(\varepsilon > 0\).
  Then there exists a trigonometric polynomial \(P\) such that \(\norm*{f - P}_{\infty} \leq \varepsilon\).
\end{thm}

\begin{proof}
  Let \(f\) be any element of \(C(\R / \Z ; \C)\);
  we know that \(f\) is bounded (by \cref{5.1.5}(a)), so that we have some \(M > 0\) such that \(\abs{f(x)} \leq M\) for all \(x \in \R\).

  Let \(\varepsilon > 0\) be arbitrary.
  Since \(f\) is uniformly continuous (by \cref{2.3.5}), there exists a \(\delta > 0\) such that \(\abs{f(x) - f(y)} \leq \varepsilon\) whenever \(\abs{x - y} \leq \delta\).
  Now use \cref{5.4.6} to find a trigonometric polynomial \(P\) which is a \((\varepsilon, \delta)\) approximation to the identity.
  Then \(f * P\) is also a trigonometric polynomial (by \cref{ac:5.4.1}).
  We now estimate \(\norm*{f - f * P}_{\infty}\).

  Let \(x\) be any real number.
  We have
  \begin{align*}
     & \abs{f(x) - f * P(x)}                                                                                         \\
     & = \abs{f(x) - P * f(x)}                                                   &  & \text{(by \cref{5.4.4}(a)(b))} \\
     & = \abs{f(x) - \int_{[0, 1]} f(x - y) P(y) \; dy}                          &  & \by{5.4.2}                     \\
     & = \abs{\int_{[0, 1]} f(x) P(y) \; dy - \int_{[0, 1]} f(x - y) P(y) \; dy} &  & \text{(by \cref{5.4.5}(a))}    \\
     & = \abs{\int_{[0, 1]} \big(f(x) - f(x - y)\big) P(y) \; dy}                                                    \\
     & \leq \int_{[0, 1]} \abs{f(x) - f(x - y)} P(y) \; dy.                      &  & \by{5.2.2}
  \end{align*}
  The right-hand side can be split as
  \begin{align*}
    \int_{[0, \delta]} \abs{f(x) - f(x - y)} P(y) \; dy & + \int_{[\delta, 1 - \delta]} \abs{f(x) - f(x - y)} P(y) \; dy \\
                                                        & + \int_{[1 - \delta, 1]} \abs{f(x) - f(x - y)} P(y) \; dy
  \end{align*}
  which we can bound from above by
  \begin{align*}
     & \leq \int_{[0, \delta]} \varepsilon P(y) \; dy + \int_{[\delta, 1 - \delta]} 2 M \varepsilon \; dy + \int_{[1 - \delta, 1]} \abs{f(x - 1) - f(x - y)} P(y) \; dy \\
     & \leq \int_{[0, \delta]} \varepsilon P(y) \; dy + \int_{[\delta, 1 - \delta]} 2 M \varepsilon \; dy + \int_{[1 - \delta, 1]} \varepsilon P(y) \; dy               \\
     & \leq \varepsilon + 2 M \varepsilon + \varepsilon                                                                                                                 \\
     & = (2M + 2) \varepsilon.
  \end{align*}
  Thus we have \(\norm*{f - f * P}_{\infty} \leq (2M + 2) \varepsilon\).
  Since \(M\) is fixed and \(\varepsilon\) is arbitrary, we can thus make \(f * P\) arbitrarily close to \(f\) in sup norm, which proves the periodic Weierstrass approximation theorem.
\end{proof}

\begin{note}
  \cref{5.4.1} asserts that any continuous periodic function can be uniformly approximated by trigonometric polynomials.
  To put it another way, if we let
  \[
    P(\R / \Z ; \C)
  \]
  denote the space of all trigonometric polynomials, then the closure of \(P(\R / \Z ; \C)\) in the \(L^\infty\) metric is \(C(\R / \Z ; \C)\).
\end{note}

\begin{note}
  It is possible to prove this theorem directly from the Weierstrass approximation theorem for polynomials (\cref{3.8.3}), and both theorems are a special case of a much more general theorem known as the \emph{Stone-Weierstrass theorem}, which we will not discuss here.
  However we shall instead prove this theorem from scratch, in order to introduce a couple of interesting notions, notably that of periodic convolution.
  The proof here, though, should strongly remind you of the arguments used to prove \cref{3.8.3}.
\end{note}

\begin{defn}[Periodic convolution]\label{5.4.2}
  Let \(f, g \in C(\R / \Z ; \C)\).
  Then we define the periodic convolution \(f * g : \R \to \C\) of \(f\) and \(g\) by the formula
  \[
    f * g(x) \coloneqq \int_{[0, 1]} f(y) g(x - y) \; dy
  \]
\end{defn}

\begin{rmk}\label{5.4.3}
  Note that \cref{5.4.2} is slightly different from the convolution for compactly supported functions defined in \cref{3.8.9}, because we are only integrating over \([0, 1]\) and not on all of \(\R\).
  Thus, in principle we have given the symbol \(f * g\) two conflicting meanings.
  However, in practice there will be no confusion, because it is not possible for a non-zero function to both be periodic and compactly supported.
\end{rmk}

\begin{lem}[Basic properties of periodic convolution]\label{5.4.4}
  Let \(f, g, h \in C(\R / \Z ; \C)\).
  \begin{enumerate}
    \item (Closure)
          The convolution \(f * g\) is continuous and \(\Z\)-periodic.
          In other words, \(f * g \in C(\R / \Z ; \C)\).
    \item (Commutativity)
          We have \(f * g = g * f\).
    \item (Bilinearity)
          We have \(f * (g + h) = f * g + f * h\) and \((f + g) * h = f * h + g * h\).
          For any complex number \(c\), we have \(c(f * g) = (cf) * g = f * (cg)\).
  \end{enumerate}
\end{lem}

\begin{proof}{(a)}
  By \cref{5.2.2} we know that \(f * g\) is continuous on \(\R\).
  Since
  \begin{align*}
    \forall x \in \R, f * g(x + 1) & = \int_{[0, 1]} f(y) g(x + 1 - y) \; dy &                         & \by{5.4.2} \\
                                   & = \int_{[0, 1]} f(y) g(x - y) \; dy     & (g \in C(\R / \Z ; \C))              \\
                                   & = f * g(x),                             &                         & \by{5.4.2}
  \end{align*}
  we know that \(f * g \in C(\R / \Z ; \C)\).
\end{proof}

\begin{proof}{(b)}
  Let \(x \in \R\) and let \(\phi : [x - 1, x] \mapsto [0, 1]\) be the function \(\phi(y) = x - y\).
  Then we have
  \begin{align*}
     & f * g(x)                                                                                                                      \\
     & = \int_{[0, 1]} f(y) g(x - y) \; dy                                           &  & \by{5.4.2}                                 \\
     & = \int_{\big[\phi(x), \phi(x - 1)\big]} f(y) g(x - y) \; dy                                                                   \\
     & = -\int_{[x - 1, x]} f\big(\phi(y)\big) g\big(x - \phi(y)\big) \phi'(y) \; dy &  & \text{(by Exercise 11.10.4 in Analysis I)} \\
     & = \int_{[x - 1, x]} f(x - y) g(y) \; dy                                                                                       \\
     & = \int_{[x - 1, x]} g(y) f(x - y) \; dy.                                                                                      \\
  \end{align*}
  Let \([x]\) be the integer defined in \cref{ex:5.1.1}.
  Then we have
  \begin{align*}
             & [x] \leq x < [x] + 1     \\
    \implies & [x] - 1 \leq x - 1 < [x]
  \end{align*}
  and
  \begin{align*}
     & f * g(x)                                                                                                                                  \\
     & = \int_{[x - 1, x]} g(y) f(x - y) \; dy                                                                                                   \\
     & = \int_{\big[x - 1, [x]\big]} g(y) f(x - y) \; dy + \int_{\big[[x], x\big]} g(y) f(x - y) \; dy                                           \\
     & = \int_{\big[[x], x\big]} g(y) f(x - y) \; dy + \int_{\big[x - 1, [x]\big]} g(y) f(x - y) \; dy                                           \\
     & = \int_{\big[[x], x\big]} g(y) f(x - y) \; dy                                                                                             \\
     & \quad + \int_{\big[x - 1 + 1, [x] + 1\big]} g(y - 1) f(x - y - 1) \; dy                                                                   \\
     & = \int_{\big[[x], x\big]} g(y) f(x - y) \; dy + \int_{\big[x, [x] + 1\big]} g(y) f(x - y) \; dy & (f, g \in C(\R / \Z ; \C))              \\
     & = \int_{\big[[x], [x] + 1\big]} g(y) f(x - y) \; dy                                                                                       \\
     & = \int_{\big[[x] - [x], [x] + 1 - [x]\big]} g(y + [x]) f(x - y + [x]) \; dy                                                               \\
     & = \int_{[0, 1]} g(y) f(x - y) \; dy                                                             & (f, g \in C(\R / \Z ; \C))              \\
     & = g * f(x).                                                                                     &                            & \by{5.4.2}
  \end{align*}
  Since \(x\) is arbitrary, we conclude that \(f * g = g * f\).
\end{proof}

\begin{proof}{(c)}
  By \cref{5.1.5}(b) we know that \(f + g, g + h, cf, cg \in C(\R / \Z ; \C)\).
  Thus \(f * (g + h), (f + g) * h, (cf) * g, f * (cg)\) are well-defined.
  Let \(x \in \R\).
  Then we have
  \begin{align*}
     & \big(f * (g + h)\big)(x)                                                                                              \\
     & = \int_{[0, 1]} f(y) \cdot (g + h)(x - y) \; dy                         &  & \by{5.4.2}                               \\
     & = \int_{[0, 1]} f(y) \cdot \big(g(x - y) + h(x - y)\big) \; dy                                                        \\
     & = \int_{[0, 1]} f(y) g(x - y) + f(y) h(x - y) \; dy                                                                   \\
     & = \int_{[0, 1]} f(y) g(x - y) \; dy + \int_{[0, 1]} f(y) h(x - y) \; dy &  & \text{(cf the proof of \cref{5.2.5}(c))} \\
     & = (f * g)(x) + (f * h)(x)                                               &  & \by{5.4.2}                               \\
     & = (f * g + f * h)(x)
  \end{align*}
  and
  \begin{align*}
    \big((cf) * g\big)(x) & = \int_{[0, 1]} (cf)(y) \cdot g(x - y) \; dy &  & \by{5.4.2}                               \\
                          & = \int_{[0, 1]} c f(y) g(x - y) \; dy                                                      \\
                          & = c \int_{[0, 1]} f(y) g(x - y) \; dy        &  & \text{(cf the proof of \cref{5.2.5}(c))} \\
                          & = c (f * g)(x).                              &  & \by{5.4.2}
  \end{align*}
  Since \(x\) is arbitrary, we conclude that \(f * (g + h) = f * g + f * h\) and \((cf) * g = c (f * g)\).
  This implies
  \begin{align*}
    (f + g) * h & = h * (f + g)   &  & \text{(by \cref{5.4.4}(b))}   \\
                & = h * f + h * g &  & \text{(from the proof above)} \\
                & = f * h + g * h &  & \text{(by \cref{5.4.4}(b))}
  \end{align*}
  and
  \begin{align*}
    f * (cg) & = (cg) * f  &  & \text{(by \cref{5.4.4}(b))}   \\
             & = c(g * f)  &  & \text{(from the proof above)} \\
             & = c(f * g). &  & \text{(by \cref{5.4.4}(b))}
  \end{align*}
\end{proof}

\begin{ac}\label{ac:5.4.1}
  Now we observe an interesting identity:
  for any \(f \in C(\R / \Z ; \C)\) and any integer \(n\), we have
  \[
    f * e_n = \hat{f}(n) e_n.
  \]
  To prove this, we compute
  \begin{align*}
    f * e_n(x) & = \int_{[0, 1]} f(y) e_n(x - y) \; dy                        &  & \by{5.4.2}                               \\
               & = \int_{[0, 1]} f(y) e^{2 \pi i n (x - y)} \; dy             &  & \by{5.3.1}                               \\
               & = e^{2 \pi i n x} \int_{[0, 1]} f(y) e^{- 2 \pi i n y} \; dy &  & \text{(cf the proof of \cref{5.2.5}(c))} \\
               & = \inner*{f, e_n} e^{2 \pi i n x}                            &  & \by{5.2.1}                               \\
               & = \hat{f}(n) e^{2 \pi i n x}                                 &  & \by{5.3.7}                               \\
               & = \hat{f}(n) e_n                                             &  & \by{5.3.1}
  \end{align*}
  as desired.
  More generally, we see from \cref{5.4.4}(c) that for any trigonometric polynomial \(P = \sum_{n = -N}^N c_n e_n\), we have
  \[
    f * P = \sum_{n = -N}^N c_n (f * e_n) = \sum_{n = -N}^N \hat{f}(n) c_n e_n.
  \]
  Thus the periodic convolution of any function in \(C(\R / \Z ; \C)\) with a trigonometric polynomial, is again a trigonometric polynomial.
  (Compare with \cref{3.8.13}.)
\end{ac}

\begin{defn}[Periodic approximation to the identity]\label{5.4.5}
  Let \(\varepsilon > 0\) and \(0 < \delta < 1 / 2\).
  A function \(f \in C(\R / \Z ; \C)\) is said to be a \emph{periodic \((\varepsilon, \delta)\) approximation to the identity} if the following properties are true:
  \begin{enumerate}
    \item \(f(x) \geq 0\) for all \(x \in \R\), and \(\int_{[0, 1]} f(x) \; dx = 1\).
    \item We have \(f(x) < \varepsilon\) for all \(\delta \leq \abs{x} \leq 1 - \delta\).
  \end{enumerate}
\end{defn}

\begin{ac}[Fejér kernel]\label{ac:5.4.2}
  Let \(N \geq 1\) be an integer.
  Then we have
  \[
    \sum_{n = -N}^N \bigg(1 - \dfrac{\abs{n}}{N}\bigg) e_n = \dfrac{1}{N} \abs{\sum_{n = 0}^{N - 1} e_n}^2.
  \]
\end{ac}

\begin{proof}
  First we claim that
  \[
    \sum_{n = 0}^{2N - 2} (N - \abs{n - N + 1}) \cdot e_n = \bigg(\sum_{n = 0}^{N - 1} e_n\bigg)^2.
  \]
  We proof the claim by induction on \(N\) and we start with \(N = 1\).
  For \(N = 1\), we have
  \begin{align*}
    \sum_{n = 0}^0 (1 - \abs{n - 1 + 1}) \cdot e_n & = \sum_{n = 0}^0 (1 - \abs{n}) e_n                                   \\
                                                   & = 1 e_0                                                              \\
                                                   & = 1                                 &  & \text{(by \cref{4.5.2}(e))} \\
                                                   & = e_0^2                             &  & \text{(by \cref{4.5.2}(e))} \\
                                                   & = \bigg(\sum_{n = 0}^0 e_n\bigg)^2.
  \end{align*}
  Thus the base case holds.
  Suppose inductively that the claim is true for some \(N \geq 1\).
  Then for \(N + 1\), we want to show that
  \[
    \sum_{n = 0}^{2(N + 1) - 2} \big(N + 1 - \abs{n - (N + 1) + 1}\big) \cdot e_n = \sum_{n = 0}^{2N} (N + 1 - \abs{n - N}) \cdot e_n = \bigg(\sum_{n = 0}^N e_n\bigg)^2.
  \]
  This is true since
  \begin{align*}
     & \bigg(\sum_{n = 0}^N e_n\bigg)^2                                                                                                               \\
     & = \Bigg(\bigg(\sum_{n = 0}^{N - 1} e_n\bigg) + e_N\Bigg)^2                                                                     &  & \by{4.6.4} \\
     & = \bigg(\sum_{n = 0}^{N - 1} e_n\bigg)^2 + 2 e_N \bigg(\sum_{n = 0}^{N - 1} e_n\bigg) + e_N^2                                  &  & \by{4.6.6} \\
     & = \sum_{n = 0}^{2N - 2} (N - \abs{n - N + 1}) \cdot e_n + 2 e_N \bigg(\sum_{n = 0}^{N - 1} e_n\bigg) + e_N^2                   &  & \byIH      \\
     & = \sum_{n = 0}^{N - 1} (N - \abs{n - N + 1}) \cdot e_n + \sum_{n = N}^{2N - 2} (N - \abs{n - N + 1}) \cdot e_n                 &  & \by{4.6.4} \\
     & \quad + 2 e_N \bigg(\sum_{n = 0}^{N - 1} e_n\bigg) + e_N^2                                                                                     \\
     & = \sum_{n = 0}^{N - 1} (n + 1) \cdot e_n + \sum_{n = N}^{2N - 2} (2N - n - 1) \cdot e_n                                                        \\
     & \quad + 2 e_N \bigg(\sum_{n = 0}^{N - 1} e_n\bigg) + e_N^2                                                                                     \\
     & = \sum_{n = 0}^{N - 1} n e_n + \sum_{n = 0}^{N - 1} e_n + \sum_{n = N}^{2N - 2} (2N - n) \cdot e_n - \sum_{n = N}^{2N - 2} e_n &  & \by{4.6.6} \\
     & \quad + 2 \bigg(\sum_{n = 0}^{N - 1} e_N e_n\bigg) + e_N^2                                                                     &  & \by{4.6.6} \\
     & = \sum_{n = 0}^{N - 1} n e_n + \sum_{n = N}^{2N - 2} (2N - n) \cdot e_n                                                        &  & \by{4.6.4} \\
     & \quad + \sum_{n = 0}^{N - 1} e_n - \sum_{n = N}^{2N - 2} e_n + 2 \bigg(\sum_{n = 0}^{N - 1} e_N e_n\bigg) + e_N^2                              \\
     & = \sum_{n = 0}^{N - 1} (N - \abs{n - N}) \cdot e_n + \sum_{n = N}^{2N - 2} (N - \abs{n - N}) \cdot e_n                                         \\
     & \quad + \sum_{n = 0}^{N - 1} e_n - \sum_{n = N}^{2N - 2} e_n + 2 \bigg(\sum_{n = 0}^{N - 1} e_N e_n\bigg) + e_N^2
  \end{align*}
  \begin{align*}
     & = \sum_{n = 0}^{2N - 2} (N - \abs{n - N}) \cdot e_n                                                               &  & \text{(conti. from above)} \\
     & \quad + \sum_{n = 0}^{N - 1} e_n - \sum_{n = N}^{2N - 2} e_n + 2 \bigg(\sum_{n = 0}^{N - 1} e_N e_n\bigg) + e_N^2                                 \\
     & = \sum_{n = 0}^{2N - 2} (N + 1 - \abs{n - N}) \cdot e_n - \sum_{n = 0}^{2N - 2} e_n                               &  & \by{4.6.6}                 \\
     & \quad + \sum_{n = 0}^{N - 1} e_n - \sum_{n = N}^{2N - 2} e_n + 2 \bigg(\sum_{n = 0}^{N - 1} e_N e_n\bigg) + e_N^2                                 \\
     & = \sum_{n = 0}^{2N - 2} (N + 1 - \abs{n - N}) \cdot e_n                                                           &  & \by{4.6.6}                 \\
     & \quad - 2 \bigg(\sum_{n = N}^{2N - 2} e_n\bigg) + 2 \bigg(\sum_{n = 0}^{N - 1} e_N e_n\bigg) + e_N^2                                              \\
     & = \sum_{n = 0}^{2N} (N + 1 - \abs{n - N}) \cdot e_n - 2 e_{2N - 1} - e_{2N}                                       &  & \by{4.6.6}                 \\
     & \quad - 2 \bigg(\sum_{n = N}^{2N - 2} e_n\bigg) + 2 \bigg(\sum_{n = 0}^{N - 1} e_N e_n\bigg) + e_N^2                                              \\
     & = \sum_{n = 0}^{2N} (N + 1 - \abs{n - N}) \cdot e_n - 2 e_{2N - 1} - e_{2N}                                                                       \\
     & \quad - 2 \bigg(\sum_{n = N}^{2N - 2} e_n\bigg) + 2 \bigg(\sum_{n = 0}^{N - 1} e_{n + N}\bigg) + e_{2N}           &  & \by{ex:4.6.16}             \\
     & = \sum_{n = 0}^{2N} (N + 1 - \abs{n - N}) \cdot e_n                                                               &  & \by{4.6.4}                 \\
     & \quad - 2 \bigg(\sum_{n = N}^{2N - 1} e_n\bigg) + 2 \bigg(\sum_{n = 0}^{N - 1} e_{n + N}\bigg)                                                    \\
     & = \sum_{n = 0}^{2N} (N + 1 - \abs{n - N}) \cdot e_n                                                                                               \\
     & \quad - 2 \bigg(\sum_{n = N}^{2N - 1} e_n\bigg) + 2 \bigg(\sum_{n = N}^{2N - 1} e_n\bigg)                                                         \\
     & = \sum_{n = 0}^{2N} (N + 1 - \abs{n - N}) \cdot e_n.
  \end{align*}
  This closes the induction.

  Using the claim above we have
  \begin{align*}
     & \sum_{n = -N}^N \bigg(1 - \dfrac{\abs{n}}{N}\bigg) e_n                                                                                 \\
     & = \dfrac{1}{N} \sum_{n = -N}^N (N - \abs{n}) e_n                                                    &  & \by{4.6.6}                    \\
     & = \dfrac{1}{N} \sum_{n = -(N - 1)}^{N - 1} (N - \abs{n}) \cdot e_n                                                                     \\
     & = \dfrac{1}{N} \sum_{n = 0}^{2N - 2} (N - \abs{n - N + 1}) \cdot e_{n - N + 1}                                                         \\
     & = \dfrac{1}{N} \sum_{n = 0}^{2N - 2} (N - \abs{n - N + 1}) \cdot e_n \cdot e_{-N + 1}               &  & \by{ex:4.6.16}                \\
     & = \dfrac{e_{-N + 1}}{N} \sum_{n = 0}^{2N - 2} (N - \abs{n - N + 1}) \cdot e_n                       &  & \by{4.6.6}                    \\
     & = \dfrac{e_{-N + 1}}{N} \bigg(\sum_{n = 0}^{N - 1} e_n\bigg)^2                                      &  & \text{(from the claim above)} \\
     & = \dfrac{e_{-N + 1}}{N} \bigg(\sum_{n = 0}^{N - 1} e_n\bigg) \bigg(\sum_{n = 0}^{N - 1} e_n\bigg)                                      \\
     & = \dfrac{1}{N} \bigg(\sum_{n = 0}^{N - 1} e_n\bigg) \bigg(\sum_{n = 0}^{N - 1} e_{-N + 1} e_n\bigg) &  & \by{4.6.6}                    \\
     & = \dfrac{1}{N} \bigg(\sum_{n = 0}^{N - 1} e_n\bigg) \bigg(\sum_{n = 0}^{N - 1} e_{n - N + 1}\bigg)  &  & \by{ex:4.6.16}                \\
     & = \dfrac{1}{N} \bigg(\sum_{n = 0}^{N - 1} e_n\bigg) \bigg(\sum_{n = 0}^{N - 1} e_{-n}\bigg)         &  & \by{4.6.4}                    \\
     & = \dfrac{1}{N} \bigg(\sum_{n = 0}^{N - 1} e_n\bigg) \bigg(\sum_{n = 0}^{N - 1} \overline{e_n}\bigg) &  & \by{4.6.15}                   \\
     & = \dfrac{1}{N} \bigg(\sum_{n = 0}^{N - 1} e_n\bigg) \bigg(\overline{\sum_{n = 0}^{N - 1} e_n}\bigg) &  & \by{4.6.9}                    \\
     & = \dfrac{1}{N} \abs{\sum_{n = 0}^{N - 1} e_n}^2.                                                    &  & \by{4.6.11}
  \end{align*}
\end{proof}

\begin{lem}\label{5.4.6}
  For every \(\varepsilon > 0\) and \(0 < \delta < 1 / 2\), there exists a trigonometric polynomial \(P\) which is an \((\varepsilon, \delta)\) approximation to the identity.
\end{lem}

\begin{proof}
  Let \(N \geq 1\) be an integer.
  We define the \emph{Fejér kernel} \(F_N\) to be the function
  \[
    F_N = \sum_{n = -N}^N \bigg(1 - \dfrac{\abs{n}}{N}\bigg) e_n.
  \]
  Clearly \(F_N\) is a trigonometric polynomial.
  We observe the identity
  \[
    F_N = \dfrac{1}{N} \abs{\sum_{n = 0}^{N - 1} e_n}^2
  \]
  by \cref{ac:5.4.2}.
  But from the geometric series formula (Lemma 7.3.3 in Analysis I) we have
  \begin{align*}
    \sum_{n = 0}^{N - 1} e_n(x) & = \sum_{n = 0}^{N - 1} \big(e_1(x)\big)^n                                                                                &  & \by{ex:4.6.16}              \\
                                & = \dfrac{\big(e_1(x)\big)^N - 1}{e_1(x) - 1}                                                                             &  & \text{(geometric series)}   \\
                                & = \dfrac{\big(e_1(x)\big)^N - e_0(x)}{e_1(x) - e_0(x)}                                                                   &  & \text{(by \cref{4.5.2}(e))} \\
                                & = \dfrac{e_N(x) - e_0(x)}{e_1(x) - e_0(x)}                                                                               &  & \by{ex:4.6.16}              \\
                                & = \dfrac{e^{2 \pi i N x} - e^0}{e^{2 \pi i x} - e^0}                                                                     &  & \by{5.3.1}                  \\
                                & = \dfrac{e^{\pi i N x} e^{\pi i N x} - e^{\pi i N x} e^{-\pi i N x}}{e^{\pi i x} e^{\pi i x} - e^{\pi i x} e^{-\pi i x}} &  & \by{ex:4.6.16}              \\
                                & = \dfrac{e^{\pi i N x} (e^{\pi i N x} - e^{-\pi i N x})}{e^{\pi i x} (e^{\pi i x} - e^{-\pi i x})}                       &  & \by{4.6.6}                  \\
                                & = \dfrac{e^{\pi i (N - 1) x} (e^{\pi i N x} - e^{-\pi i N x})}{e^{\pi i x} - e^{-\pi i x}}                               &  & \by{4.6.12}                 \\
                                & = \dfrac{2i e^{\pi i (N - 1) x} \sin(\pi N x)}{2i \sin(\pi x)}                                                           &  & \by{4.7.1}                  \\
                                & = \dfrac{e^{\pi i (N - 1) x} \sin(\pi N x)}{\sin(\pi x)}                                                                 &  & \by{4.6.12}
  \end{align*}
  when \(x\) is not an integer, and hence we have the formula
  \begin{align*}
    F_N(x) & = \dfrac{1}{N} \abs{\sum_{n = 0}^{N - 1} e_n(x)}^2                                                       &            & \by{ac:5.4.2}                 \\
           & = \dfrac{1}{N} \abs{\dfrac{e^{\pi i (N - 1) x} \sin(\pi N x)}{\sin(\pi x)}}^2                            &            & \text{(from the proof above)} \\
           & = \dfrac{\abs{e^{\pi i (N - 1) x}}^2 \abs{\sin(\pi N x)}^2}{N \abs{\sin(\pi x)}^2}                       &            & \by{ex:4.6.7}                 \\
           & = \dfrac{\abs{e^{\pi i (N - 1) x}}^2 \big(\sin(\pi N x)\big)^2}{N \big(\sin(\pi x)\big)^2}               & (x \in \R)                                 \\
           & = \dfrac{e^{\pi i (N - 1) x} e^{- \pi i (N - 1) x} \big(\sin(\pi N x)\big)^2}{N \big(\sin(\pi x)\big)^2} &            & \by{4.6.11}                   \\
           & = \dfrac{e^0 \big(\sin(\pi N x)\big)^2}{N \big(\sin(\pi x)\big)^2}                                       &            & \by{ex:4.6.16}                \\
           & = \dfrac{\big(\sin(\pi N x)\big)^2}{N \big(\sin(\pi x)\big)^2}.                                          &            & \text{(by \cref{4.5.2}(e))}
  \end{align*}
  When \(x\) is an integer, the geometric series formula does not apply, but one has \(F_N(x) = N\) in that case, as one can see by direct computation.
  In either case we see that \(F_N(x) \geq 0\) for any \(x\).
  Also, we have
  \begin{align*}
     & \int_{[0, 1]} F_N(x) \; dx                                                                                                                 \\
     & = \int_{[0, 1]} \sum_{n = -N}^N \bigg(1 - \dfrac{\abs{n}}{N}\bigg) e_n(x) \; dx                                                            \\
     & = \sum_{n = -N}^N \Bigg(\bigg(1 - \dfrac{\abs{n}}{N}\bigg) \int_{[0, 1]} e_n(x) \; dx\Bigg)                            &  & \by{5.2.2}     \\
     & = \sum_{n = -N}^N \Bigg(\bigg(1 - \dfrac{\abs{n}}{N}\bigg) \int_{[0, 1]} e_{n - 1}(x) e_1(x) \; dx\Bigg)               &  & \by{ex:4.6.16} \\
     & = \sum_{n = -N}^N \Bigg(\bigg(1 - \dfrac{\abs{n}}{N}\bigg) \int_{[0, 1]} e_{n - 1}(x) \overline{e_{-1}(x)} \; dx\Bigg) &  & \by{4.6.15}    \\
     & = \sum_{n = -N}^N \Bigg(\bigg(1 - \dfrac{\abs{n}}{N}\bigg) \inner*{e_{n - 1}, e_{-1}}\Bigg)                            &  & \by{5.2.1}     \\
     & = \bigg(1 - \dfrac{\abs{0}}{N}\bigg) 1                                                                                 &  & \by{5.3.5}     \\
     & = 1.
  \end{align*}
  Finally, since \(\sin(\pi N x) \leq 1\), we have
  \[
    F_N(x) \leq \dfrac{1}{N \big(\sin(\pi x)\big)^2} \leq \dfrac{1}{N \big(\sin(\pi \delta)\big)^2}
  \]
  whenever \(\delta < \abs{x} < 1 - \delta\)
  (this is because \(\sin\) is increasing on \([0, \pi / 2]\) and decreasing on \([\pi / 2, \pi]\)).
  Thus by choosing \(N\) large enough, we can make \(F_N (x) \leq \varepsilon\) for all \(\delta < \abs{x} < 1 - \delta\).
  Note that since
  \begin{align*}
    \big(\sin(\pi \abs{x})\big)^2 & = \begin{dcases}
                                        \big(\sin(\pi x)\big)^2  & \text{if } x \geq 0 \\
                                        \big(\sin(\pi -x)\big)^2 & \text{if } x < 0
                                      \end{dcases} \\
                                  & = \begin{dcases}
                                        \big(\sin(\pi x)\big)^2  & \text{if } x \geq 0 \\
                                        \big(-\sin(\pi x)\big)^2 & \text{if } x < 0
                                      \end{dcases} &  & \text{(by \cref{4.7.2}(c))} \\
                                  & = \big(\sin(\pi x)\big)^2
  \end{align*}
  and
  \begin{align*}
             & \begin{dcases}
                 \delta < \abs{x} \leq \dfrac{1}{2}  & \text{if } \abs{x} \leq \dfrac{1}{2} \\
                 \dfrac{1}{2} < \abs{x} < 1 - \delta & \text{if } \abs{x} > \dfrac{1}{2}    \\
               \end{dcases}                               \\
    \implies & \begin{dcases}
                 \pi \delta < \pi \abs{x} \leq \dfrac{\pi}{2}    & \text{if } \abs{x} \leq \dfrac{1}{2} \\
                 \dfrac{\pi}{2} < \pi \abs{x} < \pi - \pi \delta & \text{if } \abs{x} > \dfrac{1}{2}    \\
               \end{dcases}                   \\
    \implies & \begin{dcases}
                 \sin(\pi \delta) < \sin(\pi \abs{x}) \leq \sin(\dfrac{\pi}{2})    & \text{if } \abs{x} \leq \dfrac{1}{2} \\
                 \sin(\dfrac{\pi}{2}) > \sin(\pi \abs{x}) > \sin(\pi - \pi \delta) & \text{if } \abs{x} > \dfrac{1}{2}    \\
               \end{dcases} \\
    \implies & \begin{dcases}
                 \sin(\pi \delta) < \sin(\pi \abs{x}) \leq \sin(\dfrac{\pi}{2}) & \text{if } \abs{x} \leq \dfrac{1}{2} \\
                 \sin(\dfrac{\pi}{2}) > \sin(\pi \abs{x}) > -\sin(-\pi \delta)  & \text{if } \abs{x} > \dfrac{1}{2}    \\
               \end{dcases}    &  & \text{(by \cref{4.7.5}(a))}    \\
    \implies & \begin{dcases}
                 \sin(\pi \delta) < \sin(\pi \abs{x}) \leq \sin(\dfrac{\pi}{2}) & \text{if } \abs{x} \leq \dfrac{1}{2} \\
                 \sin(\dfrac{\pi}{2}) > \sin(\pi \abs{x}) > \sin(\pi \delta)    & \text{if } \abs{x} > \dfrac{1}{2}    \\
               \end{dcases}    &  & \text{(by \cref{4.7.2}(c))}    \\
    \implies & \sin(\pi \delta) < \sin(\pi \abs{x}),
  \end{align*}
  we have
  \begin{align*}
             & 0 < \big(\sin(\pi \delta)\big)^2 < \big(\sin(\pi \abs{x})\big)^2                  &  & \text{(by \cref{ac:4.7.2}(d))} \\
    \implies & 0 < \big(\sin(\pi \delta)\big)^2 < \big(\sin(\pi x)\big)^2                        &  & \text{(from the proof above)}  \\
    \implies & 0 < \dfrac{1}{\big(\sin(\pi x)\big)^2} < \dfrac{1}{\big(\sin(\pi \delta)\big)^2}.
  \end{align*}
\end{proof}

\exercisesection

\begin{ex}\label{ex:5.4.1}
  Show that if \(f : \R \to \C\) is both compactly supported and \(\Z\)-periodic, then it is identically zero.
\end{ex}

\begin{proof}
  Since \(f\) is compactly supported, by \cref{3.8.4} we know that
  \[
    \exists L, U \in \R : \forall x \in \R \setminus [L, U], f(x) = 0.
  \]
  Fix such \(L, U\).
  Let \([U - L]\) be the integer defined in \cref{ex:5.1.1}.
  Then we have
  \begin{align*}
             & \forall x \in [L, U], L \leq x                                                                 \\
    \implies & L + [U - L] \leq x + [U - L]                                                                   \\
    \implies & U = L + U - L < L + [U - L] + 1 \leq x + [U - L] + 1 &                         & \by{ex:5.1.1} \\
    \implies & f(x) = f(x + [U - L] + 1) = 0.                       & (f \in C(\R / \Z ; \C))
  \end{align*}
  Thus \(f = 0\).
\end{proof}

\begin{ex}\label{ex:5.4.2}
  Prove \cref{5.4.4}.
\end{ex}

\begin{proof}
  See \cref{5.4.4}.
\end{proof}

\begin{ex}\label{ex:5.4.3}
  Fill in the gaps marked in \cref{5.4.6}.
\end{ex}

\begin{proof}
  See \cref{5.4.6}.
\end{proof}
\section{The Fourier and Plancherel theorems}\label{ii:sec:5.5}

\begin{thm}[Fourier theorem]\label{ii:5.5.1}
  For any \(f \in C(\R / \Z ; \C)\), the series \(\sum_{n = -\infty}^\infty \hat{f}(n) e_n\) converges in \(L^2\) metric to \(f\).
  In other words, we have
  \[
    \lim_{N \to \infty} \norm*{f - \sum_{n = -N}^N \hat{f}(n) e_n}_2 = 0.
  \]
\end{thm}

\begin{proof}
  Let \(\varepsilon > 0\).
  We have to show that there exists an \(N_0\) such that
  \[
    \norm*{f - \sum_{n = -N}^N \hat{f}(n) e_n}_2 \leq \varepsilon
  \]
  for all \(N \geq N_0\).

  By the Weierstrass approximation theorem (\cref{ii:5.4.1}), we can find a trigonometric polynomial \(P = \sum_{n = -N_0}^{N_0} c_n e_n\) such that \(\norm*{f - P}_{\infty} \leq \varepsilon\), for some \(N_0 > 0\).
  In particular, we have \(\norm*{f - P}_2 \leq \varepsilon\) (\cref{ii:ex:5.2.3}).

  Now let \(N > N_0\), and let \(F_N \coloneqq \sum_{n = -N}^N \hat{f}(n) e_n\).
  We claim that \(\norm*{f - F_N}_2 \leq \varepsilon\).
  First, observe that for any \(\abs{m} \leq N\), we have
  \[
    \inner*{f - F_N, e_m} = \inner*{f, e_m} - \sum_{n = -N}^N \hat{f}(n) \inner*{e_n, e_m} = \hat{f}(m) - \hat{f}(m) = 0,
  \]
  where we have used \cref{ii:5.3.5} and \cref{ii:5.2.5}.
  In particular, we have
  \[
    \inner*{f - F_N, F_N - P} = 0
  \]
  since we can write \(F_N - P\) as a linear combination of the \(e_m\) for which \(\abs{m} \leq N\).
  By Pythagoras' theorem (\cref{ii:5.2.7}(d)) we therefore have
  \[
    \norm*{f - P}_2^2 = \norm*{f - F_N}_2^2 + \norm*{F_N - P}_2^2
  \]
  and in particular
  \[
    \norm*{f - F_N}_2 \leq \norm*{f - P}_2 \leq \varepsilon
  \]
  as desired.
\end{proof}

\begin{rmk}\label{ii:5.5.2}
  Note that we have only obtained convergence of the Fourier series \(\sum_{n = -\infty}^\infty \hat{f}(n) e_n\) to \(f\) in the \(L^2\) metric.
  One may ask whether one has convergence in the uniform or pointwise sense as well, but it turns out (perhaps somewhat surprisingly) that the answer is no to both of those questions.
  However, if one assumes that the function \(f\) is not only continuous, but is also differentiable, then one can recover pointwise convergence;
  if one assumes continuously differentiable, then one gets uniform convergence as well.
  These results are beyond the scope of this text and will not be proven here.
  However, we will prove one theorem about when one can improve the \(L^2\) convergence to uniform convergence.
\end{rmk}

\begin{thm}\label{ii:5.5.3}
  Let \(f \in C(\R / \Z ; \C)\), and suppose that the series \(\sum_{n = -\infty}^\infty \abs{\hat{f}(n)}\) is absolutely convergent.
  Then the series \(\sum_{n = -\infty}^\infty \hat{f}(n) e_n\) converges uniformly to \(f\).
  In other words, we have
  \[
    \lim_{N \to \infty} \norm*{f - \sum_{n = -N}^N \hat{f}(n) e_n}_{\infty} = 0.
  \]
\end{thm}

\begin{proof}
  By the Weierstrass \(M\)-test (\cref{ii:3.5.7}), we see that \(\sum_{n = -\infty}^\infty \hat{f}(n) e_n\) converges to some function \(F\), which by \cref{ii:5.1.5}(c) is also continuous and \(\Z\)-periodic.
  (Strictly speaking, the Weierstrass \(M\)-test was phrased for series from \(n = 1\) to \(n = +\infty\), but also works for series from \(n = -\infty\) to \(n = +\infty\);
  this can be seen by splitting the doubly infinite series into two pieces.)
  Thus
  \[
    \lim_{N \to \infty} \norm*{F - \sum_{n = -N}^N \hat{f}(n) e_n}_{\infty} = 0
  \]
  which implies that
  \[
    \lim_{N \to \infty} \norm*{F - \sum_{n = -N}^N \hat{f}(n) e_n}_2 = 0
  \]
  since the \(L^2\) norm is always less than or equal to the \(L^\infty\) norm (\cref{ii:ex:5.2.3}).
  But the sequence \(\sum_{n = -N}^N \hat{f}(n) e_n\) is already converging in \(L^2\) metric to \(f\) by the Fourier theorem (\cref{ii:5.5.1}), so can only converge in \(L^2\) metric to \(F\) if \(F = f\)
  (cf. \cref{ii:1.1.20}).
  Thus, \(F = f\), and so we have
  \[
    \lim_{N \to \infty} \norm*{f - \sum_{n = -N}^N \hat{f}(n) e_n}_{\infty} = 0
  \]
  as desired.
\end{proof}

\begin{thm}ncherel theorem]\label{ii:5.5.4}
  For any \(f \in C(\R / \Z ; \C)\), the series
  \[
    \sum_{n = -\infty}^\infty \abs{\hat{f}(n)}^2
  \]
  is absolutely convergent, and
  \[
    \norm*{f}_2^2 = \sum_{n = -\infty}^\infty \abs{\hat{f}(n)}^2.
  \]
\end{thm}

\begin{proof}
  Let \(\varepsilon > 0\).
  By the Fourier theorem (\cref{ii:5.5.1}) we know that
  \[
    \norm*{f - \sum_{n = -N}^N \hat{f}(n) e_n}_2 \leq \varepsilon
  \]
  if \(N\) is large enough (depending on \(\varepsilon\)).
  In particular, by the triangle inequality (\cref{ii:5.2.7}(c)(e)) this implies that
  \[
    \norm*{f}_2 - \varepsilon \leq \norm*{\sum_{n = -N}^N \hat{f}(n) e_n}_2 \leq \norm*{f}_2 + \varepsilon.
  \]
  On the other hand, by \cref{ii:5.3.6} we have
  \[
    \norm*{\sum_{n = -N}^N \hat{f}(n) e_n}_2 = \bigg(\sum_{n = -N}^N \abs{\hat{f}(n)}^2\bigg)^{1 / 2}
  \]
  and hence
  \[
    (\norm*{f}_2 - \varepsilon)^2 \leq \sum_{n = -N}^N \abs{\hat{f}(n)}^2 \leq (\norm*{f}_2 + \varepsilon)^2.
  \]
  Taking \(\limsup\), we obtain
  \[
    (\norm*{f}_2 - \varepsilon)^2 \leq \limsup_{N \to \infty} \sum_{n = -N}^N \abs{\hat{f}(n)}^2 \leq (\norm*{f}_2 + \varepsilon)^2.
  \]
  Since \(\varepsilon\) was arbitrary, we thus obtain by the squeeze test that
  \[
    \limsup_{N \to \infty} \sum_{n = -N}^N \abs{\hat{f}(n)}^2 = \norm*{f}_2^2
  \]
  and the claim follows.
\end{proof}

\begin{note}
  \cref{ii:5.5.4} is also known as \emph{Parseval's theorem}.
\end{note}

\exercisesection

\begin{ex}\label{ii:ex:5.5.1}
  Let \(f\) be a function in \(C(\R / \Z ; \C)\), and define the \emph{trigonometric Fourier coefficients} \(a_n, b_n\) for \(n = 0, 1, 2, 3, \dots\) by
  \[
    a_n = 2 \int_{[0, 1]} f(x) \cos(2 \pi n x) \; dx; \quad b_n = 2 \int_{[0, 1]} f(x) \sin(2 \pi n x) \; dx.
  \]
  \begin{enumerate}
    \item Show that the series
          \[
            \dfrac{1}{2} a_0 + \sum_{n = 1}^\infty \big(a_n \cos(2 \pi n x) + b_n \sin(2 \pi n x)\big)
          \]
          converges in \(L_2\) metric to \(f\).
    \item Show that if \(\sum_{n = 1}^\infty a_n\) and \(\sum_{n = 1}^\infty b_n\) are absolutely convergent, then the above series actually converges uniformly to \(f\), and not just in \(L_2\) metric.
  \end{enumerate}
\end{ex}

\begin{proof}{(a)}
  Observe that for all \(n \in \Z\), we have
  \begin{align*}
     & \hat{f}(n)                                                                                                                                \\
     & = \int_{[0, 1]} f(x) e^{- 2 \pi i n x} \; dx                                                                        &  & \by{ii:5.3.7}    \\
     & = \int_{[0, 1]} f(x) \big(\cos(2 \pi n x) - i \sin(2 \pi n x)\big) \; dx                                            &  & \by{ii:4.7.2}[f] \\
     & = \int_{[0, 1]} f(x) \cos(2 \pi n x) \; dx - i \int_{[0, 1]} f(x) \sin(2 \pi n x) \; dx                             &  & \by{ii:5.2.2}    \\
     & = \dfrac{1}{2} \bigg(2 \int_{[0, 1]} f(x) \cos(2 \pi n x) \; dx - 2i \int_{[0, 1]} f(x) \sin(2 \pi n x) \; dx\bigg)                       \\
     & = \dfrac{1}{2} (a_n - i b_n)
  \end{align*}
  and
  \begin{align*}
     & \hat{f}(-n)                                                                                                                                            \\
     & = \dfrac{1}{2} (a_{-n} - i b_{-n})                                                                                  &  & \text{(from the proof above)} \\
     & = \dfrac{1}{2} \bigg(2 \int_{[0, 1]} f(x) \cos(- 2 \pi n x) \; dx                                                                                      \\
     & \quad - 2i \int_{[0, 1]} f(x) \sin(- 2 \pi n x) \; dx\bigg)                                                                                            \\
     & = \dfrac{1}{2} \bigg(2 \int_{[0, 1]} f(x) \cos(2 \pi n x) \; dx                                                                                        \\
     & \quad - 2i \int_{[0, 1]} -f(x) \sin(2 \pi n x) \; dx\bigg)                                                          &  & \by{ii:4.7.2}[c]              \\
     & = \dfrac{1}{2} \bigg(2 \int_{[0, 1]} f(x) \cos(2 \pi n x) \; dx + 2i \int_{[0, 1]} f(x) \sin(2 \pi n x) \; dx\bigg) &  & \by{ii:5.2.2}                 \\
     & = \dfrac{1}{2} (a_n + i b_n).
  \end{align*}
  By Fourier theroem (\cref{ii:5.5.1}) we know that
  \[
    \lim_{N \to \infty} \norm*{f - \sum_{n = -N}^N \hat{f}(n) e_n}_2 = 0.
  \]
  Since for all \(N \in \Z^+\), we have
  \begin{align*}
     & \sum_{n = -N}^N \hat{f}(n) e_n                                                                                                                                   \\
     & = \hat{f}(0) e_0 + \sum_{n = 1}^N \hat{f}(n) e_n + \sum_{n = -N}^{-1} \hat{f}(n) e_n                                                                             \\
     & = \hat{f}(0) e_0 + \sum_{n = 1}^N \hat{f}(n) e_n + \sum_{n = 1}^N \hat{f}(-n) e_{-n}                                                                             \\
     & = \dfrac{(a_0 - i b_0) e_0}{2} + \sum_{n = 1}^N \dfrac{(a_n - i b_n) e_n}{2} + \sum_{n = 1}^N \dfrac{(a_n + i b_n) e_{-n}}{2} &  & \text{(from the proof above)} \\
     & = \dfrac{a_0 e_0}{2} + \sum_{n = 1}^N \dfrac{(a_n - i b_n) e_n}{2} + \sum_{n = 1}^N \dfrac{(a_n + i b_n) e_{-n}}{2}           &  & \by{ii:4.7.2}[e]              \\
     & = \dfrac{a_0}{2} + \sum_{n = 1}^N \dfrac{(a_n - i b_n) e_n}{2} + \sum_{n = 1}^N \dfrac{(a_n + i b_n) e_{-n}}{2}               &  & \by{ii:4.5.2}[e]              \\
     & = \dfrac{a_0}{2} + \sum_{n = 1}^N \dfrac{a_n (e_n + e_{-n}) - i b_n (e_n - e_{-n})}{2}                                        &  & \by{ii:4.6.6}                 \\
     & = \dfrac{a_0}{2} + \sum_{n = 1}^N a_n \cos(2 \pi n x) + b_n \sin(2 \pi n x),                                                  &  & \by{ii:4.7.1}
  \end{align*}
  we know that
  \[
    \lim_{N \to \infty} \norm*{f - \bigg(\dfrac{a_0}{2} + \sum_{n = 1}^N a_n \cos(2 \pi n x) + b_n \sin(2 \pi n x)\bigg)}_2 = 0.
  \]
  Thus, by \cref{ii:1.1.14} we have
  \begin{align*}
     & d_{L^2} - \lim_{N \to \infty} \bigg(\dfrac{a_0}{2} + \sum_{n = 1}^N a_n \cos(2 \pi n x) + b_n \sin(2 \pi n x)\bigg) \\
     & = \dfrac{a_0}{2} + \sum_{n = 1}^\infty a_n \cos(2 \pi n x) + b_n \sin(2 \pi n x)                                    \\
     & = f.
  \end{align*}
\end{proof}

\begin{proof}{(b)}
  Observe that
  \begin{align*}
     & \dfrac{a_0}{2} + \sum_{n = 1}^\infty \abs{a_n} + \sum_{n = 1}^\infty \abs{b_n}                                                               &  & \text{(by hypothesis)} \\
     & = \dfrac{a_0}{2} + \lim_{N \to \infty} \sum_{n = 1}^N \abs{a_n} + \lim_{N \to \infty} \sum_{n = 1}^N \abs{b_n}                               &  & \by{ii:ac:4.6.6}       \\
     & = \dfrac{a_0}{2} + \lim_{N \to \infty} \sum_{n = 1}^N \abs{a_n} + \abs{b_n}                                                                  &  & \by{ii:4.6.14}         \\
     & = \dfrac{a_0}{2} + 2 \bigg(\lim_{N \to \infty} \sum_{n = 1}^N \dfrac{\abs{a_n} + \abs{b_n}}{2}\bigg)                                         &  & \by{ii:4.6.14}         \\
     & = \dfrac{a_0}{2} + 2 \sum_{n = 1}^\infty \dfrac{\abs{a_n} + \abs{b_n}}{2}                                                                    &  & \by{ii:ac:4.6.6}       \\
     & = \dfrac{a_0}{2} + \sum_{n = 1}^\infty \dfrac{\abs{a_n} + \abs{b_n}}{2} + \sum_{n = 1}^\infty \dfrac{\abs{a_n} + \abs{b_n}}{2}                                           \\
     & = \dfrac{a_0}{2} + \sum_{n = 1}^\infty \dfrac{\abs{a_n} + \abs{i b_n}}{2} + \sum_{n = 1}^\infty \dfrac{\abs{a_n} + \abs{-i b_n}}{2}          &  & \by{ii:4.6.11}         \\
     & \geq \dfrac{a_0}{2} + \sum_{n = 1}^\infty \dfrac{\abs{a_n + i b_n}}{2} + \sum_{n = 1}^\infty \dfrac{\abs{a_n - i b_n}}{2}                    &  & \by{ii:4.6.11}         \\
     & = \dfrac{a_0}{2} + \sum_{n = 1}^\infty \dfrac{\abs{a_{-n} - i b_{-n}}}{2} + \sum_{n = 1}^\infty \dfrac{\abs{a_n - i b_n}}{2}                 &  & \by{ii:4.7.2}[c]       \\
     & = \dfrac{a_0}{2} + \sum_{n = 1}^\infty \dfrac{\abs{a_{-n} - i b_{-n}}}{2} + \sum_{n = 1}^\infty \dfrac{\abs{a_n - i b_n}}{2}                 &  & \by{ii:4.7.2}[e]       \\
     & = \lim_{N \to \infty} \dfrac{a_0}{2} + \sum_{n = 1}^N \dfrac{\abs{a_{-n} - i b_{-n}}}{2} + \sum_{n = 1}^N \dfrac{\abs{a_n - i b_n}}{2}       &  & \by{ii:4.6.14}         \\
     & = \lim_{N \to \infty} \dfrac{a_0}{2} + \sum_{n = -N}^{-1} \dfrac{\abs{a_n - i b_n}}{2} + \sum_{n = 1}^N \dfrac{\abs{a_n - i b_n}}{2}                                     \\
     & = \lim_{N \to \infty} \dfrac{a_0 - i b_0}{2} + \sum_{n = -N}^{-1} \dfrac{\abs{a_n - i b_n}}{2} + \sum_{n = 1}^N \dfrac{\abs{a_n - i b_n}}{2} &  & \by{ii:4.7.2}[e]       \\
     & = \lim_{N \to \infty} \sum_{n = -N}^N \dfrac{\abs{a_{-n} - i b_{-n}}}{2}                                                                                                 \\
     & = \sum_{n = -\infty}^\infty \dfrac{\abs{a_{-n} - i b_{-n}}}{2}.
  \end{align*}
  Since
  \[
    \hat{f}(n) = \dfrac{1}{2} (a_n - i b_n)
  \]
  for all \(n \in \Z\) (cf. the proof of \cref{ii:ex:5.5.1}(a)), we know that
  \begin{align*}
    \sum_{n = -\infty}^\infty \abs{\hat{f}(n)} & = \sum_{n = -\infty}^\infty \abs{\dfrac{a_n - i b_n}{2}}                                                                \\
                                               & = \sum_{n = -\infty}^\infty \dfrac{\abs{a_n - i b_n}}{2}                             &  & \by{ii:ex:4.6.7}              \\
                                               & \leq \dfrac{a_0}{2} + \sum_{n = 1}^\infty \abs{a_n} + \sum_{n = 1}^\infty \abs{b_n}. &  & \text{(from the proof above)}
  \end{align*}
  Thus, \(\sum_{n = -\infty}^\infty \abs{\hat{f}(n)}\) is absolutely convergent.
  By \cref{ii:5.5.3} we know that
  \[
    \lim_{N \to \infty} \norm*{f - \sum_{n = -N}^N \hat{f}(n) e_n}_{\infty} = 0.
  \]
  and \(\sum_{n = -\infty}^\infty \hat{f}(n) e_n\) converges uniformly to \(f\) on \(\R\) with respect to \(d_{l^1}|_{\C \times \C}\).
  In particular, we have (by \cref{ii:ex:5.2.3} and squeeze test)
  \[
    \lim_{N \to \infty} \norm*{f - \sum_{n = -N}^N \hat{f}(n) e_n}_2 = 0.
  \]
  By \cref{ii:ex:5.5.1}(a) we know that
  \[
    \lim_{N \to \infty} \norm*{f - \bigg(\dfrac{a_0}{2} + \sum_{n = 1}^N a_n \cos(2 \pi n x) + b_n \sin(2 \pi n x)\bigg)}_2 = 0.
  \]
  Thus, by \cref{ii:1.1.20} we have
  \[
    \lim_{N \to \infty} \norm*{f - \bigg(\dfrac{a_0}{2} + \sum_{n = 1}^N a_n \cos(2 \pi n x) + b_n \sin(2 \pi n x)\bigg)}_{\infty} = 0.
  \]
  and \(\dfrac{a_0}{2} + \sum_{n = 1}^\infty a_n \cos(2 \pi n x) + b_n \sin(2 \pi n x)\) converges uniformly to \(f\) on \(\R\) with respect to \(d_{l^1}|_{\C \times \C}\).
\end{proof}

\begin{ex}\label{ii:ex:5.5.2}
  Let \(f(x)\) be the function defined by \(f(x) = (1 - 2x)^2\) when \(x \in [0, 1)\), and extended to be \(\Z\)-periodic for the rest of the real line.
  \begin{enumerate}
    \item Using \cref{ii:ex:5.5.1}, show that the series
          \[
            \dfrac{1}{3} + \sum_{n = 1}^\infty \dfrac{4}{\pi^2 n^2} \cos(2 \pi n x)
          \]
          converges uniformly to \(f\) .
    \item Conclude that \(\sum_{n = 1}^\infty \dfrac{1}{n^2} = \dfrac{\pi^2}{6}\).
    \item Conclude that \(\sum_{n = 1}^\infty \dfrac{1}{n^4} = \dfrac{\pi^4}{90}\).
  \end{enumerate}
\end{ex}

\begin{proof}{(a)}
  For each \(n \in \N\), we define \(a_n, b_n\) as in \cref{ii:ex:5.5.1}.
  Observe that for all \(n \in \Z^+\), we have
  \begin{align*}
    a_n & = 2 \int_{[0, 1]} (1 - 2x)^2 \cos(2 \pi n x) \; dx                                           &  & \by{ii:ex:5.5.1}                \\
        & = \dfrac{2}{2 \pi n} \int_{[0, 1]} (1 - 2x)^2 \sin'(2 \pi n x) \; dx                         &  & \by{ii:4.7.2}[b]                \\
        & = \dfrac{1}{\pi n} \bigg(\big((1 - 2x)^2 \sin(2 \pi n x)\big)|_{x = 0}^{x = 1}               &  & \text{(by Proposition 11.10.1)} \\
        & \quad - \int_{[0, 1]} -4 (1 - 2x) \sin(2 \pi n x) \; dx\bigg)                                                                     \\
        & = \dfrac{4}{\pi n} \int_{[0, 1]} (1 - 2x) \sin(2 \pi n x) \; dx                              &  & \by{ii:4.7.2}[e]                \\
        & = \dfrac{-4}{2 \pi^2 n^2} \int_{[0, 1]} (1 - 2x) \cos'(2 \pi n x) \; dx                      &  & \by{ii:4.7.2}[b]                \\
        & = \dfrac{-2}{\pi^2 n^2} \bigg(\big((1 - 2x) \cos(2 \pi n x)\big)|_{x = 0}^{x = 1}                                                 \\
        & \quad - \int_{[0, 1]} -2 \cos(2 \pi n x) \; dx\bigg)                                         &  & \text{(by Proposition 11.10.1)} \\
        & = \dfrac{-2}{\pi^2 n^2} \bigg(-2 + 2 \int_{[0, 1]} \cos(2 \pi n x) \; dx\bigg)               &  & \by{ii:4.7.2}[e]                \\
        & = \dfrac{-2}{\pi^2 n^2} \bigg(-2 + \dfrac{2 \sin(2 \pi n x)}{2 \pi n}|_{x = 0}^{x = 1}\bigg) &  & \by{ii:4.7.2}[b]                \\
        & = \dfrac{4}{\pi^2 n^2}                                                                       &  & \by{ii:4.7.2}[e]
  \end{align*}
  and
  \begin{align*}
    b_n & = 2 \int_{[0, 1]} (1 - 2x)^2 \sin(2 \pi n x) \; dx                                &  & \by{ii:ex:5.5.1}                \\
        & = \dfrac{-2}{2 \pi n} \int_{[0, 1]} (1 - 2x)^2 \cos'(2 \pi n x) \; dx             &  & \by{ii:4.7.2}[b]                \\
        & = \dfrac{-1}{\pi n} \bigg(\big((1 - 2x)^2 \cos(2 \pi n x)\big)|_{x = 0}^{x = 1}   &  & \text{(by Proposition 11.10.1)} \\
        & \quad - \int_{[0, 1]} -4 (1 - 2x) \cos(2 \pi n x) \; dx\bigg)                                                          \\
        & = \dfrac{-4}{\pi n} \int_{[0, 1]} (1 - 2x) \cos(2 \pi n x) \; dx                  &  & \by{ii:4.7.2}[e]                \\
        & = \dfrac{-4}{2 \pi^2 n^2} \int_{[0, 1]} (1 - 2x) \sin'(2 \pi n x) \; dx           &  & \by{ii:4.7.2}[b]                \\
        & = \dfrac{-2}{\pi^2 n^2} \bigg(\big((1 - 2x) \sin(2 \pi n x)\big)|_{x = 0}^{x = 1}                                      \\
        & \quad - \int_{[0, 1]} -2 \sin(2 \pi n x) \; dx\bigg)                              &  & \text{(by Proposition 11.10.1)} \\
        & = \dfrac{-4}{\pi^2 n^2} \int_{[0, 1]} \sin(2 \pi n x) \; dx                       &  & \by{ii:4.7.2}[e]                \\
        & = \dfrac{-4}{\pi^2 n^2} \dfrac{-\cos(2 \pi n x)}{2 \pi n}|_{x = 0}^{x = 1}        &  & \by{ii:4.7.2}[b]                \\
        & = 0.                                                                              &  & \by{ii:4.7.2}[e]
  \end{align*}
  Since
  \begin{align*}
    \sum_{n = 1}^\infty \abs{a_n} & = \sum_{n = 1}^\infty \dfrac{4}{\pi^2 n^2} \\
    \sum_{n = 1}^\infty \abs{b_n} & = \sum_{n = 1}^\infty 0
  \end{align*}
  are absolutely convergent (by Corollary 7.3.7 in Analysis I), by \cref{ii:ex:5.5.1}(b) we know that the series
  \[
    \dfrac{a_0}{2} + \sum_{n = 1}^\infty (a_n \cos(2 \pi n x) + b_n \sin(2 \pi n x))
  \]
  converges uniformly to \(f\) on \(\R\) with respect to \(d_{l^1}|_{\C \times \C}\), and
  \begin{align*}
     & \dfrac{a_0}{2} + \sum_{n = 1}^\infty (a_n \cos(2 \pi n x) + b_n \sin(2 \pi n x))                                                                                                 \\
     & = \int_{[0, 1]} (1 - 2x)^2 \cos(0) \; dx + \sum_{n = 1}^\infty \dfrac{4}{\pi^2 n^2} \cos(2 \pi n x)                                           &  & \text{(from the proof above)} \\
     & = \int_{[0, 1]} (1 - 2x)^2 \; dx + \sum_{n = 1}^\infty \dfrac{4}{\pi^2 n^2} \cos(2 \pi n x)                                                   &  & \by{ii:4.7.2}[e]              \\
     & = 1 - \big(2x^2|_{x = 0}^{x = 1}\big) + \big(\dfrac{4x^3}{3}|_{x = 0}^{x = 1}\big) + \sum_{n = 1}^\infty \dfrac{4}{\pi^2 n^2} \cos(2 \pi n x)                                    \\
     & = 1 - 2 + \dfrac{4}{3} + \sum_{n = 1}^\infty \dfrac{4}{\pi^2 n^2} \cos(2 \pi n x)                                                                                                \\
     & = \dfrac{1}{3} + \sum_{n = 1}^\infty \dfrac{4}{\pi^2 n^2} \cos(2 \pi n x).
  \end{align*}
\end{proof}

\begin{proof}{(b)}
  We have
  \begin{align*}
             & (1 - 2 \cdot 0)^2 = \dfrac{1}{3} + \sum_{n = 1}^\infty \dfrac{4}{\pi^2 n^2} \cos(2 \pi n \cdot 0) &  & \by{ii:ex:5.5.2}[a] \\
    \implies & 1 = \dfrac{1}{3} + \sum_{n = 1}^\infty \dfrac{4}{\pi^2 n^2}                                       &  & \by{ii:4.7.2}[e]    \\
    \implies & \dfrac{2}{3} = \sum_{n = 1}^\infty \dfrac{4}{\pi^2 n^2}                                                                    \\
    \implies & \dfrac{\pi^2}{6} = \sum_{n = 1}^\infty \dfrac{1}{n^2}.
  \end{align*}
\end{proof}

\begin{proof}{(c)}
  By \cref{ii:ex:5.5.2}(a) we know that
  \[
    f(x) = (1 - 2x)^2 = \dfrac{1}{3} + \sum_{n = 1}^\infty \dfrac{4}{\pi^2 n^2} \cos(2 \pi n x)
  \]
  and the series on the right hand side converges uniformly to \(f\).
  Observe that for each \(m \in \Z\), we have
  \begin{align*}
     & \hat{f}(m)                                                                                                                                                                                       \\
     & = \int_{[0, 1]} \bigg(\dfrac{1}{3} + \sum_{n = 1}^\infty \dfrac{4}{\pi^2 n^2} \cos(2 \pi n x)\bigg) e^{-2 \pi i m x} \; dx                                                &  & \by{ii:5.3.7}     \\
     & = \int_{[0, 1]} \dfrac{e^{- 2 \pi i m x}}{3} \; dx + \int_{[0, 1]} \sum_{n = 1}^\infty \dfrac{4 e^{- 2 \pi i m x}}{\pi^2 n^2} \cos(2 \pi n x) \; dx                       &  & \by{ii:5.2.2}     \\
     & = \int_{[0, 1]} \dfrac{e^{- 2 \pi i m x}}{3} \; dx + \sum_{n = 1}^\infty \int_{[0, 1]} \dfrac{4 e^{- 2 \pi i m x}}{\pi^2 n^2} \cos(2 \pi n x) \; dx                       &  & \by{ii:3.6.2}     \\
     & = \int_{[0, 1]} \dfrac{e^{- 2 \pi i m x}}{3} \; dx + \sum_{n = 1}^\infty \int_{[0, 1]} \dfrac{2 e^{- 2 \pi i m x} (e^{2 \pi i n x} + e^{- 2 \pi i n x})}{\pi^2 n^2} \; dx &  & \by{ii:4.7.1}     \\
     & = \int_{[0, 1]} \dfrac{e^{- 2 \pi i m x}}{3} \; dx + \sum_{n = 1}^\infty \int_{[0, 1]} \dfrac{2 (e^{2 \pi i (n - m) x} + e^{- 2 \pi i (n + m) x})}{\pi^2 n^2} \; dx.      &  & \by{ii:ex:4.6.16}
  \end{align*}
  Now we split into two cases:
  \begin{itemize}
    \item If \(m = 0\), then we have
          \begin{align*}
            \hat{f}(0) & = \int_{[0, 1]} \dfrac{1}{3} \; dx + \sum_{n = 1}^\infty \int_{[0, 1]} \dfrac{2 (e^{2 \pi i n x} + e^{- 2 \pi i n x})}{\pi^2 n^2} \; dx &  & \by{ii:4.5.2}[e] \\
                       & = \dfrac{1}{3} + \sum_{n = 1}^\infty \bigg(\dfrac{2}{\pi^2 n^2} \int_{[0, 1]} e^{2 \pi i n x} + e^{- 2 \pi i n x} \; dx\bigg)                                 \\
                       & = \dfrac{1}{3} + \sum_{n = 1}^\infty \dfrac{2}{\pi^2 n^2} (0 + 0)                                                                       &  & \by{ii:5.3.5}    \\
                       & = \dfrac{1}{3}.
          \end{align*}
    \item If \(m \neq 0\), then we have
          \begin{align*}
             & \hat{f}(m)                                                                                                                                                \\
             & = 0 + \sum_{n = 1}^\infty \int_{[0, 1]} \dfrac{2 (e^{2 \pi i (n - m) x} + e^{- 2 \pi i (n + m) x})}{\pi^2 n^2} \; dx                   &  & \by{ii:5.3.5} \\
             & = \sum_{n = 1}^\infty \Bigg(\dfrac{2}{\pi^2 n^2} \bigg(\int_{[0, 1]} e^{2 \pi i (n - m) x} + e^{- 2 \pi i (n + m) x} \; dx\bigg)\Bigg)                    \\
             & = \sum_{n = 1}^\infty \bigg(\dfrac{2}{\pi^2 n^2} \big(\inner*{e_n, e_m} + \inner*{e_{-n}, e_m}\big)\bigg)                              &  & \by{ii:5.2.1} \\
             & = \dfrac{2}{\pi^2 m^2}.                                                                                                                &  & \by{ii:5.3.5}
          \end{align*}
  \end{itemize}
  From all cases above, we have
  \begin{align*}
             & \norm*{f}_2^2 = \sum_{n = -\infty}^\infty \abs{\hat{f}(n)}^2                                                                         &  & \by{ii:5.5.4}                 \\
    \implies & \int_{[0, 1]} (1 - 2x)^4 \; dx = \sum_{n = -\infty}^\infty \abs{\hat{f}(n)}^2                                                                                           \\
    \implies & \int_{[0, 1]} 1 - 8x + 24x^2 - 32x^3 + 16x^4 \; dx = \sum_{n = -\infty}^\infty \abs{\hat{f}(n)}^2                                                                       \\
    \implies & 1 - 4 (x^2|_{x = 0}^{x = 1}) + 8 (x^3|_{x = 0}^{x = 1}) - 8(x^4|_{x = 0}^{x = 1}) + \dfrac{16}{5}(x^5|_{x = 0}^{x = 1})                                                 \\
             & = \sum_{n = -\infty}^\infty \abs{\hat{f}(n)}^2                                                                                                                          \\
    \implies & \dfrac{1}{5} = \dfrac{1}{9} + \sum_{n = 1}^\infty \abs{\dfrac{2}{\pi^2 n^2}}^2 + \sum_{n = 1}^\infty \abs{\dfrac{2}{\pi^2 (-n)^2}}^2 &  & \text{(from the proof above)} \\
    \implies & \dfrac{4}{45} = 2 \sum_{n = 1}^\infty \abs{\dfrac{2}{\pi^2 n^2}}^2                                                                                                      \\
    \implies & \dfrac{4}{45} = 8 \sum_{n = 1}^\infty \dfrac{1}{\pi^4 n^4}                                                                                                              \\
    \implies & \dfrac{\pi^4}{90} = \sum_{n = 1}^\infty \dfrac{1}{n^4}.
  \end{align*}
\end{proof}

\begin{ex}\label{ii:ex:5.5.3}
  If \(f \in C(\R / \Z ; \C)\) and \(P\) is a trigonometric polynomial, show that
  \[
    \widehat{f * P}(n) = \hat{f}(n) c_n = \hat{f}(n) \hat{P}(n)
  \]
  for all integers \(n\).
  More generally, if \(f, g \in C(\R / \Z ; \C)\), show that
  \[
    \widehat{f * g}(n) = \hat{f}(n) \hat{g}(n)
  \]
  for all integers \(n\).
  (A fancy way of saying this is that the Fourier transform \emph{intertwines} convolution and multiplication.)
\end{ex}

\begin{proof}
  Let \(P = \sum_{n = -N}^N c_n e_n\) for some \(N \in \Z^+\) and some \((c_n)_{n = -N}^N\) in \(\C\).
  By \cref{ii:ac:5.4.1} we know that
  \[
    f * P = \sum_{n = -N}^N \hat{f}(n) c_n e_n.
  \]
  Thus, we have
  \begin{align*}
    \forall m \in \Z, \widehat{f * P}(m) & = \inner{f * P, e_m}                               &  & \by{ii:5.3.7}    \\
                                         & = \sum_{n = -N}^N \hat{f}(n) c_n \inner{e_n, e_m}  &  & \by{ii:5.2.5}[c] \\
                                         & = \begin{dcases}
                                               \hat{f}(m) c_m & \text{if } n = m    \\
                                               0              & \text{if } n \neq m
                                             \end{dcases}            &  & \by{ii:5.3.5}                             \\
                                         & = \hat{f}(m) \sum_{n = -N}^N c_n \inner*{e_n, e_m} &  & \by{ii:5.3.5}    \\
                                         & = \hat{f}(m) \inner*{\sum_{n = -N}^N c_n e_n, e_m} &  & \by{ii:5.2.5}[c] \\
                                         & = \hat{f}(m) \hat{P}(m).                           &  & \by{ii:5.3.7}
  \end{align*}

  Now we show that \(\widehat{f * g}(n) = \hat{f}(n) \hat{g}(n)\) for all \(n \in \Z\).
  In particular, we want to show that
  \[
    \forall \varepsilon \in \R^+, \abs{\widehat{f * g}(n) - \hat{f}(n) \hat{g}(n)} \leq \varepsilon.
  \]
  Let \(\varepsilon \in \R^+\).
  Since \(g \in C(\R / \Z ; \C)\), by \cref{ii:5.4.1} we know that there exists a trigonometric polynomial \(P\) such that
  \[
    \norm*{g - P}_{\infty} \leq \varepsilon.
  \]
  Fix such \(P\).
  Since \(f \in C(\R / \Z ; \C)\), by \cref{ii:5.5.4} we know that \(\sum_{n = -\infty}^\infty \abs{\hat{f}(n)}^2 \in \R\), thus
  \[
    \exists M \in \R^+ : \forall n \in \Z, \abs{\hat{f}(n)} \leq M.
  \]
  Fix such \(M\).
  Then for all \(n \in \Z\), we have
  \begin{align*}
    \abs{\widehat{f * g}(n) - \widehat{f * P}(n)} & = \abs{\inner*{f * g, e_n} - \inner*{f * P, e_n}} &  & \by{ii:5.3.7}    \\
                                                  & = \abs{\inner*{f * g - f * P, e_n}}               &  & \by{ii:5.2.5}[c] \\
                                                  & = \abs{\inner*{f * (g - P), e_n}}                 &  & \by{ii:5.4.4}[c] \\
                                                  & \leq \norm*{f * (g - P)}_2 \norm*{e_n}_2          &  & \by{ii:5.2.7}[b] \\
                                                  & = \norm*{f * (g - P)}_2.                          &  & \by{ii:5.3.5}
  \end{align*}
  Need some helps.
\end{proof}

\begin{ex}\label{ii:ex:5.5.4}
  Let \(f \in C(\R / \Z ; \C)\) be a function which is differentiable, and whose derivative \(f'\) is also continuous.
  Show that \(f'\) also lies in \(C(\R / \Z ; \C)\), and that \(\hat{f}'(n) = 2 \pi i n \hat{f}(n)\) for all integers \(n\).
  Here the derivative of a complex-valued function is defined in exactly the same fashion as for real-valued functions.
\end{ex}

\begin{ex}\label{ii:ex:5.5.5}
  Let \(f, g \in C(\R / \Z ; \C)\).
  Prove the \emph{Parseval identity}
  \[
    \Re\bigg(\int_0^1 f(x) \overline{g(x)} \; dx\bigg) = \Re\bigg(\sum_{n \in \Z} \hat{f}(n) \overline{\hat(g)(n)}\bigg).
  \]
  Then conclude that the real parts can be removed, thus
  \[
    \int_0^1 f(x) \overline{g(x)} \; dx = \sum_{n \in \Z} \hat{f}(n) \overline{\hat(g)(n)}.
  \]
\end{ex}

\begin{ex}\label{ii:ex:5.5.6}
  In this exercise we shall develop the theory of Fourier series for functions of any fixed period \(L\).

  Let \(L > 0\), and let \(f : \R \to \C\) be a complex-valued function which is continuous and \(L\)-periodic.
  Define the numbers \(c_n\) for every integer \(n\) by
  \[
    c_n \coloneqq \dfrac{1}{L} \int_{[0, L]} f(x) e^{- 2 \pi i n x / L} \; dx.
  \]
  \begin{enumerate}
    \item Show that the series
          \[
            \sum_{n = -\infty}^\infty c_n e^{2 \pi i n x / L}
          \]
          converges in \(L_2\) metric to \(f\).
          More precisely, show that
          \[
            \lim_{N \to \infty} \int_{[0, L]} \abs{f(x) - \sum_{n = -N}^N c_n e^{2 \pi i n x / L}}^2 \; dx = 0.
          \]
    \item If the series \(\sum_{n = -\infty}^\infty \abs{c_n}\) is absolutely convergent, show that
          \[
            \sum_{n = -\infty}^\infty c_n e^{2 \pi i n x / L}
          \]
          converges uniformly to \(f\).
    \item Show that
          \[
            \dfrac{1}{L} \int_{[0, L]} \abs{f(x)}^2 \; dx = \sum_{n = -\infty}^\infty \abs{c_n}^2.
          \]
  \end{enumerate}
\end{ex}

