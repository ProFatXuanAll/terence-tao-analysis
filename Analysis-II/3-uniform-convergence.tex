\chapter{Uniform convergence}\label{ii:ch:3}

\begin{note}
  It turns out that there are several different concepts of convergence of functions;
  here we describe the two most important ones, \emph{pointwise convergence} and \emph{uniform convergence}.
  (There are other types of convergence for functions, such as \(L^1\) convergence, \(L^2\) convergence, convergence in measure, almost everywhere convergence, and so forth, but these are beyond the scope of this text.)
  The two notions are related, but not identical;
  the relationship between the two is somewhat analogous to the relationship between continuity and uniform continuity.
\end{note}

\section{Limiting values of functions}\label{sec:3.1}

\begin{defn}[Limiting value of a function]\label{3.1.1}
  Let \((X, d_X)\) and \((Y, d_Y)\) be metric spaces, let \(E\) be a subset of \(X\), and let \(f : E \to Y\) be a function.
  If \(x_0 \in X\) is an adherent point of \(E\), and \(L \in Y\), we say that \emph{\(f(x)\) converges to \(L\) in \(Y\) as \(x\) converges to \(x_0\) in \(E\)}, or write \(\lim_{x \to x_0 ; x \in E} f(x) = L\), if for every \(\varepsilon > 0\) there exists a \(\delta > 0\) such that \(d_Y\big(f(x), L\big) < \varepsilon\) for all \(x \in E\) such that \(d_X(x, x_0) < \delta\).
\end{defn}

\begin{rmk}\label{3.1.2}
  Some authors exclude the case \(x = x_0\) from the above definition, thus requiring \(0 < d_X(x, x_0) < \delta\).
  In our current notation, this would correspond to removing \(x_0\) from \(E\), thus one would consider
  \[
    \lim_{x \to x_0 ; x \in E \setminus \{x_0\}} f(x)
  \]
  instead of
  \[
    \lim_{x \to x_0 ; x \in E} f(x).
  \]
\end{rmk}

\begin{note}
  Comparing this with \cref{2.1.1}, we see that \(f\) is continuous at \(x_0\) if and only if
  \[
    \lim_{x \to x_0 ; x \in X} f(x) = f(x_0).
  \]
  Thus \(f\) is continuous on \(X\) iff we have
  \[
    \lim_{x \to x_0 ; x \in X} f(x) = f(x_0) \text{ for all } x_0 \in X.
  \]
\end{note}

\setcounter{thm}{3}
\begin{rmk}\label{3.1.4}
  Often we shall omit the condition \(x \in X\), and abbreviate
  \[
    \lim_{x \to x_0 ; x \in X} f(x)
  \]
  as simply
  \[
    \lim_{x \to x_0} f(x)
  \]
  when it is clear what space \(x\) will range in.
\end{rmk}

\begin{prop}\label{3.1.5}
  Let \((X, d_X)\) and \((Y, d_Y)\) be metric spaces, let \(E\) be a subset of \(X\), and let \(f : E \to Y\) be a function.
  Let \(x_0 \in X\) be an adherent point of \(E\) and \(L \in Y\).
  Then the following four statements are logically equivalent:
  \begin{enumerate}
    \item \(\lim_{x \to x_0 ; x \in E} f(x) = L\).
    \item For every sequence \((x^{(n)})_{n = 1}^\infty\) in \(E\) which converges to \(x_0\) with respect to the metric \(d_X\), the sequence \(\big(f(x^{(n)})\big)_{n = 1}^\infty\) converges to \(L\) with respect to the metric \(d_Y\).
    \item For every open set \(V \subseteq Y\) which contains \(L\), there exists an open set \(U \subseteq X\) containing \(x_0\) such that \(f(U \cap E) \subseteq V\).
    \item If one defines the function \(g : E \cup \{x_0\} \to Y\) by defining \(g(x_0) \coloneqq L\), and \(g(x) \coloneqq f(x)\) for \(x \in E \setminus \{x_0\}\), then \(g\) is continuous at \(x_0\).
          Furthermore, if \(x_0 \in E\), then \(f(x_0) = L\).
  \end{enumerate}
\end{prop}

\begin{proof}
  We first show that statement (a) implies statement (b).
  Suppose that
  \[
    d_Y - \lim_{x \to x_0 ; x \in E} f(x) = L.
  \]
  By \cref{3.1.1} we have
  \[
    \forall \varepsilon \in \R^+, \exists\ \delta \in \R^+ : \Big(\forall x \in E, d_X(x, x_0) < \delta \implies d_Y\big(f(x), L\big) < \varepsilon\Big).
  \]
  Let \((x^{(n)})_{n = 1}^\infty\) be a sequence in \(E\) such that \(\lim_{n \to \infty} d_X(x^{(n)}, x_0) = 0\).
  By \cref{1.1.14} we have
  \[
    \forall \delta \in \R^+, \exists\ N \in \Z^+ : \forall n \geq N, d_X(x^{(n)}, x_0) < \delta.
  \]
  Since \((x^{(n)})_{n = 1}^\infty\) is in \(E\), we have
  \[
    \forall \varepsilon \in \R^+, \exists\ \delta \in \R^+ : \begin{dcases}
      \exists\ N \in \Z^+ : \forall n \geq N, d_X(x^{(n)}, x_0) < \delta \\
      d_X(x^{(n)}, x_0) < \delta \implies d_Y\big(f(x^{(n)}), L\big) < \varepsilon
    \end{dcases}
  \]
  and
  \[
    \forall \varepsilon \in \R^+, \exists\ N \in \Z^+ : \forall n \geq N, d_Y\big(f(x^{(n)}, L)\big) < \varepsilon.
  \]
  By \cref{1.1.14} we have \(\lim_{n \to \infty} d_Y\big(f(x^{(n)}), L\big) = 0\).
  Since \((x^{(n)})_{n = 1}^\infty\) is arbitrary, we conclude that (a) implies (b).

  Next we show that statement (b) implies statement (a).
  Suppose that if \((x^{(n)})_{n = 1}^\infty\) is a sequence in \(X\) such that \(\lim_{n \to \infty} d_X(x^{(n)}, x_0) = 0\), then \(\lim_{n \to \infty} d_Y\big(f(x), L\big) = 0\).
  Suppose for sake of contradiction that
  \[
    d_Y - \lim_{x \to x_0 ; x \in X} f(x) \neq L.
  \]
  Then by \cref{3.1.1} we have
  \[
    \exists\ \varepsilon \in \R^+ : \forall \delta \in \R^+, \exists\ x \in X : \begin{dcases}
      d_X(x, x_0) < \delta \\
      d_Y\big(f(x), L\big) \geq \varepsilon
    \end{dcases}
  \]
  Thus we can choose one sequence \((x^{(n)})_{n = 1}^\infty\) which satsifies
  \[
    \forall n \in \Z^+, \begin{dcases}
      d_X(x^{(n)}, x_0) < \dfrac{1}{n} \\
      d_Y\big(f(x^{(n)}), L\big) \geq \varepsilon
    \end{dcases}
  \]
  By squeeze test we have \(\lim_{n \to \infty} d_X(x^{(n)}, x_0) = 0\).
  But by hypothesis we know that \(\lim_{n \to \infty} d_Y\big(f(x^{(n)}), L\big) = 0\), which means
  \[
    \exists\ N \in \Z^+ : \forall n \geq N, d_Y\big(f(x^{(n)}), L\big) < \varepsilon,
  \]
  a contradiction.
  Thus we have
  \[
    d_Y - \lim_{x \to x_0 ; x \in X} f(x) = L
  \]
  and we conclude that statements (a)(b) are equivalent.

  Next we show that statement (a) implies statement (c).
  Suppose that
  \[
    d_Y - \lim_{x \to x_0 ; x \in E} f(x) = L.
  \]
  By \cref{3.1.1} we have
  \begin{align*}
             & \forall \varepsilon \in \R^+, \exists\ \delta \in \R^+ : \Big(\forall x \in E, d_X(x, x_0) < \delta \implies d_Y\big(f(x), L\big) < \varepsilon\Big)   \\
    \implies & \forall \varepsilon \in \R^+, \exists\ \delta \in \R^+ : \Big(x \in B_{(X, d_X)}(x_0, \delta) \cap E \implies d_Y\big(f(x), L\big) < \varepsilon\Big).
  \end{align*}
  Let \(V\) be an open set in \((Y, d_Y)\) such that \(L \in V\).
  Then we have
  \begin{align*}
             & V = \text{int}_{(Y, d_Y)}(V)                                                                           &  & \text{(by \cref{1.2.15}(a))} \\
    \implies & \exists\ \varepsilon \in \R^+ : B_{(Y, d_Y)}(L, \varepsilon) \subseteq V                               &  & \by{1.2.5}                   \\
    \implies & \exists\ \delta \in \R^+ :                                                                                                               \\
             & \begin{dcases}
                 x \in B_{(X, d_X)}(x_0, \delta) \cap E \implies d_Y\big(f(x), L\big) < \varepsilon \\
                 f\big(B_{(X, d_X)}(x_0, \delta) \cap E\big) \subseteq B_{(Y, d_Y)}\big(L, \varepsilon\big) \subseteq V
               \end{dcases}
  \end{align*}
  and by \cref{1.2.15}(c) we know that \(B_{(X, d_X)}(x_0, \delta)\) is open in \((X, d_X)\).
  Since \(V\) is arbitrary, we conclude that statement (a) implies statement (c).

  Next we show that statement (c) implies statement (a).
  Suppose that
  \[
    \forall V \subseteq Y, \begin{dcases}
      L \in V \\
      V \text{ is open in } (Y, d_Y)
    \end{dcases} \implies \exists\ U \subseteq X : \begin{dcases}
      x_0 \in U                      \\
      U \text{ is open in } (X, d_X) \\
      f(U \cap E) \subseteq V
    \end{dcases}
  \]
  Let \(\varepsilon \in \R^+\).
  By \cref{1.2.15}(c) we know that \(B_{(Y, d_Y)}(L, \varepsilon)\) is open in \((Y, d_Y)\).
  By hypothesis we know that there exists some \(U \subseteq X\) such that
  \[
    \begin{dcases}
      x_0 \in U                      \\
      U \text{ is open in } (X, d_X) \\
      f(U \cap E) \subseteq B_{(Y, d_Y)}(L, \varepsilon)
    \end{dcases}
  \]
  Then we have
  \begin{align*}
             & \begin{dcases}
                 x_0 \in U \\
                 U = \text{int}_{(X, d_X)}(U)
               \end{dcases}                                                                             &  & \text{(by \cref{1.2.15}(a))} \\
    \implies & \exists\ \delta \in \R^+ : B_{(X, d_X)}(x_0, \delta) \subseteq U                                         &  & \by{1.2.5}   \\
    \implies & \exists\ \delta \in \R^+ : B_{(X, d_X)}(x_0, \delta) \cap E \subseteq U \cap E                                             \\
    \implies & \exists\ \delta \in \R^+ :                                                                                                 \\
             & f\big(B_{(X, d_X)}(x_0, \delta) \cap E\big) \subseteq f(U \cap E) \subseteq B_{(Y, d_Y)}(L, \varepsilon)                   \\
    \implies & \exists\ \delta \in \R^+ :                                                                                                 \\
             & \Big(\forall x \in E, d_X(x, x_0) < \delta \implies d_Y\big(f(x), L\big) < \varepsilon\Big).
  \end{align*}
  Since \(\varepsilon\) is arbitrary, by \cref{3.1.1} we have
  \[
    d_Y - \lim_{x \to x_0 ; x \in E} f(x) = L
  \]
  and we conclude that statements (a)(c) are equivalent.

  Next we show that statement (a) implies statement (d).
  Suppose that
  \[
    d_Y - \lim_{x \to x_0 ; x \in E} f(x) = L.
  \]
  Then by \cref{3.1.1} we have
  \[
    \forall \varepsilon \in \R^+, \exists\ \delta \in \R^+ : \Big(\forall x \in E, d_X(x, x_0) < \delta \implies d_Y\big(f(x), L\big) < \varepsilon\Big).
  \]
  Let \(g : E \cup \{x_0\} \to Y\) be a function where
  \[
    \forall x \in E \cup \{x_0\}, g(x) = \begin{dcases}
      L    & \text{if } x = x_0    \\
      f(x) & \text{if } x \neq x_0
    \end{dcases}
  \]
  Then we have
  \begin{align*}
             & \forall \varepsilon \in \R^+, \exists\ \delta \in \R^+ :                                                       \\
             & \Big(\forall x \in E, d_X(x, x_0) < \delta \implies d_Y\big(f(x), L\big) < \varepsilon\Big)                    \\
    \implies & \forall \varepsilon \in \R^+, \exists\ \delta \in \R^+ :                                                       \\
             & \Big(\forall x \in E \cup \{x_0\}, d_X(x, x_0) < \delta \implies d_Y\big(g(x), g(x_0)\big) < \varepsilon\Big).
  \end{align*}
  Thus by \cref{2.1.1} \(g\) is continuous at \(x_0\) from \((E \cup \{x_0\}, d_X|_{(E \cup \{x_0\}) \times (E \cup \{x_0\})})\) to \((Y, d_Y)\).

  Now suppose that \(x_0 \in E\).
  We claim that \(f(x_0) = L\).
  Suppose for sake of contradiction that \(f(x_0) \neq L\).
  Then by \cref{1.1.2}(b) we have \(d_Y\big(f(x_0), L\big) > 0\).
  Let \(r = d_Y\big(f(x_0), L\big)\).
  By \cref{3.1.1} we have
  \[
    \exists\ \delta \in \R^+ : \forall x \in E, d_X(x, x_0) < \delta \implies d_Y\big(f(x), L\big) < r.
  \]
  Since \(x_0 \in E\), we have \(d_X(x_0, x_0) = 0 < \delta\).
  But then we have \(d_Y\big(f(x_0), L\big) < r = d_Y\big(f(x_0), L\big)\), a contradiction.
  Thus we have \(f(x_0) = L\).

  Finally we show that statement (d) implies statement (a).
  Suppose that \(g : E \cup \{x_0\} \to Y\) is a function where
  \[
    \forall x \in E \cup \{x_0\}, g(x) = \begin{dcases}
      L    & \text{if } x = x_0    \\
      f(x) & \text{if } x \neq x_0
    \end{dcases}
  \]
  and \(g\) is continuous from \((E \cup \{x_0\}, d_X|_{(E \cup \{x_0\}) \times (E \cup \{x_0\})})\) to \((Y, d_Y)\).
  Suppose also that if \(x_0 \in E\), then \(f(x_0) = L\).
  Then by \cref{2.1.1} we have
  \begin{align*}
             & \forall \varepsilon \in \R^+, \exists\ \delta \in \R^+ :                                                       \\
             & \Big(\forall x \in E \cup \{x_0\}, d_X(x, x_0) < \delta \implies d_Y\big(g(x), g(x_0)\big) < \varepsilon\Big)  \\
    \implies & \forall \varepsilon \in \R^+, \exists\ \delta \in \R^+ :                                                       \\
             & \begin{dcases}
                 \forall x \in E \setminus \{x_0\}, d_X(x, x_0) < \delta \implies d_Y\big(f(x), L\big) < \varepsilon \\
                 x_0 \in E \implies d_X(x_0, x_0) = 0 < \delta \implies f(x_0) = L \implies d_Y\big(f(x_0), L\big) < \varepsilon
               \end{dcases} \\
    \implies & \forall \varepsilon \in \R^+, \exists\ \delta \in \R^+ :                                                       \\
             & \Big(\forall x \in E, d_X(x, x_0) < \delta \implies d_Y\big(f(x), L\big) < \varepsilon\Big).
  \end{align*}
  By \cref{3.1.1} this means
  \[
    d_Y - \lim_{x \to x_0 ; x \in E} f(x) = L.
  \]
  We conclude that statements (a)(b)(c)(d) are all equivalent.
\end{proof}

\begin{rmk}\label{3.1.6}
  Observe from \cref{3.1.5}(b) and \cref{1.1.20} that a function \(f(x)\) can converge to at most one limit \(L\) as \(x\) converges to \(x_0\).
  In other words, if the limit
  \[
    \lim_{x \to x_0 ; x \in E} f(x)
  \]
  exists at all, then it can only take at most one value.
\end{rmk}

\begin{rmk}\label{3.1.7}
  The requirement that \(x_0\) be an adherent point of \(E\) is necessary for the concept of limiting value to be useful, otherwise \(x_0\) will lie in the exterior of \(E\), the notion that \(f(x)\) converges to \(L\) as \(x\) converges to \(x_0\) in \(E\) is vacuous
  (for \(\delta\) sufficiently small, there are no points \(x \in E\) so that \(d(x, x_0) < \delta\)).
\end{rmk}

\begin{rmk}\label{3.1.8}
  Strictly speaking, we should write
  \[
    d_Y - \lim_{x \to x_0 ; x \in E} f(x) \text{ instead of } \lim_{x \to x_0 ; x \in E} f(x),
  \]
  since the convergence depends on the metric \(d_Y\).
  However in practice it will be obvious what the metric \(d_Y\) is and so we will omit the \(d_Y -\) prefix from the notation.
\end{rmk}

\exercisesection

\begin{ex}\label{ex:3.1.1}
  Let \((X, d_X)\) and \((Y, d_Y)\) be metric spaces, let \(E\) be a subset of \(X\), let \(f : E \to Y\) be a function, and let \(x_0\) be an element of \(E\).
  Assume that \(x_0\) is an adherent point of \(E \setminus \{x_0\}\)
  (or equivalently, that \(x_0\) is not an \emph{isolated point} of \(E\)).
  Show that the limit \(\lim_{x \to x_0 ; x \in E} f(x)\) exists if and only if the limit \(\lim_{x \to x_0 ; x \in E \setminus \{x_0\}} f(x)\) exists and is equal to \(f(x_0)\).
  Also, show that if the limit \(\lim_{x \to x_0 ; x \in E} f(x)\) exists at all, then it must equal \(f(x_0)\).
\end{ex}

\begin{proof}
  Let \(L \in Y\).
  By \cref{1.1.2}(a) we know that
  \[
    \forall \varepsilon \in \R^+, d_Y\big(f(x_0), L\big) < \varepsilon \iff L = f(x_0).
  \]
  Thus we have
  \begin{align*}
         & d_Y - \lim_{x \to x_0 ; x \in E \setminus \{x_0\}} f(x) = f(x_0)                                                                                                    \\
    \iff & \forall \varepsilon \in \R^+, \exists\ \delta \in \R^+ :                                                                                                            \\
         & \Big(\forall x \in E \setminus \{x_0\}, d_X(x, x_0) < \delta \implies d_Y\big(f(x), f(x_0)\big) < \varepsilon\Big) &                                   & \by{3.1.1} \\
    \iff & \forall \varepsilon \in \R^+, \exists\ \delta \in \R^+ :                                                                                                            \\
         & \Big(\forall x \in E, d_X(x, x_0) < \delta \implies d_Y\big(f(x), f(x_0)\big) < \varepsilon\Big)                   & (E \setminus \{x_0\} \subseteq E)              \\
    \iff & d_Y - \lim_{x \to x_0 ; x \in E} f(x) = f(x_0).                                                                    &                                   & \by{3.1.1}
  \end{align*}
\end{proof}

\begin{ex}\label{ex:3.1.2}
  Prove \cref{3.1.5}.
\end{ex}

\begin{proof}
  See \cref{3.1.5}.
\end{proof}

\begin{ex}\label{ex:3.1.3}
  Use \cref{3.1.5}(c) to define a notion of a limiting value of a function \(f : E \to Y\) from one topological space \((X, \mathcal{F}_X)\) to another \((Y, \mathcal{F}_Y)\) where \(E \subseteq X\).
  If \(X\) is a topological space and \(Y\) is a Hausdorff topological space (see \cref{ex:2.5.4}), prove the equivalence of \cref{3.1.5}(c)(d) in this setting, as well as an analogue of \cref{3.1.6}.
  What happens to these statements of \(Y\) is not Hausdorff?
\end{ex}

\begin{proof}
  Let \((X, \mathcal{F}_X)\), \((Y, \mathcal{F}_Y)\) be topological spaces, let \(E \subseteq X\), let \(f : E \to Y\) be a function, let \(x_0 \in \overline{E}_{(X, \mathcal{F}_X)}\), and let \(L \in Y\).
  We say that \(f(x)\) converges to \(L\) in \(Y\) as \(x\) converges to \(x_0\) in \(E\) iff
  \[
    \forall V \in \mathcal{F}_Y, L \in V \implies \exists\ U \in \mathcal{F}_X : \begin{dcases}
      x_0 \in U \\
      f(U \cap E) \subseteq V
    \end{dcases}
  \]
  We want to show that if \((Y, \mathcal{F}_Y)\) is Hausdorff, then the definition above is equivalent to the follow:
  If \(g : E \cup \{x_0\} \to Y\) is a function such that
  \[
    \forall x \in E \cup \{x_0\}, g(x) = \begin{dcases}
      L    & \text{if } x = x_0    \\
      f(x) & \text{if } x \neq x_0
    \end{dcases}
  \]
  and \((E \cup \{x_0\}, \mathcal{F}_{E \cup \{x_0\}})\) is a topological subspace induced by \((X, \mathcal{F}_X)\), then \(g\) is continuous at \(x_0\) from \((E \cup \{x_0\}, \mathcal{F}_{E \cup \{x_0\}})\) to \((Y, \mathcal{F}_Y)\).

  First suppose that \(f(x)\) converges to \(L\) in \(Y\) as \(x\) converges to \(x_0\) in \(E\).
  Let \(g\) be the function in the definition and let \(V \in \mathcal{F}_Y\) such that \(L \in V\).
  By hypothesis we know that
  \[
    \exists\ U \in \mathcal{F}_X : \begin{dcases}
      x_0 \in U \\
      f(U \cap E) \subseteq V
    \end{dcases}
  \]
  Then by \cref{2.5.7} we have \(U \cap (E \cup \{x_0\}) \in \mathcal{F}_{E \cup \{x_0\}}\) and
  \begin{align*}
    g\big(U \cap (E \cup \{x_0\})\big) & = g\big((U \cap E) \cup \{x_0\}\big)                      \\
                                       & = g\big((U \cap E) \setminus \{x_0\}\big) \cup g(\{x_0\}) \\
                                       & = f\big((U \cap E) \setminus \{x_0\}\big) \cup \{L\}      \\
                                       & \subseteq f(U \cap E) \cup \{L\}                          \\
                                       & \subseteq V.
  \end{align*}
  Since \(V\) is arbitrary, by \cref{2.5.8} we know that \(g\) is continuous at \(x_0\) from \((E \cup \{x_0\}, \mathcal{F}_{E \cup \{x_0\}})\) to \((Y, \mathcal{F}_Y)\).

  Next suppose that \(x_0 \in E\) and \(f(x)\) converges to \(L\) in \(Y\) as \(x\) converges to \(x_0\) in \(E\).
  We want to show that \(f(x_0) = L\).
  Suppose for sake of contradiction that \(f(x_0) \neq L\).
  Since \((Y, \mathcal{F}_Y)\) is Hausdorff, by \cref{ex:2.5.4} we know that
  \[
    \exists\ V, W \in \mathcal{F}_Y : \begin{dcases}
      L \in V      \\
      f(x_0) \in W \\
      V \cap W = \emptyset
    \end{dcases}
  \]
  By definition we have
  \[
    \exists\ U_V, U_W \in \mathcal{F}_X : \begin{dcases}
      x_0 \in U_V               \\
      x_0 \in U_W               \\
      f(U_V \cap E) \subseteq V \\
      f(U_W \cap E) \subseteq W
    \end{dcases}
  \]
  By \cref{2.5.1} we know that \(U_V \cap U_W \in \mathcal{F}_X\).
  But then we have
  \[
    \begin{dcases}
      x_0 \in U_V \cap U_W                                       \\
      f(U_V \cap U_W \cap E) \subseteq f(U_V \cap E) \subseteq V \\
      f(U_V \cap U_W \cap E) \subseteq f(U_W \cap E) \subseteq W
    \end{dcases}
  \]
  which means \(V \cap W \neq \emptyset\), a contradiction.
  Thus we have \(f(x_0) = L\).

  Now suppose that \(g\) is the function in the definition such that \(g\) is continuous at \(x_0\) from \((E \cup \{x_0\}, \mathcal{F}_{E \cup \{x_0\}})\) to \((Y, \mathcal{F}_Y)\).
  Also suppose that if \(x_0 \in E\), then \(f(x_0) = L\).
  Let \(V \in \mathcal{F}_Y\) such that \(g(x_0) = L \in V\).
  By \cref{2.5.8} we know that
  \[
    \exists\ U \in \mathcal{F}_{E \cup \{x_0\}} : \begin{dcases}
      x_0 \in U \\
      g(U) \subseteq V
    \end{dcases}
  \]
  By \cref{2.5.7} we know that
  \[
    \exists\ U_X \in \mathcal{F}_X : U_X \cap (E \cup \{x_0\}) = U.
  \]
  Since \(x_0 \in U\), we know that \(x_0 \in U_X\).
  Thus we have
  \begin{align*}
    f(U_X \cap E) & = f\big((U_X \cap E) \setminus \{x_0\}\big) \cup f(E \cap \{x_0\})  & (f(E \cap \{x_0\}) = \emptyset \iff x_0 \notin E) \\
                  & \subseteq g\big((U_X \cap E) \setminus \{x_0\}\big) \cup g(\{x_0\}) & (x_0 \in E \iff f(x_0) = L = g(x_0))              \\
                  & = g\big(U_X \cap (E \cup \{x_0\})\big)                                                                                  \\
                  & = g(U)                                                                                                                  \\
                  & \subseteq V.
  \end{align*}
  Since \(V\) is arbitrary, we conclude that \(f(x)\) converges to \(L\) in \(Y\) as \(x\) converges to \(x_0\) in \(E\).

  If \((Y, \mathcal{F}_Y)\) is not Hausdorff, then \(x_0 \in E\) may not implies \(f(x_0) = L\).
\end{proof}

\begin{ex}\label{ex:3.1.4}
  Recall from \cref{ex:2.5.5} that the extended real line \(\R^*\) comes with a standard topology (the order topology).
  We view the natural numbers \(\N\) as a subspace of this topological space, and \(+\infty\) as an adherent point of \(\N\) in \(\R^*\).
  Let \((a_n)_{n = 0}^\infty\) be a sequence taking values in a topological space \((Y, \mathcal{F}_Y)\), and let \(L \in Y\).
  Show that \(\lim_{n \to +\infty ; n \in \N} a_n = L\) (in the sense of \cref{ex:3.1.3}) if and only if \(\lim_{n \to \infty} a_n = L\) (in the sense of \cref{2.5.4}).
  This shows that the notions of limiting values of a sequence, and limiting values of a function, are compatible.
\end{ex}

\begin{proof}
  Let \((\R^*, \mathcal{F}_{\R^*})\) be the order topology in \cref{ex:2.5.5}.
  Let \(f : \N \to Y\) be the function where \(f(n) = a_n\) for each \(n \in \N\).
  First suppose that
  \[
    \lim_{n \to +\infty ; n \in \N} a_n = \lim_{n \to +\infty ; n \in \N} f(n) = L.
  \]
  By \cref{ex:3.1.3} we have
  \[
    \forall V \in \mathcal{F}_Y, L \in V \implies \exists\ U \in \mathcal{F}_{\R^*} : \begin{dcases}
      +\infty \in U \\
      f(U \cap \N) \subseteq V
    \end{dcases}
  \]
  Since \(U \in \mathcal{F}_{\R^*}\), by \cref{ex:2.5.5} we know that there exists an interval \(I \subseteq \R^*\) such that \(I \subseteq U\) and \(+\infty \in I\).
  We know that \(I\) must in the form \((a, +\infty]\) for some \(a \in \R\).
  By Archimedean property we know that there exists some \(N \in \N\) such that \(N > a\).
  Then we have
  \begin{align*}
             & \begin{dcases}
                 I \subseteq U \subseteq \R^*            \\
                 I = (a, +\infty] \in \mathcal{F}_{\R^*} \\
                 N > a
               \end{dcases}     \\
    \implies & \forall n \geq N + 1, n \in I               \\
    \implies & \forall n \geq N + 1, n \in U \cap \N       \\
    \implies & \forall n \geq N + 1, f(n) \in f(U \cap \N) \\
    \implies & \forall n \geq N + 1, f(n) \in V.
  \end{align*}
  Since this is true for arbitrary \(V\), we have
  \[
    \lim_{n \to \infty} a_n = \lim_{n \to \infty} f(n) = L
  \]
  in the sense of \cref{2.5.4}.

  Now suppose that
  \[
    \lim_{n \to \infty} a_n = \lim_{n \to \infty} f(n) = L
  \]
  in the sense of \cref{2.5.4}.
  Then we have
  \[
    \forall V \in \mathcal{F}_Y, L \in V \implies \exists\ N \in \N : \forall n \geq N, f(n) \in V.
  \]
  Let \(I = (N, +\infty]\).
  Then we know that \(I\) is an interval of \(\R^*\) and by \cref{ex:2.5.5} we have \(I \in \mathcal{F}_{\R^*}\).
  Observe that
  \begin{align*}
             & I \cap \N = \{m \in \N : m \geq N + 1\} \\
    \implies & \forall n \in I \cap \N, f(n) \in V     \\
    \implies & f(I \cap \N) \subseteq V.
  \end{align*}
  Since this is true for arbitrary \(V \in \mathcal{F}_Y\), by \cref{ex:3.1.3} we have
  \[
    \lim_{n \to +\infty ; n \in \N} a_n = \lim_{n \to +\infty ; n \in \N} f(n) = L.
  \]
\end{proof}

\begin{ex}\label{ex:3.1.5}
  Let \((X, d_X)\), \((Y, d_Y)\), \((Z, d Z)\) be metric spaces, and let \(x_0 \in X\), \(y_0 \in Y\), \(z_0 \in Z\).
  let \(E \subseteq X\) and let \(f : E \to Y\) and \(g : Y \to Z\) be functions.
  If we have \(\lim_{x \to x_0 ; x \in E} f(x) = y_0\) and \(\lim_{y \to y_0 ; y \in f(E)} g(y) = z_0\), conclude that \(\lim_{x \to x_0 ; x \in E} g \circ f(x) = z_0\).
\end{ex}

\begin{proof}
  By \cref{3.1.1} we have
  \begin{align*}
             & d_Z - \lim_{y \to y_0 ; y \in f(E)} g(y) = z_0                                                    \\
    \implies & \forall \varepsilon \in \R^+, \exists\ \delta' \in \R^+ :                                         \\
             & \Big(\forall y \in f(E), d_Y(y, y_0) < \delta' \implies d_Z\big(g(y), z_0\big) < \varepsilon\Big)
  \end{align*}
  and
  \begin{align*}
             & d_Y - \lim_{x \to x_0 ; x \in E} f(x) = y_0                                                \\
    \implies & \forall \delta' \in \R^+, \exists\ \delta \in \R^+ :                                       \\
             & \Big(\forall x \in E, d_X(x, x_0) < \delta \implies d_Y\big(f(x), y_0\big) < \delta'\Big).
  \end{align*}
  Thus we have
  \begin{align*}
     & \forall \varepsilon \in \R^+, \exists\ \delta \in \R^+ :                                                                                             \\
     & \bigg(\forall x \in E, d_X(x, x_0) < \delta \implies d_Y\big(f(x), y_0\big) < \delta' \implies d_Z\Big(g\big(f(x)\big), z_0\Big) < \varepsilon\bigg)
  \end{align*}
  and by \cref{3.1.1} \(d_Z - \lim_{x \to x_0 ; x \in E} g \circ f(x) = z_0\).
\end{proof}

\begin{ex}\label{ex:3.1.6}
  State and prove an analogue of the limit laws in Proposition 9.3.14 in Analysis I when \(X\) is now a metric space rather than a subset of \(\R\).
\end{ex}

\begin{proof}
  Let \((X, d)\) be a metric space, let \(d_1 = d_{l^1}|_{\R \times \R}\), let \(E \subseteq X\), let \(x_0 \in \overline{E}_{(X, d)}\), let \(f : X \to \R\) and \(g : X \to \R\) be functions, and let \(c \in \R\).
  Suppose that
  \begin{align*}
     & d_1 - \lim_{x \to x_0 ; x \in E} f(x) = L \\
     & d_1 - \lim_{x \to x_0 ; x \in E} g(x) = M
  \end{align*}
  We want to show that
  \begin{align*}
     & d_1 - \lim_{x \to x_0 ; x \in E} (f + g)(x) = L + M                                          \\
     & d_1 - \lim_{x \to x_0 ; x \in E} (f - g)(x) = L - M                                          \\
     & d_1 - \lim_{x \to x_0 ; x \in E} (fg)(x) = LM                                                \\
     & d_1 - \lim_{x \to x_0 ; x \in E} \min(f, g)(x) = \min(L, M)                                  \\
     & d_1 - \lim_{x \to x_0 ; x \in E} \max(f, g)(x) = \max(L, M)                                  \\
     & d_1 - \lim_{x \to x_0 ; x \in E} (cf)(x) = cL                                                \\
     & d_1 - \lim_{x \to x_0 ; x \in E} (f / g)(x) = L / M \text{ if } \begin{dcases}
                                                                         \forall x \in E, g(x) \neq 0 \\
                                                                         M \neq 0
                                                                       \end{dcases}
  \end{align*}
  Let \(f^* : E \cup \{x_0\} \to \R\) be the function
  \[
    \forall x \in E, f^*(x) = \begin{dcases}
      L    & \text{if } x = x_0    \\
      f(x) & \text{if } x \neq x_0
    \end{dcases}
  \]
  and let \(g^* : E \cup \{x_0\} \to \R\) be the function
  \[
    \forall x \in E, g^*(x) = \begin{dcases}
      M    & \text{if } x = x_0    \\
      g(x) & \text{if } x \neq x_0
    \end{dcases}
  \]
  By \cref{3.1.5}(c) we know that \(f^*\) and \(g^*\) are continuous at \(x_0\) from \((X, d)\) to \((\R, d_1)\).
  Thus by \cref{2.2.3} we have
  \begin{align*}
     & f^* + g^* \text{ is continuous at } x_0 \text{ from } (X, d) \text{ to } (\R, d_1)                                          \\
     & f^* - g^* \text{ is continuous at } x_0 \text{ from } (X, d) \text{ to } (\R, d_1)                                          \\
     & f^* g^* \text{ is continuous at } x_0 \text{ from } (X, d) \text{ to } (\R, d_1)                                            \\
     & \min(f^*, g^*) \text{ is continuous at } x_0 \text{ from } (X, d) \text{ to } (\R, d_1)                                     \\
     & \max(f^*, g^*) \text{ is continuous at } x_0 \text{ from } (X, d) \text{ to } (\R, d_1)                                     \\
     & c f^* \text{ is continuous at } x_0 \text{ from } (X, d) \text{ to } (\R, d_1)                                              \\
     & f^* / g^* \text{ is continuous at } x_0 \text{ from } (X, d) \text{ to } (\R, d_1) \text{ if } \begin{dcases}
                                                                                                        \forall x \in E, g(x) \neq 0 \\
                                                                                                        M \neq 0
                                                                                                      \end{dcases}
  \end{align*}
  Since
  \begin{align*}
    \forall x \in E \cup \{x_0\}, & (f^* + g^*)(x) = \begin{dcases}
                                                       L + M       & \text{if } x = x_0    \\
                                                       f(x) + g(x) & \text{if } x \neq x_0
                                                     \end{dcases}                     \\
                                  & (f^* - g^*)(x) = \begin{dcases}
                                                       L - M       & \text{if } x = x_0    \\
                                                       f(x) - g(x) & \text{if } x \neq x_0
                                                     \end{dcases}                     \\
                                  & (f^* g^*)(x) = \begin{dcases}
                                                     LM        & \text{if } x = x_0    \\
                                                     f(x) g(x) & \text{if } x \neq x_0
                                                   \end{dcases}                         \\
                                  & \min(f^*, g^*)(x) = \begin{dcases}
                                                          \min(L, M)               & \text{if } x = x_0    \\
                                                          \min\big(f(x), g(x)\big) & \text{if } x \neq x_0
                                                        \end{dcases}     \\
                                  & \max(f^*, g^*)(x) = \begin{dcases}
                                                          \max(L, M)               & \text{if } x = x_0    \\
                                                          \max\big(f(x), g(x)\big) & \text{if } x \neq x_0
                                                        \end{dcases}     \\
                                  & (c f^*)(x) = \begin{dcases}
                                                   cL     & \text{if } x = x_0    \\
                                                   c f(x) & \text{if } x \neq x_0
                                                 \end{dcases}                              \\
                                  & (f^* / g^*)(x) = \begin{dcases}
                                                       L / M       & \text{if } x = x_0    \\
                                                       f(x) / g(x) & \text{if } x \neq x_0
                                                     \end{dcases} \text{ when } \begin{dcases}
                                                                                  \forall x \in E, g(x) \neq 0 \\
                                                                                  M \neq 0
                                                                                \end{dcases}
  \end{align*}
  by \cref{3.1.5}(a)(d) we know that
  \begin{align*}
     & d_1 - \lim_{x \to x_0 ; x \in E} (f + g)(x) = L + M                                          \\
     & d_1 - \lim_{x \to x_0 ; x \in E} (f - g)(x) = L - M                                          \\
     & d_1 - \lim_{x \to x_0 ; x \in E} (fg)(x) = LM                                                \\
     & d_1 - \lim_{x \to x_0 ; x \in E} \min(f, g)(x) = \min(L, M)                                  \\
     & d_1 - \lim_{x \to x_0 ; x \in E} \max(f, g)(x) = \max(L, M)                                  \\
     & d_1 - \lim_{x \to x_0 ; x \in E} (cf)(x) = cL                                                \\
     & d_1 - \lim_{x \to x_0 ; x \in E} (f / g)(x) = L / M \text{ if } \begin{dcases}
                                                                         \forall x \in E, g(x) \neq 0 \\
                                                                         M \neq 0
                                                                       \end{dcases}
  \end{align*}
\end{proof}
\section{Pointwise and uniform convergence}\label{ii:sec:3.2}

\begin{defn}[Pointwise convergence]\label{ii:3.2.1}
  Let \((f^{(n)})_{n = 1}^\infty\) be a sequence of functions from one metric space \((X, d_X)\) to another \((Y, d_Y)\), and let \(f : X \to Y\) be another function.
  We say that \emph{\((f^{(n)})_{n = 1}^\infty\) converges pointwise to \(f\) on \(X\)} if we have
  \[
    \lim_{n \to \infty} f^{(n)}(x) = f(x)
  \]
  for all \(x \in X\), i.e.,
  \[
    \lim_{n \to \infty} d_Y\big(f^{(n)}(x), f(x)\big) = 0.
  \]
  Or in other words, for every \(x\) and every \(\varepsilon > 0\) there exists \(N > 0\) such that \(d_Y\big(f^{(n)}(x), f(x)\big) < \varepsilon\) for every \(n > N\).
  We call the function \(f\) the \emph{pointwise limit} of the functions \(f^{(n)}\).
\end{defn}

\begin{rmk}\label{ii:3.2.2}
  Note that \(f^{(n)}(x)\) and \(f(x)\) are points in \(Y\), rather than functions, so we are using our prior notion of convergence in metric spaces to determine convergence of functions.
  Also note that we are not really using the fact that \((X, d_X)\) is a metric space
  (i.e., we are not using the metric \(d_X\));
  for this definition it would suffice for \(X\) to just be a plain old set with no metric structure.
  However, later on we shall want to restrict our attention to \emph{continuous} functions from \(X\) to \(Y\), and in order to do so we need a metric on \(X\) (and on \(Y\)), or at least a topological structure.
  Also when we introduce the concept of \emph{uniform convergence}, then we will definitely need a metric structure on \(X\) and \(Y\);
  there is no comparable notion for topological spaces.
\end{rmk}

\begin{note}
  From \cref{ii:1.1.20} we see that a sequence \((f^{(n)})_{n = 1}^\infty\) of functions from one metric space \((X, d_X)\) to another \((Y, d_Y)\) can have at most one pointwise limit \(f\)
  (this explains why we can refer to \(f\) as \emph{the} pointwise limit).
  However, it is of course possible for a sequence of functions to have no pointwise limit, just as a sequence of points in a metric space do not necessarily have a limit.
\end{note}

\begin{note}
  Pointwise convergence is a very natural concept, but it has a number of disadvantages:
  it does not preserve continuity, derivatives, limits, or integrals.
  The problem is that while \(f^{(n)}(x)\) converges to \(f(x)\) for each \(x\), the \emph{rate} of that convergence varies substantially with \(x\).
  To put things another way, the convergence of \(f^{(n)}\) to \(f\) is not \emph{uniform} in \(x\)
  - the \(N\) that one needs to get \(f^{(n)}(x)\) within \(\varepsilon\) of \(f\) depends on \(x\) as well as on \(\varepsilon\).
  This motivates a stronger notion of convergence.
\end{note}

\setcounter{thm}{6}
\begin{defn}[Uniform convergence]\label{ii:3.2.7}
  Let \((f^{(n)})_{n = 1}^\infty\) be a sequence of functions from one metric space \((X, d_X)\) to another \((Y, d_Y)\), and let \(f : X \to Y\) be another function.
  We say that \emph{\((f^{(n)})_{n = 1}^\infty\) converges uniformly to \(f\) on \(X\)} if for every \(\varepsilon > 0\) there exists \(N > 0\) such that \(d_Y\big(f^{(n)}(x), f(x)\big) < \varepsilon\) for every \(n > N\) and \(x \in X\).
  We call the function \(f\) the \emph{uniform limit} of the functions \(f^{(n)}\).
\end{defn}

\begin{rmk}\label{ii:3.2.8}
  Note that \cref{ii:3.2.7} is subtly different from the definition for pointwise convergence in \cref{ii:3.2.1}.
  In the definition of pointwise convergence, \(N\) was allowed to depend on \(x\);
  now it is not.
  The reader should compare this distinction to the distinction between continuity and uniform continuity
  (i.e., between \cref{ii:2.1.1} and \cref{ii:2.3.4}).
\end{rmk}

\begin{note}
  If \(f^{(n)}\) converges uniformly to \(f\) on \(X\), then it also converges pointwise to the same function \(f\).
  Thus, when the uniform limit and pointwise limit both exist, then they have to be equal.
  However, the converse is not true.
\end{note}

\begin{note}
  If a sequence \(f^{(n)} : X \to Y\) of functions converges pointwise (or uniformly) to a function \(f : X \to Y\), then the restrictions \(f^{(n)}|_E : E \to Y\) of \(f^{(n)}\) to some subset \(E\) of \(X\) will also converge pointwise (or uniformly) to \(f|_E\).
\end{note}

\exercisesection

\begin{ex}\label{ii:ex:3.2.1}
  The purpose of this exercise is to demonstrate a concrete relationship between continuity and pointwise convergence, and between uniform continuity and uniform convergence.
  Let \(f : \R \to \R\) be a function.
  For any \(a \in \R\), let \(f_a : \R \to \R\) be the shifted function \(f_a(x) \coloneqq f(x - a)\).
  \begin{enumerate}
    \item Show that \(f\) is continuous iff, whenever \((a_n)_{n = 0}^\infty\) is a sequence of real numbers which converges to zero, the shifted functions \(f_{a_n}\) converge pointwise to \(f\).
    \item Show that \(f\) is uniformly continuous iff, whenever \((a_n)_{n = 0}^\infty\) is a sequence of real numbers which converges to zero, the shifted functions \(f_{a_n}\) converge uniformly to \(f\).
  \end{enumerate}
\end{ex}

\begin{proof}{(a)}
  Suppose that \(f\) is continuous on \(\R\).
  Let \((a_n)_{n = 0}^\infty\) be a sequence in \(\R\) such that \(\lim_{n \to \infty} a_n = 0\).
  Let \(x_0 \in \R\).
  Then we have
  \begin{align*}
             & \lim_{n \to \infty} a_n = 0                                                                                                          \\
    \implies & \lim_{n \to \infty} (x_0 - a_n) = x_0                                                                                                \\
    \implies & \lim_{n \to \infty} f_{a_n}(x_0) = \lim_{n \to \infty} f(x_0 - a_n) = f(x_0). &  & \text{(\(f\) is continuous at \(x_0\) on \(\R\))}
  \end{align*}
  Since \(x_0\) was arbitrary, by \cref{ii:3.2.1} we know that \((f_{a_n})_{n = 0}^\infty\) converges pointwise to \(f\) on \(\R\) with respect to \(d_{l^1}|_{\R \times \R}\).
  Since \((a_n)_{n = 0}^\infty\) was arbitrary, we conclude that if \((a_n)_{n = 0}^\infty\) is a sequence in \(\R\) such that \(\lim_{n \to \infty} a_n = 0\), then \((f_{a_n})_{n = 0}^\infty\) converges pointwise to \(f\) on \(\R\) with respect to \(d_{l^1}|_{\R \times \R}\).

  Now suppose that if \((a_n)_{n = 0}^\infty\) is a sequence in \(\R\) such that \(\lim_{n \to \infty} a_n = 0\), then \((f_{a_n})_{n = 0}^\infty\) converges pointwise to \(f\) on \(\R\) with respect to \(d_{l^1}|_{\R \times \R}\).
  Suppose for sake of contradiction that there exists some \(x_0 \in \R\) such that \(\lim_{x \to x_0 ; x \in \R} f(x) \neq f(x_0)\).
  Then we have
  \[
    \exists \varepsilon \in \R^+ : \forall \delta \in \R^+, \exists x \in \R : \begin{dcases}
      \abs{x - x_0} < \delta \\
      \abs{f(x) - f(x_0)} > \varepsilon
    \end{dcases}
  \]
  Fix such \(\varepsilon\).
  We choose a sequence \((a_n)_{n = 0}^\infty\) in \(\R\) such that \(\abs{a_n - x_0} < \dfrac{1}{n + 1}\) for all \(n \in \N\).
  Then we have \(\lim_{n \to \infty} \abs{a_n - x_0} = \lim_{n \to \infty} (a_n - x_0) = 0\).
  By hypothesis we have
  \[
    \forall x \in \R, f(x) = \lim_{n \to \infty} f_{a_n - x_0}(x) = \lim_{n \to \infty} f(x - a_n + x_0).
  \]
  But this means
  \[
    \begin{dcases}
      \abs{x - a_n + x_0 - x} = \abs{-a_n + x_0} < \dfrac{1}{n} \\
      \abs{f(x - a_n + x_0) - f(x)} < \varepsilon
    \end{dcases}
  \]
  a contradiction.
  Thus, we have \(\lim_{x \to x_0 ; x \in \R} f(x) = f(x_0)\) for every \(x_0 \in \R\) and \(f\) is continuous on \(\R\).
\end{proof}

\begin{proof}{(b)}
  Suppose that \(f\) is uniformly continuous on \(\R\).
  Let \((a_n)_{n = 0}^\infty\) be a sequence in \(\R\) such that \(\lim_{n \to \infty} a_n = 0\).
  Since \(f\) is uniformly continuous on \(\R\), we have
  \[
    \forall \varepsilon \in \R^+, \exists \delta \in \R^+ : \forall x_1, x_2 \in \R, \abs{x_1 - x_2} < \delta \implies \abs{f(x_1) - f(x_2)} < \varepsilon.
  \]
  Fix one pair of \(\varepsilon\) and \(\delta\).
  Then we have
  \begin{align*}
             & \exists N \in \N : \forall n \geq N, \abs{a_n} < \delta                                       \\
    \implies & \exists N \in \N : \forall n \geq N, \abs{-a_n} < \delta                                      \\
    \implies & \exists N \in \N : \forall n \geq N, \forall x \in \R, \abs{x - a_n - x} < \delta             \\
    \implies & \exists N \in \N : \forall n \geq N, \forall x \in \R, \abs{f(x - a_n) - f(x)} < \varepsilon  \\
    \implies & \exists N \in \N : \forall n \geq N, \forall x \in \R, \abs{f_{a_n}(x) - f(x)} < \varepsilon.
  \end{align*}
  Since this is true for arbitrary \(\varepsilon\), by \cref{ii:3.2.7} we know that \((f_{a_n})_{n = 0}^\infty\) converges uniformly to \(f\) on \(\R\) with respect to \(d_{l^1}|_{\R \times \R}\).
  Since \((a_n)_{n = 0}^\infty\) was arbitrary, we conclude that if \((a_n)_{n = 0}^\infty\) is a sequence in \(\R\) such that \(\lim_{n \to \infty} a_n = 0\), then \((f_{a_n})_{n = 0}^\infty\) uniformly converges to \(f\) on \(\R\) with respect to \(d_{l^1}|_{\R \times \R}\).

  Now suppose that if \((a_n)_{n = 0}^\infty\) is a sequence in \(\R\) such that \(\lim_{n \to \infty} a_n = 0\), then \((f_{a_n})_{n = 0}^\infty\) uniformly converges to \(f\) on \(\R\) with respect to \(d_{l^1}|_{\R \times \R}\).
  Suppose for sake of contradiction that \(f\) is not uniformly continuous on \(\R\).
  Then we have
  \[
    \exists \varepsilon \in \R^+ : \forall \delta \in \R^+, \exists x_1, x_2 \in \R : \begin{dcases}
      \abs{x_1 - x_2} < \delta \\
      \abs{f(x_1) - f(x_2)} > \varepsilon
    \end{dcases}
  \]
  Let \((a_n)_{n = 0}^\infty\) be a sequence in \(\R\) such that \(\lim_{n \to \infty} \abs{a_n} = 0\).
  Then we have
  \begin{align*}
             & \lim_{n \to \infty} \abs{a_n} = 0 = \lim_{n \to \infty} a_n = \lim_{n \to \infty} -a_n                                 \\
    \implies & \forall \delta \in \R^+, \exists N \in \N : \forall n \geq N, \abs{-a_n} < \delta                                      \\
    \implies & \forall \delta \in \R^+, \exists N \in \N : \forall n \geq N, \forall x \in \R, \abs{x - a_n - x} < \delta             \\
    \implies & \forall \delta \in \R^+, \exists N \in \N : \forall n \geq N, \forall x \in \R, \abs{f(x - a_n) - f(x)} > \varepsilon  \\
    \implies & \forall \delta \in \R^+, \exists N \in \N : \forall n \geq N, \forall x \in \R, \abs{f_{a_n}(x) - f(x)} > \varepsilon.
  \end{align*}
  But by hypothesis we know that \((f_{a_n})_{n = 0}^\infty\) uniformly converges to \(f\) on \(\R\), which by \cref{ii:3.2.7} means
  \[
    \exists N' \in \N : \forall n \geq N', \forall x \in \R, \abs{f_{a_n}(x) - f(x)} < \varepsilon,
  \]
  a contradiction.
  Thus, \(f\) is uniformly continuous on \(\R\).
\end{proof}

\begin{ex}\label{ii:ex:3.2.2}
  \quad
  \begin{enumerate}
    \item Let \((f^{(n)})_{n = 1}^\infty\) be a sequence of functions from one metric space \((X, d_X)\) to another \((Y, d_Y)\), and let \(f : X \to Y\) be another function from \(X\) to \(Y\).
          Show that if \(f^{(n)}\) converges uniformly to \(f\), then \(f^{(n)}\) also converges pointwise to \(f\).
    \item For each integer \(n \geq 1\), let \(f^{(n)} : (-1, 1) \to \R\) be the function \(f^{(n)}(x) \coloneqq x^n\).
          Prove that \(f^{(n)}\) converges pointwise to the zero function \(0\), but does not converge uniformly to any function \(f : (-1, 1) \to \R\).
    \item Let \(g : (-1, 1) \to \R\) be the function \(g(x) \coloneqq x / (1 - x)\).
          With the notation as in (b), show that the partial sums \(\sum_{n = 1}^N f^{(n)}\) converges pointwise as \(N \to \infty\) to \(g\), but does not converge uniformly to \(g\), on the open interval \((-1, 1)\).
          What would happen if we replaced the open interval \((-1, 1)\) with the closed interval \([-1, 1]\)?
  \end{enumerate}
\end{ex}

\begin{proof}{(a)}
  Let \(x_0 \in X\).
  Since
  \begin{align*}
             & (f^{(n)})_{n = 1}^\infty \text{ converges uniformly to } f \text{ on } X                           \\
             & \text{ with respect to } d_Y                                                                       \\
    \implies & \forall \varepsilon \in \R^+, \exists N \in \Z^+ :                                                 \\
             & \forall n \geq N, \forall x \in X, d_Y\big(f^{(n)}(x), f(x)\big) < \varepsilon &  & \by{ii:3.2.7}  \\
    \implies & \forall \varepsilon \in \R^+, \exists N \in \Z^+ :                                                 \\
             & \forall n \geq N, d_Y\big(f^{(n)}(x_0), f(x_0)\big) < \varepsilon                                  \\
    \implies & \lim_{n \to \infty} d_Y\big(f^{(n)}(x_0), f(x_0)\big) = 0                      &  & \by{ii:1.1.14}
  \end{align*}
  and \(x_0\) was arbitrary, by \cref{ii:3.2.1} we know that \((f^{(n)})_{n = 1}^\infty\) converges pointwise to \(f\) on \(X\) with respect to \(d_Y\).
\end{proof}

\begin{proof}{(b)}
  Let \(z : (-1, 1) \to \R\) be the zero function, i.e., \(z(x) = 0\) for each \(x \in (-1, 1)\).
  Then we have
  \begin{align*}
             & \forall x \in (-1, 1), \lim_{n \to \infty} x^n = 0                                                \\
    \implies & \forall x \in (-1, 1), \lim_{n \to \infty} f^{(n)}(x) = 0 = z(x)                                  \\
    \implies & (f^{(n)})_{n = 1}^\infty \text{ converges pointwise to } z \text{ on } (-1, 1)                    \\
             & \text{with respect to } d_{l^1}|_{\R \times \R}.                               &  & \by{ii:3.2.1}
  \end{align*}
  Suppose for sake of contradiction that there exists a \(f : (-1, 1) \to \R\) such that \((f^{(n)})_{n = 1}^\infty\) converges uniformly to \(f\) on \((-1, 1)\) with respect to \(d_{l^1}|_{\R \times \R}\).
  By \cref{ii:ex:3.2.2}(a) and \cref{ii:1.1.20} we know that \(f = z\).
  Then we have
  \begin{align*}
             & \forall \varepsilon \in \R^+, \exists N \in \Z^+ :                                                \\
             & \forall n \geq N, \forall x \in (-1, 1), \abs{f^{(n)}(x) - z(x)} < \varepsilon &  & \by{ii:3.2.7} \\
    \implies & \forall \varepsilon \in \R^+, \exists N \in \Z^+ :                                                \\
             & \forall n \geq N, \forall x \in (-1, 1), \abs{x^n} < \varepsilon.
  \end{align*}
  Now consider \(\varepsilon = \dfrac{1}{2}\).
  Then we have
  \begin{align*}
             & \exists N \in \Z^+ : \forall n \geq N, \forall x \in (-1, 1), \abs{x^n} < \dfrac{1}{2}                                               \\
    \implies & \exists N \in \Z^+ : \begin{dcases}
                                      (\dfrac{1}{2})^{\dfrac{1}{n}} \in (-1, 1) \\
                                      \abs{\big((\dfrac{1}{2})^{\dfrac{1}{n}}\big)^n} = \dfrac{1}{2} < \dfrac{1}{2}
                                    \end{dcases}
  \end{align*}
  a contradiction.
  Thus, \((f^{(n)})_{n = 1}^\infty\) does not converge uniformly to any \(f\) on \((-1, 1)\) with respect to \(d_{l^1}|_{\R \times \R}\).
\end{proof}

\begin{proof}{(c)}
  By Lemma 7.3.3 in Analysis I we have
  \begin{align*}
    \forall x \in (-1, 1), \lim_{N \to \infty} \sum_{n = 1}^N f^{(n)}(x) & = \lim_{N \to \infty} \sum_{n = 1}^N x^n                 \\
                                                                         & = \lim_{N \to \infty} \bigg(\sum_{n = 0}^N x^n - 1\bigg) \\
                                                                         & = \bigg(\lim_{N \to \infty} \sum_{n = 0}^N x^n\bigg) - 1 \\
                                                                         & = \dfrac{1}{1 - x} - 1                                   \\
                                                                         & = \dfrac{x}{1 - x}                                       \\
                                                                         & = g(x).
  \end{align*}
  Thus, by \cref{ii:3.2.1} we know that \((\sum_{n = 1}^N f^{(n)})_{N = 1}^\infty\) converges pointwise to \(g\) on \((-1, 1)\) with respect to \(d_{l^1}|_{\R \times \R}\).
  Suppose for sake of contradiction that \((\sum_{n = 1}^N f^{(n)})_{N = 1}^\infty\) converges uniformly to \(g\) on \((-1, 1)\) with respect to \(d_{l^1}|_{\R \times \R}\).
  Then by \cref{ii:3.2.7} we know that
  \begin{align*}
             & \forall \varepsilon \in \R^+, \exists M \in \Z^+ : \forall N \geq M, \forall x \in (-1, 1), \abs{\sum_{n = 1}^N f^{(n)}(x) - g(x)} < \varepsilon                            \\
    \implies & \forall \varepsilon \in \R^+, \exists M \in \Z^+ : \forall N \geq M, \forall x \in (-1, 1), \abs{\dfrac{x(1 - x^N)}{1 - x} - \dfrac{x}{1 - x}} < \varepsilon                \\
    \implies & \forall \varepsilon \in \R^+, \exists M \in \Z^+ : \forall N \geq M, \forall x \in (-1, 1), \abs{\dfrac{-x^{N + 1}}{1 - x}} = \abs{\dfrac{x^{N + 1}}{1 - x}} < \varepsilon.
  \end{align*}
  Now consider \(\varepsilon = \dfrac{1}{2}\).
  Then we have
  \begin{align*}
             & \exists N \geq M : \forall n \geq N, \forall x \in (-1, 1), \abs{\dfrac{x^{N + 1}}{1 - x}} < \dfrac{1}{2}                                                               \\
    \implies & \exists N \geq M : \begin{dcases}
                                    (\dfrac{1}{2})^{\dfrac{1}{N + 1}} \in (-1, 1) \\
                                    \abs{\dfrac{\big((\dfrac{1}{2})^{\dfrac{1}{N + 1}}\big)^{N + 1}}{1 - (\dfrac{1}{2})^{\dfrac{1}{N + 1}}}} = \dfrac{\dfrac{1}{2}}{1 - (\dfrac{1}{2})^{\dfrac{1}{N + 1}}} < \dfrac{1}{2}
                                  \end{dcases}
  \end{align*}
  But we know that
  \begin{align*}
             & (\dfrac{1}{2})^{\dfrac{1}{N + 1}} > \dfrac{1}{2}                        \\
    \implies & 1 - (\dfrac{1}{2})^{\dfrac{1}{N + 1}} < 1 - \dfrac{1}{2} = \dfrac{1}{2} \\
    \implies & \dfrac{\dfrac{1}{2}}{1 - (\dfrac{1}{2})^{\dfrac{1}{N + 1}}} > 1,
  \end{align*}
  a contradiction.
  Thus, \((\sum_{n = 1}^N f^{(n)})_{N = 1}^\infty\) does not converge uniformly to \(g\) on \((-1, 1)\) with respect to \(d_{l^1}|_{\R \times \R}\).

  If we replace \((-1, 1)\) with \([-1, 1]\), then by Lemma 7.3.3 in Analysis I \(\sum_{n = 1}^\infty 1\) and \(\sum_{n = 1}^\infty -1\) does not converge, thus by \cref{ii:3.2.1} \((\sum_{n = 1}^N f^{(n)})_{N = 1}^\infty\) does not converge pointwise to \(g\) on \([-1, 1]\) with respect to \(d_{l^1}|_{\R \times \R}\).
\end{proof}

\begin{ex}\label{ii:ex:3.2.3}
  Let \((X, d_X)\) a metric space, and for every integer \(n \geq 1\), let \(f_n : X \to \R\) be a real-valued function.
  Suppose that \(f_n\) converges pointwise to another function \(f : X \to \R\) on \(X\)
  (in this question we give \(\R\) the standard metric \(d(x, y) = \abs{x - y}\)).
  Let \(h : \R \to \R\) be a continuous function.
  Show that the functions \(h \circ f_n\) converge pointwise to \(h \circ f\) on \(X\), where \(h \circ f_n : X \to \R\) is the function \(h \circ f_n(x) \coloneqq h\big(f_n(x)\big)\), and similarly for \(h \circ f\).
\end{ex}

\begin{proof}
  Let \(x_0 \in X\).
  We have
  \begin{align*}
             & h \text{ is continuous on } \R \text{ with respect to } d                                      \\
    \implies & h \text{ is continuous at } x_0 \text{ with respect to } d                                     \\
    \implies & \forall \varepsilon \in \R^+, \exists \delta \in \R^+ :                                        \\
             & \big(\forall x \in \R, \abs{x - x_0} < \delta \implies \abs{h(x) - h(x_0)} < \varepsilon\big).
  \end{align*}
  Fix one pair of \(\varepsilon\) and \(\delta\).
  Then we have
  \begin{align*}
             & (f_n)_{n = 1}^\infty \text{ converges pointwise to } f \text{ on } X \text{ with respect to } d                        \\
    \implies & \lim_{n \to \infty} \abs{f_n(x_0) - f(x_0)} = 0                                                     &  & \by{ii:3.2.1} \\
    \implies & \exists N \in \Z^+ : \forall n \geq N, \abs{f_n(x_0) - f(x_0)} < \delta                                                \\
    \implies & \exists N \in \Z^+ : \forall n \geq N, \abs{h\big(f_n(x_0)\big) - h\big(f(x_0)\big)} < \varepsilon.
  \end{align*}
  Since \(\varepsilon\) was arbitrary, by \cref{ii:1.1.14} we know that
  \[
    \lim_{n \to \infty} \abs{h\big(f_n(x_0)\big) - h\big(f(x_0)\big)} = \lim_{n \to \infty} \abs{(h \circ f_n)(x_0) - (h \circ f)(x_0)} = 0.
  \]
  Since \(x_0\) was arbitrary, by \cref{ii:3.2.1} we know \((h \circ f_n)_{n = 1}^\infty\) converges pointwise to \(h \circ f\) on \(X\) with respect to \(d_{l^1}|_{\R \times \R}\).
\end{proof}

\begin{ex}\label{ii:ex:3.2.4}
  Let \(f_n : X \to Y\) be a sequence of bounded functions from one metric space \((X, d_X)\) to another metric space \((Y, d_Y)\).
  Suppose that \(f_n\) converges uniformly to another function \(f : X \to Y\).
  Suppose that \(f\) is a bounded function;
  i.e., there exists a ball \(B_{(Y, d_Y)}(y_0, R)\) in \(Y\) such that \(f(x) \in B_{(Y, d_Y)}(y_0, R)\) for all \(x \in X\).
  Show that the sequence \(f_n\) is \emph{uniformly bounded};
  i.e., there exists a ball \(B_{(Y, d_Y)}(y_0, R)\) in \(Y\) such that \(f_n(x) \in B_{(Y, d_Y)}(y_0, R)\) for all \(x \in X\) and all positive integers \(n\).
\end{ex}

\begin{proof}
  Since \(f\) is bounded, by \cref{ii:1.5.3} we have
  \[
    \forall y \in Y, \exists \varepsilon \in \R^+ : f(X) \subseteq B_{(Y, d_Y)}(y, \varepsilon).
  \]
  We choose one pair of \(y\) and \(\varepsilon\).
  Since \((f_n)_{n = 1}^\infty\) converges uniformly to \(f\) on \(X\) with respect to \(d_Y\), we have
  \begin{align*}
             & \exists N \in \Z^+ : \forall n \geq N, \forall x \in X, d_Y\big(f_n(x), f(x)\big) < \varepsilon          &  & \by{ii:3.2.7}    \\
    \implies & \exists N \in \Z^+ : \forall n \geq N, \forall x \in X,                                                                        \\
             & d_Y\big(f_n(x), y\big) \leq d_Y\big(f_n(x), f(x)\big) + d_Y\big(f(x), y\big) < \varepsilon + \varepsilon &  & \by{ii:1.1.2}[d] \\
    \implies & \exists N \in \Z^+ : \forall n \geq N, \forall x \in X, f_n(x) \in B_{(Y, d_Y)}\big(y, 2\varepsilon\big)                       \\
    \implies & \exists N \in \Z^+ : \forall n \geq N, f_n(X) \subseteq B_{(Y, d_Y)}\big(y, 2\varepsilon\big)
  \end{align*}
  Now fix \(N\).
  Let \(S = \set{n \in \Z^+ : n < N}\).
  Then \(S\) is finite.
  If \(S = \emptyset\), then we have \(N = 1\) and thus by definition \((f_n)_{n = 1}^\infty\) is uniformly bounded.
  So suppose that \(S \neq \emptyset\).
  By hypothesis we know that \(f_n\) is bounded for each \(n \in S\), thus by \cref{ii:1.5.3} we have
  \[
    \forall n \in S, \exists \delta_n \in \R^+ : f_n(X) \subseteq B_{(Y, d_Y)}(y, \delta_n).
  \]
  We choose one \(\delta_n\) for each \(n \in S\).
  Since \(S\) is finite, we know that \(\delta_{\max} = \max\set{\delta_n : n \in S}\) is well-defined.
  Then we have
  \begin{align*}
             & \begin{dcases}
                 \forall n \in S, f_n(X) \subseteq B_{(Y, d_Y)}(y, \delta_{\max}) \\
                 \forall n \geq N, f_n(X) \subseteq B_{(Y, d_Y)}(y, 2\varepsilon)
               \end{dcases}                    \\
    \implies & \forall n \in \Z^+, f_n(X) \subseteq B_{(Y, d_Y)}(y, 2\varepsilon + \delta_{\max}).
  \end{align*}
  Since \(y\) was arbitrary, by \cref{ii:1.5.3} we know that \(f_n\) is bounded for each \(n \in \Z^+\), i.e.,
  \[
    \forall y \in Y, \exists r \in \R^+ : \forall n \in \Z^+, f_n(X) \subseteq B_{(Y, d_Y)}(y, r).
  \]
  And by definition \((f_n)_{n = 1}^\infty\) is uniformly bounded.
\end{proof}

\section{Uniform convergence and continuity}\label{sec:3.3}

\begin{thm}[Uniform limits preserve continuity I]\label{3.3.1}
  Suppose \((f^{(n)})_{n = 1}^\infty\) is a sequence of functions from one metric space \((X, d_X)\) to another \((Y, d_Y)\), and suppose that this sequence converges uniformly to another function \(f : X \to Y\).
  Let \(x_0\) be a point in \(X\).
  If the functions \(f^{(n)}\) are continuous at \(x_0\) for each \(n\), then the limiting function \(f\) is also continuous at \(x_0\).
\end{thm}

\begin{proof}
  We have
  \begin{align*}
             & (f^{(n)})_{n = 1}^\infty \text{ converges uniformly to } f \text{ on } X                                   \\
             & \text{with respect to } d_Y                                                                                \\
    \implies & \forall \varepsilon \in \R^+, \exists N \in \Z^+ :                                                         \\
             & \forall n \geq N, \forall x \in X, d_Y\big(f^{(n)}(x), f(x)\big) < \dfrac{\varepsilon}{3}. &  & \by{3.2.7}
  \end{align*}
  We choose one pair of \(\varepsilon\) and \(N\).
  For each \(n \in \Z^+\), since \(f^{(n)}\) is continuous at \(x_0\) from \((X, d_X)\) to \((Y, d_Y)\), by \cref{2.1.1} we have
  \begin{align*}
             & \forall n \geq N, f^{(n)} \text{ is continuous at } x_0 \text{ from } (X, d_X) \text{ to } (Y, d_Y)                                                                                            \\
    \implies & \forall n \geq N, \exists \delta \in \R^+ :                                                                                                                                                    \\
             & \Big(\forall x \in X, d_X(x, x_0) < \delta \implies d_Y\big(f^{(n)}(x), f^{(n)}(x_0)\big) < \dfrac{\varepsilon}{3}\Big)                                                                        \\
    \implies & \forall n \geq N, \exists \delta \in \R^+ :                                                                                                                                                    \\
             & \Big(\forall x \in X, d_X(x, x_0) < \delta \implies d_Y\big(f(x), f(x_0)\big)                                                                                                                  \\
             & \leq d_Y\big(f(x), f^{(n)}(x)\big) + d_Y\big(f^{(n)}(x), f^{(n)}(x_0)\big) + d_Y\big(f^{(n)}(x_0), f(x_0)\big) < \dfrac{\varepsilon}{3} + \dfrac{\varepsilon}{3} + \dfrac{\varepsilon}{3}\Big) \\
    \implies & \forall n \geq N, \exists \delta \in \R^+ :                                                                                                                                                    \\
             & \Big(\forall x \in X, d_X(x, x_0) < \delta \implies d_Y\big(f(x), f(x_0)\big) < \varepsilon.
  \end{align*}
  Since \(\varepsilon\) is arbitrary, by \cref{2.1.1} we know that \(f\) is continuous at \(x_0\) from \((X, d_X)\) to \((Y, d_Y)\).
\end{proof}

\begin{cor}[Uniform limits preserve continuity II]\label{3.3.2}
  Let \((f^{(n)})_{n = 1}^\infty\) be a sequence of functions from one metric space \((X, d_X)\) to another \((Y, d_Y)\), and suppose that this sequence converges uniformly to another function \(f : X \to Y\).
  If the functions \(f^{(n)}\) are continuous on \(X\) for each \(n\), then the limiting function \(f\) is also continuous on \(X\).
\end{cor}

\begin{proof}
  By applying \cref{3.3.1} to each \(x \in X\) we conclude that \(f\) is continuous on \(X\) from \((X, d_X)\) to \((Y, d_Y)\).
\end{proof}

\begin{prop}[Interchange of limits and uniform limits]\label{3.3.3}
  Let
  \((X, d_X)\) and \((Y, d_Y)\) be metric spaces, with \(Y\) complete, and let \(E\) be a subset of \(X\).
  Let \((f^{(n)})_{n = 1}^\infty\) be a sequence of functions from \(E\) to \(Y\), and suppose that this sequence converges uniformly in \(E\) to some function \(f : E \to Y\).
  Let \(x_0 \in X\) be an adherent point of \(E\), and suppose that for each \(n\) the limit \(\lim_{x \to x_0 ; x \in E} f^{(n)}(x)\) exists.
  Then the limit \(\lim_{x \to x_0 ; x \in E} f(x)\) also exists, and is equal to the limit of the sequence \(\big(\lim_{x \to x_0 ; x \in E} f^{(n)}(x)\big)_{n = 1}^\infty\);
  in other words we have the interchange of limits
  \[
    \lim_{n \to \infty} \lim_{x \to x_0 ; x \in E} f^{(n)}(x) = \lim_{x \to x_0 ; x \in E} \lim_{n \to \infty} f^{(n)}(x).
  \]
\end{prop}

\begin{proof}
  For each \(n \in \Z^+\), we define \(d_Y - \lim_{x \to x_0 ; x \in E} f^{(n)}(x) = L^{(n)}\).
  We claim that the sequence \((L^{(n)})_{n = 1}^\infty\) converges in \(Y\) with respect to \(d_Y\).
  Since \((Y, d_Y)\) is complete, by \cref{1.4.10} it suffices to show that \((L^{(n)})_{n = 1}^\infty\) is a Cauchy sequence in \((Y, d_Y)\).
  Let \(n_1, n_2 \in \Z^+\).
  Then by \cref{3.2.7} we have
  \begin{align*}
             & (f^{(n)})_{n = 1}^\infty \text{ converges uniformly to } f \text{ on } X \text{ with respect to } d_Y                                        \\
    \implies & \forall \varepsilon \in \R^+, \exists N \in \Z^+ : \forall n \geq N, \forall x \in X, d_Y\big(f^{(n)}(x), f(x)\big) < \dfrac{\varepsilon}{4}
  \end{align*}
  Now fix one pair of \(\varepsilon\) and \(N\).
  Since \(L^{(n)}\) exists for each \(n \in \N\), by \cref{3.1.1} we have
  \begin{align*}
             & \forall n \geq N, d_Y - \lim_{x \to x_0 ; x \in E} f^{(n)}(x) = L^{(n)}                                                                                        \\
    \implies & \forall n \geq N, \exists \delta \in \R^+ : \Big(\forall x \in X, d_X(x, x_0) < \delta \implies d_Y\big(f^{(n)}(x), L^{(n)}\big) < \dfrac{\varepsilon}{4}\Big) \\
    \implies & \forall n_1, n_2 \geq N, \exists \delta \in \R^+ :                                                                                                             \\
             & \Big(\forall x \in X, d_X(x, x_0) < \delta \implies d_Y\big(L^{(n_1)}, L^{(n_2)}\big)                                                                          \\
             & \leq d_Y\big(L^{(n_1)}, f^{(n_1)}(x)\big) + d_Y\big(f^{(n_1)}(x), f(x)\big)                                                                                    \\
             & \quad + d_Y\big(f(x), f^{(n_2)}(x)\big) + d_Y\big(f^{(n_2)}(x), L^{(n_2)}\big)                                                                                 \\
             & < \dfrac{\varepsilon}{4} + \dfrac{\varepsilon}{4} + \dfrac{\varepsilon}{4} + \dfrac{\varepsilon}{4}\Big)                                                       \\
    \implies & \forall n_1, n_2 \geq N, d_Y\big(L^{(n_1)}, L^{(n_2)}\big) < \varepsilon
  \end{align*}
  Since \(\varepsilon\) is arbitrary, we have
  \[
    \forall \varepsilon \in \R^+, \exists N \in \Z^+ : \forall n_1, n_2 \geq N, d_Y\big(L^{(n_1)}, L^{(n_2)}\big) < \varepsilon
  \]
  and by \cref{1.4.6} \((L^{(n)})_{n = 1}^\infty\) is a Cauchy sequence in \((Y, d_Y)\).

  Let \(L \in Y\) such that \(d_Y - \lim_{n \to \infty} L^{(n)} = L\).
  Again by \cref{3.2.7} we have
  \begin{align*}
             & (f^{(n)})_{n = 1}^\infty \text{ converges uniformly to } f \text{ on } X \text{with respect to } d_Y                                              \\
    \implies & \forall \varepsilon \in \R^+, \exists N_1 \in \Z^+ : \forall n \geq N_1, \forall x \in X, d_Y\big(f^{(n)}(x), f(x)\big) < \dfrac{\varepsilon}{3}.
  \end{align*}
  Again we choose one pair of \(\varepsilon\) and \(N_1\).
  Since \(L\) exists, by \cref{3.1.1} we have
  \begin{align*}
             & \lim_{n \to \infty} d_Y\big(L^{(n)}, L\big) = 0                                                                                                                                 \\
    \implies & \exists N_2 \in \Z^+ : \forall n \geq N_2, d_Y(L^{(n)}, L) < \dfrac{\varepsilon}{3}                                                                                             \\
    \implies & \exists N = \max(N_1, N_2) : \forall n \geq N,                                                                                                                                  \\
             & \begin{dcases}
                 \exists \delta \in \R^+ : \forall x \in X, d_X(x, x_0) < \delta \implies d_Y\big(f^{(n)}(x), L^{(n)}\big) < \dfrac{\varepsilon}{3} \\
                 d_Y(L^{(n)}, L) < \dfrac{\varepsilon}{3}                                                                                           \\
                 \forall x \in X, d_Y\big(f^{(n)}(x), f(x)\big) < \dfrac{\varepsilon}{3}
               \end{dcases}                                              \\
    \implies & \exists N = \max(N_1, N_2) : \forall n \geq N, \exists \delta \in \R^+ :                                                                                                        \\
             & \Big(\forall x \in X, d_X(x, x_0) < \delta \implies d_Y\big(f(x), L\big)                                                                                                        \\
             & \leq d_Y\big(f(x), f^{(n)}(x)\big) + d_Y\big(f^{(n)}(x), L^{(n)}\big) + d_Y\big(L^{(n)}, L\big) < \dfrac{\varepsilon}{3} + \dfrac{\varepsilon}{3} + \dfrac{\varepsilon}{3}\Big) \\
    \implies & \exists \delta \in \R^+ : \Big(\forall x \in X, d_X(x, x_0) < \delta \implies d_Y\big(f(x), L\big) < \varepsilon\Big).
  \end{align*}
  Since \(\varepsilon\) is arbitrary, by \cref{3.1.1} we know that \(d_Y - \lim_{x \to x_0 ; x \in E} f(x) = L\).
\end{proof}

\begin{prop}\label{3.3.4}
  Let \((f^{(n)})_{n = 1}^\infty\) be a sequence of continuous functions from one metric space \((X, d_X)\) to another \((Y, d_Y)\), and suppose that this sequence converges uniformly to another function \(f : X \to Y\).
  Let \(x^{(n)}\) be a sequence of points in \(X\) which converge to some limit \(x\).
  Then \(f^{(n)}(x^{(n)})\) converges (in \(Y\)) to \(f(x)\).
\end{prop}

\begin{proof}
  Let \(x_0 \in X\).
  Suppose that \((x^{(n)})_{n = 1}^\infty\) is a sequence in \(X\) such that
  \[
    \lim_{n \to \infty} d_X(x^{(n)}, x_0) = 0.
  \]
  By \cref{3.3.1} we know that \(f\) is continuous at \(x_0\) from \((X, d_X)\) to \((Y, d_Y)\).
  Thus by \cref{2.1.4}(a)(b) we have
  \[
    \lim_{n \to \infty} d_Y\big(f(x^{(n)}), f(x_0)\big) = 0
  \]
  and by \cref{1.1.14} we have
  \[
    \forall \varepsilon \in \R^+, \exists N_1 \in \Z^+ : \forall n \geq N_1, d_Y\big(f(x^{(n)}), f(x_0)\big) < \dfrac{\varepsilon}{2}.
  \]
  Now we choose one pair of \(\varepsilon\) and \(N_1\).
  Since \((f^{(n)})_{n = 1}^\infty\) converges uniformly to \(f\) on \(X\) with respect to \(d_Y\), by \cref{3.2.7} we have
  \begin{align*}
             & \exists N_2 \in \Z^+ : \forall n \geq N_2, \forall x \in X, d_Y\big(f^{(n)}(x), f(x)\big) < \dfrac{\varepsilon}{2}                                                       \\
    \implies & \exists N_2 \in \Z^+ : \forall n \geq N_2, d_Y\big(f^{(n)}(x^{(n)}), f(x^{(n)})\big) < \dfrac{\varepsilon}{2}                                                            \\
    \implies & \exists N = \max(N_1, N_2) : \forall n \geq N,                                                                                                                           \\
             & d_Y\big(f^{(n)}(x^{(n)}), f(x_0)\big) \leq d_Y\big(f^{(n)}(x^{(n)}), f(x^{(n)})\big) + d_Y\big(f(x^{(n)}), f(x_0)\big) < \dfrac{\varepsilon}{2} + \dfrac{\varepsilon}{2} \\
    \implies & \exists N = \max(N_1, N_2) : \forall n \geq N, d_Y\big(f^{(n)}(x^{(n)}), f(x_0)\big) < \varepsilon.
  \end{align*}
  Since \(\varepsilon\) is arbitrary, by \cref{1.1.14} we know that
  \[
    \lim_{n \to \infty} d_Y\big(f^{(n)}(x^{(n)}), f(x_0)\big) = 0.
  \]
  Since \(x_0\) is arbitrary, we conclude that for any \(x_0 \in X\), if \((x^{(n)})_{n = 1}^\infty\) is a sequence in \(X\) such that
  \[
    \lim_{n \to \infty} d_X(x^{(n)}, x_0) = 0,
  \]
  then we have
  \[
    \lim_{n \to \infty} d_Y\big(f^{(n)}(x^{(n)}), f(x_0)\big) = 0.
  \]
\end{proof}

\begin{defn}[Bounded functions]\label{3.3.5}
  A function \(f : X \to Y\) from one metric space \((X, d_X)\) to another \((Y, d_Y)\) is \emph{bounded} if \(f(X)\) is a bounded set, i.e., there exists a ball \(B_{(Y, d_Y)}(y_0, R)\) in \(Y\) such that \(f(x) \in B_{(Y, d_Y)}(y_0, R)\) for all \(x \in X\).
\end{defn}

\begin{prop}[Uniform limits preserve boundedness]\label{3.3.6}
  Let \((f^{(n)})_{n = 1}^\infty\) be a sequence of functions from one metric space \((X, d_X)\) to another \((Y, d_Y)\), and suppose that this sequence converges uniformly to another function \(f : X \to Y\).
  If the functions \(f^{(n)}\) are bounded on \(X\) for each \(n\), then the limiting function \(f\) is also bounded on \(X\).
\end{prop}

\begin{proof}
  Since \(f^{(n)}\) is bounded in \((Y, d_Y)\) for each \(n \in \Z^+\), by \cref{3.3.5} and \cref{1.5.3} we have
  \begin{align*}
             & \forall n \in \Z^+, \forall y \in Y, \exists \varepsilon \in \R^+ : f^{(n)}(X) \subseteq B_{(Y, d_Y)}(y, \varepsilon)          \\
    \implies & \forall n \in \Z^+, \forall y \in Y, \exists \varepsilon \in \R^+ : \forall x \in X, d_Y\big(f^{(n)}(x), y\big) < \varepsilon.
  \end{align*}
  Now we choose \(y\) and \(\varepsilon\) for each \(n \in \Z^+\) and we denote them as \(y^{(n)}\) and \(\varepsilon^{(n)}\).
  Since \((f^{(n)})_{n = 1}^\infty\) converges uniformly to \(f\) on \(X\) with respect to \(d_Y\), by \cref{3.2.7} we have
  \begin{align*}
             & \exists N \in \Z^+ : \forall n \geq N, \forall x \in X, d_Y\big(f^{(n)}(x), f(x)\big) < 1                                \\
             & \exists N \in \Z^+ : \forall x \in X, d_Y\big(f^{(N)}(x), f(x)\big) < 1                                                  \\
    \implies & \exists N \in \Z^+ : \forall x \in X,                                                                                    \\
             & d_Y\big(f(x), y^{(N)}\big) \leq d_Y\big(f(x), f^{(N)}(x)\big) + d_Y\big(f^{(N)}(x), y^{(N)}\big) < \varepsilon^{(N)} + 1 \\
    \implies & \exists N \in \Z^+ : \forall x \in X, d_Y\big(f(x), y^{(N)}\big) < \varepsilon^{(N)} + 1                                 \\
    \implies & \exists N \in \Z^+ : f(X) \subseteq B_{(Y, d_Y)}(y^{(N)}, \varepsilon^{(N)} + 1).
  \end{align*}
  Since \(y^{(N)}\) is arbitrary, we have
  \[
    \forall y \in Y, \exists \varepsilon \in \R^+ : f(X) \subseteq B_{(Y, d_Y)}(y, \varepsilon)
  \]
  and by \cref{1.5.3} and \cref{3.3.5} \(f\) is bounded in \((Y, d_Y)\).
\end{proof}

\begin{rmk}\label{3.3.7}
  The above propositions sound very reasonable, but one should caution that it only works if one assumes uniform convergence;
  pointwise convergence is not enough.
\end{rmk}

\exercisesection

\begin{ex}\label{ex:3.3.1}
  Prove \cref{3.3.1}.
  Explain briefly why your proof requires uniform convergence, and why pointwise convergence would not suffice.
\end{ex}

\begin{proof}
  See \cref{3.3.1}.
  Without uniform convergence we cannot make \(f^{(n)}(x)\) and \(f(x)\) arbitrary close.
\end{proof}

\begin{ex}\label{ex:3.3.2}
  Prove \cref{3.3.3}.
\end{ex}

\begin{proof}
  See \cref{3.3.3}.
\end{proof}

\begin{ex}\label{ex:3.3.3}
  Compare \cref{3.3.3} with Example 1.2.8 in Analysis I.
  Can you now explain why the interchange of limits in Example 1.2.8 in Analysis I led to a false statement, whereas the interchange of limits in \cref{3.3.3} is justified?
\end{ex}

\begin{proof}
  By \cref{ex:3.2.2}(b) we know that \(f^{(n)}(x) = x^{(n)}\) does not converge uniformly to any function \(f : (-1, 1) \to \R\), thus the interchange of limits in Example 1.2.8 in Analysis I failed.
\end{proof}

\begin{ex}\label{ex:3.3.4}
  Prove \cref{3.3.4}.
\end{ex}

\begin{proof}
  See \cref{3.3.4}.
\end{proof}

\begin{ex}\label{ex:3.3.5}
  Give an example to show that \cref{3.3.4} fails if the phrase ``converges uniformly'' is replaced by ``converges pointwise''.
\end{ex}

\begin{proof}
  For each \(n \in \Z^+\), let \(f^{(n)} : [0, 1] \to \R\) be the function where \(f^{(n)}(x) = x^n\) for each \(x \in [0, 1]\).
  Let \(f : [0, 1] \to \R\) be the function where
  \[
    \forall x \in [0, 1], f(x) = \begin{dcases}
      1 & \text{if } x = 1        \\
      0 & \text{if } x \in [0, 1)
    \end{dcases}
  \]
  By Example 3.2.4 in Analysis II we know that \((f^{(n)})_{n = 1}^\infty\) converges pointwise to \(f\) on \(X\) with respect to \(d_{l^1}|_{\R \times \R}\).
  By \cref{ex:3.2.2}(b) we know that \(f^{(n)}(x) = x^{(n)}\) does not converge uniformly to \(f\) on \(X\) with respect to \(d_{l^1}|_{\R \times \R}\).
  Let \((x^{(n)})_{n = 1}^\infty\) be the sequence where \(x^{(n)} = (\dfrac{1}{2})^{\dfrac{1}{n}}\) for each \(n \in \Z^+\).
  Then we have
  \[
    \lim_{n \to \infty} x^{(n)} = 1 = f(1).
  \]
  But
  \[
    \lim_{n \to \infty} f^{(n)}(x^{(n)}) = \lim_{n \to \infty} \big((\dfrac{1}{2})^{\dfrac{1}{n}}\big)^n = \lim_{n \to \infty} \dfrac{1}{2} = \dfrac{1}{2} \neq 1.
  \]
  Thus \cref{3.3.4} fails when the phrase ``converges uniformly'' is replaced by ``converges pointwise''.
\end{proof}

\begin{ex}\label{ex:3.3.6}
  Prove \cref{3.3.6}.
\end{ex}

\begin{proof}
  See \cref{3.3.6}.
\end{proof}

\begin{ex}\label{ex:3.3.7}
  Give an example to show that \cref{3.3.6} fails if the phrase ``converges uniformly'' is replaced by ``converges pointwise''.
\end{ex}

\begin{proof}
  By \cref{ex:3.2.2}(c) we know that \(g\) is unbounded since
  \[
    \lim_{x \to -1 ; x \in (-1, 1)} g(x) = \lim_{x \to -1 ; x \in (-1, 1)} \dfrac{x}{1 - x} = \lim_{x \to -1 ; x \in (-1, 1)} \dfrac{1}{1 - x} - 1 = +\infty.
  \]
\end{proof}

\begin{ex}\label{ex:3.3.8}
  Let \((X, d)\) be a metric space, and for every positive integer \(n\), let \(f_n : X \to \R\) and \(g_n : X \to \R\) be functions.
  Suppose that \((f_n)_{n = 1}^\infty\) converges uniformly to another function \(f : X \to \R\), and that \((g_n)_{n = 1}^\infty\) converges uniformly to another function \(g : X \to \R\).
  Suppose also that the functions \((f_n)_{n = 1}^\infty\) and \((g_n)_{n = 1}^\infty\) are uniformly bounded, i.e., there exists an \(M > 0\) such that \(\abs{f_n(x)} \leq M\) and \(\abs{g_n(x)} \leq M\) for all \(n \geq 1\) and \(x \in X\).
  Prove that the functions \(f_n g_n : X \to \R\) converge uniformly to \(fg : X \to \R\).
\end{ex}

\begin{proof}
  Let \(d_1 = d_{l^1}|_{\R \times \R}\).
  Since \((f_n)_{n = 1}^\infty\) and \((g_n)_{n = 1}^\infty\) are uniformly bounded, by \cref{3.3.5} we know that \(f_n\) and \(g_n\) are bounded in \((\R, d_1)\) for each \(n \in \Z^+\).
  By \cref{3.3.6} we know that \(f\) and \(g\) are bounded in \((\R, d_1)\), i.e.,
  \[
    \exists U \in \R^+ : \forall x \in X, \begin{dcases}
      \abs{f(x)} < U \\
      \abs{g(x)} < U
    \end{dcases}
  \]
  Since \((f_n)_{n = 1}^\infty\) and \((g_n)_{n = 1}^\infty\) converge uniformly to \(f\) on \(X\) with respect to \(d_1\), by \cref{3.2.7} we have
  \begin{align*}
     & \forall \varepsilon \in \R^+, \exists N_1 \in \Z^+ : \forall n \geq N_1, \forall x \in X, \abs{f_n(x) - f(x)} < \dfrac{\varepsilon}{2M}; \\
     & \forall \varepsilon \in \R^+, \exists N_2 \in \Z^+ : \forall n \geq N_2, \forall x \in X, \abs{g_n(x) - g(x)} < \dfrac{\varepsilon}{2U}.
  \end{align*}
  Now we fix one \(\varepsilon\) and its corresponding \(N_1, N_2\).
  Let \(N = \max(N_1, N_2)\).
  Then we have
  \begin{align*}
    \forall n \geq N, \forall x \in X, & \abs{f_n(x) g_n(x) - f(x) g(x)}                                        \\
                                       & = \abs{f_n(x) g_n(x) - f(x) g_n(x) + f(x) g_n(x) - f(x) g(x)}          \\
                                       & \leq \abs{f_n(x) g_n(x) - f(x) g_n(x)} + \abs{f(x) g_n(x) - f(x) g(x)} \\
                                       & = \abs{f_n(x) - f(x)} \abs{g_n(x)} + \abs{f(x)} \abs{g_n(x) - g(x)}    \\
                                       & < \dfrac{\varepsilon}{2M} M + U \dfrac{\varepsilon}{2U}                \\
                                       & = \varepsilon.
  \end{align*}
  Since \(\varepsilon\) is arbitrary, we have
  \[
    \forall \varepsilon \in \R^+, \exists N \in \Z^+ : \forall n \geq N, \forall x \in X, \abs{f_n(x) g_n(x) - f(x) g(x)} < \varepsilon
  \]
  and by \cref{3.2.7} \((f_n g_n)_{n = 1}^\infty\) converges uniformly to \(fg\) on \(X\) with respect to \(d_1\).
\end{proof}
\section{The metric of uniform convergence}\label{ii:sec:3.4}

\begin{note}
  We have now developed at least four, apparently separate, notions of limit in this text:
  \begin{enumerate}
    \item limits \(\lim_{n \to \infty} x^{(n)}\) of sequences of points in a metric space
          (\cref{ii:1.1.14};
          see also \cref{ii:2.5.4});
    \item limiting values \(\lim_{x \to x_0 ; x \in E} f(x)\) of functions at a point
          (\cref{ii:3.1.1});
    \item pointwise limits \(f\) of functions \(f^{(n)}\)
          (\cref{ii:3.2.1});
          and
    \item uniform limits \(f\) of functions \(f^{(n)}\)
          (\cref{ii:3.2.7}).
  \end{enumerate}

  This proliferation of limits may seem rather complicated.
  However, we can reduce the complexity slightly by observing that (d) can be viewed as a special case of (a), though in doing so it should be cautioned that because we are now dealing with functions instead of points, the convergence is not in \(X\) or in \(Y\), but rather in a new space, the space of functions from \(X\) to \(Y\).
\end{note}

\begin{rmk}\label{ii:3.4.1}
  If one is willing to work in topological spaces instead of metric spaces, we can also view (a) as a special case of (b), see \cref{ii:ex:3.1.4}, and (c) is also a special case of (a), see \cref{ii:ex:3.4.4}.
  Thus, the notion of convergence in a topological space can be used to unify all the notions of limits we have encountered so far.
\end{rmk}

\begin{defn}[Metric space of bounded functions]\label{ii:3.4.2}
  Suppose \((X, d_X)\) and \((Y, d_Y)\) are metric spaces.
  We let \(B(X \to Y)\) denote the space of bounded functions from \(X\) to \(Y\) :
  \[
    B(X \to Y) \coloneqq \set{f | f : X \to Y \text{ is a bounded function}}.
  \]
  We define a metric \(d_\infty : B(X \to Y) \times B(X \to Y) \to [0, +\infty)\) by defining
  \[
    d_\infty(f, g) \coloneqq \sup_{x \in X} d_Y\big(f(x), g(x)\big) = \sup\set{d_Y\big(f(x), g(x)\big) : x \in X}
  \]
  for all \(f, g \in B(X \to Y)\).
  This metric is sometimes known as the \emph{uniform metric}, or \emph{sup norm metric}, or the \emph{\(L^\infty\) metric}.
  We will also use \(d_{B(X \to Y)}\) as a synonym for \(d_\infty\).
  We restrict the definition of \(d_\infty\) to the case when \(X \neq \emptyset\).
  If \(X = \emptyset\), then we instead define \(d_\infty(f, g) = 0\).
\end{defn}

\begin{note}
  \(B(X \to Y)\) is a set, thanks to the power set axiom (Axiom 3.10 in Analysis I) and the axiom of specification (Axiom 3.5 in Analysis I).
\end{note}

\begin{note}
  The distance \(d_\infty(f, g)\) is always finite because \(f\) and \(g\) are assumed to be bounded on \(X\).
\end{note}

\setcounter{thm}{3}
\begin{prop}\label{ii:3.4.4}
  Let \((X, d_X)\) and \((Y, d_Y)\) be metric spaces.
  Let \((f^{(n)})_{n = 1}^\infty\) be a sequence of functions in \(B(X \to Y)\), and let \(f\) be another function in \(B(X \to Y)\).
  Then \((f^{(n)})_{n = 1}^\infty\) converges to \(f\) in the metric \(d_{B(X \to Y)}\) iff \((f^{(n)})_{n = 1}^\infty\) converges uniformly to \(f\).
\end{prop}

\begin{proof}
  We have
  \begin{align*}
         & d_{B(X \to Y)} - \lim_{n \to \infty} f^{(n)} = f                                                                               \\
    \iff & \forall \varepsilon \in \R^+, \exists N \in \Z^+ :                                                                             \\
         & \forall n \geq N, d_{B(X \to Y)}(f^{(n)}, f) \leq \dfrac{\varepsilon}{2} < \varepsilon                     &  & \by{ii:1.1.14} \\
    \iff & \forall \varepsilon \in \R^+, \exists N \in \Z^+ :                                                                             \\
         & \forall n \geq N, \sup_{x \in X} d_Y\big(f^{(n)}(x), f(x)\big) \leq \dfrac{\varepsilon}{2} < \varepsilon   &  & \by{ii:3.4.2}  \\
    \iff & \forall \varepsilon \in \R^+, \exists N \in \Z^+ :                                                                             \\
         & \forall n \geq N, \forall x \in X, d_Y\big(f^{(n)}(x), f(x)\big) \leq \dfrac{\varepsilon}{2} < \varepsilon                     \\
    \iff & (f^{(n)})_{n = 1}^\infty \text{ converges uniformly to } f \text{ on } X                                                       \\
         & \text{with respect to } d_Y.                                                                               &  & \by{ii:3.2.7}
  \end{align*}
\end{proof}

\begin{note}
  Now let \(C(X \to Y)\) be the space of bounded continuous functions from \(X\) to \(Y\) :
  \[
    C(X \to Y) \coloneqq \set{f \in B(X \to Y) | f \text{ is continuous}}.
  \]
  This set \(C(X \to Y)\) is clearly a subset of \(B(X \to Y)\).
  \cref{ii:3.3.2} asserts that this space \(C(X \to Y)\) is closed in \(\big(B(X \to Y), d_{B(X \to Y)}\big)\).
\end{note}

\begin{thm}[The space of continuous functions is complete]\label{ii:3.4.5}
  Let \((X, d_X)\) be a metric space, and let \((Y, d_Y)\) be a complete metric space.
  The space \(\big(C(X \to Y), d_{B(X \to Y)}|_{C(X \to Y) \times C(X \to Y)}\big)\) is a complete subspace of \(\big(B(X \to Y), d_{B(X \to Y)}\big)\).
  In other words, every Cauchy sequence of functions in \(C(X \to Y)\) converges to a function in \(C(X \to Y)\).
\end{thm}

\begin{proof}
  Let \(d_{C(X \to Y)} = d_{B(X \to Y)}|_{C(X \to Y) \times C(X \to Y)}\) and let \(n_1, n_2 \in \Z^+\).
  Let \((f_n)_{n = 1}^\infty\) be a Cauchy sequence in \(\big(C(X \to Y), d_{C(X \to Y)}\big)\).
  Observe that
  \begin{align*}
             & \forall \varepsilon \in \R^+, \exists N \in \Z^+ : \forall n_1, n_2 \geq N,                                                          \\
             & d_{C(X \to Y)}\big(f^{(n_1)}, f^{(n_2)}\big) < \varepsilon                                                        &  & \by{ii:1.4.6} \\
    \implies & \forall \varepsilon \in \R^+, \exists N \in \Z^+ : \forall n_1, n_2 \geq N,                                                          \\
             & \sup_{x \in X} d_Y\big(f^{(n_1)}(x), f^{(n_2)}(x)\big) < \varepsilon                                              &  & \by{ii:3.4.2} \\
    \implies & \forall x \in X, \forall \varepsilon \in \R^+, \exists N \in \Z^+ : \forall n_1, n_2 \geq N,                                         \\
             & d_Y\big(f^{(n_1)}(x), f^{(n_2)}(x)\big) \leq \sup_{x \in X} d_Y\big(f^{(n_1)}(x), f^{(n_2)}(x)\big) < \varepsilon                    \\
    \implies & \forall x \in X, \big(f_n(x)\big)_{n = 1}^\infty \text{ is a Cauchy sequence in } (Y, d_Y).                       &  & \by{ii:1.4.6}
  \end{align*}
  By hypothesis we know that \((Y, d_Y)\) is complete, thus by \cref{ii:1.4.10} we have
  \[
    \forall x \in X, d_Y - \lim_{n \to \infty} f_n(x) \in Y
  \]
  and we can define a function \(f : X \to Y\) such that
  \[
    \forall x \in X, f(x) = d_Y - \lim_{n \to \infty} f_n(x).
  \]
  By \cref{ii:1.1.14} we have
  \[
    \forall x \in X, \forall \varepsilon \in \R^+, \exists N \in \Z^+ : \forall n \geq N, d_Y\big(f_n(x), f(x)\big) < \dfrac{\varepsilon}{3}.
  \]
  We choose one \(N\) for each pairs of \(x\) and \(\varepsilon\) and denote it as \(N_{x, \varepsilon}\).
  Since \(f_n \in C(X \to Y)\) for all \(n \in \Z^+\), by \cref{ii:2.1.1} we know that
  \[
    \forall x_0 \in X, \forall \varepsilon \in \R^+, \exists \delta \in \R^+ : \forall x \in X, d_X(x, x_0) < \delta \implies d_Y\big(f_n(x), f_n(x_0)\big) < \dfrac{\varepsilon}{3}.
  \]
  If we denote \(M_{x, x_0, \varepsilon} = \max(N_{x, \varepsilon}, N_{x_0, \varepsilon})\), then by \cref{ii:1.1.2}(d) we have
  \begin{align*}
             & \forall x_0 \in X, \forall \varepsilon \in \R^+, \exists \delta \in \R^+ : \forall x \in X, d_X(x, x_0) < \delta         \\
    \implies & \begin{dcases}
                 \forall n \geq M_{x, x_0, \varepsilon}, d_Y\big(f_n(x), f(x)\big) < \dfrac{\varepsilon}{3}     \\
                 \forall n \geq M_{x, x_0, \varepsilon}, d_Y\big(f_n(x_0), f(x_0)\big) < \dfrac{\varepsilon}{3} \\
                 d_Y\big(f_n(x), f_n(x_0)\big) < \dfrac{\varepsilon}{3}
               \end{dcases}                           \\
    \implies & \forall n \geq M_{x, x_0, \varepsilon},                                                                                  \\
             & d_Y\big(f(x), f(x_0)\big) \leq d_Y\big(f_n(x), f(x)\big) + d_Y\big(f_n(x), f_n(x_0)\big) + d_Y\big(f_n(x_0), f(x_0)\big) \\
             & < \dfrac{\varepsilon}{3} + \dfrac{\varepsilon}{3} + \dfrac{\varepsilon}{3} = \varepsilon                                 \\
    \implies & d_Y\big(f(x), f(x_0)\big) < \varepsilon.
  \end{align*}
  By \cref{ii:2.1.1} this means \(f \in C(X \to Y)\).
  Since \((f_n)_{n = 1}^\infty\) was arbitrary, by \cref{ii:1.4.10} \(\big(C(X \to Y), d_{C(X \to Y)}\big)\) is complete.
\end{proof}

\exercisesection

\begin{ex}\label{ii:ex:3.4.1}
  Let \((X, d_X)\) and \((Y, d_Y)\) be metric spaces.
  Show that the space \(B(X \to Y)\) defined in \cref{ii:3.4.2}, with the metric \(d_{B(X \to Y)}\), is indeed a metric space.
\end{ex}

\begin{proof}
  If \(X = \emptyset\), then by \cref{ii:3.4.2} we have
  \begin{itemize}
    \item If \(f \in B(\emptyset \to Y)\), then \(d_{B(X \to Y)}(f, f) = 0\).
    \item If \(f, g \in B(\emptyset \to Y)\), then \(d_{B(X \to Y)}(f, g) = 0 = d_{B(X \to Y)}(g, f)\).
    \item If \(f, g, h \in B(\emptyset \to Y)\), then \(d_{B(X \to Y)}(f, h) = 0 = d_{B(X \to Y)}(f, g) + d_{B(X \to Y)}(g, h)\).
  \end{itemize}
  Thus, by \cref{ii:1.1.2} \(\big(B(\emptyset \to Y), d_{B(X \to Y)}\big)\) is a metric space.
  Now suppose that \(X \neq \emptyset\).
  Since
  \begin{align*}
    \forall f \in B(X \to Y), d_{B(X \to Y)}(f, f) & = \sup_{x \in X} d_Y\big(f(x), f(x)\big) &  & \by{ii:3.4.2}    \\
                                                   & = \sup \set{0}                           &  & \by{ii:1.1.2}[a] \\
                                                   & = 0,
  \end{align*}
  by \cref{ii:1.1.2}(a) we know that \(\big(B(X \to Y), d_{B(X \to Y)}\big)\) is reflexive.
  Since
  \begin{align*}
             & \forall f, g \in B(X \to Y), f \neq g                                                    \\
    \implies & \exists x \in X : f(x) \neq g(x)                                                         \\
    \implies & \exists x \in X : d_Y\big(f(x), g(x)\big) > 0                      &  & \by{ii:1.1.2}[b] \\
    \implies & d_{B(X \to Y)}(f, g) = \sup_{x \in X} d_Y\big(f(x), g(x)\big) > 0, &  & \by{ii:3.4.2}
  \end{align*}
  by \cref{ii:1.1.2}(b) we know that \(\big(B(X \to Y), d_{B(X \to Y)}\big)\) is positive.
  Since
  \begin{align*}
    \forall f, g \in B(X \to Y), d_{B(X \to Y)}(f, g) & = \sup_{x \in X} d_Y\big(f(x), g(x)\big) &  & \by{ii:3.4.2}    \\
                                                      & = \sup_{x \in X} d_Y\big(g(x), f(x)\big) &  & \by{ii:1.1.2}[c] \\
                                                      & = d_{B(X \to Y)}(g, f),                  &  & \by{ii:3.4.2}
  \end{align*}
  by \cref{ii:1.1.2}(c) we know that \(\big(B(X \to Y), d_{B(X \to Y)}\big)\) is symmetry.
  Since
  \begin{align*}
     & \forall f, g, h \in B(X \to Y),                                                                         \\
     & d_{B(X \to Y)}(f, g) + d_{B(X \to Y)}(g, h)                                                             \\
     & = \sup_{x \in X} d_Y\big(f(x), g(x)\big) + \sup_{x \in X} d_Y\big(g(x), h(x)\big) &  & \by{ii:3.4.2}    \\
     & \geq \sup_{x \in X} \Big(d_Y\big(f(x), g(x)\big) + d_Y\big(g(x), h(x)\big)\Big)                         \\
     & \geq \sup_{x \in X} d_Y\big(f(x), h(x)\big)                                       &  & \by{ii:1.1.2}[d] \\
     & = d_{B(X \to Y)}(f, h),                                                           &  & \by{ii:3.4.2}
  \end{align*}
  by \cref{ii:1.1.2}(d) we know that \(\big(B(X \to Y), d_{B(X \to Y)}\big)\) is transitive.
  Combine all the proofs above we conclude by \cref{ii:1.1.2} that \(\big(B(X \to Y), d_{B(X \to Y)}\big)\) is a metric space.
\end{proof}

\begin{ex}\label{ii:ex:3.4.2}
  Prove \cref{ii:3.4.4}.
\end{ex}

\begin{proof}
  See \cref{ii:3.4.4}.
\end{proof}

\begin{ex}\label{ii:ex:3.4.3}
  Prove \cref{ii:3.4.5}.
\end{ex}

\begin{proof}
  See \cref{ii:3.4.5}.
\end{proof}

\begin{ex}\label{ii:ex:3.4.4}
  Let \((X, d_X)\) and \((Y, d_Y)\) be metric spaces, and let \(Y^X \coloneqq \set{f | f : X \to Y }\) be the space of all functions from \(X\) to \(Y\)
  (cf. Axiom 3.10 in Analysis I).
  If \(x_0 \in X\) and \(V\) is an open set in \(Y\), let \(V^{(x_0)} \subseteq Y^X\) be the set
  \[
    V^{(x_0)} \coloneqq \set{f \in Y^X : f(x_0) \in V}.
  \]
  If \(E\) is a subset of \(Y^X\), we say that \(E\) is \emph{open} if for every \(f \in E\), there exists a finite number of points \(x_1, \dots, x_n \in X\) and open sets \(V_1, \dots, V_n \subseteq Y\) such that
  \[
    f \in V_1^{(x_1)} \cap \dots \cap V_n^{(x_n)} \subseteq E.
  \]
  \begin{itemize}
    \item Show that if \(\mathcal{F}\) is the collection of open sets in \(Y^X\), then \((Y^X , \mathcal{F})\) is a topological space.
    \item For each natural number \(n\), let \(f^{(n)} : X \to Y\) be a function from \(X\) to \(Y\), and let \(f : X \to Y\) be another function from \(X\) to \(Y\).
          Show that \(f^{(n)}\) converges to \(f\) in the topology \(\mathcal{F}\) (in the sense of \cref{ii:2.5.4}) iff \(f^{(n)}\) converges to \(f\) pointwise (in the sense of \cref{ii:3.2.1}).
  \end{itemize}
  The topology \(\mathcal{F}\) is known as the \emph{topology of pointwise convergence}, for obvious reasons;
  it is also known as the \emph{product topology}.
  It shows that the concept of pointwise convergence can be viewed as a special case of the more general concept of convergence in a topological space.
\end{ex}

\begin{proof}
  We know that \(\emptyset \in \mathcal{F}\) trivially.
  First we show that \(Y^X \in \mathcal{F}\).
  Let \(f \in Y^X\) and let \(x_0 \in X\).
  By \cref{ii:1.2.15}(c) we know that \(B_{(Y, d_Y)}\big(f(x_0), 1\big)\) is open in \((Y, d_Y)\).
  Then we have
  \[
    f \in \Big(B_{(Y, d_Y)}\big(f(x_0), 1\big)\Big)^{(x_0)} \subseteq Y^X.
  \]
  Since \(f\) was arbitrary, by definition we know that \(Y^X \in \mathcal{F}\).

  Next we show that the intersection of any finite collection of open sets in \((Y^X, \mathcal{F})\) is open in \((Y^X, \mathcal{F})\).
  Let \(n \in \N\) and let \((U_i)_{i = 1}^n\) be a finite collection of open sets in \((Y^X, \mathcal{F})\).
  If \(\bigcap_{i = 1}^n U_i = \emptyset\), then from the proof above we know that \(\emptyset \in \mathcal{F}\).
  So suppose that \(\bigcap_{i = 1}^n U_i \neq \emptyset\).
  Let \(f \in \bigcap_{i = 1}^n U_i\).
  Since
  \begin{align*}
             & \forall 1 \leq i \leq n, f \in U_i                                                                                  \\
    \implies & \forall 1 \leq i \leq n, \exists m_i \in \Z^+ : \begin{dcases}
                                                                 x_{(i, 1)}, \dots, x_{(i, m_i)} \in X                              \\
                                                                 V_{(i, 1)}, \dots, V_{(i, m_i)} \text{ are open sets in } (Y, d_Y) \\
                                                                 f \in \bigcap_{j = 1}^{m_i} V_{(i, j)}^{(x_{(i, j)})} \subseteq U_i
                                                               \end{dcases}  \\
    \implies & f \in \bigcap_{i = 1}^n \bigg(\bigcap_{j = 1}^{m_i} V_{(i, j)}^{(x_{(i, j)})}\bigg) \subseteq \bigcap_{i = 1}^n U_i
  \end{align*}
  and \(f\) was arbitrary, we know that \(\bigcap_{i = 1}^n U_i \in \mathcal{F}\).
  Since \(n\) was arbitrary, we know that the intersection of any finite collection of open sets in \((Y^X, \mathcal{F})\) is open in \((Y^X, \mathcal{F})\).

  Next we show that the union of arbitrary open sets in \((Y^X, \mathcal{F})\) is open in \((Y^X, \mathcal{F})\).
  Let \(S \subseteq \mathcal{F}\) and let \(f \in \bigcup S\).
  Since
  \begin{align*}
             & f \in \bigcup S                                                                      \\
    \implies & \exists U \in S : f \in U                                                            \\
    \implies & \exists U \in S : \begin{dcases}
                                   x_1, \dots, x_n \in X                              \\
                                   V_1, \dots, V_n \text{ are open sets in } (Y, d_Y) \\
                                   f \in \bigcap_{i = 1}^n V_i^{(x_i)} \subseteq U \subseteq \bigcup S
                                 \end{dcases}
  \end{align*}
  and \(f\) was arbitrary, we know that \(\bigcup S \in \mathcal{F}\).
  Since \(S\) was arbitrary, we know that the union of arbitrary open sets in \((Y^X, \mathcal{F})\) is open in \((Y^X, \mathcal{F})\).
  Combine all the proofs above we conclude by \cref{ii:2.5.1} that \((Y^X, \mathcal{F})\) is a topological space.

  Next suppose that \((f^{(n)})_{n = 1}^\infty\) is a sequence in \(Y^X\) and \(f \in Y^X\).
  Suppose also that \((f^{(n)})_{n = 1}^\infty\) converges to \(f\) in \((Y^X, \mathcal{F})\).
  Then by \cref{ii:2.5.4} we have
  \[
    \forall E \in \mathcal{F}, f \in E \implies \exists N \in \Z^+ : \forall n \geq N, f^{(n)} \in E.
  \]
  Let \(x_0 \in X\).
  Then we have
  \begin{align*}
             & \forall \varepsilon \in \R^+, B_{(Y, d_Y)}\big(f(x_0), \varepsilon\big) \text{ is open in } (Y, d_Y)            &  & \by{ii:1.2.15}[c]      \\
    \implies & \forall \varepsilon \in \R^+, f \in \Big(B_{(Y, d_Y)}\big(f(x_0), \varepsilon\big)\Big)^{(x_0)} \in \mathcal{F} &  & \text{(by definition)} \\
    \implies & \forall \varepsilon \in \R^+, \exists N \in \Z^+ : \forall n \geq N,                                                                        \\
             & f^{(n)} \in \Big(B_{(Y, d_Y)}\big(f(x_0), \varepsilon\big)\Big)^{(x_0)}                                         &  & \by{ii:2.5.4}          \\
    \implies & \forall \varepsilon \in \R^+, \exists N \in \Z^+ : \forall n \geq N,                                                                        \\
             & d_Y\big(f^{(n)}(x_0), f(x_0)\big) < \varepsilon                                                                 &  & \text{(by definition)} \\
    \implies & \lim_{n \to \infty} d_Y\big(f^{(n)}(x_0), f(x_0)\big).                                                          &  & \by{ii:1.1.14}
  \end{align*}
  Since \(x_0\) was arbitrary, by \cref{ii:3.2.1} \((f^{(n)})_{n = 1}^\infty\) converges pointwise to \(f\) on \(X\) with respect to \(d_Y\).

  Finally suppose that \((f^{(n)})_{n = 1}^\infty\) is a sequence in \(Y^X\) and \(f \in Y^X\).
  Suppose also that \((f^{(n)})_{n = 1}^\infty\) converges pointwise to \(f\) on \(X\) with respect to \(d_Y\).
  Then we have
  \begin{align*}
             & \forall x \in X, \lim_{n \to \infty} d_Y\big(f^{(n)}(x), f(x)\big)                    &  & \by{ii:3.2.1}  \\
    \implies & \forall x \in X, \forall \varepsilon \in \R^+, \exists N \in \Z^+ : \forall n \geq N,                     \\
             & d_Y\big(f^{(n)}(x), f(x)\big) < \varepsilon.                                          &  & \by{ii:1.1.14}
  \end{align*}
  We choose one \(N\) for each pair of \((x, \varepsilon)\) and denote it as \(N_{(x, \varepsilon)}\).
  Let \(E \in \mathcal{F}\) such that \(f \in E\).
  By definition we know that
  \[
    \exists m \in \Z^+ : \begin{dcases}
      x_1, \dots, x_m \in X                              \\
      V_1, \dots, V_m \text{ are open sets in } (Y, d_Y) \\
      f \in \bigcap_{i = 1}^m V_i^{(x_i)} \subseteq E
    \end{dcases}
  \]
  Then we have
  \begin{align*}
             & \forall 1 \leq i \leq m, f \in V_i^{(x_i)}                                                                                                                    \\
    \implies & \forall 1 \leq i \leq m, \begin{dcases}
                                          f(x_i) \in V_i \\
                                          V_i \text{ is open in } (Y, d_Y)
                                        \end{dcases}                                                                                                      \\
    \implies & \forall 1 \leq i \leq m, \exists \varepsilon_i \in \R^+ : B_{(Y, d_Y)}\big(f(x_i), \varepsilon_i\big) \subseteq V_i                    &  & \by{ii:1.2.15}[a] \\
    \implies & \forall 1 \leq i \leq m, \exists \varepsilon_i \in \R^+ : \exists N_{(x_i, \varepsilon_i)} \in \Z^+ :                                                         \\
             & \forall n \geq N_{(x_i, \varepsilon_i)}, f^{(n)}(x_i) \in B_{(Y, d_Y)}\big(f(x_i), \varepsilon_i\big) \subseteq V_i                                           \\
    \implies & \exists N = \max_{1 \leq i \leq m} N_{(x_i, \varepsilon_i)} : \forall n \geq N, f^{(n)} \in \bigcap_{i = 1}^m V_i^{(x_i)} \subseteq E.
  \end{align*}
  Since \(E\) was arbitrary, by \cref{ii:2.5.4} we know that \((f^{(n)})_{n = 1}^\infty\) converges to \(f\) in \((Y^X, \mathcal{F})\).
\end{proof}

\section{Series of functions; the Weierstrass \emph{M}-test}\label{sec:3.5}

\begin{note}
  Functions whose range is \(\R\) are sometimes called \emph{real-valued} functions.
\end{note}

\begin{note}
  given any finite collection \(f^{(1)}, \dots, f^{(N)}\) of functions from \(X\) to \(\R\), we can define the finite sum \(\sum_{i = 1}^N f^{(i)} : X \to \R\) by
  \[
    \bigg(\sum_{i = 1}^N f^{(i)}\bigg)(x) \coloneqq \sum_{i = 1}^N f^{(i)}(x).
  \]
\end{note}

\setcounter{thm}{1}
\begin{defn}[Infinite series]\label{3.5.2}
  Let \((X, d_X)\) be a metric space.
  Let \((f^{(n)})_{n = 1}^\infty\) be a sequence of functions from \(X\) to \(\R\), and let \(f\) be another function from \(X\) to \(\R\).
  If the partial sums \(\sum_{n = 1}^N f^{(n)}\) converges pointwise to \(f\) on \(X\) as \(N \to \infty\), we say that the infinite series \(\sum_{n = 1}^\infty f^{(n)}\) \emph{converges pointwise} to \(f\), and write \(f = \sum_{n = 1}^\infty f^{(n)}\).
  If the partial sums \(\sum_{n = 1}^N f^{(n)}\) converge uniformly to \(f\) on \(X\) as \(N \to \infty\), we say that the infinite series \(\sum_{n = 1}^\infty f^{(n)}\) \emph{converges uniformly} to \(f\), and again write \(f = \sum_{n = 1}^\infty f^{(n)}\).
  (Thus when one sees an expression such as \(f = \sum_{n = 1}^\infty f^{(n)}\), one should look at the context to see in what sense this infinite series converges.)
\end{defn}

\begin{rmk}\label{3.5.3}
  A series \(\sum_{n = 1}^\infty f^{(n)}\) converges pointwise to \(f\) on \(X\) if and only if \(\sum_{n = 1}^\infty f^{(n)}(x)\) converges to \(f(x)\) for \emph{every} \(x \in X\).
  (Thus if \(\sum_{n = 1}^\infty f^{(n)}\) does not converge pointwise to \(f\), this does not mean that it diverges pointwise;
  it may just be that it converges for some points \(x\) but diverges at other points \(x\).)
\end{rmk}

\begin{note}
  If a series \(\sum_{n = 1}^\infty f^{(n)}\) converges uniformly to \(f\), then it also converges pointwise to \(f\);
  but not vice versa.
\end{note}

\setcounter{thm}{4}
\begin{defn}[Sup norm]\label{3.5.5}
  If \(f : X \to \R\) is a bounded real-valued function, we define the \emph{sup norm} \(\norm*{f}_\infty\) of \(f\) to be the number
  \[
    \norm*{f}_\infty \coloneqq \sup\big\{\abs{f(x)} : x \in X\big\}.
  \]
  In other words, \(\norm*{f}_\infty = d_\infty(f, 0)\), where \(0 : X \to \R\) is the zero function \(0(x) \coloneqq 0\), and \(d_\infty\) was defined in \cref{3.4.2}.
  We restrict the definition of \(\norm*{f}_\infty\) to the case when \(X \neq \emptyset\).
  If \(X = \emptyset\), then we instead define \(\norm*{f}_\infty = 0\).
\end{defn}

\begin{note}
  When \(f\) is bounded then \(\norm*{f}_\infty\) will always be a non-negative real number.
\end{note}

\setcounter{thm}{6}
\begin{thm}[Weierstrass \(M\)-test]\label{3.5.7}
  Let \((X, d)\) be a metric space, and let \((f^{(n)})_{n = 1}^\infty\) be a sequence of bounded real-valued continuous functions on \(X\) such that the series \(\sum_{n = 1}^\infty \norm*{f^{(n)}}_\infty\) is convergent.
  (Note that this is a series of plain old real numbers, not of functions.)
  Then the series \(\sum_{n = 1}^\infty f^{(n)}\) converges uniformly to some function \(f\) on \(X\), and that function \(f\) is also continuous.
\end{thm}

\begin{proof}
  Let \(N_1, N_2 \in \Z^+\).
  Let \(d_{C(X \to \R)} = d_{B(X \to \R)}|_{C(X \to \R) \times C(X \to \R)}\).
  We have
  \begin{align*}
             & \sum_{n = 1}^\infty \norm*{f^{(n)}}_\infty = \lim_{N \to \infty} \sum_{n = 1}^N \norm*{f^{(n)}}_\infty                                                                                     \\
    \implies & \forall \varepsilon \in \R^+, \exists\ M \in \Z^+ : \forall N \geq M,                                                                                                                      \\
             & \abs{\sum_{n = 1}^\infty \norm*{f^{(n)}}_\infty - \sum_{n = 1}^N \norm*{f^{(n)}}_\infty} < \varepsilon &                                 & \text{(by \cref{1.1.14})}                       \\
    \implies & \forall \varepsilon \in \R^+, \exists\ M \in \Z^+ : \forall N \geq M,                                                                                                                      \\
             & \abs{\sum_{n = N + 1}^\infty \norm*{f^{(n)}}_\infty} < \varepsilon                                     &                                 & \text{(by Proposition 7.2.14(c) in Analysis I)} \\
    \implies & \forall \varepsilon \in \R^+, \exists\ M \in \Z^+ : \forall N \geq M,                                                                                                                      \\
             & \sum_{n = N + 1}^\infty \norm*{f^{(n)}}_\infty < \varepsilon                                           & (\norm*{f^{(n)}}_\infty \geq 0)                                                   \\
    \implies & \forall \varepsilon \in \R^+, \exists\ M \in \Z^+ : \forall N \geq M,                                                                                                                      \\
             & \sum_{n = N + 1}^\infty \sup_{x \in X} \abs{f^{(n)}(x)} < \varepsilon.                                 &                                 & \text{(by \cref{3.5.5})}
  \end{align*}
  Fix one \(\varepsilon\) and \(M\).
  Since \(f^{(n)} \in C(X \to \R)\), by \cref{ex:3.5.1} we know that \(\sum_{n = 1}^N f^{(n)} \in C(X \to \R)\) for each \(N \in \Z^+\).
  Thus \(d_{C(X \to \R)}\bigg(\sum_{n = 1}^{N_1} f^{(n)}, \sum_{n = 1}^{N_2} f^{(n)}\bigg)\) is well defined for each \(N_1, N_2 \geq M\) and
  \begin{align*}
    \forall N_1, N_2 \geq M, & d_{C(X \to \R)}\bigg(\sum_{n = 1}^{N_1} f^{(n)}, \sum_{n = 1}^{N_2} f^{(n)}\bigg)                                             \\
                             & = \sup_{x \in X} \abs{\sum_{n = 1}^{N_1} f^{(n)}(x) - \sum_{n = 1}^{N_2} f^{(n)}(x)}            &  & \text{(by \cref{3.4.2})} \\
                             & = \sup_{x \in X} \abs{\sum_{n = \min(N_1, N_2) + 1}^{\max(N_1, N_2)} f^{(n)}(x)}                                              \\
                             & \leq \sup_{x \in X} \bigg(\sum_{n = \min(N_1, N_2) + 1}^{\max(N_1, N_2)} \abs{f^{(n)}(x)}\bigg)                               \\
                             & \leq \sup_{x \in X} \bigg(\sum_{n = M + 1}^\infty \abs{f^{(n)}(x)}\bigg)                                                      \\
                             & \leq \sum_{n = M + 1}^\infty \sup_{x \in X} \abs{f^{(n)}(x)} < \varepsilon.
  \end{align*}
  Since \(\varepsilon\) is arbitrary, we have
  \[
    \forall \varepsilon \in \R^+, \exists\ M \in \Z^+ : \forall N_1, N_2 \geq M, d_{C(X \to \R)}\bigg(\sum_{n = 1}^{N_1} f^{(n)}, \sum_{n = 1}^{N_2} f^{(n)}\bigg) < \varepsilon.
  \]
  By \cref{1.4.6} \(\bigg(\sum_{n = 1}^N f^{(n)}\bigg)_{N = 1}^\infty\) is a Cauchy sequence in \(\big(C(X \to \R), d_{C(X \to \R)}\big)\).
  Since \((\R, d_{l^1}|_{\R \times \R})\) is complete, by \cref{3.4.5} we know that \(\bigg(\sum_{n = 1}^N f^{(n)}\bigg)_{N = 1}^\infty\) converges uniformly to some \(f \in C(X \to \R)\) on \(X\) with respect to \(d_{l^1}|_{\R \times \R}\).
\end{proof}

\begin{note}
  To put the Weierstrass \(M\)-test succinctly:
  absolute convergence of sup norms implies uniform convergence of functions.
\end{note}

\begin{eg}\label{3.5.8}
  Let \(0 < r < 1\) be a real number, and let \(f^{(n)} : [-r, r] \to \R\) be the series of functions \(f^{(n)}(x) \coloneqq x^n\).
  Then each \(f^{(n)}\) is continuous and bounded, and \(\norm*{f^{(n)}}_\infty = r^n\).
  Since the series \(\sum_{n = 1}^\infty r^n\) is absolutely convergent (e.g., by the root test, Theorem 7.5.1 in Analysis I), we thus see that \(\sum_{n = 1}^\infty f^{(n)}\) converges uniformly in \([-r, r]\) to some continuous function;
  in \cref{ex:3.2.2}(c) we see that this function must in fact be the function \(f : [-r, r] \to \R\) defined by \(f(x) \coloneqq x / (1 - x)\).
  In other words, the series \(\sum_{n = 1}^\infty x^n\) is pointwise convergent, but not uniformly convergent, on \((-1, 1)\), but is uniformly convergent on the smaller interval \([-r, r]\) for any \(0 < r < 1\).
\end{eg}

\begin{note}
  The Weierstrass \(M\)-test is especially useful in relation to power series.
\end{note}

\exercisesection

\begin{ex}\label{ex:3.5.1}
  Let \(f^{(1)}, \dots, f^{(N)}\) be a finite sequence of bounded functions from a metric space \((X, d_X)\) to \(\R\).
  Show that \(\sum_{i = 1}^N f^{(i)}\) is also bounded.
  Prove a similar claim when ``bounded'' is replaced by ``continuous''.
  What if ``continuous'' was replaced by ``uniformly continuous''?
\end{ex}

\begin{proof}
  Let \(d_1 = d_{l^1}|_{\R \times \R}\).
  We first show that \(\sum_{n = 1}^N f^{(n)}\) is bounded on \(X\) with respect to \(d_1\) for each \(N \in \Z^+\).
  Suppose that \(f^{(n)}\) is bounded on \(X\) with respect to \(d_1\) for each \(n \in \Z^+\).
  We use induction on \(N\).
  For \(N = 1\), by hypothesis we know that \(\sum_{n = 1}^1 f^{(n)} = f^{(1)}\) is bounded on \(X\).
  Thus the base case holds.
  Suppose inductively that \(\sum_{n = 1}^N f^{(n)}\) is bounded on \(X\) with respect to \(d_1\) for some \(N \geq 1\).
  By induction hypothesis we have
  \[
    \exists\ M \in \R^+ : \bigg(\sum_{n = 1}^N f^{(n)}\bigg)(X) \subseteq [-M, M].
  \]
  By hypothesis we know that \(f^{(N + 1)}\) is bounded on \(X\) with respect to \(d_1\), thus we have
  \[
    \exists\ M' \in \R^+ : f^{(N + 1)}(X) \subseteq [-M', M'].
  \]
  Then we have
  \begin{align*}
    \bigg(\sum_{n = 1}^{N + 1} f^{(n)}\bigg)(X) & = \bigg\{\sum_{n = 1}^{N + 1} f^{(n)}(x) : x \in X\bigg\}            \\
                                                & = \bigg\{\sum_{n = 1}^N f^{(n)}(x) + f^{(N + 1)}(x) : x \in X\bigg\} \\
                                                & \subseteq [-(M + M'), M + M'].
  \end{align*}
  This closes the induction.

  Next we show that \(\sum_{n = 1}^N f^{(n)}\) is continuous from \((X, d_X)\) to \((\R, d_1)\) for each \(N \in \Z^+\).
  Suppose that \(f^{(n)}\) is continuous from \((X, d_X)\) to \((\R, d_1)\) for each \(n \in \Z^+\).
  We use induction on \(N\).
  For \(N = 1\), by hypothesis we know that \(\sum_{n = 1}^1 f^{(n)} = f^{(1)}\) is continuous from \((X, d_X)\) to \((\R, d_1)\).
  Thus the base case holds.
  Suppose inductively that \(\sum_{n = 1}^N f^{(n)}\) is continuous from \((X, d_X)\) to \((\R, d_1)\) for some \(N \geq 1\).
  Then by \cref{ac:2.2.1}
  \[
    \sum_{n = 1}^{N + 1} f^{(n)} = \bigg(\sum_{n = 1}^N f^{(n)}\bigg) \oplus f^{(N + 1)}
  \]
  is continuous from \((X, d_X)\) to \((\R, d_1)\).
  This closes the induction.

  Finally we show that \(\sum_{n = 1}^N f^{(n)}\) is uniformly continuous from \((X, d_X)\) to \((\R, d_1)\) for each \(N \in \Z^+\).
  Suppose that \(f^{(n)}\) is uniformly continuous from \((X, d_X)\) to \((\R, d_1)\) for each \(n \in \Z^+\).
  We use induction on \(N\).
  For \(N = 1\), by hypothesis we know that \(\sum_{n = 1}^1 f^{(n)} = f^{(1)}\) is uniformly continuous from \((X, d_X)\) to \((\R, d_1)\).
  Thus the base case holds.
  Suppose inductively that \(\sum_{n = 1}^N f^{(n)}\) is uniformly continuous from \((X, d_X)\) to \((\R, d_1)\) for some \(N \geq 1\).
  Then by \cref{ex:2.3.5}
  \[
    \sum_{n = 1}^{N + 1} f^{(n)} = \bigg(\sum_{n = 1}^N f^{(n)}\bigg) \oplus f^{(N + 1)}
  \]
  is uniformly continuous from \((X, d_X)\) to \((\R, d_1)\).
  This closes the induction.
\end{proof}

\begin{ex}\label{ex:3.5.2}
  Prove \cref{3.5.7}.
\end{ex}

\begin{proof}
  See \cref{3.5.7}.
\end{proof}
\section{Uniform convergence and integration}\label{ii:sec:3.6}

\begin{thm}\label{ii:3.6.1}
  Let \([a, b]\) be an interval, and for each integer \(n \geq 1\), let \(f^{(n)} : [a, b] \to \R\) be a Riemann-integrable function.
  Suppose \(f^{(n)}\) converges uniformly on \([a, b]\) to a function \(f : [a, b] \to \R\).
  Then \(f\) is also Riemann integrable, and
  \[
    \lim_{n \to \infty} \int_{[a, b]} f^{(n)} = \int_{[a, b]} f.
  \]
\end{thm}

\begin{proof}
  We first show that \(f\) is Riemann integrable on \([a, b]\).
  This is the same as showing that the upper and lower Riemann integrals of \(f\) match:
  \(\underline{\int}_{[a, b]} f = \overline{\int}_{[a, b]} f\).

  Let \(\varepsilon > 0\).
  Since \(f^{(n)}\) converges uniformly to \(f\), we see that there exists an \(N > 0\) such that \(\abs{f^{(n)}(x) - f(x)} < \varepsilon\) for all \(n > N\) and \(x \in [a, b]\).
  In particular we have
  \[
    f^{(n)}(x) - \varepsilon < f(x) < f^{(n)}(x) + \varepsilon
  \]
  for all \(x \in [a, b]\).
  Integrating this on \([a, b]\) we obtain
  \[
    \underline{\int}_{[a, b]} (f^{(n)} - \varepsilon) \leq \underline{\int}_{[a, b]} f \leq \overline{\int}_{[a, b]} f \leq \overline{\int}_{[a, b]} (f^{(n)} + \varepsilon).
  \]
  Since \(f^{(n)}\) is assumed to be Riemann integrable, we thus see
  \[
    \Bigg(\int_{[a, b]} f^{(n)}\Bigg) - \varepsilon (b - a) \leq \underline{\int}_{[a, b]} f \leq \overline{\int}_{[a, b]} f \leq \Bigg(\int_{[a, b]} f^{(n)}\Bigg) + \varepsilon (b - a).
  \]
  In particular, we see that
  \[
    0 \leq \overline{\int}_{[a, b]} f - \underline{\int}_{[a, b]} f \leq 2 \varepsilon (b - a).
  \]
  Since this is true for every \(\varepsilon > 0\), we obtain \(\underline{\int}_{[a, b]} f = \overline{\int}_{[a, b]} f\) as desired.

  The above argument also shows that for every \(\varepsilon > 0\) there exists an \(N > 0\) such that
  \[
    \abs{\int_{[a, b]} f^{(n)} - \int_{[a, b]} f} \leq \varepsilon (b - a)
  \]
  for all \(n \geq N\).
  Since \(\varepsilon\) is arbitrary, we see that \(\int_{[a, b]} f^{(n)}\) converges to \(\int_{[a, b]} f\) as desired.
\end{proof}

\begin{note}
  To rephrase \cref{ii:3.6.1}:
  we can rearrange limits and integrals (on compact intervals \([a, b]\)),
  \[
    \lim_{n \to \infty} \int_{[a, b]} f^{(n)} = \int_{[a, b]} \lim_{n \to \infty} f^{(n)},
  \]
  \emph{provided that} the convergence is uniform.
\end{note}

\begin{cor}\label{ii:3.6.2}
  Let \([a, b]\) be an interval, and let \((f^{(n)})_{n = 1}^\infty\) be a sequence of Riemann integrable functions on \([a, b]\) such that the series \(\sum_{n = 1}^\infty f^{(n)}\) is uniformly convergent.
  Then we have
  \[
    \sum_{n = 1}^\infty \int_{[a, b]} f^{(n)} = \int_{[a, b]} \sum_{n = 1}^\infty f^{(n)}.
  \]
\end{cor}

\begin{proof}
  By Theorem 11.4.1(a) in Analysis I we know that
  \[
    \forall N \in \Z^+, \int_{[a, b]} \sum_{n = 1}^N f^{(n)} = \sum_{n = 1}^N \int_{[a, b]} f^{(n)}.
  \]
  Let \(f : [a, b] \to \R\) be the function such that \(\sum_{n = 1}^\infty f^{(n)}\) converges uniformly to \(f\) on \([a, b]\) with respect to \(d_{l^1}|_{\R \times \R}\).
  By \cref{ii:3.6.1} we have
  \[
    \sum_{n = 1}^\infty \int_{[a, b]} f^{(n)} = \lim_{N \to \infty} \sum_{n = 1}^N \int_{[a, b]} f^{(n)} = \lim_{N \to \infty} \int_{[a, b]} \sum_{n = 1}^N f^{(n)} = \int_{[a, b]} f = \int_{[a, b]} \sum_{n = 1}^\infty f^{(n)}.
  \]
\end{proof}

\begin{note}
  \cref{ii:3.6.2} works particularly well in conjunction with the Weierstrass \(M\)-test
  (\cref{ii:3.5.7}).
\end{note}

\exercisesection

\begin{ex}\label{ii:ex:3.6.1}
  Use \cref{ii:3.6.1} to prove \cref{ii:3.6.2}.
\end{ex}

\begin{proof}
  See \cref{ii:3.6.2}.
\end{proof}

\section{Uniform convergence and derivatives}\label{sec:3.7}

\begin{note}
  In particular we have
  \[
    \dfrac{d}{dx} \lim_{n \to \infty} f_n(x) \neq \lim_{n \to \infty} \dfrac{d}{dx} f_n(x)
  \]
  So, in summary, uniform convergence of the functions \(f_n\) says nothing about the convergence of the derivatives \(f_n'\).
\end{note}

\begin{thm}\label{3.7.1}
  Let \([a, b]\) be an interval, and for every integer \(n \geq 1\), let \(f_n : [a, b] \to \R\) be a differentiable function whose derivative \(f_n' : [a, b] \to \R\) is continuous.
  Suppose that the derivatives \(f_n'\) converge uniformly to a function \(g : [a, b] \to \R\).
  Suppose also that there exists a point \(x_0\) such that the limit \(\lim_{n \to \infty} f_n(x_0)\) exists.
  Then the functions \(f_n\) converge uniformly to a differentiable function \(f\), and the derivative of \(f\) equals \(g\).
\end{thm}

\begin{proof}
  Since \(f_n'\) is continuous, by Corollary 11.5.2 in Analysis I we know that \(f_n'\) is Riemann integrable.
  We see from the fundamental theorem of calculus (Theorem 11.9.4 in Analysis I) that
  \[
    f_n(x) - f_n(x_0) = \int_{[x_0, x]} f_n'
  \]
  when \(x \in [x_0, b]\), and
  \[
    f_n(x) - f_n(x_0) = -\int_{[x, x_0]} f_n'
  \]
  when \(x \in [a, x_0]\).
  Let \(L\) be the limit of \(f_n(x_0)\) as \(n \to \infty\):
  \[
    L \coloneqq \lim_{n \to \infty} f_n(x_0).
  \]
  By hypothesis, \(L\) exists.
  Now, since each \(f_n'\) is continuous on \([a, b]\), and \(f_n'\) converges uniformly to \(g\), we see by \cref{3.3.2} that \(g\) is also continuous.
  By \cref{3.6.1} we have
  \[
    \forall x \in [a, b], \lim_{n \to \infty} \big(f_n(x) - f_n(a)\big) = \lim_{n \to \infty} \int_{[a, x]} f_n' = \int_{[a, x]} \big(\lim_{n \to \infty} f_n'\big) = \int_{[a, x]} g.
  \]
  Now define the function \(f : [a, b] \to \R\) by setting
  \[
    f(x) \coloneqq L - \int_{[a, x_0]} g + \int_{[a, x]} g
  \]
  for all \(x \in [a, b]\).
  To finish the proof, we have to show that \(f_n\) converges uniformly to \(f\), and that \(f\) is differentiable with derivative \(g\).

  We know that \(a \neq b\) since if \(a = b\), then we have \(x_0 = a = b\) and
  \[
    \forall n \in \Z^+, \lim_{x \to x_0; x \in \{x_0\} \setminus \{x_0\}} \dfrac{f_n(x) - f_n(x_0)}{x - x_0} \text{ is undefined}
  \]
  which contradict to the hypothesis that \(f_n\) is differentiable on \([a, b]\).
  Observe that
  \begin{align*}
             & L = \lim_{n \to \infty} f_n(x_0)                                                                                              \\
    \implies & \forall \varepsilon \in \R^+, \exists\ N_1 \in \Z^+ : \forall n \geq N_1, \abs{f_n(x_0) - L} < \dfrac{\varepsilon}{3(b - a)}.
  \end{align*}
  Now we fix one pair of \(\varepsilon\) and \(N_1\).
  Since \((f_n')_{n = 1}^\infty\) converges uniformly to \(g\) on \([a, b]\) with respect to \(d_{l^1}|_{\R \times \R}\), by \cref{3.2.7} we have
  \begin{align*}
             & \exists\ N_2 \in \Z^+ : \forall n \geq N_2, \forall x \in [a, b],                                                                 \\
             & \abs{f_n'(x) - g(x)} < \dfrac{\varepsilon}{3(b - a)}                                                                              \\
    \implies & \exists\ N_2 \in \Z^+ : \forall n \geq N_2, \forall x \in [a, b],                                                                 \\
             & \dfrac{-\varepsilon}{3(b - a)} < f_n'(x) - g(x) < \dfrac{\varepsilon}{3(b - a)}                                                   \\
    \implies & \exists\ N_2 \in \Z^+ : \forall n \geq N_2, \forall x \in [a, b],                                                                 \\
             & \dfrac{-\varepsilon (x - a)}{3(b - a)} \leq \int_{[a, x]} f_n'(x) - \int_{[a, x]} g(x) \leq \dfrac{\varepsilon (x - a)}{3(b - a)} \\
    \implies & \exists\ N_2 \in \Z^+ : \forall n \geq N_2, \forall x \in [a, b],                                                                 \\
             & \dfrac{-\varepsilon (x - a)}{3(b - a)} \leq f_n(x) - f_n(a) - \int_{[a, x]} g(x) \leq \dfrac{\varepsilon (x - a)}{3(b - a)}       \\
    \implies & \exists\ N_2 \in \Z^+ : \forall n \geq N_2, \forall x \in [a, b],                                                                 \\
             & \abs{f_n(x) - f_n(a) - \int_{[a, x]} g(x)} \leq \dfrac{\varepsilon \abs{x - a}}{3(b - a)}.
  \end{align*}
  Let \(N = \max(N_1, N_2)\).
  Then we have
  \begin{align*}
     & \forall n \geq N, \forall x \in [a, b], \abs{f_n(x) - f(x)}                                                               \\
     & = \abs{f_n(x) - f_n(x_0) + f_n(x_0) - f_n(a) + f_n(a) - L + \int_{[a, x_0]} g - \int_{[a, x]} g}                          \\
     & \leq \abs{f_n(x) - f_n(a) - \int_{[a, x]} g} + \abs{f_n(x_0) - L} + \abs{f_n(x_0) - f_n(a) - \int_{[a, x_0]} g}           \\
     & < \dfrac{\varepsilon \abs{x - a}}{3(b - a)} + \dfrac{\varepsilon \abs{x_0 - a}}{3(b - a)} + \dfrac{\varepsilon}{3(b - a)} \\
     & < \dfrac{\varepsilon}{3} + \dfrac{\varepsilon}{3} + \dfrac{\varepsilon}{3} = \varepsilon.
  \end{align*}
  Since \(\varepsilon\) is arbitrary, we have
  \[
    \forall \varepsilon \in \R^+, \exists\ N \in \Z^+ : \forall n \geq N, \forall x \in [a, b], \abs{f_n(x) - f(x)} < \varepsilon
  \]
  and by \cref{3.2.7} \((f_n)_{n = 1}^\infty\) converges uniformly to \(f\) with respect to \(d_{l^1}|_{\R \times \R}\).

  Since \(f_n\) is continuous on \([a, b]\) for each \(n \in \Z^+\), by \cref{3.3.2} we know that \(f\) is also continuous on \([a, b]\).
  Since \(g\) is continuous on \([a, b]\), by fundamental theorem of calculus (Theorem 11.9.1 in Analysis) we know that
  \[
    \forall x \in X, G(x) = \int_{[a, x]} g \text{ is differentiable at } x.
  \]
  Since \(L + \int_{[a, x_0]} g\) is constant, we know that
  \[
    \forall x \in X, f(x) = L + \int_{[a, x_0]} g + G(x) = L + \int_{[a, x_0]} g + \int_{[a, x]} g \text{ is differentiable at } x
  \]
  and by fundamental theorem of calculus (Theorem 11.9.1 in Analysis) we have
  \[
    \forall x \in X, f'(x) = \bigg(\int_{[a, x]} g\bigg)' = g(x).
  \]
\end{proof}

\begin{note}
  Informally, \cref{3.7.1} says that if \(f_n'\) converges uniformly, and \(f_n(x_0)\) converges for some \(x_0\), then \(f_n\) also converges uniformly, and
  \[
    \dfrac{d}{dx} \lim_{n \to \infty} f_n(x) = \lim_{n \to \infty} \dfrac{d}{dx} f_n(x)
  \]
\end{note}

\begin{rmk}\label{3.7.2}
  It turns out that \cref{3.7.1} is still true when the functions \(f_n'\) are not assumed to be continuous, but the proof is more difficult;
  see \cref{ex:3.7.2}.
\end{rmk}

\begin{cor}\label{3.7.3}
  Let \([a, b]\) be an interval, and for every integer \(n \geq 1\), let \(f_n : [a, b] \to \R\) be a differentiable function whose derivative \(f_n' : [a, b] \to \R\) is continuous.
  Suppose that the series \(\sum_{n = 1}^\infty \norm*{f_n'}_\infty\) is absolutely convergent, where
  \[
    \norm*{f_n'}_\infty \coloneqq \sup_{x \in [a, b]} \abs{f_n'(x)}
  \]
  is the sup norm of \(f_n'\), as defined in \cref{3.5.5}.
  Suppose also that the series \(\sum_{n = 1}^\infty f_n(x_0)\) is convergent for some \(x_0 \in [a, b]\).
  Then the series \(\sum_{n = 1}^\infty f_n\) converges uniformly on \([a, b]\) to a differentiable function, and in fact
  \[
    \dfrac{d}{dx} \sum_{n = 1}^\infty f_n(x) = \sum_{n = 1}^\infty \dfrac{d}{dx} f_n(x)
  \]
  for all \(x \in [a, b]\).
\end{cor}

\begin{proof}
  Let \(F_N = \sum_{n = 1}^N f_n\) for each \(N \in \Z^+\).
  Then by Theorem 10.1.13(c) in Analysis I we have
  \[
    \forall N \in \Z^+, F_N' = \bigg(\sum_{n = 1}^N f_n\bigg)' = \sum_{n = 1}^N f_n'.
  \]
  Since \(f_n'\) is continuous on \([a, b]\) for each \(n \in \Z^+\), by Proposition 9.6.7 in Analysis I we know that \(f_n' \in B\big([a, b] \to \R\big)\) and thus \(f_n' \in C\big([a, b] \to \R\big)\).
  By \cref{ex:3.5.1} we know that
  \[
    \forall N \in \Z^+, F_N' = \sum_{n = 1}^N f_n' \in C([a, b] \to \R).
  \]
  Since \(\sum_{n = 1}^\infty \norm*{f_n'}_\infty\) converges and \(f_n' \in C\big([a, b] \to \R\big)\) for each \(n \in \Z^+\), by \cref{3.5.7} we know that there exists some \(G : [a, b] \to \R\) such that \(\big(\sum_{n = 1}^N f_n'\big)_{N = 1}^\infty\) converges uniformly to \(G\) on \([a, b]\) with respect to \(d_{l^1}|_{\R \times \R}\).
  Equivalently, \((F_N')_{N = 1}^\infty\) converges uniformly to \(G\) on \([a, b]\) with respect to \(d_{l^1}|_{\R \times \R}\).
  Since
  \[
    \sum_{n = 1}^\infty f_n(x_0) = \lim_{N \to \infty} \sum_{n = 1}^N f_n(x_0) = \lim_{N \to \infty} F_N(x_0),
  \]
  by \cref{3.7.1} we know that there exists some \(F : [a, b] \to \R\) such that \((F_N)_{N = 1}^\infty\) converges uniformly to \(F\) on \([a, b]\) with respect to \(d_{l^1}|_{\R \times \R}\) and \(F' = G\).
  Then we have
  \begin{align*}
             & \forall x \in [a, b], \begin{dcases}
                                       F(x) = \lim_{N \to \infty} F_N(x) = \lim_{N \to \infty} \sum_{n = 1}^N f_n(x) = \sum_{n = 1}^\infty f_n(x)    \\
                                       G(x) = \lim_{N \to \infty} F_N'(x) = \lim_{N \to \infty} \sum_{n = 1}^N f_n'(x) = \sum_{n = 1}^\infty f_n'(x) \\
                                       F'(x) = G(x)
                                     \end{dcases} \\
    \implies & \forall x \in [a, b], \bigg(\sum_{n = 1}^\infty f_n\bigg)'(x) = \sum_{n = 1}^\infty f_n'(x)                                         \\
    \implies & \bigg(\sum_{n = 1}^\infty f_n\bigg)' = \sum_{n = 1}^\infty f_n'.
  \end{align*}
\end{proof}

\begin{note}
  \cref{3.7.4} was discovered by Weierstrass.
\end{note}

\begin{eg}\label{3.7.4}
  Let \(f : \R \to \R\) be the function
  \[
    f(x) \coloneqq \sum_{n = 1}^\infty 4^{-n} \cos(32^n \pi x).
  \]
  Note that this series is uniformly convergent, thanks to the Weierstrass \(M\)-test, and since each individual function \(4^{-n} \cos(32^n \pi x)\) is continuous, the function \(f\) is also continuous.
  However, it is not differentiable;
  in fact it is a \emph{nowhere differentiable function}, one which is not differentiable at any point, despite being continuous everywhere!
\end{eg}

\exercisesection

\begin{ex}\label{ex:3.7.1}
  Complete the proof of \cref{3.7.1}.
  Compare this theorem with Example 1.2.10 in Analysis I, and explain why this example does not contradict the theorem.
\end{ex}

\begin{proof}
  See \cref{3.7.1}.
  Since \(\lim_{n \to \infty} \dfrac{x^3}{\dfrac{1}{n} + x^2}\) is not continuous at \(x = 0\), is does not contradict to \cref{3.7.1}.
\end{proof}

\begin{ex}\label{ex:3.7.2}
  Prove \cref{3.7.1} without assuming that \(f_n'\) is continuous.
  (This means that you cannot use the fundamental theorem of calculus.
  However, the mean value theorem (Corollary 10.2.9 in Analysis I) is still available.
  Use this to show that if \(d_\infty(f_n', f_m') \leq \varepsilon\), then \(\abs{\big(f_n(x) - f_m(x)\big) - \big(f_n(x_0) - f_m(x_0)\big)} \leq \varepsilon \abs{x - x_0}\) for all \(x \in [a, b]\), and then use this to complete the proof of \cref{3.7.1}.)
\end{ex}

\begin{proof}
  Let \(m \in \Z^+\).
  Let \(d_{C([a, b] \to \R)} = d_{B([a, b] \to \R)}|_{C([a, b] \to \R) \times C([a, b] \to \R)}\).
  Since \(f_n\) is differentiable on \([a, b]\), by Corollary 10.1.12 and Proposition 9.6.7 in Analysis I we know that \(f_n \in C([a, b] \to \R)\) for each \(n \in \Z^+\).
  Since \((f_n')_{n = 1}^\infty\) converges uniformly to \(g\) on \([a, b]\) with respect to \(d_{l^1}|_{\R \times \R}\), by \cref{3.2.7} we have
  \begin{align*}
             & \forall \varepsilon \in \R^+, \exists\ N_1 \in \Z^+ : \forall n \geq N_1, \forall x \in [a, b], \abs{f_n'(x) - g(x)} < \dfrac{\varepsilon}{6(b - a)} \\
    \implies & \forall \varepsilon \in \R^+, \exists\ N_1 \in \Z^+ : \forall n, m \geq N_1, \forall x \in [a, b],                                                   \\
             & \abs{f_n'(x) - f_m'(x)} \leq \abs{f_n'(x) - g(x)} + \abs{f_m'(x) - g(x)} < \dfrac{\varepsilon}{6(b - a)} + \dfrac{\varepsilon}{6(b - a)}             \\
    \implies & \forall \varepsilon \in \R^+, \exists\ N_1 \in \Z^+ : \forall n, m \geq N_1, \forall x \in [a, b],                                                   \\
             & \abs{(f_n' - f_m')(x)} < \dfrac{\varepsilon}{3(b - a)}.
  \end{align*}
  Now we fix one such \(\varepsilon\) and \(N_1\).
  Let \(x \in [a, b]\).
  We split into two cases:
  \begin{itemize}
    \item \(x = x_0\).
          Then we have
          \[
            \forall n, m \geq N_1, \abs{\big(f_n(x_0) - f_m(x_0)\big) - \big(f_n(x_0) - f_m(x_0)\big)} = 0 = \dfrac{\varepsilon \abs{x_0 - x_0}}{3(b - a)}.
          \]
    \item \(x \neq x_0\).
          Suppose that \(x < x_0\).
          Since \([x, x_0] \subseteq [a, b]\), by mean value theorem we know that
          \[
            \forall n, m \geq N_1, \exists\ y \in (x, x_0) : (f_n - f_m)'(y) = \dfrac{(f_n - f_m)(x_0) - (f_n - f_m)(x)}{x_0 - x}.
          \]
          Now suppose that \(x > x_0\).
          Since \([x_0, x] \subseteq [a, b]\), by mean value theorem we know that
          \[
            \forall n, m \geq N_1, \exists\ y \in (x_0, x) : (f_n - f_m)'(y) = \dfrac{(f_n - f_m)(x) - (f_n - f_m)(x_0)}{x - x_0}.
          \]
          In either cases we have
          \begin{align*}
                     & \forall n, m \geq N_1, \exists\ y \in (a, b) :                                                                                           \\
                     & \abs{\dfrac{(f_n - f_m)(x) - (f_n - f_m)(x_0)}{x - x_0}} = \abs{(f_n - f_m)'(y)}                                                         \\
            \implies & \forall n, m \geq N_1, \exists\ y \in (a, b) :                                                                                           \\
                     & \abs{\big(f_n(x) - f_m(x)\big) - \big(f_n(x_0) - f_m(x_0)\big)}                                                                          \\
                     & = \abs{f_n'(y) - f_m'(y)} \abs{x - x_0}                                                                                                  \\
                     & \leq \dfrac{\varepsilon \abs{x - x_0}}{3(b - a)}                                                                                         \\
            \implies & \forall n, m \geq N_1, \abs{\big(f_n(x) - f_m(x)\big) - \big(f_n(x_0) - f_m(x_0)\big)} \leq \dfrac{\varepsilon \abs{x - x_0}}{3(b - a)}.
          \end{align*}
  \end{itemize}
  From all cases above we conclude that
  \[
    \forall n, m \geq N_1, \forall x \in [a, b], \abs{\big(f_n(x) - f_m(x)\big) - \big(f_n(x_0) - f_m(x_0)\big)} \leq \dfrac{\varepsilon \abs{x - x_0}}{3(b - a)}.
  \]

  Let \(L = \lim_{n \to \infty} f_n(x_0)\).
  Then we have
  \begin{align*}
             & \lim_{n \to \infty} f_n(x_0) = L                                                        \\
    \implies & \exists\ N_2 \in \Z^+ : \forall n \geq N_2, \abs{f_n(x_0) - L} < \dfrac{\varepsilon}{3}
  \end{align*}
  Let \(N = \max(N_1, N_2)\).
  Then we have
  \begin{align*}
     & \forall n, m \geq N, \forall x \in [a, b],                                                                     \\
     & \abs{f_n(x) - f_m(x)}                                                                                          \\
     & = \abs{\big(f_n(x) - f_m(x)\big) - \big(f_n(x_0) - f_m(x_0)\big) + \big(f_n(x_0) - f_m(x_0)\big) + L - L}      \\
     & \leq \abs{\big(f_n(x) - f_m(x)\big) - \big(f_n(x_0) - f_m(x_0)\big)} + \abs{f_n(x_0) - L} + \abs{f_m(x_0) - L} \\
     & < \dfrac{\varepsilon \abs{x - x_0}}{3(b - a)} + \dfrac{\varepsilon}{3} + \dfrac{\varepsilon}{3}                \\
     & \leq \dfrac{\varepsilon}{3} + \dfrac{\varepsilon}{3} + \dfrac{\varepsilon}{3} = \varepsilon.
  \end{align*}
  Since \(\varepsilon\) is arbitrary, by \cref{3.4.2} we have
  \begin{align*}
             & \forall \varepsilon \in \R^+, \exists\ N \in \Z^+ : \forall n, m \geq N, \forall x \in [a, b], \abs{f_n(x) - f_m(x)} < \varepsilon \\
    \implies & \forall \varepsilon \in \R^+, \exists\ N \in \Z^+ : \forall n, m \geq N, \sup_{x \in [a, b]}\abs{f_n(x) - f_m(x)} < \varepsilon    \\
    \implies & \forall \varepsilon \in \R^+, \exists\ N \in \Z^+ : \forall n, m \geq N, d_{C([a, b] \to \R)}(f_n, f_m) < \varepsilon.
  \end{align*}
  Thus by \cref{1.4.6} \((f_n)_{n = 1}^\infty\) is a Cauchy sequence in \(\big(C([a, b] \to \R), d_{C([a, b] \to \R)}\big)\).
  Since \((\R, d_{l^1}|_{\R \times \R})\) is complete, by \cref{3.4.5} we know that there exists some \(f : [a, b] \to \R\) such that
  \[
    \begin{dcases}
      f \in C([a, b] \to \R) \\
      d_{C([a, b] \to \R)} - \lim_{n \to \infty} f_n = f
    \end{dcases}
  \]
  By \cref{3.4.4} we know that \((f_n)_{n = 1}^\infty\) convergent uniformly to \(f\) on \([a, b]\) with respect to \(d_{l^1}|_{\R \times \R}\).

  Now we show that \(f\) is differentiable on \([a, b]\) and \(f' = g\).
  Let \(y \in [a, b]\).
  Since \((f_n')_{n = 1}^\infty\) converges uniformly to \(g\) on \([a, b]\) with respect to \(d_{l^1}|_{\R \times \R}\), we have
  \begin{align*}
    g(y) & = \lim_{n \to \infty} f_n'(y)                                                                                   &  & \text{(by \cref{ex:3.2.2}(a))} \\
         & = \lim_{n \to \infty} \bigg(\lim_{x \to y ; x \in [a, b] \setminus \{y\}} \dfrac{f_n(x) - f_n(y)}{x - y}\bigg)                                      \\
         & = \lim_{x \to y ; x \in [a, b] \setminus \{y\}} \bigg(\lim_{n \to \infty} \dfrac{f_n(x) - f_n(y)}{x - y}\bigg). &  & \text{(by \cref{3.3.3})}
  \end{align*}
  Since \((f_n)_{n = 1}^\infty\) converges uniformly to \(f\) on \([a, b]\) with respect to \(d_{l^1}|_{\R \times \R}\), we have
  \begin{align*}
             & \forall x \in [a, b], f(x) = \lim_{n \to \infty} f_n(x)                                                                         &  & \text{(by \cref{ex:3.2.2}(a))} \\
    \implies & \forall x \in [a, b], f(x) - f(y) = \lim_{n \to \infty} f_n(x) - \lim_{n \to \infty} f_n(y)                                                                         \\
    \implies & \forall x \in [a, b], f(x) - f(y) = \lim_{n \to \infty} \big(f_n(x) - f_n(y)\big)                                                                                   \\
    \implies & \forall x \in [a, b] \setminus \{y\}, \dfrac{f(x) - f(y)}{x - y} = \dfrac{\lim_{n \to \infty} \big(f_n(x) - f_n(y)\big)}{x - y}                                     \\
    \implies & \forall x \in [a, b] \setminus \{y\}, \dfrac{f(x) - f(y)}{x - y} = \lim_{n \to \infty} \dfrac{f_n(x) - f_n(y)}{x - y}.
  \end{align*}
  Thus we have
  \[
    g(y) = \lim_{x \to y ; x \in [a, b] \setminus \{y\}} \bigg(\lim_{n \to \infty} \dfrac{f_n(x) - f_n(y)}{x - y}\bigg) = \lim_{x \to y ; x \in [a, b] \setminus \{y\}} \dfrac{f(x) - f(y)}{x - y} = f'(y).
  \]
  Since \(y\) is arbitrary, we conclude that \(f\) is differentiable on \([a, b]\) and \(f' = g\).
\end{proof}

\begin{ex}\label{ex:3.7.3}
  Prove \cref{3.7.3}.
\end{ex}

\begin{proof}
  See \cref{3.7.3}.
\end{proof}
\section{Uniform approximation by polynomials}\label{ii:sec:3.8}

\begin{note}
  As we have just seen, continuous functions can be very badly behaved, for instance they can be nowhere differentiable (\cref{ii:3.7.4}).
  On the other hand, functions such as polynomials are always very well behaved, in particular being always differentiable.
  Fortunately, while most continuous functions are not as well behaved as polynomials, they can always be \emph{uniformly approximated} by polynomials; this important (but difficult) result is known as the \emph{Weierstrass approximation theorem},
\end{note}

\begin{defn}\label{ii:3.8.1}
  Let \([a, b]\) be an interval.
  A \emph{polynomial on \([a, b]\)} is a
  function \(f : [a, b] \to \R\) of the form \(f(x) \coloneqq \sum_{j = 0}^n c_j x^j\), where \(n \geq 0\) is an integer and \(c_0, \dots, c_n\) are real numbers.
  If \(c_n \neq 0\), then \(n\) is called the \emph{degree} of \(f\).
\end{defn}

\setcounter{thm}{2}
\begin{thm}[Weierstrass approximation theorem]\label{ii:3.8.3}
  If \([a, b]\) is an interval, \(f : [a, b] \to \R\) is a continuous function, and \(\varepsilon > 0\), then there exists a polynomial \(P\) on \([a, b]\) such that \(d_\infty(P, f) \leq \varepsilon\)
  (i.e., \(\abs{P(x) - f(x)} \leq \varepsilon\) for all \(x \in [a, b]\)).
\end{thm}

\begin{proof}
  Let \(f : [a, b] \to \R\) be a continuous function on \([a, b]\).
  Let \(g : [0, 1] \to \R\) denote the function
  \[
    g(x) \coloneqq f\big(a + (b - a) x\big) \text{ for all } x \in [0, 1]
  \]
  Observe then that
  \[
    f(y) = g(\dfrac{y - a}{b - a}) \text{ for all } y \in [a, b].
  \]
  The function \(g\) is continuous on \([0, 1]\) since \(y \mapsto \dfrac{y - a}{b - a}\) is bijective on \([a, b]\), and so by \cref{ii:3.8.19} we may find a polynomial \(Q : [0, 1] \to \R\) such that \(\abs{Q(x) - g(x)} \leq \varepsilon\) for all \(x \in [0, 1]\).
  In particular, for any \(y \in [a, b]\), we have
  \[
    \abs{Q(\dfrac{y - a}{b - a}) - g(\dfrac{y - a}{b - a})} \leq \varepsilon.
  \]
  If we thus set \(P(y) \coloneqq Q(\dfrac{y - a}{b - a})\), then we observe that \(P\) is also a polynomial since \(y \mapsto \dfrac{y - a}{b - a}\) is bijective on \([a, b]\), and so we have \(\abs{P(y) - f(y)} \leq \varepsilon\) for all \(y \in [a, b]\), as desired.
\end{proof}

\begin{note}
  Another way of stating \cref{ii:3.8.3} is as follows.
  Recall that \(C([a, b] \to \R)\) was the space of continuous functions from \([a, b]\) to \(\R\), with the uniform metric \(d_\infty\).
  Let \(P([a, b] \to \R)\) be the space of all polynomials on \([a, b]\);
  this is a subspace of \(C([a, b] \to \R)\), since all polynomials are continuous (Exercise 9.4.7 in Analysis I).
  The Weierstrass approximation theorem then asserts that every continuous function is an adherent point of \(P([a, b] \to \R)\);
  or in other words, that the closure of the space of polynomials is the space of continuous functions (see \cref{ii:3.3.2}):
  \[
    \overline{P([a, b] \to \R)}_{\big(C([a, b] \to \R), d_\infty\big)} = C([a, b] \to \R).
  \]
  In particular, every continuous function on \([a, b]\) is the uniform limit of polynomials (see \cref{ii:3.4.4}).
  Another way of saying this is that the space of polynomials is \emph{dense} in the space of continuous functions, in the \emph{uniform topology}.
\end{note}

\begin{defn}[Compactly supported functions]\label{ii:3.8.4}
  Let \([a, b]\) be an interval.
  A function \(f : \R \to \R\) is said to be \emph{supported} on \([a, b]\) iff \(f(x) = 0\) for all \(x \notin [a, b]\).
  We say that \(f\) is \emph{compactly supported} iff it is supported on some interval \([a, b]\).
  If \(f\) is continuous and supported on \([a, b]\), we define the improper integral \(\int_{-\infty}^\infty f\) to be \(\int_{-\infty}^\infty f \coloneqq \int_{[a, b]} f\).
\end{defn}

\begin{note}
  A function can be supported on more than one interval, for instance a function which is supported on \([3, 4]\) is also automatically supported on \([2, 5]\).
\end{note}

\begin{lem}\label{ii:3.8.5}
  If \(f : \R \to \R\) is continuous and supported on an interval \([a, b]\), and is also supported on another interval \([c, d]\), then \(\int_{[a, b]} f = \int_{[c, d]} f\).
\end{lem}

\begin{proof}
  Since
  \begin{align*}
             & \begin{dcases}
                 f \text{ is supported on } [a, b] \\
                 f \text{ is supported on } [c, d]
               \end{dcases}                        \\
    \implies & \begin{dcases}
                 \forall x \notin [a, b], f(x) = 0 \\
                 \forall x \notin [c, d], f(x) = 0
               \end{dcases}                        &  & \by{ii:3.8.4}   \\
    \implies & \begin{dcases}
                 \forall x \in \R, (x < a) \lor (x > b) \implies f(x) = 0 \\
                 \forall x \in \R, (x < c) \lor (x > d) \implies f(x) = 0
               \end{dcases}
  \end{align*}
  we have
  \begin{align*}
    \int_{-\infty}^\infty f & = \int_{[a, b]} f                                                                                                            &  & \by{ii:3.8.4} \\
                            & = \begin{dcases}
                                  \int_{[a, c]} f + \int_{[c, b]} f & \text{if } a \leq c \\
                                  0 + \int_{[a, b]} f               & \text{if } a > c
                                \end{dcases}                     \\
                            & = \begin{dcases}
                                  0 + \int_{[c, b]} f               & \text{if } a \leq c \\
                                  \int_{[c, a]} f + \int_{[a, b]} f & \text{if } a > c
                                \end{dcases}                     \\
                            & = \int_{[c, b]} f                                                                                                                               \\
                            & = \begin{dcases}
                                  \int_{[c, b]} f + 0               & \text{if } b \leq d \\
                                  \int_{[c, d]} f + \int_{[d, b]} f & \text{if } b > d
                                \end{dcases}                     \\
                            & = \begin{dcases}
                                  \int_{[c, b]} f + \int_{[b, d]} f & \text{if } b \leq d \\
                                  \int_{[c, d]} f + 0               & \text{if } b > d
                                \end{dcases}                     \\
                            & = \int_{[c, d]} f.
  \end{align*}
\end{proof}

\begin{defn}[Approximation to the identity]\label{ii:3.8.6}
  Let \(\varepsilon > 0\) and \(0 < \delta < 1\).
  A function \(f : \R \to \R\) is said to be an \emph{\((\varepsilon, \delta)\)-approximation to the identity} if it obeys the following three properties:
  \begin{enumerate}
    \item \(f\) is supported on \([-1, 1]\), and \(f(x) \geq 0\) for all \(-1 \leq x \leq 1\).
    \item \(f\) is continuous, and \(\int_{-\infty}^\infty f = 1\).
    \item \(\abs{f(x)} \leq \varepsilon\) for all \(\delta \leq \abs{x} \leq 1\).
  \end{enumerate}
\end{defn}

\begin{rmk}\label{ii:3.8.7}
  For those of you who are familiar with the Dirac delta function, approximations to the identity are ways to approximate this (very discontinuous) delta function by a continuous function (which is easier to analyze).
\end{rmk}

\begin{lem}[Polynomials can approximate the identity]\label{ii:3.8.8}
  For every \(\varepsilon > 0\) and \(0 < \delta < 1\) there exists an \((\varepsilon, \delta)\)-approximation to the identity which is a polynomial \(P\) on \([-1, 1]\).
\end{lem}

\begin{proof}
  Let \(\varepsilon \in \R^+\) and let \(\delta \in (0, 1)\).
  We have
  \begin{align*}
             & \forall x \in [-1, 1], \delta \leq \abs{x} \leq 1                                                                             \\
    \implies & \delta^2 \leq x^2 \leq 1                                                                                                      \\
    \implies & 0 \leq 1 - x^2 \leq 1 - \delta^2 < 1                                                                                          \\
    \implies & \lim_{n \to \infty} \sqrt{n} (1 - \delta^2)^n = 0                               &  & \text{(by Exercise 7.5.2 in Analysis I)} \\
    \implies & \exists N \in \Z^+ : \forall n \geq N, \sqrt{n} (1 - \delta^2)^n < \varepsilon.
  \end{align*}
  Now we fix such \(N\).
  Define \(g : \R \to \R\) to be the function
  \[
    \forall x \in \R, g(x) = \begin{dcases}
      (1 - x^2)^N & \text{if } x \in [-1, 1]    \\
      0           & \text{if } x \notin [-1, 1]
    \end{dcases}
  \]
  We know that \(g(x) \geq 0\) for all \(x \in \R\).
  By \cref{ii:3.8.4} we know that \(g\) is supported on \([-1, 1]\).
  By Exercise 9.4.7 in Analysis I we know that \(g\) is continuous on \([-1, 1]\), thus by Corollary 11.5.2 in Analysis I \(g\) is Riemann integrable on \([-1, 1]\).
  By \cref{ii:ex:3.8.2}(b) we know that
  \[
    \int_{[-1, 1]} g = \int_{[-1, 1]} (1 - x^2)^N \geq \dfrac{1}{\sqrt{N}} > 0,
  \]
  so we can define \(c = (\int_{[-1, 1]} g)^{-1}\) and we have
  \[
    0 < c = \bigg(\int_{[-1, 1]} g\bigg)^{-1} \leq \sqrt{N}.
  \]
  Now we define \(f : \R \to \R\) to be the function \(f = cg\).
  Again we have \(f\) is continuous and supported on \([-1, 1]\).
  Since \(c > 0\), we know that \(f(x) \geq 0\) for all \(x \in \R\).
  By \cref{ii:3.8.4} we have
  \[
    \int_{-\infty}^\infty f = \int_{[-1, 1]} f = \int_{[-1, 1]} cg = c \int_{[-1, 1]} g = \bigg(\int_{[-1, 1]} g\bigg)^{-1} \bigg(\int_{[-1, 1]} g\bigg) = 1.
  \]
  Since
  \begin{align*}
             & \forall x \in [-1, 1], \delta \leq \abs{x} \leq 1                                                     \\
    \implies & 0 \leq 1 - x^2 \leq 1 - \delta^2 < 1                                                                  \\
    \implies & 0 \leq \sqrt{N} (1 - x^2)^N \leq \sqrt{N} (1 - \delta^2)^N < \varepsilon                              \\
    \implies & 0 \leq \abs{f(x)} = \abs{cg(x)} \leq \abs{\sqrt{N} (1 - x^2)^N} = \sqrt{N} (1 - x^2)^N < \varepsilon,
  \end{align*}
  combine all the proofs above we conclude by \cref{ii:3.8.6} that \(f\) is an \((\varepsilon, \delta)\)-approximation to the identity.
\end{proof}

\begin{defn}[Convolution]\label{ii:3.8.9}
  Let \(f : \R \to \R\) and \(g : \R \to \R\) be continuous, compactly supported functions.
  We define the \emph{convolution} \(f * g : \R \to \R\) of \(f\) and \(g\) to be the function
  \[
    (f * g)(x) \coloneqq \int_{-\infty}^\infty f(y) g(x - y) \; dy.
  \]
\end{defn}

\begin{note}
  If \(f\) and \(g\) are continuous and compactly supported, then for each \(x\) the function \(f(y) g(x - y)\) (thought of as a function of \(y\)) is also continuous and compactly supported, so \cref{ii:3.8.9} makes sense.
\end{note}

\begin{rmk}\label{ii:3.8.10}
  Convolutions play an important role in Fourier analysis and in partial differential equations (PDE), and are also important in physics, engineering, and signal processing.
\end{rmk}

\begin{prop}[Basic properties of convolution]\label{ii:3.8.11}
  Let \(f : \R \to \R\), \(g : \R \to \R\), and \(h : \R \to \R\) be continuous, compactly supported functions.
  Then the following statements are true.
  \begin{enumerate}
    \item The convolution \(f * g\) is also a continuous, compactly supported function.
    \item (Convolution is commutative)
          We have \(f * g = g * f\);
          in other words
          \begin{align*}
            f * g(x) & = \int_{-\infty}^\infty f(y) g(x - y) \; dy \\
                     & = \int_{-\infty}^\infty g(y) f(x - y) \; dy \\
                     & = g * f(x).
          \end{align*}
    \item (Convolution is linear)
          We have \(f * (g + h) = f * g + f * h\).
          Also, for any real number \(c\), we have \(f * (cg) = (cf) * g = c(f * g)\).
  \end{enumerate}
\end{prop}

\begin{proof}{(a)}
  Since \(f, g\) are compactly supported, by \cref{ii:3.8.4} we know that
  \[
    \exists L_f, L_g, U_f, U_g \in \R : \begin{dcases}
      \forall y \in \R \setminus [L_f, U_f], f(y) = 0 \\
      \forall y \in \R \setminus [L_g, U_g], g(y) = 0
    \end{dcases}
  \]
  Note that we can choose \(L_f \neq U_f\).
  Let \(L = \min(L_f, L_g)\), let \(U = \max(U_f, U_g)\) and let \(M = \max(\abs{L}, \abs{U})\).
  Then we have
  \begin{align*}
             & \forall y \in \R \setminus [-M, M], \begin{dcases}
                                                     y < -M \leq L \leq L_f \implies f(y) = 0 \\
                                                     y < -M \leq L \leq L_g \implies g(y) = 0 \\
                                                     y > M \geq U \geq U_f \implies f(y) = 0  \\
                                                     y > M \geq U \geq U_g \implies g(y) = 0
                                                   \end{dcases} \\
    \implies & f(y) = g(y) = 0
  \end{align*}
  and
  \[
    \forall y \in \R \setminus [-2M, 2M], (y < -2M \leq -M) \lor (y > 2M \geq M) \implies f(y) = g(y) = 0.
  \]
  Thus by \cref{ii:3.8.4} \(f, g\) are supported on \([-M, M]\) and \([-2M, 2M]\).
  Observe that
  \begin{align*}
             & \forall x \in (-\infty, -2M), \forall y \in \R, \begin{dcases}
                                                                 x - y < -M \text{ or } \\
                                                                 x - y \geq -M
                                                               \end{dcases}                      \\
    \implies & \forall x \in (-\infty, -2M), \forall y \in \R, \begin{dcases}
                                                                 x - y < -M \text{ or } \\
                                                                 -M > x + M \geq y
                                                               \end{dcases}                      \\
    \implies & \forall x \in (-\infty, -2M), \forall y \in \R, \begin{dcases}
                                                                 g(x - y) = 0 & \text{if } x - y < -M        \\
                                                                 f(y) = 0     & \text{if } -M > x + M \geq y
                                                               \end{dcases} \\
    \implies & \forall x \in (-\infty, -2M), \forall y \in \R, f(y) g(x - y) = 0
  \end{align*}
  and
  \begin{align*}
             & \forall x \in (2M, +\infty), \forall y \in \R, \begin{dcases}
                                                                x - y > M \text{ or } \\
                                                                x - y \leq M
                                                              \end{dcases}                      \\
    \implies & \forall x \in (2M, +\infty), \forall y \in \R, \begin{dcases}
                                                                x - y > M \text{ or } \\
                                                                M < x - M \leq y
                                                              \end{dcases}                      \\
    \implies & \forall x \in (2M, +\infty), \forall y \in \R, \begin{dcases}
                                                                g(x - y) = 0 & \text{if } x - y > M        \\
                                                                f(y) = 0     & \text{if } M < x - M \leq y
                                                              \end{dcases} \\
    \implies & \forall x \in (2M, +\infty), \forall y \in \R, f(y) g(x - y) = 0.
  \end{align*}
  This means
  \[
    \forall x \in \R \setminus [-2M, 2M], \forall y \in \R, f(y) g(x - y) = 0.
  \]
  For each \(x \in \R \setminus [-2M, 2M]\), we define \(z_x : \R \to \R\) by setting \(z_x(y) = f(y) g(x - y)\).
  Since \(z_x\) is continuous on \(\R\), by \cref{ii:3.8.4} and \cref{ii:3.8.9} we have
  \begin{align*}
             & \forall x \in \R \setminus [-2M, 2M], \forall y \in \R, z_x(y) = 0                                         \\
    \implies & \forall x \in \R \setminus [-2M, 2M], \forall y \in \R \setminus [-1, 1], z_x(y) = 0                       \\
    \implies & \forall x \in \R \setminus [-2M, 2M], z_x \text{ is supported on } [-1, 1]                                 \\
    \implies & \forall x \in \R \setminus [-2M, 2M], \int_{-\infty}^\infty z_x = \int_{[-1, 1]} z_x = 0                   \\
    \implies & \forall x \in \R \setminus [-2M, 2M], \int_{-\infty}^\infty z_x(y) \; dy = \int_{[-1, 1]} z_x(y) \; dy = 0 \\
    \implies & \forall x \in \R \setminus [-2M, 2M], (f * g)(x) = \int_{[-1, 1]} f(y) g(x - y) \; dy = 0                  \\
    \implies & f * g \text{ is supported on } [-2M, 2M]                                                                   \\
    \implies & f * g \text{ is compactly supported}.
  \end{align*}
  Since \(f, g\) are compactly supported and continuous on \(\R\), by \cref{ii:ex:3.8.3} we know that
  \[
    \exists N \in \R^+ : \forall x \in \R, \abs{f(x)} \leq N
  \]
  and
  \[
    \forall \varepsilon \in \R^+, \exists \delta \in \R^+ : \forall x_1, x_2 \in \R, \abs{x_1 - x_2} < \delta \implies \abs{g(x_1) - g(x_2)} < \dfrac{\varepsilon}{N (U_f - L_f)}.
  \]
  Fix \(N\) and one pair of \(\varepsilon\) and \(\delta\).
  Let \(x_0 \in \R\).
  Then we have
  \begin{align*}
             & \forall x \in \R, \abs{x - x_0} < \delta                                                                                        \\
    \implies & \abs{(f * g)(x) - (f * g)(x_0)} = \abs{\int_{-\infty}^\infty f(y) g(x - y) \; dy - \int_{-\infty}^\infty f(y) g(x_0 - y) \; dy} \\
             & = \abs{\int_{[L_f, U_f]} f(y) g(x - y) \; dy - \int_{[L_f, U_f]} f(y) g(x_0 - y) \; dy}                                         \\
             & = \abs{\int_{[L_f, U_f]} f(y) \big(g(x - y) - g(x_0 - y)\big) \; dy}                                                            \\
             & \leq \abs{\int_{[L_f, U_f]} N \dfrac{\varepsilon}{N (U_f - L_f)} \; dy} = \varepsilon.
  \end{align*}
  Since \(\varepsilon\) was arbitrary, we know that \(f * g\) is continuous at \(x_0\).
  Since \(x_0\) was arbitrary, we know that \(f * g\) is continuous on \(\R\).
\end{proof}

\begin{proof}{(b)}
  Let \(x_0 \in \R\).
  Since \(f\) is compactly supported, we know that
  \[
    \exists L, U \in \R : \forall y \in \R \setminus [L, U], f(y) = 0.
  \]
  Then we have
  \begin{align*}
             & \forall y \in \R \setminus [L, U], f(y) = 0             \\
    \implies & \forall y \in \R \setminus [L, U], f(y) g(x_0 - y) = 0.
  \end{align*}
  Observe that
  \begin{align*}
             & \forall y \in \R \setminus [L, U], f(y) = 0                         \\
    \implies & \forall y \in \R \setminus [-U, -L], f(-y) = 0                      \\
    \implies & \forall y \in \R \setminus [x_0 - U, x_0 - L], f(x_0 - y) = 0       \\
    \implies & \forall y \in \R \setminus [x_0 - U, x_0 - L], g(y) f(x_0 - y) = 0.
  \end{align*}
  Since \(f, g\) are continuous on \(\R\), we know that
  \begin{align*}
             & \forall y_0 \in \R, \begin{dcases}
                                     f \text{ is continuous at } y_0       \\
                                     g \text{ is continuous at } y_0       \\
                                     f \text{ is continuous at } x_0 - y_0 \\
                                     g \text{ is continuous at } x_0 - y_0 \\
                                     y \mapsto x_0 - y \text{ is continuous at } y_0
                                   \end{dcases}                    \\
    \implies & \forall y_0 \in \R, \begin{dcases}
                                     \lim_{y \to y_0 ; y \in \R} f(y) = f(y_0)             \\
                                     \lim_{y \to y_0 ; y \in \R} g(y) = g(y_0)             \\
                                     \lim_{y \to y_0 ; y \in \R} f(x_0 - y) = f(x_0 - y_0) \\
                                     \lim_{y \to y_0 ; y \in \R} g(x_0 - y) = g(x_0 - y_0)
                                   \end{dcases}             \\
    \implies & \forall y_0 \in \R, \begin{dcases}
                                     \lim_{y \to y_0 ; y \in \R} f(y) g(x_0 - y) = f(y_0) g(x_0 - y_0) \\
                                     \lim_{y \to y_0 ; y \in \R} g(y) f(x_0 - y) = g(y_0) f(x_0 - y_0)
                                   \end{dcases}
  \end{align*}
  This means
  \begin{align*}
    (f * g)(x_0) & = \int_{-\infty}^\infty f(y) g(x_0 - y) \; dy      &  & \by{ii:3.8.9} \\
                 & = \int_{[L, U]} f(y) g(x_0 - y) \; dy;             &  & \by{ii:3.8.4} \\
    (g * f)(x_0) & = \int_{-\infty}^\infty g(y) f(x_0 - y) \; dy      &  & \by{ii:3.8.9} \\
                 & = \int_{[x_0 - U, x_0 - L]} g(y) f(x_0 - y) \; dy; &  & \by{ii:3.8.4}
  \end{align*}
  Let \(\phi : \R \to \R\) be the function \(\phi = y \mapsto x_0 - y\).
  By the formula of changing variable (Exercise 11.10.4 in Analysis I) we have
  \begin{align*}
     & \int_{[L, U]} f(y) g(x_0 - y) \; dy                                                     \\
     & = \int_{[\phi(x_0 - U), \phi(x_0 - L)]} f(y) g(x_0 - y) \; dy                           \\
     & = -\int_{[x_0 - U, x_0 - L]} f\big(\phi(y)\big) g\big(x_0 - \phi(y)\big) \phi'(y) \; dy \\
     & = \int_{[x_0 - U, x_0 - L]} f(x_0 - y) g(y) \; dy                                       \\
     & = \int_{[x_0 - U, x_0 - L]} g(y) f(x_0 - y) \; dy.
  \end{align*}
  Thus \((f * g)(x_0) = (g * f)(x_0)\).
  Since \(x_0\) was arbitrary, we conclude that
  \[
    \forall x \in \R, (f * g)(x) = (g * f)(x).
  \]
\end{proof}

\begin{proof}{(c)}
  Let \(x_0 \in \R\).
  Since \(g, h\) are compactly supported, by \cref{ii:3.8.4} we know that
  \[
    \exists L_g, L_h, U_g, U_h \in \R : \begin{dcases}
      \forall y \in \R \setminus [L_g, U_g], g(y) = 0 \\
      \forall y \in \R \setminus [L_h, U_h], h(y) = 0
    \end{dcases}
  \]
  Let \(L = \min(L_g, L_h)\) and let \(U = \min(U_g, U_h)\).
  Then we have
  \begin{align*}
             & \forall y \in \R \setminus [L, U], \begin{dcases}
                                                    y < L \leq L_g & \implies g(y) = 0 \\
                                                    y > U \geq U_g & \implies g(y) = 0 \\
                                                    y < L \leq L_h & \implies h(y) = 0 \\
                                                    y > U \geq U_h & \implies h(y) = 0
                                                  \end{dcases}                      \\
    \implies & \forall y \in \R \setminus [L, U], g(y) = h(y) = 0                                         \\
    \implies & \forall y \in \R \setminus [-U, -L], g(-y) = h(-y) = 0                                     \\
    \implies & \forall y \in \R \setminus [x_0 - U, x_0 - L], g(x_0 - y) = h(x_0 - y) = 0                 \\
    \implies & \forall y \in \R \setminus [x_0 - U, x_0 - L], f(y) g(x_0 - y) = f(y) h(x_0 - y) = 0       \\
    \implies & \forall y \in \R \setminus [x_0 - U, x_0 - L], f(y) \big(g(x_0 - y) + h(x_0 - y)\big) = 0.
  \end{align*}
  Since \(f, g, h\) are continuous on \(\R\), we know that
  \begin{align*}
             & \forall y_0 \in \R, \begin{dcases}
                                     f \text{ is continuous at } y_0 \\
                                     g \text{ is continuous at } y_0 \\
                                     h \text{ is continuous at } y_0 \\
                                     y \mapsto x_0 - y \text{ is continuous at } y_0
                                   \end{dcases}                                                                     \\
    \implies & \forall y_0 \in \R, \begin{dcases}
                                     \lim_{y \to y_0 ; y \in \R} f(y) = f(y_0)             \\
                                     \lim_{y \to y_0 ; y \in \R} g(x_0 - y) = g(x_0 - y_0) \\
                                     \lim_{y \to y_0 ; y \in \R} h(x_0 - y) = h(x_0 - y_0)
                                   \end{dcases}                                                              \\
    \implies & \forall y_0 \in \R, \begin{dcases}
                                     \lim_{y \to y_0 ; y \in \R} f(y) g(x_0 - y) = f(y_0) g(x_0 - y_0) \\
                                     \lim_{y \to y_0 ; y \in \R} f(y) h(x_0 - y) = f(y_0) h(x_0 - y_0)
                                   \end{dcases}                                                  \\
    \implies & \forall y_0 \in \R, \lim_{y \to y_0 ; y \in \R} f(y) \big(g(x_0 - y) + h(x_0 - y)\big) = f(y_0) \big(g(x_0 - y_0) + h(x_0 - y_0)\big).
  \end{align*}
  Thus we have
  \begin{align*}
     & \big(f * (g + h)\big)(x_0)                                                                                                  \\
     & = \int_{-\infty}^\infty f(y) (g + h)(x_0 - y) \; dy                                                      &  & \by{ii:3.8.9} \\
     & = \int_{-\infty}^\infty f(y) \big(g(x_0 - y) + h(x_0 - y)\big) \; dy                                                        \\
     & = \int_{[x_0 - U, x_0 - L]} f(y) \big(g(x_0 - y) + h(x_0 - y)\big) \; dy                                 &  & \by{ii:3.8.4} \\
     & = \int_{[x_0 - U, x_0 - L]} f(y) g(x_0 - y) \; dy + \int_{[x_0 - U, x_0 - L]} f(y) h(x_0 - y)\big) \; dy                    \\
     & = \int_{-\infty}^\infty f(y) g(x_0 - y) \; dy + \int_{-\infty}^\infty f(y) h(x_0 - y) \; dy              &  & \by{ii:3.8.4} \\
     & = (f * g)(x_0) + (f * h)(x_0).                                                                           &  & \by{ii:3.8.9}
  \end{align*}
  Observe that
  \[
    \forall y \in \R \setminus [x_0 - U, x_0 - L], f(y) g(x_0 - y) = c f(y) g(x_0 - y) = 0.
  \]
  Since \(f\) is continuous on \(\R\), we know that \(cf\) is also continuous on \(\R\) and
  \begin{align*}
             & \forall y_0 \in \R, \begin{dcases}
                                     cf \text{ is continuous at } y_0 \\
                                     g \text{ is continuous at } y_0  \\
                                     y \mapsto x_0 - y \text{ is continuous at } y_0
                                   \end{dcases}                     \\
    \implies & \forall y_0 \in \R, \begin{dcases}
                                     \lim_{y \to y_0 ; y \in \R} cf(y) = cf(y_0) \\
                                     \lim_{y \to y_0 ; y \in \R} g(x_0 - y) = g(x_0 - y_0)
                                   \end{dcases}               \\
    \implies & \forall y_0 \in \R, \begin{dcases}
                                     \lim_{y \to y_0 ; y \in \R} cf(y) g(x_0 - y) = cf(y_0) g(x_0 - y_0)
                                   \end{dcases}
  \end{align*}
  Thus we have
  \begin{align*}
     & \big((cf) * g\big)(x_0)                                                \\
     & = \int_{-\infty}^\infty (cf)(y) g(x_0 - y) \; dy    &  & \by{ii:3.8.9} \\
     & = \int_{-\infty}^\infty c f(y) g(x_0 - y) \; dy                        \\
     & = \int_{[x_0 - U, x_0 - L]} c f(y) g(x_0 - y) \; dy &  & \by{ii:3.8.4} \\
     & = c \int_{[x_0 - U, x_0 - L]} f(y) g(x_0 - y) \; dy                    \\
     & = c \int_{-\infty}^\infty f(y) g(x_0 - y) \; dy     &  & \by{ii:3.8.4} \\
     & = c (f * g)(x_0).                                   &  & \by{ii:3.8.9}
  \end{align*}
  Using similar arguments we can show that \(\big((cg) * f\big)(x_0) = c (g * f)(x_0)\).
  By \cref{ii:3.8.11}(b) we thus have
  \[
    \big(f * (cg)\big)(x_0) = \big((cg) * f\big)(x_0) = c(g * f)(x_0) = c(f * g)(x_0) = \big((cf) * g\big)(x_0).
  \]
  Since \(x_0\) was arbitrary, we conclude that
  \[
    \forall x \in \R, \begin{dcases}
      \big(f * (g + h)\big)(x) = (f * g)(x) + (f * h)(x) \\
      \big(f * (cg)\big)(x) = \big((cf) * g\big)(x) = c(f * g)(x)
    \end{dcases}
  \]
\end{proof}

\begin{rmk}\label{ii:3.8.12}
  There are many other important properties of convolution, for instance it is associative, \((f * g) * h = f * (g * h)\), and it commutes with derivatives, \((f * g)' = f' * g = f * g'\), when \(f\) and \(g\) are differentiable.
  The Dirac delta function \(\delta\) mentioned earlier is an identity for convolution:
  \(f * \delta = \delta * f = f\).
  These results are slightly harder to prove than the ones in \cref{ii:3.8.11}, however, and we will not need them in this text.
\end{rmk}

\begin{lem}\label{ii:3.8.13}
  Let \(f : \R \to \R\) be a continuous function supported on \([0, 1]\), and let \(g : \R \to \R\) be a continuous function supported on \([-1, 1]\) which is a polynomial on \([-1, 1]\).
  Then \(f * g\) is a polynomial on \([0, 1]\).
  (Note however that it may be non-polynomial outside of \([0, 1].\))
\end{lem}

\begin{proof}
  Since \(g\) is polynomial on \([-1, 1]\), we may find an integer \(n \geq 0\) and real numbers \(c_0, c_1, \dots, c_n\) such that
  \[
    g(x) = \sum_{j = 0}^n c_j x^j \text{ for all } x \in [-1, 1].
  \]
  On the other hand, for all \(x \in [0, 1]\), we have
  \[
    f * g(x) = \int_{-\infty}^\infty f(y) g(x - y) \; dy = \int_{[0, 1]} f(y) g(x - y) \; dy
  \]
  since \(f\) is supported on \([0, 1]\).
  Since \(x \in [0, 1]\) and the variable of integration \(y\) is also in \([0, 1]\), we have \(x - y \in [-1, 1]\).
  Thus we may substitute in our formula for \(g\) to obtain
  \[
    f * g(x) = \int_{[0, 1]} f(y) \sum_{j = 0}^n c_j (x - y)^j \; dy.
  \]
  We expand this using the binomial formula (Exercise 7.1.4 in Analysis I) to obtain
  \[
    f * g(x) = \int_{[0, 1]} f(y) \sum_{j = 0}^n c_j \sum_{k = 0}^j \dfrac{j!}{k! (j - k)!} x^k (-y)^{j - k} \; dy.
  \]
  We can interchange the two summations (by Corollary 7.1.14 in Analysis I) to obtain
  \[
    f * g(x) = \int_{[0, 1]} f(y) \sum_{k = 0}^n \sum_{j = k}^n c_j \dfrac{j!}{k! (j - k)!} x^k (-y)^{j - k} \; dy.
  \]
  (why did the limits of summation change? It may help to plot \(j\) and \(k\) on a graph).
  Now we interchange the \(k\) summation with the integral, and observe that \(x^k\) is independent of \(y\), to obtain
  \[
    f * g(x) = \sum_{k = 0}^n x^k \int_{[0, 1]} f(y) \sum_{j = k}^n c_j \dfrac{j!}{k! (j - k)!} (-y)^{j - k} \; dy.
  \]
  If we thus define
  \[
    C_k \coloneqq \int_{[0, 1]} f(y) \sum_{j = k}^n c_j \dfrac{j!}{k! (j - k)!} (-y)^{j - k} \; dy
  \]
  for each \(k = 0, \dots, n\), then \(C_k\) is a number which is independent of \(x\), and we have
  \[
    f * g(x) = \sum_{k = 0}^n C_k x^k
  \]
  for all \(x \in [0, 1]\).
  Thus \(f * g\) is a polynomial on \([0, 1]\).
\end{proof}

\begin{lem}\label{ii:3.8.14}
  Let \(f : \R \to \R\) be a continuous function supported on \([0, 1]\), which is bounded by some \(M > 0\) (i.e., \(\abs{f(x)} \leq M\) for all \(x \in \R\)), and let \(\varepsilon > 0\) and \(0 < \delta < 1\) be such that one has \(\abs{f(x) - f(y)} < \varepsilon\) whenever \(x, y \in \R\) and \(\abs{x - y} < \delta\).
  Let \(g\) be any \((\varepsilon, \delta)\)-approximation to the identity.
  Then we have
  \[
    \abs{f * g(x) - f(x)} \leq (1 + 4M) \varepsilon
  \]
  for all \(x \in [0, 1]\).
\end{lem}

\begin{proof}
  Since \(g\) is an \((\varepsilon, \delta)\)-approximation to the identity, by \cref{ii:3.8.6} we have
  \begin{itemize}
    \item \(g\) is supported on \([-1, 1]\) and \(g(x) \geq 0\) for all \(x \in [-1, 1]\).
    \item \(g\) is continuous on \(\R\) and \(\int_{-\infty}^\infty g = 1\).
    \item \(\abs{g(x)} \leq \varepsilon\) for all \(\delta \leq \abs{x} \leq 1\).
  \end{itemize}
  Since \(f\) is continuous on \(\R\), by \cref{ii:3.8.9} we have
  \begin{align*}
     & \forall x \in [0, 1], (f * g)(x)                                                                                                    \\
     & = \int_{-\infty}^\infty g(y) f(x - y) \; dy                                                                                         \\
     & = \int_{[-1, 1]} g(y) f(x - y) \; dy                                                                                                \\
     & = \int_{[-1, -\delta]} g(y) f(x - y) \; dy + \int_{[-\delta, \delta]} g(y) f(x - y) \; dy + \int_{[\delta, 1]} g(y) f(x - y) \; dy.
  \end{align*}
  By \cref{ii:ex:3.8.6} we have
  \[
    1 - 2 \varepsilon \leq \int_{[-\delta, \delta]} g = \int_{[-\delta, \delta]} g(y) \; dy \leq 1.
  \]
  Since
  \begin{align*}
             & \begin{dcases}
                 \forall x \in \R, \abs{f(x)} \leq M \\
                 \forall y \in \R, \delta \leq \abs{y} \leq 1 \implies \abs{g(y)} < \varepsilon
               \end{dcases}                                                                         \\
    \implies & \forall x \in \R, \forall \delta \leq \abs{y} \leq 1, g(y) f(x - y) \leq M \varepsilon                                                                \\
    \implies & \forall x \in \R, \begin{dcases}
                                   -M \varepsilon (1 - \delta) \leq \int_{[-1, -\delta]} g(y) f(x - y) \; dy \leq M \varepsilon (1 - \delta) \\
                                   -M \varepsilon (1 - \delta) \leq \int_{[\delta, 1]} g(y) f(x - y) \; dy \leq M \varepsilon (1 - \delta)
                                 \end{dcases} \\
    \implies & \forall x \in \R, \begin{dcases}
                                   -M \varepsilon \leq \int_{[-1, -\delta]} g(y) f(x - y) \; dy \leq M \varepsilon \\
                                   -M \varepsilon \leq \int_{[\delta, 1]} g(y) f(x - y) \; dy \leq M \varepsilon
                                 \end{dcases}                           & (\delta < 1)                           \\
    \implies & \forall x \in \R, \abs{\int_{[-1, -\delta]} g(y) f(x - y) \; dy + \int_{[\delta, 1]} g(y) f(x - y) \; dy} \leq 2 M \varepsilon
  \end{align*}
  and
  \begin{align*}
             & \forall x \in [0, 1], \forall y \in [-\delta, \delta], \abs{(x - y) - x} = \abs{y} < \delta                                                                                     \\
    \implies & \forall x \in [0, 1], \forall y \in [-\delta, \delta], \abs{f(x - y) - f(x)} < \varepsilon                                    &                        & \text{(by hypothesis)} \\
    \implies & \forall x \in [0, 1], \forall y \in [-\delta, \delta],                                                                                                                          \\
             & \abs{g(y) f(x - y) - g(y) f(x)} \leq \varepsilon g(y)                                                                         & (g(y) \geq 0)                                   \\
    \implies & \forall x \in [0, 1], \forall y \in [-\delta, \delta],                                                                                                                          \\
             & g(y) f(x) - \varepsilon g(y) \leq g(y) f(x - y) \leq g(y) f(x) + \varepsilon g(y)                                                                                               \\
    \implies & \forall x \in [0, 1],                                                                                                                                                           \\
             & \big(f(x) - \varepsilon\big) \int_{[-\delta, \delta]} g(y) \; dy                                                                                                                \\
             & \leq \int_{[-\delta, \delta]} g(y) f(x - y) \; dy                                                                                                                               \\
             & \leq \big(f(x) + \varepsilon\big) \int_{[-\delta, \delta]} g(y) \; dy                                                                                                           \\
    \implies & \forall x \in [0, 1],                                                                                                                                                           \\
             & \big(f(x) - \varepsilon\big) (1 - 2 \varepsilon) \leq \int_{[-\delta, \delta]} g(y) f(x - y) \; dy \leq f(x) + \varepsilon    &                        & \by{ii:ex:3.8.6}       \\
    \implies & \forall x \in [0, 1],                                                                                                                                                           \\
             & -2 \varepsilon f(x) - \varepsilon + 2 \varepsilon^2 \leq \int_{[-\delta, \delta]} g(y) f(x - y) \; dy - f(x) \leq \varepsilon                                                   \\
    \implies & \forall x \in [0, 1],                                                                                                                                                           \\
             & -2 \varepsilon M - \varepsilon + 2 \varepsilon^2 \leq \int_{[-\delta, \delta]} g(y) f(x - y) \; dy - f(x) \leq \varepsilon    & (f(x) \leq M)                                   \\
    \implies & \forall x \in [0, 1],                                                                                                                                                           \\
             & -2 \varepsilon M - \varepsilon \leq \int_{[-\delta, \delta]} g(y) f(x - y) \; dy - f(x) \leq \varepsilon                      & (\varepsilon \in \R^+)                          \\
    \implies & \forall x \in [0, 1],                                                                                                                                                           \\
             & -\varepsilon (2M + 1) \leq \int_{[-\delta, \delta]} g(y) f(x - y) \; dy - f(x) \leq \varepsilon (2M + 1)                      & (2M + 1 > 1)                                    \\
    \implies & \forall x \in [0, 1], \abs{\int_{[-\delta, \delta]} g(y) f(x - y) \; dy - f(x)} \leq \varepsilon (2M + 1),
  \end{align*}
  we know that
  \begin{align*}
     & \forall x \in [0, 1], \abs{(f * g)(x) - f(x)}                                                                                                            \\
     & = \abs{\int_{[-1, -\delta]} g(y) f(x - y) \; dy + \int_{[-\delta, \delta]} g(y) f(x - y) \; dy - f(x) + \int_{[\delta, 1]} g(y) f(x - y) \; dy}          \\
     & \leq \abs{\int_{[-1, -\delta]} g(y) f(x - y) \; dy + \int_{[\delta, 1]} g(y) f(x - y) \; dy} + \abs{\int_{[-\delta, \delta]} g(y) f(x - y) \; dy - f(x)} \\
     & \leq 2 M \varepsilon + \varepsilon (2M + 1)                                                                                                              \\
     & = (1 + 4M) \varepsilon.
  \end{align*}
\end{proof}

\begin{cor}[Weierstrass approximation theorem I]\label{ii:3.8.15}
  Let \(f : \R \to \R\) be a continuous function supported on \([0, 1]\).
  Then for every \(\varepsilon > 0\), there exists a function \(P : \R \to \R\) which is polynomial on \([0, 1]\) and such that \(\abs{P(x) - f(x)} \leq \varepsilon\) for all \(x \in [0, 1]\).
\end{cor}

\begin{proof}
  Let \(\varepsilon \in \R^+\).
  Since \(f\) is continuous on \(\R\) and supported on \([0, 1]\), by \cref{ii:ex:3.8.3} we know that \(f\) is bounded by some \(M \in \R^+\) and \(f\) is uniformly continuous on \(\R\).
  This means
  \[
    \exists \delta \in \R^+ : \forall x_1, x_2 \in \R, \abs{x_1 - x_2} < \delta \implies \abs{f(x_1) - f(x_2)} < \dfrac{\varepsilon}{1 + 4M}.
  \]
  In particular, we can choose some \(\delta\) such that \(0 < \delta < 1\).
  By \cref{ii:3.8.8} we know that there exists a polynomial \(P\) on \([-1, 1]\) such that \(P\) is an \((\dfrac{\varepsilon}{1 + 4M}, \delta)\)-approximation to the identity.
  By \cref{ii:3.8.6} we know that \(P\) is continuous on \(\R\) and supported on \([-1, 1]\).
  Since \(f\) is continuous on \(\R\) and supported on \([0, 1]\), by \cref{ii:3.8.13} we know that \(f * P\) is a polynomial on \([0, 1]\).
  Then by \cref{ii:3.8.14} we have
  \[
    \forall x \in [0, 1], \abs{(f * P)(x) - f(x)} \leq (1 + 4M) \dfrac{\varepsilon}{1 + 4M} = \varepsilon.
  \]
  Since \(\varepsilon\) was arbitrary, we conclude that
  \[
    \forall \varepsilon \in \R^+, \exists P \in \R^{\R} : \begin{dcases}
      P \text{ is polynomial on } [0, 1] \\
      \forall x \in [0, 1], \abs{P(x) - f(x)} \leq \varepsilon
    \end{dcases}
  \]
\end{proof}

\begin{lem}\label{ii:3.8.16}
  Let \(f : [0, 1] \to \R\) be a continuous function which equals \(0\) on the boundary of \([0, 1]\), i.e., \(f(0) = f(1) = 0\).
  Let \(F : \R \to \R\) be the function defined by setting \(F(x) \coloneqq f(x)\) for \(x \in [0, 1]\) and \(F(x) \coloneqq 0\) for \(x \notin [0, 1]\).
  Then \(F\) is also continuous.
\end{lem}

\begin{proof}
  Since \(f\) is continuous on \([0, 1]\), we know that \(f\) is continuous at \(0\) and \(1\).
  Thus we have
  \begin{align*}
             & \lim_{x \to 0 ; x \in [0, 1]} f(x) = f(0) = 0                                                                                                                             \\
    \implies & \forall \varepsilon \in \R^+, \exists \delta \in \R^+ : \big(\forall x \in [0, 1], \abs{x} < \delta \implies \abs{f(x)} < \varepsilon\big)                                \\
    \implies & \forall \varepsilon \in \R^+, \exists \delta \in \R^+ : \big(\forall x \in [0, \delta], \abs{f(x)} < \varepsilon\big)                                                     \\
    \implies & \forall \varepsilon \in \R^+, \exists \delta \in \R^+ : \big(\forall x \in [0, \delta], \abs{F(x)} < \varepsilon\big)                                                     \\
    \implies & \forall \varepsilon \in \R^+, \exists \delta \in \R^+ : \big(\forall x \in \R, \abs{x} < \delta \implies \abs{F(x)} < \varepsilon\big)     & (F(x) = 0 \text{ if } x < 0) \\
    \implies & \lim_{x \to 0 ; x \in \R} F(x) = F(0) = 0.
  \end{align*}
  Similarly we have \(\lim_{x \to 1 ; x \in \R} F(x) = F(1) = 0\).
  This means \(F\) is continuous at \(0\) and \(1\).

  Since \(f\) is continuous on \([0, 1]\), we know that \(f\) is continuous on \((0, 1)\).
  Let \(x_0 \in (0, 1)\).
  Then we have
  \begin{align*}
             & \lim_{x \to x_0 ; x \in (0, 1)} f(x) = f(x_0)                                                                                                              \\
    \implies & \forall \varepsilon \in \R^+, \exists \delta \in \R^+ : \big(\forall x \in (0, 1), \abs{x - x_0} < \delta \implies \abs{f(x) - f(x_0)} < \varepsilon\big)  \\
    \implies & \forall \varepsilon \in \R^+, \exists \delta \in \R^+ : \big(\forall x \in (0, 1), \abs{x - x_0} < \delta \implies \abs{F(x) - F(x_0)} < \varepsilon\big).
  \end{align*}
  Now we fix one pair of \(\varepsilon\) and \(\delta\).
  Let \(\delta' = \min(\delta, \abs{x_0 - 0}, \abs{1 - x_0})\).
  Then we have \(\delta' \in \R^+\) and
  \begin{align*}
             & \forall x \in \R, \abs{x - x_0} < \delta' \\
    \implies & \begin{dcases}
                 \abs{x - x_0} < \abs{x_0 - 0} \\
                 \abs{x - x_0} < \abs{1 - x_0}
               \end{dcases}             \\
    \implies & \begin{dcases}
                 \abs{x - x_0} < x_0 \\
                 \abs{x - x_0} < 1 - x_0
               \end{dcases}                    \\
    \implies & \begin{dcases}
                 0 < x < 2x_0 \\
                 2x_0 - 1 < x < 1
               \end{dcases}                           \\
    \implies & \max(0, 2x_0 - 1) < x < \min(2x_0, 1)     \\
    \implies & 0 < x < 1.
  \end{align*}
  Thus
  \[
    \forall x \in \R, \abs{x - x_0} < \delta' \implies \abs{F(x) - F(x_0)} < \varepsilon.
  \]
  Since \(\varepsilon\) was arbitrary, we conclude that \(\lim_{x \to x_0 ; x \in \R} F(x) = F(x_0)\).
  Since \(x_0\) was arbitrary, we conclude that \(\lim_{x \to x_0 ; x \in \R} F(x) = F(x_0)\) for each \(x_0 \in (0, 1)\).

  Let \(x_0 \in (-\infty, 0)\) and let \(\delta = \abs{0 - x_0}\).
  Since \(F(x) = 0\) for all \(x \in (-\infty, 0)\), we have \(F(x_0) = 0\) and
  \begin{align*}
             & \forall x \in \R, \abs{x - x_0} < \delta                                \\
    \implies & \abs{x - x_0} < \abs{0 - x_0}                                           \\
    \implies & \abs{x - x_0} < -x_0                                                    \\
    \implies & 2x_0 < x < 0                                                            \\
    \implies & F(x) = 0                                                                \\
    \implies & \forall \varepsilon \in \R^+, \abs{F(x) - F(x_0)} = 0 \leq \varepsilon.
  \end{align*}
  Thus we have
  \[
    \forall \varepsilon \in \R^+, \exists \delta \in \R^+ : \forall x \in \R, \abs{x - x_0} < \delta \implies \abs{F(x) - F(x_0)} < \varepsilon
  \]
  and \(\lim_{x \to x_0 ; x \in \R} F(x) = F(x_0) = 0\).
  Since \(x_0\) was arbitrary, we conclude that
  \[
    \forall x_0 \in (-\infty, 0), \lim_{x \to x_0 ; x \in \R} F(x) = F(x_0) = 0.
  \]
  Using similar arguments we can show that \(\lim_{x \to x_0 ; x \in \R} F(x) = F(x_0) = 0\) for all \(x_0 \in (1, \infty)\).
  Combine all proofs above we have
  \[
    \forall x_0 \in \R, \lim_{x \to x_0 ; x \in \R} F(x) = F(x_0) = 0
  \]
  and thus \(F\) is continuous on \(\R\).
\end{proof}

\begin{rmk}\label{ii:3.8.17}
  The function \(F\) obtained in \cref{ii:3.8.16} is sometimes known as the \emph{extension of \(f\) by zero}.
\end{rmk}

\begin{cor}[Weierstrass approximation theorem II]\label{ii:3.8.18}
  Let \(f : [0, 1] \to \R\) be a continuous function such that \(f(0) = f(1) = 0\).
  Then for every \(\varepsilon > 0\) there exists a polynomial \(P : [0, 1] \to \R\) such that \(\abs{P(x) - f(x)} \leq \varepsilon\) for all \(x \in [0, 1]\).
\end{cor}

\begin{proof}
  Using \cref{ii:3.8.6} we can define an \(F : \R \to \R\) such that
  \[
    \forall x \in \R, F(x) = \begin{dcases}
      0    & \text{if } x \in \R \setminus [0, 1] \\
      f(x) & \text{if } x \in [0, 1]
    \end{dcases}
  \]
  and \(F\) is continuous and supported on \([0, 1]\).
  Then by \cref{ii:3.8.15} we have
  \[
    \forall \varepsilon \in \R^+, \exists P \in \R^{\R} : \begin{dcases}
      P \text{ is a polynomial on } [0, 1] \\
      \forall x \in [0, 1], \abs{P(x) - f(x)} = \abs{P(x) - F(x)} \leq \varepsilon
    \end{dcases}
  \]
\end{proof}

\begin{cor}[Weierstrass approximation theorem III]\label{ii:3.8.19}
  Let \(f : [0, 1] \to \R\) be a continuous function.
  Then for every \(\varepsilon > 0\) there exists a polynomial \(P : [0, 1] \to \R\) such that \(\abs{P(x) - f(x)} \leq \varepsilon\) for all \(x \in [0, 1]\).
\end{cor}

\begin{proof}
  Let \(F : [0, 1] \to \R\) denote the function
  \[
    F(x) \coloneqq f(x) - f(0) - x \big(f(1) - f(0)\big).
  \]
  Observe that \(F\) is also continuous, and that \(F(0) = F(1) = 0\).
  By \cref{ii:3.8.18}, we can thus find a polynomial \(Q : [0, 1] \to \R\) such that \(\abs{Q(x) - F(x)} \leq \varepsilon\) for all \(x \in [0, 1]\).
  But
  \[
    Q(x) - F(x) = Q(x) + f(0) + x \big(f(1) - f(0)\big) - f(x),
  \]
  so the claim follows by setting \(P\) to be the polynomial \(P(x) \coloneqq Q(x) + f(0) + x \big(f(1) - f(0)\big)\).
\end{proof}

\begin{rmk}\label{ii:3.8.20}
  Note that the Weierstrass approximation theorem only works on bounded intervals \([a, b]\);
  continuous functions on \(\R\) cannot be uniformly approximated by polynomials.
  For instance, the exponential function \(f : \R \to \R\) defined by \(f(x) \coloneqq e^x\) (which we shall study rigorously in Section 4.5) cannot be approximated by any polynomial, because exponential functions grow faster than any polynomial (Exercise 4.5.9) and so there is no way one can even make the sup metric between \(f\) and a polynomial finite.
\end{rmk}

\begin{rmk}\label{ii:3.8.21}
  There is a generalization of the Weierstrass approximation theorem to higher dimensions:
  if \(K\) is any compact subset of \(\R^n\) (with the Euclidean metric \(d_{l^2}\)), and \(f : K \to \R\) is a continuous function, then for every \(\varepsilon > 0\) there exists a polynomial \(P : K \to \R\) of \(n\) variables \(x_1, \dots, x_n\) such that \(d_\infty(f, P) < \varepsilon\).
  This general theorem can be proven by a more complicated variant of the arguments here, but we will not do so.
  (There is in fact an even more general version of this theorem applicable to an arbitrary metric space, known as the \emph{Stone-Weierstrass theorem}, but this is beyond the scope of this text.)
\end{rmk}

\exercisesection

\begin{ex}\label{ii:ex:3.8.1}
  Prove \cref{ii:3.8.5}.
\end{ex}

\begin{proof}
  See \cref{ii:3.8.5}.
\end{proof}

\begin{ex}\label{ii:ex:3.8.2}
  \quad
  \begin{enumerate}
    \item Prove that for any real number \(0 \leq y \leq 1\) and any natural number \(n \geq 0\), that \((1 - y)^n \geq 1 - ny\).
    \item Show that \(\int_{-1}^1 (1 - x^2)^n \; dx \geq \dfrac{1}{\sqrt{n}}\).
    \item Prove \cref{ii:3.8.8}.
  \end{enumerate}
\end{ex}

\begin{proof}{(a)}
  For each \(n \in \N\), let \(P(n)\) be the statement ``for each \(y \in \R\), if \(0 \leq y \leq 1\), then \((1 - y)^n \geq 1 - ny\).''
  We induct on \(n\) to show that \(P(n)\) is true for all \(n \in \N\).
  For \(n = 0\), we have
  \[
    \forall y \in \R, 0 \leq y \leq 1 \implies (1 - y)^0 = 1 \geq 1 - 0y = 1.
  \]
  Thus, the base case holds.
  Suppose inductively that \(P(n)\) is true for some \(n \geq 0\).
  Then we want to show that \(P(n + 1)\) is true.
  Let \(y \in \R\) such that \(0 \leq y \leq 1\).
  Then we have
  \begin{align*}
    (1 - y)^{n + 1} & = (1 - y)^n (1 - y)                                \\
                    & \geq (1 - ny) (1 - y)  &                   & \byIH \\
                    & = 1 - (n + 1)y + n y^2                             \\
                    & \geq 1 - (n + 1)y.     & (0 \leq y \leq 1)
  \end{align*}
  Since \(y\) was arbitrary, we know that \(P(n + 1)\) is true and this closes the induction.
\end{proof}

\begin{proof}{(b)}
  Let \(n \in \Z^+\).
  Since \(f(x) = 1 - x^2\) is continuous and bounded on \([-1, 1]\), by Proposition 9.4.9 and 9.6.7 in Analysis I we know that \(f^n(x) = (1 - x^2)^n\) is continuous and bounded on \([-1, 1]\).
  Thus by Corollary 11.5.2 in Analysis I we know that \(f^n\) is Riemann integrable.
  By Corollary 11.10.3 in Analysis I we have
  \[
    \int_{[-1, 1]} f^n = \int_{[-1, 1]} f^n \cdot 1 = \int_{[-1, 1]} f^n \cdot x' = \int_{[-1, 1]} f^n \; dx = \int_{-1}^1 f^n \; dx.
  \]
  Thus
  \[
    \int_{-1}^1 (1 - x^2)^n \; dx = \int_{[-1, 1]} (1 - x^2)^n = \int_{[-1, \dfrac{-1}{\sqrt{n}}]} (1 - x^2)^n + \int_{[\dfrac{-1}{\sqrt{n}}, \dfrac{1}{\sqrt{n}}]} (1 - x^2)^n + \int_{[\dfrac{1}{\sqrt{n}}, 1]} (1 - x^2)^n.
  \]
  Since
  \[
    \forall x \in [-1, 1], 1 \geq \abs{x} \geq \dfrac{1}{\sqrt{n}} \implies 1 \geq x^2 \geq \dfrac{1}{n} \implies 0 \leq 1 - x^2 \leq \dfrac{n - 1}{n},
  \]
  we know that
  \[
    \int_{-1}^1 (1 - x^2)^n \; dx \geq \int_{[\dfrac{-1}{\sqrt{n}}, \dfrac{1}{\sqrt{n}}]} (1 - x^2)^n.
  \]
  By \cref{ii:ex:3.8.2}(a) we have
  \[
    \int_{-1}^1 (1 - x^2)^n \; dx \geq \int_{[\dfrac{-1}{\sqrt{n}}, \dfrac{1}{\sqrt{n}}]} (1 - x^2)^n \geq \int_{[\dfrac{-1}{\sqrt{n}}, \dfrac{1}{\sqrt{n}}]} (1 - n x^2).
  \]
  Since
  \begin{align*}
    \int_{[\dfrac{-1}{\sqrt{n}}, \dfrac{1}{\sqrt{n}}]} (1 - n x^2) & = \int_{[\dfrac{-1}{\sqrt{n}}, \dfrac{1}{\sqrt{n}}]} 1 - n \int_{[\dfrac{-1}{\sqrt{n}}, \dfrac{1}{\sqrt{n}}]} x^2 \\
                                                                   & = \dfrac{2}{\sqrt{n}} - \dfrac{n}{3} \bigg(\dfrac{1}{n \sqrt{n}} - \dfrac{-1}{n \sqrt{n}}\bigg)                   \\
                                                                   & = \dfrac{2}{\sqrt{n}} - \dfrac{2}{3 \sqrt{n}}                                                                     \\
                                                                   & = \dfrac{4}{3 \sqrt{n}} \geq \dfrac{1}{\sqrt{n}},
  \end{align*}
  we have
  \[
    \int_{-1}^1 (1 - x^2)^n \; dx \geq \dfrac{1}{\sqrt{n}}.
  \]
\end{proof}

\begin{proof}{(c)}
  See \cref{ii:3.8.8}.
\end{proof}

\begin{ex}\label{ii:ex:3.8.3}
  Let \(f : \R \to \R\) be a compactly supported, continuous function.
  Show that \(f\) is bounded and uniformly continuous.
\end{ex}

\begin{proof}
  Since \(f\) is compactly supported, by \cref{ii:3.8.4} we know that there exists some \(a, b \in \R\) such that
  \[
    \forall x \notin [a, b], f(x) = 0.
  \]
  Since \([a, b]\) is closed and bounded in \((\R, d_{l^1}|_{\R \times \R})\), by \cref{ii:1.5.7} we know that \(\big([a, b], d_{l^1}|_{\R \times \R}\big)\) is compact.
  Since \(\big([a, b], d_{l^1}|_{\R \times \R}\big)\) is compact and \(f\) is continuous on \([a, b]\), by \cref{ii:2.3.2} we know that \(f\) is bounded.
  Since \(f\) is bounded and continuous on \([a, b]\), by \cref{ii:2.3.5} \(f\) is uniformly continuous on \([a, b]\).

  Since \(f\) is continuous at \(a\), we have
  \begin{align*}
             & \forall \varepsilon \in \R^+, \exists \delta \in \R^+ : \big(\forall x \in \R, \abs{x - a} < \delta \implies \abs{f(x) - f(a)} < \varepsilon\big)               \\
    \implies & \forall \varepsilon \in \R^+, \exists \delta \in \R^+ : \big(\forall x \in \R, x \in (a - \delta, a + \delta) \implies \abs{f(x) - f(a)} < \varepsilon\big)     \\
    \implies & \forall \varepsilon \in \R^+, \exists \delta \in \R^+ : \big(\forall x \in \R, x \in (a - \delta, a) \implies \abs{f(x) - f(a)} = \abs{f(a)} < \varepsilon\big) \\
    \implies & \forall \varepsilon \in \R^+, \abs{f(a)} < \varepsilon                                                                                                          \\
    \implies & f(a) = 0.
  \end{align*}
  Similarly, we have \(f(b) = 0\).
  If \(a = b\), then \(f\) is zero function, and we have
  \[
    \forall \varepsilon \in \R^+, \forall \delta \in \R^+, \forall x_1, x_2 \in \R, \abs{x_1 - x_2} < \delta \implies \abs{f(x_1) - f(x_2)} = 0 < \varepsilon.
  \]
  Thus \(f\) is uniformly continuous on \(\R\).
  Suppose that \(a \neq b\).
  Since \(f\) in uniformly continuous on \([a, b]\), by \cref{ii:2.3.4} we have
  \[
    \forall \varepsilon \in \R^+, \exists \delta_1 \in \R^+ : \forall x_1, x_2 \in [a, b], \abs{x_1 - x_2} < \delta_1 \implies \abs{f(x_1) - f(x_2)} < \varepsilon.
  \]
  Now fix one pair of \(\varepsilon\) and \(\delta_1\).
  Since \(\lim_{x \to a ; x \in \R} f(x) = f(a) = 0\), we have
  \[
    \exists \delta_2 \in \R^+ : \forall x \in \R, \abs{x - a} < \delta_2 < b - a \implies \abs{f(x) - f(a)} = \abs{f(x)} < \varepsilon.
  \]
  Similarly, we have
  \[
    \exists \delta_3 \in \R^+ : \forall x \in \R, \abs{x - b} < \delta_3 < b - a \implies \abs{f(x) - f(b)} = \abs{f(x)} < \varepsilon.
  \]
  Let \(\delta = \min(\delta_1, \delta_2, \delta_3)\).
  Then we have
  \begin{align*}
             & \forall x_1, x_2 \in \R, \abs{x_1 - x_2} < \delta                                                                                                          \\
    \implies & \begin{dcases}
                 \abs{x_1 - x_2} < \delta_1 \implies \abs{f(x_1) - f(x_2)} < \varepsilon & \text{if } \big(x_1, x_2 \in [a, b]\big)                                 \\
                 x_1 - x_2 < \delta_2 \implies a \leq x_1 < x_2 + \delta_2               & \text{if } \big(x_1 \in [a, b]\big) \land \big(x_2 \in (-\infty, a)\big) \\
                 x_2 - x_1 < \delta_3 \implies x_2 - \delta_3 < x_1 \leq b               & \text{if } \big(x_1 \in [a, b]\big) \land \big(x_2 \in (b, \infty)\big)  \\
                 \abs{f(x_1) - f(x_2)} = 0 < \varepsilon                                 & \text{if } \big(x_1, x_2 \notin [a, b]\big)
               \end{dcases} \\
    \implies & \begin{dcases}
                 \abs{f(x_1) - f(x_2)} < \varepsilon     & \text{if } \big(x_1, x_2 \in [a, b]\big)                                 \\
                 x_1 - a < x_2 - a + \delta_2 < \delta_2 & \text{if } \big(x_1 \in [a, b]\big) \land \big(x_2 \in (-\infty, a)\big) \\
                 \delta_3 > b - x_2 + \delta_3 > b - x_1 & \text{if } \big(x_1 \in [a, b]\big) \land \big(x_2 \in (b, \infty)\big)  \\
                 \abs{f(x_1) - f(x_2)} < \varepsilon     & \text{if } \big(x_1, x_2 \notin [a, b]\big)
               \end{dcases}                                 \\
    \implies & \begin{dcases}
                 \abs{f(x_1) - f(x_2)} < \varepsilon                          & \text{if } \big(x_1, x_2 \in [a, b]\big)                                 \\
                 \abs{x_1 - a} < \delta_2 \implies \abs{f(x_1)} < \varepsilon & \text{if } \big(x_1 \in [a, b]\big) \land \big(x_2 \in (-\infty, a)\big) \\
                 \abs{x_1 - b} < \delta_3 \implies \abs{f(x_1)} < \varepsilon & \text{if } \big(x_1 \in [a, b]\big) \land \big(x_2 \in (b, \infty)\big)  \\
                 \abs{f(x_1) - f(x_2)} < \varepsilon                          & \text{if } \big(x_1, x_2 \notin [a, b]\big)
               \end{dcases}            \\
    \implies & \abs{f(x_1) - f(x_2)} < \varepsilon.
  \end{align*}
  Since \(\varepsilon\) was arbitrary, we have
  \[
    \forall \varepsilon \in \R^+, \exists \delta \in \R^+ : \forall x_1, x_2 \in \R, \abs{x_1 - x_2} < \delta \implies \abs{f(x_1) - f(x_2)} < \varepsilon
  \]
  and \(f\) is uniformly continuous on \(\R\).
\end{proof}

\begin{ex}\label{ii:ex:3.8.4}
  Prove \cref{ii:3.8.11}.
\end{ex}

\begin{proof}
  See \cref{ii:3.8.11}.
\end{proof}

\begin{ex}\label{ii:ex:3.8.5}
  Let \(f : \R \to \R\) and \(g : \R \to \R\) be continuous, compactly supported functions.
  Suppose that \(f\) is supported on the interval \([0, 1]\), and \(g\) is constant on the interval \([0, 2]\)
  (i.e., there is a real number \(c\) such that \(g(x) = c\) for all \(x \in [0, 2]\)).
  Show that the convolution \(f * g\) is constant on the interval \([1, 2]\).
\end{ex}

\begin{proof}
  We have
  \begin{align*}
    \forall x \in [1, 2], (f * g)(x) & = \int_{-\infty}^\infty f(y) g(x - y) \; dy &                    & \by{ii:3.8.9} \\
                                     & = \int_{[0, 1]} f(y) g(x - y) \; dy         &                    & \by{ii:3.8.4} \\
                                     & = \int_{[0, 1]} c f(y) \; dy                & (x - y \in [0, 2])                 \\
                                     & = c \int_{[0, 1]} f(y) \; dy.
  \end{align*}
  Since \(\int_{[0, 1]} f(y) \; dy\) is independent of \(x\), we know that \(f * g\) is constant on \([1, 2]\).
\end{proof}

\begin{ex}\label{ii:ex:3.8.6}
  \quad
  \begin{enumerate}
    \item Let \(g\) be an \((\varepsilon, \delta)\) approximation to the identity.
          Show that \(1 - 2 \varepsilon \leq \int_{[-\delta, \delta]} g \leq 1\).
    \item Prove \cref{ii:3.8.14}.
  \end{enumerate}
\end{ex}

\begin{proof}{(a)}
  By \cref{ii:3.8.6} we know that
  \begin{itemize}
    \item \(\varepsilon \in \R^+\).
    \item \(\delta \in \R^+\) such that \(0 < \delta < 1\).
    \item \(g\) is supported on \([-1, 1]\) and \(g(x) \geq 0\) for all \(x \in [-1, 1]\).
    \item \(g\) is continuous on \(\R\) and \(\int_{-\infty}^\infty g = 1\).
    \item \(\abs{g(x)} \leq \varepsilon\) for each \(\delta \leq \abs{x} \leq 1\).
  \end{itemize}
  By \cref{ii:3.8.4} we have
  \[
    \int_{-\infty}^\infty g = \int_{[-1, 1]} g = 1.
  \]
  Thus
  \begin{align*}
             & \forall \delta \leq \abs{x} \leq 1, \abs{g(x)} \leq \varepsilon                                                                                \\
    \implies & 1 = \int_{[-1, 1]} g                                                                                                                           \\
             & = \int_{[-1, -\delta]} g + \int_{[-\delta, \delta]} g + \int_{[\delta, 1]} g                                                                   \\
             & \leq (-\delta + 1) \varepsilon + \int_{[-\delta, \delta]} g + (1 - \delta) \varepsilon                                                         \\
             & = 2 \varepsilon (1 - \delta) + \int_{[-\delta, \delta]} g                                                                                      \\
             & \leq 2 \varepsilon + \int_{[-\delta, \delta]} g                                                 & (1 - \delta < 1)                             \\
             & \leq 2 \varepsilon + \int_{[-1, -\delta]} g + \int_{[-\delta, \delta]} g + \int_{[\delta, 1]} g & (g(x) \geq 0 \text{ for all } x \in [-1, 1]) \\
             & = 2 \varepsilon + \int_{[-1, 1]} g                                                                                                             \\
             & = 2 \varepsilon + 1                                                                                                                            \\
    \implies & 1 - 2 \varepsilon \leq \int_{[-\delta, \delta]} g \leq 1.
  \end{align*}
\end{proof}

\begin{proof}{(b)}
  See \cref{ii:3.8.14}.
\end{proof}

\begin{ex}\label{ii:ex:3.8.7}
  Prove \cref{ii:3.8.15}.
\end{ex}

\begin{proof}
  See \cref{ii:3.8.15}.
\end{proof}

\begin{ex}\label{ii:ex:3.8.8}
  Let \(f : [0, 1] \to \R\) be a continuous function, and suppose that \(\int_{[0, 1]} f(x) x^n \; dx = 0\) for all non-negative integers \(n = 0, 1, 2, \dots\).
  Show that \(f\) must be the zero function \(f \equiv 0\).
\end{ex}

\begin{proof}
  Let \(P : \R \to \R\) be a polynomial with degree \(n\).
  Then by \cref{ii:3.8.1} we have
  \[
    \forall x \in \R, P(x) = \sum_{j = 0}^n c_j x^j.
  \]
  By hypothesis we have
  \begin{align*}
    \int_{[0, 1]} f(x) P(x) \; dx & = \int_{[0, 1]} f(x) \sum_{j = 0}^n c_j x^j \; dx \\
                                  & = \sum_{j = 0}^n c_j \int_{[0, 1]} f(x) x^j \; dx \\
                                  & = 0.
  \end{align*}
  Since \([0, 1]\) is closed and bounded in \((\R, d_{l^1}|_{\R \times \R})\), by \cref{ii:1.5.7} we know that \(\big([0, 1], d_{l^1}|_{\R \times \R}\big)\) is compact.
  By \cref{ii:2.3.2} we know that \(f\) is bounded, i.e., there exists a \(M \in \R^+\) such that \(\abs{f(x)} \leq M\) for all \(x \in [0, 1]\).

  Since \(f\) is continuous on \([0, 1]\), by \cref{ii:3.8.3} we know that
  \[
    \forall \varepsilon \in \R^+, \exists P \in \R^{\R} : \begin{dcases}
      P \text{ is a polynomial on } [0, 1] \\
      d_{\infty}(P, f) \leq \dfrac{\varepsilon}{M}
    \end{dcases}
  \]
  Fix one pair of \(\varepsilon\) and \(P\).
  Then we have
  \begin{align*}
             & d_\infty(P, f) \leq \dfrac{\varepsilon}{M}                                                                                                                          \\
    \implies & \sup_{x \in [0, 1]} \abs{P(x) - f(x)} \leq \dfrac{\varepsilon}{M}                                                                       &  & \by{ii:3.4.2}          \\
    \implies & \forall x \in [0, 1], \abs{P(x) - f(x)} \leq \dfrac{\varepsilon}{M}                                                                                                 \\
    \implies & \forall x \in [0, 1], \abs{f(x) P(x) - f(x) f(x)} \leq \dfrac{\varepsilon \abs{f(x)}}{M} \leq \dfrac{\varepsilon M}{M} \leq \varepsilon                             \\
    \implies & \forall x \in [0, 1], f(x) P(x) - \varepsilon \leq \big(f(x)\big)^2 \leq f(x) P(x) + \varepsilon                                                                    \\
    \implies & \forall x \in [0, 1], \int_{[0, 1]} f(x) P(x) - \varepsilon \; dx = -\varepsilon                                                                                    \\
             & \leq \int_{[0, 1]} \big(f(x)\big)^2 \; dx \leq \int_{[0, 1]} f(x) P(x) + \varepsilon \; dx = \varepsilon                                &  & \text{(by hypothesis)} \\
    \implies & \forall x \in [0, 1], -\varepsilon \leq \int_{[0, 1]} \big(f(x)\big)^2 \; dx \leq \varepsilon.
  \end{align*}
  Since \(\varepsilon\) was arbitrary, we know that
  \[
    \forall \varepsilon \in \R^+, \abs{\int_{[0, 1]} \big(f(x)\big)^2 \; dx} \leq \varepsilon \implies \abs{\int_{[0, 1]} \big(f(x)\big)^2 \; dx} = \int_{[0, 1]} \big(f(x)\big)^2 \; dx = 0.
  \]
  Since \(f\) is continuous on \([0, 1]\) and \(\big(f(x)\big)^2 \geq 0\) for all \(x \in [0, 1]\), by Exercise 11.4.2 in Analysis I we know that
  \[
    \forall x \in [0, 1], \big(f(x)\big)^2 = 0.
  \]
  Thus we have \(f(x) = 0\) for all \(x \in [0, 1]\).
\end{proof}

\begin{ex}\label{ii:ex:3.8.9}
  Prove \cref{ii:3.8.16}.
\end{ex}

\begin{proof}
  See \cref{ii:3.8.16}.
\end{proof}

