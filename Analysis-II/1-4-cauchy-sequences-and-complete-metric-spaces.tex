\section{Cauchy sequences and complete metric spaces}\label{sec:1.4}

\begin{defn}[Subsequences]\label{1.4.1}
  Suppose that \((x^{(n)})_{n = m}^\infty\) is a sequence of points in a metric space \((X, d)\).
  Suppose that \(n_1, n_2, n_3, \dots\) is an increasing sequence of integers which are at least as large as \(m\), thus
  \[
    m \leq n_1 < n_2 < n_3 < \dots.
  \]
  Then we call the sequence \((x^{(n_j)})_{j = 1}^\infty\) a \emph{subsequence} of the original sequence \((x^{(n)})_{n = m}^\infty\).
\end{defn}

\setcounter{thm}{2}
\begin{lem}\label{1.4.3}
  Let \((x^{(n)})_{n = m}^\infty\) be a sequence in \((X, d)\) which converges to some limit \(x_0\).
  Then every subsequence \((x^{(n_j)})_{j = 1}^\infty\) of that sequence also converges to \(x_0\).
\end{lem}

\begin{proof}
  \begin{align*}
             & \lim_{n \to \infty} d(x^{(n)}, x_0) = 0                                                                                                                         \\
    \implies & \forall \varepsilon \in \R^+, \exists\ N \in \N : \forall n \geq N, d(x^{(n)}, x_0) \leq \varepsilon                                & \text{(by \cref{1.1.14})} \\
    \implies & \forall \varepsilon \in \R^+, \exists\ j \in \N : \forall j \geq N, (n_j \geq j) \land \big(d(x^{(n_j)}, x_0) \leq \varepsilon\big)                             \\
    \implies & \lim_{j \to \infty} d(x^{(n_j)}, x_0) = 0.                                                                                          & \text{(by \cref{1.1.14})}
  \end{align*}
\end{proof}

\begin{defn}[Limit points]\label{1.4.4}
  Suppose that \((x^{(n)})_{n = m}^\infty\) is a sequence of points in a metric space \((X, d)\), and let \(L \in X\).
  We say that \(L\) is a \emph{limit point} of \((x^{(n)})_{n = m}^\infty\) iff for every \(N \geq m\) and \(\varepsilon > 0\) there exists an \(n \geq N\) such that \(d(x^{(n)}, L) \leq \varepsilon\).
\end{defn}

\begin{prop}\label{1.4.5}
  Let \((x^{(n)})_{n = m}^\infty\) be a sequence of points in a metric space \((X, d)\), and let \(L \in X\).
  Then the following are equivalent:
  \begin{itemize}
    \item \(L\) is a limit point of \((x^{(n)})_{n = m}^\infty\).
    \item There exists a subsequence \((x^{(n_j)})_{j = 1}^\infty\) of the original sequence \((x^{(n)})_{n = m}^\infty\) which converges to \(L\).
  \end{itemize}
\end{prop}

\begin{proof}
  We first show that if \(L\) is a limit point of \((x^{(n)})_{n = m}^\infty\) in \((X, d)\), then there exists a subsequence of \((x^{(n)})_{n = m}^\infty\) which converges to \(L\) with respect to \(d\).
  Let \(N \in \N\).
  Since \(L\) is a limit point of \((x^{(n)})_{n = m}^\infty\) in \((X, d)\), by \cref{1.4.4} we know that
  \[
    \forall \varepsilon \in \R^+, \forall N \geq m, \exists\ n \geq N : d(x^{(n)}, L) \leq \varepsilon.
  \]
  In particular, for every \(j \in \Z^+\), we have
  \[
    \forall N \geq m, \exists\ n \geq N : d(x^{(n)}, L) \leq \frac{1}{j}.
  \]
  Now we define \(n_j\) as follow:
  \[
    n_j = \begin{cases}
      \min\big\{n \in \Z^+ : d(x^{(n)}, L) \leq 1\big\}                                           & \text{if } j = 1    \\
      \min\Big\{n \in \Z^+ : (n > n_{j - 1}) \land \big(d(x^{(n)}, L) \leq \frac{1}{j}\big)\Big\} & \text{if } j \neq 1
    \end{cases}
  \]
  By well-ordering theorem \(n_j\) is well-defined for every \(j \in \Z^+\).
  By applying squeeze test to the subsequence \((x^{(n_j)})_{j = 1}^\infty\) we have
  \begin{align*}
             & 0 \leq d(x^{(n_j)}, L) \leq \frac{1}{j}                                                                     \\
    \implies & 0 = \lim_{j \to \infty} 0 \leq \lim_{j \to \infty} d(x^{(n_j)}, L) \leq \lim_{j \to \infty} \frac{1}{j} = 0 \\
    \implies & \lim_{j \to \infty} d(x^{(n_j)}, L) = 0
  \end{align*}
  and thus by \cref{1.1.14} the sequence \((x^{(n_j)})_{j = 1}^\infty\) converges to \(L\) with respect to \(d\).

  Now we show that if a subsequence of \((x^{(n)})_{n = m}^\infty\) converges to \(L\) with respect to \(d\), then \(L\) is a limit point of \((x^{(n)})_{n = m}^\infty\) in \((X, d)\).
  Let \(N \in \N\) and let \((x^{(n_j)})_{j = 1}^\infty\) be a subsequence of \((x^{(n)})_{n = 0}^\infty\) where \(\lim_{j \to \infty} d(x^{(n_j)}, L) = 0\).
  Then we have
  \begin{align*}
             & \lim_{n \to \infty} d(x^{(n_j)}, L) = 0                                                                                                  \\
    \implies & \forall \varepsilon \in \R^+, \exists\ j \geq 1 : d(x^{(n_j)}, L) \leq \varepsilon                           & \text{(by \cref{1.1.14})} \\
    \implies & \forall \varepsilon \in \R^+,  \forall N \geq m, \exists\ n \geq n_j \geq N : d(x^{(n)}, L) \leq \varepsilon                             \\
    \implies & L \text{ is a limit point of } (x^{(n)})_{n = m}^\infty \text{ in } (X, d).                                  & \text{(by \cref{1.4.4})}
  \end{align*}
\end{proof}

\begin{defn}[Cauchy sequences]\label{1.4.6}
  Let \((x^{(n)})_{n = m}^\infty\) be a sequence of points in a metric space \((X, d)\).
  We say that this sequence is a \emph{Cauchy sequence} iff for every \(\varepsilon > 0\), there exists an \(N \geq m\) such that \(d(x^{(j)}, x^{(k)}) \leq \varepsilon\) for all \(j, k \geq N\).
\end{defn}

\begin{note}
  Here the book use \(<\) instead of \(\leq\), but the two inequalities are more or less the same.
  I use \(\leq\) to ensure consistency with Definition 5.1.8 in Analysis I.
\end{note}

\begin{lem}[Convergent sequences are Cauchy sequences]\label{1.4.7}
  Let \((x^{(n)})_{n = m}^\infty\) be a sequence in \((X, d)\) which converges to some limit \(x_0\).
  Then \((x^{(n)})_{n = m}^\infty\) is also a Cauchy sequence.
\end{lem}

\begin{proof}
  Let \(N \in \N\).
  Since \(\lim_{n \to \infty} d(x^{(n)}, x_0) = 0\), by \cref{1.1.14} we know that
  \[
    \forall \varepsilon \in \R^+, \exists\ N \geq m : \forall n \geq N, d(x^{(n)}, x_0) \leq \frac{\varepsilon}{2}.
  \]
  Let \(k \in \N\) and \(k \geq N\).
  Then we have
  \begin{align*}
    d(x^{(n)}, x^{(k)}) & \leq d(x^{(n)}, x_0) + d(x_0, x^{(k)})             & \text{(by \cref{1.1.2}(d))} \\
                        & = d(x^{(n)}, x_0) + d(x^{(k)}, x_0)                & \text{(by \cref{1.1.2}(c))} \\
                        & \leq \frac{\varepsilon}{2} + \frac{\varepsilon}{2}                               \\
                        & = \varepsilon
  \end{align*}
  and by \cref{1.4.6} \((x^{(n)})_{n = m}^\infty\) is a Cauchy sequence in \((X, d)\).
\end{proof}

\setcounter{thm}{8}
\begin{lem}\label{1.4.9}
  Let \((x^{(n)})_{n = m}^\infty\) be a Cauchy sequence in \((X, d)\).
  Suppose that there is some subsequence \((x^{(n_j)})_{j = 1}^\infty\) of this sequence which converges to a limit \(x_0\) in \(X\).
  Then the original sequence \((x^{(n)})_{n = m}^\infty\) also converges to \(x_0\).
\end{lem}

\begin{proof}
  Let \(N_1, N_2, i, k \in \N\).
  Since \(\lim_{j \to \infty} d(x^{(n_j)}, x_0) = 0\), by \cref{1.1.14} we know that
  \[
    \forall \varepsilon \in \R^+, \exists\ N_1 \geq 1 : \forall j \geq N_1, d(x^{(n_j)}, x_0) \leq \frac{\varepsilon}{2}.
  \]
  Now fix such \(\varepsilon\).
  Since \((x^{(n)})_{n = m}^\infty\) is a Cauchy sequence in \((X, d)\), by \cref{1.4.6} we know that
  \[
    \exists\ N_2 \geq m : \forall i, k \geq N_2, d(x^{(i)}, x^{(k)}) \leq \frac{\varepsilon}{2}.
  \]
  Let \(N = \max(N_1, N_2)\).
  Then \(\forall i, j \geq N\), we have
  \begin{align*}
    d(x^{(i)}, x_0) & \leq d(x^{(i)}, x^{(n_j)}) + d(x^{(n_j)}, x_0)     & \text{(by \cref{1.1.2}(d))} \\
                    & \leq \frac{\varepsilon}{2} + \frac{\varepsilon}{2} & (n_j \geq j \geq N)         \\
                    & = \varepsilon.
  \end{align*}
  Since \(\varepsilon\) is arbitrary, by \cref{1.1.14} \(\lim_{n \to \infty} d(x^{(n)}, x_0) = 0\).
\end{proof}

\begin{defn}[Complete metric spaces]\label{1.4.10}
  A metric space \((X, d)\) is said to be \emph{complete} iff every Cauchy sequence in \((X, d)\) is in fact convergent in \((X, d)\).
\end{defn}

\setcounter{thm}{11}
\begin{prop}\label{1.4.12}
  \quad
  \begin{enumerate}
    \item Let \((X, d)\) be a metric space, and let \((Y, d|_{Y \times Y})\) be a subspace of \((X, d)\).
          If \((Y, d|_{Y \times Y})\) is complete, then \(Y\) must be closed in \(X\).
    \item Conversely, suppose that \((X, d)\) is a complete metric space, and \(Y\) is a closed subset of \(X\).
          Then the subspace \((Y, d|_{Y \times Y})\) is also complete.
  \end{enumerate}
\end{prop}

\begin{proof}
  We first show that the statement (a) is true.
  Suppose that \((X, d)\) is a metric space and \((Y, d|_{Y \times Y})\) is a complete subspace of \((X, d)\).
  Let \(x_0 \in \partial_{(X, d)}(Y)\).
  Then we have
  \begin{align*}
             & \begin{cases}
                 (Y, d|_{Y \times Y}) \text{ is complete} \\
                 x_0 \in \partial_{(X, d)}(Y)
               \end{cases}                                                                              \\
    \implies & \begin{cases}
                 (Y, d|_{Y \times Y}) \text{ is complete} \\
                 \exists\ (x^{(n)})_{n = m}^\infty \text{ in } Y : \lim_{n \to \infty} d(x^{(n)}, x_0) = 0
               \end{cases} & \text{(by \cref{1.2.10})}                              \\
    \implies & x_0 \in Y.                                                                                & \text{(by \cref{1.4.10})}
  \end{align*}
  Since \(x_0\) is arbitrary, we have \(\partial_{(X, d)}(Y) \subseteq Y\) and by \cref{1.2.12} \(Y\) is closed in \((X, d)\).

  Now we show that the statement (b) is true.
  Suppose that \((X, d)\) is a complete metric space, \(Y \subseteq X\) and \(Y\) is closed in \((X, d)\).
  Let \((x^{(n)})_{n = m}^\infty\) be a Cauchy sequence in \((Y, d|_{Y \times Y})\).
  Since \(Y \subseteq X\), we know that \((x^{(n)})_{n = m}^\infty\) is also a Cauchy sequence in \((X, d)\).
  Since \((X, d)\) is complete, by \cref{1.4.10} we know that \(\lim_{n \to \infty} d(x^{(n)}, x_0) = 0\) for some \(x_0 \in X\).
  Since \((x^{(n)})_{n = m}^\infty\) is in \(Y\) and \(\lim_{n \to \infty} d(x^{(n)}, x_0) = 0\), by \cref{1.2.10}(c) we know that \(x_0 \in \overline{Y}_{(X, d)}\).
  But \(Y\) is closed in \((X, d)\), thus by \cref{1.2.15}(b) we know that \(x_0 \in Y\).
  Since \((x^{(n)})_{n = m}^\infty\) is arbitrary Cauchy sequence in \((Y, d|_{Y \times Y})\), by \cref{1.4.10} \((Y, d|_{Y \times Y})\) is complete.
\end{proof}

\exercisesection

\begin{ex}\label{ex:1.4.1}
  Prove \cref{1.4.3}.
\end{ex}

\begin{proof}
  See \cref{1.4.3}.
\end{proof}

\begin{ex}\label{ex:1.4.2}
  Prove Proposition 1.4.5.
\end{ex}

\begin{proof}
  See \cref{1.4.5}.
\end{proof}

\begin{ex}\label{ex:1.4.3}
  Prove \cref{1.4.7}.
\end{ex}

\begin{proof}
  See \cref{1.4.7}.
\end{proof}

\begin{ex}\label{ex:1.4.4}
  Prove \cref{1.4.9}.
\end{ex}

\begin{proof}
  See \cref{1.4.9}.
\end{proof}

\begin{ex}\label{ex:1.4.5}
  Let \((x^{(n)})_{n = m}^\infty\) be a sequence of points in a metric space \((X, d)\), and let \(L \in X\).
  Show that if \(L\) is a limit point of the sequence \((x^{(n)})_{n = m}^\infty\), then \(L\) is an adherent point of the set \(\{x^{(n)} : n \geq m\}\).
  Is the converse true?
\end{ex}

\begin{proof}
  Let \(E = \{x^{(n)} : n \geq m\}\).
  We first show that if \(L\) is a limit point of \((x^{(n)})_{n = m}^\infty\) in \((X, d)\), then \(L \in \overline{E}_{(X, d)}\).
  Suppose that \(L\) is a limit point of \((x^{(n)})_{n = m}^\infty\) in \((X, d)\).
  By \cref{1.4.5} we know that \(\exists\ (x^{(n_j)})_{j = 1}^\infty\) in \(E\) such that \(\lim_{j \to \infty} d(x^{(n_j)}, L) = 0\).
  Thus by \cref{1.2.10}(c) \(L \in \overline{E}_{(X, d)}\).

  Now we show that if \(L \in \overline{E}_{(X, d)}\), then \(L\) may not be a limit point of \((x^{(n)})_{n = m}^\infty\) in \((X, d)\).
  Let \((X, d) = (\R, d_{l^1}|_{\R \times \R})\) and let \(x^{(n)} = 1 / n\).
  Then by \cref{1.2.9} we know that \(1 \in \overline{E}_{(X, d)}\).
  But by \cref{1.4.4} we know that \(1\) is not an limit point of \((x^{(n)})_{n = m}^\infty\) in \((\R, d_{l^1})\) since every subsequence of \((x^{(n)})_{n = 1}^\infty\) converges to \(0\) with respect to \(d_{l^1}|_{\R \times \R}\).
  Thus if \(L\) is an adherent point of \(E\) in \((X, d)\), then \(L\) may not be a limit point of \((x^{(n)})_{n = m}^\infty\) in \((X, d)\).
\end{proof}

\begin{ex}\label{ex:1.4.6}
  Show that every Cauchy sequence can have at most one limit point.
\end{ex}

\begin{proof}
  Suppose for sake of contradiction that there exists a Cauchy sequence \((x^{(n)})_{n = m}^\infty\) in some metric space \((X, d)\) which has two limit points \(L\) and \(L'\).
  Then by \cref{1.4.5} \(\exists\ (x^{(n_i)})_{i = 1}^\infty, (x^{(n_j)})_{j = 1}^\infty,\) which converges to \(L\) and \(L'\) respectively.
  Since \((x^{(n)})_{n = m}^\infty\) is a Cauchy sequence in \((X, d)\), by \cref{1.4.9} we know that \((x^{(n)})_{n = m}^\infty\) converges to \(L\) and \(L'\) with respect to \(d\), which contradict to \cref{1.1.20}.
  Thus every Cauchy sequence can have at most one limit point.
\end{proof}

\begin{ex}\label{ex:1.4.7}
  Prove \cref{1.4.12}.
\end{ex}

\begin{proof}
  See \cref{1.4.12}.
\end{proof}

\begin{ex}\label{ex:1.4.8}
  The following construction generalizes the construction of the reals from the rationals in Chapter 5, allowing one to view any metric space as a subspace of a complete metric space.
  In what follows we let \((X, d)\) be a metric space.
  \begin{enumerate}
    \item Given any Cauchy sequence \((x^{(n)})_{n = m}^\infty\) in \(X\), we introduce the \emph{formal limit} \\
          \(\text{LIM}_{n \to \infty} x_n\).
          We say that two formal limits \(\text{LIM}_{n \to \infty} x_n\) and \(\text{LIM}_{n \to \infty} y_n\) are equal if \(\text{lim}_{n \to \infty} d(x_n, y_n)\) is equal to zero.
          Show that this equality relation obeys the reflexive, symmetry, and transitive axioms.
    \item Let \(\overline{X}\) be the space of all formal limits of Cauchy sequences in \(X\), with the above equality relation.
          Define a metric \(d_{\overline{X}} : \overline{X} \times \overline{X} \to [0, \infty)\) by setting
          \[
            d_{\overline{X}}(\text{LIM}_{n \to \infty} x_n, \text{LIM}_{n \to \infty} y_n) \coloneqq \lim_{n \to \infty} d(x_n, y_n).
          \]
          Show that this function is well-defined (this means not only that the limit \\
          \(\lim_{n \to \infty} d(x_n, y_n)\) exists, but also that the axiom of substitution is obeyed;
          cf. Lemma 5.3.7), and gives \(\overline{X}\) the structure of a metric space.
    \item Show that the metric space \((\overline{X}, d_{\overline{X}})\) is complete.
    \item We identify an element \(x \in X\) with the corresponding formal limit \(\text{LIM}_{n \to \infty} x\) in \(X\);
          show that this is legitimate by verifying that \(x = y \iff \text{LIM}_{n \to \infty} x = \text{LIM}_{n \to \infty} y\).
          With this identification, show that \(d(x, y) = d_{\overline{X}}(x, y)\), and thus \((X, d)\) can now be thought of as a subspace of \((\overline{X}, d_{\overline{X}})\).
    \item Show that the closure of \(X\) in \(\overline{X}\) is \(\overline{X}\) (which explains the choice of notation \(\overline{X}\)).
    \item Show that the formal limit agrees with the actual limit, thus if \((x_n)_{n = 1}^\infty\) is any Cauchy sequence in \(X\), then we have \(\lim_{n \to \infty} x_n = \text{LIM}_{n \to \infty} x_n\) in \(\overline{X}\).
  \end{enumerate}
\end{ex}

\begin{proof}{(a)}
  Let \((x^{(n)})_{n = m}^\infty\), \((y^{(n)})_{n = m}^\infty\), \((z^{(n)})_{n = m}^\infty\) be Cauchy sequences in \((X, d)\).

  First suppose that \(\text{LIM}_{n \to \infty} x^{(n)}\) is well-defined.
  Then we have
  \begin{align*}
             & \lim_{n \to \infty} d(x^{(n)}, x^{(n)}) = \lim_{n \to \infty} 0 = 0   & \text{(by \cref{1.1.2}(a))} \\
    \implies & \text{LIM}_{n \to \infty} x^{(n)} = \text{LIM}_{n \to \infty} x^{(n)} & \text{(by definition)}
  \end{align*}
  and thus the equality relation of \cref{ex:1.4.8}(a) is reflexive.

  Next suppose that \(\text{LIM}_{n \to \infty} x^{(n)}, \text{LIM}_{n \to \infty} y^{(n)}\) are well-defined and \(\text{LIM}_{n \to \infty} x^{(n)} = \text{LIM}_{n \to \infty} y^{(n)}\).
  Then we have
  \begin{align*}
         & \text{LIM}_{n \to \infty} x^{(n)} = \text{LIM}_{n \to \infty} y^{(n)}                               \\
    \iff & \lim_{n \to \infty} d(x^{(n)}, y^{(n)}) = 0                           & \text{(by definition)}      \\
    \iff & \lim_{n \to \infty} d(y^{(n)}, x^{(n)}) = 0                           & \text{(by \cref{1.1.2}(c))} \\
    \iff & \text{LIM}_{n \to \infty} y^{(n)} = \text{LIM}_{n \to \infty} x^{(n)} & \text{(by definition)}
  \end{align*}
  and thus the equality relation of \cref{ex:1.4.8}(a) is symmetry.

  Finally suppose that \(\text{LIM}_{n \to \infty} x^{(n)}, \text{LIM}_{n \to \infty} y^{(n)}, \text{LIM}_{n \to \infty} z^{(n)}\) are well-defined.
  Suppose also that \(\text{LIM}_{n \to \infty} x^{(n)} = \text{LIM}_{n \to \infty} y^{(n)}\) and \(\text{LIM}_{n \to \infty} y^{(n)} = \text{LIM}_{n \to \infty} z^{(n)}\).
  Then we have
  \begin{align*}
             & \begin{cases}
                 \text{LIM}_{n \to \infty} x^{(n)} = \text{LIM}_{n \to \infty} y^{(n)} \\
                 \text{LIM}_{n \to \infty} y^{(n)} = \text{LIM}_{n \to \infty} z^{(n)}
               \end{cases}                                             \\
    \implies & \begin{cases}
                 \lim_{n \to \infty} d(x^{(n)}, y^{(n)}) = 0 \\
                 \lim_{n \to \infty} d(y^{(n)}, z^{(n)}) = 0
               \end{cases}                                         & \text{(by definition)}                                      \\
    \implies & \lim_{n \to \infty} \big(d(x^{(n)}, y^{(n)}) + d(y^{(n)}, z^{(n)})\big) = 0                                       \\
    \implies & 0 \leq \lim_{n \to \infty} d(x^{(n)}, z^{(n)})                                                                    \\
             & \quad \leq \lim_{n \to \infty} \big(d(x^{(n)}, y^{(n)}) + d(y^{(n)}, z^{(n)})\big) = 0 & \text{(by \cref{1.1.2})} \\
    \implies & \lim_{n \to \infty} d(x^{(n)}, z^{(n)}) = 0                                            & \text{(by squeeze test)} \\
    \implies & \text{LIM}_{n \to \infty} x^{(n)} = \text{LIM}_{n \to \infty} z^{(n)}                  & \text{(by definition)}
  \end{align*}
  and thus the equality relation of \cref{ex:1.4.8}(a) is transitive.
\end{proof}

\begin{proof}{(b)}
  Let \((x^{(n)})_{n = m}^\infty\), \((y^{(n)})_{n = m}^\infty\), \((z^{(n)})_{n = m}^\infty\) be Cauchy sequences in \((X, d)\) with formal limits in \(\overline{X}\).
  We first show that the limit
  \[
    d_{\overline{X}}(\text{LIM}_{n \to \infty} x^{(n)}, \text{LIM}_{n \to \infty} y^{(n)}) \coloneqq \lim_{n \to \infty} d(x^{(n)}, y^{(n)})
  \]
  exists.
  Let \((a^{(n)})_{n = m}^\infty\) be the sequence \(a^{(n)} = d(x^{(n)}, y^{(n)})\).
  To show that the above limit exists, it will suffice to show that \((a^{(n)})_{n = m}^\infty\) converges in \(\R\) with respect to \(d_{l^1}|_{\R \times \R}\).
  Let \(N_1, N_2, j, k \in \N\).
  Since \((x^{(n)})_{n = m}^\infty\) is a Cauchy sequence in \((X, d)\), by \cref{1.4.6} we know that
  \[
    \forall \varepsilon \in \R^+, \exists\ N_1 \geq m : \forall j, k \geq N_1, d(x^{(j)}, x^{(k)}) \leq \frac{\varepsilon}{2}.
  \]
  Similarly,
  \[
    \forall \varepsilon \in \R^+, \exists\ N_2 \geq m : \forall j, k \geq N_2, d(y^{(j)}, y^{(k)}) \leq \frac{\varepsilon}{2}.
  \]
  Let \(N = \max(N_1, N_2)\).
  Then by \cref{1.1.2} we have
  \begin{align*}
     & \forall j, k \geq N, \abs{a_j - a_k}                                                             \\
     & = \abs{d(x^{(j)}, y^{(j)}) - d(x^{(k)}, y^{(k)})}                                                \\
     & = \abs{d(x^{(j)}, y^{(j)}) + d(y^{(j)}, x^{(k)}) - d(y^{(j)}, x^{(k)}) - d(x^{(k)}, y^{(k)})}    \\
     & \leq \abs{d(x^{(j)}, y^{(j)}) + d(y^{(j)}, x^{(k)}) + d(y^{(j)}, x^{(k)}) + d(x^{(k)}, y^{(k)})} \\
     & = d(x^{(j)}, y^{(j)}) + d(y^{(j)}, x^{(k)}) + d(y^{(j)}, x^{(k)}) + d(x^{(k)}, y^{(k)})          \\
     & \leq d(x^{(j)}, x^{(k)}) + d(y^{(j)}, y^{(k)})                                                   \\
     & \leq \varepsilon / 2 + \varepsilon / 2                                                           \\
     & = \varepsilon
  \end{align*}
  and thus \((a^{(n)})_{n = m}^\infty\) is a Cauchy sequence in \((\R, d_{l^1}|_{\R \times \R})\).
  Since \((\R, d_{l^1}|_{\R \times \R})\) is complete (see Theorem 6.4.18, Analysis I), we know that \((a^{(n)})_{n = m}^\infty\) converges in \(\R\) with respect to \(d_{l^1}|_{\R \times \R}\).

  Next we show that \(d_{\overline{X}}\) obeys the axiom of substitution.
  Suppose that \(\text{LIM}_{n \to \infty} x^{(n)} = \text{LIM}_{n \to \infty} z^{(n)}\) and \(\lim_{n \to \infty} d(x^{(n)}, y^{(n)})\) exists.
  Then we have
  \begin{align*}
             & \begin{cases}
                 d(x^{(n)}, y^{(n)}) \leq d(x^{(n)}, z^{(n)}) + d(z^{(n)}, y^{(n)}) \\
                 d(z^{(n)}, y^{(n)}) \leq d(x^{(n)}, y^{(n)}) + d(x^{(n)}, z^{(n)})
               \end{cases}                              & \text{(by \cref{1.1.2}(c)(d))}                                                                                        \\
    \implies & \begin{cases}
                 d(x^{(n)}, y^{(n)}) - d(z^{(n)}, y^{(n)}) \leq d(x^{(n)}, z^{(n)}) \\
                 d(z^{(n)}, y^{(n)}) - d(x^{(n)}, y^{(n)}) \leq d(x^{(n)}, z^{(n)})
               \end{cases}                                                                                               \\
    \implies & 0 \leq \abs{d(x^{(n)}, y^{(n)}) - d(z^{(n)}, y^{(n)})} \leq d(x^{(n)}, z^{(n)})                                                                                  \\
    \implies & 0 = \lim_{n \to \infty} 0 \leq \lim_{n \to \infty} \abs{d(x^{(n)}, y^{(n)}) - d(z^{(n)}, y^{(n)})}                                                               \\
             & \quad \leq \lim_{n \to \infty} d(x^{(n)}, z^{(n)}) = 0                                             & \text{(by \cref{ex:1.4.8}(a))}                              \\
    \implies & \lim_{n \to \infty} \abs{d(x^{(n)}, y^{(n)}) - d(z^{(n)}, y^{(n)})} = 0                            & \text{(by squeeze test)}                                    \\
    \implies & \lim_{n \to \infty} \big(d(x^{(n)}, y^{(n)}) - d(z^{(n)}, y^{(n)})\big) = 0                                                                                      \\
    \implies & \lim_{n \to \infty} d(x^{(n)}, y^{(n)}) = \lim_{n \to \infty} d(z^{(n)}, y^{(n)})                  & \text{(\(\lim_{n \to \infty} d(x^{(n)}, y^{(n)})\) exists)}
  \end{align*}
  and thus \(d_{\overline{X}}(\text{LIM}_{n \to \infty} x^{(n)}, \text{LIM}_{n \to \infty} y^{(n)}) = d_{\overline{X}}(\text{LIM}_{n \to \infty} z^{(n)}, \text{LIM}_{n \to \infty} y^{(n)})\).

  Now we show that \((\overline{X}, d_{\overline{X}})\) is a metric space.
  For identify:
  By \cref{1.1.2}(a) we have
  \[
    d_{\overline{X}}(\text{LIM}_{n \to \infty} x^{(n)}, \text{LIM}_{n \to \infty} x^{(n)}) = \lim_{n \to \infty} d(x^{(n)}, x^{(n)}) = 0.
  \]
  For positivity:
  If \(\text{LIM}_{n \to \infty} x^{(n)} \neq \text{LIM}_{n \to \infty} y^{(n)}\), then by \cref{ex:1.4.8}(a) we have
  \[
    d_{\overline{X}}(\text{LIM}_{n \to \infty} x^{(n)}, \text{LIM}_{n \to \infty} y^{(n)}) = \lim_{n \to \infty} d(x^{(n)}, y^{(n)}) \neq 0.
  \]
  For symmetry:
  We have
  \begin{align*}
     & d_{\overline{X}}(\text{LIM}_{n \to \infty} x^{(n)}, \text{LIM}_{n \to \infty} y^{(n)})                                  \\
     & = \lim_{n \to \infty} d(x^{(n)}, y^{(n)})                                                 & \text{(by definition)}      \\
     & = \lim_{n \to \infty} d(y^{(n)}, x^{(n)})                                                 & \text{(by \cref{1.1.2}(c))} \\
     & = d_{\overline{X}}(\text{LIM}_{n \to \infty} y^{(n)}, \text{LIM}_{n \to \infty} x^{(n)}). & \text{(by definition)}
  \end{align*}
  For transitive:
  We have
  \begin{align*}
     & d_{\overline{X}}(\text{LIM}_{n \to \infty} x^{(n)}, \text{LIM}_{n \to \infty} z^{(n)})                                                                                             \\
     & = \lim_{n \to \infty} d(x^{(n)}, z^{(n)})                                                                                                                                          \\
     & \leq \lim_{n \to \infty} \big(d(x^{(n)}, y^{(n)}) + d(y^{(n)}, z^{(n)})\big)                                                                                                       \\
     & = \lim_{n \to \infty} d(x^{(n)}, y^{(n)}) + \lim_{n \to \infty} d(y^{(n)}, z^{(n)})                                                                                                \\
     & = d_{\overline{X}}(\text{LIM}_{n \to \infty} x^{(n)}, \text{LIM}_{n \to \infty} y^{(n)}) + d_{\overline{X}}(\text{LIM}_{n \to \infty} y^{(n)}, \text{LIM}_{n \to \infty} z^{(n)}).
  \end{align*}
  Thus by \cref{1.1.2} \((\overline{X}, d_{\overline{X}})\) is a metric space.
\end{proof}

\begin{proof}{(c)}
  Let \(I, J, K, N, k_1, k_2, n_1, n_2 \in \Z^+\).
  Let \((a^{(n)})_{n = m}^\infty\) be arbitrary Cauchy sequence in \((\overline{X}, d_{\overline{X}})\).
  Since \((a^{(n)})_{n = m}^\infty\) is a Cauchy sequence in \((\overline{X}, d_{\overline{X}})\), by \cref{1.4.6} we know that
  \begin{align*}
             & \forall \varepsilon \in \R^+, \exists\ N \geq m : \forall n_1, n_2 \geq N,                                                                                 \\
             & d_{\overline{X}}(a^{(n_1)}, a^{(n_2)}) \leq \frac{\varepsilon}{4}                                                                                          \\
    \implies & \forall \varepsilon \in \R^+, \exists\ N \geq m : \forall n_1, n_2 \geq N,                                                                                 \\
             & d_{\overline{X}}(\text{LIM}_{k \to \infty} a_k^{(n_1)}, \text{LIM}_{k \to \infty} a_k^{(n_2)}) \leq \frac{\varepsilon}{4} & \text{(by \cref{ex:1.4.8}(a))} \\
    \implies & \forall \varepsilon \in \R^+, \exists\ N \geq m : \forall n_1, n_2 \geq N,                                                                                 \\
             & \lim_{k \to \infty} d(a_k^{(n_1)}, a_k^{(n_2)}) \leq \frac{\varepsilon}{4}.                                               & \text{(by \cref{ex:1.4.8}(b))}
  \end{align*}
  Since the choice of \(N\) depends on \(\varepsilon\), we denote such \(N\) as \(N_\varepsilon\).
  We can use axiom of choice to fix \(N_\varepsilon\) for each \(\varepsilon \in \R^+\), and we rewrite the above statement as
  \[
    \forall \varepsilon \in \R^+, \forall n_1, n_2 \geq N_\varepsilon, \lim_{k \to \infty} d(a_k^{(n_1)}, a_k^{(n_2)}) \leq \frac{\varepsilon}{4}.
  \]
  Let \(L = \lim_{k \to \infty} d(a_k^{(n_1)}, a_k^{(n_2)})\).
  Since \(\Big(d(a_k^{(n_1)}, a_k^{(n_2)})\Big)_{k = 1}^\infty\) is a sequence in \(\R\) and converges to \(L\) with respect to \(d_{l^1}|_{\R \times \R}\), we have
  \begin{align*}
             & \exists\ I \geq 1 : \forall k \geq I, \abs{d(a_k^{(n_1)}, a_k^{(n_2)}) - L} \leq \frac{\varepsilon}{4}                                                                                            \\
    \implies & \exists\ I \geq 1 : \forall k \geq I, -\frac{\varepsilon}{4} \leq d(a_k^{(n_1)}, a_k^{(n_2)}) - L \leq \frac{\varepsilon}{4}                                                                      \\
    \implies & \exists\ I \geq 1 : \forall k \geq I, 0 \leq d(a_k^{(n_1)}, a_k^{(n_2)}) - L + \frac{\varepsilon}{4} \leq \frac{\varepsilon}{2}                                                                   \\
    \implies & \exists\ I \geq 1 : \forall k \geq I, 0 \leq d(a_k^{(n_1)}, a_k^{(n_2)}) \leq d(a_k^{(n_1)}, a_k^{(n_2)}) - L + \frac{\varepsilon}{4} \leq \frac{\varepsilon}{2} & (L \leq \frac{\varepsilon}{4})
  \end{align*}
  Since such \(I\) depends on the choice of \(\varepsilon\), we denote such \(I\) as \(I_\varepsilon\).
  Again we can use axiom of choice to fix \(I_\varepsilon\) for each \(\varepsilon \in \R^+\), and we rewrite the above statement as
  \[
    \forall \varepsilon \in \R^+, \forall n_1, n_2 \geq N_\varepsilon, \forall k \geq I_\varepsilon, d(a_k^{(n_1)}, a_k^{(n_2)}) \leq \frac{\varepsilon}{2}.
  \]
  If we let \(M_\varepsilon = \max(N_\varepsilon, I_\varepsilon)\), then we can further reduce the statement as
  \[
    \forall \varepsilon \in \R^+, \forall n_1, n_2, k \geq M_\varepsilon, d(a_k^{(n_1)}, a_k^{(n_2)}) \leq \frac{\varepsilon}{2}.
  \]

  Since \(a^{(n)} \in \overline{X}\) for each \(n \geq m\), by \cref{ex:1.4.8}(b) we know that there exists a Cauchy sequence \((a_k^{(n)})_{k = 1}^\infty\) in \((X, d)\) such that \(\text{LIM}_{k \to \infty} a_k^{(n)} = a^{(n)}\) for each \(n \geq m\).
  By \cref{1.4.6} we know that
  \[
    \forall \varepsilon \in \R^+, \exists\ J \geq 1 : \forall k_1, k_2 \geq J, d(a_{k_1}^{(n)}, a_{k_2}^{(n)}) \leq \frac{\varepsilon}{2}.
  \]
  Since such \(J\) depends on the choice of \(n\) and \(\varepsilon\), we denote such \(J\) as \(J_\varepsilon^{(n)}\).
  We can use axiom of choice to fix \(J_\varepsilon^{(n)}\) for each \(\varepsilon \in \R^+\) and for each \(n \geq m\), and we rewrite the above statement as
  \[
    \forall \varepsilon \in \R^+, \forall k_1, k_2 \geq J_\varepsilon^{(n)}, d(a_{k_1}^{(n)}, a_{k_2}^{(n)}) \leq \frac{\varepsilon}{2}.
  \]

  We now define a sequence \((b_k)_{k = 1}^\infty\) in \((X, d)\) by setting \(b_k = a_k^{(m + k - 1)}\) for all \(k \geq 1\).
  Informally, \((b_k)_{k = 1}^\infty\) is consist of diagonal elements in \(\big((a_k^{(n)})_{k = 1}^\infty\big)_{n = m}^\infty\).
  We claim that \((b_k)_{k = 1}^\infty\) is a Cauchy sequence in \((X, d)\).
  By \cref{1.4.6} it suffices to show that
  \[
    \forall \varepsilon \in \R^+, \exists\ K \geq 1 : \forall k_1, k_2 \geq K, d(b_{k_1}, b_{k_2}) \leq \varepsilon.
  \]
  For each \(\varepsilon \in \R^+\), we have
  \begin{align*}
     & \forall k_1, k_2 \geq M_\varepsilon, d(b_{k_1}, b_{k_2})                                                                                                                         \\
     & = d\Big(a_{k_1}^{(m + k_1 - 1)}, a_{k_2}^{(m + k_2 - 1)}\Big)                                                                                                                    \\
     & \leq d\Big(a_{k_1}^{(m + k_1 - 1)}, a_{k_2}^{(m + k_1 - 1)}\Big) + d\Big(a_{k_2}^{(m + k_1 - 1)}, a_{k_2}^{(m + k_2 - 1)}\Big) & \text{(by \cref{1.1.2}(d))}                     \\
     & \leq d\Big(a_{k_1}^{(m + k_1 - 1)}, a_{k_2}^{(m + k_1 - 1)}\Big) + \frac{\varepsilon}{2}.                                      & \text{(by the definition of \(M_\varepsilon\))}
  \end{align*}
  By choosing \(K = \max(M_\varepsilon, J_\varepsilon^{(M_\varepsilon)})\) we have
  \begin{align*}
     & \forall k_1, k_2 \geq K, d(b_{k_1}, b_{k_2})                                                                                                                 \\
     & \leq d\Big(a_{k_1}^{(m + k_1 - 1)}, a_{k_2}^{(m + k_1 - 1)}\Big) + \frac{\varepsilon}{2}                                                                     \\
     & \leq \frac{\varepsilon}{2} + \frac{\varepsilon}{2}                                       & \text{(by the definition of \(J_\varepsilon^{(M_\varepsilon)}\))} \\
     & = \varepsilon.
  \end{align*}
  Since \(\varepsilon\) is arbitrary, we have showed that
  \[
    \forall \varepsilon \in \R^+, \exists\ K \geq 1 : \forall k_1, k_2 \geq K, d(b_{k_1}, b_{k_2}) \leq \varepsilon.
  \]
  Thus by \cref{1.4.6} \((b_k)_{k = 1}^\infty\) is a Cauchy sequence in \((X, d)\).

  Now we show that \((a^{(n)})_{n = m}^\infty\) converges in \((\overline{X}, d_{\overline{X}})\).
  From the proof above we know that \((b_k)_{k = 1}^\infty\) is a Cauchy sequence in \((X, d)\), so \(\text{LIM}_{k \to \infty} b_k \in \overline{X}\).
  We claim that \((a^{(n)})_{n = m}^\infty\) converges to \(\text{LIM}_{k \to \infty} b_k\) with respect to \(d_{\overline{X}}\).
  It suffices to show that
  \begin{align*}
         & \lim_{n \to \infty} d_{\overline{X}}\Big(a^{(n)}, \text{LIM}_{k \to \infty} b_k\Big) = 0                             & \text{(by \cref{1.1.14})}      \\
    \iff & \lim_{n \to \infty} d_{\overline{X}}\Big(\text{LIM}_{k \to \infty} a_k^{(n)}, \text{LIM}_{k \to \infty} b_k\Big) = 0 & \text{(by \cref{ex:1.4.8}(a))} \\
    \iff & \lim_{n \to \infty} \big(\lim_{k \to \infty} d(a_k^{(n)}, b_k)\big) = 0                                              & \text{(by \cref{ex:1.4.8}(b))} \\
    \iff & \forall \varepsilon \in \R^+, \exists\ N \geq m : \forall n \geq m,                                                                                   \\
         & \abs{\lim_{k \to \infty} d(a_k^{(n)}, b_k) - 0} \leq \varepsilon.
  \end{align*}
  Since
  \begin{align*}
     & \forall \varepsilon \in \R^+, \exists\ M_\varepsilon \geq 1 : \forall k \geq M_\varepsilon, d(a_k^{(M_\varepsilon)}, b_k)                                                   \\
     & = d\Big(a_k^{(M_\varepsilon)}, a_k^{(m + k - 1)}\Big)                                                                                                                       \\
     & \leq \frac{\varepsilon}{2}                                                                                                & \text{(by the definition of \(M_\varepsilon\))} \\
     & < \varepsilon,
  \end{align*}
  we know that \(\lim_{k \to \infty} d(a_k^{(M_\varepsilon)}, b_k)\) exists.
  Since
  \begin{align*}
             & \forall n, k \geq M_\varepsilon, 0 \leq d(a_k^{(n)}, b_k) \leq \frac{\varepsilon}{2} < \varepsilon & \text{(by the definition of \(M_\varepsilon\))} \\
    \implies & \forall n \geq M_\varepsilon, 0 \leq \lim_{k \to \infty} d(a_k^{(n)}, b_k) \leq \varepsilon        & \text{(by comparison test)}                     \\
    \implies & \forall n \geq M_\varepsilon, \abs{\lim_{k \to \infty} d(a_k^{(n)}, b_k) - 0} \leq \varepsilon.
  \end{align*}
  by setting \(N = M_\varepsilon\) we are done.

  Since for arbitrary Cauchy sequence \((a^{(n)})_{n = m}^\infty\) in \((\overline{X}, d_{\overline{X}})\), \((a^{(n)})_{n = m}^\infty\) converges in \(\overline{X}\) with respect to \(d_{\overline{X}}\), by \cref{1.4.10} we know that \((\overline{X}, d_{\overline{X}})\) is complete.
\end{proof}

\begin{proof}{(d)}
  Since for any \(x, y \in X\), we have
  \begin{align*}
         & x = y                                                                                                           \\
    \iff & d(x, y) = 0                                                                    & \text{(by \cref{1.1.2}(a))}    \\
    \iff & \lim_{n \to \infty} d(x, y) = 0                                                                                 \\
    \iff & d_{\overline{X}}(\text{LIM}_{n \to \infty} x, \text{LIM}_{n \to \infty} y) = 0 & \text{(by \cref{ex:1.4.8}(b))} \\
    \iff & \text{LIM}_{n \to \infty} x = \text{LIM}_{n \to \infty} y.                     & \text{(by \cref{1.1.2}(a))}
  \end{align*}
  Thus
  \begin{align*}
    d_{\overline{X}}(x, y) & = d_{\overline{X}}(\text{LIM}_{n \to \infty} x, \text{LIM}_{n \to \infty} y)                                  \\
                           & = \lim_{n \to \infty} d(x, y)                                                & \text{(by \cref{ex:1.4.8}(b))} \\
                           & = d(x, y).
  \end{align*}
\end{proof}

\begin{proof}{(e)}
  From \cref{ex:1.4.8}(d) have \(d = d_{\overline{X}}|_{X \times X}\).
  Let \(Y\) be the closure of \(X\) in \((\overline{X}, d_{\overline{X}})\).
  We want to show that \(Y = \overline{X}\).
  By \cref{1.2.9} we know that \(Y \subseteq \overline{X}\).
  Thus we only need to show that \(\overline{X} \subseteq Y\).

  Let \(x_0 \in \overline{X}\).
  By \cref{ex:1.4.8}(b) there exists a Cauchy sequence \((a_n)_{n = 1}^\infty\) in \((X, d_{\overline{X}})\) such that \(\text{LIM}_{n \to \infty} a_n = x_0\).
  Since \((\overline{X}, d_{\overline{X}})\) is complete, by \cref{1.4.10} we know that \((a_n)_{n = 1}^\infty\) converges in \(\overline{X}\) with respect to \(d_{\overline{X}}\).
  But we know that \((a_n)_{n = 1}^\infty\) is a Cauchy sequence in \((X, d_{\overline{X}})\), thus by \cref{1.2.10}(c) \(x_0\) is an adherent point of \(X\) in \((\overline{X}, d_{\overline{X}})\) and \(x_0 \in Y\).
  Since \(x_0\) is arbitrary, we thus have \(\overline{X} \subseteq Y\).
\end{proof}

\begin{proof}{(f)}
  By \cref{ex:1.4.8}(d) we know that if \((x_n)_{n = 1}^\infty\) is a Cauchy sequence in \((X, d)\), then \((x_n)_{n = 1}^\infty\) is also a Cauchy sequence in \((\overline{X}, d_{\overline{X}})\).
  Thus by \cref{ex:1.4.8}(c)(e) we have \(\lim_{n \to \infty} x_n = \text{LIM}_{n \to \infty} x_n\).
\end{proof}