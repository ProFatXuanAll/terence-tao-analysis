\section{Inner products on periodic functions}\label{sec:5.2}

\begin{defn}[Inner product]\label{5.2.1}
  If \(f, g \in C(\R / \Z ; \C)\), we define the \emph{inner product} \(\inner*{f, g}\) to be the quantity
  \[
    \inner*{f, g} = \int_{[0, 1]} f(x) \overline{g(x)} \; dx.
  \]
\end{defn}

\begin{rmk}\label{5.2.2}
  In order to integrate a complex-valued function over real variables, we use the definition that
  \[
    \int_{[a, b]} f(x) \; dx \coloneqq \int_{[a, b]} \Re\big(f(x)\big) \; dx + i \int_{[a,b]} \Im\big(f(x)\big) \; dx;
  \]
  i.e., we integrate the real and imaginary parts of the function separately.
  It is easy to verify that all the standard rules of calculus (integration by parts, fundamental theorem of calculus, substitution, etc.) still hold when the functions are complex-valued instead of real-valued.
\end{rmk}

\begin{proof}
  Let \(f \in C(\R / \Z ; \C)\), let \(d_{\R} = d_{l^1}|_{\R \times \R}\) and let \(d_{\C}\) be the metric in \cref{4.6.10}.
  Let \(x_0 \in \R\) and let \((a_n)_{n = 1}^\infty\) be a sequence in \(\R\) such that \(\lim_{n \to \infty} a_n = x_0\).
  Since \(f\) is continuous on \(\R\) from \((\R, d_{\R})\) to \((\C, d_{\C})\), we know that
  \begin{align*}
             & \lim_{n \to \infty} f(a_n) = f(x_0)                                           &  & \by{2.1.4} \\
    \implies & \begin{dcases}
                 \lim_{n \to \infty} \Re(f\big(a_n)\big) = \Re\big(f(x_0)\big) \\
                 \lim_{n \to \infty} \Im(f\big(a_n)\big) = \Im\big(f(x_0)\big)
               \end{dcases} &  & \by{4.6.13}
  \end{align*}
  Since \((a_n)_{n = 0}^\infty\) is arbitrary, by \cref{2.1.4} we know that \(\Re \circ f\) is continuous at \(x_0\) from \((\R, d_{\R})\) to \((\R, d_{\R})\).
  Since \(x_0\) is arbitrary, by \cref{2.1.5} we know that \(\Re \circ f\) is continuous on \(\R\) from \((\R, d_{\R})\) to \((\R, d_{\R})\).
  Using similar arguments we can show that \(\Im \circ f\) is continuous on \(\R\) from \((\R, d_{\R})\) to \((\R, d_{\R})\).
  Since
  \begin{align*}
     & \forall x \in \R, \Re(f(x + 1)) = \Re(f(x)) \implies \Re \circ f \in C(\R / \Z ; \C); \\
     & \forall x \in \R, \Im(f(x + 1)) = \Im(f(x)) \implies \Im \circ f \in C(\R / \Z ; \C),
  \end{align*}
  by \cref{5.1.5}(a) we know that both \(\Re \circ f\) and \(\Im \circ f\) are bounded in \((\C, d_{\C})\).
  In particular, by \cref{4.6.8} we know that \((\Re \circ f)(\R) \subseteq \R\) and \((\Im \circ f)(\R) \subseteq \R\).
  Thus both \(\Re \circ f\) and \(\Im \circ f\) are bounded in \((\R, d_{\R})\).
  Since \(\Re \circ f\) and \(\Im \circ f\) are continuous and bounded on \([0, 1]\), by Corollary 11.5.2 in Analysis I we know that \(\Re \circ f\) and \(\Im \circ f\) are Riemann integrable on \([0, 1]\).
  Thus
  \[
    \int_{[0, 1]} f(x) \; dx = \int_{[0, 1]} \Re\big(f(x)\big) \; dx + i \bigg(\int_{[0, 1]} \Im\big(f(x)\big) \; dx\bigg) \in \C
  \]
  is well-defined.
  The same argument holds on arbitrary closed interval \([a, b]\) since \(f \in C(\R / \Z ; \C)\).
\end{proof}

\begin{eg}\label{5.2.3}
  Let \(f\) be the constant function \(f(x) \coloneqq 1\), and let \(g(x)\) be the function \(g(x) \coloneqq e^{2 \pi i x}\).
  Then we have
  \begin{align*}
    \inner*{f, g} & = \int_{[0, 1]} 1 \overline{e^{2 \pi i x}} \; dx       \\
                  & = \int_{[0, 1]} e^{- 2 \pi i x} \; dx                  \\
                  & = \dfrac{e^{- 2 \pi i x}}{- 2 \pi i} |_{x = 0}^{x = 1} \\
                  & = \dfrac{e^{- 2 \pi i } - e^0}{- 2 \pi i}              \\
                  & = \dfrac{1 - 1}{- 2 \pi i}                             \\
                  & = 0.
  \end{align*}
\end{eg}

\begin{rmk}\label{5.2.4}
  In general, the inner product \(\inner*{f, g}\) will be a complex number.
  (Note that \(f(x) \overline{g(x)}\) will be Riemann integrable since both functions are bounded and continuous.)
\end{rmk}

\begin{note}
  Roughly speaking, the inner product \(\inner*{f, g}\) is to the space \(C(\R / \Z ; \C)\) what the dot product \(x \cdot y\) is to Euclidean spaces such as \(\R^n\).
  A more in-depth study of inner products on vector spaces can be found in any linear algebra text but is beyond the scope of this text.
\end{note}

\begin{lem}\label{5.2.5}
  Let \(f, g, h \in C(\R / \Z ; \C)\).
  \begin{enumerate}
    \item (Hermitian property)
          We have \(\inner*{g, f} = \overline{\inner*{f, g}}\).
    \item (Positivity)
          We have \(\inner*{f, f} \geq 0\).
          Furthermore, we have \(\inner*{f, f} = 0\) iff \(f = 0\)
          (i.e., \(f(x) = 0\) for all \(x \in \R\)).
    \item (Linearity in the first variable)
          We have \(\inner*{f +g, h}\) = \(\inner*{f, h} + \inner*{g, h}\).
          For any complex number \(c\), we have \(\inner*{cf, g} = c \inner*{f, g}\).
    \item (Antilinearity in the second variable)
          We have \(\inner*{f, g + h} = \inner*{f, g} + \inner*{f, h}\).
          For any complex number \(c\), we have \(\inner*{f, cg} = c \inner*{f, g}\).
  \end{enumerate}
\end{lem}

\begin{proof}{(a)}
  We have
  \begin{align*}
    \overline{\inner*{f, g}} & = \overline{\int_{[0, 1]} f(x) \overline{g(x)} \; dx}                                                                                               &  & \by{5.2.1} \\
                             & = \overline{\int_{[0, 1]} \Re\big(f(x) \overline{g(x)}\big) \; dx + i \bigg(\int_{[0, 1]} \Im\big(f(x) \overline{g(x)}\big) \; dx\bigg)}            &  & \by{5.2.2} \\
                             & = \int_{[0, 1]} \Re\big(f(x) \overline{g(x)}\big) \; dx - i \bigg(\int_{[0, 1]} \Im\big(f(x) \overline{g(x)}\big) \; dx\bigg)                       &  & \by{4.6.8} \\
                             & = \int_{[0, 1]} \Re\big(f(x) \overline{g(x)}\big) \; dx + i \bigg(-\int_{[0, 1]} \Im\big(f(x) \overline{g(x)}\big) \; dx\bigg)                      &  & \by{4.6.6} \\
                             & = \int_{[0, 1]} \Re\big(f(x) \overline{g(x)}\big) \; dx + i \bigg(\int_{[0, 1]} -\Im\big(f(x) \overline{g(x)}\big) \; dx\bigg)                                      \\
                             & = \int_{[0, 1]} \Re\big(\overline{f(x) \overline{g(x)}}\big) \; dx + i \bigg(\int_{[0, 1]} \Im\big(\overline{f(x) \overline{g(x)}}\big) \; dx\bigg) &  & \by{4.6.8} \\
                             & = \int_{[0, 1]} \Re\big(\overline{f(x)} g(x)\big) \; dx + i \bigg(\int_{[0, 1]} \Im\big(\overline{f(x)} g(x)\big) \; dx\bigg)                       &  & \by{4.6.9} \\
                             & = \int_{[0, 1]} \Re\big(g(x) \overline{f(x)}\big) \; dx + i \bigg(\int_{[0, 1]} \Im\big(g(x) \overline{f(x)}\big) \; dx\bigg)                       &  & \by{4.6.6} \\
                             & = \int_{[0, 1]} g(x) \overline{f(x)} \; dx                                                                                                          &  & \by{5.2.2} \\
                             & = \inner*{g, f}.                                                                                                                                    &  & \by{5.2.1}
  \end{align*}
\end{proof}

\begin{proof}{(b)}
  We have
  \begin{align*}
    \inner*{f, f} & = \int_{[0, 1]} f(x) \overline{f(x)} \; dx &  & \by{5.2.1}                                  \\
                  & = \int_{[0, 1]} \abs{f(x)}^2 \; dx         &  & \by{4.6.11}                                 \\
                  & \geq \int_{[0, 1]} 0 \; dx                 &  & \text{(by Theorem 11.4.1(d) in Analysis I)} \\
                  & = 0
  \end{align*}
  and
  \begin{align*}
         & \int_{[0, 1]} \abs{f(x)}^2 \; dx = 0                                                  \\
    \iff & \forall x \in [0, 1], \abs{f(x)}^2 = 0 &  & \text{(by Exercise 11.4.2 in Analysis I)} \\
    \iff & \forall x \in [0, 1], \abs{f(x)} = 0                                                  \\
    \iff & \forall x \in [0, 1], f(x) = 0.        &  & \by{4.6.11}
  \end{align*}
\end{proof}

\begin{proof}{(c)}
  We have
  \begin{align*}
     & \inner*{f + g, h}                                                                                                                                                                                       \\
     & = \int_{[0, 1]} (f + g)(x) \overline{h(x)} \; dx                                                                                    &                                                      & \by{5.2.1} \\
     & = \int_{[0, 1]} \Re\big((f + g)(x) \overline{h(x)}\big) \; dx                                                                       &                                                      & \by{5.2.2} \\
     & \quad + i \bigg(\int_{[0, 1]} \Im\big((f + g)(x) \overline{h(x)}\big) \; dx\bigg)                                                                                                                       \\
     & = \int_{[0, 1]} \Re\big(f(x) \overline{h(x)} + g(x) \overline{h(x)}\big) \; dx                                                                                                                          \\
     & \quad + i \bigg(\int_{[0, 1]} \Im\big(f(x) \overline{h(x)} + g(x) \overline{h(x)}\big) \; dx\bigg)                                                                                                      \\
     & = \int_{[0, 1]} \Re\big(f(x) \overline{h(x)}) + \Re\big(g(x) \overline{h(x)}\big) \; dx                                             &                                                      & \by{4.6.8} \\
     & \quad + i \bigg(\int_{[0, 1]} \Im\big(f(x) \overline{h(x)}\big) + \Im\big(g(x) \overline{h(x)}\big) \; dx\bigg)                                                                                         \\
     & = \int_{[0, 1]} \Re\big(f(x) \overline{h(x)}) \; dx + \int_{[0, 1]} \Re\big(g(x) \overline{h(x)}\big) \; dx                         & (f \overline{h}, g \overline{h} \in C(\R / \Z ; \C))              \\
     & \quad + i \bigg(\int_{[0, 1]} \Im\big(f(x) \overline{h(x)}\big) \; dx + \int_{[0, 1]} \Im\big(g(x) \overline{h(x)}\big) \; dx\bigg)                                                                     \\
     & = \int_{[0, 1]} f(x) \overline{h(x)} \; dx + \int_{[0, 1]} g(x) \overline{h(x)} \; dx                                               &                                                      & \by{5.2.2} \\
     & = \inner*{f, h} + \inner*{g, h}                                                                                                     &                                                      & \by{5.2.1}
  \end{align*}
  and
  \begin{align*}
     & \inner*{cf, g}                                                                                                                                                                                   \\
     & = \int_{[0, 1]} (cf)(x) \overline{g(x)} \; dx                                                                                                &                                      & \by{5.2.1} \\
     & = \int_{[0, 1]} \Re\big((cf)(x) \overline{g(x)}\big) \; dx + i \bigg(\int_{[0, 1]} \Im\big((cf)(x) \overline{g(x)}\big) \; dx\bigg)          &                                      & \by{5.2.2} \\
     & = \int_{[0, 1]} \Re\big(cf(x) \overline{g(x)}\big) \; dx + i \bigg(\int_{[0, 1]} \Im\big(cf(x) \overline{g(x)}\big) \; dx\bigg)                                                                  \\
     & = \int_{[0, 1]} \Re(c) \Re\big(f(x) \overline{g(x)}\big) - \Im(c) \Im\big(f(x) \overline{g(x)}\big) \; dx                                    &                                      & \by{4.6.5} \\
     & \quad + i \bigg(\int_{[0, 1]} \Re(c) \Im\big(f(x) \overline{g(x)}\big) + \Im(c) \Re\big(f(x) \overline{g(x)}\big) \; dx\bigg)                                                                    \\
     & = \Re(c) \bigg(\int_{[0, 1]} \Re\big(f(x) \overline{g(x)}\big) \; dx\bigg)                                                                   & (f \overline{g} \in C(\R / \Z ; \C))              \\
     & \quad - \Im(c) \bigg(\int_{[0, 1]} \Im\big(f(x) \overline{g(x)}\big) \; dx\bigg)                                                                                                                 \\
     & \quad + i \Re(c) \bigg(\int_{[0, 1]} \Im\big(f(x) \overline{g(x)}\big) \; dx\bigg)                                                                                                               \\
     & \quad + i \Im(c) \bigg(\int_{[0, 1]} \Re\big(f(x) \overline{g(x)}\big) \; dx\bigg)                                                                                                               \\
     & = \Re(c) \Bigg(\int_{[0, 1]} \Re\big(f(x) \overline{g(x)}\big) \; dx + i \int_{[0, 1]} \Im\big(f(x) \overline{g(x)}\big) \; dx\Bigg)         &                                      & \by{4.6.6} \\
     & \quad + i \Im(c) \Bigg(\int_{[0, 1]} \Re\big(f(x) \overline{g(x)}\big) \; dx + i \int_{[0, 1]} \Im\big(f(x) \overline{g(x)}\big) \; dx\Bigg) &                                      & \by{4.6.5} \\
     & = \Re(c) \int_{[0, 1]} f(x) \overline{g(x)} \; dx + i \Im(c) \int_{[0, 1]} f(x) \overline{g(x)} \; dx                                        &                                      & \by{5.2.2} \\
     & = \Re(c) \inner*{f, g} + i \Im(c) \inner*{f, g}                                                                                              &                                      & \by{5.2.1} \\
     & = \big(\Re(c) + i \Im(c)\big) \inner*{f, g}                                                                                                  &                                      & \by{4.6.6} \\
     & = c \inner*{f, g}.                                                                                                                           &                                      & \by{4.6.8}
  \end{align*}
\end{proof}

\begin{proof}{(d)}
  We have
  \begin{align*}
     & \inner*{f, g + h}                                                                                                                                                                                       \\
     & = \int_{[0, 1]} f(x) \overline{(g + h)(x)} \; dx                                                                                    &                                                      & \by{5.2.1} \\
     & = \int_{[0, 1]} \Re\big(f(x) \overline{(g + h)(x)}\big) \; dx                                                                                                                                           \\
     & \quad + i \bigg(\int_{[0, 1]} \Im\big(f(x) \overline{(g + h)(x)}\big) \; dx\bigg)                                                   &                                                      & \by{5.2.2} \\
     & = \int_{[0, 1]} \Re\big(f(x) \overline{g(x)} + f(x) \overline{h(x)}\big) \; dx                                                      &                                                      & \by{4.6.9} \\
     & \quad + i \bigg(\int_{[0, 1]} \Im\big(f(x) \overline{g(x)} + f(x) \overline{h(x)}\big) \; dx\bigg)                                                                                                      \\
     & = \int_{[0, 1]} \Re\big(f(x) \overline{g(x)}) + \Re\big(f(x) \overline{h(x)}\big) \; dx                                             &                                                      & \by{4.6.8} \\
     & \quad + i \bigg(\int_{[0, 1]} \Im\big(f(x) \overline{g(x)}\big) + \Im\big(f(x) \overline{h(x)}\big) \; dx\bigg)                                                                                         \\
     & = \int_{[0, 1]} \Re\big(f(x) \overline{g(x)}) \; dx + \int_{[0, 1]} \Re\big(f(x) \overline{h(x)}\big) \; dx                         & (f \overline{g}, f \overline{h} \in C(\R / \Z ; \C))              \\
     & \quad + i \bigg(\int_{[0, 1]} \Im\big(f(x) \overline{g(x)}\big) \; dx + \int_{[0, 1]} \Im\big(f(x) \overline{h(x)}\big) \; dx\bigg)                                                                     \\
     & = \int_{[0, 1]} f(x) \overline{g(x)} \; dx + \int_{[0, 1]} f(x) \overline{h(x)} \; dx                                               &                                                      & \by{5.2.2} \\
     & = \inner*{f, g} + \inner*{f, h}                                                                                                     &                                                      & \by{5.2.1}
  \end{align*}
  and
  \begin{align*}
     & \inner*{f, cg}                                                                                                                                                                               \\
     & = \int_{[0, 1]} (x) \overline{(cg)(x)} \; dx                                                                                                                &  & \by{5.2.1}                  \\
     & = \int_{[0, 1]} \Re\big(f(x) \overline{(cg)(x)}\big) \; dx + i \bigg(\int_{[0, 1]} \Im\big(f(x) \overline{(cg)(x)}\big) \; dx\bigg)                         &  & \by{5.2.2}                  \\
     & = \int_{[0, 1]} \Re\big(\overline{c} f(x) \overline{g(x)}\big) \; dx + i \bigg(\int_{[0, 1]} \Im\big(\overline{c} f(x) \overline{g(x)}\big) \; dx\bigg)     &  & \by{4.6.9}                  \\
     & = \int_{[0, 1]} \Re\big((\overline{c} f)(x) \overline{g(x)}\big) \; dx + i \bigg(\int_{[0, 1]} \Im\big((\overline{c} f)(x) \overline{g(x)}\big) \; dx\bigg)                                  \\
     & = \int_{[0, 1]} (\overline{c} f)(x) \overline{g(x)} \; dx                                                                                                   &  & \by{5.2.2}                  \\
     & = \inner*{\overline{c} f, g}                                                                                                                                &  & \by{5.2.1}                  \\
     & = \overline{c} \inner*{f, g}.                                                                                                                               &  & \text{(by \cref{5.2.5}(c))}
  \end{align*}
\end{proof}

\begin{ac}\label{ac:5.2.1}
  From the positivity property (\cref{5.2.5}(b)), it makes sense to define the \(L^2\) norm \(\norm*{f}_2\) of a function \(f \in C(\R / \Z ; \C)\) by the formula
  \[
    \norm*{f}_2 \coloneqq \sqrt{\inner*{f, f}} = \bigg(\int_{[0, 1]} f(x) \overline{f(x)} \; dx\bigg)^{1 / 2} = \bigg(\int_{[0, 1]} \abs{f(x)}^2 \; dx\bigg)^{1 / 2}.
  \]
  Thus \(\norm*{f}_2 \geq 0\) for all \(f\).
  The norm \(\norm*{f}_2\) is sometimes called the \emph{root mean square} of \(f\).
\end{ac}

\begin{note}
  This \(L^2\) norm is related to, but is distinct from, the \(L^\infty\) norm
  \[
    \norm*{f}_\infty \coloneqq \sup_{x \in \R} \abs{f(x)}.
  \]
  In general, the best one can say is that \(0 \leq \norm*{f}_2 \leq \norm*{f}_\infty\).
\end{note}

\setcounter{thm}{6}
\begin{lem}\label{5.2.7}
  Let \(f, g \in C(\R / \Z ; \C)\).
  \begin{enumerate}
    \item (Non-degeneracy)
          We have \(\norm*{f}_2 = 0\) iff \(f = 0\).
    \item (Cauchy-Schwarz inequality)
          We have \(\abs{\inner*{f, g}} \leq \norm*{f}_2 \norm*{g}_2\).
    \item (Triangle inequality)
          We have \(\norm*{f + g}_2 \leq \norm*{f}_2 + \norm*{g}_2\).
    \item (Pythagoras' theorem)
          If \(\inner*{f, g} = 0\), then \(\norm*{f + g}_2^2 = \norm*{f}_2^2 + \norm*{g}_2^2\).
    \item (Homogeneity)
          We have \(\norm*{cf}_2 = \abs{c} \norm*{f}_2\) for all \(c \in \C\).
  \end{enumerate}
\end{lem}

\begin{proof}{(a)}
  We have
  \begin{align*}
         & \norm*{f}_2 = 0                                           \\
    \iff & \sqrt{\inner*{f, f}} = 0 &  & \by{ac:5.2.1}               \\
    \iff & \inner*{f, f} = 0                                         \\
    \iff & f = 0.                   &  & \text{(by \cref{5.2.5}(b))}
  \end{align*}
\end{proof}

\begin{proof}{(b)}
  If \(g\) is zero function on \([0, 1]\), then we have
  \begin{align*}
    \abs{\inner*{f, g}} & = \abs{\int_{[0, 1]} f(x) \overline{g(x)} \; dx} &  & \by{5.2.1}                  \\
                        & = \abs{\int_{[0, 1]} f(x) \cdot 0 \; dx}                                          \\
                        & = \abs{\int_{[0, 1]} 0 \; dx}                                                     \\
                        & = 0                                                                               \\
                        & = \norm*{f}_2 \norm*{g}_2.                       &  & \text{(by \cref{5.2.7}(a))}
  \end{align*}
  So suppose that \(g\) is not zero function on \([0, 1]\).
  Observe that
  \begin{align*}
             & \norm*{g}_2 \in \R                                                 &  & \by{ac:5.2.1}               \\
    \implies & \norm*{g}_2^2 \in \R                                                                                \\
    \implies & \norm*{g}_2^2 \cdot f \in C(\R / \Z ; \C)                          &  & \text{(by \cref{5.1.5}(b))} \\
    \implies & \norm*{g}_2^2 \cdot f - \inner*{f, g} \cdot g \in C(\R / \Z ; \C). &  & \text{(by \cref{5.1.5}(b))}
  \end{align*}
  If we let \(h = \norm*{g}_2^2 \cdot f - \inner*{f, g} \cdot g\), then by \cref{ac:5.2.1} we know that \(\inner*{h, h}\) is well-defined.
  Thus we have
  \begin{align*}
     & \inner*{h, h}                                                                                                                                             \\
     & = \inner*{\norm*{g}_2^2 \cdot f - \inner*{f, g} \cdot g, \norm*{g}_2^2 \cdot f - \inner*{f, g} \cdot g}                                                   \\
     & = \inner*{\norm*{g}_2^2 \cdot f, \norm*{g}_2^2 \cdot f - \inner*{f, g} \cdot g}                                          &  & \text{(by \cref{5.2.5}(c))} \\
     & \quad + \inner*{-\inner*{f, g} \cdot g, \norm*{g}_2^2 \cdot f - \inner*{f, g} \cdot g}                                                                    \\
     & = \inner*{\norm*{g}_2^2 \cdot f, \norm*{g}_2^2 \cdot f} + \inner*{\norm*{g}_2^2 \cdot f, -\inner*{f, g} \cdot g}         &  & \text{(by \cref{5.2.5}(d))} \\
     & \quad + \inner*{-\inner*{f, g} \cdot g, \norm*{g}_2^2 \cdot f} + \inner*{-\inner*{f, g} \cdot g, -\inner*{f, g} \cdot g}                                  \\
     & = \norm*{g}_2^2 \inner*{f, \norm*{g}_2^2 \cdot f} + \norm*{g}_2^2 \inner*{f, -\inner*{f, g} \cdot g}                     &  & \text{(by \cref{5.2.5}(c))} \\
     & \quad - \inner*{f, g} \inner*{g, \norm*{g}_2^2 \cdot f} - \inner*{f, g} \inner*{g, -\inner*{f, g} \cdot g}                                                \\
     & = \norm*{g}_2^2 \overline{\norm*{g}_2^2} \inner*{f, f} + \norm*{g}_2^2 \overline{-\inner*{f, g}} \inner*{f, g}           &  & \text{(by \cref{5.2.5}(d))} \\
     & \quad - \inner*{f, g} \overline{\norm*{g}_2^2} \inner*{g, f} - \inner*{f, g} \overline{-\inner*{f, g}} \inner*{g, g}                                      \\
     & = \norm*{g}_2^4 \inner*{f, f} - \norm*{g}_2^2 \overline{\inner*{f, g}} \inner*{f, g}                                     &  & \by{4.6.9}                  \\
     & \quad - \inner*{f, g} \norm*{g}_2^2 \inner*{g, f} + \inner*{f, g} \overline{\inner*{f, g}} \inner*{g, g}                                                  \\
     & = \norm*{g}_2^4 \inner*{f, f} - \norm*{g}_2^2 \overline{\inner*{f, g}} \inner*{f, g}                                                                      \\
     & \quad - \inner*{f, g} \norm*{g}_2^2 \overline{\inner*{f, g}} + \inner*{f, g} \overline{\inner*{f, g}} \inner*{g, g}      &  & \text{(by \cref{5.2.5}(a))} \\
     & = \norm*{g}_2^4 \inner*{f, f} - 2 \norm*{g}_2^2 \abs{\inner*{f, g}}^2 + \abs{\inner*{f, g}}^2 \inner*{g, g}              &  & \by{4.6.10}                 \\
     & = \norm*{g}_2^4 \norm*{f}_2^2 - 2 \norm*{g}_2^2 \abs{\inner*{f, g}}^2 + \abs{\inner*{f, g}}^2 \norm*{g}_2^2              &  & \by{ac:5.2.1}               \\
     & = \norm*{g}_2^4 \norm*{f}_2^2 - \norm*{g}_2^2 \abs{\inner*{f, g}}^2
  \end{align*}
  and
  \begin{align*}
             & \inner*{h, h} \geq 0                                                     &  & \text{(by \cref{5.2.5}(b))} \\
    \implies & \norm*{g}_2^4 \norm*{f}_2^2 - \norm*{g}_2^2 \abs{\inner*{f, g}}^2 \geq 0                                  \\
    \implies & \norm*{g}_2^2 \norm*{f}_2^2 - \abs{\inner*{f, g}}^2 \geq 0               &  & \text{(by \cref{5.2.7}(a))} \\
    \implies & \norm*{g}_2 \norm*{f}_2 \geq \abs{\inner*{f, g}}.                        &  & \text{(by \cref{5.2.5}(b))}
  \end{align*}
\end{proof}

\begin{proof}{(c)}
  We have
  \begin{align*}
    \norm*{f + g}_2^2 & = \inner*{f + g, f + g}                                         &  & \by{ac:5.2.1}                  \\
                      & = \inner*{f, f} + \inner*{f, g} + \inner*{g, f} + \inner*{g, g} &  & \text{(by \cref{5.2.5}(c)(d))} \\
                      & = \norm*{f}_2^2 + \inner*{f, g} + \inner*{g, f} + \norm*{g}_2^2 &  & \by{ac:5.2.1}                  \\
                      & \leq \norm*{f}_2^2 + 2 \norm*{f}_2 \norm*{g}_2 + \norm*{g}_2^2  &  & \text{(by \cref{5.2.7}(b))}    \\
                      & = \big(\norm*{f}_2 + \norm*{g}_2\big)^2.
  \end{align*}
  Thus
  \begin{align*}
             & \norm*{f + g}_2^2 \leq (\norm*{f}_2 + \norm*{g}_2)^2                                  \\
    \implies & \norm*{f + g}_2 \leq \norm*{f}_2 + \norm*{g}_2.      &  & \text{(by \cref{5.2.5}(b))}
  \end{align*}
\end{proof}

\begin{proof}{(d)}
  We have
  \begin{align*}
    \norm*{f + g}_2^2 & = \inner*{f + g, f + g}                                                    &  & \by{ac:5.2.1}                  \\
                      & = \inner*{f, f} + \inner*{f, g} + \inner*{g, f} + \inner*{g, g}            &  & \text{(by \cref{5.2.5}(c)(d))} \\
                      & = \inner*{f, f} + \inner*{f, g} + \overline{\inner*{f, g}} + \inner*{g, g} &  & \text{(by \cref{5.2.5}(a))}    \\
                      & = \inner*{f, f} + \inner*{g, g}                                            &  & \text{(by hypothesis)}         \\
                      & = \norm*{f}_2^2 + \norm*{g}_2^2.                                           &  & \by{ac:5.2.1}
  \end{align*}
\end{proof}

\begin{proof}{(e)}
  We have
  \begin{align*}
    \norm*{cf}_2 & = \sqrt{\inner*{cf, cf}}              &  & \by{ac:5.2.1}                  \\
                 & = \sqrt{c \overline{c} \inner*{f, f}} &  & \text{(by \cref{5.2.5}(c)(d))} \\
                 & = \sqrt{\abs{c}^2 \inner*{f, f}}      &  & \by{4.6.11}                    \\
                 & = \abs{c} \sqrt{\inner*{f, f}}                                            \\
                 & = \abs{c} \norm*{f}_2.                &  & \by{ac:5.2.1}
  \end{align*}
\end{proof}

\begin{note}
  In light of Pythagoras' theorem, we sometimes say that \(f\) and \(g\) are \emph{orthogonal} iff \(\inner*{f, g} = 0\).
\end{note}

\begin{ac}\label{ac:5.2.2}
  We can now define the \(L^2\) metric \(d_{L^2}\) on \(C(\R / \Z ; \C)\) by defining
  \[
    d_{L^2}(f, g) \coloneqq \norm*{f - g}_2 = \bigg(\int_{[0, 1]} \abs{f(x) - g(x)}^2 \; dx\bigg)^{1 / 2}.
  \]
\end{ac}

\begin{rmk}\label{5.2.8}
  One can verify that \(d_{L^2}\) is indeed a metric.
  Indeed, the \(L^2\) metric is very similar to the \(l^2\) metric on Euclidean spaces \(\R^n\), which is why the notation is deliberately chosen to be similar;
  you should compare the two metrics yourself to see the analogy.
\end{rmk}

\begin{note}
  A sequence \(f_n\) of functions in \(C(\R / \Z ; \C)\) will \emph{converge in the \(L^2\) metric} to \(f \in C(\R / \Z ; \C)\) if \(d_{L^2}(f_n, f) \to 0\) as \(n \to \infty\), or in other words that
  \[
    \lim_{n \to \infty} \int_{[0, 1]} \abs{f_n(x) - f(x)}^2 \; dx = 0.
  \]
\end{note}

\begin{rmk}\label{5.2.9}
  The notion of convergence in \(L^2\) metric is different from that of uniform or pointwise convergence.
\end{rmk}

\begin{rmk}\label{5.2.10}
  The \(L^2\) metric is not as well-behaved as the \(L_\infty\) metric.
  For instance, it turns out the space \(C(\R / \Z ; \C)\) is not complete in the \(L^2\) metric, despite being complete in the \(L_\infty\) metric.
\end{rmk}

\exercisesection

\begin{ex}\label{ex:5.2.1}
  Prove \cref{5.2.5}.
\end{ex}

\begin{proof}
  See \cref{5.2.5}.
\end{proof}

\begin{ex}\label{ex:5.2.2}
  Prove \cref{5.2.7}.
\end{ex}

\begin{proof}
  See \cref{5.2.7}.
\end{proof}

\begin{ex}\label{ex:5.2.3}
  If \(f \in C(\R / \Z ; \C)\) is a non-zero function, show that \(0 < \norm*{f}_2 \leq \norm*{f}_\infty\).
  Conversely, if \(0 < A \leq B\) are real numbers, show that there exists a non-zero function \(f \in C(\R / \Z ; \C)\) such that \(\norm*{f}_2 = A\) and \(\norm*{f}_\infty = B\).
\end{ex}

\begin{proof}
  First we show that \(f \in C(\R / \Z ; \C)\) and \(f \neq 0\) implies \(0 < \norm*{f}_2 \leq \norm*{f}_{\infty}\).
  By \cref{5.2.7}(a) we know that \(0 < \norm*{f}_2\).
  Thus we only need to show that \(\norm*{f}_2 \leq \norm*{f}_\infty\).
  By \cref{5.1.5}(a) we know that \(f\) is bounded, thus by \cref{3.5.5}
  \[
    \norm*{f}_{\infty} = \sup_{y \in \R} \abs{f(x)} = \sup_{y \in [0, 1]} \abs{f(x)} \in \R^+ \cup \set{0}.
  \]
  Since
  \begin{align*}
    \norm*{f}_2^2 & = \int_{[0, 1]} \abs{f(x)}^2 \; dx                                  &  & \by{ac:5.2.1} \\
                  & \leq \int_{[0, 1]} \big(\sup_{y \in [0, 1]} \abs{f(y)}\big)^2 \; dx                    \\
                  & = \big(\sup_{y \in [0, 1]} \abs{f(y)}\big)^2                                           \\
                  & = \norm*{f}_{\infty}^2,                                             &  & \by{3.5.5}
  \end{align*}
  we know that
  \[
    \norm*{f}_2^2 \leq \norm*{f}_{\infty}^2 \implies \norm*{f}_2 \leq \norm*{f}_{\infty}.
  \]

  Now we show that for arbitrary \(A, B \in \R\), we have
  \[
    0 < A \leq B \implies \exists f \in C(\R / \Z ; \C) : \begin{dcases}
      f \neq 0        \\
      \norm*{f}_2 = A \\
      \norm*{f}_{\infty} = B
    \end{dcases}
  \]
  So let \(A, B \in \R\) such that \(0 < A \leq B\).
  We want to find some \(f \in C(\R / \Z ; \C)\) such that
  \begin{align*}
    A^2 & = \norm*{f}_2^2 = \int_{[0, 1]} \abs{f(x)}^2 \; dx;                  \\
    B^2 & = \norm*{f}_{\infty}^2 = \big(\sup_{x \in [0, 1]} \abs{f(x)}\big)^2.
  \end{align*}
  In particular, we want our \(f\) to look like
  \[
    \forall x \in [0, 1], f(x) = \sqrt{c + d g(x)},
  \]
  where \(c, d \in \R^+\) and \(g \in C(\R / \Z ; \C)\) such that \(g(\R) \subseteq \R^+ \cup \set{0}\).
  So we are trying to solve the following equations:
  \begin{align*}
    A^2 & = \int_{[0, 1]} \abs{\sqrt{c + dg(x)}}^2 \; dx           \\
        & = \int_{[0, 1]} c + dg(x) \; dx                          \\
        & = c + d \int_{[0, 1]} g(x) \; dx;                        \\
    B^2 & = \big(\sup_{x \in [0, 1]} \abs{\sqrt{c + dg(x)}}\big)^2 \\
        & = \sup_{x \in [0, 1]} \abs{\sqrt{c + dg(x)}}^2           \\
        & = \sup_{x \in [0, 1]} \big(c + dg(x)\big)                \\
        & = c + d \big(\sup_{x \in [0, 1]} g(x)\big).
  \end{align*}
  By setting
  \begin{align*}
     & c = \dfrac{A^2}{2};                                                                                                                                      \\
     & d = \dfrac{1}{2};                                                                                                                                        \\
     & \forall x \in [0, 1], g(x) = \begin{dcases}
                                      \dfrac{(2 B^2 - A^2)^2}{A^2} x                    & \text{if } x \in [0, \dfrac{A^2}{2 B^2 - A^2})                          \\
                                      \dfrac{-(2 B^2 - A^2)^2}{A^2} x + 2 (2 B^2 - A^2) & \text{if } x \in [\dfrac{A^2}{2 B^2 - A^2}, \dfrac{2 A^2}{2 B^2 - A^2}) \\
                                      0                                                 & \text{if } x \in [\dfrac{2 A^2}{2 B^2 - A^2}, 1]
                                    \end{dcases},
  \end{align*}
  we have
  \begin{align*}
     & \int_{[0, 1]} g(x) \; dx                                                                                                                                                                                                     \\
     & = \int_{[0, \dfrac{A^2}{2 B^2 - A^2}]} \dfrac{(2 B^2 - A^2)^2}{A^2} x \; dx + \int_{[\dfrac{A^2}{2 B^2 - A^2}, \dfrac{2 A^2}{2 B^2 - A^2}]} \dfrac{-(2 B^2 - A^2)^2}{A^2} x + 2 (2 B^2 - A^2) \; dx                          \\
     & = \dfrac{(2 B^2 - A^2)^2}{A^2} \bigg(\dfrac{x^2}{2}|_{x = 0}^{x = \dfrac{A^2}{2 B^2 - A^2}}\bigg) - \dfrac{(2 B^2 - A^2)^2}{A^2} \bigg(\dfrac{x^2}{2}|_{x = \dfrac{A^2}{2 B^2 - A^2}}^{x = \dfrac{2 A^2}{2 B^2 - A^2}}\bigg) \\
     & \quad + 2 (2 B^2 - A^2) \bigg(\dfrac{2 A^2}{2 B^2 - A^2} - \dfrac{A^2}{2 B^2 - A^2}\bigg)                                                                                                                                    \\
     & = \dfrac{A^2}{2} - 2 A^2 + \dfrac{A^2}{2} + 2 A^2                                                                                                                                                                            \\
     & = A^2
  \end{align*}
  and
  \begin{align*}
    \sup_{[0, 1]} g(x) & = \dfrac{(2 B^2 - A^2)^2}{A^2} \dfrac{A^2}{2 B^2 - A^2} \\
                       & = 2 B^2 - A^2.
  \end{align*}
  Thus
  \begin{align*}
    c + d \int_{[0, 1]} g(x) \; dx           & = \dfrac{A^2}{2} + \dfrac{A^2}{2}         \\
                                             & = A^2;                                    \\
    c + d \big(\sup_{x \in [0, 1]} g(x)\big) & = \dfrac{A^2}{2} + \dfrac{2 B^2 - A^2}{2} \\
                                             & = B^2.
  \end{align*}
  Note that the idea behind the definition of \(g\) is we try to build a triangle in the interval \([0, 1]\) with height equals to \(2 B^2 - A^2\) (this explains the result of supremum), and we want that triangle's area equals to \(A^2\) (this explains the result of integration).
  One can easily show that by extended \(g\) periodically with period \(1\) we know that \(g \in C(\R / \Z ; \C)\).
\end{proof}

\begin{ex}\label{ex:5.2.4}
  Prove that the \(d_{L^2}\) metric on \(C(\R / \Z ; \C)\) does indeed turn \(C(\R / \Z ; \C)\) into a metric space.
\end{ex}

\begin{proof}
  Let \(f, g, h \in C(\R / \Z ; \C)\).
  Since
  \begin{align*}
    d_{L^2}(f, f) & = \norm*{f - f}_2 &  & \by{ac:5.2.2}               \\
                  & = \norm*{0}_2     &  & \text{(by \cref{5.2.5}(b))} \\
                  & = 0,              &  & \text{(by \cref{5.2.7}(a))}
  \end{align*}
  we know that \(\big(C(\R / \Z ; \C), d_{L^2}\big)\) satisfies \cref{1.1.2}(a).
  Since
  \begin{align*}
             & f \neq g                                             \\
    \implies & f - g \neq 0        &  & \text{(by \cref{5.2.5}(b))} \\
    \implies & \norm*{f - g}_2 > 0 &  & \text{(by \cref{5.2.7}(a))} \\
    \implies & d_{L^2}(f, g) > 0,  &  & \by{ac:5.2.2}
  \end{align*}
  we know that \(\big(C(\R / \Z ; \C), d_{L^2}\big)\) satisfies \cref{1.1.2}(b).
  Since
  \begin{align*}
    d_{L^2}(f, g) & = \norm*{f - g}_2              &  & \by{ac:5.2.2}                  \\
                  & = \sqrt{\inner*{f - g, f - g}} &  & \by{ac:5.2.1}                  \\
                  & = \sqrt{\inner*{g - f, g - f}} &  & \text{(by \cref{5.2.5}(c)(d))} \\
                  & = \norm*{g - f}_2              &  & \by{ac:5.2.1}                  \\
                  & = d_{L^2}(g, f),               &  & \by{ac:5.2.2}
  \end{align*}
  we know that \(\big(C(\R / \Z ; \C), d_{L^2}\big)\) satisfies \cref{1.1.2}(c).
  Since
  \begin{align*}
    d_{L^2}(f, g) + d_{L^2}(g, h) & = \norm*{f - g}_2 + \norm*{g - h}_2 &  & \by{ac:5.2.2}               \\
                                  & \geq \norm{f - g + g - h}_2         &  & \text{(by \cref{5.2.7}(c))} \\
                                  & = \norm{f - h}_2                                                     \\
                                  & = d_{L^2}(f, h),                    &  & \by{ac:5.2.2}
  \end{align*}
  we know that \(\big(C(\R / \Z ; \C), d_{L^2}\big)\) satisfies \cref{1.1.2}(d).
  From all proofs above we conclude by \cref{1.1.2} that \(\big(C(\R / \Z ; \C), d_{L^2}\big)\) is a metric space.
\end{proof}

\begin{ex}\label{ex:5.2.5}
  Find a sequence of continuous periodic functions which converge in \(L^2\) to a discontinuous periodic function.
\end{ex}

\begin{proof}
  By \cref{5.1.4} we can define a \(\Z\)-periodic square wave function \(f : \R \to \C\) as follow:
  \[
    \forall x \in \R, f(x) = \begin{dcases}
      1 & \text{if } x \in [n, n + \dfrac{1}{2}) \text{ for some } n \in \Z     \\
      0 & \text{if } x \in [n + \dfrac{1}{2}, n + 1) \text{ for some } n \in \Z
    \end{dcases}
  \]
  Note that \(f\) is \(1\)-periodic but \(f\) is not continuous on \(\R\).
  Let \(\N_{\geq 10} = \set{n \in \N : n \geq 10}\).
  For each \(k \in \N_{\geq 10}\), we define \(f_k : [0, 1) \to \C\) to be the function:
  \[
    \forall x \in [0, 1), f_k(x) = \begin{dcases}
      kx                 & \text{if } x \in [0, \dfrac{1}{k})                           \\
      1                  & \text{if } x \in [\dfrac{1}{k}, \dfrac{1}{2} - \dfrac{1}{k}) \\
      -kx + \dfrac{k}{2} & \text{if } x \in [\dfrac{1}{2} - \dfrac{1}{k}, \dfrac{1}{2}) \\
      0                  & \text{if } x \in [0 + \dfrac{1}{2}, 1)
    \end{dcases}
  \]
  If we extended \(f_k\) periodically with period \(1\), then \(f_k \in C(\R / \Z ; \C)\) for all \(k \in \N_{\geq 10}\).
  Note that the choice of \(10\) is to make sure \(\dfrac{1}{k} < \dfrac{1}{2} - \dfrac{1}{k} < \dfrac{1}{2}\).
  Now we show that \((f_k)_{k = 10}^\infty\) converges to \(f\) on \([0, 1)\) with respect to \(d_{L^2}\).
  In particular, we want to show that
  \begin{align*}
         & \lim_{k \to \infty} d_{L^2}(f_k, f) = 0                                                                  \\
    \iff & \lim_{k \to \infty} \bigg(\int_{[0, 1]} \abs{f_k(x) - f(x)}^2 \; dx\bigg)^{1 / 2} = 0 &  & \by{ac:5.2.2} \\
    \iff & \lim_{k \to \infty} \int_{[0, 1]} \abs{f_k(x) - f(x)}^2 \; dx = 0.
  \end{align*}
  Since for each \(k \in \N_{\geq 10}\), we have
  \begin{align*}
     & \int_{[0, 1]} \abs{f_k(x) - f(x)}^2 \; dx                                                                                                                                                                                                    \\
     & = \int_{[0, \dfrac{1}{k}]} (1 - kx)^2 \; dx + \int_{[\dfrac{1}{2} - \dfrac{1}{k}, \dfrac{1}{2}]} \bigg(1 - \dfrac{k}{2} + kx\bigg)^2 \; dx                                                                                                   \\
     & = \int_{[0, \dfrac{1}{k}]} 1 - 2kx + k^2 x^2 \; dx + \int_{[\dfrac{1}{2} - \dfrac{1}{k}, \dfrac{1}{2}]} 1 - k + \dfrac{k^2}{4} + 2kx - k^2 x + k^2 x^2 \; dx                                                                                 \\
     & = \dfrac{1}{k} - 2k \bigg(\dfrac{x^2}{2}|_{x = 0}^{x = \dfrac{1}{k}}\bigg) + k^2 \bigg(\dfrac{x^3}{3}|_{x = 0}^{x = \dfrac{1}{k}}\bigg)                                                                                                      \\
     & \quad + \dfrac{1}{k} \bigg(1 - k + \dfrac{k^2}{4}\bigg) + (2k - k^2) \bigg(\dfrac{x^2}{2}|_{x = \dfrac{1}{2} - \dfrac{1}{k}}^{x = \dfrac{1}{2}}\bigg) + k^2 \bigg(\dfrac{x^3}{3}|_{x = \dfrac{1}{2} - \dfrac{1}{k}}^{x = \dfrac{1}{2}}\bigg) \\
     & = \dfrac{1}{k} - \dfrac{1}{k} + \dfrac{1}{3k} + \dfrac{1}{k} - 1 + \dfrac{k}{4} + \dfrac{2k - k^2}{2} \bigg(\dfrac{1}{k} - \dfrac{1}{k^2}\bigg) + \dfrac{k^2}{3} \bigg(\dfrac{3}{4k} - \dfrac{3}{2k^2} + \dfrac{1}{k^3}\bigg)                \\
     & = \dfrac{2}{3k},
  \end{align*}
  we know that
  \[
    \lim_{k \to \infty} \int_{[0, 1]} \abs{f_k(x) - f(x)}^2 \; dx = \lim_{k \to \infty} \dfrac{2}{3k} = 0.
  \]
  Thus \((f_k)_{k = 10}^\infty\) converges to \(f\) on \([0, 1)\) with respect to \(d_{L^2}\).
  Since \(f\) and \(f_k\) are \(1\)-periodic for all \(k \in \N_{\geq 10}\), we know that \((f_k)_{k = 10}^\infty\) converges to \(f\) on \(\R\) with respect to \(d_{L^2}\).
\end{proof}

\begin{ex}\label{ex:5.2.6}
  Let \(f \in C(\R / \Z ; \C)\), and let \((f_n)_{n = 1}^\infty\) be a sequence of functions in \(C(\R / \Z ; \C)\).
  \begin{enumerate}
    \item Show that if \(f_n\) converges uniformly to \(f\), then \(f_n\) also converges to \(f\) in the \(L^2\) metric.
    \item Give an example where \(f_n\) converges to \(f\) in the \(L^2\) metric, but does not converge to \(f\) uniformly.
    \item Give an example where \(f_n\) converges to \(f\) in the \(L^2\) metric, but does not converge to \(f\) pointwise.
    \item Give an example where \(f_n\) converges to \(f\) pointwise, but does not converge to \(f\) in the \(L^2\) metric.
  \end{enumerate}
\end{ex}

\begin{proof}{(a)}
  We have
  \begin{align*}
             & \forall \varepsilon \in \R^+, \exists N \in \Z^+ : \forall n \geq N, \forall x \in \R,                                           &  & \by{3.2.7}    \\
             & \abs{f_n(x) - f(x)} < \dfrac{\varepsilon^{\dfrac{1}{2}}}{2}                                                                                         \\
    \implies & \forall \varepsilon \in \R^+, \exists N \in \Z^+ : \forall n \geq N, \forall x \in [0, 1],                                                          \\
             & \abs{f_n(x) - f(x)} < \dfrac{\varepsilon^{\dfrac{1}{2}}}{2}                                                                                         \\
    \implies & \forall \varepsilon \in \R^+, \exists N \in \Z^+ : \forall n \geq N, \forall x \in [0, 1],                                                          \\
             & \abs{f_n(x) - f(x)}^2 < \dfrac{\varepsilon}{4}                                                                                                      \\
    \implies & \forall \varepsilon \in \R^+, \exists N \in \Z^+ : \forall n \geq N,                                                                                \\
             & \int_{[0, 1]} \abs{f_n(x) - f(x)}^2 \; dx \leq \int_{[0, 1]} \dfrac{\varepsilon}{4} \; dx = \dfrac{\varepsilon}{4} < \varepsilon                    \\
    \implies & \forall \varepsilon \in \R^+, \exists N \in \Z^+ : \forall n \geq N,                                                                                \\
             & d_{L^2}(f_n, f) < \varepsilon                                                                                                    &  & \by{ac:5.2.2} \\
    \implies & d_{L^2} - \lim_{n \to \infty} f_n = f.                                                                                           &  & \by{1.1.14}
  \end{align*}
\end{proof}

\begin{proof}{(b)}
  Let \(f \in C(\R / \Z ; \C)\) such that \(f = 0\) and let \(\N_{\geq 2} = \set{n \in \N : n \geq 2}\).
  For all \(n \in \N_{\geq 2}\), we define \(f_n \in C(\R / \Z ; \C)\) as follow:
  \[
    \forall x \in [0, 1), f_n(x) = \begin{dcases}
      0                           & \text{if } x \in [0, \dfrac{1}{2} - \dfrac{1}{n^3})            \\
      n^4 x + n - \dfrac{n^4}{2}  & \text{if } x \in [\dfrac{1}{2} - \dfrac{1}{n^3}, \dfrac{1}{2}) \\
      -n^4 x + n + \dfrac{n^4}{2} & \text{if } x \in [\dfrac{1}{2}, \dfrac{1}{2} + \dfrac{1}{n^3}) \\
      0                           & \text{if } x \in [\dfrac{1}{2} + \dfrac{1}{n^3}, 1)
    \end{dcases}
  \]
  Since for all \(n \in \N_{\geq 2}\), we have
  \begin{align*}
     & \int_{[0, 1]} \abs{f_n(x) - f(x)}^2 \; dx                                                                                                                                                                                                                                                                      \\
     & = \int_{[\dfrac{1}{2} - \dfrac{1}{n^3}, \dfrac{1}{2}]} (n^4 x + n - \dfrac{n^4}{2})^2 \; dx + \int_{[\dfrac{1}{2}, \dfrac{1}{2} + \dfrac{1}{n^3}]} (-n^4 x + n + \dfrac{n^4}{2})^2 \; dx                                                                                                                       \\
     & = \int_{[\dfrac{1}{2} - \dfrac{1}{n^3}, \dfrac{1}{2}]} n^8 x^2 + (2n^5 - n^8) x + n^2 - n^5 + \dfrac{n^8}{4} \; dx                                                                                                                                                                                             \\
     & \quad + \int_{[\dfrac{1}{2}, \dfrac{1}{2} + \dfrac{1}{n^3}]} n^8 x^2 + (-2n^5 - n^8) x + n^2 + n^5 + \dfrac{n^8}{4} \; dx                                                                                                                                                                                      \\
     & = n^8 \bigg(\dfrac{x^3}{3}|_{x = \dfrac{1}{2} - \dfrac{1}{n^3}}^{x = \dfrac{1}{2} + \dfrac{1}{n^3}}\bigg) + (2n^5 - n^8) \bigg(\dfrac{x^2}{2}|_{x = \dfrac{1}{2} - \dfrac{1}{n^3}}^{x = \dfrac{1}{2}}\bigg) + (-2n^5 - n^8) \bigg(\dfrac{x^2}{2}|_{x = \dfrac{1}{2}}^{x = \dfrac{1}{2} + \dfrac{1}{n^3}}\bigg) \\
     & \quad + \bigg(n^2 - n^5 + \dfrac{n^8}{4}\bigg) \dfrac{1}{n^3} + \bigg(n^2 + n^5 + \dfrac{n^8}{4}\bigg) \dfrac{1}{n^3}                                                                                                                                                                                          \\
     & = \dfrac{n^5}{2} + \dfrac{2}{3n} + n^2 - \dfrac{n^5}{2} - \dfrac{1}{n} + \dfrac{n^2}{2} - n^2 - \dfrac{n^5}{2} - \dfrac{1}{n} - \dfrac{n^2}{2} + \dfrac{2}{n} + \dfrac{n^5}{2}                                                                                                                                 \\
     & = \dfrac{2}{3n},
  \end{align*}
  we know that
  \begin{align*}
    \lim_{n \to \infty} d_{L^2}(f_n, f) & = \lim_{n \to \infty} \int_{[0, 1]} \abs{f_n(x) - f(x)}^2 \; dx &  & \by{ac:5.2.2} \\
                                        & = \lim_{n \to \infty} \dfrac{2}{3n}                                                \\
                                        & = 0.
  \end{align*}
  Thus by \cref{1.1.14} we have
  \[
    d_{L^2} - \lim_{n \to \infty} f_n = f.
  \]
  But for all \(n \in \N_{\geq 2}\), we have
  \begin{align*}
             & f_n(\dfrac{1}{2}) = n                                                                                                \\
    \implies & \abs{f_n(\dfrac{1}{2}) - f(\dfrac{1}{2})} = \abs{f_n(\dfrac{1}{2})} \geq n > 1                                       \\
    \implies & \exists \varepsilon \in \R^+ : \forall n \geq \N_{\geq 2}, \exists x \in [0, 1) : \abs{f_n(x) - f(x)} > \varepsilon.
  \end{align*}
  Thus by \cref{3.2.7} \((f_n)_{n = 2}^\infty\) does not converges uniformly to \(f\) on \(\R\) with respect to \(d_{l^1}|_{\R \times \R}\).
\end{proof}

\begin{proof}{(c)}
  Using the definition of \(f, f_n\) in \cref{ex:5.2.6}(b), we see that \((f_n)_{n = 2}^\infty\) does not converges pointwise to \(f\) on \(\R\) with respect to \(d_{l^1}|_{\R \times \R}\).
\end{proof}

\begin{proof}{(d)}
  Let \(f \in C(\R / \Z ; \C)\) such that \(f = 0\) and let \(\N_{\geq 2} = \set{n \in \N : n \geq 2}\).
  For all \(n \in \N_{\geq 2}\), we define \(f_n \in C(\R / \Z ; \C)\) as follow:
  \[
    \forall x \in [0, 1), f_n(x) = \begin{dcases}
      2 n^2 x       & \text{if } x \in [0, \dfrac{1}{2n})            \\
      -2 n^2 x + 2n & \text{if } x \in [\dfrac{1}{2n}, \dfrac{1}{n}) \\
      0             & \text{if } x \in [\dfrac{1}{n}, 1)
    \end{dcases}
  \]
  Since
  \begin{align*}
             & \lim_{n \to \infty} \dfrac{1}{n} = 0                                                                 \\
    \implies & \forall \varepsilon \in \R^+, \exists N \in \Z^+ : \forall n \geq N, \dfrac{1}{n} < x                \\
    \implies & \forall x \in (0, \dfrac{1}{2}), \exists N \in \Z^+ : \forall n \geq N, \dfrac{1}{n} < x             \\
    \implies & \forall x \in (0, \dfrac{1}{2}), \exists N \in \Z^+ : \forall n \geq N, f(\dfrac{1}{n}) = f_n(x) = 0 \\
    \implies & \forall x \in (0, \dfrac{1}{2}), \lim_{n \to \infty} f_n(x) = 0 = f(x)
  \end{align*}
  and
  \begin{align*}
     & \lim_{n \to \infty} f_n(0) = 0 = f(0)                                  \\
     & \forall x \in [\dfrac{1}{2}, 1), \lim_{n \to \infty} f_n(x) = 0 = f(x)
  \end{align*}
  by \cref{3.2.1} we know that \((f_n)_{n = 2}^\infty\) converges pointwise to \(f\) on \(\R\) with respect to \(d_{l^1}|_{\R \times \R}\).
  But
  \begin{align*}
     & \int_{[0, 1]} \abs{f_n(x) - f(x)}^2 \; dx                                                                                                                    \\
     & = \int_{[0, \dfrac{1}{2n}]} (2 n^2 x)^2 \; dx + \int_{[\dfrac{1}{2n}, \dfrac{1}{n}]} (-2n^2 x + 2n)^2 \; dx                                                  \\
     & = \int_{[0, \dfrac{1}{2n}]} 4 n^4 x^2 \; dx + \int_{[\dfrac{1}{2n}, \dfrac{1}{n}]} 4 n^4 x^2 - 4 n^3 x + 4n^2 \; dx                                          \\
     & = 4n^4 \bigg(\dfrac{x^3}{3}|_{x = 0}^{x = \dfrac{1}{n}}\bigg) - 4n^3 \bigg(\dfrac{x^2}{2}|_{x = \dfrac{1}{2n}}^{x = \dfrac{1}{n}}\bigg) + 4n^2 \dfrac{1}{2n} \\
     & = \dfrac{4n}{3} - \dfrac{3n}{2} + 2n                                                                                                                         \\
     & = \dfrac{11n}{6}
  \end{align*}
  implies
  \begin{align*}
    \lim_{n \to \infty} d_{L^2}(f_n, f) & = \lim_{n \to \infty} \int_{[0, 1]} \abs{f_n(x) - f(x)}^2 \; dx &  & \by{ac:5.2.2} \\
                                        & = \lim_{n \to \infty} \dfrac{11n}{6}                                               \\
                                        & = +\infty.
  \end{align*}
  Thus \((f_n)_{n = 2}^\infty\) does not converges to \(f\) with respect to \(d_{L^2}\).
\end{proof}