\section{The inverse function theorem in several variable calculus}\label{ii:sec:6.7}

\begin{note}
  We recall the inverse function theorem in single variable calculus (Theorem 10.4.2 in Analysis I), which asserts that if a function \(f : \R \to \R\) is invertible, differentiable, and \(f'(x_0)\) is non-zero, then \(f^{-1}\) is differentiable at \(f(x_0)\), and
  \[
    (f^{-1})'\big(f(x_0)\big) = \dfrac{1}{f'(x_0)}.
  \]

  In fact, one can say something even when \(f'\) is not invertible, as long as we know that \(f\) is \emph{continuously} differentiable.
  If \(f'(x_0)\) is non-zero, then \(f'(x_0)\) must be either strictly positive or strictly negative, which implies (since we are assuming \(f'\) to be continuous) that \(f'(x)\) is either strictly positive for \(x\) near \(x_0\), or strictly negative for \(x\) near \(x_0\).
  In particular, \(f\) must be either strictly increasing near \(x_0\), or strictly decreasing near \(x_0\).
  In either case, \(f\) will become invertible if we restrict the domain and codomain of \(f\) to be sufficiently close to \(x_0\) and to \(f(x_0)\) respectively.
  (The technical terminology for this is that \(f\) is \emph{locally invertible near \(x_0\)}.)
\end{note}

\begin{lem}\label{ii:6.7.1}
  Let \(T : \R^n \to \R^n\) be a linear transformation which is also invertible.
  Then the inverse transformation \(T^{-1} : \R^n \to \R^n\) is also linear.
\end{lem}

\begin{proof}
  Let \(x, y \in \R^n\) and let \(c \in \R\).
  We have
  \begin{align*}
    T^{-1}(x + y) & = T^{-1}\Big(T\big(T^{-1}(x)\big) + T\big(T^{-1}(y)\big)\Big)                    \\
                  & = T^{-1}\Big(T\big(T^{-1}(x) + T^{-1}(y)\big)\Big)            &  & \by{ii:6.1.6} \\
                  & = T^{-1}(x) + T^{-1}(y)
  \end{align*}
  and
  \begin{align*}
    T^{-1}(cx) & = T^{-1}\Big(c T\big(T^{-1}(x)\big)\Big)                    \\
               & = T^{-1}\Big(T\big(c T^{-1}(x)\big)\Big) &  & \by{ii:6.1.6} \\
               & = c T^{-1}(x).
  \end{align*}
  Thus, by \cref{ii:6.1.6} \(T^{-1}\) is a linear transformation.
\end{proof}

\begin{thm}[Inverse function theorem]\label{ii:6.7.2}
  Let \(E\) be an open subset of \(\R^n\), and let \(f : E \to \R^n\) be a function which is continuously differentiable on \(E\).
  Suppose \(x_0 \in E\) is such that the linear transformation \(f'(x_0) : \R^n \to \R^n\) is invertible.
  Then there exists an open set \(U\) in \(E\) containing \(x_0\), and an open set \(V\) in \(\R^n\) containing \(f(x_0)\), such that \(f\) is a bijection from \(U\) to \(V\).
  In particular, there is an inverse map \(f^{-1} : V \to U\).
  Furthermore, this inverse map is differentiable at \(f(x_0)\), and
  \[
    (f^{-1})' \big(f(x_0)\big) = \big(f'(x_0)\big)^{-1}.
  \]
\end{thm}

\begin{proof}
  We first observe that once we know the inverse map \(f^{-1}\) is differentiable, the formula \((f^{-1})' \big(f(x_0)\big) = \big(f'(x_0)\big)^{-1}\) is automatic.
  This comes from starting with the identity
  \[
    I = f^{-1} \circ f
  \]
  on \(U\), where \(I : \R^n \to \R^n\) is the identity map \(I(x) \coloneqq x\), and then differentiating both sides using the chain rule at \(x_0\) to obtain
  \[
    I'(x_0) = (f^{-1})' \big(f(x_0)\big) \circ f'(x_0).
  \]
  Since \(I'(x_0) = I\), we thus have \((f^{-1})' \big(f(x_0)\big) = \big(f'(x_0)\big)^{-1}\) as desired.

  We remark that this argument shows that if \(f'(x_0)\) is \emph{not} invertible, then there is no way that an inverse \(f^{-1}\) can exist and be differentiable at \(f(x_0)\).

  Next, we observe that it suffices to prove the theorem under the additional assumption \(f(x_0) = 0\).
  The general case then follows from the special case by replacing \(f\) by a new function \(\tilde{f}(x) \coloneqq f(x) - f(x_0)\) and then applying the special case to \(\tilde{f}\)
  (note that \(V\) will have to shift by \(f(x_0)\)).
  Note that if \(V_f = \set{y \in \R^n : y - f(x_0) \in V}\), then
  \[
    \begin{dcases}
      \tilde{f} : U \to V \\
      \tilde{f}^{-1} : V \to U
    \end{dcases} \implies \begin{dcases}
      f : U \to V_f \\
      f^{-1} : V_f \to U
    \end{dcases}
  \]
  (one can show that \(f\) is bijective using proof by contradiction)
  and thus
  \begin{align*}
             & \forall x \in U, f(x) = y                                                                                                          \\
    \implies & f^{-1}(y) = x = \tilde{f}^{-1}\big(\tilde{f}(x)\big) = \tilde{f}^{-1}\big(f(x) - f(x_0)\big) = \tilde{f}^{-1}\big(y - f(x_0)\big).
  \end{align*}
  Henceforth we will always assume \(f(x_0) = 0\).

  In a similar manner, one can make the assumption \(x_0 = 0\).
  The general case then follows from this case by replacing \(f\) by a new function
  \(\tilde{f}(x) \coloneqq f(x + x_0)\) and applying the special case to \(\tilde{f}\)
  (note that \(E\) and \(U\) will have to shift by \(x_0\)).
  Note that if \(U_f = \set{x \in E : x - x_0 \in U}\), then
  \[
    \begin{dcases}
      \tilde{f} : U \to V \\
      \tilde{f}^{-1} : V \to U
    \end{dcases} \implies \begin{dcases}
      f : U_f \to V \\
      f^{-1} : V \to U_f
    \end{dcases}
  \]
  (one can show that \(f\) is bijective using proof by contradiction)
  and thus
  \begin{align*}
             & \forall x \in U, \tilde{f}(x) = f(x + x_0) = y                                                                                         \\
    \implies & f^{-1}(y) = x + x_0 = \tilde{f}^{-1}\big(\tilde{f}(x)\big) + x_0 = \tilde{f}^{-1}\big(f(x + x_0)\big) + x_0 = \tilde{f}^{-1}(y) + x_0.
  \end{align*}
  Henceforth we will always assume \(x_0 = 0\).
  Thus, we now have that \(f(0) = 0\) and that \(f'(0)\) is invertible.

  Finally, one can assume that \(f'(0) = I\), where \(I : \R^n \to \R^n\) is the identity transformation \(I(x) = x\).
  The general case then follows from this case by replacing \(f\) with a new function \(\tilde{f} : E \to \R^n\) defined by \(\tilde{f}(x) \coloneqq \big(f'(0)\big)^{-1} \big(f(x)\big)\), and applying the special case to this case.
  Note from \cref{ii:6.7.1} that \(\big(f'(0)\big)^{-1}\) is a linear transformation.
  In particular, we note that \(\tilde{f}(0) = 0\) and that
  \begin{align*}
    \tilde{f}'(0) & = \Big(\big(f'(0)\big)^{-1}\Big)'\big(f(0)\big) \circ f'(0) &  & \by{ii:6.4.1}    \\
                  & = \big(f'(0)\big)^{-1} \circ f'(0)                          &  & \by{ii:ex:6.4.1} \\
                  & = I,
  \end{align*}
  so by the special case of the inverse function theorem we know that there exists an open set \(U'\) containing \(0\), and an open set \(V'\) containing \(0\), such that \(\tilde{f}\) is a bijection from \(U'\) to \(V'\), and that \(\tilde{f}^{-1} : V' \to U'\) is differentiable at \(0\) with derivative \(I\).
  But we have
  \begin{align*}
             & \tilde{f}(x) = \big(f'(0)\big)^{-1} \big(f(x)\big)                                 \\
    \implies & f'(0) \big(\tilde{f}(x)\big) = f'(0) \Big(\big(f'(0)\big)^{-1} \big(f(x)\big)\Big) \\
    \implies & f(x) = f'(0) \big(\tilde{f}(x)\big),
  \end{align*}
  and hence \(f\) is a bijection from \(U'\) to \(f'(0)(V')\)
  (note that \(f'(0)\) is also a bijection).
  Since \(f'(0)\) and its inverse are both continuous, \(f'(0)(V')\) is open (see \cref{ii:2.1.5}(a)(c)), and it certainly contains \(0\).
  Now consider the inverse function \(f^{-1} : f'(0)(V') \to U'\).
  Note that
  \begin{align*}
             & f = f'(0) \circ \tilde{f}                                                             \\
    \implies & f^{-1} = \tilde{f}^{-1} \circ \big(f'(0)\big)^{-1}                                    \\
    \implies & \forall y \in f'(0)(V'), f^{-1}(y) = \tilde{f}^{-1}\Big(\big(f'(0)\big)^{-1}(y)\Big).
  \end{align*}
  In particular, we see that \(f^{-1}\) is differentiable at \(0\).

  So all we have to do now is prove the inverse function theorem in the special case, when \(x_0 = 0\), \(f(x_0) = 0\), and \(f'(x_0) = I\).
  Let \(g : E \to \R^n\) denote the function \(g(x) = f(x) - x\).
  Then \(g(0) = 0\) and \(g'(0) = 0\).
  In particular
  \[
    \dfrac{\partial g}{\partial x_j}(0) = 0
  \]
  for \(j = 1, \dots, n\).
  Since \(g\) is continuously differentiable, there thus exists a ball \(B(0, r)\) in \(E\) such that
  \[
    \norm*{\dfrac{\partial g}{\partial x_j}(x)} \leq \dfrac{1}{2 n^2}
  \]
  for all \(x \in B(0, r)\).
  (There is nothing particularly special about \(\dfrac{1}{2 n^2}\), we just need a nice small number here.)
  In particular, for any \(x \in B(0, r)\) and \(v = (v_1, \dots, v_n)\) we have
  \begin{align*}
    \norm*{D_v g(x)} & = \norm*{\sum_{j = 1}^n v_j \dfrac{\partial g}{\partial x_j} (x)}         \\
                     & \leq \sum_{j = 1}^n \abs{v_j} \norm*{\dfrac{\partial g}{\partial x_j}(x)} \\
                     & \leq \sum_{j = 1}^n \norm*{v} \dfrac{1}{2 n^2}                            \\
                     & \leq \dfrac{1}{2n} \norm*{v}.
  \end{align*}
  But now for any \(x, y \in B(0, r)\), we have by the fundamental theorem of calculus
  \begin{align*}
    g(y) - g(x) & = g\big(x + t(y - x)\big) \big|_{t = 0}^{t = 1}        \\
                & = \int_0^1 \dfrac{d}{dt} g\big(x + t(y - x)\big) \; dt \\
                & = \int_0^1 D_{y - x} g\big(x + t(y - x)\big) \; dt
  \end{align*}
  where the integral of a vector-valued function is defined by integrating each component separately.
  By the previous remark, the vectors \(D_{y - x} g\big(x + t(y - x)\big)\) have a magnitude of at most \(\dfrac{1}{2n} \norm*{y - x}\).
  Thus, every component of these vectors has magnitude at most \(\dfrac{1}{2n} \norm*{y - x}\).
  Thus, every component of \(g(y) - g(x)\) has magnitude at most \(\dfrac{1}{2n} \norm*{y - x}\), and hence \(g(y) - g(x)\) itself has magnitude at most \(\dfrac{1}{2} \norm*{y - x}\)
  (actually, it will be substantially less than this, but this bound will be enough for our purposes).
  In other words, \(g\) is a contraction.
  By \cref{ii:6.6.6}, the map \(f = g + I\) is thus one-to-one on \(B(0, r)\), and the image \(f\big(B(0, r)\big)\) contains \(B(0, \dfrac{r}{2})\).
  In particular, we have an inverse map \(f^{-1} : B(0, \dfrac{r}{2}) \to B(0, r)\) defined on \(B(0, \dfrac{r}{2})\).

  Applying the contraction bound with \(y = 0\) we obtain, in particular, that
  \[
    \norm*{g(x)} \leq \dfrac{1}{2} \norm*{x}
  \]
  for all \(x \in B(0, r)\), and so by the triangle inequality
  \[
    \dfrac{1}{2} \norm*{x} \leq \norm*{f(x)} \leq \dfrac{3}{2} \norm*{x}
  \]
  for all \(x \in B(0, r)\).

  Now we set \(V \coloneqq B(0, \dfrac{r}{2})\) and \(U \coloneqq f^{-1}(V) \cap B(0, r)\).
  Then by construction \(f\) is a bijection from \(U\) to \(V\).
  \(V\) is clearly open, and \(U\) is also open since \(f\) is continuous.
  (Notice that if a set is open relative to \(B(0, r)\), then it is open in \(\R^n\) as well.)
  Now we want to show that \(f^{-1} : V \to U\) is differentiable at \(0\) with derivative \(I^{-1} = I\).
  In other words, we wish to show that
  \[
    \lim_{x \to 0 ; x \in V \setminus \set{0}} \dfrac{\norm*{f^{-1}(x) - f^{-1}(0) - I(x - 0)}}{\norm*{x}} = 0.
  \]
  Since \(f(0) = 0\), we have \(f^{-1}(0) = 0\), and the above simplifies to
  \[
    \lim_{x \to 0 ; x \in V \setminus \set{0}} \dfrac{\norm*{f^{-1}(x) - x}}{\norm*{x}} = 0.
  \]
  Let \((x_n)_{n = 1}^\infty\) be any sequence in \(V \setminus \set{0}\) that converges to \(0\).
  By \cref{ii:3.1.5}(b), it suffices to show that
  \[
    \lim_{n \to \infty} \dfrac{\norm*{f^{-1}(x_n) - x_n}}{\norm*{x_n}} = 0.
  \]
  Write \(y_n \coloneqq f^{-1}(x_n)\).
  Then \(y_n \in B(0, r)\) and \(x_n = f(y_n)\).
  In particular, we have
  \[
    \dfrac{1}{2} \norm*{y_n} \leq \norm*{x_n} \leq \dfrac{3}{2} \norm*{y_n}
  \]
  and so since \(\norm*{x_n}\) goes to \(0\), \(\norm*{y_n}\) goes to \(0\) also, and their ratio remains bounded.
  It will thus suffice to show that
  \[
    \lim_{n \to \infty} \dfrac{\norm*{y_n - f(y_n)}}{\norm*{y_n}} = 0.
  \]
  But since \(y_n\) is going to \(0\), and \(f\) is differentiable at \(0\), we have
  \[
    \lim_{n \to \infty} \dfrac{\norm*{f(y_n) - f(0) - f'(0)(y_n - 0)}}{\norm*{y_n}} = 0
  \]
  as desired (since \(f(0) = 0\) and \(f'(0) = I\)).
\end{proof}

\begin{note}
  The inverse function theorem gives a useful criterion for when a function is (locally) invertible at a point \(x_0\)
  - all we need is for its derivative \(f'(x_0)\) to be invertible
  (and then we even get further information, for instance we can compute the derivative of \(f^{-1}\) at \(f(x_0)\)).
  Of course, this begs the question of how one can tell whether the linear transformation \(f'(x_0)\) is invertible or not.
  Recall that we have \(f'(x_0) = L_{D f(x_0)}\), so by \cref{ii:6.1.13,ii:6.1.16} we see that the linear transformation \(f'(x_0)\) is invertible iff the matrix \(D f(x_0)\) is.
  There are many ways to check whether a matrix such as \(D f(x_0)\) is invertible;
  for instance, one can use determinants, or alternatively Gaussian elimination methods.
  We will not pursue this matter here, but refer the reader to any linear algebra text.
\end{note}

\begin{note}
  If \(f'(x_0)\) exists but is non-invertible, then the inverse function theorem does not apply.
  In such a situation it is not possible for \(f^{-1}\) to exist and be differentiable at \(f(x_0)\);
  this was remarked in the proof of \cref{ii:6.7.2}.
  But it is still possible for \(f\) to be invertible.
  For instance, the single-variable function \(f : \R \to \R\) defined by \(f(x) = x^3\) is invertible despite \(f'(0)\) not being invertible.
\end{note}

\exercisesection

\begin{ex}\label{ii:ex:6.7.1}
  Let \(f : \R \to \R\) be the function defined by \(f(x) \coloneqq x + x^2 \sin(1 / x^4)\) for \(x \neq 0\) and \(f(0) \coloneqq 0\).
  Show that \(f\) is differentiable and \(f'(0) = 1\), but \(f\) is not increasing on any open set containing \(0\).
\end{ex}

\begin{proof}
  Let \(x \in \R \setminus \set{0}\).
  Then we have
  \begin{align*}
             & (x \mapsto x^{-4})' = (-4) x^{-5}                                               \\
    \implies & \big(x \mapsto \sin(x^{-4})\big)' = (-4) x^{-5} \cos(x^{-4})                    \\
    \implies & \big(x \mapsto x^2 \sin(x^{-4})\big)' = 2x \sin(x^{-4}) - 4 x^{-3} \cos(x^{-4}) \\
    \implies & f'(x) = 1 + 2 x \sin(x^{-4}) - 4 x^{-3} \cos(x^{-4}).
  \end{align*}
  Observe that
  \[
    \forall x \in \R \setminus \set{0}, \abs{x \sin(x^{-4})} = \abs{x} \abs{\sin(x^{-4})} \leq \abs{x} \cdot 1.
  \]
  Thus, we have
  \[
    \lim_{x \to 0 ; x \in \R \setminus \set{0}} x = 0 \implies \lim_{x \to 0 ; x \in \R \setminus \set{0}} x \sin(x^{-4}) = 0
  \]
  and by squeeze test
  \begin{align*}
     & \lim_{x \to 0 ; x \in \R \setminus \set{0}} \dfrac{f(x) - f(0)}{x - 0}        \\
     & = \lim_{x \to 0 ; x \in \R \setminus \set{0}} \dfrac{x + x^2 \sin(x^{-4})}{x} \\
     & = \lim_{x \to 0 ; x \in \R \setminus \set{0}} 1 + x \sin(x^{-4})              \\
     & = 1.
  \end{align*}
  We conclude that \(f\) is differentiable on \(\R\) and \(f'(0) = 1\).

  Let \(E\) be an open set in \(\R\) containing \(0\).
  By \cref{ii:1.2.15}(a) we know that
  \[
    \exists r \in \R^+ : B(0, r) \subseteq E \implies (-r, r) \subseteq E.
  \]
  Fix such \(r\).
  Since
  \begin{align*}
             & \lim_{n \to \infty} \dfrac{1}{n} = 0                                     \\
    \implies & \lim_{n \to \infty} \dfrac{1}{2 n \pi} = 0                               \\
    \implies & \lim_{n \to \infty} \sqrt[4]{\dfrac{1}{2 n \pi}} = 0                     \\
    \implies & \exists N \in \Z^+ : \forall n \geq N, \sqrt[4]{\dfrac{1}{2 n \pi}} < r,
  \end{align*}
  by fixing such \(N\) we know that
  \begin{align*}
             & \sqrt[4]{\dfrac{1}{2 N \pi}} \in (-r, r) \subseteq E                                                                                     \\
    \implies & f'\bigg(\sqrt[4]{\dfrac{1}{2 N \pi}}\bigg) = 1 + 2 \sqrt[4]{\dfrac{1}{2 N \pi}} \sin(2 N \pi) - 4 (2 N \pi)^{\dfrac{3}{4}} \cos(2 N \pi) \\
             & = 1 - 4 (2 N \pi)^{\dfrac{3}{4}} \leq 1 - 4 = -3 < 0.
  \end{align*}
  Thus, \(f\) is not increasing at \(\sqrt[4]{\dfrac{1}{2 N \pi}}\), and not increasing on \(E\).
\end{proof}

\begin{ex}\label{ii:ex:6.7.2}
  Prove \cref{ii:6.7.1}.
\end{ex}

\begin{proof}
  See \cref{ii:6.7.1}.
\end{proof}

\begin{ex}\label{ii:ex:6.7.3}
  Let \(f : \R^n \to \R^n\) be a continuously differentiable function such that \(f'(x)\) is an invertible linear transformation for every \(x \in \R^n\).
  Show that whenever \(V\) is an open set in \(\R^n\), that \(f(V)\) is also open.
\end{ex}

\begin{proof}
  Let \(d = d_{l^2}|_{\R^n \times \R^n}\), let \(V\) be an open set in \((\R^n, d)\) and let \(y \in f(V)\).
  We know that there exists a \(x \in V\) such that \(f(x) = y\).
  Since \(f\) is continuously differentiable on \(\R^n\) and \(f'(x)\) is invertible, by inverse function theorem (\cref{ii:6.7.2}) we know that
  \[
    \exists U, W \subseteq \R^n : \begin{dcases}
      U, W \text{ are open sets in } (\R^n, d) \\
      x \in U                                  \\
      y \in W                                  \\
      f : U \to W \text{ is a bijection}       \\
      (f^{-1})'(y) = (f^{-1})'\big(f(x)\big) = \big(f'(x)\big)^{-1}
    \end{dcases}.
  \]
  Fix such \(U, W\).
  Since \(V\) is open in \((\R^n, d)\), we know that there exists a \(r \in \R^+\) such that \(B_{(\R^n, d)}(x, r) \subseteq V\).
  Fix such \(r\).
  By \cref{ii:1.2.15}(f) we know that \(U \cap B_{(\R^n, d)}(x, r)\) is open in \((\R^n, d)\).
  Since \(U \cap B_{(\R^n, d)}(x, r) \subseteq U\), we know that \(f\) is a bijection from \(U \cap B_{(\R^n, d)}(x, r)\) to \(f\big(U \cap B_{(\R^n, d)}(x, r)\big)\).
  By hypotheses we know that \(f^{-1}\) is differentiable on \(f\big(U \cap B_{(\R^n, d)}(x, r)\big)\), by \cref{ii:ex:6.4.2} we know that \(f^{-1}\) is continuous \(f\big(U \cap B_{(\R^n, d)}(x, r)\big)\).
  Thus, by \cref{ii:2.1.5}(a)(c) we know that \(f\big(U \cap B_{(\R^n, d)}(x, r)\big)\) is open in \((\R, d)\).
  Now we have
  \begin{align*}
             & x \in U \cap B_{(\R^n, d)}(x, r)\big)                                                                                                                     \\
    \implies & y \in f\big(U \cap B_{(\R^n, d)}(x, r)\big)                                                                                                               \\
    \implies & \exists r' \in \R^+ : B_{(\R^n, d)}(y, r') \subseteq f\big(U \cap B_{(\R^n, d)}(x, r)\big) &                                          & \by{ii:1.2.15}[a] \\
    \implies & \exists r' \in \R^+ : B_{(\R^n, d)}(y, r') \subseteq f(V).                                 & (U \cap B_{(\R^n, d)}(x, r) \subseteq V)
  \end{align*}
  Since \(y\) was arbitrary, we know that \(f(V)\) is open in \((\R^n, d)\).
  Since \(V\) was arbitrary, we know that if \(V\) is an open set in \((\R^n, d)\), then \(f(V)\) is also an open set in \((\R^n, d)\).
\end{proof}

\begin{ex}\label{ii:ex:6.7.4}
  Let the notation and hypotheses be as in \cref{ii:6.7.2}.
  Show that, after shrinking the open sets \(U, V\) if necessary (while still having \(x_0 \in U\), \(f(x_0) \in V\) of course), the derivative map \(f'(x)\) is invertible for all \(x \in U\), and that the inverse map \(f^{-1}\) is differentiable at every point of \(V\) with \((f^{-1})' \big(f(x)\big) = \big(f'(x)\big)^{-1}\) for all \(x \in U\).
  Finally, show that \(f^{-1}\) is continuously differentiable on \(V\).
\end{ex}
