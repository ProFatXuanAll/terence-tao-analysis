\section{Double derivatives and Clairaut's theorem}\label{sec:6.5}

\begin{defn}[Twice continuous differentiability]\label{6.5.1}
  Let \(E\) be an open subset of \(\R^n\), and let \(f : E \to \R^m\) be a function.
  We say that \(f\) is \emph{continuously differentiable} if the partial derivatives \(\dfrac{\partial f}{\partial x_1}, \dots, \dfrac{\partial f}{\partial x_n}\) exist and are continuous on \(E\).
  We say that \(f\) is \emph{twice continuously differentiable} if it is continuously differentiable, and the partial derivatives \(\dfrac{\partial f}{\partial x_1}, \dots, \dfrac{\partial f}{\partial x_n}\) are themselves continuously differentiable.
\end{defn}

\begin{rmk}\label{6.5.2}
  Continuously differentiable functions are sometimes called \(C^1\) functions;
  twice continuously differentiable functions are sometimes called \(C^2\) functions.
  One can also define \(C^3\), \(C^4\), etc. but we shall not do so here.
\end{rmk}

\setcounter{thm}{3}
\begin{thm}[Clairaut's theorem]\label{6.5.4}
  Let \(E\) be an open subset of \(\R^n\), and let \(f : E \to \R^m\) be a twice continuously differentiable function on \(E\).
  Then we have \(\dfrac{\partial}{\partial x_j} \dfrac{\partial f}{\partial x_i}(x_0) = \dfrac{\partial}{\partial x_i} \dfrac{\partial f}{\partial x_j}(x_0)\) for all \(1 \leq i, j \leq n\).
\end{thm}

\begin{proof}
  By working with one component of \(f\) at a time we can assume that \(m = 1\).
  The claim is trivial if \(i = j\), so we shall assume that \(i \neq j\).
  We shall prove the theorem for \(x_0 = 0\);
  the general case is similar.
  (Actually, once one proves Clairaut's theorem for \(x_0 = 0\), one can immediately obtain it for general \(x_0\) by applying the theorem with \(f(x)\) replaced by \(f(x + x_0)\).)

  Let \(a\) be the number \(a \coloneqq \dfrac{\partial}{\partial x_j} \dfrac{\partial f}{\partial x_i}(0)\), and \(a'\) denote the quantity \(a' \coloneqq \dfrac{\partial}{\partial x_i} \dfrac{\partial f}{\partial x_j}(0)\).
  Our task is to show that \(a' = a\).

  Let \(\varepsilon > 0\).
  Because the double derivatives of \(f\) are continuous, we can find a \(\delta > 0\) such that
  \[
    \abs{\dfrac{\partial}{\partial x_j} \dfrac{\partial f}{\partial x_i}(x) - a} \leq \varepsilon
  \]
  and
  \[
    \abs{\dfrac{\partial}{\partial x_i} \dfrac{\partial f}{\partial x_j}(x) - a'} \leq \varepsilon
  \]
  whenever \(\norm*{x} \leq 2\delta\).

  Now we consider the quantity
  \[
    X \coloneqq f(\delta e_i + \delta e_j) - f(\delta e_i) - f(\delta e_j) + f(0).
  \]
  From the fundamental theorem of calculus in the \(e_i\) variable, we have
  \[
    f(\delta e_i + \delta e_j) - f(\delta e_j) = \int_0^{\delta} \dfrac{\partial f}{\partial x_i}(x_i e_i + \delta e_j) \; d x_i
  \]
  and
  \[
    f(\delta e_i) - f(0) = \int_0^{\delta} \dfrac{\partial f}{\partial x_i}(x_i e_i) \; d x_i
  \]
  and hence
  \[
    X = \int_0^{\delta} \bigg(\dfrac{\partial f}{\partial x_i} (x_i e_i + \delta e_j) - \dfrac{\partial f}{\partial x_i} (x_i e_i)\bigg) \; d x_i.
  \]
  But by the mean value theorem, for each \(x_i\) we have
  \[
    \dfrac{\partial f}{\partial x_i} (x_i e_i + \delta e_j) - \dfrac{\partial f}{\partial x_i} (x_i e_i) = \delta \dfrac{\partial}{\partial x_j} \dfrac{\partial f}{\partial x_i} (x_i e_i + x_j e_j)
  \]
  for some \(0 \leq x_j \leq \delta\).
  By our construction of \(\delta\), we thus have
  \[
    \abs{\dfrac{\partial f}{\partial x_i} (x_i e_i + \delta e_j) - \dfrac{\partial f}{\partial x_i} (x_i e_i) - \delta a} \leq \varepsilon \delta.
  \]
  Integrating this from \(0\) to \(\delta\), we thus obtain
  \[
    \abs{X - \delta^2 a} \leq \varepsilon \delta^2.
  \]

  We can run the same argument with the rôle of \(i\) and \(j\) reversed (note that \(X\) is symmetric in \(i\) and \(j\)), to obtain
  \[
    \abs{X - \delta^2 a'} \leq \varepsilon \delta^2.
  \]
  From the triangle inequality we thus obtain
  \[
    \abs{\delta^2 a - \delta^2 a'} \leq 2 \varepsilon \delta^2,
  \]
  and thus
  \[
    \abs{a - a'} \leq 2 \varepsilon.
  \]
  But this is true for all \(\varepsilon > 0\), and \(a\) and \(a'\) do not depend on \(\varepsilon\), and so we must have \(a = a'\), as desired.
\end{proof}

\exercisesection

\begin{ex}\label{ex:6.5.1}
  Let \(f : \R^2 \to \R\) be the function defined by \(f(x, y) \coloneqq \dfrac{x y^3}{x^2 + y^2}\) when \((x, y) \neq (0, 0)\), and \(f(0, 0) \coloneqq 0\).
  Show that \(f\) is continuously differentiable, and the double derivatives \(\dfrac{\partial}{\partial y} \dfrac{\partial f}{\partial x}\) and \(\dfrac{\partial}{\partial x} \dfrac{\partial f}{\partial y}\) exist, but are not equal to each other at \((0, 0)\).
  Explain why this does not contradict Clairaut's theorem.
\end{ex}

\begin{proof}
  Let \((a, b) \in \R^2 \setminus \set{(0, 0)}\).
  We have
  \begin{align*}
     & \dfrac{\partial f}{\partial x}(a, b)                                                                                                                   \\
     & = \lim_{t \to 0 ; t \neq 0} \dfrac{f\big((a, b) + t(1, 0)\big) - f(a, b)}{t}                                                           &  & \by{6.3.7} \\
     & = \lim_{t \to 0 ; t \neq 0} \dfrac{f(a + t, b) - f(a, b)}{t}                                                                                           \\
     & = \lim_{t \to 0 ; t \neq 0} \dfrac{\dfrac{(a + t) b^3}{(a + t)^2 + b^2} - \dfrac{ab^3}{a^2 + b^2}}{t}                                                  \\
     & = \lim_{t \to 0 ; t \neq 0} \dfrac{b^3 (a + t) (a^2 + b^2) - a b^3 \big((a + t)^2 + b^2\big)}{t \big((a + t)^2 + b^2\big) (a^2 + b^2)}                 \\
     & = \lim_{t \to 0 ; t \neq 0} \dfrac{t b^5 - t a^2 b^3 - t^2 a b^3}{t \big((a + t)^2 + b^2\big) (a^2 + b^2)}                                             \\
     & = \lim_{t \to 0 ; t \neq 0} \dfrac{b^5 - a^2 b^3 - t a b^3}{\big((a + t)^2 + b^2\big) (a^2 + b^2)}                                                     \\
     & = \dfrac{b^5 - a^2 b^3}{(a^2 + b^2)^2}
  \end{align*}
  and
  \begin{align*}
     & \dfrac{\partial f}{\partial y}(a, b)                                                                                                                     \\
     & = \lim_{t \to 0 ; t \neq 0} \dfrac{f\big((a, b) + t(0, 1)\big) - f(a, b)}{t}                                                             &  & \by{6.3.7} \\
     & = \lim_{t \to 0 ; t \neq 0} \dfrac{f(a, b + t) - f(a, b)}{t}                                                                                             \\
     & = \lim_{t \to 0 ; t \neq 0} \dfrac{\dfrac{a (b + t)^3}{a^2 + (b + t)^2} - \dfrac{ab^3}{a^2 + b^2}}{t}                                                    \\
     & = \lim_{t \to 0 ; t \neq 0} \dfrac{a (b + t)^3 (a^2 + b^2) - a b^3 \big(a^2 + (b + t)^2\big)}{t \big(a^2 + (b + t)^2\big) (a^2 + b^2)}                   \\
     & = \lim_{t \to 0 ; t \neq 0} \dfrac{(3 b^2 t + 3 b t^2 + t^3) (a^3 + a b^2) - a b^3 (2bt + t^2)}{t \big(a^2 + (b + t)^2\big) (a^2 + b^2)}                 \\
     & = \lim_{t \to 0 ; t \neq 0} \dfrac{(3 b^2 + 3 b t + t^2) (a^3 + a b^2) - a b^3 (2b + t)}{\big(a^2 + (b + t)^2\big) (a^2 + b^2)}                          \\
     & = \dfrac{3 a^3 b^2 + a b^4}{(a^2 + b^2)^2}.
  \end{align*}
  Since \((a, b)\) is arbitrary, by \cref{2.2.2} we know that \(\dfrac{\partial f}{\partial x}\) and \(\dfrac{\partial f}{\partial y}\) are continuous on \(\R^2 \setminus \set{(0, 0)}\) from \((\R^2, d_{l^2}|_{\R^2 \times \R^2})\) to \((\R, d_{l^1}|_{\R \times \R})\).
  Observe that
  \begin{align*}
    \abs{\dfrac{b^5 - a^2 b^3}{(a^2 + b^2)^2}} & = \dfrac{\abs{b (b^4 - a^2 b^2)}}{(a^2 + b^2)^2}            \\
                                               & = \dfrac{\abs{b} \abs{b^4 - a^2 b^2}}{(a^2 + b^2)^2}        \\
                                               & \leq \dfrac{\abs{b} (b^4 + 2 a^2 b^2 + a^4)}{(a^2 + b^2)^2} \\
                                               & = \dfrac{\abs{b} (a^2 + b^2)^2}{(a^2 + b^2)^2}              \\
                                               & = \abs{b}
  \end{align*}
  and
  \begin{align*}
    \abs{\dfrac{3 a^3 b^2 + a b^4}{(a^2 + b^2)^2}} & = \dfrac{\abs{a (2 a^2 b^2 + b^4) + a (a^2 b^2)}}{(a^2 + b^2)^2}                              \\
                                                   & \leq \dfrac{\abs{a} \abs{2 a^2 b^2 + b^4} + \abs{a} \abs{a^2 b^2}}{(a^2 + b^2)^2}             \\
                                                   & \leq \dfrac{\abs{a} (a^4 + 2 a^2 b^2 + b^4) + \abs{a} (a^4 + 2 a^2 b^2 + b^4)}{(a^2 + b^2)^2} \\
                                                   & = \dfrac{\abs{a} (a^2 + b^2)^2 + \abs{a} (a^2 + b^2)^2}{(a^2 + b^2)^2}                        \\
                                                   & = 2 \abs{a}.
  \end{align*}
  Thus we have
  \begin{align*}
             & \lim_{(a, b) \to (0, 0) ; (a, b) \neq (0, 0)} \abs{b} = 0                                                                  \\
    \implies & \lim_{(a, b) \to (0, 0) ; (a, b) \neq (0, 0)} \abs{\dfrac{b^5 - a^2 b^3}{(a^2 + b^2)^2}} = 0 &  & \text{(by squeeze test)} \\
    \implies & \lim_{(a, b) \to (0, 0) ; (a, b) \neq (0, 0)} \dfrac{b^5 - a^2 b^3}{(a^2 + b^2)^2} = 0                                     \\
    \implies & \lim_{(a, b) \to (0, 0) ; (a, b) \neq (0, 0)} \dfrac{\partial f}{\partial x}(a, b) = 0
  \end{align*}
  and
  \begin{align*}
             & \lim_{(a, b) \to (0, 0) ; (a, b) \neq (0, 0)} 2 \abs{a} = 0                                                                    \\
    \implies & \lim_{(a, b) \to (0, 0) ; (a, b) \neq (0, 0)} \abs{\dfrac{3 a^3 b^2 + a b^4}{(a^2 + b^2)^2}} = 0 &  & \text{(by squeeze test)} \\
    \implies & \lim_{(a, b) \to (0, 0) ; (a, b) \neq (0, 0)} \dfrac{3 a^3 b^2 + a b^4}{(a^2 + b^2)^2} = 0                                     \\
    \implies & \lim_{(a, b) \to (0, 0) ; (a, b) \neq (0, 0)} \dfrac{\partial f}{\partial y}(a, b) = 0.
  \end{align*}
  Since
  \begin{align*}
    \dfrac{\partial f}{\partial x}(0, 0) & = \lim_{t \to 0 ; t \neq 0} \dfrac{f\big((0, 0) + t(1, 0)\big) - f(0, 0)}{t} &  & \by{6.3.7} \\
                                         & = \lim_{t \to 0 ; t \neq 0} \dfrac{\dfrac{t 0^3}{t^2 + 0^2} - 0}{t}                          \\
                                         & = 0
  \end{align*}
  and
  \begin{align*}
    \dfrac{\partial f}{\partial y}(0, 0) & = \lim_{t \to 0 ; t \neq 0} \dfrac{f\big((0, 0) + t(0, 1)\big) - f(0, 0)}{t} &  & \by{6.3.7} \\
                                         & = \lim_{t \to 0 ; t \neq 0} \dfrac{\dfrac{0 t^3}{0^2 + t^2} - 0}{t}                          \\
                                         & = 0,
  \end{align*}
  by \cref{2.1.1} we know that \(\dfrac{\partial f}{\partial x}\) and \(\dfrac{\partial f}{\partial y}\) are continuous at \((0, 0)\) from \((\R^2, d_{l^2}|_{\R^2 \times \R^2})\) to \((\R, d_{l^1}|_{\R \times \R})\).
  Combine the proof above we conclude by \cref{6.5.1} that \(f\) is continuously differentiable on \(\R^2\).

  Let \((a, b) \in \R^2 \setminus \set{(0, 0)}\).
  Observe that
  \begin{align*}
     & \dfrac{\partial}{\partial y} \dfrac{\partial f}{\partial x}(a, b)                                                                                                                \\
     & = \lim_{t \to 0 ; t \neq 0} \dfrac{\dfrac{\partial f}{\partial x}\big((a, b) + t(0, 1)\big) - \dfrac{\partial f}{\partial x}(a, b)}{t}                                           \\
     & = \lim_{t \to 0 ; t \neq 0} \dfrac{\dfrac{\partial f}{\partial x}(a, b + t) - \dfrac{\partial f}{\partial x}(a, b)}{t}                                                           \\
     & = \lim_{t \to 0 ; t \neq 0} \dfrac{\dfrac{(b + t)^5 - a^2 (b + t)^3}{\big(a^2 + (b + t)^2\big)^2} - \dfrac{b^5 - a^2 b^3}{(a^2 + b^2)^2}}{t}                                     \\
     & = \lim_{t \to 0 ; t \neq 0} \dfrac{\big((b + t)^5 - a^2 (b + t)^3\big) (a^2 + b^2)^2 - (b^5 - a^2 b^3) \big(a^2 + (b + t)^2\big)^2}{t \big(a^2 + (b + t)^2\big)^2 (a^2 + b^2)^2} \\
     & = \dfrac{-3 a^6 b^2 + 3 a^4 b^4 + 7 a^2 b^6 + b^8}{(a^2 + b^2)^4}
  \end{align*}
  and
  \begin{align*}
     & \dfrac{\partial}{\partial x} \dfrac{\partial f}{\partial y}(a, b)                                                                                                                        \\
     & = \lim_{t \to 0 ; t \neq 0} \dfrac{\dfrac{\partial f}{\partial y}\big((a, b) + t(1, 0)\big) - \dfrac{\partial f}{\partial y}(a, b)}{t}                                                   \\
     & = \lim_{t \to 0 ; t \neq 0} \dfrac{\dfrac{\partial f}{\partial y}(a + t, b) - \dfrac{\partial f}{\partial y}(a, b)}{t}                                                                   \\
     & = \lim_{t \to 0 ; t \neq 0} \dfrac{\dfrac{3 (a + t)^3 b^2 + (a + t) b^4}{\big((a + t)^2 + b^2\big)^2} - \dfrac{3 a^3 b^2 + a b^4}{(a^2 + b^2)^2}}{t}                                     \\
     & = \lim_{t \to 0 ; t \neq 0} \dfrac{\big(3 (a + t)^3 b^2 + (a + t) b^4\big) (a^2 + b^2)^2 - (3 a^3 b^2 + a b^4) \big((a + t)^2 + b^2\big)^2}{t \big((a + t)^2 + b^2\big)^2 (a^2 + b^2)^2} \\
     & = \dfrac{-3 a^6 b^2 + 3 a^4 b^4 + 7 a^2 b^6 + b^8}{(a^2 + b^2)^4}.
  \end{align*}
  Thus by \cref{6.3.7} \(\dfrac{\partial}{\partial y} \dfrac{\partial f}{\partial x}\) and \(\dfrac{\partial}{\partial x} \dfrac{\partial f}{\partial y}\) exist for all \((a, b) \in \R^2 \setminus \set{(0, 0)}\).
  Since
  \begin{align*}
     & \dfrac{\partial}{\partial y} \dfrac{\partial f}{\partial x}(0, 0)                                                                      \\
     & = \lim_{t \to 0 ; t \neq 0} \dfrac{\dfrac{\partial f}{\partial x}\big((0, 0) - t(0, 1)\big) - \dfrac{\partial f}{\partial x}(0, 0)}{t} \\
     & = \lim_{t \to 0 ; t \neq 0} \dfrac{\dfrac{t^5 - 0^2 t^3}{(0^2 + t^2)^2} - 0}{t}                                                        \\
     & = \lim_{t \to 0 ; t \neq 0} \dfrac{t^5}{t^5}                                                                                           \\
     & = 1
  \end{align*}
  and
  \begin{align*}
     & \dfrac{\partial}{\partial x} \dfrac{\partial f}{\partial y}(0, 0)                                                                      \\
     & = \lim_{t \to 0 ; t \neq 0} \dfrac{\dfrac{\partial f}{\partial y}\big((0, 0) - t(1, 0)\big) - \dfrac{\partial f}{\partial y}(0, 0)}{t} \\
     & = \lim_{t \to 0 ; t \neq 0} \dfrac{\dfrac{3 t^3 0^2 + t 0^4}{(t^2 + 0^2)^2} - 0}{t}                                                    \\
     & = 0,
  \end{align*}
  we know that both \(\dfrac{\partial}{\partial y} \dfrac{\partial f}{\partial x}\) and \(\dfrac{\partial}{\partial x} \dfrac{\partial f}{\partial y}\) exist at \((0, 0)\) but are not equal to each other.
  This does not contradict to \cref{6.5.4} since both \(\dfrac{\partial}{\partial y} \dfrac{\partial f}{\partial x}\) and \(\dfrac{\partial}{\partial x} \dfrac{\partial f}{\partial y}\) are not continuous at \((0, 0)\).
\end{proof}
