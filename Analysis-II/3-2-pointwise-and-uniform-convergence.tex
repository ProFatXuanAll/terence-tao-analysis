\section{Pointwise and uniform convergence}\label{sec:3.2}

\begin{defn}[Pointwise convergence]\label{3.2.1}
  Let \((f^{(n)})_{n = 1}^\infty\) be a sequence of functions from one metric space \((X, d_X)\) to another \((Y, d_Y)\), and let \(f : X \to Y\) be another function.
  We say that \emph{\((f^{(n)})_{n = 1}^\infty\) converges pointwise to \(f\) on \(X\)} if we have
  \[
    \lim_{n \to \infty} f^{(n)}(x) = f(x)
  \]
  for all \(x \in X\), i.e.,
  \[
    \lim_{n \to \infty} d_Y\big(f^{(n)}(x), f(x)\big) = 0.
  \]
  Or in other words, for every \(x\) and every \(\varepsilon > 0\) there exists \(N > 0\) such that \(d_Y\big(f^{(n)}(x), f(x)\big) < \varepsilon\) for every \(n > N\).
  We call the function \(f\) the \emph{pointwise limit} of the functions \(f^{(n)}\).
\end{defn}

\begin{rmk}\label{3.2.2}
  Note that \(f^{(n)}(x)\) and \(f(x)\) are points in \(Y\), rather than functions, so we are using our prior notion of convergence in metric spaces to determine convergence of functions.
  Also note that we are not really using the fact that \((X, d_X)\) is a metric space
  (i.e., we are not using the metric \(d_X\));
  for this definition it would suffice for \(X\) to just be a plain old set with no metric structure.
  However, later on we shall want to restrict our attention to \emph{continuous} functions from \(X\) to \(Y\), and in order to do so we need a metric on \(X\) (and on \(Y\)), or at least a topological structure.
  Also when we introduce the concept of \emph{uniform convergence}, then we will definitely need a metric structure on \(X\) and \(Y\);
  there is no comparable notion for topological spaces.
\end{rmk}

\begin{note}
  From \cref{1.1.20} we see that a sequence \((f^{(n)})_{n = 1}^\infty\) of functions from one metric space \((X, d_X)\) to another \((Y, d_Y)\) can have at most one pointwise limit \(f\)
  (this explains why we can refer to \(f\) as \emph{the} pointwise limit).
  However, it is of course possible for a sequence of functions to have no pointwise limit, just as a sequence of points in a metric space do not necessarily have a limit.
\end{note}

\begin{note}
  Pointwise convergence is a very natural concept, but it has a number of disadvantages:
  it does not preserve continuity, derivatives, limits, or integrals.
  The problem is that while \(f^{(n)}(x)\) converges to \(f(x)\) for each \(x\), the \emph{rate} of that convergence varies substantially with \(x\).
  To put things another way, the convergence of \(f^{(n)}\) to \(f\) is not \emph{uniform} in \(x\)
  - the \(N\) that one needs to get \(f^{(n)}(x)\) within \(\varepsilon\) of \(f\) depends on \(x\) as well as on \(\varepsilon\).
  This motivates a stronger notion of convergence.
\end{note}

\setcounter{thm}{6}
\begin{defn}[Uniform convergence]\label{3.2.7}
  Let \((f^{(n)})_{n = 1}^\infty\) be a sequence of functions from one metric space \((X, d_X)\) to another \((Y, d_Y)\), and let \(f : X \to Y\) be another function.
  We say that \emph{\((f^{(n)})_{n = 1}^\infty\) converges uniformly to \(f\) on \(X\)} if for every \(\varepsilon > 0\) there exists \(N > 0\) such that \(d_Y\big(f^{(n)}(x), f(x)\big) < \varepsilon\) for every \(n > N\) and \(x \in X\).
  We call the function \(f\) the \emph{uniform limit} of the functions \(f^{(n)}\).
\end{defn}

\begin{rmk}\label{3.2.8}
  Note that \cref{3.2.7} is subtly different from the definition for pointwise convergence in \cref{3.2.1}.
  In the definition of pointwise convergence, \(N\) was allowed to depend on \(x\);
  now it is not.
  The reader should compare this distinction to the distinction between continuity and uniform continuity
  (i.e., between \cref{2.1.1} and \cref{2.3.4}).
\end{rmk}

\begin{note}
  If \(f^{(n)}\) converges uniformly to \(f\) on \(X\), then it also converges pointwise to the same function \(f\).
  Thus when the uniform limit and pointwise limit both exist, then they have to be equal.
  However, the converse is not true.
\end{note}

\begin{note}
  If a sequence \(f^{(n)} : X \to Y\) of functions converges pointwise (or uniformly) to a function \(f : X \to Y\), then the restrictions \(f^{(n)}|_E : E \to Y\) of \(f^{(n)}\) to some subset \(E\) of \(X\) will also converge pointwise (or uniformly) to \(f|_E\).
\end{note}

\exercisesection

\begin{ex}\label{ex:3.2.1}
  The purpose of this exercise is to demonstrate a concrete relationship between continuity and pointwise convergence, and between uniform continuity and uniform convergence.
  Let \(f : \R \to \R\) be a function.
  For any \(a \in \R\), let \(f_a : \R \to \R\) be the shifted function \(f_a(x) \coloneqq f(x - a)\).
  \begin{enumerate}
    \item Show that \(f\) is continuous if and only if, whenever \((a_n)_{n = 0}^\infty\) is a sequence of real numbers which converges to zero, the shifted functions \(f_{a_n}\) converge pointwise to \(f\).
    \item Show that \(f\) is uniformly continuous if and only if, whenever \((a_n)_{n = 0}^\infty\) is a sequence of real numbers which converges to zero, the shifted functions \(f_{a_n}\) converge uniformly to \(f\).
  \end{enumerate}
\end{ex}

\begin{proof}{(a)}
  Suppose that \(f\) is continuous on \(\R\).
  Let \((a_n)_{n = 0}^\infty\) be a sequence in \(\R\) such that \(\lim_{n \to \infty} a_n = 0\).
  Let \(x_0 \in \R\).
  Then we have
  \begin{align*}
             & \lim_{n \to \infty} a_n = 0                                                                                                       \\
    \implies & \lim_{n \to \infty} (x_0 - a_n) = x_0                                                                                             \\
    \implies & \lim_{n \to \infty} f_{a_n}(x_0) = \lim_{n \to \infty} f(x_0 - a_n) = f(x_0). & \text{(\(f\) is continuous at \(x_0\) on \(\R\))}
  \end{align*}
  Since \(x_0\) is arbitrary, by \cref{3.2.1} we know that \((f_{a_n})_{n = 0}^\infty\) converges pointwise to \(f\) on \(\R\) with respect to \(d_{l^1}|_{\R \times \R}\).
  Since \((a_n)_{n = 0}^\infty\) is arbitrary, we conclude that if \((a_n)_{n = 0}^\infty\) is a sequence in \(\R\) such that \(\lim_{n \to \infty} a_n = 0\), then \((f_{a_n})_{n = 0}^\infty\) converges pointwise to \(f\) on \(\R\) with respect to \(d_{l^1}|_{\R \times \R}\).

  Now suppose that if \((a_n)_{n = 0}^\infty\) is a sequence in \(\R\) such that \(\lim_{n \to \infty} a_n = 0\), then \((f_{a_n})_{n = 0}^\infty\) converges pointwise to \(f\) on \(\R\) with respect to \(d_{l^1}|_{\R \times \R}\).
  Suppose for sake of contradiction that there exists some \(x_0 \in \R\) such that \(\lim_{x \to x_0 ; x \in \R} f(x) \neq f(x_0)\).
  Then we have
  \[
    \exists\ \varepsilon \in \R^+ : \forall \delta \in \R^+, \exists\ x \in \R : \begin{cases}
      \abs{x - x_0} < \delta \\
      \abs{f(x) - f(x_0)} > \varepsilon
    \end{cases}
  \]
  Fix such \(\varepsilon\).
  We choose a sequence \((a_n)_{n = 0}^\infty\) in \(\R\) such that \(\abs{a_n - x_0} < \frac{1}{n + 1}\) for all \(n \in \N\).
  Then we have \(\lim_{n \to \infty} \abs{a_n - x_0} = \lim_{n \to \infty} (a_n - x_0) = 0\).
  By hypothesis we have
  \[
    \forall x \in \R, f(x) = \lim_{n \to \infty} f_{a_n - x_0}(x) = \lim_{n \to \infty} f(x - a_n + x_0).
  \]
  But this means
  \[
    \begin{cases}
      \abs{x - a_n + x_0 - x} = \abs{-a_n + x_0} < \frac{1}{n} \\
      \abs{f(x - a_n + x_0) - f(x)} < \varepsilon
    \end{cases}
  \]
  a contradiction.
  Thus we have \(\lim_{x \to x_0 ; x \in \R} f(x) = f(x_0)\) for every \(x_0 \in \R\) and \(f\) is continuous on \(\R\).
\end{proof}

\begin{proof}{(b)}
  Suppose that \(f\) is uniformly continuous on \(\R\).
  Let \((a_n)_{n = 0}^\infty\) be a sequence in \(\R\) such that \(\lim_{n \to \infty} a_n = 0\).
  Since \(f\) is uniformly continuous on \(\R\), we have
  \[
    \forall \varepsilon \in \R^+, \exists\ \delta \in \R^+ : \forall x_1, x_2 \in \R, \abs{x_1 - x_2} < \delta \implies \abs{f(x_1) - f(x_2)} < \varepsilon.
  \]
  Fix one pair of \(\varepsilon\) and \(\delta\).
  Then we have
  \begin{align*}
             & \exists\ N \in \N : \forall n \geq N, \abs{a_n} < \delta                                       \\
    \implies & \exists\ N \in \N : \forall n \geq N, \abs{-a_n} < \delta                                      \\
    \implies & \exists\ N \in \N : \forall n \geq N, \forall x \in \R, \abs{x - a_n - x} < \delta             \\
    \implies & \exists\ N \in \N : \forall n \geq N, \forall x \in \R, \abs{f(x - a_n) - f(x)} < \varepsilon  \\
    \implies & \exists\ N \in \N : \forall n \geq N, \forall x \in \R, \abs{f_{a_n}(x) - f(x)} < \varepsilon.
  \end{align*}
  Since this is true for arbitrary \(\varepsilon\), by \cref{3.2.7} we know that \((f_{a_n})_{n = 0}^\infty\) converges uniformly to \(f\) on \(\R\) with respect to \(d_{l^1}|_{\R \times \R}\).
  Since \((a_n)_{n = 0}^\infty\) is arbitrary, we conclude that if \((a_n)_{n = 0}^\infty\) is a sequence in \(\R\) such that \(\lim_{n \to \infty} a_n = 0\), then \((f_{a_n})_{n = 0}^\infty\) uniformly converges to \(f\) on \(\R\) with respect to \(d_{l^1}|_{\R \times \R}\).

  Now suppose that if \((a_n)_{n = 0}^\infty\) is a sequence in \(\R\) such that \(\lim_{n \to \infty} a_n = 0\), then \((f_{a_n})_{n = 0}^\infty\) uniformly converges to \(f\) on \(\R\) with respect to \(d_{l^1}|_{\R \times \R}\).
  Suppose for sake of contradiction that \(f\) is not uniformly continuous on \(\R\).
  Then we have
  \[
    \exists\ \varepsilon \in \R^+ : \forall \delta \in \R^+, \exists\ x_1, x_2 \in \R : \begin{cases}
      \abs{x_1 - x_2} < \delta \\
      \abs{f(x_1) - f(x_2)} > \varepsilon
    \end{cases}
  \]
  Let \((a_n)_{n = 0}^\infty\) be a sequence in \(\R\) such that \(\lim_{n \to \infty} \abs{a_n} = 0\).
  Then we have
  \begin{align*}
             & \lim_{n \to \infty} \abs{a_n} = 0 = \lim_{n \to \infty} a_n = \lim_{n \to \infty} -a_n                                  \\
    \implies & \forall \delta \in \R^+, \exists\ N \in \N : \forall n \geq N, \abs{-a_n} < \delta                                      \\
    \implies & \forall \delta \in \R^+, \exists\ N \in \N : \forall n \geq N, \forall x \in \R, \abs{x - a_n - x} < \delta             \\
    \implies & \forall \delta \in \R^+, \exists\ N \in \N : \forall n \geq N, \forall x \in \R, \abs{f(x - a_n) - f(x)} > \varepsilon  \\
    \implies & \forall \delta \in \R^+, \exists\ N \in \N : \forall n \geq N, \forall x \in \R, \abs{f_{a_n}(x) - f(x)} > \varepsilon.
  \end{align*}
  But by hypothesis we know that \((f_{a_n})_{n = 0}^\infty\) uniformly converges to \(f\) on \(\R\), which by \cref{3.2.7} means
  \[
    \exists\ N' \in \N : \forall n \geq N', \forall x \in \R, \abs{f_{a_n}(x) - f(x)} < \varepsilon,
  \]
  a contradiction.
  Thus \(f\) is uniformly continuous on \(\R\).
\end{proof}

\begin{ex}\label{ex:3.2.2}
  \quad
  \begin{enumerate}
    \item Let \((f^{(n)})_{n = 1}^\infty\) be a sequence of functions from one metric space \((X, d_X)\) to another \((Y, d_Y)\), and let \(f : X \to Y\) be another function from \(X\) to \(Y\).
          Show that if \(f^{(n)}\) converges uniformly to \(f\), then \(f^{(n)}\) also converges pointwise to \(f\).
    \item For each integer \(n \geq 1\), let \(f^{(n)} : (-1, 1) \to \R\) be the function \(f^{(n)}(x) \coloneqq x^n\).
          Prove that \(f^{(n)}\) converges pointwise to the zero function \(0\), but does not converge uniformly to any function \(f : (-1, 1) \to \R\).
    \item Let \(g : (-1, 1) \to \R\) be the function \(g(x) \coloneqq x / (1 - x)\).
          With the notation as in (b), show that the partial sums \(\sum_{n = 1}^N f^{(n)}\) converges pointwise as \(N \to \infty\) to \(g\), but does not converge uniformly to \(g\), on the open interval \((-1, 1)\).
          What would happen if we replaced the open interval \((-1, 1)\) with the closed interval \([-1, 1]\)?
  \end{enumerate}
\end{ex}

\begin{proof}{(a)}
  Let \(x_0 \in X\).
  Since
  \begin{align*}
             & (f^{(n)})_{n = 1}^\infty \text{ converges uniformly to } f \text{ on } X                                   \\
             & \text{ with respect to } d_Y                                                                               \\
    \implies & \forall \varepsilon \in \R^+, \exists\ N \in \Z^+ :                                                        \\
             & \forall n \geq N, \forall x \in X, d_Y\big(f^{(n)}(x), f(x)\big) < \varepsilon & \text{(by \cref{3.2.7})}  \\
    \implies & \forall \varepsilon \in \R^+, \exists\ N \in \Z^+ :                                                        \\
             & \forall n \geq N, d_Y\big(f^{(n)}(x_0), f(x_0)\big) < \varepsilon                                          \\
    \implies & \lim_{n \to \infty} d_Y\big(f^{(n)}(x_0), f(x_0)\big) = 0                      & \text{(by \cref{1.1.14})}
  \end{align*}
  and \(x_0\) is arbitrary, by \cref{3.2.1} we know that \((f^{(n)})_{n = 1}^\infty\) converges pointwise to \(f\) on \(X\) with respect to \(d_Y\).
\end{proof}

\begin{proof}{(b)}
  Let \(z : (-1, 1) \to \R\) be the zero function, i.e., \(z(x) = 0\) for each \(x \in (-1, 1)\).
  Then we have
  \begin{align*}
             & \forall x \in (-1, 1), \lim_{n \to \infty} x^n = 0                                                        \\
    \implies & \forall x \in (-1, 1), \lim_{n \to \infty} f^{(n)}(x) = 0 = z(x)                                          \\
    \implies & (f^{(n)})_{n = 1}^\infty \text{ converges pointwise to } z \text{ on } (-1, 1)                            \\
             & \text{with respect to } d_{l^1}|_{\R \times \R}.                               & \text{(by \cref{3.2.1})}
  \end{align*}
  Suppose for sake of contradiction that there exists a \(f : (-1, 1) \to \R\) such that \((f^{(n)})_{n = 1}^\infty\) converges uniformly to \(f\) on \((-1, 1)\) with respect to \(d_{l^1}|_{\R \times \R}\).
  By \cref{ex:3.2.2}(a) and \cref{1.1.20} we know that \(f = z\).
  Then we have
  \begin{align*}
             & \forall \varepsilon \in \R^+, \exists\ N \in \Z^+ :                                                       \\
             & \forall n \geq N, \forall x \in (-1, 1), \abs{f^{(n)}(x) - z(x)} < \varepsilon & \text{(by \cref{3.2.7})} \\
    \implies & \forall \varepsilon \in \R^+, \exists\ N \in \Z^+ :                                                       \\
             & \forall n \geq N, \forall x \in (-1, 1), \abs{x^n} < \varepsilon.
  \end{align*}
  Now consider \(\varepsilon = \frac{1}{2}\).
  Then we have
  \begin{align*}
             & \exists\ N \in \Z^+ : \forall n \geq N, \forall x \in (-1, 1), \abs{x^n} < \frac{1}{2}                                              \\
    \implies & \exists\ N \in \Z^+ : \begin{cases}
                                       (\frac{1}{2})^{\frac{1}{n}} \in (-1, 1) \\
                                       \abs{\big((\frac{1}{2})^{\frac{1}{n}}\big)^n} = \frac{1}{2} < \frac{1}{2}
                                     \end{cases}
  \end{align*}
  a contradiction.
  Thus \((f^{(n)})_{n = 1}^\infty\) does not converge uniformly to any \(f\) on \((-1, 1)\) with respect to \(d_{l^1}|_{\R \times \R}\).
\end{proof}

\begin{proof}{(c)}
  By Lemma 7.3.3 in Analysis I we have
  \begin{align*}
    \forall x \in (-1, 1), \lim_{N \to \infty} \sum_{n = 1}^N f^{(n)}(x) & = \lim_{N \to \infty} \sum_{n = 1}^N x^n                 \\
                                                                         & = \lim_{N \to \infty} \bigg(\sum_{n = 0}^N x^n - 1\bigg) \\
                                                                         & = \bigg(\lim_{N \to \infty} \sum_{n = 0}^N x^n\bigg) - 1 \\
                                                                         & = \frac{1}{1 - x} - 1                                    \\
                                                                         & = \frac{x}{1 - x}                                        \\
                                                                         & = g(x).
  \end{align*}
  Thus by \cref{3.2.1} we know that \((\sum_{n = 1}^N f^{(n)})_{N = 1}^\infty\) converges pointwise to \(g\) on \((-1, 1)\) with respect to \(d_{l^1}|_{\R \times \R}\).
  Suppose for sake of contradiction that \((\sum_{n = 1}^N f^{(n)})_{N = 1}^\infty\) converges uniformly to \(g\) on \((-1, 1)\) with respect to \(d_{l^1}|_{\R \times \R}\).
  Then by \cref{3.2.7} we know that
  \begin{align*}
             & \forall \varepsilon \in \R^+, \exists\ M \in \Z^+ : \forall N \geq M, \forall x \in (-1, 1), \abs{\sum_{n = 1}^N f^{(n)}(x) - g(x)} < \varepsilon                          \\
    \implies & \forall \varepsilon \in \R^+, \exists\ M \in \Z^+ : \forall N \geq M, \forall x \in (-1, 1), \abs{\frac{x(1 - x^N)}{1 - x} - \frac{x}{1 - x}} < \varepsilon                \\
    \implies & \forall \varepsilon \in \R^+, \exists\ M \in \Z^+ : \forall N \geq M, \forall x \in (-1, 1), \abs{\frac{-x^{N + 1}}{1 - x}} = \abs{\frac{x^{N + 1}}{1 - x}} < \varepsilon.
  \end{align*}
  Now consider \(\varepsilon = \frac{1}{2}\).
  Then we have
  \begin{align*}
             & \exists\ N \geq M : \forall n \geq N, \forall x \in (-1, 1), \abs{\frac{x^{N + 1}}{1 - x}} < \frac{1}{2}                                                               \\
    \implies & \exists\ N \geq M : \begin{cases}
                                     (\frac{1}{2})^{\frac{1}{N + 1}} \in (-1, 1) \\
                                     \abs{\frac{\big((\frac{1}{2})^{\frac{1}{N + 1}}\big)^{N + 1}}{1 - (\frac{1}{2})^{\frac{1}{N + 1}}}} = \frac{\frac{1}{2}}{1 - (\frac{1}{2})^{\frac{1}{N + 1}}} < \frac{1}{2}
                                   \end{cases}
  \end{align*}
  But we know that
  \begin{align*}
             & (\frac{1}{2})^{\frac{1}{N + 1}} > \frac{1}{2}                       \\
    \implies & 1 - (\frac{1}{2})^{\frac{1}{N + 1}} < 1 - \frac{1}{2} = \frac{1}{2} \\
    \implies & \frac{\frac{1}{2}}{1 - (\frac{1}{2})^{\frac{1}{N + 1}}} > 1,
  \end{align*}
  a contradiction.
  Thus \((\sum_{n = 1}^N f^{(n)})_{N = 1}^\infty\) does not converge uniformly to \(g\) on \((-1, 1)\) with respect to \(d_{l^1}|_{\R \times \R}\).

  If we replace \((-1, 1)\) with \([-1, 1]\), then by Lemma 7.3.3 in Analysis I \(\sum_{n = 1}^\infty 1\) and \(\sum_{n = 1}^\infty -1\) does not converge, thus by \cref{3.2.1} \((\sum_{n = 1}^N f^{(n)})_{N = 1}^\infty\) does not converge pointwise to \(g\) on \([-1, 1]\) with respect to \(d_{l^1}|_{\R \times \R}\).
\end{proof}

\begin{ex}\label{ex:3.2.3}
  Let \((X, d_X)\) a metric space, and for every integer \(n \geq 1\), let \(f_n : X \to \R\) be a real-valued function.
  Suppose that \(f_n\) converges pointwise to another function \(f : X \to \R\) on \(X\)
  (in this question we give \(\R\) the standard metric \(d(x, y) = \abs{x - y}\)).
  Let \(h : \R \to \R\) be a continuous function.
  Show that the functions \(h \circ f_n\) converge pointwise to \(h \circ f\) on \(X\), where \(h \circ f_n : X \to \R\) is the function \(h \circ f_n(x) \coloneqq h\big(f_n(x)\big)\), and similarly for \(h \circ f\).
\end{ex}

\begin{proof}
  Let \(x_0 \in X\).
  We have
  \begin{align*}
             & h \text{ is continuous on } \R \text{ with respect to } d                                      \\
    \implies & h \text{ is continuous at } x_0 \text{ with respect to } d                                     \\
    \implies & \forall \varepsilon \in \R^+, \exists\ \delta \in \R^+ :                                       \\
             & \big(\forall x \in \R, \abs{x - x_0} < \delta \implies \abs{h(x) - h(x_0)} < \varepsilon\big).
  \end{align*}
  Fix one pair of \(\varepsilon\) and \(\delta\).
  Then we have
  \begin{align*}
             & (f_n)_{n = 1}^\infty \text{ converges pointwise to } f \text{ on } X \text{ with respect to } d                                 \\
    \implies & \lim_{n \to \infty} \abs{f_n(x_0) - f(x_0)} = 0                                                      & \text{(by \cref{3.2.1})} \\
    \implies & \exists\ N \in \Z^+ : \forall n \geq N, \abs{f_n(x_0) - f(x_0)} < \delta                                                        \\
    \implies & \exists\ N \in \Z^+ : \forall n \geq N, \abs{h\big(f_n(x_0)\big) - h\big(f(x_0)\big)} < \varepsilon.
  \end{align*}
  Since \(\varepsilon\) is arbitrary, by \cref{1.1.14} we know that
  \[
    \lim_{n \to \infty} \abs{h\big(f_n(x_0)\big) - h\big(f(x_0)\big)} = \lim_{n \to \infty} \abs{(h \circ f_n)(x_0) - (h \circ f)(x_0)} = 0.
  \]
  Since \(x_0\) is arbitrary, by \cref{3.2.1} we know \((h \circ f_n)_{n = 1}^\infty\) converges pointwise to \(h \circ f\) on \(X\) with respect to \(d_{l^1}|_{\R \times \R}\).
\end{proof}

\begin{ex}\label{ex:3.2.4}
  Let \(f_n : X \to Y\) be a sequence of bounded functions from one metric space \((X, d_X)\) to another metric space \((Y, d_Y)\).
  Suppose that \(f_n\) converges uniformly to another function \(f : X \to Y\).
  Suppose that \(f\) is a bounded function;
  i.e., there exists a ball \(B_{(Y, d_Y)}(y_0, R)\) in \(Y\) such that \(f(x) \in B_{(Y, d_Y)}(y_0, R)\) for all \(x \in X\).
  Show that the sequence \(f_n\) is \emph{uniformly bounded};
  i.e., there exists a ball \(B_{(Y, d_Y)}(y_0, R)\) in \(Y\) such that \(f_n(x) \in B_{(Y, d_Y)}(y_0, R)\) for all \(x \in X\) and all positive integers \(n\).
\end{ex}

\begin{proof}
  Since \(f\) is bounded, by \cref{1.5.3} we have
  \[
    \forall y \in Y, \exists\ \varepsilon \in \R^+ : f(X) \subseteq B_{(Y, d_Y)}(y, \varepsilon).
  \]
  We choose one pair of \(y\) and \(\varepsilon\).
  Since \((f_n)_{n = 1}^\infty\) converges uniformly to \(f\) on \(X\) with respect to \(d_Y\), we have
  \begin{align*}
             & \exists\ N \in \Z^+ : \forall n \geq N, \forall x \in X, d_Y\big(f_n(x), f(x)\big) < \varepsilon          & \text{(by \cref{3.2.7})}    \\
    \implies & \exists\ N \in \Z^+ : \forall n \geq N, \forall x \in X,                                                                                \\
             & d_Y\big(f_n(x), y\big) \leq d_Y\big(f_n(x), f(x)\big) + d_Y\big(f(x), y\big) < \varepsilon + \varepsilon  & \text{(by \cref{1.1.2}(d))} \\
    \implies & \exists\ N \in \Z^+ : \forall n \geq N, \forall x \in X, f_n(x) \in B_{(Y, d_Y)}\big(y, 2\varepsilon\big)                               \\
    \implies & \exists\ N \in \Z^+ : \forall n \geq N, f_n(X) \subseteq B_{(Y, d_Y)}\big(y, 2\varepsilon\big)
  \end{align*}
  Now fix \(N\).
  Let \(S = \{n \in \Z^+ : n < N\}\).
  Then \(S\) is finite.
  If \(S = \emptyset\), then we have \(N = 1\) and thus by definition \((f_n)_{n = 1}^\infty\) is uniformly bounded.
  So suppose that \(S \neq \emptyset\).
  By hypothesis we know that \(f_n\) is bounded for each \(n \in S\), thus by \cref{1.5.3} we have
  \[
    \forall n \in S, \exists\ \delta_n \in \R^+ : f_n(X) \subseteq B_{(Y, d_Y)}(y, \delta_n).
  \]
  We choose one \(\delta_n\) for each \(n \in S\).
  Since \(S\) is finite, we know that \(\delta_{\max} = \max\{\delta_n : n \in S\}\) is well-defined.
  Then we have
  \begin{align*}
             & \begin{cases}
                 \forall n \in S, f_n(X) \subseteq B_{(Y, d_Y)}(y, \delta_{\max}) \\
                 \forall n \geq N, f_n(X) \subseteq B_{(Y, d_Y)}(y, 2\varepsilon)
               \end{cases}                    \\
    \implies & \forall n \in \Z^+, f_n(X) \subseteq B_{(Y, d_Y)}(y, 2\varepsilon + \delta_{\max}).
  \end{align*}
  Since \(y\) is arbitrary, by \cref{1.5.3} we know that \(f_n\) is bounded for each \(n \in \Z^+\), i.e.,
  \[
    \forall y \in Y, \exists\ r \in \R^+ : \forall n \in \Z^+, f_n(X) \subseteq B_{(Y, d_Y)}(y, r).
  \]
  And by definition \((f_n)_{n = 1}^\infty\) is uniformly bounded.
\end{proof}