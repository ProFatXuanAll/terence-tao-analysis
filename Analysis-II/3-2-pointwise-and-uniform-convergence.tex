\section{Pointwise and uniform convergence}\label{sec 3.2}

\begin{definition}[Pointwise convergence]\label{3.2.1}
    Let \((f^{(n)})_{n = 1}^\infty\) be a sequence of functions from one metric space \((X, d_X)\) to another \((Y, d_Y)\), and let \(f : X \to Y\) be another function.
    We say that \emph{\((f^{(n)})_{n = 1}^\infty\) converges pointwise to \(f\) on \(X\)} if we have
    \[
        \lim_{n \to \infty} f^{(n)}(x) = f(x)
    \]
    for all \(x \in X\), i.e.,
    \[
        \lim_{n \to \infty} d_Y\big(f^{(n)}(x), f(x)\big) = 0.
    \]
    Or in other words, for every \(x\) and every \(\varepsilon > 0\) there exists \(N > 0\) such that \(d_Y\big(f^{(n)}(x), f(x)\big) < \varepsilon\) for every \(n > N\).
    We call the function \(f\) the \emph{pointwise limit} of the functions \(f^{(n)}\).
\end{definition}

\begin{remark}\label{3.2.2}
    Note that \(f^{(n)}(x)\) and \(f(x)\) are points in \(Y\), rather than functions, so we are using our prior notion of convergence in metric spaces to determine convergence of functions.
    Also note that we are not really using the fact that \((X, d_X)\) is a metric space
    (i.e., we are not using the metric \(d_X\));
    for this definition it would suffice for \(X\) to just be a plain old set with no metric structure.
    However, later on we shall want to restrict our attention to \emph{continuous} functions from \(X\) to \(Y\), and in order to do so we need a metric on \(X\) (and on \(Y\)), or at least a topological structure.
    Also when we introduce the concept of \emph{uniform convergence}, then we will definitely need a metric structure on \(X\) and \(Y\);
    there is no comparable notion for topological spaces.
\end{remark}