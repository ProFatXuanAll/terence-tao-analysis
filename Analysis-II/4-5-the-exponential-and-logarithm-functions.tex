\section{The exponential and logarithm functions}\label{sec 4.5}

\begin{definition}[Exponential function]\label{4.5.1}
    For every real number \(x\), we define the \emph{exponential function} \(\exp(x)\) to be the real number
    \[
        \exp(x) \coloneqq \sum_{n = 0}^\infty \frac{x^n}{n!}.
    \]
\end{definition}

\begin{theorem}[Basic properties of exponential]\label{4.5.2}
    \quad
    \begin{enumerate}
        \item For every real number \(x\), the series \(\sum_{n = 0}^\infty \frac{x^n}{n!}\) is absolutely convergent.
              In particular, \(\exp(x)\) exists and is real for every \(x \in \mathbf{R}\), the power series \(\sum_{n = 0}^\infty \frac{x^n}{n!}\) has an infinite radius of convergence, and \(\exp\) is a real analytic function on \((-\infty, \infty)\).
        \item \(\exp\) is differentiable on \(R\), and for every \(x \in \mathbf{R}\), \(\exp'(x) = \exp(x)\).
        \item \(\exp\) is continuous on \(R\), and for every interval \([a, b]\), we have \(\int_{[a, b]} \exp(x) \; dx = \exp(b) - \exp(a)\).
        \item For every \(x, y \in \mathbf{R}\), we have \(\exp(x + y) = \exp(x) \exp(y)\).
        \item We have \(\exp(0) = 1\).
              Also, for every \(x \in \mathbf{R}\), \(\exp(x)\) is positive, and \(\exp(-x) = 1 / \exp(x)\).
        \item \(\exp\) is strictly monotone increasing:
              in other words, if \(x, y\) are real numbers, then we have \(\exp(y) > \exp(x)\) if and only if \(y > x\).
    \end{enumerate}
\end{theorem}

\begin{proof}{(a)}
    If \(x = 0\), then we have
    \[
        1 = 1 + 0 = \frac{0^0}{0!} + \sum_{n = 1}^\infty \frac{0^n}{n!} = \sum_{n = 0}^\infty \frac{0^n}{n!} = \exp(0).
    \]
    So suppose that \(x \in \mathbf{R} \setminus \{0\}\).
    Since
    \begin{align*}
                 & \limsup_{n \to \infty} \abs*{\frac{x^{n + 1}}{(n + 1)!} \frac{n!}{x^n}} = \limsup_{n \to \infty} \frac{\abs*{x}}{n + 1} = 0 < 1                          \\
        \implies & \sum_{n = 0}^\infty \frac{x^n}{n!} \text{ is absolutely convergent},                                                            & \text{(by ratio test)}
    \end{align*}
    we know that \(\exp(x)\) exists for all \(x \in \mathbf{R} \setminus \{0\}\).
    Combine all proofs above we have
    \[
        \forall\ x \in \mathbf{R}, \begin{cases}
            \sum_{n = 0}^\infty \frac{x^n}{n!} \text{ is absolutely convergent} \\
            \exp(x) \in \mathbf{R}
        \end{cases}
    \]
    and by Definition \ref{4.2.1} \(\exp\) is real analytic on \((-\infty, \infty)\).
\end{proof}

\begin{note}
    One can write the exponential function in a more compact form, introducing famous \emph{Euler's number} \(e = 2.71828183 \dots\), also known as the \emph{base of the natural logarithm}.
\end{note}

\begin{definition}[Euler's number]\label{4.5.3}
    The number \(e\) is defined to be
    \[
        e = \exp(1) = \sum_{n = 0}^\infty \frac{1}{n!} = \frac{1}{0!} + \frac{1}{1!} + \frac{1}{2!} + \frac{1}{3!} + \dots.
    \]
\end{definition}

\begin{proposition}\label{4.5.4}
    For every real number \(x\), we have \(\exp(x) = e^x\).
\end{proposition}

\begin{note}
    In light of Proposition \ref{4.5.3} we can and will use \(e^x\) and \(\exp(x)\) interchangeably.
\end{note}

\begin{note}
    Since \(e > 1\), we see that \(e^x \to +\infty\) as \(x \to +\infty\), and \(e^x \to 0\) as \(x \to -\infty\).
    From this and the intermediate value theorem (Theorem 9.7.1 in Analysis I) we see that the range of the function \(\exp\) is \((0, \infty)\).
    Since \(\exp\) is increasing, it is injective, and hence \(\exp\) is a bijection from \(\mathbf{R}\) to \((0, \infty)\), and thus has an inverse from \((0, \infty) \to \mathbf{R}\).
\end{note}

\begin{definition}[Logarithm]\label{4.5.5}
    We define the \emph{natural logarithm function}
    \[
        \log : (0, \infty) \to \mathbf{R}
    \]
    (also called \(\ln\)) to be the inverse of the exponential function.
    Thus \(\exp\big(\log(x)\big) = x\) and \(\log\big(\exp(x)\big) = x\).
\end{definition}

\begin{note}
    Since \(\exp\) is continuous and strictly monotone increasing, we see that \(\log\) is also continuous and strictly monotone increasing (see Proposition 9.8.3 in Analysis I).
    Since \(\exp\) is also differentiable, and the derivative is never zero, we see from the inverse function theorem (Theorem 10.4.2 in Analysis I) that \(\log\) is also differentiable.
\end{note}

\begin{theorem}[Logarithm properties]\label{4.5.6}
    \quad
    \begin{enumerate}
        \item or every \(x \in (0, \infty)\), we have \(\ln'(x) = \frac{1}{x}\).
              In particular, by the fundamental theorem of calculus, we have \(\int_{[a, b]} \frac{1}{x} \; dx = \ln(b) - \ln(a)\) for any interval \([a, b]\) in \((0, \infty)\).
        \item We have \(\ln(xy) = \ln(x) + \ln(y)\) for all \(x, y \in (0, \infty)\).
        \item We have \(\ln(1) = 0\) and \(\ln(1 / x) = -\ln(x)\) for all \(x \in (0, \infty)\).
        \item For any \(x \in (0, \infty)\) and \(y \in R\), we have \(\ln(x^y) = y \ln(x)\).
        \item For any \(x \in (-1, 1)\), we have
              \[
                  \ln(1 - x) = - \sum_{n = 1}^\infty \frac{x^n}{n}.
              \]
              In particular, \(\ln\) is analytic at \(1\), with the power series expansion
              \[
                  \ln(x) = \sum_{n = 1}^\infty \frac{(-1)^{n + 1}}{n} (x - 1)^n
              \]
              for \(x \in (0, 2)\), with radius of convergence \(1\).
    \end{enumerate}
\end{theorem}

\exercisesection

\begin{exercise}\label{ex 4.5.1}
    Prove Theorem \ref{4.5.2}.
\end{exercise}

\begin{proof}
    See Theorem \ref{4.5.2}.
\end{proof}

\begin{exercise}\label{ex 4.5.2}
    Show that for every integer \(n \geq 3\), we have
    \[
        0 < \frac{1}{(n + 1)!} + \frac{1}{(n + 2)!} + \dots < \frac{1}{n!}.
    \]
    Conclude that \(n! e\) is not an integer for every \(n \geq 3\).
    Deduce from this that \(e\) is irrational.
\end{exercise}

\begin{exercise}\label{ex 4.5.3}
    Prove Proposition \ref{4.5.4}.
\end{exercise}

\begin{proof}
    See Proposition \ref{4.5.4}.
\end{proof}

\begin{exercise}\label{ex 4.5.4}
    Let \(f : \mathbf{R} \to \mathbf{R}\) be the function defined by setting \(f(x) \coloneqq \exp(-1 / x)\) when \(x > 0\), and \(f(x) \coloneqq 0\) when \(x \leq 0\).
    Prove that \(f\) is infinitely differentiable, and \(f^{(k)}(0) = 0\) for every integer \(k \geq 0\), but that \(f\) is not real analytic at \(0\).
\end{exercise}

\begin{exercise}\label{ex 4.5.5}
    Prove Theorem \ref{4.5.6}.
\end{exercise}

\begin{proof}
    See Theorem \ref{4.5.6}.
\end{proof}

\begin{exercise}\label{ex 4.5.6}
    Prove that the natural logarithm function is real analytic on \((0, +\infty)\).
\end{exercise}

\begin{exercise}\label{ex 4.5.7}
    Let \(f : \mathbf{R} \to (0, \infty)\) be a positive, real analytic function such that \(f'(x) = f(x)\) for all \(x \in \mathbf{R}\).
    Show that \(f(x) = C e^x\) for some positive constant \(C\);
    justify your reasoning.
\end{exercise}

\begin{exercise}\label{ex 4.5.8}
    Let \(m > 0\) be an integer.
    Show that
    \[
        \lim_{x \to +\infty} \frac{e^x}{x^m} +\infty.
    \]
\end{exercise}

\begin{exercise}\label{ex 4.5.9}
    Let \(P(x)\) be a polynomial, and let \(c > 0\).
    Show that there exists a real number \(N > 0\) such that \(e^{cx} > \abs*{P(x)}\) for all \(x > N\);
    thus an exponentially growing function, no matter how small the growth rate \(c\), will eventually overtake any given polynomial \(P(x)\), no matter how large.
\end{exercise}

\begin{exercise}\label{ex 4.5.10}
    Let \(f : (0, +\infty) \times \mathbf{R} \to \mathbf{R}\) be the exponential function \(f(x, y) \coloneqq x^y\).
    Show that \(f\) is continuous.
\end{exercise}