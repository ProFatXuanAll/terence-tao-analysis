\section{Trigonometric functions}\label{sec 4.7}

\begin{note}
    There are several other useful special functions in mathematics, such as the hyperbolic trigonometric functions and hypergeometric functions, the gamma and zeta functions, and elliptic functions, but they occur more rarely than trigonometric functions.
\end{note}

\begin{note}
    Trigonometric functions are often defined using geometric concepts, notably those of circles, triangles, and angles.
    However, it is also possible to define them using more analytic concepts, and in particular the (complex) exponential function.
\end{note}

\begin{definition}[Trigonometric functions]\label{4.7.1}
    If \(z\) is a complex number, then we define
    \[
        \cos(z) \coloneqq \frac{e^{iz} + e^{-iz}}{2}
    \]
    and
    \[
        \sin(z) \coloneqq \frac{e^{iz} - e^{-iz}}{2i}.
    \]
    We refer to \(\cos\) and \(\sin\) as the \emph{cosine} and \emph{sine} functions respectively.
\end{definition}

\begin{note}
    Definition \ref{4.7.1} were discovered by Leonhard Euler (1707 -- 1783) in 1748, who recognized the link between the complex exponential and the trigonometric functions.
    Note that since we have defined the sine and cosine for complex numbers \(z\), we automatically have defined them also for real numbers \(x\).
    In fact in most applications one is only interested in the trigonometric functions when applied to real numbers.
\end{note}

\begin{additional corollary}\label{ac 4.7.1}
From the Definition \ref{4.6.15}, we have
\[
    e^{i z} = 1 + i z - \frac{z^2}{2!} - \frac{i z^3}{3!} + \frac{z^4}{4!} + \dots
\]
and
\[
    e^{- i z} = 1 - i z - \frac{z^2}{2!} + \frac{i z^3}{3!} + \frac{z^4}{4!} - \dots
\]
and so from the above formulae we have
\[
    \cos(z) = 1 - \frac{z^2}{2!} + \frac{z^4}{4!} - \dots = \sum_{n = 0}^\infty \frac{(-1)^n z^{2n}}{(2n)!}
\]
and
\[
    \sin(z) = z - \frac{z^3}{3!} + \frac{z^5}{5!} - \dots = \sum_{n = 0}^\infty \frac{(-1)^n z^{2n + 1}}{(2n + 1)!}.
\]
In particular, \(\cos(x)\) and \(\sin(x)\) are always real whenever \(x\) is real.
From the ratio test we see that the two power series \(\sum_{n = 0}^\infty \frac{(-1)^n x^{2n}}{(2n)!}\), \(\sum_{n = 0}^\infty \frac{(-1)^n x^{2n + 1}}{(2n + 1)!}\) are absolutely convergent for every \(x\), thus \(\sin(x)\) and \(\cos(x)\) are real analytic at \(0\) with an infinite radius of convergence.
From Exercise \ref{ex 4.2.8} we thus see that the sine and cosine functions are real analytic on all of \(\mathbf{R}\).
(They are also complex analytic on all of \(\mathbf{C}\), but we will not pursue this matter in this text.)
In particular the sine and cosine functions are continuous and differentiable (see Proposition \ref{4.2.6}).
\end{additional corollary}

\begin{theorem}[Trigonometric identities]\label{4.7.2}
    Let \(x, y\) be real numbers.
    \begin{enumerate}
        \item We have \(\big(\sin(x)\big)^2 + \big(\cos(x)\big)^2 = 1\).
              In particular, we have \(\sin(x) \in [-1, 1]\) and \(\cos(x) \in [-1, 1]\) for all \(x \in \mathbf{R}\).
        \item We have \(\sin'(x) = \cos(x)\) and \(\cos'(x) = -\sin(x)\).
        \item We have \(\sin(-x) = -\sin(x)\) and \(\cos(-x) = \cos(x)\).
        \item We have \(\cos(x + y) = \cos(x) \cos(y) - \sin(x) \sin(y)\) and \(\sin(x + y) = \sin(x) \cos(y) + \cos(x) \sin(y)\).
        \item We have \(\sin(0) = 0\) and \(\cos(0) = 1\).
        \item We have \(e^{i x} = \cos(x) + i \sin(x)\) and \(e^{- i x} = \cos(x) - i \sin(x)\).
              In particular \(\cos(x) = \Re(e^{i x})\) and \(\sin(x) = \Im(e^{i x})\).
    \end{enumerate}
\end{theorem}

\begin{lemma}\label{4.7.3}
    There exists a positive number \(x\) such that \(\sin(x)\) is equal to \(0\).
\end{lemma}

\begin{proof}
    Suppose for sake of contradiction that \(\sin(x) \neq 0\) for all \(x \in (0, \infty)\).
    Observe that this would also imply that \(\cos(x) \neq 0\) for all \(x \in (0, \infty)\), since if \(\cos(x) = 0\) then \(\sin(2x) = 0\) by Theorem \ref{4.7.2}(d).
    Since \(\cos(0) = 1\), this implies by the intermediate value theorem (Theorem 9.7.1 in Analysis I) that \(\cos(x) > 0\) for all \(x > 0\) (why?).
    Also, since \(\sin(0) = 0\) and \(\sin'(0) = 1 > 0\), we see that \(\sin\) increasing near \(0\), hence is positive to the right of \(0\).
    By the intermediate value theorem again we conclude that \(\sin(x) > 0\) for all \(x > 0\)
    (otherwise \(\sin\) would have a zero on \((0, \infty)\)).

    In particular if we define the cotangent function \(\cot(x) \coloneqq \cos(x) / \sin(x)\), then \(\cot(x)\) would be positive and differentiable on all of \((0, \infty)\).
    From the quotient rule (Theorem 10.1.13(h) in Analysis I) and Theorem \ref{4.7.2} we see that the derivative of \(\cot(x)\) is \(-1 / \sin(x)^2\) (why?).
    In particular, we have \(\cot'(x) \leq -1\) for all \(x > 0\).
    By the fundamental theorem of calculus (Theorem 11.9.1 in Analysis I) this implies that \(\cot(x + s) \leq \cot(x) - s\) for all \(x > 0\) and \(s > 0\).
    But letting \(s \to \infty\) we see that this contradicts our assertion that \(\cot\) is positive on \((0, \infty)\) (why?).
\end{proof}

\begin{note}
    Let \(E\) be the set \(E \coloneqq \{x \in (0, +\infty) : \sin(x) = 0\}\), i.e., \(E\) is the set of roots of \(\sin\) on \((0, +\infty)\).
    By Lemma \ref{4.7.3}, \(E\) is non-empty.
    Since \(\sin'(0) > 0\), there exists a \(c > 0\) such that \(E \subseteq [c, +\infty)\) (see Exercise \ref{ex 4.7.2}).
    Also, since \(\sin\) is continuous in \([c, +\infty)\), \(E\) is closed in \([c, +\infty)\) (why? use Theorem \ref{2.1.5}(d)).
    Since \([c, +\infty)\) is closed in \(\mathbf{R}\), we conclude that \(E\) is closed in \(\mathbf{R}\).
    Thus \(E\) contains all its adherent points, and thus contains \(\inf(E)\).
\end{note}

\begin{definition}\label{4.7.4}
    We define \(\pi\) to be the number
    \[
        \pi \coloneqq \inf\{x \in (0, \infty) : \sin(x) = 0\}.
    \]
\end{definition}

\begin{additional corollary}\label{ac 4.7.2}
We have \(\pi \in E \subseteq [c, +\infty)\) (so in particular \(\pi > 0\)) and \(\sin(\pi) = 0\).
By definition of \(\pi\), \(\sin\) cannot have any zeroes in \((0, \pi)\), and so in particular must be positive on \((0, \pi)\)
(cf. the arguments in Lemma \ref{4.7.3} using the intermediate value theorem).
Since \(\cos'(x) = -\sin(x)\), we thus conclude that \(\cos(x)\) is strictly decreasing on \((0, \pi)\).
Since \(\cos(0) = 1\), this implies in particular that \(\cos(\pi) < 1\);
since \(\big(\sin(\pi)\big)^2 + \big(\cos(\pi)\big)^2 = 1\) and \(\sin(\pi) = 0\), we thus conclude that \(\cos(\pi) = -1\).
\end{additional corollary}

\begin{additional corollary}[Euler's formula]\label{ac 4.7.3}
\[
    e^{\pi i} = \cos(\pi) + i \sin(\pi) = -1.
\]
\end{additional corollary}

\begin{theorem}[Periodicity of trigonometric functions]\label{4.7.5}
    Let \(x\) be a real number.
    \begin{enumerate}
        \item We have \(\cos(x + \pi) = -\cos(x)\) and \(\sin(x + \pi) = -\sin(x)\).
              In particular we have \(\cos(x + 2\pi) = \cos(x)\) and \(\sin(x + 2\pi) = \sin(x)\), i.e., \(\sin\) and \(\cos\) are periodic with period \(2\pi\).
        \item We have \(\sin(x) = 0\) if and only if \(x / \pi\) is an integer.
        \item We have \(\cos(x) = 0\) if and only if \(x / \pi\) is an integer plus \(1 / 2\).
    \end{enumerate}
\end{theorem}

\exercisesection

\begin{exercise}\label{ex 4.7.1}
    Prove Theorem \ref{4.7.2}.
\end{exercise}

\begin{proof}
    See Theorem \ref{4.7.2}.
\end{proof}

\begin{exercise}\label{ex 4.7.2}
    Let \(f : \mathbf{R} \to \mathbf{R}\) be a function which is differentiable at \(x_0\), with \(f(x_0) = 0\) and \(f'(x_0) \neq 0\).
    Show that there exists a \(c > 0\) such that \(f(y)\) is non-zero whenever \(0 < \abs*{x_0 - y} < c\).
    Conclude in particular that there exists a \(c > 0\) such that \(\sin(x) \neq 0\) for all \(0 < x < c\).
\end{exercise}

\begin{exercise}\label{ex 4.7.3}
    Prove Theorem \ref{4.7.5}.
\end{exercise}

\begin{proof}
    See Theorem \ref{4.7.5}.
\end{proof}

\begin{exercise}\label{ex 4.7.4}
    Let \(x, y\) be real numbers such that \(x^2 + y^2 = 1\).
    Show that there is exactly one real number \(\theta \in (-\pi, \pi]\) such that \(x = \sin(\theta)\) and \(y = \cos(\theta)\).
\end{exercise}

\begin{exercise}\label{ex 4.7.5}
    Show that if \(r, s > 0\) are positive real numbers, and \(\theta, \alpha\) are real numbers such that \(r e^{i \theta} = s e^{i \alpha}\), then \(r = s\) and \(\theta = \alpha + 2 \pi k\) for some integer \(k\).
\end{exercise}

\begin{exercise}\label{ex 4.7.6}
    Let \(z\) be a non-zero complex number.
    Using Exercise \ref{ex 4.7.4}, show that there is exactly one pair of real numbers \(r, \theta\) such that \(r > 0\), \(\theta \in (-\pi, \pi]\), and \(z = r e^{i \theta}\).
    (This is sometimes known as the \emph{standard polar representation} of \(z\).)
\end{exercise}

\begin{exercise}\label{ex 4.7.7}
    For any real number \(\theta\) and integer \(n\), prove the \emph{de Moivre identities}
    \[
        \cos(n \theta) = \Re\big((\cos \theta + i \sin \theta)^n\big); \quad \sin(n \theta) = \Im\big((\cos \theta + i \sin \theta)^n\big).
    \]
\end{exercise}

\begin{exercise}\label{ex 4.7.8}
    Let \(\tan : (- \pi / 2, \pi / 2) \to \mathbf{R}\) be the tangent function \(\tan(x) \coloneqq \sin(x) / \cos(x)\).
    Show that \(\tan\) is differentiable and monotone increasing, with
    \[
        \frac{d}{dx} \tan(x) = 1 + \big(\tan(x)\big)^2,
    \]
    and that \(\lim_{x \to \pi / 2} \tan(x) = +\infty\) and \(\lim_{x \to -\pi / 2} \tan(x) = -\infty\).
    Conclude that \(\tan\) is in fact a bijection from \((- \pi / 2, \pi / 2) \to \mathbf{R}\), and thus has an inverse function \(\tan^{-1} : \mathbf{R} \to (- \pi / 2, \pi / 2)\)
    (this function is called the \emph{arctangent function}).
    Show that \(\tan^{-1}\) is differentiable and \(\frac{d}{dx} \tan^{-1}(x) = \frac{1}{1 + x^2}\).
\end{exercise}

\begin{exercise}\label{ex 4.7.9}
    Recall the arctangent function \(\tan^{-1}\) from Exercise \ref{ex 4.7.8}.
    By modifying the proof of Theorem \ref{4.5.6}(e), establish the identity
    \[
        \tan^{-1}(x) = \sum_{n = 0}^\infty \frac{(-1)^n x^{2n + 1}}{2n + 1}
    \]
    for all \(x \in (-1, 1)\).
    Using Abel's theorem (Theorem \ref{4.3.1}) to extend this identity to the case \(x = 1\), conclude in particular the identity
    \[
        \pi = 4 - \frac{4}{3} + \frac{4}{5} - \frac{4}{7} + \dots = 4 \sum_{n = 0}^\infty \frac{(-1)^n}{2n + 1}.
    \]
    (Note that the series converges by the alternating series test, Proposition 7.2.12 in Analysis I.)
    Conclude in particular that \(4 - \frac{4}{3} < \pi < 4\).
    (One can of course compute \(\pi = 3.1415926 \dots\) to much higher accuracy, though if one wishes to do so it is advisable to use a different formula than the one above, which converges very slowly.)
\end{exercise}

\begin{exercise}\label{ex 4.7.10}
    Let \(f : \mathbf{R} \to \mathbf{R}\) be the function
    \[
        f(x) \coloneqq \sum_{n = 1}^\infty 4^{-n} \cos(32^n \pi x).
    \]
    \begin{enumerate}
        \item Show that this series is uniformly convergent, and that \(f\) is continuous.
        \item Show that for every integer \(j\) and every integer \(m \geq 1\), we have
              \[
                  \abs*{f\bigg(\frac{j + 1}{32^m}\bigg) - f\bigg(\frac{j}{32^m}\bigg)} \geq 4^{-m}.
              \]
        \item Using (b), show that for every real number \(x_0\), the function \(f\) is not differentiable at \(x_0\).
        \item Explain briefly why the result in (c) does not contradict Corollary \ref{3.7.3}.
    \end{enumerate}
\end{exercise}