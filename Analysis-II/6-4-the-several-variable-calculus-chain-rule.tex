\section{The several variable calculus chain rule}\label{ii:sec:6.4}

\begin{thm}[Several variable calculus chain rule]\label{ii:6.4.1}
  Let \(E\) be a subset of \(\R^n\), and let \(F\) be a subset of \(\R^m\).
  Let \(f : E \to F\) be a function, and let \(g : F \to \R^p\) be another function.
  Let \(x_0\) be a point in the interior of \(E\).
  Suppose that \(f\) is differentiable at \(x_0\), and that \(f(x_0)\) is in the interior of \(F\).
  Suppose also that \(g\) is differentiable at \(f(x_0)\).
  Then \(g \circ f : E \to \R^p\) is also differentiable at \(x_0\), and we have the formula
  \[
    (g \circ f)'(x_0) = g'\big(f(x_0)\big) \circ f'(x_0).
  \]
\end{thm}

\begin{proof}
  By \cref{ii:6.2.2} we want to show that
  \[
    \lim_{x \to x_0 ; x \in E \setminus \set{x_0}} \dfrac{\norm*{(g \circ f)(x) - (g \circ f)(x_0) - \Big(g'\big(f(x_0)\big) \circ f'(x_0)\Big)(x - x_0)}}{\norm*{x - x_0}} = 0.
  \]
  Equivalently, we want to show that
  \begin{align*}
             & \forall \varepsilon \in \R^+, \exists \delta \in \R^+ : \forall x \in E \setminus \set{x_0}, \norm*{x - x_0} < \delta                  \\
    \implies & \dfrac{\norm*{(g \circ f)(x) - (g \circ f)(x_0) - \Big(g'\big(f(x_0)\big) \circ f'(x_0)\Big)(x - x_0)}}{\norm*{x - x_0}} < \varepsilon \\
    \implies & \norm*{(g \circ f)(x) - (g \circ f)(x_0) - \Big(g'\big(f(x_0)\big) \circ f'(x_0)\Big)(x - x_0)} < \varepsilon \norm*{x - x_0}.
  \end{align*}
  Since \(f'(x_0)\) exists, we know that
  \[
    \lim_{x \to x_0 ; x \in E \setminus \set{x_0}} \dfrac{\norm*{f(x) - f(x_0) - f'(x_0)(x - x_0)}}{\norm*{x - x_0}} = 0.
  \]
  Equivalently, we know that
  \begin{align*}
             & \forall \varepsilon_f \in \R^+, \exists \delta \in \R^+ : \forall x \in E \setminus \set{x_0}, \norm*{x - x_0} < \delta \\
    \implies & \dfrac{\norm*{f(x) - f(x_0) - f'(x_0)(x - x_0)}}{\norm*{x - x_0}} < \varepsilon_f                                       \\
    \implies & \norm*{f(x) - f(x_0) - f'(x_0)(x - x_0)} < \varepsilon_f \norm*{x - x_0}                                                \\
    \implies & \norm*{f(x) - f(x_0)} < \varepsilon_f \norm*{x - x_0} + \norm*{f'(x_0)(x - x_0)}.
  \end{align*}
  Since \(g'\big(f(x_0)\big)\) exists, we know that
  \[
    \lim_{y \to f(x_0) ; x \in F \setminus \set{f(x_0)}} \dfrac{\norm*{g(y) - g\big(f(x_0)\big) - g'\big(f(x_0)\big)\big(y - f(x_0)\big)}}{\norm*{y - f(x_0)}} = 0.
  \]
  Equivalently, we know that
  \begin{align*}
             & \forall \varepsilon \in \R^+, \exists \delta_g \in \R^+ : \forall y \in F \setminus \set{f(x_0)}, \norm*{y - f(x_0)} < \delta_g \\
    \implies & \dfrac{\norm*{g(y) - g\big(f(x_0)\big) - g'\big(f(x_0)\big)\big(y - f(x_0)\big)}}{\norm*{y - f(x_0)}} < \varepsilon             \\
    \implies & \norm*{g(y) - g\big(f(x_0)\big) - g'\big(f(x_0)\big)\big(y - f(x_0)\big)} < \varepsilon \norm*{y - f(x_0)}.
  \end{align*}
  Fix one pair of \(\varepsilon\) and \(\delta_g\).
  Then we have
  \begin{align*}
             & \exists \delta \in \R^+ : \forall x \in E \setminus \set{x_0}, \norm*{x - x_0} < \delta                                      \\
    \implies & \begin{dcases}
                 \norm*{f(x) - f(x_0)} < \varepsilon \norm*{x - x_0} + \norm*{f'(x_0)(x - x_0)}; \\
                 \norm*{f(x) - f(x_0)} < \delta_g;
               \end{dcases}                                              \\
    \implies & \begin{dcases}
                 \norm*{f(x) - f(x_0)} < \varepsilon \norm*{x - x_0} + \norm*{f'(x_0)(x - x_0)}; \\
                 \norm*{g\big(f(x)\big) - g\big(f(x_0)\big) - g'\big(f(x_0)\big) \big(f(x) - f(x_0)\big)} < \varepsilon \norm*{f(x) - f(x_0)};
               \end{dcases} \\
    \implies & \norm*{g\big(f(x)\big) - g\big(f(x_0)\big) - g'\big(f(x_0)\big) \big(f(x) - f(x_0)\big)}                                     \\
             & < \varepsilon^2 \norm*{x - x_0} + \varepsilon \norm*{f'(x_0)(x - x_0)}                                                       \\
    \implies & \norm*{g\big(f(x)\big) - g\big(f(x_0)\big) - \Big(g'\big(f(x_0)\big) \circ f'(x_0)\Big) (x - x_0)}                           \\
             & < \varepsilon^2 \norm*{x - x_0} + \varepsilon \norm*{f'(x_0)(x - x_0)}                                                       \\
             & \quad + \norm*{g'\big(f(x_0)\big) \big(f(x) - f(x_0)\big) - \Big(g'\big(f(x_0)\big) \circ f'(x_0)\Big) (x - x_0)}.
  \end{align*}
  Since \(g'\big(f(x_0)\big)\) is a linear transformation, by \cref{ii:6.1.6} we know that
  \begin{align*}
     & \norm*{g'\big(f(x_0)\big) \big(f(x) - f(x_0)\big) - \Big(g'\big(f(x_0)\big) \circ f'(x_0)\Big) (x - x_0)} \\
     & = \norm*{g'\big(f(x_0)\big) \big(f(x) - f(x_0) - f'(x_0) (x - x_0)\big)}.
  \end{align*}
  By \cref{ii:ex:6.1.4} we know that
  \begin{align*}
    \exists M \in \R^+ : & \norm*{g'\big(f(x_0)\big) \big(f(x) - f(x_0) - f'(x_0) (x - x_0)\big)} \\
                         & \leq M \norm*{f(x) - f(x_0) - f'(x_0) (x - x_0)}                       \\
                         & \leq M \varepsilon \norm*{x - x_0}.
  \end{align*}
  Fix such \(M\).
  Then we have
  \begin{align*}
             & \exists \delta \in \R^+ : \forall x \in E \setminus \set{x_0}, \norm*{x - x_0} < \delta                 \\
    \implies & \norm*{g\big(f(x)\big) - g\big(f(x_0)\big) - \Big(g'\big(f(x_0)\big) \circ f'(x_0)\Big) (x - x_0)}      \\
             & < \varepsilon^2 \norm*{x - x_0} + \varepsilon \norm*{f'(x_0)(x - x_0)} + M \varepsilon \norm*{x - x_0}.
  \end{align*}
  Since \(\varepsilon\) is arbitrary, we conclude that
  \begin{align*}
             & \forall \varepsilon \in \R^+, \exists \delta \in \R^+ : \forall x \in E \setminus \set{x_0}, \norm*{x - x_0} < \delta \\
    \implies & \norm*{g\big(f(x)\big) - g\big(f(x_0)\big) - \Big(g'\big(f(x_0)\big) \circ f'(x_0)\Big) (x - x_0)} < \varepsilon.
  \end{align*}
\end{proof}

\begin{note}
  As a corollary of the chain rule and \cref{ii:6.1.16} (and \cref{ii:6.1.13}), we see that
  \[
    D (g \circ f)(x_0) = D g\big(f(x_0)\big) \cdot D f(x_0);
  \]
  i.e., we can write the chain rule in terms of matrices and matrix multiplication, instead of in terms of linear transformations and composition.
\end{note}

\begin{eg}\label{ii:6.4.2}
  Let \(f : \R^n \to \R\) and \(g : \R^n \to \R\) be differentiable functions.
  We form the combined function \(h : \R^n \to \R^2\) by defining \(h(x) \coloneqq \big(f(x), g(x)\big)\).
  Now let \(k : \R^2 \to \R\) be the multiplication function \(k(a, b) \coloneqq ab\).

  We first show that
  \[
    D h(x_0) = \begin{pmatrix}
      \nabla f(x_0) \\
      \nabla g(x_0)
    \end{pmatrix}.
  \]
  Let \(x_0 \in \R^n\).
  Since
  \begin{align*}
     & \norm*{h(x) - h(x_0) - (x - x_0) D h(x_0)^\top}                                                                                  \\
     & = \norm*{\big(f(x), g(x)\big) - \big(f(x_0), g(x_0)\big) - \big((x - x_0) \nabla f(x_0)^\top, (x - x_0) \nabla g(x_0)^\top\big)} \\
     & = \norm*{\big(f(x) - f(x_0) - (x - x_0) \nabla f(x_0)^\top, g(x) - g(x_0) - (x - x_0) \nabla g(x_0)^\top\big)}                   \\
     & \leq \norm*{f(x) - f(x_0) - (x - x_0) \nabla f(x_0)^\top} + \norm*{g(x) - g(x_0) - (x - x_0) \nabla g(x_0)^\top}
  \end{align*}
  (note that the last line follow by \cref{ii:ex:1.1.8}),
  by squeeze test we know that
  \begin{align*}
             & \begin{dcases}
                 \lim_{x \to x_0 ; x \in \R^n \setminus \set{x_0}} \dfrac{\norm*{f(x) - f(x_0) - f'(x_0)(x - x_0)}}{\norm*{x - x_0}} = 0 \\
                 \lim_{x \to x_0 ; x \in \R^n \setminus \set{x_0}} \dfrac{\norm*{g(x) - g(x_0) - g'(x_0)(x - x_0)}}{\norm*{x - x_0}} = 0
               \end{dcases}       \\
    \implies & \begin{dcases}
                 \lim_{x \to x_0 ; x \in \R^n \setminus \set{x_0}} \dfrac{\norm*{f(x) - f(x_0) - (x - x_0) \nabla f(x_0)^\top}}{\norm*{x - x_0}} = 0 \\
                 \lim_{x \to x_0 ; x \in \R^n \setminus \set{x_0}} \dfrac{\norm*{g(x) - g(x_0) - (x - x_0) \nabla g(x_0)^\top}}{\norm*{x - x_0}} = 0
               \end{dcases}       \\
    \implies & \lim_{x \to x_0 ; x \in \R^n \setminus \set{x_0}} \dfrac{\norm*{h(x) - h(x_0) - (x - x_0) D h(x_0)^\top}}{\norm*{x - x_0}} = 0.
  \end{align*}
  Since \(x_0\) is arbitrary, we conclude that the identity is true.

  Now we show that
  \[
    D k(a, b) = (b, a).
  \]
  Let \((a, b) \in \R^2\).
  Observe that for any \((x, y) \in \R^2 \setminus \set{(a, b)}\), we have
  \begin{align*}
     & \dfrac{\norm*{k(x, y) - k(a, b) - \big((x, y) - (a, b)\big) (b, a)^\top}}{\norm*{(x, y) - (a, b)}} \\
     & = \dfrac{\norm*{xy - ab - xb - ay + ab + ba}}{\norm*{(x - a, y - b)}}                              \\
     & = \dfrac{\norm*{(x - a)(y - b)}}{\norm*{(x - a, y - b)}}                                           \\
     & = \sqrt{\dfrac{(x - a)^2 (y - b)^2}{(x - a)^2 + (y - b)^2}}                                        \\
     & \leq \sqrt{\dfrac{2 (x - a)^2 (y - b)^2}{(x - a)^2 + (y - b)^2}}                                   \\
     & \leq \sqrt{\dfrac{(x - a)^4 + 2 (x - a)^2 (y - b)^2 + (y - b)^4}{(x - a)^2 + (y - b)^2}}           \\
     & = \sqrt{\dfrac{\big((x - a)^2 + (y - b)^2\big)^2}{(x - a)^2 + (y - b)^2}}                          \\
     & = \sqrt{(x - a)^2 + (y - b)^2}.
  \end{align*}
  Since
  \begin{align*}
             & \lim_{(x, y) \to (a, b) ; (x, y) \in \R^2 \setminus \set{(a, b)}} (x - a)^2 + (y - b)^2 = 0                                                                              \\
    \implies & \lim_{(x, y) \to (a, b) ; (x, y) \in \R^2 \setminus \set{(a, b)}} \sqrt{(x - a)^2 + (y - b)^2} = 0                                                                       \\
    \implies & \lim_{(x, y) \to (a, b) ; (x, y) \in \R^2 \setminus \set{(a, b)}} \dfrac{\norm*{k(x, y) - k(a, b) - \big((x, y) - (a, b)\big) (b, a)^\top}}{\norm*{(x, y) - (a, b)}} = 0
  \end{align*}
  (note that the last line follow by squeeze test),
  by \cref{ii:6.2.2} we know that the identity is true.

  By the chain rule, we thus see that
  \[
    D (k \circ h)(x_0) = \big(g(x_0), f(x_0)\big) \begin{pmatrix}
      \nabla f(x_0) \\
      \nabla g(x_0)
    \end{pmatrix} = g(x_0) \nabla f(x_0) + f(x_0) \nabla g(x_0).
  \]
  But \(k \circ h = fg\), and \(D (fg) = \nabla (fg)\).
  We have thus proven the \emph{product rule}
  \[
    \nabla (fg) = g \nabla f + f \nabla g.
  \]
  A similar argument gives the sum rule \(\nabla (f + g) = \nabla f + \nabla g\), or the difference rule \(\nabla (f - g) = \nabla f - \nabla g\), as well as the quotient rule (\cref{ii:ex:6.4.4}).
\end{eg}

\begin{note}
  We do record one further useful application of the chain rule.
  Let \(T : \R^n \to \R^m\) be a linear transformation.
  From \cref{ii:ex:6.4.1} we observe that \(T\) is continuously differentiable at every point, and in fact \(T'(x) = T\) for every \(x\).
  (This equation may look a little strange, but perhaps it is easier to swallow if you view it in the form \(\dfrac{d}{dx} (Tx) = T\).)
  Thus, for any differentiable function \(f : E \to \R^n\), we see that \(T f : E \to \R^m\) is also differentiable, and hence by the chain rule
  \[
    (T f)'(x_0) = T\big(f'(x_0)\big).
  \]
  This is a generalization of the single-variable calculus rule \((cf)' = c(f')\) for constant scalars \(c\).
\end{note}

\begin{ac}\label{ii:ac:6.4.1}
  If \(f : \R^n \to \R^m\) is some differentiable function, and \(x_j : \R \to \R\) are differentiable functions for each \(j = 1, \dots n\), then
  \[
    \dfrac{d}{dt} f\big(x_1(t), x_2(t), \dots, x_n(t)\big) = \sum_{j = 1}^n x_j'(t) \dfrac{\partial f}{\partial x_j} \big(x_1(t), x_2(t), \dots, x_n(t)\big).
  \]
\end{ac}

\begin{proof}
  Let \(h : \R \to \R^n\) be the function
  \[
    \forall t \in \R, h(t) = \big(x_1(t), \dots, x_n(t)\big).
  \]
  We claim that \(h\) is differentiable on \(\R\), and
  \[
    \forall t \in \R, D h(t) = \big(x_1'(t), \dots, x_n'(t)\big)^\top.
  \]
  Let \(t_0 \in \R\).
  Since
  \begin{align*}
     & \dfrac{\norm*{h(t) - h(t_0) - (t - t_0) D_h(t_0)^\top}}{\norm*{t - t_0}}                                                                                  \\
     & = \dfrac{\norm*{\big(x_1(t), \dots, x_n(t)\big) - \big(x_1(t_0), \dots, x_n(t_0)\big) - (t - t_0) \big(x_1'(t), \dots, x_n'(t)\big)^\top}}{\abs{t - t_0}} \\
     & = \dfrac{\norm*{\big(x_1(t) - x_1(t_0) - x_1'(t_0)(t - t_0), \dots, x_n(t) - x_n(t_0) - x_n'(t_0)(t - t_0)\big)}}{\abs{t - t_0}}                          \\
     & \leq \dfrac{\sum_{i = 1}^n \abs{\big(x_i(t) - x_i(t_0) - x_i'(t_0)(t - t_0)\big)}}{\abs{t - t_0}}                                                         \\
     & = \sum_{i = 1}^n \dfrac{\abs{\big(x_i(t) - x_i(t_0) - x_i'(t_0)(t - t_0)\big)}}{\abs{t - t_0}},
  \end{align*}
  we know that
  \begin{align*}
             & \forall 1 \leq i \leq n, \lim_{t \to t_0 ; t \in \R \setminus \set{t_0}} \dfrac{\abs{x_i(t) - x_i(t_0) - x_i'(t_0)(x - x_0)}}{\abs{t - t_0}} = 0                               \\
    \implies & \lim_{t \to t_0 ; t \in \R \setminus \set{t_0}} \sum_{i = 1}^n \dfrac{\abs{x_i(t) - x_i(t_0) - x_i'(t_0)(x - x_0)}}{\abs{t - t_0}} = 0           &  & \text{(by limit laws)}   \\
    \implies & \lim_{t \to t_0 ; t \in \R \setminus \set{t_0}} \dfrac{\norm*{h(t) - h(t_0) - (t - t_0) D_h(t_0)^\top}}{\norm*{t - t_0}}.                        &  & \text{(by squeeze test)}
  \end{align*}
  Since \(t_0\) is arbitrary, by \cref{ii:6.2.2} we know that
  \[
    \forall t \in \R, D h(t) = \big(x_1'(t), \dots, x_n'(t)\big)^\top.
  \]
  Using chain rule we have
  \begin{align*}
    \forall t \in \R, & D (f \circ h)(t)                                                                                             \\
                      & = D f\big(h(t)\big) \cdot D h(t)                                                          &  & \by{ii:6.4.1}    \\
                      & = D f\big(x_1(t), \dots, x_n(t)\big) \cdot \big(x_1'(t), \dots, x_n'(t)\big)^\top                            \\
                      & = D_{\big(x_1'(t), \dots, x_n'(t)\big)} f\big(x_1(t), \dots, x_n(t)\big)                  &  & \by{ii:6.3.5}    \\
                      & = \sum_{j = 1}^n x_j'(t) \dfrac{\partial f}{\partial x_j}\big(x_1(t), \dots, x_n(t)\big). &  & \by{ii:ac:6.3.2}
  \end{align*}
\end{proof}

\exercisesection

\begin{ex}\label{ii:ex:6.4.1}
  Let \(T : \R^n \to \R^m\) be a linear transformation.
  Show that \(T\) is continuously differentiable at every point, and in fact \(T'(x) = T\) for every \(x\).
  What is \(D T\)?
\end{ex}

\begin{proof}
  Let \(x_0 \in \R^n\).
  Since
  \begin{align*}
     & \lim_{x \to x_0 ; x \in \R^n \setminus \set{x_0}} \dfrac{\norm*{T(x) - T(x_0) - T(x - x_0)}}{\norm*{x - x_0}}                      \\
     & = \lim_{x \to x_0 ; x \in \R^n \setminus \set{x_0}} \dfrac{\norm*{T(x) - T(x_0) - T(x) - T(x_0)}}{\norm*{x - x_0}} &  & \by{ii:6.1.6} \\
     & = \lim_{x \to x_0 ; x \in \R^n \setminus \set{x_0}} \dfrac{\norm*{0_{\R^m}}}{\norm*{x - x_0}}                                      \\
     & = 0
  \end{align*}
  and \(x_0\) is arbitrary, we conclude by \cref{ii:6.2.2} that
  \[
    \forall x \in \R^n, T'(x) = T.
  \]
  By \cref{ii:6.1.13} we know that
  \[
    \forall x \in \R^n, D T(x) = \big(T(e_1)^\top, \dots, T(e_n)^\top\big).
  \]
\end{proof}

\begin{ex}\label{ii:ex:6.4.2}
  Let \(E\) be a subset of \(\R^n\).
  Prove that if a function \(f : E \to \R^m\) is differentiable at an interior point \(x_0\) of \(E\), then it is also continuous at \(x_0\).
\end{ex}

\begin{proof}
  By \cref{ii:6.2.2} we have
  \begin{align*}
             & \forall \varepsilon \in \R^+, \exists \delta_1 \in \R^+ : \forall x \in E \setminus \set{x_0},                                             \\
             & \bigg(\norm*{x - x_0} < \delta_1 \implies \dfrac{\norm*{f(x) - f(x_0) - f'(x_0)(x - x_0)}}{\norm*{x - x_0}} < \dfrac{\varepsilon}{2}\bigg) \\
    \implies & \forall \varepsilon \in \R^+, \exists \delta_1 \in \R^+ : \forall x \in E \setminus \set{x_0},                                             \\
             & \bigg(\norm*{x - x_0} < \delta_1 \implies \norm*{f(x) - f(x_0) - f'(x_0)(x - x_0)} < \dfrac{\varepsilon}{2} \norm*{x - x_0}\bigg)          \\
    \implies & \forall \varepsilon \in \R^+, \exists \delta_1 \in \R^+ : \forall x \in E \setminus \set{x_0},                                             \\
             & \bigg(\norm*{x - x_0} < \delta_1 \implies \norm*{f(x) - f(x_0)} < \dfrac{\varepsilon}{2} \norm*{x - x_0} + \norm*{f'(x_0)(x - x_0)}\bigg)  \\
    \implies & \forall \varepsilon \in \R^+, \exists \delta_1 \in \R^+ : \forall x \in E,                                                                 \\
             & \bigg(\norm*{x - x_0} < \delta_1 \implies \norm*{f(x) - f(x_0)} < \dfrac{\varepsilon}{2} \norm*{x - x_0} + \norm*{f'(x_0)(x - x_0)}\bigg).
  \end{align*}
  By \cref{ii:ex:6.1.4} we know that every linear transformation is continuous.
  Thus we know that
  \begin{align*}
             & \forall \varepsilon \in \R^+, \exists \delta_2 \in \R^+ : \forall x \in E,                                                 \\
             & \bigg(\norm*{x - x_0} < \delta_2 \implies \norm*{f'(x_0)(x) - f'(x_0)(x_0)} < \dfrac{\varepsilon}{2}\bigg)                 \\
    \implies & \forall \varepsilon \in \R^+, \exists \delta_2 \in \R^+ : \forall x \in E,                                                 \\
             & \bigg(\norm*{x - x_0} < \delta_2 \implies \norm*{f'(x_0)(x - x_0)} < \dfrac{\varepsilon}{2}\bigg).         &  & \by{ii:6.1.6}
  \end{align*}
  Let \(\delta = \min(\delta_1, \delta_2, 1)\).
  Then we have
  \begin{align*}
             & \forall \varepsilon \in \R^+, \exists \delta \in \R^+ : \forall x \in E, \norm*{x - x_0} < \delta                                                          \\
    \implies & \norm*{f(x) - f(x_0)} < \dfrac{\varepsilon}{2} \norm*{x - x_0} + \norm*{f'(x_0)(x - x_0)} < \dfrac{\varepsilon}{2} + \dfrac{\varepsilon}{2} = \varepsilon.
  \end{align*}
  Thus \(f\) is continuous at \(x_0\) from \((E, d_{l^2}|_{E \times E})\) to \((\R^m, d_{l^2}|_{\R^m \times \R^m})\).
\end{proof}

\begin{ex}\label{ii:ex:6.4.3}
  Prove \cref{ii:6.4.1}.
\end{ex}

\begin{proof}
  See \cref{ii:6.4.1}.
\end{proof}

\begin{ex}\label{ii:ex:6.4.4}
  State and prove some version of the quotient rule for functions of several variables (i.e., functions of the form \(f : E \to \R\) for some subset \(E\) of \(\R^n\)).
  In other words, state a rule which gives a formula for the gradient of \(f / g\);
  compare your answer with Theorem 10.1.13(h) in Analysis I.
  Be sure to make clear what all your assumptions are.
\end{ex}

\begin{proof}
  Let \(E \subseteq \R^n\) and let \(x_0\) be an interior point of \(E\).
  Let \(f : E \to \R\) and \(g : E \to \R\) be functions where \(g(x) \neq 0\) for all \(x \in E\).
  If \(f, g\) are differentiable at \(x_0\), then \(f / g\) is differentiable at \(x_0\), and
  \[
    \nabla \bigg(\dfrac{f}{g}\bigg)(x_0) = \dfrac{g(x_0) \nabla f(x_0) - f(x_0) \nabla g(x_0)}{\big(g(x_0)\big)^2}.
  \]
  If \(f, g\) are differentiable on \(E\), then \(f / g\) is differentiable on \(E\), and
  \[
    \nabla \bigg(\dfrac{f}{g}\bigg) = \dfrac{g \nabla f - f \nabla g}{g^2}.
  \]

  Let \(k : \R \times (\R \setminus \set{0}) \to \R\) be the function
  \[
    \forall (a, b) \in \R \times (\R \setminus \set{0}), k(a, b) = \dfrac{a}{b}.
  \]
  We claim that \(k\) is differentiable on \(\R \times (\R \setminus \set{0})\) and
  \[
    \forall (a, b) \in \R \times (\R \setminus \set{0}), D k(a, b) = \bigg(\dfrac{1}{b}, \dfrac{-a}{b^2}\bigg).
  \]
  Let \((a, b) \in \R \times (\R \setminus \set{0})\).
  Since for each \((x, y) \in \big(\R \times (\R \setminus \set{0})\big) \setminus \set{(a, b)}\), we have
  \begin{align*}
     & \dfrac{\norm*{k(x, y) - k(a, b) - \big((x, y) - (a, b)\big) \bigg(\dfrac{1}{b}, \dfrac{-a}{b^2}\bigg)^\top}}{\norm*{(x, y) - (a, b)}} \\
     & = \dfrac{\norm*{\dfrac{x}{y} - \dfrac{a}{b} - \dfrac{x - a}{b} + \dfrac{a(y - b)}{b^2}}}{\norm*{(x - a, y - b)}}                      \\
     & = \dfrac{\norm*{\dfrac{b^2 x - bxy + ay^2 - aby}{b^2y}}}{\norm*{(x - a, y - b)}}                                                      \\
     & = \dfrac{\norm*{\dfrac{(ay - bx)(y - b)}{b^2y}}}{\norm*{(x - a, y - b)}}                                                              \\
     & = \sqrt{\dfrac{(ay - bx)^2 (y - b)^2}{b^4 y^2 \big((x - a)^2 + (y - b)^2\big)}}                                                       \\
     & \leq \sqrt{\dfrac{(ay - bx)^2 \big((x - a)^2 + (y - b)^2\big)}{b^4 y^2 \big((x - a)^2 + (y - b)^2\big)}}                              \\
     & = \sqrt{\dfrac{(ay - bx)^2}{b^4 y^2}},
  \end{align*}
  we know that
  \begin{align*}
             & \lim_{(x, y) \to (a, b) ; (x, y) \in \big(\R \times (\R \setminus \set{0})\big) \setminus \set{(a, b)}} (ay - bx)^2 = 0                                                                                                                     \\
    \implies & \lim_{(x, y) \to (a, b) ; (x, y) \in \big(\R \times (\R \setminus \set{0})\big) \setminus \set{(a, b)}} \dfrac{(ay - bx)^2}{b^4 y^2} = 0                                                                                                    \\
    \implies & \lim_{(x, y) \to (a, b) ; (x, y) \in \big(\R \times (\R \setminus \set{0})\big) \setminus \set{(a, b)}} \sqrt{\dfrac{(ay - bx)^2}{b^4 y^2}} = 0                                                                                             \\
    \implies & \lim_{(x, y) \to (a, b) ; (x, y) \in \big(\R \times (\R \setminus \set{0})\big) \setminus \set{(a, b)}} \dfrac{\norm*{k(x, y) - k(a, b) - \big((x, y) - (a, b)\big) \big(\dfrac{1}{b}, \dfrac{-a}{b^2}\big)^\top}}{\norm*{(x, y) - (a, b)}} \\
             & = 0.
  \end{align*}
  (note that the last line was done by squeeze test)
  Since \((a, b)\) is arbitrary, by \cref{ii:6.2.2} we conclude that
  \[
    \forall (a, b) \in \R \times (\R \setminus \set{0}), D k(a, b) = \bigg(\dfrac{1}{b}, \dfrac{-a}{b^2}\bigg).
  \]

  Let \(h\) be the function in \cref{ii:6.4.2}.
  Using chain rule we have
  \begin{align*}
    D (k \circ h)(x_0) & = D k\big(h(x_0)\big) \cdot D h(x_0)                                                       &  & \by{ii:6.4.1} \\
                       & = D k\big(f(x_0), g(x_0)\big) \cdot \begin{pmatrix}
                                                               \nabla f(x_0) \\
                                                               \nabla g(x_0)
                                                             \end{pmatrix}                                        &  & \by{ii:6.4.2}   \\
                       & = \bigg(\dfrac{1}{g(x_0)}, \dfrac{-f(x_0)}{\big(g(x_0)\big)^2}\bigg) \cdot \begin{pmatrix}
                                                                                                      \nabla f(x_0) \\
                                                                                                      \nabla g(x_0)
                                                                                                    \end{pmatrix}                  \\
                       & = \dfrac{\nabla f(x_0)}{g(x_0)} - \dfrac{f(x_0) \nabla g(x_0)}{\big(g(x_0)\big)^2}                         \\
                       & = \dfrac{g(x_0) \nabla f(x_0) - f(x_0) \nabla g(x_0)}{\big(g(x_0)\big)^2}.
  \end{align*}
  But \(k \circ h = f / g\).
  Thus we have
  \[
    \nabla \bigg(\dfrac{f}{g}\bigg)(x_0) = \dfrac{g(x_0) \nabla f(x_0) - f(x_0) \nabla g(x_0)}{\big(g(x_0)\big)^2}.
  \]
\end{proof}

\begin{ex}\label{ii:ex:6.4.5}
  Let \(\vec{x} : \R \to \R^3\) be a differentiable function, and let \(r : \R \to \R\) be the function \(r(t) \coloneqq \norm*{\vec{x}(t)}\), where \(\norm*{\vec{x}}\) denotes the length of \(\vec{x}\) as measured in the usual \(l^2\) metric.
  Let \(t_0\) be a real number.
  Show that if \(r(t_0) \neq 0\), then \(r\) is differentiable at \(t_0\), and
  \[
    r'(t_0) = \dfrac{\vec{x}'(t_0) \cdot \vec{x}(t_0)}{r(t_0)}.
  \]
\end{ex}

\begin{proof}
  Let \(f : \R^3 \to \R\) be the function
  \[
    \forall x \in \R^3, f(x) = \norm*{x}.
  \]
  Since
  \[
    \forall y \in \R^3 \setminus \set{0_{\R^3}}, \forall 1 \leq i \leq 3, \dfrac{\partial f}{\partial x_i}(y) = \dfrac{y_i}{\norm*{y}},
  \]
  and \(\dfrac{\partial f}{\partial x_i}\) is continuous on \(\R^3 \setminus \set{0_{\R^3}}\), by \cref{ii:6.3.8} we know that
  \begin{align*}
    \forall y \in \R^3 \setminus \set{0_{\R^3}}, D f(y) & = \nabla f(y)                                                                                                               \\
                                                        & = \bigg(\dfrac{\partial f}{\partial x_1}(y), \dfrac{\partial f}{\partial x_2}(y), \dfrac{\partial f}{\partial x_3}(y)\bigg) \\
                                                        & = \bigg(\dfrac{y_1}{\norm*{y}}, \dfrac{y_2}{\norm*{y}}, \dfrac{y_3}{\norm*{y}}\bigg)                                        \\
                                                        & = \dfrac{y}{\norm*{y}}.
  \end{align*}
  By chain rule (\cref{ii:6.4.1}) we know that
  \begin{align*}
             & \forall t \in \R, \vec{x}(t) \neq 0                                                                                                  \\
    \implies & D (f \circ \vec{x})(t) = D f\big(\vec{x}(t)\big) \cdot D \vec{x}(t) = \dfrac{\vec{x}(t)}{\norm*{\vec{x}(t)}} \cdot \vec{x}'(t)^\top.
  \end{align*}
  Since \(r = f \circ \vec{x}\), we know that
  \begin{align*}
             & \forall t \in \R, \vec{x}(t) \neq 0                       \\
    \implies & \norm*{t} \neq 0                                          \\
    \implies & r(t) \neq 0                                               \\
    \implies & D r(t) = \dfrac{\vec{x}(t) \cdot \vec{x}'(t)^\top}{r(t)}.
  \end{align*}
\end{proof}
