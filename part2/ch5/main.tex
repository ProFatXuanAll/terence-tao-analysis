\chapter{Fourier series}\label{ii:ch:5}

\begin{note}
  Power series are already immensely useful, especially when dealing with special functions such as the exponential and trigonometric functions discussed earlier.
  However, there are some circumstances where power series are not so useful, because one has to deal with functions (e.g., \(\sqrt{x}\)) which are not real analytic, and so do not have power series.
\end{note}

\begin{note}
  Fortunately, there is another type of series expansion, known as \emph{Fourier series}, which is also a very powerful tool in analysis
  (though used for slightly different purposes).
  Instead of analyzing compactly supported functions, it instead analyzes \emph{periodic functions};
  instead of decomposing into polynomials, it decomposes into \emph{trigonometric polynomials}.
  Roughly speaking, the theory of Fourier series asserts that just about every periodic function can be decomposed as an (infinite) sum of sines and cosines.
\end{note}

\begin{rmk}\label{ii:5.0.1}
  Jean-Baptiste Fourier (1768--1830) was, among other things, an administrator accompanying Napoleon on his invasion of Egypt, and then a Prefect in France during Napoleons reign.
  After the Napoleonic wars, he returned to mathematics.
  He introduced Fourier series in an important 1807 paper in which he used them to solve what is now known as the heat equation.
  At the time, the claim that every periodic function could be expressed as a sum of sines and cosines was extremely controversial, even such leading mathematicians as Euler declared that it was impossible.
  Nevertheless, Fourier managed to show that this was indeed the case, although the proof was not completely rigorous and was not totally accepted for almost another hundred years.
\end{rmk}

\begin{note}
  For instance, the convergence of Fourier series is usually not uniform (i.e., not in the \(L^\infty\) metric), but instead we have convergence in a different metric, the \(L^2\)-metric.
  We will need to use complex numbers heavily in our theory, while they played only a tangential rôle in power series.
\end{note}

\begin{note}
  The theory of Fourier series (and of related topics such as Fourier integrals and the Laplace transform) is vast, and deserves an entire course in itself.
  It has many, many applications, most directly to differential equations, signal processing, electrical engineering, physics, and analysis, but also to algebra and number theory.
\end{note}

\subimport{./}{sec1.tex}
\subimport{./}{sec2.tex}
\subimport{./}{sec3.tex}
\subimport{./}{sec4.tex}
\subimport{./}{sec5.tex}
