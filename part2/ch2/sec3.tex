\section{Continuity and compactness}\label{ii:sec:2.3}

\begin{thm}[Continuous maps preserve compactness]\label{ii:2.3.1}
  Let \(f : X \to Y\) be a continuous map from one metric space \((X, d_X)\) to another \((Y, d_Y)\).
  Let \(K \subseteq X\) be any compact subset of \(X\).
  Then the image \(f(K) \coloneqq \set{f(x) : x \in K}\) of \(K\) is also compact.
\end{thm}

\begin{proof}
  Let \(\bigcup_{\alpha \in I} V_\alpha\) be an open cover of \(f(K)\) in \((Y, d_Y)\), i.e., \(I \subseteq Y\) and for each \(\alpha \in I\), \(V_{\alpha}\) is an open set in \((Y, d_Y)\) such that \(f(K) \subseteq \bigcup_{\alpha \in I} V_\alpha\).
  Since
  \begin{align*}
             & f \text{ is continuous from } (X, d_X) \text{ to } (Y, d_Y)                                                                                          \\
    \implies & f|_K \text{ is continuous from } (K, d_X|_{K \times K}) \text{ to } (Y, d_Y)                                                 &  & \by{ii:2.1.3}      \\
    \implies & \forall \alpha \in I, f|_K^{-1}(V_\alpha) \text{ is open in } (K, d_X|_{K \times K})                                         &  & \by{ii:2.1.5}[a,d] \\
    \implies & K = \bigcup_{\alpha \in I} f|_K^{-1}(V_\alpha) \text{ is open in } (K, d_X|_{K \times K})                                    &  & \by{ii:1.2.15}[g]  \\
    \implies & \exists F \subseteq I : (F \text{ is finite}) \land \bigg(K = \bigcup_{\alpha \in F} f|_K^{-1}(V_\alpha)\bigg)               &  & \by{ii:1.5.8}      \\
    \implies & \exists F \subseteq I : (F \text{ is finite}) \land \bigg(f(K) = f\big(\bigcup_{\alpha \in F} f|_K^{-1}(V_\alpha)\big)\bigg)                         \\
    \implies & \exists F \subseteq I : (F \text{ is finite}) \land \bigg(f(K) \subseteq \bigcup_{\alpha \in F} V_\alpha)\bigg),
  \end{align*}
  we know that there exists an finite subcover of \(f(K)\) with respect to \(\bigcup_{\alpha \in I} V_\alpha\) in \((Y, d_Y)\).
  Since \(I\) was arbitrary open cover of \(f(K)\) in \((Y, d_Y)\), by \cref{ii:ex:1.5.11} we know that \(\big(f(K), d_Y|_{Y \times Y}\big)\) is compact.
\end{proof}

\begin{prop}[Maximum principle]\label{ii:2.3.2}
  Let \((X, d)\) be a compact metric space, and let \(f : X \to \R\) be a continuous function.
  Then \(f\) is bounded.
  Furthermore, if \(X\) is non-empty, \(f\) attains its maximum at some point \(x_{\max} \in X\), and also attains its minimum at some point \(x_{\min} \in X\).
\end{prop}

\begin{proof}
  We have
  \begin{align*}
             & \begin{dcases}
                 (X, d) \text{ is compact} \\
                 f \text{ is continuous from } (X, d) \text{ to } (\R, d_{l^1}|_{\R \times \R})
               \end{dcases}                     \\
    \implies & \big(f(X), d_{l^1}|_{f(X) \times f(X)}\big) \text{ is compact}                 &  & \by{ii:2.3.1} \\
    \implies & \begin{dcases}
                 f(X) \text{ is closed in } (\R, d_{l^1}|_{\R \times \R}) \\
                 \big(f(X), d_{l^1}|_{f(X) \times f(X)}\big) \text{ is bounded}
               \end{dcases}                 &  & \by{ii:1.5.6}                                     \\
    \implies & \begin{dcases}
                 f(X) \text{ is closed in } (\R, d_{l^1}|_{\R \times \R}) \\
                 f(X) \text{ is bounded subset of } \R
               \end{dcases}                    &  & \by{ii:ex:1.5.1}
  \end{align*}

  Now we show that if \(X \neq \emptyset\), then
  \[
    \exists x_{\min}, x_{\max} \in X : \forall x \in X, f(x_{\min}) \leq f(x) \leq f(x_{\max}).
  \]
  Let \(U = \sup\big(f(X)\big)\) and let \(L = \inf\big(f(X)\big)\).
  Since \(f\) is bounded subset of \(\R\), we know that \(U, L \in \R\).
  By the definition of \(U\) and \(L\) we know that
  \[
    \forall n \in \Z^+, \exists u_n, l_n \in f(X) : \begin{dcases}
      U - u_n < \dfrac{1}{n} \\
      l_n - L < \dfrac{1}{n}
    \end{dcases}
  \]
  Thus, we have
  \begin{align*}
             & \begin{dcases}
                 0 = \lim_{n \to \infty} U - u_n = \lim_{n \to \infty} \dfrac{1}{n} = 0 \\
                 0 = \lim_{n \to \infty} l_n - L = \lim_{n \to \infty} \dfrac{1}{n} = 0
               \end{dcases} \\
    \implies & \begin{dcases}
                 \lim_{n \to \infty} u_n = U \\
                 \lim_{n \to \infty} l_n = L
               \end{dcases}
  \end{align*}
  Since \(f(X)\) is closed in \((\R, d_{l^1}|_{\R \times \R})\) and \((u_n)_{n = 1}^\infty\), \((l_n)_{n = 1}^\infty\) are convergent sequences in \(\R\) with respect to \(d_{l^1}|_{\R \times \R}\), by \cref{ii:1.2.15}(b) we know that \(U, L \in f(X)\).
  Since \(U, L \in f(X)\), we know that
  \[
    \exists x_{\min}, x_{\max} \in X : \big(f(x_{\min}) = L\big) \land \big(f(x_{\max}) = U\big).
  \]
\end{proof}

\begin{rmk}\label{ii:2.3.3}
  As was already noted in Exercise 9.6.1 in Analysis I, this principle can fail if \(X\) is not compact.
  \cref{ii:2.3.2} should be compared with Lemma 9.6.3 in Analysis I and Proposition 9.6.7 in Analysis I.
\end{rmk}

\begin{defn}[Uniform continuity]\label{ii:2.3.4}
  Let \(f : X \to Y\) be a map from one metric space \((X, d_X)\) to another \((Y, d_Y)\).
  We say that \(f\) is \emph{uniformly continuous} if, for every \(\varepsilon > 0\), there exists a \(\delta > 0\) such that \(d_Y\big(f(x), f(x')\big) < \varepsilon\) whenever \(x, x' \in X\) are such that \(d_X(x, x') < \delta\).
\end{defn}

\begin{note}
  Every uniformly continuous function is continuous, but not conversely.
  But if the domain \(X\) is compact, then the two notions are equivalent.
\end{note}

\begin{thm}\label{ii:2.3.5}
  Let \((X, d_X)\) and \((Y, d_Y)\) be metric spaces, and suppose that \((X, d_X)\) is compact.
  If \(f : X \to Y\) is function, then \(f\) is continuous iff it is uniformly continuous.
\end{thm}

\begin{proof}
  If \(f\) is uniformly continuous then it is also continuous by \cref{ii:ex:2.3.3}.
  Now suppose that \(f\) is continuous.
  Fix \(\varepsilon > 0\).
  For every \(x_0 \in X\), the function \(f\) is continuous at \(x_0\) from \((X, d_X)\) to \((Y, d_Y)\).
  Thus, there exists a \(\delta(x_0) > 0\), depending on \(x_0\), such that \(d_Y\big(f(x), f(x_0)\big) < \varepsilon / 2\) whenever \(d_X(x, x_0) < \delta(x_0)\).
  In particular, by the triangle inequality this implies that \(d_Y(f(x), f(x')) < \varepsilon\) whenever \(x \in B_{(X, d_X)}\big(x_0, \delta(x_0) / 2\big)\) and \(d_X(x', x) < \delta(x_0) / 2\).

  Now consider the (possibly infinite) collection of balls
  \[
    \set{B_{(X, d_X)}\big(x_0, \delta(x_0) / 2\big) : x_0 \in X}.
  \]
  Each ball in this collection is of course open in \((X, d_X)\), and the union of all these balls covers \(X\), since each point \(x_0\) in \(X\) is contained in its own ball \(B_{(X, d_X)}\big(x_0, \delta(x_0) / 2\big)\).
  Hence, by \cref{ii:1.5.8}, there exist a finite number of points \(x_1, \dots, x_n\) such that the balls \(B_{(X, d_X)}\big(x_j, \delta(x_j) / 2\big)\) for \(j = 1, \dots, n\) cover \(X\):
  \[
    X \subseteq \bigcup_{j = 1}^n B_{(X, d_X)}\big(x_j, \delta(x_j) / 2\big).
  \]
  Now let \(\delta \coloneqq \min_{j = 1}^n \delta(x_j) / 2\).
  Since each of the \(\delta(x_j)\) are positive, and there are only a finite number of \(j\), we see that \(\delta > 0\).
  Now let \(x, x'\) be any two points in \(X\) such that \(d_X(x, x') < \delta\).
  Since the balls \(B_{(X, d_X)}\big(x_j, \delta(x_j) / 2\big)\) cover \(X\), we see that there must exist \(1 \leq j \leq n\) such that \(x \in B_{(X, d_X)}\big(x_j, \delta(x_j) / 2\big)\).
  Since \(d_X(x, x') < \delta\), we have \(d_X(x, x') < \delta(x_j) / 2\), and so by the previous discussion we have \(d_Y\big(f(x), f(x')\big) < \varepsilon\).
  We have thus found a \(\delta\) such that \(d_Y\big(f(x), f(x')\big) < \varepsilon\) whenever \(d(x, x') < \delta\), and this proves uniform continuity as desired.
\end{proof}

\exercisesection

\begin{ex}\label{ii:ex:2.3.1}
  Prove \cref{ii:2.3.1}.
\end{ex}

\begin{proof}
  See \cref{ii:2.3.1}.
\end{proof}

\begin{ex}\label{ii:ex:2.3.2}
  Prove \cref{ii:2.3.2}.
\end{ex}

\begin{proof}
  See \cref{ii:2.3.2}.
\end{proof}

\begin{ex}\label{ii:ex:2.3.3}
  Show that every uniformly continuous function is continuous, but give an example that shows that not every continuous function is uniformly continuous.
\end{ex}

\begin{proof}
  Let \((X, d_X)\), \((Y, d_Y)\) be two metric spaces and let \(f : X \to Y\) be a function which is uniformly continuous from \((X, d_X)\) to \((Y, d_Y)\).
  Then we have
  \begin{align*}
             & \forall \varepsilon \in \R^+, \exists \delta \in \R^+ :                                                                  \\
             & \Big(\forall x, x_0 \in X, d_X(x, x_0) < \delta \implies d_Y\big(f(x), f(x_0)\big) < \varepsilon\Big) &  & \by{ii:2.3.4} \\
    \implies & \forall x_0 \in X, f \text{ is continuous at } x_0 \text{ from } (X, d_X) \text{ to } (Y, d_Y)        &  & \by{ii:2.1.1} \\
    \implies & f \text{ is continuous from } (X, d_X) \text{ to } (Y, d_Y).                                          &  & \by{ii:2.1.1}
  \end{align*}

  For the converse example, see Example 9.9.11 in Analysis I.
\end{proof}

\begin{ex}\label{ii:ex:2.3.4}
  Let \((X, d_X)\), \((Y, d_Y)\), \((Z, d_Z)\) be metric spaces, and let \(f : X \to Y\) and \(g : Y \to Z\) be two uniformly continuous functions.
  Show that \(g \circ f : X \to Z\) is also uniformly continuous.
\end{ex}

\begin{proof}
  Since \(f\) is uniformly continuous from \((X, d_X)\) to \((Y, d_Y)\) and \(g\) is uniformly continuous from \((Y, d_Y)\) to \((Z, d_Z)\), by \cref{ii:2.3.4} we have
  \begin{align*}
             & \begin{dcases}
                 \forall \delta' \in \R^+, \exists \delta, \in \R^+ :                                                     \\
                 \Big(\forall x_1, x_2 \in X, d_X(x_1, x_2) < \delta \implies d_Y\big(f(x_1), f(x_2)\big) < \delta'\Big); \\
                 \forall \varepsilon \in \R^+, \exists \delta' \in \R^+ :                                                 \\
                 \Big(\forall y_1, y_2 \in Y, d_Y(y_1, y_2) < \delta' \implies d_Z\big(g(y_1), g(y_2)\big) < \varepsilon\Big);
               \end{dcases}                         \\
    \implies & \forall \varepsilon \in \R^+, \exists \delta \in \R^+ :                                                                              \\
             & \bigg(\forall x_1, x_2 \in X, d_X(x_1, x_2) < \delta \implies d_Z\Big(g\big(f(x_1)\big), g\big(f(x_2)\big)\Big) < \varepsilon\bigg).
  \end{align*}
  Thus, by \cref{ii:2.3.4} \(g \circ f\) is uniformly continuous from \((X, d_X)\) to \((Z, d_Z)\).
\end{proof}

\begin{ex}\label{ii:ex:2.3.5}
  Let \((X, d_X)\) be a metric space, and let \(f : X \to \R\) and \(g : X \to \R\) be uniformly continuous functions.
  Show that the direct sum \(f \oplus g : X \to \R^2\) defined by \(f \oplus g(x) \coloneqq \big(f(x), g(x)\big)\) is uniformly continuous.
\end{ex}

\begin{proof}
  Let \(d_1 = d_{l^1}|_{\R \times \R}\) and let \(d_2 = d_{l^1}|_{\R^2 \times \R^2}\).
  Since \(f, g\) are uniformly continuous from \((X, d_X)\) to \((\R, d_1)\), by \cref{ii:2.3.4} we have
  \begin{align*}
             & \forall \varepsilon \in \R^+, \exists \delta \in \R^+ :                                                                                            \\
             & \begin{dcases}
                 \forall x_1, x_2 \in X, d_X(x_1, x_2) < \delta \implies d_1\big(f(x_1), f(x_2)\big) < \dfrac{\varepsilon}{2} \\
                 \forall x_1, x_2 \in X, d_X(x_1, x_2) < \delta \implies d_1\big(g(x_1), g(x_2)\big) < \dfrac{\varepsilon}{2}
               \end{dcases}                                       \\
    \implies & \forall \varepsilon \in \R^+, \exists \delta \in \R^+ :                                                                                            \\
             & \Big(\forall x_1, x_2 \in X, d_X(x_1, x_2) < \delta \implies d_1\big(f(x_1), f(x_2)\big) + d_1\big(g(x_1), g(x_2)\big) < \varepsilon\Big)          \\
    \implies & \forall \varepsilon \in \R^+, \exists \delta \in \R^+ :                                                                                            \\
             & \bigg(\forall x_1, x_2 \in X, d_X(x_1, x_2) < \delta \implies d_2\Big(\big(f(x_1), g(x_1)\big), \big(f(x_2), g(x_2)\big)\Big) < \varepsilon\bigg).
  \end{align*}
  Thus, by \cref{ii:2.3.4} \(f \oplus g\) is uniformly continuous from \((X, d_X)\) to \((\R^2, d_2)\).
\end{proof}

\begin{ex}\label{ii:ex:2.3.6}
  Show that the addition function \((x, y) \mapsto x + y\) and the subtraction function \((x, y) \mapsto x - y\) are uniformly continuous from \(\R^2\) to \(\R\), but the multiplication function \((x, y) \mapsto xy\) is not.
  Conclude that if \(f : X \to \R\) and \(g : X \to \R\) are uniformly continuous functions on a metric space \((X, d)\), then \(f + g : X \to \R\) and \(f - g : X \to \R\) are also uniformly continuous.
  Give an example to show that \(fg : X \to \R\) need not be uniformly continuous.
  What is the situation for \(\max(f, g)\), \(\min(f, g)\), \(f / g\), and \(cf\) for a real number \(c\)?
\end{ex}

\begin{proof}
  Let \(d_1 = d_{l^1}|_{\R \times \R}\) and let \(d_2 = d_{l^1}|_{\R^2 \times \R^2}\).
  We first show that \((x, y) \mapsto x + y\) and \((x, y) \mapsto x - y\) are uniformly continuous from \((\R^2, d_2)\) to \((\R, d_1)\).
  Since
  \begin{align*}
             & \forall \varepsilon \in \R^+, \forall (x_1, y_1), (x_2, y_2) \in \R^2,                                      \\
             & \big(\abs{x_1 - x_2} < \dfrac{\varepsilon}{2}\big) \land \big(\abs{y_1 - y_2} < \dfrac{\varepsilon}{2}\big) \\
    \implies & \begin{dcases}
                 \abs{x_1 - x_2 + y_1 - y_2} \leq \abs{x_1 - x_2} + \abs{y_1 - y_2} < \varepsilon \\
                 \abs{x_1 - x_2 - y_1 + y_2} \leq \abs{x_1 - x_2} + \abs{y_2 - y_1} < \varepsilon \\
               \end{dcases}                            \\
    \implies & \begin{dcases}
                 d_1(x_1 + y_1, x_2 + y_2) \leq d_2\big((x_1, y_1), (x_2, y_2)\big) < \varepsilon \\
                 d_1(x_1 - y_1, x_2 - y_2) \leq d_2\big((x_1, y_1), (x_2, y_2)\big) < \varepsilon
               \end{dcases}
  \end{align*}
  by choosing \(\delta = \varepsilon\) we have
  \begin{align*}
     & \forall \varepsilon \in \R^+, \exists \delta \in \R^+ : \forall (x_1, y_1), (x_2, y_2) \in \R^2, \\
     & d_2\big((x_1, y_1), (x_2, y_2)\big) < \delta \implies \begin{dcases}
                                                               d_1(x_1 + y_1, x_2 + y_2) < \varepsilon \\
                                                               d_1(x_1 - y_1, x_2 - y_2) < \varepsilon
                                                             \end{dcases}
  \end{align*}
  and thus by \cref{ii:2.3.4} \((x, y) \mapsto x + y\) and \((x, y) \mapsto x - y\) are uniformly continuous from \((\R^2, d_2)\) to \((\R, d_1)\).

  Next we show that \((x, y) \mapsto xy\) is not uniformly continuous from \((\R^2, d_2)\) to \((\R, d_1)\).
  Since
  \[
    \forall n \in \Z^+, \begin{dcases}
      d_2\big((n, n), (n + \dfrac{1}{n}, n)\big) = \dfrac{1}{n} \\
      d_1\big(n \times n, (n + \dfrac{1}{n}) \times n\big) = 1
    \end{dcases}
  \]
  no matter which \(\delta \in \R^+\) we choose, we cannot make \(d_1(n^2, n^2 + 1) < \dfrac{1}{2}\) for every \(n \in \Z^+\).
  Thus, by \cref{ii:2.3.4} \((x, y) \mapsto xy\) is not uniformly continuous from \((\R^2, d_2)\) to \((\R, d_1)\).

  Next we show that \((x, y) \mapsto \max(x, y)\) and \((x, y) \mapsto \min(x, y)\) are uniformly continuous from \((X, d)\) to \((\R, d_1)\).
  Since
  \begin{align*}
             & \forall \varepsilon \in \R^+, \forall (x_1, y_1), (x_2, y_2) \in \R^2,                \\
             & \big(\abs{x_1 - x_2} < \varepsilon\big) \land \big(\abs{y_2 - y_1} < \varepsilon\big) \\
    \implies & \begin{dcases}
                 x_2 - \varepsilon < x_1 < x_2 + \varepsilon \\
                 y_2 - \varepsilon < y_1 < y_2 + \varepsilon \\
               \end{dcases}                                           \\
    \implies & \begin{dcases}
                 \max(x_2, y_2) - \varepsilon < \max(x_1, y_1) < \max(x_2, y_2) + \varepsilon \\
                 \min(x_2, y_2) - \varepsilon < \min(x_1, y_1) < \min(x_2, y_2) + \varepsilon
               \end{dcases}          \\
    \implies & \begin{dcases}
                 \abs{\max(x_1, y_1) - \max(x_2, y_2)} < \varepsilon \\
                 \abs{\min(x_1, y_1) - \min(x_2, y_2)} < \varepsilon
               \end{dcases}
  \end{align*}
  by choosing \(\delta = \varepsilon\) we have
  \begin{align*}
     & \forall \varepsilon \in \R^+, \exists \delta \in \R^+ :                               \\
     & \forall (x_1, y_1), (x_2, y_2) \in \R^2, d_2\big((x_1, y_1), (x_2, y_2)\big) < \delta \\
     & \implies \begin{dcases}
                  d_1\big(\max(x_1, y_1), \max(x_2, y_2)\big) < \varepsilon \\
                  d_1\big(\min(x_1, y_1), \min(x_2, y_2)\big) < \varepsilon
                \end{dcases}
  \end{align*}
  and thus by \cref{ii:2.3.4} \((x, y) \mapsto \max(x, y)\) and \((x, y) \mapsto \min(x, y)\) are uniformly continuous from \((\R^2, d_2)\) to \((\R, d_1)\).

  Next we show that \(f + g\), \(f - g\), \(\max(f, g)\), \(\min(f, g)\) are uniformly continuous from \((X, d)\) to \((\R, d_1)\) given \(f, g\) are uniformly continuous from \((X, d)\) to \((\R, d_1)\).
  \begin{align*}
             & f, g \text{ are uniformly continuous from } (X, d) \text{ to } (\R, d_1)                                \\
    \implies & f \oplus g \text{ is uniformly continuous from } (X, d) \text{ to } (\R^2, d_2)   &  & \by{ii:ex:2.3.5} \\
    \implies & \begin{dcases}
                 f \oplus g(x) \mapsto f(x) + g(x) \text{ is uniformly continuous}              \\
                 \text{from } (X, d) \text{ to } (\R, d_1);                                     \\
                 f \oplus g(x) \mapsto f(x) - g(x) \text{ is uniformly continuous}              \\
                 \text{from } (X, d) \text{ to } (\R, d_1);                                     \\
                 f \oplus g(x) \mapsto \max\big(f(x), g(x)\big) \text{ is uniformly continuous} \\
                 \text{from } (X, d) \text{ to } (\R, d_1);                                     \\
                 f \oplus g(x) \mapsto \min\big(f(x), g(x)\big) \text{ is uniformly continuous} \\
                 \text{from } (X, d) \text{ to } (\R, d_1).
               \end{dcases} &  & \by{ii:ex:2.3.4}                          \\
    \implies & f + g, f - g, \max(f, g), \min(f, g) \text{ are uniformly continuous}                                   \\
             & \text{from } (X, d) \text{ to } (\R, d_1).
  \end{align*}

  Next we give a example where \(fg\) is not continuous from \((X, d)\) to \((\R, d_1)\) given \(f, g\) are uniformly continuous from \((X, d)\) to \((\R, d_1)\).
  Let \(f(x) = g(x) = x\).
  Then \(fg(x) = x^2\) and by Example 9.9.11 in Analysis I we know that \(x^2\) is not uniformly continuous from \((\R, d_1)\) to \((\R, d_1)\).

  Next we give a example where \(f / g\) is not continuous from \((X, d)\) to \((\R, d_1)\) given \(f, g\) are uniformly continuous from \((X, d)\) to \((\R, d_1)\).
  Let \(f(x) = 1\) and let \(g(x) = x\).
  Then \(f / g(x) = 1 / x\) and by Example 9.9.10 in Analysis I we know that \(1 / x\) is not uniformly continuous from \((\R, d_1)\) to \((\R, d_1)\).

  Finally we show that for each \(c \in \R\), \(cf\) is uniformly continuous from \((X, d)\) to \((\R, d_1)\) given \(f\) is uniformly continuous from \((X, d)\) to \((\R, d_1)\).
  If \(c = 0\), then \(f\) is a constant function.
  Thus
  \[
    \forall \varepsilon \in \R^+, \forall x \in X, d(x_1, x_2) < \varepsilon \implies d_1\big(f(x_1), f(x_2)\big) = 0 < \varepsilon
  \]
  and by \cref{ii:2.3.4} \(f\) is uniformly continuous from \((X, d)\) to \((\R, d_1)\).
  Suppose that \(c \neq 0\).
  Since \(f\) is uniformly continuous from \((X, d)\) to \((\R, d_1)\), by \cref{ii:2.3.4} we have
  \begin{align*}
             & \forall \varepsilon \in \R^+, \exists \delta \in \R^+:                                                                     \\
             & \Big(\forall x_1, x_2 \in X, d(x_1, x_2) < \delta \implies d_1\big(f(x_1), f(x_2)\big) < \dfrac{\varepsilon}{\abs{c}}\Big) \\
    \implies & \forall \varepsilon \in \R^+, \exists \delta \in \R^+:                                                                     \\
             & \Big(\forall x_1, x_2 \in X, d(x_1, x_2) < \delta \implies d_1\big(cf(x_1), cf(x_2)\big) < \varepsilon\Big)
  \end{align*}
  and thus by \cref{ii:2.3.4} \(cf\) is uniformly continuous from \((X, d)\) to \((\R, d_1)\).
\end{proof}
