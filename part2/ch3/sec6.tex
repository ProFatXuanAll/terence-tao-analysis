\section{Uniform convergence and integration}\label{ii:sec:3.6}

\begin{thm}\label{ii:3.6.1}
  Let \([a, b]\) be an interval, and for each integer \(n \geq 1\), let \(f^{(n)} : [a, b] \to \R\) be a Riemann-integrable function.
  Suppose \(f^{(n)}\) converges uniformly on \([a, b]\) to a function \(f : [a, b] \to \R\).
  Then \(f\) is also Riemann integrable, and
  \[
    \lim_{n \to \infty} \int_{[a, b]} f^{(n)} = \int_{[a, b]} f.
  \]
\end{thm}

\begin{proof}
  We first show that \(f\) is Riemann integrable on \([a, b]\).
  This is the same as showing that the upper and lower Riemann integrals of \(f\) match:
  \(\underline{\int}_{[a, b]} f = \overline{\int}_{[a, b]} f\).

  Let \(\varepsilon > 0\).
  Since \(f^{(n)}\) converges uniformly to \(f\), we see that there exists an \(N > 0\) such that \(\abs{f^{(n)}(x) - f(x)} < \varepsilon\) for all \(n > N\) and \(x \in [a, b]\).
  In particular, we have
  \[
    f^{(n)}(x) - \varepsilon < f(x) < f^{(n)}(x) + \varepsilon
  \]
  for all \(x \in [a, b]\).
  Integrating this on \([a, b]\) we obtain
  \[
    \underline{\int}_{[a, b]} (f^{(n)} - \varepsilon) \leq \underline{\int}_{[a, b]} f \leq \overline{\int}_{[a, b]} f \leq \overline{\int}_{[a, b]} (f^{(n)} + \varepsilon).
  \]
  Since \(f^{(n)}\) is assumed to be Riemann integrable, we thus see
  \[
    \Bigg(\int_{[a, b]} f^{(n)}\Bigg) - \varepsilon (b - a) \leq \underline{\int}_{[a, b]} f \leq \overline{\int}_{[a, b]} f \leq \Bigg(\int_{[a, b]} f^{(n)}\Bigg) + \varepsilon (b - a).
  \]
  In particular, we see that
  \[
    0 \leq \overline{\int}_{[a, b]} f - \underline{\int}_{[a, b]} f \leq 2 \varepsilon (b - a).
  \]
  Since this is true for every \(\varepsilon > 0\), we obtain \(\underline{\int}_{[a, b]} f = \overline{\int}_{[a, b]} f\) as desired.

  The above argument also shows that for every \(\varepsilon > 0\) there exists an \(N > 0\) such that
  \[
    \abs{\int_{[a, b]} f^{(n)} - \int_{[a, b]} f} \leq \varepsilon (b - a)
  \]
  for all \(n \geq N\).
  Since \(\varepsilon\) was arbitrary, we see that \(\int_{[a, b]} f^{(n)}\) converges to \(\int_{[a, b]} f\) as desired.
\end{proof}

\begin{note}
  To rephrase \cref{ii:3.6.1}:
  we can rearrange limits and integrals (on compact intervals \([a, b]\)),
  \[
    \lim_{n \to \infty} \int_{[a, b]} f^{(n)} = \int_{[a, b]} \lim_{n \to \infty} f^{(n)},
  \]
  \emph{provided that} the convergence is uniform.
\end{note}

\begin{cor}\label{ii:3.6.2}
  Let \([a, b]\) be an interval, and let \((f^{(n)})_{n = 1}^\infty\) be a sequence of Riemann integrable functions on \([a, b]\) such that the series \(\sum_{n = 1}^\infty f^{(n)}\) is uniformly convergent.
  Then we have
  \[
    \sum_{n = 1}^\infty \int_{[a, b]} f^{(n)} = \int_{[a, b]} \sum_{n = 1}^\infty f^{(n)}.
  \]
\end{cor}

\begin{proof}
  By Theorem 11.4.1(a) in Analysis I we know that
  \[
    \forall N \in \Z^+, \int_{[a, b]} \sum_{n = 1}^N f^{(n)} = \sum_{n = 1}^N \int_{[a, b]} f^{(n)}.
  \]
  Let \(f : [a, b] \to \R\) be the function such that \(\sum_{n = 1}^\infty f^{(n)}\) converges uniformly to \(f\) on \([a, b]\) with respect to \(d_{l^1}|_{\R \times \R}\).
  By \cref{ii:3.6.1} we have
  \[
    \sum_{n = 1}^\infty \int_{[a, b]} f^{(n)} = \lim_{N \to \infty} \sum_{n = 1}^N \int_{[a, b]} f^{(n)} = \lim_{N \to \infty} \int_{[a, b]} \sum_{n = 1}^N f^{(n)} = \int_{[a, b]} f = \int_{[a, b]} \sum_{n = 1}^\infty f^{(n)}.
  \]
\end{proof}

\begin{note}
  \cref{ii:3.6.2} works particularly well in conjunction with the Weierstrass \(M\)-test
  (\cref{ii:3.5.7}).
\end{note}

\exercisesection

\begin{ex}\label{ii:ex:3.6.1}
  Use \cref{ii:3.6.1} to prove \cref{ii:3.6.2}.
\end{ex}

\begin{proof}
  See \cref{ii:3.6.2}.
\end{proof}
