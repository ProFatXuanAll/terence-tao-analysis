\section{Limits at infinity}\label{sec 9.10}

\begin{definition}[Infinite adherent points]\label{9.10.1}
    Let \(X\) be a subset of \(\mathbf{R}\).
    We say that \(+\infty\) is \emph{adherent} to \(X\) iff for every \(M \in \mathbf{R}\) there exists an \(x \in X\) such that \(x > M\);
    we say that \(-\infty\) is \emph{adherent} to \(X\) iff for every \(M \in \mathbf{R}\) there exists an \(x \in X\) such that \(x < M\).
\end{definition}

\begin{note}
    In other words, \(+\infty\) is adherent to \(X\) iff \(X\) has no upper bound, or equivalently iff \(\sup(X) = +\infty\).
    Similarly \(-\infty\) is adherent to \(X\) iff \(X\) has no lower bound, or iff \(\inf(X) = -\infty\).
    Thus a set is bounded if and only if \(+\infty\) and \(-\infty\) are not adherent points.
\end{note}

\begin{remark}\label{9.10.2}
    Definition \ref{9.10.1} may seem rather different from Definition \ref{9.1.8}, but can be unified using the topological structure of the extended real line \(\mathbf{R}^*\).
\end{remark}

\begin{definition}[Limits at infinity]\label{9.10.3}
    Let \(X\) be a subset of \(\mathbf{R}\) with \(+\infty\) as an adherent point, and let \(f : X \to \mathbf{R}\) be a function.
    We say that \emph{\(f(x)\) converges to \(L\)} as \(x \to +\infty\) in \(X\), and write \(\lim_{x \to +\infty ; x \in X} f(x) = L\), iff for every \(\varepsilon > 0\) there exists an \(M\) such that \(f\) is \(\varepsilon\)-close to \(L\) on \(X \cap (M, +\infty)\)
    (i.e., \(\abs*{f(x) - L} \leq \varepsilon\) for all \(x \in X\) such that \(x > M\)).
    Similarly we say that \emph{\(f(x)\) converges to \(L\)} as \(x \to -\infty\) iff for every \(\varepsilon > 0\) there exists an \(M\) such that \(f\) is \(\varepsilon\)-close to \(L\) on \(X \cap (-\infty, M)\).
\end{definition}