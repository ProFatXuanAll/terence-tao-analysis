\section{Suprema and Infima of sequences}

\begin{definition}[Sup and inf of sequences]\label{6.3.1}
Let \((a_n)_{n = m}^\infty\) be a sequence of real numbers.
Then we define \(\sup(a_n)_{n = m}^\infty\) to be the supremum of the set \(\{a_n : n \geq m\}\), and \(\inf(a_n)_{n = m}^\infty\) to the infimum of the same set \(\{a_n : n \geq m\}\).
\end{definition}

\begin{remark}\label{6.3.2}
The quantities \(\sup(a_n)_{n = m}^\infty\) and \(\inf(a_n)_{n = m}^\infty\) are sometimes written as \(\sup_{n \geq m} a_n\) and \(\inf_{n \geq m} a_n\) respectively.
\end{remark}

\setcounter{theorem}{3}
\begin{example}\label{6.3.4}
Let \(a_n \coloneqq 1 / n\);
thus \((a_n)_{n = 1}^\infty\) is the sequence \(1, 1 / 2, 1 / 3, \dots\).
Then the set \(\{a_n : n \geq 1\}\) is the countable set \(\{1, 1 / 2, 1 / 3, 1 / 4, \dots\}\).
Thus \(\sup(a_n)_{n = 1}^\infty = 1\) and \(\inf(a_n)_{n = 1}^\infty = 0\).
\end{example}

\begin{proof}
We first show that \(\sup(a_n)_{n = 1}^\infty = 1\).
By the given condition we have \(\forall\ n \in \mathds{N}\) and \(n \geq 1\), \(a_n \leq 1\).
So if \(x \in \mathds{R}\) and \(x < 1\), then \(x < a_1\), which means \(x\) is not an upper bound of \((a_n)_{n = 1}^\infty\).
Thus \(\sup(a_n)_{n = 1}^\infty = 1\).

Now we show that \(\inf(a_n)_{n = 1}^\infty = 0\).
Because \(\forall\ n \in \mathds{N}\) and \(n \geq 1\), we have \(-a_n = -1 / n \leq 0\).
So \(0\) is an upper bound of \(\{-a_n : n \geq 1\}\), and \(\sup(\{-a_n : n \geq 1\}) \leq 0\).
Then we have
\begin{align*}
\inf(a_n)_{n = 1}^\infty &= \inf(\{a_n : n \geq 1\}) & \text{(by Definition \ref{6.3.1})} \\
&= -\sup(-\{a_n : n \geq 1\}) & \text{(by Definition \ref{6.2.6})} \\
&= -\sup(\{-a_n : n \geq 1\}) & \text{(by Definition \ref{6.2.6})} \\
&\geq 0.
\end{align*}
So \(0\) is a lower bound of \(\{a_n : n \geq 1\}\), i.e., \(0 \leq \inf(a_n)_{n = 1}^\infty\).
Suppose for sake of contradiction that \(\exists\ x \in \mathds{R}\) such that \(x > 0\) and \(x = \inf(a_n)_{n = 1}^\infty\).
Then we must have \(\forall\ n \in \mathds{N}\) and \(n \geq 1\), \(0 < x \leq a_n\).
But by Proposition \ref{5.4.12}, \(\exists\ q \in \mathds{Q}\) and \(q > 0\) such that \(q \leq x\).
Let such \(q = a / b\), where \(a, b \in \mathds{Z}\) and \(a, b > 0\).
Since \(b \in \mathds{Z}\) and \(b > 0\), we have both \(1 / b, 1 / (b + 1) \in \{a_n : n \geq 1\}\).
Since \(1 / (b + 1) < 1 / b \leq a / b\), we have \(1 / (b + 1) < x\), which contradict to \(x \leq a_n\).
Thus \(\nexists\ x \in \mathds{R}\) such that \(x > 0\) and \(x = \inf(a_n)_{n = 1}^\infty\), therefore \(\inf(a_n)_{n = 1}^\infty = 0\).
\end{proof}

\begin{note}
It is a little inaccurate to think of the supremum and infimum as the ``largest element of the sequence'' and ``smallest element of the sequence'' respectively.
\end{note}

\begin{note}
It is possible for the supremum or infimum of a sequence to be \(+\infty\) or \(-\infty\).
However, if a sequence \((a_n)_{n = m}^\infty\) is bounded, say bounded by \(M\), then all the elements \(a_n\) of the sequence lie between \(-M\) and \(M\), so that the set \(\{a_n : n \geq m\}\) has \(M\) as an upper bound and \(-M\) as a lower bound.
Since this set is clearly non-empty, we can thus conclude that the supremum and infimum of a bounded sequence are real numbers (i.e., not \(+\infty\) and \(-\infty\)).
\end{note}

\exercisesection

\begin{exercise}\label{ex 6.3.1}
Verify the claim in Example \ref{6.3.4}.
\end{exercise}

\begin{proof}
See Example \ref{6.3.4}.
\end{proof}