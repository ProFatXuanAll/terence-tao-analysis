\section{Gaps in the rational numbers}

\begin{proposition}[Interspersing of integers by rationals]\label{4.4.1}
Let \(x\) be a rational number.
Then there exists an integer \(n\) such that \(n \leq x < n + 1\).
In fact, this integer is unique (i.e., for each \(x\) there is only one \(n\) for which \(n \leq x < n + 1\)).
In particular, there exists a natural number \(N\) such that \(N > x\)
(i.e., there is no such thing as a rational number which is larger than all the natural numbers).
\end{proposition}

\begin{proof}
We first prove that \(\forall\ x \in \mathds{Q}\), \(\exists\ n \in \mathds{Z}\) such that \(n \leq x < n + 1\).
By Lemma \ref{4.2.7}, exactly one of the following three statements is true:
\begin{enumerate}
    \item \(x = 0\).
    Then by Definition \ref{4.2.8}, \(0 \leq 0\).
    By Definition \ref{2.2.11} and Definition \ref{2.2.1}, \(0 + 1 = 1 \implies 0 < 1\).
    So \(0 \leq 0 < 1\).
    \item \(x\) is a positive rational number.
    Then by Definition \ref{4.2.6}, \(x = a / b\), where \(a, b \in \mathds{Z}^+\).
    By Proposition \ref{2.3.9}, \(a = nb + r\), where \(n, r \in \mathds{N}\) and \(0 \leq r < b\).
    So
    \begin{align*}
    x &= a / b \\
    &= (nb + r) / b \\
    &= (nb + r)b^{-1} & \text{(by Definition \ref{4.3.11})} \\
    &= (nb)b^{-1} + rb^{-1} & \text{(by Proposition \ref{4.2.4})} \\
    &= n(bb^{-1}) + rb^{-1} & \text{(by Proposition \ref{4.2.4})} \\
    &= n1 + rb^{-1} & \text{(by Proposition \ref{4.2.4})} \\
    &= n + rb^{-1}. & \text{(by Proposition \ref{4.2.4})}
    \end{align*}
    By Proposition \ref{4.2.9}, \(0 \leq r < b \implies 0b^{-1} \leq rb^{-1} < bb^{-1}\).
    Since \(0b^{-1} = 0\) and \(bb^{-1} = 1\) by Proposition \ref{4.2.4}, \(0 \leq rb^{-1} < 1\).
    Again by Proposition \ref{4.2.9}, \(0 \leq rb^{-1} < 1 \implies n + 0 \leq n + rb^{-1} < n + 1\).
    Again By Proposition \ref{4.2.4}, \(n + 0 = n\).
    So \(n \leq x < n + 1\).
    \item \(x\) is a negative rational number.
    Then by Definition \ref{4.2.6}, \(x = (-a) / b\), where \(a, b \in \mathds{Z}^+\).
    By Proposition \ref{2.3.9}, \(a = mb + r\), where \(m, r \in \mathds{N}\) and \(0 \leq r < b\).
    So
    \begin{align*}
    x &= (-a) / b \\
    &= -(mb + r) / b \\
    &= ((-1)(mb + r)) / b & \text{(by Additional Corollary \ref{ac 4.2.3})} \\
    &= ((-1)(mb + r))b^{-1} & \text{(by Definition \ref{4.3.11})} \\
    &= (-1)((mb + r)b^{-1}) & \text{(by Proposition \ref{4.2.4})} \\
    &= (-1)((mb)b^{-1} + rb^{-1}) & \text{(by Proposition \ref{4.2.4})} \\
    &= (-1)(m(bb^{-1}) + rb^{-1}) & \text{(by Proposition \ref{4.2.4})} \\
    &= (-1)(m1 + rb^{-1}) & \text{(by Proposition \ref{4.2.4})} \\
    &= (-1)(m + rb^{-1}) & \text{(by Proposition \ref{4.2.4})} \\
    &= (-1)m + (-1)(rb^{-1}) & \text{(by Proposition \ref{4.2.4})} \\
    &= (-m) + (-(rb^{-1})). & \text{(by Additional Corollary \ref{ac 4.2.3})} \\
    \end{align*}
    By Proposition \ref{4.2.9}, \(0 \leq r < b \implies 0b^{-1} \leq rb^{-1} < bb^{-1}\).
    Since \(0b^{-1} = 0\) and \(bb^{-1} = 1\) by Proposition \ref{4.2.4}, \(0 \leq rb^{-1} < 1\).
    By Exercise \ref{ex 4.2.6}, \(0 \leq rb^{-1} < 1 \implies (-1)1 < (-1)(rb^{-1}) \leq (-1)0\).
    By Additional Corollary \ref{ac 4.2.3}, \((-1)(rb^{-1}) = -(rb^{-1})\).
    Since \((-1)1 = (-1)\) and \((-1)0 = 0\), \(-1 < -(rb^{-1}) \leq 0\).
    Again by Proposition \ref{4.2.9}, \(-1 < -(rb^{-1}) \leq 0 \implies (-m) + (-1) < (-m) + (-(rb^{-1})) \leq (-m) + 0\).
    Again By Proposition \ref{4.2.4}, \((-m) + 0 = m\).
    So \((-m) + (-1) < x \leq -m\).
    Now we futher split into two cases:
    \begin{enumerate}[label=(\roman*)]
        \item \(x = -m\).
        Then by Definition \ref{4.2.8}, \(x = -m \implies -m \leq x\).
        And by Proposition \ref{4.2.4} and Definition \ref{4.2.8}, \(((-m) + 1) - (-m) = 1 > 0 \implies -m < (-m) + 1\).
        Let \(n = -m\).
        So \(-m \leq x < (-m) + 1 \implies n \leq x < n + 1\).
        \item \(x < -m\).
        Then by Definition \ref{4.2.8}, \((-m) + 1 < x \implies (-m) + 1 \leq x\).
        Let \(n = (-m) + (-1)\), then by Proposition \ref{4.1.6}, \(n + 1 = ((-m) + (-1)) + 1 = (-m) + ((-1) + 1) = (-m) + 0 = -m\).
        So \((-m) + (-1) \leq x < -m \implies n \leq x < n + 1\).
    \end{enumerate}
\end{enumerate}
From all cases above, we conclude that \(\exists\ n \in \mathds{Z}\), \(n \leq x < n + 1\).

Now we prove that \(\forall\ x \in \mathds{Q}\), \(\exists!\ n\) for which \(n \leq x < n + 1\).
Suppose that \(\exists\ n_1, n_2 \in \mathds{N}\) such that \(n_1 \leq x < n_1 + 1\) and \(n_2 \leq x < n_2 + 1\).
We want to show that \(n_1 = n_2\).
Then we get \(n_1 < n_2 + 1\) and \(n_2 < n_1 + 1\).
So
\begin{align*}
& (n_1 < n_2 + 1) \land (n_2 < n_1 + 1) \\
\implies & (n_1 + 1 \leq n_2 + 1) \land (n_2 + 1 \leq n_1 + 1) & \text{(by Proposition \ref{2.2.12})} \\
\implies & (n_1 \leq n_2) \land (n_2 \leq n_1) & \text{(by Proposition \ref{2.2.12})} \\
\implies & n_1 = n_2. & \text{(by Proposition \ref{2.2.12})}
\end{align*}

Finally, we prove that \(\forall\ x \in \mathds{Q}\), \(\exists\ N \in \mathds{N}\) such that \(N > x\).
Because \(\forall\ x \in \mathds{Q}\), \(\exists!\ n \in \mathds{N}\), \(n \leq x < n + 1\).
By setting \(N = n + 1\), we get \(x < N\).
By Proposition \ref{4.2.9}, \(x < N \implies N > x\).
\end{proof}

\exercisesection

\begin{exercise}\label{ex 4.4.1}
Prove Proposition \ref{4.4.1}.
\end{exercise}

\begin{proof}
See Proposition \ref{4.4.1}.
\end{proof}