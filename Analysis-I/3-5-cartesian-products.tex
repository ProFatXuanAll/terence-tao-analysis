\section{Cartesian products}\label{i:sec:3.5}

\begin{defn}[Ordered pair]\label{i:3.5.1}
  If \(x\) and \(y\) are any objects (possibly equal), we define the \emph{ordered pair} \((x, y)\) to be a new object, consisting of \(x\) as its first component and \(y\) as its second component.
  Two ordered pairs \((x, y)\) and \((x', y')\) are considered equal iff both their components match, i.e.
  \[
    (x, y) = (x', y') \iff (x = x' \text{ and } y = y').
  \]
  This is consistent with the usual axioms of equality (\cref{i:ex:3.5.3}).
\end{defn}

\begin{rmk}\label{i:3.5.2}
  Strictly speaking, \cref{i:3.5.1} is partly an axiom, because we have simply postulated that given any two objects \(x\) and \(y\), that an object of the form \((x, y)\) exists.
  However, it is possible to define an ordered pair using the axioms of set theory in such a way that we do not need any further postulates (see \cref{i:ex:3.5.1}).
\end{rmk}

\begin{rmk}\label{i:3.5.3}
  We have now ``overloaded'' the parenthesis symbols \(()\) once again;
  they now are not only used to denote grouping of operators and arguments of functions, but also to enclose ordered pairs.
  This is usually not a problem in practice as one can still determine what usage the symbols \(()\) were intended for from context.
\end{rmk}

\begin{defn}[Cartesian product]\label{i:3.5.4}
  If \(X\) and \(Y\) are sets, then we define the \emph{Cartesian product} \(X \times Y\) to be the collection of ordered pairs, whose first component lies in \(X\) and second component lies in \(Y\), thus
  \[
    X \times Y \coloneqq \set{(x, y) : x \in X, y \in Y}
  \]
  or equivalently,
  \[
    a \in X \times Y \iff (a = (x, y) \text{ for some } x \in X \text{ and } y \in Y).
  \]
\end{defn}

\begin{rmk}\label{i:3.5.5}
  One can show that the Cartesian product \(X \times Y\) is indeed a set;
  see \cref{i:ex:3.5.1}.
\end{rmk}

\begin{note}
  Let \(f : X \times Y \to Z\) be a function whose domain \(X \times Y\) is a Cartesian product of two other sets \(X\) and \(Y\).
  Then \(f\) can either be thought of as a function of one variable, mapping the single input of an ordered pair \((x, y)\) in \(X \times Y\) to an output \(f(x, y)\) in \(Z\), or as a function of two variables, mapping an input \(x \in X\) and another input \(y \in Y\) to a single output \(f(x, y)\) in \(Z\).
  While the two notions are technically different, we will not bother to distinguish the two, and think of \(f\) simultaneously as a function of one variable with domain \(X \times Y\) and as a function of two variables with domains \(X\) and \(Y\).
  Thus for instance the addition operation \(+\) on the natural numbers can now be re-interpreted as a function \(+ : N \times N \to N\), defined by \((x, y) \mapsto x + y\).
\end{note}

\setcounter{thm}{6}
\begin{defn}[Ordered \(n\)-tuple and \(n\)-fold Cartesian product]\label{i:3.5.7}
  Let \(n\) be a natural number.
  An \emph{ordered \(n\)-tuple} \((x_i)_{1 \leq i \leq n}\) (also denoted \((x_1, \dots, x_n)\)) is a collection of objects \(x_i\), one for every natural number \(i\) between \(1\) and \(n\);
  we refer to \(x_i\) as the \emph{\(i^{th}\) component} of the ordered \(n\)-tuple.
  Two ordered \(n\)-tuples \((x_i)_{1 \leq i \leq n}\) and \((y_i)_{1 \leq i \leq n}\) are said to be equal iff \(x_i = y_i\) for all \(1 \leq i \leq n\).
  If \((X_i)_{1 \leq i \leq n}\) is an ordered \(n\)-tuple of sets, we define their \emph{Cartesian product} \(\prod_{1 \leq i \leq n} X_i\) (also denoted \(\prod_{i=1}^n X_i\) or \(X_1 \times \dots \times X_n\)) by
  \[
    \prod_{1 \leq i \leq n} X_i \coloneqq \set{(x_i)_{1 \leq i \leq n} : x_i \in X_i \text{ for all } 1 \leq i \leq n}.
  \]
\end{defn}

\begin{note}
  Again, \cref{i:3.5.7} simply postulates that an ordered \(n\)-tuple and a Cartesian product always exist when needed, but using the axioms of set theory one can explicitly construct these objects (\cref{i:ex:3.5.2}).
\end{note}

\begin{rmk}\label{i:3.5.8}
  One can show that \(\prod_{1 \leq i \leq n} X_i\) is indeed a set.
  Indeed, from the power set axiom we can consider the set of all functions \(i \mapsto x_i\) from the domain \(\set{1 \leq i \leq n}\) to the codomain \(\bigcup_{1 \leq i \leq n} X_i\), and then we can restrict using the axiom of specification to restrict to those functions \(i \mapsto x_i\) for which \(x_i \in X_i\) for all \(1 \leq i \leq n\).
  One can generalize this construction to infinite Cartesian products, see \cref{i:8.4.1}.
\end{rmk}

\begin{note}
  Strictly speaking, the sets \(X_1 \times X_2 \times X_3\), \((X_1 \times X_2) \times X_3\), and \(X_1 \times (X_2 \times X_3)\) are distinct.
  However, they are clearly very related to each other (for instance, there are obvious bijections between any two of the three sets), and it is common in practice to neglect the minor distinctions between these sets and pretend that they are in fact equal.
  Thus a function \(f : X_1 \times X_2 \times X_3 \to Y\) can be thought of as a function of one variable \((x_1, x_2, x_3) \in X_1 \times X_2 \times X_3\), or as a function of three variables \(x_1 \in X_1\), \(x_2 \in X_2\), \(x_3 \in X_3\), or as a function of two variables \(x_1 \in X_1\), \((x_2, x_3) \in X_2 \times X_3\), and so forth;
  we will not bother to distinguish between these different perspectives.
\end{note}

\setcounter{thm}{9}
\begin{rmk}\label{i:3.5.10}
  An ordered \(n\)-tuple \(x_1, \dots, x_n\) of objects is also called an \emph{ordered sequence} of \(n\) elements, or a \emph{finite sequence} for short.
  In \cref{i:ch:5} we shall also introduce the very useful concept of an \emph{infinite sequence}.
\end{rmk}

\begin{eg}\label{i:3.5.11}
  If \(x\) is an object, then \((x)\) is a \(1\)-tuple, which we shall identify with \(x\) itself (even though the two are, strictly speaking, not the same object).
  Then if \(X_1\) is any set, then the Cartesian product \(\prod_{1 \leq i \leq 1} X_i\) is just \(X_1\).
  Also, the \emph{empty Cartesian product} \(\prod_{1 \leq i \leq 0} X_i\) gives, not the empty set \(\set{}\), but rather the singleton set \(\set{()}\) whose only element is the \emph{\(0\)-tuple} \(()\), also known as the \emph{empty tuple}.

  If \(n\) is a natural number, we often write \(X^n\) as shorthand for the \(n\)-fold Cartesian product \(X^n \coloneqq \prod_{1 \leq i \leq n} X\).
  Thus \(X^1\) is essentially the same set as \(X\) (if we ignore the distinction between an object \(x\) and the \(1\)-tuple \((x)\)), while \(X^2\) is the Cartesian product \(X \times X\).
  The set \(X^0\) is a singleton set \(\set{()}\).
\end{eg}

\begin{proof}[\pf{i:3.5.11}]
  First we show that \(\prod_{1 \leq i \leq 1} X_i = X_1\).
  This is true since
  \[
    \prod_{1 \leq i \leq 1} X_i = \set{(x) : x \in X_1} = \set{x : x \in X_1} = X_1.
  \]

  Next we show that \((x_i)_{1 \leq i \leq 0} = ()\).
  By \cref{i:ex:3.5.2} we see that \(x : \set{i \in \N : 1 \leq i \leq 0} \to Y\) is a surjective function where \(Y\) is an arbitrary set.
  Clearly the domain of \(x\) is the empty set.
  Since the \(i^{th}\) component of \(x\) and \(()\) do not exist, we see that the statement ``for \(1 \leq i \leq n\), the \(i^{th}\) component of \(x\) and \(()\) are the same'' is vacuously true for arbitrary natural number \(n\).
  Thus \((x_i)_{1 \leq i \leq 0} = ()\).

  Next we show that \(\prod_{1 \leq i \leq 0} X_i = \set{()}\).
  This is true since
  \begin{align*}
    \prod_{1 \leq i \leq 0} X_i & = \set{(x_i)_{1 \leq i \leq 0} : x_i \in X_i \text{ for all } 1 \leq i \leq 0} &  & \by{i:3.5.7}                  \\
                                & = \set{() : x_i \in X_i \text{ for all } 1 \leq i \leq 0}                      &  & \text{(from the proof above)} \\
                                & = \set{()}.                                                                    &  & \by{i:3.2}
  \end{align*}

  Next we show that \(X^1 = X\).
  This is true since
  \begin{align*}
    X^1 & = \prod_{1 \leq i \leq 1} X &  & \text{(by definition)}        \\
        & = X.                        &  & \text{(from the proof above)}
  \end{align*}

  Finally we show that \(X^0 = \set{()}\).
  This is true since
  \begin{align*}
    X^0 & = \prod_{1 \leq i \leq 0} X &  & \text{(by definition)}        \\
        & = \set{()}.                 &  & \text{(from the proof above)}
  \end{align*}
\end{proof}

\setcounter{thm}{11}
\begin{lem}[Finite choice]\label{i:3.5.12}
  Let \(n \geq 1\) be a natural number, and for each natural number \(1 \leq i \leq n\), let \(X_i\) be a non-empty set.
  Then there exists an \(n\)-tuple \((x_i)_{1 \leq i \leq n}\) such that \(x_i \in X_i\) for all \(1 \leq i \leq n\).
  In other words, if each \(X_i\) is non-empty, then the set \(\prod_{1 \leq i \leq n} X_i\) is also non-empty.
\end{lem}

\begin{proof}[\pf{i:3.5.12}]
  We induct on \(n\) (starting with the base case \(n = 1\); the claim is also vacuously true with \(n = 0\) but is not particularly interesting in that case).
  When \(n = 1\) the claim follows from \cref{i:3.1.6}.
  Now suppose inductively that the claim has already been proven for some \(n\);
  we will now prove it for \(n\pp\).
  Let \(X_1, \dots, X_{n\pp}\) be a collection of non-empty sets.
  By induction hypothesis, we can find an \(n\)-tuple \((x_i)_{1 \leq i \leq n}\) such that \(x_i \in X_i\) for all \(1 \leq i \leq n\).
  Also, since \(X_{n\pp}\) is non-empty, by \cref{i:3.1.6} we may find an object \(a\) such that \(a \in X_{n\pp}\).
  If we thus define the \(n\pp\)-tuple \((y_i)_{1 \leq i \leq n\pp}\) by setting \(y_i \coloneqq x_i\) when \(1 \leq i \leq n\) and \(y_i \coloneqq a\) when \(i = n\pp\) it is clear that \(y_i \in X_i\) for all \(1 \leq i \leq n\pp\), thus closing the induction.
\end{proof}

\begin{rmk}\label{i:3.5.13}
  It is intuitively plausible that this lemma should be extended to allow for an infinite number of choices, but this cannot be done automatically;
  it requires an additional axiom, the \emph{axiom of choice}.
  See \cref{i:sec:8.4}.
\end{rmk}

\exercisesection

\begin{ex}\label{i:ex:3.5.1}
  \begin{enumerate}
    \item Suppose we \emph{define} the ordered pair \((x, y)\) for any objects \(x\) and \(y\) by the formula \((x, y) \coloneqq \set{\set{x}, \set{x, y}}\)
          (thus using several applications of \cref{i:3.3}).
          Show that such a definition indeed obeys the \cref{i:3.5.1}.
          Thus this definition can be validly used as a definition of an ordered pair.
    \item For an additional challenge, show that the alternate definition \((x, y) := \set{x, \set{x, y}}\) also verifies \cref{i:3.5.1} and is thus also an acceptable definition of ordered pair.
    \item Show that regardless of the definition of ordered pair, the Cartesian product \(X \times Y\) is a set.
  \end{enumerate}
\end{ex}

\begin{proof}[\pf{i:ex:3.5.1}(a)]
  Suppose that \(x, x', y, y'\) are objects.
  By definition we have
  \begin{align*}
    (x, y)   & = \set{\set{x}, \set{x, y}}     \\
    (x', y') & = \set{\set{x'}, \set{x', y'}}.
  \end{align*}
  Then we have
  \begin{align*}
         & \begin{dcases}
             x = x' \\
             y = y'
           \end{dcases}                                                                    \\
    \iff & \begin{dcases}
             x = x' \\
             \pa{x = y = y' = x'} \lor \pa{x \neq y = y' \neq x'}
           \end{dcases}                              \\
    \iff & \begin{dcases}
             \set{x} = \set{x'} \\
             \set{x, y} = \set{x', y'}
           \end{dcases}                                &  & \by{i:3.3}                      \\
    \iff & \set{\set{x}, \set{x, y}} = \set{\set{x'}, \set{x', y'}} &  & \by{i:3.3}         \\
    \iff & (x, y) = (x', y').                                       &  & \by{i:ex:3.5.1}[a]
  \end{align*}
  Thus \cref{i:ex:3.5.1}(a) is a valid definition of ordered pairs.
\end{proof}

\begin{proof}[\pf{i:ex:3.5.1}(b)]
  Suppose that \(x, x', y, y'\) are objects.
  By definition we have
  \begin{align*}
    (x, y)   & = \set{x, \set{x, y}}     \\
    (x', y') & = \set{x', \set{x', y'}}.
  \end{align*}
  Then we have
  \begin{align*}
         & \begin{dcases}
             x = x' \\
             y = y'
           \end{dcases}                                                                \\
    \iff & \begin{dcases}
             x = x' \\
             \pa{x = y = y' = x'} \lor \pa{x \neq y = y' \neq x'}
           \end{dcases}                          \\
    \iff & \begin{dcases}
             x = x' \\
             \set{x, y} = \set{x', y'}
           \end{dcases}                            &  & \by{i:3.3}                      \\
    \iff & \set{x, \set{x, y}} = \set{x', \set{x', y'}}         &  & \by{i:3.3}         \\
    \iff & (x, y) = (x', y').                                   &  & \by{i:ex:3.5.1}[b]
  \end{align*}
  Thus \cref{i:ex:3.5.1}(b) is a valid definition of ordered pairs.
\end{proof}

\begin{proof}[\pf{i:ex:3.5.1}(c)]
  We use \cref{i:ex:3.5.1}(a) as the definition of ordered pairs.
  Suppose that \(X, Y\) are sets.
  For each \(x \in X\), we can use \cref{i:3.3} to create singleton set \(\set{x}\).
  Using \cref{i:3.3} again we can create pair sets \(\set{x, y}\) and \(\set{\set{x}, \set{x, y}}\) for each \(x \in X\) and each \(y \in Y\).
  Since \(Y\) is a set, each object in \(Y\) is uniquely identified (\cref{i:3.1.1}).
  Thus for each \(x \in X\) and each \(y \in Y\), the statement ``\(P_y(x, y') \coloneqq y' = y\)'' is only true for one \(y' \in Y\), namely \(y\).
  Thus for each \(y \in Y\), we can use \cref{i:3.6} to create the set \(S_y\) using \(P_y\):
  \begin{align*}
    S_y & = \set{\set{\set{x}, \set{x, y'}} : P_y(x, y') \text{ is true for some } x \in X} &  & \by{i:3.6}       \\
        & = \set{\set{\set{x}, \set{x, y}} : x \in X}                                                             \\
        & = \set{(x, y) : x \in X}.                                                         &  & \by{i:ex:3.5.1}.
  \end{align*}
  By \cref{i:3.11} we see that \(\bigcup_{y \in Y} S_y = \set{(x, y) : x \in X, y \in Y}\), which equals to \(X \times Y\) (\cref{i:3.5.4}).
  Thus \(X \times Y\) is a set.
\end{proof}

\begin{ex}\label{i:ex:3.5.2}
  Suppose we \emph{define} an ordered \(n\)-tuple to be a surjective function
  \[
    x : \set{i \in \N : 1 \leq i \leq n} \to X
  \]
  whose codomain is some arbitrary set \(X\) (so different ordered \(n\)-tuples are allowed to have different codomains);
  we then write \(x_i\) for \(x(i)\), and also write \(x\) as \((x_i)_{1 \leq i \leq n}\).
  Using this definition, verify that we have \((x_i)_{1 \leq i \leq n} = (y_i)_{1 \leq i \leq n}\) iff \(x_i = y_i\) for all \(1 \leq i \leq n\).
  Also, show that if \((X_i)_{1 \leq i \leq n}\) are an ordered \(n\)-tuple of sets, then the Cartesian product, as defined in \cref{i:3.5.7}, is indeed a set.
  (Technically, this construction of ordered \(n\)-tuple is not compatible with the construction of ordered pair in \cref{i:ex:3.5.1}, but this does not cause a difficulty in practice;
  for instance, one can use the definition of an ordered \(2\)-tuple here to replace the construction in \cref{i:ex:3.5.1}, or one can make a rather pedantic distinction between an ordered \(2\)-tuple and an ordered pair in one's mathematical arguments.)
\end{ex}

\begin{proof}[\pf{i:ex:3.5.2}]
  First we show that \cref{i:ex:3.5.2} is a valid definition of \(n\)-tuple.
  Let \(n \in \N\), let \(X\) be a set and let \(I = \set{i \in \N : 1 \leq i \leq n}\).
  Let \(x : I \to X, y : I \to X\) be two surjective functions.
  By definition we have \(x = (x_i)_{1 \leq i \leq n}\) and \(y = (y_i)_{1 \leq i \leq n}\).
  Then we have
  \begin{align*}
         & (x_i)_{1 \leq i \leq n} = (y_i)_{1 \leq i \leq n}                      \\
    \iff & x = y                                             &  & \by{i:ex:3.5.2} \\
    \iff & \forall i \in I, x(i) = y(i)                      &  & \by{i:3.3.1}    \\
    \iff & \forall i \in I, x_i = y_i.                       &  & \by{i:ex:3.5.2}
  \end{align*}
  Thus \cref{i:ex:3.5.2} is a valid definition of \(n\)-tuple.

  Now we show that \(\prod_{i = 1}^n X_i\) defined in \cref{i:3.5.7} is indeed a set.
  Suppose that \((X_i)_{1 \leq i \leq n}\) is an ordered \(n\)-tuple of sets.
  Let \(I = \set{i \in \N : 1 \leq i \leq n}\).
  By definition we have a surjective function \(X : I \to Y\) where \(Y\) is a set and \(X(i) = X_i \in Y\) for each \(i \in I\).
  Then we can create the following set
  \begin{align*}
    \bigcup \set{X(i) : i \in I} & = \bigcup_{i \in I} X(i) &  & \by{i:3.11}     \\
                                 & = \bigcup_{i \in I} X_i. &  & \by{i:ex:3.5.2}
  \end{align*}
  By \cref{i:ex:3.4.7} there exists a set of all partial function with domain \(I\) and codomain \(\bigcup_{i \in I} X_i\), and we denote this set \(A\).
  By \cref{i:3.5} there exists a set \(B = \set{x \in A : \forall i \in I, x_i \in X_i}\).
  Clearly every \(x \in B\) has the form \(x = (x_i)_{1 \leq i \leq n}\).
  And every \(n\)-tuple \((x_i)_{1 \leq i \leq n}\) with \(x_i \in X_i\) for all \(i \in I\) is also in \(B\) thanks to \cref{i:ex:3.4.7}.
  Thus by \cref{i:3.5.7} we see that \(B = \prod_{i = 1}^n X_i\) and therefore \(\prod_{i = 1}^n X_i\) is indeed a set.
\end{proof}

\begin{ex}\label{i:ex:3.5.3}
  Show that the definitions of equality for ordered pair and ordered \(n\)-tuple are consistent with the reflexivity, symmetry, and transitivity axioms in the sense that if these axioms of equality are already assumed to hold for the individual components \(x, y\) of an ordered pair \((x, y)\), then they hold for an ordered pair itself.
\end{ex}

\begin{proof}[\pf{i:ex:3.5.3}]
  Let \((x, y), (x', y'), (x'', y'')\) be ordered pairs.
  We first show that \cref{i:3.5.1} is reflexive.
  Since
  \begin{align*}
             & (x = x) \land (y = y)                   \\
    \implies & (x, y) = (x, y),      &  & \by{i:3.5.1}
  \end{align*}
  we see that \cref{i:3.5.1} is reflexive.

  Next we show that \cref{i:3.5.1} is symmetry.
  Since
  \begin{align*}
         & (x, y) = (x', y')                         \\
    \iff & (x = x') \land (y = y') &  & \by{i:3.5.1} \\
    \iff & (x' = x) \land (y' = y)                   \\
    \iff & (x', y') = (x, y),      &  & \by{i:3.5.1}
  \end{align*}
  we see that \cref{i:3.5.1} is symmetry.

  Next we show that \cref{i:3.5.1} is transitive.
  Since
  \begin{align*}
             & ((x, y) = (x', y')) \land ((x', y') = (x'', y''))                           \\
    \implies & (x = x') \land (y = y') \land (x' = x'') \land (y' = y'') &  & \by{i:3.5.1} \\
    \implies & (x = x'') \land (y = y'')                                                   \\
    \implies & (x, y) = (x'', y''),                                      &  & \by{i:3.5.1}
  \end{align*}
  we see that \cref{i:3.5.1} is transitive.

  Let \(i, n \in \N\) and let \((x_i)_{1 \leq i \leq n}, (y_i)_{1 \leq i \leq n}, (z_i)_{1 \leq i \leq n}\) be ordered \(n\)-tuples.
  Next we show that \cref{i:3.5.7} is reflexive.
  Since
  \begin{align*}
             & \forall 1 \leq i \leq n, x_i = x_i                                   \\
    \implies & (x_i)_{1 \leq i \leq n} = (x_i)_{1 \leq i \leq n}, &  & \by{i:3.5.7}
  \end{align*}
  we see that \cref{i:3.5.7} is reflexive.

  Next we show that \cref{i:3.5.7} is symmetry.
  Since
  \begin{align*}
         & (x_i)_{1 \leq i \leq n} = (y_i)_{1 \leq i \leq n}                    \\
    \iff & \forall 1 \leq i \leq n, x_i = y_i                 &  & \by{i:3.5.7} \\
    \iff & \forall 1 \leq i \leq n, y_i = x_i                                   \\
    \iff & (y_i)_{1 \leq i \leq n} = (x_i)_{1 \leq i \leq n}, &  & \by{i:3.5.7}
  \end{align*}
  we see that \cref{i:3.5.7} is symmetry.

  Finally we show that \cref{i:3.5.7} is transitive.
  Since
  \begin{align*}
             & \pa{(x_i)_{1 \leq i \leq n} = (y_i)_{1 \leq i \leq n}} \land \pa{(y_i)_{1 \leq i \leq n} = (z_i)_{1 \leq i \leq n}}                   \\
    \implies & \forall 1 \leq i \leq n, (x_i = y_i) \land (y_i = z_i)                                                              &  & \by{i:3.5.7} \\
    \implies & \forall 1 \leq i \leq n, x_i = z_i                                                                                                    \\
    \implies & (x_i)_{1 \leq i \leq n} = (z_i)_{1 \leq i \leq n},                                                                  &  & \by{i:3.5.7}
  \end{align*}
  we see that \cref{i:3.5.7} is transitive.
\end{proof}

\begin{ex}\label{i:ex:3.5.4}
  Let \(A, B, C\) be sets.
  Show that \(A \times (B \cup C) = (A \times B) \cup (A \times C)\), that \(A \times (B \cap C) = (A \times B) \cap (A \times C)\), and that \(A \times (B \setminus C) = (A \times B) \setminus (A \times C)\).
\end{ex}

\begin{proof}[\pf{i:ex:3.5.4}]
  We first show that \(A \times (B \cup C) = (A \times B) \cup (A \times C)\).
  Since
  \begin{align*}
         & (x, y) \in A \times (B \cup C)                                                 \\
    \iff & (x \in A) \land (y \in B \cup C)                             &  & \by{i:3.5.4} \\
    \iff & (x \in A) \land ((y \in B) \lor (y \in C))                   &  & \by{i:3.4}   \\
    \iff & ((x \in A) \land (y \in B)) \lor ((x \in A) \land (y \in C))                   \\
    \iff & ((x, y) \in A \times B) \lor ((x, y) \in A \times C)         &  & \by{i:3.5.4} \\
    \iff & (x, y) \in (A \times B) \cup (A \times C),                   &  & \by{i:3.4}
  \end{align*}
  by \cref{i:3.1.4} we have \(A \times (B \cup C) = (A \times B) \cup (A \times C)\).

  Next we show that \(A \times (B \cap C) = (A \times B) \cap (A \times C)\).
  Since
  \begin{align*}
         & (x, y) \in A \times (B \cap C)                                                   \\
    \iff & (x \in A) \land (y \in B \cap C)                              &  & \by{i:3.5.4}  \\
    \iff & (x \in A) \land ((y \in B) \land (y \in C))                   &  & \by{i:3.1.23} \\
    \iff & ((x \in A) \land (y \in B)) \land ((x \in A) \land (y \in C))                    \\
    \iff & ((x, y) \in A \times B) \land ((x, y) \in A \times C)         &  & \by{i:3.5.4}  \\
    \iff & (x, y) \in (A \times B) \cap (A \times C),                    &  & \by{i:3.1.23}
  \end{align*}
  by \cref{i:3.1.4} we have \(A \times (B \cap C) = (A \times B) \cap (A \times C)\).

  Finally we show that \(A \times (B \setminus C) = (A \times B) \setminus (A \times C)\).
  Since
  \begin{align*}
         & (x, y) \in A \times (B \setminus C)                                                 \\
    \iff & (x \in A) \land (y \in B \setminus C)                            &  & \by{i:3.5.4}  \\
    \iff & (x \in A) \land ((y \in B) \land (y \notin C))                   &  & \by{i:3.1.27} \\
    \iff & ((x \in A) \land (y \in B)) \land ((x \in A) \land (y \notin C))                    \\
    \iff & ((x, y) \in A \times B) \land ((x, y) \notin A \times C)         &  & \by{i:3.5.4}  \\
    \iff & (x, y) \in (A \times B) \setminus (A \times C),                  &  & \by{i:3.1.27}
  \end{align*}
  by \cref{i:3.1.4} we have \(A \times (B \setminus C) = (A \times B) \setminus (A \times C)\).
\end{proof}

\begin{ex}\label{i:ex:3.5.5}
  Let \(A, B, C, D\) be sets.
  Show that \((A \times B) \cap (C \times D) = (A \cap C) \times (B \cap D)\).
  Is it true that \((A \times B) \cup (C \times D) = (A \cup C) \times (B \cup D)?\)
  Is it true that \((A \times B) \setminus (C \times D) = (A \setminus C) \times (B \setminus D)?\)
\end{ex}

\begin{proof}[\pf{i:ex:3.5.5}]
  We first show that \((A \times B) \cap (C \times D) = (A \cap C) \times (B \cap D)\).
  Since
  \begin{align*}
         & (x, y) \in (A \times B) \cap (C \times D)                                        \\
    \iff & ((x, y) \in A \times B) \land ((x, y) \in C \times D)         &  & \by{i:3.1.23} \\
    \iff & ((x \in A) \land (y \in B)) \land ((x \in C) \land (y \in D)) &  & \by{i:3.5.4}  \\
    \iff & ((x \in A) \land (x \in C)) \land ((y \in B) \land (y \in D))                    \\
    \iff & (x \in A \cap C) \land (y \in B \cap D)                       &  & \by{i:3.1.23} \\
    \iff & (x, y) \in (A \cap C) \times (B \cap D),                      &  & \by{i:3.5.4}
  \end{align*}
  by \cref{i:3.1.4} we have \((A \times B) \cap (C \times D) = (A \cap C) \times (B \cap D)\).

  We do not have \((A \times B) \cup (C \times D) = (A \cup C) \times (B \cup D)\).
  Let \(A = \set{1}, B = \set{2}, C = \set{3}, D = \set{4}\).
  Then we have
  \begin{align*}
    (A \times B) \cup (C \times D) & = \set{(1, 2)} \cup \set{(3, 4)}        &  & \by{i:3.5.4} \\
                                   & = \set{(1, 2), (3, 4)}.                 &  & \by{i:3.4}   \\
    (A \cup C) \times (B \cup D)   & = \set{1, 3} \times \set{2, 4}          &  & \by{i:3.4}   \\
                                   & = \set{(1, 2), (1, 4), (3, 2), (3, 4)}. &  & \by{i:3.5.4}
  \end{align*}
  Clearly \((A \times B) \cup (C \times D) \neq (A \cup C) \times (B \cup D)\) by \cref{i:3.1.4}.

  We do not have \((A \times B) \setminus (C \times D) = (A \setminus C) \times (B \setminus D)\).
  Let \(A = \set{1, 2}, B = \set{3, 4}, C = \set{1}, D = \set{3}\).
  Then we have
  \begin{align*}
    (A \times B) \setminus (C \times D)    & = \set{(1, 3), (1, 4), (2, 3), (2, 4)} \setminus \set{(1, 3)} &  & \by{i:3.5.4}  \\
                                           & = \set{(1, 4), (2, 3), (2, 4)}.                               &  & \by{i:3.1.27} \\
    (A \setminus C) \times (B \setminus D) & = \set{2} \times \set{4}                                      &  & \by{i:3.1.27} \\
                                           & = \set{(2, 4)}.                                               &  & \by{i:3.5.4}
  \end{align*}
  Clearly \((A \times B) \setminus (C \times D) \neq (A \setminus C) \times (B \setminus D)\) by \cref{i:3.1.4}.
\end{proof}

\begin{ex}\label{i:ex:3.5.6}
  Let \(A, B, C, D\) be non-empty sets.
  Show that \(A \times B \subseteq C \times D\) iff \(A \subseteq C\) and \(B \subseteq D\), and that \(A \times B = C \times D\) iff \(A = C\) and \(B = D\).
  What happens if the hypotheses that the \(A, B, C, D\) are all non-empty are removed?
\end{ex}

\begin{proof}[\pf{i:ex:3.5.6}]
  We first show that if \(A, B, C, D\) are non-empty sets, then \(A \times B \subseteq C \times D \iff (A \subseteq C) \land (B \subseteq D)\).
  This is true since
  \begin{align*}
         & A \times B \subseteq C \times D                                                                                      \\
    \iff & \pa{(x, y) \in A \times B \implies (x, y) \in C \times D}             &  & \by{i:3.1.15}                             \\
    \iff & \pa{((x \in A) \land (y \in B)) \implies ((x \in C) \land (y \in D))} &  & \by{i:3.5.4}                              \\
    \iff & \pa{x \in A \implies x \in C} \land \pa{y \in B \implies y \in D}     &  & (A \neq \emptyset \land B \neq \emptyset) \\
    \iff & (A \subseteq C) \land (B \subseteq D).                                &  & \by{i:3.1.15}
  \end{align*}
  This statement is not true when \((A = \emptyset) \lor (B = \emptyset)\).
  For example, if \(A = \set{1}\) and \(B = C = D = \emptyset\), then \(A \times B = \set{()} \subseteq C \times D = \set{()}\) but \(A \not \subseteq C\).

  Next we show that if \(A, B, C, D\) are non-empty sets, then \(A \times B = C \times D \iff (A = C) \land (B = D)\).
  This is true since
  \begin{align*}
         & A \times B = C \times D                                                                                      \\
    \iff & \pa{(x, y) \in A \times B \iff (x, y) \in C \times D}             &  & \by{i:3.1.4}                          \\
    \iff & \pa{((x \in A) \land (y \in B)) \iff ((x \in C) \land (y \in D))} &  & \by{i:3.5.4}                          \\
    \iff & (x \in A \iff x \in C) \land (y \in B \iff y \in D)               &  & \text{(\(A, B, C, D\) are non-empty)} \\
    \iff & (A = C) \land (B = D).                                            &  & \by{i:3.1.4}
  \end{align*}
  This statement is not true when \(((A = \emptyset) \land (C = \emptyset)) \lor ((B = \emptyset) \land (D = \emptyset))\).
  For example, if \(A = \set{1}\), \(C = \set{2}\) and \(B = D = \emptyset\), then \(A \times B = \set{()} = C \times D\) but \(A \neq C\).
\end{proof}

\begin{ex}\label{i:ex:3.5.7}
  Let \(X, Y\) be sets, and let \(\pi_{X \times Y \to X} : X \times Y \to X\) and \(\pi_{X \times Y \to Y} : X \times Y \to Y\) be the maps \(\pi_{X \times Y \to X}(x, y) \coloneqq x\) and \(\pi_{X \times Y \to Y}(x, y) \coloneqq y\);
  these maps are known as the \emph{co-ordinate functions} on \(X \times Y\).
  Show that for any functions \(f : Z \to X\) and \(g : Z \to Y\), there exists a unique function \(h : Z \to X \times Y\) such that \(\pi_{X \times Y \to X} \circ h = f\) and \(\pi_{X \times Y \to Y} \circ h = g\).
  (Compare this to \cref{i:ex:3.3.8}(d), and to \cref{i:ex:3.1.7}.)
  This function \(h\) is known as the \emph{direct sum} of \(f\) and \(g\) and is denoted \(h = f \oplus g\).
\end{ex}

\begin{proof}[\pf{i:ex:3.5.7}]
  We first show the existence of such function \(h\).
  Suppose that \(X, Y, Z\) are sets and \(f : Z \to X, g : Z \to Y\) be functions.
  Let \(\pi_{X \times Y \to X} : X \times Y \to X, \pi_{X \times Y \to Y} : X \times Y \to Y\) be functions where \(\pi_{X \times Y \to X}(x, y) = x\) and \(\pi_{X \times Y \to Y}(x, y) = y\).
  Both \(\pi_{X \times Y \to X}\) and \(\pi_{X \times Y \to Y}\) are well-defined by \cref{i:3.6}.
  We now define a function \(h : Z \to X \times Y\) where
  \[
    \forall z \in Z, h(z) = (f(z), g(z)).
  \]
  To show that \(h\) is well-defined, by \cref{i:3.3.1} we have to show that \(h\) pass the vertical line test.
  Since \(f, g\) are functions, by \cref{i:3.3.1} we know that for each \(z \in Z\), \(f(z)\) and \(g(z)\) are unique objects in \(X\) and \(Y\), respectively.
  Thus by \cref{i:ex:3.5.3} \((f(z), g(z)) \in X \times Y\) is unique for each \(z \in Z\).
  Therefore \(h(z) \in X \times Y\) is unique for each \(z \in Z\) and thus \(h\) is well-defined.
  Now we have
  \begin{align*}
    \forall z \in Z, (\pi_{X \times Y \to X} \circ h)(z) & = \pi_{X \times Y \to X}(h(z))       &  & \by{i:3.3.10} \\
                                                         & = \pi_{X \times Y \to X}(f(z), g(z))                    \\
                                                         & = f(z).                                                 \\
    \forall z \in Z, (\pi_{X \times Y \to Y} \circ h)(z) & = \pi_{X \times Y \to Y}(h(z))       &  & \by{i:3.3.10} \\
                                                         & = \pi_{X \times Y \to Y}(f(z), g(z))                    \\
                                                         & = g(z).
  \end{align*}
  Thus by \cref{i:3.3.7} we have \(\pi_{X \times Y \to X} \circ h = f\) and \(\pi_{X \times Y \to Y} \circ h = g\).

  Now we show the uniqueness of such function \(h\).
  Suppose that there exists another function \(h' : Z \to X \times Y\) such that \(\pi_{X \times Y \to X} \circ h' = f\) and \(\pi_{X \times Y \to Y} \circ h' = g\).
  Then we have
  \begin{align*}
             & \forall z \in Z, \begin{dcases}
                                  f(z) = \pa{\pi_{X \times Y \to X} \circ h'}(z) = \pi_{X \times Y \to X}\pa{h'(z)} \\
                                  g(z) = \pa{\pi_{X \times Y \to Y} \circ h'}(z) = \pi_{X \times Y \to Y}\pa{h'(z)}
                                \end{dcases} &  & \by{i:3.3.7,i:3.3.10}     \\
    \implies & \forall z \in Z, h'(z) = (f(z), g(z)) = h(z)                                                            \\
    \implies & h' = h.                                                                               &  & \by{i:3.3.7}
  \end{align*}
  Thus \(h\) is unique.
\end{proof}

\begin{ex}\label{i:ex:3.5.8}
  Let \(X_1, \dots, X_n\) be sets.
  Show that the Cartesian product \(\prod_{i = 1}^n X_i\) is empty iff at least one of the \(X_i\) is empty.
\end{ex}

\begin{proof}[\pf{i:ex:3.5.8}]
  We have
  \begin{align*}
         & \emptyset = \prod_{i = 1}^n X_i = \set{(x_i)_{1 \leq i \leq n} : x_i \in X_i \text{ for all } 1 \leq i \leq n} &  & \by{i:3.5.7} \\
    \iff & \exists i \in \N : (1 \leq i \leq n) \land (\forall x_i, x_i \notin X_i)                                       &  & \by{i:3.5.7} \\
    \iff & \exists i \in N : (1 \leq i \leq n) \land (X_i = \emptyset).                                                   &  & \by{i:3.2}
  \end{align*}
\end{proof}

\begin{ex}\label{i:ex:3.5.9}
  Suppose that \(I\) and \(J\) are two sets, and for all \(\alpha \in I\) let \(A_\alpha\) be a set, and for all \(\beta \in J\) let \(B_\beta\) be a set.
  Show that \(\pa{\bigcup_{\alpha \in I} A_\alpha} \cap \pa{\bigcup_{\beta \in J} B_\beta} = \bigcup_{(\alpha, \beta) \in I \times J} (A_\alpha \cap B_\beta)\).
\end{ex}

\begin{proof}[\pf{i:ex:3.5.9}]
  Since
  \begin{align*}
         & x \in \pa{\bigcup_{\alpha \in I} A_\alpha} \cap \pa{\bigcup_{\beta \in J} B_\beta}                           \\
    \iff & \pa{x \in \bigcup_{\alpha \in I} A_\alpha} \land \pa{x \in \bigcup_{\beta \in J} B_\beta} &  & \by{i:3.1.23} \\
    \iff & \pa{\exists \alpha \in I : x \in A_\alpha} \land \pa{\exists \beta \in J : x \in B_\beta} &  & \by{i:3.11}   \\
    \iff & \exists (\alpha, \beta) \in I \times J : (x \in A_\alpha) \land (x \in B_\beta)           &  & \by{i:3.5.4}  \\
    \iff & \exists (\alpha, \beta) \in I \times J : x \in A_\alpha \cap B_\beta                      &  & \by{i:3.1.23} \\
    \iff & x \in \bigcup_{(\alpha, \beta) \in I \times J} (A_\alpha \cap B_\beta),                   &  & \by{i:3.11}
  \end{align*}
  by \cref{i:3.1.4} we have \(\pa{\bigcup_{\alpha \in I} A_\alpha} \cap \pa{\bigcup_{\beta \in J} B_\beta} = \bigcup_{(\alpha, \beta) \in I \times J} (A_\alpha \cap B_\beta)\).
\end{proof}

\begin{ex}\label{i:ex:3.5.10}
  If \(f : X \to Y\) is a function, define the \emph{graph} of \(f\) to be the subset of \(X \times Y\) defined by \(\set{(x, f(x)) : x \in X}\).
  Show that two functions \(f : X \to Y\), \(\tilde{f} : X \to Y\) are equal iff they have the same graph.
  Conversely, if \(G\) is any subset of \(X \times Y\) with the property that for each \(x \in X\), the set \(\set{y \in Y : (x, y) \in G}\) has exactly one element (or in other words, \(G\) obeys the \emph{vertical line test}), show that there is exactly one function \(f : X \to Y\) whose graph is equal to \(G\).
\end{ex}

\begin{proof}[\pf{i:ex:3.5.10}]
  The first statement is true since
  \begin{align*}
         & f = \tilde{f}                                                                       \\
    \iff & \forall x \in X, f(x) = \tilde{f}(x)                              &  & \by{i:3.3.7} \\
    \iff & \forall x \in X, (x, f(x)) = \pa{x, \tilde{f}(x)}                 &  & \by{i:3.5.1} \\
    \iff & \set{(x, f(x)) : x \in X} = \set{\pa{x, \tilde{f}(x)} : x \in X}. &  & \by{i:3.6}
  \end{align*}

  Next we show that there exists a function whose graph is equal to \(G\).
  For each \(x \in X\), we use \cref{i:3.6} to create the set \(S_x = \set{y \in Y : (x, y) \in G}\).
  Define \(P(x, y)\) to be the statement ``\(y \in S_x\)'' for each \((x, y) \in X \times Y\).
  Then by the hypothesis of \(G\) we know that for each \(x \in X\), there is only one \(y \in Y\) such that \(P(x, y)\) is true.
  Thus we can use \cref{i:3.3.1} to create a function \(f : X \to Y\) such that
  \[
    \forall (x, y) \in X \times Y, y = f(x) \iff P(x, y) \iff y \in S_x \iff (x, y) \in G.
  \]
  By definition the graph of \(f\) is \(\set{(x, f(x)) : x \in X}\).
  Thus we have
  \[
    \forall (x, y) \in X \times Y, (x, y) \in \set{(x, f(x)) : x \in X} \iff y = f(x) \iff (x, y) \in G.
  \]
  By \cref{i:3.1.4} this means \(G = \set{(x, f(x)) : x \in X}\).
  Thus \(f\) is a function whose graph is equal to \(G\).

  Now we show that \(f\) is unique.
  Suppose that there exists another function \(\tilde{f} : X \to Y\) such that the graph of \(\tilde{f}\) is equal to \(G\).
  But then \(f\) and \(\tilde{f}\) have the same graph, and by the first part of the proof we see that \(f = \tilde{f}\).
  Thus \(f\) is unique.
\end{proof}

\begin{ex}\label{i:ex:3.5.11}
  Show that \cref{i:3.10} can in fact be deduced from \cref{i:3.4.9} and the other axioms of set theory, and thus \cref{i:3.4.9} can be used as an alternate formulation of the power set axiom.
\end{ex}

\begin{proof}[\pf{i:ex:3.5.11}]
  Suppose that \(X, Y\) are sets.
  Then by \cref{i:ex:3.5.1}(c), \(X \times Y\) is a set.
  Thus we can use \cref{i:3.4.9} to create a set of subsets of \(X \times Y\), i.e., \(S_1 = \set{S : S \subseteq X \times Y}\).
  If \(f : X \to Y\) is a function, then by \cref{i:3.3.1} every element on \(X\) must be assigned exactly one object in \(Y\).
  Thus we use \cref{i:3.5} to rule out those sets violating the vertical line test and we derive the following set
  \[
    S_2 = \set{G \in S_1 \mid \forall x \in X, \exists! y \in Y : (x, y) \in G}.
  \]
  By \cref{i:ex:3.5.10} we see that every element \(G \in S_2\) is a graph and we know that there exists exactly one function \(f : X \to Y\) whose graph is \(G\).
  Thus we can use \cref{i:3.6} to create the following set
  \[
    S_3 = \set{f : X \to Y \mid \text{the graph of } f \text{ is in } S_2}.
  \]
  Now we claim that every function \(f : X \to Y\) is an element of \(S_3\), and thus by \cref{i:3.10} we have \(S_3 = X^Y\) and \(X^Y\) is indeed a set.
  But by \cref{i:ex:3.5.10} every function with domain \(X\) and codomain \(Y\) is uniquely identified by one graph and vice versa.
  Thus the claim is true.
\end{proof}

\begin{ex}\label{i:ex:3.5.12}
  Let \(X\) be an arbitrary set, let \(f : \N \times X \to X\) be a function, and let \(c \in X\).
  Show that there exists a function \(a : \N \to X\) such that
  \[
    a(0) = c
  \]
  and
  \[
    a(n\pp) = f(n, a(n)) \text{ for all } n \in \N,
  \]
  and furthermore that this function is unique.
  For an additional challenge, prove this result without using any properties of the natural numbers other than the Peano axioms directly.
\end{ex}

\begin{proof}[\pf{i:ex:3.5.12}]
  For each \(N \in \N\), let \(P(N)\) be the statement ``there exists an unique function \(a_N : \set{n \in \N : n \leq N} \to X\) such that \(a_N(0) = c\) and \(a_N(n\pp) = f(n, a_N(n))\) for all \(n \in \N\) such that \(n < N\).''
  We use induction to prove that \(P(N)\) is true for all \(N \in \N\).

  For \(N = 0\), we define \(a_0 : \set{0} \to X\) to be the function \(a_0(0) = c\).
  By \cref{i:ac:2.2.4} we see that \(\set{0} = \set{n \in \N : n \leq 0}\) and the statement ``\(a_0(n\pp) = f(n, a_0(n))\) for all \(n \in \N\) such that \(n < 0\)'' is vacuously true.
  Thus to close the base case, we only need to show that \(a_0\) is unique.
  Let \(b_0 : \set{0} \to X\) be another function satisfying \(b_0(0) = c\) and \(b_0(n\pp) = f(n, b_0(n))\) for all \(n \in \N\) such that \(n < 0\).
  Since \(0\) is the only element in \(\set{0}\) and \(b_0(0) = c = a_0(0)\), by \cref{i:3.3.7} we have \(b_0 = a_0\).
  Thus \(a_0\) is unique and the base case holds.

  Suppose inductively that \(P(N)\) is true for some \(N \in \N\).
  We want to show that \(P(N\pp)\) is true.
  By induction hypothesis there exists an unique function \(a_N : \set{n \in \N : n \leq N} \to X\) such that \(a_N(0) = c\) and \(a_N(n\pp) = f(n, a_N(n))\) for all \(n \in \N\) such that \(n < N\).
  Now we define a function \(a_{N\pp} : \set{n \in \N : n \leq N\pp} \to X\) as follow:
  \[
    \forall n \in \N \text{ such that } n \leq N\pp, a_{N\pp}(n) = \begin{dcases}
      a_N(n)       & \text{if } n \neq N\pp                    \\
      f(m, a_N(m)) & \text{if } n = N\pp \text{ and } m\pp = n
    \end{dcases}.
  \]
  Since each \(n \in \set{n \in \N : n \leq N\pp}\) is assigned with an unique object in \(X\), we see that \(a_{N\pp}\) is well-define.
  Since \(0 \neq N\pp\) (\cref{i:2.3}), we have \(a_{N\pp}(0) = a_N(0) = c\).
  We claim that we must have \(a_{N\pp}(n) = f(n, a_{N\pp}(n))\) for natural number \(n < N\pp\).
  So let \(n \in N\) and \(n < N\pp\).
  We split into two cases:
  \begin{itemize}
    \item If \(n\pp \neq N\pp\), then we have
          \begin{align*}
            a_{N\pp}(n\pp) & = a_N(n\pp)          &  & (n\pp \neq N\pp) \\
                           & = f(n, a_N(n))       &  & \byIH            \\
                           & = f(n, a_{N\pp}(n)). &  & (n < N\pp)
          \end{align*}
    \item If \(n\pp = N\pp\), then we have
          \begin{align*}
            a_{N\pp}(n\pp) & = f(n, a_N(n))       &  & (n\pp = N\pp) \\
                           & = f(n, a_{N\pp}(n)). &  & (n < N\pp)
          \end{align*}
  \end{itemize}
  From all cases above we see that \(a_{N\pp}(n\pp) = f(n, a_{N\pp}(n))\).
  Thus our claim is true.
  To close the induction, we need to show that \(a_{N\pp}\) is unique.
  So suppose that there exists another function \(b_{N\pp} : \set{n \in \N : n \leq N\pp} \to X\) where \(b_{N\pp}(0) = c\) and \(b_{N\pp}(n\pp) = f(n, b_{N\pp}(n))\) for all \(n \in \N\) such that \(n < N\).
  Clearly we have \(b_{N\pp}(0) = c = a_{N\pp}(0)\).
  If we have show that \(b_{N\pp}(n) = a_{N\pp}(n)\) for some \(n \in \N\) and \(n < N\pp\), then we see that
  \[
    b_{N\pp}(n\pp) = f(n, b_{N\pp}(n)) = f(n, a_{N\pp}(n)) = a_{N\pp}(n\pp).
  \]
  Thus we have \(b_{N\pp}(n) = a_{N\pp}(n)\) for all \(n \in \set{n \in \N : n \leq N\pp}\).
  By \cref{i:3.1.4} this means \(b_{N\pp} = a_{N\pp}\), and this closes the induction.

  Now we define \(a : \N \to X\) as follow:
  \[
    \forall n \in \N, a(n) = a_n(n).
  \]
  Since \(P(n)\) is true for all \(n \in \N\), we know that \(a_n\) is well-defined and \(a_n(n)\) is unique for each \(n \in \N\), thus \(a\) is well-defined.
  Clearly we have \(a(0) = a_0(0) = c\).
  We claim that \(a(n\pp) = f(n, a(n))\) for all \(n \in \N\).
  This is true since
  \begin{align*}
    \forall n \in \N, a(n\pp) & = a_{n\pp}(n\pp)                    \\
                              & = f(n, a_{n\pp}(n))                 \\
                              & = f(n, a_n(n))      &  & (n < n\pp) \\
                              & = f(n, a(n)).
  \end{align*}
  Now we show that \(a\) is unique.
  Suppose there exists another \(b : \N \to X\) where \(b(0) = c\) and \(b(n\pp) = f(n, b(n))\) for all \(n \in \N\).
  Clealy \(b(0) = c = a(0)\).
  If \(b(n) = a(n)\) is true for some \(n \in \N\), then we have
  \[
    b(n\pp) = f(n, b(n)) = f(n, a(n)) = a(n\pp).
  \]
  Thus by \cref{i:2.5} we know that \(b(n) = a(n)\) for all \(n \in \N\).
  By \cref{i:3.1.4} this means \(b = a\).
  Thus \(a\) is unique.

  Now we prove the additional challenge.
  We claim that for every natural number \(N \in \N\), there exists a unique pair \(A_N, B_N\) of subsets of \(\N\) which obeys the following properties:
  \begin{enumerate}
    \item \(A_N \cap B_N = \emptyset\);
    \item \(A_N \cup B_N = \N\);
    \item \(0 \in A_N\);
    \item \(N\pp \in B_N\);
    \item Whenever \(n \in B_N\), we have \(n\pp \in B_N\);
    \item Whenever \(n \in A_N\) and \(n \neq N\), we have \(n\pp \in A_N\).
  \end{enumerate}
  we induct on \(N\) to prove the claim.

  For \(N = 0\), let \(A_0 = \set{0}\) and let \(B_0 = \N \setminus A_0\).
  By \cref{i:3.1.28}(g) we see that (a)(b) holds for \(A_0, B_0\).
  By \cref{i:3.3} we have \(0 \in A_0\) and \(0\pp = 1 \notin A_0\).
  Thus \(1 \in B_0\).
  So (c)(d) holds for \(A_0, B_0\).
  If \(n \in B_0\), then \(n \in \N\) and \(n\pp \neq 0\) (\cref{i:2.4}).
  Thus we have \(n\pp \notin A_0\) and therefore \(n\pp \in B_0\).
  So (e) holds for \(A_0, B_0\).
  Since there is no natural number \(n\) satisfying \(n \in A_0\) and \(n \neq 0\), we see that (f) holds for \(A_0, B_0\).
  To close the base case, we are left to show that \(A_0, B_0\) are unique.
  So suppose that there exist another pair of sets \(A_0', B_0'\) such that (a)--(f) hold.
  By (c)(d) we know that \(0 \in A_0'\) and \(1 \in B_0'\).
  Thus by (e) we see that \(n \in B_0'\) for all \(n \in \N\) and \(n \geq 1\).
  Therefore we have \(A_0' = \set{0} = A_0\) by (a).
  By \cref{i:ex:3.1.9} we have \(B_0' = \N \setminus A_0' = \N \setminus A_0 = B_0\).
  Thus \(A_0, B_0\) is unique and the base case holds.

  Suppose inductively that, for some natural number \(N\), there exists a unique pair of set \(A_N, B_N\) such that (a)--(f) hold.
  We define \(A_{N\pp} = A_N \cup \set{N\pp}\) and \(B_{N\pp} = B_N \setminus \set{N\pp}\).
  Since
  \begin{align*}
     & A_{N\pp} \cap B_{N\pp}                                                                                          \\
     & = (A_N \cup \set{N\pp}) \cap (B_N \setminus \set{N\pp})                                                         \\
     & = (A_N \cap (B_N \setminus \set{N\pp})) \cup (\set{N\pp} \cap (B_N \setminus \set{N\pp})) &  & \by{i:3.1.28}[f] \\
     & = (A_N \cap (B_N \setminus \set{N\pp})) \cup \emptyset                                    &  & \by{i:3.1.28}[g] \\
     & = A_N \cap (B_N \setminus \set{N\pp})                                                     &  & \by{i:3.1.28}[a] \\
     & \subseteq A_n \cap B_N                                                                    &  & \by{i:3.1.15}    \\
     & = \emptyset,                                                                              &  & \byIH
  \end{align*}
  by \cref{i:3.2} we see that \(A_{N\pp} \cap B_{N\pp} = \emptyset\) and thus (a) is true for \(A_{N\pp}, B_{N\pp}\).
  Since
  \begin{align*}
     & A_{N\pp} \cup B_{N\pp}                                                        \\
     & = (A_N \cup \set{N\pp}) \cup (B_N \setminus \set{N\pp})                       \\
     & = A_N \cup (\set{N\pp} \cup (B_N \setminus \set{N\pp})) &  & \by{i:3.1.28}[e] \\
     & = A_N \cup B_N                                          &  & \by{i:3.1.28}[g] \\
     & = \N,                                                   &  & \byIH
  \end{align*}
  we see that (b) is true for \(A_{N\pp}, B_{N\pp}\).
  Since \(A_{N\pp} = A_N \cup \set{N\pp}\), by \cref{i:ex:3.1.7} we have \(A_N \subseteq A_{N\pp}\).
  By induction hypothesis we know that \(0 \in A_N\).
  Thus by \cref{i:3.1.15} we have \(0 \in A_{N\pp}\) and (c) is true for \(A_{N\pp}, B_{N\pp}\).
  By induction hypothesis we have \(N\pp \in B_N\).
  Thus by (e) we see that \((N\pp)\pp \in B_N\).
  Since \(B_{N\pp} = B_N \setminus \set{N\pp}\), we see that \((N\pp)\pp \in B_{N\pp}\).
  Thus (d) is true for \(A_{N\pp}, B_{N\pp}\).
  If \(n \in B_{N\pp} = B_N \setminus \set{N\pp}\), then by \cref{i:3.1.27,i:3.3} we have \(n \in B_N\) and \(n \neq N\pp\).
  We must have \(n\pp \neq N\pp\), for otherwise we would have \(n\pp = N\pp\) and \(n = N\) by \cref{i:2.4}.
  But by induction hypotheses we would have \(N\pp = n\pp \in B_N\), a contradiction.
  Thus \(n \in B_{N\pp} \implies n\pp \neq N\pp\).
  Hence
  \begin{align*}
             & \forall n \in B_{N\pp}, (n \in B_n) \land (n\pp \neq N\pp)                                     \\
    \implies & \forall n \in B_{N\pp}, (n\pp \in B_N) \land (n\pp \neq N\pp)         &  & \byIH               \\
    \implies & \forall n \in B_{N\pp}, n\pp \in B_N \setminus \set{N\pp} = B_{N\pp}. &  & \by{i:3.1.27,i:3.3}
  \end{align*}
  So (e) is true for \(A_{N\pp}, B_{N\pp}\).
  Since
  \begin{align*}
             & (n \in A_{N\pp} = A_N \cup \set{N\pp}) \land (n \neq N\pp)                                 \\
    \implies & ((n \in A_N) \lor (n = N\pp)) \land (n \neq N\pp)                    &  & \by{i:3.3,i:3.4} \\
    \implies & ((n \in A_N) \land (n \neq N)) \lor ((n = N\pp) \land (n \neq N\pp))                       \\
    \implies & (n \in A_N) \land (n \neq N)                                                               \\
    \implies & n\pp \in A_N                                                         &  & \byIH            \\
    \implies & n\pp \in A_{N\pp},                                                   &  & \by{i:3.1.15}
  \end{align*}
  we see that (f) is true for \(A_{N\pp}, B_{N\pp}\).
  To close the induction, we only need to show that the pair of sets \(A_{N\pp}, B_{N\pp}\) is unique.
  So suppose that there exist another pair of sets \(A_{N\pp}', B_{N\pp}'\) such that (a)--(f) hold.
  Then by (c) and (f) we see that \(A_{N\pp}' = A_{N\pp}\).
  Thus by (b) and \cref{i:ex:3.1.9} we see that \(B_{N\pp}' = B_{N\pp}\).
  Thus the pair \(A_{N\pp}, B_{N\pp}\) is unique and this closes the induction.

  From the construction of \(A_N\) we see that \(A_N = \set{n \in N : 0 \leq N}\).
  Thus we can simply apply \(A_N\) to define \(a_N : N \to X\) as in the first part of the proof.
  The rest claim than follows.
\end{proof}

\begin{ex}\label{i:ex:3.5.13}
  Suppose we have a set \(\N'\) of ``alternative natural numbers'', an ``alternative zero'' \(0'\), and an ``alternative increment operation'' which takes any alternative natural number \(n' \in N\) and returns another alternative natural number \(n'\pp' \in \N'\), such that the Peano axioms (\crefrange{i:2.1}{i:2.5}) all hold with the natural numbers, zero, and increment replaced by their alternative counterparts.
  Show that there exists a bijection \(f : \N \to \N'\) from the natural numbers to the alternative natural numbers such that \(f(0) = 0'\), and such that for any \(n \in \N\) and \(n' \in \N'\), we have \(f(n) = n'\) iff \(f(n\pp) = n'\pp'\).
\end{ex}

\begin{proof}[\pf{i:ex:3.5.13}]
  Define \(g : \N \times \N' \to \N'\) as follow:
  \[
    \forall (n, n') \in \N \times \N', g(n, n') = n'\pp'.
  \]
  By \cref{i:2.4} we see that \(g\) pass the vertical line test, thus \(g\) is well-defined.
  By \cref{i:ex:3.5.12} there exists a unique function \(a : \N \to \N'\) such that
  \[
    a(0) = 0' \quad \text{and} \quad \forall n \in \N, a(n\pp) = g(n, a(n)) = a(n)\pp'.
  \]

  First we show that \(a\) is injective.
  Let \(n_1, n_2 \in \N\).
  we induct on \(n_1\) to show that \(n_1 \neq n_2 \implies a(n_1) \neq a(n_2)\).
  For \(n_1 = 0\), we have \(n_2 \neq 0\) and \(a(n_1) = a(0) = 0'\).
  Since \(n_2 \neq 0\), by \cref{i:2.2.10} there exists an \(m \in \N\) such that \(m\pp = n_2\).
  By the definition of \(a\) we see that \(a(n_2) = a(m\pp) = a(m)\pp'\).
  By \cref{i:2.3} we know that \(a(m)\pp' \neq 0'\).
  Thus we have \(a(n_1) \neq a(n_2)\) and the base case holds.
  Suppose inductively that \(n_1 \neq n_2 \implies a(n_1) \neq a(n_2)\) for some \(n_1 \in \N\).
  We want to show that \(n_1\pp \neq n_2 \implies a(n_1\pp) \neq a(n_2)\).
  We split into two cases:
  \begin{itemize}
    \item If \(n_2 = 0\), then we can use the identical proof as in the base case but replacing \((n_1, n_2)\) with \((n_2, n_1\pp)\) to see that \(a(n_1\pp) \neq a(n_2)\).
    \item If \(n_2 \neq 0\), then by \cref{i:2.2.10} there exists an \(m \in \N\) such that \(m\pp = n_2\).
          Thus
          \begin{align*}
                     & n_1\pp \neq m\pp = n_2           &  & \by{i:2.2.10}                       \\
            \implies & n_1 \neq m                       &  & \by{i:2.4}                          \\
            \implies & a(n_1) \neq a(m)                 &  & \byIH                               \\
            \implies & a(n_1)\pp' \neq a(m)\pp'         &  & \by{i:2.4}                          \\
            \implies & a(n_1\pp) \neq a(m\pp) = a(n_2). &  & \text{(by the definition of \(a\))}
          \end{align*}
  \end{itemize}
  From all cases above we see that \(a(n_1\pp) \neq a(n_2)\).
  This closes the induction and by \cref{i:3.3.14} \(a\) is injective.

  Next we claim that \(a\) is surjective, i.e. (\cref{i:3.3.17}), for each \(n' \in \N'\), there exists an \(n \in \N\) such that \(a(n) = n'\).
  Since \cref{i:2.5} holds for \(\N'\), we can use induction on \(n'\) to prove the claim.
  For \(n' = 0'\), we see that \(a(0) = 0'\).
  So the base case holds.
  Suppose inductively that for some \(n' \in \N'\), there exists an \(n \in \N\) such that \(a(n) = n'\).
  Then we have
  \begin{align*}
    a(n\pp) & = a(n)\pp' &  & \text{(by the definition of \(a\))} \\
            & = n'\pp'   &  & \byIH
  \end{align*}
  and the claim is true for \(n'\pp'\).
  This closes the induction and thus \(a\) is surjective.
  Since \(a\) is both injective and surjective, by \cref{i:3.3.20} we know that \(a\) is bijective.

  By the definition of \(a\) we see that for all \(n, n' \in \N \times \N'\), we have \(a(n) = n' \iff a(n\pp) = n'\pp'\).
  Thus by setting \(f = a\) we are done.
\end{proof}
