\chapter{Introduction}\label{i:ch:1}

\begin{note}
  This is my notes on the book ``Analysis I'', 3rd edition written by Terence Tao.
  My notes cover almost all axioms, corollaries, exercises, lemmata, proofs, propositions, remarks, and theories appeared in the book.
  I have written all the proofs for each exercise.
  In addition, I have written some ``additional corollaries'' which helps proving complex statements.

  All statements are wrapped inside a \LaTeX\ environment.
  The following table provides the mapping between the typeset (name) and the meaning of each environment.
  \begin{table}[h]
    \centering
    \begin{tabular}{|c|c|}
      \hline
      Environment Typesets  & Meaning     \\
      \hline
      \namecref{i:2.1}      & Axiom       \\
      \hline
      \namecref{i:ch:1}     & Chapter     \\
      \hline
      \namecref{i:2.2.9}    & Corollary   \\
      \hline
      \namecref{i:2.2.1}    & Definition  \\
      \hline
      \namecref{i:3.1.10}   & Example     \\
      \hline
      \namecref{i:ex:2.2.1} & Exercise    \\
      \hline
      \namecref{i:2.2.2}    & Lemma       \\
      \hline
      Note                  & Note        \\
      \hline
      \namecref{i:2.2.4}    & Proposition \\
      \hline
      \namecref{i:sec:2.1}  & Section     \\
      \hline
      \namecref{i:3.6.12}   & Theorem     \\
      \hline
    \end{tabular}
  \end{table}
\end{note}

\begin{note}
  \emph{circularity}:
  Using an advanced fact to prove a more elementary fact, and then later using the elementary fact to prove the advanced fact.
  When do a mathematics proofs, one should avoid \emph{circularity}.
\end{note}

\begin{note}
  From a logical point of view, there is no difference between a lemma, proposition, theorem, or corollary
  - they are all claims waiting to be proved.
  However, we use these terms to suggest different levels of importance and difficulty.
  A lemma is an easily proved claim which is helpful for proving other propositions and theorems, but is usually not particularly interesting in its own right.
  A proposition is a statement which is interesting in its own right, while a theorem is a more important statement than a proposition which says something definitive on the subject, and often takes more effort to prove than a proposition or lemma.
  A corollary is a quick consequence of a proposition or theorem that was proven recently.
\end{note}
