\section{Equivalent Cauchy sequences}\label{sec:5.2}

\begin{defn}[\(\varepsilon\)-close sequences]\label{5.2.1}
  Let \((a_n)_{n = 0}^{\infty}\) and \((b_n)_{n = 0}^{\infty}\) be two sequences, and let \(\varepsilon > 0\).
  We say that the sequence \((a_n)_{n = 0}^{\infty}\) is \emph{\(\varepsilon\)-close} to \((b_n)_{n = 0}^{\infty}\) iff \(a_n\) is \(\varepsilon\)-close to \(b_n\) for each \(n \in \N\).
  In other words, the sequence \(a_0, a_1, a_2, \dots\) is \(\varepsilon\)-close to the sequence \(b_0, b_1, b_2, \dots\) iff \(\abs{a_n - b_n} \leq \varepsilon\) for all \(n = 0, 1, 2, \dots\).
\end{defn}

\setcounter{thm}{2}
\begin{defn}[\(Eventually \varepsilon\)-close sequences]\label{5.2.3}
  Let \((a_n)_{n = 0}^{\infty}\) and \((b_n)_{n = 0}^{\infty}\) be two sequences, and let \(\varepsilon > 0\).
  We say that the sequence \((a_n)_{n = 0}^{\infty}\) is \emph{eventually \(\varepsilon\)-close} to \((b_n)_{n = 0}^{\infty}\) iff there exists an \(N \geq 0\) such that the sequences \((a_n)_{n = N}^{\infty}\) and \((b_n)_{n = N}^{\infty}\) are \(\varepsilon\)-close.
  In other words, \(a_0, a_1, a_2, \dots\) is eventually \(\varepsilon\)-close to \(b_0, b_1, b_2, \dots\) iff there exists an \(N \geq 0\) such that \(\abs{a_n - b_n} \leq \varepsilon\) for all \(n \geq N\).
\end{defn}

\begin{rmk}\label{5.2.4}
  Again, the notations for \(\varepsilon\)-close sequences and eventually \(\varepsilon\)-close sequences are not standard in the literature, and we will not use them outside of this section.
\end{rmk}

\setcounter{thm}{5}
\begin{defn}[Equivalent sequences]\label{5.2.6}
  Two sequences \((a_n)_{n = 0}^{\infty}\) and \((b_n)_{n = 0}^{\infty}\) are \emph{equivalent} iff for each rational \(\varepsilon > 0\), the sequences \((a_n)_{n = 0}^{\infty}\) and \((b_n)_{n = 0}^{\infty}\) are eventually \(\varepsilon\)-close.
  In other words, \(a_0, a_1, a_2, \dots\) and \(b_0, b_1, b_2, \dots\) are equivalent iff for every rational \(\varepsilon > 0\), there exists an \(N \geq 0\) such that \(\abs{a_n - b_n} \leq \varepsilon\) for all \(n \geq N\).
\end{defn}

\begin{rmk}\label{5.2.7}
  As with \cref{5.1.8}, the quantity \(\varepsilon > 0\) is currently restricted to be a positive rational, rather than a positive real.
  However, we shall eventually see that it makes no difference whether \(\varepsilon\) ranges over the positive rationals or positive reals.
\end{rmk}

\begin{ac}\label{ac:5.2.1}
  Equivalence defined as \cref{5.2.6} is reflexive, symmetric and transitive.
\end{ac}

\begin{proof}
  Let \((a_n)_{n = m}^\infty\), \((b_n)_{n = m}^\infty\), \((c_n)_{n = m}^\infty\) be sequences of rationals.
  We have
  \begin{align*}
             & \forall \varepsilon \in \Q^+, \forall n \geq m, \abs{a_n - a_n} = \abs{0} = 0 \leq \varepsilon                            \\
    \implies & (a_n)_{n = m}^\infty = (a_n)_{n = m}^\infty                                                    & \text{(by \cref{5.2.6})}
  \end{align*}
  and thus \cref{5.2.6} is reflexive.

  Now suppose that \((a_n)_{n = m}^\infty = (b_n)_{n = m}^\infty\).
  Then we have
  \begin{align*}
             & (a_n)_{n = m}^\infty = (b_n)_{n = m}^\infty                                                   \\
    \implies & \forall \varepsilon \in \Q^+, \exists\ N \in \N \land N \geq m:                               \\
             & \forall n \geq N, \abs{a_n - b_n} \leq \varepsilon              & \text{(by \cref{5.2.6})}    \\
    \implies & \forall \varepsilon \in \Q^+, \exists\ N \in \N \land N \geq m:                               \\
             & \forall n \geq N, \abs{b_n - a_n} \leq \varepsilon              & \text{(by \cref{4.3.3}(f))} \\
    \implies & (b_n)_{n = m}^\infty = (a_n)_{n = m}^\infty                     & \text{(by \cref{5.2.6})}
  \end{align*}
  and thus \cref{5.2.6} is symmetric.

  Finally suppose that \((a_n)_{n = m}^\infty = (b_n)_{n = m}^\infty\) and \((b_n)_{n = m}^\infty = (c_n)_{n = m}^\infty\).
  Then we have
  \begin{align*}
             & \big((a_n)_{n = m}^\infty = (b_n)_{n = m}^\infty\big) \land \big((b_n)_{n = m}^\infty = (c_n)_{n = m}^\infty\big)                                  \\
    \implies & \forall \varepsilon \in \Q^+, \exists\ N_1, N_2 \in \N \land N_1, N_2 \geq m:                                     & \text{(by \cref{5.2.6})}       \\
             & \begin{cases}
                 \abs{a_n - b_n} \leq \frac{\varepsilon}{2} & \forall n \geq N_1 \\
                 \abs{b_n - c_n} \leq \frac{\varepsilon}{2} & \forall n \geq N_2 \\
               \end{cases}                                                                                    \\
    \implies & \forall \varepsilon \in \Q^+, \exists\ N = \max(N_1, N_2) \geq m:                                                 & \text{(by \cref{2.2.13})}      \\
             & \forall n \geq N, (\abs{a_n - b_n} \leq \frac{\varepsilon}{2}) \land (\abs{b_n - c_n} \leq \frac{\varepsilon}{2})                                  \\
    \implies & \forall \varepsilon \in \Q^+, \exists\ N = \max(N_1, N_2) \geq m:                                                                                  \\
             & \forall n \geq N, \abs{a_n - b_n} + \abs{b_n - c_n} \leq \varepsilon                                              & \text{(by \cref{4.2.9}(c)(d))} \\
    \implies & \forall \varepsilon \in \Q^+, \exists\ N = \max(N_1, N_2) \geq m:                                                                                  \\
             & \forall n \geq N, \abs{a_n - c_n} \leq \abs{a_n - b_n} + \abs{b_n - c_n} \leq \varepsilon                         & \text{(by \cref{4.3.3}(b))}    \\
    \implies & (a_n)_{n = m}^\infty = (c_n)_{n = m}^\infty                                                                       & \text{(by \cref{5.2.6})}
  \end{align*}
  and thus \cref{5.2.6} is transitive.
\end{proof}

\begin{prop}\label{5.2.8}
  Let \((a_n)_{n = 1}^{\infty}\) and \((b_n)_{n = 1}^{\infty}\) be the sequences \(a_n = 1 + 10^{-n}\) and \(b_n = 1 - 10^{-n}\).
  Then the sequences \(a_n, b_n\) are equivalent.
\end{prop}

\begin{proof}
  We need to prove that for every \(\varepsilon > 0\), the two sequences \((a_n)_{n = 1}^{\infty}\) and \((b_n)_{n = 1}^{\infty}\) are eventually \(\varepsilon\)-close to each other.
  So we fix an \(\varepsilon > 0\).
  We need to find an \(N > 0\) such that \((a_n)_{n = 1}^{\infty}\) and \((b_n)_{n = 1}^{\infty}\) are \(\varepsilon\)-close;
  in other words, we need to find an \(N > 0\) such that
  \[
    \abs{a_n - b_n} \leq \varepsilon \text{ for all } n \geq N.
  \]
  However, we have
  \[
    \abs{a_n - b_n} = \abs{(1 + 10^{-n}) - (1 - 10^{-n})} = 2 \times 10^{-n}.
  \]
  Since \(10^{-n}\) is a decreasing function of \(n\) (i.e., \(10^{-m} < 10^{-n}\) whenever \(m > n\);
  this is easily proven by induction), and \(n \geq N\), we have \(2 \times 10^{-n} \leq 2 \times 10^{-N}\).
  Thus we have
  \[
    \abs{a_n - b_n} \leq 2 \times 10^{-N} \text{ for all } n \geq N.
  \]
  Thus in order to obtain \(\abs{a_n - b_n} \leq \varepsilon\) for all \(n \geq N\), it will be sufficient to choose \(N\) so that \(2 \times 10^{-N} \leq \varepsilon\).
  This is easy to do using logarithms, but we have not yet developed logarithms yet, so we will use a cruder method.
  First, we observe \(10^N\) is always greater than \(N\) for any \(N \geq 1\) (see \cref{ex:4.3.5}).
  Thus \(10^{-N} \leq 1 / N\), and so \(2 \times 10^{-N} \leq 2 / N\).
  Thus to get \(2 \times 10^{-N} \leq \varepsilon\), it will suffice to choose \(N\) so that \(2 / N \leq \varepsilon\), or equivalently that \(N \geq 2 / \varepsilon\).
  But by \cref{4.4.1} we can always choose such an \(N\), and the claim follows.
\end{proof}

\begin{rmk}\label{5.2.9}
  \cref{5.2.8}, in decimal notation, asserts that
  \[
    1.0000 \dots = 0.9999 \dots.
  \]
\end{rmk}

\exercisesection

\begin{ex}\label{ex:5.2.1}
  Show that if \((a_n)_{n = 1}^{\infty}\) and \((b_n)_{n = 1}^{\infty}\) are equivalent sequences of rationals, then \((a_n)_{n = 1}^{\infty}\) is a Cauchy sequence if and only if \((b_n)_{n = 1}^{\infty}\) is a Cauchy sequence.
\end{ex}

\begin{proof}
  Let \(j, k \in \Z^+\).
  Since \((a_n)_{n = 1}^\infty = (b_n)_{n = 1}^\infty\), by \cref{5.2.6} we have
  \[
    \forall \varepsilon \in \Q^+, \exists\ N_1 \in \Z^+ : \forall n \geq N_1, \abs{a_n - b_n} \leq \frac{\varepsilon}{3}.
  \]
  Then we have
  \begin{align*}
             & (a_n)_{n = 1}^\infty \text{ is a Cauchy sequence}                                                                         \\
    \implies & \exists\ N_2 \in \Z^+ : \forall j, k \geq N,                                                                              \\
             & \abs{a_j - a_k} \leq \frac{\varepsilon}{3}                                               & \text{(by \cref{5.1.8})}       \\
    \implies & \exists\ N = \max(N_1, N_2) \in \Z^+ : \forall j, k \geq N,                              & \text{(by \cref{2.2.13})}      \\
             & \abs{a_j - a_k} \leq \frac{\varepsilon}{3}                                                                                \\
    \implies & \exists\ N = \max(N_1, N_2) \in \Z^+ : \forall j, k \geq N,                                                               \\
             & \abs{a_j - a_k} + \abs{a_j - b_j} + \abs{a_k - b_k}                                                                       \\
             & \leq \frac{\varepsilon}{3} + \frac{\varepsilon}{3} + \frac{\varepsilon}{3} = \varepsilon & \text{(by \cref{4.2.9}(c)(d))} \\
    \implies & \exists\ N = \max(N_1, N_2) \in \Z^+ : \forall j, k \geq N,                                                               \\
             & \abs{a_j - a_k} + \abs{b_j - a_j} + \abs{a_k - b_k} \leq \varepsilon                     & \text{(by \cref{4.3.3}(f))}    \\
    \implies & \exists\ N = \max(N_1, N_2) \in \Z^+ : \forall j, k \geq N,                                                               \\
             & \abs{b_j - b_k} = \abs{a_j - a_k + b_j - a_j + a_k - b_k}                                                                 \\
             & \leq \abs{a_j - a_k} + \abs{b_j - a_j} + \abs{a_k - b_k} \leq \varepsilon                & \text{(by \cref{4.3.3}(b))}    \\
    \implies & (b_n)_{n = 1}^\infty \text{ is a Cauchy sequence}.                                       & \text{(by \cref{5.1.8})}
  \end{align*}
  Using similar arguments we can show that \((b_n)_{n = 1}^\infty\) is a Cauchy sequence implies \((a_n)_{n = 1}^\infty\) is a Cauchy sequence.
  Thus we conclude that \((a_n)_{n = 1}^\infty\) is a Cauchy sequence iff \((b_n)_{n = 1}^\infty\) is a Cauchy sequence.
\end{proof}

\begin{ex}\label{ex:5.2.2}
  Let \(\varepsilon > 0\).
  Show that if \((a_n)_{n = 1}^{\infty}\) and \((b_n)_{n = 1}^{\infty}\) are eventually \(\varepsilon\)-close, then \((a_n)_{n = 1}^{\infty}\) is bounded if and only if \((b_n)_{n = 1}^{\infty}\) is bounded.
\end{ex}

\begin{proof}
  Since \((a_n)_{n = 1}^{\infty}\) and \((b_n)_{n = 1}^{\infty}\) are eventually \(\varepsilon\)-close, by \cref{5.2.3} we have
  \[
    \exists\ N \in \Z^+ : \forall n \geq N, \abs{a_n - b_n} \leq \varepsilon.
  \]
  Then we have
  \begin{align*}
             & (a_n)_{n = 1}^\infty \text{ is bounded}                                                                                       \\
    \implies & \exists\ M \in \Q \setminus \Q^- : \forall n \geq 1, \abs{a_n} \leq M                        & \text{(by \cref{5.1.12})}      \\
    \implies & \exists\ M \in \Q \setminus \Q^- : \forall n \geq \max(1, N), \abs{a_n} \leq M               & \text{(by \cref{2.2.13})}      \\
    \implies & \exists\ M \in \Q \setminus \Q^- : \forall n \geq \max(1, N), \abs{-a_n} \leq M              & \text{(by \cref{4.3.3}(d))}    \\
    \implies & \exists\ M \in \Q \setminus \Q^- : \forall n \geq \max(1, N),                                                                 \\
             & \abs{-a_n} + \abs{a_n - b_n} \leq M + \varepsilon                                            & \text{(by \cref{4.2.9}(c)(d))} \\
    \implies & \exists\ M \in \Q \setminus \Q^- : \forall n \geq \max(1, N),                                                                 \\
             & \abs{-a_n + a_n - b_n} \leq \abs{-a_n} + \abs{a_n - b_n} \leq M + \varepsilon                & \text{(by \cref{4.3.3}(b))}    \\
    \implies & \exists\ M \in \Q \setminus \Q^- : \forall n \geq \max(1, N),                                                                 \\
             & \abs{-b_n} \leq M + \varepsilon                                                              & \text{(by \cref{4.2.4})}       \\
    \implies & \exists\ M \in \Q \setminus \Q^- : \forall n \geq \max(1, N), \abs{b_n} \leq M + \varepsilon & \text{(by \cref{4.3.3}(d))}    \\
    \implies & \exists\ M \in \Q \setminus \Q^- : \forall n \geq 1,                                                                          \\
             & \abs{b_n} \leq M + \varepsilon + \max_{1 \leq n \leq N - 1}\{\abs{b_n}\}                     & \text{(by \cref{5.1.14})}      \\
    \implies & (a_n)_{n = 1}^\infty \text{ is bounded}.                                                     & \text{(by \cref{5.1.12})}
  \end{align*}
  Using similar arguments we can show that \((b_n)_{n = 1}^\infty\) is bounded implies \((a_n)_{n = 1}^\infty\) is bounded.
  Thus we conclude that \((a_n)_{n = 1}^\infty\) is bounded iff \((b_n)_{n = 1}^\infty\) is bounded.
\end{proof}