\section{Basic definitions}\label{sec 10}

\begin{note}
    We can now define derivatives analyti cally, using limits, in contrast to the geometric definition of derivatives, which uses tangents.
    The advantage of working analytically is that
    (a) we do not need to know the axioms of geometry, and
    (b) these definitions can be modified to handle functions of several variables, or functions whose values are vectors instead of scalar.
    Furthermore, one's geometric intuition becomes difficult to rely on once one has more than three dimensions in play.
    (Conversely, one can use one's experience in analytic rigour to extend one's geometric intuition to such abstract settings;
    as mentioned earlier, the two viewpoints complement rather than oppose each other.)
\end{note}

\begin{definition}[Differentiability at a point]\label{10.1.1}
    Let \(X\) be a subset of \(\mathbf{R}\), and let \(x_0 \in X\) be an element of \(X\) which is also a limit point of \(X\).
    Let \(f : X \to \mathbf{R}\) be a function.
    If the limit
    \[
        \lim_{x \to x_0 ; x \in X \setminus \{x_0\}} \frac{f(x) - f(x_0)}{x - x_0}
    \]
    converges to some real number \(L\), then we say that \(f\) is \emph{differentiable at \(x_0\) on \(X\) with derivative \(L\)}, and write \(f'(x_0) \coloneqq L\).
    If the limit does not exist, or if \(x_0\) is not an element of \(X\) or not a limit point of \(X\), we leave \(f'(x_0)\) undefined, and say that \(f\) is \emph{not differentiable at \(x_0\) on \(X\)}.
\end{definition}

\begin{remark}\label{10.1.2}
    Note that we need \(x_0\) to be a limit point in order for \(x_0\) to be adherent to \(X \setminus \{x_0\}\), otherwise the limit
    \[
        \lim_{x \to x_0 ; x \in X \setminus \{x_0\}} \frac{f(x) - f(x_0)}{x - x_0}
    \]
    would automatically be undefined.
    In particular, we do not define the derivative of a function at an isolated point;
    In practice, the domain \(X\) will almost always be an interval, and so by Lemma \ref{9.1.21} all elements \(x_0\) of \(X\) will automatically be limit points and we will not have to care much about these issues.
\end{remark}

\setcounter{theorem}{3}
\begin{remark}\label{10.1.4}
    This point is trivial, but it is worth mentioning:
    if \(f : X \to \mathbf{R}\) is differentiable at \(x_0\), and \(g : X \to \mathbf{R}\) is equal to \(f\) (i.e., \(g(x) = f(x)\) for all \(x \in X\)), then \(g\) is also differentiable at \(x_0\) and \(g'(x_0) = f'(x_0)\).
    However, if two functions \(f\) and \(g\) merely have the same value at \(x_0\), i.e., \(g(x_0) = f(x_0)\), this does not imply that \(g'(x_0) = f'(x_0)\).
    Thus there is a big difference between two functions being equal on their whole domain, and merely being equal at one point.
\end{remark}

\begin{remark}\label{10.1.5}
    One sometimes writes \(\frac{df}{dx}\) instead of \(f'\).
    This notation is of course very familiar and convenient, but one has to be a little careful, because it is only safe to use as long as \(x\) is the only variable used to represent the input for \(f\);
    otherwise one can get into all sorts of trouble.
    Because of this possible source of confusion, we will refrain from using the notation \(\frac{df}{dx}\) whenever it could possibly lead to confusion.
    (This confusion becomes even worse in the calculus of several variables, and the standard notation of \(\frac{\partial f}{\partial x}\) can lead to some serious ambiguities.
    There are ways to resolve these ambiguities, most notably by introducing the notion of differentiation along vector fields, but this is beyond the scope of this text.)
\end{remark}

\begin{example}\label{10.1.6}
    Let \(f : \mathbf{R} \to \mathbf{R}\) be the function \(f(x) \coloneqq \abs*{x}\), and let \(x_0 = 0\).
    To see whether \(f\) is differentiable at \(0\) on \(\mathbf{R}\), we compute the limit
    \[
        \lim_{x \to 0 ; x \in \mathbf{R} \setminus \{0\}} \frac{f(x) - f(0)}{x - 0} = \lim_{x \to 0 ; x \in \mathbf{R} \setminus \{0\}} \frac{\abs*{x}}{x}.
    \]
    Now we take left limits and right limits.
    The right limit is
    \[
        \lim_{x \to 0 ; x \in (0, \infty)} \frac{\abs*{x}}{x} = \lim_{x \to 0 ; x \in (0, \infty)} \frac{x}{x} = \lim_{x \to 0 ; x \in (0, \infty)} 1 = 1,
    \]
    while the left limit is
    \[
        \lim_{x \to 0 ; x \in (-\infty, 0)} \frac{\abs*{x}}{x} = \lim_{x \to 0 ; x \in (-\infty, 0)} \frac{-x}{x} = \lim_{x \to 0 ; x \in (-\infty, 0)} -1 = -1,
    \]
    and these limits do not match.
    Thus \(\lim_{x \to 0 ; x \in (0, \infty)} \frac{\abs*{x}}{x}\) does not exist, and \(f\) is not differentiable at \(0\) on \(\mathbf{R}\).
    However, if one restricts \(f\) to \([0, \infty)\), then the restricted function \(f|_{[0, \infty)}\) is differentiable at \(0\) on \([0, \infty)\), with derivative \(1\):
    \[
    \lim_{x \to 0 ; x \in [0, \infty) \setminus \{0\}} \frac{f(x) - f(0)}{x - 0} = \lim_{x \to 0 ; x \in (0, \infty)} \frac{\abs*{x}}{x} = 1.
        \]
        Similarly, when one restricts \(f\) to \((-\infty, 0]\), the restricted function \(f|_{(-\infty, 0]}\) is differentiable at \(0\) on \((-\infty, 0]\), with derivative \(-1\).
    Thus even when a function is not differentiable, it is sometimes possible to restore the differentiability by restricting the domain of the function.
\end{example}

\begin{proposition}[Newton's approximation]\label{10.1.7}
    Let \(X\) be a subset of \(\mathbf{R}\), let \(x_0 \in X\) be a limit point of \(X\), let \(f : X \to \mathbf{R}\) be a function, and let \(L\) be a real number.
    Then the following statements are logically equivalent:
    \begin{enumerate}
        \item \(f\) is differentiable at \(x_0\) on \(X\) with derivative \(L\).
        \item For every \(\varepsilon > 0\), there exists a \(\delta > 0\) such that \(f(x)\) is \(\varepsilon \abs*{x - x_0}\)-close to \(f(x_0) + L(x - x_0)\) whenever \(x \in X\) is \(\delta\)-close to \(x_0\), i.e., we have
              \[
                  \abs*{f(x) - (f(x_0) + L(x - x_0))} \leq \varepsilon \abs*{x - x_0}
              \]
              whenever \(x \in X\) and \(\abs*{x - x_0} \leq \delta\).
    \end{enumerate}
\end{proposition}

\begin{proof}
    We first show that the first statement implies the second statement.
    Since \(f\) is differentiable at \(x_0\) on \(X\) with derivative \(L\), by Definition \ref{10.1.1} we have
    \[
        \lim_{x \to x_0 ; x \in X \setminus \{x_0\}} \frac{f(x) - f(x_0)}{x - x_0} = L.
    \]
    By Definition \ref{9.3.6} this means
    \[
        \forall\ \varepsilon \in \mathbf{R}^+, \exists\ \delta \in \mathbf{R}^+ : \bigg(\forall\ x \in X \setminus \{x_0\}, \abs*{x - x_0} < \delta \implies \abs*{\frac{f(x) - f(x_0)}{x - x_0} - L} \leq \varepsilon\bigg).
    \]
    Thus we have
    \begin{align*}
                 & \forall\ x \in X \setminus \{x_0\}, \abs*{x - x_0} \leq \delta / 2 < \delta             \\
        \implies & \abs*{\frac{f(x) - f(x_0)}{x - x_0} - L} \leq \varepsilon                               \\
        \implies & \abs*{\frac{f(x) - f(x_0)}{x - x_0} - L} \abs*{x - x_0} \leq \varepsilon \abs*{x - x_0} \\
        \implies & \abs*{(f(x) - f(x_0)) - L(x - x_0)} \leq \varepsilon \abs*{x - x_0}                     \\
        \implies & \abs*{f(x) - (f(x_0) + L(x - x_0))} \leq \varepsilon \abs*{x - x_0}.
    \end{align*}
    If \(x = x_0\), then we have
    \begin{align*}
                 & 0 = \abs*{x_0 - x_0} \leq \delta / 2 < \delta                                      \\
        \implies & 0 = \abs*{f(x_0) - (f(x_0) + L(x_0 - x_0))} \leq \varepsilon \abs*{x_0 - x_0} = 0.
    \end{align*}
    Thus we have \(\forall\ \varepsilon \in \mathbf{R}^+\), \(\exists\ \delta \in \mathbf{R}^+\) such that
    \[
        \forall\ x \in X, \abs*{x - x_0} \leq \delta / 2 \implies \abs*{f(x) - (f(x_0) + L(x - x_0))} \leq \varepsilon \abs*{x - x_0}.
    \]

    Now we show that the second statement implies the first statement.
    By hypothesis we have \(\forall\ \varepsilon \in \mathbf{R}^+\), \(\exists\ \delta \in \mathbf{R}^+\) such that
    \[
        \forall\ x \in X, \abs*{x - x_0} \leq \delta \implies \abs*{f(x) - (f(x_0) + L(x - x_0))} \leq \varepsilon \abs*{x - x_0}.
    \]
    In particular, we have
    \[
        \forall\ x \in X \setminus \{x_0\}, \abs*{x - x_0} \leq \delta \implies \abs*{f(x) - (f(x_0) + L(x - x_0))} \leq \varepsilon \abs*{x - x_0}.
    \]
    Thus we have
    \begin{align*}
                 & \forall\ x \in X \setminus \{x_0\}, \abs*{x - x_0} \leq \delta        \\
        \implies & \abs*{f(x) - (f(x_0) + L(x - x_0))} \leq \varepsilon \abs*{x - x_0}   \\
        \implies & \abs*{f(x) - (f(x_0) + L(x - x_0))} / \abs*{x - x_0} \leq \varepsilon \\
        \implies & \abs*{\frac{f(x) - f(x_0)}{x - x_0} - L} \leq \varepsilon.
    \end{align*}
    By Definition \ref{9.3.6} this means
    \[
        \lim_{x \to x_0 ; x \in X \setminus \{x_0\}} \frac{f(x) - f(x_0)}{x - x_0} = L
    \]
    and by Definition \ref{10.1.1} we know that \(f\) is differentiable at \(x_0\) on \(X\) with derivative \(L\).
\end{proof}

\begin{remark}\label{10.1.8}
    Newton's approximation is of course named after the great scientist and mathematician Isaac Newton (1642 -- 1727), one of the founders of differential and integral calculus.
\end{remark}

\begin{remark}\label{10.1.9}
    We can phrase Proposition \ref{10.1.7} in a more informal way:
    if \(f\) is differentiable at \(x_0\), then one has the approximation \(f(x) \approx f(x_0) + f'(x_0)(x - x_0)\), and conversely.
\end{remark}

\begin{proposition}[Differentiability implies continuity]\label{10.1.10}
    Let \(X\) be a subset of \(\mathbf{R}\), let \(x_0 \in X\) be a limit point of \(X\), and let \(f : X \to \mathbf{R}\) be a function.
    If \(f\) is differentiable at \(x_0\), then \(f\) is also continuous at \(x_0\).
\end{proposition}

\begin{proof}
    Since \(f\) is differentiable at \(x_0\), by Definition \ref{10.1.1} we have
    \[
        L = \lim_{x \to x_0 ; x \in X \setminus \{x_0\}} \frac{f(x) - f(x_0)}{x - x_0}
    \]
    for some \(L \in \mathbf{R}\).
    By Proposition \ref{10.1.7}, we have \(\forall\ \varepsilon \in \mathbf{R}^+\), \(\exists\ \delta \in \mathbf{R}^+\) such that
    \begin{align*}
                 & \forall\ x \in X, \abs*{x - x_0} \leq \delta                                                                \\
        \implies & \abs*{f(x) - (f(x_0) + L(x - x_0))} \leq \varepsilon \abs*{x - x_0}                                         \\
        \implies & \abs*{f(x) - (f(x_0) + L(x - x_0))} + \abs*{L(x - x_0)} \leq \varepsilon \abs*{x - x_0} + \abs*{L(x - x_0)} \\
        \implies & \abs*{f(x) - (f(x_0) + L(x - x_0))} + \abs*{L(x - x_0)} \leq (\varepsilon + \abs*{L}) \abs*{x - x_0}        \\
        \implies & \abs*{f(x) - (f(x_0) + L(x - x_0)) + L(x - x_0)}                                                            \\
                 & \leq \abs*{f(x) - (f(x_0) + L(x - x_0))} + \abs*{L(x - x_0)}                                                \\
                 & \leq (\varepsilon + \abs*{L}) \abs*{x - x_0}                                                                \\
        \implies & \abs*{f(x) - f(x_0)} \leq (\varepsilon + \abs*{L}) \abs*{x - x_0}.
    \end{align*}
    Let \(\delta' = \min(\delta, \varepsilon / (\varepsilon + \abs*{L}))\).
    Then we have
    \begin{align*}
                 & \forall\ x \in X, \abs*{x - x_0} < \delta' \leq \delta                                                                                                                       \\
        \implies & \abs*{f(x) - f(x_0)} \leq (\varepsilon + \abs*{L}) \abs*{x - x_0}                                                                                                            \\
        \implies & \abs*{f(x) - f(x_0)} \leq (\varepsilon + \abs*{L}) \abs*{x - x_0} \leq (\varepsilon + \abs*{L}) \delta' \leq (\varepsilon + \abs*{L}) \varepsilon / (\varepsilon + \abs*{L}) \\
        \implies & \abs*{f(x) - f(x_0)} \leq \varepsilon.
    \end{align*}
    Thus by Definition \ref{9.3.6} we have \(\lim_{x \to x_0 ; x \in X} f(x) = f(x_0)\), and by Definition \ref{9.4.1} \(f\) is continuous at \(x_0\).
\end{proof}

\begin{definition}[Differentiability on a domain]\label{10.1.11}
    Let \(X\) be a subset of \(\mathbf{R}\), and let \(f : X \to \mathbf{R}\) be a function.
    We say that \(f\) is \emph{differentiable on} \(X\) if, for every limit point \(x_0 \in X\), the function \(f\) is differentiable at \(x_0\) on \(X\).
\end{definition}

\begin{corollary}\label{10.1.12}
    Let \(X\) be a subset of \(\mathbf{R}\), and let \(f : X \to \mathbf{R}\) be a function which is differentiable on \(X\).
    Then \(f\) is also continuous on \(X\).
\end{corollary}

\begin{proof}
    By Lemma \ref{9.1.11} we know that \(\forall\ x_0 \in X\), \(x_0\) is an adherent point.
    By Exercise \ref{ex 9.1.9} we know that \(x_0\) is either a limit point or an isolated point.
    By Proposition \ref{10.1.10} and Definition \ref{10.1.11} we know that if \(x_0\) is a limit point then \(f\) is continuous at \(x_0\).
    So we only need to show that if \(x_0\) is an isolated point, then \(f\) is also continuous at \(x_0\).
    Suppose that \(x_0\) is an isolated point of \(X\).
    By Definition \ref{9.1.18} we know that \(\exists\ \varepsilon' \in \mathbf{R}^+\) such that \(\abs*{x - x_0} > \varepsilon'\) for all \(x \in X \setminus \{x_0\}\).
    To show that \(f\) is continuous at \(x_0\), by Definition \ref{9.4.1} and Definition \ref{9.3.6} we need to show that
    \[
        \forall\ \varepsilon \in \mathbf{R}^+, \exists\ \delta \in \mathbf{R}^+ : \bigg(\forall\ x \in X, \abs*{x - x_0} < \delta \implies \abs*{f(x) - f(x_0)} \leq \varepsilon\bigg).
    \]
    Let \(\delta = \varepsilon'\).
    Since \(x_0\) is an isolated point, the only \(x \in X\) satisfying \(\abs*{x - x_0} < \delta\) is \(x_0\).
    Thus we have \(0 = \abs*{f(x_0) - f(x_0)} \leq \varepsilon\) and \(\lim_{x \to x_0 ; x \in X} f(x) = f(x_0)\).
\end{proof}

\begin{theorem}[Differential calculus]\label{10.1.13}
    Let \(X\) be a subset of \(\mathbf{R}\), let \(x_0 \in X\) be a limit point of \(X\), and let \(f : X \to \mathbf{R}\) and \(g : X \to \mathbf{R}\) be functions.
    \begin{enumerate}
        \item If \(f\) is a constant function, i.e., there exists a real number \(c\) such that \(f(x) = c\) for all \(x \in X\), then \(f\) is differentiable at \(x_0\) and \(f'(x_0) = 0\).
        \item If \(f\) is the identity function, i.e., \(f(x) = x\) for all \(x \in X\), then \(f\) is differentiable at \(x_0\) and \(f'(x_0) = 1\).
        \item (Sum rule)
              If \(f\) and \(g\) are differentiable at \(x_0\), then \(f + g\) is also differentiable at \(x_0\), and \((f + g)'(x_0) = f'(x_0) + g'(x_0)\).
        \item (Product rule)
              If \(f\) and \(g\) are differentiable at \(x_0\), then \(fg\) is also differentiable at \(x_0\), and \((fg)'(x_0) = f'(x_0)g(x_0) + f(x_0)g'(x_0)\).
        \item If \(f\) is differentiable at \(x_0\) and \(c\) is a real number, then \(cf\) is also differentiable at \(x_0\), and \((cf)'(x_0) = cf'(x_0)\).
        \item (Difference rule)
              If \(f\) and \(g\) are differentiable at \(x_0\), then \(f - g\) is also differentiable at \(x_0\), and \((f - g)'(x_0) = f'(x_0) - g'(x_0)\).
        \item If \(g\) is differentiable at \(x_0\), and \(g\) is non-zero on \(X\) (i.e., \(g(x) \neq 0\) for all \(x \in X\)), then \(1 / g\) is also differentiable at \(x_0\), and \((\frac{1}{g})'(x_0) = -\frac{g'(x_0)}{g(x_0)^2}\).
        \item (Quotient rule)
              If \(f\) and \(g\) are differentiable at \(x_0\), and \(g\) is non-zero on \(X\), then \(f / g\) is also differentiable at \(x_0\), and
              \[
                  (\frac{f}{g})'(x_0) = \frac{f'(x_0) g(x_0) - f(x_0) g'(x_0)}{g(x_0)^2}.
              \]
    \end{enumerate}
\end{theorem}

\begin{proof}{(a)}
    We have \(\forall\ \varepsilon \in \mathbf{R}^+\), \(\forall\ \delta \in \mathbf{R}^+\) such that
    \begin{align*}
                 & \forall\ x \in X \setminus \{x_0\}, \abs*{x - x_0} < \delta                                       \\
        \implies & \abs*{\frac{f(x) - f(x_0)}{x - x_0} - 0} = \abs*{\frac{c - c}{x - x_0} - 0} = 0 \leq \varepsilon.
    \end{align*}
    Thus by Definition \ref{9.3.6} we have
    \[
        \lim_{x \to x_0 ; x \in X \setminus \{x_0\}} \frac{f(x) - f(x_0)}{x - x_0} = 0
    \]
    and by Definition \ref{10.1.1} we have \(f'(x_0) = 0\).
\end{proof}

\begin{proof}{(b)}
    We have \(\forall\ \varepsilon \in \mathbf{R}^+\), \(\forall\ \delta \in \mathbf{R}^+\) such that
    \begin{align*}
                 & \forall\ x \in X \setminus \{x_0\}, \abs*{x - x_0} < \delta                                         \\
        \implies & \abs*{\frac{f(x) - f(x_0)}{x - x_0} - 1} = \abs*{\frac{x - x_0}{x - x_0} - 1} = 0 \leq \varepsilon.
    \end{align*}
    Thus by Definition \ref{9.3.6} we have
    \[
        \lim_{x \to x_0 ; x \in X \setminus \{x_0\}} \frac{f(x) - f(x_0)}{x - x_0} = 1
    \]
    and by Definition \ref{10.1.1} we have \(f'(x_0) = 1\).
\end{proof}

\begin{proof}{(c)}
    By Definition \ref{10.1.1} and Proposition \ref{9.3.14} we have
    \begin{align*}
          & f'(x_0) + g'(x_0)                                                                                                                                       \\
        = & \lim_{x \to x_0 ; x \in X \setminus \{x_0\}} \frac{f(x) - f(x_0)}{x - x_0} + \lim_{x \to x_0 ; x \in X \setminus \{x_0\}} \frac{g(x) - g(x_0)}{x - x_0} \\
        = & \lim_{x \to x_0 ; x \in X \setminus \{x_0\}} \frac{f(x) - f(x_0)}{x - x_0} + \frac{g(x) - g(x_0)}{x - x_0}                                              \\
        = & \lim_{x \to x_0 ; x \in X \setminus \{x_0\}} \frac{f(x) - f(x_0) + g(x) - g(x_0)}{x - x_0}                                                              \\
        = & \lim_{x \to x_0 ; x \in X \setminus \{x_0\}} \frac{f(x) + g(x) - (f(x_0) + g(x_0))}{x - x_0}                                                            \\
        = & \lim_{x \to x_0 ; x \in X \setminus \{x_0\}} \frac{(f + g)(x) - (f + g)(x_0)}{x - x_0}                                                                  \\
        = & (f + g)'(x_0).
    \end{align*}
    Thus \(f + g\) is differentiable at \(x_0\) and \((f + g)'(x_0) = f'(x_0) + g'(x_0)\).
\end{proof}

\begin{proof}{(d)}
    By Definition \ref{10.1.1} and Proposition \ref{9.3.14} we have
    \begin{align*}
          & f'(x_0) g(x_0) + f(x_0) g'(x_0)                                                                                                                                                                         \\
        = & \bigg(\lim_{x \to x_0 ; x \in X \setminus \{x_0\}} \frac{f(x) - f(x_0)}{x - x_0}\bigg) \bigg(\lim_{x \to x_0 ; x \in X \setminus \{x_0\}} g(x)\bigg)                                                    \\
          & + f(x_0) \bigg(\lim_{x \to x_0 ; x \in X \setminus \{x_0\}} \frac{g(x) - g(x_0)}{x - x_0}\bigg)                                                                                                         \\
        = & \bigg(\lim_{x \to x_0 ; x \in X \setminus \{x_0\}} \frac{(f(x) - f(x_0)) g(x)}{x - x_0}\bigg) + \bigg(\lim_{x \to x_0 ; x \in X \setminus \{x_0\}} \frac{f(x_0) (g(x) - g(x_0))}{x - x_0}\bigg)         \\
        = & \bigg(\lim_{x \to x_0 ; x \in X \setminus \{x_0\}} \frac{f(x) g(x) - f(x_0) g(x)}{x - x_0}\bigg) + \bigg(\lim_{x \to x_0 ; x \in X \setminus \{x_0\}} \frac{f(x_0) g(x) - f(x_0) g(x_0)}{x - x_0}\bigg) \\
        = & \lim_{x \to x_0 ; x \in X \setminus \{x_0\}} \bigg(\frac{f(x) g(x) - f(x_0) g(x)}{x - x_0} + \frac{f(x_0) g(x) - f(x_0) g(x_0)}{x - x_0}\bigg)                                                          \\
        = & \lim_{x \to x_0 ; x \in X \setminus \{x_0\}} \frac{f(x) g(x) - f(x_0) g(x) + f(x_0) g(x) - f(x_0) g(x_0)}{x - x_0}                                                                                      \\
        = & \lim_{x \to x_0 ; x \in X \setminus \{x_0\}} \frac{f(x) g(x) - f(x_0) g(x_0)}{x - x_0}                                                                                                                  \\
        = & \lim_{x \to x_0 ; x \in X \setminus \{x_0\}} \frac{(fg)(x) - (fg)(x_0)}{x - x_0}                                                                                                                        \\
        = & (fg)'(x_0).
    \end{align*}
    Thus \(fg\) is differentiable at \(x_0\) and \((fg)'(x_0) = f'(x_0) g(x_0) + f(x_0) g'(x_0)\).
\end{proof}

\begin{proof}{(e)}
    By Definition \ref{10.1.1} and Proposition \ref{9.3.14} we have
    \begin{align*}
          & cf'(x_0)                                                                                 \\
        = & c \bigg(\lim_{x \to x_0 ; x \in X \setminus \{x_0\}} \frac{f(x) - f(x_0)}{x - x_0}\bigg) \\
        = & \lim_{x \to x_0 ; x \in X \setminus \{x_0\}} \bigg(c \frac{f(x) - f(x_0)}{x - x_0}\bigg) \\
        = & \lim_{x \to x_0 ; x \in X \setminus \{x_0\}} \frac{cf(x) - cf(x_0)}{x - x_0}             \\
        = & \lim_{x \to x_0 ; x \in X \setminus \{x_0\}} \frac{(cf)(x) - (cf)(x_0)}{x - x_0}         \\
        = & (cf)'(x_0).
    \end{align*}
    Thus \(cf\) is differentiable at \(x_0\) and \((cf)'(x_0) = cf'(x_0)\).
\end{proof}

\begin{proof}{(f)}
    \begin{align*}
          & f'(x_0) - g'(x_0)                                             \\
        = & f'(x_0) + (-g'(x_0))                                          \\
        = & f'(x_0) + ((-g)'(x_0)) & \text{(by Theorem \ref{10.1.13}(e))} \\
        = & (f + (-g))'(x_0)       & \text{(by Theorem \ref{10.1.13}(c))} \\
        = & (f - g)'(x_0).         & \text{(by Definition \ref{9.2.1})}
    \end{align*}
    Thus \(f - g\) is differentiable at \(x_0\) and \((f - g)'(x_0) = f'(x_0) - g'(x_0)\).
\end{proof}

\begin{proof}{(g)}
    By Definition \ref{10.1.1} and Proposition \ref{9.3.14} we have
    \begin{align*}
          & -\frac{g'(x_0)}{g(x_0)^2}                                                                                                                                                     \\
        = & \bigg(\lim_{x \to x_0 ; x \in X \setminus \{x_0\}} \frac{g(x) - g(x_0)}{x - x_0}\bigg) \bigg(\frac{-1}{g(x_0)^2}\bigg)                                                        \\
        = & \bigg(\lim_{x \to x_0 ; x \in X \setminus \{x_0\}} \frac{g(x) - g(x_0)}{x - x_0}\bigg) \bigg(\frac{-g(x_0)}{g(x_0) g(x_0)^2}\bigg)                                            \\
        = & \bigg(\lim_{x \to x_0 ; x \in X \setminus \{x_0\}} \frac{g(x) - g(x_0)}{x - x_0}\bigg) \bigg(\lim_{x \to x_0 ; x \in X \setminus \{x_0\}} \frac{-g(x_0)}{g(x) g(x_0)^2}\bigg) \\
        = & \lim_{x \to x_0 ; x \in X \setminus \{x_0\}} \Bigg(\bigg(\frac{g(x) - g(x_0)}{x - x_0}\bigg) \bigg(\frac{-g(x_0)}{g(x) g(x_0)^2}\bigg)\Bigg)                                  \\
        = & \lim_{x \to x_0 ; x \in X \setminus \{x_0\}} \frac{\frac{g(x_0)(g(x_0) - g(x))}{g(x) g(x_0)^2}}{x - x_0}                                                                      \\
        = & \lim_{x \to x_0 ; x \in X \setminus \{x_0\}} \frac{\frac{g(x_0) - g(x)}{g(x) g(x_0)}}{x - x_0}                                                                                \\
        = & \lim_{x \to x_0 ; x \in X \setminus \{x_0\}} \frac{\frac{1}{g(x)} - \frac{1}{g(x_0)}}{x - x_0}                                                                                \\
        = & (\frac{1}{g})'(x_0).
    \end{align*}
    Thus \(1 / g\) is differentiable at \(x_0\) and \((1 / g)'(x_0) = -\frac{g'(x_0)}{g(x_0)^2}\).
\end{proof}

\begin{proof}{(h)}
    \begin{align*}
          & (\frac{f}{g})'(x_0)                                                                                      \\
        = & (f \cdot \frac{1}{g})'(x_0)                                       & \text{(by Definition \ref{9.2.1})}   \\
        = & f'(x_0) \frac{1}{g}(x_0) + f(x_0) (\frac{1}{g})'(x_0)             & \text{(by Theorem \ref{10.1.13}(d))} \\
        = & \frac{f'(x_0)}{g(x_0)} + f(x_0) (\frac{1}{g})'(x_0)               & \text{(by Definition \ref{9.2.1})}   \\
        = & \frac{f'(x_0)}{g(x_0)} + f(x_0) \frac{-g'(x_0)}{g(x_0)^2}         & \text{(by Theorem \ref{10.1.13}(g))} \\
        = & \frac{f'(x_0) g(x_0)}{g(x_0)^2} - \frac{f(x_0) g'(x_0)}{g(x_0)^2}                                        \\
        = & \frac{f'(x_0) g(x_0) - f(x_0) g'(x_0)}{g(x_0)^2}.                                                        \\
    \end{align*}
    Thus \(f / g\) is differentiable at \(x_0\) and \((f / g)'(x_0) = \frac{f'(x_0) g(x_0) - f(x_0) g'(x_0)}{g(x_0)^2}\).
\end{proof}

\begin{remark}\label{10.1.14}
    The product rule is also known as the \emph{Leibniz rule}, after Gottfried Leibniz (1646 -- 1716), who was the other founder of differential and integral calculus besides Newton.
\end{remark}