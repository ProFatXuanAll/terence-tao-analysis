\section{Sums of non-negative numbers}\label{i:sec:7.3}

\begin{note}
  When all the terms in a series are non-negative, there is no distinction between conditional convergence and absolute convergence.
\end{note}

\begin{prop}\label{i:7.3.1}
  Let \(\sum_{n = m}^\infty a_n\) be a formal series of non-negative real numbers.
  Then this series is convergent iff there is a real number \(M\) such that
  \[
    \sum_{n = m}^N a_n \leq M \text{ for all integers } N \geq m.
  \]
\end{prop}

\begin{proof}
  Suppose \(\sum_{n = m}^\infty a_n\) is a series of non-negative numbers.
  Then the partial sums \(S_N \coloneqq \sum_{n = m}^N a_n\) are increasing, i.e., \(S_{N + 1} \geq S_N\) for all \(N \geq m\).
  From \cref{i:6.3.8} and \cref{i:6.1.17}, we thus see that the sequence \((S_N)_{n = m}^\infty\) is convergent iff it has an upper bound \(M\).
\end{proof}

\begin{cor}[Comparison test]\label{i:7.3.2}
  Let \(\sum_{n = m}^\infty a_n\) and \(\sum_{n = m}^\infty b_n\) be two formal series of real numbers, and suppose that \(\abs{a_n} \leq b_n\) for all \(n \geq m\).
  Then if \(\sum_{n = m}^\infty b_n\) is convergent, then \(\sum_{n = m}^\infty a_n\) is absolutely convergent, and in fact
  \[
    \abs{\sum_{n = m}^\infty a_n} \leq \sum_{n = m}^\infty \abs{a_n} \leq \sum_{n = m}^\infty b_n.
  \]
\end{cor}

\begin{proof}
  Let \(N \in \N\).
  Then we have
  \begin{align*}
             & \sum_{n = m}^\infty b_n \text{ converges}                                                                                  \\
    \implies & \exists M \in \R : \forall N \geq m, \sum_{n = m}^N b_n \leq M                                 &  & \by{i:7.3.1}           \\
    \implies & \exists M \in \R : \forall N \geq m, \sum_{n = m}^N \abs{a_n} \leq \sum_{n = m}^N b_n \leq M   &  & \text{(by hypothesis)} \\
    \implies & \sum_{n = m}^\infty \abs{a_n} \text{ converges}                                                &  & \by{i:7.3.1}           \\
    \implies & \sum_{n = m}^\infty \abs{a_n} \leq \sum_{n = m}^\infty b_n                                     &  & \by{i:6.4.13}          \\
    \implies & \abs{\sum_{n = m}^\infty a_n} \leq \sum_{n = m}^\infty \abs{a_n} \leq \sum_{n = m}^\infty b_n. &  & \by{i:7.2.9}
  \end{align*}
\end{proof}

\begin{note}
  We can also run the comparison test in the contrapositive:
  if we have \(\abs{a_n} \leq b_n\) for all \(n \geq m\), and \(\sum_{n = m}^\infty a_n\) is not absolutely convergent, then \(\sum_{n = m}^\infty b_n\) is not conditionally convergent.
\end{note}

\begin{lem}[Geometric series]\label{i:7.3.3}
  Let \(x\) be a real number.
  If \(\abs{x} \geq 1\), then the series \(\sum_{n = 0}^\infty x^n\) is divergent.
  If however \(\abs{x} < 1\), then the series is absolutely convergent and
  \[
    \sum_{n = 0}^\infty x^n = 1 / (1 - x).
  \]
\end{lem}

\begin{proof}
  We first show that if \(\abs{x} \geq 1\), then \(\sum_{n = 0}^\infty x^n\) is divergent.
  We split into two cases:
  \begin{enumerate}
    \item If \(x = 1\), then \(\lim_{n \to \infty} x^n = 1\). By zero test (\cref{i:7.2.6}) \(\sum_{n = 0}^\infty x^n\) diverges.
    \item If \(x = -1\) or \(\abs{x} > 1\), then by \cref{i:6.5.2} \(\lim_{n \to \infty} x^n\) diverges.
          Thus, by zero test (\cref{i:7.2.6}) \(\sum_{n = 0}^\infty x^n\) diverges.
  \end{enumerate}
  From all cases above, we conclude that if \(\abs{x} \geq 1\), then \(\sum_{n = 0}^\infty x^n\) diverges.

  Next we show that \(\sum_{n = 0}^N x^n = (1 - x^{N + 1}) / (1 - x)\).
  We induct on \(N\).
  For \(N = 0\), by \cref{i:7.1.1} we have
  \[
    \sum_{n = 0}^0 x^n = x^0 = 1 = \dfrac{1 - x}{1 - x} = \dfrac{1 - x^1}{1 - x}.
  \]
  So the base case holds.
  Suppose inductivly that for some \(N \geq 0\) we have \(\sum_{n = 0}^N x^n = (1 - x^{N + 1}) / (1 - x)\).
  Then for \(N + 1\), we have
  \begin{align*}
    \sum_{n = 0}^{N + 1} x^n & = \sum_{n = 0}^N x^n + x^{N + 1}                                      &  & \by{i:7.1.1} \\
                             & = \dfrac{1 - x^{N + 1}}{1 - x} + x^{N + 1}                            &  & \byIH        \\
                             & = \dfrac{1 - x^{N + 1}}{1 - x} + \dfrac{(1 - x) x^{N + 1}}{1 - x}                       \\
                             & = \dfrac{1 - x^{N + 1}}{1 - x} + \dfrac{x^{N + 1} - x^{N + 2}}{1 - x}                   \\
                             & = \dfrac{1 - x^{N + 2}}{1 - x}.
  \end{align*}
  This closes the induction.
  Using similar arguments, we can show that
  \[
    \sum_{n = 0}^N \abs{x^n} = \dfrac{1 - \abs{x^{N + 1}}}{1 - \abs{x}}.
  \]

  Now we use the induction result to show that if \(\abs{x} < 1\), then \(\sum_{n = 0}^\infty x^n\) is absolutely convergent and \(\sum_{n = 0}^\infty x^n = 1 / (1 - x)\).
  \begin{align*}
    \sum_{n = 0}^\infty x^n & = \lim_{N \to \infty} \sum_{n = 0}^N x^n                               &  & \by{i:7.2.2}              \\
                            & = \lim_{N \to \infty} \dfrac{1 - x^{N + 1}}{1 - x}                     &  & \text{(from claim above)} \\
                            & = \dfrac{\lim_{N \to \infty} 1 - x^{N + 1}}{1 - x}                     &  & \by{i:6.1.19}[f]          \\
                            & = \dfrac{\lim_{N \to \infty} 1 - \lim_{N \to \infty} x^{N + 1}}{1 - x} &  & \by{i:6.1.19}[d]          \\
                            & = \dfrac{1 - (\lim_{N \to \infty} x^{N + 1})}{1 - x}                   &  & \by{i:ac:6.5.1}           \\
                            & = \dfrac{1 - 0}{1 - x}                                                 &  & \by{i:6.5.2}              \\
                            & = \dfrac{1}{1 - x}.
  \end{align*}
  Using similar arguments, we can show that \(\sum_{n = 0}^\infty \abs{x^n} = 1 / (1 - \abs{x})\).
  Thus, we conclude that if \(\abs{x} < 1\), then \(\sum_{n = 0}^\infty x^n\) is absolutely convergent and \(\sum_{n = 0}^\infty x^n = 1 / (1 - x)\).
\end{proof}

\begin{prop}[Cauchy criterion]\label{i:7.3.4}
  Let \((a_n)_{n = 1}^\infty\) be a decreasing sequence of non-negative real numbers
  (so \(a_n \geq 0\) and \(a_{n + 1} \leq a_n\) for all \(n \geq 1\)).
  Then the series \(\sum_{n = 1}^\infty a_n\) is convergent iff the series
  \[
    \sum_{k = 0}^\infty 2^k a_{2^k} = a_1 + 2a_2 + 4a_4 + 8a_8 + \dots
  \]
  is convergent.
\end{prop}

\begin{proof}
  Let \(S_N \coloneqq \sum_{n = 1}^N a_n\) be the partial sums of \(\sum_{n = 1}^\infty a_n\), and let \(T_K \coloneqq \sum_{k = 0}^K 2^k a_{2^k}\) be the partial sums of \(\sum_{k = 0}^\infty 2^k a_{2^k}\).
  In light of \cref{i:7.3.1}, our task is to show that the sequence \((S_N)_{N = 1}^\infty\) is bounded iff the sequence \((T_K)_{K = 0}^\infty\) is bounded.
  From \cref{i:7.3.6} we see that if \((S_N)_{N = 1}^\infty\) is bounded, then \((S_{2^K})_{K = 0}^\infty\) is bounded, and hence \((T_K)_{K = 0}^\infty\) is bounded.
  Conversely, if \((T_K)_{K = 0}^\infty\) is bounded, then \cref{i:7.3.6} implies that \((S_{2^{K + 1} - 1})_{K = 0}^\infty\) is bounded, i.e., there is an \(M\) such that \(S_{2^{K + 1} - 1} \leq M\) for all natural numbers \(K\).
  But one can easily show (using induction) that \(2^{K + 1} - 1 \geq K + 1\), and hence that \(S_{K + 1} \leq M\) for all natural numbers \(K\), hence \((S_N)_{N = 1}^\infty\) is bounded.
\end{proof}

\begin{rmk}\label{i:7.3.5}
  An interesting feature of this criterion is that it only uses a small number of elements of the sequence \(a_n\)
  (namely, those elements whose index \(n\) is a power of \(2\), \(n = 2^k\))
  in order to determine whether the whole series is convergent or not.
\end{rmk}

\begin{lem}\label{i:7.3.6}
  For any natural number \(K\), we have \(S_{2^{K + 1} - 1} \leq T_K \leq 2S_{2^K}\).
\end{lem}

\begin{proof}
  We induct on \(K\).
  First, we prove the claim when \(K = 0\), i.e.
  \[
    S_1 \leq T_0 \leq 2S_1.
  \]
  This becomes
  \[
    a_1 \leq a_1 \leq 2a_1
  \]
  which is clearly true, since \(a_1\) is non-negative.
  Now suppose the claim has been proven for \(K\), and now we try to prove it for \(K + 1\):
  \[
    S_{2^{K + 2} - 1} \leq T_{K + 1} \leq 2S_{2^{K + 1}}.
  \]
  Clearly, we have
  \[
    T_{K + 1} = T_K + 2^{K + 1} a_{2^{K + 1}}.
  \]
  Also, we have
  (using \cref{i:7.1.4}(a) and (f), and the hypothesis that the \(a_n\) are decreasing)
  \[
    S_{2^{K + 1}} = S_{2^K} + \sum_{n = 2^K + 1}^{2^{K + 1}} a_n \geq S_{2^K} + \sum_{n = 2^K + 1}^{2^{K + 1}} a_{2^{K + 1}} = S_{2^K} + 2^K a_{2^{K + 1}}
  \]
  and hence
  \[
    2S_{2^{K + 1}} \geq 2S_{2^K} + 2^{K + 1} a_{2^{K + 1}}.
  \]
  Similarly, we have
  \begin{align*}
    S_{2^{K + 2} - 1} & = S_{2^{K + 1} - 1} + \sum_{n = 2^{K + 1}}^{2^{K + 2} - 1} a_n              \\
                      & \leq S_{2^{K + 1} - 1} + \sum_{n = 2^{K + 1}}^{2^{K + 2} - 1} a_{2^{K + 1}} \\
                      & = S_{2^{K + 1} - 1} + 2^{K + 1} a_{2^{K + 1}}.
  \end{align*}
  Combining these inequalities with the induction hypothesis
  \[
    S_{2^{K + 1} - 1} \leq T_K \leq 2S_{2^K}
  \]
  we obtain
  \[
    S_{2^{K + 2} - 1} \leq T_{K + 1} \leq 2S_{2^{K + 1}}
  \]
  as desired.
  This proves the claim.
\end{proof}

\begin{cor}\label{i:7.3.7}
  Let \(q > 0\) be a real number.
  Then the series \(\sum_{n = 1}^\infty 1 / n^q\) is convergent when \(q > 1\) and divergent when \(q \leq 1\).
\end{cor}

\begin{proof}
  The sequence \((1 / n^q)_{n = 1}^\infty\) is non-negative and decreasing (by \cref{i:6.7.3}), and so the Cauchy criterion (\cref{i:7.3.4}) applies.
  Thus, this series is convergent iff
  \[
    \sum_{k = 0}^\infty 2^k \dfrac{1}{(2^k)^q}
  \]
  is convergent.
  But by the laws of exponentiation (\cref{i:6.7.3}) we can rewrite this as the geometric series
  \[
    \sum_{k = 0}^\infty (2^{1 - q})^k.
  \]
  By \cref{i:7.3.3}, the geometric series \(\sum_{k = 0}^\infty x^k\) converges iff \(\abs{x} < 1\).
  Thus, the series \(\sum_{n = 1}^\infty 1 / n^q\) will converge iff \(\abs{2^{1 - q}} < 1\), which happens iff \(q > 1\).
\end{proof}

\begin{note}
  In particular, the series \(\sum_{n = 1}^\infty 1 / n\) (also known as the \emph{harmonic series}) is divergent, as claimed earlier.
  However, the series is \(\sum_{n = 1}^\infty 1 / n^2\) convergent.
\end{note}

\begin{rmk}\label{i:7.3.8}
  The quantity \(\sum_{n = 1}^\infty 1 / n^q\), when it converges, is called \(\zeta(q)\), the \\
  \emph{Riemann-zeta function of \(q\)}.
  This function is very important in number theory, particularly in the distribution of the primes;
  there is a very famous unsolved problem regarding this function, called the \emph{Riemann hypothesis}, but to discuss it further is far beyond the scope of this text.
  I will mention however that there is a US\$ \(1\) million prize
  - and instant fame among all mathematicians -
  attached to the solution to this problem.
\end{rmk}

\exercisesection

\begin{ex}\label{i:ex:7.3.1}
  Use \cref{i:7.3.1} to prove \cref{i:7.3.2}.
\end{ex}

\begin{proof}
  See \cref{i:7.3.2}.
\end{proof}

\begin{ex}\label{i:ex:7.3.2}
  Prove \cref{i:7.3.3}.
\end{ex}

\begin{proof}
  See \cref{i:7.3.3}.
\end{proof}

\begin{ex}\label{i:ex:7.3.3}
  Let \(\sum_{n = 0}^\infty a_n\) be an absolutely convergent series of real numbers such that \(\sum_{n = 0}^\infty \abs{a_n} = 0\).
  Show that \(a_n = 0\) for every natural number \(n\).
\end{ex}

\begin{proof}
  Let \(N, k \in \N\).
  Then we have \(\forall 0 \leq k \leq N\),
  \begin{align*}
             & 0 \leq \abs{a_k} \leq \sum_{n = 0}^N \abs{a_n}                                                                                \\
    \implies & \lim_{N \to \infty} 0 \leq \lim_{N \to \infty} \abs{a_k} \leq \lim_{N \to \infty} \sum_{n = 0}^N \abs{a_n} &  & \by{i:6.4.13} \\
    \implies & 0 \leq \abs{a_k} \leq \sum_{n = 0}^\infty \abs{a_n} = 0                                                    &  & \by{i:7.2.2}  \\
    \implies & \abs{a_k} = 0 = a_k.
  \end{align*}
  Since \(N\) was arbitrary, we have \(a_k = 0\) for every \(k \geq 0\).
\end{proof}
