\section{Multiplication}\label{i:sec:2.3}

\begin{defn}[Multiplication of natural numbers]\label{i:2.3.1}
  Let \(m\) be a natural number.
  To multiply zero to \(m\), we define \(0 \times m \coloneqq 0\).
  Now suppose inductively that we have defined how to multiply \(n\) to \(m\).
  Then we can multiply \(n\pp\) to \(m\) by defining \((n\pp) \times m \coloneqq (n \times m) + m\).
\end{defn}

\begin{ac}\label{i:ac:2.3.1}
  The product of two natural numbers is a natural number.
\end{ac}

\begin{proof}[\pf{i:ac:2.3.1}]
  Let \(n, m\) be two natural numbers.
  We induct on \(n\).
  For \(n = 0\), by \cref{i:2.3.1}, we have \(0 \times m = 0\), which is a natural number by \cref{i:2.1}.
  So the base case holds.
  Suppose inductively that, for some natural number \(n\), we know that \(n \times m\) is a natural number.
  We want to show that \((n\pp) \times m\) is a natural number.
  By \cref{i:2.3.1}, \((n\pp) \times m = (n \times m) + m\).
  By the induction hypothesis, \(n \times m\) is a natural number.
  By \cref{i:ac:2.2.1}, \((n \times m) + m\) is a natural number.
  Thus, \((n\pp) \times m\) is a natural number.
  This closes the induction.
\end{proof}

\begin{ac}\label{i:ac:2.3.2}
  Let \(n\) be a natural number.
  Then \(n \times 0 = 0\).
\end{ac}

\begin{proof}[\pf{i:ac:2.3.2}]
  We induct on \(n\).
  For \(n = 0\), by \cref{i:2.3.1}, we have \(0 \times 0 = 0\).
  So the base case holds.
  Suppose inductively that, for some natural number \(n\), we have \(n \times 0 = 0\).
  Then, for \(n\pp\), we have
  \begin{align*}
    (n\pp) \times 0 & = (n \times 0) + 0 &  & \by{i:2.3.1} \\
                    & = 0 + 0            &  & \byIH        \\
                    & = 0.               &  & \by{i:2.2.1}
  \end{align*}
  This closes the induction.
\end{proof}

\begin{ac}\label{i:ac:2.3.3}
  Let \(n, m\) be natural numbers.
  Then \(n \times (m\pp) = (n \times m) + n\).
\end{ac}

\begin{proof}[\pf{i:ac:2.3.3}]
  We induct on \(n\) and fix \(m\).
  For \(n = 0\), by \cref{i:2.3.1}, we have \(0 \times (m\pp) = 0\).
  So the base case holds.
  Suppose inductively that, for some natural number \(n\), we have \(n \times (m\pp) = (n \times m) + n\).
  Then, for \(n\pp\), we have
  \begin{align*}
    (n\pp) \times (m\pp)
     & = (n \times (m\pp)) + (m\pp)  &  & \by{i:2.3.1} \\
     & = ((n \times m) + n) + (m\pp) &  & \byIH        \\
     & = (n \times m) + (n + (m\pp)) &  & \by{i:2.2.5} \\
     & = (n \times m) + ((n + m)\pp) &  & \by{i:2.2.3} \\
     & = (n \times m) + ((m + n)\pp) &  & \by{i:2.2.4} \\
     & = (n \times m) + (m + (n\pp)) &  & \by{i:2.2.3} \\
     & = ((n \times m) + m) + (n\pp) &  & \by{i:2.2.5} \\
     & = ((n\pp) \times m) + (n\pp). &  & \by{i:2.3.1}
  \end{align*}
  This closes the induction.
\end{proof}

\begin{lem}[Multiplication is commutative]\label{i:2.3.2}
  Let \(n, m\) be natural numbers.
  Then \(n \times m = m \times n\).
\end{lem}

\begin{proof}[\pf{i:2.3.2}]
  We induct on \(n\) and fix \(m\).
  For \(n = 0\), by \cref{i:2.3.1}, we have \(0 \times m = 0\), and by \cref{i:ac:2.3.2}, we have \(m \times 0 = 0\).
  So the base case holds.
  Suppose inductively that, for some natural number \(n\), we have \(n \times m = m \times n\).
  Then, for \(n\pp\), we have
  \begin{align*}
    (n\pp) \times m & = (n \times m) + m &  & \by{i:2.3.1}    \\
                    & = (m \times n) + m &  & \byIH           \\
                    & = m \times (n\pp). &  & \by{i:ac:2.3.3}
  \end{align*}
  This closes the induction.
\end{proof}

\begin{note}
  We will now abbreviate \(n \times m\) as \(nm\), and use the usual convention that multiplication takes precedence over addition, thus for instance \(ab + c\) means \((a \times b) + c\), not \(a \times (b + c)\).
\end{note}

\begin{lem}[Positive natural numbers have no zero divisors]\label{i:2.3.3}
  Let \(n, m\) be natural numbers.
  Then \(n \times m = 0\) iff at least one of \(n, m\) is equal to zero.
  In particular, if \(n\) and \(m\) are both positive, then \(nm\) is also positive.
\end{lem}

\begin{proof}[\pf{i:2.3.3}]
  First, suppose that \(n \times m = 0\).
  Suppose for the sake of contradiction that \(n \neq 0 \neq m\).
  By \cref{i:2.2.7}, this means \(n, m\) are positive natural numbers.
  Then by \cref{i:2.2.10}, there exist some natural numbers \(a, b\), such that \(n = a\pp\) and \(m = b\pp\).
  Thus, we have
  \begin{align*}
    n \times m & = (a\pp) \times (b\pp)                        \\
               & = a \times (b\pp) + (b\pp). &  & \by{i:2.3.1}
  \end{align*}
  By \cref{i:2.3}, we know that \(b\pp \neq 0\).
  Thus, by \cref{i:2.2.8}, we know that \(n \times m\) is a positive natural number.
  But this contradict to \(n \times m = 0\).
  Thus, we must have either \(n = 0\) or \(m = 0\).

  Now suppose that \(n = 0\) or \(m = 0\).
  If \(n = 0\), then by \cref{i:2.3.1}, we have \(n \times m = 0 \times m = 0\).
  If \(m = 0\), then by \cref{i:ac:2.3.2}, we have \(n \times m = n \times 0 = 0\).
  In either cases, we have \(n \times m = 0\).

  From all proofs above, we conclude that \(n \times m = 0 \iff (n = 0) \lor (m = 0)\).
  Thus, we have \(n \times m \neq 0 \iff (n \neq 0) \land (m \neq 0)\).
  By \cref{i:2.2.7}, we see that \(n, m\) are positive natural numbers iff \(n \times m \neq 0\).
\end{proof}

\begin{ac}\label{i:ac:2.3.4}
  Let \(n\) be a natural number.
  Then \(n1 = 1n = n\).
\end{ac}

\begin{proof}[\pf{i:ac:2.3.4}]
  By \cref{i:2.3.2}, we know that \(n1 = 1n\).
  Thus, we only need to show that \(n1 = n\).
  We induct on \(n\).
  For \(n = 0\), by \cref{i:2.3.3}, we have \(0 \times 1 = 0\).
  So the base case holds.
  Suppose inductively that \(n1 = n\) is true for some natural number \(n\).
  Then, for \(n + 1\), by \cref{i:2.3.1}, we have \((n + 1) \times 1 = n1 + 1\).
  By the induction hypothesis, we have \(n1 = n\).
  Thus, we have \((n + 1) \times 1 = n + 1\), and this closes the induction.
\end{proof}

\begin{prop}[Distributive law]\label{i:2.3.4}
  For any natural numbers \(a, b, c\), we have \(a(b + c) = ab + ac\) and \((b + c)a = ba + ca\).
\end{prop}

\begin{proof}[\pf{i:2.3.4}]
  Since multiplication is commutative we only need to show the first identity \(a(b + c) = ab + ac\).
  We keep \(a\) and \(b\) fixed, and use induction on \(c\).
  Let's prove the base case \(c = 0\), i.e., \(a(b + 0) = ab + a0\).
  The left-hand side is \(ab\), while the right-hand side is \(ab + 0 = ab\), so we are done with the base case.
  Now let us suppose inductively that \(a(b + c) = ab + ac\), and let us prove that \(a(b + (c\pp)) = ab + a(c\pp)\).
  The left-hand side is \(a((b + c)\pp) = a(b + c) + a\) by \cref{i:ac:2.3.3}, while the right-hand side is \(ab + ac + a = a(b + c) + a\) by the induction hypothesis, and so we can close the induction.
\end{proof}

\begin{prop}[Multiplication is associative]\label{i:2.3.5}
  For any natural numbers \(a, b, c\), we have \((a \times b) \times c = a \times (b \times c)\).
\end{prop}

\begin{proof}[\pf{i:2.3.5}]
  We keep \(a\) and \(b\) fixed, and use induction on \(c\).
  For \(c = 0\), by \cref{i:ac:2.3.2}, we have \((a \times b) \times 0 = 0 = a \times 0 = a \times (b \times 0)\).
  So the base case holds.
  Suppose inductively that, for some natural number \(c\), we have \((a \times b) \times c = a \times (b \times c)\).
  Then, for \(c\pp\), we have
  \begin{align*}
    (a \times b) \times (c\pp) & = (a \times b) \times c + a \times b &  & \by{i:ac:2.3.3} \\
                               & = a \times (b \times c) + a \times b &  & \byIH           \\
                               & = a \times (b \times c + b)          &  & \by{i:2.3.4}    \\
                               & = a \times (b \times (c\pp)).        &  & \by{i:ac:2.3.3}
  \end{align*}
  This closes the induction.
\end{proof}

\begin{prop}[Multiplication preserves order]\label{i:2.3.6}
  If \(a, b\) are natural numbers such that \(a < b\), and \(c\) is positive, then \(ac < bc\).
\end{prop}

\begin{proof}[\pf{i:2.3.6}]
  Since \(a < b\), we have \(b = a + d\) for some positive \(d\) by \cref{i:2.2.12}(f).
  Multiplying by \(c\) and using the distributive law (\cref{i:2.3.4}), we obtain \(bc = ac + dc\).
  Since \(d\) is positive, and \(c\) is positive, \(dc\) is positive (\cref{i:2.3.3}), and hence \(ac < bc\) (by \cref{i:2.2.11}), as desired.
\end{proof}

\begin{cor}[Cancellation law]\label{i:2.3.7}
  Let \(a, b, c\) be natural numbers such that \(ac = bc\) and \(c\) is non-zero.
  Then \(a = b\).
\end{cor}

\begin{proof}[\pf{i:2.3.7}]
  By the trichotomy of order (\cref{i:2.2.13}), we have three cases: \(a < b\), \(a = b\), \(a > b\).
  Suppose first that \(a < b\), then by \cref{i:2.3.6}, we have \(ac < bc\), a contradiction.
  We can obtain a similar contradiction when \(a > b\).
  Thus, the only possibility is that \(a = b\), as desired.
\end{proof}

\begin{rmk}\label{i:2.3.8}
  Just as \cref{i:2.2.6} will allow for a ``virtual subtraction'' which will eventually let us define genuine subtraction, \cref{i:2.3.7} provides a ``virtual division'' which will be needed to define genuine division later on.
\end{rmk}

\begin{prop}[Euclid's division lemma]\label{i:2.3.9}
  Let \(n\) be a natural number, and let \(q\) be a positive number.
  Then there exist natural numbers \(m, r\) such that \(0 \leq r < q\) and \(n = mq + r\).
\end{prop}

\begin{proof}[\pf{i:2.3.9}]
  We induct on \(n\) and fix \(q\).
  For \(n = 0\), let \(r = m = 0\).
  Then we have
  \begin{align*}
    mq + r & = 0q + 0                   \\
           & = 0 + 0  &  & \by{i:2.3.1} \\
           & = 0,     &  & \by{i:2.2.1}
  \end{align*}
  and
  \begin{align*}
    0 & \leq 0 = r &  & \by{i:2.2.12}[a] \\
      & < q.       &  & \by{i:2.2.11}
  \end{align*}
  So the base case holds.
  Suppose inductively that, for some natural number \(n\), there exist some natural numbers \(m, r\), such that \(n = mq + r\) and \(0 \leq r < q\).
  Then, for \(n\pp\), we have
  \begin{align*}
    n\pp & = (mq + r)\pp  &  & \byIH        \\
         & = mq + (r\pp). &  & \by{i:2.2.3} \\
  \end{align*}
  Since \(r < q\), by \cref{i:2.2.12}(e), we have \(r\pp \leq q\).
  Now we split into two cases:
  \begin{itemize}
    \item If \(r\pp < q\), then we have \(0 \leq r < r\pp < q\), and we are done in this case.
    \item If \(r\pp = q\), then by \cref{i:2.3.1}, we have
          \[
            n\pp = mq + (r\pp) = mq + q = (m\pp) \times q = (m\pp) \times q + r'
          \]
          where \(r' = 0\) and \(0 \leq r' < q\), by \cref{i:ac:2.2.4}, and we are also done in this case.
  \end{itemize}
  From all cases above, we can find some natural numbers \(m, r\), such that \(n\pp = mq + r\) and \(0 \leq r < q\).
  This closes the induction.
\end{proof}

\begin{rmk}\label{i:2.3.10}
  In other words, we can divide a natural number \(n\) by a positive number \(q\) to obtain a quotient \(m\) (which is another natural number) and a remainder \(r\) (which is less than \(q\)).
  This algorithm marks the beginning of \emph{number theory}, which is a beautiful and important subject but one which is beyond the scope of this text.
\end{rmk}

\begin{defn}[Exponentiation for natural numbers]\label{i:2.3.11}
  Let \(m\) be a natural number.
  To raise \(m\) to the power \(0\), we define \(m^0 \coloneqq 1\); in particular, we define \(0^0 \coloneqq 1\).
  Now suppose recursively that \(m^n\) has been defined for some natural number \(n\), then we define \(m^{n\pp} \coloneqq m^n \times m\).
\end{defn}

\begin{ac}\label{i:ac:2.3.5}
  For any natural number \(n\), we have \(n^1 = n\).
\end{ac}

\begin{proof}[\pf{i:ac:2.3.5}]
  We have
  \begin{align*}
    n^1 & = n^0 \times n = 1 \times n &  & \by{i:2.3.11}   \\
        & = n.                        &  & \by{i:ac:2.3.4}
  \end{align*}
\end{proof}

\exercisesection

\begin{ex}\label{i:ex:2.3.1}
  Prove \cref{i:2.3.2}.
\end{ex}

\begin{proof}[\pf{i:ex:2.3.1}]
  See \cref{i:2.3.2}
\end{proof}

\begin{ex}\label{i:ex:2.3.2}
  Prove \cref{i:2.3.3}
\end{ex}

\begin{proof}[\pf{i:ex:2.3.2}]
  See \cref{i:2.3.3}
\end{proof}

\begin{ex}\label{i:ex:2.3.3}
  Prove \cref{i:2.3.5}
\end{ex}

\begin{proof}[\pf{i:ex:2.3.3}]
  See \cref{i:2.3.5}
\end{proof}

\begin{ex}\label{i:ex:2.3.4}
  Prove the identity \((a + b)^2 = a^2 + 2ab + b^2\) for all natural numbers \(a, b\).
\end{ex}

\begin{proof}[\pf{i:ex:2.3.4}]
  We have
  \begin{align*}
    (a + b)^2 & = (a + b)^1 \times (a + b)                         &  & \by{i:2.3.11}   \\
              & = (a + b) \times (a + b)                           &  & \by{i:ac:2.3.5} \\
              & = a(a + b) + b(a + b) = aa + ab + ba + bb          &  & \by{i:2.3.4}    \\
              & = a^1 \times a + ab + ba + b^1 \times b            &  & \by{i:ac:2.3.5} \\
              & = a^2 + ab + ba + b^2                              &  & \by{i:2.3.11}   \\
              & = a^2 + ab + ab + b^2                              &  & \by{i:2.3.2}    \\
              & = a^2 + 1 \times ab + 1 \times ab + b^2            &  & \by{i:ac:2.3.4} \\
              & = a^2 + (1 + 1) \times ab + b^2 = a^2 + 2ab + b^2. &  & \by{i:2.3.4}
  \end{align*}
\end{proof}

\begin{ex}\label{i:ex:2.3.5}
  Prove \cref{i:2.3.9}
\end{ex}

\begin{proof}[\pf{i:ex:2.3.5}]
  See \cref{i:2.3.9}
\end{proof}
