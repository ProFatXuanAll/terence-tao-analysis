\section{Multiplication}\label{sec:2.3}

\begin{defn}[Multiplication of natural numbers]\label{2.3.1}
  Let \(m\) be a natural number.
  To multiply zero to \(m\), we define \(0 \times m \coloneqq 0\).
  Now suppose inductively that we have defined how to multiply \(n\) to \(m\).
  Then we can multiply \(n++\) to \(m\) by defining \((n++) \times m \coloneqq (n \times m) + m\).
\end{defn}

\begin{ac}\label{ac:2.3.1}
  The product of two natural numbers is a natural number.
\end{ac}

\begin{proof}
  Let \(n\), \(m\) be two natural numbers.
  We use induction on \(n\).
  For \(n = 0\), by \cref{2.3.1} we have \(0 \times m = 0\), which is a natural number by \cref{2.1}.
  So the base case holds.
  Suppose inductively that for some natural number \(n\) we have \(n \times m\) is a natural number.
  We want to show that \((n++) \times m\) is also a natural number.
  By \cref{2.3.1}, \((n++) \times m = (n \times m) + m\).
  By induction hypothesis \(n \times m\) is a natural number.
  By \cref{ac:2.2.1}, \((n \times m) + m\) is again a natural number.
  This closes the induction.
\end{proof}

\begin{ac}\label{ac:2.3.2}
  Let \(n\) be a natural number.
  Then \(n \times 0 = 0\).
\end{ac}

\begin{proof}
  We use induction on \(n\).
  For \(n = 0\), by \cref{2.3.1} we have \(0 \times 0 = 0\).
  So the base case holds.
  Suppose inductively that for some natural number \(n\) we have \(n \times 0 = 0\).
  Then for \(n++\) we have
  \begin{align*}
    (n++) \times 0 & = (n \times 0) + 0 &  & \by{2.3.1} \\
                   & = 0 + 0            &  & \byIH      \\
                   & = 0.               &  & \by{2.2.1}
  \end{align*}
  This closes the induction.
\end{proof}

\begin{ac}\label{ac:2.3.3}
  Let \(n\), \(m\) be natural numbers.
  Then \(n \times (m++) = (n \times m) + n\).
\end{ac}

\begin{proof}
  We use induction on \(n\) (fixed \(m\)).
  For \(n = 0\), by \cref{2.3.1} we have \(0 \times (m++) = 0\).
  So the base case holds.
  Suppose inductively that for some natural number \(n\) we have \(n \times (m++) = (n \times m) + n\).
  Then for \(n++\), we have
  \begin{align*}
    (n++) \times (m++)
     & = (n \times (m++)) + (m++)  &  & \by{2.3.1} \\
     & = (n \times m) + n + (m++)  &  & \byIH      \\
     & = (n \times m) + (n++) + m  &  & \by{2.2.5} \\
     & = (n \times m) + m + (n++)  &  & \by{2.2.4} \\
     & = ((n++) \times m) + (n++). &  & \by{2.3.1}
  \end{align*}
  This closes the induction.
\end{proof}

\begin{lem}[Multiplication is commutative]\label{2.3.2}
  Let \(n\), \(m\) be natural numbers.
  Then \(n \times m = m \times n\)
\end{lem}

\begin{proof}
  We use induction on \(n\) (fixed \(m\)).
  For \(n = 0\), by \cref{2.3.1} we have \(0 \times m = 0\), and by \cref{ac:2.3.2} we have \(m \times 0 = 0\).
  So the base case holds.
  Suppose inductively that for some natural number \(n\) we have \(n \times m = m \times n\).
  Then for \(n++\), we have
  \begin{align*}
    (n++) \times m & = (n \times m) + m &  & \by{2.3.1}    \\
                   & = (m \times n) + m &  & \byIH         \\
                   & = m \times (n++).  &  & \by{ac:2.3.3}
  \end{align*}
  This closes the induction.
\end{proof}

\begin{note}
  We will now abbreviate \(n \times m\) as \(nm\), and use the usual convention that multiplication takes precedence over addition, thus for instance \(ab + c\) means \((a \times b) + c\), not \(a \times (b + c)\).
\end{note}

\begin{lem}[Positive natural numbers have no zero divisors]\label{2.3.3}
  Let \(n\), \(m\) be natural numbers.
  Then \(n \times m = 0\) if and only if at least one of \(n\), \(m\) is equal to zero.
  In particular, if \(n\) and \(m\) are both positive, then \(nm\) is also positive.
\end{lem}

\begin{proof}
  Let \(n, m, a, b, c, d\) be natural numbers where \(n = a++\), \(m = b++\), \(c = ab + a\) and \(d = b++\).
  Then
  \begin{align*}
         & n \text{ is positive} \land m \text{ is positive}                                                \\
    \iff & n \neq 0 \land m \neq 0                           &  & \by{2.2.7}                                \\
    \iff & n = a++ \land m = b++                             &  & \by{2.2.10}                               \\
    \iff & nm = (a++)(b++)                                                                                  \\
         & = ab + a + (b++)                                  &  & \text{(by \cref{2.3.1} and \cref{2.3.2})} \\
    \iff & nm = c + d \land d \neq 0                         &  & \text{(by \cref{2.3})}                    \\
    \iff & nm \text{ is positive}.                           &  & \by{2.2.8}
  \end{align*}
  And we must also have
  \begin{align*}
         & (n \text{ is positive} \land m \text{ is positive} \iff nm \text{ is positive})                                 \\
    \iff & (\lnot (n \text{ is positive} \land m \text{ is positive}) \iff \lnot (nm \text{ is positive}))                 \\
    \iff & (n = 0 \lor m = 0 \iff nm = 0).                                                                 &  & \by{2.2.7}
  \end{align*}
\end{proof}

\begin{ac}\label{ac:2.3.4}
  Let \(n\) be a natural number.
  Then \(n1 = 1n = n\).
\end{ac}

\begin{proof}
  By \cref{2.3.2} we know that \(n1 = 1n\), thus we only need to show that \(n1 = n\).
  We use induction on \(n\).
  For \(n=0\), by \cref{2.3.3} we have \(0 \times 1 = 0\), so the base case holds.
  Suppose inductively that \(n1 = n\) is true for some \(n\).
  Then for \(n + 1\), by \cref{2.3.1} we have \((n + 1) \times 1 = n1 + 1\).
  By induction hypothesis we have \(n1 = n\).
  Thus we have \(n1 + 1 = n + 1\), and this closes the induction.
\end{proof}

\begin{prop}[Distributive law]\label{2.3.4}
  For any natural numbers \(a\), \(b\), \(c\), we have \(a(b + c) = ab + ac\) and \((b + c)a = ba + ca\).
\end{prop}

\begin{proof}
  Since multiplication is commutative we only need to show the first identity \(a(b + c) = ab + ac\).
  We keep \(a\) and \(b\) fixed, and use induction on \(c\).
  Let's prove the base case \(c = 0\), i.e., \(a(b + 0) = ab + a0\).
  The left-hand side is \(ab\), while the right-hand side is \(ab + 0 = ab\), so we are done with the base case.
  Now let us suppose inductively that \(a(b + c) = ab + ac\), and let us prove that \(a(b + (c++)) = ab + a(c++)\).
  The left-hand side is \(a((b + c)++) = a(b + c) + a\), while the right-hand side is \(ab + ac + a = a(b + c) + a\) by the induction hypothesis, and so we can close the induction.
\end{proof}

\begin{prop}[Multiplication is associative]\label{2.3.5}
  For any natural numbers \(a\), \(b\), \(c\), we have \((a \times b) \times c = a \times (b \times c)\).
\end{prop}

\begin{proof}
  We keep \(a\) and \(b\) fixed, and use induction on \(c\).
  For \(c = 0\), by \cref{ac:2.3.2} we have \((a \times b) \times 0 = 0\) and \(a \times (b \times 0) = a \times 0 = 0\).
  So the base case holds.
  Suppose inductively that for some natural number \(c\) we have \((a \times b) \times c = a \times (b \times c)\).
  Then for \(c++\), we have
  \begin{align*}
    (a \times b) \times (c++) & = (a \times b) \times c + a \times b &  & \by{ac:2.3.3} \\
                              & = a \times (b \times c) + a \times b &  & \byIH         \\
                              & = a \times (b \times c + b)          &  & \by{2.3.4}    \\
                              & = a \times (b \times (c++)).         &  & \by{ac:2.3.3}
  \end{align*}
  This closes the induction.
\end{proof}

\begin{prop}[Multiplication preserves order]\label{2.3.6}
  If \(a\), \(b\) are natural numbers such that \(a < b\), and \(c\) is positive, then \(ac < bc\).
\end{prop}

\begin{proof}
  Since \(a < b\), we have \(b = a + d\) for some positive \(d\) by \cref{2.2.12}(f).
  Multiplying by \(c\) and using the distributive law (by \cref{2.3.4}) we obtain \(bc = ac + dc\).
  Since \(d\) is positive, and \(c\) is positive, \(dc\) is positive (by \cref{2.3.3}), and hence \(ac < bc\) (by \cref{2.2.11}) as desired.
\end{proof}

\begin{cor}[Cancellation law]\label{2.3.7}
  Let \(a\), \(b\), \(c\) be natural numbers such that \(ac = bc\) and \(c\) is non-zero.
  Then \(a = b\).
\end{cor}

\begin{proof}
  By the trichotomy of order (\cref{2.2.13}), we have three cases: \(a < b\), \(a = b\), \(a > b\).
  Suppose first that \(a < b\), then by \cref{2.3.6} we have \(ac < bc\), a contradiction.
  We can obtain a similar contradiction when \(a > b\).
  Thus the only possibility is that \(a = b\), as desired.
\end{proof}

\begin{rmk}\label{2.3.8}
  Just as \cref{2.2.6} will allow for a ``virtual subtraction'' which will eventually let us define genuine subtraction, \cref{2.3.7} provides a ``virtual division'' which will be needed to define genuine division later on.
\end{rmk}

\begin{prop}[Euclidean algorithm]\label{2.3.9}
  Let \(n\) be a natural number, and let \(q\) be a positive number.
  Then there exist natural numbers \(m\), \(r\) such that \(0 \leq r < q\) and \(n = mq + r\).
\end{prop}

\begin{proof}
  We use induction on \(n\) (fixed \(q\)).
  For \(n = 0\), let \(r = m = 0\), then we have \(0 = 0q + 0\) and \(0 \leq 0 < q\).
  So the base case holds.
  Suppose inductively that for some natural number \(n\) we have \(n = mq + r\), \(0 \leq r < q\).
  Then for \(n++\), we have
  \begin{align*}
    n++ & = (mq + r)++ &  & \byIH      \\
        & = mq + (r++) &  & \by{2.2.3} \\
  \end{align*}
  Since \(r < q\), by \cref{2.2.12}(e) we have \(r++ \leq q\).
  \begin{itemize}
    \item If \(r++ < q\), then we have \(0 \leq r < r++\) and we are done in this case.
    \item If \(r++ = q\), then by \cref{2.3.1} we have
          \[
            n++ = mq + q = (m++) \times q = (m++) \times q + r'
          \]
          where \(r' = 0\) and \(0 \leq r' < q\) by \cref{2.2.11}, and we are also done in this case.
  \end{itemize}
  From all cases above we can find some natural numbers \(m, r\) such that \(n++ = mq + r\) and \(0 \leq r < q\).
  This closes the induction.
\end{proof}

\begin{rmk}\label{2.3.10}
  In other words, we can divide a natural number \(n\) by a positive number \(q\) to obtain a quotient \(m\) (which is another natural number) and a remainder \(r\) (which is less than \(q\)).
  This algorithm marks the beginning of \emph{number theory}, which is a beautiful and important subject but one which is beyond the scope of this text.
\end{rmk}

\begin{defn}[Exponentiation for natural numbers]\label{2.3.11}
  Let \(m\) be a natural number.
  To raise \(m\) to the power \(0\), we define \(m^0 \coloneqq 1\); in particular, we define \(0^0 \coloneqq 1\).
  Now suppose recursively that \(m^n\) has been defined for some natural number \(n\), then we define \(m^{n++} \coloneqq m^n \times m\).
\end{defn}

\exercisesection

\begin{ex}\label{ex:2.3.1}
  Prove \cref{2.3.2}.
\end{ex}

\begin{proof}
  See \cref{2.3.2}
\end{proof}

\begin{ex}\label{ex:2.3.2}
  Prove \cref{2.3.3}
\end{ex}

\begin{proof}
  See \cref{2.3.3}
\end{proof}

\begin{ex}\label{ex:2.3.3}
  Prove \cref{2.3.5}
\end{ex}

\begin{proof}
  See \cref{2.3.5}
\end{proof}

\begin{ex}\label{ex:2.3.4}
  Prove the identity \((a + b)^2 = a^2 + 2ab + b^2\) for all natural numbers \(a\), \(b\).
\end{ex}

\begin{proof}
  \begin{align*}
    (a + b)^2 & = (a + b)^1 \times (a + b)                             &  & \by{2.3.11} \\
              & = (a + b)^0 \times (a + b) \times (a + b)              &  & \by{2.3.11} \\
              & = 1 \times (a + b) \times (a + b)                      &  & \by{2.3.11} \\
              & = (1 \times (a + b)) \times (a + b)                    &  & \by{2.3.5}  \\
              & = (1 \times a + 1 \times b)) \times (a + b)            &  & \by{2.3.4}  \\
              & = ((0 \times a + a) + (0 \times b + b)) \times (a + b) &  & \by{2.3.1}  \\
              & = (a + b) \times (a + b)                               &  & \by{2.3.1}  \\
              & = a(a + b) + b(a + b)                                  &  & \by{2.3.5}  \\
              & = aa + ab + ba + bb                                    &  & \by{2.3.5}  \\
              & = a^2 + ab + ba + b^2                                  &  & \by{2.3.11} \\
              & = a^2 + ab + ab + b^2                                  &  & \by{2.3.2}  \\
              & = a^2 + (0 + ab) + ab + b^2                            &  & \by{2.2.1}  \\
              & = a^2 + 0ab + ab + ab + b^2                            &  & \by{2.3.1}  \\
              & = a^2 + 1ab + ab + b^2                                 &  & \by{2.3.2}  \\
              & = a^2 + 2ab + b^2                                      &  & \by{2.3.1}  \\
  \end{align*}
\end{proof}

\begin{ex}\label{ex:2.3.5}
  Prove \cref{2.3.9}
\end{ex}

\begin{proof}
  See \cref{2.3.9}
\end{proof}