\section{Uniform continuity}\label{sec 9.9}

\setcounter{theorem}{1}
\begin{definition}[Uniform continuity]\label{9.9.2}
    Let \(X\) be a subset of \(\mathbf{R}\), and let \(f : X \to \mathbf{R}\) be a function.
    We say that \(f\) is \emph{uniformly continuous} on \(X\) if, for every \(\varepsilon > 0\), there exists a \(\delta > 0\) such that \(f(x)\) and \(f(x_0)\) are \(\varepsilon\)-close whenever \(x, x_0 \in X\) are two points in \(X\) which are \(\delta\)-close.
\end{definition}

\begin{remark}\label{9.9.3}
    This definition should be compared with the notion of continuity.
    From Proposition \ref{9.4.7}(c), we know that a function \(f\) is \emph{continuous} if for every \(\varepsilon > 0\), and every \(x_0 \in X\), there is a \(\delta > 0\) such that \(f(x)\) and \(f(x_0)\) are \(\varepsilon\)-close whenever \(x \in X\) is \(\delta\)-close to \(x_0\).
    The difference between uniform continuity and continuity is that in uniform continuity one can take a single \(\delta\) which works for all \(x_0 \in X\);
    for ordinary continuity, each \(x_0 \in X\) might use a different \(\delta\).
    Thus every uniformly continuous function is continuous, but not conversely.
\end{remark}

\setcounter{theorem}{4}
\begin{definition}[Equivalent sequences]\label{9.9.5}
    Let \(m\) be an integer, let \((a_n)_{n = m}^\infty\) and \((b_n)_{n = m}^\infty\) be two sequences of real numbers, and let \(\varepsilon > 0\) be given.
    We say that \((a_n)_{n = m}^\infty\) is \emph{\(\varepsilon\)-close} to \((b_n)_{n = m}^\infty\) iff \(a_n\) is \(\varepsilon\)-close to \(b_n\) for each \(n \geq m\).
    We say that \((a_n)_{n = m}^\infty\) is \emph{eventually \(\varepsilon\)-close} to \((b_n)_{n = m}^\infty\) iff there exists an \(N \geq m\) such that the sequences \((a_n)_{n = N}^\infty\) and \((b_n)_{n = N}^\infty\) are \(\varepsilon\)-close.
    Two sequences \((a_n)_{n = m}^\infty\) and \((b_n)_{n = m}^\infty\) are \emph{equivalent} iff for each \(\varepsilon > 0\), the sequences \((a_n)_{n = m}^\infty\) and \((b_n)_{n = m}^\infty\) are eventually \(\varepsilon\)-close.
\end{definition}

\begin{remark}\label{9.9.6}
    One could debate whether \(\varepsilon\) should be assumed to be rational or real, but a minor modification of Proposition \ref{6.1.4} shows that this does not make any difference to the above definitions.
\end{remark}

\begin{lemma}\label{9.9.7}
    Let \((a_n)_{n = 1}^\infty\) and \((b_n)_{n = 1}^\infty\) be sequences of real numbers
    (not necessarily bounded or convergent).
    Then \((a_n)_{n = 1}^\infty\) and \((b_n)_{n = 1}^\infty\) are equivalent if and only if \(\lim_{n \to \infty} (a_n - b_n) = 0\).
\end{lemma}

\begin{proof}
    \begin{align*}
             & (a_n)_{n = 1}^\infty = (b_n)_{n = 1}^\infty                                                                                                                          \\
        \iff & \forall\ \varepsilon \in \mathbf{R}^+, \exists\ N \in \mathbf{Z}^+ : \forall\ n \geq N, \abs*{a_n - b_n} \leq \varepsilon       & \text{(by Definition \ref{9.9.5})} \\
        \iff & \forall\ \varepsilon \in \mathbf{R}^+, \exists\ N \in \mathbf{Z}^+ : \forall\ n \geq N, \abs*{(a_n - b_n) - 0} \leq \varepsilon                                      \\
        \iff & \lim_{n \to \infty} (a_n - b_n) = 0.                                                                                            & \text{(by Definition \ref{6.1.5})}
    \end{align*}
\end{proof}

\begin{proposition}\label{9.9.8}
    Let \(X\) be a subset of \(\mathbf{R}\), and let \(f : X \to \mathbf{R}\) be a function.
    Then the following two statements are logically equivalent:
    \begin{enumerate}
        \item \(f\) is uniformly continuous on \(X\).
        \item Whenever \((x_n)_{n = 0}^\infty\) and \((y_n)_{n = 0}^\infty\) are two equivalent sequences consisting of elements of \(X\), the sequences \((f(x_n))_{n = 0}^\infty\) and \((f(y_n))_{n = 0}^\infty\) are also equivalent.
    \end{enumerate}
\end{proposition}

\begin{proof}
    We first show that the first statement implies the second statement.
    By Definition \ref{9.9.2}, \(f\) is uniformly continuous on \(X\) iff
    \[
        \forall\ \varepsilon \in \mathbf{R}^+, \exists\ \delta \in \mathbf{R}^+ : \bigg(\forall\ x, y \in X, \abs*{x - y} < \delta \implies \abs*{f(x) - f(y)} \leq \varepsilon\bigg).
    \]
    By Definition \ref{9.9.5}, \((x_n)_{n = 0}^\infty = (y_n)_{n = 0}^\infty\) iff
    \[
        \forall\ \varepsilon \in \mathbf{R}^+, \exists\ N \in \mathbf{N} : \bigg(\forall\ n \in \mathbf{N} \land n \geq N, \abs*{x_n - y_n} \leq \varepsilon\bigg).
    \]
    In particular, we have
    \[
        \exists\ N \in \mathbf{N} : \bigg(\forall\ n \in \mathbf{N} \land n \geq N, \abs*{x_n - y_n} \leq \delta / 2 < \delta\bigg).
    \]
    Since \(\forall\ n \geq 0\) we have \(x_n, y_n \in X\), by replacing \(x, y\) with \(x_n, y_n\) we see that
    \[
        \forall\ \varepsilon \in \mathbf{R}^+, \exists\ \delta \in \mathbf{R}^+ : \bigg(\forall\ x_n, y_n \in X, \abs*{x_n - y_n} < \delta \implies \abs*{f(x_n) - f(y_n)} \leq \varepsilon\bigg).
    \]
    Thus we have
    \[
        \forall\ \varepsilon \in \mathbf{R}^+, \exists\ N \in \mathbf{N} : \bigg(\forall\ n \in \mathbf{N} \land n \geq N, \abs*{f(x_n) - f(y_n)} \leq \varepsilon\bigg)
    \]
    and by Definition \ref{9.9.5} \((f(x_n))_{n = 0}^\infty = (f(y_n))_{n = 0}^\infty\).

    Now we show that the second statement implies the first statement.
    We have \(\forall\ n \in \mathbf{N}\), \(x_n, y_n \in X\) and \((x_n)_{n = 0}^\infty = (y_n)_{n = 0}^\infty \implies (f(x_n))_{n = 0}^\infty = (f(y_n))_{n = 0}^\infty\).
    Suppose for sake of contradiction that \(f\) is not uniformly continuous on \(X\).
    Then we have
    \[
        \exists\ \varepsilon \in \mathbf{R}^+ : \forall\ \delta \in \mathbf{R}^+, \bigg(\forall\ x, y \in X, \abs*{x - y} < \delta \land \abs*{f(x) - f(y)} > \varepsilon\bigg).
    \]
    By replacing \(x, y\) with \(x_n, y_n\) we have
    \[
        \exists\ \varepsilon \in \mathbf{R}^+ : \forall\ \delta \in \mathbf{R}^+, \bigg(\forall\ x_n, y_n \in X, \abs*{x_n - y_n} < \delta \land \abs*{f(x_n) - f(y_n)} > \varepsilon\bigg).
    \]
    Since \((x_n)_{n = 0}^\infty = (y_n)_{n = 0}^\infty\), we have
    \[
        \exists\ N_1 \in \mathbf{N} : \bigg(\forall\ n \in \mathbf{N} \land n \geq N_1, \abs*{x_n - y_n} \leq \delta / 2 < \delta\bigg).
    \]
    Since \((f(x_n))_{n = 0}^\infty = (f(y_n))_{n = 0}^\infty\), we have
    \[
        \exists\ N_2 \in \mathbf{N} : \bigg(\forall\ n \in \mathbf{N} \land n \geq N_2, \abs*{f(x_n) - f(y_n)} \leq \varepsilon\bigg).
    \]
    Let \(N = \max(N_1, N_2)\).
    Then we have
    \[
        \forall\ n \in \mathbf{N} \land n \geq N, \abs*{x_n - y_n} < \delta \land \abs*{f(x_n) - f(y_n)} \leq \varepsilon,
    \]
    a contradiction.
    Thus \(f\) is uniformly continuous on \(X\).
\end{proof}

\begin{remark}\label{9.9.9}
    The reader should compare Proposition \ref{9.9.8} with Proposition \ref{9.3.9}.
    Proposition \ref{9.3.9} asserted that if \(f\) was continuous, then \(f\) maps convergent sequences to convergent sequences.
    In contrast, Proposition \ref{9.9.8} asserts that if \(f\) is \emph{uniformly} continuous, then \(f\) maps \emph{equivalent} pairs of sequences to equivalent pairs of sequences.
    To see how the two Propositions are connected, observe from Lemma \ref{9.9.7} that \((x_n)_{n = 0}^\infty\) will converge to \(x_*\) if and only if the sequences \((x_n)_{n = 0}^\infty\) and \((x_*)_{n = 0}^\infty\) are equivalent.
\end{remark}

\setcounter{theorem}{11}
\begin{proposition}\label{9.9.12}
    Let \(X\) be a subset of \(\mathbf{R}\), and let \(f : X \to \mathbf{R}\) be a uniformly continuous function.
    Let \((x_n)_{n = 0}^\infty\) be a Cauchy sequence consisting entirely of elements in \(X\).
    Then \((f(x_n))_{n = 0}^\infty\) is also a Cauchy sequence.
\end{proposition}

\begin{proof}
    Since \(f\) is uniformly continuous, by Definition \ref{9.9.2} we have
    \[
        \forall\ \varepsilon \in \mathbf{R}^+, \exists\ \delta \in \mathbf{R}^+ : \bigg(\forall\ x, y \in X, \abs*{x - y} < \delta \implies \abs*{f(x) - f(y)} \leq \varepsilon\bigg).
    \]
    Since \((x_n)_{n = 0}^\infty\) is a Cauchy sequence, by Definition \ref{6.1.3} we have
    \[
        \forall\ \varepsilon \in \mathbf{R}^+, \exists\ N \in \mathbf{N} : \bigg(\forall\ i, j \in \mathbf{N} \land i, j \geq N, \abs*{x_i - x_j} \leq \varepsilon\bigg).
    \]
    In particular, we have
    \[
        \exists\ N \in \mathbf{N} : \bigg(\forall\ i, j \in \mathbf{N} \land i, j \geq N, \abs*{x_i - x_j} \leq \delta / 2 < \delta\bigg).
    \]
    Since \(x_i, x_j \in X\), we have
    \[
        \abs*{x_i - x_j} < \delta \implies \abs*{f(x_i) - f(x_j)} \leq \varepsilon.
    \]
    Thus we have
    \[
        \forall\ \varepsilon \in \mathbf{R}^+, \exists\ N \in \mathbf{N} : \bigg(\forall\ i, j \in \mathbf{N} \land i, j \geq N, \abs*{f(x_i) - f(x_j)} \leq \varepsilon\bigg).
    \]
    and by Definition \ref{6.1.3} \((f(x_n))_{n = 0}^\infty\) is a Cauchy sequence.
\end{proof}

\setcounter{theorem}{13}
\begin{corollary}\label{9.9.14}
    Let \(X\) be a subset of \(\mathbf{R}\), let \(f : X \to \mathbf{R}\) be a uniformly continuous function, and let \(x_0\) be an adherent point of \(X\).
    Then the limit \(\lim_{x \to x_0 ; x \in X} f(x)\) exists
    (in particular, it is a real number ).
\end{corollary}

\begin{proof}
    Since \(x_0\) is an adherent point of \(X\), by Lemma \ref{9.1.14} there exists a sequence \((a_n)_{n = 0}^\infty\), consisting entirely of elements in \(X\), which converges to \(x_0\).
    Since \(\lim_{n \to \infty} a_n = x_0\), by Proposition \ref{6.1.12} \((a_n)_{n = 0}^\infty\) is a Cauchy sequence.
    Since \(f\) is uniformly continuous, by Proposition \ref{9.9.12} \((f(a_n))_{n = 0}^\infty\) is also a Cauchy sequence.
    Let \(L = \lim_{n \to \infty} f(a_n)\).
    By Proposition \ref{9.9.8}, we know that if \((a_n)_{n = 0}^\infty\) and \((b_n)_{n = 0}^\infty\) are equivalent, then \((f(a_n))_{n = 0}^\infty\) and \((f(b_n))_{n = 0}^\infty\) are equivalent.
    By Proposition \ref{6.1.7}, we know that \(\lim_{n \to \infty} f(b_n) = L\).
    Since \((b_n)_{n = 0}^\infty\) is arbitrary, by Proposition \ref{9.3.9} we know that \(f\) converges to \(L\) at \(x_0\) in \(X\), and thus \(\lim_{x \to x_0 ; x \in X} f(x) = L\) exists.

    Now we show that \(f : (0, 2) \to \mathbf{R}\) defined by \(f(x) = 1 / x\) is not uniformly continuous.
    Suppose for sake of contradiction that \(f\) is uniformly continuous.
    Since \(0\) is an adherent point of \((0, 2)\), we know that \(\lim_{x \to 0 ; x \in (0, 2)} f(x)\) exists.
    Since \(f\) is continuous, \(f(0) = 1 / 0\) must exist, a contradiction.
    Thus \(f\) is not uniformly continuous.
\end{proof}

\begin{proposition}\label{9.9.15}
    Let \(X\) be a subset of \(\mathbf{R}\), and let \(f : X \to \mathbf{R}\) be a uniformly continuous function.
    Suppose that \(E\) is a bounded subset of \(X\).
    Then \(f(E)\) is also bounded.
\end{proposition}

\begin{proof}
    Suppose for sake of contradiction that \(f(E)\) is not bounded.
    Thus for every real number \(M\) there exists an element \(x \in E\) such that \(\abs*{f(x)} \geq M\).

    In particular, for every natural number \(n\), the set \(\{x \in E : \abs*{f(x)} \geq n\}\) is non-empty.
    We can thus choose a sequence \((x_n)_{n = 0}^\infty\) in \(E\) such that \(\abs*{f(x_n)} \geq n\) for all \(n\).
    Since \((x_n)_{n = 0}^\infty\) in \(E\), by Bolzano-Weierstrass theorem (Theorem \ref{6.6.8}) there exists a subsequence \((x_{n_j})_{j = 0}^\infty\) which converges, where \(n_0 < n_1 < n_2 < \dots\) is an increasing sequence of natural numbers.
    In particular, we see that \(n_j \geq j\) for all \(j \in \mathbf{N}\) (use induction).

    Let \(\lim_{n \to \infty} x_n = x_*\).
    Since \((x_n)_{n = 0}^\infty\) in \(E\), by Lemma \ref{9.1.14} we know that \(x_*\) is an adherent point of \(E\).
    Since \(f\) is continuous on \(X\), by Exercise \ref{ex 9.4.6} \(f\) is continuous on \(E\).
    In particular, \(f\) is uniformly continuous on \(E\).
    Thus by Corollary \ref{9.9.14} we know that \(\lim_{x \to x_* ; x \in E} f(x)\) exists.
    By Proposition \ref{9.3.9} we see that
    \[
        \lim_{j \to \infty} f(x_{n_j}) = \lim_{x \to x_* ; x \in E} f(x).
    \]
    Thus the sequence \((f(x_{n_j}))_{j = 0}^\infty\) is convergent, and hence it is bounded.
    On the other hand, we know from the construction that \(\abs*{f(x_{n_j})} \geq n_j \geq j\) for all \(j\), and hence the sequence \((f(x_{n_j}))_{j = 0}^\infty\) is not bounded, a contradiction.
\end{proof}

\begin{theorem}\label{9.9.16}
    Let \(a < b\) be real numbers, and let \(f : [a, b] \to \mathbf{R}\) be a function which is continuous on \([a, b]\).
    Then \(f\) is also uniformly continuous.
\end{theorem}

\begin{proof}
    Suppose for sake of contradiction that \(f\) is not uniformly continuous.
    By Proposition \ref{9.9.8}, there must therefore exist two equivalent sequences \((x_n)_{n = 0}^\infty\) and \((y_n)_{n = 0}^\infty\) in \([a, b]\) such that the sequences \((f(x_n))_{n = 0}^\infty\) and \((f(y_n))_{n = 0}^\infty\) are not equivalent.
    In particular, we can find an \(\varepsilon > 0\) such that \((f(x_n))_{n = 0}^\infty\) and \((f(y_n))_{n = 0}^\infty\) are not eventually \(\varepsilon\)-close.

    Fix this value of \(\varepsilon\), and let \(E\) be the set
    \[
        E \coloneqq \{n \in \mathbf{N} : f(x_n) \text{ and } f(y_n) \text{ are not \(\varepsilon\)-close}\}.
    \]
    We must have \(E\) infinite, since if \(E\) were finite then \((f(x_n))_{n = 0}^\infty\) and \((f(y_n))_{n = 0}^\infty\) would be eventually \(\varepsilon\)-close.
    By Proposition \ref{8.1.5}, \(E\) is countable;
    in fact from the proof of that proposition we see that we can find an infinite sequence
    \[
        n_0 < n_1 < n_2 < \dots
    \]
    consisting entirely of elements in \(E\).
    In particular, we have
    \[
        \abs*{f(x_{n_j}) - f(y_{n_j})} > \varepsilon \text{ for all } j \in \mathbf{N}. \tag{9.3}\label{eq 9.3}
    \]
    On the other hand, the sequence \((x_{n_j})_{j = 0}^\infty\) is a sequence in \([a, b]\), and so by the Heine-Borel theorem (Theorem \ref{9.1.24}) there must be a subsequence \((x_{n_{j_k}})_{k = 0}^\infty\) which converges to some limit \(L\) in \([a, b]\).
    In particular, \(f\) is continuous at \(L\), and so by Proposition \ref{9.4.7},
    \[
        \lim_{k \to \infty} f(x_{n_{j_k}}) = f(L). \tag{9.4}\label{eq 9.4}
    \]
    Note that \((x_{n_{j_k}})_{k = 0}^\infty\) is a subsequence of \((x_n)_{n = 0}^\infty\), and \((y_{n_{j_k}})_{k = 0}^\infty\) is a subsequence of \((y_n)_{n = 0}^\infty\), by Lemma \ref{6.6.4}.
    On the other hand, from Lemma \ref{9.9.7} we have
    \[
        \lim_{n \to \infty} (x_n - y_n) = 0.
    \]
    By Proposition \ref{6.6.5}, we thus have
    \[
        \lim_{k \to \infty} (x_{n_{j_k}} - y_{n_{j_k}}) = 0.
    \]
    Since \(x_{n_{j_k}}\) converges to \(L\) as \(k \to \infty\), we thus have by limit laws
    \[
        \lim_{k \to \infty} y_{n_{j_k}} = L.
    \]
    and hence by continuity of \(f\) at \(L\)
    \[
        \lim_{k \to \infty} f(y_{n_{j_k}}) = f(L).
    \]
    Subtracting this from (9.4) using limit laws, we obtain
    \[
        \lim_{k \to \infty} (f(x_{n_{j_k}}) - f(y_{n_{j_k}})) = 0.
    \]
    But this contradicts (9.3) (by Lemma \ref{9.9.7}).
    From this contradiction we conclude that \(f\) is in fact uniformly continuous.
\end{proof}

\begin{remark}\label{9.9.17}
    One should compare Lemma \ref{9.6.3}, Proposition \ref{9.9.15}, and Theorem \ref{9.9.16} with each other.
    Note in particular that Lemma \ref{9.6.3} follows from combining Proposition \ref{9.9.15} and Theorem \ref{9.9.16}.
\end{remark}

\exercisesection

\begin{exercise}\label{ex 9.9.1}
    Prove Lemma \ref{9.9.7}.
\end{exercise}

\begin{proof}
    See Lemma \ref{9.9.7}.
\end{proof}

\begin{exercise}\label{ex 9.9.2}
    Prove Proposition \ref{9.9.8}.
\end{exercise}

\begin{proof}
    See Proposition \ref{9.9.8}.
\end{proof}

\begin{exercise}\label{ex 9.9.3}
    Prove Proposition \ref{9.9.12}.
\end{exercise}

\begin{proof}
    See Proposition \ref{9.9.12}.
\end{proof}

\begin{exercise}\label{ex 9.9.4}
    Use Proposition \ref{9.9.12} to prove Corollary \ref{9.9.14}.
    Use this corollary to give an alternate demonstration of the results in Example 9.9.10.
\end{exercise}

\begin{proof}
    See Corollary \ref{9.9.14}.
\end{proof}

\begin{exercise}\label{ex 9.9.5}
    Prove Proposition \ref{9.9.15}.
\end{exercise}

\begin{proof}
    See Proposition \ref{9.9.15}.
\end{proof}

\begin{exercise}\label{ex 9.9.6}
    Let \(X, Y, Z\) be subsets of \(\mathbf{R}\).
    Let \(f : X \to Y\) be a function which is uniformly continuous on \(X\), and let \(g : Y \to Z\) be a function which is uniformly continuous on \(Y\).
    Show that the function \(g \circ f : X \to Z\) is uniformly continuous on \(X\).
\end{exercise}

\begin{proof}
    Since \(f\) is continuous on \(X\) and \(g\) is continuous on \(Y\), by Proposition \ref{9.4.13} we know that \(g \circ f\) is continuous on \(X\).
    Since \(g\) is uniformly continuous, by Definition \ref{9.9.2} we have
    \[
        \forall\ \varepsilon \in \mathbf{R}^+, \exists\ \delta' \in \mathbf{R}^+ : \bigg(\forall\ y_1, y_2 \in Y, \abs*{y_1 - y_2} < \delta' \implies \abs*{g(y_1) - g(y_2)} \leq \varepsilon\bigg).
    \]
    Similarly, since \(f\) is uniformly continuous, by Definition \ref{9.9.2} we have
    \[
        \forall\ \varepsilon' \in \mathbf{R}^+, \exists\ \delta \in \mathbf{R}^+ : \bigg(\forall\ x_1, x_2 \in X, \abs*{x_1 - x_2} < \delta \implies \abs*{f(x_1) - f(x_2)} \leq \varepsilon'\bigg).
    \]
    In particular, we have
    \[
        \exists\ \delta \in \mathbf{R}^+ : \bigg(\forall\ x_1, x_2 \in X, \abs*{x_1 - x_2} < \delta \implies \abs*{f(x_1) - f(x_2)} \leq \delta' / 2 < \delta'\bigg).
    \]
    Since \(f(x_1), f(x_2) \in Y\), we have
    \[
        \abs*{f(x_1) - f(x_2)} < \delta' \implies \abs*{g(f(x_1)) - g(f(x_2))} \leq \varepsilon.
    \]
    Thus we have showed that
    \[
        \forall\ \varepsilon \in \mathbf{R}^+, \exists\ \delta \in \mathbf{R}^+ : \bigg(\forall\ x_1, x_2 \in X, \abs*{x_1 - x_2} < \delta \implies \abs*{g(f(x_1)) - g(f(x_2))} \leq \varepsilon\bigg).
    \]
    and by Definition \ref{9.9.2} \(g \circ f\) is uniformly continuous on \(X\).
\end{proof}