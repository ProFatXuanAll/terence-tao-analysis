\section{Uniform continuity}\label{sec 9.9}

\setcounter{theorem}{1}
\begin{definition}[Uniform continuity]\label{9.9.2}
    Let \(X\) be a subset of \(\mathbf{R}\), and let \(f : X \to \mathbf{R}\) be a function.
    We say that \(f\) is \emph{uniformly continuous} if, for every \(\varepsilon > 0\), there exists a \(\delta > 0\) such that \(f(x)\) and \(f(x_0)\) are \(\varepsilon\)-close whenever \(x, x_0 \in X\) are two points in \(X\) which are \(\delta\)-close.
\end{definition}

\begin{remark}\label{9.9.3}
    This definition should be compared with the notion of continuity.
    From Proposition \ref{9.4.7}(c), we know that a function \(f\) is \emph{continuous} if for every \(\varepsilon > 0\), and every \(x_0 \in X\), there is a \(\delta > 0\) such that \(f(x)\) and \(f(x_0)\) are \(\varepsilon\)-close whenever \(x \in X\) is \(\delta\)-close to \(x_0\).
    The difference between uniform continuity and continuity is that in uniform continuity one can take a single \(\delta\) which works for all \(x_0 \in X\);
    for ordinary continuity, each \(x_0 \in X\) might use a different \(\delta\).
    Thus every uniformly continuous function is continuous, but not conversely.
\end{remark}
