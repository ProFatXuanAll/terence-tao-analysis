\chapter{The Riemann integral}\label{ch 11}

\begin{note}
    In Chapter \ref{ch 10} we reviewed \emph{differentiation} - one of the two pillars of single variable calculus.
    The other pillar is, of course, \emph{integration}, which is the focus of the current chapter.
    More precisely, we will turn to the \emph{definite integral}, the integral of a function on a fixed interval, as opposed to the \emph{indefinite integral}, otherwise known as the \emph{antiderivative}.
    These two are of course linked by the \emph{Fundamental theorem of calculus}.
\end{note}

\begin{note}
    To actually \emph{define} this integral \(\int_I f\) is somewhat delicate (especially if one does not want to assume any axioms concerning geometric notions such as area), and not all functions \(f\) are integrable.
    It turns out that there are at least two ways to define this integral:
    the \emph{Riemann integral}, named after Georg Riemann (1826 -- 1866), which suffices for most applications, and the \emph{Lebesgue integral}, named after Henri Lebesgue (1875 -- 1941), which supercedes the Riemann integral and works for a much larger class of functions.
    There is also the \emph{Riemann-Steiltjes integral} \(\int_I f(x) d \alpha(x)\), a generalization of the Riemann integral due to Thomas Stieltjes (1856 -- 1894).
\end{note}

\section{Partitions}\label{sec 11.1}

\begin{definition}\label{11.1.1}
    Let \(X\) be a subset of \(\mathbf{R}\).
    We say that \(X\) is \emph{connected} iff the following property is true:
    whenever \(x, y\) are elements in \(X\) such that \(x < y\), the bounded interval \([x, y]\) is a subset of \(X\)
    (i.e., every number between \(x\) and \(y\) is also in \(X\)).
\end{definition}
\section{Piecewise constant functions}\label{sec 11.2}

\begin{definition}[Constant functions]\label{11.2.1}
    Let \(X\) be a subset of \(\mathbf{R}\), and let \(f : X \to \mathbf{R}\) be a function.
    We say that \(f\) is \emph{constant} iff there exists a real number \(c\) such that \(f(x) = c\) for all \(x \in X\).
    If \(E\) is a subset of \(X\), we say that \(f\) is \emph{constant on} \(E\) if the restriction \(f|_E\) of \(f\) to \(E\) is constant, in other words there exists a real number \(c\) such that \(f(x) = c\) for all \(x \in E\).
    We refer to \(c\) as the \emph{constant value} of \(f\) on \(E\).
\end{definition}
\section{Upper and lower Riemann integrals}\label{sec:11.3}

\begin{defn}[Majorization of functions]\label{11.3.1}
  Let \(f : I \to \R\) and \(g : I \to \R\).
  We say that \(g\) \emph{majorizes} \(f\) on \(I\) if we have \(g(x) \geq f(x)\) for all \(x \in I\), and that \(g\) \emph{minorizes} \(f\) on \(I\) if \(g(x) \leq f(x)\) for all \(x \in I\).
\end{defn}

\begin{defn}[Upper and lower Riemann integrals]\label{11.3.2}
  Let \(f : I \to \R\) be a bounded function defined on a bounded interval \(I\).
  We define the \emph{upper Riemann integral} \(\overline{\int}_I f\) by the formula
  \[
    \overline{\int}_I f \coloneqq \inf\bigg\{p.c. \int_I g : g \text{ is a piecewise constant function on \(I\) which majorizes } f\bigg\}
  \]
  and the \emph{lower Riemann integral} \(\underline{\int}_I f\) by the formula
  \[
    \underline{\int}_I f \coloneqq \sup\bigg\{p.c. \int_I g : g \text{ is a piecewise constant function on \(I\) which minorizes } f\bigg\}.
  \]
\end{defn}

\begin{lem}\label{11.3.3}
  Let \(f : I \to \R\) be a function on a bounded interval \(I\) which is bounded by some real number \(M\), i.e., \(-M \leq f(x) \leq M\) for all \(x \in I\).
  Then we have
  \[
    -M \abs{I} \leq \underline{\int}_I f \leq \overline{\int}_I f \leq M \abs{I}.
  \]
  in particular, both the lower and upper Riemann integrals are real numbers (i.e., they are not infinite).
\end{lem}

\begin{proof}
  The function \(g : I \to \R\) defined by \(g(x) = M\) is constant, hence piecewise constant, and majorizes \(f\);
  thus \(\overline{\int}_I f \leq p.c. \int_I g = M \abs{I}\) by definition of the upper Riemann integral.
  A similar argument gives \(-M \abs{I} \leq \underline{\int}_I f\).
  Finally, we have to show that \(\underline{\int}_I f \leq \overline{\int}_I f\).
  Let \(g\) be any piecewise constant function majorizing \(f\), and let \(h\) be any piecewise constant function minorizing \(f\).
  Then \(g\) majorizes \(h\), and hence \(p.c. \int_I h \leq p.c. \int_I g\).
  Taking suprema in \(h\), we obtain that \(\underline{\int}_I f \leq p.c. \int_I g\).
  Taking infima in \(g\), we thus obtain \(\underline{\int}_I f \leq \overline{\int}_I f\), as desired.
\end{proof}

\begin{defn}[Riemann integral]\label{11.3.4}
  Let \(f : I \to \R\) be a bounded function on a bounded interval \(I\).
  If \(\underline{\int}_I f = \overline{\int}_I f\), then we say that \(f\) is \emph{Riemann integrable on \(I\)} and define
  \[
    \int_I f \coloneqq \underline{\int}_I f = \overline{\int}_I f.
  \]
  If the upper and lower Riemann integrals are unequal, we say that \(f\) is not Riemann integrable.
\end{defn}

\begin{rmk}\label{11.3.5}
  Compare this definition to the relationship between the \(\limsup\), \(liminf\), and limit of a sequence \(a_n\) that was established in \cref{6.4.12}(f);
  the \(\limsup\) is always greater than or equal to the \(\liminf\), but they are only equal when the sequence converges, and in this case they are both equal to the limit of the sequence.
  The definition given above may differ from the definition you may have encountered in your calculus courses, based on Riemann sums.
  However, the two definitions turn out to be equivalent.
\end{rmk}

\begin{rmk}\label{11.3.6}
  Note that we do not consider unbounded functions to be Riemann integrable;
  an integral involving such functions is known as an \emph{improper integral}.
  It is possible to still evaluate such integrals using more sophisticated integration methods (such as the Lebesgue integral).
\end{rmk}

\begin{lem}\label{11.3.7}
  Let \(f : I \to \R\) be a piecewise constant function on a bounded interval \(I\).
  Then \(f\) is Riemann integrable, and \(\int_I f = p.c. \int_I f\).
\end{lem}

\begin{proof}
  Since \(f(x) \leq f(x)\) for every \(x \in I\), by \cref{11.3.2} we have
  \[
    \overline{\int}_I f \leq p.c. \int_I f
  \]
  and
  \[
    p.c. \int_I f \leq \underline{\int}_I f.
  \]
  By \cref{11.3.3} we know that
  \[
    p.c. \int_I f \leq \underline{\int}_I f \leq \overline{\int}_I f \leq p.c. \int_I f.
  \]
  Thus by \cref{11.3.4} we have
  \[
    \int_I f = \underline{\int}_I f = \overline{\int}_I f = p.c. \int_I f.
  \]
\end{proof}

\begin{rmk}\label{11.3.8}
  Because of \cref{11.3.7}, we will not refer to the piecewise constant integral \(p.c. \int_I\) again, and just use the Riemann integral \(\int_I\) throughout
  (until this integral is itself superceded by the Lebesgue integral).
  We observe one special case of \cref{11.3.7}:
  if \(I\) is a point or the empty set, then \(\int_I f = 0\) for all functions \(f : I \to \R\).
  (Note that all such functions are automatically constant.)
\end{rmk}

\begin{defn}[Riemann sums]\label{11.3.9}
  Let \(f : I \to \R\) be a bounded function on a bounded interval \(I\), and let \(\mathbf{P}\) be a partition of \(I\).
  We define the \emph{upper Riemann sum} \(U(f, \mathbf{P})\) and the \emph{lower Riemann sum} \(L(f, \mathbf{P})\) by
  \[
    U(f, \mathbf{P}) \coloneqq \sum_{J \in \mathbf{P} : J \neq \emptyset} \big(\sup_{x \in J} f(x)\big) \abs{J}
  \]
  and
  \[
    L(f, \mathbf{P}) \coloneqq \sum_{J \in \mathbf{P} : J \neq \emptyset} \big(\inf_{x \in J} f(x)\big) \abs{J}.
  \]
\end{defn}

\begin{rmk}\label{11.3.10}
  The restriction \(J \neq \emptyset\) is required because the quantities \(\inf_{x \in J} f(x)\) and \(\sup_{x \in J} f(x)\) are infinite (or negative infinite) if \(J\) is empty.
\end{rmk}

\begin{lem}\label{11.3.11}
  Let \(f : I \to \R\) be a bounded function on a bounded interval \(I\), and let \(g\) be a function which majorizes \(f\) and which is piecewise constant with respect to some partition \(\mathbf{P}\) of \(I\).
  Then
  \[
    p.c. \int_I g \geq U(f, \mathbf{P}).
  \]
  Similarly, if \(h\) is a function which minorizes \(f\) and is piecewise constant with respect to \(\mathbf{P}\), then
  \[
    p.c. \int_I h \leq L(f, \mathbf{P}).
  \]
\end{lem}

\begin{proof}
  Since \(g\) majorizes \(f\) and \(h\) minorizes \(f\), by \cref{11.3.1} we have \(h(x) \leq f(x) \leq g(x)\) for every \(x \in I\).
  Since \(\mathbf{P}\) is a partition of \(I\), by \cref{11.1.10} for every \(J \in \mathbf{P}\), we have \(h(x) \leq f(x) \leq g(x)\) for all \(x \in J\).
  In particular, when \(J \neq \emptyset\) we have
  \[
    h(x) \leq \inf_{x \in J} f(x) \leq f(x) \leq \sup_{x \in J} f(x) \leq g(x)
  \]
  for every \(x \in J\).
  Let \(c_{g|_J}, c_{h|_J}\) be constant values of \(g|_J, h|_J\), respectively.
  Then we have
  \begin{align*}
    U(f, \mathbf{P}) & = \sum_{J \in \mathbf{P} : J \neq \emptyset} \big(\sup_{x \in J} f(x)\big) \abs{J} & \text{(by \cref{11.3.9})}       \\
                     & \leq \sum_{J \in \mathbf{P} : J \neq \emptyset} c_{g|_J} \abs{J}                   & \text{(by \cref{7.1.11}(h))}    \\
                     & = \sum_{J \in \mathbf{P}} c_{g|_J} \abs{J}                                         & \text{(by \cref{7.1.11}(a)(e))} \\
                     & = p.c. \int_{[\mathbf{P}]} g                                                       & \text{(by \cref{11.2.9})}       \\
                     & = p.c. \int_I g                                                                    & \text{(by \cref{11.2.14})}
  \end{align*}
  and
  \begin{align*}
    L(f, \mathbf{P}) & = \sum_{J \in \mathbf{P} : J \neq \emptyset} \big(\inf_{x \in J} f(x)\big) \abs{J} & \text{(by \cref{11.3.9})}       \\
                     & \geq \sum_{J \in \mathbf{P} : J \neq \emptyset} c_{h|_J} \abs{J}                   & \text{(by \cref{7.1.11}(h))}    \\
                     & = \sum_{J \in \mathbf{P}} c_{h|_J} \abs{J}                                         & \text{(by \cref{7.1.11}(a)(e))} \\
                     & = p.c. \int_{[\mathbf{P}]} h                                                       & \text{(by \cref{11.2.9})}       \\
                     & = p.c. \int_I h.                                                                   & \text{(by \cref{11.2.14})}
  \end{align*}
\end{proof}

\begin{prop}\label{11.3.12}
  Let \(f : I \to \R\) be a bounded function on a bounded interval \(I\).
  Then
  \[
    \overline{\int}_I f = \inf\{U(f, \mathbf{P}) : \mathbf{P} \text{ is a partition of } I\}
  \]
  and
  \[
    \underline{\int}_I f = \sup\{L(f, \mathbf{P}) : \mathbf{P} \text{ is a partition of } I\}.
  \]
\end{prop}

\begin{proof}
  Let \(g\) be a function which majorizes \(f\) and which is piecewise constant with respect to some partition \(\mathbf{P}_g\) of \(I\).
  Let \(h\) be a function which minorizes \(f\) and which is piecewise constant with respect to some partition \(\mathbf{P}_h\) of \(I\).
  Both functions are well defined since \(f\) is bounded function on a bounded interval \(I\).
  By \cref{11.3.11} we have
  \[
    \inf\big\{U(f, \mathbf{P}) : \mathbf{P} \text{ is a partition of } I\big\} \leq U(f, \mathbf{P}_g) \leq p.c. \int_I g
  \]
  and
  \[
    \sup\big\{L(f, \mathbf{P}) : \mathbf{P} \text{ is a partition of } I\big\} \geq L(f, \mathbf{P}_h) \geq p.c. \int_I h.
  \]
  Since \(g, h\) are arbitrary, by \cref{11.3.2} we have
  \[
    \inf\big\{U(f, \mathbf{P}) : \mathbf{P} \text{ is a partition of } I\big\} \leq \overline{\int}_I f
  \]
  and
  \[
    \sup\big\{L(f, \mathbf{P}) : \mathbf{P} \text{ is a partition of } I\big\} \geq \underline{\int}_I f.
  \]

  Let \(\mathbf{P}\) be a partition of \(I\).
  Let \(G : I \to \R\) be a function where \(G(x) = \sup_{x \in J} f(x)\) for all \(J \in \mathbf{P}\).
  Let \(H : I \to \R\) be a function where \(H(x) = \inf_{x \in J} f(x)\) for all \(J \in \mathbf{P}\).
  By \cref{11.2.3} we know that \(G, H\) are piecewise constant functions with respect to \(\mathbf{P}\).
  Thus we have
  \begin{align*}
    U(f, \mathbf{P}) & = \sum_{J \in \mathbf{P} : J \neq \emptyset} \big(\sup_{x \in J} f(x)\big) \abs{J} & \text{(by \cref{11.3.9})}       \\
                     & = \sum_{J \in \mathbf{P}} \big(\sup_{x \in J} f(x)\big) \abs{J}                    & \text{(by \cref{7.1.11}(a)(e))} \\
                     & = p.c. \int_{[\mathbf{P}]} G                                                       & \text{(by \cref{11.2.9})}       \\
                     & = p.c. \int_I G                                                                    & \text{(by \cref{11.2.14})}
  \end{align*}
  and
  \begin{align*}
    L(f, \mathbf{P}) & = \sum_{J \in \mathbf{P} : J \neq \emptyset} \big(\inf_{x \in J} f(x)\big) \abs{J} & \text{(by \cref{11.3.9})}       \\
                     & = \sum_{J \in \mathbf{P}} \big(\inf_{x \in J} f(x)\big) \abs{J}                    & \text{(by \cref{7.1.11}(a)(e))} \\
                     & = p.c. \int_{[\mathbf{P}]} H                                                       & \text{(by \cref{11.2.9})}       \\
                     & = p.c. \int_I H.                                                                   & \text{(by \cref{11.2.14})}
  \end{align*}
  By \cref{11.3.2} we have
  \[
    \overline{\int}_I f \leq p.c. \int_I G = U(f, \mathbf{P})
  \]
  and
  \[
    \underline{\int}_I f \geq p.c. \int_I H = L(f, \mathbf{P}).
  \]
  Since \(\mathbf{P}\) is arbitrary, we have
  \[
    \overline{\int}_I f \leq \inf\big\{U(f, \mathbf{P}) : \mathbf{P} \text{ is a partition of } I\big\} \leq U(f, \mathbf{P})
  \]
  and
  \[
    \underline{\int}_I f \geq \sup\big\{L(f, \mathbf{P}) : \mathbf{P} \text{ is a partition of } I\big\} \leq L(f, \mathbf{P}).
  \]
  Combine all results above we have
  \[
    \overline{\int}_I f = \inf\big\{U(f, \mathbf{P}) : \mathbf{P} \text{ is a partition of } I\big\}
  \]
  and
  \[
    \underline{\int}_I f = \sup\big\{L(f, \mathbf{P}) : \mathbf{P} \text{ is a partition of } I\big\}.
  \]
\end{proof}

\exercisesection

\begin{ex}\label{ex:11.3.1}
  Let \(f : I \to \R\), \(g : I \to \R\), and \(h : I \to \R\) be functions.
  Show that if \(f\) majorizes \(g\) and \(g\) majorizes \(h\), then \(f\) majorizes \(h\).
  Show that if \(f\) and \(g\) majorize each other, then they must be equal.
\end{ex}

\begin{proof}
  We first show that if \(f\) majorizes \(g\) and \(g\) majorizes \(h\), then \(f\) majorizes \(h\).
  Since
  \begin{align*}
             & \forall x \in I, f(x) \geq g(x) \geq h(x) & \text{(by \cref{11.3.1})} \\
    \implies & f(x) \geq h(x),
  \end{align*}
  by \cref{11.3.1} we know that \(f\) majorize \(h\).

  Now we show that if \(f\) and \(g\) majorize each other, then they must be equal.
  Since
  \begin{align*}
             & \forall x \in I, \big(f(x) \geq g(x)\big) \land \big(g(x) \geq f(x)\big) & \text{(by \cref{11.3.1})} \\
    \implies & f(x) = g(x),
  \end{align*}
  by \cref{3.3.7} we know that \(f = g\).
\end{proof}

\begin{ex}\label{ex:11.3.2}
  Let \(f : I \to \R\), \(g : I \to \R\), and \(h : I \to \R\) be functions.
  If \(f\) majorizes \(g\), is it true that \(f + h\) majorizes \(g + h\)?
  Is it true that \(f \cdot h\) majorizes \(g \cdot h\)?
  If \(c\) is a real number, is it true that \(cf\) majorizes \(cg\)?
\end{ex}

\begin{proof}
  We first show that if \(f\) majorizes \(g\), then \(f + h\) majorizes \(g + h\).
  Since
  \begin{align*}
             & \forall x \in I, f(x) \geq g(x) & \text{(by \cref{11.3.1})} \\
    \implies & f(x) + h(x) \geq g(x) + h(x)                                \\
    \implies & (f + h)(x) \geq (g + h)(x),     & \text{(by \cref{9.2.1})}
  \end{align*}
  by \cref{11.3.1} we know that \(f + h\) majorizes \(g + h\).

  Now we show that \(f \cdot h\) may not majorized \(g \cdot h\) and \(cf\) may not majorize \(cg\).
  Let \(c = h(x) = -1\).
  Then we have
  \begin{align*}
             & \forall x \in I, f(x) \geq g(x)                         & \text{(by \cref{11.3.1})} \\
    \implies & cf(x) = f(x) h(x) \leq cg(x) = g(x) h(x)                                            \\
    \implies & (cf)(x) = (f \cdot h)(x) \leq (cg)(x) = (g \cdot h)(x). & \text{(by \cref{9.2.1})}
  \end{align*}
  In this case \(f \cdot h\) does not majorized \(g \cdot h\) and \(cf\) does not majorized \(cg\).
\end{proof}

\begin{ex}\label{ex:11.3.3}
  Prove \cref{11.3.7}.
\end{ex}

\begin{proof}
  See \cref{11.3.7}.
\end{proof}

\begin{ex}\label{ex:11.3.4}
  Prove \cref{11.3.11}.
\end{ex}

\begin{proof}
  See \cref{11.3.11}.
\end{proof}

\begin{ex}\label{ex:11.3.5}
  Prove \cref{11.3.12}.
\end{ex}

\begin{prop}
  See \cref{11.3.12}.
\end{prop}
\section{Basic properties of the Riemann integral}\label{sec 11.4}

\begin{theorem}[Laws of Riemann integration]\label{11.4.1}
    Let \(I\) be a bounded interval, and let \(f : I \to \mathbf{R}\) and \(g : I \to \mathbf{R}\) be Riemann integrable functions on \(I\).
    \begin{enumerate}
        \item The function \(f + g\) is Riemann integrable, and we have \(\int_I f + \int_I g\).
        \item For any real number \(c\), the function \(cf\) is Riemann integrable, and we have \(\int_I (cf) = c(\int_I f)\).
        \item The function \(f - g\) is Riemann integrable, and we have \(\int_I (f - g) = \int_I f - \int_I g\).
        \item If \(f(x) \geq 0\) for all \(x \in I\), then \(\int_I f \geq 0\).
        \item If \(f(x) \geq g(x)\) for all \(x \in I\), then \(\int_I f \geq \int_I g\).
        \item If \(f\) is the constant function \(f(x) = c\) for all \(x \in I\), then \(\int_I f = c \abs*{I}\).
        \item Let \(J\) be a bounded interval containing \(I\) (i.e., \(I \subseteq J\)), and let \(F : J \to \mathbf{R}\) be the function
              \[
                  F(x) \coloneqq \begin{cases}
                      f(x) & \text{if } x \in I    \\
                      0    & \text{if } x \notin I \\
                  \end{cases}
              \]
              Then \(F\) is Riemann integrable on \(J\), and \(\int_J F = \int_I f\).
        \item Suppose that \(\{J, K\}\) is a partition of \(I\) into two intervals \(J\) and \(K\).
              Then the functions \(f|_J : J \to \mathbf{R}\) and \(f|_K : K \to \mathbf{R}\) are Riemann integrable on \(J\) and \(K\) respectively, and we have
              \[
                  \int_I f = \int_J f|_J + \int_K f|_K.
              \]
    \end{enumerate}
\end{theorem}

\begin{proof}{(a)}
    Since \(f, g\) are Riemann integrable, by Definition \ref{11.3.4} we have
    \[
        \int_I f = \overline{\int}_I f = \underline{\int}_I f
    \]
    and
    \[
        \int_I g = \overline{\int}_I g = \underline{\int}_I g.
    \]
    Let \(f_U : I \to \mathbf{R}\) and \(g_U : I \to \mathbf{R}\) be piecewise constant functions on \(I\) which majorizes \(f\) and \(g\) respectively.
    Let \(f_L : I \to \mathbf{R}\) and \(g_L : I \to \mathbf{R}\) be piecewise constant functions on \(I\) which minorizes \(f\) and \(g\) respectively.
    \(f_U, g_U, f_L, g_L\) are well-defined since by Definition \ref{11.3.4} \(f, g\) are bounded functions on a bounded interval \(I\).
    By Definition \ref{11.3.2} we have
    \[
        p.c. \int_I f_L \leq \underline{\int}_I f = \int_I f = \overline{\int}_I f \leq p.c. \int_I f_U
    \]
    and
    \[
        p.c. \int_I g_L \leq \underline{\int}_I g = \int_I g = \overline{\int}_I g \leq p.c. \int_I g_U.
    \]
    By Definition \ref{11.3.4} both \(f, g\) are bounded functions, so \(f + g\) is bounded function, and \(\underline{\int}_I (f + g), \overline{\int}_I (f + g)\) are well-defined (by Definition \ref{11.3.2}).
    By Exercise \ref{ex 11.3.2} we know that \(f_U + g_U\) majorizes \(f + g_U\) and \(f + g_U\) majorizes \(f + g\), thus \(f_U + g_U\) majorizes \(f + g\).
    Similarly \(f_L + g_L\) minorizes \(f + g\).
    Then we have
    \begin{align*}
                 & \overline{\int}_I (f + g) \leq p.c. \int_I (f + g)                       & \text{(by Definition \ref{11.3.2})}  \\
        \implies & \overline{\int}_I (f + g) \leq p.c. \int_I f_U + p.c. \int_I g_U         & \text{(by Theorem \ref{11.2.16}(a))} \\
        \implies & \overline{\int}_I (f + g) - p.c. \int_I g_U \leq p.c. \int_I f_U                                                \\
        \implies & \overline{\int}_I (f + g) - p.c. \int_I g_U \leq \overline{\int}_I f     & \text{(by Definition \ref{11.3.2})}  \\
        \implies & \overline{\int}_I (f + g) - \overline{\int}_I f \leq p.c. \int_I g_U                                            \\
        \implies & \overline{\int}_I (f + g) - \overline{\int}_I f \leq \overline{\int}_I g & \text{(by Definition \ref{11.3.2})}  \\
        \implies & \overline{\int}_I (f + g) \leq \overline{\int}_I f + \overline{\int}_I g &                                      \\
        \implies & \overline{\int}_I (f + g) \leq \int_I f + \int_I g                       & \text{(by Definition \ref{11.3.4})}
    \end{align*}
    and
    \begin{align*}
                 & \underline{\int}_I (f + g) \geq p.c. \int_I (f + g)                         & \text{(by Definition \ref{11.3.2})}  \\
        \implies & \underline{\int}_I (f + g) \geq p.c. \int_I f_L + p.c. \int_I g_L           & \text{(by Theorem \ref{11.2.16}(a))} \\
        \implies & \underline{\int}_I (f + g) - p.c. \int_I g_L \geq p.c. \int_I f_L                                                  \\
        \implies & \underline{\int}_I (f + g) - p.c. \int_I g_L \geq \underline{\int}_I f      & \text{(by Definition \ref{11.3.2})}  \\
        \implies & \underline{\int}_I (f + g) - \underline{\int}_I f \geq p.c. \int_I g_L                                             \\
        \implies & \underline{\int}_I (f + g) - \underline{\int}_I f \geq \underline{\int}_I g & \text{(by Definition \ref{11.3.2})}  \\
        \implies & \underline{\int}_I (f + g) \geq \underline{\int}_I f + \underline{\int}_I g &                                      \\
        \implies & \underline{\int}_I (f + g) \geq \int_I f + \int_I g.                        & \text{(by Definition \ref{11.3.4})}
    \end{align*}
    By Lemma \ref{11.3.3} we have
    \[
        \int_I f + \int_I g \leq \underline{\int}_I (f + g) \leq \overline{\int}_I (f + g) \leq \int_I f + \int_I g
    \]
    and thus by Definition \ref{11.3.4} we have
    \[
        \int_I (f + g) = \underline{\int}_I (f + g) = \overline{\int}_I (f + g) = \int_I f + \int_I g.
    \]
\end{proof}

\begin{proof}{(b)}
    Since \(f\) is Riemann integrable, by Definition \ref{11.3.4} we have
    \[
        \int_I f = \overline{\int}_I f = \underline{\int}_I f.
    \]
    First suppose that \(c = 0\).
    Then we have \((cf)(x) = 0\) for all \(x \in 0\), thus we have
    \begin{align*}
        \int_I (cf) & = p.c. \int_I (cf) & \text{(by Lemma \ref{11.3.7})} \\
                    & = 0                                                 \\
                    & = c \int_I f.
    \end{align*}

    Next suppose that \(c > 0\).
    Let \(f_U : I \to \mathbf{R}\) be a piecewise constant function on \(I\) which majorizes \(f\).
    Let \(f_L : I \to \mathbf{R}\) be a piecewise constant function on \(I\) which minorizes \(f\).
    \(f_U, f_L\) are well-defined since by Definition \ref{11.3.4} \(f\) is a bounded function on a bounded interval \(I\).
    Then by Definition \ref{11.3.2} we have
    \[
        p.c. \int_I f_L \leq \underline{\int}_I f = \int_I f = \overline{\int}_I f \leq p.c. \int_I f_U.
    \]
    Since \(f\) is a bounded function, \(cf\) is also a bounded function, by Definition \ref{11.3.2} both \(\overline{\int}_I (cf), \underline{\int}_I (cf)\) are well-defined.
    Since \(c > 0\), by Definition \ref{11.3.1} we know that \(c f_U\) majorizes \(c f\) and \(c f_L\) minorizes \(c f\).
    Then we have
    \begin{align*}
                 & \overline{\int}_I (cf) \leq p.c. \int_I (c f_U)                         & \text{(by Definition \ref{11.3.2})}  \\
        \implies & \overline{\int}_I (cf) \leq c \bigg(p.c. \int_I f_U\bigg)               & \text{(by Theorem \ref{11.2.16}(b))} \\
        \implies & \frac{1}{c} \bigg(\overline{\int}_I (cf)\bigg) \leq p.c. \int_I f_U                                            \\
        \implies & \frac{1}{c} \bigg(\overline{\int}_I (cf)\bigg) \leq \overline{\int}_I f & \text{(by Definition \ref{11.3.2})}  \\
        \implies & \overline{\int}_I (cf) \leq c\bigg(\overline{\int}_I f\bigg)                                                   \\
        \implies & \overline{\int}_I (cf) \leq c\bigg(\int_I f\bigg)                       & \text{(by Definition \ref{11.3.4})}
    \end{align*}
    and
    \begin{align*}
                 & \underline{\int}_I (cf) \geq p.c. \int_I (c f_L)                          & \text{(by Definition \ref{11.3.2})}  \\
        \implies & \underline{\int}_I (cf) \geq c \bigg(p.c. \int_I f_L\bigg)                & \text{(by Theorem \ref{11.2.16}(b))} \\
        \implies & \frac{1}{c} \bigg(\underline{\int}_I (cf)\bigg) \geq p.c. \int_I f_L                                             \\
        \implies & \frac{1}{c} \bigg(\underline{\int}_I (cf)\bigg) \geq \underline{\int}_I f & \text{(by Definition \ref{11.3.2})}  \\
        \implies & \underline{\int}_I (cf) \geq c\bigg(\underline{\int}_I f\bigg)                                                   \\
        \implies & \underline{\int}_I (cf) \geq c\bigg(\int_I f\bigg).                       & \text{(by Definition \ref{11.3.4})}
    \end{align*}
    By Lemma \ref{11.3.3} we have
    \[
        c\bigg(\int_I f\bigg) \leq \underline{\int}_I (cf) \leq \overline{\int}_I (cf) \leq c\bigg(\int_I f\bigg)
    \]
    and thus by Definition \ref{11.3.4} we have
    \[
        \int_I (cf) = \underline{\int}_I (cf) = \overline{\int}_I (cf) = c\bigg(\int_I f\bigg).
    \]

    Next suppose that \(c = -1\).
    Using the same definition of \(f_U, f_L\) we have
    \begin{align*}
                 & \overline{\int}_I (cf) \leq p.c. \int_I (c f_U)                         & \text{(by Definition \ref{11.3.2})}  \\
                 & \overline{\int}_I (cf) \leq p.c. \int_I (c f_U)                         & \text{(by Lemma \ref{11.3.3})}       \\
        \implies & \overline{\int}_I (cf) \leq c \bigg(p.c. \int_I f_U\bigg)               & \text{(by Theorem \ref{11.2.16}(b))} \\
        \implies & \frac{1}{c} \bigg(\overline{\int}_I (cf)\bigg) \geq p.c. \int_I f_U                                            \\
        \implies & \frac{1}{c} \bigg(\overline{\int}_I (cf)\bigg) \geq \overline{\int}_I f & \text{(by Definition \ref{11.3.2})}  \\
        \implies & \overline{\int}_I (cf) \leq c\bigg(\overline{\int}_I f\bigg)                                                   \\
        \implies & \overline{\int}_I (cf) \leq c\bigg(\int_I f\bigg)                       & \text{(by Definition \ref{11.3.4})}
    \end{align*}
    and
    \begin{align*}
                 & \underline{\int}_I (cf) \geq p.c. \int_I (c f_L)                          & \text{(by Definition \ref{11.3.2})}  \\
                 & \underline{\int}_I (cf) \geq p.c. \int_I (c f_L)                          & \text{(by Lemma \ref{11.3.3})}       \\
        \implies & \underline{\int}_I (cf) \geq c \bigg(p.c. \int_I f_L\bigg)                & \text{(by Theorem \ref{11.2.16}(b))} \\
        \implies & \frac{1}{c} \bigg(\underline{\int}_I (cf)\bigg) \leq p.c. \int_I f_L                                             \\
        \implies & \frac{1}{c} \bigg(\underline{\int}_I (cf)\bigg) \leq \underline{\int}_I f & \text{(by Definition \ref{11.3.2})}  \\
        \implies & \underline{\int}_I (cf) \geq c\bigg(\underline{\int}_I f\bigg)                                                   \\
        \implies & \underline{\int}_I (cf) \geq c\bigg(\int_I f\bigg).                       & \text{(by Definition \ref{11.3.4})}
    \end{align*}
    Again we have
    \[
        \int_I (cf) = \underline{\int}_I (cf) = \overline{\int}_I (cf) = c\bigg(\int_I f\bigg).
    \]

    Finally suppose \(c < 0\).
    Then we have
    \begin{align*}
        \int_I (cf) & = \int_I ((-1)(-c)f)              & \text{(by Definition \ref{9.2.1})} \\
                    & = (-1) \bigg(\int_I ((-c)f)\bigg)                                      \\
                    & = (-1)(-c) \bigg(\int_I f\bigg)   & (-c > 0)                           \\
                    & = c \bigg(\int_I f\bigg).
    \end{align*}
    We conclude that \(\forall\ c \in \mathbf{R}\), \(\int_I (cf) = c (\int_I f)\).
\end{proof}

\begin{proof}{(c)}
    We have
    \begin{align*}
        \int_I f - \int_I g & = \int_I f + \int_I (-g) & \text{(by Theorem \ref{11.4.1}(b))} \\
                            & = \int_I (f + (-g))      & \text{(by Theorem \ref{11.4.1}(a))} \\
                            & = \int_I (f - g).        & \text{(by Definition \ref{9.2.1})}
    \end{align*}
\end{proof}

\begin{proof}{(d)}
    Let \(f_U : I \to \mathbf{R}\) be a piecewise constant function on \(I\) which majorizes \(f\).
    \(f_U\) is well-defined since by Definition \ref{11.3.4} \(f\) is a bounded function on a bounded interval \(I\).
    Then by Definition \ref{11.3.2} we have
    \[
        \overline{\int}_I f \leq p.c. \int_I f_U.
    \]
    Since \(\forall\ x \in I\), \(0 \leq f(x) \leq f_U(x)\), we have
    \begin{align*}
                 & 0 \leq p.c. \int_I f_U     & \text{(by Theorem \ref{11.2.16}(d))} \\
        \implies & 0 \leq \overline{\int}_I f & \text{(by Definition \ref{11.3.2})}  \\
        \implies & 0 \leq \int_I f.           & \text{(by Definition \ref{11.3.4})}
    \end{align*}
\end{proof}

\begin{proof}{(e)}
    We have \(\forall\ x \in I\), \(f(x) - g(x) \geq 0\) and by Theorem \ref{11.4.1}(c) \(f - g\) is Riemann integrable.
    Thus
    \begin{align*}
                 & \int_I (f - g) \geq 0      & \text{(by Theorem \ref{11.4.1}(d))} \\
        \implies & \int_I f - \int_I g \geq 0 & \text{(by Theorem \ref{11.4.1}(c))} \\
        \implies & \int_I f \geq \int_I g.
    \end{align*}
\end{proof}

\begin{proof}{(f)}
    We have
    \begin{align*}
        \int_I f & = p.c. \int_I f & \text{(by Lemma \ref{11.3.7})}       \\
                 & = c \abs*{I}.   & \text{(by Theorem \ref{11.2.16}(f))}
    \end{align*}
\end{proof}

\begin{proof}{(g)}
    Let \(f_U : I \to \mathbf{R}\) be a piecewise constant function on \(I\) which majorizes \(f\).
    Let \(f_L : I \to \mathbf{R}\) be a piecewise constant function on \(I\) which minorizes \(f\).
    \(f_U, f_L\) are well-defined since by Definition \ref{11.3.4} \(f\) is a bounded function on a bounded interval \(I\).
    Then by Definition \ref{11.3.2} we have
    \[
        p.c. \int_I f_L \leq \underline{\int}_I f = \int_I f = \overline{\int}_I f \leq p.c. \int_I f_U.
    \]
    Let \(F_U : J \to \mathbf{R}\) be the function
    \[
        F_U(x) = \begin{cases}
            f_U(x) & \text{if } x \in I    \\
            0      & \text{if } x \notin I
        \end{cases}
    \]
    and let \(F_L : J \to \mathbf{R}\) be the function
    \[
        F_L(x) = \begin{cases}
            f_L(x) & \text{if } x \in I     \\
            0      & \text{if } x \notin I.
        \end{cases}
    \]
    We know that \(F_U\) majorizes \(F\) and \(F_L\) minorizes \(F\), and by Theorem \ref{11.2.16}(g) we have \(p.c. \int_J F_U = p.c. \int_I f_U\) and \(p.c. \int_J F_L = p.c. \int_I f_L\).
    Thus \(F\) is a bounded function on a bounded interval \(I\), and we have
    \begin{align*}
                 & \overline{\int}_J F \leq p.c. \int_J F_U     & \text{(by Definition \ref{11.3.2})}  \\
        \implies & \overline{\int}_J F \leq p.c. \int_I f_U     & \text{(by Theorem \ref{11.2.16}(g))} \\
        \implies & \overline{\int}_J F \leq \overline{\int}_I f & \text{(by Definition \ref{11.3.2})}  \\
        \implies & \overline{\int}_J F \leq \int_I f            & \text{(by Definition \ref{11.3.4})}
    \end{align*}
    and
    \begin{align*}
                 & \underline{\int}_J F \geq p.c. \int_J F_L      & \text{(by Definition \ref{11.3.2})}  \\
        \implies & \underline{\int}_J F \geq p.c. \int_I f_L      & \text{(by Theorem \ref{11.2.16}(g))} \\
        \implies & \underline{\int}_J F \geq \underline{\int}_I f & \text{(by Definition \ref{11.3.2})}  \\
        \implies & \underline{\int}_J F \geq \int_I f.            & \text{(by Definition \ref{11.3.4})}
    \end{align*}
    By Lemma \ref{11.3.3} we have
    \[
        \int_I f \leq \underline{\int}_J F \leq \overline{\int}_J F \leq \int_I f
    \]
    and thus by Definition \ref{11.3.4} we have
    \[
        \int_J F = \underline{\int}_J F = \overline{\int}_J F = \int_I f.
    \]
\end{proof}

\begin{proof}
    Let \(f_U : I \to \mathbf{R}\) be a piecewise constant function on \(I\) which majorizes \(f\).
    Let \(f_L : I \to \mathbf{R}\) be a piecewise constant function on \(I\) which minorizes \(f\).
    \(f_U, f_L\) are well-defined since by Definition \ref{11.3.4} \(f\) is a bounded function on a bounded interval \(I\).
    Then by Definition \ref{11.3.2} we have
    \[
        p.c. \int_I f_L \leq \underline{\int}_I f = \int_I f = \overline{\int}_I f \leq p.c. \int_I f_U.
    \]
    By Theorem \ref{11.2.16}(h) we know that \(f_U|_J : J \to \mathbf{R}\), \(f_L|_J : J \to \mathbf{R}\) are piecewise constant function on \(J\) and \(f_U|_K : K \to \mathbf{R}\), \(f_L|_K : K \to \mathbf{R}\) are piecewise constant functions on \(K\).
    We also have \(f_U|_J\) majorizes \(f|_J\) and \(f_L|_J\) minorizes \(f|_J\), similarly \(f_U|_K\) majorizes \(f|_K\) and \(f_L|_K\) minorizes \(f|_K\).
    Thus \(f|_J\), \(f|_K\) are bounded functions on bounded intervals \(J, K\) respectively, and \(\overline{\int}_J f|_J\), \(\overline{\int}_K f|_K\), \(\underline{\int}_J f|_J\), \(\underline{\int}_K f|_K\) are well-defined.
    Now we have
    \begin{align*}
                 & \overline{\int}_J f|_J + \overline{\int}_K f|_K \leq p.c. \int_J f_U|_J + p.c. \int_K f_U|_K & \text{(by Definition \ref{11.3.2})}  \\
        \implies & \overline{\int}_J f|_J + \overline{\int}_K f|_K \leq p.c. \int_I f_U                         & \text{(by Theorem \ref{11.2.16}(h))} \\
        \implies & \overline{\int}_J f|_J + \overline{\int}_K f|_K \leq \overline{\int}_I f                     & \text{(by Definition \ref{11.3.2})}  \\
        \implies & \overline{\int}_J f|_J + \overline{\int}_K f|_K \leq \int_I f                                & \text{(by Definition \ref{11.3.4})}
    \end{align*}
    and
    \begin{align*}
                 & \underline{\int}_J f|_J + \underline{\int}_K f|_K \geq p.c. \int_J f_L|_J + p.c. \int_K f_L|_K & \text{(by Definition \ref{11.3.2})}  \\
        \implies & \underline{\int}_J f|_J + \underline{\int}_K f|_K \geq p.c. \int_I f_L                         & \text{(by Theorem \ref{11.2.16}(h))} \\
        \implies & \underline{\int}_J f|_J + \underline{\int}_K f|_K \geq \underline{\int}_I f                    & \text{(by Definition \ref{11.3.2})}  \\
        \implies & \underline{\int}_J f|_J + \underline{\int}_K f|_K \geq \int_I f.                               & \text{(by Definition \ref{11.3.4})}
    \end{align*}
    By Lemma \ref{11.3.3} we have
    \[
        \int_I f \leq \underline{\int}_J f|_J + \underline{\int}_K f|_K \leq \overline{\int}_J f|_J + \overline{\int}_K f|_K \leq \int_I f
    \]
    and thus by Definition \ref{11.3.4} we have
    \[
        \int_J f|_J + \int_K f|_K = \underline{\int}_J f|_J + \underline{\int}_K f|_K = \overline{\int}_J F + \overline{\int}_J F = \int_I f.
    \]
\end{proof}

\begin{remark}\label{11.4.2}
    We often abbreviate \(\int_J f|_J\) as \(\int_J f\) even though \(f\) is really defined on a larger domain than just \(J\).
    We also observe from Theorem \ref{11.4.1}(h) and Remark \ref{11.3.8} that if \(f : [a, b] \to \mathbf{R}\) is Riemann integrable on a closed interval \([a, b]\), then \(\int_{[a, b]} f = \int_{(a, b]} f = \int_{[a, b)} f = \int_{(a, b)} f\).
\end{remark}

\begin{theorem}[Max and min preserve integrability]\label{11.4.3}
    Let \(I\) be a bounded interval, and let \(f : I \to \mathbf{R}\) and \(g : I \to \mathbf{R}\) be a Riemann integrable function.
    Then the functions \(\max(f, g) : I \to \mathbf{R}\) and \(\min(f, g) : I \to \mathbf{R}\) defined by \(\max(f, g)(x) \coloneqq \max(f (x), g(x))\) and \(\min(f, g)(x) \coloneqq \min(f (x), g(x))\) are also Riemann integrable.
\end{theorem}

\begin{proof}
    We shall just prove the claim for \(\max(f, g)\), the case of \(\min(f, g)\) being similar.
    First note that since \(f\) and \(g\) are bounded, then \(\max(f, g)\) is also bounded.

    Let \(\varepsilon > 0\).
    Since \(\int_I f = \underline{\int}_I f\), there exists a piecewise constant function \(\underline{f} : I \to \mathbf{R}\) which minorizes \(f\) on \(I\) such that
    \[
        \int_I \underline{f} \geq \int_I f - \varepsilon.
    \]
    Similarly we can find a piecewise constant \(g : I \to \mathbf{R}\) which minorizes \(g\) on \(I\) such that
    \[
        \int_I \underline{g} \geq \int_I g - \varepsilon,
    \]
    and we can find piecewise functions \(\overline{f}, \overline{g}\) which majorize \(f, g\) respectively on \(I\) such that
    \[
        \int_I \overline{f} \leq \int_I f + \varepsilon
    \]
    and
    \[
        \int_I \overline{g} \leq \int_I g + \varepsilon.
    \]
    In particular, if \(h : I \to \mathbf{R}\) denotes the function
    \[
        h \coloneqq (\overline{f} - \underline{f}) + (\overline{g} - \underline{g})
    \]
    we have
    \[
        \int_I h \leq 4 \varepsilon.
    \]
    On the other hand, \(\max(\underline{f}, \underline{g})\) is a piecewise constant function on \(I\) which minorizes \(\max(f, g)\), while \(\max(\overline{f}, \overline{g})\) is similarly a piecewise constant function on \(I\) which majorizes \(\max(f, g)\).
    Thus
    \[
        \int_I \max(\underline{f}, \underline{g}) \leq \underline{\int}_I \max(f, g) \leq \overline{\int}_I \max(f, g) \leq \int_I \max(\overline{f}, \overline{g}),
    \]
    and so
    \[
        0 \leq \overline{\int}_I \max(f, g) - \underline{\int}_I \max(f, g) \leq \int_I \max(\overline{f}, \overline{g}) - \max(\underline{f}, \underline{g}).
    \]
    But we have
    \[
        \overline{f}(x) = \underline{f}(x) + (\overline{f} - \underline{f})(x) \leq \underline{f}(x) + h(x)
    \]
    and similarly
    \[
        \overline{g}(x) = \underline{g}(x) + (\overline{g} - \underline{g})(x) \leq \underline{g}(x) + h(x)
    \]
    and thus
    \[
        \max(\overline{f}(x), \overline{g}(x)) \leq \max(\underline{f}(x), \underline{g}(x)) + h(x).
    \]
    Inserting this into the previous inequality, we obtain
    \[
        0 \leq \overline{\int}_I \max(f, g) - \underline{\int}_I \max(f, g) \leq \int_I h \leq 4 \varepsilon.
    \]
    To summarize, we have shown that
    \[
        0 \leq \overline{\int}_I \max(f, g) - \underline{\int}_I \max(f, g) \leq 4 \varepsilon
    \]
    for every \(\varepsilon\).
    Since \(\overline{\int}_I \max(f, g) - \underline{\int}_I \max(f, g)\) does not depend on \(\varepsilon\), we thus see that
    \[
        \overline{\int}_I \max(f, g) - \underline{\int}_I \max(f, g) = 0
    \]
    and hence that \(\max(f, g)\) is Riemann integrable.
\end{proof}
\section{Riemann integrability of continuous functions}\label{sec 11.5}

\begin{theorem}\label{11.5.1}
    Let \(I\) be a bounded interval, and let \(f\) be a function which is uniformly continuous on \(I\).
    Then \(f\) is Riemann integrable.
\end{theorem}

\begin{proof}
    From Proposition \ref{9.9.15} we see that \(f\) is bounded.
    Now we have to show that \(\underline{\int}_I f = \overline{\int}_I f\).

    If \(I\) is a point or the empty set then the theorem is trivial, so let us assume that \(I\) is one of the four intervals \([a, b]\), \((a, b)\), \((a, b]\), or \([a, b)\) for some real numbers \(a < b\).

            Let \(\varepsilon > 0\) be arbitrary.
            By uniform continuity, there exists a \(\delta > 0\) such that \(\abs*{f(x) - f(y)} < \varepsilon\) whenever \(x, y \in I\) are such that \(\abs*{x - y} < \delta\).
            By the Archimedean principle, there exists an integer \(N > 0\) such that \((b - a) / N < \delta\).

            Note that we can partition \(I\) into \(N\) intervals \(J_1, \dots, J_N\), each of length \((b - a) / N\).
            (How? One has to treat each of the cases \([a, b]\), \((a, b)\), \((a, b]\), \([a, b)\) slightly differently.)
    By Proposition \ref{11.3.12}, we thus have
    \[
        \overline{\int}_I f \leq \sum_{k = 1}^N (\sup_{x \in J_k} f(x)) \abs*{J_k}
    \]
    and
    \[
        \underline{\int}_I f \geq \sum_{k = 1}^N (\inf_{x \in J_k} f(x)) \abs*{J_k}
    \]
    so in particular
    \[
        \overline{\int}_I f - \underline{\int}_I f \leq \sum_{k = 1}^N (\sup_{x \in J_k} f(x) - \inf_{x \in J_k} f(x)) \abs*{J_k}.
    \]
    However, we have \(\abs*{f(x) - f(y)} < \varepsilon\) for all \(x, y \in J_k\), since \(\abs*{J_k} = (b - a) / N < \delta\).
    In particular we have
    \[
        f(x) < f(y) + \varepsilon \text{ for all } x, y \in J_k.
    \]
    Taking suprema in \(x\), we obtain
    \[
        \sup_{x \in J_k} f(x) \leq f(y) + \varepsilon \text{ for all } y \in J_k,
    \]
    and then taking infima in \(y\) we obtain
    \[
        \sup_{x \in J_k} f(x) \leq \inf_{y \in J_k} f(y) + \varepsilon.
    \]
    Inserting this bound into our previous inequality, we obtain
    \[
        \overline{\int}_I f - \underline{\int}_I f \leq \sum_{k = 1}^N \varepsilon \abs*{J_k},
    \]
    but by Theorem \ref{11.1.13} we thus have
    \[
        \overline{\int}_I f - \underline{\int}_I f \leq \varepsilon (b - a).
    \]
    But \(\varepsilon > 0\) was arbitrary, while \((b - a)\) is fixed.
    Thus \(\overline{\int}_I f - \underline{\int}_I f\) cannot be positive.
    By Lemma \ref{11.3.3} and the definition of Riemann integrability we thus have that \(f\) is Riemann integrable.
\end{proof}
\section{Riemann integrability of monotone functions}\label{sec 11.6}

\begin{proposition}\label{11.6.1}
    Let \([a, b]\) be a closed and bounded interval and let \(f : [a, b] \to \mathbf{R}\) be a monotone function.
    Then \(f\) is Riemann integrable on \([a, b]\).
\end{proposition}

\begin{proof}
    Without loss of generality we may take \(f\) to be monotone increasing (instead of monotone decreasing).
    From Exercise \ref{ex 9.8.1} we know that \(f\) is bounded.
    Now let \(N > 0\) be an integer, and partition \([a, b]\) into \(N\) half-open intervals \(\{[a + \frac{b - a}{N} j, a + \frac{b - a}{N} (j + 1)) : 0 \leq j \leq N - 1\}\) of length \((b - a) / N\), together with the point \(\{b\}\).
    Then by Proposition \ref{11.3.12} we have
    \[
        \overline{\int}_I f \leq \sum_{j = 0}^{N - 1} \Bigg(\sup_{x \in [a + \frac{b - a}{N} j, a + \frac{b - a}{N} (j + 1))} f(x)\Bigg) \frac{b - a}{N},
    \]
    (the point \(\{b\}\) clearly giving only a zero contribution).
    Since \(f\) is monotone increasing, we thus have
    \[
        \overline{\int}_I f \leq \sum_{j = 0}^{N - 1} f\bigg(a + \frac{b - a}{N} (j + 1)\bigg) \frac{b - a}{N}.
    \]
    Similarly we have
    \[
        \underline{\int}_I f \geq \sum_{j = 0}^{N - 1} f\bigg(a + \frac{b - a}{N} j\bigg) \frac{b - a}{N}.
    \]
    Thus we have
    \[
        \overline{\int}_I f - \underline{\int}_I f \leq \sum_{j = 0}^{N - 1} \Bigg(f\bigg(a + \frac{b - a}{N} (j + 1)\bigg) - f\bigg(a + \frac{b - a}{N} j\bigg)\Bigg) \frac{b - a}{N}.
    \]
    Using telescoping series (Lemma \ref{7.2.15}) we thus have
    \begin{align*}
        \overline{\int}_I f - \underline{\int}_I f & \leq \Bigg(f\bigg(a + \frac{b - a}{N} N\bigg) - f\bigg(a + \frac{b - a}{N} 0\bigg)\Bigg) \frac{b - a}{N} \\
                                                   & = (f(b) - f(a)) \frac{b - a}{N}.
    \end{align*}
    But \(N\) was arbitrary, so we can conclude as in the proof of Theorem \ref{11.5.1} that \(f\) is Riemann integrable.
\end{proof}

\begin{remark}\label{11.6.2}
    From Exercise \ref{ex 9.8.5} we know that there exist monotone functions which are not piecewise continuous, so Proposition \ref{11.6.1} is not subsumed by Proposition \ref{11.5.6}.
\end{remark}

\begin{corollary}\label{11.6.3}
    Let \(I\) be a bounded interval, and let \(f : I \to \mathbf{R}\) be both monotone and bounded.
    Then \(f\) is Riemann integrable on \(I\).
\end{corollary}

\begin{proof}
    If \(I\) is a point or an empty set then the claim is trivial;
    if \(I\) is a closed interval the claim follows from Proposition \ref{11.6.1}.
    So let us assume that \(I\) is of the form \((a, b]\), \((a, b)\), or \([a, b)\) for some \(a < b\).

    We have a bound \(M\) for \(f\), so that \(-M \leq f(x) \leq M\) for all \(x \in I\).
    Now let \(0 < \varepsilon < (b - a) / 2\) be a small number.
    The function \(f\) when restricted to the interval \([a + \varepsilon, b - \varepsilon]\) is monotone, and hence Riemann integrable by Proposition \ref{11.6.1}.
    In particular, we can find a piecewise constant function \(h : [a + \varepsilon, b - \varepsilon] \to \mathbf{R}\) which majorizes \(f\) on \([a + \varepsilon, b - \varepsilon]\) such that
    \[
        \int_{[a + \varepsilon, b - \varepsilon]} h \leq \int_{[a + \varepsilon, b - \varepsilon]} f + \varepsilon.
    \]
    Define \(\tilde{h} : I \to \mathbf{R}\) by
    \[
        \tilde{h}(x) \coloneqq \begin{cases}
            h(x) & \text{if } x \in [a + \varepsilon, b - \varepsilon]             \\
            M    & \text{if } x \in I \setminus [a + \varepsilon, b - \varepsilon]
        \end{cases}
    \]
    Clearly \(\tilde{h}\) is piecewise constant on \(I\) and majorizes \(f\);
    by Theorem \ref{11.2.16} we have
    \[
        \int_I \tilde{h} = \varepsilon M + \int_{[a + \varepsilon, b - \varepsilon]} h + \varepsilon M \leq \int_{[a + \varepsilon, b - \varepsilon]} f + (2M + 1) \varepsilon.
    \]
    In particular we have
    \[
        \overline{\int}_I f \leq \int_{[a + \varepsilon, b - \varepsilon]} f + (2M + 1) \varepsilon
    \]
    (since \(\overline{\int}_{I \cap [a, a + \varepsilon]} f \leq M \varepsilon\) and \(\overline{\int}_{I \cap [b, b - \varepsilon]} f \leq M \varepsilon\)).
    A similar argument gives
    \[
        \underline{\int}_I f \geq \int_{[a + \varepsilon, b - \varepsilon]} f - (2M + 1) \varepsilon.
    \]
    and hence
    \[
        \overline{\int}_I f - \underline{\int}_I f \leq (4M + 2) \varepsilon.
    \]
    But \(\varepsilon\) is arbitrary, and so we can argue as in the proof of Theorem \ref{11.5.1} to conclude Riemann integrability.
\end{proof}

\begin{proposition}[Integral test]\label{11.6.4}
    Let \(f : [0, \infty) \to \mathbf{R}\) be a monotone decreasing function which is non-negative
    (i.e., \(f(x) \geq 0\) for all \(x \geq 0\)).
    Then the sum \(\sum_{n = 0}^\infty f(n)\) is convergent if and only if \(\sup_{N > 0} \int_{[0, N]} f\) is finite.
\end{proposition}

\begin{proof}
    Let \(N \in \mathbf{N}\) and \(N > 0\).
    Since \(f\) is monotone decreasing, by Proposition \ref{11.6.1} we know that \(f\) is Riemann integrable on both \([0, N]\) and every interval \([a, b] \subseteq [0, N]\).
    Then we have
    \begin{align*}
        \int_{[0, N]} f & = \sum_{n = 0}^{N - 1} \int_{[n, n + 1)} f|_{[n, n + 1)} + \int_{[N, N]} f|_{[N, N]} & \text{(by Exercise \ref{ex 11.4.3})}      \\
                        & = \sum_{n = 0}^{N - 1} \int_{[n, n + 1)} f|_{[n, n + 1)}                             & \text{(by Definition \ref{11.1.8})}       \\
                        & \leq \sum_{n = 0}^{N - 1} \int_{[n, n + 1)} f(n)                                     & \text{(by Theorem \ref{11.4.1}(e))}       \\
                        & = \sum_{n = 0}^{N - 1} f(n) \abs*{n + 1 - n}                                         & \text{(by Definition \ref{11.2.9})}       \\
                        & = \sum_{n = 0}^{N - 1} f(n)                                                                                                      \\
                        & \leq \sum_{n = 0}^N f(n)                                                             & (\forall\ x \in [0, \infty), f(x) \geq 0)
    \end{align*}
    and
    \begin{align*}
        \int_{[0, N]} f & = \sum_{n = 0}^{N - 1} \int_{[n, n + 1)} f|_{[n, n + 1)} + \int_{[N, N]} f|_{[N, N]} & \text{(by Exercise \ref{ex 11.4.3})} \\
                        & = \sum_{n = 0}^{N - 1} \int_{[n, n + 1)} f|_{[n, n + 1)}                             & \text{(by Definition \ref{11.1.8})}  \\
                        & \geq \sum_{n = 0}^{N - 1} \int_{[n, n + 1)} f(n + 1)                                 & \text{(by Theorem \ref{11.4.1}(e))}  \\
                        & = \sum_{n = 0}^{N - 1} f(n + 1) \abs*{n + 1 - n}                                     & \text{(by Definition \ref{11.2.9})}  \\
                        & = \sum_{n = 0}^{N - 1} f(n + 1)                                                                                             \\
                        & = \sum_{n = 1}^N f(n).                                                               & \text{(by Lemma \ref{7.1.4}(b))}
    \end{align*}

    Suppose that \(\sum_{n = 0}^\infty f(n)\) is convergent.
    Then by Definition \ref{7.2.2} we know that
    \[
        \sum_{n = 0}^\infty f(n) = \lim_{m \to \infty} \sum_{n = 0}^m f(n)
    \]
    and by Proposition \ref{6.1.12} \((\sum_{n = 0}^m f(n))_{m = 0}^\infty\) is a Cauchy sequence.
    By Lemma \ref{5.1.15} we know that \((\sum_{n = 0}^m f(n))_{m = 0}^\infty\) is bounded by some \(M \in \mathbf{R}\).
    By Comparison principle (Lemma \ref{6.4.13}) we have
    \[
        \sup \bigg(\int_{[0, N] f}\bigg)_{N = 1}^\infty \leq \sup \bigg(\sum_{n = 0}^N f(n)\bigg)_{N = 1}^\infty \leq M
    \]
    and thus \(\sup_{N > 0} \int_{[0, N]} f\) is finite.

    Now suppose that \(\sup_{N > 0} \int_{[0, N]} f\) is finite.
    By Comparison principle (Lemma \ref{6.4.13}) we have
    \[
        \sup \bigg(\sum_{n = 1}^N f(n)\bigg)_{N = 1}^\infty \leq \sup \bigg(\int_{[0, N] f}\bigg)_{N = 1}^\infty
    \]
    By Proposition \ref{7.3.1} we thus have \(\sum_{n = 0}^\infty f(n)\) is convergent.
\end{proof}

\begin{corollary}\label{11.6.5}
    Let \(p\) be a real number.
    Then \(\sum_{n = 1}^\infty \frac{1}{n^p}\) converges absolutely when \(p > 1\) and diverges when \(p \leq 1\).
\end{corollary}

\begin{proof}
    First suppose that \(p > 1\).
    Let \(f : [1, \infty) \to \mathbf{R}\) be the function \(f(x) = \frac{1}{x^p}\).
    By Proposition \ref{6.7.3}(a)(e) we know that \(f\) is positive and monotone decreasing.
    Let \(N \in \mathbf{N}\) and \(N > 1\).
    Then by Proposition \ref{11.6.1} we know that \(\int_{[1, N]} f\) is Riemann integrable.
    Let \(x \in [1, \infty)\).
    Since \(p > 1\), we have \(p - 1 > 0\) and by Archimedean property (Corollary \ref{5.4.13}) we know that \(\exists\ M \in \mathbf{N}\) and \(M > 0\) such that \(M(N - 1)(p - 1) > 1 = 1^p > x^p\).
    This means \(\frac{1}{x^p} < \frac{1}{M(N - 1)(p - 1)}\) and thus
    \begin{align*}
        \int_{[1, N]} f & \leq \int_{[1, N]} \frac{1}{M(N - 1)(p - 1)}   & \text{(by Theorem \ref{11.4.1}(e))} \\
                        & = p.c. \int_{[1, N]} \frac{1}{M(N - 1)(p - 1)} & \text{(by Lemma \ref{11.3.7})}      \\
                        & = \frac{N - 1}{M(N - 1)(p - 1)}                & \text{(by Definition \ref{11.2.9})} \\
                        & = \frac{1}{M(p - 1)}                                                                 \\
                        & \leq \frac{1}{p - 1}.
    \end{align*}
    Since \(\int_{[1, N]} f \leq \frac{1}{p - 1}\) for all \(N > 1\), we have \(\sup_{N > 1} \int_{[1, N]} f \leq \frac{1}{p - 1}\), and thus by Proposition \ref{11.6.4} we know that \(\sum_{n = 1}^\infty f(n)\) is convergent.
\end{proof}

\exercisesection

\begin{exercise}\label{ex 11.6.1}
    Use Proposition \ref{11.6.1} to prove Corollary \ref{11.6.3}.
\end{exercise}

\begin{proof}
    See Corollary \ref{11.6.3}.
\end{proof}

\begin{exercise}\label{ex 11.6.2}
    Formulate a reasonable notion of a piecewise monotone function, and then show that all bounded piecewise monotone functions are Riemann integrable.
\end{exercise}

\begin{proof}
    Let \(I\) be a bounded interval, and let \(f : I \to \mathbf{R}\).
    We say that \(f\) is \emph{piecewise monotone on \(I\)} iff there exists a partition \(\mathbf{P}\) of \(I\) such that \(f|_J\) is monotone on \(J\) for all \(J \in \mathbf{P}\).

    Now we show that all bounded piecewise monotone functions are Riemann integrable.
    Suppose that \(f : I \to \mathbf{R}\) is a bounded piecewise monotone function.
    Then by the definition \(\exists\ \mathbf{P}\) such that \(\mathbf{P}\) is a partition of \(I\) and \(f|_J\) is monotone on \(J\) for all \(J \in \mathbf{P}\).
    Since \(f\) is bounded, \(f|_J\) is also bounded, by Corollary \ref{11.6.3} we know that \(f|_J\) is Riemann integrable on \(J\).
    Let \(F_J : I \to \mathbf{R}\) be the function
    \[
        F_J(x) = \begin{cases}
            f|_J(x) & \text{if } x \in J    \\
            0       & \text{if } x \notin J
        \end{cases}
    \]
    Then by Theorem \ref{11.4.1}(g) we know that \(F_J\) is Riemann integrable and
    \begin{align*}
        \sum_{J \in \mathbf{P}} \int_I F_J & = \sum_{J \in \mathbf{P}} \int_J f|_J & \text{(by Theorem \ref{11.4.1}(g))}  \\
                                           & = \int_I f.                           & \text{(by Exercise \ref{ex 11.4.3})}
    \end{align*}
    Thus \(f\) is Riemann integrable on \(I\).
\end{proof}

\begin{exercise}\label{ex 11.6.3}
    Prove Proposition \ref{11.6.4}.
\end{exercise}

\begin{proof}
    See Proposition \ref{11.6.4}.
\end{proof}

\begin{exercise}\label{ex 11.6.4}
    Give examples to show that both directions of the integral test break down if \(f\) is not assumed to be monotone decreasing.
\end{exercise}

\begin{proof}
    Let \(f_1 : [0, \infty) \to \mathbf{R}\) be the function
    \[
        f_1(x) = \begin{cases}
            1 & \text{if } x \in \mathbf{N}    \\
            0 & \text{if } x \notin \mathbf{N}
        \end{cases}
    \]
    Then we know that \(f_1\) is not monotone decreasing and \(\sum_{n = 0}^\infty f_1(n)\) diverges.
    But \(\int_{[0, N]} f_1 = 0\) for all \(N \in \mathbf{N}\) and \(N > 0\), thus \(\sup_{N > 0} \int_{[0, N]} f_1\) is finite.

    Let \(f_2 : [0, \infty) \to \mathbf{R}\) be the function
    \[
        f_2(x) = \begin{cases}
            \frac{1}{x^2} & \text{if } x \in \mathbf{N}    \\
            \frac{1}{x}   & \text{if } x \notin \mathbf{N}
        \end{cases}
    \]
    Then we know that \(f_2\) is not monotone decreasing.
    By Corollary \ref{11.6.5} we know that \(\sup_{N > 0} \int_{[0, N]} \frac{1}{x}\) is not finite, and since \(\int_{[0, N]} f_2 = \int_{[0, N]} \frac{1}{x}\) we also have \(\sup_{N > 0} \int_{[0, N]} f_2\) is not finite.
    But by Corollary \ref{11.6.5} we know that \(\sum_{n = 0}^\infty \frac{1}{x^2}\) converges.
\end{proof}

\begin{exercise}\label{ex 11.6.5}
    Use Proposition \ref{11.6.4} to prove Corollary \ref{11.6.5}.
\end{exercise}

\begin{proof}
    See Corollary \ref{11.6.5}.
\end{proof}
\section{A non-Riemann integrable function}\label{sec 11.7}