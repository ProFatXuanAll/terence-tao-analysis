\section{Limiting values of functions}\label{sec 9.3}

\begin{definition}[\(\varepsilon\)-closeness]\label{9.3.1}
    Let \(X\) be a subset of \(\mathbf{R}\), let \(f : X \to \mathbf{R}\) be a function, let \(L\) be a real number, and let \(\varepsilon > 0\) be a real number.
    We say that the function \(f\) is \emph{\(\varepsilon\)-close} to \(L\) iff \(f(x)\) is \(\varepsilon\)-close to \(L\) for every \(x \in X\).
\end{definition}

\setcounter{theorem}{2}
\begin{definition}[Local \(\varepsilon\)-closeness]\label{9.3.3}
    Let \(X\) be a subset of \(\mathbf{R}\), let \(f : X \to \mathbf{R}\) be a function, let \(L\) be a real number, \(x_0\) be an adherent point of \(X\), and \(\varepsilon > 0\) be a real number.
    We say that \(f\) is \emph{\(\varepsilon\)-close to \(L\) near \(x_0\)} iff there exists a \(\delta > 0\) such that \(f\) becomes \(\varepsilon\)-close to \(L\) when restricted to the set \(\{x \in X : \abs*{x - x_0} < \delta\}\).
\end{definition}
