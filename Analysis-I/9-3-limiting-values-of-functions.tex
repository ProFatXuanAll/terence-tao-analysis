\section{Limiting values of functions}\label{sec 9.3}

\begin{definition}[\(\varepsilon\)-closeness]\label{9.3.1}
    Let \(X\) be a subset of \(\mathbf{R}\), let \(f : X \to \mathbf{R}\) be a function, let \(L\) be a real number, and let \(\varepsilon > 0\) be a real number.
    We say that the function \(f\) is \emph{\(\varepsilon\)-close to \(L\)} iff \(f(x)\) is \(\varepsilon\)-close to \(L\) for every \(x \in X\).
\end{definition}

\setcounter{theorem}{2}
\begin{definition}[Local \(\varepsilon\)-closeness]\label{9.3.3}
    Let \(X\) be a subset of \(\mathbf{R}\), let \(f : X \to \mathbf{R}\) be a function, let \(L\) be a real number, \(x_0\) be an adherent point of \(X\), and \(\varepsilon > 0\) be a real number.
    We say that \(f\) is \emph{\(\varepsilon\)-close to \(L\) near \(x_0\)} iff there exists a \(\delta > 0\) such that \(f\) becomes \(\varepsilon\)-close to \(L\) when restricted to the set \(\{x \in X : \abs*{x - x_0} < \delta\}\).
\end{definition}

\setcounter{theorem}{5}
\begin{definition}[Convergence of functions at a point]\label{9.3.6}
    Let \(X\) be a subset of \(\mathbf{R}\), let \(f : X \to \mathbf{R}\) be a function, let \(E\) be a subset of \(X\), \(x_0\) be an adherent point of \(E\), and let \(L\) be a real number.
    We say that \emph{\(f\) converges to \(L\) at \(x_0\) in \(E\)}, and write \(\lim_{x \to x_0 ; x \in E} f(x) = L\), iff \(f\), after restricting to \(E\), is \(\varepsilon\)-close to \(L\) near \(x_0\) for every \(\varepsilon > 0\).
    If \(f\) does not converge to any number \(L\) at \(x_0\), we say that \emph{\(f\) diverges at \(x_0\)}, and leave \(\lim_{x \to x_0 ; x \in E} f(x)\) undefined.
\end{definition}

\begin{note}
    In other words, we have \(\lim_{x \to x_0 ; x \in E} f(x) = L\) iff for every \(\varepsilon > 0\), there exists a \(\delta > 0\) such that \(\abs*{f(x) - L} \leq \varepsilon\) for all \(x \in E\) such that \(\abs*{x - x_0} < \delta\).
\end{note}

\begin{remark}\label{9.3.7}
    In many cases we will omit the set \(E\) from the above notation (i.e., we will just say that \(f\) converges to \(L\) at \(x_0\), or that \(\lim_{x \to x_0} f(x) = L\)), although this is slightly dangerous.
    For instance, it sometimes makes a difference whether \(E\) actually contains \(x_0\) or not.
    To give an example, if \(f : \mathbf{R} \to \mathbf{R}\) is the function defined by setting \(f(x) = 1\) when \(x = 0\) and \(f(x) = 0\) when \(x \neq 0\), then one has \(\lim_{x \to 0 ; x \in \mathbf{R} \setminus \{0\}} f(x) = 0\), but \(\lim_{x \to 0 ; x \in \mathbf{R}} f(x)\) is undefined.
    Some authors only define the limit \(\lim_{x \to x_0 ; x \in E} f(x)\) when \(E\) does not contain \(x_0\) (so that \(x_0\) is now a limit point of \(E\) rather than an adherent point), or would use \(\lim_{x \to x_0 ; x \in E} f(x)\) to denote what we would call \(\lim_{x \in x_0 ; x \in E \setminus \{x_0\}} f(x)\), but we have chosen a slightly more general notation, which allows the possibility that \(E\) contains \(x_0\).
\end{remark}