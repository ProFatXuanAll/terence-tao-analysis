\section{Real exponentiation, part I}

\begin{definition}[Exponentiating a real by a natural number]\label{5.6.1}
Let \(x\) be a real number.
To raise \(x\) to the power \(0\), we define \(x^0 \coloneqq 1\).
Now suppose recursively that \(x^n\) has been defined for some natural number \(n\), then we define \(x^{n + 1} \coloneqq x^n \times x\).
\end{definition}

\begin{definition}[Exponentiating a real by an integer]\label{5.6.2}
Let \(x\) be a non-zero real number.
Then for any negative integer \(-n\), we define \(x^{-n} \coloneqq 1 / x^n\).
\end{definition}

\begin{proposition}\label{5.6.3}
All the properties in Propositions \ref{4.3.10} and \ref{4.3.12} remain valid if \(x\) and \(y\) are assumed to be real numbers instead of rational numbers.
\end{proposition}

\begin{meta-proof}
If one inspects the proof of Propositions \ref{4.3.10} and \ref{4.3.12} we see that they rely on the laws of algebra and the laws of order for the rationals (Propositions \ref{4.2.4} and \ref{4.2.9}).
But by Propositions \ref{5.3.11}, \ref{5.4.7}, and the identity \(xx^{-1} = x^{-1} x = 1\) we know that all these laws of algebra and order continue to hold for real numbers as well as rationals.
Thus we can modify the proof of Proposition \ref{4.3.10} and \ref{4.3.12} to hold in the case when \(x\) and \(y\) are real.
\end{meta-proof}

\begin{note}
Instead of giving an actual proof of Proposition \ref{5.6.3}, we shall give a meta-proof
(an argument appealing to the nature of proofs, rather than the nature of real and rational numbers).
\end{note}

\begin{definition}\label{5.6.4}
Let \(x \geq 0\) be a non-negative real, and let \(n \geq 1\) be a positive integer.
We define \(x^{1 / n}\), also known as the \emph{\(n^{\text{th}}\) root of \(x\)}, by the formula
\[
    x^{1 / n} \coloneqq \sup\{y \in \mathds{R} : y \geq 0 \text{ and } y^n \leq x\}.
\]
We often write \(\sqrt{x}\) for \(x^{1 / 2}\).
\end{definition}

\begin{note}
we do not define the \(n^{\text{th}}\) root of a negative number.
In fact, we will leave the \(n^{\text{th}}\) roots of negative numbers undefined for the rest of the text
(one can define these \(n^{\text{th}}\) roots once one defines the complex numbers, but we shall refrain from doing so).
\end{note}

\begin{lemma}[Existence of \(n^{\text{th}}\) roots]\label{5.6.5}
Let \(x \geq 0\) be a non-negative real, and let \(n \geq 1\) be a positive integer.
Then the set \(E \coloneqq \{y \in R : y \geq 0 \text{ and } y^n \leq x\}\) is non-empty and is also bounded above.
In particular, \(x^{1 / n}\) is a real number.
\end{lemma}

\begin{proof}
The set \(E\) contains \(0\), so it is certainly not empty.
Now we show it has an upper bound.
We divide into two cases: \(x \leq 1\) and \(x > 1\).
First suppose that we are in the case where \(x \leq 1\).
Then we claim that the set \(E\) is bounded above by \(1\).
To see this, suppose for sake of contradiction that there was an element \(y \in E\) for which \(y > 1\).
But then \(y^n > 1\), and hence \(y^n > x\), a contradiction.
Thus \(E\) has an upper bound.
Now suppose that we are in the case where \(x > 1\).
Then we claim that the set \(E\) is bounded above by \(x\).
To see this, suppose for contradiction that there was an element \(y \in E\) for which \(y > x\).
Since \(x > 1\), we thus have \(y > 1\).
Since \(y > x\) and \(y > 1\), we have \(y^n > x\), a contradiction.
Thus in both cases \(E\) has an upper bound, and so \(x^{1 / n}\) is finite.
\end{proof}

\begin{lemma}\label{5.6.6}
Let \(x, y \geq 0\) be non-negative reals, and let \(n, m \geq 1\) be positive integers.
\begin{enumerate}
    \item If \(y = x^{1 / n}\), then \(y^n = x\).
    \item Conversely, if \(y^n = x\), then \(y = x^{1 / n}\).
    \item \(x^{1 / n}\) is a non-negative real number, and is positive if and only if \(x\) is positive.
    \item We have \(x > y\) if and only if \(x^{1 / n} > y^{1 / n}\).
    \item If \(x > 1\), then \(x^{1 / k}\) is a decreasing function of \(k\).
    If \(x < 1\), then \(x^{1 / k}\) is an increasing function of \(k\).
    If \(x = 1\), then \(x^{1 / k} = 1\) for all \(k\).
    \item We have \((xy)^{1 / n} = x^{1 / n} y^{1 / n}\).
    \item We have \((x^{1 / n})^{1 / m} = x^{1 / nm}\).
\end{enumerate}
\end{lemma}

\begin{proof}{(c)}
We will prove the statement (c) before the other statements in Lemma \ref{5.6.6}.
First we prove that \(x^{1 / n}\) is non-negative.
Let \(E = \{z \in \mathds{R} : (z \geq 0) \land (z^n \leq x)\}\).
Since \(x^{1 / n} = \sup(E)\), we have \(x^{1 / n} \geq z \ \forall\ z \in E\) by Definition \ref{5.5.1}.
Since \(\forall\ z \in E \implies z \geq 0\), by Proposition \ref{5.4.7}, we have \(x^{1 / n} \geq z \geq 0\).
So \(x^{1 / n}\) is non-negative.

Now we prove that \(x^{1 / n}\) is positive iff \(x\) is positive.
By the given conditions we already know that \(x \geq 0\).
If \(x^{1 / n}\) is positive, then \(x\) is also positive, otherwised \(x = 0\) means \(E = \{z \in \mathds{R} : (z \geq 0) \land (z^n \leq 0)\} = \{0\}\), and \(x^{1 / n} = \sup(\{0\}) = 0\), a contradiction.
If \(x\) is positive, then we can split into two cases:
\begin{enumerate}[label=(\Roman*)]
    \item \(x > 1\).
    \item \(x \leq 1\).
\end{enumerate}

% then \(x^{1 / n}\) is also positive, otherwised \(x^{1 / n} = \sup(E) = 0\) means \(E = \{0\} = \{z \in \mathds{R} : (z \geq 0) \land (z^n \leq x)\}\) and \(x = 0\), a contradiction.
\end{proof}

\begin{proof}{(a)}
% Let \(E = \{z \in \mathds{R} : (z \geq 0) \land (z^n \leq x)\}\).
% Since \(y = x^{1 / n} = \sup(E) \in \mathds{R}\), we have \(y^n \in \mathds{R}\) by Proposition \ref{5.3.10}.
% By Proposition \ref{5.4.7}, we have exactly one of the following three statements is true:
% \begin{enumerate*}[label=(\roman*)]
%     \item \(y^n < x\)
%     \item \(y^n > x\)
%     \item \(y^n = x\).
% \end{enumerate*}

% Now we argue by contradiction.
% We show that both \(y^n < x\) and \(y^n > x\) lead to contradictions.
% First suppose that \(y^n < x\).
% Then we have \(y \in E\).

\end{proof}

\begin{proof}{(b)}

\end{proof}

\begin{proof}{(d)}

\end{proof}

\begin{proof}{(e)}

\end{proof}

\begin{proof}{(f)}

\end{proof}

\begin{proof}{(g)}

\end{proof}

\exercisesection

\begin{exercise}\label{ex 5.6.1}
Prove Lemma \ref{5.6.6}.
\end{exercise}

\begin{proof}
See Lemma \ref{5.6.6}.
\end{proof}