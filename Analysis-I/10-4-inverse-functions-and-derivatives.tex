\section{Inverse functions and derivatives}\label{sec 10.4}

\begin{lemma}\label{10.4.1}
    Let \(f : X \to Y\) be an invertible function, with inverse \(f^{-1} : Y \to X\).
    Suppose that \(x_0 \in X\) and \(y_0 \in Y\) are such that \(y_0 = f(x_0)\)
    (which also implies that \(x_0 = f^{-1}(y_0)\)).
    If \(f\) is differentiable at \(x_0\), and \(f^{-1}\) is differentiable at \(y_0\), then
    \[
        (f^{-1})'(y_0) = \frac{1}{f'(x_0)}.
    \]
\end{lemma}

\begin{proof}
    From the chain rule (Theorem \ref{10.1.15}) we have
    \[
        (f^{-1} \circ f)'(x_0) = (f^{-1})'(y_0) f'(x_0).
    \]
    But \(f^{-1} \circ f\) is the identity function on \(X\), and hence by Theorem \ref{10.1.13}(b) \((f^{-1} \circ f)'(x_0) = 1\).
    The claim follows.
\end{proof}

\begin{note}
    As a particular corollary of Lemma \ref{10.4.1}, we see that if \(f\) is differentiable at \(x_0\) with \(f'(x_0) = 0\), then \(f^{-1}\) cannot be differentiable at \(y_0 = f(x_0)\), since \(1 / f'(x_0)\) is undefined in that case.
\end{note}

\begin{note}
    If one writes \(y = f(x)\), so that \(x = f^{-1}(y)\), then one can write the conclusion of Lemma \ref{10.4.1} in the more appealing form \(dx / dy = 1 / (dy / dx)\).
    However, as mentioned before, this way of writing things, while very convenient and easy to remember, can be misleading and cause errors if applied too carelessly (especially when one begins to work in the calculus of several variables).
\end{note}

\begin{note}
    Lemma \ref{10.4.1} seems to answer the question of how to differentiate the inverse of a function, however it has one significant drawback:
    the lemma only works if one assumes a \emph{priori} that \(f^{-1}\) is differentiable.
    Thus, if one does not already know that \(f^{-1}\) is differentiable, one cannot use Lemma \ref{10.4.1} to compute the derivative of \(f^{-1}\) .
\end{note}