\section{Russell's paradox}\label{sec:3.2}

\begin{ax}[Universal specification]\label{3.8}
  (Dangerous!)
  Suppose for every object \(x\) we have a property \(P(x)\) pertaining to \(x\) (so that for every \(x\), \(P(x)\) is either a true statement or a false statement).
  Then there exists a set \(\set{x : P(x) \text{ is true}}\) such that for every object \(y\),
  \[
    y \in \set{x : P(x) \text{ is true}} \iff P(y) \text{ is true}.
  \]
\end{ax}

\begin{note}
  Compare to \cref{3.5}, an object \(x\) does not need to be in a set \(A\) to apply this axiom.
  So \cref{3.8} is more powerful than \cref{3.5}.
\end{note}

\begin{note}
  \cref{3.8} is also known as the \emph{axiom of comprehension}.
  It asserts that every property corresponds to a set.
  \cref{3.8} also implies most of the axioms in \cref{sec:3.1} (see \cref{ex:3.2.1}).
  Unfortunately, this axiom cannot be introduced into set theory, because it creates a logical contradiction known as \emph{Russell's paradox}, discovered by the philosopher and logician Bertrand Russell (1872--1970) in 1901.
  The paradox runs as follows.
  Let \(P(x)\) be the statement
  \[
    P(x) \iff \text{``\(x\) is a set, and \(x \notin x\)''};
  \]
  i.e., \(P(x)\) is true only when \(x\) is a set which does not contain itself.
  Now use the axiom of universal specification to create the set
  \[
    \Omega \coloneqq \set{x : P(x) \text{ is true}} = \set{x : x \text{ is a set and } x \notin x},
  \]
  i.e., the set of all sets which do not contain themselves.
  Now ask the question: does \(\Omega\) contain itself, i.e. is \(\Omega \in \Omega\)?
  If \(\Omega\) did contain itself, then by definition this means that \(P(\Omega)\) is true, i.e., \(\Omega\) is a set and \(\Omega \notin \Omega\).
  On the other hand, if \(\Omega\) did not contain itself, then \(P(\Omega)\) would be true, and hence \(\Omega \in \Omega\).
  Thus in either case we have both \(\Omega \in \Omega\) and \(\Omega \notin \Omega\), which is absurd.
\end{note}

\begin{note}
  The problem with \cref{3.8} is that it creates sets which are far too ``large''.
  Since sets are themselves objects (\cref{3.1}), this means that sets are allowed to contain themselves, which is a somewhat silly state of affairs.
  One way to informally resolve this issue is to think of objects as being arranged in a hierarchy.
  At the bottom of the hierarchy are the \emph{primitive objects} - the objects that are not sets.
  Then on the next rung of the hierarchy there are sets whose elements consist only of primitive objects, let's call these ``primitive sets'' for now.
  Then there are sets whose elements consist only of primitive objects and primitive sets, and we can form sets out of these objects, and so forth.
  The point is that at each stage of the hierarchy we only see sets whose elements consist of objects at lower stages of the hierarchy, and so at no stage do we ever construct a set which contains itself.
\end{note}

\begin{note}
  In pure set theory, there will be no primitive objects, but there will be one primitive set \(\emptyset\) on the next rung of the hierarchy.
\end{note}

\begin{ax}[Regularity]\label{3.9}
  If \(A\) is a non-empty set, then there is at least one element \(x\) of \(A\) which is either not a set, or is disjoint from \(A\).
\end{ax}

\begin{note}
  The point of \cref{3.9} (which is also known as the \emph{axiom of foundation}) is that it is asserting that at least one of the elements of \(A\) is so low on the hierarchy of objects that it does not contain any of the other elements of \(A\).
  One particular consequence of \cref{3.9} is that sets are no longer allowed to contain themselves (\cref{ex:3.2.2}).
\end{note}

\exercisesection

\begin{ex}\label{ex:3.2.1}
  Show that the universal specification axiom, \cref{3.8}, if assumed to be true, would imply \crefrange{3.2}{3.6}.
  (If we assume that all natural numbers are objects, we also obtain \cref{3.7}.)
  Thus, \cref{3.8}, if permitted, would simplify the foundations of set theory tremendously (and can be viewed as one basis for an intuitive model of set theory known as ``naive set theory'').
  Unfortunately, as we have seen, \cref{3.8} is ``too good to be true''!
\end{ex}

\begin{proof}[\pf{ex:3.2.1}]
  We first show that \cref{3.8} implies \cref{3.2}.
  Using \cref{3.8} we can create a set \(E = \set{x : P(x)}\) where \(P(x)\) is a property which is false for all object \(x\).
  Then we must have \(x \notin E\) for every object \(x\).
  Hence we can construct empty set using \cref{3.8} and therefore \cref{3.8} implies \cref{3.2}.

  Next we show that \cref{3.8} implies \cref{3.3}.
  Suppose that \(a, b\) are objects.
  Using \cref{3.8} we can create a set \(A = \set{x : P(x)}\) where \(P(x)\) is a property which is false for all object \(x\) other than \(a\).
  Then the only element in the set \(A\) is \(a\).
  Hence we can construct any singleton set using \cref{3.8}.
  Using \cref{3.8} again we can create a set \(B = \set{x : Q(x)}\) where \(Q(x)\) is a property which is false for all object \(x\) other than \(a, b\).
  Then \(a, b\) are the only elements in the set \(B\).
  Hence we can construct any pair set using \cref{3.8}.
  Therefore \cref{3.8} implies \cref{3.3}.

  Next we show that \cref{3.8} implies \cref{3.4}.
  Suppose that \(A, B\) are sets.
  By \cref{3.8}, there exists a set \(C = \set{x : P(x)}\) where \(P(x) = (x \in A) \lor (x \in B)\).
  Clearly we have \(C = A \cup B\).
  Therefore \cref{3.8} implies \cref{3.4}.

  Next we show that \cref{3.8} implies \cref{3.5}.
  Suppose that \(A\) is a set.
  Using \cref{3.8} we can create a property \(P(x)\) which is true for some object \(x\).
  Then using \cref{3.8} again we can create a set \(B = \set{x : Q(x)}\) where \(Q(x) = (x \in A) \land P(x)\).
  Clearly we have \(B = \set{x \in A : Q(x)}\).
  Therefore \cref{3.8} implies \cref{3.5}.

  Next we show that \cref{3.8} implies \cref{3.6}.
  Suppose that \(A\) is a set and \(P(x, y)\) is a property such that for each \(x \in A\), there is at most one object \(y\) such that \(P(x, y)\).
  By \cref{3.8}, there exists a set \(B = \set{y : Q(y)}\) where \(Q(y)\) is the statement ``\(P(x, y)\) is true for some \(x \in A\)''.
  Clearly we have \(B = \set{y : P(x, y) \text{ is true for some } x \in A}\).
  Therefore \cref{3.8} implies \cref{3.6}.

  Finally, suppose all natural numbers are objects.
  Using \cref{3.8} we can create a set \(N = \set{x : P(x)}\) where \(P(x)\) is the statement ``\(x\) is a natural number and \(x\) satisfy \crefrange{2.1}{2.5}''.
  Clearly we have \(N = \N\).
  Therefore \cref{3.8} implies \cref{3.7}.
\end{proof}

\begin{ex}\label{ex:3.2.2}
  Use the axiom of regularity, \cref{3.9} (and the singleton set axiom, \cref{3.3}) to show that if \(A\) is a set, then \(A \notin A\).
  Furthermore, show that if \(A\) and \(B\) are two sets, then either \(A \notin B\) or \(B \notin A\) (or both).
\end{ex}

\begin{proof}[\pf{ex:3.2.2}]
  We fisrt show that if \(A\) is a set, then \(A \notin A\).
  Suppose for sake of contradiction that there exist a set \(A\) such that \(A \in A\) is true.
  By \cref{3.3}, there exist a set \(\set{A}\) and \(A \in \set{A}\) is true.
  Then \(A \in A \cap \set{A}\) is true.
  But by \cref{3.9}, the only element \(A\) in \(\set{A}\) must be disjoint from \(\set{A}\), which mean \(A \cap \set{A} = \emptyset\), a contradiction.
  Thus there does not exist a set \(A\) such that \(A \in A\) is true, i.e., \(A \notin A\) for any set \(A\).

  Now we show that if \(A\) and \(B\) are two sets, then \((A \notin B) \lor (B \notin A)\) is true.
  Since
  \[
    (A \notin B) \lor (B \notin A) \iff (\lnot(A \in B)) \lor (\lnot(B \in A)) \iff \lnot ((A \in B) \land (B \in A)),
  \]
  it suffices to show that \((A \in B) \land (B \in A)\) is false.
  So suppose for sake of contradiction that \((A \in B) \land (B \in A)\) is true.
  By \cref{3.3} we can create a pair set \(\set{A, B}\).
  By \cref{3.9} we know that there exists one element in \(\set{A, B}\) such that either it is not a set or it is disjoint from \(\set{A, B}\).
  Since \(A, B\) are sets, we must have either \(A \cap \set{A, B} = \emptyset\) or \(B \cap \set{A, B} = \emptyset\).
  But since \((A \in B) \land (B \in A)\), we must have \(A \in B \cap \set{A, B}\) and \(B \in A \cap \set{A, B}\), a contradiction.
  Thus \((A \in B) \land (B \in A)\) is false.
\end{proof}

\begin{ex}\label{ex:3.2.3}
  Show (assuming the other axioms of set theory) that the universal specification axiom, \cref{3.8}, is equivalent to an axiom postulating the existence of a ``universal set'' \(\Omega\) consisting of all objects (i.e., for all objects \(x\), we have \(x \in \Omega\)).
  In other words, if \cref{3.8} is true, then a universal set exists, and conversely, if a universal set exists, then \cref{3.8} is true.
  (This may explain why \cref{3.8} is called the axiom of universal specification.)
  Note that if a universal set \(\Omega\) existed, then we would have \(\Omega \in \Omega\) by \cref{3.1}, contradicting \cref{ex:3.2.2}.
  Thus the axiom of foundation specifically rules out the axiom of universal specification.
\end{ex}

\begin{proof}[\pf{ex:3.2.3}]
  If \cref{3.8} is true, then there exists a set \(\Omega = \set{x : x \text{ is a object}}\), and \(\Omega \in \Omega\).
  Thus \cref{3.8} implies a universal set exists.
  If a universal set \(\Omega\) exists, then by \cref{3.5} we must have a set \(A = \set{x \in \Omega : P(x)}\) where \(P(x)\) is a property of object \(x \in \Omega\).
  Clearly we have \(A = \set{x : P(x)}\) since \(\Omega\) is consist of all objects.
  Thus a universal set exists implies \cref{3.8} is true.
  We conclude that \cref{3.8} is true iff a universal set exists.
\end{proof}
