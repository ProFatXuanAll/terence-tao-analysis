\section{Finite series}\label{sec:7.1}

\begin{defn}[Finite series]\label{7.1.1}
  Let \(m, n\) be integers, and let \((a_i)_{i = m}^n\) be a finite sequence of real numbers, assigning a real number \(a_i\) to each integer \(i\) between \(m\) and \(n\) inclusive (i.e., \(m \leq i \leq n\)).
  Then we define the finite sum (or finite series) \(\sum_{i = m}^n a_i\) by the recursive formula
  \begin{align*}
     & \sum_{i = m}^n a_i \coloneqq 0 \text{ whenever } n < m ;                                                      \\
     & \sum_{i = m}^{n + 1} a_i \coloneqq \Bigg(\sum_{i = m}^n a_i\Bigg) + a_{n + 1} \text{ whenever } n \geq m - 1.
  \end{align*}
\end{defn}

\begin{note}
  we sometimes express \(\sum_{i = m}^n a_i\) less formally as
  \[
    \sum_{i = m}^n a_i = a_m + a_{m + 1} + \dots + a_n.
  \]
\end{note}

\begin{rmk}\label{7.1.2}
  The difference between ``sum'' and ``series'' is a subtle linguistic one.
  Strictly speaking, a series is an \emph{expression} of the form \(\sum_{i = m}^n a_i\);
  this series is mathematically (but not semantically) equal to a real number, which is then the \emph{sum} of that series.
  For instance, \(1 + 2 + 3 + 4 + 5\) is a series, whose sum is \(15\);
  if one were to be very picky about semantics, one would not consider \(15\) a series and one would not consider \(1 + 2 + 3 + 4 + 5\) a sum, despite the two expressions having the same value.
  However, we will not be very careful about this distinction as it is purely linguistic and has no bearing on the mathematics;
  the expressions \(1 + 2 + 3 + 4 + 5\) and \(15\) are the same number, and thus \emph{mathematically} interchangeable, in the sense of the axiom of substitution, even if they are not semantically interchangeable.
\end{rmk}

\begin{rmk}\label{7.1.3}
  Note that the variable \(i\) (sometimes called the \emph{index of summation}) is a \emph{bound variable} (sometimes called a \emph{dummy variable});
  the expression \(\sum_{i = m}^n a_i\) does not actually depend on any quantity named \(i\).
  In particular, one can replace the index of summation \(i\) with any other symbol, and obtain the same sum:
  \[
    \sum_{i = m}^n a_i = \sum_{j = m}^n a_j.
  \]
\end{rmk}

\begin{lem}\label{7.1.4}
  \begin{enumerate}
    \item Let \(m \leq n < p\) be integers, and let \(a_i\) be a real number assigned to each integer \(m \leq i \leq p\).
          Then we have
          \[
            \sum_{i = m}^n a_i + \sum_{i = n + 1}^p a_i = \sum_{i = m}^p a_i.
          \]
    \item Let \(m \leq n\) be integers, \(k\) be another integer, and let \(a_i\) be a real number assigned to each integer \(m \leq i \leq n\).
          Then we have
          \[
            \sum_{i = m}^n a_i = \sum_{j = m + k}^{n + k} a_{j - k}.
          \]
    \item Let \(m \leq n\) be integers, and let \(a_i, b_i\) be real numbers assigned to each integer \(m \leq i \leq n\).
          Then we have
          \[
            \sum_{i = m}^n (a_i + b_i) = \Bigg(\sum_{i = m}^n a_i\Bigg) + \Bigg(\sum_{i = m}^n b_i\Bigg).
          \]
    \item Let \(m \leq n\) be integers, and let \(a_i\) be a real number assigned to each integer \(m \leq i \leq n\), and let \(c\) be another real number.
          Then we have
          \[
            \sum_{i = m}^n (ca_i) = c\Bigg(\sum_{i = m}^n a_i\Bigg).
          \]
    \item (Triangle inequality for finite series)
          Let \(m \leq n\) be integers, and let \(a_i\) be a real number assigned to each integer \(m \leq i \leq n\).
          Then we have
          \[
            \abs{\sum_{i = m}^n a_i} \leq \sum_{i = m}^n \abs{a_i}.
          \]
    \item (Comparison test for finite series) Let \(m \leq n\) be integers, and let \(a_i\), \(b_i\) be real numbers assigned to each integer \(m \leq i \leq n\).
          Suppose that \(a_i \leq b_i\) for all \(m \leq i \leq n\).
          Then we have
          \[
            \sum_{i = m}^n a_i \leq \sum_{i = m}^n b_i
          \]
  \end{enumerate}
\end{lem}

\begin{proof}{(a)}
  Let \(k = p - m\).
  By hypothesis we know that \(k > 0\).
  Now we use induction on \(k\) to show that \cref{7.1.4}(a) is true and we start with \(k = 1\).
  For \(k = 1\), we have \(p = m + 1\) and by \cref{7.1.1} we have
  \[
    \sum_{i = m}^n a_i + \sum_{i = n + 1}^p a_i = \sum_{i = m}^m a_i + \sum_{i = m + 1}^p a_i = a_m + a_{m + 1} = \sum_{i = m}^p a_i.
  \]
  Thus the base case holds.
  Suppose inductively that for some \(k \geq 1\) \cref{7.1.4}(a) is true.
  Then for \(k + 1 = p - m\), we have \(p - 1 = k + m\) and
  \begin{align*}
    \sum_{i = m}^n a_i + \sum_{i = n + 1}^p a_i & = \Bigg(\sum_{i = m}^n a_i\Bigg) + \Bigg(\sum_{i = n + 1}^{p - 1} a_i\Bigg) + a_p &  & \by{7.1.1} \\
                                                & = \Bigg(\sum_{i = m}^{p - 1} a_i\Bigg) + a_p                                      &  & \byIH      \\
                                                & = \sum_{i = m}^p a_i.                                                             &  & \by{7.1.1}
  \end{align*}
  This closes the induction.
\end{proof}

\begin{proof}{(b)}
  Let \(p = n - m\).
  By hypothesis we know that \(p \geq 0\).
  Now we use induction on \(p\) to show that \cref{7.1.4}(b) is true.
  For \(p = 0\), we have \(n = m\) and
  \begin{align*}
    \sum_{j = m + k}^{m + k} a_{j - k} & = \Bigg(\sum_{j = m + k}^{m + k - 1} a_{j - k}\Bigg) + a_{m + k - k} &  & \by{7.1.1} \\
                                       & = 0 + a_{m + k - k}                                                  &  & \by{7.1.1} \\
                                       & = 0 + a_m                                                                            \\
                                       & = \Bigg(\sum_{i = m}^{m - 1} a_i\Bigg) + a_m                         &  & \by{7.1.1} \\
                                       & = \sum_{i = m}^m a_i.                                                &  & \by{7.1.1}
  \end{align*}
  So the base case holds.
  Suppose inductively that for some \(p \geq 0\) \cref{7.1.4}(b) is true.
  Then for \(p + 1 = n - m\), we have \(p = n - m - 1\) and
  \begin{align*}
    \sum_{j = m + k}^{n + k} a_{j - k} & = \Bigg(\sum_{j = m + k}^{n + k - 1} a_{j - k}\Bigg) + a_{n + k - k} &  & \by{7.1.1} \\
                                       & = \Bigg(\sum_{j = m + k}^{n + k - 1} a_{j - k}\Bigg) + a_n                           \\
                                       & = \Bigg(\sum_{i = m}^{n - 1} a_i\Bigg) + a_n                         &  & \byIH      \\
                                       & = \sum_{i = m}^n a_i.                                                &  & \by{7.1.1}
  \end{align*}
  This closes the induction.
\end{proof}

\begin{proof}{(c)}
  Let \(p = n - m\).
  By hypothesis we know that \(p \geq 0\).
  Now we use induction on \(p\) to show that \cref{7.1.4}(c) is true.
  For \(p = 0\), we have \(n = m\) and
  \begin{align*}
    \sum_{i = m}^m (a_i + b_i) & = \Bigg(\sum_{i = m}^{m - 1} (a_i + b_i)\Bigg) + a_m + b_m                                &  & \by{7.1.1} \\
                               & = 0 + a_m + b_m                                                                           &  & \by{7.1.1} \\
                               & = \Bigg(\sum_{i = m}^{m - 1} a_i\Bigg) + \Bigg(\sum_{i = m}^{m - 1} b_i\Bigg) + a_m + b_m &  & \by{7.1.1} \\
                               & = \Bigg(\sum_{i = m}^m a_i\Bigg) + \Bigg(\sum_{i = m}^m b_i\Bigg).                        &  & \by{7.1.1}
  \end{align*}
  So the base case holds.
  Suppose inductively that for some \(p \geq 0\) \cref{7.1.4}(c) is true.
  Then for \(p + 1 = n - m\), we have \(p = n - m - 1\) and
  \begin{align*}
    \sum_{i = m}^n (a_i + b_i) & = \Bigg(\sum_{i = m}^{n - 1} (a_i + b_i)\Bigg) + a_n + b_n                                &  & \by{7.1.1} \\
                               & = \Bigg(\sum_{i = m}^{n - 1} a_i\Bigg) + \Bigg(\sum_{i = m}^{n - 1} b_i\Bigg) + a_n + b_n &  & \byIH      \\
                               & = \Bigg(\sum_{i = m}^n a_i\Bigg) + \Bigg(\sum_{i = m}^n b_i\Bigg).                        &  & \by{7.1.1}
  \end{align*}
  This closes the induction.
\end{proof}

\begin{proof}{(d)}
  Let \(p = n - m\).
  By hypothesis we know that \(p \geq 0\).
  Now we use induction on \(p\) to show that \cref{7.1.4}(d) is true.
  For \(p = 0\), we have \(n = m\) and
  \begin{align*}
    \sum_{i = m}^m ca_i & = \Bigg(\sum_{i = m}^{m - 1} ca_i\Bigg) + ca_m             &  & \by{7.1.1} \\
                        & = 0 + ca_m                                                 &  & \by{7.1.1} \\
                        & = c \times 0 + ca_m                                                        \\
                        & = c \Bigg(\sum_{i = m}^{m - 1} a_i\Bigg) + ca_m            &  & \by{7.1.1} \\
                        & = c \Bigg(\Bigg(\sum_{i = m}^{m - 1} a_i\Bigg) + a_m\Bigg)                 \\
                        & = c \Bigg(\sum_{i = m}^m a_i\Bigg).                        &  & \by{7.1.1}
  \end{align*}
  So the base case holds.
  Suppose inductively that for some \(p \geq 0\) \cref{7.1.4}(d) is true.
  Then for \(p + 1 = n - m\), we have \(p = n - m - 1\) and
  \begin{align*}
    \sum_{i = m}^n ca_i & = \Bigg(\sum_{i = m}^{n - 1} ca_i\Bigg) + ca_n             &  & \by{7.1.1} \\
                        & = c \Bigg(\sum_{i = m}^{n - 1} a_i\Bigg) + ca_n            &  & \byIH      \\
                        & = c \Bigg(\Bigg(\sum_{i = m}^{n - 1} a_i\Bigg) + a_n\Bigg)                 \\
                        & = c \Bigg(\sum_{i = m}^n a_i\Bigg).                        &  & \by{7.1.1}
  \end{align*}
  This closes the induction.
\end{proof}

\begin{proof}{(e)}
  Let \(p = n - m\).
  By hypothesis we know that \(p \geq 0\).
  Now we use induction on \(p\) to show that \cref{7.1.4}(e) is true.
  For \(p = 0\), we have \(n = m\) and
  \begin{align*}
    \abs{\sum_{i = m}^m a_i} & = \abs{\Bigg(\sum_{i = m}^{m - 1} a_i\Bigg) + a_m}       &  & \by{7.1.1} \\
                             & = \abs{0 + a_m}                                          &  & \by{7.1.1} \\
                             & = 0 + \abs{a_m}                                                          \\
                             & = \Bigg(\sum_{i = m}^{m - 1} \abs{a_i}\Bigg) + \abs{a_m} &  & \by{7.1.1} \\
                             & = \sum_{i = m}^m \abs{a_i}.                              &  & \by{7.1.1}
  \end{align*}
  So the base case holds.
  Suppose inductively that for some \(p \geq 0\) \cref{7.1.4}(e) is true.
  Then for \(p + 1 = n - m\), we have \(p = n - m - 1\) and
  \begin{align*}
    \abs{\sum_{i = m}^n a_i} & = \abs{\Bigg(\sum_{i = m}^{n - 1} a_i\Bigg) + a_n} &  & \by{7.1.1} \\
                             & \leq \abs{\sum_{i = m}^{n - 1} a_i} + \abs{a_n}                    \\
                             & \leq \sum_{i = m}^{n - 1} \abs{a_i} + \abs{a_n}    &  & \byIH      \\
                             & = \sum_{i = m}^n \abs{a_i}.                        &  & \by{7.1.1}
  \end{align*}
  This closes the induction.
\end{proof}

\begin{proof}{(f)}
  Let \(p = n - m\).
  By hypothesis we know that \(p \geq 0\).
  Now we use induction on \(p\) to show that \cref{7.1.4}(f) is true.
  For \(p = 0\), we have \(n = m\) and
  \begin{align*}
    \sum_{i = m}^m a_i & = \Bigg(\sum_{i = m}^{m - 1} a_i\Bigg) + a_m &  & \by{7.1.1}             \\
                       & = 0 + a_m                                    &  & \by{7.1.1}             \\
                       & \leq 0 + b_m                                 &  & \text{(by hypothesis)} \\
                       & = \Bigg(\sum_{i = m}^{m - 1} b_i\Bigg) + b_m &  & \by{7.1.1}             \\
                       & = \sum_{i = m}^m b_i.                        &  & \by{7.1.1}             \\
  \end{align*}
  So the base case holds.
  Suppose inductively that for some \(p \geq 0\) \cref{7.1.4}(f) is true.
  Then for \(p + 1 = n - m\), we have \(p = n - m - 1\) and
  \begin{align*}
    \sum_{i = m}^n a_i & = \Bigg(\sum_{i = m}^{n - 1} a_i\Bigg) + a_n    &  & \by{7.1.1}             \\
                       & \leq \Bigg(\sum_{i = m}^{n - 1} b_i\Bigg) + a_n &  & \byIH                  \\
                       & \leq \Bigg(\sum_{i = m}^{n - 1} b_i\Bigg) + b_n &  & \text{(by hypothesis)} \\
                       & = \sum_{i = m}^n b_i.                           &  & \by{7.1.1}             \\
  \end{align*}
  This closes the induction.
\end{proof}

\begin{rmk}\label{7.1.5}
  In the future we may omit some of the parentheses in series expressions, for instance we may write \(\sum_{i = m}^n (a_i + b_i)\) simply as \(\sum_{i = m}^n a_i + b_i\).
  This is reasonably safe from being mis-interpreted, because the alternative interpretation \((\sum_{i = m}^n a_i) + b_i\) does not make any sense
  (the index \(i\) in \(b_i\) is meaningless outside of the summation, since \(i\) is only a dummy variable).
\end{rmk}

\begin{defn}[Summations over finite sets]\label{7.1.6}
  Let \(X\) be a finite set with \(n\) elements (where \(n \in \N\)), and let \(f : X \to \R\) be a function from \(X\) to the real numbers
  (i.e., \(f\) assigns a real number \(f(x)\) to each element \(x\) of \(X\)).
  Then we can define the finite sum \(\sum_{x \in X} f(x)\) as follows.
  We first select any bijection \(g\) from \(\set{i \in \N : 1 \leq i \leq n}\) to \(X\);
  such a bijection exists since \(X\) is assumed to have \(n\) elements.
  We then define
  \[
    \sum_{x \in X} f(x) \coloneqq \sum_{i = 1}^n f(g(i)).
  \]
  In some cases we would like to define the sum \(\sum_{x \in X} f(x)\) when \(f : Y \to \R\) is defined on a larger set \(Y\) than \(X\).
  In such cases we use exactly the same definition as is given above.
\end{defn}

\setcounter{thm}{7}
\begin{prop}[Finite summations are well-defined]\label{7.1.8}
  Let \(X\) be a finite set with \(n\) elements (where \(n \in \N\)), let \(f : X \to \R\) be a function, and let \(g : \set{i \in \N : 1 \leq i \leq n} \to X\) and \(h : \set{i \in \N : 1 \leq i \leq n} \to X\) be bijections.
  Then we have
  \[
    \sum_{i = 1}^n f(g(i)) = \sum_{i = 1}^n f(h(i)).
  \]
\end{prop}

\begin{proof}
  We use induction on \(n\);
  more precisely, we let \(P(n)\) be the assertion that ``For any set \(X\) of \(n\) elements, any function \(f : X \to \R\), and any two bijections \(g, h\) from \(\set{i \in \N : 1 \leq i \leq n}\) to \(X\), we have \(\sum_{i = 1}^n f(g(i)) = \sum_{i = 1}^n f(h(i))\)''.
  (More informally, \(P(n)\) is the assertion that \cref{7.1.8} is true for that value of \(n\).)
  We want to prove that \(P(n)\) is true for all natural numbers \(n\).

  We first check the base case \(P(0)\).
  In this case \(\sum_{i = 1}^0 f(g(i))\) and \(\sum_{i = 1}^0 f(h(i))\) both equal to \(0\), by definition of finite series (\cref{7.1.1}), so we are done.

  Now suppose inductively that \(P(n)\) is true;
  we now prove that \(P(n + 1)\) is true.
  Thus, let \(X\) be a set with \(n + 1\) elements, let \(f : X \to \R\) be a function, and let \(g\) and \(h\) be bijections from \(\set{i \in N : 1 \leq i \leq n + 1}\) to \(X\).
  We have to prove that
  \[
    \sum_{i = 1}^{n + 1} f(g(i)) = \sum_{i = 1}^{n + 1} f(h(i)). \tag{7.1}\label{eq 7.1}
  \]
  Let \(x \coloneqq g(n + 1)\);
  thus \(x\) is an element of \(X\).
  By definition of finite series (\cref{7.1.1}), we can expand the left-hand side of \eqref{eq 7.1} as
  \[
    \sum_{i = 1}^{n + 1} f(g(i)) = \Bigg(\sum_{i = 1}^n f(g(i))\Bigg) + f(x).
  \]
  Now let us look at the right-hand side of \eqref{eq 7.1}.
  Ideally we would like to have \(h(n + 1)\) also equal to \(x\)
  - this would allow us to use the inductive hypothesis \(P(n)\) much more easily
  - but we cannot assume this.
  However, since \(h\) is a bijection, we do know that there is \emph{some} index \(j\), with \(1 \leq j \leq n + 1\), for which \(h(j) = x\).
  We now use \cref{7.1.4} and the definition of finite series (\cref{7.1.1}) to write
  \begin{align*}
    \sum_{i = 1}^{n + 1} f(h(i)) & = \Bigg(\sum_{i = 1}^j f(h(i))\Bigg) + \Bigg(\sum_{i = j + 1}^{n + 1} f(h(i))\Bigg)                 \\
                                 & = \Bigg(\sum_{i = 1}^{j - 1} f(h(i))\Bigg) + f(h(j)) + \Bigg(\sum_{i = j + 1}^{n + 1} f(h(i))\Bigg) \\
                                 & = \Bigg(\sum_{i = 1}^{j - 1} f(h(i))\Bigg) + f(x) + \Bigg(\sum_{i = j}^n f(h(i + 1))\Bigg).
  \end{align*}
  We now define the function \(\tilde{h} : \set{i \in \N : 1 \leq i \leq n} \to X - \set{x}\) by setting \(\tilde{h}(i) \coloneqq h(i)\) when \(i < j\) and \(\tilde{h}(i) \coloneqq h(i + 1)\) when \(i \geq j\).
  We can thus write the right-hand side of \eqref{eq 7.1} as
  \[
    = \Bigg(\sum_{i = 1}^{j - 1} f(\tilde{h}(i))\Bigg) + f(x) + \Bigg(\sum_{i = j}^n f(\tilde{h}(i))\Bigg) = \Bigg(\sum_{i = 1}^n f(\tilde{h}(i))\Bigg) + f(x)
  \]
  where we have used \cref{7.1.4} once again.
  Thus to finish the proof of \eqref{eq 7.1} we have to show that
  \[
    \sum_{i = 1}^n f(g(i)) = \sum_{i = 1}^n f(\tilde{h}(i)). \tag{7.2}\label{eq 7.2}
  \]
  But the function \(g\) (when restricted to \(\set{i \in \N : 1 \leq i \leq n}\)) is a bijection from \(\set{i \in \N : 1 \leq i \leq n} \to X - \set{x}\).
  The function \(\tilde{h}\) is also a bijection from \(\set{i \in \N : 1 \leq i \leq n} \to X - \set{x}\) (cf. \cref{3.6.9}).
  Since \(X - \set{x}\) has \(n\) elements (by \cref{3.6.9}), the claim \eqref{eq 7.2} then follows directly from the induction hypothesis \(P(n)\).
\end{proof}

\begin{rmk}\label{7.1.9}
  The issue is somewhat more complicated when summing over infinite sets;
  See \cref{sec:8.2}.
\end{rmk}

\begin{rmk}\label{7.1.10}
  Suppose that \(X\) is a set, that \(P(x)\) is a property pertaining to an element \(x\) of \(X\), and \(f : \set{y \in X : P(y) \text{ is true}} \to \R\) is a function.
  Then we will often abbreviate
  \[
    \sum_{x \in \set{y \in X : P(y) \text{ is true}}} f(x)
  \]
  as \(\sum_{x \in X : P(x) \text{ is true}} f(x)\) or even as \(\sum_{P(x) \text{ is true}} f(x)\) when there is no change of confusion.
\end{rmk}

\begin{prop}[Basic properties of summation over finite sets]\label{7.1.11}
  \begin{enumerate}
    \item If \(X\) is empty, and \(f : X \to \R\) is a function (i.e., \(f\) is the empty function), we have
          \[
            \sum_{x \in X} f(x) = 0.
          \]
    \item If \(X\) consists of a single element, \(X = \set{x_0}\), and \(f : X \to \R\) is a function, we have
          \[
            \sum_{x \in X} f(x) = f(x_0).
          \]
    \item (Substitution, part I) If \(X\) is a finite set, \(f : X \to \R\) is a function, and \(g : Y \to X\) is a bijection, then
          \[
            \sum_{x \in X} f(x) = \sum_{y \in Y} f(g(y)).
          \]
    \item (Substitution, part II) Let \(n \leq m\) be integers, and let \(X\) be the set \(X \coloneqq \set{i \in \Z : n \leq i \leq m}\).
          If \(a_i\) is a real number assigned to each integer \(i \in X\), then we have
          \[
            \sum_{i = n}^m a_i = \sum_{i \in X} a_i.
          \]
    \item Let \(X, Y\) be disjoint finite sets (so \(X \cap Y = \emptyset\)), and \(f : X \cup Y \to \R\) is a function.
          Then we have
          \[
            \sum_{z \in X \cup Y} f(z) = \Bigg(\sum_{x \in X} f(x)\Bigg) + \Bigg(\sum_{y \in Y} f(y)\Bigg).
          \]
    \item (Linearity, part I) Let \(X\) be a finite set, and let \(f : X \to \R\) and \(g : X \to \R\) be functions.
          Then
          \[
            \sum_{x \in X} (f(x) + g(x)) = \sum_{x \in X} f(x) + \sum_{x \in X} g(x).
          \]
    \item (Linearity, part II) Let \(X\) be a finite set, let \(f : X \to \R\) be a function, and let \(c\) be a real number.
          Then
          \[
            \sum_{x \in X} cf(x) = c \sum_{x \in X} f(x).
          \]
    \item (Monotonicity) Let \(X\) be a finite set, and let \(f : X \to \R\) and \(g : X \to \R\) be functions such that \(f(x) \leq g(x)\) for all \(x \in \mathbf{X}\).
          Then we have
          \[
            \sum_{x \in X} f(x) \leq \sum_{x \in X} g(x).
          \]
    \item (Triangle inequality) Let \(X\) be a finite set, and let \(f : X \to \R\) be a function, then
          \[
            \abs{\sum_{x \in X} f(x)} \leq \sum_{x \in X} \abs{f(x)}.
          \]
  \end{enumerate}
\end{prop}

\begin{proof}{(a)}
  Let \(g : \set{i \in \N : 1 \leq i \leq 0} \to \emptyset\) be a function.
  Then \(g\) is a bijection and
  \begin{align*}
    \sum_{x \in X} f(x) & = \sum_{i = 1}^0 f(g(i)) &  & \by{7.1.6} \\
                        & = 0.                     &  & \by{7.1.1}
  \end{align*}
\end{proof}

\begin{proof}{(b)}
  Let \(g : \set{1} \to \set{x_0}\) be a function.
  Then \(g\) is a bijection and
  \begin{align*}
    \sum_{x \in X} f(x) & = \sum_{i = 1}^1 f(g(i))                       &  & \by{7.1.6} \\
                        & = \bigg(\sum_{i = 1}^0 f(g(i))\bigg) + f(g(1)) &  & \by{7.1.1} \\
                        & = 0 + f(g(1))                                  &  & \by{7.1.1} \\
                        & = f(x_0).
  \end{align*}
\end{proof}

\begin{proof}{(c)}
  Let \(h : \set{i \in \N : 1 \leq i \leq \#(Y)} \to Y\) be a bijection.
  Since \(X\) is finite and \(g\) is a bijection between \(X\) and \(Y\), we know that \(Y\) is finite and thus such \(h\) is well-defined.
  Then we know that \(g \circ h : \set{i \in \N : 1 \leq i \leq \#(Y)} \to X\) is also a bijection and
  \begin{align*}
    \sum_{x \in X} f(x) & = \sum_{i = 1}^{\#(Y)} f((g \circ h)(i)) &  & \by{7.1.6} \\
                        & = \sum_{i = 1}^{\#(Y)} f(g(h(i)))                        \\
                        & = \sum_{i = 1}^{\#(Y)} (f \circ g)(h(i))                 \\
                        & = \sum_{y \in Y} (f \circ g)(y)          &  & \by{7.1.6} \\
                        & = \sum_{y \in Y} f(g(y)).
  \end{align*}
\end{proof}

\begin{proof}{(d)}
  Let \(f : X \to \set{a_i \in \R : n \leq i \leq m}\) be a function where \(f = i \mapsto a_i\).
  Let \(g : \set{i \in \N : 1 \leq i \leq m - n + 1} \to X\) be a function where \(g = i \mapsto i + n - 1\).
  Then \(g\) is a bijection and
  \begin{align*}
    \sum_{i \in X} a_i & = \sum_{i \in X} f(i)                                                               \\
                       & = \sum_{i = 1}^{m - n + 1} f(g(i))                               &  & \by{7.1.6}    \\
                       & = \sum_{i = 1}^{m - n + 1} f(i + n - 1)                                             \\
                       & = \sum_{i = 1}^{m - n + 1} a_{i + n - 1}                                            \\
                       & = \sum_{i = 1 + n - 1}^{m - n + 1 + n - 1} a_{i + n - 1 - n + 1} &  & \by{7.1.4}[b] \\
                       & = \sum_{i = n}^m a_i.
  \end{align*}
\end{proof}

\begin{proof}{(e)}
  Let \(g : \set{i \in \N : 1 \leq i \leq \#(X)} \to X\) and \(h : \set{i \in \N : 1 \leq i \leq \#(Y)} \to Y\) be bijections.
  Since \(X, Y\) are finite, we know that \(g, h\) are well-defined and \(X \cup Y\) is finite.
  Let \(k : \set{i \in \N : 1 \leq i \leq \#(X \cup Y)} \to X \cup Y\) be a bijection where
  \[
    k(i) = \begin{dcases}
      g(i)         & \text{if } 1 \leq i \leq \#(X)                  \\
      h(i - \#(X)) & \text{if } \#(X) + 1 \leq i \leq \#(X) + \#(Y).
    \end{dcases}
  \]
  Since \(X \cup Y\) is finite, we know that \(k\) is well-defined and \(\#(X \cup Y) = \#(X) + \#(Y)\).
  Then we have
  \begin{align*}
    \sum_{z \in X \cup Y} f(z) & = \sum_{i = 1}^{\#(X \cup Y)} f(k(i))                                                &  & \by{7.1.6}    \\
                               & = \sum_{i = 1}^{\#(X)} f(k(i)) + \sum_{i = \#(X) + 1}^{\#(X \cup Y)} f(k(i))         &  & \by{7.1.4}[a] \\
                               & = \sum_{i = 1}^{\#(X)} f(g(i)) + \sum_{i = \#(X) + 1}^{\#(X \cup Y)} f(h(i - \#(X)))                    \\
                               & = \sum_{i = 1}^{\#(X)} f(g(i)) + \sum_{i = 1}^{\#(Y)} f(h(i))                        &  & \by{7.1.4}[b] \\
                               & = \sum_{x \in X} f(x) + \sum_{y \in Y} f(y).                                         &  & \by{7.1.6}
  \end{align*}
\end{proof}

\begin{proof}{(f)}
  Let \(h : \set{i \in \N : 1 \leq i \leq \#(X)} \to X\) be a bijection.
  Since \(X\) is finite, we know that \(h\) is well-defined and
  \begin{align*}
    \sum_{x \in X} (f(x) + g(x)) & = \sum_{x \in X} (f + g)(x)                                                      \\
                                 & = \sum_{i = 1}^{\#(X)} (f + g)(h(i))                          &  & \by{7.1.6}    \\
                                 & = \sum_{i = 1}^{\#(X)} (f(h(i)) + g(h(i)))                                       \\
                                 & = \sum_{i = 1}^{\#(X)} f(h(i)) + \sum_{i = 1}^{\#(X)} g(h(i)) &  & \by{7.1.4}[c] \\
                                 & = \sum_{x \in X} f(x) + \sum_{x \in X} g(x).                  &  & \by{7.1.6}
  \end{align*}
\end{proof}

\begin{proof}{(g)}
  Let \(g : \set{i \in \N : 1 \leq i \leq \#(X)} \to X\) be a bijection.
  Since \(X\) is finite, we know that \(g\) is well-defined and
  \begin{align*}
    \sum_{x \in X} cf(x) & = \sum_{x \in X} (cf)(x)                             \\
                         & = \sum_{i = 1}^{\#(X)} (cf)(g(i)) &  & \by{7.1.6}    \\
                         & = \sum_{i = 1}^{\#(X)} cf(g(i))                      \\
                         & = c \sum_{i = 1}^{\#(X)} f(g(i))  &  & \by{7.1.4}[d] \\
                         & = c \sum_{x \in X} f(x).          &  & \by{7.1.6}
  \end{align*}
\end{proof}

\begin{proof}{(h)}
  Let \(h : \set{i \in \N : 1 \leq i \leq \#(X)} \to X\) be a bijection.
  Since \(X\) is finite, we know that \(h\) is well-defined and
  \begin{align*}
    \sum_{x \in X} f(x) & = \sum_{i = 1}^{\#(X)} f(h(i))    &  & \by{7.1.6}    \\
                        & \leq \sum_{i = 1}^{\#(X)} g(h(i)) &  & \by{7.1.4}[f] \\
                        & = \sum_{x \in X} g(x).            &  & \by{7.1.6}
  \end{align*}
\end{proof}

\begin{proof}{(i)}
  Let \(g : \set{i \in \N : 1 \leq i \leq \#(X)} \to X\) be a bijection.
  Since \(X\) is finite, we know that \(g\) is well-defined and
  \begin{align*}
    \abs{\sum_{x \in X} f(x)} & = \abs{\sum_{i = 1}^{\#(X)} f(g(i))}    &  & \by{7.1.6}    \\
                              & \leq \sum_{i = 1}^{\#(X)} \abs{f(g(i))} &  & \by{7.1.4}[e] \\
                              & = \sum_{x \in X} \abs{f(x)}.            &  & \by{7.1.6}
  \end{align*}
\end{proof}

\begin{rmk}\label{7.1.12}
  The substitution rule in \cref{7.1.11}(c) can be thought of as making the substitution \(x \coloneqq g(y)\) (hence the name).
  Note that the assumption that \(g\) is a bijection is essential.
  From \cref{7.1.11}(c) and (d) we see that
  \[
    \sum_{i = n}^m a_i = \sum_{i = n}^m a_{f(i)}
  \]
  for any bijection \(f\) from the set \(\set{i \in \Z : n \leq i \leq m}\) to itself.
  Informally, this means that we can rearrange the elements of a finite sequence at will and still obtain the same value.
\end{rmk}

\begin{lem}\label{7.1.13}
  Let \(X, Y\) be finite sets, and let \(f : X \times Y \to \R\) be a function.
  Then
  \[
    \sum_{x \in X} \bigg(\sum_{y \in Y} f(x, y)\bigg) = \sum_{(x, y) \in X \times Y} f(x, y).
  \]
\end{lem}

\begin{proof}
  Let \(n\) be the number of elements in \(X\).
  We will use induction on \(n\) (cf. \cref{7.1.8});
  i.e., we let \(P(n)\) be the assertion that \cref{7.1.13} is true for any set \(X\) with \(n\) elements, and any finite set \(Y\) and any function \(f : X \times Y \to \R\).
  We wish to prove \(P(n)\) for all natural numbers \(n\).

  The base case \(P(0)\) is easy, following from \cref{7.1.11}(a).
  Now suppose that \(P(n)\) is true;
  we now show that \(P(n + 1)\) is true.
  Let \(X\) be a set with \(n + 1\) elements.
  In particular, by \cref{3.6.9}, we can write \(X = X' \cup \set{x_0}\), where \(x_0\) is an element of \(X\) and \(X' \coloneqq X - \set{x_0}\) has \(n\) elements.
  Then by \cref{7.1.11}(e) we have
  \[
    \sum_{x \in X} \bigg(\sum_{y \in Y} f(x, y)\bigg) = \sum_{x \in X'} \bigg(\sum_{y \in Y} f(x, y)\bigg) + \bigg(\sum_{y \in Y} f(x_0, y)\bigg);
  \]
  by the induction hypothesis this is equal to
  \[
    \sum_{(x, y) \in X' \times Y} f(x, y) + \bigg(\sum_{y \in Y} f(x_0, y)\bigg).
  \]
  By \cref{7.1.11}(c) this is equal to
  \[
    \sum_{(x, y) \in X' \times Y} f(x, y) + \bigg(\sum_{(x, y) \in \set{x_0} \times Y} f(x, y)\bigg).
  \]
  By \cref{7.1.11}(e) this is equal to
  \[
    \sum_{(x, y) \in X \times Y} f(x, y)
  \]
  as desired.
\end{proof}

\begin{cor}[Fubini's theorem for finite series]\label{7.1.14}
  Let \(X, Y\) be finite sets, and let \(f : X \times Y \to \R\) be a function.
  Then
  \begin{align*}
    \sum_{x \in X} \bigg(\sum_{y \in Y} f(x, y)\bigg) & = \sum_{(x, y) \in X \times Y} f(x, y)               \\
                                                      & = \sum_{(y, x) \in Y \times X} f(x, y)               \\
                                                      & = \sum_{y \in Y} \bigg(\sum_{x \in X} f(x, y)\bigg).
  \end{align*}
\end{cor}

\begin{proof}
  In light of \cref{7.1.13}, it suffices to show that
  \[
    \sum_{(x, y) \in X \times Y} f(x, y) = \sum_{(y, x) \in Y \times X} f(x, y).
  \]
  But this follows from \cref{7.1.11}(c) by applying the bijection \(h : Y \times X \to X \times Y\) defined by \(h(y, x) \coloneqq (x, y)\).
\end{proof}

\begin{rmk}\label{7.1.15}
  We anticipate something interesting to happen when we move from finite sums to infinite sums.
  However, see \cref{8.2.2}.
\end{rmk}

\begin{ac}[Products over finite sets]\label{ac:7.1.1}
  Let \(m, n\) be integers, and let \((a_i)_{i = m}^n\) be a finite sequence of real numbers, assigning a real number \(a_i\) to each integer \(i\) between \(m\) and \(n\) inclusive (i.e., \(m \leq i \leq n\)).
  Then we define the finite product \(\prod_{i = m}^n a_i\) by the recursive formula
  \begin{align*}
     & \prod_{i = m}^n a_i \coloneqq 1 \text{ whenever } n < m;                                                             \\
     & \prod_{i = m}^{n + 1} a_i \coloneqq \Bigg(\prod_{i = m}^n a_i\Bigg) \times a_{n + 1} \text{ whenever } n \geq m - 1.
  \end{align*}
\end{ac}

\begin{ac}\label{ac:7.1.2}
  \begin{enumerate}
    \item Let \(m \leq n < p\) be integers, and let \(a_i\) be a real number assigned to each integer \(m \leq i \leq p\).
          Then we have
          \[
            \prod_{i = m}^n a_i \times \prod_{i = n + 1}^p a_i = \prod_{i = m}^p a_i.
          \]
    \item Let \(m \leq n\) be integers, \(k\) be another integer, and let \(a_i\) be a real number assigned to each integer \(m \leq i \leq n\).
          Then we have
          \[
            \prod_{i = m}^n a_i = \prod_{j = m + k}^{n + k} a_{j - k}.
          \]
    \item Let \(m \leq n\) be integers, and let \(a_i, b_i\) be real numbers assigned to each integer \(m \leq i \leq n\).
          Then we have
          \[
            \prod_{i = m}^n (a_i \times b_i) = \Bigg(\prod_{i = m}^n a_i\Bigg) \times \Bigg(\prod_{i = m}^n b_i\Bigg).
          \]
    \item Let \(m \leq n\) be integers, and let \(a_i\) be a real number assigned to each integer \(m \leq i \leq n\), and let \(c\) be another real number.
          Then we have
          \[
            \prod_{i = m}^n (ca_i) = c^{n - m + 1} \Bigg(\prod_{i = m}^n a_i\Bigg).
          \]
    \item Let \(m \leq n\) be integers, and let \(a_i\) be a real number assigned to each integer \(m \leq i \leq n\).
          Then we have
          \[
            \abs{\prod_{i = m}^n a_i} = \prod_{i = m}^n \abs{a_i}.
          \]
  \end{enumerate}
\end{ac}

\begin{proof}{(a)}
  Let \(k = p - m\).
  By hypothesis we know that \(k > 0\).
  Now we use induction on \(k\) to show that \cref{ac:7.1.2}(a) is true and we start with \(k = 1\).
  For \(k = 1\), we have \(p = m + 1\) and by \cref{ac:7.1.1} we have
  \[
    \prod_{i = m}^n a_i \times \prod_{i = n + 1}^p a_i = \prod_{i = m}^m a_i \times \prod_{i = m + 1}^p a_i = a_m \times a_{m + 1} = \prod_{i = m}^p a_i.
  \]
  Thus the base case holds.
  Suppose inductively that for some \(k \geq 1\) \cref{ac:7.1.2}(a) is true.
  Then for \(k + 1 = p - m\), we have \(p - 1 = k + m\) and
  \begin{align*}
     & \prod_{i = m}^n a_i \times \prod_{i = n + 1}^p a_i                                                               \\
     & = \Bigg(\prod_{i = m}^n a_i\Bigg) \times \Bigg(\prod_{i = n + 1}^{p - 1} a_i\Bigg) \times a_p &  & \by{ac:7.1.1} \\
     & = \Bigg(\prod_{i = m}^{p - 1} a_i\Bigg) \times a_p                                            &  & \byIH         \\
     & = \prod_{i = m}^p a_i.                                                                        &  & \by{ac:7.1.1}
  \end{align*}
  This closes the induction.
\end{proof}

\begin{proof}{(b)}
  Let \(p = n - m\).
  By hypothesis we know that \(p \geq 0\).
  Now we use induction on \(p\) to show that \cref{ac:7.1.2}(b) is true.
  For \(p = 0\), we have \(n = m\) and
  \begin{align*}
    \prod_{j = m + k}^{m + k} a_{j - k} & = \Bigg(\prod_{j = m + k}^{m + k - 1} a_{j - k}\Bigg) \times a_{m + k - k} &  & \by{ac:7.1.1} \\
                                        & = 1 \times a_{m + k - k}                                                   &  & \by{ac:7.1.1} \\
                                        & = 1 \times a_m                                                                                \\
                                        & = \Bigg(\prod_{i = m}^{m - 1} a_i\Bigg) \times a_m                         &  & \by{ac:7.1.1} \\
                                        & = \prod_{i = m}^m a_i.                                                     &  & \by{ac:7.1.1}
  \end{align*}
  So the base case holds.
  Suppose inductively that for some \(p \geq 0\) \cref{ac:7.1.2}(b) is true.
  Then for \(p + 1 = n - m\), we have \(p = n - m - 1\) and
  \begin{align*}
    \prod_{j = m + k}^{n + k} a_{j - k} & = \Bigg(\prod_{j = m + k}^{n + k - 1} a_{j - k}\Bigg) \times a_{n + k - k} &  & \by{ac:7.1.1} \\
                                        & = \Bigg(\prod_{j = m + k}^{n + k - 1} a_{j - k}\Bigg) \times a_n                              \\
                                        & = \Bigg(\prod_{i = m}^{n - 1} a_i\Bigg) \times a_n                         &  & \byIH         \\
                                        & = \prod_{i = m}^n a_i.                                                     &  & \by{ac:7.1.1}
  \end{align*}
  This closes the induction.
\end{proof}

\begin{proof}{(c)}
  Let \(p = n - m\).
  By hypothesis we know that \(p \geq 0\).
  Now we use induction on \(p\) to show that \cref{ac:7.1.2}(c) is true.
  For \(p = 0\), we have \(n = m\) and
  \begin{align*}
    \prod_{i = m}^m (a_i \times b_i) & = \Bigg(\prod_{i = m}^{m - 1} (a_i \times b_i)\Bigg) \times a_m \times b_m                                 &  & \by{ac:7.1.1} \\
                                     & = 1 \times a_m \times b_m                                                                                  &  & \by{ac:7.1.1} \\
                                     & = \Bigg(\prod_{i = m}^{m - 1} a_i\Bigg) \times \Bigg(\prod_{i = m}^{m - 1} b_i\Bigg) \times a_m \times b_m &  & \by{ac:7.1.1} \\
                                     & = \Bigg(\prod_{i = m}^m a_i\Bigg) \times \Bigg(\prod_{i = m}^m b_i\Bigg).                                  &  & \by{ac:7.1.1}
  \end{align*}
  So the base case holds.
  Suppose inductively that for some \(p \geq 0\) \cref{ac:7.1.2}(c) is true.
  Then for \(p + 1 = n - m\), we have \(p = n - m - 1\) and
  \begin{align*}
    \prod_{i = m}^n (a_i \times b_i) & = \Bigg(\prod_{i = m}^{n - 1} (a_i \times b_i)\Bigg) \times a_n \times b_n                                 &  & \by{ac:7.1.1} \\
                                     & = \Bigg(\prod_{i = m}^{n - 1} a_i\Bigg) \times \Bigg(\prod_{i = m}^{n - 1} b_i\Bigg) \times a_n \times b_n &  & \byIH         \\
                                     & = \Bigg(\prod_{i = m}^n a_i\Bigg) \times \Bigg(\prod_{i = m}^n b_i\Bigg).                                  &  & \by{ac:7.1.1}
  \end{align*}
  This closes the induction.
\end{proof}

\begin{proof}{(d)}
  Let \(p = n - m\).
  By hypothesis we know that \(p \geq 0\).
  Now we use induction on \(p\) to show that \cref{ac:7.1.2}(d) is true.
  For \(p = 0\), we have \(n = m\) and
  \begin{align*}
    \prod_{i = m}^m ca_i & = \Bigg(\prod_{i = m}^{m - 1} ca_i\Bigg) \times ca_m &  & \by{ac:7.1.1} \\
                         & = 1 \times ca_m                                      &  & \by{ac:7.1.1} \\
                         & = c \times a_m                                                          \\
                         & = c \Bigg(\prod_{i = m}^m a_i\Bigg)                  &  & \by{ac:7.1.1} \\
                         & = c^{m - m + 1} \Bigg(\prod_{i = m}^m a_i\Bigg).
  \end{align*}
  So the base case holds.
  Suppose inductively that for some \(p \geq 0\) \cref{ac:7.1.2}(d) is true.
  Then for \(p + 1 = n - m\), we have \(p = n - m - 1\) and
  \begin{align*}
    \prod_{i = m}^n ca_i & = \Bigg(\prod_{i = m}^{n - 1} ca_i\Bigg) \times ca_n                         &  & \by{ac:7.1.1} \\
                         & = c^{n - 1 - m + 1} \Bigg(\prod_{i = m}^{n - 1} a_i\Bigg) \times ca_n        &  & \byIH         \\
                         & = c^{n - m + 1} \Bigg(\Bigg(\prod_{i = m}^{n - 1} a_i\Bigg) \times a_n\Bigg)                    \\
                         & = c^{n - m + 1} \Bigg(\prod_{i = m}^n a_i\Bigg).                             &  & \by{ac:7.1.1}
  \end{align*}
  This closes the induction.
\end{proof}

\begin{proof}{(e)}
  Let \(p = n - m\).
  By hypothesis we know that \(p \geq 0\).
  Now we use induction on \(p\) to show that \cref{ac:7.1.2}(e) is true.
  For \(p = 0\), we have \(n = m\) and
  \begin{align*}
    \abs{\prod_{i = m}^m a_i} & = \abs{\Bigg(\prod_{i = m}^{m - 1} a_i\Bigg) \times a_m}       &  & \by{ac:7.1.1} \\
                              & = \abs{1a_m}                                                   &  & \by{ac:7.1.1} \\
                              & = \abs{1}\abs{a_m}                                                                \\
                              & = \Bigg(\prod_{i = m}^{m - 1} \abs{a_i}\Bigg) \times \abs{a_m} &  & \by{ac:7.1.1} \\
                              & = \prod_{i = m}^m \abs{a_i}.                                   &  & \by{ac:7.1.1}
  \end{align*}
  So the base case holds.
  Suppose inductively that for some \(p \geq 0\) \cref{ac:7.1.2}(e) is true.
  Then for \(p + 1 = n - m\), we have \(p = n - m - 1\) and
  \begin{align*}
    \abs{\prod_{i = m}^n a_i} & = \abs{\Bigg(\prod_{i = m}^{n - 1} a_i\Bigg) \times a_n}       &  & \by{ac:7.1.1} \\
                              & = \abs{\prod_{i = m}^{n - 1} a_i} \times \abs{a_n}                                \\
                              & = \Bigg(\prod_{i = m}^{n - 1} \abs{a_i}\Bigg) \times \abs{a_n} &  & \byIH         \\
                              & = \prod_{i = m}^n \abs{a_i}.                                   &  & \by{ac:7.1.1}
  \end{align*}
  This closes the induction.
\end{proof}

\begin{ac}\label{ac:7.1.3}
  Let \(X\) be a finite set with \(n\) elements (where \(n \in \N\)), and let \(f : X \to \R\) be a function from \(X\) to the real numbers
  (i.e., \(f\) assigns a real number \(f(x)\) to each element \(x\) of \(X\)).
  Then we can define the finite product \(\prod_{x \in X} f(x)\) as follows.
  We first select any bijection \(g\) from \(\set{i \in \N : 1 \leq i \leq n}\) to \(X\);
  such a bijection exists since \(X\) is assumed to have \(n\) elements.
  We then define
  \[
    \prod_{x \in X} f(x) \coloneqq \prod_{i = 1}^n f(g(i))
  \]
\end{ac}

\begin{ac}[Finite products are well-defined]\label{ac:7.1.4}
  Let \(X\) be a finite set with \(n\) elements (where \(n \in \N\)), let \(f : X \to \R\) be a function, and let \(g : \set{i \in \N : 1 \leq i \leq n} \to X\) and \(h : \set{i \in \N : 1 \leq i \leq n} \to X\) be bijections.
  Then we have
  \[
    \prod_{i = 1}^n f(g(i)) = \prod_{i = 1}^n f(h(i)).
  \]
\end{ac}

\begin{proof}
  Let \(P(n)\) be the assertion that ``For any set \(X\) of \(n\) elements, any function \(f : X \to \R\), and any two bijections \(g, h\) from \(\set{i \in \N : 1 \leq i \leq n}\) to \(X\), we have \(\prod_{i = 1}^n f(g(i)) = \prod_{i = 1}^n f(h(i))\)''.
  (More informally, \(P(n)\) is the assertion that \cref{ac:7.1.4} is true for that value of \(n\).)
  We use induction on \(n\);

  We first check the base case \(P(0)\).
  In this case \(\prod_{i = 1}^0 f(g(i))\) and \(\prod_{i = 1}^0 f(h(i))\) both equal to \(1\), by \cref{ac:7.1.1}, so we are done.

  Now suppose inductively that \(P(n)\) is true;
  we now prove that \(P(n + 1)\) is true.
  Thus, let \(X\) be a set with \(n + 1\) elements, let \(f : X \to \R\) be a function, and let \(g\) and \(h\) be bijections from \(\set{i \in N : 1 \leq i \leq n + 1}\) to \(X\).
  We have to prove that
  \[
    \prod_{i = 1}^{n + 1} f(g(i)) = \prod_{i = 1}^{n + 1} f(h(i)). \tag{ac 7.1}\label{eq ac 7.1}
  \]
  Let \(x \coloneqq g(n + 1)\);
  thus \(x\) is an element of \(X\).
  By \cref{ac:7.1.1}, we can expand the left-hand side of \eqref{eq ac 7.1} as
  \[
    \prod_{i = 1}^{n + 1} f(g(i)) = \Bigg(\prod_{i = 1}^n f(g(i))\Bigg) \times f(x).
  \]
  Now let us look at the right-hand side of \eqref{eq ac 7.1}.
  Since \(h\) is a bijection, we do know that there is \emph{some} index \(j\), with \(1 \leq j \leq n + 1\), for which \(h(j) = x\).
  We now use \cref{ac:7.1.1,ac:7.1.2} to write
  \begin{align*}
    \prod_{i = 1}^{n + 1} f(h(i)) & = \Bigg(\prod_{i = 1}^j f(h(i))\Bigg) \times \Bigg(\prod_{i = j + 1}^{n + 1} f(h(i))\Bigg)                      \\
                                  & = \Bigg(\prod_{i = 1}^{j - 1} f(h(i))\Bigg) \times f(h(j)) \times \Bigg(\prod_{i = j + 1}^{n + 1} f(h(i))\Bigg) \\
                                  & = \Bigg(\prod_{i = 1}^{j - 1} f(h(i))\Bigg) \times f(x) \times \Bigg(\prod_{i = j}^n f(h(i + 1))\Bigg).
  \end{align*}
  We now define the function \(\tilde{h} : \set{i \in \N : 1 \leq i \leq n} \to X - \set{x}\) by setting \(\tilde{h}(i) \coloneqq h(i)\) when \(i < j\) and \(\tilde{h}(i) \coloneqq h(i + 1)\) when \(i \geq j\).
  We can thus write the right-hand side of \eqref{eq ac 7.1} as
  \[
    = \Bigg(\prod_{i = 1}^{j - 1} f(\tilde{h}(i))\Bigg) \times f(x) \times \Bigg(\prod_{i = j}^n f(\tilde{h}(i))\Bigg) = \Bigg(\prod_{i = 1}^n f(\tilde{h}(i))\Bigg) \times f(x)
  \]
  where we have used \cref{ac:7.1.2} once again.
  Thus to finish the proof of \eqref{eq ac 7.1} we have to show that
  \[
    \prod_{i = 1}^n f(g(i)) = \prod_{i = 1}^n f(\tilde{h}(i)). \tag{ac 7.2}\label{eq ac 7.2}
  \]
  But the function \(g\) (when restricted to \(\set{i \in \N : 1 \leq i \leq n}\)) is a bijection from \(\set{i \in \N : 1 \leq i \leq n} \to X - \set{x}\).
  The function \(\tilde{h}\) is also a bijection from \(\set{i \in \N : 1 \leq i \leq n} \to X - \set{x}\) (cf. \cref{3.6.9}).
  Since \(X - \set{x}\) has \(n\) elements (by \cref{3.6.9}), the claim \eqref{eq ac 7.2} then follows directly from the induction hypothesis \(P(n)\).
\end{proof}

\begin{ac}[Basic properties of product over finite sets]\label{ac:7.1.5}
  \begin{enumerate}
    \item If \(X\) is empty, and \(f : X \to \R\) is a function (i.e., \(f\) is the empty function), we have
          \[
            \prod_{x \in X} f(x) = 1.
          \]
    \item If \(X\) consists of a single element, \(X = \set{x_0}\), and \(f : X \to \R\) is a function, we have
          \[
            \prod_{x \in X} f(x) = f(x_0).
          \]
    \item (Substitution, part I) If \(X\) is a finite set, \(f : X \to \R\) is a function, and \(g : Y \to X\) is a bijection, then
          \[
            \prod_{x \in X} f(x) = \prod_{y \in Y} f(g(y)).
          \]
    \item (Substitution, part II) Let \(n \leq m\) be integers, and let \(X\) be the set \(X \coloneqq \set{i \in \Z : n \leq i \leq m}\).
          If \(a_i\) is a real number assigned to each integer \(i \in X\), then we have
          \[
            \prod_{i = n}^m a_i = \prod_{i \in X} a_i.
          \]
    \item Let \(X, Y\) be disjoint finite sets (so \(X \cap Y = \emptyset\)), and \(f : X \cup Y \to \R\) is a function.
          Then we have
          \[
            \prod_{z \in X \cup Y} f(z) = \Bigg(\prod_{x \in X} f(x)\Bigg) \times \Bigg(\prod_{y \in Y} f(y)\Bigg).
          \]
    \item Let \(X\) be a finite set, and let \(f : X \to \R\) and \(g : X \to \R\) be functions.
          Then
          \[
            \prod_{x \in X} (f(x) \times g(x)) = \prod_{x \in X} f(x) \times \prod_{x \in X} g(x).
          \]
    \item Let \(X\) be a finite set, let \(f : X \to \R\) be a function, and let \(c\) be a real number.
          Then
          \[
            \prod_{x \in X} cf(x) = c^{\#(X)} \prod_{x \in X} f(x).
          \]
    \item Let \(X\) be a finite set, and let \(f : X \to \R\) be a function, then
          \[
            \abs{\prod_{x \in X} f(x)} = \prod_{x \in X} \abs{f(x)}.
          \]
  \end{enumerate}
\end{ac}

\begin{proof}{(a)}
  Let \(g : \set{i \in \N : 1 \leq i \leq 0} \to \emptyset\) be a function.
  Then \(g\) is a bijection and
  \begin{align*}
    \prod_{x \in X} f(x) & = \prod_{i = 1}^0 f(g(i)) &  & \by{ac:7.1.3} \\
                         & = 1.                      &  & \by{ac:7.1.1}
  \end{align*}
\end{proof}

\begin{proof}{(b)}
  Let \(g : \set{1} \to \set{x_0}\) be a function.
  Then \(g\) is a bijection and
  \begin{align*}
    \prod_{x \in X} f(x) & = \prod_{i = 1}^1 f(g(i))                            &  & \by{ac:7.1.3} \\
                         & = \bigg(\prod_{i = 1}^0 f(g(i))\bigg) \times f(g(1)) &  & \by{ac:7.1.1} \\
                         & = 1 \times f(g(1))                                   &  & \by{ac:7.1.1} \\
                         & = f(x_0).
  \end{align*}
\end{proof}

\begin{proof}{(c)}
  Let \(h : \set{i \in \N : 1 \leq i \leq \#(Y)} \to Y\) be a bijection.
  Since \(X\) is finite and \(g\) is a bijection between \(X\) and \(Y\), we know that \(Y\) is finite and thus such \(h\) is well-defined.
  Then we know that \(g \circ h : \set{i \in \N : 1 \leq i \leq \#(Y)} \to X\) is also a bijection and
  \begin{align*}
    \prod_{x \in X} f(x) & = \prod_{i = 1}^{\#(Y)} f((g \circ h)(i)) &  & \by{ac:7.1.3} \\
                         & = \prod_{i = 1}^{\#(Y)} f(g(h(i)))                           \\
                         & = \prod_{i = 1}^{\#(Y)} (f \circ g)(h(i))                    \\
                         & = \prod_{y \in Y} (f \circ g)(y)          &  & \by{ac:7.1.3} \\
                         & = \prod_{y \in Y} f(g(y)).
  \end{align*}
\end{proof}

\begin{proof}{(d)}
  Let \(f : X \to \set{a_i \in \R : n \leq i \leq m}\) be a function where \(f = i \mapsto a_i\).
  Let \(g : \set{i \in \N : 1 \leq i \leq m - n + 1} \to X\) be a function where \(g = i \mapsto i + n - 1\).
  Then \(g\) is a bijection and
  \begin{align*}
    \prod_{i \in X} a_i & = \prod_{i \in X} f(i)                                                                  \\
                        & = \prod_{i = 1}^{m - n + 1} f(g(i))                               &  & \by{ac:7.1.3}    \\
                        & = \prod_{i = 1}^{m - n + 1} f(i + n - 1)                                                \\
                        & = \prod_{i = 1}^{m - n + 1} a_{i + n - 1}                                               \\
                        & = \prod_{i = 1 + n - 1}^{m - n + 1 + n - 1} a_{i + n - 1 - n + 1} &  & \by{ac:7.1.2}[b] \\
                        & = \prod_{i = n}^m a_i.
  \end{align*}
\end{proof}

\begin{proof}{(e)}
  Let \(g : \set{i \in \N : 1 \leq i \leq \#(X)} \to X\) and \(h : \set{i \in \N : 1 \leq i \leq \#(Y)} \to Y\) be bijections.
  Since \(X, Y\) are finite, we know that \(g, h\) are well-defined and \(X \cup Y\) is finite.
  Let \(k : \set{i \in \N : 1 \leq i \leq \#(X \cup Y)} \to X \cup Y\) be a bijection where
  \[
    k(i) = \begin{dcases}
      g(i)         & \text{if } 1 \leq i \leq \#(X)                  \\
      h(i - \#(X)) & \text{if } \#(X) + 1 \leq i \leq \#(X) + \#(Y).
    \end{dcases}
  \]
  Since \(X \cup Y\) is finite, we know that \(k\) is well-defined and \(\#(X \cup Y) = \#(X) + \#(Y)\).
  Then we have
  \begin{align*}
     & \prod_{z \in X \cup Y} f(z)                                                                                       \\
     & = \prod_{i = 1}^{\#(X \cup Y)} f(k(i))                                                      &  & \by{ac:7.1.3}    \\
     & = \prod_{i = 1}^{\#(X)} f(k(i)) \times \prod_{i = \#(X) + 1}^{\#(X \cup Y)} f(k(i))         &  & \by{ac:7.1.2}[a] \\
     & = \prod_{i = 1}^{\#(X)} f(g(i)) \times \prod_{i = \#(X) + 1}^{\#(X \cup Y)} f(h(i - \#(X)))                       \\
     & = \prod_{i = 1}^{\#(X)} f(g(i)) \times \prod_{i = 1}^{\#(Y)} f(h(i))                        &  & \by{ac:7.1.2}[b] \\
     & = \prod_{x \in X} f(x) \times \prod_{y \in Y} f(y).                                         &  & \by{ac:7.1.3}
  \end{align*}
\end{proof}

\begin{proof}{(f)}
  Let \(h : \set{i \in \N : 1 \leq i \leq \#(X)} \to X\) be a bijection.
  Since \(X\) is finite, we know that \(h\) is well-defined and
  \begin{align*}
     & \prod_{x \in X} (f(x) \times g(x))                                                         \\
     & = \prod_{x \in X} (f \times g)(x)                                                          \\
     & = \prod_{i = 1}^{\#(X)} (f \times g)(h(i))                           &  & \by{ac:7.1.3}    \\
     & = \prod_{i = 1}^{\#(X)} (f(h(i)) \times g(h(i)))                                           \\
     & = \prod_{i = 1}^{\#(X)} f(h(i)) \times \prod_{i = 1}^{\#(X)} g(h(i)) &  & \by{ac:7.1.2}[c] \\
     & = \prod_{x \in X} f(x) \times \prod_{x \in X} g(x).                  &  & \by{ac:7.1.3}
  \end{align*}
\end{proof}

\begin{proof}{(g)}
  Let \(g : \set{i \in \N : 1 \leq i \leq \#(X)} \to X\) be a bijection.
  Since \(X\) is finite, we know that \(g\) is well-defined and
  \begin{align*}
    \prod_{x \in X} cf(x) & = \prod_{x \in X} (cf)(x)                                       \\
                          & = \prod_{i = 1}^{\#(X)} (cf)(g(i))        &  & \by{ac:7.1.3}    \\
                          & = \prod_{i = 1}^{\#(X)} cf(g(i))                                \\
                          & = c^{\#(X)} \prod_{i = 1}^{\#(X)} f(g(i)) &  & \by{ac:7.1.2}[d] \\
                          & = c^{\#(X)} \prod_{x \in X} f(x).         &  & \by{ac:7.1.3}
  \end{align*}
\end{proof}

\begin{proof}{(h)}
  Let \(g : \set{i \in \N : 1 \leq i \leq \#(X)} \to X\) be a bijection.
  Since \(X\) is finite, we know that \(g\) is well-defined and
  \begin{align*}
    \abs{\prod_{x \in X} f(x)} & = \abs{\prod_{i = 1}^{\#(X)} f(g(i))} &  & \by{ac:7.1.3}    \\
                               & = \prod_{i = 1}^{\#(X)} \abs{f(g(i))} &  & \by{ac:7.1.2}[e] \\
                               & = \prod_{x \in X} \abs{f(x)}.         &  & \by{ac:7.1.3}
  \end{align*}
\end{proof}

\begin{ac}\label{ac:7.1.6}
  Let \(X, Y\) be finite sets, and let \(f : X \times Y \to \R\) be a function.
  Then
  \[
    \prod_{x \in X} \bigg(\prod_{y \in Y} f(x, y)\bigg) = \prod_{(x, y) \in X \times Y} f(x, y).
  \]
\end{ac}

\begin{proof}
  Let \(n\) be the number of elements in \(X\).
  We will use induction on \(n\) (cf. \cref{ac:7.1.4});
  i.e., we let \(P(n)\) be the assertion that \cref{ac:7.1.6} is true for any set \(X\) with \(n\) elements, and any finite set \(Y\) and any function \(f : X \times Y \to \R\).
  We wish to prove \(P(n)\) for all natural numbers \(n\).

  The base case \(P(0)\) is easy, following from \cref{ac:7.1.5}(a).
  Now suppose that \(P(n)\) is true;
  we now show that \(P(n + 1)\) is true.
  Let \(X\) be a set with \(n + 1\) elements.
  In particular, by \cref{3.6.9}, we can write \(X = X' \cup \set{x_0}\), where \(x_0\) is an element of \(X\) and \(X' \coloneqq X - \set{x_0}\) has \(n\) elements.
  Then by \cref{7.1.5}(e) we have
  \[
    \prod_{x \in X} \bigg(\prod_{y \in Y} f(x, y)\bigg) = \prod_{x \in X'} \bigg(\prod_{y \in Y} f(x, y)\bigg) \times \bigg(\prod_{y \in Y} f(x_0, y)\bigg);
  \]
  by the induction hypothesis this is equal to
  \[
    \prod_{(x, y) \in X' \times Y} f(x, y) \times \bigg(\prod_{y \in Y} f(x_0, y)\bigg).
  \]
  By \cref{7.1.11}(c) this is equal to
  \[
    \prod_{(x, y) \in X' \times Y} f(x, y) \times \bigg(\prod_{(x, y) \in \set{x_0} \times Y} f(x, y)\bigg).
  \]
  By \cref{7.1.11}(e) this is equal to
  \[
    \prod_{(x, y) \in X \times Y} f(x, y)
  \]
  as desired.
\end{proof}

\begin{ac}\label{ac:7.1.7}
  Let \(X, Y\) be finite sets, and let \(f : X \times Y \to \R\) be a function.
  Then
  \begin{align*}
    \prod_{x \in X} \bigg(\prod_{y \in Y} f(x, y)\bigg) & = \prod_{(x, y) \in X \times Y} f(x, y)                \\
                                                        & = \prod_{(y, x) \in Y \times X} f(x, y)                \\
                                                        & = \prod_{y \in Y} \bigg(\prod_{x \in X} f(x, y)\bigg).
  \end{align*}
\end{ac}

\begin{proof}
  In light of \cref{ac:7.1.6}, it suffices to show that
  \[
    \prod_{(x, y) \in X \times Y} f(x, y) = \prod_{(y, x) \in Y \times X} f(x, y).
  \]
  But this follows from \cref{ac:7.1.5}(c) by applying the bijection \(h : Y \times X \to X \times Y\) defined by \(h(y, x) \coloneqq (x, y)\).
\end{proof}

\exercisesection

\begin{ex}\label{ex:7.1.1}
  Prove \cref{7.1.4}.
\end{ex}

\begin{proof}
  See \cref{7.1.4}.
\end{proof}

\begin{ex}\label{ex:7.1.2}
  Prove \cref{7.1.11}.
\end{ex}

\begin{proof}
  See \cref{7.1.11}.
\end{proof}

\begin{ex}\label{ex:7.1.3}
  Form a definition for the finite products \(\prod_{i = 1}^n a_i\) and \(\prod_{x \in X} f(x)\).
  Which of the above result for finite series have analoges for finite products?
\end{ex}

\begin{proof}
  See \crefrange{ac:7.1.1}{ac:7.1.7}.
\end{proof}

\begin{ex}\label{ex:7.1.4}
  Define the \emph{factorial function} \(n!\) for natural numbers \(n\) by the recursive definition \(0! \coloneqq 1\) and \((n + 1)! \coloneqq n! \times (n + 1)\).
  If \(x\) and \(y\) are real numbers, prove the \emph{binomial formula}
  \[
    (x + y)^n = \sum_{j = 0}^n \dfrac{n!}{j!(n - j)!} x^j y^{n - j}
  \]
  for all natural numbers \(n\).
\end{ex}

\begin{proof}
  We use induction on \(n\).
  For \(n = 0\), we have
  \begin{align*}
    (x + y)^0 & = 1                                                                                                                         \\
              & = \dfrac{0!}{0!(0 - 0)!} x^0 y^{0 - 0}                                                          &  & \text{(by definition)} \\
              & = \sum_{j = 0}^{-1} \dfrac{0!}{j!(0 - j)!} x^j y^{0 - j} + \dfrac{0!}{0!(0 - 0)!} x^0 y^{0 - 0} &  & \by{7.1.1}             \\
              & = \sum_{j = 0}^0 \dfrac{0!}{j!(0 - j)!} x^j y^{0 - j}                                           &  & \by{7.1.1}
  \end{align*}
  So the base case holds.
  Suppose inductively that for some \(n \geq 0\) the statement holds.
  Then for \(n + 1\), we have
  \begin{align*}
    (x + y)^{n + 1} & = (x + y)^n \times (x + y)                                                                                              \\
                    & = \bigg(\sum_{j = 0}^n \dfrac{n!}{j!(n - j)!} x^j y^{n - j}\bigg) \times (x + y)            &  & \byIH                  \\
                    & = \bigg(\sum_{j = 0}^n \dfrac{n!}{j!(n - j)!} x^{j + 1} y^{n - j}\bigg)                                                 \\
                    & \quad + \bigg(\sum_{j = 0}^n \dfrac{n!}{j!(n - j)!} x^j y^{n + 1 - j}\bigg)                                             \\
                    & = \bigg(\sum_{j = 0}^{n - 1} \dfrac{n!}{j!(n - j)!} x^{j + 1} y^{n - j}\bigg)               &  & \by{7.1.1}             \\
                    & \quad + \bigg(\dfrac{n!}{n!0!} x^{n + 1} y^0\bigg)                                                                      \\
                    & \quad + \bigg(\sum_{j = 1}^n \dfrac{n!}{j!(n - j)!} x^j y^{n + 1 - j}\bigg)                                             \\
                    & \quad + \bigg(\dfrac{n!}{0!n!} x^0 y^{n + 1}\bigg)                                                                      \\
                    & = \bigg(\sum_{j = 0}^{n - 1} \dfrac{n!}{j!(n - j)!} x^{j + 1} y^{n - j}\bigg) + x^{n + 1}   &  & \text{(by definition)} \\
                    & \quad + \bigg(\sum_{j = 1}^n \dfrac{n!}{j!(n - j)!} x^j y^{n + 1 - j}\bigg) + y^{n + 1}                                 \\
                    & = \bigg(\sum_{j = 1}^n \dfrac{n!}{(j - 1)!(n + 1 - j)!} x^j y^{n + 1 - j}\bigg) + x^{n + 1} &  & \by{7.1.4}[b]          \\
                    & \quad + \bigg(\sum_{j = 1}^n \dfrac{n!}{j!(n - j)!} x^j y^{n + 1 - j}\bigg) + y^{n + 1}
  \end{align*}
  and
  \begin{align*}
     & \bigg(\sum_{j = 1}^n \dfrac{n!}{(j - 1)!(n + 1 - j)!} x^j y^{n + 1 - j}\bigg)                                                                                           \\
     & \quad + \bigg(\sum_{j = 1}^n \dfrac{n!}{j!(n - j)!} x^j y^{n + 1 - j}\bigg)                                                                                             \\
     & = \sum_{j = 1}^n \bigg(\dfrac{n!}{(j - 1)!(n + 1 - j)!} x^j y^{n + 1 - j} + \dfrac{n!}{j!(n - j)!} x^j y^{n + 1 - j}\bigg)                           &  & \by{7.1.4}[c] \\
     & = \sum_{j = 1}^n \bigg(\dfrac{j \times n!}{j!(n + 1 - j)!} x^j y^{n + 1 - j} + \dfrac{(n + 1 - j) \times n!}{j!(n + 1 - j)!} x^j y^{n + 1 - j}\bigg)                    \\
     & = \sum_{j = 1}^n \bigg(\dfrac{j \times n! + (n + 1 - j) \times n!}{j!(n + 1 - j)!} x^j y^{n + 1 - j}\bigg)                                                              \\
     & = \sum_{j = 1}^n \bigg(\dfrac{(n + 1)!}{j!(n + 1 - j)!} x^j y^{n + 1 - j}\bigg).
  \end{align*}
  We also have
  \begin{align*}
     & \sum_{j = 0}^{n + 1} \dfrac{(n + 1)!}{j!(n + 1 - j)!} x^j y^{n + 1 - j}                                                                                   \\
     & = \dfrac{(n + 1)!}{(n + 1)! 0!} x^{n + 1} y^0 + \bigg(\sum_{j = 0}^n \dfrac{(n + 1)!}{j!(n + 1 - j)!} x^j y^{n + 1 - j}\bigg) &  & \by{7.1.1}             \\
     & = x^{n + 1} + \bigg(\sum_{j = 0}^n \dfrac{(n + 1)!}{j!(n + 1 - j)!} x^j y^{n + 1 - j}\bigg)                                   &  & \text{(by definition)} \\
     & = x^{n + 1} + \bigg(\sum_{j = 0}^0 \dfrac{(n + 1)!}{j!(n + 1 - j)!} x^j y^{n + 1 - j}\bigg)                                   &  & \by{7.1.4}[a]          \\
     & \quad + \bigg(\sum_{j = 1}^n \dfrac{(n + 1)!}{j!(n + 1 - j)!} x^j y^{n + 1 - j}\bigg)                                                                     \\
     & = x^{n + 1} + \dfrac{(n + 1)!}{0! (n + 1)!} x^0 y^{n + 1}                                                                     &  & \by{7.1.1}             \\
     & \quad + \bigg(\sum_{j = 1}^n \dfrac{(n + 1)!}{j!(n + 1 - j)!} x^j y^{n + 1 - j}\bigg)                                                                     \\
     & = x^{n + 1} + y^{n + 1} + \bigg(\sum_{j = 1}^n \dfrac{(n + 1)!}{j!(n + 1 - j)!} x^j y^{n + 1 - j}\bigg).                      &  & \text{(by definition)}
  \end{align*}
  Thus we have
  \[
    (x + y)^{n + 1} = \sum_{j = 0}^{n + 1} \dfrac{(n + 1)!}{j!(n + 1 - j)!} x^j y^{n + 1 - j}.
  \]
  and this closes the induction.
\end{proof}

\begin{ex}\label{ex:7.1.5}
  Let \(X\) be a finite set, let \(m\) be an integer, and for each \(x \in X\) let \((a_n(x))_{n = m}^\infty\) be a convergent sequence of real numbers.
  Show that the sequence \((\sum_{x \in X} a_n(x))_{n = m}^\infty\) is convergent, and
  \[
    \lim_{n \to \infty} \sum_{x \in X} a_n(x) = \sum_{x \in X} \lim_{n \to \infty} a_n(x).
  \]
  Thus we may always interchange finite sums with convergent limits.
  Things however get trickier with infinite sums.
\end{ex}

\begin{proof}
  Let \(k = \#(X)\).
  We use induction on \(k\).
  For \(k = 0\), we have \(X = \emptyset\).
  So
  \begin{align*}
    \lim_{n \to \infty} \sum_{x \in X} a_n(x) & = \lim_{n \to \infty} 0                      &  & \by{7.1.11} \\
                                              & = 0                                                           \\
                                              & = \sum_{x \in X} \lim_{n \to \infty} a_n(x). &  & \by{7.1.11}
  \end{align*}
  Thus the base case holds.
  Suppose inductively that for some \(k \geq 0\) the statement is true.
  Then for \(k + 1\), we have to show that the statement is also true.
  Let \(x_0 \in X\) and \(X' = X \setminus \set{x_0}\).
  So \(\#(X') = \#(X) - 1 = n\), and we have
  \begin{align*}
     & \lim_{n \to \infty} \sum_{x \in X} a_n(x)                                                                                                    \\
     & = \lim_{n \to \infty} \sum_{x \in \set{x_0} \cup X'} a_n(x)                                                                                  \\
     & = \lim_{n \to \infty} \bigg(\sum_{x \in \set{x_0}} a_n(x) + \sum_{x \in X'} a_n(x)\bigg)                                 &  & \by{7.1.11}[e] \\
     & = \bigg(\lim_{n \to \infty} \sum_{x \in \set{x_0}} a_n(x)\bigg) + \bigg(\lim_{n \to \infty} \sum_{x \in X'} a_n(x)\bigg) &  & \by{6.1.19}[a] \\
     & = \bigg(\lim_{n \to \infty} a_n(x_0)\bigg) + \bigg(\lim_{n \to \infty} \sum_{x \in X'} a_n(x)\bigg)                      &  & \by{7.1.11}[b] \\
     & = \bigg(\sum_{x \in \set{x_0}} \lim_{n \to \infty} a_n(x)\bigg) + \bigg(\lim_{n \to \infty} \sum_{x \in X'} a_n(x)\bigg) &  & \by{7.1.11}[b] \\
     & = \bigg(\sum_{x \in \set{x_0}} \lim_{n \to \infty} a_n(x)\bigg) + \bigg(\sum_{x \in X'} \lim_{n \to \infty} a_n(x)\bigg) &  & \byIH          \\
     & = \bigg(\sum_{x \in \set{x_0} \cup X'} \lim_{n \to \infty} a_n(x)\bigg)                                                  &  & \by{7.1.11}[e] \\
     & = \sum_{x \in X} \lim_{n \to \infty} a_n(x).
  \end{align*}
  This closes the induction.
\end{proof}
