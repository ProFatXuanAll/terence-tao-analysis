\section{Cardinality of sets}\label{i:sec:3.6}

\begin{note}
  In \cref{i:ch:2} we defined the natural numbers axiomatically, assuming that they were equipped with a \(0\) and an increment operation, and assuming five axioms on these numbers.
  Philosophically, this is quite different from one of our main conceptualizations of natural numbers
  - that of \emph{cardinality}, or measuring \emph{how many} elements there are in a set.
  Indeed, the Peano axiom approach treats natural numbers more like \emph{ordinals} than \emph{cardinals}.
  (The \emph{cardinals} are One, Two, Three, ..., and are used to count how many things there are in a set.
  The \emph{ordinals} are First, Second, Third, ..., and are used to order a sequence of objects.
  There is a subtle difference between the two, especially when comparing infinite cardinals with infinite ordinals, but this is beyond the scope of this text.)
  We paid a lot of attention to what number came \emph{next} after a given number \(n\)
  - which is an operation which is quite natural for ordinals, but less so for cardinals
  - but did not address the issue of whether these numbers could be used to \emph{count} sets.
  The purpose of \cref{i:sec:3.6} is to address this issue by noting that the natural numbers \emph{can} be used to count the cardinality of sets, as long as the set is finite.

  One way to define cardinality is to say that two sets have the same size if they have the same number of elements, but we have not yet defined what the ``number of elements'' in a set is.
  Besides, this runs into problems when a set is infinite.

  The right way to define the concept of ``two sets having the same size'' is not immediately obvious, but can be worked out with some thought.
  One intuitive reason why two sets have the same size is that one can match the elements of the first set with the elements in the second set in a one-to-one correspondence
  (Indeed, this is how we first learn to count a set:
  we correspond the set we are trying to count with another set, such as a set of fingers on your hand).
  We will use this intuitive understanding as our rigorous basis for ``having the same size.''
\end{note}

\begin{defn}[Equal cardinality]\label{i:3.6.1}
  We say that two sets \(X\) and \(Y\) have \emph{equal cardinality} iff there exists a bijection \(f : X \to Y\) from \(X\) to \(Y\).
\end{defn}

\setcounter{thm}{2}
\begin{rmk}\label{i:3.6.3}
  The fact that two sets have equal cardinality does not preclude one of the sets from containing the other.
  For instance, if \(X\) is the set of natural numbers and \(Y\) is the set of even natural numbers, then the map \(f : X \to Y\) defined by \(f(n) \coloneqq 2n\) is a bijection from \(X\) to \(Y\), and so \(X\) and \(Y\) have equal cardinality, despite \(Y\) being a subset of \(X\) and seeming intuitively as if it should only have ``half'' of the elements of \(X\).
\end{rmk}

\begin{prop}\label{i:3.6.4}
  Let \(X, Y, Z\) be sets.
  Then \(X\) has equal cardinality with \(X\).
  If \(X\) has equal cardinality with \(Y\), then \(Y\) has equal cardinality with \(X\).
  If \(X\) has equal cardinality with \(Y\) and \(Y\) has equal cardinality with \(Z\), then \(X\) has equal cardinality with \(Z\).
\end{prop}

\begin{proof}[\pf{i:3.6.4}]
  We first show that \cref{i:3.6.1} is reflexive.
  Let \(f : X \to X\) be a function where \(f = x \mapsto x\).
  By \cref{i:3.6} \(f\) is well-defined.
  Since
  \[
    \forall x_1, x_2 \in X, x_1 \neq x_2 \implies x_1 = f(x_1) \neq f(x_2) = x_2,
  \]
  by \cref{i:3.3.14} we know that \(f\) is injective.
  Since
  \[
    \forall x \in X, f(x) = x,
  \]
  by \cref{i:3.3.17} we know that \(f\) is surjective.
  Thus by \cref{i:3.3.20} \(f\) is bijective, and by \cref{i:3.6.1} \(X\) has equal cardinality with \(X\).

  Next we show that \cref{i:3.6.1} is symmetric.
  Suppose that \(X\) has equal cardinality with \(Y\).
  Then by \cref{i:3.6.1} there exists a bijective function \(f : X \to Y\).
  Since \(f\) is bijective, by \cref{i:ex:3.3.6} we know that \(f^{-1} : Y \to X\) is also bijective.
  Thus by \cref{i:3.6.1} \(Y\) has equal cardinality with \(X\).

  Finally we show that \cref{i:3.6.1} is transitive.
  Suppose that \(X\) has equal cardinality with \(Y\) and \(Y\) has equal cardinality with \(Z\).
  Then by \cref{i:3.6.1} there exist two bijective functions \(f : X \to Y\) and \(g : Y \to Z\).
  Since \(f\) and \(g\) are bijective, by \cref{i:ex:3.3.7} we know that \(g \circ f : X \to Z\) is also bijective.
  Thus by \cref{i:3.6.1} \(X\) has equal cardinality with \(Z\).
\end{proof}

\begin{defn}\label{i:3.6.5}
  Let \(n\) be a natural number.
  A set \(X\) is said to have \emph{cardinality} \(n\), iff it has equal cardinality with \(\set{i \in \N : 1 \leq i \leq n}\).
  We also say that \(X\) \emph{has \(n\) elements} iff it has cardinality \(n\).
\end{defn}

\begin{rmk}\label{i:3.6.6}
  One can use the set \(\set{i \in \N : i < n}\) instead of \(\set{i \in \N : 1 \leq i \leq n}\), since these two sets clearly have equal cardinality.
\end{rmk}

\begin{proof}[\pf{i:3.6.6}]
  Let \(A = \set{i \in \N : i < n}\) and let \(B = \set{i \in \N : 1 \leq i \leq n}\).
  Let \(f : A \to B\) be the function defined by \(f = n \mapsto n\pp\).
  By \cref{i:2.4,i:3.3.14} we see that \(f\) is injective.
  Since \(0 \notin B\), by \cref{i:2.2.10} we know that for every \(i \in B\), there exists an \(j \in \N\) such that \(j\pp = i\).
  Then we have
  \begin{align*}
             & i \in B                               \\
    \implies & j\pp = i \leq n &  & \by{i:3.5}       \\
    \implies & j < n           &  & \by{i:2.2.12}[e] \\
    \implies & j \in A.        &  & \by{i:3.5}
  \end{align*}
  Thus \(f(j)\) is well-defined by \cref{i:3.3.1} and \(f(j) = j\pp = i\).
  By \cref{i:3.3.17} this means \(f\) is surjective.
  Since \(f\) is both injective and surjective, by \cref{i:3.3.20} we know that \(f\) is bijective.
  Thus by \cref{i:3.6.1} \(A, B\) have equal cardinality.
\end{proof}

\setcounter{thm}{7}
\begin{prop}[Uniqueness of cardinality]\label{i:3.6.8}
  Let \(X\) be a set with some cardinality \(n\).
  Then \(X\) cannot have any other cardinality, i.e., \(X\) cannot have cardinality \(m\) for any \(m \neq n\).
\end{prop}

\begin{proof}[\pf{i:3.6.8}]
  We induct on \(n\).
  First suppose that \(n = 0\).
  Then \(X\) must be empty (\cref{i:ex:3.3.3}), and so \(X\) cannot have any non-zero cardinality.
  Now suppose that the proposition is already proven for some \(n\);
  we now prove it for \(n\pp\).
  Let \(X\) have cardinality \(n\pp\);
  and suppose that \(X\) also has some other cardinality \(m \neq n\pp\).
  By \cref{i:2.2.10} there exists a \(p \in \N\) such that \(p\pp = m\).
  By \cref{i:3.6.9}, \(X\) is non-empty, and if \(x\) is any element of \(X\), then \(X \setminus \set{x}\) has cardinality \(n\) and also has cardinality \(p\), by \cref{i:3.6.9}.
  By induction hypothesis, this means that \(n = p\), which implies that \(p\pp = m = n\pp\), a contradiction.
  This closes the induction.
\end{proof}

\begin{lem}\label{i:3.6.9}
  Let \(X\) be a set with cardinality \(n \geq 1\).
  Then \(X\) is non-empty, and if \(x\) is any element of \(X\), then the set \(X \setminus \set{x}\) (i.e., \(X\) with the element \(x\) removed) has cardinality \(m\), where \(m\pp = n\).
  (Such \(m\) exists by \cref{i:2.2.10}.)
\end{lem}

\begin{proof}[\pf{i:3.6.9}]
  If \(X\) is empty then it clearly cannot have the same cardinality as the non-empty set \(\set{i \in \N : 1 \leq i \leq n}\), as there is no bijection from the empty set to a non-empty set (\cref{i:ex:3.3.3}).
  Now let \(x\) be an element of \(X\).
  Since \(X\) has the same cardinality as \(\set{i \in \N : 1 \leq i \leq n}\) (\cref{i:3.6.5}), we thus have a bijection \(f\) from \(X\) to \(\set{i \in \N : 1 \leq i \leq n}\).
  In particular, \(f(x)\) is a natural number between \(1\) and \(n\).
  Now define the function \(g : X \setminus \set{x} \to \set{i \in \N : 1 \leq i \leq m}\) by the following rule:
  \[
    \forall y \in X \setminus \set{x}, g(y) \coloneqq \begin{dcases}
      f(y) & \text{if } f(y) < f(x)                          \\
      k    & \text{if } f(y) > f(x) \text{ and } k\pp = f(y)
    \end{dcases}.
  \]
  Note that \(f(y)\) cannot equal \(f(x)\) since \(y \neq x\) and \(f\) is a bijection.
  Also note that since \(f(y) \geq 1\), \(k\) is well-defined by \cref{i:2.2.10}.

  We show that \(g\) is injective.
  Let \(y_1, y_2 \in X \setminus \set{x}\) and \(y_1 \neq y_2\).
  Now we split into four cases:
  \begin{itemize}
    \item If \(f(y_1) < f(x)\) and \(f(y_2) < f(x)\), then \(g(y_1) = f(y_1) \neq f(y_2) = g(y_2)\) since \(f\) is bijective.
    \item If \(f(y_1) < f(x)\) and \(f(y_2) > f(x)\), then there exists a \(k \in \N\) such that \(k\pp = f(y_2)\).
          Thus we have
          \begin{align*}
                     & g(y_1) = f(y_1) < f(x) < f(y_2) = k\pp                                  \\
            \implies & g(y_1) < f(x) < f(x)\pp \leq k\pp      &  & \by{i:ac:2.2.3,i:2.2.12}[e] \\
            \implies & g(y_1) < f(x) \leq k = g(y_2)          &  & \by{i:2.2.12}[d]            \\
            \implies & g(y_1) < g(y_2).                       &  & \by{i:2.2.12}[b]
          \end{align*}
          In particular we have \(g(y_1) \neq g(y_2)\) by \cref{i:2.2.13}.
    \item If \(f(y_1) > f(x)\) and \(f(y_2) < f(x)\), then we can use the second case and switch the role of \(f(y_1)\) and \(f(y_2)\) to derive \(g(y_2) \neq g(y_1)\).
    \item If \(f(y_1) > f(x)\) and \(f(y_2) > f(x)\), then there exist some \(k_1, k_2 \in \N\) such that \(k_1\pp = f(y_1)\) and \(k_2\pp = f(y_2)\).
          Since \(f\) is bijective, we must have \(f(y_1) \neq f(y_2)\).
          Therefore by \cref{i:2.4} we have \(g(y_1) = k_1 \neq k_2 = g(y_2)\).
  \end{itemize}
  From all cases above we see that \(g(y_1) \neq g(y_2)\).
  Therefore by \cref{i:3.3.14} \(g\) is injective.

  Next we show that \(g\) is surjective.
  Let \(i \in \set{i \in \N : 1 \leq i \leq m}\).
  Since \(f\) is bijective and \(i \leq m < n\), we know that \(f^{-1}(i) \in X\) is well-defined.
  Now we split into two cases:
  \begin{itemize}
    \item If \(1 \leq i < f(x)\), then we have \(f^{-1}(i) \neq f^{-1}(f(x)) = x\) since \(f\) is bijective.
          Thus by \cref{i:3.1.27} we have \(f^{-1}(i) \in X \setminus \set{x}\).
          By \cref{i:3.3.1} we know that \(g\pa{f^{-1}(i)}\) is well-defined and by the definition of \(g\) we have \(i = f\pa{f^{-1}(i)} < f(x) \implies g\pa{f^{-1}(i)} = f\pa{f^{-1}(i)} = i\).
          Thus we have found some \(y \in X \setminus \set{x}\) such that \(g(y) = i\).
    \item If \(f(x) \leq i \leq m\), then by \cref{i:2.2.12}(d)(e) we have \(f(x) < i\pp \leq m\pp = n\).
          Since \(f\) is bijective, we know that \(f^{-1}(i\pp)\) is well-defined and \(f(x) \neq i\pp \implies f^{-1}(f(x)) = x \neq f^{-1}(i\pp)\).
          Thus by \cref{i:3.1.27} we have \(f^{-1}(i\pp) \in X \setminus \set{x}\).
          By \cref{i:3.3.1} we know that \(g\pa{f^{-1}(i\pp)}\) is well-defined and by the definition of \(g\) we have \(i\pp = f\pa{f^{-1}(i\pp)} > f(x) \implies g\pa{f^{-1}(i\pp)} = i\).
          Thus we have found some \(y \in X \setminus \set{x}\) such that \(g(y) = i\).
  \end{itemize}
  From all cases above we see that there exists a \(y \in X \setminus \set{x}\) such that \(g(y) = i\).
  Therefore by \cref{i:3.3.17} \(g\) is surjective.
  Since \(g\) is both injective and surjective, by \cref{i:3.3.20} we know that \(g\) is bijective.

  By \cref{i:3.6.5} this means \(X \setminus \set{x}\) has equal cardinality with \(\set{i \in \N : 1 \leq i \leq m}\).
  In particular \(X \setminus \set{x}\) has cardinality \(m\), as desired.
\end{proof}

\begin{defn}[Finite sets]\label{i:3.6.10}
  A set is \emph{finite} iff it has cardinality \(n\) for some natural number \(n\);
  otherwise, the set is called \emph{infinite}.
  If \(X\) is a finite set, we use \(\#(X)\) to denote the cardinality of \(X\).
\end{defn}

\setcounter{thm}{11}
\begin{thm}\label{i:3.6.12}
  The set of natural numbers \(\N\) is infinite.
\end{thm}

\begin{proof}[\pf{i:3.6.12}]
  Suppose for sake of contradiction that the set of natural numbers \(\N\) was finite, so it had some cardinality \(\#(\N) = n\).
  Then there is a bijection \(f\) from \(\set{i \in \N : 1 \leq i \leq n}\) to \(\N\).
  One can show that the sequence \(f(1), f(2), \dots, f(n)\) is bounded, or more precisely that there exists a natural number \(M\) such that \(f(i) \leq M\) for all \(1 \leq i \leq n\) (\cref{i:ex:3.6.3}).
  But then the natural number \(M+1\) is not equal to any of the \(f(i)\), contradicting the hypothesis that \(f\) is a bijection.
\end{proof}

\begin{rmk}\label{i:3.6.13}
  One can also use similar arguments to show that any unbounded set is infinite;
  for instance the rationals \(\Q\) and the reals \(\R\) are infinite.
  However, it is possible for some sets to be ``more'' infinite than others.
  See \cref{i:sec:8.3}.
\end{rmk}

\begin{prop}[Cardinal arithmetic]\label{i:3.6.14}
  \begin{enumerate}
    \item Let \(X\) be a finite set, and let \(x\) be an object which is not an element of \(X\).
          Then \(X \cup \set{x}\) is finite and \(\#(X \cup \set{x}) = \#(X) + 1\).
    \item Let \(X\) and \(Y\) be finite sets.
          Then \(X \cup Y\) is finite and \(\#(X \cup Y) \leq \#(X) + \#(Y)\).
          If in addition \(X\) and \(Y\) are disjoint (i.e., \(X \cap Y = \emptyset\)), then \(\#(X \cup Y) = \#(X) + \#(Y)\).
    \item Let \(X\) be a finite set, and let \(Y\) be a subset of \(X\).
          Then \(Y\) is finite, and \(\#(Y) \leq \#(X)\).
          If in addition \(Y \neq X\) (i.e., \(Y\) is a proper subset of \(X\)), then we have \(\#(Y) < \#(X)\).
    \item If \(X\) is a finite set, and \(f : X \to Y\) is a function, then \(f(X)\) is a finite set with \(\#(f(X)) \leq \#(X)\).
          If in addition \(f\) is one-to-one, then \(\#(f(X)) = \#(X)\).
    \item Let \(X\) and \(Y\) be finite sets.
          Then Cartesian product \(X \times Y\) is finite and \(\#(X \times Y) = \#(X) \times \#(Y)\).
    \item Let \(X\) and \(Y\) be finite sets.
          Then the set \(Y^X\) (defined in \cref{i:3.10}) is finite and \(\#\pa{Y^X} = \#(Y)^{\#(X)}\).
  \end{enumerate}
\end{prop}

\begin{proof}[\pf{i:3.6.14}(a)]
  Since \(X\) is finite, by \cref{i:3.6.10} there exists an \(n \in \N\) such that \(\#(X) = n\).
  By \cref{i:3.6.5} there exists a bijective function \(f : X \to \set{i \in \N : 1 \leq i \leq n}\).
  Now we define a function \(g : X \cup \set{x} \to \set{i \in \N : 1 \leq i \leq n + 1}\) as follow:
  \[
    \forall y \in X \cup \set{x}, g(y) = \begin{dcases}
      f(y)  & \text{if } y \neq x \\
      n + 1 & \text{if } y = x
    \end{dcases}.
  \]
  Since \(x \notin X\), we see that \(g\) passes the vertical line test and therefore \(g\) is well-defined.

  Now we claim that \(g\) is bijective.
  Since \(f\) is bijective, we know that for all \(j \in \set{i \in \N : 1 \leq i \leq n}\), there exists a \(y \in X\) such that \(f(y) = j\).
  With that and \(g(x) = n + 1\), we see that \(g\) is surjective by \cref{i:3.3.17}.
  To show that \(g\) is bijective, by \cref{i:3.3.20} we also need to show that \(g\) is injective.
  So let \(y, y' \in X \cup \set{x}\) and \(y \neq y'\).
  Now we split into three cases:
  \begin{itemize}
    \item If \(y \neq x \neq y'\), then we have \(g(y) = f(y) \neq f(y') = g(y')\) since \(f\) is bijective.
    \item If \(y \neq x = y'\), then by \cref{i:ac:2.2.3} we have \(g(y) = f(y) \leq n < n + 1 = g(x) = g(y')\).
          By \cref{i:2.2.13} this means \(g(y) \neq g(y')\).
    \item If \(y = x \neq y'\), then by \cref{i:ac:2.2.3} we have \(g(y') = f(y') \leq n < n + 1 = g(x) = g(y)\).
          By \cref{i:2.2.13} this means \(g(y) \neq g(y')\).
  \end{itemize}
  For all cases above we have \(g(y) \neq g(y')\), thus by \cref{i:3.3.14} \(g\) is injective.
  This means \(g\) is bijective, therefore by \cref{i:3.6.10} we know that \(X \cup \set{x}\) is finite and \(\#(X \cup \set{x}) = n + 1 = \#(X) + 1\).
\end{proof}

\begin{proof}[\pf{i:3.6.14}(b)]
  Since \(X\) is finite, by \cref{i:3.6.10} there exists an \(n \in \N\) such that \(\#(X) = n\).
  We induct on \(n\) to show that \(X \cup Y\) is finite and \(\#(X \cup Y) \leq \#(X) + \#(Y)\).
  For \(n = 0\), by \cref{i:ex:3.6.2} we have \(X = \emptyset\).
  By \cref{i:3.1.28}(a) we have \(\emptyset \cup Y = Y\).
  Since \(Y\) is finite, we know that \(\emptyset \cup Y\) is finite.
  Thus we have
  \begin{align*}
    \#(\emptyset \cup Y) & = \#(Y)                  &  & \by{i:3.1.28}[a] \\
                         & = 0 + \#(Y)              &  & \by{i:2.2.1}     \\
                         & = \#(\emptyset) + \#(Y). &  & \by{i:ex:3.6.2}
  \end{align*}
  So the base case holds.
  Suppose inductively that \(X \cup Y\) is finite and \(\#(X \cup Y) \leq \#(X) + \#(Y)\) for some \(\#(X) = n\).
  We want to show that when \(\#(X) = n\pp\), we have \(X \cup Y\) is finite and \(\#(X \cup Y) \leq \#(X) + \#(Y)\).
  So let \(X\) be a set with cardinality \(n\pp\).
  We split into two cases:
  \begin{itemize}
    \item If \(X \setminus Y = \emptyset\), then we have
          \begin{align*}
            X \cup Y & = (X \setminus Y) \cup (X \cap Y) \cup (Y \setminus X) &  & \by{i:ex:3.1.10} \\
                     & = (X \cap Y) \cup (Y \setminus X)                      &  & \by{i:3.1.28}[a] \\
                     & \subseteq Y.                                           &  & \by{i:3.1.15}
          \end{align*}
          But by \cref{i:ex:3.1.7} we have \(Y \subseteq X \cup Y\).
          Thus by \cref{i:3.1.18} we have \(X \cup Y = Y\).
          Since \(Y\) is finite, \(X \cup Y\) is also finite.
          By \cref{i:2.2.11} this means \(\#(X \cup Y) = \#(Y) \leq \#(X) + \#(Y)\).
    \item If \(X \setminus Y \neq \emptyset\), then by \cref{i:3.1.6,i:3.1.27} there exists a \(z \in X\) such that \(z \notin Y\).
          This implies \((X \setminus \set{z}) \cup Y = (X \cup Y) \setminus \set{z}\).
          By \cref{i:3.6.9} we know that \(\#(X \setminus \set{z}) = n\), thus by induction hypothesis we know that \((X \setminus \set{z}) \cup Y\) is finite and \(\#((X \setminus \set{z}) \cup Y) \leq \#(X \setminus \set{z}) + \#(Y)\).
          By \cref{i:3.6.14}(a) we know that \(X \cup Y = (X \setminus \set{z}) \cup Y \cup \set{z}\) is finite.
          Thus we have
          \begin{align*}
            \#(X \cup Y) & = \#(((X \cup Y) \setminus \set{z}) \cup \set{z}) &  & \by{i:3.1.28}[e] \\
                         & = \#(((X \setminus \set{z}) \cup Y) \cup \set{z}) &  & \by{i:3.1.28}[e] \\
                         & = \#((X \setminus \set{z}) \cup Y) + 1            &  & \by{i:3.6.14}[a] \\
                         & \leq \#(X \setminus \set{z}) + \#(Y) + 1          &  & \by{i:2.2.12}[d] \\
                         & = \#((X \setminus \set{z}) \cup \set{z}) + \#(Y)  &  & \by{i:3.6.14}[a] \\
                         & = \#(X) + \#(Y).                                  &  & \by{i:3.1.28}[e]
          \end{align*}
  \end{itemize}
  From all cases above we see that \(X \cup Y\) is finite and \(\#(X \cup Y) \leq \#(X) + \#(Y)\).
  This closes the induction.

  Now we show that \(X \cap Y = \emptyset \implies \#(X \cup Y) = \#(X) + \#(Y)\).
  So suppose that \(X \cap Y = \emptyset\).
  Since \(X, Y\) are finite, by \cref{i:3.6.5} there exist two bijective functions \(f : X \to \set{i \in \N : 1 \leq i \leq \#(X)}\) and \(g : Y \to \set{i \in \N : 1 \leq i \leq \#(Y)}\).
  Now we define a function \(h : X \cup Y \to \set{i \in \N : 1 \leq i \leq \#(X) + \#(Y)}\) as follow:
  \[
    \forall z \in X \cup Y, h(z) = \begin{dcases}
      f(z)         & \text{if } z \in X \\
      g(z) + \#(X) & \text{if } z \in Y
    \end{dcases}.
  \]
  Since \(X \cap Y = \emptyset\), we know that \(h\) passes the vertical line test, therefore \(h\) is well-defined.
  We claim that \(h\) is injective.
  So let \(z_1, z_2 \in X \cup Y\) and \(z_1 \neq z_2\).
  Since \(X \cap Y = \emptyset\), we know that \(z_1\) and \(z_2\) can be either in \(X\) or \(Y\) but not both.
  Thus we can split into four cases:
  \begin{itemize}
    \item If both \(z_1\) and \(z_2\) are in \(X\), then we have \(h(z_1) = f(z_1) \neq f(z_2) = h(z_2)\) since \(f\) is bijective.
    \item If both \(z_1\) and \(z_2\) are in \(Y\), then we have \(g(z_1) \neq g(z_2)\) since \(g\) is bijective.
          By \cref{i:2.2.13} this means \(g(z_1) < g(z_2)\) or \(g(z_1) > g(z_2)\), thus we can apply \cref{i:2.2.12}(d) to derive \(g(z_1) + \#(X) < g(z_2) + \#(X)\) or \(g(z_1) + \#(X) > g(z_2) + \#(X)\).
          By \cref{i:2.2.13} again we have \(h(z_1) = g(z_1) + \#(X) \neq g(z_2) + \#(X) = h(z_2)\).
    \item If \(z_1 \in X\) and \(z_2 \in Y\), then by \cref{i:ac:2.2.3} and \cref{i:2.2.12}(b)(d) we have
          \[
            1 \leq h(z_1) = f(z_1) \leq \#(X) < \#(X) + 1 \leq h(z_2) = g(z_2) + \#(X) \leq \#(X) + \#(Y).
          \]
          By \cref{i:2.2.13} this means \(h(z_1) \neq h(z_2)\).
    \item If \(z_1 \in Y\) and \(z_2 \in X\), then by \cref{i:ac:2.2.3} and \cref{i:2.2.12}(b)(d) we have
          \[
            1 \leq h(z_2) = f(z_2) \leq \#(X) < \#(X) + 1 \leq h(z_1) = g(z_1) + \#(X) \leq \#(X) + \#(Y).
          \]
          By \cref{i:2.2.13} this means \(h(z_1) \neq h(z_2)\).
  \end{itemize}
  From all cases above we see that \(h(z_1) \neq h(z_2)\).
  Therefore \(h\) is injective.
  Now we claim that \(h\) is surjective.
  Let \(j \in \set{i \in \N : 1 \leq i \leq \#(X) + \#(Y)}\).
  We split into two cases:
  \begin{itemize}
    \item If \(1 \leq j \leq \#(X)\), then there exists a \(z \in X\) such that \(f(z) = j\).
          This is true since \(f\) is bijective.
          By the definition of \(h\) we know that \(h(z) = f(z) = j\).
          Thus we conclude that there exists a \(z \in X \cup Y\) such that \(h(z) = j\).
    \item If \(\#(X) + 1 \leq j \leq \#(X) + \#(Y)\), then by \cref{i:2.2.11} we have \(j = \#(X) + 1 + k\) for some \(k \in \N\).
          By \cref{i:2.2.12}(d) we know that \(\#(X) + 1 \leq \#(X) + 1 + k \leq \#(X) + \#(Y) \implies 1 \leq 1 + k \leq \#(Y)\).
          Thus there exists a \(z \in Y\) such that \(g(z) = 1 + k\) since \(g\) is bijective.
          By the definition of \(h\) we know that \(h(z) = g(z) + \#(X) = k + 1 + \#(X) = j\).
          Thus we conclude that there exists a \(z \in X \cup Y\) such that \(h(z) = j\).
  \end{itemize}
  From all cases above we conclude that there exists a \(z \in X \cup Y\) such that \(h(z) = j\).
  Thus \(h\) is surjective.
  Since \(h\) is both injective and surjective, by \cref{i:3.3.20} we know that \(h\) is bijective.
  By \cref{i:3.6.5} this means \(X \cup Y\) has cardinality \(\#(X) + \#(Y)\).
  Thus by \cref{i:3.6.10} we have \(\#(X \cup Y) = \#(X) + \#(Y)\).
\end{proof}

\begin{proof}[\pf{i:3.6.14}(c)]
  Since \(X\) is finite, by \cref{i:3.6.10} there exists an \(n \in \N\) such that \(\#(X) = n\).
  We induct on \(n\) to show that any subset \(Y\) of \(X\) is finite and \(\#(Y) \leq \#(X)\).
  For \(n = 0\), by \cref{i:ex:3.6.2} we have \(X = \emptyset\).
  The only subset of \(\emptyset\) is \(\emptyset\), and we have \(\#(\emptyset) = 0 = \#(\emptyset)\).
  So the base case holds.
  Suppose inductively that for some \(\#(X) = n\), we have ``\(Y \subseteq X\) implies \(Y\) is finite and \(\#(Y) \leq \#(X)\).''
  We want to show that when \(\#(X) = n\pp\), we have ``\(Y \subseteq X\) implies \(Y\) is finite and \(\#(Y) \leq \#(X)\).''
  So let \(X\) be a set with cardinality \(n\pp\).
  Let \(Y \subseteq X\) and let \(z \in X\).
  We split into two cases:
  \begin{itemize}
    \item If \(z \in Y\), then we have \(Y \setminus \set{z} \subseteq X \setminus \set{z}\).
          By \cref{i:3.6.9} we know that \(\#(X \setminus \set{z}) = n\).
          Thus we can use induction hypothesis to derive \(Y \setminus \set{z}\) is finite and \(\#(Y \setminus \set{z}) \leq \#(X \setminus \set{z})\).
          By \cref{i:3.1.28}(g) we have \(Y = (Y \setminus \set{z}) \cup \set{z}\), thus by \cref{i:3.6.14}(a) we know that \(Y\) is finite.
          Therefore we have
          \begin{align*}
            \#(Y) & = \#((Y \setminus \set{z}) \cup \set{z}) &  & \by{i:3.1.28}[g] \\
                  & = \#(Y \setminus \set{z}) + 1            &  & \by{i:3.6.14}[a] \\
                  & \leq \#(X \setminus \set{z}) + 1         &  & \by{i:2.2.12}[d] \\
                  & = \#((X \setminus \set{z}) \cup \set{z}) &  & \by{i:3.6.14}[a] \\
                  & = \#(X).                                 &  & \by{i:3.1.28}[g]
          \end{align*}
    \item If \(z \notin Y\), then we have \(Y \subseteq X \setminus \set{z}\).
          By \cref{i:3.6.9} we know that \(\#(X \setminus \set{z}) = n\).
          Thus we can use induction hypothesis to derive \(Y\) is finite and \(\#(Y) \leq \#(X \setminus \set{z})\).
          Then we have
          \begin{align*}
            \#(Y) & \leq \#(X \setminus \set{z})             &  & \byIH            \\
                  & < \#(X \setminus \set{z}) + 1            &  & \by{i:ac:2.2.3}  \\
                  & = \#((X \setminus \set{z}) \cup \set{z}) &  & \by{i:3.6.14}[a] \\
                  & = \#(X).                                 &  & \by{i:3.1.28}[g]
          \end{align*}
  \end{itemize}
  From all cases above we conclude that \(Y\) is finite and \(\#(Y) \leq \#(X)\).
  This closes the induction.

  Now suppose that \(Y \neq X\).
  Then we know that \(X \setminus Y \neq \emptyset\).
  Since \(X \setminus Y \subseteq X\), from first part of the proof we know that \(X \setminus Y\) is finite.
  Thus by \cref{i:ex:3.6.2} we have \(X \setminus Y \neq \emptyset \iff \#(X \setminus Y) > 0\).
  Since
  \begin{align*}
    \#(X) & = \#(Y \cup (X \setminus Y)) &  & \by{i:3.1.28}[g] \\
          & = \#(Y) + \#(X \setminus Y), &  & \by{i:3.6.14}[b]
  \end{align*}
  by \cref{i:2.2.12}(f) we have \(\#(Y) < \#(X)\).
\end{proof}

\begin{proof}[\pf{i:3.6.14}(d)]
  Since \(X\) is finite, by \cref{i:3.6.10} there exists an \(n \in \N\) such that \(\#(X) = n\).
  We induct on \(n\) to show that \(\#(f(X))\) is finite and \(\#(f(X)) \leq \#(X)\).
  For \(n = 0\), we have
  \begin{align*}
             & X = \emptyset         &  & \by{i:ex:3.6.2} \\
    \implies & f(X) = \emptyset      &  & \by{i:3.4.1}    \\
    \implies & \#(f(X)) = 0 = \#(X). &  & \by{i:ex:3.6.2}
  \end{align*}
  Thus, the base case holds.
  Suppose inductively that \(f(X)\) is finite and \(\#(f(X)) \leq \#(X)\) for some \(\#(X) = n\).
  We show that the statement is still true when \(\#(X) = n\pp\).
  So let \(X\) be a set with cardinality \(n\pp\).
  Let \(x \in X\).
  By \cref{i:3.6.9} we have \(\#(X \setminus \set{x}) = n\).
  Thus we can use induction hypothesis to derive ``\(f(X \setminus \set{x})\) is finite and \(\#(f(X \setminus \set{x})) \leq \#(X \setminus \set{x})\).''
  Now we split into two cases:
  \begin{itemize}
    \item If \(f(X \setminus \set{x}) = f(X)\), then \(f(X)\) is finite and we have
          \begin{align*}
            \#(f(X)) & = \#(f(X \setminus \set{x}))                                   \\
                     & \leq \#(X \setminus \set{x})             &  & \byIH            \\
                     & < \#(X \setminus \set{x}) + 1            &  & \by{i:ac:2.2.3}  \\
                     & = \#((X \setminus \set{x}) \cup \set{x}) &  & \by{i:3.6.14}[a] \\
                     & = \#(X).                                 &  & \by{i:3.1.28}[g]
          \end{align*}
    \item If \(f(X \setminus \set{x}) \neq f(X)\), then by \cref{i:ex:3.4.3,i:3.6.14}(a) we know that \(f(X) = f(X \setminus \set{x}) \cup \set{f(x)}\) is finite.
          Thus
          \begin{align*}
            \#(f(X)) & = \#(f(X \setminus \set{x}) \cup \set{f(x)}) &  & \by{i:ex:3.4.3}  \\
                     & = \#(f(X \setminus \set{x})) + 1             &  & \by{i:3.6.14}[a] \\
                     & \leq \#(X \setminus \set{x}) + 1             &  & \byIH            \\
                     & = \#((X \setminus \set{x}) \cup \set{x})     &  & \by{i:3.6.14}[a] \\
                     & = \#(X).                                     &  & \by{i:3.1.28}[g]
          \end{align*}
  \end{itemize}
  From all cases above we conclude that \(f(X)\) is finite and \(\#(f(X)) \leq \#(X)\).
  This closes the induction.

  Now suppose that \(f\) is injective.
  Define \(g : X \to f(X)\) as follow:
  \[
    \forall x \in X, g(x) = f(x).
  \]
  Since \(f\) is a function, \(g\) is well-defined.
  By \cref{i:3.4.1} we know that \(g\) is surjective.
  Since \(f\) is injective, by \cref{i:3.3.14} we know that
  \[
    \forall x_1, x_2 \in X, x_1 \neq x_2 \implies g(x_1) = f(x_1) \neq f(x_2) = g(x_2).
  \]
  Thus \(g\) is injective.
  By \cref{i:3.3.20} this means \(g\) is bijective.
  Therefore by \cref{i:3.6.1} \(X\) and \(f(X)\) have equal cardinality.
  From first part of the proof we know that \(f(X)\) is finite.
  Thus by \cref{i:3.6.10} we have \(\#(f(X)) = \#(X)\).
\end{proof}

\begin{proof}[\pf{i:3.6.14}(e)]
  We first show that for any object \(x\), \(\set{x} \times Y\) is finite and \(\#(\set{x} \times Y) = \#(Y)\).
  Define \(f : \set{x} \times Y \to Y\) as follow:
  \[
    \forall (z, y) \in \set{x} \times Y, f(z, y) = y.
  \]
  By \cref{i:ex:3.5.7} we see that \(f = \pi_{\set{x} \times Y \to Y}\).
  We claim that \(f\) is injective.
  Let \((x_1, y_1), (x_2, y_2) \in \set{x} \times Y\) and \((x_1, y_1) \neq (x_2, y_2)\).
  Since
  \begin{align*}
             & \begin{dcases}
                 (x_1, y_1), (x_2, y_2) \in \set{x} \times Y \\
                 (x_1, y_1) \neq (x_2, y_2)
               \end{dcases}        \\
    \implies & \begin{dcases}
                 x_1, x_2 \in \set{x} \\
                 (x_1 \neq x_2) \lor (y_1 \neq y_2)
               \end{dcases}             &  & \by{i:3.5.1,i:3.5.4} \\
    \implies & \begin{dcases}
                 x_1 = x_2 = x \\
                 (x_1 \neq x_2) \lor (y_1 \neq y_2)
               \end{dcases}             &  & \by{i:3.3}           \\
    \implies & y_1 \neq y_2                                       \\
    \implies & y_1 = f(x_1, y_1) \neq f(x_2, y_2) = y_2,
  \end{align*}
  by \cref{i:3.3.14} we see that \(f\) is injective.
  Now we show that \(f\) is surjective.
  Let \(y \in Y\).
  Then we have
  \begin{align*}
             & \begin{dcases}
                 x \in \set{x} \\
                 y \in Y
               \end{dcases}            &  & \by{i:3.3}       \\
    \implies & (x, y) \in \set{x} \times Y &  & \by{i:3.5.4} \\
    \implies & f(x, y) = y.
  \end{align*}
  Thus by \cref{i:3.3.17} we see that \(f\) is surjective.
  Since \(f\) is both injective and surjective, by \cref{i:3.3.20} we know that \(f\) is bijective.
  Thus by \cref{i:3.6.1} \(\set{x} \times Y\) and \(Y\) have equal cardinality.
  Since \(Y\) is finite, by \cref{i:3.6.4} we know that \(\set{x} \times Y\) is also finite.
  Thus by \cref{i:3.6.10} we have \(\#(\set{x} \times Y) = \#(Y)\).

  Now we show that \(X \times Y\) is finite and \(\#(X \times Y) = \#(X) \times \#(Y)\).
  By \cref{i:3.6.10} there exists an \(n \in \N\) such that \(\#(X) = n\).
  We induct on \(n\) to show that \(X \times Y\) is finite and \(\#(X \times Y) = \#(X) \times \#(Y)\).
  For \(n = 0\), by \cref{i:ex:3.6.2} we have \(X = \emptyset\).
  Thus by \cref{i:3.5.4} we have \(\emptyset \times Y = \emptyset\) and therefore \(\emptyset \times Y\) is finite.
  Then we have
  \begin{align*}
    \#(\emptyset \times Y) & = \#(\emptyset)               &  & \by{i:3.5.4}    \\
                           & = 0                           &  & \by{i:ex:3.6.2} \\
                           & = 0 \times \#(Y)              &  & \by{i:2.3.1}    \\
                           & = \#(\emptyset) \times \#(Y). &  & \by{i:ex:3.6.2}
  \end{align*}
  Thus, the base case holds.
  Suppose inductively that \(X \times Y\) is finite and \(\#(X \times Y) = \#(X) \times \#(Y)\) for some \(\#(X) = n\).
  We show that the statement is still true for \(\#(X) = n\pp\).
  So let \(X\) be a set with cardinality \(n\pp\).
  Let \(x \in X\).
  By \cref{i:3.6.9} we have \(\#(X \setminus \set{x}) = n\).
  Thus by induction hypothesis we know that \((X \setminus \set{x}) \times Y\) is finite.
  From first part of the proof we know that \(\set{x} \times Y\) is finite.
  Since
  \begin{align*}
    X \times Y & = ((X \setminus \set{x}) \cup \set{x}) \times Y             &  & \by{i:3.1.28}[g] \\
               & = ((X \setminus \set{x}) \times Y) \cup (\set{x} \times Y), &  & \by{i:ex:3.5.4}
  \end{align*}
  by \cref{i:3.6.14}(b) we know that \(X \times Y\) is finite.
  Thus we have
  \begin{align*}
    \#(X \times Y) & = \#(((X \setminus \set{x}) \times Y) \cup (\set{x} \times Y)) &  & \by{i:ex:3.5.4}           \\
                   & = \#((X \setminus \set{x}) \times Y) + \#(\set{x} \times Y)    &  & \by{i:3.6.14}[b]          \\
                   & = \#(X \setminus \set{x}) \times \#(Y) + \#(\set{x} \times Y)  &  & \byIH                     \\
                   & = \#(X \setminus \set{x}) \times \#(Y) + \#(Y)                 &  & \text{(from proof above)} \\
                   & = (\#(X \setminus \set{x}) + 1) \times \#(Y)                   &  & \by{i:2.3.1}              \\
                   & = \#((X \setminus \set{x}) \cup \set{x}) \times \#(Y)          &  & \by{i:3.6.14}[a]          \\
                   & = \#(X) \times \#(Y).                                          &  & \by{i:3.1.28}[g]
  \end{align*}
  This closes the induction.
\end{proof}

\begin{proof}[\pf{i:3.6.14}(f)]
  We first show that \(Y^{\set{x}}\) is finite and \(\#\pa{Y^{\set{x}}} = \#(Y)\) for any object \(x\).
  Define \(f : Y^{\set{x}} \to Y\) as follow:
  \[
    \forall g \in Y^{\set{x}}, f(g) = g(x).
  \]
  Since \(g \in Y^{\set{x}}\) implies \(g(x)\) is an unique object defined in \(Y\), we know that \(f(g)\) passes the vertical line test.
  Thus by \cref{i:3.3.1} \(f\) is well-defined.
  We now show that \(f\) is bijective.
  We start by showing \(f\) is injective.
  Let \(g_1, g_2 \in Y^{\set{x}}\) and \(g_1 \neq g_2\).
  Since
  \begin{align*}
             & g_1 \neq g_2                                                 \\
    \implies & \exists z \in \set{x} : g_1(z) \neq g_2(z) &  & \by{i:3.3.7} \\
    \implies & g_1(x) \neq g_2(x)                         &  & \by{i:3.3}   \\
    \implies & g_1(x) = f(g_1) \neq f(g_2) = g_2(x),
  \end{align*}
  by \cref{i:3.3.14} we see that \(f\) is injective.
  Now we show that \(f\) is surjective.
  Let \(y \in Y\).
  Let \(g : \set{x} \to Y\) be defined by \(g(x) = y\).
  By \cref{i:3.10} we know that \(g \in Y^{\set{x}}\).
  Therefore \(f(g) = y\).
  By \cref{i:3.3.17} we see that \(f\) is surjective.
  Since \(f\) is both injective and surjective, by \cref{i:3.3.20} we know that \(f\) is bijective.
  Thus by \cref{i:3.6.1} \(Y^{\set{x}}\) and \(Y\) have equal cardinality.
  Since \(Y\) is finite, by \cref{i:3.6.10} we know that \(Y^{\set{x}}\) is finite, and we have \(\#\pa{Y^{\set{x}}} = \#(Y)\).

  Now we show that \(Y^X\) is finite and \(\#\pa{Y^X} = \#(Y)^{\#(X)}\).
  By \cref{i:3.6.10} there exists an \(n \in \N\) such that \(\#(X) = n\).
  We induct on \(n\) to show that \(Y^X\) is finite and \(\#\pa{Y^X} = \#(Y)^{\#(X)}\).
  For \(n = 0\), by \cref{i:ex:3.6.2} we have \(X = \emptyset\).
  By \cref{i:3.3.9} there is only one function in \(Y^\emptyset\), thus by \cref{i:3.3} \(Y^\emptyset\) is a singleton set and therefore \(Y^\emptyset\) is finite.
  By \cref{i:3.6.10} we have \(\#\pa{Y^\emptyset} = 1\), thus
  \begin{align*}
    \#(Y)^{\#(\emptyset)} & = \#(Y)^0             &  & \by{i:ex:3.6.2} \\
                          & = 1                   &  & \by{i:2.3.11}   \\
                          & = \#\pa{Y^\emptyset}. &  & \by{i:3.6.10}
  \end{align*}
  So the base case holds.

  Suppose inductively that the statement is true for some \(\#(X) = n\).
  We show that the statement is still true for \(\#(X) = n\pp\).
  So let \(X\) be a set with cardinality \(n\pp\).
  Let \(x \in X\).
  If \(f \in Y^X\), then we define two functions \(f_1 : X \setminus \set{x} \to Y\) and \(f_2 : \set{x} \to Y\) as follow:
  \[
    \forall z \in X \setminus \set{x}, f_1(z) = f(z) \quad \text{and} \quad \forall z \in \set{x}, f_2(z) = f(z).
  \]
  Since \(f\) is a function, we know that \(f_1, f_2\) are well-defined and unique to \(f\).
  Now we define \(h : Y^X \to Y^{X \setminus \set{x}} \times Y^{\set{x}}\) as follow:
  \[
    \forall f \in Y^X : h(f) = (f_1, f_2).
  \]
  Since \(f_1, f_2\) are unique to \(f\), we know that \(h\) is well-defined.
  We claim that \(h\) is bijective.
  We start by showing \(h\) is injective.
  So let \(f, g \in Y^X\) and \(f \neq g\).
  Since
  \begin{align*}
             & f \neq g                                                                                                             \\
    \implies & \exists z \in X : f(z) \neq g(z)                                                                   &  & \by{i:3.3.7} \\
    \implies & (\exists z \in X \setminus \set{x} : f(z) \neq g(z)) \lor (\exists z \in \set{x} : f(z) \neq g(z))                   \\
    \implies & (f_1 \neq g_1) \lor (f_2 \neq g_2)                                                                 &  & \by{i:3.3.7} \\
    \implies & (f_1, f_2) \neq (g_1, g_2)                                                                         &  & \by{i:3.5.1} \\
    \implies & h(f) \neq h(g),
  \end{align*}
  by \cref{i:3.3.14} we see that \(h\) is injective.
  Now we show that \(h\) is surjective.
  Let \(p \in Y^{X \setminus \set{x}}\) and let \(q \in Y^{\set{x}}\).
  Define \(f : X \to Y\) as follow:
  \[
    \forall z \in X : f(z) = \begin{dcases}
      p(z) & \text{if } z \in X \setminus \set{x} \\
      q(z) & \text{if } z \in \set{x}
    \end{dcases}
  \]
  Since \(p, q\) are functions, we know that \(f\) is well-defined and thus \(f \in Y^X\).
  Since
  \begin{align*}
             & \begin{dcases}
                 \forall z \in X \setminus \set{x}, f_1(z) = f(z) = p(z) \\
                 \forall z \in \set{x}, f_2(z) = f(z) = q(z)
               \end{dcases}                      \\
    \implies & (f_1 = p) \land (f_2 = q)                                  &  & \by{i:3.3.7} \\
    \implies & (f_1, f_2) = (p, q)                                        &  & \by{i:3.5.1} \\
    \implies & h(f) = (p, q),
  \end{align*}
  by \cref{i:3.3.17} we see that \(h\) is surjective.
  Since \(h\) is both injective and surjective, by \cref{i:3.3.20} we know that \(h\) is bijective.
  Thus by \cref{i:3.6.1} we know that \(Y^X\) and \(Y^{X \setminus \set{x}} \times Y^{\set{x}}\) have equal cardinality.
  By \cref{i:3.6.9} we know that \(\#(X \setminus \set{x}) = n\).
  Thus by induction hypothesis we know that \(Y^{X \setminus \set{x}}\) is finite.
  From the first part of the proof we know that \(Y^{\set{x}}\) is finite.
  Thus by \cref{i:3.6.14}(e) we know that \(Y^{X \setminus \set{x}} \times Y^{\set{x}}\) is finite.
  By \cref{i:3.6.10} this means \(Y^X\) is finite and we have \(\#\pa{Y^X} = \#\pa{Y^{X \setminus \set{x}} \times Y^{\set{x}}}\).
  We now finish our induction as follow:
  \begin{align*}
    \#\pa{Y^X} & = \#\pa{Y^{X \setminus \set{x}} \times Y^{\set{x}}}         &  & \text{(from the proof above)}             \\
               & = \#\pa{Y^{X \setminus \set{x}}} \times \#\pa{Y^{\set{x}}}  &  & \by{i:3.6.14}[e]                          \\
               & = \#(Y)^{\#(X \setminus \set{x})} \times \#\pa{Y^{\set{x}}} &  & \byIH                                     \\
               & = \#(Y)^{\#(X \setminus \set{x})} \times \#(Y)              &  & \text{(from the first part of the proof)} \\
               & = \#(Y)^{\#(X \setminus \set{x}) + 1}                       &  & \by{i:2.3.11}                             \\
               & = \#(Y)^{\#((X \setminus \set{x}) \cup \set{x})}            &  & \by{i:3.6.14}[a]                          \\
               & = \#(Y)^{\#(X)}.                                            &  & \by{i:3.1.28}[g]
  \end{align*}
  This closes the induction.
\end{proof}

\begin{rmk}\label{i:3.6.15}
  \cref{i:3.6.14} suggests that there is another way to define the arithmetic operations of natural numbers;
  not defined recursively as in \cref{i:2.2.1,i:2.3.1,i:2.3.11}, but instead using the notions of union, Cartesian product, and power set.
  This is the basis of \emph{cardinal arithmetic}, which is an alternative foundation to arithmetic than the Peano arithmetic we have developed here;
  we will not develop this arithmetic in this text, but we give some examples of how one would work with this arithmetic in \cref{i:ex:3.6.5,i:ex:3.6.6}.
\end{rmk}

\exercisesection

\begin{ex}\label{i:ex:3.6.1}
  Prove \cref{i:3.6.4}.
\end{ex}

\begin{proof}[\pf{i:ex:3.6.1}]
  See \cref{i:3.6.4}.
\end{proof}

\begin{ex}\label{i:ex:3.6.2}
  Show that a set \(X\) has cardinality \(0\) iff \(X\) is the empty set.
\end{ex}

\begin{proof}[\pf{i:ex:3.6.2}]
  By \cref{i:3.6.5} we know that \(\#(X) = 0\) iff there exists a bijective function \(f : X \to \set{i \in \N : 1 \leq i \leq 0}\).
  Since \(\set{i \in \N : 1 \leq i \leq 0} = \emptyset\), by \cref{i:ex:3.3.3} we know that \(f\) is a bijective function iff \(X = \emptyset\).
  Thus we have \(\#(X) = 0\) iff \(X = \emptyset\).
\end{proof}

\begin{ex}\label{i:ex:3.6.3}
  Let \(n\) be a natural number, and let \(f : \set{i \in \N : 1 \leq i \leq n} \to \N\) be a function.
  Show that there exists a natural number \(M\) such that \(f(i) \leq M\) for all \(1 \leq i \leq n\).
  Thus finite subsets of the natural numbers are bounded.
\end{ex}

\begin{proof}[\pf{i:ex:3.6.3}]
  We induct on \(n\).
  For \(n = 0\), any function \(f : \set{i \in \N : 1 \leq i \leq 0} \to \N\) is the empty function to \(\N\) (\cref{i:3.3.9}).
  Thus the following statement is vacuously true:
  \[
    \forall M \in \N, \forall i \in \emptyset, f(i) \leq M.
  \]
  So the base case holds.
  Suppose inductively that for some \(n\) the statement is true.
  We want to show that the statement is still true for \(n\pp\).
  Let \(f : \set{i \in \N : 1 \leq i \leq n\pp} \to \N\) be a function.
  Define \(g : \set{i \in \N : 1 \leq i \leq n} \to \N\) as follow:
  \[
    \forall j \in \set{i \in \N : 1 \leq i \leq n}, g(j) = f(j).
  \]
  By induction hypothesis we know that there exists an \(M \in \N\) such that \(g(i) \leq M\) for all \(1 \leq i \leq n\).
  Thus by the definition of \(g\) we know that there exists an \(M \in \N\) such that \(f(i) \leq M\) for all \(1 \leq i \leq n\).
  Fix one such \(M\).
  Now we split into two cases:
  \begin{itemize}
    \item If \(f(n\pp) \leq M\), then we have \(f(i) \leq M\) for all \(1 \leq i \leq n\pp\).
    \item If \(f(n\pp) > M\), then by \cref{i:2.2.12}(b) we know that \(f(i) < f(n\pp)\) for all \(1 \leq i \leq n\).
          By setting \(M' = f(n\pp)\) we see that \(f(i) \leq M'\) for all \(1 \leq i \leq n\pp\).
  \end{itemize}
  From all cases above we conclude that there exists an \(M \in \N\) such that \(f(u) \leq M\) for all \(1 \leq i \leq n\).
  This closes the induction.
\end{proof}

\begin{ex}\label{i:ex:3.6.4}
  Prove \cref{i:3.6.14}.
\end{ex}

\begin{proof}[\pf{i:ex:3.6.4}]
  See \cref{i:3.6.14}.
\end{proof}

\begin{ex}\label{i:ex:3.6.5}
  Let \(A, B\) be sets.
  Show that \(A \times B\) and \(B \times A\) have equal cardinality by constructing an explicit bijection between the two sets.
  Then use \cref{i:3.6.14} to conclude an alternate proof of \cref{i:2.3.2}.
\end{ex}

\begin{proof}[\pf{i:ex:3.6.5}]
  Define \(f : A \times B \to B \times A\) as follow:
  \[
    \forall (a, b) \in A \times B : f(a, b) = (b, a).
  \]
  By \cref{i:3.5.1} we know that \(f\) passes the vertical line test, thus \(f\) is well-defined.
  We now show that such \(f\) is bijective.
  We start by showing \(f\) is injective.
  Let \((a_1, b_1), (a_2, b_2) \in A \times B\) and \((a_1, b_1) \neq (a_2, b_2)\).
  Since
  \begin{align*}
             & (a_1, b_1) \neq (a_2, b_2)                           \\
    \implies & (a_1 \neq a_2) \lor (b_1 \neq b_2) &  & \by{i:3.5.1} \\
    \implies & (b_1 \neq b_2) \lor (a_1 \neq a_2)                   \\
    \implies & (b_1, a_1) \neq (b_2, a_2)         &  & \by{i:3.5.1} \\
    \implies & f(a_1, b_1) \neq f(a_2, b_2),
  \end{align*}
  by \cref{i:3.3.14} we know that \(f\) is injective.
  Now we show that \(f\) is surjective.
  Let \((b, a) \in B \times A\).
  Clearly \((a, b) \in A \times B\), thus we have \(f(a, b) = (b, a)\).
  Therefore \(f\) is surjective by \cref{i:3.3.17}.
  Since \(f\) is both injective and surjective, by \cref{i:3.3.20} we know that \(f\) is bijective.
  By \cref{i:3.6.1} we conclude that \(A \times B\) and \(B \times A\) have equal cardinality.

  Now we give an alternative proof of \cref{i:2.3.2}.
  Let \(n, m \in \N\), let \(A = \set{i \in \N : 1 \leq i \leq n}\) and let \(B = \set{i \in \N : 1 \leq i \leq m}\).
  By \cref{i:3.6.10} we know that \(\#(A) = n\) and \(\#(B) = m\).
  From the proof above we know that \(A \times B\) and \(B \times A\) have equal cardinality, thus by \cref{i:3.6.8} we have \(\#(A \times B) = \#(B \times A)\).
  Then we have
  \begin{align*}
    n \times m & = \#(A) \times \#(B)                       \\
               & = \#(A \times B)     &  & \by{i:3.6.14}[e] \\
               & = \#(B \times A)     &  & \by{i:3.6.8}     \\
               & = \#(B) \times \#(A) &  & \by{i:3.6.14}[e] \\
               & = m \times n.
  \end{align*}
  Thus \cref{i:2.3.2} is true.
\end{proof}

\begin{ex}\label{i:ex:3.6.6}
  Let \(A, B, C\) be sets.
  Show that the sets \((A^B)^C\) and \(A^{B \times C}\) have equal cardinality by constructing an explicit bijection between the two sets.
  Conclude that \((a^b)^c = a^{bc}\) for any natural numbers \(a, b, c\).
  Use a similar argument to also conclude \(a^b \times a^c = a^{b+c}\).
\end{ex}

\begin{proof}[\pf{i:ex:3.6.6}]
  We first show that \((A^B)^C\) and \(A^{B \times C}\) have equal cardinality.
  By \cref{i:3.5.4} \(B \times C\) is a set and by \cref{i:3.10} \(A^B, (A^B)^C, A^{B \times C}\) are sets.
  Therefore it makes sense to ask whether \((A^B)^C\) and \(A^{B \times C}\) have equal cardinality.
  Define \(f : (A^B)^C \to A^{B \times C}\) as follow:
  \[
    \forall g \in (A^B)^C, f(g) = h_g
  \]
  where \(h_g : B \times C \to A\) is defined as follow:
  \[
    \forall (b, c) \in B \times C, h_g(b, c) = (g(c))(b).
  \]
  We need to show that \(f\) is well-defined.
  So let \(g \in (A^B)^C\) and let \((b, c) \in B \times C\).
  Define \(h_g : B \times C \to A\) as above.
  Since \(g\) is a function with domain \(C\) and codomain \(A^B\), we know that \(g(c)\) is well-defined and is a function in \(A^B\).
  Since \(g(c)\) has domain \(B\) and codomain \(A\), we know that \((g(c))(b)\) is well-defined and is an object in \(A\).
  Thus \(h_g\) passes the vertical line test and \(h_g\) is well-defined.
  If \(h_g' : B \times C \to A\) is another function satisfying \(h_g'(b, c) = (g(c))(b)\) for any \((b, c) \in B \times C\), then by \cref{i:3.3.7} we have \(h_g = h_g'\).
  Thus \(f\) passes the vertical line test and \(f\) is well-defined.

  We now show that \(f\) is bijective.
  We start by showing that \(f\) is injective.
  Let \(g_1, g_2 \in (A^B)^C\) and \(g_1 \neq g_2\).
  Since
  \begin{align*}
             & g_1 \neq g_2                                                                         \\
    \implies & \exists c \in C : g_1(c) \neq g_2(c)                               &  & \by{i:3.3.7} \\
    \implies & \exists c \in C : (\exists b \in B : (g_1(c))(b) \neq (g_2(c))(b)) &  & \by{i:3.3.7} \\
    \implies & \exists (b, c) \in B \times C : (g_1(c))(b) \neq (g_2(c))(b)       &  & \by{i:3.5.1} \\
    \implies & \exists (b, c) \in B \times C : h_{g_1}(b, c) \neq h_{g_2}(b, c)                     \\
    \implies & h_{g_1} \neq h_{g_2}                                               &  & \by{i:3.3.7} \\
    \implies & f(g_1) \neq f(g_2),
  \end{align*}
  by \cref{i:3.3.14} we know that \(f\) is injective.
  Next we show that \(f\) is surjective.
  Let \(h \in A^{B \times C}\).
  Define \(g : C \to A^B\) as follow:
  \[
    \forall (b, c) \in B \times C, (g(c))(b) = h(b, c).
  \]
  Since \(h\) is a function, \(g\) pass the vertical line test and thus \(g\) is well-defined.
  By the definition of \(f\) we see that \(f(g) = h_g = h\).
  Thus by \cref{i:3.3.17} \(f\) is surjective.
  Since \(f\) is both injective and surjective, by \cref{i:3.3.20} we know that \(f\) is bijective.
  By \cref{i:3.6.1} we conclude that \((A^B)^C\) and \(A^{B \times C}\) have equal cardinality.

  Now we show that \((a^b)^c = a^{bc}\) for any \(a, b, c \in \N\).
  Suppose that \(A, B, C\) are finite sets with cardinality \(\#(A) = a\), \(\#(B) = b\) and \(\#(C) = c\).
  Since \(B, C\) are finite, by \cref{i:3.6.14}(e) we know that \(B \times C\) is finite.
  By \cref{i:3.6.14}(f) we know that \(A^{B \times C}\) is finite.
  From the proof above we see that \((A^B)^C\) and \(A^{B \times C}\) have equal cardinality.
  Thus \((A^B)^C\) is finite and by \cref{i:3.6.8} we have \(\#\pa{(A^B)^C} = \#\pa{A^{B \times C}}\).
  Then we have
  \begin{align*}
    (a^b)^c & = \pa{\#(A)^{\#(B)}}^{\#(C)} &  & \by{i:3.6.10}                 \\
            & = \pa{\#\pa{A^B}}^{\#(C)}    &  & \by{i:3.6.14}[f]              \\
            & = \#\pa{\pa{A^B}^C}          &  & \by{i:3.6.14}[f]              \\
            & = \#\pa{A^{B \times C}}      &  & \text{(from the proof above)} \\
            & = \#(A)^{\#(B \times C)}     &  & \by{i:3.6.14}[f]              \\
            & = \#(A)^{\#(B) \times \#(C)} &  & \by{i:3.6.14}[e]              \\
            & = a^{bc}.                    &  & \by{i:3.6.10}
  \end{align*}

  Next we show that \(A^B \times A^C\) and \(A^{B \cup C}\) have equal cardinality if \(B \cap C = \emptyset\).
  So suppose that \(B \cap C = \emptyset\).
  By \cref{i:3.10} \(A^B, A^C, A^{B \cup C}\) are sets and by \cref{i:3.5.4} \(A^B \times A^C\) is a set.
  Therefore it makes sense to ask whether \(A^B \times A^C\) and \(A^{B \cup C}\) have equal cardinality.
  Define \(f : A^B \times A^C \to A^{B \cup C}\) as follow:
  \[
    \forall (g, h) \in A^B \times A^C, \forall x \in B \cup C, (f(g, h))(x) = \begin{dcases}
      g(x) & \text{if } x \in B \\
      h(x) & \text{if } x \in C
    \end{dcases}.
  \]
  Since \(g, h\) are functions and \(B \cap C = \emptyset\), \((f(g, h))(x)\) is well-defined.
  If there exist some \(g' \in A^B\) and \(h' \in A^C\) such that \((f(g, h))(x) = g'(x)\) if \(x \in B\) and \((f(g, h))(x) = h'(x)\) if \(x \in C\), then by \cref{i:3.3.7} we have \(g = g'\) and \(h = h'\).
  Thus \(f\) passes the vertical line test and \(f\) is well-defined.
  We now show that \(f\) is bijective.
  We start by showing that \(f\) is injective.
  Let \((g_1, h_1), (g_2, h_2) \in A^B \times A^C\) and \((g_1, h_1) \neq (g_2, h_2)\).
  Since
  \begin{align*}
             & (g_1, h_1) \neq (g_2, h_2)                                                                           \\
    \implies & (g_1 \neq g_2) \lor (h_1 \neq h_2)                                                 &  & \by{i:3.5.1} \\
    \implies & (\exists x \in B : g_1(x) \neq g_2(x)) \lor (\exists x \in C : h_1(x) \neq h_2(x)) &  & \by{i:3.3.7} \\
    \implies & \exists x \in B \cup C : (g_1(x) \neq g_2(x)) \lor (h_1(x) \neq h_2(x))            &  & \by{i:3.4}   \\
    \implies & \exists x \in B \cup C : (f(g_1, h_1))(x) \neq (f(g_2, h_2))(x)                                      \\
    \implies & f(g_1, h_1) \neq f(g_2, h_2),                                                      &  & \by{i:3.3.7}
  \end{align*}
  by \cref{i:3.3.14} we know that \(f\) is injective.
  Next we show that \(f\) is surjective.
  Let \(k \in A^{B \cup C}\).
  Define functions \(g : B \to A\) and \(h : C \to A\) as follow:
  \[
    \forall x \in B, g(x) = k(x) \quad \text{and} \quad \forall x \in C, h(x) = k(x).
  \]
  By the definition of \(f\) we see that \(f(g, h) = k\).
  Thus by \cref{i:3.3.17} \(f\) is surjective.
  Since \(f\) is both injective and surjective, by \cref{i:3.3.20} we know that \(f\) is bijective.
  By \cref{i:3.6.1} we conclude that \(A^B \times A^C\) and \(A^{B \cup C}\) have equal cardinality if \(B \cap C = \emptyset\).

  Now we show that \(a^b \times a^c = a^{b + c}\) for any \(a, b, c \in \N\).
  Suppose that \(A, B, C\) are finite sets with cardinality \(\#(A) = a\), \(\#(B) = b\) and \(\#(C) = c\).
  Suppose also that \(B \cap C = \emptyset\).
  Since \(B, C\) are finite, by \cref{i:3.6.14}(b) we know that \(B \cup C\) is finite.
  By \cref{i:3.6.14}(f) we know that \(A^{B \cup C}\) is finite.
  From the proof above we know that \(A^B \times A^C\) and \(A^{B \cup C}\) have equal cardinality.
  Thus by \cref{i:3.6.8} \(A^B \times A^C\) is finite and \(\#(A^B \times A^C) = \#\pa{A^{B \cup C}}\).
  Then we have
  \begin{align*}
    a^b \times a^c & = \#(A)^{\#(B)} \times \#(A)^{\#(C)} &  & \by{i:3.6.10}                 \\
                   & = \#\pa{A^B} \times \#\pa{A^C}       &  & \by{i:3.6.14}[f]              \\
                   & = \#\pa{A^B \times A^C}              &  & \by{i:3.6.14}[e]              \\
                   & = \#\pa{A^{B \cup C}}                &  & \text{(from the proof above)} \\
                   & = \#(A)^{\#(B \cup C)}               &  & \by{i:3.6.14}[f]              \\
                   & = \#(A)^{\#(B) + \#(C)}              &  & \by{i:3.6.14}[b]              \\
                   & = a^{b + c}.                         &  & \by{i:3.6.10}
  \end{align*}
\end{proof}

\begin{ex}\label{i:ex:3.6.7}
  Let \(A\) and \(B\) be sets.
  Let us say that \(A\) has \emph{lesser or equal} cardinality to \(B\) if there exists an injection \(f : A \to B\) from \(A\) to \(B\).
  Show that if \(A\) and \(B\) are finite sets, then \(A\) has lesser or equal cardinality to \(B\) iff \(\#(A) \leq \#(B)\).
\end{ex}

\begin{proof}[\pf{i:ex:3.6.7}]
  First suppose that \(A\) has lesser or equal cardinality to \(B\).
  By definition there exists a injective function \(f : A \to B\).
  By \cref{i:3.4.1} we have \(f(A) \subseteq B\).
  Thus
  \begin{align*}
    \#(A) & = \#(f(A))  &  & \by{i:3.6.14}[d] \\
          & \leq \#(B). &  & \by{i:3.6.14}[c]
  \end{align*}

  Now suppose that \(\#(A) \leq \#(B)\).
  By \cref{i:3.6.5} there exists two bijective functions \(g_A : A \to \set{i \in \N : 1 \leq i \leq \#(A)}\) and \(g_B : B \to \set{i \in \N : 1 \leq i \leq \#(B)}\).
  Now we define \(f : A \to B\) as follow:
  \[
    \forall x \in A, f(x) = g_B^{-1}(g_A(x)).
  \]
  Since both \(g_A\) and \(g_B\) are bijective, we know that \(g_B^{-1}(g_A(x))\) is unique to each \(x\).
  Thus \(f\) pass the vertical line test and \(f\) is well-defined.
  We claim that \(f\) is injective.
  So let \(x_1, x_2 \in A\) and \(x_1 \neq x_2\).
  Since
  \begin{align*}
             & x_1 \neq x_2                                                  \\
    \implies & g_A(x_1) \neq g_A(x_2)                     &  & \by{i:3.3.20} \\
    \implies & g_B^{-1}(g_A(x_1)) \neq g_B^{-1}(g_A(x_2)) &  & \by{i:3.3.20} \\
    \implies & f(x_1) \neq f(x_2),
  \end{align*}
  by \cref{i:3.3.14} we see that \(f\) is injective.
  Thus \(A\) has lesser or equal cardinality to \(B\).
  We conclude that if \(A, B\) are finite sets, then \(A\) has lesser or equal cardinality to \(B\) iff \(\#(A) \leq \#(B)\).
\end{proof}

\begin{ex}\label{i:ex:3.6.8}
  Let \(A, B\) be sets such that \(A \neq \emptyset\) and there exists an injection \(f : A \to B\) from \(A\) to \(B\) (i.e., \(A\) has lesser or equal cardinality to \(B\)).
  Show that there exists a surjection \(g : B \to A\) from \(B\) to \(A\).
  (The converse to this statement requires the axiom of choice;
  see \cref{i:ex:8.4.3}.)
\end{ex}

\begin{proof}[\pf{i:ex:3.6.8}]
  Since \(f\) is injective, by \cref{i:3.3.14} we know that if \(b \in f(A)\), then there exists only one element \(a \in A\) such that \(f(a) = b\).
  So it makes sense to talk about the inverse of \(b\) when \(b \in f(A)\), and we denote the inverse of \(b\) as \(f^{-1}(b)\).

  Now we construct a surjective function.
  Let \(a \in A\).
  Define \(g : B \to A\) as follow:
  \[
    \forall x \in B, g(x) = \begin{dcases}
      f^{-1}(x) & \text{if } x \in f(A)    \\
      a         & \text{if } x \notin f(A)
    \end{dcases}.
  \]
  Clearly \(g\) passes the vertical line test, thus \(g\) is well-defined.
  Now we show that \(g\) is surjective.
  Let \(y \in A\).
  By \cref{i:3.4.1} we have \(f(y) \in f(A) \subseteq B\), thus by \cref{i:ex:3.3.6} we have \(g(f(y)) = f^{-1}(f(y)) = y\).
  Therefore by \cref{i:3.3.17} \(g\) is surjective.
\end{proof}

\begin{ex}\label{i:ex:3.6.9}
  Let \(A\) and \(B\) be finite sets.
  Show that \(A \cup B\) and \(A \cap B\) are also finite sets, and that \(\#(A) + \#(B) = \#(A \cup B) + \#(A \cap B)\).
\end{ex}

\begin{proof}[\pf{i:ex:3.6.9}]
  By \cref{i:3.6.14}(b) \(A \cup B\) is finite.
  By \cref{i:ex:3.1.7} we have \(A \cap B \subseteq A\), thus by \cref{i:3.6.14}(c) \(A \cap B\) is finite since \(A\) is finite.
  Using similar argument we can show that \(A \setminus B\) and \(B \setminus A\) are finite.
  Thus we have
  \begin{align*}
     & \#(A \cup B) + \#(A \cap B)                                                                       \\
     & = \#((A \setminus B) \cup (A \cap B) \cup (B \setminus A)) + \#(A \cap B)   &  & \by{i:ex:3.1.10} \\
     & = \#(A \setminus B) + \#(A \cap B) + \#(B \setminus A) + \#(A \cap B)       &  & \by{i:3.6.14}[b] \\
     & = \#((A \setminus B) \cup (A \cap B)) + \#((B \setminus A) \cup (A \cap B)) &  & \by{i:3.6.14}[b] \\
     & = \#(A) + \#(B).                                                            &  & \by{i:3.1.28}[g]
  \end{align*}
\end{proof}

\begin{ex}\label{i:ex:3.6.10}
  Let \(A_1, \dots, A_n\) be finite sets such that \(\#\pa{\bigcup_{i \in \set{1, \dots, n}} A_i} > n\).
  Show that there exists \(i \in \set{1, \dots, n}\) such that \(\#(A_i) \geq 2\).
  (This is known as the \emph{pigeonhole principle}.)
\end{ex}

\begin{proof}[\pf{i:ex:3.6.10}]
  We induct on \(n\).
  We start with \(n = 1\) since for \(n = 0\) the statement is vacuously true.
  For \(n = 1\), we have
  \begin{align*}
             & \#\pa{\bigcup_{i \in \set{1, \dots, 1}} A_i} > 1                       \\
    \implies & \#(A_1) > 1                                      &  & \by{i:3.11}      \\
    \implies & \#(A_1) \geq 2.                                  &  & \by{i:2.2.12}[e]
  \end{align*}
  Thus, the base case holds.
  Suppose inductively that for some \(n\) the statement is true.
  We show that the statement is still true for \(n\pp\).
  By induction hypothesis we know that \(\bigcup_{i \in \set{1, \dots, n}} A_i\) is finite.
  Since \(A_{n\pp}\) is finite, by \cref{i:3.6.14}(b) we know that \(\pa{\bigcup_{i \in \set{1, \dots, n}} A_i} \cup A_{n\pp} = \bigcup_{i \in \set{1, \dots, n\pp}} A_i\) is finite.
  Thus we have
  \begin{align*}
    \#\pa{\bigcup_{i \in \set{1, \dots, n}} A_i} + \#(A_{n\pp}) & \geq \#\pa{\pa{\bigcup_{i \in \set{1, \dots, n}} A_i} \cup A_{n\pp}} &  & \by{i:3.6.14}(b) \\
                                                                & = \#\pa{\bigcup_{i \in \set{1, \dots, n\pp}} A_i}                    &  & \by{i:3.11}      \\
                                                                & > n\pp.
  \end{align*}
  Now we split into three cases:
  \begin{itemize}
    \item If \(\#(A_{n\pp}) = 0\), then we have
          \begin{align*}
            \#\pa{\bigcup_{i \in \set{1, \dots, n\pp}} A_i} & = \#\pa{\bigcup_{i \in \set{1, \dots, n}} A_i} &  & \by{i:3.1.28}[a] \\
                                                            & > n\pp                                                               \\
                                                            & > n.                                           &  & \by{i:ac:2.2.3}
          \end{align*}
          By induction hypothesis, there exists an \(i \in \set{1, \dots, n}\) such that \(\#(A_i) \geq 2\).
          Therefore there exists an \(i \in \set{1, \dots, n\pp}\) such that \(\#(A_i) \geq 2\).
    \item If \(\#(A_{n\pp}) = 1\), then we have
          \begin{align*}
                     & \#\pa{\bigcup_{i \in \set{1, \dots, n}} A_i} + 1 > n\pp                       \\
            \implies & \#\pa{\bigcup_{i \in \set{1, \dots, n}} A_i} > n.       &  & \by{i:2.2.12}[b]
          \end{align*}
          By induction hypothesis there exists an \(i \in \set{1, \dots, n}\) such that \(\#(A_i) \geq 2\).
          Therefore there exists an \(i \in \set{1, \dots, n\pp}\) such that \(\#(A_i) \geq 2\).
    \item If \(\#(A_{n\pp}) > 1\), then \(\#(A_{n\pp}) \geq 2\).
          Thus there exists an \(i \in \set{1, \dots, n\pp}\) such that \(\#(A_i) \geq 2\).
  \end{itemize}
  From all cases above we conclude that there exists an \(i \in \set{1, \dots, n\pp}\) such that \(\#(A_i) \geq 2\).
  This closes the induction.
\end{proof}
