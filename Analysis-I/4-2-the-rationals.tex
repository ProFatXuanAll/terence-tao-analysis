\section{The rationals}\label{i:sec:4.2}

\begin{defn}\label{i:4.2.1}
  A \emph{rational number} is an expression of the form \(a // b\), where \(a\) and \(b\) are integers and \(b\) is non-zero;
  \(a // 0\) is not considered to be a rational number.
  Two rational numbers are considered to be equal, \(a // b = c // d\), iff \(ad = cb\).
  The set of all rational numbers is denoted \(\Q\).
\end{defn}

\begin{note}
  There is no reasonable way we can divide by zero, since one cannot have both the identities \((a / b) \times b = a\) and \(c \times 0 = 0\) hold simultaneously if \(b\) is allowed to be zero and \(a\) is non-zero.
  Similarly, the identities \(a / a = 1\) and \(2 \times (a / a) = (2 \times a) / a\) cannot hold simultaneously if \(0 / 0\) is defined.
  However, we can eventually get a reasonable notion of dividing by a quantity which approaches zero
  - think of L'H\^opital's rule (see \cref{i:sec:10.5}), which suffices for doing things like defining differentiation.
\end{note}

\begin{ac}\label{i:ac:4.2.1}
  The definition of equality for the rational numbers is reflexive, symmetric and transitive.
\end{ac}

\begin{proof}[\pf{i:ac:4.2.1}]
  Let \(a // b\), \(c // d\), \(e // f\) be rational numbers where \(a, b, c, d, e, f \in \Z\) and \(b, d, f \neq 0\).
  Since
  \begin{align*}
             & ab = ab          &  & \by{i:4.1.3} \\
    \implies & a // b = a // b, &  & \by{i:4.2.1}
  \end{align*}
  we know that \cref{i:4.2.1} is reflexive.

  Next suppose that \(a // b = c // d\).
  Then we have
  \begin{align*}
             & a // b = c // d                    \\
    \implies & ad = cb          &  & \by{i:4.2.1} \\
    \implies & cb = ad          &  & \by{i:4.1.3} \\
    \implies & c // d = a // b. &  & \by{i:4.2.1}
  \end{align*}
  Thus \cref{i:4.2.1} is symmetric.

  Finally suppose that \(a // b = c // d\) and \(c // d = e // f\).
  Then we have
  \begin{align*}
             & (a // b = c // d) \land (c // d = e // f)                   \\
    \implies & (ad = cb) \land (cf = ed)                 &  & \by{i:4.2.1} \\
    \implies & (adf = cbf) \land (cfb = edb)             &  & \by{i:4.1.3} \\
    \implies & (afd = cbf) \land (cbf = ebd)             &  & \by{i:4.1.6} \\
    \implies & afd = ebd                                 &  & \by{i:4.1.3} \\
    \implies & af = eb                                   &  & \by{i:4.1.9} \\
    \implies & a // b = e // f.                          &  & \by{i:4.2.1}
  \end{align*}
  Thus \cref{i:4.2.1} is transitive.
\end{proof}

\begin{defn}\label{i:4.2.2}
  If \(a // b\) and \(c // d\) are rational numbers, we define their sum
  \[
    (a // b) + (c // d) \coloneqq (ad + bc) // (bd)
  \]
  their product
  \[
    (a // b) \times (c // d) \coloneqq (ac) // (bd)
  \]
  and the negation
  \[
    -(a // b) \coloneqq (-a) // b.
  \]

  Note that if \(b\) and \(d\) are non-zero, then \(bd\) is also non-zero, by \cref{i:4.1.8}, so the sum or product of two rational numbers remains a rational number.
\end{defn}

\begin{lem}\label{i:4.2.3}
  The sum, product, and negation operations on rational numbers are well-defined, in the sense that if one replaces \(a // b\) with another rational number \(a' // b'\) which is equal to \(a // b\), then the output of the above operations remains unchanged, and similarly for \(c // d\).
\end{lem}

\begin{proof}[\pf{i:4.2.3}]
  We first show that the addition on rationals numbers is well-defined.
  Suppose \(a // b = a' // b'\), so that \(b\) and \(b'\) are non-zero and \(ab' = a'b\).
  We now show that \((a // b) + (c // d) = (a' // b') + (c // d)\).
  By \cref{i:4.2.2}, the left-hand side is \((ad + bc) // bd\) and the right-hand side is \((a'd + b'c) // b'd\).
  So by \cref{i:4.2.1}, we have to show that
  \[
    (ad + bc)b'd = (a'd + b'c)bd,
  \]
  which expands to
  \[
    ab'd^2 + bb'cd = a'bd^2 + bb'cd.
  \]
  But since \(ab' = a'b\), the claim follows.
  Similarly suppose \(c // d = c' // d'\), so that \(d\) and \(d'\) are non-zero and \(cd' = c'd\).
  We show that \((a // b) + (c // d) = (a // b) + (c' // d')\).
  By \cref{i:4.2.2}, the left-hand side is \((ad + bc) // bd\) and the right-hand side is \((ad' + bc') // bd'\).
  So by \cref{i:4.2.1}, we have to show that
  \[
    (ad + bc)bd' = (ad' + bc')bd,
  \]
  which expands to
  \[
    abdd' + b^2cd' = abdd' + b^2c'd.
  \]
  But since \(cd' = c'd\), the claim follows.

  Next we show that the multiplication on rationals numbers is well-defined.
  Suppose \(a // b = a' // b'\), so that \(b\) and \(b'\) are non-zero and \(ab' = a'b\).
  We now show that \((a // b) \times (c // d) = (a' // b') \times (c // d)\).
  By \cref{i:4.2.2}, the left-hand side is \((ac) // (bd)\) and the right-hand side is \((a'c) // (b'd)\).
  So by \cref{i:4.2.1}, we have to show that
  \[
    (ac)(b'd) = (a'c)(bd),
  \]
  which is equivalent to
  \[
    ab'cd = a'bcd.
  \]
  But since \(ab' = a'b\), the claim follows.
  Similarly suppose \(c // d = c' // d'\), so that \(d\) and \(d'\) are non-zero and \(cd' = c'd\).
  We now show that \((a // b) \times (c // d) = (a // b) \times (c' // d')\).
  By \cref{i:4.2.2}, the left-hand side is \((ac) // (bd)\) and the right-hand side is \((ac') // (bd')\).
  So by \cref{i:4.2.1}, we have to show that
  \[
    (ac)(bd') = (ac')(bd),
  \]
  which is equivalent to
  \[
    abcd' = abc'd.
  \]
  But since \(cd' = c'd\), the claim follows.

  Finally we show that the negation on rationals numbers is well-defined.
  Suppose \(a // b = a' // b'\), so that \(b\) and \(b'\) are non-zero and \(ab' = a'b\).
  We now show that \(-(a // b) = -(a' // b')\).
  By \cref{i:4.2.2}, the left-hand side is \((-a) // b\) and the right-hand side is \((-a') // b'\).
  So by \cref{i:4.2.1}, we have to show that
  \[
    (-a)b' = (-a')b,
  \]
  which by \cref{i:ac:4.1.5} is equivalent to
  \[
    (-1)ab' = (-1)a'b.
  \]
  But since \(ab' = a'b\), the claim follows from \cref{i:4.1.3}.
\end{proof}

\begin{ac}\label{i:ac:4.2.2}
  The rational numbers \(a // 1\) behave in a manner identical to the integers \(a\):
  \begin{align*}
    (a // 1) + (b // 1)      & = (a + b) // 1; \\
    (a // 1) \times (b // 1) & = (ab // 1);    \\
    -(a // 1)                & = (-a) // 1.
  \end{align*}
  Also, \(a // 1\) and \(b // 1\) are only equal when \(a\) and \(b\) are equal.
  Because of this, we will identify \(a\) with \(a // 1\) for each integer \(a\): \(a \equiv a // 1\);
  the above identities then guarantee that the arithmetic of the integers is consistent with the arithmetic of the rationals.
  Thus just as we embedded the natural numbers inside the integers, we embed the integers inside the rational numbers.
  In particular, all natural numbers are rational numbers, for instance \(0\) is equal to \(0 // 1\) and \(1\) is equal to \(1 // 1\).

  Observe that a rational number \(a // b\) is equal to \(0 = 0 // 1\) iff \(a \times 1 = b \times 0\), i.e., if the numerator \(a\) is equal to \(0\).
  Thus if \(a\) and \(b\) are non-zero then so is \(a // b\).
\end{ac}

\begin{proof}[\pf{i:ac:4.2.2}]
  Let \(a, b \in \Z\).
  First we show that \((a // 1) + (b // 1) = (a + b) // 1\).
  This is true since
  \begin{align*}
    (a // 1) + (b // 1) & = (a1 + b1) // (1 \times 1) &  & \by{i:4.2.2} \\
                        & = (a + b) // 1.             &  & \by{i:4.1.6}
  \end{align*}

  Next we show that \((a // 1) \times (b // 1) = (ab) // 1\).
  This is true since
  \begin{align*}
    (a // 1) \times (b // 1) & = (ab) // (1 \times 1) &  & \by{i:4.2.2} \\
                             & = (ab) // 1.           &  & \by{i:4.1.6}
  \end{align*}

  Next we show that \(-(a // 1) = (-a) // 1\).
  This is true by \cref{i:4.2.2}.

  Finally we show that \(a // 1 = b // 1 \iff a = b\).
  This is true since
  \begin{align*}
         & a // 1 = b // 1                   \\
    \iff & a1 = b1         &  & \by{i:4.2.1} \\
    \iff & a = b.          &  & \by{i:4.1.6}
  \end{align*}
\end{proof}

\begin{ac}\label{i:ac:4.2.3}
  We now define a new operation on the rationals: reciprocal.
  If \(x = a // b\) is a non-zero rational (so that \(a, b \neq 0\)) then we define the \emph{reciprocal} \(x^{-1}\) of \(x\) to be the rational number \(x^{-1} \coloneqq b // a\).
  The reciprocal operation on rational numbers is consistent with \cref{i:4.2.1}:
  if two rational numbers \(a // b\), \(a' // b'\) are equal, then their reciprocals are also equal.
  In contrast to reciprocal, an operation such as ``numerator'' is not well-defined:
  the rationals \(3 // 4\) and \(6 // 8\) are equal, but have unequal numerators, so we have to be careful when referring to such terms as ``the numerator of \(x\)''.
  We however leave the reciprocal of \(0\) undefined.
\end{ac}

\begin{proof}[\pf{i:ac:4.2.3}]
  By \cref{i:4.2.1} and the definition of reciprocal we have \(a, a', b, b' \neq 0\).
  Then we have
  \begin{align*}
             & a // b = a' // b'                    \\
    \implies & ab' = a'b          &  & \by{i:4.2.1} \\
    \implies & b'a = ba'          &  & \by{i:4.1.6} \\
    \implies & b' // a' = b // a. &  & \by{i:4.2.1}
  \end{align*}
\end{proof}

\begin{prop}[Laws of algebra for rationals]\label{i:4.2.4}
  Let \(x, y, z\) be rationals.
  Then the following laws of algebra hold:
  \begin{align*}
    x + y               & = y + x       \\
    (x + y) + z         & = x + (y + z) \\
    x + 0 = 0 + x       & = x           \\
    x + (-x) = (-x) + x & = 0           \\
    xy                  & = yx          \\
    (xy)z               & = x(yz)       \\
    x1 = 1x             & = x           \\
    x(y + z)            & = xy + xz     \\
    (y + z)x            & = yx + zx.
  \end{align*}
  If \(x\) is non-zero, we also have
  \[
    xx^{-1} = x^{-1}x = 1.
  \]
\end{prop}

\begin{proof}[\pf{i:4.2.4}]
  To prove this identity, one writes \(x = a // b\), \(y = c // d\), \(z = e // f\) for some integers \(a\), \(c\), \(e\) and non-zero integers \(b\), \(d\), \(f\), and verifies each identity in turn using the algebra of the integers.

  First we show that \(x + y = y + x\).
  \begin{align*}
    x + y & = (a // b) + (c // d)                   \\
          & = (ad + bc) // bd     &  & \by{i:4.2.2} \\
          & = (bc + ad) // bd     &  & \by{i:4.1.6} \\
          & = (cb + da) // db     &  & \by{i:4.1.6} \\
          & = (c // d) + (a // b) &  & \by{i:4.2.2} \\
          & = y + x.
  \end{align*}
  Thus the addition on rationals is commutative.

  Next we show that \((x + y) + z = x + (y + z)\).
  \begin{align*}
    (x + y) + z & = ((a // b) + (c // d)) + (e // f)                     \\
                & = ((ad + bc) // bd) + (e // f)       &  & \by{i:4.2.2} \\
                & = ((ad + bc)f + (bd)e) // ((bd)f)    &  & \by{i:4.2.2} \\
                & = ((ad)f + (bc)f + (bd)e) // ((bd)f) &  & \by{i:4.1.6} \\
                & = (a(df) + b(cf) + b(de)) // (b(df)) &  & \by{i:4.1.6} \\
                & = (a(df) + b(cf + de)) // (b(df))    &  & \by{i:4.1.6} \\
                & = (a // b) + ((cf + de) // df)       &  & \by{i:4.2.2} \\
                & = (a // b) + ((c // d) + (e // f))   &  & \by{i:4.2.2} \\
                & = x + (y + z).
  \end{align*}
  Thus the addition on rationals is associative.

  Next we show that \(x + 0 = 0 + x = x\).
  Since the addition on rationals is commutative, we know that \(x + 0 = 0 + x\).
  Thus we only need to show that \(x + 0 = x\).
  \begin{align*}
    x + 0 & = (a // b) + (0 // 1) &  & \by{i:ac:4.2.2} \\
          & = (a1 + b0) // b1     &  & \by{i:4.2.2}    \\
          & = (a + 0) // b        &  & \by{i:4.1.6}    \\
          & = a // b              &  & \by{i:4.1.6}    \\
          & = x.
  \end{align*}
  Thus \(0\) is the additive identity on rationals.

  Next we show that \(x + (-x) = (-x) + x = 0\).
  Since the addition on rationals is commutative, we know that \(x + (-x) = (-x) + x\).
  Thus we only need to show that \(x + (-x) = 0\).
  \begin{align*}
    x + (-x) & = (a // b) + ((-a) // b) &  & \by{i:4.2.2}    \\
             & = (ab + b(-a)) // b^2    &  & \by{i:4.2.2}    \\
             & = (ab + (-a)b) // b^2    &  & \by{i:4.1.6}    \\
             & = (ab + (-(ab)) // b^2   &  & \by{i:ac:4.1.5} \\
             & = 0 // b^2               &  & \by{i:4.1.6}    \\
             & = 0.                     &  & \by{i:ac:4.2.2}
  \end{align*}
  Thus the additive inverse of rational \(x\) is \(-x\).

  Next we show that \(xy = yx\).
  \begin{align*}
    xy & = (a // b) \times (c // d)                   \\
       & = ac // bd                 &  & \by{i:4.2.2} \\
       & = ca // db                 &  & \by{i:4.1.6} \\
       & = (c // d) \times (a // b) &  & \by{i:4.2.2} \\
       & = yx.
  \end{align*}
  Thus the multiplication on rationals is commutative.

  Next we show that \((xy)z = x(yz)\).
  \begin{align*}
    (xy)z & = ((a // b) \times (c // d)) \times (e // f)                   \\
          & = (ac // bd) \times (e // f)                 &  & \by{i:4.2.2} \\
          & = ((ac)e) // ((bd)f)                         &  & \by{i:4.2.2} \\
          & = (a(ce)) // (b(df))                         &  & \by{i:4.1.6} \\
          & = (a // b) \times (ce // df)                 &  & \by{i:4.2.2} \\
          & = (a // b) \times ((c // d) \times (e // f)) &  & \by{i:4.2.2} \\
          & = x(yz).
  \end{align*}
  Thus the multiplication on rationals is associative.

  Next we show that \(x1 = 1x = x\).
  Since the multiplication on rationals is commutative, we know that \(x1 = 1x\).
  Thus we only need to show that \(x1 = x\).
  \begin{align*}
    x1 & = (a // b) \times (1 // 1) &  & \by{i:ac:4.2.2} \\
       & = a1 // b1                 &  & \by{i:4.2.2}    \\
       & = a // b                   &  & \by{i:4.1.6}    \\
       & = x.
  \end{align*}
  Thus \(1\) is the multiplicative identity on rationals.

  Next we show that \(x(y + z) = xy + xz\).
  \begin{align*}
    x(y + z) & = (a // b) \times ((c // d) + (e // f))                                     \\
             & = (a // b) \times ((cf + de) // df)                       &  & \by{i:4.2.2} \\
             & = (a(cf + de)) // (b(df))                                 &  & \by{i:4.2.2} \\
             & = (b(a(cf + de))) // (b^2(df))                            &  & \by{i:4.2.3} \\
             & = ((ba)(cf + de)) // (b^2(df))                            &  & \by{i:4.1.6} \\
             & = ((ba)(cf) + (ba)(de)) // (b^2(df))                      &  & \by{i:4.1.6} \\
             & = ((ab)(fc) + (ba)(ed)) // (b^2(fd))                      &  & \by{i:4.1.6} \\
             & = (a(bf)c + b(ae)d) // (b(bf)d)                           &  & \by{i:4.1.6} \\
             & = ((ac)(bf) + (bd)(ae)) // ((bd)(bf))                     &  & \by{i:4.1.6} \\
             & = (ac // bd) + (ae // bf)                                 &  & \by{i:4.2.2} \\
             & = ((a // b) \times (c // d)) + ((a // b) \times (e // f)) &  & \by{i:4.2.2} \\
             & = xy + xz.
  \end{align*}
  Thus the multiplication and addition on rationals are left distributive.

  Next we show that \((y + z)x = yx + zx\).
  \begin{align*}
    (y + z)x & = x(y + z) &  & \text{(multiplication is commutative)}                     \\
             & = xy + xz  &  & \text{(multiplication and addition are left distributive)} \\
             & = yx + zx. &  & \text{(multiplication is commutative)}
  \end{align*}
  Thus the multiplication and addition on rationals are right distributive.

  Finally we show that if \(x \neq 0\), then \(xx^{-1} = x^{-1}x = 1\).
  Since the multiplication on rationals is commutative, we know that \(xx^{-1} = x^{-1}x\).
  Thus we only need to show that \(xx^{-1} = 1\).
  \begin{align*}
    xx^{-1} & = (a // b) \times (b // a) &  & \by{i:ac:4.2.3} \\
            & = ab // ba                 &  & \by{i:4.2.2}    \\
            & = ab // ab                 &  & \by{i:4.1.6}    \\
            & = 1 // 1                   &  & \by{i:4.2.1}    \\
            & = 1.                       &  & \by{i:ac:4.2.2}
  \end{align*}
  Thus the multiplicative inverse of rational \(x\) is \(x^{-1}\).
\end{proof}

\begin{rmk}\label{i:4.2.5}
  The above set (\cref{i:4.2.4}) of ten identities have a name;
  they are asserting that the rationals \(\Q\) form a \emph{field}.
  This is better than being a commutative ring because of the tenth identity \(xx^{-1} = x^{-1}x = 1\).
  Note that \cref{i:4.2.4} supercedes \cref{i:4.1.6}.
\end{rmk}

\begin{ac}\label{i:ac:4.2.4}
  We can now define the \emph{quotient} \(x / y\) of two rational numbers \(x\) and \(y\), \emph{provided that} \(y\) is non-zero, by the formula
  \[
    x / y \coloneqq x \times y^{-1}.
  \]
  Using the above formula, it is easy to see that \(a / b = a // b\) for every integer \(a\) and every non-zero integer \(b\).
  Thus we can now discard the \(//\) notation, and use the more customary \(a / b\) instead of \(a // b\).

  In a similar spirit, we define subtraction on the rationals by the formula
  \[
    x - y \coloneqq x + (-y),
  \]
  just as we did with the integers.
\end{ac}

\begin{proof}[\pf{i:ac:4.2.4}]
  We have
  \begin{align*}
    a / b & = (a // 1) / (b // 1)           &  & \by{i:ac:4.2.2} \\
          & = (a // 1) \times (b // 1)^{-1} &  & \by{i:ac:4.2.4} \\
          & = (a // 1) \times (1 // b)      &  & \by{i:ac:4.2.3} \\
          & = a1 // 1b                      &  & \by{i:4.2.2}    \\
          & = a // b.                       &  & \by{i:4.1.6}
  \end{align*}
\end{proof}

\begin{ac}\label{i:ac:4.2.5}
  Let \(x = a / b\) be a rational number where \(a, b \in \Z\) and \(b \neq 0\).
  Then we have
  \[
    -x = (-a) / b = a / (-b) = (-1) (a / b) = (-1) x
  \]
  and \(-(-x) = x\).
  If \(y \in \Q\), then we have
  \[
    -xy = (-x) y = x (-y).
  \]
  If \(y \neq 0\), then we have
  \[
    -(x / y) = (-x) / y = x / (-y).
  \]
\end{ac}

\begin{proof}[\pf{i:ac:4.2.5}]
  By \cref{i:4.2.2,i:ac:4.2.4} we have \(-x = (-a) / b\) and by \cref{i:4.2.3} we have \((-1) x = (-1) (a / b)\).
  We first show that \((-a) / b = a / (-b)\).
  This is true since
  \begin{align*}
    (-a) / b & = ((-a) / b) \times 1             &  & \by{i:4.2.4}    \\
             & = ((-a) / b) \times (1 / 1)       &  & \by{i:ac:4.2.2} \\
             & = ((-a) / b) \times ((-1) / (-1)) &  & \by{i:4.2.1}    \\
             & = ((-a) (-1)) / (b (-1))          &  & \by{i:4.2.2}    \\
             & = ((-1) (-a)) / ((-1) b)          &  & \by{i:4.1.6}    \\
             & = (-(-a)) / (-b)                  &  & \by{i:ac:4.1.5} \\
             & = a / (-b).                       &  & \by{i:ac:4.1.6}
  \end{align*}

  Next we show that \(-x = (-1) x\).
  This is true since
  \begin{align*}
    -x & = -(a / b)                  &  & \by{i:ac:4.2.4} \\
       & = (-a) / b                  &  & \by{i:4.2.2}    \\
       & = ((-1) a) / b              &  & \by{i:ac:4.1.5} \\
       & = ((-1) a) / (1b)           &  & \by{i:4.1.6}    \\
       & = ((-1) / 1) \times (a / b) &  & \by{i:4.2.2}    \\
       & = (-1) (a / b)              &  & \by{i:ac:4.2.2} \\
       & = (-1) x.                   &  & \by{i:ac:4.2.4}
  \end{align*}

  Next we show that \(-(-x) = x\).
  This is true since
  \begin{align*}
    -(-x) & = (-1) ((-1) x) &  & \text{(from the proof above)} \\
          & = ((-1) (-1)) x &  & \by{i:4.2.4}                  \\
          & = 1 x           &  & \by{i:ac:4.1.7}               \\
          & = x.            &  & \by{i:4.2.4}
  \end{align*}

  Next we show that \(-xy = (-x) y = x (-y)\).
  This is true since
  \begin{align*}
    -xy & = (-1) (xy)  &  & \text{(from the proof above)} \\
        & = ((-1) x) y &  & \by{i:4.2.4}                  \\
        & = (-x) y     &  & \text{(from the proof above)} \\
        & = (x (-1)) y &  & \by{i:4.2.4}                  \\
        & = x ((-1) y) &  & \by{i:4.2.4}                  \\
        & = x (-y).    &  & \text{(from the proof above)}
  \end{align*}

  Next we show that if \(y \neq 0\), then \(-(x / y) = (-x) / y\).
  Suppose that \(y \neq 0\).
  Then we have
  \begin{align*}
    -(x / y) & = -(x \times y^{-1})            &  & \by{i:ac:4.2.4}               \\
             & = (-1) \times (x \times y^{-1}) &  & \text{(from the proof above)} \\
             & = ((-1) \times x) \times y^{-1} &  & \by{i:4.2.4}                  \\
             & = (-x) \times y^{-1}            &  & \text{(from the proof above)} \\
             & = (-x) / y.                     &  & \by{i:ac:4.2.4}
  \end{align*}

  Finally we show that if \(y \neq 0\), then \((-x) / y = x / (-y)\).
  Suppose that \(y = c / d\) where \(c, d \neq 0\).
  Then we have
  \begin{align*}
    (-x) / y & = (-x) \times y^{-1}             &  & \by{i:ac:4.2.4}               \\
             & = ((-1) \times x) \times y^{-1}  &  & \text{(from the proof above)} \\
             & = ((-1) \times x) \times (d / c) &  & \by{i:ac:4.2.3}               \\
             & = (x \times (-1)) \times (d / c) &  & \by{i:4.2.4}                  \\
             & = x \times ((-1) \times (d / c)) &  & \by{i:4.2.4}                  \\
             & = x \times (d / (-c))            &  & \text{(from the proof above)} \\
             & = x \times ((-c) / d)^{-1}       &  & \by{i:ac:4.2.3}               \\
             & = x \times (-y)^{-1}             &  & \by{i:4.2.2}                  \\
             & = x / (-y).                      &  & \by{i:ac:4.2.4}
  \end{align*}
\end{proof}

\begin{defn}\label{i:4.2.6}
  A rational number \(x\) is said to be \emph{positive} iff we have \(x = a / b\) for some positive integers \(a\) and \(b\).
  It is said to be \emph{negative} iff we have \(x = -y\) for some positive rational \(y\)
  (i.e., \(x = (-a) / b\) for some positive integers \(a\) and \(b\)).

  Thus for instance, every positive integer is a positive rational number, and every negative integer is a negative rational number, so our new definition is consistent with our old one.
\end{defn}

\begin{lem}[Trichotomy of rationals]\label{i:4.2.7}
  Let \(x\) be a rational number.
  Then exactly one of the following three statements is true:
  \begin{enumerate}
    \item \(x\) is equal to \(0\).
    \item \(x\) is a positive rational number.
    \item \(x\) is a negative rational number.
  \end{enumerate}
\end{lem}

\begin{proof}[\pf{i:4.2.7}]
  We first show that at least one of (a), (b), (c) is true.
  Let \(x = a / b\), where \(a, b \in \Z\) and \(b \neq 0\).
  By \cref{i:4.1.11}(f), \(a\) can only satisified one of the following three statements:
  \(a = 0\), \(a > 0\) and \(a < 0\).
  Similarly, \(b\) can only satisified one of the following two statements:
  \(b > 0\) and \(b < 0\).
  We first consider \(a\):
  \begin{itemize}
    \item If \(a = 0\), then \(x = 0 / b = 0\).
    \item If \(a > 0\), then we need to consider \(b\):
          \begin{itemize}
            \item If \(b > 0\), then by \cref{i:4.2.6} \(x\) is positive.
            \item If \(b < 0\), then by \cref{i:4.1.4} \(b = -c\) for some \(c \in \Z^+\).
                  Thus by \cref{i:ac:4.2.5} we have \(a / b = a / (-c) = (-a) / c\), which means \(x\) is negative by \cref{i:4.2.6}.
          \end{itemize}
    \item If \(a < 0\), then by \cref{i:4.1.4} \(a = -c\) for some \(c \in \Z^+\).
          Now we consider \(b\):
          \begin{itemize}
            \item If \(b > 0\), then \(a / b = (-c) / b\), which means \(x\) is negative by \cref{i:4.2.6}.
            \item If \(b < 0\), then by \cref{i:4.1.4} \(b = -d\) for some \(d \in \Z^+\).
                  Thus by \cref{i:4.2.1} we have \(a / b = (-c) / (-d) = (-1) / (-1) \times (c / d) = c / d\), which means \(x\) is positive by \cref{i:4.2.6}.
          \end{itemize}
  \end{itemize}
  From all cases above we conclude that at least one of (a), (b), (c) is true.

  Now we show that at most one of (a), (b), (c) is true.
  \begin{itemize}
    \item If \(x\) is both positive and equal to \(0\), then by \cref{i:ac:4.2.2,i:4.2.6}, \(x = a / b = 0 / 1\), where \(a, b \in \Z^+\).
          But by \cref{i:ac:4.2.2}, \(a / b = 0 / 1\) means \(a = 0\), contradicted to \(a > 0\).
    \item If \(x\) is both negative and equal to \(0\), then by \cref{i:ac:4.2.2,i:4.2.6}, \(x = (-a) / b = 0 / 1\), where \(a, b \in \Z^+\).
          But by \cref{i:ac:4.2.2} \((-a) / b = 0 / 1\) means \(-a = 0\), so that \(a = 0\) by \cref{i:ac:4.1.8}, contradicted to \(a > 0\).
    \item If \(x\) is both positive and negative, then by \cref{i:4.2.6}, \(x = a / b = (-c) / d\), where \(a, b, c, d \in \Z^+\).
          By \cref{i:4.2.1} \(a / b = (-c) / d\) means \(ad = b(-c) = (-1)(bc)\).
          By \cref{i:2.3.3} we know that \(ad\) and \(bc\) are positive.
          But by \cref{i:4.1.4} we know that \((-1)(bc) = -(bc)\) is negative, which contradict to \cref{i:4.1.5}.
  \end{itemize}
  From all cases above we conclude that no more than one of (a), (b), (c) is true at the same time.
\end{proof}

\begin{defn}[Ordering of the rationals]\label{i:4.2.8}
  Let \(x\) and \(y\) be rational numbers.
  We say that \(x > y\) iff \(x - y\) is a positive rational number, and \(x < y\) iff \(x - y\) is a negative rational number.
  We write \(x \geq y\) iff either \(x > y\) or \(x = y\), and similarly define \(x \leq y\) iff either \(x < y\) or \(x = y\).
\end{defn}

\begin{ac}\label{i:ac:4.2.6}
  If \(x\) and \(y\) are two positive rationals, then \(x + y\) is also a positive rational number.
  If \(x\) and \(y\) are two negative rationals, then \(x + y\) is also a negative rational number.
\end{ac}

\begin{proof}[\pf{i:ac:4.2.6}]
  We first show that if \(x\) and \(y\) are two positive rationals, then \(x + y\) is also positive.
  By \cref{i:4.2.6} we have \(x = a / b\) and \(y = c / d\) where \(a, b, c, d \in \Z^+\).
  Then by \cref{i:4.2.2} we have \(x + y = (ad + bc) / bd\).
  By \cref{i:2.3.3} we know that \(ad, bc, bd \in \Z^+\).
  Since \(ad, bc \in \Z^+\), by \cref{i:2.2.8} we know that \(ad + bc \in \Z^+\).
  Thus by \cref{i:4.2.6} we know that \(x + y = (ad + bc) / bd\) is a positive rational number.

  Now we show that if \(x\) and \(y\) are two negative rationals, then \(x + y\) is also negative.
  By \cref{i:4.2.6} we have \(x = (-a) / b\) and \(y = (-c) / d\) where \(a, b, c, d \in \Z^+\).
  Then by \cref{i:4.2.2} we have \(x + y = ((-a)d + b(-c)) / bd\).
  By \cref{i:ac:4.1.5,i:4.1.6} we have
  \[
    (-a)d + b(-c) = (-1) (ad) + (-1) (cb) = (-1)(ad + cb) = -(ad + cb).
  \]
  By \cref{i:2.3.3} we know that \(ad, cb, bd \in \Z^+\).
  Since \(ad, cb \in \Z^+\), by \cref{i:2.2.8} we have \(ad + cb \in \Z^+\).
  Thus by \cref{i:4.1.4} we have \(-(ad + cb) \in \Z^-\) and by \cref{i:4.2.6} \(x + y = -(ad + cb) / bd\) is a negative rational number.
\end{proof}

\begin{ac}\label{i:ac:4.2.7}
  Let \(x\) and \(y\) be two rationals.
  If \(x\) and \(y\) are positive, then \(xy\) is positive.
  If \(x\) and \(y\) are negative, then \(xy\) is positive.
\end{ac}

\begin{proof}[\pf{i:ac:4.2.7}]
  We first show that if \(x\) and \(y\) are two positive rationals, then \(xy\) is a positive rational number.
  By \cref{i:4.2.6} we know that \(x = a / b\) and \(y = c / d\) where \(a, b, c, d \in \Z^+\).
  By \cref{i:4.2.2} we have \(xy = ac / bd\).
  By \cref{i:2.3.3} we have \(ac, bd \in \Z^+\), thus by \cref{i:4.2.6} we know that \(xy\) is a positive rational number.

  Now we show that if \(x\) and \(y\) are two negative rationals, then \(xy\) is a positive rational number.
  By \cref{i:4.2.6} we know that \(x = (-a) / b\) and \(y = (-c) / d\) where \(a, b, c, d \in \Z^+\).
  By \cref{i:4.2.2} we have \(xy = (-a)(-c) / bd\).
  By \cref{i:ac:4.1.7} we have \((-a)(-c) = ac\).
  By \cref{i:2.3.3} we have \(ac, bd \in \Z^+\), thus by \cref{i:4.2.6} we know that \(xy\) is a positive rational number.
\end{proof}

\begin{ac}\label{i:ac:4.2.8}
  Let \(x\) and \(y\) be two rationals.
  If \(x\) is negative and \(y\) is positive, then \(xy\) is negative.
  If \(x\) is positive and \(y\) is negative, then \(xy\) is negative.
\end{ac}

\begin{proof}[\pf{i:ac:4.2.8}]
  By \cref{i:4.2.4} we know that \(xy = yx\), thus we only need to show that if \(x\) is negative and \(y\) is positive, then \(xy\) is negative.
  By \cref{i:4.2.6} we know that \(x = (-a) / b\) and \(y = c / d\) where \(a, b, c, d \in \Z^+\).
  By \cref{i:4.2.2} we have \(xy = (-a)c / bd\).
  By \cref{i:ac:4.1.5} we have \((-a)c = -(ac)\).
  By \cref{i:2.3.3} we know that \(ac, bd \in \Z^+\), thus by \cref{i:4.2.6} we know that \(xy\) is a negative rational number.
\end{proof}

\begin{ac}\label{i:ac:4.2.9}
  \(x\) is a positive rational number iff \(x > 0\).
  \(x\) is a negative rational number iff \(x < 0\).
\end{ac}

\begin{proof}[\pf{i:ac:4.2.9}]
  By \cref{i:4.2.1} we have \(x = x - 0\).
  Thus \(x\) is a positive rational number iff \(x - 0\) is a positive rational number, iff \(x > 0\) (\cref{i:4.2.8}).
  Similarly \(x\) is a negative rational number iff \(x - 0\) is a negative rational number, iff \(x < 0\) (\cref{i:4.2.8}).
\end{proof}

\begin{prop}[Basic properties of order on the rationals]\label{i:4.2.9}
  Let \(x, y, z\) be rational numbers.
  Then the following properties hold.
  \begin{enumerate}
    \item (Order trichotomy)
          Exactly one of the three statements \(x = y\), \(x < y\), or \(x > y\) is true.
    \item (Order is anti-symmetric)
          One has \(x < y\) iff \(y > x\).
    \item (Order is transitive)
          If \(x < y\) and \(y < z\), then \(x < z\).
    \item (Addition preserves order)
          If \(x < y\), then \(x + z < y + z\).
    \item (Positive multiplication preserves order)
          If \(x < y\) and \(z\) is positive, then \(xz < yz\).
  \end{enumerate}
\end{prop}

\begin{proof}[\pf{i:4.2.9}(a)]
  By \cref{i:4.2.7} \(x - y\) satisfy exactly one of the following three statements:
  \begin{itemize}
    \item \(x - y = 0\).
          Then by \cref{i:4.2.4} we have \(x = y\).
    \item \(x - y\) is a positive rational number.
          Then by \cref{i:4.2.8} we have \(x > y\).
    \item \(x - y\) is a negative rational number.
          Then by \cref{i:4.2.8} we have \(x < y\).
  \end{itemize}
\end{proof}

\begin{proof}[\pf{i:4.2.9}(b)]
  Since
  \begin{align*}
    x - y & = -(-(x - y))           &  & \by{i:ac:4.2.5} \\
          & = -((-1) (x - y))       &  & \by{i:ac:4.2.5} \\
          & = -((-1) (x + (-y)))    &  & \by{i:ac:4.2.4} \\
          & = -((-1) x + (-1) (-y)) &  & \by{i:4.2.4}    \\
          & = -((-x) + y)           &  & \by{i:ac:4.2.5} \\
          & = -(y + (-x))           &  & \by{i:4.2.4}    \\
          & = -(y - x)              &  & \by{i:ac:4.2.4} \\
          & = (-1)(y - x),          &  & \by{i:ac:4.2.5}
  \end{align*}
  we have
  \begin{align*}
         & x < y                                                    \\
    \iff & x - y \in \Q^-           &  & \by{i:4.2.8}               \\
    \iff & (-1)(x - y) \in \Q^+     &  & \by{i:ac:4.2.7,i:ac:4.2.8} \\
    \iff & (-1)(-1)(y - x) \in \Q^+ &  & \by{i:4.2.3}               \\
    \iff & y - x \in \Q^+           &  & \by{i:ac:4.2.5}            \\
    \iff & y > x.                   &  & \by{i:4.2.8}
  \end{align*}
\end{proof}

\begin{proof}[\pf{i:4.2.9}(c)]
  We have
  \begin{align*}
             & (x < y) \land (y < z)                                        \\
    \implies & (x - y \in \Q^-) \land (y - z \in \Q^-) &  & \by{i:4.2.8}    \\
    \implies & (x - y) + (y - z) \in \Q^-              &  & \by{i:ac:4.2.6} \\
    \implies & x + z \in \Q^-                          &  & \by{i:4.2.4}    \\
    \implies & x < z.                                  &  & \by{i:4.2.8}
  \end{align*}
\end{proof}

\begin{proof}[\pf{i:4.2.9}(d)]
  We have
  \begin{align*}
             & x < y                                                 \\
    \implies & x - y \in \Q^-                   &  & \by{i:4.2.8}    \\
    \implies & x + (-y) \in \Q^-                &  & \by{i:ac:4.2.4} \\
    \implies & x + z + (-z) + (-y) \in \Q^-     &  & \by{i:4.2.4}    \\
    \implies & x + z + (-y) + (-z) \in \Q^-     &  & \by{i:4.2.4}    \\
    \implies & x + z + (-1) y + (-1) z \in \Q^- &  & \by{i:ac:4.2.5} \\
    \implies & x + z + (-1) (y + z) \in \Q^-    &  & \by{i:4.2.4}    \\
    \implies & x + z + (-(y + z)) \in \Q^-      &  & \by{i:ac:4.2.5} \\
    \implies & x + z - (y + z) \in \Q^-         &  & \by{i:ac:4.2.4} \\
    \implies & x + z < y + z.                   &  & \by{i:4.2.8}
  \end{align*}
\end{proof}

\begin{proof}[\pf{i:4.2.9}(e)]
  We have
  \begin{align*}
             & x < y                                  \\
    \implies & x - y \in \Q^-    &  & \by{i:4.2.8}    \\
    \implies & (x - y)z \in \Q^- &  & \by{i:ac:4.2.8} \\
    \implies & xz - yz \in \Q^-  &  & \by{i:4.2.4}    \\
    \implies & xz < yz.          &  & \by{i:4.2.8}
  \end{align*}
\end{proof}

\begin{rmk}\label{i:4.2.10}
  The above five properties in \cref{i:4.2.9}, combined with the field axioms in \cref{i:4.2.4}, have a name:
  they assert that the rationals \(\Q\) form an \emph{ordered field}.
  It is important to keep in mind that \cref{i:4.2.9}(e) only works when \(z\) is positive, see \cref{i:ex:4.2.6}.
\end{rmk}

\exercisesection

\begin{ex}\label{i:ex:4.2.1}
  Show that the definition of equality for the rational numbers is reflexive, symmetric, and transitive.
\end{ex}

\begin{proof}[\pf{i:ex:4.2.1}]
  See \cref{i:ac:4.2.1}.
\end{proof}

\begin{ex}\label{i:ex:4.2.2}
  Prove the remaining components of \cref{i:4.2.3}.
\end{ex}

\begin{proof}[\pf{i:ex:4.2.2}]
  See \cref{i:4.2.3}.
\end{proof}

\begin{ex}\label{i:ex:4.2.3}
  Prove the remaining components of \cref{i:4.2.4}.
\end{ex}

\begin{proof}[\pf{i:ex:4.2.3}]
  See \cref{i:4.2.4}.
\end{proof}

\begin{ex}\label{i:ex:4.2.4}
  Prove \cref{i:4.2.7}.
\end{ex}

\begin{proof}[\pf{i:ex:4.2.4}]
  See \cref{i:4.2.7}.
\end{proof}

\begin{ex}\label{i:ex:4.2.5}
  Prove \cref{i:4.2.9}.
\end{ex}

\begin{proof}[\pf{i:ex:4.2.5}]
  See \cref{i:4.2.9}.
\end{proof}

\begin{ex}\label{i:ex:4.2.6}
  Show that if \(x, y, z\) are rational numbers such that \(x < y\) and \(z\) is negative, then \(xz > yz\).
\end{ex}

\begin{proof}[\pf{i:ex:4.2.6}]
  We have
  \begin{align*}
             & x < y                                  \\
    \implies & x - y \in \Q^-    &  & \by{i:4.2.8}    \\
    \implies & (x - y)z \in \Q^+ &  & \by{i:ac:4.2.7} \\
    \implies & xz - yz \in \Q^+  &  & \by{i:4.2.4}    \\
    \implies & xz > yz.          &  & \by{i:4.2.8}
  \end{align*}
\end{proof}
