\section{Absolute value and exponentiation}\label{sec:4.3}

\begin{defn}[Absolute value]\label{4.3.1}
  If \(x\) is a rational number, the \emph{absolute value} \(\abs{x}\) of \(x\) is defined as follows.
  If \(x\) is positive, then \(\abs{x} \coloneqq x\).
  If \(x\) is negative, then \(\abs{x} \coloneqq -x\).
  If \(x\) is zero, then \(\abs{x} \coloneqq 0\).
\end{defn}

\begin{defn}[Distance]\label{4.3.2}
  Let \(x\) and \(y\) be rational numbers.
  The quantity \(\abs{x - y}\) is called the \emph{distance between \(x\) and \(y\)} and is sometimes denoted \(d(x, y)\), thus \(d(x, y) \coloneqq \abs{x - y}\).
\end{defn}

\begin{prop}[Basic properties of absolute value and distance]\label{4.3.3}
  Let \(x, y, z\) be rational numbers.
  \begin{enumerate}
    \item (Non-degeneracy of absolute value)
          We have \(\abs{x} \geq 0\).
          Also, \(\abs{x} = 0\) iff \(x\) is \(0\).
    \item (Triangle inequality for absolute value)
          We have \(\abs{x + y} \leq \abs{x} + \abs{y}\).
    \item We have the inequalities \(-y \leq x \leq y\) iff \(y \geq \abs{x}\).
          In particular, we have \(-\abs{x} \leq x \leq \abs{x}\).
    \item (Multiplicativity of absolute value)
          We have \(\abs{xy} = \abs{x} \abs{y}\).
          In particular, \(\abs{-x} = \abs{x}\).
    \item (Non-degeneracy of distance)
          We have \(d(x, y) \geq 0\).
          Also, \(d(x, y) = 0\) iff \(x = y\).
    \item (Symmetry of distance)
          \(d(x, y) = d(y, x)\).
    \item (Triangle inequality for distance)
          \(d(x, z) \leq d(x, y) + d(y, z)\).
  \end{enumerate}
\end{prop}

\begin{proof}[\pf{4.3.3}(a)]
  By \cref{4.2.7} we know that exactly one of the three statements is true:
  \begin{itemize}
    \item \(x = 0\).
          Then by \cref{4.3.1} we have \(\abs{x} = 0\).
    \item \(x \in \Q^+\).
          Then by \cref{4.3.1} we have \(\abs{x} = x \in \Q^+\).
          By \cref{ac:4.2.9} we have \(\abs{x} = x > 0\).
    \item \(x \in \Q^-\).
          Then by \cref{4.3.1} we have \(\abs{x} = -x\).
          By \cref{ac:4.2.5,ac:4.2.7} we know that \(-x = (-1)x \in \Q^+\).
          Thus by \cref{ac:4.2.9} we have \(\abs{x} = -x > 0\).
  \end{itemize}
  From all cases above we conclude that \(\abs{x} \geq 0\) and \(\abs{x} = 0 \iff x = 0\).
\end{proof}

\begin{proof}[\pf{4.3.3}(b)]
  By \cref{4.2.7} exactly one of the following three statements is true:
  \begin{itemize}
    \item \(x + y = 0\).
          By \cref{4.3.3}(a) we have \(\abs{x + y} = 0 \leq \abs{x}\) and \(0 \leq \abs{y}\).
          Thus
          \begin{align*}
            \abs{x + y} & = 0 \leq \abs{y}        &  & \by{4.3.3}[a] \\
                        & = 0 + \abs{y}           &  & \by{4.2.4}    \\
                        & \leq \abs{x} + \abs{y}. &  & \by{4.2.9}[d]
          \end{align*}
    \item \(x + y \in \Q^+\).
          By \cref{ac:4.2.6} we know that we cannot have both \(x \in \Q^-\) and \(y \in \Q^-\).
          So we can use \cref{4.2.7} to split into two further cases:
          \begin{itemize}
            \item If exactly one of \(x, y\) is a negative rational number, say \(x\), then by \cref{ac:4.2.9} we have \(x < 0\) and \(y > 0\).
                  Note that \(y \neq 0\), otherwise we would have \(x + y = x \in \Q^-\) by \cref{4.2.4}, which contradicts to \(x + y \in \Q^+\).
                  Since \(x \neq 0\), by \cref{4.3.3}(a) we have \(0 < \abs{x}\).
                  Thus
                  \begin{align*}
                    \abs{x + y} & = x + y              &  & \by{4.3.1}    \\
                                & < 0 + y              &  & \by{4.2.9}[d] \\
                                & < \abs{x} + y        &  & \by{4.2.9}[d] \\
                                & = \abs{x} + \abs{y}. &  & \by{4.3.1}
                  \end{align*}
            \item If none of \(x, y\) are negative rational numbers, then by \cref{ac:4.2.9} we have \(x \geq 0\) and \(y \geq 0\).
                  Thus
                  \begin{align*}
                    \abs{x + y} & = x + y              &  & \by{4.3.1} \\
                                & = \abs{x} + \abs{y}. &  & \by{4.3.1}
                  \end{align*}
          \end{itemize}
    \item \(x + y \in \Q^-\).
          Then by \cref{ac:4.2.7} we have \(-(x + y) \in \Q^+\).
          Since
          \begin{align*}
            -(x + y) & = (-1) (x + y)    &  & \by{ac:4.2.5} \\
                     & = (-1) x + (-1) y &  & \by{4.2.4}    \\
                     & = (-x) + (-y),    &  & \by{ac:4.2.5}
          \end{align*}
          we can use the second case to derive \(\abs{-(x + y)} \leq \abs{-x} + \abs{-y}\).
          But by \cref{4.3.1} we have \(\abs{x + y} = -(x + y) = \abs{-(x + y)}\).
          And \(\abs{-x} = \abs{x}\) since
          \begin{align*}
            \abs{-x} & = \begin{dcases}
                           -x    & \text{if } (-x) > 0 \\
                           0     & \text{if } (-x) = 0 \\
                           -(-x) & \text{if } (-x) < 0
                         \end{dcases} &  & \by{4.3.1}                   \\
                     & = \begin{dcases}
                           -x & \text{if } x < 0 \\
                           0  & \text{if } x = 0 \\
                           x  & \text{if } x > 0
                         \end{dcases}       &  & \by{ac:4.2.5,ex:4.2.6} \\
                     & = \abs{x}.                     &  & \by{4.3.1}
          \end{align*}
          Similarly \(\abs{-y} = \abs{y}\).
          Thus we have \(\abs{x + y} \leq \abs{x} + \abs{y}\).
  \end{itemize}
  For all cases above we conclude that \(\abs{x + y} \leq \abs{x} + \abs{y}\).
\end{proof}

\begin{proof}[\pf{4.3.3}(c)]
  We have
  \begin{align*}
         & -y \leq x \leq y                                \\
    \iff & (x \leq y) \land (-x \leq y) &  & \by{ex:4.2.6} \\
    \iff & \abs{x} \leq y.              &  & \by{4.3.1}
  \end{align*}
  In particular, we have \(\abs{x} \leq \abs{x} \iff -\abs{x} \leq x \leq \abs{x}\).
\end{proof}

\begin{proof}[\pf{4.3.3}(d)]
  By \cref{4.2.7} we know that exactly one of the following three statements is true:
  \begin{itemize}
    \item \(x = 0\).
          Then we have
          \begin{align*}
            \abs{0y} & = \abs{0}          &  & \by{4.2.2} \\
                     & = 0                &  & \by{4.3.1} \\
                     & = 0 \abs{y}        &  & \by{4.2.2} \\
                     & = \abs{0} \abs{y}. &  & \by{4.3.1}
          \end{align*}
    \item \(x \in \Q^+\).
          By \cref{4.2.7} again we know that exactly one of the following three statements is true:
          \begin{itemize}
            \item \(y = 0\).
                  By \cref{4.2.4} we know that \(xy = yx\), thus this is the same case as \(x = 0\).
            \item \(y \in \Q^+\).
                  By \cref{ac:4.2.7} we know that \(xy \in \Q^+\).
                  Thus
                  \begin{align*}
                    \abs{xy} & = xy               &  & \by{4.3.1}    \\
                             & = \abs{x} \abs{y}. &  & \by{4.3.3}[a]
                  \end{align*}
            \item \(y \in \Q^-\).
                  By \cref{ac:4.2.8} we know that \(xy \in \Q^-\).
                  Thus
                  \begin{align*}
                    \abs{xy} & = -xy              &  & \by{4.3.1}    \\
                             & = x(-y)            &  & \by{ac:4.2.5} \\
                             & = \abs{x} \abs{y}. &  & \by{4.3.1}    \\
                  \end{align*}
          \end{itemize}
    \item \(x \in \Q^-\).
          By \cref{4.2.7}, exactly one of the following three statements is true:
          \begin{itemize}
            \item \(y = 0\).
                  By \cref{4.2.4} we know that \(xy = yx\), thus this is the same case as \(x = 0\).
            \item \(y \in \Q^+\).
                  By \cref{4.2.4} we know that \(xy = yx\), thus this is the same case as \(x \in \Q^+\) and \(y \in \Q^-\).
            \item \(y \in \Q^-\).
                  By \cref{ac:4.2.7}, \(xy\) is a positive.
                  Thus
                  \begin{align*}
                    \abs{xy} & = xy               &  & \by{4.3.1}    \\
                             & = (-1)(-1)xy       &  & \by{4.2.2}    \\
                             & = (-1)x(-1)y       &  & \by{4.2.4}    \\
                             & = (-x)(-y)         &  & \by{ac:4.2.5} \\
                             & = \abs{x} \abs{y}. &  & \by{4.3.1}
                  \end{align*}
          \end{itemize}
  \end{itemize}
  From all cases above we conclude that \(\abs{xy} = \abs{x} \abs{y}\).
  In particular, we have
  \begin{align*}
    \abs{-x} & = \abs{(-1)x}      &  & \by{ac:4.2.5}                 \\
             & = \abs{-1} \abs{x} &  & \text{(from the proof above)} \\
             & = -(-1) \abs{x}    &  & \by{4.3.1}                    \\
             & = 1 \abs{x}        &  & \by{ac:4.1.6}                 \\
             & = \abs{x}.         &  & \by{4.2.4}
  \end{align*}
\end{proof}

\begin{proof}[\pf{4.3.3}(e)]
  Since \(x - y \in \Q\), by \cref{4.3.3}(a) we have \(d(x, y) = \abs{x - y} \geq 0\) and
  \begin{align*}
         & d(x, y) = 0                        \\
    \iff & \abs{x - y} = 0 &  & \by{4.3.2}    \\
    \iff & x - y = 0       &  & \by{4.3.3}[a] \\
    \iff & x = y.          &  & \by{4.2.4}
  \end{align*}
\end{proof}

\begin{proof}[\pf{4.3.3}(f)]
  We have
  \begin{align*}
    d(x, y) & = \abs{x - y}    &  & \by{4.3.2}    \\
            & = \abs{-(x - y)} &  & \by{4.3.3}[d] \\
            & = \abs{y - x}    &  & \by{4.2.4}    \\
            & = d(y, x).       &  & \by{4.3.2}
  \end{align*}
\end{proof}

\begin{proof}[\pf{4.3.3}(g)]
  We have
  \begin{align*}
    d(x, z) & = \abs{x - z}                  &  & \by{4.3.2}    \\
            & = \abs{x - y + y - z}          &  & \by{4.2.4}    \\
            & \leq \abs{x - y} + \abs{y - z} &  & \by{4.3.3}[b] \\
            & = d(x, y) + d(y, z).           &  & \by{4.3.2}
  \end{align*}
\end{proof}

\begin{ac}\label{ac:4.3.1}
  Let \(x, y\) be rational numbers.
  Then \(\abs{x} - \abs{y} \leq \abs{x + y}\).
\end{ac}

\begin{proof}[\pf{ac:4.3.1}]
  \begin{align*}
             & \abs{x + y + (-y)} \leq \abs{x + y} + \abs{-y}               &  & \by{4.3.3}[b] \\
    \implies & \abs{x} \leq \abs{x + y} + \abs{-y}                          &  & \by{4.2.4}    \\
    \implies & \abs{x} \leq \abs{x + y} + \abs{y}                           &  & \by{4.3.3}[d] \\
    \implies & \abs{x} + (-\abs{y}) \leq \abs{x + y} + \abs{y} + (-\abs{y}) &  & \by{4.2.9}[d] \\
    \implies & \abs{x} + (-\abs{y}) \leq \abs{x + y}                        &  & \by{4.2.4}    \\
    \implies & \abs{x} - \abs{y} \leq \abs{x + y}.                          &  & \by{ac:4.2.4}
  \end{align*}
\end{proof}

\begin{defn}[\(\varepsilon\)-closeness]\label{4.3.4}
  Let \(\varepsilon > 0\) be a rational number, and let \(x\), \(y\) be rational numbers.
  We say that \(y\) is \emph{\(\varepsilon\)-close} to \(x\) iff we have \(d(y, x) \leq \varepsilon\).
\end{defn}

\begin{rmk}\label{4.3.5}
  This definition is not standard in mathematics textbooks;
  we will use it as ``scaffolding'' to construct the more important notions of limits (and of Cauchy sequences) later on, and once we have those more advanced notions we will discard the notion of \(\varepsilon\)-close.
\end{rmk}

\begin{note}
  We do not bother defining a notion of \(\varepsilon\)-close when \(\varepsilon\) is zero or negative, because if \(\varepsilon\) is zero then \(x\) and \(y\) are only \(\varepsilon\)-close when they are equal, and when \(\varepsilon\) is negative then \(x\) and \(y\) are never \(\varepsilon\)-close.
\end{note}

\begin{note}
  In any event it is a long-standing tradition in analysis that the Greek letters \(\varepsilon\), \(\delta\) should only denote small positive numbers.
\end{note}

\setcounter{thm}{6}
\begin{prop}\label{4.3.7}
  Let \(x, y, z, w\) be rational numbers.
  (extended to cover the \(0\)-close case)
  \begin{enumerate}
    \item If \(x = y\), then \(x\) is \(\varepsilon\)-close to \(y\) for every \(\varepsilon > 0\).
          Conversely, if \(x\) is \(\varepsilon\)-close to \(y\) for every \(\varepsilon > 0\), then we have \(x = y\).
    \item Let \(\varepsilon > 0\).
          If \(x\) is \(\varepsilon\)-close to \(y\), then \(y\) is \(\varepsilon\)-close to \(x\).
    \item Let \(\varepsilon, \delta > 0\).
          If \(x\) is \(\varepsilon\)-close to \(y\), and \(y\) is \(\delta\)-close to \(z\), then \(x\) and \(z\) are \((\varepsilon + \delta)\)-close.
    \item Let \(\varepsilon, \delta > 0\).
          If \(x\) and \(y\) are \(\varepsilon\)-close, and \(z\) and \(w\) are \(\delta\)-close, then \(x + z\) and \(y + w\) are \((\varepsilon + \delta)\)-close, and \(x - z\) and \(y - w\) are also \((\varepsilon + \delta)\)-close.
    \item Let \(\varepsilon > 0\).
          If \(x\) and \(y\) are \(\varepsilon\)-close, they are also \(\varepsilon'\)-close for every \(\varepsilon' > \varepsilon\).
    \item Let \(\varepsilon > 0\).
          If \(y\) and \(z\) are both \(\varepsilon\)-close to \(x\), and \(w\) is between \(y\) and \(z\) (i.e., \(y \leq w \leq z\) or \(z \leq w \leq y\)), then \(w\) is also \(\varepsilon\)-close to \(x\).
    \item Let \(\varepsilon > 0\).
          If \(x\) and \(y\) are \(\varepsilon\)-close, and \(z\) is non-zero, then \(xz\) and \(yz\) are \(\varepsilon\abs{z}\)-close.
    \item Let \(\varepsilon, \delta > 0\).
          If \(x\) and \(y\) are \(\varepsilon\)-close, and \(z\) and \(w\) are \(\delta\)-close, then \(xz\) and \(yw\) are \((\varepsilon\abs{z} + \delta\abs{x} + \varepsilon\delta)\)-close.
  \end{enumerate}
\end{prop}

\begin{proof}[\pf{4.3.7}(a)]
  We first show that if \(x = y\), then \(x\) is \(\varepsilon\)-close to \(y\) for every \(\varepsilon \in \Q^+\).
  \begin{align*}
             & x = y                                                                                            \\
    \implies & x - y = 0                                                               &  & \by{4.2.4}          \\
    \implies & d(x, y) = \abs{x - y} = 0                                               &  & \by{4.3.2,4.3.3}[a] \\
    \implies & \forall \varepsilon \in \Q^+, d(x, y) \leq \varepsilon                  &  & \by{ac:4.2.9}       \\
    \implies & \forall \varepsilon \in \Q^+, x \text{ is \(\varepsilon\)-close to } y. &  & \by{4.3.4}
  \end{align*}

  Now we show that if \(x\) is \(\varepsilon\)-close to \(y\) for every \(\varepsilon \in \Q^+\), then \(x = y\).
  Suppose for sake of contradiction that \(x \neq y\).
  Then by \cref{4.3.3}(e) we have \(d(x, y) > 0\).
  But then we have \(d(x, y) \in \Q^+\), so \(d(x, y) < d(x, y)\), a contradiction.
  Thus we must have \(x = y\).
\end{proof}

\begin{proof}[\pf{4.3.7}(b)]
  We have
  \begin{align*}
         & x \text{ is \(\varepsilon\)-close to } y                     \\
    \iff & d(x, y) \leq \varepsilon                  &  & \by{4.3.4}    \\
    \iff & d(y, x) \leq \varepsilon                  &  & \by{4.3.3}[f] \\
    \iff & y \text{ is \(\varepsilon\)-close to } x. &  & \by{4.3.4}
  \end{align*}
\end{proof}

\begin{proof}[\pf{4.3.7}(c)]
  We have
  \begin{align*}
             & (x \text{ is \(\varepsilon\)-close to } y) \land (y \text{ is \(\delta\)-close to } z)                      \\
    \implies & \big(d(x, y) \leq \varepsilon\big) \land \big(d(y, z) \leq \delta\big)                 &  & \by{4.3.4}      \\
    \implies & d(x, y) + d(y, z) \leq \varepsilon + d(y, z) \leq \varepsilon + \delta                 &  & \by{4.2.9}[c,d] \\
    \implies & d(x, z) \leq d(x, y) + d(y, z) \leq \varepsilon + \delta                               &  & \by{4.3.3}[g]   \\
    \implies & x \text{ is \((\varepsilon + \delta)\)-close to } z.                                   &  & \by{4.3.4}
  \end{align*}
\end{proof}

\begin{proof}[\pf{4.3.7}(d)]
  Since
  \begin{align*}
             & (x \text{ is \(\varepsilon\)-close to } y) \land (z \text{ is \(\delta\)-close to } w)                      \\
    \implies & \big(d(x, y) \leq \varepsilon\big) \land \big(d(z, w) \leq \delta\big)                 &  & \by{4.3.4}      \\
    \implies & d(x, y) + d(z, w) \leq \varepsilon + d(z, w) \leq \varepsilon + \delta                 &  & \by{4.2.9}[c,d] \\
    \implies & \abs{x - y} + \abs{z - w} \leq \varepsilon + \delta,                                   &  & \by{4.3.2}
  \end{align*}
  we have
  \begin{align*}
             & \abs{x - y} + \abs{z - w} \leq \varepsilon + \delta                                             \\
    \implies & \abs{x - y + z - w} \leq \abs{x - y} + \abs{z - w} \leq \varepsilon + \delta &  & \by{4.3.3}[b] \\
    \implies & \abs{x + z - (y + w)} \leq \varepsilon + \delta                              &  & \by{4.2.4}    \\
    \implies & d(x + z, y + w) \leq \varepsilon + \delta                                    &  & \by{4.3.2}    \\
    \implies & (x + z) \text{ is \((\varepsilon + \delta)\)-close to } (y + w)              &  & \by{4.3.4}
  \end{align*}
  and
  \begin{align*}
             & \abs{x - y} + \abs{z - w} \leq \varepsilon + \delta                                             \\
    \implies & \abs{x - y} + \abs{-(z - w)} \leq \varepsilon + \delta                       &  & \by{4.3.3}[d] \\
    \implies & \abs{x - y} + \abs{w - z} \leq \varepsilon + \delta                          &  & \by{4.2.4}    \\
    \implies & \abs{x - y + w - z} \leq \abs{x - y} + \abs{w - z} \leq \varepsilon + \delta &  & \by{4.3.3}[b] \\
    \implies & \abs{x - z - (y - w)} \leq \varepsilon + \delta                              &  & \by{4.2.4}    \\
    \implies & d(x - z, y - w) \leq \varepsilon + \delta                                    &  & \by{4.3.2}    \\
    \implies & (x - z) \text{ is \((\varepsilon + \delta)\)-close to } (y - w).             &  & \by{4.3.4}
  \end{align*}
\end{proof}

\begin{proof}[\pf{4.3.7}(e)]
  We have
  \begin{align*}
             & (x \text{ is \(\varepsilon\)-close to } y) \land (\varepsilon' > \varepsilon)                      \\
    \implies & \big(d(x, y) \leq \varepsilon\big) \land (\varepsilon' > \varepsilon)         &  & \by{4.3.4}      \\
    \implies & d(x, y) < \varepsilon'                                                        &  & \by{4.2.9}[b,c] \\
    \implies & x \text{ is \(\varepsilon'\)-close to } y.                                    &  & \by{4.3.4}
  \end{align*}
\end{proof}

\begin{proof}[\pf{4.3.7}(f)]
  We have
  \begin{align*}
             & (y \text{ is } \varepsilon\text{-close to } x) \land (z \text{ is } \varepsilon\text{-close to } x)                    \\
    \implies & \big(d(y, x) \leq \varepsilon\big) \land \big(d(z, x) \leq \varepsilon\big)                         &  & \by{4.3.4}    \\
    \implies & (\abs{y - x} \leq \varepsilon) \land (\abs{z - x} \leq \varepsilon)                                 &  & \by{4.3.2}    \\
    \implies & (-\varepsilon \leq y - x \leq \varepsilon) \land (-\varepsilon \leq z - x \leq \varepsilon).        &  & \by{4.3.3}[c]
  \end{align*}
  If \(y \leq w \leq z\), then we have
  \begin{align*}
             & y \leq w \leq z                                                                   \\
    \implies & y - x \leq w - x \leq z - x                                    &  & \by{4.2.9}[d] \\
    \implies & -\varepsilon \leq y - x \leq w - x \leq z - x \leq \varepsilon &  & \by{4.2.9}[c] \\
    \implies & \abs{w - x} \leq \varepsilon                                   &  & \by{4.3.3}[c] \\
    \implies & d(w, x) \leq \varepsilon                                       &  & \by{4.3.2}    \\
    \implies & w \text{ is } \varepsilon\text{-close to } x.                  &  & \by{4.3.4}
  \end{align*}
  The case for \(z \leq w \leq y\) can be proven similarly.
\end{proof}

\begin{proof}[\pf{4.3.7}(g)]
  We have
  \begin{align*}
             & (x \text{ is } \varepsilon\text{-close to } y) \land (z \neq 0)                       \\
    \implies & (\abs{x - y} \leq \varepsilon) \land (z \neq 0)                 &  & \by{4.3.2,4.3.4} \\
    \implies & (\abs{x - y} \leq \varepsilon) \land (\abs{z} > 0)              &  & \by{4.3.3}[a]    \\
    \implies & \abs{x - y} \abs{z} \leq \varepsilon \abs{z}                    &  & \by{4.2.9}[e]    \\
    \implies & \abs{(x - y)z} \leq \varepsilon \abs{z}                         &  & \by{4.3.3}[d]    \\
    \implies & \abs{xz - yz} \leq \varepsilon \abs{z}                          &  & \by{4.2.4}       \\
    \implies & d(xz, yz) \leq \varepsilon \abs{z}                              &  & \by{4.3.2}       \\
    \implies & xz \text{ is } (\varepsilon \abs{z})\text{-close to } yz.       &  & \by{4.3.4}
  \end{align*}
\end{proof}

\begin{proof}[\pf{4.3.7}(h)]
  If we write \(a \coloneqq y - x\), then we have \(y = x + a\) and that \(\abs{a} \leq \varepsilon\).
  Similarly, if we define \(b \coloneqq w - z\), then \(w = z + b\) and \(\abs{b} \leq \delta\).

  Since \(y = x + a\) and \(w = z + b\), by \cref{4.2.4} we have
  \[
    yw = (x + a)(z + b) = xz + az + xb + ab.
  \]
  Thus by \cref{4.3.3}(b)(d) we have
  \[
    \abs{yw - xz} = \abs{az + bx + ab} \leq \abs{az} + \abs{bx} + \abs{ab} = \abs{a}\abs{z} + \abs{b}\abs{x} + \abs{a}\abs{b}.
  \]
  Since \(\abs{a} \leq \varepsilon\) and \(\abs{b} \leq \delta\), we thus have
  \[
    \abs{yw - xz} \leq \varepsilon\abs{z} + \delta\abs{x} + \varepsilon\delta
  \]
  and thus that \(yw\) and \(xz\) are \((\varepsilon\abs{z} + \delta\abs{x} + \varepsilon\delta)\)-close.
\end{proof}

\begin{rmk}\label{4.3.8}
  One should compare statements (a)-(c) of \cref{4.3.7} with the reflexive, symmetric, and transitive axioms of equality.
  It is often useful to think of the notion of ``\(\varepsilon\)-close'' as an approximate substitute for that of equality in analysis.
\end{rmk}

\begin{defn}[Exponentiation to a natural number]\label{4.3.9}
  Let \(x\) be a rational number.
  To raise \(x\) to the power \(0\), we define \(x^0 \coloneqq 1\);
  in particular we define \(0^0 \coloneqq 1\).
  Now suppose inductively that \(x^n\) has been defined for some natural number \(n\), then we define \(x^{n+1} \coloneqq x^n \times x\).
\end{defn}

\begin{prop}[Properties of exponentiation, I]\label{4.3.10}
  Let \(x, y\) be rational numbers, and let \(n, m\) be natural numbers.
  \begin{enumerate}
    \item We have \(x^n x^m = x^{n + m}\), \((x^n)^m = x^{nm}\), and \((xy)^n = x^n y^n\).
    \item Suppose \(n > 0\).
          Then we have \(x^n = 0\) iff \(x = 0\).
    \item If \(x \geq y \geq 0\), then \(x^n \geq y^n \geq 0\).
          If \(x > y \geq 0\) and \(n > 0\), then \(x^n > y^n \geq 0\).
    \item We have \(\abs{x^n} = \abs{x}^n\).
  \end{enumerate}
\end{prop}

\begin{proof}[\pf{4.3.10}(a)]
  We first show that \(x^n x^m = x^{n + m}\).
  We induct on \(n\).
  For \(n = 0\), we have
  \begin{align*}
    x^0 x^m & = 1 x^m      &  & \by{4.3.9} \\
            & = x^m        &  & \by{4.2.4} \\
            & = x^{0 + m}. &  & \by{2.2.1}
  \end{align*}
  So the base case holds.
  Suppose inductively that for some \(n \in \N\) we have \(x^n x^m = x^{n + m}\).
  Then for \(n + 1\), we have
  \begin{align*}
    x^{n + 1} x^m & = (x^n x) x^m      &  & \by{4.3.9} \\
                  & = x^n (x x^m)      &  & \by{4.2.4} \\
                  & = x^n (x^m x)      &  & \by{4.2.4} \\
                  & = (x^n x^m) x      &  & \by{4.2.4} \\
                  & = x^{n + m} x      &  & \byIH      \\
                  & = x^{(n + m) + 1}  &  & \by{4.3.9} \\
                  & = x^{n + (m + 1)}  &  & \by{2.2.5} \\
                  & = x^{n + (1 + m)}  &  & \by{2.2.4} \\
                  & = x^{(n + 1) + m}. &  & \by{2.2.5}
  \end{align*}
  This closes the induction.

  Next we show that \((x^n)^m = x^{nm}\).
  We induct on \(m\).
  For \(m = 0\), we have
  \begin{align*}
    (x^n)^0 & = 1       &  & \by{4.3.9}    \\
            & = x^0     &  & \by{4.3.9}    \\
            & = x^{n0}. &  & \by{ac:2.3.2}
  \end{align*}
  So the base case holds.
  Suppose inductively that for some \(m \in \N\) we have \((x^n)^m = x^{nm}\).
  Then for \(m + 1\), we have
  \begin{align*}
    (x^n)^{m + 1} & = (x^n)^m (x^n) &  & \by{4.3.9}                    \\
                  & = x^{nm} x^n    &  & \byIH                         \\
                  & = x^{nm + n}    &  & \text{(from the proof above)} \\
                  & = x^{n(m + 1)}. &  & \by{ac:2.3.3}
  \end{align*}
  This closes the induction.

  Finally we show that \((xy)^n = x^n y^n\).
  We induct on \(n\).
  For \(n = 0\), we have
  \begin{align*}
    (xy)^0 & = 1        &  & \by{4.3.9} \\
           & = y^0      &  & \by{4.3.9} \\
           & = 1y^0     &  & \by{4.2.4} \\
           & = x^0 y^0. &  & \by{4.3.9}
  \end{align*}
  So the base case holds.
  Suppose inductively that for some \(n \in \N\) we have \((xy)^n = x^n y^n\).
  Then for \(n + 1\), we have
  \begin{align*}
    (xy)^{n + 1} & = (xy)^n (xy)          &  & \by{4.3.9} \\
                 & = (x^n y^n) (xy)       &  & \byIH      \\
                 & = x^n (y^n x) y        &  & \by{4.2.4} \\
                 & = x^n (x y^n) y        &  & \by{4.2.4} \\
                 & = (x^n x)(y^n y)       &  & \by{4.2.4} \\
                 & = x^{n + 1} y^{n + 1}. &  & \by{4.3.9}
  \end{align*}
  This closes the induction.
\end{proof}

\begin{proof}[\pf{4.3.10}(b)]
  We induct on \(n\) and we start with \(n = 1\).
  For \(n = 1\), we have
  \begin{align*}
         & x^1 = 0                   \\
    \iff & x^0 x = 0 &  & \by{4.3.9} \\
    \iff & 1x = 0    &  & \by{4.3.9} \\
    \iff & x = 0.    &  & \by{4.2.4}
  \end{align*}
  So the base case holds.
  Suppose inductively that for some \(n \in \N \setminus \set{0}\) we have \(x^n = 0 \iff x = 0\).
  Then for \(n + 1\), we have
  \begin{align*}
         & x^{n + 1} = 0                                      \\
    \iff & x^n x = 0              &  & \by{4.3.9}             \\
    \iff & (x^n = 0) \lor (x = 0) &  & \by{ac:4.2.7,ac:4.2.8} \\
    \iff & x = 0.                 &  & \byIH
  \end{align*}
  This closes the induction.
\end{proof}

\begin{proof}[\pf{4.3.10}(c)]
  We first show that if \(x \geq y \geq 0\), then \(x^n \geq y^n \geq 0\).
  We induct on \(n\).
  For \(n = 0\), we have
  \begin{align*}
             & x \geq y \geq 0                              \\
    \implies & x^0 = 1 \geq y^0 = 1 \geq 0. &  & \by{4.3.9}
  \end{align*}
  So the base case holds.
  Suppose inductively that for some \(n \in \N\) we have \(x^n \geq y^n \geq 0\).
  Then for \(n + 1\), we have
  \begin{align*}
             & (x^n \geq y^n \geq 0) \land (x \geq y \geq 0)                  &  & \byIH            \\
    \implies & (x^n x \geq y^n x \geq 0x) \land (y^n x \geq y^n y \geq y^n 0) &  & \by{4.2.9}[e]    \\
    \implies & x^n x \geq y^n x \geq y^n y \geq y^n 0                         &  & \by{4.2.9}[c]    \\
    \implies & x^{n + 1} \geq y^{n + 1} \geq 0.                               &  & \by{4.2.2,4.3.9}
  \end{align*}
  This closes the induction.

  Now we show that if \(x > y \geq 0\) and \(n > 0\), then \(x^n > y^n \geq 0\).
  We split into two cases:
  \begin{itemize}
    \item If \(y = 0\), then by \cref{4.3.10}(b) we know that \(x \neq 0 \iff x^n \neq 0\) and \(y = 0 \iff y^n = 0\).
          By \cref{4.2.9}(e) we know that \(x > 0 \implies x^n > 0\).
          Thus by \cref{4.2.9}(c) we have \(x^n > y^n = 0\).
    \item If \(y > 0\), then we have \(x > y > 0\).
          We induct on \(n\) to show that \(x^n > y^n > 0\) and we start with \(n = 1\).
          For \(n = 1\), we have
          \begin{align*}
                     & \begin{dcases}
                         x > y > 0            \\
                         x^1 = x^0 x = 1x = x \\
                         y^1 = y^0 y = 1y = y
                       \end{dcases} &  & \by{4.2.4,4.3.9}      \\
            \implies & x^1 > y^1 > 0.          &  & \by{4.3.9}
          \end{align*}
          So the base case holds.
          Suppose inductively that for some \(n \in \N \setminus \set{0}\) we have \(x^n > y^n > 0\).
          Then for \(n + 1\), we have
          \begin{align*}
                     & (x^n > y^n > 0) \land (x > y > 0)                  &  & \byIH            \\
            \implies & (x^n x > y^n x > 0x) \land (y^n x > y^n y > y^n 0) &  & \by{4.2.9}[e]    \\
            \implies & x^n x > y^n x > y^n y > y^n 0                      &  & \by{4.2.9}[c]    \\
            \implies & x^{n + 1} > y^{n + 1} > 0.                         &  & \by{4.2.2,4.3.9}
          \end{align*}
          This closes the induction.
  \end{itemize}
  From all cases above we conclude that \(x > y \geq 0 \implies x^n > y^n \geq 0\).
\end{proof}

\begin{proof}[\pf{4.3.10}(d)]
  We induct on \(n\).
  For \(n = 0\), we have
  \begin{align*}
    \abs{x^0} & = \abs{1}    &  & \by{4.3.9} \\
              & = 1          &  & \by{4.3.1} \\
              & = \abs{x}^0. &  & \by{4.3.9}
  \end{align*}
  So the base case holds.
  Suppose inductively that for some \(n \in \N\) we have \(\abs{x^n} = \abs{x}^n\).
  Then for \(n + 1\), we have
  \begin{align*}
    \abs{x^{n + 1}} & = \abs{x^n x}       &  & \by{4.3.9}    \\
                    & = \abs{x^n} \abs{x} &  & \by{4.3.3}[d] \\
                    & = \abs{x}^n \abs{x} &  & \byIH         \\
                    & = \abs{x}^{n + 1}.  &  & \by{4.3.9}
  \end{align*}
  This closes the induction.
\end{proof}

\begin{defn}[Exponentiation to a negative number]\label{4.3.11}
  Let \(x\) be a non-zero rational number.
  Then for any negative integer \(-n\), we define \(x^{-n} \coloneqq 1 / x^n\).

  Note that when \(n = 1\), the definition of \(x^{-1}\) provided by \cref{4.3.11} coincides with the reciprocal of \(x\) defined previously, so there is no incompatibility of notation caused by this new definition.
\end{defn}


\begin{ac}\label{ac:4.3.2}
  Let \(x\) be a non-zero rational number.
  Then for any negative integer \(-n\), we have \(x^{-n + 1} = x^{-n} \times x\).
\end{ac}

\begin{proof}[\pf{ac:4.3.2}]
  By \cref{4.1.11}(f), exactly one of the following three statements is true:
  \begin{itemize}
    \item \(-n + 1 = 0\).
          Then by \cref{4.1.6} we have \(n = 1\) and
          \begin{align*}
            x^{-1 + 1} & = x^0       &  & \by{4.1.6} \\
                       & = 1         &  & \by{4.3.9} \\
                       & = x^{-1} x. &  & \by{4.2.4}
          \end{align*}
    \item \(-n + 1 > 0\).
          Then we have
          \begin{align*}
                     & -n + 1 > 0                     \\
            \implies & n < 1      &  & \by{4.1.11}[b] \\
            \implies & n = 0      &  & \by{2.3}       \\
            \implies & -n = 0     &  & \by{ac:4.1.8}
          \end{align*}
          which contradict to \(-n\) is negative.
          Thus we cannot have \(-n + 1 > 0\).
    \item \(-n + 1 < 0\).
          Then by \cref{4.1.11}(d) we have \(n - 1 > 0\) and by \cref{4.3.9} we have \(x^{n - 1} x = x^n\).
          Therefore,
          \begin{align*}
                     & x^{-n + 1} x^{n - 1} = 1              &  & \by{4.2.4} \\
            \implies & (x^{-n + 1} x^{n - 1}) x = 1x         &  & \by{4.2.3} \\
            \implies & x^{-n + 1} (x^{n - 1} x) = 1x         &  & \by{4.2.4} \\
            \implies & x^{-n + 1} x^{n - 1 + 1} = 1x         &  & \by{4.3.9} \\
            \implies & x^{-n + 1} x^n = 1x                   &  & \by{4.1.6} \\
            \implies & (x^{-n + 1} x^n) x^{-n} = (1x) x^{-n} &  & \by{4.2.3} \\
            \implies & x^{-n + 1} (x^n x^{-n}) = 1(x x^{-n}) &  & \by{4.2.4} \\
            \implies & x^{-n + 1} 1 = 1x x^{-n}              &  & \by{4.2.4} \\
            \implies & x^{-n + 1} = x x^{-n}                 &  & \by{4.2.4} \\
            \implies & x^{-n + 1} = x^{-n} x.                &  & \by{4.2.4}
          \end{align*}
  \end{itemize}
  From all cases above we conclude that \(x^{-n + 1} = x^{-n} x\).
\end{proof}

\begin{prop}[Properties of exponentiation, II]\label{4.3.12}
  Let \(x\), \(y\) be nonzero rational numbers, and let \(n\), \(m\) be integers.
  \begin{enumerate}
    \item We have \(x^n x^m = x^{n + m}\), \((x^n)^m = x^{nm}\), and \((xy)^n = x^n y^n\).
    \item If \(x \geq y > 0\), then \(x^n \geq y^n > 0\) if \(n\) is positive, and \(0 < x^n \leq y^n\) if \(n\) is negative.
    \item If \(x, y > 0\), \(n \neq 0\), and \(x^n = y^n\), then \(x = y\).
    \item We have \(\abs{x^n} = \abs{x}^n\).
  \end{enumerate}
\end{prop}

\begin{proof}[\pf{4.3.12}(a)]
  We first show that \(x^n x^m = x^{n + m}\).
  By \cref{4.1.11}(f) exactly one of \(n > 0\), \(n = 0\) or \(n < 0\) is true.
  Similarly, exactly one of \(m > 0\), \(m = 0\) or \(m < 0\) is true, and exactly one of \(n + m > 0\), \(n + m = 0\) or \(n + m < 0\) is true.
  Thus we can split into following cases:
  \begin{itemize}
    \item If \(n = 0\) or \(m = 0\), then we have
          \begin{align*}
            x^0 x^m & = 1 x^m     &  & \by{4.3.9} \\
                    & = x^m       &  & \by{4.2.4} \\
                    & = x^{0 + m} &  & \by{4.1.6}
          \end{align*}
          and
          \begin{align*}
            x^n x^0 & = x^0 x^n    &  & \by{4.2.4}                    \\
                    & = x^{0 + n}  &  & \text{(from the proof above)} \\
                    & = x^{n + 0}. &  & \by{4.1.6}
          \end{align*}
    \item If \(n > 0\) and \(m > 0\), then by \cref{4.3.10}(a) we have \(x^n x^m = x^{n + m}\).
    \item If \(n > 0\), \(m < 0\) and \(n + m \geq 0\), then by \cref{4.1.11}(d) we have \(-m > 0\).
          Thus
          \begin{align*}
            x^n x^m & = \dfrac{1 x^n}{x^{-m}}            &  & \by{4.3.11}    \\
                    & = \dfrac{x^n}{x^{-m}}              &  & \by{4.2.4}     \\
                    & = \dfrac{x^{n + m + (-m)}}{x^{-m}} &  & \by{4.1.6}     \\
                    & = \dfrac{x^{n + m} x^{-m}}{x^{-m}} &  & \by{4.3.10}[a] \\
                    & = x^{n + m}.                       &  & \by{4.2.3}
          \end{align*}
    \item If \(n > 0\), \(m < 0\) and \(n + m < 0\), then by \cref{4.1.11}(d) we have \(-(n + m) > 0\).
          Thus
          \begin{align*}
            x^n x^m & = \dfrac{1 x^n}{x^{-m}}              &  & \by{4.3.11}    \\
                    & = \dfrac{1 x^n}{x^{n + (-n) + (-m)}} &  & \by{4.1.6}     \\
                    & = \dfrac{1 x^n}{x^{n + (-(n + m))}}  &  & \by{ac:4.1.5}  \\
                    & = \dfrac{1 x^n}{x^n x^{-(n + m)}}    &  & \by{4.3.10}[a] \\
                    & = \dfrac{1}{x^{-(n + m)}}            &  & \by{4.2.3}     \\
                    & = x^{n + m}.                         &  & \by{4.3.11}
          \end{align*}
    \item If \(n < 0\) and \(m > 0\), then we can use the previous two cases to conclude that
          \begin{align*}
            x^n x^m & = x^m x^n    &  & \by{4.2.4}                           \\
                    & = x^{m + n}  &  & \text{(from the previous two cases)} \\
                    & = x^{n + m}. &  & \by{4.1.6}
          \end{align*}
    \item If \(n < 0\) and \(m < 0\), then by \cref{4.1.11}(d) we have \(-n > 0\), \(-m > 0\) and \(-(n + m) > 0\).
          Thus
          \begin{align*}
            x^n x^m & = \dfrac{1}{x^{-n} x^{-m}}   &  & \by{4.3.11}    \\
                    & = \dfrac{1}{x^{(-n) + (-m)}} &  & \by{4.3.10}[a] \\
                    & = \dfrac{1}{x^{-(n + m)}}    &  & \by{ac:4.1.5}  \\
                    & = x^{n + m}.                 &  & \by{4.3.11}
          \end{align*}
  \end{itemize}
  From all cases above we conclude that \(x^n x^m = x^{n + m}\).

  Next we show that \((x^n)^m = x^{nm}\).
  By \cref{4.1.5} exactly one of the following two statements is true:
  \begin{itemize}
    \item \(m \geq 0\).
          Then we induct on \(m\) to show that \((x^n)^m = x^{nm}\).
          For \(m = 0\), we have
          \begin{align*}
            (x^n)^0 & = 1       &  & \by{4.3.9} \\
                    & = x^0     &  & \by{4.2.4} \\
                    & = x^{n0}. &  & \by{4.1.6}
          \end{align*}
          So the base case holds.
          Suppose inductively that for some \(m \in \N\) we have \((x^n)^m = x^{nm}\).
          Then for \(n + 1\), we have
          \begin{align*}
            (x^n)^{m + 1} & = (x^n)^m x^n   &  & \by{ac:4.3.2}  \\
                          & = x^{nm} x^n    &  & \byIH          \\
                          & = x^{nm + n}    &  & \by{4.3.12}[a] \\
                          & = x^{n(m + 1)}. &  & \by{ac:2.3.3}
          \end{align*}
          This closes the induction.
    \item \(m < 0\).
          Then by \cref{ac:4.2.7} we have \(-m > 0\) and
          \begin{align*}
            (x^n)^m & = \dfrac{1}{(x^n)^{-m}} &  & \by{4.3.11}                  \\
                    & = \dfrac{1}{x^{n(-m)}}  &  & \text{(from the first case)} \\
                    & = \dfrac{1}{x^{-nm}}    &  & \by{ac:4.1.5}                \\
                    & = x^{nm}.               &  & \by{4.3.11}
          \end{align*}
  \end{itemize}
  From all cases above we conclude that \((x^n)^m = x^{nm}\).

  Finally we show that \((xy)^n = x^n y^n\).
  By \cref{4.1.5} exactly one of the following two statements is true:
  \begin{itemize}
    \item \(n \geq 0\).
          Then by \cref{4.3.10}(a) we know that \((xy)^n = x^n y^n\).
    \item \(n < 0\).
          Then by \cref{ac:4.2.7} we have \(-n > 0\) and
          \begin{align*}
            (xy)^n & = \dfrac{1}{(xy)^{-n}}     &  & \by{4.3.11}    \\
                   & = \dfrac{1}{x^{-n} y^{-n}} &  & \by{4.3.10}[a] \\
                   & = x^n y^n.                 &  & \by{4.3.11}
          \end{align*}
  \end{itemize}
  From all cases above we conclude that \((xy)^n = x^n y^n\).
\end{proof}

\begin{proof}[\pf{4.3.12}(b)]
  By \cref{4.1.5} exactly one of the following two statements is true:
  \begin{itemize}
    \item \(n \geq 0\).
          Then by \cref{4.3.10}(b)(c) we have \(x^n \geq y^n > 0\).
    \item \(n < 0\).
          Then by \cref{ac:4.2.7} we have \(-n > 0\) and
          \begin{align*}
                     & x^{-n} \geq y^{-n} > 0                                                   &  & \by{4.3.10}[b,c] \\
            \implies & \dfrac{1}{x^n} \geq \dfrac{1}{y^n} > 0                                   &  & \by{4.3.11}      \\
            \implies & (\dfrac{1}{x^n} \geq \dfrac{1}{y^n} > 0) \land (x^n > 0) \land (y^n > 0) &  & \by{ac:4.2.8}    \\
            \implies & \dfrac{x^n y^n}{x^n} \geq \dfrac{x^n y^n}{y^n} > 0(x^n y^n)              &  & \by{4.2.9}[e]    \\
            \implies & y^n \geq x^n > 0(x^n y^n)                                                &  & \by{4.2.2}       \\
            \implies & y^n \geq x^n > 0.                                                        &  & \by{4.2.4}
          \end{align*}
  \end{itemize}
\end{proof}

\begin{proof}[\pf{4.3.12}(c)]
  Suppose for sake of contradiction that \(x \neq y\)
  Then by \cref{4.2.9} exactly one of the following two statements is true:
  \begin{enumerate}[label=(\Roman*)]
    \item \(x > y\).
          By hypothesis we know that \(n \neq 0\).
          Then by \cref{4.1.5} exactly one of the following two statements is true:
          \begin{itemize}
            \item \(n > 0\).
                  But by \cref{4.3.10}(c) we must have \(x^n > y^n\), a contradiction.
            \item \(n < 0\).
                  Then by \cref{ac:4.2.7} we have \(-n > 0\) and
                  \begin{align*}
                             & x^{-n} > y^{-n} > 0 &  & \by{4.3.10}[b,c] \\
                    \implies & y^n > x^n > 0,      &  & \by{4.3.12}[b]
                  \end{align*}
                  a contradiction.
          \end{itemize}
    \item \(x < y\).
          By \cref{4.2.9} we know that \(x < y \implies y > x\), which is just the same as case (I).
  \end{enumerate}
  From all cases above we derive contradictions, thus we must have \(x = y\).
\end{proof}

\begin{proof}[\pf{4.3.12}(d)]
  By \cref{4.1.5} exactly one of the following two statements is true:
  \begin{itemize}
    \item \(n \geq 0\).
          Then by \cref{4.3.10}(d) we have \(\abs{x^n} = \abs{x}^n\).
    \item \(n < 0\).
          Then by \cref{ac:4.2.7} we have \(-n > 0\) and
          \begin{align*}
            \abs{x^n} & = \abs{x^{-(-n)}}                       &  & \by{ac:4.1.6}  \\
                      & = \abs{x^{(-1)(-n)}}                    &  & \by{ac:4.1.5}  \\
                      & = \abs{(x^{-1})^{-n}}                   &  & \by{4.3.12}[a] \\
                      & = \abs{x^{-1}}^{-n}                     &  & \by{4.3.10}[d] \\
                      & = \abs{\dfrac{1}{x^1}}^{-n}             &  & \by{4.3.11}    \\
                      & = \bigg(\dfrac{1}{\abs{x^1}}\bigg)^{-n} &  & \by{4.2.2}     \\
                      & = \bigg(\dfrac{1}{\abs{x}^1}\bigg)^{-n} &  & \by{4.3.10}[d] \\
                      & = (\abs{x}^{-1})^{-n}                   &  & \by{4.3.11}    \\
                      & = \abs{x}^{-1(-n)}                      &  & \by{4.3.12}[a] \\
                      & = \abs{x}^n.                            &  & \by{ac:4.1.6}
          \end{align*}
  \end{itemize}
  From all cases above we conclude that \(\abs{x^n} = \abs{x}^n\).
\end{proof}

\exercisesection

\begin{ex}\label{ex:4.3.1}
  Prove \cref{4.3.3}.
\end{ex}

\begin{proof}[\pf{ex:4.3.1}]
  See \cref{4.3.3}.
\end{proof}

\begin{ex}\label{ex:4.3.2}
  Prove the remaining claims in \cref{4.3.7}.
\end{ex}

\begin{proof}[\pf{ex:4.3.2}]
  See \cref{4.3.7}.
\end{proof}

\begin{ex}\label{ex:4.3.3}
  Prove \cref{4.3.10}.
\end{ex}

\begin{proof}[\pf{ex:4.3.3}]
  See \cref{4.3.10}.
\end{proof}

\begin{ex}\label{ex:4.3.4}
  Prove \cref{4.3.12}.
\end{ex}

\begin{proof}[\pf{ex:4.3.4}]
  See \cref{4.3.12}.
\end{proof}

\begin{ex}\label{ex:4.3.5}
  Prove that \(2^N \geq N\) for all positive integers \(N\).
\end{ex}

\begin{proof}[\pf{ex:4.3.5}]
  We induct on \(N\) and start with \(N = 1\).
  For \(N = 1\), by \cref{4.3.9} we have
  \[
    2^1 = 2^0 \times 2 = 1 \times 2 = 2 \geq 1,
  \]
  so the base case holds.
  Suppose inductively that for some \(N \geq 1\) we have \(2^N \geq N\).
  Then for \(N + 1\), we have
  \begin{align*}
             & 0 < N                          \\
    \implies & N < 2N        &  & \by{2.2.11} \\
    \implies & N + 1 \leq 2N &  & \by{2.2.12}
  \end{align*}
  and
  \begin{align*}
    2^{N + 1} & = 2 \times 2^N &  & \by{4.3.9} \\
              & \geq 2N        &  & \byIH      \\
              & \geq N + 1.
  \end{align*}
  This closes the induction.
\end{proof}
