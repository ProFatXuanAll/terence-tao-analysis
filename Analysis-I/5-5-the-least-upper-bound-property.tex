\section{The least upper bound property}\label{i:sec:5.5}

\begin{defn}[Upper bound]\label{i:5.5.1}
  Let \(E\) be a subset of \(\R\), and let \(M\) be a real number.
  We say that \(M\) is an \emph{upper bound} for \(E\), iff we have \(x \leq M\) for every element \(x\) in \(E\).
\end{defn}

\setcounter{thm}{2}
\begin{eg}\label{i:5.5.3}
  Let \(\R^+\) be the set of positive reals: \(\R^+ \coloneqq \set{x \in \R : x > 0}\).
  Then \(\R^+\) does not have any upper bounds at all.
  (More precisely, \(\R^+\) has no upper bounds which are real numbers.)
\end{eg}

\begin{proof}[\pf{i:5.5.3}]
  Suppose for sake of contradiction that there exists an \(M \in \R\) such that \(M\) is an upper bound for \(\R^+\).
  Then we have \(x \leq M\) for all \(x \in \R^+\).
  Since \(x > 0\), by \cref{i:5.4.7} we have \(M > 0\), thus \(M \in \R^+\).
  But this means \(M + 1 \in \R^+\), and we must have \(M > M + 1\), a contradiction.
  Thus such \(M\) does not exist and \(\R^+\) has no upper bounds in \(\R\).
\end{proof}

\begin{ac}\label{i:ac:5.5.1}
  Let \(x, y \in \R\).
  We define the following eight subsets of \(\R\):
  \begin{align*}
    \R_{\leq x}   & \coloneqq \set{r \in \R : r \leq x};        & \R_{< x}   & \coloneqq \set{r \in \R : r < x};     & \R^+ & \coloneqq \R_{> 0}; \\
    \R_{\geq x}   & \coloneqq \set{r \in \R : r \geq x};        & \R_{> x}   & \coloneqq \set{r \in \R : r > x};     & \R^- & \coloneqq \R_{< 0}; \\
    \R_{x \leq y} & \coloneqq \set{r \in \R : x \leq r \leq y}; & \R_{x < y} & \coloneqq \set{r \in \R : x < r < y}. &      &
  \end{align*}
\end{ac}

\begin{eg}\label{i:5.5.4}
  Let \(\emptyset\) be the empty set.
  Then every number \(M\) is an upper bound for \(\emptyset\), because \(M\) is greater than every element of the empty set
  (this is a vacuously true statement, but still true).
\end{eg}

\begin{note}
  It is clear that if \(M\) is an upper bound of \(E\), then any larger number \(M' \geq M\) is also an upper bound of \(E\).
  On the other hand, it is not so clear whether it is also possible for any number smaller than \(M\) to also be an upper bound of \(E\).
  This motivates the \cref{i:5.5.5}.
\end{note}

\begin{defn}[Least upper bound]\label{i:5.5.5}
  Let \(E\) be a subset of \(\R\), and \(M\) be a real number.
  We say that \(M\) is a \emph{least upper bound} for \(E\) iff
  \begin{enumerate}
    \item \(M\) is an upper bound for \(E\), and also
    \item any other upper bound \(M'\) for \(E\) must be larger than or equal to \(M\).
  \end{enumerate}
\end{defn}

\setcounter{thm}{6}
\begin{eg}\label{i:5.5.7}
  The empty set does not have a least upper bound.
\end{eg}

\begin{proof}[\pf{i:5.5.7}]
  Suppose for sake of contradiction that there exists an \(M \in \R\) such that \(M\) is a least upper bound of \(\emptyset\).
  By \cref{i:5.5.5} we know that \(x \leq M\) for all \(x \in \emptyset\).
  But by \cref{i:5.5.4} we know that \(M - 1\) is also a upper bound of \(\emptyset\), so by \cref{i:5.5.5} we have \(M < M - 1\), a contradiction.
  Thus \(\emptyset\) does not have a least upper bound.
\end{proof}

\begin{prop}[Uniqueness of least upper bound]\label{i:5.5.8}
  Let \(E\) be a subset of \(\R\).
  Then \(E\) can have at most one least upper bound.
\end{prop}

\begin{proof}[\pf{i:5.5.8}]
  Let \(M_1\) and \(M_2\) be two least upper bounds of \(E\).
  Since \(M_1\) is a least upper bound and \(M_2\) is an upper bound, then by definition of least upper bound we have \(M_2 \geq M_1\).
  Since \(M_2\) is a least upper bound and \(M_1\) is an upper bound, we similarly have \(M_1 \geq M_2\).
  Thus \(M_1 = M_2\).
  Thus there is at most one least upper bound.
\end{proof}

\begin{thm}[Existence of least upper bound]\label{i:5.5.9}
  Let \(E\) be a non-empty subset of \(\R\).
  If \(E\) has an upper bound (i.e., \(E\) has some upper bound \(M\)), then it must have exactly one least upper bound.
\end{thm}

\begin{proof}[\pf{i:5.5.9}]
  Let \(E\) be a non-empty subset of \(\R\) with an upper bound \(M\).
  By \cref{i:5.5.8}, we know that \(E\) has at most one least upper bound;
  we have to show that \(E\) has at least one least upper bound.
  Since \(E\) is non-empty, we can choose some element \(x_0\) in \(E\).

  Let \(n \in \Z^+\).
  We know that \(E\) has an upper bound \(M\).
  By the Archimedean property (\cref{i:5.4.13}), we can find a \(K \in \Z^+\) such that \(K / n \geq M\), and hence \(K / n\) is also an upper bound for \(E\).
  (Note that \(K\) is positive, and \(M\) can be either zero or negative, but \(K / n\) is positive, so we are fine.)
  By the Archimedean property again, there exists another \(L \in \Z\) such that \(L / n < x_0\).
  (Note that if \(x_0 \geq 0\), then we can set \(L = -1\); if \(x_0 < 0\), then \(-x_0\) is positive, so by Archimedean property we have some \(-L \in \Z^+\) such that \(-L / n > -x_0\).)
  Since \(x_0\) lies in \(E\), we see that \(L / n\) is not an upper bound for \(E\).
  Since \(K / n\) is an upper bound but \(L / n\) is not, we see that \(K > L\).

  Since \(K / n\) is an upper bound for \(E\) and \(L / n\) is not, we can find an integer \(L < m_n \leq K\) with the property that \(m_n / n\) is an upper bound for \(E\), but \((m_n - 1) / n\) is not (see \cref{i:ex:5.5.2}).
  In fact, this integer \(m_n\) is unique (\cref{i:ex:5.5.3}).
  We subscript \(m_n\) by \(n\) to emphasize the fact that this integer \(m\) depends on the choice of \(n\).
  This gives a well-defined (and unique) sequence \(m_1, m_2, m_3, \dots\) of integers, with each of the \(m_n / n\) being upper bounds and each of the \((m_n - 1) / n\) not being upper bounds.

  Now let \(N \in \Z^+\), and let \(n, n' \in \Z_{\geq N}\).
  Since \(m_n / n\) is an upper bound for \(E\) and \((m_{n'} - 1) / n'\) is not, by \cref{i:5.5.1} we must have \(m_n / n > (m_{n'} - 1) / n'\).
  After a little algebra, this implies that
  \[
    \dfrac{m_n}{n} - \dfrac{m_{n'}}{n'} > -\dfrac{1}{n'} \geq -\dfrac{1}{N}.
  \]
  Similarly, since \(m_{n'} / n'\) is an upper bound for \(E\) and \((m_n - 1) / n\) is not, we have \(m_{n'} / n' > (m_n - 1) / n\), and hence
  \[
    \dfrac{m_n}{n} - \dfrac{m_{n'}}{n'} < \dfrac{1}{n} \leq \dfrac{1}{N}.
  \]
  Putting these two bounds together, we see that
  \[
    \abs{\dfrac{m_n}{n} - \dfrac{m_{n'}}{n'}} \leq \dfrac{1}{N} \text{ for all } n, n' \geq N \geq 1.
  \]
  This implies that \(\dfrac{m_n}{n}\) is a Cauchy sequence (\cref{i:ex:5.5.4}).
  Since the \(\dfrac{m_n}{n}\) are rational numbers, we can now define the real number \(S\) as
  \[
    S \coloneqq \LIM_{n \to \infty} \dfrac{m_n}{n}.
  \]
  From \cref{i:ex:5.3.5} we conclude that
  \[
    S = \LIM_{n \to \infty} \dfrac{m_n - 1}{n}.
  \]
  To finish the proof of the theorem, we need to show that \(S\) is the least upper bound for \(E\).
  First we show that it is an upper bound.
  Let \(x\) be any element of \(E\).
  Then, since \(m_n / n\) is an upper bound for \(E\), we have \(x \leq m_n / n\) for all \(n \in \Z^+\).
  Applying \cref{i:ex:5.4.8}, we conclude that \(x \leq \LIM_{n \to \infty} m_n / n = S\).
  Thus \(S\) is indeed an upper bound for \(E\).

  Now we show it is a least upper bound.
  Suppose \(y\) is an upper bound for \(E\).
  Since \((m_n - 1) / n\) is not an upper bound, we conclude that \(y \geq (m_n - 1) / n\) for all \(n \in \Z^+\).
  Applying \cref{i:ex:5.4.8}, we conclude that \(y \geq \LIM_{n \to \infty} (m_n - 1) / n = S\).
  Thus the upper bound \(S\) is less than or equal to every upper bound of \(E\), and \(S\) is thus a least upper bound of \(E\).
\end{proof}

\begin{defn}[Supremum]\label{i:5.5.10}
  Let \(E\) be a subset of the real numbers.
  If \(E\) is non-empty and has some upper bound, we define \(\sup(E)\) to be the least upper bound of \(E\)
  (this is well-defined by \cref{i:5.5.9}).
  We introduce two additional symbols, \(+\infty\) and \(-\infty\).
  If \(E\) is non-empty and has no upper bound, we set \(\sup(E) \coloneqq +\infty\);
  if \(E\) is empty, we set \(\sup(E) \coloneqq -\infty\).
  We refer to \(\sup(E)\) as the \emph{supremum} of \(E\), and also denote it by \(\sup E\).
\end{defn}

\begin{rmk}\label{i:5.5.11}
  At present, \(+\infty\) and \(-\infty\) are meaningless symbols;
  we have no operations on them at present, and none of our results involving real numbers apply to \(+\infty\) and \(-\infty\), because these are not real numbers.
  In \cref{i:sec:6.2} we add \(+\infty\) and \(-\infty\) to the reals to form the \emph{extended real number system}, but this system is not as convenient to work with as the real number system, because many of the laws of algebra break down.
  For instance, it is not a good idea to try to define \(+\infty + -\infty\);
  setting this equal to \(0\) causes some problems.
\end{rmk}

\begin{prop}\label{i:5.5.12}
  There exists a positive real number \(x\) such that \(x^2 = 2\).
\end{prop}

\begin{proof}[\pf{i:5.5.12}]
  Let \(E\) be the set \(\set{y \in \R : y \geq 0 \text{ and } y^2 < 2}\);
  thus \(E\) is the set of all non-negative real numbers whose square is less than \(2\).
  Observe that \(E\) has an upper bound of \(2\) (because if \(y > 2\), then \(y^2 > 4 > 2\) and hence \(y \notin E\)).
  Also, \(E\) is non-empty (for instance, \(1\) is an element of \(E\)).
  Thus by the least upper bound property (\cref{i:5.5.9}), we have a real number \(x \coloneqq \sup(E)\) which is the least upper bound of \(E\).
  Then \(x\) is greater than or equal to \(1\) (since \(1 \in E\)) and less than or equal to \(2\)
  (since \(2\) is an upper bound for \(E\)).
  So \(x\) is positive.
  Now we show that \(x^2 = 2\).

  We argue this by contradiction.
  We show that both \(x^2 < 2\) and \(x^2 > 2\) lead to contradictions.
  First suppose that \(x^2 < 2\).
  Let \(\varepsilon \in \Q_{0 < 1}\) be a small number;
  then we have
  \[
    (x + \varepsilon)^2 = x^2 + 2\varepsilon x + \varepsilon^2 \leq x^2 + 4\varepsilon + \varepsilon = x^2 + 5\varepsilon
  \]
  since \(x \leq 2\) and \(\varepsilon^2 \leq \varepsilon\).
  Since \(x^2 < 2\), we see that we can use the Archimedean property (\cref{i:5.4.13}) to choose an \(\varepsilon \in \Q_{0 < 1}\) such that \(x^2 + 5\varepsilon < 2\), thus \((x + \varepsilon)^2 < 2\).
  By construction of \(E\), this means that \(x + \varepsilon \in E\);
  but this contradicts the fact that \(x\) is an upper bound of \(E\).

  Now suppose that \(x^2 > 2\).
  Let \(\varepsilon \in \Q_{0 < 1}\) be a small number;
  then we have
  \[
    (x - \varepsilon)^2 = x^2 - 2\varepsilon x + \varepsilon^2 \geq x^2 - 2\varepsilon x \geq x^2 - 4\varepsilon
  \]
  since \(x \leq 2\) and \(\varepsilon^2 \geq 0\).
  Since \(x^2 > 2\), we can choose \(\varepsilon \in \Q_{0 < 1}\) such that \(x^2 - 4\varepsilon > 2\), and thus \((x - \varepsilon)^2 > 2\).
  But then this implies that \(x - \varepsilon \geq y\) for all \(y \in E\).
  (Why? If \(x - \varepsilon < y\) then \((x - \varepsilon)^2 < y^2 \leq 2\), a contradiction.)
  Thus \(x - \varepsilon\) is an upper bound for \(E\), which contradicts the fact that \(x\) is the \emph{least} upper bound of \(E\).
  From these two contradictions we see that \(x^2 = 2\), as desired.
\end{proof}

\begin{rmk}\label{i:5.5.13}
  Comparing \cref{i:5.5.12} with \cref{i:4.4.4}, we see that certain numbers are real but not rational.
  The proof of \cref{i:5.5.12} also shows that the rationals \(\Q\) do not obey the least upper bound property, otherwise one could use that property to construct a square root of \(2\), which by \cref{i:4.4.4} is not possible.
\end{rmk}

\begin{rmk}\label{i:5.5.14}
  In \cref{i:ch:6} we will use the least upper bound property to develop the theory of limits, which allows us to do many more things than just take square roots.
\end{rmk}

\begin{rmk}\label{i:5.5.15}
  We can of course talk about lower bounds, and greatest lower bounds, of sets \(E\);
  the greatest lower bound of a set \(E\) is also known as the \emph{infimum} of \(E\) and is denoted \(\inf(E)\) or \(\inf E\).
  Everything we say about suprema has a counterpart for infima;
  A precise relationship between the two notions is given by \cref{i:ex:5.5.1}.
  See also \cref{i:sec:6.2}.
\end{rmk}

\begin{note}
  Supremum means ``highest'' and infimum means ``lowest'', and the plurals are suprema and infima.
  Supremum is to superior, and infimum to inferior, as maximum is to major, and minimum to minor.
  The root words are ``super'', which means ``above'', and ``infer'', which means ``below''
  (this usage only survives in a few rare English words such as ``infernal'', with the Latin prefix ``sub'' having mostly replaced ``infer'' in English).
\end{note}

\exercisesection

\begin{ex}\label{i:ex:5.5.1}
  Let \(E\) be a subset of the real numbers \(\R\), and suppose that \(E\) has a least upper bound \(M\) which is a real number, i.e., \(M = \sup(E)\).
  Let \(-E\) be the set
  \[
    -E \coloneqq \set{-x : x \in E}.
  \]
  Show that \(-M\) is the greatest lower bound of \(-E\), i.e., \(-M = \inf(-E)\).
\end{ex}

\begin{proof}[\pf{i:ex:5.5.1}]
  We first show that \(-M\) is a lower bound for \(-E\).
  This is true since
  \begin{align*}
             & \forall x \in E, x \leq M           &  & \by{i:5.5.1}  \\
    \implies & \forall x \in E, -x \geq -M         &  & \by{i:5.4.7}  \\
    \implies & \forall -x \in -E, -x \geq -M                          \\
    \implies & -M \text{ is a lower bound of } -E. &  & \by{i:5.5.15}
  \end{align*}

  Next we show that \(-M\) is a greatest lower bound for \(-E\).
  Let \(L \in \R\) be any lower bound for \(-E\).
  Then we have
  \begin{align*}
             & \forall x \in E, L \leq -x                                      \\
    \implies & \forall x \in E, x \leq -L                   &  & \by{i:5.4.7}  \\
    \implies & -L \text{ is an upper bound of } E           &  & \by{i:5.5.1}  \\
    \implies & M \leq -L                                    &  & \by{i:5.5.5}  \\
    \implies & -M \geq L                                    &  & \by{i:5.4.7}  \\
    \implies & -M \text{ is a greatest lower bound of } -E. &  & \by{i:5.5.15}
  \end{align*}

  Now we show that the greatest lower bound is unique.
  Let \(M, M'\) be two greatest lower bounds of \(-E\).
  Then we have \(M \leq M'\) and \(M \geq M'\), which means \(M = M'\).
  So the greatest lower bound is unique.
\end{proof}

\begin{ex}\label{i:ex:5.5.2}
  Let \(E\) be a non-empty subset of \(\R\), let \(n \in \Z^+\), and let \(L < K\) be integers.
  Suppose that \(K / n\) is an upper bound for \(E\), but that \(L / n\) is not an upper bound for \(E\).
  Without using \cref{i:5.5.9}, show that there exists an integer \(L < m \leq K\) such that \(m / n\) is an upper bound for \(E\), but that \((m - 1) / n\) is not an upper bound for \(E\).
\end{ex}

\begin{proof}[\pf{i:ex:5.5.2}]
  Let \(d = K - L\), so \(d\) is positive by \cref{i:4.1.11}(a).
  Now we induct on \(d\) to show that for every \(d \in \Z^+\), there exists an \(m \in \Z_{> L} \cap \Z_{\leq K}\) such that \(m / n\) is an upper bound for \(E\), but that \((m - 1) / n\) is not.
  We start with \(d = 1\).
  For \(d = 1\), we have \(K - 1 = L\).
  Then let \(m = K\).
  So by hypothesis we have \(m = K \in \Z_{> L} \cap \Z_{\leq K}\), \(m / n = K / n\) is an upper bound for \(E\), and \((m - 1) / n = (K - 1) / n = L / n\) is not an upper bound for \(E\).
  Thus the base case holds.
  Suppose inductively that the statement holds for some \(d \in \Z^+\).
  Now we show that for \(d + 1\) the statement is also true.
  So suppose that for some \(K, L \in \Z\), we have \(K - L = d + 1\), \(K / n\) is an upper bound of \(E\), but \(L / n\) is not.
  Since \(K / n\) is an upper bound for \(E\), we can ask whether \((K - 1) / n\) is an upper bound for \(E\).
  \begin{itemize}
    \item If \((K - 1) / n\) is not an upper bound for \(E\), then we can choose \(m = K\) and we are done.
    \item If \((K - 1) / n\) is an upper bound for \(E\), then by \cref{i:5.5.1} we have \(L < K - 1\).
          Since \(K - 1 - L = d\), by induction hypothesis we know that there exists an \(m \in \Z_{> L} \cap \Z_{\leq K}\) such that \(m / n\) is an upper bound for \(E\), but \((m - 1) / n\) is not.
  \end{itemize}
  From all cases above we found an \(m \in \Z_{> L} \cap \Z_{\leq K}\) such that \(m / n\) is an upper bound for \(E\), but \((m - 1) / n\) is not.
  This closes the induction.
\end{proof}

\begin{ex}\label{i:ex:5.5.3}
  Let \(E\) be a non-empty subset of \(\R\), let \(n \in \Z^+\), and let \(m, m' \in \Z\) with the properties that \(m / n\) and \(m' / n\) are upper bounds for \(E\), but \((m - 1) / n\) and \((m' - 1) / n\) are not upper bounds for \(E\).
  Show that \(m = m'\).
  This shows that the integer \(m\) constructed in \cref{i:ex:5.5.2} is unique.
\end{ex}

\begin{proof}[\pf{i:ex:5.5.3}]
  Suppose for sake of contradiction that \(m \neq m'\).
  Then by \cref{i:4.1.11}(f) we have either \(m < m'\) or \(m > m'\).
  \begin{itemize}
    \item If \(m < m'\), then we have \(m \leq m' - 1\).
          By \cref{i:4.2.9}(e) we have \(m / n \leq (m' - 1) / n\), which means \((m' - 1) / n\) is an upper bound for \(E\).
          But this contradict to the hypothesis.
    \item If \(m > m'\), then by switching the row of \(m\) and \(m'\) in the previous case we can also derive contradiction.
  \end{itemize}
  From all cases above we derive contradictions.
  Thus we must have \(m = m'\).
\end{proof}

\begin{ex}\label{i:ex:5.5.4}
  Let \((q_n)_{n = 1}^\infty\) be a sequence of rational numbers with the property that \(\abs{q_n - q_{n'}} \leq \dfrac{1}{M}\) whenever \(M \in \Z^+\) and \(n, n' \in \Z_{\geq M}\).
  Show that \((q_n)_{n = 1}^\infty\) is a Cauchy sequence.
  Furthermore, if \(S \coloneqq \LIM_{n \to \infty} q_n\), show that \(\abs{q_M - S} \leq \dfrac{1}{M}\) for every \(M \in \Z^+\).
\end{ex}

\begin{proof}[\pf{i:ex:5.5.4}]
  We first show that \((q_n)_{n = 1}^\infty\) is a Cauchy sequence.
  Let \(\varepsilon \in \Q^+\).
  By Archimedean property (\cref{i:5.4.13}) we know that there exists an \(M \in \Z^+\) such that \(M \varepsilon > 1\).
  By \cref{i:4.2.9}(e) we have \(\varepsilon > 1 / M\).
  But by hypothesis we have \(\abs{q_n - q_{n'}} \leq \dfrac{1}{M} < \varepsilon\) for all \(n, n' \in \Z_{\geq M}\).
  Since \(\varepsilon\) is arbitrary, by \cref{i:5.1.8} this means \((q_n)_{n = 1}^{\infty}\) is a Cauchy sequence.

  Now we show that if \(S = \LIM_{n \to \infty} q_n\), then \(\abs{q_M - S} \leq \dfrac{1}{M}\) for every \(M \in \Z^+\).
  From the proof above we know that \((q_n)_{n = 1}^\infty\) is a Cauchy sequence, thus \(S = \LIM_{n \to \infty} q_n\) is well-defined.
  By hypothesis we have \(\abs{q_M - q_n} \leq 1 / M\) for all \(M \in \Z^+\) and for all \(n \geq M\).
  Then we have
  \begin{align*}
             & \forall M \in \Z^+, \forall n \in \Z_{\geq M}, \abs{q_M - q_n} \leq 1 / M                                  \\
    \implies & \forall M \in \Z^+, \forall n \in \Z_{\geq M}, \abs{q_n - q_M} \leq 1 / M             &  & \by{i:4.3.1}    \\
    \implies & \forall M \in \Z^+, \forall n \in \Z_{\geq M}, -1 / M \leq q_n - q_M \leq 1 / M       &  & \by{i:4.3.3}[c] \\
    \implies & \forall M \in \Z^+, \forall n \in \Z_{\geq M}, -1 / M + q_M \leq q_n \leq 1 / M + q_M &  & \by{i:4.2.9}[d] \\
    \implies & \forall M \in \Z^+, -1 / M + q_M \leq S \leq 1 / M + q_M                              &  & \by{i:ex:5.4.8} \\
    \implies & \forall M \in \Z^+, -1 / M \leq S - q_M \leq 1 / M                                    &  & \by{i:5.4.7}[d] \\
    \implies & \forall M \in \Z^+, \abs{S - q_M} \leq 1 / M                                          &  & \by{i:ex:5.4.6} \\
    \implies & \forall M \in \Z^+, \abs{q_M - S} \leq 1 / M.                                         &  & \by{i:5.4.5}
  \end{align*}
\end{proof}

\begin{ex}\label{i:ex:5.5.5}
  Establish an analogue of \cref{i:5.4.14}, in which ``rational'' is replaced by ``irrational''.
\end{ex}

\begin{proof}[\pf{i:ex:5.5.5}]
  Let \(x, y, z \in \R\) where \(x < y\) and \(z^2 = 2\).
  (\(z\) is well-defined thanks to \cref{i:5.5.12}.)
  So by \cref{i:5.4.7} we have \(x - z < y - z\).
  But by \cref{i:5.4.14} there exists a \(q \in \Q\) such that \(x - z < q < y - z\).
  So by \cref{i:5.4.7} again we have \(x < q + z < y\).
  Because \(z\) is irrational, \(q + z\) is also irrational
  (otherwised we have \(a = q + z \in \Q\) and \(z = a - q \in \Q\), contradicts to \cref{i:4.4.4}).
  So we have an irrational number in between any two real numbers \(x, y\) where \(x < y\).
\end{proof}
