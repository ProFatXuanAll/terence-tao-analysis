\section{A non-Riemann integrable function}\label{sec:11.7}

\begin{prop}\label{11.7.1}
  Let \(f : [0, 1] \to \R\) be the discontinuous function
  \[
    f(x) \coloneqq \begin{dcases}
      1 & \text{if } x \in \Q    \\
      0 & \text{if } x \notin \Q
    \end{dcases}
  \]
  Then \(f\) is bounded but not Riemann integrable.
\end{prop}

\begin{proof}
  It is clear that \(f\) is bounded, so let us show that it is not Riemann integrable.

  Let \(\mathbf{P}\) be any partition of \([0, 1]\).
  For any \(J \in \mathbf{P}\), observe that if \(J\) is not a point or the empty set, then
  \[
    \sup_{x \in J} f(x) = 1
  \]
  (by \cref{5.4.14}).
  In particular we have
  \[
    \bigg(\sup_{x \in J} f(x)\bigg) \abs{J} = \abs{J}.
  \]
  (Note this is also true when \(J\) is a point, since both sides are zero.)
  In particular we see that
  \[
    U(f, \mathbf{P}) = \sum_{J \in \mathbf{P} : J \neq \emptyset} \abs{J} = \abs{[0, 1]} = 1
  \]
  by \cref{11.1.13};
  note that the empty set does not contribute anything to the total length.
  In particular we have \(\overline{\int}_{[0, 1]} f = 1\), by \cref{11.3.12}.

  A similar argument gives that
  \[
    \inf_{x \in J} f(x) = 0
  \]
  for all \(J\) (other than points or the empty set), and so
  \[
    L(f, \mathbf{P}) = \sum_{J \in \mathbf{P} : J \neq \emptyset} 0 = 0.
  \]
  In particular we have \(\underline{\int}_{[0, 1]} f = 0\), by \cref{11.3.12}.
  Thus the upper and lower Riemann integrals do not match, and so this function is not Riemann integrable.
\end{proof}

\begin{rmk}\label{11.7.2}
  It is only rather ``artificial'' bounded functions which are not Riemann integrable.
  Because of this, the Riemann integral is good enough for a large majority of cases.
  There are ways to generalize or improve this integral, though.
  One of these is the \emph{Lebesgue integral}.
  Another is the \emph{Riemann-Stieltjes integral} \(\int_I f d\alpha\), where \(\alpha : I \to \R\) is a monotone increasing function.
\end{rmk}