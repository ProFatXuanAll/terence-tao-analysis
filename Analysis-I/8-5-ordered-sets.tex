\section{Ordered sets}\label{sec 8.5}

\begin{definition}[Partially ordered sets]\label{8.5.1}
    A \emph{partially ordered set} (or \emph{poset}) is a set \(X\), together with a relation \(\leq_X\) on \(X\)
    (thus for any two objects \(x, y \in X\), the statement \(x \leq_X y\) is either a true statement or a false statement).
    Furthermore, this relation is assumed to obey the following three properties:
    \begin{itemize}
        \item (Reflexivity) For any \(x \in X\), we have \(x \leq_X x\).
        \item (Anti-symmetry) If \(x, y \in X\) are such that \(x \leq_X y\) and \(y \leq_X x\), then \(x = y\).
        \item (Transitivity) If \(x, y, z \in X\) are such that \(x \leq_X y\) and \(y \leq_X z\), then \(x \leq_X z\).
    \end{itemize}
    We refer to \(\leq_X\) as the \emph{ordering relation}.
    In most situations it is understood what the set \(X\) is from context, and in those cases we shall simply write \(\leq\) instead of \(\leq_X\).
    We write \(x <_X y\) (or \(x < y\) for short) if \(x \leq_X y\) and \(x \neq y\).
\end{definition}

\begin{note}
    Strictly speaking, a partially ordered set is not a set \(X\), but rather a pair \((X, \leq_X)\).
    But in many cases the ordering \(\leq_X\) will be clear from context, and so we shall refer to \(X\) itself as the partially ordered set even though this is technically incorrect.
\end{note}