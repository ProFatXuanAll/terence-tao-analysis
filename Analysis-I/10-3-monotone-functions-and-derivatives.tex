\section{Monotone functions and derivatives}\label{sec 10.3}

\begin{proposition}\label{10.3.1}
    Let \(X\) be a subset of \(\mathbf{R}\), let \(x_0 \in X\) be a limit point of \(X\), and let \(f : X \to \mathbf{R}\) be a function.
    If \(f\) is monotone increasing and \(f\) is differentiable at \(x_0\), then \(f'(x_0) \geq 0\).
    If f is monotone decreasing and \(f\) is differentiable at \(x_0\), then \(f'(x_0) \leq 0\).
\end{proposition}

\begin{proof}
    First suppose that \(f\) is monotone increasing.
    Since \(f\) is differentiable at \(x_0\), by Definition \ref{10.1.1} we have
    \[
        f'(x_0) = \lim_{x \to x_0 ; x \in X \setminus \{x_0\}} \frac{f(x) - f(x_0)}{x - x_0}.
    \]
    Since \(x, x_0 \in \mathbf{R}\) and \(x \in X \setminus \{x_0\}\), we have either \(x < x_0\) and \(x > x_0\).
    \begin{enumerate}
        \item If \(x < x_0\), then we have
              \begin{align*}
                           & (x - x_0 < 0) \land \big(f(x) - f(x_0) \leq 0\big) & \text{(by Definition \ref{9.8.1})} \\
                  \implies & \frac{f(x) - f(x_0)}{x - x_0} \geq 0.                                                   \\
              \end{align*}
        \item If \(x > x_0\), then we have
              \begin{align*}
                           & (x - x_0 > 0) \land \big(f(x) - f(x_0) \geq 0\big) & \text{(by Definition \ref{9.8.1})} \\
                  \implies & \frac{f(x) - f(x_0)}{x - x_0} \geq 0.                                                   \\
              \end{align*}
              From all cases above we have \(x \in X \setminus \{x_0\} \implies \frac{f(x) - f(x_0)}{x - x_0} \geq 0\).
              Thus by Proposition \ref{9.3.14} we have \(f'(x_0) \geq 0\).
    \end{enumerate}

    Now suppose that \(f\) is monotone decreasing.
    Since \(f\) is differentiable at \(x_0\), by Definition \ref{10.1.1} we have
    \[
        f'(x_0) = \lim_{x \to x_0 ; x \in X \setminus \{x_0\}} \frac{f(x) - f(x_0)}{x - x_0}.
    \]
    Since \(x, x_0 \in \mathbf{R}\) and \(x \in X \setminus \{x_0\}\), we have either \(x < x_0\) and \(x > x_0\).
    \begin{enumerate}
        \item If \(x < x_0\), then we have
              \begin{align*}
                           & (x - x_0 < 0) \land \big(f(x) - f(x_0) \geq 0\big) & \text{(by Definition \ref{9.8.1})} \\
                  \implies & \frac{f(x) - f(x_0)}{x - x_0} \leq 0.                                                   \\
              \end{align*}
        \item If \(x > x_0\), then we have
              \begin{align*}
                           & (x - x_0 > 0) \land \big(f(x) - f(x_0) \leq 0\big) & \text{(by Definition \ref{9.8.1})} \\
                  \implies & \frac{f(x) - f(x_0)}{x - x_0} \leq 0.                                                   \\
              \end{align*}
              From all cases above we have \(x \in X \setminus \{x_0\} \implies \frac{f(x) - f(x_0)}{x - x_0} \leq 0\).
              Thus by Proposition \ref{9.3.14} we have \(f'(x_0) \leq 0\).
    \end{enumerate}
\end{proof}

\begin{remark}\label{10.3.2}
    We have to assume that \(f\) is differentiable at \(x_0\);
    There exist monotone functions which are not always differentiable, and of course if \(f\) is not differentiable at \(x_0\) we cannot possibly conclude that \(f'(x_0) \geq 0\) or \(f'(x_0) \leq 0\).
\end{remark}