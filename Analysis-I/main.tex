% We use chapter structure.
\documentclass[11pt,a4paper]{book}

%==============================================================================
% Preamble.
%==============================================================================

% Correctly showing characters outside ASCII.
\usepackage[T1]{fontenc}
% File is written and read with utf8 encoding.
\usepackage[utf8]{inputenc}
% Set paging layout.
\usepackage[paperwidth=160mm,paperheight=240mm,margin=15mm]{geometry}
% Including `amsfonts'.
% Must be loaded before `mathtools'.
\usepackage{amssymb}
% Including `amsmath' and fixing bugs for `amsmath'.
\usepackage{mathtools}
% Must be loaded after `amsmath' and `mathtools'.
\usepackage{amsthm}
% Define page header and footer layout.
\usepackage{fancyhdr}
% LaTex 3 new command tools.
\usepackage{xparse}

% Use `fancyhdr' page style.  All page style settings must be called only after
% this command.  See `fancyhdr' for details.
\pagestyle{fancy}

% Make header height enough to fit chapter / section titles.
\setlength{\headheight}{15pt}
% Change chapter and section marks' formatting.  Note that I use `\markright'
% to make sure header use chapter information when section information is not
% available (this is need for appendix).
\renewcommand{\chaptermark}[1]{\markright{\textsf{\small Chap. \thechapter \quad #1}}}
\renewcommand{\sectionmark}[1]{\markright{\textsf{\small Sec. \thesection \quad #1}}}
% Cleanup page header settings.
\fancyhead{}
% Put page number on the left of page header.
\fancyhead[L]{\textbf{\textsf{\small \thepage}}}
% Put chapter / section information on the right of page header.
\fancyhead[R]{\rightmark}
% Cleanup page footer settings.
\fancyfoot{}

% Reset `plain' page style which is used for the first page of each chapter.
% I set this to make page number style consistent.
\fancypagestyle{plain}{%
  \fancyhf{}% clear all fields
  \fancyfoot[C]{\textbf{\textsf{\small \thepage}}}%
  \renewcommand{\headrulewidth}{0pt}%
}

% Automatically adjust character spacing at margins.
\usepackage{microtype}
% Provide further utilities and fix bugs for `enumerate', `itemize' and `description'.
\usepackage{enumitem}
% Provide better quoting environment.
\usepackage{dirtytalk}
% Parsing list inside `\newcommand'.
\usepackage{listofitems}
% Nice looking if-then-else structure with comparison functionality.
\usepackage{ifthen}
% Automatically add hyperlinks to labels/refs.
% Must be loaded after all packages above and before `cleveref'.
% Recommend to use with `natbib' when you need bibtex.
\usepackage{hyperref}

\hypersetup{       % This macro come with `hyperref'.
	colorlinks=true, % Color hyperlinks.
	linkcolor=blue,  % Color local hyperlinks with blue.
	urlcolor=cyan,   % Color url links with cyan.
}

% Must be loaded after `hyperref'.  We always capitalize each cross-references'
% type name.  See `cleveref' for details.
\usepackage[capitalize]{cleveref}

% Allow page break in the middle of multi-line equations.
\allowdisplaybreaks

%------------------------------------------------------------------------------
% Define environments.
%------------------------------------------------------------------------------

% Text inside the body of theorem-like environments are set to Roman font.
% Theorem-like environments share their counters, counters follow section and reset in every sections
% (except for axioms, axioms counters are reset in each chapter).
% Exercises has their owned counter.
% Notes do not use counter.
% See `amsthm' for details.
\theoremstyle{definition}
\newtheorem{ax}{Ax.}[chapter]
\newtheorem{ac}{A.Cor.}[section]
\newtheorem{ex}{Ex.}[section]
\newtheorem{thm}{Thm.}[section]
\newtheorem{cor}[thm]{Cor.}
\newtheorem{defn}[thm]{Def.}
\newtheorem{eg}[thm]{E.g.}
\newtheorem{lem}[thm]{Lem.}
\newtheorem{prop}[thm]{Prop.}
\newtheorem{rmk}[thm]{Rmk.}
\newtheorem*{note}{Note}

\theoremstyle{remark}
\newtheorem*{meta-proof}{Meta-proof}

% Define plural form for theorem-like environments. See `cleveref' for details.
\crefname{ax}{Ax.}{Ax.}
\crefname{ac}{A.Cor.}{A.Cor.}
\crefname{chapter}{Ch.}{Ch.}
\crefname{cor}{Cor.}{Cor.}
\crefname{defn}{Def.}{Def.}
\crefname{eg}{E.g.}{E.g.}
\crefname{ex}{Ex.}{Ex.}
\crefname{lem}{Lem.}{Lem.}
\crefname{note}{Note}{Notes}
\crefname{prop}{Prop.}{Prop.}
\crefname{rmk}{Rmk.}{Rmk.}
\crefname{section}{Sec.}{Sec.}
\crefname{thm}{Thm.}{Thm.}

% In `enumerate' enviroments, items' label are alphabets and surrounded by
% parentheses.  See `enumitem' for details.
\renewcommand{\labelenumi}{\textnormal{(}\alph{enumi}\textnormal{)}}

% Formatting exercises section.
\NewDocumentCommand{\exercisesection}{}{
  \begin{center}
    --- Exercises ---
  \end{center}
}

%------------------------------------------------------------------------------
% Define operators and symbols.
%------------------------------------------------------------------------------

% Absolute value.
\DeclarePairedDelimiter{\absTmp}{\lvert}{\rvert}
\NewDocumentCommand{\abs}{m}{\absTmp*{#1}}
% Ceiling.
\DeclarePairedDelimiter{\ceilTmp}{\lceil}{\rceil}
\NewDocumentCommand{\ceil}{m}{\ceilTmp*{#1}}
% Floor.
\DeclarePairedDelimiter{\floorTmp}{\lfloor}{\rfloor}
\NewDocumentCommand{\floor}{m}{\floorTmp*{#1}}
% Evaluate.
\DeclarePairedDelimiter{\evalTmp}{.}{\rvert}
\NewDocumentCommand{\eval}{m}{\evalTmp*{#1}}
% Parenthese.
\DeclarePairedDelimiter{\paTmp}{\lparen}{\rparen}
\NewDocumentCommand{\pa}{m}{\paTmp*{#1}}
% Bracket.
\DeclarePairedDelimiter{\brTmp}{\lbrack}{\rbrack}
\NewDocumentCommand{\br}{m}{\brTmp*{#1}}
% Brace.
\DeclarePairedDelimiter{\BTmp}{\lbrace}{\rbrace}
\NewDocumentCommand{\B}{m}{\BTmp*{#1}}
% Set.
\NewDocumentCommand{\set}{m}{\B{#1}}

% Define common symbols.
% See `amsmath' section 9.2 for details.

% Fields.
\NewDocumentCommand{\field}{m}{\mathbf{#1}}
% General field.
\NewDocumentCommand{\F}{}{\field{F}}
% Complex number.
\NewDocumentCommand{\C}{}{\mathbb{C}}
% Natural number.
\NewDocumentCommand{\N}{}{\mathbb{N}}
% Rational number.
\NewDocumentCommand{\Q}{}{\mathbb{Q}}
% Real number.
\NewDocumentCommand{\R}{}{\mathbb{R}}
% Integer number.
\NewDocumentCommand{\Z}{}{\mathbb{Z}}

% Plus plus operation.
\NewDocumentCommand{\pp}{}{\!\!+\!\!+}

% Proof statements reference text.
\NewDocumentCommand{\byOptionalArgumentProcess}{m}{(#1)}
\NewDocumentCommand{\by}{m >{\SplitList{,}} o}{%
	\IfNoValueTF{#2}{%
		\text{(by \cref{#1})}%
	}{%
		\text{(by \cref{#1}\ProcessList{#2}{\byOptionalArgumentProcess})}%
	}%
}

% Proof statements reference induction hypothesis.
\NewDocumentCommand{\byIH}{}{\text{(by induction hypothesis)}}

%==============================================================================
% Document.
%==============================================================================

\begin{document}

%------------------------------------------------------------------------------
% Front matters.
%------------------------------------------------------------------------------

\frontmatter

% Author informations.
\title{Analysis I}
\author{ProFatXuanAll}
\maketitle

% Table of contents.
\tableofcontents

%------------------------------------------------------------------------------
% Main matters.
%------------------------------------------------------------------------------

\mainmatter

% All chapters are in separated files.
% We include them here.
\chapter{Introduction}\label{i:ch:1}

\begin{note}
  \emph{circularity}:
  Using an advanced fact to prove a more elementary fact, and then later using the elementary fact to prove the advanced fact.
  When do a mathematics proofs, one should avoid \emph{circularity}.
\end{note}

\begin{note}
  From a logical point of view, there is no difference between a lemma, proposition, theorem, or corollary
  - they are all claims waiting to be proved.
  However, we use these terms to suggest different levels of importance and difficulty.
  A lemma is an easily proved claim which is helpful for proving other propositions and theorems but is usually not particularly interesting in its own right.
  A proposition is a statement which is interesting in its own right.
  A theorem is a more important statement than a proposition which says something definitive on the subject and often takes more effort to prove than a proposition or lemma.
  A corollary is an immediate consequence of a proposition or theorem that was proven recently.
\end{note}

\chapter{Natural Number}\label{i:ch:2}

\section{The Peano axioms}\label{i:sec:2.1}

\begin{note}
  We now present one standard way to define the natural numbers, in terms of the \emph{Peano axioms}, which were first laid out by Giuseppe Peano (1858--1932).
  This is not the only way to define the natural numbers.
  For instance, another approach is to talk about the cardinality of finite sets.
  For instance, one could take a set of five elements and define \(5\) to be the number of elements in that set.
\end{note}

\begin{note}
  In some texts, the natural numbers start at \(1\) instead of \(0\), but this is a matter of notational convention more than anything else.
  In this text, we shall refer to the set \(\set{1, 2, 3,...}\) as the set of positive integers \(\Z^+\) rather than the natural numbers.
  Natural numbers are sometimes also known as \emph{whole numbers}.
\end{note}

\begin{note}
  In mathematics, we try not to define a variable more than once in any given setting, as it can often lead to confusion;
  many of the statements which were true for the old value of the variable can now become false, and vice versa.
\end{note}

\begin{ax}\label{i:2.1}
  \(0\) is a natural number.
\end{ax}

\begin{ax}\label{i:2.2}
  If \(n\) is a natural number, then \(n\pp\) is a natural number.
\end{ax}

\begin{ax}\label{i:2.3}
  \(0\) is not the successor of any natural number;
  i.e., we have \(n\pp \neq 0\) for every natural number \(n\).
\end{ax}

\begin{ax}\label{i:2.4}
  Different natural numbers must have different successors;
  i.e., if \(n\) and \(m\) are natural numbers and \(n \neq m\), then \(n\pp \neq m\pp\).
  Equivalently, if \(n\pp = m\pp\), then we must have \(n = m\).
\end{ax}

\begin{ax}[Principle of mathematical induction]\label{i:2.5}
  Let \(P(n)\) be any property pertaining to a natural number \(n\).
  Suppose that \(P(0)\) is true, and suppose that whenever \(P(n)\) is true, \(P(n\pp)\) is also true.
  Then \(P(n)\) is true for every natural number \(n\).
\end{ax}

\begin{note}
  \crefrange{i:2.1}{i:2.5} are known as the \emph{Peano axioms} for the natural numbers.
\end{note}

\begin{note}
  ``a priori'' is Latin for ``beforehand''
  - it refers to what one already knows or assumes to be true before one begins a proof or argument.
  The opposite is ``a posteriori''
  - what one knows to be true after the proof or argument is concluded.
\end{note}

\begin{note}
  \cref{i:2.5} should technically be called an \emph{axiom schema} rather than an \emph{axiom}
  - it is a template for producing an (infinite) number of axioms rather than being a single axiom in its own right.
\end{note}

\begin{note}
  A remarkable accomplishment of modern analysis is that by starting from these five very primitive axioms and some additional axioms from set theory, we can build all the other number systems, create functions, and do all the algebra and calculus that we are used to.
\end{note}

\section{Addition}\label{i:sec:2.2}

\begin{defn}[Addition of natural numbers]\label{i:2.2.1}
  Let \(m\) be a natural numbers.
  To add zero to \(m\), we define \(0 + m \coloneqq m\).
  Now suppose inductively that we have defined how to add \(n\) to \(m\).
  Then we can add \(n\pp\) to \(m\) by defining \((n\pp) + m \coloneqq (n + m)\pp\).
\end{defn}

\begin{note}
  From a logical point of view, there is no difference between a lemma, proposition, theorem, or corollary
  - they are all claims waiting to be proved.
  However, we use these terms to suggest different levels of importance and difficulty.
  A lemma is an easily proved claim which is helpful for proving other propositions and theorems, but is usually not particularly interesting in its own right.
  A proposition is a statement which is interesting in its own right, while a theorem is a more important statement than a proposition which says something definitive on the subject, and often takes more effort to prove than a proposition or lemma.
  A corollary is a quick consequence of a proposition or theorem that was proven recently.
\end{note}

\begin{ac}\label{i:ac:2.2.1}
  The sum \(n + m\) of two natural numbers \(n, m\) is again a natural number.
\end{ac}

\begin{proof}[\pf{i:ac:2.2.1}]
  Let \(m, n\) be a natural number.
  We use induction on \(n\).
  For \(n = 0\), by \cref{i:2.2.1} we have \(0 + m = m\), which is a natural number by definition.
  So the base case holds.
  Suppose inductively that for some natural number \(n\) we know that \(n + m\) is a natural number.
  We want to show that \((n\pp) + m\) is also a natural number.
  By \cref{i:2.2.1} we have \((n\pp) + m = (n + m)\pp\).
  By induction hypothesis we know that \(n + m\) is a natural number.
  Thus by \cref{i:2.2} we know that \((n + m)\pp\) is again a natural number.
  This closes the induction.
\end{proof}

\begin{lem}\label{i:2.2.2}
  For any natural number \(n\), \(n + 0 = n\).
\end{lem}

\begin{proof}[\pf{i:2.2.2}]
  We use induction on \(n\).
  The base case \(0 + 0 = 0\) follows since we know that \(0 + m = m\) for every natural number \(m\) (\cref{i:2.2.1}), and \(0\) is a natural number (\cref{i:2.1}).
  Now suppose inductively that \(n + 0 = n\).
  We wish to show that \((n\pp) + 0 = n\pp\).
  But by \cref{i:2.2.1}, \((n\pp) + 0\) is equal to \((n + 0)\pp\), which is equal to \(n\pp\) since \(n + 0 = n\).
  This closes the induction.
\end{proof}

\begin{lem}\label{i:2.2.3}
  For any natural numbers \(n\) and \(m\), \(n + (m\pp) = (n + m)\pp\).
\end{lem}

\begin{proof}[\pf{i:2.2.3}]
  We induct on \(n\) (keeping \(m\) fixed).
  We first consider the base case \(n = 0\).
  In this case we have to prove \(0 + (m\pp) = (0 + m)\pp\).
  But by \cref{i:2.2.1}, \(0 + (m\pp) = m\pp\) and \(0 + m = m\), so both sides are equal to \(m\pp\) and are thus equal to each other.
  Now we assume inductively that \(n + (m\pp) = (n + m)\pp\);
  we now have to show that \((n\pp) + (m\pp) = ((n\pp) + m)\pp\).
  The left-hand side is \((n + (m\pp))\pp\) by \cref{i:2.2.1}, which is equal to \(((n+m)\pp)\pp\) by the inductive hypothesis.
  Similarly, we have \((n\pp) + m = (n + m)\pp\) by \cref{i:2.2.1}, and so the right-hand side is also equal to \(((n + m)\pp)\pp\).
  Thus both sides are equal to each other, and we have closed the induction.
\end{proof}

\begin{ac}\label{i:ac:2.2.2}
  For any natural number \(n\), we have \(n\pp = n + 1\).
\end{ac}

\begin{proof}[\pf{i:ac:2.2.2}]
  Since \(n\) is a natural number, by \cref{i:2.2} we know that \(n\pp\) is also a natural number.
  Thus we can apply \cref{i:2.2.2} to derive the following fact:
  \begin{align*}
    n\pp & = (n\pp) + 0 &  & \by{i:2.2.2} \\
         & = (n + 0)\pp &  & \by{i:2.2.1} \\
         & = n + (0\pp) &  & \by{i:2.2.3} \\
         & = n + 1.
  \end{align*}
\end{proof}

\begin{prop}[Addition is commutative]\label{i:2.2.4}
  For any natural numbers \(n\) and \(m\), \(n + m = m + n\).
\end{prop}

\begin{proof}[\pf{i:2.2.4}]
  We shall use induction on \(n\) (keeping \(m\) fixed).
  First we do the base case \(n = 0\), i.e., we show \(0 + m = m + 0\).
  By \cref{i:2.2.1}, \(0 + m = m\), while by \cref{i:2.2.2}, \(m + 0 = m\).
  Thus the base case is done.
  Now suppose inductively that \(n + m = m + n\), now we have to prove that \((n\pp) + m = m + (n\pp)\) to close the induction.
  By \cref{i:2.2.1}, \((n\pp) + m = (n + m)\pp\).
  By \cref{i:2.2.3}, \(m + (n\pp) = (m + n)\pp\), but this is equal to \((n + m)\pp\) by the inductive hypothesis \(n+m=m+n\).
  Thus \((n\pp) + m = m + (n\pp)\) and we have closed the induction.
\end{proof}

\begin{prop}[Addition is associative]\label{i:2.2.5}
  For any natural numbers \(a\), \(b\), \(c\), we have \((a + b) + c = a + (b + c)\).
\end{prop}

\begin{proof}[\pf{i:2.2.5}]
  We use induction on \(c\) and keep both \(a\) and \(b\) fixed.
  For \(c = 0\), we have
  \begin{align*}
    (a + b) + 0 & = a + b        &  & \by{i:2.2.2} \\
                & = a + (b + 0). &  & \by{i:2.2.2}
  \end{align*}
  Thus the base case holds.
  Suppose inductively that \((a + b) + c = a + (b + c)\) for some natural number \(c\).
  We want to show that \((a + b) + (c\pp) = a + (b + (c\pp))\).
  But this is true since
  \begin{align*}
    (a + b) + (c\pp) & = ((a + b) + c)\pp  &  & \by{i:2.2.3} \\
                     & = (a + (b + c))\pp  &  & \byIH        \\
                     & = a + (b + c)\pp    &  & \by{i:2.2.3} \\
                     & = a + (b + (c\pp)). &  & \by{i:2.2.3}
  \end{align*}
  This closes the induction.
\end{proof}

\begin{note}
  Because of associativity showed in \cref{i:2.2.5} we can write sums such as \(a + b + c\) without having to worry about which order the numbers are being added together.
\end{note}

\begin{prop}[Cancellation law]\label{i:2.2.6}
  Let \(a, b, c\) be natural numbers such that \(a + b = a + c\).
  Then we have \(b = c\).
\end{prop}

\begin{proof}[\pf{i:2.2.6}]
  We prove this by induction on \(a\).
  First consider the base case \(a = 0\).
  Then we have \(0 + b = 0 + c\), which by \cref{i:2.2.1} implies that \(b = c\) as desired.
  Now suppose inductively that we have the cancellation law for \(a\) (so that \(a + b = a + c\) implies \(b = c\));
  we now have to prove the cancellation law for \(a\pp\).
  In other words, we assume that \((a\pp) + b = (a\pp) + c\) and need to show that \(b = c\).
  By \cref{i:2.2.1}, \((a\pp) + b = (a + b)\pp\) and \((a\pp) + c = (a + c)\pp\) and so we have \((a + b)\pp = (a + c)\pp\).
  By \cref{i:2.4}, we have \(a + b = a + c\).
  Since we already have the cancellation law for \(a\), we thus have \(b = c\) as desired.
  This closes the induction.
\end{proof}

\begin{defn}[Positive natural numbers]\label{i:2.2.7}
  A natural number \(n\) is said to be \emph{positive} iff it is not equal to \(0\).
\end{defn}

\begin{prop}\label{i:2.2.8}
  If \(a\) is a positive natural number and \(b\) is a natural number, then \(a + b\) is positive (and hence \(b + a\) is also, by \cref{i:2.2.4}).
\end{prop}

\begin{proof}[\pf{i:2.2.8}]
  We use induction on \(b\).
  If \(b = 0\), then \(a + b = a + 0 = a\), which is positive by \cref{i:2.2.7}, so this proves the base case.
  Now suppose inductively that \(a + b\) is positive.
  Then \(a + (b\pp) = (a + b)\pp\), which cannot be zero by \cref{i:2.3}, and is hence positive.
  This closes the induction.
\end{proof}

\begin{cor}\label{i:2.2.9}
  If \(a\) and \(b\) are natural numbers such that \(a + b = 0\), then \(a = 0\) and \(b = 0\).
\end{cor}

\begin{proof}[\pf{i:2.2.9}]
  Suppose for sake of contradiction that \(a \neq 0\) or \(b \neq 0\).
  If \(a \neq 0\) then \(a\) is positive, and hence \(a + b = 0\) is positive by \cref{i:2.2.8}, a contradiction.
  Similarly if \(b \neq 0\) then \(b\) is positive, and again \(a + b = 0\) is positive by \cref{i:2.2.8}, a contradiction.
  Thus \(a\) and \(b\) must both be zero.
\end{proof}

\begin{lem}\label{i:2.2.10}
  Let \(a\) be a positive natural number.
  Then there exists exactly one natural number \(b\) such that \(b\pp = a\).
\end{lem}

\begin{proof}[\pf{i:2.2.10}]
  Let \(P(n)\) be the statement ``either \(n = 0\) or there exists a natural number \(m\) such that \(m\pp = n\)''.
  We use induction to show that \(P(n)\) is true for all natural number \(n\).
  Clearly \(P(0)\) is true.
  So suppose that \(P(n)\) is true for some natural number \(n\).
  We want to show that \(P(n\pp)\) is true.
  By \cref{i:2.3} we know that \(n\pp \neq 0\).
  So we have to show that there exists a natural number \(m\) such that \(m\pp = n\pp\).
  By \cref{i:2.4} we see that \(m = n\).
  Thus \(P(n\pp)\) is true and this closes the induction.

  Now we prove the existence of \(b\).
  From first part of the proof we know that \(P(a)\) is true.
  Since \(a\) is a positive natural number, by \cref{i:2.2.7} we know that \(a \neq 0\).
  Thus there must exist a natural number \(b\) such that \(b\pp = a\).

  Finally we prove the uniqueness of \(b\).
  Suppose that there exists another natural number \(c\) such that \(c\pp = a\).
  But this means \(b\pp = c\pp\).
  Thus by \cref{i:2.4} we have \(b = c\).
\end{proof}

\begin{defn}[Ordering of the natural numbers]\label{i:2.2.11}
  Let \(n\) and \(m\) be natural numbers.
  We say that \(n\) is \emph{greater than or equal to} \(m\), and write \(n \geq m\) or \(m \leq n\), iff we have \(n = m + a\) for some natural number \(a\).
  We say that \(n\) is \emph{strictly greater than} \(m\), and write \(n > m\) or \(m < n\), iff \(n \geq m\) and \(n \neq m\).
\end{defn}

\begin{ac}\label{i:ac:2.2.3}
  We have \(n\pp > n\) for any natural number \(n\).
  Therefore there is no largest natural number \(n\), because the next number \(n\pp\) is always larger.
\end{ac}

\begin{proof}[\pf{i:ac:2.2.3}]
  Let \(n\) be a natural number.
  By \cref{i:ac:2.2.2} we have \(n\pp = n + 1\).
  Since \(1\) is a natural number, by \cref{i:2.2.11} we have \(n\pp \geq n\).
  To show that \(n\pp > n\), by \cref{i:2.2.11} we only need to show that \(n\pp \neq n\).

  We use induction to show that \(n\pp \neq n\).
  For \(n = 0\), we have \(0\pp \neq 0\) by \cref{i:2.3}.
  Thus the base case holds.
  Suppose inductively that \(n\pp \neq n\) for some natural number \(n\).
  Then we have
  \begin{align*}
             & n\pp \neq n          &  & \byIH      \\
    \implies & (n\pp)\pp \neq n\pp. &  & \by{i:2.4}
  \end{align*}
  This closes the induction and we conclude that \(n\pp > n\) for all natural number \(n\).
\end{proof}

\begin{ac}\label{i:ac:2.2.4}
  We have \(n \geq 0\) for every natural number \(n\).
  If \(n\) is a positive natural number, then \(n > 0\).
\end{ac}

\begin{proof}[\pf{i:ac:2.2.4}]
  Let \(n\) be a natural number.
  By \cref{i:2.2.1} we have \(n = 0 + n\).
  Thus by \cref{i:2.2.11} we have \(n \geq 0\).

  Now suppose that \(n\) is a positive natural number.
  By \cref{i:2.2.7} this means \(n \neq 0\).
  From first paragraph we see that \(n \geq 0\).
  Thus by \cref{i:2.2.11} we have \(n > 0\).
\end{proof}

\begin{prop}[Basic properties of order for natural numbers]\label{i:2.2.12}
  Let \(a\), \(b\), \(c\) be natural numbers.
  Then
  \begin{enumerate}
    \item (Order is reflexive) \(a \geq a\).
    \item (Order is transitive) If \(a \geq b\) and \(b \geq c\), then \(a \geq c\).
    \item (Order is anti-symmetric) If \(a \geq b\) and \(b \geq a\), then \(a = b\).
    \item (Addition preserves order) \(a \geq b\) iff \(a + c \geq b + c\).
    \item \(a < b\) iff \(a\pp \leq b\).
    \item \(a < b\) iff \(b = a + d\) for some \emph{positive} number \(d\).
  \end{enumerate}
\end{prop}

\begin{proof}[\pf{i:2.2.12}(a)]
  Let \(a\) be a natural number.
  Then
  \begin{align*}
             & \begin{dcases}
                 0 \text{ is a natural number} \\
                 a = a + 0
               \end{dcases} &  & \by{i:2.1,i:2.2.2}                \\
    \implies & a \geq a.                        &  & \by{i:2.2.11}
  \end{align*}
\end{proof}

\begin{proof}[\pf{i:2.2.12}(b)]
  Let \(a, b, c\) be natural numbers and suppose that \(a \geq b\) and \(b \geq c\).
  By \cref{i:2.2.11} there exist some natural numbers \(d\) and \(e\) such that \(a = b + d\) and \(b = c + e\).
  Then we have
  \begin{align*}
             & \begin{dcases}
                 a = b + d = (c + e) + d = c + (e + d) \\
                 e + d \text{ is a natural number}
               \end{dcases} &  & \by{i:2.2.5,i:ac:2.2.1}                   \\
    \implies & a \geq c.                                &  & \by{i:2.2.11}
  \end{align*}
\end{proof}

\begin{proof}[\pf{i:2.2.12}(c)]
  Let \(a, b\) be natural numbers and suppose that \(a \geq b\) and \(b \geq a\).
  By \cref{i:2.2.11} there exist some natural numbers \(c\) and \(d\) such that \(a = b + c\) and \(b = a + d\).
  Then we have
  \begin{align*}
             & \begin{dcases}
                 a = a + 0 \\
                 a = b + c = (a + d) + c = a + (d + c)
               \end{dcases} &  & \by{i:2.2.2,i:2.2.5}                  \\
    \implies & 0 = d + c                             &  & \by{i:2.2.6} \\
    \implies & d = c = 0                             &  & \by{i:2.2.9} \\
    \implies & a = b + 0 = b.                        &  & \by{i:2.2.2}
  \end{align*}
\end{proof}

\begin{proof}[\pf{i:2.2.12}(d)]
  Let \(a, b, c\) be natural numbers.
  Then we have
  \begin{align*}
             & a \geq b                                                               \\
    \implies & a = b + d \text{ for some natural number } d &  & \by{i:2.2.11}        \\
    \implies & a + c = c + a = c + (b + d)                  &  & \by{i:2.2.4}         \\
             & = (c + b) + d = (b + c) + d                  &  & \by{i:2.2.4,i:2.2.5} \\
    \implies & a + c \geq b + c                             &  & \by{i:2.2.11}
  \end{align*}
  and
  \begin{align*}
             & a + c \geq b + c                                                                 \\
    \implies & a + c = (b + c) + d \text{ for some natural number } d &  & \by{i:2.2.11}        \\
    \implies & c + a = (c + b) + d = c + (b + d)                      &  & \by{i:2.2.4,i:2.2.5} \\
    \implies & a = b + d                                              &  & \by{i:2.2.6}         \\
    \implies & a \geq b.                                              &  & \by{i:2.2.11}
  \end{align*}
  Thus we conclude that \(a \geq b \iff a + c \geq b + c\).
\end{proof}

\begin{proof}[\pf{i:2.2.12}(e)]
  Let \(a, b\) be natural numbers.
  First suppose that \(a < b\).
  Then we have
  \begin{align*}
             & a < b                                                                                         \\
    \implies & \begin{dcases}
                 a \leq b \\
                 a \neq b
               \end{dcases}                                                      &  & \by{i:2.2.11}          \\
    \implies & \begin{dcases}
                 b = a + c \text{ for some natural number } c \\
                 a \neq b
               \end{dcases}                     &  & \by{i:2.2.11}                                           \\
    \implies & c \neq 0                                                            &  & \by{i:2.2.2}         \\
    \implies & c \text{ is a positive natural number}                              &  & \by{i:2.2.8}         \\
    \implies & \text{there exists a natural number } d \text{ such that } d\pp = c &  & \by{i:2.2.10}        \\
    \implies & b = a + (d\pp) = (a + d)\pp = (a\pp) + d                            &  & \by{i:2.2.1,i:2.2.3} \\
    \implies & a\pp \leq b.                                                        &  & \by{i:2.2.11}
  \end{align*}

  Now suppose that \(a\pp \leq b\).
  Then we have
  \begin{align*}
             & a\pp \leq b                                                                 \\
    \implies & b = (a\pp) + c \text{ for some natural number } c &  & \by{i:2.2.11}        \\
    \implies & b = (a\pp) + c = (a + c)\pp = a + (c\pp)          &  & \by{i:2.2.1,i:2.2.3} \\
    \implies & a \leq b.                                         &  & \by{i:2.2.11}
  \end{align*}
  By \cref{i:2.3} we know that \(c\pp \neq 0\).
  Thus by \cref{i:2.2.6} we must have \(b \neq a\).
  (If \(b = a\), then \(b = a + c\pp = a + 0\) implies \(c\pp = 0\), a contradiction.)
  By \cref{i:2.2.11} this means \(a < b\).
  From all proofs above we conclude that \(a < b \iff a\pp \leq b\).
\end{proof}

\begin{proof}[\pf{i:2.2.12}(f)]
  Let \(a, b\) be natural numbers.
  Then we have
  \begin{align*}
         & a < b                                                                            \\
    \iff & a\pp \leq b                                            &  & \by{i:2.2.12}[e]     \\
    \iff & b = (a\pp) + c \text{ for some natural number } c      &  & \by{i:2.2.11}        \\
    \iff & b = (a\pp) + c = (a + c)\pp = a + (c\pp)               &  & \by{i:2.2.1,i:2.2.3} \\
    \iff & b = a + d \text{ for some positive natural number } d. &  & \by{i:2.3,i:2.2.10}
  \end{align*}
\end{proof}

\begin{prop}[Trichotomy of order for natural numbers]\label{i:2.2.13}
  Let \(a\) and \(b\) be natural numbers.
  Then exactly one of the following statements is true: \(a < b\), \(a = b\), or \(a > b\).
\end{prop}

\begin{proof}[\pf{i:2.2.13}]
  First we show that we cannot have more than one of the statements \(a < b\), \(a = b\), \(a > b\) holding at the same time.
  If \(a < b\) then \(a \neq b\) by \cref{i:2.2.11}, and if \(a > b\) then \(a \neq b\) by \cref{i:2.2.11}.
  If \(a > b\) and \(a < b\) then by \cref{i:2.2.12}(c) we have \(a = b\), a contradiction.
  Thus no more than one of the statements is true.

  Now we show that at least one of the statements is true.
  We keep \(b\) fixed and induct on \(a\).
  When \(a = 0\), by \cref{i:2.2.1} we have \(b = 0 + b\) and thus by \cref{i:2.2.11} we have \(0 \leq b\) for any natural number \(b\).
  So we have either \(0 = b\) or \(0 < b\), which proves the base case.
  Now suppose we have proven the proposition for \(a\), and now we prove the proposition for \(a\pp\).
  From the trichotomy for \(a\), there are three cases: \(a < b\), \(a = b\), and \(a > b\).
  \begin{itemize}
    \item If \(a > b\), then by \cref{i:2.2.12}(d) we have \(a\pp \geq b\pp\).
          Since \(b\pp > b\), by \cref{i:2.2.12}(b) we have \(a\pp \geq b\).
          Then we must have \(a\pp > b\), otherwise by \cref{i:2.2.11} we would have \(a\pp = b\) and by \cref{i:2.2.12}(e) this implies \(a < b\) and contradicts to \(a > b\)
          (the contradiction is proved by the first paragraph).
    \item If \(a = b\), then by \cref{i:2.4} we have \(a\pp = b\pp\).
          Since \(b\pp > b\), by \cref{i:2.2.12}(b) we have \(a\pp \geq b\).
          Then we must have \(a\pp > b\), otherwise by \cref{i:2.2.11} we would have \(a\pp = b\) and by \cref{i:2.2.12}(e) this implies \(a < b\) and contradicts to \(a = b\)
          (the contradiction is proved by the first paragraph).
    \item If \(a < b\), then by \cref{i:2.2.12}(e) we have \(a\pp \leq b\).
          Thus either \(a\pp = b\) or \(a\pp < b\), and in either case we are done.
  \end{itemize}
  This closes the induction.
\end{proof}

\begin{prop}[Strong principle of induction]\label{i:2.2.14}
  Let \(m_0\) be a natural number, and let \(P(m)\) be a property pertaining to an arbitrary natural number \(m\).
  Suppose that for each \(m \geq m_0\), we have the following implication: if \(P(m')\) is true for all natural numbers \(m_0 \leq m' < m\), then \(P(m)\) is also true.
  (In particular, this means that \(P(m_0)\) is true, since in this case the hypothesis is vacuous.)
  Then we can conclude that \(P(m)\) is true for all natural numbers \(m \geq m_0\).
\end{prop}

\begin{proof}[\pf{i:2.2.14}]
  Let \(n\) be a natural number and let \(Q(n)\) be the statement ``\(P(m)\) is true for all natural number \(m\) satisfying \(m_0 \leq m < n\)''.
  We use induction to show that \(Q(n)\) is true for all natural number \(n\).

  For \(n = 0\), we want to show that \(Q(0)\) is true.
  However, we know that \(0 \leq m_0\) for all natural number \(m_0\).
  Thus, either \(0 = m_0\) or \(0 < m_0\) and so we split into cases.
  \begin{itemize}
    \item If \(0 < m_0\), the statement ``\(P(m)\) is true for all natural number \(m\) satisfying \(m_0 \leq m < n\)'' is vacuously true since there does not exist a natural number \(m\) satisfying \(0 < m_0 \leq m < n = 0\).
          Thus \(Q(0)\) is true in this case.
    \item If \(0 = m_0\), then the statement ``\(P(m)\) is true for all natural number \(m\) satisfying \(m_0 \leq m < n\)'' is vacuously true since there does not exist a natural number \(m\) satisfying \(0 = m_0 \leq m < n = 0\).
          Hence, \(Q(0)\) is true in this case.
  \end{itemize}
  From all cases above we see that \(Q(0)\) is true.
  Thus the base case holds.

  Suppose inductively that \(Q(n)\) is true for some natural number \(n\).
  We need to show that \(Q(n\pp)\) is true.
  Using induction hypothesis (\(Q(n)\) is true) and the hypothesis of \(P\) we see that \(P(n)\) is true.
  Since \(n < n\pp\), we know that \(P(m)\) is true for all natural number \(m\) satisfying \(m_0 \leq m \leq n < n\pp\).
  So \(P(m)\) is true for all natural number \(m\) satisfying \(m_0 \leq m < n\pp\) which in turn implies that \(Q(n\pp)\) is true.
  This closes the induction and hence we can conclude that \(Q(n)\) is true for any natural number \(n\).

  Since \(Q(n)\) is true for all natural number \(n\), by the hypothesis of \(P\) we know that \(P(n)\) is true for all natural number \(n\).
  In particular, we see that \(P(n)\) is true for all natural number \(n\) satisfying \(n \geq m_0\).
\end{proof}

\begin{rmk}\label{i:2.2.15}
  In applications we usually use \cref{i:2.2.14} with \(m_0 = 0\) or \(m_0 = 1\).
\end{rmk}

\exercisesection

\begin{ex}\label{i:ex:2.2.1}
  Prove \cref{i:2.2.5}.
\end{ex}

\begin{proof}[\pf{i:ex:2.2.1}]
  See \cref{i:2.2.5}.
\end{proof}

\begin{ex}\label{i:ex:2.2.2}
  Prove \cref{i:2.2.10}.
\end{ex}

\begin{proof}[\pf{i:ex:2.2.2}]
  See \cref{i:2.2.10}.
\end{proof}

\begin{ex}\label{i:ex:2.2.3}
  Prove \cref{i:2.2.12}.
\end{ex}

\begin{proof}[\pf{i:ex:2.2.3}]
  See \cref{i:2.2.12}.
\end{proof}

\begin{ex}\label{i:ex:2.2.4}
  Justify the three statements marked in the proof of \cref{i:2.2.13}.
\end{ex}

\begin{proof}[\pf{i:ex:2.2.4}]
  See \cref{i:2.2.13}.
\end{proof}

\begin{ex}\label{i:ex:2.2.5}
  Prove \cref{i:2.2.14}.
\end{ex}

\begin{proof}[\pf{i:ex:2.2.5}]
  See \cref{i:2.2.14}.
\end{proof}

\begin{ex}[Principle of backwards induction]\label{i:ex:2.2.6}
  Let \(n\) be a natural number, and let \(P(m)\) be a property pertaining to the natural numbers such that whenever \(P(m\pp)\) is true, then \(P(m)\) is true.
  Suppose that \(P(n)\) is also true.
  Prove that \(P(m)\) is true for all natural numbers \(m \leq n\);
\end{ex}

\begin{proof}[\pf{i:ex:2.2.6}]
  We use induction on \(n\).
  For \(n = 0\), the only natural number \(m\) satisfying \(m \leq n = 0\) is \(0\).
  By hypothesis \(P(0)\) is true.
  Therefore, the base case holds trivially.

  Suppose inductively that for some natural number \(n\) we have the implication ``\(P(n)\) is true implies \(P(m)\) is true for all natural number \(m\) satisfying \(m \leq n\)''.
  We want to show the implication ``\(P(n\pp)\) is true implies \(P(m)\) is true for all natural number \(m\) satisfying \(m \leq n\pp\)''.
  But when \(P(n\pp)\) is true, by the hypothesis of \(P\) we know that \(P(n)\) is true.
  Thus we can apply induction hypothesis to derive ``\(P(m)\) is true for all natural number \(m\) satisfying \(m \leq n\)''.
  Combining the statement ``\(P(n\pp)\) is true'' we see that the statement ``\(P(m)\) is true for all natural number \(m\) satisfying \(m \leq n\pp\)'' is true.
  This closes the induction.
\end{proof}

\begin{ex}\label{i:ex:2.2.7}
  Let \(n\) be a natural number, and let \(P(m)\) be a property pertaining to the natural numbers such that whenever \(P(m)\) is true, \(P(m\pp)\) is true.
  Show that if \(P(n)\) is true, then \(P(m)\) is true for any natural number \(m\) satisfying \(m \geq n\).
  This principle is sometimes referred to as \emph{the principle of induction starting from the base case \(n\)}.
\end{ex}

\begin{proof}[\pf{i:ex:2.2.7}]
  Suppose that \(P(n)\) is true.
  Let \(Q(k) = P(n + k)\) for every natural number \(k\).
  We use induction to show that \(Q(k)\) is true for every natural number \(k\).

  For \(k = 0\), by \cref{i:2.2.2} we have \(Q(0) = P(n + 0) = P(n)\).
  Since \(P(n)\) is true by hypothesis, we know that \(Q(0)\) is true.
  Thus the base case holds.

  Suppose inductively that \(Q(k)\) is true for some natural number \(k\).
  We want to show that \(Q(k\pp)\) is true.
  By \cref{i:2.2.3} we have \(Q(k\pp) = P(n + (k\pp)) = P((n + k)\pp)\).
  By induction hypothesis we have \(Q(k) = P(n + k)\) is true.
  Thus we can use the hypothesis of \(P\) to show that \(P((n + k)\pp)\) is also true.
  This closes the induction.
  Thus we conclude that \(P(n + k)\) is true for every natural number \(k\).

  By \cref{i:2.2.11} we know that for every natural number \(m\), we have \(m \geq n \iff m = n + k\) for some natural number \(k\).
  Thus we see that \(P(m)\) is true for every natural number \(m\) satisfying \(m \geq n\).
\end{proof}

\section{Multiplication}\label{i:sec:2.3}

\begin{defn}[Multiplication of natural numbers]\label{i:2.3.1}
  Let \(m\) be a natural number.
  To multiply zero to \(m\), we define \(0 \times m \coloneqq 0\).
  Now suppose inductively that we have defined how to multiply \(n\) to \(m\).
  Then we can multiply \(n\pp\) to \(m\) by defining \((n\pp) \times m \coloneqq (n \times m) + m\).
\end{defn}

\begin{ac}\label{i:ac:2.3.1}
  The product of two natural numbers is a natural number.
\end{ac}

\begin{proof}[\pf{i:ac:2.3.1}]
  Let \(n, m\) be two natural numbers.
  We induct on \(n\).
  For \(n = 0\), by \cref{i:2.3.1}, we have \(0 \times m = 0\), which is a natural number by \cref{i:2.1}.
  So the base case holds.
  Suppose inductively that, for some natural number \(n\), we know that \(n \times m\) is a natural number.
  We want to show that \((n\pp) \times m\) is a natural number.
  By \cref{i:2.3.1}, \((n\pp) \times m = (n \times m) + m\).
  By induction hypothesis, \(n \times m\) is a natural number.
  By \cref{i:ac:2.2.1}, \((n \times m) + m\) is a natural number.
  Thus, \((n\pp) \times m\) is a natural number.
  This closes the induction.
\end{proof}

\begin{ac}\label{i:ac:2.3.2}
  Let \(n\) be a natural number.
  Then \(n \times 0 = 0\).
\end{ac}

\begin{proof}[\pf{i:ac:2.3.2}]
  We induct on \(n\).
  For \(n = 0\), by \cref{i:2.3.1}, we have \(0 \times 0 = 0\).
  So the base case holds.
  Suppose inductively that, for some natural number \(n\), we have \(n \times 0 = 0\).
  Then, for \(n\pp\), we have
  \begin{align*}
    (n\pp) \times 0 & = (n \times 0) + 0 &  & \by{i:2.3.1} \\
                    & = 0 + 0            &  & \byIH        \\
                    & = 0.               &  & \by{i:2.2.1}
  \end{align*}
  This closes the induction.
\end{proof}

\begin{ac}\label{i:ac:2.3.3}
  Let \(n, m\) be natural numbers.
  Then \(n \times (m\pp) = (n \times m) + n\).
\end{ac}

\begin{proof}[\pf{i:ac:2.3.3}]
  We induct on \(n\) and fix \(m\).
  For \(n = 0\), by \cref{i:2.3.1}, we have \(0 \times (m\pp) = 0\).
  So the base case holds.
  Suppose inductively that, for some natural number \(n\), we have \(n \times (m\pp) = (n \times m) + n\).
  Then, for \(n\pp\), we have
  \begin{align*}
    (n\pp) \times (m\pp)
     & = (n \times (m\pp)) + (m\pp)  &  & \by{i:2.3.1} \\
     & = ((n \times m) + n) + (m\pp) &  & \byIH        \\
     & = (n \times m) + (n + (m\pp)) &  & \by{i:2.2.5} \\
     & = (n \times m) + ((n + m)\pp) &  & \by{i:2.2.3} \\
     & = (n \times m) + ((m + n)\pp) &  & \by{i:2.2.4} \\
     & = (n \times m) + (m + (n\pp)) &  & \by{i:2.2.3} \\
     & = ((n \times m) + m) + (n\pp) &  & \by{i:2.2.5} \\
     & = ((n\pp) \times m) + (n\pp). &  & \by{i:2.3.1}
  \end{align*}
  This closes the induction.
\end{proof}

\begin{lem}[Multiplication is commutative]\label{i:2.3.2}
  Let \(n, m\) be natural numbers.
  Then \(n \times m = m \times n\).
\end{lem}

\begin{proof}[\pf{i:2.3.2}]
  We induct on \(n\) and fix \(m\).
  For \(n = 0\), by \cref{i:2.3.1}, we have \(0 \times m = 0\), and by \cref{i:ac:2.3.2}, we have \(m \times 0 = 0\).
  So the base case holds.
  Suppose inductively that, for some natural number \(n\), we have \(n \times m = m \times n\).
  Then, for \(n\pp\), we have
  \begin{align*}
    (n\pp) \times m & = (n \times m) + m &  & \by{i:2.3.1}    \\
                    & = (m \times n) + m &  & \byIH           \\
                    & = m \times (n\pp). &  & \by{i:ac:2.3.3}
  \end{align*}
  This closes the induction.
\end{proof}

\begin{note}
  We will now abbreviate \(n \times m\) as \(nm\), and use the usual convention that multiplication takes precedence over addition, thus for instance \(ab + c\) means \((a \times b) + c\), not \(a \times (b + c)\).
\end{note}

\begin{lem}[Positive natural numbers have no zero divisors]\label{i:2.3.3}
  Let \(n, m\) be natural numbers.
  Then \(n \times m = 0\) iff at least one of \(n, m\) is equal to zero.
  In particular, if \(n\) and \(m\) are both positive, then \(nm\) is also positive.
\end{lem}

\begin{proof}[\pf{i:2.3.3}]
  First suppose that \(n \times m = 0\).
  Suppose for sake of contradiction that \(n \neq 0 \neq m\).
  By \cref{i:2.2.7}, this means \(n, m\) are positive natural numbers.
  Then by \cref{i:2.2.10}, there exist some natural numbers \(a, b\), such that \(n = a\pp\) and \(m = b\pp\).
  Thus, we have
  \begin{align*}
    n \times m & = (a\pp) \times (b\pp)                        \\
               & = a \times (b\pp) + (b\pp). &  & \by{i:2.3.1}
  \end{align*}
  By \cref{i:2.3}, we know that \(b\pp \neq 0\).
  Thus by \cref{i:2.2.8}, we know that \(n \times m\) is a positive natural number.
  But this contradict to \(n \times m = 0\).
  Thus, we must have either \(n = 0\) or \(m = 0\).

  Now suppose that \(n = 0\) or \(m = 0\).
  If \(n = 0\), then by \cref{i:2.3.1}, we have \(n \times m = 0 \times m = 0\).
  If \(m = 0\), then by \cref{i:ac:2.3.2}, we have \(n \times m = n \times 0 = 0\).
  In either cases, we have \(n \times m = 0\).

  From all proofs above we conclude that \(n \times m = 0 \iff (n = 0) \lor (m = 0)\).
  Thus, we have \(n \times m \neq 0 \iff (n \neq 0) \land (m \neq 0)\).
  By \cref{i:2.2.7}, we see that \(n, m\) are positive natural numbers iff \(n \times m \neq 0\).
\end{proof}

\begin{ac}\label{i:ac:2.3.4}
  Let \(n\) be a natural number.
  Then \(n1 = 1n = n\).
\end{ac}

\begin{proof}[\pf{i:ac:2.3.4}]
  By \cref{i:2.3.2}, we know that \(n1 = 1n\).
  Thus, we only need to show that \(n1 = n\).
  We induct on \(n\).
  For \(n = 0\), by \cref{i:2.3.3}, we have \(0 \times 1 = 0\).
  So the base case holds.
  Suppose inductively that \(n1 = n\) is true for some natural number \(n\).
  Then, for \(n + 1\), by \cref{i:2.3.1}, we have \((n + 1) \times 1 = n1 + 1\).
  By induction hypothesis, we have \(n1 = n\).
  Thus, we have \((n + 1) \times 1 = n + 1\), and this closes the induction.
\end{proof}

\begin{prop}[Distributive law]\label{i:2.3.4}
  For any natural numbers \(a, b, c\), we have \(a(b + c) = ab + ac\) and \((b + c)a = ba + ca\).
\end{prop}

\begin{proof}[\pf{i:2.3.4}]
  Since multiplication is commutative we only need to show the first identity \(a(b + c) = ab + ac\).
  We keep \(a\) and \(b\) fixed, and use induction on \(c\).
  Let's prove the base case \(c = 0\), i.e., \(a(b + 0) = ab + a0\).
  The left-hand side is \(ab\), while the right-hand side is \(ab + 0 = ab\), so we are done with the base case.
  Now let us suppose inductively that \(a(b + c) = ab + ac\), and let us prove that \(a(b + (c\pp)) = ab + a(c\pp)\).
  The left-hand side is \(a((b + c)\pp) = a(b + c) + a\) by \cref{i:ac:2.3.3}, while the right-hand side is \(ab + ac + a = a(b + c) + a\) by the induction hypothesis, and so we can close the induction.
\end{proof}

\begin{prop}[Multiplication is associative]\label{i:2.3.5}
  For any natural numbers \(a, b, c\), we have \((a \times b) \times c = a \times (b \times c)\).
\end{prop}

\begin{proof}[\pf{i:2.3.5}]
  We keep \(a\) and \(b\) fixed, and use induction on \(c\).
  For \(c = 0\), by \cref{i:ac:2.3.2}, we have \((a \times b) \times 0 = 0 = a \times 0 = a \times (b \times 0)\).
  So the base case holds.
  Suppose inductively that, for some natural number \(c\), we have \((a \times b) \times c = a \times (b \times c)\).
  Then, for \(c\pp\), we have
  \begin{align*}
    (a \times b) \times (c\pp) & = (a \times b) \times c + a \times b &  & \by{i:ac:2.3.3} \\
                               & = a \times (b \times c) + a \times b &  & \byIH           \\
                               & = a \times (b \times c + b)          &  & \by{i:2.3.4}    \\
                               & = a \times (b \times (c\pp)).        &  & \by{i:ac:2.3.3}
  \end{align*}
  This closes the induction.
\end{proof}

\begin{prop}[Multiplication preserves order]\label{i:2.3.6}
  If \(a, b\) are natural numbers such that \(a < b\), and \(c\) is positive, then \(ac < bc\).
\end{prop}

\begin{proof}[\pf{i:2.3.6}]
  Since \(a < b\), we have \(b = a + d\) for some positive \(d\) by \cref{i:2.2.12}(f).
  Multiplying by \(c\) and using the distributive law (\cref{i:2.3.4}) we obtain \(bc = ac + dc\).
  Since \(d\) is positive, and \(c\) is positive, \(dc\) is positive (\cref{i:2.3.3}), and hence \(ac < bc\) (by \cref{i:2.2.11}), as desired.
\end{proof}

\begin{cor}[Cancellation law]\label{i:2.3.7}
  Let \(a, b, c\) be natural numbers such that \(ac = bc\) and \(c\) is non-zero.
  Then \(a = b\).
\end{cor}

\begin{proof}[\pf{i:2.3.7}]
  By the trichotomy of order (\cref{i:2.2.13}), we have three cases: \(a < b\), \(a = b\), \(a > b\).
  Suppose first that \(a < b\), then by \cref{i:2.3.6}, we have \(ac < bc\), a contradiction.
  We can obtain a similar contradiction when \(a > b\).
  Thus the only possibility is that \(a = b\), as desired.
\end{proof}

\begin{rmk}\label{i:2.3.8}
  Just as \cref{i:2.2.6} will allow for a ``virtual subtraction'' which will eventually let us define genuine subtraction, \cref{i:2.3.7} provides a ``virtual division'' which will be needed to define genuine division later on.
\end{rmk}

\begin{prop}[Euclid's division lemma]\label{i:2.3.9}
  Let \(n\) be a natural number, and let \(q\) be a positive number.
  Then there exist natural numbers \(m, r\) such that \(0 \leq r < q\) and \(n = mq + r\).
\end{prop}

\begin{proof}[\pf{i:2.3.9}]
  We induct on \(n\) and fix \(q\).
  For \(n = 0\), let \(r = m = 0\).
  Then we have
  \begin{align*}
    mq + r & = 0q + 0                   \\
           & = 0 + 0  &  & \by{i:2.3.1} \\
           & = 0,     &  & \by{i:2.2.1}
  \end{align*}
  and
  \begin{align*}
    0 & \leq 0 = r &  & \by{i:2.2.12}[a] \\
      & < q.       &  & \by{i:2.2.11}
  \end{align*}
  So the base case holds.
  Suppose inductively that, for some natural number \(n\), there exist some natural numbers \(m, r\), such that \(n = mq + r\) and \(0 \leq r < q\).
  Then, for \(n\pp\), we have
  \begin{align*}
    n\pp & = (mq + r)\pp  &  & \byIH        \\
         & = mq + (r\pp). &  & \by{i:2.2.3} \\
  \end{align*}
  Since \(r < q\), by \cref{i:2.2.12}(e), we have \(r\pp \leq q\).
  Now we split into two cases:
  \begin{itemize}
    \item If \(r\pp < q\), then we have \(0 \leq r < r\pp < q\), and we are done in this case.
    \item If \(r\pp = q\), then by \cref{i:2.3.1}, we have
          \[
            n\pp = mq + (r\pp) = mq + q = (m\pp) \times q = (m\pp) \times q + r'
          \]
          where \(r' = 0\) and \(0 \leq r' < q\), by \cref{i:ac:2.2.4}, and we are also done in this case.
  \end{itemize}
  From all cases above we can find some natural numbers \(m, r\), such that \(n\pp = mq + r\) and \(0 \leq r < q\).
  This closes the induction.
\end{proof}

\begin{rmk}\label{i:2.3.10}
  In other words, we can divide a natural number \(n\) by a positive number \(q\) to obtain a quotient \(m\) (which is another natural number) and a remainder \(r\) (which is less than \(q\)).
  This algorithm marks the beginning of \emph{number theory}, which is a beautiful and important subject but one which is beyond the scope of this text.
\end{rmk}

\begin{defn}[Exponentiation for natural numbers]\label{i:2.3.11}
  Let \(m\) be a natural number.
  To raise \(m\) to the power \(0\), we define \(m^0 \coloneqq 1\); in particular, we define \(0^0 \coloneqq 1\).
  Now suppose recursively that \(m^n\) has been defined for some natural number \(n\), then we define \(m^{n\pp} \coloneqq m^n \times m\).
\end{defn}

\begin{ac}\label{i:ac:2.3.5}
  For any natural number \(n\), we have \(n^1 = n\).
\end{ac}

\begin{proof}[\pf{i:ac:2.3.5}]
  We have
  \begin{align*}
    n^1 & = n^0 \times n = 1 \times n &  & \by{i:2.3.11}   \\
        & = n.                        &  & \by{i:ac:2.3.4}
  \end{align*}
\end{proof}

\exercisesection

\begin{ex}\label{i:ex:2.3.1}
  Prove \cref{i:2.3.2}.
\end{ex}

\begin{proof}[\pf{i:ex:2.3.1}]
  See \cref{i:2.3.2}
\end{proof}

\begin{ex}\label{i:ex:2.3.2}
  Prove \cref{i:2.3.3}
\end{ex}

\begin{proof}[\pf{i:ex:2.3.2}]
  See \cref{i:2.3.3}
\end{proof}

\begin{ex}\label{i:ex:2.3.3}
  Prove \cref{i:2.3.5}
\end{ex}

\begin{proof}[\pf{i:ex:2.3.3}]
  See \cref{i:2.3.5}
\end{proof}

\begin{ex}\label{i:ex:2.3.4}
  Prove the identity \((a + b)^2 = a^2 + 2ab + b^2\) for all natural numbers \(a, b\).
\end{ex}

\begin{proof}[\pf{i:ex:2.3.4}]
  We have
  \begin{align*}
    (a + b)^2 & = (a + b)^1 \times (a + b)                         &  & \by{i:2.3.11}   \\
              & = (a + b) \times (a + b)                           &  & \by{i:ac:2.3.5} \\
              & = a(a + b) + b(a + b) = aa + ab + ba + bb          &  & \by{i:2.3.4}    \\
              & = a^1 \times a + ab + ba + b^1 \times b            &  & \by{i:ac:2.3.5} \\
              & = a^2 + ab + ba + b^2                              &  & \by{i:2.3.11}   \\
              & = a^2 + ab + ab + b^2                              &  & \by{i:2.3.2}    \\
              & = a^2 + 1 \times ab + 1 \times ab + b^2            &  & \by{i:ac:2.3.4} \\
              & = a^2 + (1 + 1) \times ab + b^2 = a^2 + 2ab + b^2. &  & \by{i:2.3.4}
  \end{align*}
\end{proof}

\begin{ex}\label{i:ex:2.3.5}
  Prove \cref{i:2.3.9}
\end{ex}

\begin{proof}[\pf{i:ex:2.3.5}]
  See \cref{i:2.3.9}
\end{proof}


\chapter{Set Theory}\label{ch:3}

\section{Fundamentals}\label{sec:3.1}

\begin{defn}\label{3.1.1}
  We define a \emph{set} \(A\) to be any unordered collection of objects.
  If \(x\) is an object, we say that \emph{\(x\) is an element of \(A\)} or \(x \in A\) if \(x\) lies in the collection;
  otherwise we say that \(x \notin A\).
\end{defn}

\begin{ax}[Sets are objects]\label{3.1}
  If \(A\) is a set, then \(A\) is also an object.
  In particular, given two sets \(A\) and \(B\), it is meaningful to ask whether \(A\) is also an element of \(B\).
\end{ax}

\setcounter{thm}{2}
\begin{rmk}\label{3.1.3}
  There is a special case of set theory, called ``pure set theory'', in which \emph{all} objects are sets;
  for instance the number \(0\) might be identified with the empty set \(\emptyset = \{\}\), the number \(1\) might be identified with \(\{0\} = \{\{\}\}\), the number \(2\) might be identified with \(\{0, 1\} = \{\{\}, \{\{\}\}\}\), and so forth.
  From a logical point of view, pure set theory is a simpler theory, since one only has to deal with sets and not with objects;
  however, from a conceptual point of view it is often easier to deal with impure set theories in which some objects are not considered to be sets.
  The two types of theories are more or less equivalent for the purposes of doing mathematics, and so we shall take an agnostic position as to whether all objects are sets or not.
\end{rmk}

\begin{defn}[Equality of sets]\label{3.1.4}
  Two sets \(A\) and \(B\) are \emph{equal}, \(A = B\), iff every element of \(A\) is an element of \(B\) and vice versa.
  To put it another way, \(A = B\) if and only if every element \(x\) of \(A\) belongs also to \(B\), and every element \(y\) of \(B\) belongs also to \(A\).
\end{defn}

\begin{ac}\label{ac:3.1.1}
  The definition of equality in \cref{3.1.4} is reflexive, symmetric and transitive.
\end{ac}

\begin{proof}
  We first prove that \cref{3.1.4} is reflexive.
  Let \(A\) be a set.
  Since
  \[
    \forall x : x \in A \implies x \in A,
  \]
  we have \cref{3.1.4} is reflexive, i.e., \(A = A\).

  Next we prove that \cref{3.1.4} is symmetric.
  Let \(A, B\) be sets and suppose \(A = B\).
  Since
  \[
    (\forall x : x \in A \iff x \in B) \iff (\forall x : x \in B \iff x \in A),
  \]
  we have \cref{3.1.4} is symmetric, i.e., \(A = B \iff B = A\).

  Finally, we prove that \cref{3.1.4} is transitive.
  Let \(A, B, C\) be sets and suppose \(A = B\) and \(B = C\).
  Since
  \[
    (\forall x : (x \in A \implies x \in B) \land (x \in B \implies x \in C)) \implies (\forall x : x \in A \implies x \in C),
  \]
  we have \cref{3.1.4} is transitive, i.e., \(A = B \land B = C \implies A = C\).
\end{proof}

\begin{note}
  Observe that if \(x \in A\) and \(A = B\), then \(x \in B\), by \cref{3.1.4}.
  Thus the ``is an element of'' relation \(\in\) obeys the axiom of substitution.
  Because of this, any new operation we define on sets will also obey the axiom of substitution, as long as we can define that operation purely in terms of the relation \(\in\).
\end{note}

\begin{note}
  Next, we turn to the issue of exactly which objects are sets and which objects are not.
  The situation is analogous to how we defined the natural numbers in the previous chapter;
  we started with a single natural number, \(0\), and started building more numbers out of \(0\) using the increment operation.
  We will try something similar here, starting with a single set, the \emph{empty set},
  and building more sets out of the empty set by various operations.
  We begin by postulating the existence of the empty set.
\end{note}

\begin{ax}[Empty set]\label{3.2}
  There exists a set \(\emptyset\), known as the empty set, which contains no elements, i.e., for every object \(x\) we have \(x \notin \emptyset\).
\end{ax}

\begin{note}
  The empty set is also denoted \(\{\}\).
\end{note}

\begin{ac}\label{ac:3.1.2}
  There can only be one empty set;
  if there were two sets \(\emptyset\) and \(\emptyset'\) which were both empty, then they would be equal to each other.
\end{ac}

\begin{proof}
  Suppose there exist two empty set \(\emptyset\) and \(\emptyset'\).
  Then we have
  \begin{align*}
         & (\forall x : (x \in \emptyset \implies x \in \emptyset') \land (x \in \emptyset' \implies x \in \emptyset)) &  & \text{(vacuously true)}  \\
    \iff & (\forall x : x \in \emptyset \iff x \in \emptyset')                                                                                       \\
    \iff & \emptyset = \emptyset'.                                                                                     &  & \text{(by \cref{3.1.4})}
  \end{align*}
\end{proof}

\begin{note}
  If a set is not equal to the empty set, we call it \emph{non-empty}.
\end{note}

\setcounter{thm}{5}
\begin{lem}[Single choice]\label{3.1.6}
  Let \(A\) be a non-empty set.
  Then there exists an object \(x\) such that \(x \in A\).
\end{lem}

\begin{proof}
  We prove by contradiction.
  Suppose there does not exist any object \(x\) such that \(x \in A\).
  Then for all objects \(x\), we have \(x \notin A\).
  Also, by \cref{3.2} we have \(x \notin \emptyset\).
  Thus \(x \in A \iff x \in \emptyset\) (both statements are equally false), and so \(A = \emptyset\) by \cref{3.1.4}, a contradiction.
\end{proof}

\begin{rmk}\label{3.1.7}
  The above Lemma asserts that given any non-empty set \(A\), we are allowed to ``choose'' an element \(x\) of \(A\) which demonstrates this non-emptyness.
  Later on (in \cref{3.5.12}) we will show that given any finite number of non-empty sets, say \(A_1, \dots, A_n\), it is possible to choose one element \(x_1, \dots, x_n\) from each set \(A_1, \dots, A_n\);
  this is known as ``finite choice''.
  However, in order to choose elements from an infinite number of sets, we need an additional axiom, the \emph{axiom of choice} (\cref{8.1}).
\end{rmk}

\begin{rmk}\label{3.1.8}
  Note that the empty set is \emph{not} the same thing as the natural number \(0\).
  One is a set;
  the other is a number.
  However, it is true that the \emph{cardinality} of the empty set is \(0\).
\end{rmk}

\begin{ax}[Singleton sets and pair sets]\label{3.3}
  If \(a\) is an object, then there exists a set \(\{a\}\) whose only element is \(a\), i.e., for every object \(y\), we have \(y \in \{a\}\) if and only if \(y = a\);
  we refer to \(\{a\}\) as the \emph{singleton set} whose element is \(a\).
  Furthermore, if \(a\) and \(b\) are objects, then there exists a set \(\{a, b\}\) whose only elements are \(a\) and \(b\);
  i.e., for every object \(y\), we have \(y \in \{a, b\}\) if and only if \(y = a\) or \(y = b\);
  we refer to this set as the \emph{pair set} formed by \(a\) and \(b\).
\end{ax}

\begin{rmk}\label{3.1.9}
  There is only one singleton set for each object \(a\).
  Similarly, given any two objects \(a\) and \(b\), there is only one pair set formed by \(a\) and \(b\).
  Thus the singleton set axiom is in fact redundant, being a consequence of the pair set axiom.
  Conversely, the pair set axiom will follow from the singleton set axiom and the pairwise union axiom (\cref{3.4}).
\end{rmk}

\begin{proof}
  We first show the uniqueness of singleton set.
  Suppose there exists two sets \(A\) and \(A'\) which are singleton sets of object \(a\).
  Then we have
  \begin{align*}
         & (\forall x : x \in A \iff x = a) \land (\forall x : x \in A' \iff x = a) &  & \text{(by \cref{3.3})}   \\
    \iff & \forall x : x \in A \iff x \in A'                                                                      \\
    \iff & A = A'.                                                                  &  & \text{(by \cref{3.1.4})}
  \end{align*}

  Next we show the uniqueness of pair set.
  Suppose there exists two sets \(X\) and \(X'\) which are pair sets of object \(a\) and \(b\).
  Then we have
  \begin{align*}
         & (\forall x : x \in X \iff (x = a) \lor (x = b))                                      \\
         & \land (\forall x : x \in X' \iff (x = a) \lor (x = b)) &  & \text{(by \cref{3.3})}   \\
    \iff & \forall x : x \in X \iff x \in X'                                                    \\
    \iff & X = X'.                                                &  & \text{(by \cref{3.1.4})}
  \end{align*}
\end{proof}

\begin{eg}\label{3.1.10}
  Since \(\emptyset\) is a set (and hence an object), so is the singleton set \(\{\emptyset\}\), i.e., the set whose only element is \(\emptyset\), is a set (and it is not the same set as \(\emptyset\), \(\{\emptyset\} \neq \emptyset\)).
  Similarly, the singleton set \(\{\{\emptyset\}\}\) and the pair set \(\{\emptyset, \{\emptyset\}\}\) are also sets.
  These three sets are not equal to each other.
\end{eg}

\begin{ax}[Pairwise union]\label{3.4}
  Given any two sets \(A\), \(B\), there exists a set \(A \cup B\), called the \emph{union} \(A \cup B\) of \(A\) and \(B\), whose elements consist of all the elements which belong to \(A\) or \(B\) or both.
  In other words, for any object \(x\),
  \[
    x \in A \cup B \iff (x \in A \lor x \in B).
  \]
\end{ax}

\setcounter{thm}{11}
\begin{rmk}\label{3.1.12}
  If \(A\), \(B\), \(A'\) are sets, and \(A\) is equal to \(A'\), then \(A \cup B\) is equal to \(A' \cup B\).
  Similarly if \(B'\) is a set which is equal to \(B\), then \(A \cup B\) is equal to \(A \cup B'\).
  Thus the operation of union obeys the axiom of substitution, and is thus well-defined on sets.
\end{rmk}

\begin{proof}
  Suppose \(A, A', B\) are sets such that \(A = A'\).
  Then we have
  \begin{align*}
         & \forall x : x \in A \cup B                               \\
    \iff & x \in A \lor x \in B       &  & \text{(by \cref{3.4})}   \\
    \iff & x \in A' \lor x \in B      &  & \text{(by \cref{3.1.4})} \\
    \iff & x \in A' \cup B.           &  & \text{(by \cref{3.4})}
  \end{align*}

  Similarly, suppose \(A, B, B'\) are sets such that \(B = B'\).
  Then we have
  \begin{align*}
         & \forall x : x \in A \cup B                               \\
    \iff & x \in A \lor x \in B       &  & \text{(by \cref{3.4})}   \\
    \iff & x \in A \lor x \in B'      &  & \text{(by \cref{3.1.4})} \\
    \iff & x \in A \cup B'.           &  & \text{(by \cref{3.4})}
  \end{align*}
\end{proof}

\begin{lem}\label{3.1.13}
  If \(a\) and \(b\) are objects, then \(\{a, b\} = \{a\} \cup \{b\}\).
  If \(A\), \(B\), \(C\) are sets, then the union operation is commutative (i.e., \(A \cup B = B \cup A\)) and associative (i.e., \((A \cup B) \cup C = A \cup (B \cup C)\)).
  Also, we have \(A \cup A = A \cup \emptyset = \emptyset \cup A = A\).
\end{lem}

\begin{proof}
  We first show that \(\{a, b\} = \{a\} \cup \{b\}\).
  By \cref{3.3}, the sets \(\{a\}, \{b\}, \{a, b\}\) exist.
  And by \cref{3.4}, the set \(\{a\} \cup \{b\}\) exists.
  Then we have
  \begin{align*}
         & (\forall x : x \in \{a, b\} \iff x = a \lor x = b)             &  & \text{(by \cref{3.3})}   \\
    \iff & (\forall x : x \in \{a, b\} \iff x \in \{a\} \lor x \in \{b\}) &  & \text{(by \cref{3.3})}   \\
    \iff & (\forall x : x \in \{a, b\} \iff x \in \{a\} \cup \{b\})       &  & \text{(by \cref{3.4})}   \\
    \iff & \{a, b\} = \{a\} \cup \{b\}.                                   &  & \text{(by \cref{3.1.4})}
  \end{align*}

  Next we show the commutative identity of union sets.
  Suppose that \(A, B\) are sets.
  By \cref{3.4}, the sets \(A \cup B\) and \(B \cup A\) exists.
  Then we have
  \begin{align*}
         & (\forall x : x \in A \cup B \iff x \in A \lor x \in B) &  & \text{(by \cref{3.4})}   \\
    \iff & (\forall x : x \in A \cup B \iff x \in B \lor x \in A)                               \\
    \iff & (\forall x : x \in A \cup B \iff x \in B \cup A)       &  & \text{(by \cref{3.4})}   \\
    \iff & A \cup B = B \cup A.                                   &  & \text{(by \cref{3.1.4})}
  \end{align*}

  Next we show the associativity identity of union sets.
  By \cref{3.1.4}, we need to show that every element \(x\) of \((A \cup B) \cup C\) is an element of \(A \cup (B \cup C)\), and vice versa.
  So suppose first that \(x\) is an element of \((A \cup B) \cup C\).
  By \cref{3.4}, this means that at least one of \(x \in A \cup B\) or \(x \in C\) is true.
  We now divide into two cases.
  If \(x \in C\), then by \cref{3.4} again \(x \in B \cup C\), and so by \cref{3.4} again we have \(x \in A \cup (B \cup C)\).
  Now suppose instead \(x \in A \cup B\), then by \cref{3.4} again \(x \in A\) or \(x \in B\).
  If \(x \in A\) then \(x \in A \cup (B \cup C)\) by \cref{3.4}, while if \(x \in B\) then by consecutive applications of \cref{3.4} we have \(x \in B \cup C\) and hence \(x \in A \cup (B \cup C)\).
  Thus in all cases we see that every element of \((A \cup B) \cup C\) lies in \(A \cup (B \cup C)\).
  A similar argument shows that every element of \(A \cup (B \cup C)\) lies in \((A \cup B) \cup C\), and so \((A \cup B) \cup C = A \cup (B \cup C) \) as desired.

  Finally we show that \(A \cup A = A \cup \emptyset = \emptyset \cup A = A\).
  Suppose that \(A\) is a set.
  Then we have
  \begin{align*}
         & (\forall x : x \in A \iff x \in A \lor x \in A)                                       \\
    \iff & (\forall x : x \in A \iff x \in A \cup A)               &  & \text{(by \cref{3.4})}   \\
    \iff & (A = A \cup A)                                          &  & \text{(by \cref{3.1.4})} \\
    \iff & (\forall x : x \in A \iff x \in A \lor x \in \emptyset) &  & \text{(vacuously true)}  \\
    \iff & A = A \cup \emptyset                                    &  & \text{(by \cref{3.1.4})} \\
    \iff & (\forall x : x \in A \iff x \in \emptyset \lor x \in A) &  & \text{(vacuously true)}  \\
    \iff & A = \emptyset \cup A.                                   &  & \text{(by \cref{3.1.4})}
  \end{align*}
\end{proof}

\begin{note}
  Because of \cref{3.1.13}, we do not need to use parentheses to denote multiple unions, thus for instance we can write \(A \cup B \cup C\) instead of \((A \cup B) \cup C\) or \(A \cup (B \cup C)\).
  Similarly for unions of four sets, \(A \cup B \cup C \cup D\), etc.
\end{note}

\begin{rmk}\label{3.1.14}
  While the operation of union has some similarities with addition, the two operations are \emph{not} identical.
\end{rmk}

\begin{note}
  \cref{3.4} allows us to define triplet sets, quadruplet sets, and so forth: if \(a, b, c\) are three objects, we define \(\{a, b, c\} \coloneqq \{a\} \cup \{b\} \cup \{c\}\);
  if \(a, b, c, d\) are four objects, then we define \(\{a, b, c, d\} \coloneqq \{a\} \cup \{b\} \cup \{c\} \cup \{d\}\), and so forth.
  On the other hand, we are not yet in a position to define sets consisting of \(n\) objects for any given natural number \(n\);
  this would require iterating the above construction ``\(n\) times'', but the concept of \(n\)-fold iteration has not yet been rigorously defined.
  For similar reasons, we cannot yet define sets consisting of infinitely many objects, because that would require iterating the axiom of pairwise union (\cref{3.4}) infinitely often, and it is not clear at this stage that one can do this rigorously.
  Later on, we will introduce other axioms of set theory which allow one to construct arbitrarily large, and even infinite, sets.
\end{note}

\begin{defn}[Subsets]\label{3.1.15}
  Let \(A\), \(B\) be sets.
  We say that \(A\) is a \emph{subset} of \(B\), denoted \(A \subseteq B\), iff every element of \(A\) is also an element of \(B\), i.e.
  \[
    \text{For any object } x, x \in A \implies x \in B.
  \]
  We say that \(A\) is a \emph{proper subset} of \(B\), denoted \(A \subsetneq B\), if \(A \subseteq B\) and \(A \neq B\).
\end{defn}

\begin{rmk}\label{3.1.16}
  Because these definitions involve only the notions of equality and the ``is an element of'' relation, both of which already obey the axiom of substitution, the notion of subset also automatically obeys the axiom of substitution.
  Thus for instance if \(A \subseteq B\) and \(A = A'\), then \(A' \subseteq B\).
\end{rmk}

\begin{eg}\label{3.1.17}
  Given any set \(A\), we always have \(A \subseteq A\) and \(\emptyset \subseteq A\).
\end{eg}

\begin{proof}
  Suppose that \(A\) is a set.
  Then we have
  \begin{align*}
    \top \iff & (\forall x : x \in A \implies x \in A)                                \\
    \iff      & A \subseteq A.                         &  & \text{(by \cref{3.1.15})}
  \end{align*}
  And we also have
  \begin{align*}
         & (\forall x : x \in \emptyset \implies x \in A) &  & \text{(vacuously true)}   \\
    \iff & \emptyset \subseteq A.                         &  & \text{(by \cref{3.1.15})}
  \end{align*}
\end{proof}

\begin{prop}[Sets are partially ordered by set inclusion]\label{3.1.18}
  Let \(A\), \(B\), \(C\) be sets.
  If \(A \subseteq B\) and \(B \subseteq C\) then \(A \subseteq C\).
  \(A \subseteq B\) and \(B \subseteq A\) if and only if \(A = B\).
  Finally, if \(A \subsetneq B\) and \(B \subsetneq C\) then \(A \subsetneq C\).
\end{prop}

\begin{proof}
  We first show that \(A \subseteq B \land B \subseteq C \implies A \subseteq C\).
  Suppose that \(A \subseteq B\) and \(B \subseteq C\).
  To prove that \(A \subseteq C\), we have to prove that every element of \(A\) is an element of \(C\).
  So, let us pick an arbitrary element \(x\) of \(A\).
  Then, since \(A \subseteq B\), \(x\) must then be an element of \(B\).
  But then since \(B \subseteq C\), \(x\) is an element of \(C\).
  Thus every element of \(A\) is indeed an element of \(C\), as claimed.

  Next we show that \(A \subseteq B \land B \subseteq A \iff A = B\).
  Suppose that \(A, B\) are sets and \(A \subseteq B \land B \subseteq A\).
  Then we have
  \begin{align*}
         & A \subseteq B \land B \subseteq A                                                                        \\
    \iff & (\forall x : (x \in A \implies x \in B) \land (x \in B \implies x \in A)) &  & \text{(by \cref{3.1.15})} \\
    \iff & (\forall x : x \in A \iff x \in B)                                                                       \\
    \iff & A = B.                                                                    &  & \text{(by \cref{3.1.4})}
  \end{align*}

  Finally we show that \(A \subsetneq B \land B \subsetneq C \implies A \subsetneq C\).
  Suppose that \(A, B, C\) are sets and \(A \subsetneq B \land B \subsetneq C\).
  Then we have
  \begin{align*}
             & A \subsetneq B \land B \subsetneq C                                                                                                          \\
    \implies & (\forall x : x \in A \implies x \in B) \land (A \neq B)                                                                                      \\
             & \land (\forall x : x \in B \implies x \in C) \land (B \neq C)                   &  & \text{(by \cref{3.1.15})}                               \\
    \implies & (\forall x : x \in A \implies x \in B)                                                                                                       \\
             & \land \lnot(\forall x : x \in A \iff x \in B)                                                                                                \\
             & \land (\forall x : x \in B \implies x \in C)                                                                                                 \\
             & \land (B \neq C)                                                                                                                             \\
    \implies & (\forall x : x \in A \implies x \in B)                                                                                                       \\
             & \land (\exists\ x : (x \in A \land x \notin B) \lor (x \in B \land x \notin A))                                                              \\
             & \land (\forall x : x \in B \implies x \in C)                                                                                                 \\
             & \land (B \neq C)                                                                                                                             \\
    \implies & (\forall x : x \in A \implies x \in B)                                                                                                       \\
             & \land (\exists\ x : x \in B \land x \notin A)                                   &  & \text{(since \(\forall x : x \in A \implies x \in B\))} \\
             & \land (\forall x : x \in B \implies x \in C)                                                                                                 \\
             & \land (B \neq C)                                                                                                                             \\
    \implies & (\forall x : x \in A \implies x \in B)                                                                                                       \\
             & \land (\exists\ x : x \in C \land x \notin A)                                   &  & \text{(since \(\forall x : x \in B \implies x \in C\))} \\
             & \land (\forall x : x \in B \implies x \in C)                                                                                                 \\
             & \land (B \neq C)                                                                                                                             \\
    \implies & (\forall x : x \in A \implies x \in C)                                                                                                       \\
             & \land (A \neq C) \land (B \neq C)                                                                                                            \\
    \implies & A \subseteq C \land A \neq C \land B \neq C                                     &  & \text{(by \cref{3.1.15})}                               \\
    \implies & A \subsetneq C \land B \neq C                                                   &  & \text{(by \cref{3.1.15})}                               \\
    \implies & A \subsetneq C.
  \end{align*}
\end{proof}

\setcounter{thm}{19}
\begin{rmk}\label{3.1.20}
  There is one important difference between the subset relation \(\subsetneq\) and the less than relation \(<\).
  Given any two distinct natural numbers \(n\), \(m\), we know that one of them is smaller than the other (\cref{2.2.13});
  however, given two distinct sets, it is not in general true that one of them is a subset of the other.
  we say that sets are only \emph{partially ordered}, whereas the natural numbers are \emph{totally ordered}.
\end{rmk}

\begin{rmk}\label{3.1.21}
  We should also caution that the subset relation \(\subseteq\) is not the same as the element relation \(\in\).
  It is important to distinguish sets from their elements, as they can have different properties.
  For instance, it is possible to have an infinite set consisting of finite numbers (the set \(\N\) of natural numbers is one such example), and it is also possible to have a finite set consisting of infinite objects
  (consider for instance the finite set \(\{\N, \Z, \Q, \R\}\), which has four elements, all of which are infinite).
\end{rmk}

\begin{ax}[Axiom of specification]\label{3.5}
  Let \(A\) be a set, and for each \(x \in A\), let \(P(x)\) be a property pertaining to \(x\) (i.e., \(P(x)\) is either a true statement or a false statement).
  Then there exists a set, called \(\{x \in A : P(x) \text{ is true}\}\) (or simply \(\{x \in A : P(x)\}\) for short), whose elements are precisely the elements \(x\) in \(A\) for which \(P(x)\) is true.
  In other words, for any object \(y\),
  \[
    y \in \{x \in A : P(x) \text{ is true}\} \iff (y \in A \text{ and } P(y) \text{ is true}).
  \]
\end{ax}

\begin{note}
  \cref{3.5} is also known as the \emph{axiom of separation}.
  We sometimes write \(\{x \in A \mid P(x)\}\) instead of \(\{x \in A : P(x)\}\);
  this is useful when we are using the colon ``:'' to denote something else.
\end{note}

\setcounter{thm}{22}
\begin{defn}[Intersections]\label{3.1.23}
  The intersection \(S_1 \cap S_2\) of two sets is defined to be the set
  \[
    S_1 \cap S_2 \coloneqq \{x \in S_1 : x \in S_2\}.
  \]
  In other words, \(S_1 \cap S_2\) consists of all the elements which belong to both \(S_1\) and \(S_2\).
  Thus, for all objects \(x\),
  \[
    x \in S_1 \cap S_2 \iff x \in S_1 \text{ and } x \in S_2.
  \]
\end{defn}

\begin{note}
  Two sets \(A\), \(B\) are said to be \emph{disjoint} if \(A \cap B = \emptyset\).
  This is not the same concept as being \emph{distinct}, \(A \neq B\).
  Meanwhile, the sets \(\emptyset\) and \(\emptyset\) are disjoint but not distinct.
\end{note}

\setcounter{thm}{26}
\begin{defn}[Difference sets]\label{3.1.27}
  Given two sets \(A\) and \(B\), we define the set \(A - B\) or \(A \setminus B\) to be the set \(A\) with any elements of \(B\) removed:
  \[
    A \setminus B \coloneqq \{x \in A : x \notin B\}.
  \]
\end{defn}

\begin{prop}[Sets form a boolean algebra]\label{3.1.28}
  Let \(A\), \(B\), \(C\) be sets, and let \(X\) be a set containing \(A\), \(B\), \(C\) as subsets.
  \begin{enumerate}
    \item (Minimal element) We have \(A \cup \emptyset = A\) and \(A \cap \emptyset = \emptyset\).
    \item (Maximal element) We have \(A \cup X = X\) and \(A \cap X = A\).
    \item (Identity) We have \(A \cap A = A\) and \(A \cup A = A\).
    \item (Commutativity) We have \(A \cup B = B \cup A\) and \(A \cap B = B \cap A\).
    \item (Associativity) We have \((A \cup B) \cup C = A \cup (B \cup C)\) and \((A \cap B) \cap C = A \cap (B \cap C)\).
    \item (Distributivity) We have \(A \cap (B \cup C) = (A \cap B) \cup (A \cap C)\) and \(A \cup (B \cap C) = (A \cup B) \cap (A \cup C)\).
    \item (Partition) We have \(A \cup (X \setminus A) = X\) and \(A \cap (X \setminus A) = \emptyset\).
    \item (De Morgan laws) We have \(X \setminus (A \cup B) = (X \setminus A) \cap (X \setminus B)\) and \(X \setminus (A \cap B) = (X \setminus A) \cup (X \setminus B)\).
  \end{enumerate}
\end{prop}

\begin{proof}{(a)}
  Suppose that \(A\) is a set.
  By \cref{3.1.13} we have \(A \cup \emptyset = A\).
  We only need to show that \(A \cap \emptyset = \emptyset\).
  \begin{align*}
    \top \iff & (\bot \iff \bot)                                                 &  & \text{(vacuously true)}   \\
    \iff      & (\forall x : x \in \emptyset \iff x \in \emptyset)               &  & \text{(vacuously true)}   \\
    \iff      & (\forall x : x \in \emptyset \iff x \in A \land x \in \emptyset)                                \\
    \iff      & (\forall x : x \in \emptyset \iff x \in A \cap \emptyset)        &  & \text{(by \cref{3.1.23})} \\
    \iff      & \emptyset = A \cap \emptyset.                                    &  & \text{(by \cref{3.1.4})}
  \end{align*}
\end{proof}

\begin{proof}{(b)}
  Suppose that \(A, X\) are sets and \(A \subseteq X\).
  Then we have
  \begin{align*}
    \top \iff                     & (\forall x : x \in X \implies x \in A \lor x \in X)                                \\
    \iff                          & (\forall x : x \in X \implies x \in A \cup X).      &  & \text{(by \cref{3.4})}    \\
    A \subseteq X \iff            & (\forall x : x \in A \implies x \in X)              &  & \text{(by \cref{3.1.15})} \\
    \iff                          & (\forall x : x \in A \lor x \in X \implies x \in X)                                \\
    \iff                          & (\forall x : x \in A \cup X \implies x \in X).      &  & \text{(by \cref{3.4})}    \\
    \top \land A \subseteq X \iff & (\forall x : x \in X \iff x \in A \cup X)                                          \\
    \iff                          & X = A \cup X.                                       &  & \text{(by \cref{3.1.4})}
  \end{align*}
  And we also have
  \begin{align*}
    \top \iff                     & (\forall x : x \in A \land x \in X \implies x \in A)                                \\
    \iff                          & (\forall x : x \in A \cap X \implies x \in A).       &  & \text{(by \cref{3.1.23})} \\
    A \subseteq X \iff            & (\forall x : x \in A \implies x \in X)               &  & \text{(by \cref{3.1.15})} \\
    \iff                          & (\forall x : x \in A \implies x \in A \land x \in X)                                \\
    \iff                          & (\forall x : x \in A \implies x \in A \cap X).       &  & \text{(by \cref{3.1.23})} \\
    \top \land A \subseteq X \iff & (\forall x : x \in A \cap X \iff x \in A)                                           \\
    \iff                          & A \cap X = A.                                        &  & \text{(by \cref{3.1.4})}
  \end{align*}
\end{proof}

\begin{proof}{(c)}
  Suppose that \(A\) is a set.
  By \cref{3.1.13} we have \(A \cup A = A\).
  We only need to show that \(A \cap A = A\).
  \begin{align*}
    \top \iff & (\forall x : x \in A \iff x \in A)                                              \\
    \iff      & (\forall x : x \in A \land x \in A \iff x \in A)                                \\
    \iff      & (\forall x : x \in A \cap A \iff x \in A)        &  & \text{(by \cref{3.1.23})} \\
    \iff      & A \cap A = A.                                    &  & \text{(by \cref{3.1.4})}
  \end{align*}
\end{proof}

\begin{proof}{(d)}
  Suppose that \(A, B\) are sets.
  Then we have
  \begin{align*}
         & (\forall x : x \in A \cup B \iff x \in A \lor x \in B) &  & \text{(by \cref{3.4})}   \\
    \iff & (\forall x : x \in A \cup B \iff x \in B \lor x \in A)                               \\
    \iff & (\forall x : x \in A \cup B \iff x \in B \cup A)       &  & \text{(by \cref{3.4})}   \\
    \iff & A \cup B = B \cup A.                                   &  & \text{(by \cref{3.1.4})}
  \end{align*}
  And we also have
  \begin{align*}
         & (\forall x : x \in A \cap B \iff x \in A \land x \in B) &  & \text{(by \cref{3.1.23})} \\
    \iff & (\forall x : x \in A \cap B \iff x \in B \land x \in A)                                \\
    \iff & (\forall x : x \in A \cap B \iff x \in B \cap A)        &  & \text{(by \cref{3.1.23})} \\
    \iff & A \cap B = B \cap A.                                    &  & \text{(by \cref{3.1.4})}
  \end{align*}
\end{proof}

\begin{proof}{(e)}
  Suppose that \(A, B, C\) are sets.
  Then we have
  \begin{align*}
         & (\forall x : x \in (A \cup B) \cup C \iff (x \in A \lor x \in B) \lor x \in C) &  & \text{(by \cref{3.4})}   \\
    \iff & (\forall x : x \in (A \cup B) \cup C \iff x \in A \lor (x \in B \lor x \in C))                               \\
    \iff & (\forall x : x \in (A \cup B) \cup C \iff x \in A \cup (B \cup C))             &  & \text{(by \cref{3.4})}   \\
    \iff & (A \cup B) \cup C = A \cup (B \cup C).                                         &  & \text{(by \cref{3.1.4})}
  \end{align*}
  And we also have
  \begin{align*}
         & (\forall x : x \in (A \cap B) \cap C \iff (x \in A \land x \in B) \land x \in C) &  & \text{(by \cref{3.1.23})} \\
    \iff & (\forall x : x \in (A \cap B) \cap C \iff x \in A \land (x \in B \land x \in C))                                \\
    \iff & (\forall x : x \in (A \cap B) \cap C \iff x \in A \cap (B \cap C))               &  & \text{(by \cref{3.1.23})} \\
    \iff & (A \cap B) \cap C = A \cap (B \cap C).                                           &  & \text{(by \cref{3.1.4})}
  \end{align*}
\end{proof}

\begin{proof}{(f)}
  Suppose that \(A, B, C\) are sets.
  Then we have
  \begin{align*}
         & \forall x : x \in A \cap (B \cup C)                                                 \\
    \iff & x \in A \land x \in B \cup C                         &  & \text{(by \cref{3.1.23})} \\
    \iff & x \in A \land (x \in B \lor x \in C)                 &  & \text{(by \cref{3.4})}    \\
    \iff & (x \in A \land x \in B) \lor (x \in A \land x \in C)                                \\
    \iff & (x \in A \cap B) \lor (x \in A \cap C)               &  & \text{(by \cref{3.1.23})} \\
    \iff & x \in (A \cap B) \cup (A \cap C).                    &  & \text{(by \cref{3.4})}
  \end{align*}
  Thus by \cref{3.1.4} we have \(A \cap (B \cup C) = (A \cap B) \cup (A \cap C)\).
  Similarly we have
  \begin{align*}
         & \forall x : x \in A \cup (B \cap C)                                                \\
    \iff & x \in A \lor x \in B \cap C                         &  & \text{(by \cref{3.4})}    \\
    \iff & x \in A \lor (x \in B \land x \in C)                &  & \text{(by \cref{3.1.23})} \\
    \iff & (x \in A \lor x \in B) \land (x \in A \lor x \in C)                                \\
    \iff & (x \in A \cup B) \land (x \in A \cup C)             &  & \text{(by \cref{3.4})}    \\
    \iff & x \in (A \cup B) \cap (A \cup C).                   &  & \text{(by \cref{3.1.23})}
  \end{align*}
  Thus by \cref{3.1.4} we have \(A \cup (B \cap C) = (A \cup B) \cap (A \cup C)\).
\end{proof}

\begin{proof}{(g)}
  Suppose that \(A, X\) are sets and \(A \subseteq X\).
  Then we have
  \begin{align*}
         & \forall x : x \in A \cup (X \setminus A)                                                 \\
    \iff & x \in A \lor x \in (X \setminus A)                     &  & \text{(by \cref{3.4})}       \\
    \iff & x \in A \lor (x \in X \land x \notin A)                &  & \text{(by \cref{3.1.27})}    \\
    \iff & (x \in A \lor x \in X) \land (x \in A \lor x \notin A)                                   \\
    \iff & (x \in A \lor x \in X) \land \top                                                        \\
    \iff & x \in A \lor x \in X                                                                     \\
    \iff & x \in A \cup X                                         &  & \text{(by \cref{3.4})}       \\
    \iff & x \in X.                                               &  & \text{(by \cref{3.1.28})(b)}
  \end{align*}
  Thus by \cref{3.1.4} we have \(A \cup (X \setminus A) = X\).
  Similarly we have
  \begin{align*}
         & \forall x : x \in A \cap (X \setminus A)                                \\
    \iff & x \in A \land x \in (X \setminus A)      &  & \text{(by \cref{3.1.23})} \\
    \iff & x \in A \land (x \in X \land x \notin A) &  & \text{(by \cref{3.1.27})} \\
    \iff & (x \in A \land x \notin A) \land x \in X                                \\
    \iff & \bot \land x \in X                                                      \\
    \iff & \bot                                                                    \\
    \iff & x \in \emptyset.                         &  & \text{(vacuously true)}
  \end{align*}
  Thus by \cref{3.1.4} we have \(A \cap (X \setminus A) = \emptyset\).
\end{proof}

\begin{proof}{(h)}
  Suppose that \(A, B, X\) are sets such that \(A \subseteq X\) and \(B \subseteq X\).
  Then we have
  \begin{align*}
         & \forall x : x \in X \setminus (A \cup B)                                                   \\
    \iff & x \in X \land x \notin (A \cup B)                           &  & \text{(by \cref{3.1.27})} \\
    \iff & x \in X \land \lnot (x \in A \cup B)                                                       \\
    \iff & x \in X \land \lnot (x \in A \lor x \in B)                  &  & \text{(by \cref{3.4})}    \\
    \iff & x \in X \land (x \notin A \land x \notin B)                                                \\
    \iff & (x \in X \land x \notin A) \land (x \in X \land x \notin B)                                \\
    \iff & (x \in X \setminus A) \land (x \in X \setminus B)           &  & \text{(by \cref{3.1.27})} \\
    \iff & x \in (X \setminus A) \cap (X \setminus B).                 &  & \text{(by \cref{3.1.23})}
  \end{align*}
  Thus by \cref{3.1.4} we have \(X \setminus (A \cup B) = (X \setminus A) \cap (X \setminus B)\).
  Similarly we have
  \begin{align*}
         & \forall x : x \in X \setminus (A \cap B)                                                  \\
    \iff & x \in X \land x \notin (A \cap B)                          &  & \text{(by \cref{3.1.27})} \\
    \iff & x \in X \land \lnot (x \in A \cap B)                                                      \\
    \iff & x \in X \land \lnot (x \in A \land x \in B)                &  & \text{(by \cref{3.1.23})} \\
    \iff & x \in X \land (x \notin A \lor x \notin B)                                                \\
    \iff & (x \in X \land x \notin A) \lor (x \in X \land x \notin B)                                \\
    \iff & (x \in X \setminus A) \lor (x \in X \setminus B)           &  & \text{(by \cref{3.1.27})} \\
    \iff & x \in (X \setminus A) \cup (X \setminus B).                &  & \text{(by \cref{3.4})}
  \end{align*}
  Thus by \cref{3.1.4} we have \(X \setminus (A \cap B) = (X \setminus A) \cup (X \setminus B)\).
\end{proof}

\begin{rmk}\label{3.1.29}
  The de Morgan laws are named after the logician Augustus De Morgan (1806 -- 1871), who identified them as one of the basic laws of set theory.
\end{rmk}

\begin{rmk}\label{3.1.30}
  The reader may observe a certain symmetry in the above laws between \(\cup\) and \(\cap\), and between \(X\) and \(\emptyset\).
  This is an example of \emph{duality} - two distinct properties or objects being dual to each other.
  In this case, the duality is manifested by the complementation relation \(A \mapsto X \setminus A\);
  the de Morgan laws assert that this relation converts unions into intersections and vice versa.
  (It also interchanges \(X\) and the empty set.)
  \cref{3.1.28} are collectively known as the \emph{laws of Boolean algebra}, after the mathematician George Boole (1815 -- 1864), and are also applicable to a number of other objects other than sets;
  it plays a particularly important role in logic.
\end{rmk}

\begin{ax}[Replacement]\label{3.6}
  Let \(A\) be a set.
  For any object \(x \in A\), and any object \(y\), suppose we have a statement \(P(x, y)\) pertaining to \(x\) and \(y\), such that for each \(x \in A\) there is at most one \(y\) for which \(P(x, y)\) is true.
  Then there exists a set \(\{y : P(x, y) \text{ is true for some } x \in A\}\), such that for any object \(z\),
  \[
    z \in \{y: P(x, y) \text{ is true for some } x \in A\} \iff P(x, y) \text{ is true for some } x \in A.
  \]
\end{ax}

\begin{note}
  The keyword here is ``suppose'';
  We have to assume that there exists a set \(E = \{x \in A : \exists!\ y \text{ such that } P(x, y) \text{ is true}\}\).
  \(E\) must exist first so we can apply \cref{3.6}.
  This means we assert the existence of a function \(f : E \to \{y : P(x, y) \text{ is true for some } x\}\).
\end{note}

\begin{note}
  We often abbreviate a set of the form
  \[
    \{y : y = f(x) \text{ for some } x \in A\}
  \]
  as \(\{f(x) : x \in A\}\) or \(\{f(x) \mid x \in A\}\).
  We can of course combine the axiom of replacement with the axiom of specification, thus for instance we can create sets such as \(\{f(x) : x \in A; P(x) \text{ is true}\}\) by starting with the set \(A\), using the axiom of specification to create the set \(\{x \in A : P(x) \text{ is true}\}\), and then applying the axiom of replacement to create \(\{f(x) : x \in A; P(x) \text{ is true}\}\).
\end{note}

\begin{ax}[Infinity]\label{3.7}
  There exists a set \(\N\), whose elements are called natural numbers, as well as an object \(0\) in \(\N\), and an object \(n++\) assigned to every natural number \(n \in \N\), such that the Peano axioms (\crefrange{2.1}{2.5}) hold.
\end{ax}

\exercisesection

\begin{ex}\label{ex:3.1.1}
  Show that the definition of equality in \cref{3.1.4} is reflexive, symmetric, and transitive.
\end{ex}

\begin{proof}
  See \cref{ac:3.1.1}.
\end{proof}

\begin{ex}\label{ex:3.1.2}
  Using only \cref{3.1.4}, \cref{3.1}, \cref{3.2}, and \cref{3.3}, prove that the sets \(\emptyset\), \(\{\emptyset\}\), \(\{\{\emptyset\}\}\), and \(\{\emptyset, \{\emptyset\}\}\) are all distinct
  (i.e., no two of them are equal to each other).
\end{ex}

\begin{proof}
  We first show that \(\emptyset \neq \{\emptyset\}\), \(\emptyset \neq \{\{\emptyset\}\}\) and \(\emptyset \neq \{\emptyset, \{\emptyset\}\}\).
  \begin{align*}
             & \emptyset \in \{\emptyset\} \land \emptyset \notin \emptyset                &  & \text{(by \cref{3.2})}   \\
    \implies & \{\emptyset\} \neq \emptyset.                                               &  & \text{(by \cref{3.1.4})} \\
             & \{\emptyset\} \in \{\{\emptyset\}\} \land \{\emptyset\} \notin \emptyset    &  & \text{(by \cref{3.2})}   \\
    \implies & \{\{\emptyset\}\} \neq \emptyset.                                           &  & \text{(by \cref{3.1.4})} \\
             & \emptyset \in \{\emptyset, \{\emptyset\}\} \land \emptyset \notin \emptyset &  & \text{(by \cref{3.2})}   \\
    \implies & \{\emptyset\} \neq \emptyset.                                               &  & \text{(by \cref{3.1.4})}
  \end{align*}

  Next we show that \(\{\emptyset\} \neq \{\{\emptyset\}\}\) and \(\{\emptyset\} \neq \{\emptyset, \{\emptyset\}\}\).
  \begin{align*}
             & \{\emptyset\} \in \{\{\emptyset\}\} \land \{\emptyset\} \notin \{\emptyset\}            &  & \text{(by \cref{3.3})}   \\
    \implies & \{\emptyset\} \neq \{\{\emptyset\}\}.                                                   &  & \text{(by \cref{3.1.4})} \\
             & \{\emptyset\} \in \{\emptyset, \{\emptyset\}\} \land \{\emptyset\} \notin \{\emptyset\} &  & \text{(by \cref{3.2})}   \\
    \implies & \{\emptyset\} \neq \{\{\emptyset\}\}.                                                   &  & \text{(by \cref{3.1.4})}
  \end{align*}

  Finally we show that \(\{\{\emptyset\}\} \neq \{\emptyset, \{\emptyset\}\}\).
  \begin{align*}
             & \emptyset \in \{\emptyset, \{\emptyset\}\} \land \emptyset \notin \{\{\emptyset\}\} &  & \text{(by \cref{3.3})}   \\
    \implies & \{\{\emptyset\}\} \neq \{\emptyset, \{\emptyset\}\}.                                &  & \text{(by \cref{3.1.4})}
  \end{align*}
\end{proof}

\begin{ex}\label{ex:3.1.3}
  Prove the remaining claims in \cref{3.1.13}.
\end{ex}

\begin{proof}
  See \cref{3.1.13}.
\end{proof}

\begin{ex}\label{ex:3.1.4}
  Prove the remaining claims in \cref{3.1.18}.
\end{ex}

\begin{proof}
  See \cref{3.1.18}.
\end{proof}

\begin{ex}\label{ex:3.1.5}
  Let \(A\), \(B\) be sets.
  Show that the three statements \(A \subseteq B\), \(A \cup B = B\), \(A \cap B = A\) are logically equivalent (any one of them implies the other two).
\end{ex}

\begin{proof}
  We first show that \(A \subseteq B \iff A \cup B = B\).
  Suppose that \(A, B\) are sets.
  Then we have
  \begin{align*}
    A \subseteq B \implies & A \cup B = B.                                   &  & \text{(by \cref{3.1.28}(b))} \\
    A \cup B = B \implies  & (\forall x : x \in A \cup B \iff x \in B)       &  & \text{(by \cref{3.1.4})}     \\
    \implies               & (\forall x : x \in A \lor x \in B \iff x \in B) &  & \text{(by \cref{3.4})}       \\
    \implies               & (\forall x : x \in A \implies x \in B)                                            \\
    \implies               & A \subseteq B.                                  &  & \text{(by \cref{3.1.15})}    \\
    A \cup B = B \iff      & A \subseteq B.
  \end{align*}

  Now we show that \(A \subseteq B \iff A \cap B = A\).
  Suppose that \(A, B\) are sets.
  Then we have
  \begin{align*}
    A \subseteq B \implies & A \cap B = A.                                    &  & \text{(by \cref{3.1.28}(b))} \\
    A \cap B = A \implies  & (\forall x : x \in A \cap B \iff x \in A)        &  & \text{(by \cref{3.1.4})}     \\
    \implies               & (\forall x : x \in A \land x \in B \iff x \in A) &  & \text{(by \cref{3.1.23})}    \\
    \implies               & (\forall x : x \in A \implies x \in B)                                             \\
    \implies               & A \subseteq B.                                   &  & \text{(by \cref{3.1.15})}    \\
    A \cap B = A \iff      & A \subseteq B.
  \end{align*}
\end{proof}

\begin{ex}\label{ex:3.1.6}
  Prove \cref{3.1.28}.
\end{ex}

\begin{proof}
  See \cref{3.1.28}.
\end{proof}

\begin{ex}\label{ex:3.1.7}
  Let \(A\), \(B\), \(C\) be sets.
  Show that \(A \cap B \subseteq A\) and \(A \cap B \subseteq B\).
  Furthermore, show that \(C \subseteq A\) and \(C \subseteq B\) if and only if \(C \subseteq A \cap B\).
  In a similar spirit, show that \(A \subseteq A \cup B\) and \(B \subseteq A \cup B\), and furthermore that \(A \subseteq C\) and \(B \subseteq C\) if and only if \(A \cup B \subseteq C\).
\end{ex}

\begin{proof}
  We first show that \(A \cap B \subseteq A\) and \(A \cap B \subseteq B\).
  Suppose that \(A, B\) are sets.
  Then we have
  \begin{align*}
             & (\forall x : x \in A \cap B \iff x \in A \land x \in B) &  & \text{(by \cref{3.1.23})} \\
    \implies & ((\forall x : x \in A \cap B \implies x \in A)                                         \\
             & \land (\forall x : x \in A \cap B \implies x \in B))                                   \\
    \implies & (A \cap B \subseteq A) \land (A \cap B \subseteq B).    &  & \text{(by \cref{3.1.15})}
  \end{align*}

  Next we show that \(C \subseteq A \land C \subseteq B \iff C \subseteq A \cap B\).
  Suppose that \(A, B, C\) are sets.
  Then we have
  \begin{align*}
         & (C \subseteq A \land C \subseteq B)                                                                                \\
    \iff & (\forall x : x \in C \implies x \in A) \land (\forall x : x \in C \implies x \in B) &  & \text{(by \cref{3.1.15})} \\
    \iff & (\forall x : x \in C \implies x \in A \land x \in B)                                                               \\
    \iff & (\forall x : x \in C \implies x \in A \cap B)                                       &  & \text{(by \cref{3.1.23})} \\
    \iff & (C \subseteq A \cap B).                                                             &  & \text{(by \cref{3.1.15})}
  \end{align*}

  Next we show that \(A \subseteq A \cup B\) and \(B \subseteq A \cup B\).
  Suppose that \(A, B\) are sets.
  Then we have
  \begin{align*}
    \top \iff & (\forall x : x \in A \implies \forall x : x \in A \lor x \in B)                                \\
    \iff      & (\forall x : x \in A \implies \forall x : x \in A \cup B)       &  & \text{(by \cref{3.4})}    \\
    \iff      & (A \subseteq A \cup B).                                         &  & \text{(by \cref{3.1.15})} \\
    \top \iff & (\forall x : x \in B \implies \forall x : x \in A \lor x \in B)                                \\
    \iff      & (\forall x : x \in B \implies \forall x : x \in A \cup B)       &  & \text{(by \cref{3.4})}    \\
    \iff      & (B \subseteq A \cup B).                                         &  & \text{(by \cref{3.1.15})}
  \end{align*}

  Finally we show that \(A \subseteq C \land B \subseteq C \iff A \cup B \subseteq C\).
  Suppose that \(A, B, C\) are sets.
  Then we have
  \begin{align*}
         & (A \subseteq C \land B \subseteq C)                                                                                \\
    \iff & (\forall x : x \in A \implies x \in C) \land (\forall x : x \in B \implies x \in C) &  & \text{(by \cref{3.1.15})} \\
    \iff & (\forall x : x \in A \lor x \in B \implies x \in C)                                                                \\
    \iff & (\forall x : x \in A \cup B \implies x \in C)                                       &  & \text{(by \cref{3.4})}    \\
    \iff & (A \cup B \subseteq C).                                                             &  & \text{(by \cref{3.1.15})}
  \end{align*}
\end{proof}

\begin{ex}\label{ex:3.1.8}
  Let \(A\), \(B\) be sets.
  Prove the \emph{absorption laws} \(A \cap (A \cup B) = A\) and \(A \cup (A \cap B) = A\).
\end{ex}

\begin{proof}
  Suppose that \(A, B\) are sets.
  Then we have
  \begin{align*}
         & \forall x : x \in A \cap (A \cup B)                                 \\
    \iff & x \in A \land x \in A \cup B         &  & \text{(by \cref{3.1.23})} \\
    \iff & x \in A \land (x \in A \lor x \in B) &  & \text{(by \cref{3.4})}    \\
    \iff & x \in A.
  \end{align*}
  Thus by \cref{3.1.4} we have \(A \cap (A \cup B) = A\).
  Similarly we have
  \begin{align*}
         & \forall x : x \in A \cup (A \cap B)                                 \\
    \iff & x \in A \lor x \in A \cap B          &  & \text{(by \cref{3.4})}    \\
    \iff & x \in A \lor (x \in A \land x \in B) &  & \text{(by \cref{3.1.23})} \\
    \iff & x \in A.
  \end{align*}
  Thus by \cref{3.1.4} we have \(A \cup (A \cap B) = A\).
\end{proof}

\begin{ex}\label{ex:3.1.9}
  Let \(A\), \(B\), \(X\) be sets such that \(A \cup B = X\) and \(A \cap B = \emptyset\).
  Show that \(A = X \setminus B\) and \(B = X \setminus A\).
\end{ex}

\begin{proof}
  Suppose that \(A, B, X\) are sets such that \(A \cup B = X\) and \(A \cap B = \emptyset\).
  We first show that \(A \cap B = \emptyset \iff \forall x : (x \in A \implies x \notin B) \land (x \in B \implies x \notin A)\).
  \begin{align*}
         & (A \cap B = \emptyset)                                                                                         \\
    \iff & (\forall x : x \notin A \cap B)                                                 &  & \text{(by \cref{3.2})}    \\
    \iff & (\forall x : \lnot (x \in A \cap B))                                                                           \\
    \iff & (\forall x : \lnot (x \in A \land x \in B))                                     &  & \text{(by \cref{3.1.23})} \\
    \iff & (\forall x : x \notin A \lor x \notin B)                                                                       \\
    \iff & (\forall x : (x \in A \implies x \notin B) \land (x \in B \implies x \notin A))                                \\
  \end{align*}

  Now we show that \(A = X \setminus B\).
  From the proof above we have
  \begin{align*}
             & (A \cup B = X) \land (A \cap B = \emptyset)                                                          \\
    \implies & (A \subseteq X) \land (A \cap B = \emptyset)                        &  & \text{(by \cref{ex:3.1.7})} \\
    \implies & (\forall x : x \in A \implies x \in X) \land (A \cap B = \emptyset) &  & \text{(by \cref{3.1.15})}   \\
    \implies & (\forall x : x \in A \implies x \in X)                                                               \\
             & \land (\forall x : x \in A \implies x \notin B)                                                      \\
    \implies & (\forall x : x \in A \implies x \in X \land x \notin B)                                              \\
    \implies & (\forall x : x \in A \implies x \in X \setminus B).                 &  & \text{(by \cref{3.1.27})}
  \end{align*}
  And we also have
  \begin{align*}
             & \forall x : x \in X \setminus B                                        \\
    \implies & x \in X \land x \notin B                &  & \text{(by \cref{3.1.27})} \\
    \implies & x \in A \cup B \land x \notin B                                        \\
    \implies & (x \in A \lor x \in B) \land x \notin B &  & \text{(by \cref{3.4})}    \\
    \implies & x \in A \land x \notin B                                               \\
    \implies & x \in A.
  \end{align*}
  Thus by \cref{3.1.4} we have \(A = X \setminus B\).
  Similarly argument show that \(B = X \setminus A\), as desired.
\end{proof}

\begin{ex}\label{ex:3.1.10}
  Let \(A\) and \(B\) be sets.
  Show that the three sets \(A \setminus B\), \(A \cap B\), and \(B \setminus A\) are disjoint, and that their union is \(A \cup B\).
\end{ex}

\begin{proof}
  Suppose that \(A, B\) are sets.
  Then we have
  \begin{align*}
             & \forall x : x \in A \setminus B                                \\
    \implies & x \in A \land x \notin B        &  & \text{(by \cref{3.1.27})} \\
    \implies & x \notin A \cap B.              &  & \text{(by \cref{3.1.23})}
  \end{align*}
  And we also have
  \begin{align*}
             & \forall x : x \in A \cap B                                \\
    \implies & x \in A \land x \in B      &  & \text{(by \cref{3.1.23})} \\
    \implies & x \notin A \setminus B.    &  & \text{(by \cref{3.1.27})}
  \end{align*}
  Thus \(A \setminus B\) and \(A \cap B\) are disjoint.
  Similarly argument show that \(B \setminus A\) and \(A \cap B\) are disjoint.
  For \(A \setminus B\) and \(B \setminus A\), we have
  \begin{align*}
             & \forall x : x \in A \setminus B                                \\
    \implies & x \in A \land x \notin B        &  & \text{(by \cref{3.1.27})} \\
    \implies & x \notin B \setminus A.         &  & \text{(by \cref{3.1.27})}
  \end{align*}
  Similarly argument show that \(\forall x : x \in B \setminus A \implies x \notin A \setminus B\).
  Thus \(A \setminus B\) and \(B \setminus A\) are disjoint.

  Now we show that \((A \setminus B) \cup (A \cap B) \cup (B \setminus A) = A \cup B\).
  \begin{align*}
         & \forall x : x \in (A \setminus B) \cup (A \cap B) \cup (B \setminus A)                                                 \\
    \iff & x \in (A \setminus B) \lor x \in (A \cap B) \lor x \in (B \setminus A)                  &  & \text{(by \cref{3.4})}    \\
    \iff & (x \in A \land x \notin B) \lor (x \in A \cap B) \lor (x \in B \land x \notin A)        &  & \text{(by \cref{3.1.27})} \\
    \iff & (x \in A \land x \notin B) \lor (x \in A \land x \in B) \lor (x \in B \land x \notin A) &  & \text{(by \cref{3.1.23})} \\
    \iff & ((x \in A \lor (x \in A \land x \in B))                                                                                \\
         & \land (x \notin B \lor (x \in A \land x \in B)))                                                                       \\
         & \lor (x \in B \land x \notin A)                                                                                        \\
    \iff & ((x \in A) \land (x \notin B \lor x \in A)) \lor (x \in B \land x \notin A)                                            \\
    \iff & (x \in A) \lor (x \in B \land x \notin A)                                                                              \\
    \iff & x \in A \lor x \in B                                                                                                   \\
    \iff & x \in A \cup B.                                                                         &  & \text{(by \cref{3.4})}
  \end{align*}
\end{proof}

\begin{ex}\label{ex:3.1.11}
  Show that the axiom of replacement implies the axiom of specification.
\end{ex}

\begin{proof}
  By \cref{3.6}, \(z \in \{y : P(x, y) \text{ is true for some } x \in A\} \iff P(x, z)\) is true for some \(x \in A\).
  Change all \(y\) and \(z\) into \(x\), and replace \(P(x, x)\) with \(P(x)\), we derive \(x \in \{x : P(x) \text{ is true for some } x \in A\} \iff P(x)\) is true for some \(x \in A\), which is the same as \cref{3.5}.
  Thus we conclude that \cref{3.6} implies \cref{3.5}.
\end{proof}
\section{Russell's paradox}\label{i:sec:3.2}

\begin{ax}[Universal specification]\label{i:3.8}
  (Dangerous!)
  Suppose for every object \(x\) we have a property \(P(x)\) pertaining to \(x\) (so that for every \(x\), \(P(x)\) is either a true statement or a false statement).
  Then there exists a set \(\set{x : P(x) \text{ is true}}\) such that for every object \(y\),
  \[
    y \in \set{x : P(x) \text{ is true}} \iff P(y) \text{ is true}.
  \]
\end{ax}

\begin{note}
  Compare to \cref{i:3.5}, an object \(x\) does not need to be in a set \(A\) to apply this axiom.
  So \cref{i:3.8} is more powerful than \cref{i:3.5}.
\end{note}

\begin{note}
  \cref{i:3.8} is also known as the \emph{axiom of comprehension}.
  It asserts that every property corresponds to a set.
  \cref{i:3.8} also implies most of the axioms in \cref{i:sec:3.1} (see \cref{i:ex:3.2.1}).
  Unfortunately, this axiom cannot be introduced into set theory, because it creates a logical contradiction known as \emph{Russell's paradox}, discovered by the philosopher and logician Bertrand Russell (1872--1970) in 1901.
  The paradox runs as follows.
  Let \(P(x)\) be the statement
  \[
    P(x) \iff \text{``\(x\) is a set, and \(x \notin x\)''};
  \]
  i.e., \(P(x)\) is true only when \(x\) is a set which does not contain itself.
  Now use the axiom of universal specification to create the set
  \[
    \Omega \coloneqq \set{x : P(x) \text{ is true}} = \set{x : x \text{ is a set and } x \notin x},
  \]
  i.e., the set of all sets which do not contain themselves.
  Now ask the question: does \(\Omega\) contain itself, i.e. is \(\Omega \in \Omega\)?
  If \(\Omega\) did contain itself, then by definition of \(\Omega\) this means that \(P(\Omega)\) is true, i.e., \(\Omega\) is a set and \(\Omega \notin \Omega\).
  On the other hand, if \(\Omega\) did not contain itself, then by definition of \(P\), \(P(\Omega)\) would be true, and hence by the definition of \(\Omega\), \(\Omega \in \Omega\).
  Thus in either case we have both \(\Omega \in \Omega\) and \(\Omega \notin \Omega\), which is absurd.
\end{note}

\begin{note}
  The problem with \cref{i:3.8} is that it creates sets which are far too ``large''.
  Since sets are themselves objects (\cref{i:3.1}), this means that sets are allowed to contain themselves, which is a somewhat silly state of affairs.
  One way to informally resolve this issue is to think of objects as being arranged in a hierarchy.
  At the bottom of the hierarchy are the \emph{primitive objects} - the objects that are not sets.
  Then on the next rung of the hierarchy there are sets whose elements consist only of primitive objects, let's call these ``primitive sets'' for now.
  Then there are sets whose elements consist only of primitive objects and primitive sets, and we can form sets out of these objects, and so forth.
  The point is that at each stage of the hierarchy we only see sets whose elements consist of objects at lower stages of the hierarchy, and so at no stage do we ever construct a set which contains itself.
\end{note}

\begin{note}
  In pure set theory, there will be no primitive objects, but there will be one primitive set \(\emptyset\) on the next rung of the hierarchy.
\end{note}

\begin{ax}[Regularity]\label{i:3.9}
  If \(A\) is a non-empty set, then there is at least one element \(x\) of \(A\) which is either not a set, or is disjoint from \(A\).
\end{ax}

\begin{note}
  The point of \cref{i:3.9} (which is also known as the \emph{axiom of foundation}) is that it is asserting that at least one of the elements of \(A\) is so low on the hierarchy of objects that it does not contain any of the other elements of \(A\).
  One particular consequence of \cref{i:3.9} is that sets are no longer allowed to contain themselves (\cref{i:ex:3.2.2}).
\end{note}

\exercisesection

\begin{ex}\label{i:ex:3.2.1}
  Show that the universal specification axiom, \cref{i:3.8}, if assumed to be true, would imply \crefrange{i:3.2}{i:3.6}.
  (If we assume that all natural numbers are objects, we also obtain \cref{i:3.7}.)
  Thus, \cref{i:3.8}, if permitted, would simplify the foundations of set theory tremendously (and can be viewed as one basis for an intuitive model of set theory known as ``naive set theory'').
  Unfortunately, as we have seen, \cref{i:3.8} is ``too good to be true''!
\end{ex}

\begin{proof}[\pf{i:ex:3.2.1}]
  We first show that \cref{i:3.8} implies \cref{i:3.2}.
  Using \cref{i:3.8} we can create a set \(E = \set{x : P(x)}\) where \(P(x)\) is a property which is false for all object \(x\).
  Then we must have \(x \notin E\) for every object \(x\).
  Hence we can construct empty set using \cref{i:3.8} and therefore \cref{i:3.8} implies \cref{i:3.2}.

  Next we show that \cref{i:3.8} implies \cref{i:3.3}.
  Suppose that \(a, b\) are objects.
  Using \cref{i:3.8} we can create a set \(A = \set{x : P(x)}\) where \(P(x)\) is a property which is false for all object \(x\) other than \(a\).
  Then the only element in the set \(A\) is \(a\).
  Hence we can construct any singleton set using \cref{i:3.8}.
  Using \cref{i:3.8} again we can create a set \(B = \set{x : Q(x)}\) where \(Q(x)\) is a property which is false for all object \(x\) other than \(a, b\).
  Then \(a, b\) are the only elements in the set \(B\).
  Hence we can construct any pair set using \cref{i:3.8}.
  Therefore \cref{i:3.8} implies \cref{i:3.3}.

  Next we show that \cref{i:3.8} implies \cref{i:3.4}.
  Suppose that \(A, B\) are sets.
  By \cref{i:3.8}, there exists a set \(C = \set{x : P(x)}\) where \(P(x) = (x \in A) \lor (x \in B)\).
  Clearly we have \(C = A \cup B\).
  Therefore \cref{i:3.8} implies \cref{i:3.4}.

  Next we show that \cref{i:3.8} implies \cref{i:3.5}.
  Suppose that \(A\) is a set.
  Using \cref{i:3.8} we can create a property \(P(x)\) which is true for some object \(x\).
  Then using \cref{i:3.8} again we can create a set \(B = \set{x : Q(x)}\) where \(Q(x) = (x \in A) \land P(x)\).
  Clearly we have \(B = \set{x \in A : Q(x)}\).
  Therefore \cref{i:3.8} implies \cref{i:3.5}.

  Next we show that \cref{i:3.8} implies \cref{i:3.6}.
  Suppose that \(A\) is a set and \(P(x, y)\) is a property such that for each \(x \in A\), there is at most one object \(y\) such that \(P(x, y)\).
  By \cref{i:3.8}, there exists a set \(B = \set{y : Q(y)}\) where \(Q(y)\) is the statement ``\(P(x, y)\) is true for some \(x \in A\)''.
  Clearly we have \(B = \set{y : P(x, y) \text{ is true for some } x \in A}\).
  Therefore \cref{i:3.8} implies \cref{i:3.6}.

  Finally, suppose all natural numbers are objects.
  Using \cref{i:3.8} we can create a set \(N = \set{x : P(x)}\) where \(P(x)\) is the statement ``\(x\) is a natural number and \(x\) satisfy \crefrange{i:2.1}{i:2.5}''.
  Clearly we have \(N = \N\).
  Therefore \cref{i:3.8} implies \cref{i:3.7}.
\end{proof}

\begin{ex}\label{i:ex:3.2.2}
  Use the axiom of regularity, \cref{i:3.9} (and the singleton set axiom, \cref{i:3.3}) to show that if \(A\) is a set, then \(A \notin A\).
  Furthermore, show that if \(A\) and \(B\) are two sets, then either \(A \notin B\) or \(B \notin A\) (or both).
\end{ex}

\begin{proof}[\pf{i:ex:3.2.2}]
  We fisrt show that if \(A\) is a set, then \(A \notin A\).
  Suppose for sake of contradiction that there exist a set \(A\) such that \(A \in A\) is true.
  By \cref{i:3.3}, there exist a set \(\set{A}\) and \(A \in \set{A}\) is true.
  Then \(A \in A \cap \set{A}\) is true.
  But by \cref{i:3.9}, the only element \(A\) in \(\set{A}\) must be disjoint from \(\set{A}\), which mean \(A \cap \set{A} = \emptyset\), a contradiction.
  Thus there does not exist a set \(A\) such that \(A \in A\) is true, i.e., \(A \notin A\) for any set \(A\).

  Now we show that if \(A\) and \(B\) are two sets, then \((A \notin B) \lor (B \notin A)\) is true.
  Since
  \[
    (A \notin B) \lor (B \notin A) \iff (\lnot(A \in B)) \lor (\lnot(B \in A)) \iff \lnot ((A \in B) \land (B \in A)),
  \]
  it suffices to show that \((A \in B) \land (B \in A)\) is false.
  So suppose for sake of contradiction that \((A \in B) \land (B \in A)\) is true.
  By \cref{i:3.3} we can create a pair set \(\set{A, B}\).
  By \cref{i:3.9} we know that there exists one element in \(\set{A, B}\) such that either it is not a set or it is disjoint from \(\set{A, B}\).
  Since \(A, B\) are sets, we must have either \(A \cap \set{A, B} = \emptyset\) or \(B \cap \set{A, B} = \emptyset\).
  But since \((A \in B) \land (B \in A)\), we must have \(A \in B \cap \set{A, B}\) and \(B \in A \cap \set{A, B}\), a contradiction.
  Thus \((A \in B) \land (B \in A)\) is false.
\end{proof}

\begin{ex}\label{i:ex:3.2.3}
  Show (assuming the other axioms of set theory) that the universal specification axiom, \cref{i:3.8}, is equivalent to an axiom postulating the existence of a ``universal set'' \(\Omega\) consisting of all objects (i.e., for all objects \(x\), we have \(x \in \Omega\)).
  In other words, if \cref{i:3.8} is true, then a universal set exists, and conversely, if a universal set exists, then \cref{i:3.8} is true.
  (This may explain why \cref{i:3.8} is called the axiom of universal specification.)
  Note that if a universal set \(\Omega\) existed, then we would have \(\Omega \in \Omega\) by \cref{i:3.1}, contradicting \cref{i:ex:3.2.2}.
  Thus the axiom of foundation specifically rules out the axiom of universal specification.
\end{ex}

\begin{proof}[\pf{i:ex:3.2.3}]
  If \cref{i:3.8} is true, then there exists a set \(\Omega = \set{x : x \text{ is a object}}\), and \(\Omega \in \Omega\).
  Thus \cref{i:3.8} implies a universal set exists.
  If a universal set \(\Omega\) exists, then by \cref{i:3.5} we must have a set \(A = \set{x \in \Omega : P(x)}\) where \(P(x)\) is a property of object \(x \in \Omega\).
  Clearly we have \(A = \set{x : P(x)}\) since \(\Omega\) is consist of all objects.
  Thus a universal set exists implies \cref{i:3.8} is true.
  We conclude that \cref{i:3.8} is true iff a universal set exists.
\end{proof}

\section{Functions}\label{i:sec:3.3}

\begin{defn}[Functions]\label{i:3.3.1}
  Let \(X, Y\) be sets, and let \(P(x, y)\) be a property pertaining to an object \(x \in X\) and an object \(y \in Y\), such that for every \(x \in X\), there is exactly one \(y \in Y\) for which \(P(x, y)\) is true (this is sometimes known as the \emph{vertical line test}).
  Then we define the \emph{function \(f : X \to Y\) defined by \(P\) on the domain \(X\) and codomain \(Y\)} to be the object which, given any input \(x \in X\), assigns an output \(f(x) \in Y\), defined to be the unique object \(f(x) \in Y\) for which \(P(x, f(x))\) is true.
  Thus, for any \(x \in X\) and \(y \in Y\),
  \[
    y = f(x) \iff P(x, y) \text{ is true}.
  \]
\end{defn}

\begin{note}
  Implicit in \cref{i:3.3.1} is the assumption that whenever one is given two sets \(X, Y\) and a property \(P\) obeying the vertical line test, one can form a function object.
  Strictly speaking, this assumption of the existence of the function as a mathematical object should be stated as an explicit axiom;
  however we will not do so here, as it turns out to be redundant.
  (More precisely, in view of \cref{i:ex:3.5.10} below, it is always possible to encode a function \(f\) as an ordered triple \((X, Y, \set{(x, f(x)) : x \in X})\) consisting of the domain, codomain, and graph of the function, which gives a way to build functions as objects using the operations provided by the preceding axioms.)
\end{note}

\begin{note}
  Functions are also referred to as \emph{maps} or \emph{transformations}, depending on the context.
  They are also sometimes called \emph{morphisms}, although to be more precise, a morphism refers to a more general class of object, which may or may not correspond to actual functions, depending on the context.
\end{note}

\begin{note}
  One common way to define a function is simply to specify its domain, its codomain, and how one generates the output \(f(x)\) from each input;
  this is known as an \emph{explicit} definition of a function.
  In other cases we only define a function \(f\) by specifying what property \(P(x, y)\) links the input \(x\) with the output \(f(x)\);
  this is an \emph{implicit} definition of a function.
  An implicit definition is only valid if we know that for every input there is exactly one output which obeys the implicit relation.
\end{note}

\begin{note}
  In many cases we omit specifying the domain and codomain of a function for brevity.
  However, too much of this abbreviation can be dangerous;
  sometimes it is important to know what the domain and codomain of the function is.
\end{note}

\begin{ac}\label{i:ac:3.3.1}
  We observe that functions obey the axiom of substitution:
  if \(x = x'\), then \(f(x) = f(x')\).
  In other words, equal inputs imply equal outputs.
  On the other hand, unequal inputs do not necessarily ensure unequal outputs.
  For example, \emph{constant function} simply assign each input with the same output.
\end{ac}

\begin{proof}[\pf{i:ac:3.3.1}]
  Suppose that \(f : X \to Y\) is a function defined by \(P\) on the domain \(X\) and codomain \(Y\).
  Let \(x, x' \in X\) where \(x = x'\).
  By \cref{i:3.3.1}, we know that \(f(x)\) is the unique object for which \(P(x, f(x))\) is true.
  Similarly, we know that \(f(x')\) is the unique object for which \(P(x', f(x'))\) is true.
  Since \(x = x'\), we know that \(P(x', f(x))\) must be true.
  Then the uniqueness of \(f(x')\) implies \(f(x) = f(x')\).
\end{proof}

\setcounter{thm}{4}
\begin{rmk}\label{i:3.3.5}
  We are now using parentheses () to denote several different things in mathematics;
  on one hand, we are using them to clarify the order of operations, but on the other hand we also use parentheses to enclose the argument of a function \(f(x)\) or of a property such as \(P(x)\).
  However, the two usages of parentheses usually are unambiguous from context.
  For instance, if \(a\) is a number, then \(a(b + c)\) denotes the expression \(a \times (b + c)\), whereas if \(f\) is a function, then \(f(b + c)\) denotes the output of \(f\) when the input is \(b + c\).
  Sometimes the argument of a function is denoted by subscripting instead of parentheses;
  for instance, a sequence of natural numbers \(a_0, a_1, a_2, a_3, \dots\) is, strictly speaking, a function from \(\N\) to \(\N\), but is denoted by \(n \mapsto a_n\) rather than \(n \mapsto a(n)\).
\end{rmk}

\begin{rmk}\label{i:3.3.6}
  Strictly speaking, functions are not necessarily sets, and sets are not necessarily functions;
  it does not make sense to ask whether an object \(x\) is an element of a function \(f\), and it does not make sense to apply a set \(A\) to an input \(x\) to create an output \(A(x)\).
  On the other hand, it is possible to start with a function \(f : X \to Y\) and construct its \emph{graph} \(\set{(x, f(x)) : x \in X}\), which describes the function completely once the domain \(X\) and codomain \(Y\) are specified.
  See \cref{i:sec:3.5}.
\end{rmk}

\begin{defn}[Equality of functions]\label{i:3.3.7}
  Two functions \(f : X \to Y, g : X' \to Y'\) are said to be \emph{equal} iff they have the same domain and codomain (i.e., \(X = X'\) and \(Y = Y'\)), and \(f(x) = g(x)\) for all \(x \in X\).
  (If \(f(x)\) and \(g(x)\) agree for some values of \(x\), but not others, then we do not consider \(f\) and \(g\) to be equal.)
  According to this definition, two functions that have different domains or different codomains are, strictly speaking, distinct functions.
  However, when it is safe to do so without causing confusion, it is sometimes useful to ``abuse notation'' by identifying together functions of different domains or codomains if their values agree on their common domain of definition;
  this is analogous to the practice of ``overloading'' an operator in software engineering.
  See the discussion after \cref{i:9.4.1} for an instance of this.
\end{defn}

\setcounter{thm}{8}
\begin{eg}\label{i:3.3.9}
  A rather boring example of a function is the \emph{empty function} \(f : \emptyset \to X\) from the empty set to a given set \(X\).
  Since the empty set has no elements, we do not need to specify what \(f\) does to any input.
  Nevertheless, just as the empty set is a set, the empty function is a function, albeit not a particularly interesting one.
  Note that for each set \(X\), there is only one function from \(\emptyset\) to \(X\), since \cref{i:3.3.7} asserts that all functions from \(\emptyset\) to \(X\) are equal.
\end{eg}

\begin{defn}[Composition]\label{i:3.3.10}
  Let \(f : X \to Y\) and \(g : Y \to Z\) be two functions, such that the codomain of \(f\) is the same set as the domain of \(g\).
  We then define the \emph{composition} \(g \circ f : X \to Z\) of the two functions \(g\) and \(f\) to be the function defined explicitly by the formula
  \[
    (g \circ f)(x) \coloneqq g(f(x)).
  \]
  If the codomain of \(f\) does not match the domain of \(g\), we leave the composition \(g \circ f\) undefined.
\end{defn}

\begin{note}
  Composition is not commutative:
  \(f \circ g\) and \(g \circ f\) are not necessarily the same function.
\end{note}

\setcounter{thm}{11}
\begin{lem}[Composition is associative]\label{i:3.3.12}
  Let \(f : Z \to W\), \(g : Y \to Z\), and \(h : X \to Y\) be functions.
  Then \(f \circ (g \circ h) = (f \circ g) \circ h\).
\end{lem}

\begin{proof}[\pf{i:3.3.12}]
  Since \(g \circ h\) is a function from \(X\) to \(Z\), \(f \circ (g \circ h)\) is a function from \(X\) to \(W\).
  Similarly, \(f \circ g\) is a function from \(Y\) to \(W\), and hence \((f \circ g) \circ h\) is a function from \(X\) to \(W\).
  Thus, \(f \circ (g \circ h)\) and \((f \circ g) \circ h\) have the same domain and codomain.
  In order to check that they are equal, we see from \cref{i:3.3.7} that we have to verify that \((f \circ (g \circ h))(x) = ((f \circ g) \circ h)(x)\) for all \(x \in X\).
  But by \cref{i:3.3.10},
  \begin{align*}
    (f \circ (g \circ h))(x)
     & = f((g \circ h)(x))         \\
     & = f(g(h(x)))                \\
     & = (f \circ g)(h(x))         \\
     & = ((f \circ g) \circ h)(x),
  \end{align*}
  as desired.
\end{proof}

\begin{rmk}\label{i:3.3.13}
  Note that while \(g\) appears to the left of \(f\) in the expression \(g \circ f\), the function \(g \circ f\) applies the right-most function \(f\) first, before applying \(g\).
  This is often confusing at first;
  it arises because we traditionally place a function \(f\) to the left of its input \(x\) rather than to the right.
  (There are some alternate mathematical notations in which the function is placed to the right of the input, thus we would write \(xf\) instead of \(f(x)\), but this notation has often proven to be more confusing than clarifying, and has not as yet become particularly popular.)
\end{rmk}

\begin{defn}[One-to-one function]\label{i:3.3.14}
  A function \(f\) is \emph{one-to-one} (or \emph{injective}) if different elements map to different elements:
  \[
    x \neq x' \implies f(x) \neq f(x').
  \]

  Equivalently, a function is one-to-one if
  \[
    f(x) = f(x') \implies x = x'.
  \]
\end{defn}

\begin{note}
  The notion of a one-to-one function depends not just on what the function does, but also what its domain is.
\end{note}

\setcounter{thm}{15}
\begin{rmk}\label{i:3.3.16}
  If a function \(f : X \to Y\) is not one-to-one, then one can find distinct \(x\) and \(x'\) in the domain \(X\) such that \(f(x) = f(x')\), thus one can find two inputs which map to one output.
  Because of this, we say that \(f\) is \emph{two-to-one} instead of \emph{one-to-one}.
\end{rmk}

\begin{defn}[Onto functions]\label{i:3.3.17}
  A function \(f\) is \emph{onto} (or \emph{surjective}) if \(f(X) = Y\), i.e., every element in \(Y\) comes from applying \(f\) to some element in \(X\):
  \[
    \text{For every } y \in Y, \text{there exists } x \in X \text{ such that } f(x) = y.
  \]
\end{defn}

\begin{note}
  The notion of an onto function depends not just on what the function does, but also what its codomain is.
\end{note}

\setcounter{thm}{18}
\begin{rmk}\label{i:3.3.19}
  The concepts of injectivity and surjectivity are in many ways dual to each other.
  See \cref{i:ex:3.3.2,i:ex:3.3.4,i:ex:3.3.5} for some evidence of this.
\end{rmk}

\begin{defn}[Bijective functions]\label{i:3.3.20}
  A function \(f : X \to Y\) which is both one-to-one and onto is also called \emph{bijective} or \emph{invertible}.

  If \(f\) is bijective, then for every \(y \in Y\), there is exactly one \(x\) such that \(f(x) = y\) (there is at least one because of surjectivity, and at most one because of injectivity).
  This value of \(x\) is denoted \(f^{-1}(y)\); thus \(f^{-1}\) is a function from \(Y\) to \(X\).
  We call \(f^{-1}\) the \emph{inverse} of \(f\).
\end{defn}

\begin{note}
  The notion of a bijective function depends not just on what the function does, but also what its domain and codomain are.
\end{note}

\setcounter{thm}{22}
\begin{rmk}\label{i:3.3.23}
  If a function \(x \mapsto f(x)\) is bijective, then we sometimes call \(f\) a \emph{perfect matching} or a \emph{one-to-one correspondence} (not to be confused with the notion of a one-to-one function), and denote the action of \(f\) using the notation \(x \leftrightarrow f(x)\) instead of \(x \mapsto f(x)\).
\end{rmk}

\exercisesection

\begin{ex}\label{i:ex:3.3.1}
  Show that the definition of equality in \cref{i:3.3.7} is reflexive, symmetric, and transitive.
  Also verify the substitution property: if \(f, \tilde{f} : X \to Y\) and \(g, \tilde{g} : Y \to Z\) are functions such that \(f = \tilde{f}\) and \(g = \tilde{g}\), then \(g \circ f = \tilde{g} \circ \tilde{f}\).
  Of course, these statements are immediate from the axioms of equality applied directly to the functions in question, but the point of the exercise is to show that they can also be established by instead applying the axioms of equality to elements of the domain and codomain of these functions, rather than to the functions itself.
\end{ex}

\begin{proof}[\pf{i:ex:3.3.1}]
  We first show that \cref{i:3.3.7} is reflexive.
  Suppose that \(f : X \to Y\) is a function.
  Then we have
  \begin{align*}
             & \begin{dcases}
                 X = X \\
                 Y = Y \\
                 \forall x \in X, f(x) = f(x)
               \end{dcases} &  & \by{i:ac:3.1.1,i:3.3.1}      \\
    \implies & f = f.                       &  & \by{i:3.3.7}
  \end{align*}
  Thus, \cref{i:3.3.7} is reflexive.

  Next we show that \cref{i:3.3.7} is symmetric.
  Suppose that \(f : X \to Y, g : X' \to Y'\) are functions such that \(f = g\).
  Then we have
  \begin{align*}
         & f = g                                          \\
    \iff & \begin{dcases}
             X = X' \\
             Y = Y' \\
             \forall x \in X, f(x) = g(x)
           \end{dcases} &  & \by{i:3.3.7}                 \\
    \iff & \begin{dcases}
             X' = X \\
             Y' = Y \\
             \forall x \in X, g(x) = f(x)
           \end{dcases} &  & \by{i:ac:3.1.1}              \\
    \iff & g = f.                       &  & \by{i:3.3.7}
  \end{align*}
  Thus, \cref{i:3.3.7} is symmetric.

  Next we show that \cref{i:3.3.7} is transitive.
  Suppose that \(f : X_f \to Y_f, g : X_g \to Y_g, h : X_h \to Y_h\) are functions such that \((f = g) \land (g = h)\).
  Then we have
  \begin{align*}
             & \begin{dcases}
                 f = g \\
                 g = h
               \end{dcases}                                       \\
    \implies & \begin{dcases}
                 X_f = X_g                      \\
                 Y_f = Y_g                      \\
                 \forall x \in X_f, f(x) = g(x) \\
                 X_g = X_h                      \\
                 Y_g = Y_h                      \\
                 \forall x \in X_g, g(x) = h(x)
               \end{dcases} &  & \by{i:3.3.7}                      \\
    \implies & \begin{dcases}
                 X_f = X_h \\
                 Y_f = Y_h \\
                 \forall x \in X_f, f(x) = g(x)
               \end{dcases}    &  & \by{i:ac:3.1.1}                \\
    \implies & f = h.                            &  & \by{i:3.3.7}
  \end{align*}
  Thus, \cref{i:3.3.7} is transitive.

  Finally we show that Axiom of substitution holds for composition.
  Suppose that \(f : X \to Y, \tilde{f} : X \to Y, g : Y \to Z, \tilde{g} : Y \to Z\) are functions such that \((f = \tilde{f}) \land (g = \tilde{g})\).
  By \cref{i:3.3.10}, \(g \circ f : X \to Z\) and \(\tilde{g} \circ \tilde{f} : X \to Z\) are well-defined.
  Since \(X = X\) and \(Z = Z\), to prove that \(g \circ f = \tilde{g} \circ \tilde{f}\), by \cref{i:3.3.7}, we only need to show that \((g \circ f)(x) = \pa{\tilde{g} \circ \tilde{f}}(x)\) for all \(x \in X\).
  But this is true since.
  \begin{align*}
    \forall x \in X, (g \circ f)(x) & = g(f(x))                            &  & \by{i:3.3.10} \\
                                    & = g\pa{\tilde{f}(x)}                 &  & \by{i:3.3.7}  \\
                                    & = \tilde{g}\pa{\tilde{f}(x)}         &  & \by{i:3.3.7}  \\
                                    & = \pa{\tilde{g} \circ \tilde{f}}(x). &  & \by{i:3.3.10}
  \end{align*}
\end{proof}

\begin{ex}\label{i:ex:3.3.2}
  Let \(f : X \to Y\) and \(g : Y \to Z\) be functions.
  Show that if \(f\) and \(g\) are both injective, then so is \(g \circ f\);
  similarly, show that if \(f\) and \(g\) are both surjective, then so is \(g \circ f\).
\end{ex}

\begin{proof}[\pf{i:ex:3.3.2}]
  We first show that \(f, g\) are injective implies \(g \circ f\) is injective.
  Suppose that \(f : X \to Y, g : Y \to Z\) are injective functions.
  Then we have
  \begin{align*}
             & \forall x, x' \in X : x \neq x'                         \\
    \implies & f(x) \neq f(x')                      &  & \by{i:3.3.14} \\
    \implies & g(f(x)) \neq g(f(x'))                &  & \by{i:3.3.14} \\
    \implies & (g \circ f)(x) \neq (g \circ f)(x'). &  & \by{i:3.3.10}
  \end{align*}
  Thus, by \cref{i:3.3.14}, \(g \circ f\) is injective.

  Now we show that \(f, g\) are surjective implies \(g \circ f\) is surjective.
  Suppose that \(f : X \to Y, g : Y \to Z\) are surjective functions.
  Then we have
  \begin{align*}
             & \begin{dcases}
                 \forall z \in Z, \exists y \in Y : z = g(y) \\
                 \forall w \in Y, \exists x \in X : w = f(x)
               \end{dcases}                   &  & \by{i:3.3.17}                                   \\
    \implies & \forall z \in Z, \exists x \in X : z = g(f(x)) = (g \circ f)(x). &  & \by{i:3.3.10}
  \end{align*}
  Thus, by \cref{i:3.3.17}, \(g \circ f\) is surjective.
\end{proof}

\begin{ex}\label{i:ex:3.3.3}
  When is the empty function into a given set \(X\) injective?
  surjective?
  bijective?
\end{ex}

\begin{proof}[\pf{i:ex:3.3.3}]
  Suppose that \(f : \emptyset \to X\) is the empty function for the given set \(X\).
  \(f\) is always injective since the statement ``for all \(x, x' \in \emptyset\), \(f(x) = f(x') \implies x = x'\)'' is vacuously true by \cref{i:3.2}.
  For surjective we can split into two cases:
  \begin{itemize}
    \item If \(X \neq \emptyset\), then \(f\) is not surjective, since \(\forall y \in X\), \(\nexists x \in \emptyset\) such that \(f(x) = y\).
    \item If \(X = \emptyset\), then \(f\) is surjective, since \(\forall y \in \emptyset\), \(\exists x \in \emptyset\) such that \(f(x) = y\) (which is vacuously true).
  \end{itemize}
  From the proof above we see that the empty function \(f\) is bijective iff \(X = \emptyset\).
\end{proof}

\begin{ex}\label{i:ex:3.3.4}
  In this section we give some cancellation laws for composition.
  Let \(f : X \to Y\), \(\tilde{f} : X \to Y\), \(g : Y \to Z\), and \(\tilde{g} : Y \to Z\) be functions.
  Show that if \(g \circ f = g \circ \tilde{f}\) and g is injective, then \(f = \tilde{f}\).
  Is the same statement true if \(g\) is not injective?
  Show that if \(g \circ f = \tilde{g} \circ f\) and \(f\) is surjective, then \(g = \tilde{g}\).
  Is the same statement true if \(f\) is not surjective?
\end{ex}

\begin{proof}[\pf{i:ex:3.3.4}]
  We first show that \(g\) is injective, and \(g \circ f = g \circ \tilde{f}\) implies \(f = \tilde{f}\).
  Suppose that \(f : X \to Y, \tilde{f} : X \to Y, g : Y \to Z\) are functions, such that \(g\) is injective and \(g \circ f = g \circ \tilde{f}\).
  Then we have
  \begin{align*}
             & g \circ f = g \circ \tilde{f}                                                  \\
    \implies & \forall x \in X, (g \circ f)(x) = \pa{g \circ \tilde{f}}(x) &  & \by{i:3.3.7}  \\
    \implies & \forall x \in X, g(f(x)) = g\pa{\tilde{f}(x)}               &  & \by{i:3.3.10} \\
    \implies & \forall x \in X, f(x) = \tilde{f}(x)                        &  & \by{i:3.3.14} \\
    \implies & f = \tilde{f}.                                              &  & \by{i:3.3.7}
  \end{align*}
  The statement is not true when \(g\) is not injective.
  For example, define \(f = x \mapsto x, \tilde{f} = x \mapsto \abs{x}, g = x \mapsto x^2\).
  Then we see that \(g \circ f = x \mapsto x^2 = x \mapsto \abs{x}^2 = g \circ \tilde{f}\).
  But \(f(-1) = -1 \neq \abs{-1} = \tilde{f}(1)\) implies \(f \neq \tilde{f}\).

  Now we show that \(f\) is surjective, and \(g \circ f = \tilde{g} \circ f\) implies \(g = \tilde{g}\).
  Suppose that\(f : X \to Y, g : Y \to Z, \tilde{g} : Y \to Z\) are functions, such that \(f\) is surjective and \(g \circ f = \tilde{g} \circ f\).
  Then we have
  \begin{align*}
             & \forall y \in Y, \exists x \in X : y = f(x)       &  & \by{i:3.3.17} \\
    \implies & \forall y \in Y, \exists x \in X : g(y) = g(f(x)) &  & \by{i:3.3.1}  \\
             & = (g \circ f)(x)                                  &  & \by{i:3.3.10} \\
             & = (\tilde{g} \circ f)(x)                          &  & \by{i:3.3.7}  \\
             & = \tilde{g}(f(x)) = \tilde{g}(y)                  &  & \by{i:3.3.10} \\
    \implies & g = \tilde{g}.                                    &  & \by{i:3.3.7}
  \end{align*}
  The statement is not true when \(f\) is not surjective.
  For example, define \(f = x \mapsto \abs{x}, g = x \mapsto x, \tilde{g} = x \mapsto \abs{x}\).
  Then we see that \(g \circ f = x \mapsto \abs{x} = x \mapsto \abs{(\abs{x})} = \tilde{g} \circ f\).
  But \(g(-1) = -1 \neq \abs{-1} = \tilde{g}(-1)\) implies \(g \neq \tilde{g}\).
\end{proof}

\begin{ex}\label{i:ex:3.3.5}
  Let \(f : X \to Y\) and \(g : Y \to Z\) be functions.
  Show that if \(g \circ f\) is injective, then \(f\) must be injective.
  Is it true that \(g\) must also be injective?
  Show that if \(g \circ f\) is surjective, then \(g\) must be surjective.
  Is it true that \(f\) must also be surjective?
\end{ex}

\begin{proof}[\pf{i:ex:3.3.5}]
  We first show that \(g \circ f\) is injective implies \(f\) is injective.
  Suppose \(f : X \to Y, g : Y \to Z\) are functions such that \(g \circ f\) is injective.
  Then we have
  \begin{align*}
             & \forall x, x' \in X, x \neq x'                                                       \\
    \implies & g(f(x)) = (g \circ f)(x) \neq (g \circ f)(x') = g(f(x')) &  & \by{i:3.3.10,i:3.3.14} \\
    \implies & f(x) \neq f(x').                                         &  & \by{i:ac:3.3.1}
  \end{align*}
  Thus, by \cref{i:3.3.14}, \(f\) is injective.
  And we don't need \(g\) to be injective to finish the proof.

  Now we show that \(g \circ f\) is surjective implies \(g\) is surjective.
  Suppose \(f : X \to Y, g : Y \to Z\) are functions such that \(g \circ f\) is surjective.
  Then we have
  \begin{align*}
             & \forall z \in Z, \exists x \in X : z = (g \circ f)(x) = g(f(x)) &  & \by{i:3.3.10,i:3.3.17} \\
    \implies & \forall z \in Z, \exists f(x) \in Y : z = g(f(x))               &  & \by{i:3.3.1}           \\
    \implies & g \text{ is surjective}.                                        &  & \by{i:3.3.17}
  \end{align*}
  And we don't need \(f\) to be surjective to finish the proof.
\end{proof}

\begin{ex}\label{i:ex:3.3.6}
  Let \(f : X \to Y\) be a bijective function, and let \(f^{-1} : Y \to X\) be its inverse.
  Verify the cancellation laws \(f^{-1}(f(x)) = x\) for all \(x \in X\) and \(f\pa{f^{-1}(y)} = y\) for all \(y \in Y\).
  Conclude that \(f^{-1}\) is also invertible, and has \(f\) as its inverse (thus \(\pa{f^{-1}}^{-1} = f\)).
\end{ex}

\begin{proof}[\pf{i:ex:3.3.6}]
  We first show that \(f^{-1}(f(x)) = x\) for all \(x \in X\).
  \begin{align*}
             & \forall x \in X, \exists! y \in Y : \begin{dcases}
                                                     f(x) = y \\
                                                     f^{-1}(y) = x
                                                   \end{dcases}              &  & \by{i:3.3.20}     \\
    \implies & \forall x \in X, \exists! y \in Y : f^{-1}\pa{f(x)} = f(y) = x. &  & \by{i:ac:3.3.1}
  \end{align*}

  Next we show that \(f\pa{f^{-1}(y)} = y\) for all \(y \in Y\).
  \begin{align*}
             & \forall y \in Y, \exists! x \in X : \begin{dcases}
                                                     f^{-1}(y) = x \\
                                                     f(x) = y
                                                   \end{dcases}              &  & \by{i:3.3.20}     \\
    \implies & \forall y \in Y, \exists! x \in X : f\pa{f^{-1}(y)} = f(x) = y. &  & \by{i:ac:3.3.1}
  \end{align*}

  Next we show that \(f^{-1}\) is bijective.
  Since
  \begin{align*}
             & \forall y, y' \in Y, y \neq y'                         \\
    \implies & \begin{dcases}
                 \exists! x \in X : f(x) = y    \\
                 \exists! x' \in X : f(x') = y' \\
                 x \neq x'
               \end{dcases}   &  & \by{i:3.3.20}                      \\
    \implies & x = f^{-1}(y) \neq f^{-1}(y') = x', &  & \by{i:3.3.20}
  \end{align*}
  we know that \(f^{-1}\) is injective by \cref{i:3.3.14}.
  Since
  \begin{align*}
             & \forall x \in X, \exists! y \in Y : f(x) = y      &  & \by{i:3.3.1}  \\
    \implies & \forall x \in X, \exists y \in Y : f^{-1}(y) = x, &  & \by{i:3.3.20}
  \end{align*}
  we know that \(f^{-1}\) is surjective by \cref{i:3.3.17}.
  Since \(f^{-1}\) is both injective and surjective, we know that \(f^{-1}\) is bijective by \cref{i:3.3.20}.

  Finally we show that \(\pa{f^{-1}}^{-1} = f\).
  Clearly, \(\pa{f^{-1}}^{-1} : X \to Y\) has the same domain and codomain as \(f\).
  Since
  \begin{align*}
             & \forall x \in X, \exists! y \in Y : \begin{dcases}
                                                     f(x) = y      \\
                                                     f^{-1}(y) = x \\
                                                     \pa{f^{-1}}^{-1}(x) = y
                                                   \end{dcases} &  & \by{i:3.3.20} \\
    \implies & \forall x \in X, f(x) = \pa{f^{-1}}^{-1}(x),
  \end{align*}
  we know that \(f = \pa{f^{-1}}^{-1}\) by \cref{i:3.3.7}.
\end{proof}

\begin{ex}\label{i:ex:3.3.7}
  Let \(f : X \to Y\) and \(g : Y \to Z\) be functions.
  Show that if \(f\) and \(g\) are bijective, then so is \(g \circ f\), and we have \((g \circ f)^{-1} = f^{-1} \circ g^{-1}\).
\end{ex}

\begin{proof}[\pf{i:ex:3.3.7}]
  Let \(f : X \to Y, g : Y \to Z\) be bijectives.
  First we show that \(g \circ f\) is bijective.
  This is true since
  \begin{align*}
             & f, g \text{ are bijective}                                                                    \\
    \implies & (f, g \text{ are injective}) \land (f, g \text{ are surjective})         &  & \by{i:3.3.20}   \\
    \implies & (g \circ f \text{ is injective}) \land (g \circ f \text{ is surjective}) &  & \by{i:ex:3.3.2} \\
    \implies & g \circ f \text{ is bijective}.                                          &  & \by{i:3.3.20}
  \end{align*}

  Now we show that \((g \circ f)^{-1} = f^{-1} \circ g^{-1}\).
  From the proof above we know that \(g \circ f\) is bijective, so \((g \circ f)^{-1}\) is well-defined.
  Since \(g \circ f\) has domain \(X\) and codomain \(Z\), by \cref{i:3.3.20}, we know that \((g \circ f)^{-1}\) has domain \(Z\) and codomain \(X\).
  By \cref{i:3.3.20}, we know that \(g^{-1} : Z \to Y\) and \(f^{-1} : Y \to X\) are well-defined.
  Thus, by \cref{i:3.3.10}, we know that \(f^{-1} \circ g^{-1} : Z \to X\) is well-defined.
  Since \((g \circ f)^{-1}\) and \(f^{-1} \circ g^{-1}\) both have the same domain and codomain, by \cref{i:3.3.7}, we only need to show that \((g \circ f)^{-1}(z) = (f^{-1} \circ g^{-1})(z)\) for every \(z \in Z\).
  Since
  \begin{align*}
    \forall x \in X, \pa{\pa{f^{-1} \circ g^{-1}} \circ (g \circ f)}(x) & = \pa{f^{-1} \circ \pa{g^{-1} \circ \pa{g \circ f}}}(x) &  & \by{i:3.3.12}   \\
                                                                        & = f^{-1}\pa{g^{-1}\pa{g(f(x))}}                         &  & \by{i:3.3.10}   \\
                                                                        & = f^{-1}(f(x))                                          &  & \by{i:ex:3.3.6} \\
                                                                        & = x,                                                    &  & \by{i:ex:3.3.6}
  \end{align*}
  we see that
  \begin{align*}
    \forall z \in Z, & \pa{f^{-1} \circ g^{-1}}(z)                                                                                         \\
                     & = \pa{f^{-1} \circ g^{-1}}\pa{\pa{(g \circ f) \circ (g \circ f)^{-1}}(z)}        &  & \by{i:ex:3.3.6}               \\
                     & = \pa{\pa{\pa{f^{-1} \circ g^{-1}} \circ (g \circ f)} \circ (g \circ f)^{-1}}(z) &  & \by{i:3.3.12}                 \\
                     & = \pa{\pa{f^{-1} \circ g^{-1}} \circ (g \circ f)}\pa{(g \circ f)^{-1}(z)}        &  & \by{i:3.3.10}                 \\
                     & = (g \circ f)^{-1}(z).                                                           &  & \text{(from the proof above)}
  \end{align*}
  Thus, by \cref{i:3.3.7}, we conclude that \((g \circ f)^{-1} = f^{-1} \circ g^{-1}\).
\end{proof}

\begin{ex}\label{i:ex:3.3.8}
  If \(X\) is a subset of \(Y\), let \(\iota_{X \to Y} : X \to Y\) be the \emph{inclusive map from \(X\) to \(Y\)}, defined by mapping \(x \mapsto x\) for all \(x \in X\), i.e., \(\iota_{X \to Y}(x) \coloneqq x\) for all \(x \in X\).
  The map \(\iota_{X \to X}\) is, in particular, called the \emph{identity map} on \(X\).
  \begin{enumerate}
    \item Show that if \(X \subseteq Y \subseteq Z\) then \(\iota_{Y \to Z} \circ \iota_{X \to Y} = \iota_{X \to Z}\).
    \item Show that if \(f : A \to B\) is any function, then \(f = f \circ \iota_{A \to A} = \iota_{B \to B} \circ f\).
    \item Show that if \(f : A \to B\) is a bijective function, then \(f \circ f^{-1} = \iota_{B \to B}\) and \(f^{-1} \circ f = \iota_{A \to A}\).
    \item Show that if \(X\) and \(Y\) are disjoint sets, and \(f : X \to Z\) and \(g : Y \to Z\) are functions, then there is a unique function \(h : X \cup Y \to Z\) such that \(h \circ \iota_{X \to X \cup Y} = f\) and \(h \circ \iota_{Y \to X \cup Y} = g\).
  \end{enumerate}
\end{ex}

\begin{proof}[\pf{i:ex:3.3.8}(a)]
  If \(X, Y, Z\) are sets such that \(X \subseteq Y \subseteq Z\), then we have
  \begin{align*}
    \forall x \in X, (\iota_{Y \to Z} \circ \iota_{X \to Y})(x) & = \iota_{Y \to Z}(\iota_{X \to Y}(x)) &  & \by{i:3.3.10}   \\
                                                                & = \iota_{Y \to Z}(x)                  &  & \by{i:ex:3.3.8} \\
                                                                & = x                                   &  & \by{i:ex:3.3.8} \\
                                                                & = \iota_{X \to Z}(x).                 &  & \by{i:ex:3.3.8}
  \end{align*}
  Thus, by \cref{i:3.3.7,i:3.3.10}, we have \(\iota_{Y \to Z} \circ \iota_{X \to Y} = \iota_{X \to Z}\).
\end{proof}

\begin{proof}[\pf{i:ex:3.3.8}(b)]
  Let \(f : A \to B\) be a function.
  Then we have
  \begin{align*}
    \forall a \in A, f(a) & = \begin{dcases}
                                f(\iota_{A \to A}(a)) \\
                                \iota_{B \to B}(f(a))
                              \end{dcases}        &  & \by{i:ex:3.3.8} \\
                          & = \begin{dcases}
                                (f \circ \iota_{A \to A})(a) \\
                                (\iota_{B \to B} \circ f)(a)
                              \end{dcases}. &  & \by{i:3.3.10}
  \end{align*}
  Thus, by \cref{i:3.3.7,i:3.3.10}, we have \(f = f \circ \iota_{A \to A} = \iota_{B \to B} \circ f\).
\end{proof}

\begin{proof}[\pf{i:ex:3.3.8}(c)]
  Suppose \(f : A \to B\) is bijective.
  Then we have
  \begin{align*}
    \forall a \in A, a & = f^{-1}(f(a))        &  & \by{i:ex:3.3.6} \\
                       & = (f^{-1} \circ f)(a) &  & \by{i:3.3.10}   \\
                       & = \iota_{A \to A}(a). &  & \by{i:ex:3.3.8} \\
    \forall b \in B, b & = f(f^{-1}(b))        &  & \by{i:ex:3.3.6} \\
                       & = (f \circ f^{-1})(b) &  & \by{i:3.3.10}   \\
                       & = \iota_{B \to B}(b). &  & \by{i:ex:3.3.8}
  \end{align*}
  Thus, by \cref{i:3.3.7,i:3.3.10}, we have \(f^{-1} \circ f = \iota_{A \to A}\) and \(f \circ f^{-1} = \iota_{B \to B}\).
\end{proof}

\begin{proof}[\pf{i:ex:3.3.8}(d)]
  Suppose that \(X, Y, Z\) are sets such that \(X \cap Y = \emptyset\).
  Let \(f : X \to Z, g : Y \to Z\) be functions.
  We now define a function \(h : X \cup Y \to Z\) as follow:
  \[
    \forall w \in X \cup Y, h(w) = \begin{dcases}
      f(w) & \text{ if } w \in X \\
      g(w) & \text{ if } w \in Y
    \end{dcases}.
  \]
  This function is well-defined since \(X \cap Y = \emptyset\).
  Thus, each \(w \in X \cup Y\) can either be in \(X\) or \(Y\) but not both.
  Then we have
  \begin{align*}
    \forall w \in X, h(w) & = h(\iota_{X \to X \cup Y}(w))        &  & \by{i:ex:3.3.8}                     \\
                          & = (h \circ \iota_{X \to X \cup Y})(w) &  & \by{i:3.3.10}                       \\
                          & = f(w).                               &  & \text{(by the definition of \(h\))} \\
    \forall w \in Y, h(w) & = h(\iota_{Y \to X \cup Y}(w))        &  & \by{i:ex:3.3.8}                     \\
                          & = (h \circ \iota_{Y \to X \cup Y})(w) &  & \by{i:3.3.10}                       \\
                          & = g(w).                               &  & \text{(by the definition of \(h\))}
  \end{align*}
  Thus, by \cref{i:3.3.7,i:3.3.10}, we have \(h \circ \iota_{X \to X \cup Y} = f\) and \(h \circ \iota_{Y \to X \cup Y} = g\).

  Now suppose there exists another function \(h' : X \cup Y \to Z\) such that \(h' \circ \iota_{X \to X \cup Y} = f\) and \(h' \circ \iota_{Y \to X \cup Y} = g\).
  Then we have
  \begin{align*}
    \forall x \in X, f(x)  & = (h' \circ \iota_{X \to X \cup Y})(x)                    \\
                           & = h'(\iota_{X \to X \cup Y}(x))        &  & \by{i:3.3.10} \\
                           & = h'(x)                                                   \\
                           & = (h \circ \iota_{X \to X \cup Y})(x)                     \\
                           & = h(\iota_{X \to X \cup Y}(x))         &  & \by{i:3.3.10} \\
                           & = h(x).                                                   \\
    \forall y \in Y : g(y) & = (h' \circ \iota_{Y \to X \cup Y})(y)                    \\
                           & = h'(\iota_{Y \to X \cup Y}(y))        &  & \by{i:3.3.10} \\
                           & = h'(y)                                                   \\
                           & = (h \circ \iota_{Y \to X \cup Y})(y)                     \\
                           & = h(\iota_{Y \to X \cup Y}(y))         &  & \by{i:3.3.10} \\
                           & = h(y).
  \end{align*}
  Thus, by \cref{i:3.3.7}, we have \(h = h'\), so \(h\) is unique.
\end{proof}

\section{Images and inverse images}\label{sec:3.4}

\begin{defn}[Images of sets]\label{3.4.1}
  If \(f : X \to Y\) is a function from \(X\) to \(Y\), and \(S\) is a set in \(X\), we define \(f(S)\) to be the set
  \[
    f(S) \coloneqq \{f(x) : x \in S\};
  \]
  this set is a subset of \(Y\), and is sometimes called the \emph{image} of \(S\) under the map \(f\).
  We sometimes call \(f(S)\) the \emph{forward image} of \(S\) to distinguish it from the concept of the \emph{inverse image} \(f^{-1}(S)\) of \(S\).
\end{defn}

\setcounter{thm}{3}
\begin{defn}[Inverse images]\label{3.4.4}
  If \(U\) is a subset of \(Y\), we define the set \(f^{-1}(U)\) to be the set
  \[
    f^{-1}(U) \coloneqq \{x \in X : f(x) \in U\}.
  \]
  In other words, \(f^{-1}(U)\) consists of all the elements of \(X\) which map into \(U\):
  \[
    f(x) \in U \iff x \in f^{-1}(U).
  \]
  We call \(f^{-1}(U)\) the \emph{inverse image} of \(U\).
\end{defn}

\setcounter{thm}{6}
\begin{rmk}\label{3.4.6}
  If \(f\) is a bijective function, then we have defined \(f^{-1}\) in two slightly different ways, but this is not an issue because both definitions are equivalent.
\end{rmk}

\begin{ax}[Power set axiom]\label{3.10}
  Let \(X\) and \(Y\) be sets.
  Then there exists a set, denoted \(Y^X\), which consists of all the functions from \(X\) to \(Y\), thus
  \[
    f \in Y^X \iff (f \text{ is a function with domain } X \text{ and range } Y).
  \]
\end{ax}

\begin{note}
  The reason we use the notation \(Y^X\) to denote this set is that if \(Y\) has \(n\) elements and \(X\) has \(m\) elements, then one can show that \(Y^X\) has \(n^m\) elements.
\end{note}

\setcounter{thm}{8}
\begin{lem}\label{3.4.9}
  Let \(X\) be a set.
  Then the set
  \[
    \{Y : Y \text{ is a subset of } X\}
  \]
  is a set.
\end{lem}

\begin{proof}
  Suppose that \(X\) is a set.
  By \cref{3.10}, there exists a set \(\{0, 1\}^X\) which consists of all the functions from \(X\) to \(\{0, 1\}\).
  \[
    f \in \{0, 1\}^X \iff (f \text{ is a function with domain } X \text{ and range } \{0, 1\}).
  \]
  By \cref{3.6}, we can replace each \(f \in \{0, 1\}^X\) with \(f^{-1}(\{1\})\), i.e., there exists a set
  \[
    S = \{f^{-1}(\{1\}) : f \in \{0, 1\}^X\}.
  \]
  By \cref{3.4.4} we have
  \begin{align*}
             & \forall Y \in S                                                              \\
    \implies & \exists\ f \in \{0, 1\}^X : Y = f^{-1}(\{1\}) = \{x \in X : f(x) \in \{1\}\} \\
    \implies & Y \subseteq X.
  \end{align*}
  Now \(\forall Y' \subseteq X\) we can define the following sets:
  \begin{align*}
    A_0 & = \{0 : x \in X \setminus Y'\}. & \text{(by \cref{3.1.27} and \cref{3.6})} \\
    A_1 & = \{1 : x \in Y'\}.             & \text{(by \cref{3.6})}
  \end{align*}
  So we have
  \begin{align*}
             & \forall Y' \subseteq X                                     \\
    \implies & \exists\ A_0, A_1                                          \\
    \implies & \exists\ f : X \to A_0 \cup A_1 & \text{(by \cref{3.6})}   \\
    \implies & f \in \{0, 1\}^X                & \text{(by \cref{3.10})}  \\
    \implies & f^{-1}(A_1) = Y'                & \text{(by \cref{3.4.4})} \\
    \implies & Y' \in S.
  \end{align*}
  Since \(\forall Y : Y \in S \iff Y \subseteq X\), we have show that \(S = \{Y : Y \subseteq X\}\) exists.
\end{proof}

\begin{rmk}\label{3.4.10}
  The set \(\{Y : Y \text{ is a subset of } X\}\) is know as the \emph{power set} of \(X\) and is denoted \(2^X\).
\end{rmk}

\begin{ax}[Union]\label{3.11}
  Let \(A\) be a set, all of whose elements are themselves sets.
  Then there exists a set \(\bigcup A\) whose elements are precisely those objects which are elements of the elements of \(A\), thus for all objects \(x\)
  \[
    x \in \bigcup A \iff (x \in S \text{ for some } S \in A)
  \]
\end{ax}

\begin{note}
  The axiom of union (\cref{3.11}), combined with the axiom of pair set (\cref{3.3}), implies the axiom of pairwise union (\cref{3.4}).
  Another important consequence of \cref{3.11} is that if one has some set \(I\), and for every element \(\alpha \in I\) we have some set \(A_{\alpha}\), then we can form the union set \(\bigcup_{\alpha \in I} A_{\alpha}\) by defining
  \[
    \bigcup_{\alpha \in I} A_{\alpha} \coloneqq \bigcup \{A_{\alpha} : \alpha \in I\},
  \]
  which is a set thanks to the axiom of replacement (\cref{3.6}) and the axiom of union (\cref{3.11}).
  More generally, we see that for any object \(y\),
  \[
    y \in \bigcup_{\alpha \in I} A_{\alpha} \iff (y \in A_{\alpha} \text{ for some } \alpha \in I).
  \]
  In situations like this, we often refer to \(I\) as an \emph{index set}, and the elements \(\alpha\) of this index set as \emph{labels};
  the sets \(A_{\alpha}\) are then called a \emph{family of sets}, and are \emph{indexed} by the labels \(\alpha \in I\).
  Note that if \(I\) was empty, then \(\bigcup_{\alpha \in I} A_{\alpha}\) would automatically also be empty.
\end{note}

\begin{note}
  We can similarly form intersections of families of sets, as long as the index set is non-empty.
  More specifically, given any non-empty set \(I\), and given an assignment of a set \(A_{\alpha}\) to each \(\alpha \in I\), we can define the intersection \(\bigcap_{\alpha \in I} A_{\alpha}\) by first choosing some element \(\beta\) of \(I\) (which we can do since \(I\) is non-empty), and setting
  \[
    \bigcap_{\alpha \in I} A_{\alpha} \coloneqq \{x \in A_{\beta} : x \in A_{\alpha} \text{ for all } \alpha \in I\},
  \]
  which is a set by the axiom of specification (\cref{3.5}).
  This definition may look like it depends on the choice of \(\beta\), but it does not.
  Observe that for any object \(y\),
  \[
    y \in \bigcap_{\alpha \in I} A_{\alpha} \iff (y \in A_{\alpha} \text{ for all } \alpha \in I).
  \]
\end{note}

\setcounter{thm}{11}
\begin{rmk}\label{3.4.12}
  The axioms of set theory that we have introduced (\crefrange{3.1}{3.11}, excluding the dangerous \cref{3.8}) are known as the \emph{Zermelo-Fraenkel axioms of set theory}, after Ernst Zermelo (1871 -- 1953) and Abraham Fraenkel (1891 -- 1965).
  There is one further axiom we will eventually need, the famous \emph{axiom of choice}, giving rise to the \emph{Zermelo-Fraenkel-Choice (ZFC) axioms of set theory}, but we will not need this axiom for some time.
\end{rmk}

\exercisesection

\begin{ex}\label{ex:3.4.1}
  Let \(f : X \to Y\) be a bijective function, and let \(f^{-1} : Y \to X\) be its inverse.
  Let \(V\) be any subset of \(Y\).
  Prove that the forward image of \(V\) under \(f^{-1}\) is the same set as the inverse image of \(V\) under \(f\);
  thus the fact that both sets are denoted by \(f^{-1}(V)\) will not lead to any inconsistency.
\end{ex}

\begin{proof}
  Suppose that \(X, Y, V\) are sets and \(f : X \to Y\) is a function such that \(V \subseteq Y\) and \(f\) is bijective.
  Let \(f^{-1} : Y \to X\) be the inverse of \(f\).
  Let \(A\) be the set of the forward image of \(V\) under \(f^{-1}\).
  Let \(B\) be the set of the inverse image of \(V\) under \(f\).
  Then we have
  \begin{align*}
    \forall x \in A \iff & \exists\ v \in V : f^{-1}(v) = x & \text{(by \cref{3.4.1})}  \\
    \iff                 & f(f^{-1}(v)) = v = f(x)          & \text{(by \cref{3.3.20})} \\
    \iff                 & x \in B.                         & \text{(by \cref{3.4.4})}
  \end{align*}
  Thus by \cref{3.1.4} we have \(A = B\).
\end{proof}

\begin{ex}\label{ex:3.4.2}
  Let \(f : X \to Y\) be a function from one set \(X\) to another set \(Y\), let \(S\) be a subset of \(X\), and let \(U\) be a subset of \(Y\).
  What, in general, can one say about \(f^{-1}(f(S))\) and \(S\)?
  What about \(f(f^{-1}(U))\) and \(U\)?
\end{ex}

\begin{proof}
  We first show that \(S \subseteq f^{-1}(f(S))\).
  Suppose that \(X, Y, S\) are sets such that \(S \subseteq X\) and \(f : X \to Y\) is a function.
  Then we have
  \begin{align*}
    \forall x \in S \implies & f(x) \in f(S)       & \text{(by \cref{3.4.1})} \\
    \implies                 & x \in f^{-1}(f(S)). & \text{(by \cref{3.4.4})}
  \end{align*}
  Thus by \cref{3.1.15} we have \(S \subseteq f^{-1}(f(S))\).

  Now we show that \(f(f^{-1}(U)) \subseteq U\).
  Suppose that \(X, Y, U\) are sets such that \(U \subseteq Y\) and \(f : X \to Y\) is a function.
  Then we have
  \begin{align*}
    \forall y \in f(f^{-1}(U)) \implies & \exists\ x \in f^{-1}(U) : f(x) = y & \text{(by \cref{3.4.1})} \\
    \implies                            & y \in U.                            & \text{(by \cref{3.4.4})}
  \end{align*}
  Thus by \cref{3.1.15} we have \(f(f^{-1}(U)) \subseteq U\).
\end{proof}

\begin{ex}\label{ex:3.4.3}
  Let \(A\), \(B\) be two subsets of a set \(X\), and let \(f : X \to Y\) be a function.
  Show that \(f(A \cap B) \subseteq f(A) \cap f(B)\), that \(f(A) \setminus f(B) \subseteq f(A \setminus B)\), \(f(A \cup B) = f(A) \cup f(B)\).
  For the first two statements, is it true that the \(\subseteq\) relation can be imporved to \(=\)?
\end{ex}

\begin{proof}
  We first show that \(f(A \cap B) \subseteq f(A) \cap f(B)\).
  Suppose that \(A, B, X, Y\) are sets such that \(A \subseteq X\) and \(B \subseteq X\).
  Suppose that \(f : X \to Y\) is a function.
  Then we have
  \begin{align*}
             & \forall y \in f(A \cap B)                                                           \\
    \implies & y \in \{f(x) : x \in A \cap B\}                         & \text{(by \cref{3.4.1})}  \\
    \implies & y \in \{f(x) : x \in A \land x \in B\}                  & \text{(by \cref{3.1.23})} \\
    \implies & y \in \{f(x) : x \in A\} \land y \in \{f(x) : x \in B\}                             \\
    \implies & y \in f(A) \land y \in f(B)                             & \text{(by \cref{3.4.1})}  \\
    \implies & y \in f(A) \cap f(B).                                   & \text{(by \cref{3.1.23})}
  \end{align*}
  Thus by \cref{3.1.15} we have \(f(A \cap B) \subseteq f(A) \cap f(B)\).

  Next we show that \(f(A) \setminus f(B) \subseteq f(A \setminus B)\).
  Suppose that \(A, B, X, Y\) are sets such that \(A \subseteq X\) and \(B \subseteq X\).
  Suppose that \(f : X \to Y\) is a function.
  Then we have
  \begin{align*}
             & \forall y \in f(A) \setminus f(B)                                                      \\
    \implies & y \in f(A) \land y \notin f(B)                             & \text{(by \cref{3.1.27})} \\
    \implies & y \in \{f(x) : x \in A\} \land y \notin \{f(x) : x \in B\} & \text{(by \cref{3.4.1})}  \\
    \implies & y \in \{f(x) : x \in A \land x \notin B\}                                              \\
    \implies & y \in \{f(x) : x \in A \setminus B\}                       & \text{(by \cref{3.1.27})} \\
    \implies & y \in f(A \setminus B).                                    & \text{(by \cref{3.4.1})}
  \end{align*}
  Thus by \cref{3.1.15} we have \(f(A) \setminus f(B) \subseteq f(A \setminus B)\).

  Finally we show that \(f(A \cup B) = f(A) \cup f(B)\).
  Suppose that \(A, B, X, Y\) are sets such that \(A \subseteq X\) and \(B \subseteq X\).
  Suppose that \(f : X \to Y\) is a function.
  Then we have
  \begin{align*}
         & \forall y \in f(A \cup B)                                                         \\
    \iff & y \in \{f(x) : x \in A \cup B\}                        & \text{(by \cref{3.4.1})} \\
    \iff & y \in \{f(x) : x \in A \lor x \in B\}                  & \text{(by \cref{3.4})}   \\
    \iff & y \in \{f(x) : x \in A\} \lor y \in \{f(x) : x \in B\}                            \\
    \iff & y \in f(A) \lor y \in f(B)                             & \text{(by \cref{3.4.1})} \\
    \iff & y \in f(A) \cup f(B).                                  & \text{(by \cref{3.4})}
  \end{align*}
  Thus by \cref{3.1.4} we have \(f(A \cup B) = f(A) \cup f(B)\).
\end{proof}

\begin{ex}\label{ex:3.4.4}
  Let \(f : X \to Y\) be a function from one set \(X\) to another set \(Y\), and let \(U\), \(V\) be subsets of \(Y\). Show that \(f^{-1}(U \cup V) = f^{-1}(U) \cup f^{-1}(V)\), that
  \(f^{-1}(U \cap V) = f^{-1}(U) \cap f^{-1}(V)\), and that \(f^{-1}(U \setminus V) = f^{-1}(U) \setminus f^{-1}(V)\).
\end{ex}

\begin{proof}
  We first show that \(f^{-1}(U \cup V) = f^{-1}(U) \cup f^{-1}(V)\).
  Suppose that \(U, V, X, Y\) are sets such that \(U \subseteq Y\) and \(V \subseteq Y\).
  Suppose that \(f : X \to Y\) is a function.
  Then we have
  \begin{align*}
         & \forall x \in f^{-1}(U \cup V)                                  \\
    \iff & f(x) \in U \cup V                    & \text{(by \cref{3.4.4})} \\
    \iff & f(x) \in U \lor f(x) \in V           & \text{(by \cref{3.4})}   \\
    \iff & x \in f^{-1}(U) \lor x \in f^{-1}(V) & \text{(by \cref{3.4.4})} \\
    \iff & x \in f^{-1}(U) \cup f^{-1}(V).      & \text{(by \cref{3.4})}
  \end{align*}
  Thus by \cref{3.1.4} we have \(f^{-1}(U \cup V) = f^{-1}(U) \cup f^{-1}(V)\).

  Next we show that \(f^{-1}(U \cap V) = f^{-1}(U) \cap f^{-1}(V)\).
  Suppose that \(U, V, X, Y\) are sets such that \(U \subseteq Y\) and \(V \subseteq Y\).
  Suppose that \(f : X \to Y\) is a function.
  Then we have
  \begin{align*}
         & \forall x \in f^{-1}(U \cap V)                                    \\
    \iff & f(x) \in U \cap V                     & \text{(by \cref{3.4.4})}  \\
    \iff & f(x) \in U \land f(x) \in V           & \text{(by \cref{3.1.23})} \\
    \iff & x \in f^{-1}(U) \land x \in f^{-1}(V) & \text{(by \cref{3.4.4})}  \\
    \iff & x \in f^{-1}(U) \cap f^{-1}(V).       & \text{(by \cref{3.1.23})}
  \end{align*}
  Thus by \cref{3.1.4} we have \(f^{-1}(U \cap V) = f^{-1}(U) \cap f^{-1}(V)\).

  Finally we show that \(f^{-1}(U \setminus V) = f^{-1}(U) \setminus f^{-1}(V)\).
  Suppose that \(U, V, X, Y\) are sets such that \(U \subseteq Y\) and \(V \subseteq Y\).
  Suppose that \(f : X \to Y\) is a function.
  Then we have
  \begin{align*}
         & \forall x \in f^{-1}(U \setminus V)                                  \\
    \iff & f(x) \in U \setminus V                   & \text{(by \cref{3.4.4})}  \\
    \iff & f(x) \in U \land f(x) \notin V           & \text{(by \cref{3.1.23})} \\
    \iff & x \in f^{-1}(U) \land x \notin f^{-1}(V) & \text{(by \cref{3.4.4})}  \\
    \iff & x \in f^{-1}(U) \setminus f^{-1}(V).     & \text{(by \cref{3.1.23})}
  \end{align*}
  Thus by \cref{3.1.4} we have \(f^{-1}(U \setminus V) = f^{-1}(U) \setminus f^{-1}(V)\).
\end{proof}

\begin{ex}\label{ex:3.4.5}
  Let \(f : X \to Y\) be a function from one set \(X\) to another set \(Y\).
  Show that \(f(f^{-1}(S)) = S\) for every \(S \subseteq Y\) if and only if \(f\) is surjective.
  Show that \(f^{-1}(f(S)) = S\) for every \(S \subseteq X\) if and only if \(f\) is injective.
\end{ex}

\begin{proof}
  We first show that \(\forall S \subseteq Y : f(f^{-1}(S)) = S \iff f\) is surjective.
  Suppose that \(X, Y, S\) are sets such that \(S \subseteq Y\) and \(f : X \to Y\) is a function.
  Then we have
  \begin{align*}
         & f \text{ is surjective}                                                                                         \\
    \iff & (\forall S \subseteq Y : y \in S \implies \exists\ x \in X : f(x) = y)            & \text{(by \cref{3.3.17})}   \\
    \iff & (\forall S \subseteq Y : y \in S \implies \exists\ x \in f^{-1}(S) : f(x) = y)    & \text{(by \cref{3.4.4})}    \\
    \iff & (\forall S \subseteq Y : y \in S \implies y \in f(f^{-1}(S)))                     & \text{(by \cref{3.4.1})}    \\
    \iff & (\forall S \subseteq Y : S \subseteq f(f^{-1}(S)))                                & \text{(by \cref{3.1.15})}   \\
    \iff & (\forall S \subseteq Y : S \subseteq f(f^{-1}(S)) \land f(f^{-1}(S)) \subseteq S) & \text{(by \cref{ex:3.4.2})} \\
    \iff & (\forall S \subseteq Y : S = f(f^{-1}(S))).                                       & \text{(by \cref{3.1.18})}
  \end{align*}

  Now we show that \(\forall S \subseteq X : f^{-1}(f(S)) = S \iff f\) is injective.
  Suppose that \(X, Y, S\) are sets such that \(S \subseteq X\) and \(f : X \to Y\) is a function.
  If \(f\) is injective, then \(\forall S \subseteq X\) we have
  \begin{align*}
             & x \in f^{-1}(f(S))                                                             \\
    \implies & f(x) \in f(S)                                      & \text{(by \cref{3.4.4})}  \\
    \implies & \exists\ x' \in S : (f(x) = f(x') \implies x = x') & \text{(by \cref{3.3.14})} \\
    \implies & x \in S.
  \end{align*}
  Thus \(f^{-1}(f(S)) \subseteq S\).
  By \cref{ex:3.4.2} we have \(S \subseteq f^{-1}(f(S))\), thus by \cref{3.1.18} we have \(S = f^{-1}(f(S))\).
  On the other hand, if \(\forall S \subseteq X : f^{-1}(f(S)) = S\), then we have
  \begin{align*}
             & \forall x, x' \in S : f(x) = f(x')                                              \\
    \implies & x \in f^{-1}(f(\{x\})) = f^{-1}(f(\{x'\})) = \{x'\}                             \\
    \implies & x = x'                                                                          \\
    \implies & f \text{ is injective}.                             & \text{(by \cref{3.3.14})}
  \end{align*}
  Thus we conclude that \(\forall S \subseteq X : f^{-1}(f(S)) = S \iff f\) is injective.
\end{proof}

\begin{ex}\label{ex:3.4.6}
  Prove \cref{3.4.9}.
\end{ex}

\begin{proof}
  See \cref{3.4.9}.
\end{proof}

\begin{ex}\label{ex:3.4.7}
  Let \(X\), \(Y\) be sets.
  Define a \emph{partial function} from \(X\) to \(Y\) to be any function \(f : X' \to Y'\) whose domain \(X'\) is a subset of \(X\), and whose range \(Y'\) is a subset of \(Y\).
  Show that the collection of all partial functions from \(X\) to \(Y\) is itself a set.
\end{ex}

\begin{proof}
  Suppose that \(X, Y\) are sets.
  Then by \cref{3.4.9}, the set \(A = \{X' : X' \subseteq X\}\) exists, so does \(B = \{Y' : Y' \subseteq Y\}\).
  Now we have
  \begin{align*}
    C_1 & = \{Y'^{X'} : X' \in A \land Y' \in B\}             & \text{(by \cref{3.10})} \\
    C_2 & = \bigcup C_1 = \{f \in Y'^{X'} : Y'^{X'} \in C_1\} & \text{(by \cref{3.11})} \\
  \end{align*}
  If \(f : X' \to Y'\) is a partial function whose domain \(X' \subseteq X\) and whose range \(Y' \subseteq Y\), then we have \(Y'^{X'} \in C_1\), and thus \(f \in C_2\).
\end{proof}

\begin{ex}\label{ex:3.4.8}
  Show that \cref{3.4} can be deduced from \cref{3.1}, \cref{3.3} and \cref{3.11}.
\end{ex}

\begin{proof}
  By \cref{3.1}, \(A\) is a set and \(B\) is a set.
  And if \(x\) is a object, we can say \(x \in A\) or \(x \in B\).
  By \cref{3.3}, there exists a set \(\{A, B\}\) whose only elements are \(A\) and \(B\).
  By \cref{3.11}, \(x \in \bigcup \{A, B\} \iff x \in A \lor x \in B\).
  By defining \(A \cup B \coloneqq \bigcup \{A, B\}\), we show that \(x \in A \cup B \iff x \in A \lor x \in B\) is true.
\end{proof}

\begin{ex}\label{ex:3.4.9}
  Show that if \(\beta\) and \(\beta'\) are two elements of a set \(I\), and to each \(\alpha \in I\) we assign a set \(A_{\alpha}\), then
  \[
    \{x \in A_{\beta} : x \in A_{\alpha} \ \forall \alpha \in I\} = \{x \in A_{\beta'} : x \in A_{\alpha} \ \forall \alpha \in I\},
  \]
  and so the definition of \(\bigcap_{\alpha \in I} A_{\alpha}\) does not depend on \(\beta\).
\end{ex}

\begin{proof}
  Suppose that \(I\) is a set and \(\forall \alpha \in I : A_{\alpha}\) is a set.
  Let \(\beta, \beta' \in I\) and \(B, B'\) be sets
  \begin{align*}
    B  & = \{x \in A_{\beta} : x \in A_{\alpha} \ \forall \alpha \in I\}   \\
    B' & = \{x \in A_{\beta'} : x \in A_{\alpha} \ \forall \alpha \in I\}.
  \end{align*}
  We now show that \(B = B'\).
  \begin{align*}
         & \forall x \in B                                                                 \\
    \iff & x \in A_{\beta} \land x \in A_{\alpha} \ \forall \alpha \in I                   \\
    \iff & x \in A_{\alpha} \ \forall \alpha \in I                        & (\beta \in I)  \\
    \iff & x \in A_{\beta'} \land x \in A_{\alpha} \ \forall \alpha \in I & (\beta' \in I) \\
    \iff & x \in B'.
  \end{align*}
  Thus by \cref{3.1.4} we have \(B = B'\).
\end{proof}

\begin{ex}\label{ex:3.4.10}
  Suppose that \(I\) and \(J\) are two sets, and for all \(\alpha \in I \cup J\) let \(A_{\alpha}\) be a set.
  Show that \((\bigcup_{\alpha \in I} A_{\alpha}) \cup (\bigcup_{\alpha \in J} A_{\alpha}) = \bigcup_{\alpha \in I \cup J} A_{\alpha}\).
  If \(I\) and \(J\) are non-empty, show that \((\bigcap_{\alpha \in I} A_{\alpha}) \cap (\bigcap_{\alpha \in J} A_{\alpha}) = \bigcap_{\alpha \in I \cup J} A_{\alpha}\).
\end{ex}

\begin{proof}
  We first show that \((\bigcup_{\alpha \in I} A_{\alpha}) \cup (\bigcup_{\alpha \in J} A_{\alpha}) = \bigcup_{\alpha \in I \cup J} A_{\alpha}\).
  Suppose that \(I\) and \(J\) are two sets, and \(\forall \alpha \in I \cup J : A_{\alpha}\) be a set.
  Then we have
  \begin{align*}
         & \forall x \in (\bigcup_{\alpha \in I} A_{\alpha}) \cup (\bigcup_{\alpha \in J} A_{\alpha})                           \\
    \iff & x \in \bigcup_{\alpha \in I} A_{\alpha} \lor x \in \bigcup_{\alpha \in J} A_{\alpha}       & \text{(by \cref{3.4})}  \\
    \iff & (\exists\ \alpha \in I : x \in A_{\alpha}) \lor (\exists\ \alpha \in J : x \in A_{\alpha}) & \text{(by \cref{3.11})} \\
    \iff & \exists\ \alpha \in I \lor \alpha \in J : x \in A_{\alpha}                                                           \\
    \iff & \exists\ \alpha \in I \cup J : x \in A_{\alpha}                                            & \text{(by \cref{3.4})}  \\
    \iff & x \in \bigcup_{\alpha \in I \cup J} A_{\alpha}.                                            & \text{(by \cref{3.11})}
  \end{align*}
  Thus by \cref{3.1.4} we have \((\bigcup_{\alpha \in I} A_{\alpha}) \cup (\bigcup_{\alpha \in J} A_{\alpha}) = \bigcup_{\alpha \in I \cup J} A_{\alpha}\).

  Now we show that \(I \neq \emptyset \land J \neq \emptyset \implies (\bigcap_{\alpha \in I} A_{\alpha}) \cap (\bigcap_{\alpha \in J} A_{\alpha}) = \bigcap_{\alpha \in I \cup J} A_{\alpha}\).
  Suppose that \(I\) and \(J\) are two non-empty sets, and \(\forall \alpha \in I \cup J : A_{\alpha}\) be a set.
  Then we have
  \begin{align*}
         & \forall x \in (\bigcap_{\alpha \in I} A_{\alpha}) \cap (\bigcap_{\alpha \in J} A_{\alpha})                               \\
    \iff & x \in \bigcap_{\alpha \in I} A_{\alpha} \land x \in \bigcap_{\alpha \in J} A_{\alpha}      & \text{(by \cref{3.1.23})}   \\
    \iff & (\forall \alpha \in I : x \in A_{\alpha}) \land (\forall \alpha \in J : x \in A_{\alpha})  & \text{(by \cref{ex:3.4.9})} \\
    \iff & \forall \alpha \in I \lor \alpha \in J : x \in A_{\alpha}                                                                \\
    \iff & \forall \alpha \in I \cup J : x \in A_{\alpha}                                             & \text{(by \cref{3.4})}      \\
    \iff & x \in \bigcap_{\alpha \in I \cup J} A_{\alpha}.                                            & \text{(by \cref{ex:3.4.9})}
  \end{align*}
  Thus by \cref{3.1.4} we have \((\bigcap_{\alpha \in I} A_{\alpha}) \cap (\bigcap_{\alpha \in J} A_{\alpha}) = \bigcap_{\alpha \in I \cup J} A_{\alpha}\).
\end{proof}

\begin{ex}\label{ex:3.4.11}
  Let \(X\) be a set, let \(I\) be a non-empty set, and for all \(\alpha \in I\) let \(A_{\alpha}\) be a subset of \(X\).
  Show that
  \[
    X \setminus \bigcup_{\alpha \in I} A_{\alpha} = \bigcap_{\alpha \in I} (X \setminus A_{\alpha})
  \]
  and
  \[
    X \setminus \bigcap_{\alpha \in I} A_{\alpha} = \bigcup_{\alpha \in I} (X \setminus A_{\alpha}).
  \]
  This should be compared with de Morgan's laws in \cref{3.1.28}
  (although one cannot derive the above identities directly from de Morgan's laws, as \(I\) could be infinite).
\end{ex}

\begin{proof}
  We first show that \(X \setminus \bigcup_{\alpha \in I} A_{\alpha} = \bigcap_{\alpha \in I} (X \setminus A_{\alpha})\).
  Suppose that \(X, I\) are sets, \(I \neq \emptyset\), \(\forall \alpha \in I : A_{\alpha}\) is a set and \(A_{\alpha} \subseteq X\).
  Then we have
  \begin{align*}
         & \forall x \in X \setminus \bigcup_{\alpha \in I} A_{\alpha}                                 \\
    \iff & x \in X \land x \notin \bigcup_{\alpha \in I} A_{\alpha}      & \text{(by \cref{3.1.27})}   \\
    \iff & x \in X \land \lnot(\exists\ \alpha \in I : x \in A_{\alpha}) & \text{(by \cref{3.11})}     \\
    \iff & x \in X \land (\forall \alpha \in I : x \notin A_{\alpha})                                  \\
    \iff & \forall \alpha \in I : x \in X \land x \notin A_{\alpha}                                    \\
    \iff & \forall \alpha \in I : x \in X \setminus A_{\alpha}           & \text{(by \cref{3.1.27})}   \\
    \iff & x \in \bigcap_{\alpha \in I} (X \setminus A_{\alpha})         & \text{(by \cref{ex:3.4.9})} \\
  \end{align*}
  Thus by \cref{3.1.4} we have \(X \setminus \bigcup_{\alpha \in I} A_{\alpha} = \bigcap_{\alpha \in I} (X \setminus A_{\alpha})\).

  Now we show that \(X \setminus \bigcap_{\alpha \in I} A_{\alpha} = \bigcup_{\alpha \in I} (X \setminus A_{\alpha})\).
  Suppose that \(X, I\) are sets, \(I \neq \emptyset\), \(\forall \alpha \in I : A_{\alpha}\) is a set and \(A_{\alpha} \subseteq X\).
  Then we have
  \begin{align*}
         & \forall x \in X \setminus \bigcap_{\alpha \in I} A_{\alpha}                                \\
    \iff & x \in X \land x \notin \bigcap_{\alpha \in I} A_{\alpha}     & \text{(by \cref{3.1.27})}   \\
    \iff & x \in X \land \lnot(\forall \alpha \in I : x \in A_{\alpha}) & \text{(by \cref{ex:3.4.9})} \\
    \iff & x \in X \land (\exists\ \alpha \in I : x \notin A_{\alpha})                                \\
    \iff & \exists\ \alpha \in I : x \in X \land x \notin A_{\alpha}                                  \\
    \iff & \exists\ \alpha \in I : x \in X \setminus A_{\alpha}         & \text{(by \cref{3.1.27})}   \\
    \iff & x \in \bigcup_{\alpha \in I} (X \setminus A_{\alpha})        & \text{(by \cref{3.11})}     \\
  \end{align*}
  Thus by \cref{3.1.4} we have \(X \setminus \bigcap_{\alpha \in I} A_{\alpha} = \bigcup_{\alpha \in I} (X \setminus A_{\alpha})\).
\end{proof}
\section{Cartesian products}\label{i:sec:3.5}

\begin{defn}[Ordered pair]\label{i:3.5.1}
  If \(x\) and \(y\) are any objects (possibly equal), we define the \emph{ordered pair} \((x, y)\) to be a new object, consisting of \(x\) as its first component and \(y\) as its second component.
  Two ordered pairs \((x, y)\) and \((x', y')\) are considered equal iff both their components match, i.e.
  \[
    (x, y) = (x', y') \iff (x = x' \text{ and } y = y').
  \]
  This is consistent with the usual axioms of equality (\cref{i:ex:3.5.3}).
\end{defn}

\begin{rmk}\label{i:3.5.2}
  Strictly speaking, \cref{i:3.5.1} is partly an axiom, because we have simply postulated that given any two objects \(x\) and \(y\), that an object of the form \((x, y)\) exists.
  However, it is possible to define an ordered pair using the axioms of set theory in such a way that we do not need any further postulates (see \cref{i:ex:3.5.1}).
\end{rmk}

\begin{rmk}\label{i:3.5.3}
  We have now ``overloaded'' the parenthesis symbols \(()\) once again;
  they now are not only used to denote grouping of operators and arguments of functions, but also to enclose ordered pairs.
  This is usually not a problem in practice as one can still determine what usage the symbols \(()\) were intended for from context.
\end{rmk}

\begin{defn}[Cartesian product]\label{i:3.5.4}
  If \(X\) and \(Y\) are sets, then we define the \emph{Cartesian product} \(X \times Y\) to be the collection of ordered pairs, whose first component lies in \(X\) and second component lies in \(Y\), thus
  \[
    X \times Y \coloneqq \set{(x, y) : x \in X, y \in Y}
  \]
  or equivalently,
  \[
    a \in X \times Y \iff (a = (x, y) \text{ for some } x \in X \text{ and } y \in Y).
  \]
\end{defn}

\begin{rmk}\label{i:3.5.5}
  One can show that the Cartesian product \(X \times Y\) is indeed a set;
  see \cref{i:ex:3.5.1}.
\end{rmk}

\begin{note}
  Let \(f : X \times Y \to Z\) be a function whose domain \(X \times Y\) is a Cartesian product of two other sets \(X\) and \(Y\).
  Then \(f\) can either be thought of as a function of one variable, mapping the single input of an ordered pair \((x, y)\) in \(X \times Y\) to an output \(f(x, y)\) in \(Z\), or as a function of two variables, mapping an input \(x \in X\) and another input \(y \in Y\) to a single output \(f(x, y)\) in \(Z\).
  While the two notions are technically different, we will not bother to distinguish the two, and think of \(f\) simultaneously as a function of one variable with domain \(X \times Y\) and as a function of two variables with domains \(X\) and \(Y\).
  Thus, for instance the addition operation \(+\) on the natural numbers can now be re-interpreted as a function \(+ : N \times N \to N\), defined by \((x, y) \mapsto x + y\).
\end{note}

\setcounter{thm}{6}
\begin{defn}[Ordered \(n\)-tuple and \(n\)-fold Cartesian product]\label{i:3.5.7}
  Let \(n\) be a natural number.
  An \emph{ordered \(n\)-tuple} \((x_i)_{1 \leq i \leq n}\) (also denoted \((x_1, \dots, x_n)\)) is a collection of objects \(x_i\), one for every natural number \(i\) between \(1\) and \(n\);
  we refer to \(x_i\) as the \emph{\(i^{th}\) component} of the ordered \(n\)-tuple.
  Two ordered \(n\)-tuples \((x_i)_{1 \leq i \leq n}\) and \((y_i)_{1 \leq i \leq n}\) are said to be equal iff \(x_i = y_i\) for all \(1 \leq i \leq n\).
  If \((X_i)_{1 \leq i \leq n}\) is an ordered \(n\)-tuple of sets, we define their \emph{Cartesian product} \(\prod_{1 \leq i \leq n} X_i\) (also denoted \(\prod_{i=1}^n X_i\) or \(X_1 \times \dots \times X_n\)) by
  \[
    \prod_{1 \leq i \leq n} X_i \coloneqq \set{(x_i)_{1 \leq i \leq n} : x_i \in X_i \text{ for all } 1 \leq i \leq n}.
  \]
\end{defn}

\begin{note}
  Again, \cref{i:3.5.7} simply postulates that an ordered \(n\)-tuple and a Cartesian product always exist when needed, but using the axioms of set theory one can explicitly construct these objects (\cref{i:ex:3.5.2}).
\end{note}

\begin{rmk}\label{i:3.5.8}
  One can show that \(\prod_{1 \leq i \leq n} X_i\) is indeed a set.
  Indeed, from the power set axiom we can consider the set of all functions \(i \mapsto x_i\) from the domain \(\set{1 \leq i \leq n}\) to the codomain \(\bigcup_{1 \leq i \leq n} X_i\), and then we can restrict using the axiom of specification to restrict to those functions \(i \mapsto x_i\) for which \(x_i \in X_i\) for all \(1 \leq i \leq n\).
  One can generalize this construction to infinite Cartesian products, see \cref{i:8.4.1}.
\end{rmk}

\begin{note}
  Strictly speaking, the sets \(X_1 \times X_2 \times X_3\), \((X_1 \times X_2) \times X_3\), and \(X_1 \times (X_2 \times X_3)\) are distinct.
  However, they are clearly very related to each other (for instance, there are obvious bijections between any two of the three sets), and it is common in practice to neglect the minor distinctions between these sets and pretend that they are in fact equal.
  Thus, a function \(f : X_1 \times X_2 \times X_3 \to Y\) can be thought of as a function of one variable \((x_1, x_2, x_3) \in X_1 \times X_2 \times X_3\), or as a function of three variables \(x_1 \in X_1\), \(x_2 \in X_2\), \(x_3 \in X_3\), or as a function of two variables \(x_1 \in X_1\), \((x_2, x_3) \in X_2 \times X_3\), and so forth;
  we will not bother to distinguish between these different perspectives.
\end{note}

\setcounter{thm}{9}
\begin{rmk}\label{i:3.5.10}
  An ordered \(n\)-tuple \(x_1, \dots, x_n\) of objects is also called an \emph{ordered sequence} of \(n\) elements, or a \emph{finite sequence} for short.
  In \cref{i:ch:5} we shall also introduce the very useful concept of an \emph{infinite sequence}.
\end{rmk}

\begin{eg}\label{i:3.5.11}
  If \(x\) is an object, then \((x)\) is a \(1\)-tuple, which we shall identify with \(x\) itself (even though the two are, strictly speaking, not the same object).
  Then if \(X_1\) is any set, then the Cartesian product \(\prod_{1 \leq i \leq 1} X_i\) is just \(X_1\).
  Also, the \emph{empty Cartesian product} \(\prod_{1 \leq i \leq 0} X_i\) gives, not the empty set \(\set{}\), but rather the singleton set \(\set{()}\) whose only element is the \emph{\(0\)-tuple} \(()\), also known as the \emph{empty tuple}.

  If \(n\) is a natural number, we often write \(X^n\) as shorthand for the \(n\)-fold Cartesian product \(X^n \coloneqq \prod_{1 \leq i \leq n} X\).
  Thus, \(X^1\) is essentially the same set as \(X\) (if we ignore the distinction between an object \(x\) and the \(1\)-tuple \((x)\)), while \(X^2\) is the Cartesian product \(X \times X\).
  The set \(X^0\) is a singleton set \(\set{()}\).
\end{eg}

\begin{proof}[\pf{i:3.5.11}]
  First we show that \(\prod_{1 \leq i \leq 1} X_i = X_1\).
  This is true since
  \[
    \prod_{1 \leq i \leq 1} X_i = \set{(x) : x \in X_1} = \set{x : x \in X_1} = X_1.
  \]

  Next we show that \((x_i)_{1 \leq i \leq 0} = ()\).
  By \cref{i:ex:3.5.2}, we see that \(x : \set{i \in \N : 1 \leq i \leq 0} \to Y\) is a surjective function, where \(Y\) is an arbitrary set.
  Clearly, the domain of \(x\) is the empty set.
  Since the \(i^{th}\) component of \(x\) and \(()\) do not exist, we see that the statement ``for \(1 \leq i \leq n\), the \(i^{th}\) component of \(x\) and \(()\) are the same'' is vacuously true for arbitrary natural number \(n\).
  Thus, we have \((x_i)_{1 \leq i \leq 0} = ()\).

  Next we show that \(\prod_{1 \leq i \leq 0} X_i = \set{()}\).
  This is true since
  \begin{align*}
    \prod_{1 \leq i \leq 0} X_i & = \set{(x_i)_{1 \leq i \leq 0} : x_i \in X_i \text{ for all } 1 \leq i \leq 0} &  & \by{i:3.5.7}                  \\
                                & = \set{() : x_i \in X_i \text{ for all } 1 \leq i \leq 0}                      &  & \text{(from the proof above)} \\
                                & = \set{()}.                                                                    &  & \by{i:3.2}
  \end{align*}

  Next we show that \(X^1 = X\).
  This is true since
  \begin{align*}
    X^1 & = \prod_{1 \leq i \leq 1} X &  & \text{(by definition)}        \\
        & = X.                        &  & \text{(from the proof above)}
  \end{align*}

  Finally we show that \(X^0 = \set{()}\).
  This is true since
  \begin{align*}
    X^0 & = \prod_{1 \leq i \leq 0} X &  & \text{(by definition)}        \\
        & = \set{()}.                 &  & \text{(from the proof above)}
  \end{align*}
\end{proof}

\setcounter{thm}{11}
\begin{lem}[Finite choice]\label{i:3.5.12}
  Let \(n \geq 1\) be a natural number, and for each natural number \(1 \leq i \leq n\), let \(X_i\) be a non-empty set.
  Then there exists an \(n\)-tuple \((x_i)_{1 \leq i \leq n}\) such that \(x_i \in X_i\) for all \(1 \leq i \leq n\).
  In other words, if each \(X_i\) is non-empty, then the set \(\prod_{1 \leq i \leq n} X_i\) is also non-empty.
\end{lem}

\begin{proof}[\pf{i:3.5.12}]
  We induct on \(n\) (starting with the base case \(n = 1\); the claim is also vacuously true with \(n = 0\) but is not particularly interesting in that case).
  When \(n = 1\) the claim follows from \cref{i:3.1.6}.
  Now suppose inductively that the claim has already been proven for some \(n\);
  we will now prove it for \(n\pp\).
  Let \(X_1, \dots, X_{n\pp}\) be a collection of non-empty sets.
  By the induction hypothesis, we can find an \(n\)-tuple \((x_i)_{1 \leq i \leq n}\) such that \(x_i \in X_i\) for all \(1 \leq i \leq n\).
  Also, since \(X_{n\pp}\) is non-empty, by \cref{i:3.1.6} we may find an object \(a\) such that \(a \in X_{n\pp}\).
  If we thus define the \(n\pp\)-tuple \((y_i)_{1 \leq i \leq n\pp}\) by setting \(y_i \coloneqq x_i\) when \(1 \leq i \leq n\) and \(y_i \coloneqq a\) when \(i = n\pp\) it is clear that \(y_i \in X_i\) for all \(1 \leq i \leq n\pp\), thus closing the induction.
\end{proof}

\begin{rmk}\label{i:3.5.13}
  It is intuitively plausible that this lemma should be extended to allow for an infinite number of choices, but this cannot be done automatically;
  it requires an additional axiom, the \emph{axiom of choice}.
  See \cref{i:sec:8.4}.
\end{rmk}

\exercisesection

\begin{ex}\label{i:ex:3.5.1}
  \begin{enumerate}
    \item Suppose we \emph{define} the ordered pair \((x, y)\) for any objects \(x\) and \(y\) by the formula \((x, y) \coloneqq \set{\set{x}, \set{x, y}}\)
          (thus using several applications of \cref{i:3.3}).
          Show that such a definition indeed obeys the \cref{i:3.5.1}.
          Thus, this definition can be validly used as a definition of an ordered pair.
    \item For an additional challenge, show that the alternate definition \((x, y) := \set{x, \set{x, y}}\) also verifies \cref{i:3.5.1} and is thus also an acceptable definition of ordered pair.
    \item Show that regardless of the definition of ordered pair, the Cartesian product \(X \times Y\) is a set.
  \end{enumerate}
\end{ex}

\begin{proof}[\pf{i:ex:3.5.1}(a)]
  Suppose that \(x, x', y, y'\) are objects.
  By definition, we have
  \begin{align*}
    (x, y)   & = \set{\set{x}, \set{x, y}}     \\
    (x', y') & = \set{\set{x'}, \set{x', y'}}.
  \end{align*}
  Then we have
  \begin{align*}
         & \begin{dcases}
             x = x' \\
             y = y'
           \end{dcases}                                                                    \\
    \iff & \begin{dcases}
             x = x' \\
             \pa{x = y = y' = x'} \lor \pa{x \neq y = y' \neq x'}
           \end{dcases}                              \\
    \iff & \begin{dcases}
             \set{x} = \set{x'} \\
             \set{x, y} = \set{x', y'}
           \end{dcases}                                &  & \by{i:3.3}                      \\
    \iff & \set{\set{x}, \set{x, y}} = \set{\set{x'}, \set{x', y'}} &  & \by{i:3.3}         \\
    \iff & (x, y) = (x', y').                                       &  & \by{i:ex:3.5.1}[a]
  \end{align*}
  Thus, \cref{i:ex:3.5.1}(a) is a valid definition of ordered pairs.
\end{proof}

\begin{proof}[\pf{i:ex:3.5.1}(b)]
  Suppose that \(x, x', y, y'\) are objects.
  By definition, we have
  \begin{align*}
    (x, y)   & = \set{x, \set{x, y}}     \\
    (x', y') & = \set{x', \set{x', y'}}.
  \end{align*}
  Then we have
  \begin{align*}
         & \begin{dcases}
             x = x' \\
             y = y'
           \end{dcases}                                                                \\
    \iff & \begin{dcases}
             x = x' \\
             \pa{x = y = y' = x'} \lor \pa{x \neq y = y' \neq x'}
           \end{dcases}                          \\
    \iff & \begin{dcases}
             x = x' \\
             \set{x, y} = \set{x', y'}
           \end{dcases}                            &  & \by{i:3.3}                      \\
    \iff & \set{x, \set{x, y}} = \set{x', \set{x', y'}}         &  & \by{i:3.3}         \\
    \iff & (x, y) = (x', y').                                   &  & \by{i:ex:3.5.1}[b]
  \end{align*}
  Thus, \cref{i:ex:3.5.1}(b) is a valid definition of ordered pairs.
\end{proof}

\begin{proof}[\pf{i:ex:3.5.1}(c)]
  We use \cref{i:ex:3.5.1}(a) as in the definition of ordered pairs.
  Suppose that \(X, Y\) are sets.
  For each \(x \in X\), we can use \cref{i:3.3} to create singleton set \(\set{x}\).
  Using \cref{i:3.3} again, we can create pair sets \(\set{x, y}\) and \(\set{\set{x}, \set{x, y}}\) for each \(x \in X\) and each \(y \in Y\).
  Since \(Y\) is a set, each object in \(Y\) is uniquely identified (\cref{i:3.1.1}).
  Thus, for each \(x \in X\) and each \(y \in Y\), the statement ``\(P_y(x, y') \coloneqq y' = y\)'' is only true for one \(y' \in Y\), namely \(y\).
  Thus, for each \(y \in Y\), we can use \cref{i:3.6} to create the set \(S_y\) using \(P_y\):
  \begin{align*}
    S_y & = \set{\set{\set{x}, \set{x, y'}} : P_y(x, y') \text{ is true for some } x \in X} &  & \by{i:3.6}       \\
        & = \set{\set{\set{x}, \set{x, y}} : x \in X}                                                             \\
        & = \set{(x, y) : x \in X}.                                                         &  & \by{i:ex:3.5.1}.
  \end{align*}
  By \cref{i:3.11}, we see that \(\bigcup_{y \in Y} S_y = \set{(x, y) : x \in X, y \in Y}\), which equals to \(X \times Y\) (\cref{i:3.5.4}).
  Thus, \(X \times Y\) is a set.
\end{proof}

\begin{ex}\label{i:ex:3.5.2}
  Suppose we \emph{define} an ordered \(n\)-tuple to be a surjective function
  \[
    x : \set{i \in \N : 1 \leq i \leq n} \to X
  \]
  whose codomain is some arbitrary set \(X\) (so different ordered \(n\)-tuples are allowed to have different codomains);
  we then write \(x_i\) for \(x(i)\), and also write \(x\) as \((x_i)_{1 \leq i \leq n}\).
  Using this definition, verify that we have \((x_i)_{1 \leq i \leq n} = (y_i)_{1 \leq i \leq n}\) iff \(x_i = y_i\) for all \(1 \leq i \leq n\).
  Also, show that if \((X_i)_{1 \leq i \leq n}\) are an ordered \(n\)-tuple of sets, then the Cartesian product, as defined in \cref{i:3.5.7}, is indeed a set.
  (Technically, this construction of ordered \(n\)-tuple is not compatible with the construction of ordered pair in \cref{i:ex:3.5.1}, but this does not cause a difficulty in practice;
  for instance, one can use the definition of an ordered \(2\)-tuple here to replace the construction in \cref{i:ex:3.5.1}, or one can make a rather pedantic distinction between an ordered \(2\)-tuple and an ordered pair in one's mathematical arguments.)
\end{ex}

\begin{proof}[\pf{i:ex:3.5.2}]
  First we show that \cref{i:ex:3.5.2} is a valid definition of \(n\)-tuple.
  Let \(n \in \N\), let \(X\) be a set, and let \(I = \set{i \in \N : 1 \leq i \leq n}\).
  Let \(x : I \to X, y : I \to X\) be two surjective functions.
  By definition, we have \(x = (x_i)_{1 \leq i \leq n}\) and \(y = (y_i)_{1 \leq i \leq n}\).
  Then we have
  \begin{align*}
         & (x_i)_{1 \leq i \leq n} = (y_i)_{1 \leq i \leq n}                      \\
    \iff & x = y                                             &  & \by{i:ex:3.5.2} \\
    \iff & \forall i \in I, x(i) = y(i)                      &  & \by{i:3.3.1}    \\
    \iff & \forall i \in I, x_i = y_i.                       &  & \by{i:ex:3.5.2}
  \end{align*}
  Thus, \cref{i:ex:3.5.2} is a valid definition of \(n\)-tuple.

  Now we show that \(\prod_{i = 1}^n X_i\) defined in \cref{i:3.5.7} is indeed a set.
  Suppose that \((X_i)_{1 \leq i \leq n}\) is an ordered \(n\)-tuple of sets.
  Let \(I = \set{i \in \N : 1 \leq i \leq n}\).
  By definition, we have a surjective function \(X : I \to Y\), where \(Y\) is a set, and \(X(i) = X_i \in Y\) for each \(i \in I\).
  Then we can create the following set
  \begin{align*}
    \bigcup \set{X(i) : i \in I} & = \bigcup_{i \in I} X(i) &  & \by{i:3.11}     \\
                                 & = \bigcup_{i \in I} X_i. &  & \by{i:ex:3.5.2}
  \end{align*}
  By \cref{i:ex:3.4.7}, there exists a set of all partial function with domain \(I\) and codomain \(\bigcup_{i \in I} X_i\), and we denote this set \(A\).
  By \cref{i:3.5}, there exists a set \(B = \set{x \in A : \forall i \in I, x_i \in X_i}\).
  Clearly, every \(x \in B\) has the form \(x = (x_i)_{1 \leq i \leq n}\).
  And every \(n\)-tuple \((x_i)_{1 \leq i \leq n}\) with \(x_i \in X_i\) for all \(i \in I\) is also in \(B\), thanks to \cref{i:ex:3.4.7}.
  Thus, by \cref{i:3.5.7}, we see that \(B = \prod_{i = 1}^n X_i\).
  Therefore, \(\prod_{i = 1}^n X_i\) is indeed a set.
\end{proof}

\begin{ex}\label{i:ex:3.5.3}
  Show that the definitions of equality for ordered pair and ordered \(n\)-tuple are consistent with the reflexivity, symmetry, and transitivity axioms in the sense that if these axioms of equality are already assumed to hold for the individual components \(x, y\) of an ordered pair \((x, y)\), then they hold for an ordered pair itself.
\end{ex}

\begin{proof}[\pf{i:ex:3.5.3}]
  Let \((x, y), (x', y'), (x'', y'')\) be ordered pairs.
  We first show that \cref{i:3.5.1} is reflexive.
  Since
  \begin{align*}
             & (x = x) \land (y = y)                   \\
    \implies & (x, y) = (x, y),      &  & \by{i:3.5.1}
  \end{align*}
  we see that \cref{i:3.5.1} is reflexive.

  Next we show that \cref{i:3.5.1} is symmetry.
  Since
  \begin{align*}
         & (x, y) = (x', y')                         \\
    \iff & (x = x') \land (y = y') &  & \by{i:3.5.1} \\
    \iff & (x' = x) \land (y' = y)                   \\
    \iff & (x', y') = (x, y),      &  & \by{i:3.5.1}
  \end{align*}
  we see that \cref{i:3.5.1} is symmetry.

  Next we show that \cref{i:3.5.1} is transitive.
  Since
  \begin{align*}
             & ((x, y) = (x', y')) \land ((x', y') = (x'', y''))                           \\
    \implies & (x = x') \land (y = y') \land (x' = x'') \land (y' = y'') &  & \by{i:3.5.1} \\
    \implies & (x = x'') \land (y = y'')                                                   \\
    \implies & (x, y) = (x'', y''),                                      &  & \by{i:3.5.1}
  \end{align*}
  we see that \cref{i:3.5.1} is transitive.

  Let \(i, n \in \N\), and let \((x_i)_{1 \leq i \leq n}, (y_i)_{1 \leq i \leq n}, (z_i)_{1 \leq i \leq n}\) be ordered \(n\)-tuples.
  Next we show that \cref{i:3.5.7} is reflexive.
  Since
  \begin{align*}
             & \forall 1 \leq i \leq n, x_i = x_i                                   \\
    \implies & (x_i)_{1 \leq i \leq n} = (x_i)_{1 \leq i \leq n}, &  & \by{i:3.5.7}
  \end{align*}
  we see that \cref{i:3.5.7} is reflexive.

  Next we show that \cref{i:3.5.7} is symmetry.
  Since
  \begin{align*}
         & (x_i)_{1 \leq i \leq n} = (y_i)_{1 \leq i \leq n}                    \\
    \iff & \forall 1 \leq i \leq n, x_i = y_i                 &  & \by{i:3.5.7} \\
    \iff & \forall 1 \leq i \leq n, y_i = x_i                                   \\
    \iff & (y_i)_{1 \leq i \leq n} = (x_i)_{1 \leq i \leq n}, &  & \by{i:3.5.7}
  \end{align*}
  we see that \cref{i:3.5.7} is symmetry.

  Finally we show that \cref{i:3.5.7} is transitive.
  Since
  \begin{align*}
             & \pa{(x_i)_{1 \leq i \leq n} = (y_i)_{1 \leq i \leq n}} \land \pa{(y_i)_{1 \leq i \leq n} = (z_i)_{1 \leq i \leq n}}                   \\
    \implies & \forall 1 \leq i \leq n, (x_i = y_i) \land (y_i = z_i)                                                              &  & \by{i:3.5.7} \\
    \implies & \forall 1 \leq i \leq n, x_i = z_i                                                                                                    \\
    \implies & (x_i)_{1 \leq i \leq n} = (z_i)_{1 \leq i \leq n},                                                                  &  & \by{i:3.5.7}
  \end{align*}
  we see that \cref{i:3.5.7} is transitive.
\end{proof}

\begin{ex}\label{i:ex:3.5.4}
  Let \(A, B, C\) be sets.
  Show that \(A \times (B \cup C) = (A \times B) \cup (A \times C)\), that \(A \times (B \cap C) = (A \times B) \cap (A \times C)\), and that \(A \times (B \setminus C) = (A \times B) \setminus (A \times C)\).
\end{ex}

\begin{proof}[\pf{i:ex:3.5.4}]
  We first show that \(A \times (B \cup C) = (A \times B) \cup (A \times C)\).
  Since
  \begin{align*}
         & (x, y) \in A \times (B \cup C)                                                 \\
    \iff & (x \in A) \land (y \in B \cup C)                             &  & \by{i:3.5.4} \\
    \iff & (x \in A) \land ((y \in B) \lor (y \in C))                   &  & \by{i:3.4}   \\
    \iff & ((x \in A) \land (y \in B)) \lor ((x \in A) \land (y \in C))                   \\
    \iff & ((x, y) \in A \times B) \lor ((x, y) \in A \times C)         &  & \by{i:3.5.4} \\
    \iff & (x, y) \in (A \times B) \cup (A \times C),                   &  & \by{i:3.4}
  \end{align*}
  we have \(A \times (B \cup C) = (A \times B) \cup (A \times C)\) by \cref{i:3.1.4}.

  Next we show that \(A \times (B \cap C) = (A \times B) \cap (A \times C)\).
  Since
  \begin{align*}
         & (x, y) \in A \times (B \cap C)                                                   \\
    \iff & (x \in A) \land (y \in B \cap C)                              &  & \by{i:3.5.4}  \\
    \iff & (x \in A) \land ((y \in B) \land (y \in C))                   &  & \by{i:3.1.23} \\
    \iff & ((x \in A) \land (y \in B)) \land ((x \in A) \land (y \in C))                    \\
    \iff & ((x, y) \in A \times B) \land ((x, y) \in A \times C)         &  & \by{i:3.5.4}  \\
    \iff & (x, y) \in (A \times B) \cap (A \times C),                    &  & \by{i:3.1.23}
  \end{align*}
  we have \(A \times (B \cap C) = (A \times B) \cap (A \times C)\) by \cref{i:3.1.4}.

  Finally we show that \(A \times (B \setminus C) = (A \times B) \setminus (A \times C)\).
  Since
  \begin{align*}
         & (x, y) \in A \times (B \setminus C)                                                 \\
    \iff & (x \in A) \land (y \in B \setminus C)                            &  & \by{i:3.5.4}  \\
    \iff & (x \in A) \land ((y \in B) \land (y \notin C))                   &  & \by{i:3.1.27} \\
    \iff & ((x \in A) \land (y \in B)) \land ((x \in A) \land (y \notin C))                    \\
    \iff & ((x, y) \in A \times B) \land ((x, y) \notin A \times C)         &  & \by{i:3.5.4}  \\
    \iff & (x, y) \in (A \times B) \setminus (A \times C),                  &  & \by{i:3.1.27}
  \end{align*}
  we have \(A \times (B \setminus C) = (A \times B) \setminus (A \times C)\) by \cref{i:3.1.4}.
\end{proof}

\begin{ex}\label{i:ex:3.5.5}
  Let \(A, B, C, D\) be sets.
  Show that \((A \times B) \cap (C \times D) = (A \cap C) \times (B \cap D)\).
  Is it true that \((A \times B) \cup (C \times D) = (A \cup C) \times (B \cup D)?\)
  Is it true that \((A \times B) \setminus (C \times D) = (A \setminus C) \times (B \setminus D)?\)
\end{ex}

\begin{proof}[\pf{i:ex:3.5.5}]
  We first show that \((A \times B) \cap (C \times D) = (A \cap C) \times (B \cap D)\).
  Since
  \begin{align*}
         & (x, y) \in (A \times B) \cap (C \times D)                                        \\
    \iff & ((x, y) \in A \times B) \land ((x, y) \in C \times D)         &  & \by{i:3.1.23} \\
    \iff & ((x \in A) \land (y \in B)) \land ((x \in C) \land (y \in D)) &  & \by{i:3.5.4}  \\
    \iff & ((x \in A) \land (x \in C)) \land ((y \in B) \land (y \in D))                    \\
    \iff & (x \in A \cap C) \land (y \in B \cap D)                       &  & \by{i:3.1.23} \\
    \iff & (x, y) \in (A \cap C) \times (B \cap D),                      &  & \by{i:3.5.4}
  \end{align*}
  we have \((A \times B) \cap (C \times D) = (A \cap C) \times (B \cap D)\) by \cref{i:3.1.4}.

  We do not have \((A \times B) \cup (C \times D) = (A \cup C) \times (B \cup D)\).
  Let \(A = \set{1}, B = \set{2}, C = \set{3}, D = \set{4}\).
  Then we have
  \begin{align*}
    (A \times B) \cup (C \times D) & = \set{(1, 2)} \cup \set{(3, 4)}        &  & \by{i:3.5.4} \\
                                   & = \set{(1, 2), (3, 4)}.                 &  & \by{i:3.4}   \\
    (A \cup C) \times (B \cup D)   & = \set{1, 3} \times \set{2, 4}          &  & \by{i:3.4}   \\
                                   & = \set{(1, 2), (1, 4), (3, 2), (3, 4)}. &  & \by{i:3.5.4}
  \end{align*}
  Clearly, \((A \times B) \cup (C \times D) \neq (A \cup C) \times (B \cup D)\) by \cref{i:3.1.4}.

  We do not have \((A \times B) \setminus (C \times D) = (A \setminus C) \times (B \setminus D)\).
  Let \(A = \set{1, 2}, B = \set{3, 4}, C = \set{1}, D = \set{3}\).
  Then we have
  \begin{align*}
    (A \times B) \setminus (C \times D)    & = \set{(1, 3), (1, 4), (2, 3), (2, 4)} \setminus \set{(1, 3)} &  & \by{i:3.5.4}  \\
                                           & = \set{(1, 4), (2, 3), (2, 4)}.                               &  & \by{i:3.1.27} \\
    (A \setminus C) \times (B \setminus D) & = \set{2} \times \set{4}                                      &  & \by{i:3.1.27} \\
                                           & = \set{(2, 4)}.                                               &  & \by{i:3.5.4}
  \end{align*}
  Clearly, \((A \times B) \setminus (C \times D) \neq (A \setminus C) \times (B \setminus D)\) by \cref{i:3.1.4}.
\end{proof}

\begin{ex}\label{i:ex:3.5.6}
  Let \(A, B, C, D\) be non-empty sets.
  Show that \(A \times B \subseteq C \times D\) iff \(A \subseteq C\) and \(B \subseteq D\), and that \(A \times B = C \times D\) iff \(A = C\) and \(B = D\).
  What happens if the hypotheses that the \(A, B, C, D\) are all non-empty are removed?
\end{ex}

\begin{proof}[\pf{i:ex:3.5.6}]
  We first show that if \(A, B, C, D\) are non-empty sets, then \(A \times B \subseteq C \times D \iff (A \subseteq C) \land (B \subseteq D)\).
  This is true since
  \begin{align*}
         & A \times B \subseteq C \times D                                                                                      \\
    \iff & \pa{(x, y) \in A \times B \implies (x, y) \in C \times D}             &  & \by{i:3.1.15}                             \\
    \iff & \pa{((x \in A) \land (y \in B)) \implies ((x \in C) \land (y \in D))} &  & \by{i:3.5.4}                              \\
    \iff & \pa{x \in A \implies x \in C} \land \pa{y \in B \implies y \in D}     &  & (A \neq \emptyset \land B \neq \emptyset) \\
    \iff & (A \subseteq C) \land (B \subseteq D).                                &  & \by{i:3.1.15}
  \end{align*}
  This statement is not true when \((A = \emptyset) \lor (B = \emptyset)\).
  For example, if \(A = \set{1}\) and \(B = C = D = \emptyset\), then \(A \times B = \set{()} \subseteq C \times D = \set{()}\), but \(A \not \subseteq C\).

  Next we show that if \(A, B, C, D\) are non-empty sets, then \(A \times B = C \times D \iff (A = C) \land (B = D)\).
  This is true since
  \begin{align*}
         & A \times B = C \times D                                                                                      \\
    \iff & \pa{(x, y) \in A \times B \iff (x, y) \in C \times D}             &  & \by{i:3.1.4}                          \\
    \iff & \pa{((x \in A) \land (y \in B)) \iff ((x \in C) \land (y \in D))} &  & \by{i:3.5.4}                          \\
    \iff & (x \in A \iff x \in C) \land (y \in B \iff y \in D)               &  & \text{(\(A, B, C, D\) are non-empty)} \\
    \iff & (A = C) \land (B = D).                                            &  & \by{i:3.1.4}
  \end{align*}
  This statement is not true when \(((A = \emptyset) \land (C = \emptyset)) \lor ((B = \emptyset) \land (D = \emptyset))\).
  For example, if \(A = \set{1}\), \(C = \set{2}\) and \(B = D = \emptyset\), then \(A \times B = \set{()} = C \times D\), but \(A \neq C\).
\end{proof}

\begin{ex}\label{i:ex:3.5.7}
  Let \(X, Y\) be sets, and let \(\pi_{X \times Y \to X} : X \times Y \to X\) and \(\pi_{X \times Y \to Y} : X \times Y \to Y\) be the maps \(\pi_{X \times Y \to X}(x, y) \coloneqq x\) and \(\pi_{X \times Y \to Y}(x, y) \coloneqq y\);
  these maps are known as the \emph{co-ordinate functions} on \(X \times Y\).
  Show that for any functions \(f : Z \to X\) and \(g : Z \to Y\), there exists a unique function \(h : Z \to X \times Y\) such that \(\pi_{X \times Y \to X} \circ h = f\) and \(\pi_{X \times Y \to Y} \circ h = g\).
  (Compare this to \cref{i:ex:3.3.8}(d), and to \cref{i:ex:3.1.7}.)
  This function \(h\) is known as the \emph{direct sum} of \(f\) and \(g\) and is denoted \(h = f \oplus g\).
\end{ex}

\begin{proof}[\pf{i:ex:3.5.7}]
  We first show the existence of such function \(h\).
  Suppose that \(X, Y, Z\) are sets, and \(f : Z \to X, g : Z \to Y\) are functions.
  Let \(\pi_{X \times Y \to X} : X \times Y \to X, \pi_{X \times Y \to Y} : X \times Y \to Y\) be functions where \(\pi_{X \times Y \to X}(x, y) = x\) and \(\pi_{X \times Y \to Y}(x, y) = y\).
  Both \(\pi_{X \times Y \to X}\) and \(\pi_{X \times Y \to Y}\) are well-defined by \cref{i:3.6}.
  We now define a function \(h : Z \to X \times Y\) where
  \[
    \forall z \in Z, h(z) = (f(z), g(z)).
  \]
  To show that \(h\) is well-defined, by \cref{i:3.3.1}, we have to show that \(h\) pass the vertical line test.
  Since \(f, g\) are functions, by \cref{i:3.3.1}, we know that for each \(z \in Z\), \(f(z)\) and \(g(z)\) are unique objects in \(X\) and \(Y\), respectively.
  Thus, by \cref{i:ex:3.5.3}, \((f(z), g(z)) \in X \times Y\) is unique for each \(z \in Z\).
  Therefore, \(h(z) \in X \times Y\) is unique for each \(z \in Z\),
  Thus, \(h\) is well-defined.
  Now we have
  \begin{align*}
    \forall z \in Z, (\pi_{X \times Y \to X} \circ h)(z) & = \pi_{X \times Y \to X}(h(z))       &  & \by{i:3.3.10} \\
                                                         & = \pi_{X \times Y \to X}(f(z), g(z))                    \\
                                                         & = f(z).                                                 \\
    \forall z \in Z, (\pi_{X \times Y \to Y} \circ h)(z) & = \pi_{X \times Y \to Y}(h(z))       &  & \by{i:3.3.10} \\
                                                         & = \pi_{X \times Y \to Y}(f(z), g(z))                    \\
                                                         & = g(z).
  \end{align*}
  Thus, by \cref{i:3.3.7}, we have \(\pi_{X \times Y \to X} \circ h = f\) and \(\pi_{X \times Y \to Y} \circ h = g\).

  Now we show the uniqueness of such function \(h\).
  Suppose that there exists another function \(h' : Z \to X \times Y\) such that \(\pi_{X \times Y \to X} \circ h' = f\) and \(\pi_{X \times Y \to Y} \circ h' = g\).
  Then we have
  \begin{align*}
             & \forall z \in Z, \begin{dcases}
                                  f(z) = \pa{\pi_{X \times Y \to X} \circ h'}(z) = \pi_{X \times Y \to X}\pa{h'(z)} \\
                                  g(z) = \pa{\pi_{X \times Y \to Y} \circ h'}(z) = \pi_{X \times Y \to Y}\pa{h'(z)}
                                \end{dcases} &  & \by{i:3.3.7,i:3.3.10}     \\
    \implies & \forall z \in Z, h'(z) = (f(z), g(z)) = h(z)                                                            \\
    \implies & h' = h.                                                                               &  & \by{i:3.3.7}
  \end{align*}
  Thus, \(h\) is unique.
\end{proof}

\begin{ex}\label{i:ex:3.5.8}
  Let \(X_1, \dots, X_n\) be sets.
  Show that the Cartesian product \(\prod_{i = 1}^n X_i\) is empty iff at least one of the \(X_i\) is empty.
\end{ex}

\begin{proof}[\pf{i:ex:3.5.8}]
  We have
  \begin{align*}
         & \emptyset = \prod_{i = 1}^n X_i = \set{(x_i)_{1 \leq i \leq n} : x_i \in X_i \text{ for all } 1 \leq i \leq n} &  & \by{i:3.5.7} \\
    \iff & \exists i \in \N : (1 \leq i \leq n) \land (\forall x_i, x_i \notin X_i)                                       &  & \by{i:3.5.7} \\
    \iff & \exists i \in N : (1 \leq i \leq n) \land (X_i = \emptyset).                                                   &  & \by{i:3.2}
  \end{align*}
\end{proof}

\begin{ex}\label{i:ex:3.5.9}
  Suppose that \(I\) and \(J\) are two sets, and for all \(\alpha \in I\) let \(A_\alpha\) be a set, and for all \(\beta \in J\) let \(B_\beta\) be a set.
  Show that \(\pa{\bigcup_{\alpha \in I} A_\alpha} \cap \pa{\bigcup_{\beta \in J} B_\beta} = \bigcup_{(\alpha, \beta) \in I \times J} (A_\alpha \cap B_\beta)\).
\end{ex}

\begin{proof}[\pf{i:ex:3.5.9}]
  Since
  \begin{align*}
         & x \in \pa{\bigcup_{\alpha \in I} A_\alpha} \cap \pa{\bigcup_{\beta \in J} B_\beta}                           \\
    \iff & \pa{x \in \bigcup_{\alpha \in I} A_\alpha} \land \pa{x \in \bigcup_{\beta \in J} B_\beta} &  & \by{i:3.1.23} \\
    \iff & \pa{\exists \alpha \in I : x \in A_\alpha} \land \pa{\exists \beta \in J : x \in B_\beta} &  & \by{i:3.11}   \\
    \iff & \exists (\alpha, \beta) \in I \times J : (x \in A_\alpha) \land (x \in B_\beta)           &  & \by{i:3.5.4}  \\
    \iff & \exists (\alpha, \beta) \in I \times J : x \in A_\alpha \cap B_\beta                      &  & \by{i:3.1.23} \\
    \iff & x \in \bigcup_{(\alpha, \beta) \in I \times J} (A_\alpha \cap B_\beta),                   &  & \by{i:3.11}
  \end{align*}
  we have \(\pa{\bigcup_{\alpha \in I} A_\alpha} \cap \pa{\bigcup_{\beta \in J} B_\beta} = \bigcup_{(\alpha, \beta) \in I \times J} (A_\alpha \cap B_\beta)\) by \cref{i:3.1.4}.
\end{proof}

\begin{ex}\label{i:ex:3.5.10}
  If \(f : X \to Y\) is a function, define the \emph{graph} of \(f\) to be the subset of \(X \times Y\) defined by \(\set{(x, f(x)) : x \in X}\).
  Show that two functions \(f : X \to Y\), \(\tilde{f} : X \to Y\) are equal iff they have the same graph.
  Conversely, if \(G\) is any subset of \(X \times Y\) with the property that for each \(x \in X\), the set \(\set{y \in Y : (x, y) \in G}\) has exactly one element (or in other words, \(G\) obeys the \emph{vertical line test}), show that there is exactly one function \(f : X \to Y\) whose graph is equal to \(G\).
\end{ex}

\begin{proof}[\pf{i:ex:3.5.10}]
  The first statement is true since
  \begin{align*}
         & f = \tilde{f}                                                                       \\
    \iff & \forall x \in X, f(x) = \tilde{f}(x)                              &  & \by{i:3.3.7} \\
    \iff & \forall x \in X, (x, f(x)) = \pa{x, \tilde{f}(x)}                 &  & \by{i:3.5.1} \\
    \iff & \set{(x, f(x)) : x \in X} = \set{\pa{x, \tilde{f}(x)} : x \in X}. &  & \by{i:3.6}
  \end{align*}

  Next we show that there exists a function whose graph is equal to \(G\).
  For each \(x \in X\), we use \cref{i:3.6} to create the set \(S_x = \set{y \in Y : (x, y) \in G}\).
  Define \(P(x, y)\) to be the statement ``\(y \in S_x\)'' for each \((x, y) \in X \times Y\).
  Then, by the hypothesis of \(G\), we know that for each \(x \in X\), there is only one \(y \in Y\) such that \(P(x, y)\) is true.
  Thus, we can use \cref{i:3.3.1} to create a function \(f : X \to Y\) such that
  \[
    \forall (x, y) \in X \times Y, y = f(x) \iff P(x, y) \iff y \in S_x \iff (x, y) \in G.
  \]
  By definition, the graph of \(f\) is \(\set{(x, f(x)) : x \in X}\).
  Thus, we have
  \[
    \forall (x, y) \in X \times Y, (x, y) \in \set{(x, f(x)) : x \in X} \iff y = f(x) \iff (x, y) \in G.
  \]
  By \cref{i:3.1.4}, this means \(G = \set{(x, f(x)) : x \in X}\).
  Thus, \(f\) is a function whose graph is equal to \(G\).

  Now we show that \(f\) is unique.
  Suppose that there exists another function \(\tilde{f} : X \to Y\) such that the graph of \(\tilde{f}\) is equal to \(G\).
  But then \(f\) and \(\tilde{f}\) have the same graph, and by the first part of the proof we see that \(f = \tilde{f}\).
  Thus, \(f\) is unique.
\end{proof}

\begin{ex}\label{i:ex:3.5.11}
  Show that \cref{i:3.10} can in fact be deduced from \cref{i:3.4.9} and the other axioms of set theory, and thus \cref{i:3.4.9} can be used as an alternate formulation of the power set axiom.
\end{ex}

\begin{proof}[\pf{i:ex:3.5.11}]
  Suppose that \(X, Y\) are sets.
  Then by \cref{i:ex:3.5.1}(c), \(X \times Y\) is a set.
  Thus, we can use \cref{i:3.4.9} to create a set of subsets of \(X \times Y\), i.e., \(S_1 = \set{S : S \subseteq X \times Y}\).
  If \(f : X \to Y\) is a function, then by \cref{i:3.3.1}, every element on \(X\) must be assigned exactly one object in \(Y\).
  Thus, we use \cref{i:3.5} to rule out those sets violating the vertical line test, and we derive the following set
  \[
    S_2 = \set{G \in S_1 \mid \forall x \in X, \exists! y \in Y : (x, y) \in G}.
  \]
  By \cref{i:ex:3.5.10}, we see that every element \(G \in S_2\) is a graph, and we know that there exists exactly one function \(f : X \to Y\) whose graph is \(G\).
  Thus, we can use \cref{i:3.6} to create the following set
  \[
    S_3 = \set{f : X \to Y \mid \text{the graph of } f \text{ is in } S_2}.
  \]
  Now we claim that every function \(f : X \to Y\) is an element of \(S_3\), and thus by \cref{i:3.10} we have \(S_3 = X^Y\), and \(X^Y\) is indeed a set.
  But by \cref{i:ex:3.5.10}, every function with domain \(X\) and codomain \(Y\) is uniquely identified by one graph and vice versa.
  Thus, the claim is true.
\end{proof}

\begin{ex}\label{i:ex:3.5.12}
  Let \(X\) be an arbitrary set, let \(f : \N \times X \to X\) be a function, and let \(c \in X\).
  Show that there exists a function \(a : \N \to X\) such that
  \[
    a(0) = c
  \]
  and
  \[
    a(n\pp) = f(n, a(n)) \text{ for all } n \in \N,
  \]
  and furthermore that this function is unique.
  For an additional challenge, prove this result without using any properties of the natural numbers other than the Peano axioms directly.
\end{ex}

\begin{proof}[\pf{i:ex:3.5.12}]
  For each \(N \in \N\), let \(P(N)\) be the statement ``there exists an unique function \(a_N : \set{n \in \N : n \leq N} \to X\), such that \(a_N(0) = c\), and \(a_N(n\pp) = f(n, a_N(n))\) for all \(n \in \N\) such that \(n < N\).''
  We induct on \(N\) to prove that \(P(N)\) is true for all \(N \in \N\).

  For \(N = 0\), we define \(a_0 : \set{0} \to X\) to be the function \(a_0(0) = c\).
  By \cref{i:ac:2.2.4}, we see that \(\set{0} = \set{n \in \N : n \leq 0}\), and the statement ``\(a_0(n\pp) = f(n, a_0(n))\) for all \(n \in \N\) such that \(n < 0\)'' is vacuously true.
  Thus, to ensure that the base case holds, we only need to show that \(a_0\) is unique.
  Let \(b_0 : \set{0} \to X\) be another function satisfying \(b_0(0) = c\), and \(b_0(n\pp) = f(n, b_0(n))\) for all \(n \in \N\) such that \(n < 0\).
  Since \(0\) is the only element in \(\set{0}\), and \(b_0(0) = c = a_0(0)\), we must have \(b_0 = a_0\) by \cref{i:3.3.7}.
  Thus, \(a_0\) is unique, and the base case holds.

  Suppose inductively that \(P(N)\) is true for some \(N \in \N\).
  We want to show that \(P(N\pp)\) is true.
  By the induction hypothesis, there exists an unique function \(a_N : \set{n \in \N : n \leq N} \to X\), such that \(a_N(0) = c\), and \(a_N(n\pp) = f(n, a_N(n))\) for all \(n \in \N\) such that \(n < N\).
  Now we define a function \(a_{N\pp} : \set{n \in \N : n \leq N\pp} \to X\) as follow:
  \[
    \forall n \in \N \text{ such that } n \leq N\pp, a_{N\pp}(n) = \begin{dcases}
      a_N(n)       & \text{if } n \neq N\pp                    \\
      f(m, a_N(m)) & \text{if } n = N\pp \text{ and } m\pp = n
    \end{dcases}.
  \]
  Since each \(n \in \set{n \in \N : n \leq N\pp}\) is assigned with an unique object in \(X\), we see that \(a_{N\pp}\) is well-defined.
  Since \(0 \neq N\pp\) (\cref{i:2.3}), we have \(a_{N\pp}(0) = a_N(0) = c\).
  We claim that we must have \(a_{N\pp}(n) = f(n, a_{N\pp}(n))\) for each natural number \(n < N\pp\).
  So let \(n \in N\) where \(n < N\pp\).
  We split into two cases:
  \begin{itemize}
    \item If \(n\pp \neq N\pp\), then we have
          \begin{align*}
            a_{N\pp}(n\pp) & = a_N(n\pp)          &  & (n\pp \neq N\pp) \\
                           & = f(n, a_N(n))       &  & \byIH            \\
                           & = f(n, a_{N\pp}(n)). &  & (n < N\pp)
          \end{align*}
    \item If \(n\pp = N\pp\), then we have
          \begin{align*}
            a_{N\pp}(n\pp) & = f(n, a_N(n))       &  & (n\pp = N\pp) \\
                           & = f(n, a_{N\pp}(n)). &  & (n < N\pp)
          \end{align*}
  \end{itemize}
  From all cases above we see that \(a_{N\pp}(n\pp) = f(n, a_{N\pp}(n))\).
  Thus, our claim is true.
  To close the induction, we need to show that \(a_{N\pp}\) is unique.
  So suppose that there exists another function \(b_{N\pp} : \set{n \in \N : n \leq N\pp} \to X\), where \(b_{N\pp}(0) = c\), and \(b_{N\pp}(n\pp) = f(n, b_{N\pp}(n))\) for all \(n \in \N\) such that \(n < N\).
  Clearly, we have \(b_{N\pp}(0) = c = a_{N\pp}(0)\).
  If we have show that \(b_{N\pp}(n) = a_{N\pp}(n)\) for some \(n \in \N\) and \(n < N\pp\), then we see that
  \[
    b_{N\pp}(n\pp) = f(n, b_{N\pp}(n)) = f(n, a_{N\pp}(n)) = a_{N\pp}(n\pp).
  \]
  Thus, we have \(b_{N\pp}(n) = a_{N\pp}(n)\) for all \(n \in \set{n \in \N : n \leq N\pp}\).
  By \cref{i:3.1.4}, this means \(b_{N\pp} = a_{N\pp}\), and this closes the induction.

  Now we define \(a : \N \to X\) as follow:
  \[
    \forall n \in \N, a(n) = a_n(n).
  \]
  Since \(P(n)\) is true for all \(n \in \N\), we know that \(a_n\) is well-defined, and \(a_n(n)\) is unique for each \(n \in \N\).
  Thus, \(a\) is well-defined.
  Clearly, we have \(a(0) = a_0(0) = c\).
  We claim that \(a(n\pp) = f(n, a(n))\) for all \(n \in \N\).
  This is true since
  \begin{align*}
    \forall n \in \N, a(n\pp) & = a_{n\pp}(n\pp)                    \\
                              & = f(n, a_{n\pp}(n))                 \\
                              & = f(n, a_n(n))      &  & (n < n\pp) \\
                              & = f(n, a(n)).
  \end{align*}
  Now we show that \(a\) is unique.
  Suppose there exists another \(b : \N \to X\), where \(b(0) = c\), and \(b(n\pp) = f(n, b(n))\) for all \(n \in \N\).
  Clealy, \(b(0) = c = a(0)\).
  If \(b(n) = a(n)\) is true for some \(n \in \N\), then we have
  \[
    b(n\pp) = f(n, b(n)) = f(n, a(n)) = a(n\pp).
  \]
  Thus, by \cref{i:2.5}, we know that \(b(n) = a(n)\) for all \(n \in \N\).
  By \cref{i:3.1.4}, this means \(b = a\).
  Thus, \(a\) is unique.

  Now we prove the additional challenge.
  We claim that for every natural number \(N \in \N\), there exists a unique pair \(A_N, B_N\) of subsets of \(\N\) which obeys the following properties:
  \begin{enumerate}
    \item \(A_N \cap B_N = \emptyset\);
    \item \(A_N \cup B_N = \N\);
    \item \(0 \in A_N\);
    \item \(N\pp \in B_N\);
    \item Whenever \(n \in B_N\), we have \(n\pp \in B_N\);
    \item Whenever \(n \in A_N\) and \(n \neq N\), we have \(n\pp \in A_N\).
  \end{enumerate}
  We induct on \(N\) to prove the claim.

  For \(N = 0\), let \(A_0 = \set{0}\), and let \(B_0 = \N \setminus A_0\).
  By \cref{i:3.1.28}(g), we see that (a)(b) holds for \(A_0, B_0\).
  By \cref{i:3.3}, we have \(0 \in A_0\) and \(0\pp = 1 \notin A_0\).
  Thus, \(1 \in B_0\).
  So (c)(d) holds for \(A_0, B_0\).
  If \(n \in B_0\), then \(n \in \N\) and \(n\pp \neq 0\) (\cref{i:2.4}).
  Thus, we have \(n\pp \notin A_0\), and therefore \(n\pp \in B_0\).
  So (e) holds for \(A_0, B_0\).
  Since there is no natural number \(n\) satisfying \(n \in A_0\) and \(n \neq 0\), we see that (f) holds for \(A_0, B_0\).
  To ensure that the base case holds, we are left to show that \(A_0, B_0\) are unique.
  So suppose that there exist another pair of sets \(A_0', B_0'\) such that (a)--(f) hold.
  By (c)(d), we know that \(0 \in A_0'\) and \(1 \in B_0'\).
  Thus, by (e), we see that \(n \in B_0'\) for all \(n \in \N\) and \(n \geq 1\).
  Therefore, we have \(A_0' = \set{0} = A_0\) by (a).
  By \cref{i:ex:3.1.9}, we have \(B_0' = \N \setminus A_0' = \N \setminus A_0 = B_0\).
  Thus, \(A_0, B_0\) is unique, and the base case holds.

  Suppose inductively that, for some natural number \(N\), there exists a unique pair of set \(A_N, B_N\) such that (a)--(f) hold.
  We define \(A_{N\pp} = A_N \cup \set{N\pp}\) and \(B_{N\pp} = B_N \setminus \set{N\pp}\).
  Since
  \begin{align*}
     & A_{N\pp} \cap B_{N\pp}                                                                                          \\
     & = (A_N \cup \set{N\pp}) \cap (B_N \setminus \set{N\pp})                                                         \\
     & = (A_N \cap (B_N \setminus \set{N\pp})) \cup (\set{N\pp} \cap (B_N \setminus \set{N\pp})) &  & \by{i:3.1.28}[f] \\
     & = (A_N \cap (B_N \setminus \set{N\pp})) \cup \emptyset                                    &  & \by{i:3.1.28}[g] \\
     & = A_N \cap (B_N \setminus \set{N\pp})                                                     &  & \by{i:3.1.28}[a] \\
     & \subseteq A_n \cap B_N                                                                    &  & \by{i:3.1.15}    \\
     & = \emptyset,                                                                              &  & \byIH
  \end{align*}
  we see that \(A_{N\pp} \cap B_{N\pp} = \emptyset\) by \cref{i:3.2}.
  Thus, (a) is true for \(A_{N\pp}, B_{N\pp}\).
  Since
  \begin{align*}
     & A_{N\pp} \cup B_{N\pp}                                                        \\
     & = (A_N \cup \set{N\pp}) \cup (B_N \setminus \set{N\pp})                       \\
     & = A_N \cup (\set{N\pp} \cup (B_N \setminus \set{N\pp})) &  & \by{i:3.1.28}[e] \\
     & = A_N \cup B_N                                          &  & \by{i:3.1.28}[g] \\
     & = \N,                                                   &  & \byIH
  \end{align*}
  we see that (b) is true for \(A_{N\pp}, B_{N\pp}\).
  Since \(A_{N\pp} = A_N \cup \set{N\pp}\), we have \(A_N \subseteq A_{N\pp}\) by \cref{i:ex:3.1.7}.
  By the induction hypothesis, we know that \(0 \in A_N\).
  Thus, by \cref{i:3.1.15}, we have \(0 \in A_{N\pp}\), and (c) is true for \(A_{N\pp}, B_{N\pp}\).
  By the induction hypothesis, we have \(N\pp \in B_N\).
  Thus, by (e), we see that \((N\pp)\pp \in B_N\).
  Since \(B_{N\pp} = B_N \setminus \set{N\pp}\), we see that \((N\pp)\pp \in B_{N\pp}\).
  Thus, (d) is true for \(A_{N\pp}, B_{N\pp}\).
  If \(n \in B_{N\pp} = B_N \setminus \set{N\pp}\), then we have \(n \in B_N\) and \(n \neq N\pp\) by \cref{i:3.1.27,i:3.3}.
  We must have \(n\pp \neq N\pp\), for otherwise we would have \(n\pp = N\pp\) and \(n = N\) by \cref{i:2.4}.
  But by induction hypotheses, we would have \(N\pp = n\pp \in B_N\), a contradiction.
  Thus, \(n \in B_{N\pp} \implies n\pp \neq N\pp\).
  Hence,
  \begin{align*}
             & \forall n \in B_{N\pp}, (n \in B_n) \land (n\pp \neq N\pp)                                     \\
    \implies & \forall n \in B_{N\pp}, (n\pp \in B_N) \land (n\pp \neq N\pp)         &  & \byIH               \\
    \implies & \forall n \in B_{N\pp}, n\pp \in B_N \setminus \set{N\pp} = B_{N\pp}. &  & \by{i:3.1.27,i:3.3}
  \end{align*}
  So (e) is true for \(A_{N\pp}, B_{N\pp}\).
  Since
  \begin{align*}
             & (n \in A_{N\pp} = A_N \cup \set{N\pp}) \land (n \neq N\pp)                                 \\
    \implies & ((n \in A_N) \lor (n = N\pp)) \land (n \neq N\pp)                    &  & \by{i:3.3,i:3.4} \\
    \implies & ((n \in A_N) \land (n \neq N)) \lor ((n = N\pp) \land (n \neq N\pp))                       \\
    \implies & (n \in A_N) \land (n \neq N)                                                               \\
    \implies & n\pp \in A_N                                                         &  & \byIH            \\
    \implies & n\pp \in A_{N\pp},                                                   &  & \by{i:3.1.15}
  \end{align*}
  we see that (f) is true for \(A_{N\pp}, B_{N\pp}\).
  To close the induction, we only need to show that the pair of sets \(A_{N\pp}, B_{N\pp}\) is unique.
  So suppose that there exist another pair of sets \(A_{N\pp}', B_{N\pp}'\) such that (a)--(f) hold.
  Then, by (c) and (f), we see that \(A_{N\pp}' = A_{N\pp}\).
  Thus, by (b) and \cref{i:ex:3.1.9}, we see that \(B_{N\pp}' = B_{N\pp}\).
  Thus, the pair \(A_{N\pp}, B_{N\pp}\) is unique and this closes the induction.

  From the construction of \(A_N\) we see that \(A_N = \set{n \in N : 0 \leq N}\).
  Thus, we can simply apply \(A_N\) to define \(a_N : N \to X\) as in the first part of the proof.
  The rest of the claims follows from the first part of the proof.
\end{proof}

\begin{ex}\label{i:ex:3.5.13}
  Suppose we have a set \(\N'\) of ``alternative natural numbers'', an ``alternative zero'' \(0'\), and an ``alternative increment operation'' which takes any alternative natural number \(n' \in N\) and returns another alternative natural number \(n'\pp' \in \N'\), such that the Peano axioms (\crefrange{i:2.1}{i:2.5}) all hold with the natural numbers, zero, and increment replaced by their alternative counterparts.
  Show that there exists a bijection \(f : \N \to \N'\) from the natural numbers to the alternative natural numbers such that \(f(0) = 0'\), and such that for any \(n \in \N\) and \(n' \in \N'\), we have \(f(n) = n'\) iff \(f(n\pp) = n'\pp'\).
\end{ex}

\begin{proof}[\pf{i:ex:3.5.13}]
  Define \(g : \N \times \N' \to \N'\) as follow:
  \[
    \forall (n, n') \in \N \times \N', g(n, n') = n'\pp'.
  \]
  By \cref{i:2.4}, we see that \(g\) pass the vertical line test.
  Thus, \(g\) is well-defined.
  By \cref{i:ex:3.5.12}, there exists a unique function \(a : \N \to \N'\) such that
  \[
    a(0) = 0' \quad \text{and} \quad \forall n \in \N, a(n\pp) = g(n, a(n)) = a(n)\pp'.
  \]

  First we show that \(a\) is injective.
  Let \(n_1, n_2 \in \N\).
  We induct on \(n_1\) to show that \(n_1 \neq n_2 \implies a(n_1) \neq a(n_2)\).
  For \(n_1 = 0\), we have \(n_2 \neq 0\) and \(a(n_1) = a(0) = 0'\).
  Since \(n_2 \neq 0\), by \cref{i:2.2.10}, there exists an \(m \in \N\) such that \(m\pp = n_2\).
  By the definition of \(a\), we see that \(a(n_2) = a(m\pp) = a(m)\pp'\).
  By \cref{i:2.3}, we know that \(a(m)\pp' \neq 0'\).
  Thus, we have \(a(n_1) \neq a(n_2)\), and the base case holds.
  Suppose inductively that \(n_1 \neq n_2 \implies a(n_1) \neq a(n_2)\) for some \(n_1 \in \N\).
  We want to show that \(n_1\pp \neq n_2 \implies a(n_1\pp) \neq a(n_2)\).
  We split into two cases:
  \begin{itemize}
    \item If \(n_2 = 0\), then we can use the identical proof as in the base case, but we replace \((n_1, n_2)\) with \((n_2, n_1\pp)\) to see that \(a(n_1\pp) \neq a(n_2)\).
    \item If \(n_2 \neq 0\), then by \cref{i:2.2.10}, there exists an \(m \in \N\) such that \(m\pp = n_2\).
          Thus,
          \begin{align*}
                     & n_1\pp \neq m\pp = n_2           &  & \by{i:2.2.10}                       \\
            \implies & n_1 \neq m                       &  & \by{i:2.4}                          \\
            \implies & a(n_1) \neq a(m)                 &  & \byIH                               \\
            \implies & a(n_1)\pp' \neq a(m)\pp'         &  & \by{i:2.4}                          \\
            \implies & a(n_1\pp) \neq a(m\pp) = a(n_2). &  & \text{(by the definition of \(a\))}
          \end{align*}
  \end{itemize}
  From all cases above we see that \(a(n_1\pp) \neq a(n_2)\).
  This closes the induction, and we see that \(a\) is injective by \cref{i:3.3.14}.

  Next we claim that \(a\) is surjective, i.e. (\cref{i:3.3.17}), for each \(n' \in \N'\), there exists an \(n \in \N\) such that \(a(n) = n'\).
  Since \cref{i:2.5} holds for \(\N'\), we can induct on \(n'\) to prove the claim.
  For \(n' = 0'\), we see that \(a(0) = 0'\).
  So the base case holds.
  Suppose inductively that for some \(n' \in \N'\), there exists an \(n \in \N\) such that \(a(n) = n'\).
  Then we have
  \begin{align*}
    a(n\pp) & = a(n)\pp' &  & \text{(by the definition of \(a\))} \\
            & = n'\pp',  &  & \byIH
  \end{align*}
  and the claim is true for \(n'\pp'\).
  This closes the induction, and thus \(a\) is surjective.
  Since \(a\) is both injective and surjective, we know that \(a\) is bijective by \cref{i:3.3.20}.

  By the definition of \(a\), we see that for all \(n, n' \in \N \times \N'\), we have \(a(n) = n' \iff a(n\pp) = n'\pp'\).
  Thus, by setting \(f = a\), we are done.
\end{proof}

\section{Cardinality of sets}\label{sec:3.6}

\begin{defn}[Equal cardinality]\label{3.6.1}
  We say that two sets \(X\) and \(Y\) have \emph{equal cardinality} iff there exists a bijection \(f : X \to Y\) from \(X\) to \(Y\).
\end{defn}

\setcounter{thm}{2}
\begin{rmk}\label{3.6.3}
  The fact that two sets have equal cardinality does not preclude one of the sets from containing the other.
  For instance, if \(X\) is the set of natural numbers and \(Y\) is the set of even natural numbers, then the map \(f : X \to Y\) defined by \(f(n) \coloneqq 2n\) is a bijection from \(X\) to \(Y\), and so \(X\) and \(Y\) have equal cardinality, despite \(Y\) being a subset of \(X\) and seeming intuitively as if it should only have ``half'' of the elements of \(X\).
\end{rmk}

\begin{prop}\label{3.6.4}
  Let \(X\), \(Y\), \(Z\) be sets.
  Then \(X\) has equal cardinality with \(X\).
  If \(X\) has equal cardinality with \(Y\), then \(Y\) has equal cardinality with \(X\).
  If \(X\) has equal cardinality with \(Y\) and \(Y\) has equal cardinality with \(Z\), then \(X\) has equal cardinality with \(Z\).
\end{prop}

\begin{proof}
  We first show that \cref{3.6.1} is reflexive.
  Suppose that \(X\) is a set.
  Let \(f : X \to X\) be a function where \(f = x \mapsto x\).
  By \cref{3.6} \(f\) is well-defined.
  Such \(f\) is injective since \(\forall x, x' \in X\), \(f(x) = f(x') \implies x = x'\), and \(f\) is also surjective since \(\forall x \in X\), \(\exists\ x \in X\) such that \(f(x) = x\).
  Thus \(f\) is bijective, and by \cref{3.6.1} \(X\) has equal cardinality with \(X\).

  Next we show that \cref{3.6.1} is symmetric.
  Suppose that \(X, Y\) are sets such that \(X\) has equal cardinality with \(Y\).
  Then by \cref{3.6.1} there exists a function \(f : X \to Y\) such that \(f\) is bijective.
  Since \(f\) is bijective, by \cref{ex:3.3.6} \(f^{-1} : Y \to X\) is also bijective.
  Thus by \cref{3.6.1} \(Y\) has equal cardinality with \(X\).

  Finally we show that \cref{3.6.1} is transitive.
  Suppose that \(X, Y, Z\) are sets such that \(X\) has equal cardinality with \(Y\) and \(Y\) has equal cardinality with \(Z\).
  Then by \cref{3.6.1} there exist two functions \(f : X \to Y\) and \(g : Y \to Z\) such that \(f\) and \(g\) are bijective.
  Since \(f\) and \(g\) are bijective, by \cref{ex:3.3.7} \(g \circ f : X \to Z\) is also bijective.
  Thus by \cref{3.6.1} \(X\) has equal cardinality with \(Z\).
\end{proof}

\begin{defn}\label{3.6.5}
  Let \(n\) be a natural number.
  A set \(X\) is said to have \emph{cardinality} \(n\), iff it has equal cardinality with \(\{i \in \N : 1 \leq i \leq n\}\).
  We also say that \(X\) \emph{has \(n\) elements} iff it has cardinality \(n\).
\end{defn}

\begin{rmk}\label{3.6.6}
  One can use the set \(\{i \in \N : i < n\}\) instead of \(\{i \in \N : 1 \leq i \leq n\}\), since these two sets clearly have equal cardinality.
\end{rmk}

\setcounter{thm}{7}
\begin{prop}[Uniqueness of cardinality]\label{3.6.8}
  Let \(X\) be a set with some cardinality \(n\).
  Then \(X\) cannot have any other cardinality, i.e., \(X\) cannot have cardinality \(m\) for any \(m \neq n\).
\end{prop}

\begin{proof}
  We induct on \(n\).
  First suppose that \(n = 0\).
  Then \(X\) must be empty, and so \(X\) cannot have any non-zero cardinality.
  Now suppose that the proposition is already proven for some \(n\);
  we now prove it for \(n++\).
  Let \(X\) have cardinality \(n++\);
  and suppose that \(X\) also has some other cardinality \(m \neq n++\).
  By \cref{3.6.9}, \(X\) is non-empty, and if \(x\) is any element of \(X\), then \(X \setminus \{x\}\) has cardinality \(n\) and also has cardinality \(p\), where \(p++ = m\), by \cref{3.6.9}.
  By induction hypothesis, this means that \(n = p\), which implies that \(p++ = m = n++\), a contradiction.
  This closes the induction.
\end{proof}

\begin{lem}\label{3.6.9}
  Suppose that \(n \geq 1\), and \(X\) has cardinality \(n\).
  Then \(X\) is non-empty, and if \(x\) is any element of \(X\), then the set \(X \setminus \{x\}\) (i.e., \(X\) with the element \(x\) removed) has cardinality \(m\), where \(m++ = n\).
\end{lem}

\begin{proof}
  If \(X\) is empty then it clearly cannot have the same cardinality as the non-empty set \(\{i \in \N : 1 \leq i \leq n\}\), as there is no bijection from the empty set to a non-empty set.
  Now let \(x\) be an element of \(X\).
  Since \(X\) has the same cardinality as \(\{i \in \N : 1 \leq i \leq n\}\), we thus have a bijection \(f\) from \(X\) to \(\{i \in \N : 1 \leq i \leq n\}\).
  In particular, \(f(x)\) is a natural number between \(1\) and \(n\).
  Now define the function \(g : X \setminus \{x\} \to \{i \in \N : 1 \leq i \leq m\}\) by the following rule: for any \(y \in X \setminus \{x\}\), we define \(g(y) \coloneqq f(y)\) if \(f(y) < f(x)\), and define \(g(y)++ \coloneqq f(y)\) if \(f(y) > f(x)\).
  (Note that \(f(y)\) cannot equal \(f(x)\) since \(y \neq x\) and \(f\) is a bijection.)
  It is easy to check that this map is also a bijection, and so \(X \setminus \{x\}\) has equal cardinality with \(\{i \in \N : 1 \leq i \leq m\}\).
  In particular \(X \setminus \{x\}\) has cardinality \(m\), as desired.
\end{proof}

\begin{defn}[Finite sets]\label{3.6.10}
  A set is \emph{finite} iff it has cardinality \(n\) for some natural number \(n\);
  otherwise, the set is called \emph{infinite}.
  If \(X\) is a finite set, we use \(\#(X)\) to denote the cardinality of \(X\).
\end{defn}

\setcounter{thm}{11}
\begin{thm}\label{3.6.12}
  The set of natural numbers \(\N\) is infinite.
\end{thm}

\begin{proof}
  Suppose for sake of contradiction that the set of natural numbers \(\N\) was finite, so it had some cardinality \(\#(\N) = n\).
  Then there is a bijection \(f\) from \(\{i \in \N : 1 \leq i \leq n\}\) to \(\N\).
  One can show that the sequence \(f(1), f(2), \dots, f(n)\) is bounded, or more precisely that there exists a natural number \(M\) such that \(f(i) \leq M\) for all \(1 \leq i \leq n\) (\cref{ex:3.6.3}).
  But then the natural number \(M+1\) is not equal to any of the \(f(i)\), contradicting the hypothesis that \(f\) is a bijection.
\end{proof}

\begin{rmk}\label{3.6.13}
  One can also use similar arguments to show that any unbounded set is infinite;
  for instance the rationals \(\Q\) and the reals \(\R\) are infinite.
  However, it is possible for some sets to be ``more'' infinite than others.
\end{rmk}

\begin{prop}[Cardinal arithmetic]\label{3.6.14}
  \leavevmode
  \begin{enumerate}
    \item Let \(X\) be a finite set, and let \(x\) be an object which is not an element of \(X\).
          Then \(X \cup \{x\}\) is finite and \(\#(X \cup \{x\}) = \#(X) + 1\).
    \item Let \(X\) and \(Y\) be finite sets.
          Then \(X \cup Y\) is finite and \(\#(X \cup Y) \leq \#(X) + \#(Y)\).
          If in addition \(X\) and \(Y\) are disjoint (i.e., \(X \cap Y = \emptyset\)), then \(\#(X \cup Y) = \#(X) + \#(Y)\).
    \item Let \(X\) be a finite set, and let \(Y\) be a subset of \(X\).
          Then \(Y\) is finite, and \(\#(Y) \leq \#(X)\).
          If in addition \(Y \neq X\) (i.e., \(Y\) is a proper subset of \(X\)), then we have \(\#(Y) < \#(X)\).
    \item If \(X\) is a finite set, and \(f : X \to Y\) is a function, then \(f(X)\) is a finite set with \(\#(f(X)) \leq \#(X)\).
          If in addition \(f\) is one-to-one, then \(\#(f(X)) = \#(X)\).
    \item Let \(X\) and \(Y\) be finite sets.
          Then Cartesian product \(X \times Y\) is finite and \(\#(X \times Y) = \#(X) \times \#(Y)\).
    \item Let \(X\) and \(Y\) be finite sets.
          Then the set \(Y^X\) (defined in \cref{3.10}) is finite and \(\#(Y^X) = \#(Y)^{\#(X)}\).
  \end{enumerate}
\end{prop}

\begin{proof}{(a)}
  Suppose that \(X\) is a finite set and \(x \notin X\).
  By \cref{3.6.10} \(\exists\ n \in \N\) such that \(\#(X) = n\).
  By \cref{3.6.5} \(\exists\ f : X \to \{i \in \N : 1 \leq i \leq n\}\) such that \(f\) is bijective.
  Now we define a function \(g : X \cup \{x\} \to \{i \in \N : 1 \leq i \leq n + 1\}\) as follow:
  \[
    g(y) = \begin{dcases}
      f(y)  & \text{if } y \in X \\
      n + 1 & \text{otherwise}
    \end{dcases}
  \]

  Now we need to show that \(g\) is bijective.
  Since \(f\) is bijective, \(\forall i \in \{i \in \N : 1 \leq i \leq n\}, \exists\ y \in X\) such that \(f(y) = i\).
  With that and \(g(x) = n + 1\) we thus have \(g\) is surjective.
  For any \(y, y' \in X \cup \{x\}\), if \(g(y) = g(y')\), then we have two cases:
  \begin{itemize}
    \item If \(g(y) \in X\), then \(y = y'\) since \(f\) is bijective.
    \item If \(g(y) = x\), then \(y = y' = x\) by definition of \(g\).
  \end{itemize}
  For all cases above we have \(g(y) = g(y') \implies y = y'\), thus \(g\) is injective.
  Since \(g\) is bijective, we have \(\#(X \cup \{x\}) = n + 1 = \#(X) + 1\).
\end{proof}

\begin{proof}{(b)}
  Suppose that \(X, Y\) are finite sets.
  If \(X = \emptyset \lor X = Y\), then \(X \cup Y = Y\) is finite.
  Similarly if \(Y = \emptyset \lor Y = X\), then \(X \cup Y = X\) is finite.
  Thus assume that \(X \neq \emptyset \land Y \neq \emptyset\) and \(X, Y\) are distinct.
  By \cref{3.6.10} \(\exists\ n \in \N\) such that \(\#(X) = n\).
  We use induction on \(n\) to show that \(X \cup Y\) is finite and \(\#(X \cup Y) \leq \#(X) + \#(Y)\).
  We start with \(n = 1\) since \(X \neq \emptyset\).
  For \(n = 1\), we have
  \begin{align*}
    \#(X \cup Y) & = \#(Y) + 1         &  & \text{(by \cref{3.6.14}(a))} \\
                 & = \#(X) + \#(Y)     &  & \by{3.6.5}                   \\
                 & \leq \#(X) + \#(Y).
  \end{align*}
  Thus \(X \cup Y\) is finite and the base case holds.
  Suppose inductively that the statement is true for some \(\#(X) = n\).
  We show that the statement is still true for \(\#(X) = n++\).
  Let \(x \in X\).
  By \cref{3.6.9} we have \(\#(X \setminus \{x\}) = n\).
  If \(x \in Y\), then we have
  \begin{align*}
    \#(X \cup Y) & = \#((X \setminus \{x\}) \cup \{x\} \cup Y)   &  & \text{(by \cref{3.1.28}(g))} \\
                 & = \#((X \setminus \{x\}) \cup (\{x\} \cup Y)) &  & \text{(by \cref{3.1.28}(e))} \\
                 & = \#((X \setminus \{x\}) \cup Y)                                                \\
                 & \leq \#(X \setminus \{x\}) + \#(Y)            &  & \byIH                        \\
                 & < \#(X) + \#(Y).
  \end{align*}
  If \(x \notin Y\), then we have
  \begin{align*}
    \#(X \cup Y) & = \#((X \setminus \{x\}) \cup \{x\} \cup Y)  &  & \text{(by \cref{3.1.28}(g))}    \\
                 & = \#((X \setminus \{x\}) \cup Y \cup \{x\})  &  & \text{(by \cref{3.1.28}(d)(e))} \\
                 & = \#((X \setminus \{x\}) \cup Y) + 1         &  & \text{(by \cref{3.6.14}(a))}    \\
                 & \leq \#(X \setminus \{x\}) + \#(Y) + 1       &  & \byIH                           \\
                 & = \#((X \setminus \{x\}) \cup \{x\}) + \#(Y) &  & \text{(by \cref{3.6.14}(a))}    \\
                 & = \#(X) + \#(Y).                             &  & \text{(by \cref{3.1.28}(g))}
  \end{align*}
  In either cases we have \(\#(X \cup Y) \leq \#(X) + \#(Y)\).
  Thus \(X \cup Y\) is finite and this closes the induction.

  Now suppose that \(X, Y\) are finite sets and \(X \cap Y = \emptyset\).
  By \cref{3.6.10} \(\exists\ n \in \N\) such that \(\#(X) = n\).
  From proof above we already know that \(\#(X \cup Y) \leq \#(X) + \#(Y)\).
  We now use induction on \(n\) to show that \(\#(X \cup Y) = \#(X) + \#(Y)\).
  For \(n = 0\), we have
  \begin{align*}
    \#(X \cup Y) & = \#(Y)          &  & \text{(by \cref{3.1.28}(a))} \\
                 & = 0 + \#(Y)                                        \\
                 & = \#(X) + \#(Y). &  & \by{3.6.5}
  \end{align*}
  Thus the base case holds.
  Suppose inductively that the statement is true for some \(\#(X) = n\).
  We show that the statement is still true for \(\#(X) = n++\).
  Let \(x \in X\).
  By \cref{3.6.9} we have \(\#(X \setminus \{x\}) = n\).
  Since \(X \cap Y = \emptyset\), \(x \notin Y\).
  So we have
  \begin{align*}
    \#(X \cup Y) & = \#((X \setminus \{x\}) \cup \{x\} \cup Y)  &  & \text{(by \cref{3.1.28}(g))}    \\
                 & = \#((X \setminus \{x\}) \cup Y \cup \{x\})  &  & \text{(by \cref{3.1.28}(d)(e))} \\
                 & = \#((X \setminus \{x\}) \cup Y) + 1         &  & \text{(by \cref{3.6.14}(a))}    \\
                 & = \#(X \setminus \{x\}) + \#(Y) + 1          &  & \byIH                           \\
                 & = \#((X \setminus \{x\}) \cup \{x\}) + \#(Y) &  & \text{(by \cref{3.6.14}(a))}    \\
                 & = \#(X) + \#(Y).                             &  & \text{(by \cref{3.1.28}(g))}
  \end{align*}
  This closes the induction.
\end{proof}

\begin{proof}{(c)}
  Suppose that \(X\) is a finite sets.
  By \cref{3.6.10} \(\exists\ n \in \N\) such that \(\#(X) = n\).
  We use induction on \(n\) to show that \(\forall Y \subseteq X\), \(Y\) is finite and \(\#(Y) \leq \#(X)\).
  For \(n = 0\), we have
  \begin{align*}
             & \forall Y \subseteq \emptyset                             \\
    \implies & Y = \emptyset                 &  & \text{(by \cref{3.2})} \\
    \implies & \#(Y) = 0 \leq 0 = \#(X).     &  & \by{3.6.5}
  \end{align*}
  Thus the base case holds.
  Suppose inductively that the statement is true for some \(\#(X) = n\).
  We show that the statement is still true for \(\#(X) = n++\).
  Let \(Y \subseteq X\).
  If \(Y = X\), then by \cref{3.6.8} \(\#(Y) = \#(X)\).
  If \(Y \neq X\), then \(\exists\ x \in X \setminus Y\) such that \(Y \subseteq X \setminus \{x\}\) and
  \begin{align*}
    \#(X) & = \#((X \setminus \{x\}) \cup \{x\}) &  & \text{(by \cref{3.1.28}(g))} \\
          & = \#(X \setminus \{x\}) + 1          &  & \text{(by \cref{3.6.14}(a))} \\
          & \geq \#(Y) + 1                       &  & \byIH                        \\
          & > \#(Y).
  \end{align*}
  From all cases above we have \(\forall Y \subseteq X \implies \#(Y) \leq \#(X)\).
  Thus \(Y\) is finite and this closes the induction.

  We now use induction on \(n\) to show that \(\forall Y \subseteq X : Y \neq X \implies \#(Y) < \#(X)\).
  We start with \(n = 1\) since for \(n = 0\) we have \(X = \emptyset\) and \(\nexists Y : Y \subseteq X \land Y \neq X\).
  For \(n = 1\), we have
  \begin{align*}
             & \forall Y : Y \subseteq X \land Y \neq X                             \\
    \implies & Y = \emptyset                            &  & \text{(by \cref{3.3})} \\
    \implies & \#(Y) = 0 < 1 = \#(X).                   &  & \by{3.6.5}
  \end{align*}
  Thus the base case holds.
  Suppose inductively that the statement is true for some \(\#(X) = n\).
  We show that the statement is still true for \(\#(X) = n++\).
  Let \(Y \subseteq X \land Y \neq X\).
  If \(Y = \emptyset\), then \(\#(Y) = 0 < n++ = \#(X)\).
  If \(Y \neq \emptyset\), then \(\exists\ x \in Y\) such that
  \begin{align*}
             & \#(Y \setminus \{x\}) < \#(X \setminus \{x\})                           &  & \byIH                        \\
    \implies & \#(Y \setminus \{x\}) + 1 < \#(X \setminus \{x\}) + 1                                                     \\
    \implies & \#((Y \setminus \{x\}) \cup \{x\}) < \#((X \setminus \{x\}) \cup \{x\}) &  & \text{(by \cref{3.6.14}(a))} \\
    \implies & \#(Y) < \#(X).                                                          &  & \text{(by \cref{3.1.28}(g))}
  \end{align*}
  This closes the induction.
\end{proof}

\begin{proof}{(d)}
  Suppose that \(X\) is a finite sets.
  By \cref{3.6.10} \(\exists\ n \in \N\) such that \(\#(X) = n\).
  We use induction on \(n\) to show that for any set \(Y\) and any function \(f : X \to Y\) we have \(\#(f(X)) \leq \#(X)\).
  For \(n = 0\), let \(Y\) be arbitrary set and \(f : X \to Y\) be arbitrary function.
  Then we have
  \begin{align*}
             & X = \emptyset                                \\
    \implies & f(X) = \emptyset             &  & \by{3.4.1} \\
    \implies & \#(f(X)) = 0 \leq 0 = \#(X). &  & \by{3.6.5}
  \end{align*}
  Thus the base case holds.
  Suppose inductively that the statement is true for some \(\#(X) = n\).
  We show that the statement is still true for \(\#(X) = n++\).
  Let \(x \in X\), \(Y\) be arbitrary set and \(f : X \to Y\) be arbitrary function.
  If \(f(X \setminus \{x\}) = f(X)\), then we have
  \begin{align*}
             & \#(f(X \setminus \{x\})) \leq \#(X \setminus \{x\}) &  & \byIH                        \\
    \implies & \#(f(X)) \leq \#(X \setminus \{x\})                                                   \\
    \implies & \#(f(X)) < \#(X \setminus \{x\}) + 1                                                  \\
    \implies & \#(f(X)) < \#((X \setminus \{x\}) \cup \{x\})       &  & \text{(by \cref{3.6.14}(a))} \\
    \implies & \#(f(X)) < \#(X).                                   &  & \text{(by \cref{3.1.28}(g))}
  \end{align*}
  If \(f(X \setminus \{x\}) \neq f(X)\), then we have
  \begin{align*}
    \#(f(X)) & = \#(f(X \setminus \{x\}) \cup \{f(x)\}) &  & \text{(by \cref{ex:3.4.3})}  \\
             & = \#(f(X \setminus \{x\})) + 1           &  & \text{(by \cref{3.6.14}(a))} \\
             & \leq \#(X \setminus \{x\}) + 1           &  & \byIH                        \\
             & = \#((X \setminus \{x\}) \cup \{x\})     &  & \text{(by \cref{3.6.14}(a))} \\
             & = \#(X).                                 &  & \text{(by \cref{3.1.28}(g))}
  \end{align*}
  From all cases above we have \(\#(f(X)) \leq \#(X)\).
  This closes the induction.

  We now use induction on \(n\) to show that for any set \(Y\) and any one-to-one function \(f : X \to Y\) we have \(\#(f(X)) = \#(X)\).
  For \(n = 0\), let \(Y\) be arbitrary set and \(f : X \to Y\) be arbitrary one-to-one function.
  Then we have
  \begin{align*}
             & X = \emptyset                         \\
    \implies & f(X) = \emptyset      &  & \by{3.4.1} \\
    \implies & \#(f(X)) = 0 = \#(X). &  & \by{3.6.5}
  \end{align*}
  Thus the base case holds.
  Suppose inductively that the statement is true for some \(\#(X) = n\).
  We show that the statement is still true for \(\#(X) = n++\).
  Let \(x \in X\), \(Y\) be arbitrary set and \(f : X \to Y\) be arbitrary one-to-one function.
  Since \(f\) is one-to-one, we must have \(f(X \setminus \{x\}) \neq f(X)\) and
  \begin{align*}
    \#(f(X)) & = \#(f(X \setminus \{x\}) \cup \{f(x)\}) &  & \text{(by \cref{ex:3.4.3})}  \\
             & = \#(f(X \setminus \{x\})) + 1           &  & \text{(by \cref{3.6.14}(a))} \\
             & = \#(X \setminus \{x\}) + 1              &  & \byIH                        \\
             & = \#((X \setminus \{x\}) \cup \{x\})     &  & \text{(by \cref{3.6.14}(a))} \\
             & = \#(X).                                 &  & \text{(by \cref{3.1.28}(g))}
  \end{align*}
  This closes the induction.
\end{proof}

\begin{proof}{(e)}
  Suppose that \(X, Y\) are finite sets.
  We first show that \(\forall x : \#(\{x\} \times Y) = \#(Y)\).
  By \cref{3.6.1}, we only need to find a function \(f : \{x\} \times Y \to Y\) such that \(f\) is bijective.
  We now define \(f : \{x\} \times Y \to Y\) as \(f(x', y) = y\).
  We need to show that \(f\) is bijective.
  We start by showing \(f\) is injective.
  \begin{align*}
             & \forall (x_1, y_1), (x_2, y_2) \in \{x\} \times Y : f(x_1, y_1) = f(x_2, y_2)                             \\
    \implies & x_1 = x_2 \land y_1 = y_2                                                     &  & \text{(by \cref{3.3})} \\
    \implies & (x_1, y_1) = (x_2, y_2).                                                      &  & \by{3.5.1}
  \end{align*}
  Thus \(f\) is injective.
  Now we show that \(f\) is surjective.
  \begin{align*}
             & \forall y \in Y                                                   \\
    \implies & (x, y) \in \{x\} \times Y                         &  & \by{3.5.4} \\
    \implies & \exists\ (x, y) \in \{x\} \times Y : f(x, y) = y.
  \end{align*}
  Thus \(f\) is surjective.
  Since \(f\) is both injective and surjective, \(f\) is bijective and thus by \cref{3.6.1} we have \(\#(\{x\} \times Y) = \#(Y)\).

  Now we show that \(\#(X \times Y) = \#(X) \times \#(Y)\).
  By \cref{3.6.10} \(\exists\ n \in \N\) such that \(\#(X) = n\).
  We use induction on \(n\) to show that \(\#(X \times Y) = \#(X) \times \#(Y)\).
  For \(n = 0\), we have
  \begin{align*}
    \#(X \times Y) & = \#(\emptyset \times Y) &  & \by{3.6.5} \\
                   & = \#(\emptyset)          &  & \by{3.5.4} \\
                   & = 0                      &  & \by{3.6.5} \\
                   & = \#(X) \times \#(Y).
  \end{align*}
  Thus the base case holds.
  Suppose inductively that the statement is true for some \(\#(X) = n\).
  We show that the statement is still true for \(\#(X) = n++\).
  Let \(x \in X\).
  Then we have
  \begin{align*}
    \#(X \times Y) & = \#\bigg(\big((X \setminus \{x\}) \cup \{x\}\big) \times Y\bigg)                                   \\
                   & = \#((X \setminus \{x\}) \times Y \cup \{x\} \times Y)            &  & \text{(by \cref{ex:3.5.4})}  \\
                   & = \#((X \setminus \{x\}) \times Y) + \#(\{x\} \times Y)           &  & \text{(by \cref{3.6.14}(b))} \\
                   & = \#(X \setminus \{x\}) \times \#(Y) + \#(\{x\} \times Y)         &  & \byIH                        \\
                   & = \#(X \setminus \{x\}) \times \#(Y) + \#(Y)                      &  & \text{(from proof above)}    \\
                   & = (\#(X \setminus \{x\}) + 1) \times \#(Y)                                                          \\
                   & = \#((X \setminus \{x\}) \cup \{x\}) \times \#(Y)                 &  & \text{(by \cref{3.6.14}(a))} \\
                   & = \#(X) \times \#(Y).                                             &  & \text{(by \cref{3.1.28}(g))}
  \end{align*}
  This closes the induction.
\end{proof}

\begin{proof}{(f)}
  Suppose that \(X, Y\) are finite sets.
  We first show that \(\forall x : \#(Y^{\{x\}}) = \#(Y)\).
  We define a function \(f : Y^{\{x\}} \to Y\) by setting \(\forall g \in Y^{\{x\}} : f(g) = g(x)\).
  We now show that \(f\) is bijective.
  We start by showing \(f\) is injective.
  \begin{align*}
             & \forall g, g' \in Y^{\{x\}} : f(g) = f(g')                             \\
    \implies & g(x) = g'(x)                                                           \\
    \implies & \forall x' \in \{x\} : g(x') = g'(x').     &  & \text{(by \cref{3.3})} \\
    \implies & g = g'.                                    &  & \by{3.3.7}
  \end{align*}
  Thus \(f\) is injective.
  Now we show that \(f\) is surjective.
  \begin{align*}
             & \forall y \in Y, \exists\ (g : \{x\} \to Y) : g(x) = y &  & \text{(by \cref{3.6})}  \\
    \implies & g \in Y^{\{x\}}.                                       &  & \text{(by \cref{3.10})}
  \end{align*}
  Thus \(f\) is surjective.
  Since \(f\) is both injective and surjective, \(f\) is bijective and thus by \cref{3.6.1} we have \(\#(Y^{\{x\}}) = \#(Y)\).

  Now we show that \(\#(Y^X) = \#(Y)^{\#(X)}\).
  By \cref{3.6.10} \(\exists\ n \in \N\) such that \(\#(X) = n\).
  We use induction on \(n\) to show that \(\#(Y^X) = \#(Y)^{\#(X)}\).
  For \(n = 0\), by \cref{3.6.5} we have \(X = \emptyset\) and
  \[
    \forall f, f' \in Y^\emptyset, \forall x \in \emptyset : f(x) = f'(x).
  \]
  Thus by \cref{3.3} \(Y^\emptyset\) is a singleton set.
  We can construct a bijection \(g : \{i \in \N : 1 \leq i \leq 1\} \to Y^\emptyset\) and thus by \cref{3.6.5} \(\#(Y^\emptyset) = 1\).
  Again by \cref{3.6.5} we have \(\#(X) = 0\), and thus by \cref{2.3.11} we have \(\#(Y)^0 = 1\).
  So the base case holds.

  Suppose inductively that the statement is true for some \(\#(X) = n\).
  We show that the statement is still true for \(\#(X) = n++\).
  Let \(x \in X\).
  We define a function \(h : Y^X \to Y^{X \setminus \{x\}} \times Y^{\{x\}}\) as follow:
  \[
    \forall f \in Y^X : h(f) = \bigg(g : X \setminus \{x\} \to f(X \setminus \{x\}), g' : \{x\} \to f(\{x\})\bigg),
  \]
  where \(\forall x' \in X \setminus \{x\} : g(x') = f(x')\).
  We show that such \(h\) is bijective.
  We start by showing \(h\) is injective.
  \begin{align*}
             & \forall f_1, f_2 \in Y^X : h(f_1) = h(f_2)                                                      \\
    \implies & (g_{f_1}, g_{f_1}') = (g_{f_2}, g_{f_2}')                                                       \\
    \implies & g_{f_1} = g_{f_2} \land g_{f_2} = g_{f_2}'                    &  & \by{3.5.1}                   \\
    \implies & (\forall x' \in X \setminus \{x\} : g_{f_1}(x') = g_{f_2}(x') &  & \by{3.3.7}                   \\
             & \land (\forall x' \in \{x\} : g_{f_1}'(x') = g_{f_2}'(x')                                       \\
    \implies & (\forall x' \in X \setminus \{x\} : f_1(x') = f_2(x')                                           \\
             & \land (\forall x' \in \{x\} : f_1(x') = f_2(x')                                                 \\
    \implies & \forall x' \in X : f_1(x') = f_2(x')                          &  & \text{(by \cref{3.1.28}(g))} \\
    \implies & f_1 = f_2.                                                    &  & \by{3.3.7}
  \end{align*}
  Thus \(h\) is injective.
  Now we show that \(h\) is surjective.
  \(\forall (g, g') \in Y^{X \setminus \{x\}} \times Y^{\{x\}}\), we define a function \(k : X \to Y\) as follow:
  \[
    \forall x' \in X : k(x') = \begin{dcases}
      g(x')  & \text{if } x' \in X \setminus \{x\} \\
      g'(x') & \text{if } x' \in \{x\}
    \end{dcases}
  \]
  Then \(k \in Y^X\).
  Thus \(h\) is surjective.
  Since \(h\) is both injective and surjective, \(h\) is bijective, and we have \(\#(Y^X) = \#(Y^{(X \setminus \{x\})} \times Y^{\{x\}})\).
  We now finish our induction as follow:
  \begin{align*}
    \#(Y^X) & = \#(Y^{(X \setminus \{x\})} \times Y^{\{x\}})       &  & \text{(by proof above)}      \\
            & = \#(Y^{(X \setminus \{x\})}) \times \#(Y^{\{x\}})   &  & \text{(by \cref{3.6.14}(e))} \\
            & = \#(Y)^{\#(X \setminus \{x\})} \times \#(Y^{\{x\}}) &  & \byIH                        \\
            & = \#(Y)^{\#(X \setminus \{x\})} \times \#(Y)         &  & \text{(by proof above)}      \\
            & = \#(Y)^{\#(X \setminus \{x\}) + 1}                  &  & \text{(by \cref{2.3.11})}    \\
            & = \#(Y)^{\#((X \setminus \{x\}) \cup \{x\})}         &  & \text{(by \cref{3.6.14}(a))} \\
            & = \#(Y)^{\#(X)}.                                     &  & \text{(by \cref{3.1.28}(g))}
  \end{align*}
  This closes the induction.
\end{proof}

\begin{rmk}\label{3.6.15}
  \cref{3.6.14} suggests that there is another way to define the arithmetic operations of natural numbers;
  not defined recursively as in \cref{2.2.1,2.3.1,2.3.11}, but instead using the notions of union, Cartesian product, and power set.
  This is the basis of \emph{cardinal arithmetic}, which is an alternative foundation to arithmetic than the Peano arithmetic we have developed here.
\end{rmk}

\exercisesection

\begin{ex}\label{ex:3.6.1}
  Prove \cref{3.6.4}.
\end{ex}

\begin{proof}
  See \cref{3.6.4}.
\end{proof}

\begin{ex}\label{ex:3.6.2}
  Show that a set \(X\) has cardinality \(0\) if and only if \(X\) is the empty set.
\end{ex}

\begin{proof}
  \begin{align*}
         & \#(X) = 0                                                                                                  \\
    \iff & \exists\ f : X \to \{i \in \N : 1 \leq i \leq 0\} \land f \text{ is bijective} &  & \by{3.6.5}             \\
    \iff & \exists\ f : X \to \emptyset                                                   &  & \text{(by \cref{3.2})} \\
    \iff & X = \emptyset.                                                                 &  & \text{(by \cref{3.6})}
  \end{align*}
\end{proof}

\begin{ex}\label{ex:3.6.3}
  Let \(n\) be a natural number, and let \(f : \{i \in \N : 1 \leq i \leq n\} \to \N\) be a function.
  Show that there exists a natural number \(M\) such that \(f(i) \leq M\) for all \(1 \leq i \leq n\).
  Thus finite subsets of the natural numbers are bounded.
\end{ex}

\begin{proof}
  Suppose that \(n \in \N\).
  We use induction on \(n\) to show that for any function \(f : \{i \in \N : 1 \leq i \leq n\} \to \N\), \(\exists\ M \in \N\) such that \(f(i) \leq M\).
  For \(n = 0\), for any function \(f : \{i \in \N : 1 \leq i \leq 0\} \to \N\) we have
  \begin{align*}
             & f : \{i \in \N : 1 \leq i \leq 0\} \to \N                                             \\
    \implies & f : \emptyset \to \N                                     &  & \text{(by \cref{3.2})}  \\
    \implies & \forall M \in \N, \forall i \in \emptyset : f(i) \leq M. &  & \text{(trivially true)} \\
  \end{align*}
  Thus the base case holds.
  Suppose inductively that for some \(n\) the statement is true.
  Then for \(n++\), for any function \(f : \{i \in \N : 1 \leq i \leq n++\} \to \N\) we have
  \begin{itemize}
    \item By induction hypothesis, \(\exists\ M \in \N\) such that \(f(\{i \in \N : 1 \leq i \leq n\}) \leq M\).
    \item By \cref{2.2.13}, exactly one of \(M < f(n++)\), \(M = f(n++)\) or \(M > f(n++)\) is true.
  \end{itemize}
  If \(f(n++) \leq M\), then we have \(\forall i \in \{i \in \N : 1 \leq i \leq n++\} : f(i) \leq M\).
  If \(f(n++) > M\), then we can set \(M' = f(n++)\) and thus \(\forall i \in \{i \in \N : 1 \leq i \leq n++\} : f(i) \leq M'\).
  In all cases above we can conclude that \(\exists\ M \in \N\) such that \(\forall i \in \{i \in \N : 1 \leq i \leq n++\} : f(i) \leq M\).
  This closes the induction.
\end{proof}

\begin{ex}\label{ex:3.6.4}
  Prove \cref{3.6.14}.
\end{ex}

\begin{proof}
  See \cref{3.6.14}.
\end{proof}

\begin{ex}\label{ex:3.6.5}
  Let \(A\) and \(B\) be sets.
  Show that \(A \times B\) and \(B \times A\) have equal cardinality by constructing an explicit bijection between the two sets.
  Then use \cref{3.6.14} to conclude an alternate proof of \cref{2.3.2}.
\end{ex}

\begin{proof}
  Suppose that \(A, B\) are sets.
  By \cref{3.5.4} we have \(A \times B, B \times A\) are sets.
  We define a function \(f : A \times B \to B \times A\) by setting \(\forall (a, b) \in A \times B : f(a, b) = (b, a)\).
  We now show that such \(f\) is bijective.
  We start by showing \(f\) is injective.
  \begin{align*}
             & \forall (a, b), (a', b') \in A \times B : f(a, b) = f(a', b')                 \\
    \implies & (b, a) = (b', a')                                                             \\
    \implies & b = b' \land a = a'                                           &  & \by{3.5.1} \\
    \implies & (a, b) = (a', b').                                            &  & \by{3.5.1}
  \end{align*}
  Thus \(f\) is injective.
  Now we show that \(f\) is surjective.
  This is true since
  \[
    \forall (b, a) \in B \times A, \exists\ (a, b) \in A \times B : f(a, b) = (b, a).
  \]
  Thus \(f\) is surjective.
  Since \(f\) is both injective and surjective, we conclude that \(f\) is bijective.
  Since \(f\) is bijective, by \cref{3.6.1} we conclude that \(A \times B\) and \(B \times A\) have same cardinality.

  Now suppose that \(A, B\) are two finite set.
  By \cref{3.6.5}, \(\exists\ n, m \in \N\) such that \(\#(A) = n \land \#(B) = m\).
  Then we have
  \begin{align*}
    \#(A \times B) & = \#(A) \times \#(B) &  & \text{(by \cref{3.6.14}(e))} \\
                   & = n \times m                                           \\
                   & = \#(B \times A)     &  & \text{(by proof above)}      \\
                   & = \#(B) \times \#(A) &  & \text{(by \cref{3.6.14}(e))} \\
                   & = m \times n.
  \end{align*}
  Thus \cref{2.3.2} is true.
\end{proof}

\begin{ex}\label{ex:3.6.6}
  Let \(A, B, C\) be sets.
  Show that the sets \((A^B)^C\) and \(A^{B \times C}\) have equal cardinality by constructing an explicit bijection between the two sets.
  Conclude that \((a^b)^c = a^{bc}\) for any natural numbers \(a, b, c\).
  Use a similar argument to also conclude \(a^b \times a^c = a^{b+c}\).
\end{ex}

\begin{proof}
  We first show that \((A^B)^C\) and \(A^{B \times C}\) have equal cardinality.
  Suppose that \(A, B, C\) are sets.
  By \cref{3.5.4}, \(B \times C\) is a set.
  By \cref{3.10} \(A^B, (A^B)^C, A^{B \times C}\) are sets.
  We define a function \(f : (A^B)^C \to A^{B \times C}\) by setting \(\big(f(g)\big)(b, c) = \big(g(c)\big)(b)\) where \(b \in B\), \(c \in C\) and \(g : C \to A^B\).
  We now show that \(f\) is bijective.
  We start by showing that \(f\) is injective.
  \begin{align*}
             & \forall h, h' \in (A^B)^C : f(h) = f(h')                                                     \\
    \implies & \forall (b, c) \in B \times C : \big(f(h)\big)(b, c) = \big(f(h')\big)(b, c) &  & \by{3.3.7} \\
    \implies & \forall (b, c) \in B \times C : \big(h(c)\big)(b) = \big(h'(c)\big)(b)                       \\
    \implies & \forall c \in C : h(c) = h'(c)                                               &  & \by{3.3.7} \\
    \implies & h = h'.                                                                      &  & \by{3.3.7}
  \end{align*}
  Thus \(f\) is injective.
  We now show that \(f\) is surjective.
  \(\forall h \in A^{B \times C}\), we define a function \(k : C \to A^B\) by setting \(h(b, c) = (k(c))(b)\) where \(b \in B\) and \(c \in C\).
  Then \(k \in (A^B)^C\) and thus \(f\) is surjective.
  Since \(f\) is both injective and surjective, we conclude that \(f\) is bijective.
  Since \(f\) is bijective, by \cref{3.6.1} we conclude that \((A^B)^C\) and \(A^{B \times C}\) have same cardinality.

  Now we show that \(\forall a, b, c \in \N : (a^b)^c = a^{bc}\).
  Suppose that \(A, B, C\) are finite set.
  By \cref{3.6.5}, \(\exists\ a, b, c \in \N\) such that \(\#(A) = a \land \#(B) = b \land \#(C) = c\).
  Then we have
  \begin{align*}
    \#((A^B)^C) & = \#(A^B)^{\#(C)}            &  & \text{(by \cref{3.6.14}(f))} \\
                & = \#(A^B)^c                                                    \\
                & = (\#(A)^{\#(B)})^c          &  & \text{(by \cref{3.6.14}(f))} \\
                & = (a^b)^c                                                      \\
                & = \#(A^{B \times C})         &  & \text{(by proof above)}      \\
                & = \#(A)^{\#(B \times C)}     &  & \text{(by \cref{3.6.14}(f))} \\
                & = \#(A)^{\#(B) \times \#(C)} &  & \text{(by \cref{3.6.14}(e))} \\
                & = a^{bc}.
  \end{align*}
  Thus we conclude that \(\forall a, b, c \in \N : (a^b)^c = a^{bc}\).

  Next we show that \(A^B \times A^C\) and \(A^{B \cup C}\) have equal cardinality if \(B \cap C = \emptyset\).
  Now suppose that \(A, B, C\) are sets where \(B \cap C = \emptyset\).
  By \cref{3.10} \(A^B, A^C, A^{B \cup C}\) are sets.
  By \cref{3.5.4}, \(A^B \times A^C\) is a set.
  We define a function \(f : A^B \times A^C \to A^{B \cup C}\) by setting
  \[
    f(g, h)(x) = \begin{dcases}
      g(x) & \text{if } x \in B \\
      h(x) & \text{if } x \in C
    \end{dcases}
  \]
  where \(x \in B \cup C\), \(g : B \to A\) and \(h : C \to A\).
  We now show that \(f\) is bijective.
  We start by showing that \(f\) is injective.
  \begin{align*}
             & \forall (g, h), (g', h') \in A^B \times A^C : f(g, h) = f(g', h')                             \\
    \implies & \forall x \in B \cup C : f(g, h)(x) = f(g', h')(x)                &  & \by{3.3.7}             \\
    \implies & (\forall x \in B : f(g, h)(x) = f(g', h')(x))                                                 \\
             & \land (\forall x \in C : f(g, h)(x) = f(g', h')(x))               &  & \text{(by \cref{3.4})} \\
    \implies & (\forall x \in B : g(x) = g'(x))                                                              \\
             & \land (\forall x \in C : h(x) = h'(x))                                                        \\
    \implies & g = g' \land h = h'                                               &  & \by{3.3.7}             \\
    \implies & (g, h) = (g', h').                                                &  & \by{3.5.1}
  \end{align*}
  Thus \(f\) is injective.
  We now show that \(f\) is surjective.
  \(\forall k \in A^{B \cup C}\), we define functions \(g : B \to A\) and \(h : C \to A\) by setting
  \[
    k(x) = \begin{dcases}
      g(x) & \text{if } x \in B \\
      h(x) & \text{if } x \in C
    \end{dcases}
  \]
  Since \(g \in A^B \land h \in A^C\), by \cref{3.5.4} we have \((g, h) \in A^B \times A^C\).
  Thus \(f\) is surjective.
  Since \(f\) is both injective and surjective, we conclude that \(f\) is bijective.
  Since \(f\) is bijective, by \cref{3.6.1} we conclude that \(A^B \times A^C\) and \(A^{B \cup C}\) have same cardinality.

  Now we show that \(\forall a, b, c \in \N : a^b \times a^c = a^{b + c}\).
  Suppose that \(A, B, C\) are finite set where \(B \cap C = \emptyset\).
  By \cref{3.6.5}, \(\exists\ a, b, c \in \N\) such that \(\#(A) = a \land \#(B) = b \land \#(C) = c\).
  Then we have
  \begin{align*}
    \#(A^B \times A^C) & = \#(A^B) \times \#(A^C)             &  & \text{(by \cref{3.6.14}(e))} \\
                       & = \#(A)^{\#(B)} \times \#(A)^{\#(C)} &  & \text{(by \cref{3.6.14}(f))} \\
                       & = a^b \times a^c                                                       \\
                       & = \#(A^{B \cup C})                   &  & \text{(by proof above)}      \\
                       & = \#(A)^{\#(B \cup C)}               &  & \text{(by \cref{3.6.14}(f))} \\
                       & = \#(A)^{\#(B) + \#(C)}              &  & \text{(by \cref{3.6.14}(b))} \\
                       & = a^{b + c}.
  \end{align*}
  Thus we conclude that \(\forall a, b, c \in \N : a^b \times a^c = a^{b + c}\).
\end{proof}

\begin{ex}\label{ex:3.6.7}
  Let \(A\) and \(B\) be sets.
  Let us say that \(A\) has \emph{lesser or equal} cardinality to \(B\) if there exists an injection \(f : A \to B\) from \(A\) to \(B\).
  Show that if \(A\) and \(B\) are finite sets, then \(A\) has lesser or equal cardinality to \(B\) if and only if \(\#(A) \leq \#(B)\).
\end{ex}

\begin{proof}
  Suppose that \(A, B\) are finite sets.
  Then we have
  \begin{align*}
             & A \text{ has lesser or equal cardinality to } B                                     \\
    \implies & \exists\ f : A \to B \land f \text{ is injective}                                   \\
    \implies & f(A) \subseteq B                                  &  & \by{3.4.1}                   \\
             & \land \#(f(A)) = \#(A)                            &  & \text{(by \cref{3.6.14}(d))} \\
    \implies & \#(A) = \#(f(A)) \leq \#(B).                      &  & \text{(by \cref{3.6.14}(c))}
  \end{align*}
  And
  \begin{align*}
             & \#(A) \leq \#(B)                                                                                                       \\
    \implies & \exists\ g : \{i \in \N : 1 \leq i \leq \#(A)\} \to A                                                                  \\
             & \land g \text{ is bijective}                                                          &  & \by{3.6.5}                  \\
             & \land \exists\ g' : \{i \in \N : 1 \leq i \leq \#(B)\} \to B                                                           \\
             & \land g' \text{ is bijective}                                                         &  & \by{3.6.5}                  \\
             & \land \{i \in \N : 1 \leq i \leq \#(A)\} \subseteq \{i \in \N : 1 \leq i \leq \#(B)\}                                  \\
    \implies & g'(g^{-1}(A)) \subseteq B                                                             &  & \by{3.4.1}                  \\
             & \land g' \circ g^{-1} \text{ is bijective}                                            &  & \text{(by \cref{ex:3.3.2})} \\
    \implies & A \text{ has lesser or equal cardinality to } B.                                      &  & \text{(by \cref{3.3.20})}
  \end{align*}
  Thus we conclude that if \(A, B\) are finite sets, then \(A\) has lesser or equal cardinality to \(B\) iff \(\#(A) \leq \#(B)\).
\end{proof}

\begin{ex}\label{ex:3.6.8}
  Let \(A\) and \(B\) be sets and \(A \neq \emptyset\) such that there exists an injection \(f : A \to B\) from \(A\) to \(B\) (i.e., \(A\) has lesser or equal cardinality to \(B\)).
  Show that there exists a surjection \(g : B \to A\) from \(B\) to \(A\).
\end{ex}

\begin{proof}
  Suppose that \(A, B\) are sets, \(A \neq \emptyset\) and \(f : A \to B\) where \(f\) is injection.
  We now define a function \(g : B \to A\) as follow:
  \[
    \forall b \in B : \begin{dcases}
      g(b) \in A \setminus \{a\} & \text{if } b \in f(A \setminus \{a\})  \\
      g(b) = a                   & \text{if } b \notin f(A) \lor b = f(a)
    \end{dcases}
  \]
  where \(a \in A\) is a fixed value.
  We now show that \(g\) is surjective.
  \begin{align*}
             & \forall a' \in A : (a' = a) \lor (a' \neq a)                                                           \\
    \implies & (\exists\ b \in B : b \notin f(A) \lor b = f(a'))     &  & \text{(\(a' = a \land f\) is injective)}    \\
             & \lor (\exists\ b \in B : b \in f(A \setminus \{a'\})) &  & \text{(\(a' \neq a \land f\) is injective)} \\
    \implies & \exists\ b \in B : g(b) = a'.
  \end{align*}
  Thus \(g\) is surjective.
\end{proof}

\begin{ex}\label{ex:3.6.9}
  Let \(A\) and \(B\) be finite sets.
  Show that \(A \cup B\) and \(A \cap B\) are also finite sets, and that \(\#(A) + \#(B) = \#(A \cup B) + \#(A \cap B)\).
\end{ex}

\begin{proof}
  Suppose that \(A, B\) are finite sets.
  By \cref{3.6.14}(b) we have \(A \cup B\) is finite.
  By \cref{ex:3.1.7} we have \(A \cap B \subseteq A\), and thus by \cref{3.6.14}(c) we have \(A \cup B\) is finite.
  By \cref{3.6.10} \(\exists\ n \in \N\) such that \(\#(A) = n\).
  We use induction on \(n\) to show that \(\#(A) + \#(B) = \#(A \cup B) + \#(A \cap B)\).
  For \(n = 0\), we have \(A = \emptyset\) and
  \begin{align*}
    \#(\emptyset) + \#(B) & = 0 + \#(B)                                    &  & \by{3.6.5}                   \\
                          & = \#(B)                                                                          \\
                          & = \#(\emptyset \cup B)                         &  & \text{(by \cref{3.1.28}(a))} \\
                          & = \#(\emptyset \cup B) + 0                                                       \\
                          & = \#(\emptyset \cup B) + \#(\emptyset)         &  & \by{3.6.5}                   \\
                          & = \#(\emptyset \cup B) + \#(\emptyset \cap B). &  & \text{(by \cref{3.1.28}(a))}
  \end{align*}
  Thus the base case holds.
  Suppose inductively that for some \(n\) the statement is true.
  We now show that for \(n++\) the statement is also true.
  Since \(\#(A) = n++\), By \cref{3.6.8} we must have \(A \neq \emptyset\).
  Let \(a \in A\).
  Then we have
  \begin{align*}
     & \#(A) + \#(B)                                                                                            \\
     & = \#((A \setminus \{a\}) \cup \{a\}) + \#(B)                           &  & \text{(by \cref{3.1.28}(g))} \\
     & = \#(A \setminus \{a\}) + 1 + \#(B)                                    &  & \text{(by \cref{3.6.14}(a))} \\
     & = \#((A \setminus \{a\}) \cup B) + \#((A \setminus \{a\}) \cap B) + 1. &  & \byIH
  \end{align*}
  We now divide into two cases:
  \begin{itemize}
    \item If \(a \in B\), then we have
          \begin{align*}
             & \#((A \setminus \{a\}) \cup B) + \#((A \setminus \{a\}) \cap B) + 1                                         \\
             & = \#(A \cup B) + \#((A \setminus \{a\}) \cap B) + 1                                                         \\
             & = \#(A \cup B) + \#(((A \setminus \{a\}) \cap B) \cup \{a\})              &  & \text{(by \cref{3.6.14}(a))} \\
             & = \#(A \cup B) + \#(((A \setminus \{a\}) \cup \{a\}) \cap (B \cup \{a\})) &  & \text{(by \cref{3.1.28}(f))} \\
             & = \#(A \cup B) + \#(A \cap B).                                            &  & \text{(by \cref{3.1.28}(g))}
          \end{align*}
    \item If \(a \notin B\), then we have
          \begin{align*}
             & \#((A \setminus \{a\}) \cup B) + \#((A \setminus \{a\}) \cap B) + 1                                   \\
             & = \#((A \setminus \{a\}) \cup B) + \#(A \cap B) + 1                                                   \\
             & = \#(((A \setminus \{a\}) \cup B) \cup \{a\}) + \#(A \cap B)        &  & \text{(by \cref{3.6.14}(a))} \\
             & = \#(A \cup B) + \#(A \cap B).                                      &  & \text{(by \cref{3.1.28}(d))}
          \end{align*}
  \end{itemize}
  From all cases above we have \(\#(A) + \#(B) = \#(A \cup B) + \#(A \cap B)\) and this closes the induction.
\end{proof}

\begin{ex}\label{ex:3.6.10}
  Let \(A_1, \dots, A_n\) be finite sets such that \(\#(\bigcup_{i \in \{1, \dots, n\}} A_i) > n\).
  Show that there exists \(i \in \{1, \dots, n\}\) such that \(\#(A_i) \geq 2\).
  (This is known as the \emph{pigeonhole principle}.)
\end{ex}

\begin{proof}
  Suppose that \(n \in \N\), \(A_1, \dots, A_n\) are finite sets and \(\#(\bigcup_{i \in \{1, \dots, n\}} A_i) > n\).
  We use induction on \(n\) to show that \(\exists\ i \in \{1, \dots, n\} : \#(A_i) \geq 2\).
  We start with \(n = 1\) since for \(n = 0\) the statement is vacuously true.
  For \(n = 1\), we have
  \begin{align*}
             & \#(\bigcup_{i \in \{1, \dots, 1\}} A_i) > 1                              \\
    \implies & \#(A_1) > 1                                 &  & \text{(by \cref{3.11})} \\
    \implies & \#(A_1) \geq 2.
  \end{align*}
  Thus the base case holds.
  Suppose inductively that for some \(n\) the statement is true.
  Then for \(n++\), we have
  \begin{align*}
             & \#(\bigcup_{i \in \{1, \dots, n++\}} A_i) > n++                                                \\
    \implies & \#((\bigcup_{i \in \{1, \dots, n\}} A_i) \cup A_{n++}) > n++ &  & \text{(by \cref{3.11})}      \\
    \implies & \#(\bigcup_{i \in \{1, \dots, n\}} A_i) + \#(A_{n++}) > n++. &  & \text{(by \cref{3.6.14}(b))}
  \end{align*}
  Now we split into three cases:
  \begin{itemize}
    \item If \(\#(A_{n++}) = 0\), then we have \(\#(\bigcup_{i \in \{1, \dots, n\}} A_i) > n++ > n\).
          By induction hypothesis, \(\exists\ i \in \{1, \dots, n\} : \#(A_i) \geq 2\).
          Then we have \(\exists\ i \in \{1, \dots, n++\} : \#(A_i) \geq 2\).
    \item If \(\#(A_{n++}) = 1\), then we have \(\#(\bigcup_{i \in \{1, \dots, n\}} A_i) + 1 > n++\).
          This means \(\#(\bigcup_{i \in \{1, \dots, n\}} A_i) > n\) and by induction hypothesis, \(\exists\ i \in \{1, \dots, n\} : \#(A_i) \geq 2\).
          Then we have \(\exists\ i \in \{1, \dots, n++\} : \#(A_i) \geq 2\).
    \item If \(\#(A_{n++}) > 1\), then \(\#(A_{n++}) \geq 2\) and we have \(\exists\ i \in \{1, \dots, n++\} : \#(A_i) \geq 2\).
  \end{itemize}
  From all cases above we have \(\exists\ i \in \{1, \dots, n++\} : \#(A_i) \geq 2\).
  This closes the induction.
\end{proof}
\chapter{Integers and rationals}\label{i:ch:4}

\section{The integers}\label{sec:4.1}

\begin{defn}[Integers]\label{4.1.1}
  An \emph{integer} is an expression of the form \(a \text{---} b\), where \(a\) and \(b\) are natural numbers.
  Two integers are considered to be equal, \(a \text{---} b = c \text{---} d\), iff \(a + d = c + b\).
  We let \(\Z\) denote the set of all integers.
\end{defn}

\begin{note}
  In the language of set theory, what we are doing here is starting with the space \(\N \times \N\) of ordered pairs \((a, b)\) of natural numbers.
  Then we place an \emph{equivalence relation} \(\sim\) on these pairs by declaring \((a, b) \sim (c, d)\) iff \(a + d = c + b\).
  The set-theoretic interpretation of the symbol \(a \text{---} b\) is that it is the space of all pairs equivalent to \((a, b): a \text{---} b \coloneqq \set{(c, d) \in \N \times \N : (a, b) \sim (c, d)}\).
  However, this interpretation plays no role in how we manipulate the integers and we will not refer to it again.
  A similar set-theoretic interpretation can be given to the construction of the rational numbers later in this chapter, or the real numbers in the next chapter.
\end{note}

\begin{ac}\label{ac:4.1.1}
  The definition of equality on the integers is reflexive, symmetric and transitive.
\end{ac}

\begin{proof}
  Let \((a, b), (c, d), (e, f) \in \N \times \N\).
  Since
  \begin{align*}
             & a + b = a + b                                    \\
    \implies & a \text{---} b = a \text{---} b, &  & \by{4.1.1}
  \end{align*}
  we know that \cref{4.1.1} is reflexive.

  Suppose that \(a \text{---} b = c \text{---} d\).
  Then we have
  \begin{align*}
             & a \text{---} b = c \text{---} d                 \\
    \implies & a + d = c + b                   &  & \by{4.1.1} \\
    \implies & c + b = a + d                                   \\
    \implies & c \text{---} d = a \text{---} b &  & \by{4.1.1}
  \end{align*}
  and thus \cref{4.1.1} is symmetric.

  Finally suppose that \(a \text{---} b = c \text{---} d\) and \(c \text{---} d = e \text{---} f\).
  Then we have
  \begin{align*}
             & (a \text{---} b = c \text{---} d) \land (c \text{---} d = e \text{---} f)                 \\
    \implies & (a + d = c + b) \land (c + f = e + d)                                     &  & \by{4.1.1} \\
    \implies & (a + d + f = c + b + f) \land (c + f + b = e + d + b)                     &  & \by{2.2.1} \\
    \implies & (a + f + d = c + f + b) \land (c + f + b = e + b + d)                     &  & \by{2.2.4} \\
    \implies & a + f + d = e + b + d                                                                     \\
    \implies & a + f = e + b                                                             &  & \by{2.2.6} \\
    \implies & a \text{---} b = e \text{---} f
  \end{align*}
  and thus \cref{4.1.1} is transitive.
\end{proof}

\begin{defn}\label{4.1.2}
  The sum of two integers, \((a \text{---} b) + (c \text{---} d)\), is defined by the formula
  \[
    (a \text{---} b) + (c \text{---} d) \coloneqq (a + c) \text{---} (b + d).
  \]
  The product of two integers, \((a \text{---} b) \times (c \text{---} d)\), is defined by
  \[
    (a \text{---} b) \times (c \text{---} d) \coloneqq (ac + bd) \text{---} (ad + bc).
  \]
\end{defn}

\begin{lem}[Addition and multiplication are well-defined]\label{4.1.3}
  Let \(a, b, a', b', c, d\) be natural numbers.
  If \((a \text{---} b) = (a' \text{---} b')\), then \((a \text{---} b) + (c \text{---} d) = (a' \text{---} b') + (c \text{---} d)\) and \((a \text{---} b) \times (c \text{---} d) = (a' \text{---} b') \times (c \text{---} d)\), and also \((c \text{---} d) + (a \text{---} b) = (c \text{---} d) + (a' \text{---} b')\) and \((c \text{---} d) \times (a \text{---} b) = (c \text{---} d) \times (a' \text{---} b')\).
  Thus addition and multiplication are well-defined operations (equal inputs give equal outputs).
\end{lem}

\begin{proof}
  To prove that \((a \text{---} b) + (c \text{---} d) = (a' \text{---} b') + (c \text{---} d)\), we evaluate both sides as \((a + c) \text{---} (b + d)\) and \((a' + c) \text{---} (b' + d)\).
  Thus we need to show that \(a + c + b' + d = a' + c + b + d\).
  But since \((a \text{---} b) = (a' \text{---} b')\), we have \(a + b' = a' + b\), and so by adding \(c + d\) to both sides we obtain the claim.

  Now we show that \((a \text{---} b) \times (c \text{---} d) = (a' \text{---} b') \times (c \text{---} d)\).
  We evaluate both sides to \((ac + bd) \text{---} (ad + bc)\) and \((a'c + b'd) \text{---} (a'd + b'c)\), so we have to show that \(ac + bd + a'd + b'c = a'c + b'd + ad + bc\).
  But the left-hand side factors as \(c(a + b') + d(a' + b)\), while the right-hand side factors as \(c(a' + b) + d(a + b')\).
  Since \(a + b' = a' + b\), the two sides are equal.
  The other two identities are proven similarly.
\end{proof}

\begin{note}
  The integers \(n \text{---} 0\) behave in the same way as the natural numbers \(n\);
  indeed one can check that \((n \text{---} 0) + (m \text{---} 0) = (n + m) \text{---} 0\) and \((n \text{---} 0) \times (m \text{---} 0) = nm \text{---} 0\).
  Furthermore, \((n \text{---} 0)\) is equal to \((m \text{---} 0)\) iff \(n = m\).
  (The mathematical term for this is that there is an \emph{isomorphism} between the natural numbers \(n\) and those integers of the form \(n \text{---} 0\).)
  Thus we may \emph{identify} the natural numbers with integers by setting \(n \equiv n \text{---} 0\);
  this does not affect our definitions of addition or multiplication or equality since they are consistent with each other.
  Of course, if we set \(n\) equal to \(n \text{---} 0\), then it will also be equal to any other integer which is equal to \(n \text{---} 0\).
\end{note}

\begin{note}
  We can now define incrementation on the integers by defining \(x++ \coloneqq x + 1\) for any integer \(x\);
  this is of course consistent with our definition of the increment operation for natural numbers.
  However, this is no longer an important operation for us, as it has been now superceded by the more general notion of addition.
\end{note}

\begin{defn}[Negation of integers]\label{4.1.4}
  If \((a \text{---} b)\) is an integer, we define the negation \(-(a \text{---} b)\) to be the integer \((b \text{---} a)\).
  In particular if \(n = n \text{---} 0\) is a positive natural number, we can define its negation \(-n = 0 \text{---} n\).
\end{defn}

\begin{ac}\label{ac:4.1.2}
  The definition of negation on the integers is well-defined.
\end{ac}

\begin{proof}
  Let \(a, b, a', b' \in \N\) and \(a \text{---} b = a' \text{---} b'\).
  Then we have
  \begin{align*}
             & a \text{---} b = a' \text{---} b'                       \\
    \implies & a + b' = a' + b                         &  & \by{4.1.1} \\
    \implies & b' + a = b + a'                         &  & \by{2.2.4} \\
    \implies & b' \text{---} a' = b \text{---} a       &  & \by{4.1.1} \\
    \implies & -(a' \text{---} b') = -(a \text{---} b) &  & \by{4.1.4} \\
  \end{align*}
  and thus \cref{4.1.4} is well-defined.
\end{proof}

\begin{lem}[Trichotomy of integers]\label{4.1.5}
  Let \(x\) be an integer.
  Then exactly one of the following three statements is true:
  \begin{enumerate}
    \item \(x\) is zero.
    \item \(x\) is equal to a positive natural number \(n\).
    \item \(x\) is the negation \(-n\) of a positive natural number \(n\).
  \end{enumerate}
\end{lem}

\begin{proof}
  We first show that at least one of (a), (b), (c) is true.
  By definition, \(x = a \text{---} b\) for some natural numbers \(a, b\).
  By \cref{2.2.13}, we have three cases: \(a > b\), \(a = b\), or \(a < b\).
  If \(a > b\) then \(a = b + c\) for some positive natural number \(c\), which means that \(a \text{---} b = c \text{---} 0 = c\), which is (b).
  If \(a = b\), then \(a \text{---} b = a \text{---} a = 0 \text{---} 0 = 0\), which is (a).
  If \(a < b\), then \(b > a\), so that \(b \text{---} a = n\) for some natural number \(n\) by the previous reasoning, and thus \(a \text{---} b = -n\), which is (c).
  Now we show that no more than one of (a), (b), (c) can hold at a time.
  By \cref{2.2.7}, a positive natural number is non-zero, so (a) and (b) cannot simultaneously be true.
  If (a) and (c) were simultaneously true, then \(0 = -n\) for some positive natural \(n\);
  thus \((0 \text{---} 0) = (0 \text{---} n)\), so that \(0 + n = 0 + 0\), so that \(n = 0\), a contradiction.
  If (b) and (c) were simultaneously true, then \(n = -m\) for some positive \(n, m\), so that \((n \text{---} 0) = (0 \text{---} m)\), so that \(n + m = 0 + 0\), which contradicts \cref{2.2.8}.
  Thus exactly one of (a), (b), (c) is true for any integer \(x\).
\end{proof}

\begin{note}
  If \(n\) is a positive natural number, we call \(n\) a \emph{positive integer} and \(-n\) a \emph{negative integer}.
  Thus every integer is positive, zero, or negative, but not more than one of these at a time.
\end{note}

\begin{note}
  One could well ask why we don't use \cref{4.1.5} to \emph{define} the integers;
  i.e., why didn't we just say an integer is anything which is either a positive natural number, zero, or the negative of a natural number.
  The reason is that if we did so, the rules for adding and multiplying integers would split into many different cases (e.g., negative times positive equals negative; negative plus positive is either negative, positive, or zero, depending on which term is larger, etc.) and to verify all the properties would end up being much messier.
\end{note}

\begin{prop}[Laws of algebra for integers]\label{4.1.6}
  Let \(x\), \(y\), \(z\) be integers.
  Then we have
  \begin{align*}
    x + y               & = y + x       \\
    (x + y) + z         & = x + (y + z) \\
    x + 0 = 0 + x       & = x           \\
    x + (-x) = (-x) + x & = 0           \\
    xy                  & = yx          \\
    (xy)z               & = x(yz)       \\
    x1 = 1x             & = x           \\
    x(y + z)            & = xy + xz     \\
    (y + z)x            & = yx + zx.
  \end{align*}
\end{prop}

\begin{proof}
  There are two ways to prove these identities.
  One is to use \cref{4.1.5} and split into a lot of cases depending on whether \(x, y, z\) are zero, positive, or negative.
  This becomes very messy.
  A shorter way is to write \(x = (a \text{---} b), y = (c \text{---} d)\), and \(z = (e \text{---} f)\) for some natural numbers \(a, b, c, d, e, f\), and expand these identities in terms of \(a, b, c, d, e, f\) and use the algebra of the natural numbers.
  This allows each identity to be proven in a few lines.

  We first show that \(x + y = y + x\).
  \begin{align*}
    x + y & = (a \text{---} b) + (c \text{---} d) &                \\
          & = (a + c) \text{---} (b + d)          &   & \by{4.1.2} \\
          & = (c + a) \text{---} (d + b)          &   & \by{2.2.4} \\
          & = (c \text{---} d) + (a \text{---} b) &   & \by{4.1.2} \\
          & = y + x.
  \end{align*}
  Thus the addition on integers is commutative.

  Next we show that \((x + y) + z = x + (y + z)\).
  \begin{align*}
    (x + y) + z & = \big((a \text{---} b) + (c \text{---} d)\big) + (e \text{---} f)                 \\
                & = \big((a + c) \text{---} (b + d)\big) + (e \text{---} f)          &  & \by{4.1.2} \\
                & = \big((a + c) + e\big) \text{---} \big((b + d) + f\big)           &  & \by{4.1.2} \\
                & = \big(a + (c + e)\big) \text{---} \big(b + (d + f)\big)           &  & \by{2.2.5} \\
                & = (a \text{---} b) + \big((c + e) \text{---} (d + f)\big)          &  & \by{4.1.2} \\
                & = (a \text{---} b) + \big((c \text{---} d) + (e \text{---} f)\big) &  & \by{4.1.2} \\
                & = x + (y + z).
  \end{align*}
  Thus the addition on integers is associative.

  Next we show that \(x + 0 = 0 + x = x\).
  Since the addition on integers is commutative, we know that \(x + 0 = 0 + x\).
  So we only need to show that \(x + 0 = x\).
  \begin{align*}
    x + 0 & = (a \text{---} b) + (0 \text{---} 0)                 \\
          & = (a + 0) \text{---} (b + 0)          &  & \by{4.1.2} \\
          & = (a \text{---} b)                    &  & \by{2.2.2} \\
          & = x.
  \end{align*}
  Thus \(0\) is the additive identity on integers.

  Next we show that \(x + (-x) = (-x) + x = 0\).
  Since the addition on integers is commutative, we know that \(x + (-x) = (-x) + x\).
  So we only need to show that \(x + (-x) = 0\).
  \begin{align*}
    x + (-x) & = (a \text{---} b) + (-x)                             \\
             & = (a \text{---} b) + (b \text{---} a) &  & \by{4.1.4} \\
             & = (a + b) \text{---} (b + a)          &  & \by{4.1.2} \\
             & = (a + b) \text{---} (a + b)          &  & \by{2.2.4} \\
             & = (a \text{---} a) + (b \text{---} b) &  & \by{4.1.2} \\
             & = (0 \text{---} 0) + (0 \text{---} 0) &  & \by{4.1.1} \\
             & = 0 + 0                                               \\
             & = 0.                                  &  & \by{2.2.2}
  \end{align*}
  Thus the additive inverse of integer \(x\) is \(-x\).

  Next we show that \(xy = yx\).
  \begin{align*}
    xy & = (a \text{---} b) \times (c \text{---} d)                 \\
       & = (ac + bd) \text{---} (ad + bc)           &  & \by{4.1.2} \\
       & = (ca + db) \text{---} (da + cb)           &  & \by{2.3.2} \\
       & = (ca + db) \text{---} (cb + da)           &  & \by{2.2.4} \\
       & = (c \text{---} d) \times (a \text{---} b) &  & \by{4.1.2} \\
       & = yx.
  \end{align*}
  Thus the multiplication on integers is commutative.

  Next we show that \((xy)z = x(yz)\).
  \begin{align*}
    (xy)z & = \big((a \text{---} b) \times (c \text{---} d)\big) \times (e \text{---} f)                 \\
          & = \big((ac + bd) \text{---} (ad + bc)\big) \times (e \text{---} f)           &  & \by{4.1.2} \\
          & = \big((ac + bd)e + (ad + bc)f\big)                                                          \\
          & \quad \text{---} \big((ac + bd)f + (ad + bc)e\big)                           &  & \by{4.1.2} \\
          & = \big((ac)e + (bd)e + (ad)f + (bc)f\big)                                                    \\
          & \quad \text{---} \big((ac)f + (bd)f + (ad)e + (bc)e\big)                     &  & \by{2.3.4} \\
          & = \big(a(ce) + b(de) + a(df) + b(cf)\big)                                                    \\
          & \quad \text{---} \big(a(cf) + b(df) + a(de) + b(ce)\big)                     &  & \by{2.3.5} \\
          & = \big(a(ce) + a(df) + b(cf) + b(de)\big)                                                    \\
          & \quad \text{---} \big(a(cf) + a(de) + b(ce) + b(df)\big)                     &  & \by{2.2.4} \\
          & = \big(a(ce + df) + b(cf + de)\big)                                                          \\
          & \quad \text{---} \big(a(cf + de) + b(ce + df)\big)                           &  & \by{2.3.4} \\
          & = (a \text{---} b) \times \big((ce + df) \text{---} (cf + de)\big)           &  & \by{4.1.2} \\
          & = (a \text{---} b) \times \big((c \text{---} d) \times (e \text{---} f)\big) &  & \by{4.1.2} \\
          & = x(yz).
  \end{align*}
  Thus the multiplication on integers is associative.

  Next we show that \(x1 = 1x = x\).
  Since the multiplication on integers is commutative, we know that \(x1 = 1x\).
  So we only need to show that \(x1 = x\).
  \begin{align*}
    x1 & = (a \text{---} b) \times (1 \text{---} 0)                    \\
       & = (a1 + b0) \text{---} (a0 + b1)           &  & \by{4.1.2}    \\
       & = (a + b0) \text{---} (a0 + b)             &  & \by{ac:2.3.4} \\
       & = (a + 0) \text{---} (0 + b)               &  & \by{ac:2.3.2} \\
       & = (a + 0) \text{---} (b + 0)               &  & \by{2.2.4}    \\
       & = (a \text{---} b)                         &  & \by{2.2.2}    \\
       & = x.
  \end{align*}
  Thus \(1\) is the multiplicative identity on integers.

  Next we show that \(x(y + z) = xy + xz\).
  \begin{align*}
    x(y + z) & = (a \text{---} b) \times \big((c \text{---} d) + (e \text{---} f)\big)                               \\
             & = (a \text{---} b) \times \big((c + e) \text{---} (d + f)\big)                        &  & \by{4.1.2} \\
             & = \big(a(c + e) + b(d + f)\big) \text{---} \big(a(d + f) + b(c + e)\big)              &  & \by{4.1.2} \\
             & = (ac + ae + bd + bf) \text{---} (ad + af + bc + be)                                  &  & \by{2.3.4} \\
             & = (ac + bd + ae + bf) \text{---} (ad + bc + af + be)                                  &  & \by{2.2.4} \\
             & = \big((ac + bd) \text{---} (ad + bc)\big) + \big((ae + bf) \text{---} (af + be)\big) &  & \by{4.1.2} \\
             & = (a \text{---} b) \times (c \text{---} d) + (a \text{---} b) \times (e \text{---} f) &  & \by{4.1.2} \\
             & = xy + xz.
  \end{align*}
  Thus the multiplication and addition on integers are left distributive.

  Finally we show that \((y + z)x = yx + zx\).
  \begin{align*}
    (y + z)x & = x(y + z) &  & \text{(multiplication is commutative)}                     \\
             & = xy + xz  &  & \text{(multiplication and addition are left distributive)} \\
             & = yx + zx. &  & \text{(multiplication is commutative)}
  \end{align*}
  Thus the multiplication and addition on integers are right distributive.
\end{proof}

\begin{rmk}\label{4.1.7}
  The above set of nine identities have a name; they are asserting that the integers form a \emph{commutative ring}.
  (If one deleted the identity \(xy = yx\), then they would only assert that the integers form a \emph{ring}).
  Note that some of these identities were already proven for the natural numbers, but this does not automatically mean that they also hold for the integers because the integers are a larger set than the natural numbers.
  On the other hand, this proposition supercedes many of the propositions derived earlier for natural numbers.
\end{rmk}

\begin{note}
  We now define the operation of \emph{subtraction} \(x - y\) of two integers by the formula
  \[
    x - y \coloneqq x + (-y).
  \]
  We do not need to verify the substitution axiom for this operation, since we have defined subtraction in terms of two other operations on integers, namely addition and negation, and we have already verified that those operations are well-defined.
\end{note}

\begin{note}
  One can easily check now that if \(a\) and \(b\) are natural numbers, then
  \[
    a - b = a + -b = (a \text{---} 0) + (0 \text{---} b) = a \text{---} b,
  \]
  and so \(a \text{---} b\) is just the same thing as \(a - b\).
  Because of this we can now discard the ----- notation, and use the familiar operation of subtraction instead.
  (As remarked before, we could not use subtraction immediately because it would be circular.)
\end{note}

\begin{ac}\label{ac:4.1.3}
  Let \(x\) be an integer.
  Then \(-x = (-1)x\).
\end{ac}

\begin{proof}
  Let \(x = (a \text{---} b)\) where \(a, b \in \N\).
  Then we have
  \begin{align*}
    -x & = -(a \text{---} b)                                   \\
       & = (b \text{---} a)                 &  & \by{4.1.4}    \\
       & = (1b \text{---} 1a)               &  & \by{ac:2.3.4} \\
       & = (0 + 1b) \text{---} (0 + 1a)     &  & \by{2.2.1}    \\
       & = (0a + 1b) \text{---} (0b + 1a)   &  & \by{2.3.3}    \\
       & = (0 \text{---} 1)(a \text{---} b) &  & \by{4.1.2}    \\
       & = (-1)(a \text{---} b)             &  & \by{4.1.4}    \\
       & = (-1)x.
  \end{align*}
\end{proof}

\begin{ac}\label{ac:4.1.4}
  Let \(x\) be an integer.
  Then \(x = -(-x)\).
\end{ac}

\begin{proof}
  Let \(x = (a \text{---} b)\) where \(a, b \in \N\).
  Then we have
  \begin{align*}
    -(-x) & = -(b \text{---} a) &  & \by{4.1.4} \\
          & = (a \text{---} b)  &  & \by{4.1.4} \\
          & = x.
  \end{align*}
\end{proof}

\begin{ac}\label{ac:4.1.5}
  Let \(x, y\) be integers.
  Then \((-x)(-y) = xy\).
\end{ac}

\begin{proof}
  We have
  \begin{align*}
    (-x)(-y) & = \big((-1)x\big) \big((-1)y\big) &  & \by{ac:4.1.3} \\
             & = (-1)\big(x(-1)\big)y            &  & \by{4.1.6}    \\
             & = (-1)\big((-1)x\big)y            &  & \by{4.1.6}    \\
             & = \big((-1)(-1)\big)(xy)          &  & \by{4.1.6}    \\
             & = \big(-(-1)\big)(xy)             &  & \by{ac:4.1.3} \\
             & = 1(xy)                           &  & \by{ac:4.1.4} \\
             & = xy.                             &  & \by{4.1.6}
  \end{align*}
\end{proof}

\begin{prop}[Integers have no zero divisors]\label{4.1.8}
  Let \(a\) and \(b\) be integers such that \(ab = 0\).
  Then either \(a = 0\) or \(b = 0\) (or both).
\end{prop}

\begin{proof}
  Suppose for sake of contradiction that \(a \neq 0 \land b \neq 0\).
  By \cref{4.1.5} we now split into four cases:
  \begin{enumerate}
    \item \(a\) is positive and \(b\) is positive.
          Then by \cref{2.3.3} we know that \(a = 0 \lor b = 0\), a contradiction.
    \item \(a\) is negative and \(b\) is positive.
          Then by \cref{4.1.4} \(-a\) is positive and
          \begin{align*}
                     & ab + (-ab) = 0       &  & \by{4.1.6}    \\
            \implies & 0 + (-ab) = 0                           \\
            \implies & -ab = 0              &  & \by{2.2.1}    \\
            \implies & (-1)(ab) = 0         &  & \by{ac:4.1.3} \\
            \implies & \big((-1)a\big)b = 0 &  & \by{4.1.6}    \\
            \implies & (-a)b = 0            &  & \by{ac:4.1.3} \\
            \implies & -a = 0 \lor b = 0    &  & \by{2.3.3}    \\
            \implies & a = -0 \lor b = 0    &  & \by{ac:4.1.4} \\
            \implies & a = 0 \lor b = 0,    &  & \by{4.1.4}
          \end{align*}
          a contradiction.
    \item \(a\) is positive and \(b\) is negative.
          Then by \cref{4.1.4} \(-b\) is positive and by \cref{4.1.6} we have \(-ab = -ba\).
          From case (b) we know that \(b = 0 \lor a = 0\), a contradiction.
    \item \(a\) is negative and \(b\) is negative.
          Then by \cref{4.1.4} \(-a\) and \(-b\) are positive.
          By \cref{ac:4.1.5} we know that \((-a)(-b) = ab\).
          This means \(-a = 0 \lor -b = 0\) and by \cref{ac:4.1.3,ac:4.1.4} we have \(a = 0 \lor b = 0\), a contradiction.
  \end{enumerate}
  From all cases above we derived contradiction.
  Thus we must have
  \[
    \lnot(a \neq 0 \land b \neq 0) \iff a = 0 \lor b = 0.
  \]
\end{proof}

\begin{cor}[Cancellation law for integers]\label{4.1.9}
  If \(a, b, c\) are integers such that \(ac = bc\) and \(c\) is non-zero, then \(a = b\).
\end{cor}

\begin{proof}
  We have
  \begin{align*}
             & ac = bc                                                              \\
    \implies & ac + (-bc) = bc + (-bc) = 0             &            & \by{4.1.6}    \\
    \implies & ac + (-1)(bc) = 0                       &            & \by{ac:4.1.3} \\
    \implies & ac + \big((-1)b\big)c = 0               &            & \by{4.1.6}    \\
    \implies & ac + (-b)c = 0                          &            & \by{ac:4.1.3} \\
    \implies & \big(a + (-b)\big)c = 0                 &            & \by{4.1.6}    \\
    \implies & \big(a + (-b) = 0\big) \lor (c = 0)     &            & \by{4.1.8}    \\
    \implies & a + (-b) = 0                            & (c \neq 0)                 \\
    \implies & a + (-b) + b = 0 + b = b                &            & \by{4.1.6}    \\
    \implies & a + \big((-b) + b\big) = a + 0 = a = b. &            & \by{4.1.6}
  \end{align*}
\end{proof}

\begin{defn}[Ordering of the integers]\label{4.1.10}
  Let \(n\) and \(m\) be integers.
  We say that \(n\) is \emph{greater than or equal} to \(m\), and write \(n \geq m\) or \(m \leq n\), iff we have \(n = m + a\) for some natural number \(a\).
  We say that \(n\) is \emph{strictly greater than} \(m\), and write \(n > m\) or \(m < n\), iff \(n \geq m\) and \(n \neq m\).
\end{defn}

\begin{lem}[Properties of order]\label{4.1.11}
  Let \(a, b, c\) be integers.
  \begin{enumerate}
    \item \(a > b\) iff \(a - b\) is a positive natural number.
    \item (Addition preserves order) If \(a > b\), then \(a + c > b + c\).
    \item (Positive multiplication preserves order) If \(a > b\) and \(c\) is positive, then \(ac > bc\).
    \item (Negation reverses order) If \(a > b\), then \(-a < -b\).
    \item (Order is transitive) If \(a > b\) and \(b > c\), then \(a > c\).
    \item (Order trichotomy) Exactly one of the statements \(a > b\), \(a < b\), or \(a = b\) is true.
  \end{enumerate}
\end{lem}

\begin{proof}{(a)}
  We have
  \begin{align*}
         & a > b                                                            \\
    \iff & (\exists m \in \N : a = b + m) \land (a \neq b) &  & \by{4.1.10} \\
    \iff & \exists m \in \N : a - b = m \neq 0             &  & \by{4.1.6}  \\
    \iff & a - b \text{ is positive}.                      &  & \by{2.2.7}
  \end{align*}
\end{proof}

\begin{proof}{(b)}
  We have
  \begin{align*}
             & a > b                                                            \\
    \implies & (\exists m \in \N : a = b + m) \land (a \neq b) &  & \by{4.1.10} \\
    \implies & (a + c = b + m + c) \land (a + c \neq b + c)    &  & \by{4.1.3}  \\
    \implies & (a + c = b + c + m) \land (a + c \neq b + c)    &  & \by{4.1.6}  \\
    \implies & a + c > b + c.                                  &  & \by{4.1.10}
  \end{align*}
\end{proof}

\begin{proof}{(c)}
  We have
  \begin{align*}
             & a > b                                                                             \\
    \implies & (\exists m \in \N : a - b = m) \land (m > 0)    &  & \text{(by \cref{4.1.11}(a))} \\
    \implies & \big((a - b)c = ac - bc = mc\big) \land (m > 0) &  & \by{4.1.6}                   \\
    \implies & (ac = bc + mc) \land (m > 0)                    &  & \by{4.1.6}                   \\
    \implies & (ac = bc + mc) \land (mc > 0c)                  &  & \by{2.3.6}                   \\
    \implies & (ac = bc + mc) \land (mc > 0)                   &  & \by{2.3.1}                   \\
    \implies & ac > bc.                                        &  & \text{(by \cref{4.1.11}(a))}
  \end{align*}
\end{proof}

\begin{proof}{(d)}
  We have
  \begin{align*}
             & a > b                                                                                \\
    \implies & (\exists m \in \N : a - b = m) \land (m > 0)       &  & \text{(by \cref{4.1.11}(a))} \\
    \implies & (b - a = -m) \land (m > 0)                         &  & \by{4.1.4}                   \\
    \implies & \big((-b) + b - a = (-b) + (-m)\big) \land (m > 0) &  & \by{4.1.3}                   \\
    \implies & \big(-a = (-b) + (-m)\big) \land (m > 0)           &  & \by{4.1.6}                   \\
    \implies & \big((-a) + m = (-b) + (-m) + m\big) \land (m > 0) &  & \by{4.1.3}                   \\
    \implies & \big((-a) + m = -b\big) \land (m > 0)              &  & \by{4.1.6}                   \\
    \implies & -b > -a.                                           &  & \text{(by \cref{4.1.11}(a))}
  \end{align*}
\end{proof}

\begin{proof}{(e)}
  We have
  \begin{align*}
             & (a > b) \land (b > c)                                                                 \\
    \implies & \exists m, n \in \N : (a = b + m) \land (b = c + n)                                   \\
             & \land (m > 0) \land (n > 0)                         &  & \text{(by \cref{4.1.11}(a))} \\
    \implies & (a = c + n + m) \land (m > 0) \land (n > 0)         &  & \by{4.1.3}                   \\
    \implies & (a = c + n + m) \land (n + m > 0)                   &  & \by{2.2.8}                   \\
    \implies & a > c.                                              &  & \text{(by \cref{4.1.11}(a))}
  \end{align*}
\end{proof}

\begin{proof}{(f)}
  By \cref{4.1.5}, \(a - b\) can be exactly one of the following three statements:
  \begin{itemize}
    \item \(a - b = 0\).
          Then we have
          \begin{align*}
                 & a - b = 0                            \\
            \iff & a + (-b) = 0                         \\
            \iff & a + (-b) + b = 0 + b &  & \by{4.1.3} \\
            \iff & a + 0 = 0 + b        &  & \by{4.1.6} \\
            \iff & a = b.               &  & \by{4.1.6}
          \end{align*}
    \item \(a - b\) is a positive natural number.
          Then by \cref{4.1.11}(a) we have \(a > b\).
    \item \(a - b = -n\) where \(n\) is a positive natural number.
          Then we have
          \begin{align*}
                 & a - b = -n                          \\
            \iff & -(a - b) = -(-n) &  & \by{ac:4.1.2} \\
            \iff & (b - a) = n      &  & \by{4.1.4}
          \end{align*}
          and by \cref{4.1.11}(a) we have \(b > a\).
          By \cref{4.1.10} we have \(b > a \iff a < b\).
  \end{itemize}
  Thus we conclude that exactly one of the statements \(a = b\), \(a > b\), or \(a < b\) is true.
\end{proof}

\exercisesection

\begin{ex}\label{ex:4.1.1}
  Verify that the definition of equality on the integers is both reflexive and symmetric.
\end{ex}

\begin{proof}
  See \cref{ac:4.1.1}.
\end{proof}

\begin{ex}\label{ex:4.1.2}
  Show that the definition of negation on the integers is well-defined in the sense that if \((a \text{---} b) = (a' \text{---} b')\), then \(-(a \text{---} b) = -(a' \text{---} b')\)
  (so equal integers have equal negations).
\end{ex}

\begin{proof}
  See \cref{ac:4.1.2}.
\end{proof}

\begin{ex}\label{ex:4.1.3}
  Show that \((-1) \times a = -a\) for every integer \(a\).
\end{ex}

\begin{proof}
  See \cref{ac:4.1.3}.
\end{proof}

\begin{ex}\label{ex:4.1.4}
  Prove the remaining identities in \cref{4.1.6}.
\end{ex}

\begin{proof}
  See \cref{4.1.6}.
\end{proof}

\begin{ex}\label{ex:4.1.5}
  Prove \cref{4.1.8}.
\end{ex}

\begin{proof}
  See \cref{4.1.8}.
\end{proof}

\begin{ex}\label{ex:4.1.6}
  Prove \cref{4.1.9}.
\end{ex}

\begin{proof}
  See \cref{4.1.9}.
\end{proof}

\begin{ex}\label{ex:4.1.7}
  Prove \cref{4.1.11}.
\end{ex}

\begin{proof}
  See \cref{4.1.11}.
\end{proof}

\begin{ex}\label{ex:4.1.8}
  Show that the principle of induction (\cref{2.5}) does not apply directly to the integers.
  More precisely, give an example of a property \(P(n)\) pertaining to an integer \(n\) such that \(P(0)\) is true, and that \(P(n)\) implies \(P(n++)\) for all integers \(n\), but that \(P(n)\) is not true for all integers \(n\).
  Thus induction is not as useful a tool for dealing with the integers as it is with the natural numbers.
  (The situation becomes even worse with the rational and real numbers, which we shall define shortly.)
\end{ex}

\begin{proof}
  For sake of contradiction, we claim that \(\forall n \in \Z\), \(n + 1 > 0\).
  And we use \cref{2.5} to prove the above claim.
  For \(n = 0\), by \cref{4.1.10} we know that \((0 + 1 = 1) \land (1 \neq 0) \implies 1 > 0\), so the base case holds.
  Suppose inductively that for some \(n\) the statement \(n + 1 > 0\) is true.
  Then for \(n + 1\), we need to show that \((n + 1) + 1 > 0\).
  By induction hypothesis we know that \(n + 1 > 0\).
  Then by \cref{4.1.11}(b) we have \((n + 1) + 1 > 0 + 1 = 1\).
  Since \((n + 1) + 1 > 1\) and \(1 > 0\), by \cref{4.1.11}(e) we have \((n + 1) + 1 > 0\).
  This close induction.

  Now let \(n = -1\).
  By \cref{4.1.6} we have \((-1) + 1 = 0\).
  But this means \(0 > 0\), which by \cref{4.1.10} we have \(0 \neq 0\), a contradiction.
  This means induction is not as useful a tool for dealing with the integers as it is with the natural numbers.
\end{proof}
\section{The rationals}\label{sec:4.2}

\begin{defn}\label{4.2.1}
  A \emph{rational number} is an expression of the form \(a // b\), where \(a\) and \(b\) are integers and \(b\) is non-zero;
  \(a // 0\) is not considered to be a rational number.
  Two rational numbers are considered to be equal, \(a // b = c // d\), iff \(ad = cb\).
  The set of all rational numbers is denoted \(\Q\).
\end{defn}

\begin{note}
  There is no reasonable way we can divide by zero, since one cannot have both the identities \((a / b) \times b = a\) and \(c \times 0 = 0\) hold simultaneously if \(b\) is allowed to be zero and \(a\) is non-zero.
  Similarly, the identities \(a / a = 1\) and \(2 \times (a / a) = (2 \times a) / a\) cannot hold simultaneously if \(0 / 0\) is defined.
  However, we can eventually get a reasonable notion of dividing by a quantity which approaches zero
  - think of L'H\^opital's rule (see \cref{sec:10.5}), which suffices for doing things like defining differentiation.
\end{note}

\begin{ac}\label{ac:4.2.1}
  The definition of equality for the rational numbers is reflexive, symmetric and transitive.
\end{ac}

\begin{proof}[\pf{ac:4.2.1}]
  Let \(a // b\), \(c // d\), \(e // f\) be rational numbers where \(a, b, c, d, e, f \in \Z\) and \(b, d, f \neq 0\).
  Since
  \begin{align*}
             & ab = ab          &  & \by{4.1.3} \\
    \implies & a // b = a // b, &  & \by{4.2.1}
  \end{align*}
  we know that \cref{4.2.1} is reflexive.

  Next suppose that \(a // b = c // d\).
  Then we have
  \begin{align*}
             & a // b = c // d                  \\
    \implies & ad = cb          &  & \by{4.2.1} \\
    \implies & cb = ad          &  & \by{4.1.3} \\
    \implies & c // d = a // b. &  & \by{4.2.1}
  \end{align*}
  Thus \cref{4.2.1} is symmetric.

  Finally suppose that \(a // b = c // d\) and \(c // d = e // f\).
  Then we have
  \begin{align*}
             & (a // b = c // d) \land (c // d = e // f)                 \\
    \implies & (ad = cb) \land (cf = ed)                 &  & \by{4.2.1} \\
    \implies & (adf = cbf) \land (cfb = edb)             &  & \by{4.1.3} \\
    \implies & (afd = cbf) \land (cbf = ebd)             &  & \by{4.1.6} \\
    \implies & afd = ebd                                 &  & \by{4.1.3} \\
    \implies & af = eb                                   &  & \by{4.1.9} \\
    \implies & a // b = e // f.                          &  & \by{4.2.1}
  \end{align*}
  Thus \cref{4.2.1} is transitive.
\end{proof}

\begin{defn}\label{4.2.2}
  If \(a // b\) and \(c // d\) are rational numbers, we define their sum
  \[
    (a // b) + (c // d) \coloneqq (ad + bc) // (bd)
  \]
  their product
  \[
    (a // b) \times (c // d) \coloneqq (ac) // (bd)
  \]
  and the negation
  \[
    -(a // b) \coloneqq (-a) // b.
  \]

  Note that if \(b\) and \(d\) are non-zero, then \(bd\) is also non-zero, by \cref{4.1.8}, so the sum or product of two rational numbers remains a rational number.
\end{defn}

\begin{lem}\label{4.2.3}
  The sum, product, and negation operations on rational numbers are well-defined, in the sense that if one replaces \(a // b\) with another rational number \(a' // b'\) which is equal to \(a // b\), then the output of the above operations remains unchanged, and similarly for \(c // d\).
\end{lem}

\begin{proof}[\pf{4.2.3}]
  We first show that the addition on rationals numbers is well-defined.
  Suppose \(a // b = a' // b'\), so that \(b\) and \(b'\) are non-zero and \(ab' = a'b\).
  We now show that \((a // b) + (c // d) = (a' // b') + (c // d)\).
  By \cref{4.2.2}, the left-hand side is \((ad + bc) // bd\) and the right-hand side is \((a'd + b'c) // b'd\).
  So by \cref{4.2.1}, we have to show that
  \[
    (ad + bc)b'd = (a'd + b'c)bd,
  \]
  which expands to
  \[
    ab'd^2 + bb'cd = a'bd^2 + bb'cd.
  \]
  But since \(ab' = a'b\), the claim follows.
  Similarly suppose \(c // d = c' // d'\), so that \(d\) and \(d'\) are non-zero and \(cd' = c'd\).
  We show that \((a // b) + (c // d) = (a // b) + (c' // d')\).
  By \cref{4.2.2}, the left-hand side is \((ad + bc) // bd\) and the right-hand side is \((ad' + bc') // bd'\).
  So by \cref{4.2.1}, we have to show that
  \[
    (ad + bc)bd' = (ad' + bc')bd,
  \]
  which expands to
  \[
    abdd' + b^2cd' = abdd' + b^2c'd.
  \]
  But since \(cd' = c'd\), the claim follows.

  Next we show that the multiplication on rationals numbers is well-defined.
  Suppose \(a // b = a' // b'\), so that \(b\) and \(b'\) are non-zero and \(ab' = a'b\).
  We now show that \((a // b) \times (c // d) = (a' // b') \times (c // d)\).
  By \cref{4.2.2}, the left-hand side is \((ac) // (bd)\) and the right-hand side is \((a'c) // (b'd)\).
  So by \cref{4.2.1}, we have to show that
  \[
    (ac)(b'd) = (a'c)(bd),
  \]
  which is equivalent to
  \[
    ab'cd = a'bcd.
  \]
  But since \(ab' = a'b\), the claim follows.
  Similarly suppose \(c // d = c' // d'\), so that \(d\) and \(d'\) are non-zero and \(cd' = c'd\).
  We now show that \((a // b) \times (c // d) = (a // b) \times (c' // d')\).
  By \cref{4.2.2}, the left-hand side is \((ac) // (bd)\) and the right-hand side is \((ac') // (bd')\).
  So by \cref{4.2.1}, we have to show that
  \[
    (ac)(bd') = (ac')(bd),
  \]
  which is equivalent to
  \[
    abcd' = abc'd.
  \]
  But since \(cd' = c'd\), the claim follows.

  Finally we show that the negation on rationals numbers is well-defined.
  Suppose \(a // b = a' // b'\), so that \(b\) and \(b'\) are non-zero and \(ab' = a'b\).
  We now show that \(-(a // b) = -(a' // b')\).
  By \cref{4.2.2}, the left-hand side is \((-a) // b\) and the right-hand side is \((-a') // b'\).
  So by \cref{4.2.1}, we have to show that
  \[
    (-a)b' = (-a')b,
  \]
  which by \cref{ac:4.1.5} is equivalent to
  \[
    (-1)ab' = (-1)a'b.
  \]
  But since \(ab' = a'b\), the claim follows from \cref{4.1.3}.
\end{proof}

\begin{ac}\label{ac:4.2.2}
  The rational numbers \(a // 1\) behave in a manner identical to the integers \(a\):
  \begin{align*}
    (a // 1) + (b // 1)      & = (a + b) // 1; \\
    (a // 1) \times (b // 1) & = (ab // 1);    \\
    -(a // 1)                & = (-a) // 1.
  \end{align*}
  Also, \(a // 1\) and \(b // 1\) are only equal when \(a\) and \(b\) are equal.
  Because of this, we will identify \(a\) with \(a // 1\) for each integer \(a\): \(a \equiv a // 1\);
  the above identities then guarantee that the arithmetic of the integers is consistent with the arithmetic of the rationals.
  Thus just as we embedded the natural numbers inside the integers, we embed the integers inside the rational numbers.
  In particular, all natural numbers are rational numbers, for instance \(0\) is equal to \(0 // 1\) and \(1\) is equal to \(1 // 1\).

  Observe that a rational number \(a // b\) is equal to \(0 = 0 // 1\) iff \(a \times 1 = b \times 0\), i.e., if the numerator \(a\) is equal to \(0\).
  Thus if \(a\) and \(b\) are non-zero then so is \(a // b\).
\end{ac}

\begin{proof}[\pf{ac:4.2.2}]
  Let \(a, b \in \Z\).
  First we show that \((a // 1) + (b // 1) = (a + b) // 1\).
  This is true since
  \begin{align*}
    (a // 1) + (b // 1) & = (a1 + b1) // (1 \times 1) &  & \by{4.2.2} \\
                        & = (a + b) // 1.             &  & \by{4.1.6}
  \end{align*}

  Next we show that \((a // 1) \times (b // 1) = (ab) // 1\).
  This is true since
  \begin{align*}
    (a // 1) \times (b // 1) & = (ab) // (1 \times 1) &  & \by{4.2.2} \\
                             & = (ab) // 1.           &  & \by{4.1.6}
  \end{align*}

  Next we show that \(-(a // 1) = (-a) // 1\).
  This is true by \cref{4.2.2}.

  Finally we show that \(a // 1 = b // 1 \iff a = b\).
  This is true since
  \begin{align*}
         & a // 1 = b // 1                 \\
    \iff & a1 = b1         &  & \by{4.2.1} \\
    \iff & a = b.          &  & \by{4.1.6}
  \end{align*}
\end{proof}

\begin{ac}\label{ac:4.2.3}
  We now define a new operation on the rationals: reciprocal.
  If \(x = a // b\) is a non-zero rational (so that \(a, b \neq 0\)) then we define the \emph{reciprocal} \(x^{-1}\) of \(x\) to be the rational number \(x^{-1} \coloneqq b // a\).
  The reciprocal operation on rational numbers is consistent with \cref{4.2.1}:
  if two rational numbers \(a // b\), \(a' // b'\) are equal, then their reciprocals are also equal.
  In contrast to reciprocal, an operation such as ``numerator'' is not well-defined:
  the rationals \(3 // 4\) and \(6 // 8\) are equal, but have unequal numerators, so we have to be careful when referring to such terms as ``the numerator of \(x\)''.
  We however leave the reciprocal of \(0\) undefined.
\end{ac}

\begin{proof}[\pf{ac:4.2.3}]
  By \cref{4.2.1} and the definition of reciprocal we have \(a, a', b, b' \neq 0\).
  Then we have
  \begin{align*}
             & a // b = a' // b'                  \\
    \implies & ab' = a'b          &  & \by{4.2.1} \\
    \implies & b'a = ba'          &  & \by{4.1.6} \\
    \implies & b' // a' = b // a. &  & \by{4.2.1}
  \end{align*}
\end{proof}

\begin{prop}[Laws of algebra for rationals]\label{4.2.4}
  Let \(x, y, z\) be rationals.
  Then the following laws of algebra hold:
  \begin{align*}
    x + y               & = y + x       \\
    (x + y) + z         & = x + (y + z) \\
    x + 0 = 0 + x       & = x           \\
    x + (-x) = (-x) + x & = 0           \\
    xy                  & = yx          \\
    (xy)z               & = x(yz)       \\
    x1 = 1x             & = x           \\
    x(y + z)            & = xy + xz     \\
    (y + z)x            & = yx + zx.
  \end{align*}
  If \(x\) is non-zero, we also have
  \[
    xx^{-1} = x^{-1}x = 1.
  \]
\end{prop}

\begin{proof}[\pf{4.2.4}]
  To prove this identity, one writes \(x = a // b\), \(y = c // d\), \(z = e // f\) for some integers \(a\), \(c\), \(e\) and non-zero integers \(b\), \(d\), \(f\), and verifies each identity in turn using the algebra of the integers.

  First we show that \(x + y = y + x\).
  \begin{align*}
    x + y & = (a // b) + (c // d)                 \\
          & = (ad + bc) // bd     &  & \by{4.2.2} \\
          & = (bc + ad) // bd     &  & \by{4.1.6} \\
          & = (cb + da) // db     &  & \by{4.1.6} \\
          & = (c // d) + (a // b) &  & \by{4.2.2} \\
          & = y + x.
  \end{align*}
  Thus the addition on rationals is commutative.

  Next we show that \((x + y) + z = x + (y + z)\).
  \begin{align*}
    (x + y) + z & = ((a // b) + (c // d)) + (e // f)                   \\
                & = ((ad + bc) // bd) + (e // f)       &  & \by{4.2.2} \\
                & = ((ad + bc)f + (bd)e) // ((bd)f)    &  & \by{4.2.2} \\
                & = ((ad)f + (bc)f + (bd)e) // ((bd)f) &  & \by{4.1.6} \\
                & = (a(df) + b(cf) + b(de)) // (b(df)) &  & \by{4.1.6} \\
                & = (a(df) + b(cf + de)) // (b(df))    &  & \by{4.1.6} \\
                & = (a // b) + ((cf + de) // df)       &  & \by{4.2.2} \\
                & = (a // b) + ((c // d) + (e // f))   &  & \by{4.2.2} \\
                & = x + (y + z).
  \end{align*}
  Thus the addition on rationals is associative.

  Next we show that \(x + 0 = 0 + x = x\).
  Since the addition on rationals is commutative, we know that \(x + 0 = 0 + x\).
  Thus we only need to show that \(x + 0 = x\).
  \begin{align*}
    x + 0 & = (a // b) + (0 // 1) &  & \by{ac:4.2.2} \\
          & = (a1 + b0) // b1     &  & \by{4.2.2}    \\
          & = (a + 0) // b        &  & \by{4.1.6}    \\
          & = a // b              &  & \by{4.1.6}    \\
          & = x.
  \end{align*}
  Thus \(0\) is the additive identity on rationals.

  Next we show that \(x + (-x) = (-x) + x = 0\).
  Since the addition on rationals is commutative, we know that \(x + (-x) = (-x) + x\).
  Thus we only need to show that \(x + (-x) = 0\).
  \begin{align*}
    x + (-x) & = (a // b) + ((-a) // b) &  & \by{4.2.2}    \\
             & = (ab + b(-a)) // b^2    &  & \by{4.2.2}    \\
             & = (ab + (-a)b) // b^2    &  & \by{4.1.6}    \\
             & = (ab + (-(ab)) // b^2   &  & \by{ac:4.1.5} \\
             & = 0 // b^2               &  & \by{4.1.6}    \\
             & = 0.                     &  & \by{ac:4.2.2}
  \end{align*}
  Thus the additive inverse of rational \(x\) is \(-x\).

  Next we show that \(xy = yx\).
  \begin{align*}
    xy & = (a // b) \times (c // d)                 \\
       & = ac // bd                 &  & \by{4.2.2} \\
       & = ca // db                 &  & \by{4.1.6} \\
       & = (c // d) \times (a // b) &  & \by{4.2.2} \\
       & = yx.
  \end{align*}
  Thus the multiplication on rationals is commutative.

  Next we show that \((xy)z = x(yz)\).
  \begin{align*}
    (xy)z & = ((a // b) \times (c // d)) \times (e // f)                 \\
          & = (ac // bd) \times (e // f)                 &  & \by{4.2.2} \\
          & = ((ac)e) // ((bd)f)                         &  & \by{4.2.2} \\
          & = (a(ce)) // (b(df))                         &  & \by{4.1.6} \\
          & = (a // b) \times (ce // df)                 &  & \by{4.2.2} \\
          & = (a // b) \times ((c // d) \times (e // f)) &  & \by{4.2.2} \\
          & = x(yz).
  \end{align*}
  Thus the multiplication on rationals is associative.

  Next we show that \(x1 = 1x = x\).
  Since the multiplication on rationals is commutative, we know that \(x1 = 1x\).
  Thus we only need to show that \(x1 = x\).
  \begin{align*}
    x1 & = (a // b) \times (1 // 1) &  & \by{ac:4.2.2} \\
       & = a1 // b1                 &  & \by{4.2.2}    \\
       & = a // b                   &  & \by{4.1.6}    \\
       & = x.
  \end{align*}
  Thus \(1\) is the multiplicative identity on rationals.

  Next we show that \(x(y + z) = xy + xz\).
  \begin{align*}
    x(y + z) & = (a // b) \times ((c // d) + (e // f))                                   \\
             & = (a // b) \times ((cf + de) // df)                       &  & \by{4.2.2} \\
             & = (a(cf + de)) // (b(df))                                 &  & \by{4.2.2} \\
             & = (b(a(cf + de))) // (b^2(df))                            &  & \by{4.2.3} \\
             & = ((ba)(cf + de)) // (b^2(df))                            &  & \by{4.1.6} \\
             & = ((ba)(cf) + (ba)(de)) // (b^2(df))                      &  & \by{4.1.6} \\
             & = ((ab)(fc) + (ba)(ed)) // (b^2(fd))                      &  & \by{4.1.6} \\
             & = (a(bf)c + b(ae)d) // (b(bf)d)                           &  & \by{4.1.6} \\
             & = ((ac)(bf) + (bd)(ae)) // ((bd)(bf))                     &  & \by{4.1.6} \\
             & = (ac // bd) + (ae // bf)                                 &  & \by{4.2.2} \\
             & = ((a // b) \times (c // d)) + ((a // b) \times (e // f)) &  & \by{4.2.2} \\
             & = xy + xz.
  \end{align*}
  Thus the multiplication and addition on rationals are left distributive.

  Next we show that \((y + z)x = yx + zx\).
  \begin{align*}
    (y + z)x & = x(y + z) &  & \text{(multiplication is commutative)}                     \\
             & = xy + xz  &  & \text{(multiplication and addition are left distributive)} \\
             & = yx + zx. &  & \text{(multiplication is commutative)}
  \end{align*}
  Thus the multiplication and addition on rationals are right distributive.

  Finally we show that if \(x \neq 0\), then \(xx^{-1} = x^{-1}x = 1\).
  Since the multiplication on rationals is commutative, we know that \(xx^{-1} = x^{-1}x\).
  Thus we only need to show that \(xx^{-1} = 1\).
  \begin{align*}
    xx^{-1} & = (a // b) \times (b // a) &  & \by{ac:4.2.3} \\
            & = ab // ba                 &  & \by{4.2.2}    \\
            & = ab // ab                 &  & \by{4.1.6}    \\
            & = 1 // 1                   &  & \by{4.2.1}    \\
            & = 1.                       &  & \by{ac:4.2.2}
  \end{align*}
  Thus the multiplicative inverse of rational \(x\) is \(x^{-1}\).
\end{proof}

\begin{rmk}\label{4.2.5}
  The above set (\cref{4.2.4}) of ten identities have a name;
  they are asserting that the rationals \(\Q\) form a \emph{field}.
  This is better than being a commutative ring because of the tenth identity \(xx^{-1} = x^{-1}x = 1\).
  Note that \cref{4.2.4} supercedes \cref{4.1.6}.
\end{rmk}

\begin{ac}\label{ac:4.2.4}
  We can now define the \emph{quotient} \(x / y\) of two rational numbers \(x\) and \(y\), \emph{provided that} \(y\) is non-zero, by the formula
  \[
    x / y \coloneqq x \times y^{-1}.
  \]
  Using the above formula, it is easy to see that \(a / b = a // b\) for every integer \(a\) and every non-zero integer \(b\).
  Thus we can now discard the \(//\) notation, and use the more customary \(a / b\) instead of \(a // b\).

  In a similar spirit, we define subtraction on the rationals by the formula
  \[
    x - y \coloneqq x + (-y),
  \]
  just as we did with the integers.
\end{ac}

\begin{proof}[\pf{ac:4.2.4}]
  We have
  \begin{align*}
    a / b & = (a // 1) / (b // 1)           &  & \by{ac:4.2.2} \\
          & = (a // 1) \times (b // 1)^{-1} &  & \by{ac:4.2.4} \\
          & = (a // 1) \times (1 // b)      &  & \by{ac:4.2.3} \\
          & = a1 // 1b                      &  & \by{4.2.2}    \\
          & = a // b.                       &  & \by{4.1.6}
  \end{align*}
\end{proof}

\begin{ac}\label{ac:4.2.5}
  Let \(x = a // b\) be a rational number where \(a, b \in \Z\) and \(b \neq 0\).
  Then we have
  \[
    -x = (-a) // b = a // (-b) = (-1) (a // b) = (-1) x.
  \]
  If \(y \in \Q\) and \(y \neq 0\), then we have
  \[
    -(x / y) = (-x) / y = x / (-y).
  \]
\end{ac}

\begin{proof}[\pf{ac:4.2.5}]
  By \cref{4.2.2} we have \(-x = (-a) // b\) and by \cref{4.2.3} we have \((-1) x = (-1) (a // b)\).
  We first show that \((-a) // b = a // (-b)\).
  This is true since
  \begin{align*}
    (-a) // b & = ((-a) // b) \times 1              &  & \by{4.2.4}    \\
              & = ((-a) // b) \times (1 // 1)       &  & \by{ac:4.2.2} \\
              & = ((-a) // b) \times ((-1) // (-1)) &  & \by{4.2.1}    \\
              & = ((-a) (-1)) // (b (-1))           &  & \by{4.2.2}    \\
              & = ((-1) (-a)) // ((-1) b)           &  & \by{4.1.6}    \\
              & = (-(-a)) // (-b)                   &  & \by{ac:4.1.5} \\
              & = a // (-b).                        &  & \by{ac:4.1.6}
  \end{align*}

  Next we show that \(-x = (-1) x\).
  This is true since
  \begin{align*}
    -x & = -(a // b)                                      \\
       & = (-a) // b                   &  & \by{4.2.2}    \\
       & = ((-1) a) // b               &  & \by{ac:4.1.5} \\
       & = ((-1) a) // 1b              &  & \by{4.1.6}    \\
       & = ((-1) // 1) \times (a // b) &  & \by{4.2.2}    \\
       & = (-1) (a // b)               &  & \by{ac:4.2.2} \\
       & = (-1) x.
  \end{align*}

  Next we show that if \(y \neq 0\), then \(-(x / y) = (-x) / y\).
  Suppose that \(y = c // d\) where \(c, d \neq 0\).
  Then we have
  \begin{align*}
    -(x / y) & = -\pa{(a // b) \times (c // d)^{-1}} &  & \by{ac:4.2.4} \\
             & = -((a // b) \times (d // c))         &  & \by{ac:4.2.3} \\
             & = -(ad // bc)                         &  & \by{4.2.2}    \\
             & = (-(ad)) // bc                       &  & \by{4.2.2}    \\
             & = ((-a) d) // bc                      &  & \by{ac:4.1.5} \\
             & = ((-a) // b) \times (d // c)         &  & \by{4.2.2}    \\
             & = ((-a) // b) \times (c // d)^{-1}    &  & \by{ac:4.2.3} \\
             & = (-x) \times y^{-1}                  &  & \by{4.2.2}    \\
             & = (-x) / y.                           &  & \by{ac:4.2.4}
  \end{align*}

  Finally we show that if \(y \neq 0\), then \((-x) / y = x / (-y)\).
  Suppose that \(y = c // d\) where \(c, d \neq 0\).
  Then we have
  \begin{align*}
    (-x) / y & = (-x) \times y^{-1}              &  & \by{ac:4.2.4}                 \\
             & = ((-1) \times x) \times y^{-1}   &  & \text{(from the proof above)} \\
             & = ((-1) \times x) \times (d // c) &  & \by{ac:4.2.3}                 \\
             & = (x \times (-1)) \times (d // c) &  & \by{4.2.4}                    \\
             & = x \times ((-1) \times (d // c)) &  & \by{4.2.4}                    \\
             & = x \times (d // (-c))            &  & \text{(from the proof above)} \\
             & = x \times ((-c) // d)^{-1}       &  & \by{ac:4.2.3}                 \\
             & = x \times (-y)^{-1}              &  & \by{4.2.2}                    \\
             & = x / (-y).                       &  & \by{ac:4.2.4}
  \end{align*}
\end{proof}

\begin{defn}\label{4.2.6}
  A rational number \(x\) is said to be \emph{positive} iff we have \(x = a / b\) for some positive integers \(a\) and \(b\).
  It is said to be \emph{negative} iff we have \(x = -y\) for some positive rational \(y\)
  (i.e., \(x = (-a) / b\) for some positive integers \(a\) and \(b\)).

  Thus for instance, every positive integer is a positive rational number, and every negative integer is a negative rational number, so our new definition is consistent with our old one.
\end{defn}

\begin{lem}[Trichotomy of rationals]\label{4.2.7}
  Let \(x\) be a rational number.
  Then exactly one of the following three statements is true:
  \begin{enumerate}
    \item \(x\) is equal to \(0\).
    \item \(x\) is a positive rational number.
    \item \(x\) is a negative rational number.
  \end{enumerate}
\end{lem}

\begin{proof}[\pf{4.2.7}]
  We first show that at least one of (a), (b), (c) is true.
  Let \(x = a // b\), where \(a, b \in \Z\) and \(b \neq 0\).
  By \cref{4.1.11}(f), \(a\) can only satisified one of the following three statements:
  \(a = 0\), \(a > 0\) and \(a < 0\).
  Similarly, \(b\) can only satisified one of the following two statements:
  \(b > 0\) and \(b < 0\).
  We first consider \(a\):
  \begin{itemize}
    \item If \(a = 0\), then \(x = 0 // b = 0\).
    \item If \(a > 0\), then we need to consider \(b\):
          \begin{itemize}
            \item If \(b > 0\), then by \cref{4.2.6} \(x\) is positive.
            \item If \(b < 0\), then by \cref{4.1.4} \(b = -c\) for some \(c \in \Z^+\).
                  Thus by \cref{ac:4.2.5} we have \(a // b = a // (-c) = (-a) // c\), which means \(x\) is negative by \cref{4.2.6}.
          \end{itemize}
    \item If \(a < 0\), then by \cref{4.1.4} \(a = -c\) for some \(c \in \Z^+\).
          Now we consider \(b\):
          \begin{itemize}
            \item If \(b > 0\), then \(a // b = (-c) // b\), which means \(x\) is negative by \cref{4.2.6}.
            \item If \(b < 0\), then by \cref{4.1.4} \(b = -d\) for some \(d \in \Z^+\).
                  Thus by \cref{4.2.1} we have \(a // b = (-c) // (-d) = (-1) // (-1) \times (c // d) = c // d\), which means \(x\) is positive by \cref{4.2.6}.
          \end{itemize}
  \end{itemize}
  From all cases above we conclude that at least one of (a), (b), (c) is true.

  Now we show that at most one of (a), (b), (c) is true.
  \begin{itemize}
    \item If \(x\) is both positive and equal to \(0\), then by \cref{ac:4.2.2,4.2.6}, \(x = a / b = 0 / 1\), where \(a, b \in \Z^+\).
          But by \cref{ac:4.2.2}, \(a / b = 0 / 1\) means \(a = 0\), contradicted to \(a > 0\).
    \item If \(x\) is both negative and equal to \(0\), then by \cref{ac:4.2.2,4.2.6}, \(x = (-a) / b = 0 / 1\), where \(a, b \in \Z^+\).
          But by \cref{ac:4.2.2} \((-a) / b = 0 / 1\) means \(-a = 0\), so that \(a = 0\) by \cref{ac:4.1.8}, contradicted to \(a > 0\).
    \item If \(x\) is both positive and negative, then by \cref{4.2.6}, \(x = a / b = (-c) / d\), where \(a, b, c, d \in \Z^+\).
          By \cref{4.2.1} \(a / b = (-c) / d\) means \(ad = b(-c) = (-1)(bc)\).
          By \cref{2.3.3} we know that \(ad\) and \(bc\) are positive.
          But by \cref{4.1.4} we know that \((-1)(bc) = -(bc)\) is negative, which contradict to \cref{4.1.5}.
  \end{itemize}
  From all cases above we conclude that no more than one of (a), (b), (c) is true at the same time.
\end{proof}

\begin{defn}[Ordering of the rationals]\label{4.2.8}
  Let \(x\) and \(y\) be rational numbers.
  We say that \(x > y\) iff \(x - y\) is a positive rational number, and \(x < y\) iff \(x - y\) is a negative rational number.
  We write \(x \geq y\) iff either \(x > y\) or \(x = y\), and similarly define \(x \leq y\) iff either \(x < y\) or \(x = y\).
\end{defn}

\begin{ac}\label{ac:4.2.6}
  If \(x\) and \(y\) are two positive rationals, then \(x + y\) is also a positive rational number.
  If \(x\) and \(y\) are two negative rationals, then \(x + y\) is also a negative rational number.
\end{ac}

\begin{proof}[\pf{ac:4.2.6}]
  We first show that if \(x\) and \(y\) are two positive rationals, then \(x + y\) is also positive.
  By \cref{4.2.6} we have \(x = a / b\) and \(y = c / d\) where \(a, b, c, d \in \Z^+\).
  Then by \cref{4.2.2} we have \(x + y = (ad + bc) / bd\).
  By \cref{2.3.2} we know that \(ad, bc, bd \in \Z^+\).
  Since \(ad, bc \in \Z^+\), by \cref{2.2.8} we know that \(ad + bc \in \Z^+\).
  Thus by \cref{4.2.6} we know that \(x + y = (ad + bc) / bd\) is a positive rational number.

  Now we show that if \(x\) and \(y\) are two negative rationals, then \(x + y\) is also negative.
  By \cref{4.2.6} we have \(x = (-a) / b\) and \(y = (-c) / d\) where \(a, b, c, d \in \Z^+\).
  Then by \cref{4.2.2} we have \(x + y = ((-a)d + b(-c)) / bd\).
  By \cref{4.2.4} and \cref{ac:4.2.5} we have
  \[
    (-a)d + b(-c) = (-a)d + (-c)b = ((-1)a)d + ((-1)c)b = (-1)(ad + cb) = -(ad + cb).
  \]
  By \cref{2.3.2} we know that \(ad, cb, bd \in \Z^+\).
  Since \(ad, cb \in \Z^+\), by \cref{2.2.8} we have \(ad + cb \in \Z^+\).
  Thus by \cref{4.1.4} we have \(-(ad + cb) \in \Z^-\) and by \cref{4.2.6}, \(x + y = -(ad + cb) / bd\) is a negative rational number.
\end{proof}

\begin{ac}\label{ac:4.2.7}
  Let \(x\) and \(y\) be two rationals.
  If \(x\) and \(y\) are positive, then \(xy\) is positive.
  If \(x\) and \(y\) are negative, then \(xy\) is positive.
\end{ac}

\begin{proof}[\pf{ac:4.2.7}]
  We first show that if \(x\) and \(y\) are two positive rationals, then \(xy\) is a positive rational number.
  By \cref{4.2.6} we know that \(x = a / b\) and \(y = c / d\) where \(a, b, c, d \in \Z^+\).
  By \cref{4.2.2} we have \(xy = ac / bd\).
  By \cref{2.3.2} we have \(ac, bd \in \Z^+\), thus by \cref{4.2.6} we know that \(xy\) is a positive rational number.

  Now we show that if \(x\) and \(y\) are two negative rationals, then \(xy\) is a positive rational number.
  By \cref{4.2.6} we know that \(x = (-a) / b\) and \(y = (-c) / d\) where \(a, b, c, d \in \Z^+\).
  By \cref{4.2.2} we have \(xy = (-a)(-c) / bd\).
  By \cref{ac:4.1.6} we have \((-a)(-c) = ac\).
  By \cref{2.3.2} we have \(ac, bd \in \Z^+\), thus by \cref{4.2.6} we know that \(xy\) is a positive rational number.
\end{proof}

\begin{ac}\label{ac:4.2.8}
  Let \(x\) and \(y\) be two rationals.
  If \(x\) is negative and \(y\) is positive, then \(xy\) is negative.
  If \(x\) is positive and \(y\) is negative, then \(xy\) is negative.
\end{ac}

\begin{proof}[\pf{ac:4.2.8}]
  By \cref{4.2.4} we know that \(xy = yx\), thus we only need to show that if \(x\) is negative and \(y\) is positive, then \(xy\) is negative.
  By \cref{4.2.6} we know that \(x = (-a) / b\) and \(y = c / d\) where \(a, b, c, d \in \Z^+\).
  By \cref{4.2.2} we have \(xy = (-a)c / bd\).
  By \cref{ac:4.1.5} we have \((-a)c = -(ac)\).
  By \cref{2.3.2} we know that \(ac, bd \in \Z^+\), thus by \cref{4.2.6} we know that \(xy\) is a negative rational number.
\end{proof}

\begin{ac}\label{ac:4.2.9}
  \(x\) is a positive rational number iff \(x > 0\).
  \(x\) is a negative rational number iff \(x < 0\).
\end{ac}

\begin{proof}[\pf{ac:4.2.9}]
  We first show that \(x\) is a positive rational number iff \(x > 0\).
  By \cref{4.2.6} we know that \(x = a / b\) where \(a, b \in \Z^+\).
  Thus
  \begin{align*}
         & x = a / b \in \Q^+         &  & \by{4.2.6}    \\
    \iff & x - 0 = a / b - 0 \in \Q^+ &  & \by{ac:4.2.5} \\
    \iff & x > 0.                     &  & \by{4.2.8}
  \end{align*}

  Now we show that \(x\) is a negative rational number iff \(x < 0\).
  By \cref{4.2.6} we know that \(x = (-a) / b\) where \(a, b \in \Z^+\).
  Thus
  \begin{align*}
         & x = (-a) / b \in \Q^-         &  & \by{4.2.6}    \\
    \iff & x - 0 = (-a) / b - 0 \in \Q^- &  & \by{ac:4.2.5} \\
    \iff & x < 0.                        &  & \by{4.2.8}
  \end{align*}
\end{proof}

\begin{prop}[Basic properties of order on the rationals]\label{4.2.9}
  Let \(x, y, z\) be rational numbers.
  Then the following properties hold.
  \begin{enumerate}
    \item (Order trichotomy)
          Exactly one of the three statements \(x = y\), \(x < y\), or \(x > y\) is true.
    \item (Order is anti-symmetric)
          One has \(x < y\) iff \(y > x\).
    \item (Order is transitive)
          If \(x < y\) and \(y < z\), then \(x < z\).
    \item (Addition preserves order)
          If \(x < y\), then \(x + z < y + z\).
    \item (Positive multiplication preserves order)
          If \(x < y\) and \(z\) is positive, then \(xz < yz\).
  \end{enumerate}
\end{prop}

\begin{proof}[\pf{4.2.9}(a)]
  By \cref{4.2.7} \(x - y\) is exactly one of the following three cases:
  \begin{enumerate}[label=(\Roman*)]
    \item \(x - y = 0\).
          Then by \cref{4.2.4} we have \(x = y\).
    \item \(x - y\) is positive.
          Then by \cref{4.2.8} we have \(x > y\).
    \item \(x - y\) is negative.
          Then by \cref{4.2.8} we have \(x < y\).
  \end{enumerate}
\end{proof}

\begin{proof}[\pf{4.2.9}(b)]
  Since
  \begin{align*}
    x - y & = (-1)(-1)(x - y)                 &  & \by{4.2.2}    \\
          & = (-1)(-1)\big(x + (-1)y\big)     &  & \by{ac:4.2.5} \\
          & = (-1)\big((-1)x + (-1)(-1)y\big) &  & \by{4.2.4}    \\
          & = (-1)\big((-1)x + y\big)         &  & \by{4.2.2}    \\
          & = (-1)\big(y + (-1)x\big)         &  & \by{4.2.4}    \\
          & = (-1)(y - x),                    &  & \by{ac:4.2.5}
  \end{align*}
  we have
  \begin{align*}
         & x < y                                                  \\
    \iff & x - y \text{ is negative}           &  & \by{4.2.8}    \\
    \iff & (-1)(x - y) \text{ is positive}     &  & \by{ac:4.2.7} \\
    \iff & (-1)(-1)(y - x) \text{ is positive} &  & \by{ac:4.2.3} \\
    \iff & y - x \text{ is positive}           &  & \by{4.2.2}    \\
    \iff & y > x.                              &  & \by{4.2.8}
  \end{align*}
\end{proof}

\begin{proof}[\pf{4.2.9}(c)]
  We have
  \begin{align*}
             & (x < y) \land (y < z)                                                            \\
    \implies & (x - y \text{ is negative}) \land (y - z \text{ is negative}) &  & \by{4.2.8}    \\
    \implies & (x - y) + (y - z) \text{ is negative}                         &  & \by{ac:4.2.6} \\
    \implies & x + z \text{ is negative}                                     &  & \by{4.2.4}    \\
    \implies & x < z.                                                        &  & \by{4.2.8}
  \end{align*}
\end{proof}

\begin{proof}[\pf{4.2.9}(d)]
  We have
  \begin{align*}
             & x < y                                                  \\
    \implies & x - y \text{ is negative}           &  & \by{4.2.8}    \\
    \implies & x + z - z - y \text{ is negative}   &  & \by{4.2.4}    \\
    \implies & x + z - y - z \text{ is negative}   &  & \by{4.2.4}    \\
    \implies & x + z - (y + z) \text{ is negative} &  & \by{ac:4.2.5} \\
    \implies & x + z < y + z.                      &  & \by{4.2.8}
  \end{align*}
\end{proof}

\begin{proof}[\pf{4.2.9}(e)]
  We have
  \begin{align*}
             & x < y                                           \\
    \implies & x - y \text{ is negative}    &  & \by{4.2.8}    \\
    \implies & (x - y)z \text{ is negative} &  & \by{ac:4.2.8} \\
    \implies & xz - yz \text{ is negative}  &  & \by{4.2.4}    \\
    \implies & xz < yz.                     &  & \by{4.2.8}
  \end{align*}
\end{proof}

\begin{rmk}\label{4.2.10}
  The above five properties in \cref{4.2.9}, combined with the field axioms in \cref{4.2.4}, have a name:
  they assert that the rationals \(\Q\) form an \emph{ordered field}.
  It is important to keep in mind that \cref{4.2.9}(e) only works when \(z\) is positive.
\end{rmk}

\exercisesection

\begin{ex}\label{ex:4.2.1}
  Show that the definition of equality for the rational numbers is reflexive, symmetric, and transitive.
\end{ex}

\begin{proof}[\pf{ex:4.2.1}]
  See \cref{ac:4.2.1}.
\end{proof}

\begin{ex}\label{ex:4.2.2}
  Prove the remaining components of \cref{4.2.3}.
\end{ex}

\begin{proof}[\pf{ex:4.2.2}]
  See \cref{4.2.3}.
\end{proof}

\begin{ex}\label{ex:4.2.3}
  Prove the remaining components of \cref{4.2.4}.
\end{ex}

\begin{proof}[\pf{ex:4.2.3}]
  See \cref{4.2.4}.
\end{proof}

\begin{ex}\label{ex:4.2.4}
  Prove \cref{4.2.7}.
\end{ex}

\begin{proof}[\pf{ex:4.2.4}]
  See \cref{4.2.7}.
\end{proof}

\begin{ex}\label{ex:4.2.5}
  Prove \cref{4.2.9}.
\end{ex}

\begin{proof}[\pf{ex:4.2.5}]
  See \cref{4.2.9}.
\end{proof}

\begin{ex}\label{ex:4.2.6}
  Show that if \(x, y, z\) are rational numbers such that \(x < y\) and \(z\) is negative, then \(xz > yz\).
\end{ex}

\begin{proof}[\pf{ex:4.2.6}]
  We have
  \begin{align*}
             & x < y                                           \\
    \implies & x - y \text{ is negative}    &  & \by{4.2.8}    \\
    \implies & (x - y)z \text{ is positive} &  & \by{ac:4.2.7} \\
    \implies & xz - yz \text{ is positive}  &  & \by{4.2.4}    \\
    \implies & xz > yz.                     &  & \by{4.2.8}
  \end{align*}
\end{proof}

\section{Absolute value and exponentiation}\label{sec:4.3}

\begin{defn}[Absolute value]\label{4.3.1}
  If \(x\) is a rational number, the \emph{absolute value} \(\abs{x}\) of \(x\) is defined as follows.
  If \(x\) is positive, then \(\abs{x} \coloneqq x\).
  If \(x\) is negative, then \(\abs{x} \coloneqq -x\).
  If \(x\) is zero, then \(\abs{x} \coloneqq 0\).
\end{defn}

\begin{defn}[Distance]\label{4.3.2}
  Let \(x\) and \(y\) be rational numbers.
  The quantity \(\abs{x - y}\) is called the \emph{distance between \(x\) and \(y\)} and is sometimes denoted \(d(x, y)\), thus \(d(x, y) \coloneqq \abs{x - y}\).
\end{defn}

\begin{prop}[Basic properties of absolute value and distance]\label{4.3.3}
  Let \(x, y, z\) be rational numbers.
  \begin{enumerate}
    \item (Non-degeneracy of absolute value)
          We have \(\abs{x} \geq 0\).
          Also, \(\abs{x} = 0\) iff \(x\) is \(0\).
    \item (Triangle inequality for absolute value)
          We have \(\abs{x + y} \leq \abs{x} + \abs{y}\).
    \item We have the inequalities \(-y \leq x \leq y\) iff \(y \geq \abs{x}\).
          In particular, we have \(-\abs{x} \leq x \leq \abs{x}\).
    \item (Multiplicativity of absolute value)
          We have \(\abs{xy} = \abs{x} \abs{y}\).
          In particular, \(\abs{-x} = \abs{x}\).
    \item (Non-degeneracy of distance)
          We have \(d(x, y) \geq 0\).
          Also, \(d(x, y) = 0\) iff \(x = y\).
    \item (Symmetry of distance)
          \(d(x, y) = d(y, x)\).
    \item (Triangle inequality for distance)
          \(d(x, z) \leq d(x, y) + d(y, z)\).
  \end{enumerate}
\end{prop}

\begin{proof}[\pf{4.3.3}(a)]
  By \cref{4.2.7} we know that exactly one of the three statements is true:
  \begin{itemize}
    \item \(x = 0\).
          Then by \cref{4.3.1} we have \(\abs{x} = 0\).
    \item \(x \in \Q^+\).
          Then by \cref{4.3.1} we have \(\abs{x} = x \in \Q^+\).
          By \cref{ac:4.2.9} we have \(\abs{x} = x > 0\).
    \item \(x \in \Q^-\).
          Then by \cref{4.3.1} we have \(\abs{x} = -x\).
          By \cref{ac:4.2.5,ac:4.2.7} we know that \(-x = (-1)x \in \Q^+\).
          Thus by \cref{ac:4.2.9} we have \(\abs{x} = -x > 0\).
  \end{itemize}
  From all cases above we conclude that \(\abs{x} \geq 0\) and \(\abs{x} = 0 \iff x = 0\).
\end{proof}

\begin{proof}[\pf{4.3.3}(b)]
  By \cref{4.2.7} exactly one of the following three statements is true:
  \begin{itemize}
    \item \(x + y = 0\).
          By \cref{4.3.3}(a) we have \(\abs{x + y} = 0 \leq \abs{x}\) and \(0 \leq \abs{y}\).
          Thus
          \begin{align*}
            \abs{x + y} & = 0 \leq \abs{y}        &  & \by{4.3.3}[a] \\
                        & = 0 + \abs{y}           &  & \by{4.2.4}    \\
                        & \leq \abs{x} + \abs{y}. &  & \by{4.2.9}[d]
          \end{align*}
    \item \(x + y \in \Q^+\).
          By \cref{ac:4.2.6} we know that we cannot have both \(x \in \Q^-\) and \(y \in \Q^-\).
          So we can use \cref{4.2.7} to split into two further cases:
          \begin{itemize}
            \item If exactly one of \(x, y\) is a negative rational number, say \(x\), then by \cref{ac:4.2.9} we have \(x < 0\) and \(y > 0\).
                  Note that \(y \neq 0\), otherwise we would have \(x + y = x \in \Q^-\) by \cref{4.2.4}, which contradicts to \(x + y \in \Q^+\).
                  Since \(x \neq 0\), by \cref{4.3.3}(a) we have \(0 < \abs{x}\).
                  Thus
                  \begin{align*}
                    \abs{x + y} & = x + y              &  & \by{4.3.1}    \\
                                & < 0 + y              &  & \by{4.2.9}[d] \\
                                & < \abs{x} + y        &  & \by{4.2.9}[d] \\
                                & = \abs{x} + \abs{y}. &  & \by{4.3.1}
                  \end{align*}
            \item If none of \(x, y\) are negative rational numbers, then by \cref{ac:4.2.9} we have \(x \geq 0\) and \(y \geq 0\).
                  Thus
                  \begin{align*}
                    \abs{x + y} & = x + y              &  & \by{4.3.1} \\
                                & = \abs{x} + \abs{y}. &  & \by{4.3.1}
                  \end{align*}
          \end{itemize}
    \item \(x + y \in \Q^-\).
          Then by \cref{ac:4.2.7} we have \(-(x + y) \in \Q^+\).
          Since
          \begin{align*}
            -(x + y) & = (-1) (x + y)    &  & \by{ac:4.2.5} \\
                     & = (-1) x + (-1) y &  & \by{4.2.4}    \\
                     & = (-x) + (-y),    &  & \by{ac:4.2.5}
          \end{align*}
          we can use the second case to derive \(\abs{-(x + y)} \leq \abs{-x} + \abs{-y}\).
          But by \cref{4.3.1} we have \(\abs{x + y} = -(x + y) = \abs{-(x + y)}\).
          And \(\abs{-x} = \abs{x}\) since
          \begin{align*}
            \abs{-x} & = \begin{dcases}
                           -x    & \text{if } (-x) > 0 \\
                           0     & \text{if } (-x) = 0 \\
                           -(-x) & \text{if } (-x) < 0
                         \end{dcases} &  & \by{4.3.1}                   \\
                     & = \begin{dcases}
                           -x & \text{if } x < 0 \\
                           0  & \text{if } x = 0 \\
                           x  & \text{if } x > 0
                         \end{dcases}       &  & \by{ac:4.2.5,ex:4.2.6} \\
                     & = \abs{x}.                     &  & \by{4.3.1}
          \end{align*}
          Similarly \(\abs{-y} = \abs{y}\).
          Thus we have \(\abs{x + y} \leq \abs{x} + \abs{y}\).
  \end{itemize}
  For all cases above we conclude that \(\abs{x + y} \leq \abs{x} + \abs{y}\).
\end{proof}

\begin{proof}[\pf{4.3.3}(c)]
  We have
  \begin{align*}
         & -y \leq x \leq y                                \\
    \iff & (x \leq y) \land (-x \leq y) &  & \by{ex:4.2.6} \\
    \iff & \abs{x} \leq y.              &  & \by{4.3.1}
  \end{align*}
  In particular, we have \(\abs{x} \leq \abs{x} \iff -\abs{x} \leq x \leq \abs{x}\).
\end{proof}

\begin{proof}[\pf{4.3.3}(d)]
  By \cref{4.2.7} we know that exactly one of the following three statements is true:
  \begin{itemize}
    \item \(x = 0\).
          Then we have
          \begin{align*}
            \abs{0y} & = \abs{0}          &  & \by{4.2.2} \\
                     & = 0                &  & \by{4.3.1} \\
                     & = 0 \abs{y}        &  & \by{4.2.2} \\
                     & = \abs{0} \abs{y}. &  & \by{4.3.1}
          \end{align*}
    \item \(x \in \Q^+\).
          By \cref{4.2.7} again we know that exactly one of the following three statements is true:
          \begin{itemize}
            \item \(y = 0\).
                  By \cref{4.2.4} we know that \(xy = yx\), thus this is the same case as \(x = 0\).
            \item \(y \in \Q^+\).
                  By \cref{ac:4.2.7} we know that \(xy \in \Q^+\).
                  Thus
                  \begin{align*}
                    \abs{xy} & = xy               &  & \by{4.3.1}    \\
                             & = \abs{x} \abs{y}. &  & \by{4.3.3}[a]
                  \end{align*}
            \item \(y \in \Q^-\).
                  By \cref{ac:4.2.8} we know that \(xy \in \Q^-\).
                  Thus
                  \begin{align*}
                    \abs{xy} & = -xy              &  & \by{4.3.1}    \\
                             & = x(-y)            &  & \by{ac:4.2.5} \\
                             & = \abs{x} \abs{y}. &  & \by{4.3.1}    \\
                  \end{align*}
          \end{itemize}
    \item \(x \in \Q^-\).
          By \cref{4.2.7}, exactly one of the following three statements is true:
          \begin{itemize}
            \item \(y = 0\).
                  By \cref{4.2.4} we know that \(xy = yx\), thus this is the same case as \(x = 0\).
            \item \(y \in \Q^+\).
                  By \cref{4.2.4} we know that \(xy = yx\), thus this is the same case as \(x \in \Q^+\) and \(y \in \Q^-\).
            \item \(y \in \Q^-\).
                  By \cref{ac:4.2.7}, \(xy\) is a positive.
                  Thus
                  \begin{align*}
                    \abs{xy} & = xy               &  & \by{4.3.1}    \\
                             & = (-1)(-1)xy       &  & \by{4.2.2}    \\
                             & = (-1)x(-1)y       &  & \by{4.2.4}    \\
                             & = (-x)(-y)         &  & \by{ac:4.2.5} \\
                             & = \abs{x} \abs{y}. &  & \by{4.3.1}
                  \end{align*}
          \end{itemize}
  \end{itemize}
  From all cases above we conclude that \(\abs{xy} = \abs{x} \abs{y}\).
  In particular, we have
  \begin{align*}
    \abs{-x} & = \abs{(-1)x}      &  & \by{ac:4.2.5}                 \\
             & = \abs{-1} \abs{x} &  & \text{(from the proof above)} \\
             & = -(-1) \abs{x}    &  & \by{4.3.1}                    \\
             & = 1 \abs{x}        &  & \by{ac:4.1.6}                 \\
             & = \abs{x}.         &  & \by{4.2.4}
  \end{align*}
\end{proof}

\begin{proof}[\pf{4.3.3}(e)]
  Since \(x - y \in \Q\), by \cref{4.3.3}(a) we have \(d(x, y) = \abs{x - y} \geq 0\) and
  \begin{align*}
         & d(x, y) = 0                        \\
    \iff & \abs{x - y} = 0 &  & \by{4.3.2}    \\
    \iff & x - y = 0       &  & \by{4.3.3}[a] \\
    \iff & x = y.          &  & \by{4.2.4}
  \end{align*}
\end{proof}

\begin{proof}[\pf{4.3.3}(f)]
  We have
  \begin{align*}
    d(x, y) & = \abs{x - y}    &  & \by{4.3.2}    \\
            & = \abs{-(x - y)} &  & \by{4.3.3}[d] \\
            & = \abs{y - x}    &  & \by{4.2.4}    \\
            & = d(y, x).       &  & \by{4.3.2}
  \end{align*}
\end{proof}

\begin{proof}[\pf{4.3.3}(g)]
  We have
  \begin{align*}
    d(x, z) & = \abs{x - z}                  &  & \by{4.3.2}    \\
            & = \abs{x - y + y - z}          &  & \by{4.2.4}    \\
            & \leq \abs{x - y} + \abs{y - z} &  & \by{4.3.3}[b] \\
            & = d(x, y) + d(y, z).           &  & \by{4.3.2}
  \end{align*}
\end{proof}

\begin{ac}\label{ac:4.3.1}
  Let \(x, y\) be rational numbers.
  Then \(\abs{x} - \abs{y} \leq \abs{x + y}\).
\end{ac}

\begin{proof}[\pf{ac:4.3.1}]
  \begin{align*}
             & \abs{x + y + (-y)} \leq \abs{x + y} + \abs{-y}               &  & \by{4.3.3}[b] \\
    \implies & \abs{x} \leq \abs{x + y} + \abs{-y}                          &  & \by{4.2.4}    \\
    \implies & \abs{x} \leq \abs{x + y} + \abs{y}                           &  & \by{4.3.3}[d] \\
    \implies & \abs{x} + (-\abs{y}) \leq \abs{x + y} + \abs{y} + (-\abs{y}) &  & \by{4.2.9}[d] \\
    \implies & \abs{x} + (-\abs{y}) \leq \abs{x + y}                        &  & \by{4.2.4}    \\
    \implies & \abs{x} - \abs{y} \leq \abs{x + y}.                          &  & \by{ac:4.2.4}
  \end{align*}
\end{proof}

\begin{defn}[\(\varepsilon\)-closeness]\label{4.3.4}
  Let \(\varepsilon > 0\) be a rational number, and let \(x\), \(y\) be rational numbers.
  We say that \(y\) is \emph{\(\varepsilon\)-close} to \(x\) iff we have \(d(y, x) \leq \varepsilon\).
\end{defn}

\begin{rmk}\label{4.3.5}
  This definition is not standard in mathematics textbooks;
  we will use it as ``scaffolding'' to construct the more important notions of limits (and of Cauchy sequences) later on, and once we have those more advanced notions we will discard the notion of \(\varepsilon\)-close.
\end{rmk}

\begin{note}
  We do not bother defining a notion of \(\varepsilon\)-close when \(\varepsilon\) is zero or negative, because if \(\varepsilon\) is zero then \(x\) and \(y\) are only \(\varepsilon\)-close when they are equal, and when \(\varepsilon\) is negative then \(x\) and \(y\) are never \(\varepsilon\)-close.
\end{note}

\begin{note}
  In any event it is a long-standing tradition in analysis that the Greek letters \(\varepsilon\), \(\delta\) should only denote small positive numbers.
\end{note}

\setcounter{thm}{6}
\begin{prop}\label{4.3.7}
  Let \(x, y, z, w\) be rational numbers.
  (extended to cover the \(0\)-close case)
  \begin{enumerate}
    \item If \(x = y\), then \(x\) is \(\varepsilon\)-close to \(y\) for every \(\varepsilon > 0\).
          Conversely, if \(x\) is \(\varepsilon\)-close to \(y\) for every \(\varepsilon > 0\), then we have \(x = y\).
    \item Let \(\varepsilon > 0\).
          If \(x\) is \(\varepsilon\)-close to \(y\), then \(y\) is \(\varepsilon\)-close to \(x\).
    \item Let \(\varepsilon, \delta > 0\).
          If \(x\) is \(\varepsilon\)-close to \(y\), and \(y\) is \(\delta\)-close to \(z\), then \(x\) and \(z\) are \((\varepsilon + \delta)\)-close.
    \item Let \(\varepsilon, \delta > 0\).
          If \(x\) and \(y\) are \(\varepsilon\)-close, and \(z\) and \(w\) are \(\delta\)-close, then \(x + z\) and \(y + w\) are \((\varepsilon + \delta)\)-close, and \(x - z\) and \(y - w\) are also \((\varepsilon + \delta)\)-close.
    \item Let \(\varepsilon > 0\).
          If \(x\) and \(y\) are \(\varepsilon\)-close, they are also \(\varepsilon'\)-close for every \(\varepsilon' > \varepsilon\).
    \item Let \(\varepsilon > 0\).
          If \(y\) and \(z\) are both \(\varepsilon\)-close to \(x\), and \(w\) is between \(y\) and \(z\) (i.e., \(y \leq w \leq z\) or \(z \leq w \leq y\)), then \(w\) is also \(\varepsilon\)-close to \(x\).
    \item Let \(\varepsilon > 0\).
          If \(x\) and \(y\) are \(\varepsilon\)-close, and \(z\) is non-zero, then \(xz\) and \(yz\) are \(\varepsilon\abs{z}\)-close.
    \item Let \(\varepsilon, \delta > 0\).
          If \(x\) and \(y\) are \(\varepsilon\)-close, and \(z\) and \(w\) are \(\delta\)-close, then \(xz\) and \(yw\) are \((\varepsilon\abs{z} + \delta\abs{x} + \varepsilon\delta)\)-close.
  \end{enumerate}
\end{prop}

\begin{proof}[\pf{4.3.7}(a)]
  We first show that if \(x = y\), then \(x\) is \(\varepsilon\)-close to \(y\) for every \(\varepsilon \in \Q^+\).
  \begin{align*}
             & x = y                                                                                            \\
    \implies & x - y = 0                                                               &  & \by{4.2.4}          \\
    \implies & d(x, y) = \abs{x - y} = 0                                               &  & \by{4.3.2,4.3.3}[a] \\
    \implies & \forall \varepsilon \in \Q^+, d(x, y) \leq \varepsilon                  &  & \by{ac:4.2.9}       \\
    \implies & \forall \varepsilon \in \Q^+, x \text{ is \(\varepsilon\)-close to } y. &  & \by{4.3.4}
  \end{align*}

  Now we show that if \(x\) is \(\varepsilon\)-close to \(y\) for every \(\varepsilon \in \Q^+\), then \(x = y\).
  Suppose for sake of contradiction that \(x \neq y\).
  Then by \cref{4.3.3}(e) we have \(d(x, y) > 0\).
  But then we have \(d(x, y) \in \Q^+\), so \(d(x, y) < d(x, y)\), a contradiction.
  Thus we must have \(x = y\).
\end{proof}

\begin{proof}[\pf{4.3.7}(b)]
  We have
  \begin{align*}
         & x \text{ is \(\varepsilon\)-close to } y                     \\
    \iff & d(x, y) \leq \varepsilon                  &  & \by{4.3.4}    \\
    \iff & d(y, x) \leq \varepsilon                  &  & \by{4.3.3}[f] \\
    \iff & y \text{ is \(\varepsilon\)-close to } x. &  & \by{4.3.4}
  \end{align*}
\end{proof}

\begin{proof}[\pf{4.3.7}(c)]
  We have
  \begin{align*}
             & (x \text{ is \(\varepsilon\)-close to } y) \land (y \text{ is \(\delta\)-close to } z)                      \\
    \implies & \big(d(x, y) \leq \varepsilon\big) \land \big(d(y, z) \leq \delta\big)                 &  & \by{4.3.4}      \\
    \implies & d(x, y) + d(y, z) \leq \varepsilon + d(y, z) \leq \varepsilon + \delta                 &  & \by{4.2.9}[c,d] \\
    \implies & d(x, z) \leq d(x, y) + d(y, z) \leq \varepsilon + \delta                               &  & \by{4.3.3}[g]   \\
    \implies & x \text{ is \((\varepsilon + \delta)\)-close to } z.                                   &  & \by{4.3.4}
  \end{align*}
\end{proof}

\begin{proof}[\pf{4.3.7}(d)]
  Since
  \begin{align*}
             & (x \text{ is \(\varepsilon\)-close to } y) \land (z \text{ is \(\delta\)-close to } w)                      \\
    \implies & \big(d(x, y) \leq \varepsilon\big) \land \big(d(z, w) \leq \delta\big)                 &  & \by{4.3.4}      \\
    \implies & d(x, y) + d(z, w) \leq \varepsilon + d(z, w) \leq \varepsilon + \delta                 &  & \by{4.2.9}[c,d] \\
    \implies & \abs{x - y} + \abs{z - w} \leq \varepsilon + \delta,                                   &  & \by{4.3.2}
  \end{align*}
  we have
  \begin{align*}
             & \abs{x - y} + \abs{z - w} \leq \varepsilon + \delta                                             \\
    \implies & \abs{x - y + z - w} \leq \abs{x - y} + \abs{z - w} \leq \varepsilon + \delta &  & \by{4.3.3}[b] \\
    \implies & \abs{x + z - (y + w)} \leq \varepsilon + \delta                              &  & \by{4.2.4}    \\
    \implies & d(x + z, y + w) \leq \varepsilon + \delta                                    &  & \by{4.3.2}    \\
    \implies & (x + z) \text{ is \((\varepsilon + \delta)\)-close to } (y + w)              &  & \by{4.3.4}
  \end{align*}
  and
  \begin{align*}
             & \abs{x - y} + \abs{z - w} \leq \varepsilon + \delta                                             \\
    \implies & \abs{x - y} + \abs{-(z - w)} \leq \varepsilon + \delta                       &  & \by{4.3.3}[d] \\
    \implies & \abs{x - y} + \abs{w - z} \leq \varepsilon + \delta                          &  & \by{4.2.4}    \\
    \implies & \abs{x - y + w - z} \leq \abs{x - y} + \abs{w - z} \leq \varepsilon + \delta &  & \by{4.3.3}[b] \\
    \implies & \abs{x - z - (y - w)} \leq \varepsilon + \delta                              &  & \by{4.2.4}    \\
    \implies & d(x - z, y - w) \leq \varepsilon + \delta                                    &  & \by{4.3.2}    \\
    \implies & (x - z) \text{ is \((\varepsilon + \delta)\)-close to } (y - w).             &  & \by{4.3.4}
  \end{align*}
\end{proof}

\begin{proof}[\pf{4.3.7}(e)]
  We have
  \begin{align*}
             & (x \text{ is \(\varepsilon\)-close to } y) \land (\varepsilon' > \varepsilon)                      \\
    \implies & \big(d(x, y) \leq \varepsilon\big) \land (\varepsilon' > \varepsilon)         &  & \by{4.3.4}      \\
    \implies & d(x, y) < \varepsilon'                                                        &  & \by{4.2.9}[b,c] \\
    \implies & x \text{ is \(\varepsilon'\)-close to } y.                                    &  & \by{4.3.4}
  \end{align*}
\end{proof}

\begin{proof}[\pf{4.3.7}(f)]
  We have
  \begin{align*}
             & (y \text{ is } \varepsilon\text{-close to } x) \land (z \text{ is } \varepsilon\text{-close to } x)                    \\
    \implies & \big(d(y, x) \leq \varepsilon\big) \land \big(d(z, x) \leq \varepsilon\big)                         &  & \by{4.3.4}    \\
    \implies & (\abs{y - x} \leq \varepsilon) \land (\abs{z - x} \leq \varepsilon)                                 &  & \by{4.3.2}    \\
    \implies & (-\varepsilon \leq y - x \leq \varepsilon) \land (-\varepsilon \leq z - x \leq \varepsilon).        &  & \by{4.3.3}[c]
  \end{align*}
  If \(y \leq w \leq z\), then we have
  \begin{align*}
             & y \leq w \leq z                                                                   \\
    \implies & y - x \leq w - x \leq z - x                                    &  & \by{4.2.9}[d] \\
    \implies & -\varepsilon \leq y - x \leq w - x \leq z - x \leq \varepsilon &  & \by{4.2.9}[c] \\
    \implies & \abs{w - x} \leq \varepsilon                                   &  & \by{4.3.3}[c] \\
    \implies & d(w, x) \leq \varepsilon                                       &  & \by{4.3.2}    \\
    \implies & w \text{ is } \varepsilon\text{-close to } x.                  &  & \by{4.3.4}
  \end{align*}
  The case for \(z \leq w \leq y\) can be proven similarly.
\end{proof}

\begin{proof}[\pf{4.3.7}(g)]
  We have
  \begin{align*}
             & (x \text{ is } \varepsilon\text{-close to } y) \land (z \neq 0)                       \\
    \implies & (\abs{x - y} \leq \varepsilon) \land (z \neq 0)                 &  & \by{4.3.2,4.3.4} \\
    \implies & (\abs{x - y} \leq \varepsilon) \land (\abs{z} > 0)              &  & \by{4.3.3}[a]    \\
    \implies & \abs{x - y} \abs{z} \leq \varepsilon \abs{z}                    &  & \by{4.2.9}[e]    \\
    \implies & \abs{(x - y)z} \leq \varepsilon \abs{z}                         &  & \by{4.3.3}[d]    \\
    \implies & \abs{xz - yz} \leq \varepsilon \abs{z}                          &  & \by{4.2.4}       \\
    \implies & d(xz, yz) \leq \varepsilon \abs{z}                              &  & \by{4.3.2}       \\
    \implies & xz \text{ is } (\varepsilon \abs{z})\text{-close to } yz.       &  & \by{4.3.4}
  \end{align*}
\end{proof}

\begin{proof}[\pf{4.3.7}(h)]
  If we write \(a \coloneqq y - x\), then we have \(y = x + a\) and that \(\abs{a} \leq \varepsilon\).
  Similarly, if we define \(b \coloneqq w - z\), then \(w = z + b\) and \(\abs{b} \leq \delta\).

  Since \(y = x + a\) and \(w = z + b\), by \cref{4.2.4} we have
  \[
    yw = (x + a)(z + b) = xz + az + xb + ab.
  \]
  Thus by \cref{4.3.3}(b)(d) we have
  \[
    \abs{yw - xz} = \abs{az + bx + ab} \leq \abs{az} + \abs{bx} + \abs{ab} = \abs{a}\abs{z} + \abs{b}\abs{x} + \abs{a}\abs{b}.
  \]
  Since \(\abs{a} \leq \varepsilon\) and \(\abs{b} \leq \delta\), we thus have
  \[
    \abs{yw - xz} \leq \varepsilon\abs{z} + \delta\abs{x} + \varepsilon\delta
  \]
  and thus that \(yw\) and \(xz\) are \((\varepsilon\abs{z} + \delta\abs{x} + \varepsilon\delta)\)-close.
\end{proof}

\begin{rmk}\label{4.3.8}
  One should compare statements (a)-(c) of \cref{4.3.7} with the reflexive, symmetric, and transitive axioms of equality.
  It is often useful to think of the notion of ``\(\varepsilon\)-close'' as an approximate substitute for that of equality in analysis.
\end{rmk}

\begin{defn}[Exponentiation to a natural number]\label{4.3.9}
  Let \(x\) be a rational number.
  To raise \(x\) to the power \(0\), we define \(x^0 \coloneqq 1\);
  in particular we define \(0^0 \coloneqq 1\).
  Now suppose inductively that \(x^n\) has been defined for some natural number \(n\), then we define \(x^{n+1} \coloneqq x^n \times x\).
\end{defn}

\begin{prop}[Properties of exponentiation, I]\label{4.3.10}
  Let \(x, y\) be rational numbers, and let \(n, m\) be natural numbers.
  \begin{enumerate}
    \item We have \(x^n x^m = x^{n + m}\), \((x^n)^m = x^{nm}\), and \((xy)^n = x^n y^n\).
    \item Suppose \(n > 0\).
          Then we have \(x^n = 0\) iff \(x = 0\).
    \item If \(x \geq y \geq 0\), then \(x^n \geq y^n \geq 0\).
          If \(x > y \geq 0\) and \(n > 0\), then \(x^n > y^n \geq 0\).
    \item We have \(\abs{x^n} = \abs{x}^n\).
  \end{enumerate}
\end{prop}

\begin{proof}[\pf{4.3.10}(a)]
  We first show that \(x^n x^m = x^{n + m}\).
  We induct on \(n\).
  For \(n = 0\), we have
  \begin{align*}
    x^0 x^m & = 1 x^m      &  & \by{4.3.9} \\
            & = x^m        &  & \by{4.2.4} \\
            & = x^{0 + m}. &  & \by{2.2.1}
  \end{align*}
  So the base case holds.
  Suppose inductively that for some \(n \in \N\) we have \(x^n x^m = x^{n + m}\).
  Then for \(n + 1\), we have
  \begin{align*}
    x^{n + 1} x^m & = (x^n x) x^m      &  & \by{4.3.9} \\
                  & = x^n (x x^m)      &  & \by{4.2.4} \\
                  & = x^n (x^m x)      &  & \by{4.2.4} \\
                  & = (x^n x^m) x      &  & \by{4.2.4} \\
                  & = x^{n + m} x      &  & \byIH      \\
                  & = x^{(n + m) + 1}  &  & \by{4.3.9} \\
                  & = x^{n + (m + 1)}  &  & \by{2.2.5} \\
                  & = x^{n + (1 + m)}  &  & \by{2.2.4} \\
                  & = x^{(n + 1) + m}. &  & \by{2.2.5}
  \end{align*}
  This closes the induction.

  Next we show that \((x^n)^m = x^{nm}\).
  We induct on \(m\).
  For \(m = 0\), we have
  \begin{align*}
    (x^n)^0 & = 1       &  & \by{4.3.9}    \\
            & = x^0     &  & \by{4.3.9}    \\
            & = x^{n0}. &  & \by{ac:2.3.2}
  \end{align*}
  So the base case holds.
  Suppose inductively that for some \(m \in \N\) we have \((x^n)^m = x^{nm}\).
  Then for \(m + 1\), we have
  \begin{align*}
    (x^n)^{m + 1} & = (x^n)^m (x^n) &  & \by{4.3.9}                    \\
                  & = x^{nm} x^n    &  & \byIH                         \\
                  & = x^{nm + n}    &  & \text{(from the proof above)} \\
                  & = x^{n(m + 1)}. &  & \by{ac:2.3.3}
  \end{align*}
  This closes the induction.

  Finally we show that \((xy)^n = x^n y^n\).
  We induct on \(n\).
  For \(n = 0\), we have
  \begin{align*}
    (xy)^0 & = 1        &  & \by{4.3.9} \\
           & = y^0      &  & \by{4.3.9} \\
           & = 1y^0     &  & \by{4.2.4} \\
           & = x^0 y^0. &  & \by{4.3.9}
  \end{align*}
  So the base case holds.
  Suppose inductively that for some \(n \in \N\) we have \((xy)^n = x^n y^n\).
  Then for \(n + 1\), we have
  \begin{align*}
    (xy)^{n + 1} & = (xy)^n (xy)          &  & \by{4.3.9} \\
                 & = (x^n y^n) (xy)       &  & \byIH      \\
                 & = x^n (y^n x) y        &  & \by{4.2.4} \\
                 & = x^n (x y^n) y        &  & \by{4.2.4} \\
                 & = (x^n x)(y^n y)       &  & \by{4.2.4} \\
                 & = x^{n + 1} y^{n + 1}. &  & \by{4.3.9}
  \end{align*}
  This closes the induction.
\end{proof}

\begin{proof}[\pf{4.3.10}(b)]
  We induct on \(n\) and we start with \(n = 1\).
  For \(n = 1\), we have
  \begin{align*}
         & x^1 = 0                   \\
    \iff & x^0 x = 0 &  & \by{4.3.9} \\
    \iff & 1x = 0    &  & \by{4.3.9} \\
    \iff & x = 0.    &  & \by{4.2.4}
  \end{align*}
  So the base case holds.
  Suppose inductively that for some \(n \in \N \setminus \set{0}\) we have \(x^n = 0 \iff x = 0\).
  Then for \(n + 1\), we have
  \begin{align*}
         & x^{n + 1} = 0                                      \\
    \iff & x^n x = 0              &  & \by{4.3.9}             \\
    \iff & (x^n = 0) \lor (x = 0) &  & \by{ac:4.2.7,ac:4.2.8} \\
    \iff & x = 0.                 &  & \byIH
  \end{align*}
  This closes the induction.
\end{proof}

\begin{proof}[\pf{4.3.10}(c)]
  We first show that if \(x \geq y \geq 0\), then \(x^n \geq y^n \geq 0\).
  We induct on \(n\).
  For \(n = 0\), we have
  \begin{align*}
             & x \geq y \geq 0                              \\
    \implies & x^0 = 1 \geq y^0 = 1 \geq 0. &  & \by{4.3.9}
  \end{align*}
  So the base case holds.
  Suppose inductively that for some \(n \in \N\) we have \(x^n \geq y^n \geq 0\).
  Then for \(n + 1\), we have
  \begin{align*}
             & (x^n \geq y^n \geq 0) \land (x \geq y \geq 0)                  &  & \byIH            \\
    \implies & (x^n x \geq y^n x \geq 0x) \land (y^n x \geq y^n y \geq y^n 0) &  & \by{4.2.9}[e]    \\
    \implies & x^n x \geq y^n x \geq y^n y \geq y^n 0                         &  & \by{4.2.9}[c]    \\
    \implies & x^{n + 1} \geq y^{n + 1} \geq 0.                               &  & \by{4.2.2,4.3.9}
  \end{align*}
  This closes the induction.

  Now we show that if \(x > y \geq 0\) and \(n > 0\), then \(x^n > y^n \geq 0\).
  We split into two cases:
  \begin{itemize}
    \item If \(y = 0\), then by \cref{4.3.10}(b) we know that \(x \neq 0 \iff x^n \neq 0\) and \(y = 0 \iff y^n = 0\).
          By \cref{4.2.9}(e) we know that \(x > 0 \implies x^n > 0\).
          Thus by \cref{4.2.9}(c) we have \(x^n > y^n = 0\).
    \item If \(y > 0\), then we have \(x > y > 0\).
          We induct on \(n\) to show that \(x^n > y^n > 0\) and we start with \(n = 1\).
          For \(n = 1\), we have
          \begin{align*}
                     & \begin{dcases}
                         x > y > 0            \\
                         x^1 = x^0 x = 1x = x \\
                         y^1 = y^0 y = 1y = y
                       \end{dcases} &  & \by{4.2.4,4.3.9}      \\
            \implies & x^1 > y^1 > 0.          &  & \by{4.3.9}
          \end{align*}
          So the base case holds.
          Suppose inductively that for some \(n \in \N \setminus \set{0}\) we have \(x^n > y^n > 0\).
          Then for \(n + 1\), we have
          \begin{align*}
                     & (x^n > y^n > 0) \land (x > y > 0)                  &  & \byIH            \\
            \implies & (x^n x > y^n x > 0x) \land (y^n x > y^n y > y^n 0) &  & \by{4.2.9}[e]    \\
            \implies & x^n x > y^n x > y^n y > y^n 0                      &  & \by{4.2.9}[c]    \\
            \implies & x^{n + 1} > y^{n + 1} > 0.                         &  & \by{4.2.2,4.3.9}
          \end{align*}
          This closes the induction.
  \end{itemize}
  From all cases above we conclude that \(x > y \geq 0 \implies x^n > y^n \geq 0\).
\end{proof}

\begin{proof}[\pf{4.3.10}(d)]
  We induct on \(n\).
  For \(n = 0\), we have
  \begin{align*}
    \abs{x^0} & = \abs{1}    &  & \by{4.3.9} \\
              & = 1          &  & \by{4.3.1} \\
              & = \abs{x}^0. &  & \by{4.3.9}
  \end{align*}
  So the base case holds.
  Suppose inductively that for some \(n \in \N\) we have \(\abs{x^n} = \abs{x}^n\).
  Then for \(n + 1\), we have
  \begin{align*}
    \abs{x^{n + 1}} & = \abs{x^n x}       &  & \by{4.3.9}    \\
                    & = \abs{x^n} \abs{x} &  & \by{4.3.3}[d] \\
                    & = \abs{x}^n \abs{x} &  & \byIH         \\
                    & = \abs{x}^{n + 1}.  &  & \by{4.3.9}
  \end{align*}
  This closes the induction.
\end{proof}

\begin{defn}[Exponentiation to a negative number]\label{4.3.11}
  Let \(x\) be a non-zero rational number.
  Then for any negative integer \(-n\), we define \(x^{-n} \coloneqq 1 / x^n\).

  Note that when \(n = 1\), the definition of \(x^{-1}\) provided by \cref{4.3.11} coincides with the reciprocal of \(x\) defined previously, so there is no incompatibility of notation caused by this new definition.
\end{defn}

\begin{prop}[Properties of exponentiation, II]\label{4.3.12}
  Let \(x\), \(y\) be nonzero rational numbers, and let \(n\), \(m\) be integers.
  \begin{enumerate}
    \item We have \(x^n x^m = x^{n + m}\), \((x^n)^m = x^{nm}\), and \((xy)^n = x^n y^n\).
    \item If \(x \geq y > 0\), then \(x^n \geq y^n > 0\) if \(n\) is positive, and \(0 < x^n \leq y^n\) if \(n\) is negative.
    \item If \(x, y > 0\), \(n \neq 0\), and \(x^n = y^n\), then \(x = y\).
    \item We have \(\abs{x^n} = \abs{x}^n\).
  \end{enumerate}
\end{prop}

\begin{proof}[\pf{4.3.12}(a)]
  We first show that \(x^n x^m = x^{n + m}\).
  By \cref{4.1.11}(f) exactly one of \(n > 0\), \(n = 0\) or \(n < 0\) is true.
  Similarly, exactly one of \(m > 0\), \(m = 0\) or \(m < 0\) is true, and exactly one of \(n + m > 0\), \(n + m = 0\) or \(n + m < 0\) is true.
  Thus we can split into following cases:
  \begin{itemize}
    \item If \(n = 0\) or \(m = 0\), then we have
          \begin{align*}
            x^0 x^m & = 1 x^m     &  & \by{4.3.9} \\
                    & = x^m       &  & \by{4.2.4} \\
                    & = x^{0 + m} &  & \by{4.1.6}
          \end{align*}
          and
          \begin{align*}
            x^n x^0 & = x^0 x^n    &  & \by{4.2.4}                    \\
                    & = x^{0 + n}  &  & \text{(from the proof above)} \\
                    & = x^{n + 0}. &  & \by{4.1.6}
          \end{align*}
    \item If \(n > 0\) and \(m > 0\), then by \cref{4.3.10}(a) we have \(x^n x^m = x^{n + m}\).
    \item If \(n > 0\), \(m < 0\) and \(n + m \geq 0\), then by \cref{4.1.11}(d) we have \(-m > 0\).
          Thus
          \begin{align*}
            x^n x^m & = \dfrac{1 x^n}{x^{-m}}            &  & \by{4.3.11}    \\
                    & = \dfrac{x^n}{x^{-m}}              &  & \by{4.2.4}     \\
                    & = \dfrac{x^{n + m + (-m)}}{x^{-m}} &  & \by{4.1.6}     \\
                    & = \dfrac{x^{n + m} x^{-m}}{x^{-m}} &  & \by{4.3.10}[a] \\
                    & = x^{n + m}.                       &  & \by{4.2.3}
          \end{align*}
    \item If \(n > 0\), \(m < 0\) and \(n + m < 0\), then by \cref{4.1.11}(d) we have \(-(n + m) > 0\).
          Thus
          \begin{align*}
            x^n x^m & = \dfrac{1 x^n}{x^{-m}}              &  & \by{4.3.11}    \\
                    & = \dfrac{1 x^n}{x^{n + (-n) + (-m)}} &  & \by{4.1.6}     \\
                    & = \dfrac{1 x^n}{x^{n + (-(n + m))}}  &  & \by{ac:4.1.5}  \\
                    & = \dfrac{1 x^n}{x^n x^{-(n + m)}}    &  & \by{4.3.10}[a] \\
                    & = \dfrac{1}{x^{-(n + m)}}            &  & \by{4.2.3}     \\
                    & = x^{n + m}.                         &  & \by{4.3.11}
          \end{align*}
    \item If \(n < 0\) and \(m > 0\), then we can use the previous two cases to conclude that
          \begin{align*}
            x^n x^m & = x^m x^n    &  & \by{4.2.4}                           \\
                    & = x^{m + n}  &  & \text{(from the previous two cases)} \\
                    & = x^{n + m}. &  & \by{4.1.6}
          \end{align*}
    \item If \(n < 0\) and \(m < 0\), then by \cref{4.1.11}(d) we have \(-n > 0\), \(-m > 0\) and \(-(n + m) > 0\).
          Thus
          \begin{align*}
            x^n x^m & = \dfrac{1}{x^{-n} x^{-m}}   &  & \by{4.3.11}    \\
                    & = \dfrac{1}{x^{(-n) + (-m)}} &  & \by{4.3.10}[a] \\
                    & = \dfrac{1}{x^{-(n + m)}}    &  & \by{ac:4.1.5}  \\
                    & = x^{n + m}.                 &  & \by{4.3.11}
          \end{align*}
  \end{itemize}
  From all cases above we conclude that \(x^n x^m = x^{n + m}\).

  Next we show that \((x^n)^m = x^{nm}\).
  By \cref{4.1.5} exactly one of the following two statements is true:
  \begin{itemize}
    \item \(m \geq 0\).
          Then we induct on \(m\) to show that \((x^n)^m = x^{nm}\).
          For \(m = 0\), we have
          \begin{align*}
            (x^n)^0 & = 1       &  & \by{4.3.9} \\
                    & = x^0     &  & \by{4.3.9} \\
                    & = x^{n0}. &  & \by{4.1.6}
          \end{align*}
          So the base case holds.
          Suppose inductively that for some \(m \in \N\) we have \((x^n)^m = x^{nm}\).
          Then for \(m + 1\), we have \(m + 1 > 0\) and
          \begin{align*}
            (x^n)^{m + 1} & = (x^n)^m x^n   &  & \by{4.3.9}                    \\
                          & = x^{nm} x^n    &  & \byIH                         \\
                          & = x^{nm + n}    &  & \text{(from the proof above)} \\
                          & = x^{n(m + 1)}. &  & \by{ac:2.3.3}
          \end{align*}
          This closes the induction.
    \item \(m < 0\).
          Then by \cref{ac:4.2.7} we have \(-m > 0\) and
          \begin{align*}
            (x^n)^m & = \dfrac{1}{(x^n)^{-m}} &  & \by{4.3.11}                  \\
                    & = \dfrac{1}{x^{n(-m)}}  &  & \text{(from the first case)} \\
                    & = \dfrac{1}{x^{-nm}}    &  & \by{ac:4.1.5}                \\
                    & = x^{nm}.               &  & \by{4.3.11}
          \end{align*}
  \end{itemize}
  From all cases above we conclude that \((x^n)^m = x^{nm}\).

  Finally we show that \((xy)^n = x^n y^n\).
  By \cref{4.1.5} exactly one of the following two statements is true:
  \begin{itemize}
    \item \(n \geq 0\).
          Then by \cref{4.3.10}(a) we know that \((xy)^n = x^n y^n\).
    \item \(n < 0\).
          Then by \cref{ac:4.2.7} we have \(-n > 0\) and
          \begin{align*}
            (xy)^n & = \dfrac{1}{(xy)^{-n}}     &  & \by{4.3.11}    \\
                   & = \dfrac{1}{x^{-n} y^{-n}} &  & \by{4.3.10}[a] \\
                   & = x^n y^n.                 &  & \by{4.3.11}
          \end{align*}
  \end{itemize}
  From all cases above we conclude that \((xy)^n = x^n y^n\).
\end{proof}

\begin{proof}[\pf{4.3.12}(b)]
  First suppose that \(n > 0\).
  Then by \cref{4.3.10}(b)(c) we have \(x^n \geq y^n > 0\).
  Now suppose that \(n < 0\).
  Then by \cref{ac:4.2.7} we have \(-n > 0\) and
  \begin{align*}
             & x^{-n} \geq y^{-n} > 0                                                      &  & \by{4.3.10}[b,c] \\
    \implies & \dfrac{1}{x^n} \geq \dfrac{1}{y^n} > 0                                      &  & \by{4.3.11}      \\
    \implies & \pa{\dfrac{1}{x^n} \geq \dfrac{1}{y^n} > 0} \land (x^n > 0) \land (y^n > 0) &  & \by{ac:4.2.8}    \\
    \implies & \dfrac{x^n y^n}{x^n} \geq \dfrac{x^n y^n}{y^n} > 0(x^n y^n)                 &  & \by{4.2.9}[e]    \\
    \implies & y^n \geq x^n > 0.                                                           &  & \by{4.2.3}
  \end{align*}
\end{proof}

\begin{proof}[\pf{4.3.12}(c)]
  Suppose for sake of contradiction that \(x \neq y\).
  Then by \cref{4.2.9}(a) exactly one of the following two statements is true:
  \begin{itemize}
    \item \(x > y\).
          By hypothesis we know that \(n \neq 0\).
          Then by \cref{4.1.11}(f) exactly one of the following two statements is true:
          \begin{itemize}
            \item \(n > 0\).
                  But by \cref{4.3.10}(c) we must have \(x^n > y^n\), which contradicts to \(x^n = y^n\).
            \item \(n < 0\).
                  Then by \cref{ac:4.2.7} we have \(-n > 0\) and by \cref{4.3.10}(b)(c) we have \(x^{-n} > y^{-n} > 0\).
                  Since
                  \begin{align*}
                    x^n x^{-n} & = x^{n + (-n)} &  & \by{4.3.12}[a] \\
                               & = x^0          &  & \by{4.1.6}     \\
                               & = 1            &  & \by{4.3.9}     \\
                               & > 0
                  \end{align*}
                  and \(x^{-n} > 0\), by \cref{ac:4.2.8} we know that \(x^n > 0\).
                  Similarly we have \(y^n > 0\).
                  Thus by \cref{ac:4.2.7} \(x^n y^n > 0\).
                  But then we have
                  \begin{align*}
                             & (x^{-n} > y^{-n} > 0) \land (x^n y^n > 0)                       \\
                    \implies & x^{-n} x^n y^n > y^{-n} x^n y^n > 0 x^n y^n &  & \by{4.2.9}[e]  \\
                    \implies & x^{-n} x^n y^n > x^n y^{-n} y^n > 0         &  & \by{4.2.4}     \\
                    \implies & x^{(-n) + n} y^n > x^n y^{(-n) + n} > 0     &  & \by{4.3.12}[a] \\
                    \implies & x^0 y^n > x^n y^0 > 0                       &  & \by{4.1.6}     \\
                    \implies & 1 y^n > x^n 1 > 0                           &  & \by{4.3.9}     \\
                    \implies & y^n > x^n > 0,                              &  & \by{4.2.4}
                  \end{align*}
                  which contradicts to \(x^n = y^n\).
          \end{itemize}
    \item \(x < y\).
          By \cref{4.2.9} we know that \(x < y \implies y > x\), and we can use the first cases to derive either \(y^n > x^n\) or \(x^n > y^n\).
          But again this contradicts to \(x^n = y^n\).
  \end{itemize}
  From all cases above we derive contradictions, thus we must have \(x = y\).
\end{proof}

\begin{proof}[\pf{4.3.12}(d)]
  First we claim that \(\abs{x^{-1}} = \abs{x}^{-1}\).
  By \cref{ac:4.2.3} we know that \(x^{-1} \neq 0\).
  By \cref{4.2.4} we have \(x x^{-1} = 1\).
  Thus by \cref{ac:4.2.7,ac:4.2.8} both \(x, x^{-1}\) are positive or negative.
  So we use \cref{4.2.9}(a) to split into two cases:
  \begin{itemize}
    \item If \(x > 0\) and \(x^{-1} > 0\), then we have
          \begin{align*}
            \abs{x^{-1}} & = x^{-1}               &  & \by{4.3.1}  \\
                         & = \dfrac{1}{x^1}       &  & \by{4.3.11} \\
                         & = \dfrac{1}{x}         &  & \by{4.3.9}  \\
                         & = \dfrac{1}{\abs{x}}   &  & \by{4.3.1}  \\
                         & = \dfrac{1}{\abs{x}^1} &  & \by{4.3.9}  \\
                         & = \abs{x}^{-1}.        &  & \by{4.3.11}
          \end{align*}
    \item If \(x < 0\) and \(x^{-1} < 0\), then we have
          \begin{align*}
            \abs{x^{-1}} & = -(x^{-1})            &  & \by{4.3.1}    \\
                         & = (-1) (x^{-1})        &  & \by{ac:4.2.5} \\
                         & = \dfrac{-1}{x}        &  & \by{ac:4.2.4} \\
                         & = \dfrac{1}{-x}        &  & \by{ac:4.2.5} \\
                         & = \dfrac{1}{\abs{x}}   &  & \by{4.3.1}    \\
                         & = \dfrac{1}{\abs{x}^1} &  & \by{4.3.9}    \\
                         & = \abs{x}^{-1}.        &  & \by{4.3.11}
          \end{align*}
  \end{itemize}
  From all cases above we conclude that \(\abs{x^{-1}} = \abs{x}^{-1}\).

  Now we show that \(\abs{x^n} = \abs{x}^n\).
  By \cref{4.1.11}(f) exactly one of the following two statements is true:
  \begin{itemize}
    \item \(n \geq 0\).
          Then by \cref{4.3.10}(d) we have \(\abs{x^n} = \abs{x}^n\).
    \item \(n < 0\).
          Then by \cref{ac:4.2.7} we have \(-n > 0\) and
          \begin{align*}
            \abs{x^n} & = \abs{x^{-(-n)}}     &  & \by{ac:4.1.6}                 \\
                      & = \abs{x^{(-1) (-n)}} &  & \by{ac:4.1.5}                 \\
                      & = \abs{(x^{-1})^{-n}} &  & \by{4.3.12}[a]                \\
                      & = \abs{x^{-1}}^{-n}   &  & \by{4.3.10}[d]                \\
                      & = (\abs{x}^{-1})^{-n} &  & \text{(from the claim above)} \\
                      & = \abs{x}^{(-1) (-n)} &  & \by{4.3.12}[a]                \\
                      & = \abs{x}^n.          &  & \by{ac:4.1.6}
          \end{align*}
  \end{itemize}
  From all cases above we conclude that \(\abs{x^n} = \abs{x}^n\).
\end{proof}

\exercisesection

\begin{ex}\label{ex:4.3.1}
  Prove \cref{4.3.3}.
\end{ex}

\begin{proof}[\pf{ex:4.3.1}]
  See \cref{4.3.3}.
\end{proof}

\begin{ex}\label{ex:4.3.2}
  Prove the remaining claims in \cref{4.3.7}.
\end{ex}

\begin{proof}[\pf{ex:4.3.2}]
  See \cref{4.3.7}.
\end{proof}

\begin{ex}\label{ex:4.3.3}
  Prove \cref{4.3.10}.
\end{ex}

\begin{proof}[\pf{ex:4.3.3}]
  See \cref{4.3.10}.
\end{proof}

\begin{ex}\label{ex:4.3.4}
  Prove \cref{4.3.12}.
\end{ex}

\begin{proof}[\pf{ex:4.3.4}]
  See \cref{4.3.12}.
\end{proof}

\begin{ex}\label{ex:4.3.5}
  Prove that \(2^N \geq N\) for all positive integers \(N\).
\end{ex}

\begin{proof}[\pf{ex:4.3.5}]
  We induct on \(N\) and start with \(N = 1\).
  For \(N = 1\), by \cref{4.3.9} we have
  \[
    2^1 = 2^0 \times 2 = 1 \times 2 = 2 \geq 1,
  \]
  so the base case holds.
  Suppose inductively that for some \(N \in \N \setminus \set{0}\) we have \(2^N \geq N\).
  Then for \(N + 1\), we have
  \begin{align*}
             & 0 < N                  &  & \by{ac:2.2.4}  \\
    \implies & N = 0 + N < N + N = 2N &  & \by{2.2.12}[d] \\
    \implies & N + 1 \leq 2N          &  & \by{2.2.12}[e]
  \end{align*}
  and
  \begin{align*}
             & 2^N \geq N                      &  & \byIH           \\
    \implies & 2^N \times 2 \geq 2N \geq N + 1 &  & \by{4.2.9}[c,e] \\
    \implies & 2^{N + 1} \geq N + 1.           &  & \by{4.3.9}
  \end{align*}
  This closes the induction.
\end{proof}

\section{Gaps in the rational numbers}\label{i:sec:4.4}

\begin{ac}[Euclidean algorithm]\label{i:ac:4.4.1}
  Let \(n \in \Z\) and let \(q \in \Z^+\).
  Then there exist an unique pair of \((m, r) \in \Z \times \N\) such that \(0 \leq r < q\) and \(n = mq + r\).
\end{ac}

\begin{proof}[\pf{i:ac:4.4.1}]
  We first show that there exists at least one pairs of \((m, r) \in \Z \times \N\) satisfy the statement.
  By \cref{i:4.1.5} exactly one of the following two statements is true:
  \begin{itemize}
    \item \(n \geq 0\).
          Then by \cref{i:2.3.9} we know that there exists a pair of \((m, r) \in \N \times \N\) such that \(0 \leq r < q\) and \(n = mq + r\).
          By \cref{i:ac:4.1.2} we have \(m \in \Z\).
          Thus there exists a pair of \((m, r) \in \Z \times \N\) such that \(0 \leq r < q\) and \(n = mq + r\).
    \item \(n < 0\).
          Then by \cref{i:ac:4.2.7} we have \(-n > 0\) and by \cref{i:2.3.9} there exists a pair of \((m, r) \in \N \times \N\) such that \(0 \leq r < q\) and \(-n = mq + r\).
          Fix one such pair \((m, r)\).
          Since \(0 \leq r\), by \cref{i:4.1.11}(d) we have \(-r \leq 0\), and by \cref{i:4.1.11}(b) we have \(q - r \leq q\).
          Since \(r < q\), by \cref{i:4.1.11}(a) we have \(q - r > 0\).
          Thus by \cref{i:4.1.11}(e) we have \(0 < q - r \leq q\).
          Now observe that
          \begin{align*}
            n & = -(-n)                         &  & \by{i:ac:4.1.6}         \\
              & = -(mq + r)                     &  & \by{i:2.3.9}            \\
              & = -(mq + q + (-q) + r)          &  & \by{i:4.1.6}            \\
              & = -((m + 1) q + (r - q))        &  & \by{i:4.1.6,i:ac:4.1.4} \\
              & = (-1) ((m + 1) q + (r - q))    &  & \by{i:ac:4.1.5}         \\
              & = (-1) (m + 1) q + (-1) (r - q) &  & \by{i:4.1.6}            \\
              & = -(m + 1) q + (q - r).         &  & \by{i:ac:4.1.5,i:4.1.6}
          \end{align*}
          Since \(-(m + 1) \in \Z\) and \(q > q - r \geq 0\), by setting \(m' = -m - 1\) and \(r' = q - r\) we see that \(n = m'q + r'\) satisfy the statement.
  \end{itemize}
  From all cases above we conclude that at least one pairs of \((m, r) \in \Z \times \N\) satisfy the statement.

  Now we show the uniqueness of such \((m, r) \in \Z \times \N\).
  Let \((m, r), (m', r') \in \Z \times \N\) such that
  \[
    (n = mq + r = m'q + r') \land (0 \leq r < q) \land (0 \leq r' < q).
  \]
  Suppose for sake of contradiction that \(r \neq r'\).
  By \cref{i:4.1.5} exactly one of the following two statements is true:
  \begin{itemize}
    \item \(r > r'\).
          Let \(a = r - r'\)
          By \cref{i:4.1.11}(a) we know that \(a \in \Z^+\).
          Then we have
          \begin{align*}
                     & mq + r = m'q + r'                                     \\
            \implies & r - r' = (m' - m)q    &  & \by{i:4.1.6}               \\
            \implies & m' - m > 0            &  & \by{i:ac:4.2.8,i:ac:4.2.9} \\
            \implies & m' - m \geq 1         &  & \by{i:2.2.12}[e]           \\
            \implies & (m' - m)q \geq q      &  & \by{i:2.3.6}               \\
            \implies & r - r' \geq q                                         \\
            \implies & r \geq q + r' \geq q. &  & \by{i:4.1.11}[b]
          \end{align*}
          But this contradict to \(r < q\).
    \item \(r < r'\).
          By \cref{i:4.1.10} we have \(r' > r\).
          Using similar arguments as above, we derive a contradiction.
  \end{itemize}
  From all cases above we derive contradictions, thus we must have \(r = r'\).
  This means
  \begin{align*}
             & mq + r = m'q + r'                   \\
    \implies & mq + r = m'q + r  &  & \by{i:4.1.3} \\
    \implies & mq = m'q          &  & \by{i:4.1.6} \\
    \implies & m = m'.           &  & \by{i:4.1.9}
  \end{align*}
  Thus such \((m, r) \in \Z \times \N\) is unique.
\end{proof}

\begin{prop}[Interspersing of integers by rationals]\label{i:4.4.1}
  Let \(x\) be a rational number.
  Then there exists an integer \(n\) such that \(n \leq x < n + 1\).
  In fact, this integer is unique (i.e., for each \(x\) there is only one \(n\) for which \(n \leq x < n + 1\)).
  In particular, there exists a natural number \(N\) such that \(N > x\)
  (i.e., there is no such thing as a rational number which is larger than all the natural numbers).
\end{prop}

\begin{proof}[\pf{i:4.4.1}]
  By \cref{i:4.2.1} we know that \(x = a / b\) where \(a, b \in \Z\) and \(b > 0\).
  Since \(a \in \Z\) and \(b \in \Z^+\), by \cref{i:ac:4.4.1} we know that there exists a pair of \((m, r) \in \Z \times \N\) such that \(a = mb + r\) and \(0 \leq r < b\).
  Then we have
  \begin{align*}
             & (a = mb + r) \land (0 \leq r < b)                                                                \\
    \implies & \pa{x = \dfrac{a}{b} = m + \dfrac{r}{b}} \land (0 \leq r < b)               &  & \by{i:4.2.4}    \\
    \implies & \pa{x = \dfrac{a}{b} = m + \dfrac{r}{b}} \land \pa{0 \leq \dfrac{r}{b} < 1} &  & \by{i:4.2.9}[e] \\
    \implies & m \leq x = m + \dfrac{r}{b} < m + 1.                                        &  & \by{i:4.2.9}[d]
  \end{align*}

  Next we show the uniqueness of such \(n\).
  Suppose there exists another \(n' \in \Z\) such that \(n' \leq x < n' + 1\).
  Then we have
  \begin{align*}
             & (n \leq x < n + 1) \land (n' \leq x < n' + 1)                       \\
    \implies & (n < n' + 1) \land (n' < n + 1)               &  & \by{i:4.2.9}[c]  \\
    \implies & (n' + 1 - n > 0) \land (n + 1 - n' > 0)       &  & \by{i:4.2.9}[d]  \\
    \implies & (n' + 1 - n \geq 1) \land (n + 1 - n' \geq 1) &  & \by{i:2.2.12}[e] \\
    \implies & (n' \geq n) \land (n \geq n')                 &  & \by{i:4.2.9}[d]  \\
    \implies & n' = n.                                       &  & \by{i:4.2.9}[a]
  \end{align*}
  Thus such \(n\) is unique.

  Finally we show that there exists an \(N \in \N\) such that \(N > x\).
  By \cref{i:4.2.9}(a) we split into three cases:
  \begin{itemize}
    \item \(x = 0\).
          Then by setting \(N = 1\) we have \(N > x\).
    \item \(x < 0\).
          Then by setting \(N = 0\) we have \(N > x\).
    \item \(x > 0\).
          From the proof above we know that there exists an \(n \in \Z\) such that \(n \leq x < n + 1\).
          Since \(x > 0\), by \cref{i:4.2.9}(c) we know that \(n + 1 > 0\).
          Thus \(n + 1 \in \N\).
          By setting \(N = n + 1\) we have \(N > x\).
  \end{itemize}
  From all cases above we conclude that there exists an \(N \in \N\) such that \(x < N\).
\end{proof}

\begin{rmk}\label{i:4.4.2}
  The integer \(n\) for which \(n \leq x < n + 1\) is sometimes referred to as the \emph{integer part} of \(x\) and is sometimes denoted \(n = \floor{x}\).
\end{rmk}

\begin{prop}[Interspersing of rationals by rationals]\label{i:4.4.3}
  If \(x\) and \(y\) are two rationals such that \(x < y\), then there exists a third rational \(z\) such that \(x < z < y\).
\end{prop}

\begin{proof}[\pf{i:4.4.3}]
  We set \(z \coloneqq (x + y) / 2\).
  Since \(x < y\), and \(1 / 2 = 1 // 2\) is positive, we have from \cref{i:4.2.9} that \(x / 2 < y / 2\).
  If we add \(y / 2\) to both sides using \cref{i:4.2.9} we obtain \(x / 2 + y / 2 < y / 2 + y / 2\), i.e., \(z < y\).
  If we instead add \(x / 2\) to both sides we obtain \(x / 2 + x / 2 < y / 2 + x / 2\), i.e., \(x < z\).
  Thus \(x < z < y\) as desired.
\end{proof}

\begin{note}
  Despite the rationals having this denseness property, they are still incomplete;
  there are still an infinite number of ``gaps'' or ``holes'' between the rationals, although this denseness property does ensure that these holes are in some sense infinitely small.
\end{note}

\begin{ac}\label{i:ac:4.4.2}
  Let \(n \in \N\).
  Define \(n\) to be \emph{even} if \(n = 2m\) for some \(m \in \N\), and \emph{odd} if \(n = 2m + 1\) for some \(m \in \N\).
  Then every natural number is either even or odd, but not both.
\end{ac}

\begin{proof}[\pf{i:ac:4.4.2}]
  We induct on \(n\).
  For \(n = 0\), by \cref{i:2.3.1,i:2.3.2} we have \(0 = 0 \times 2 = 2 \times 0\).
  By \cref{i:2.3} we have \(0 \neq 2m + 1\).
  Thus \(0\) is even and is not odd, so the base case holds.
  Suppose inductively that for some \(n \in N\), there exists some \(m \in \N\) such that either \(n = 2m\) or \(n = 2m + 1\) is true, but not both.
  Then we need to show that the same statement holds for \(n + 1\).
  By induction hypothesis we can split into two cases:
  \begin{itemize}
    \item \(n = 2m\) for some \(m \in \N\).
          Then \(n + 1 = 2m + 1\), which means \(n + 1\) is odd.
    \item \(n = 2m + 1\) for some \(m \in \N\).
          Then by \cref{i:2.2.5,i:2.3.4} we have \(n + 1 = 2m + 2 = 2(m + 1)\), which means \(n + 1\) is even.
  \end{itemize}
  Thus \(n + 1\) is either even or odd.
  Now we show that we cannot have \(n + 1\) be both even and odd.
  Suppose for sake of contradiction that \(n + 1\) is both even and odd.
  Then there exist \(m, k \in \N\) such that \(n + 1 = 2m = 2k + 1\).
  By \cref{i:2.3} we know that \(2m = n + 1 > 0\), thus by \cref{i:2.3.3} we must have \(m > 0\).
  By \cref{i:2.2.12}(e) we have \(m \geq 1\), so by \cref{i:4.1.11}(b) we have \(m - 1 \geq 0\), and by definition \(2(m - 1) + 1\) is odd.
  Now we use \cref{i:4.1.6} to rewrite \(2m = 2 (m - 1 + 1) = 2 (m - 1) + 2\).
  Therefore by \cref{i:2.2.10} \(n + 1 = 2m \implies n = 2 (m - 1) + 1\), which means \(n\) is odd.
  But by \cref{i:2.2.10} again we have \(n + 1 = 2k + 1 \implies n = 2k\), which means \(n\) is even, contradict to induction hypothesis that \(n\) cannot be both even and odd.
  Therefore \(n + 1\) cannot be both even and odd.
  This closes the induction.
\end{proof}

\begin{ac}\label{i:ac:4.4.3}
  Let \(n\) be a natural number.
  If \(n\) is even, then \(n^2\) is also even.
  If \(n\) is odd, then \(n^2\) is also odd.
\end{ac}

\begin{proof}[\pf{i:ac:4.4.3}]
  We first show that \(n\) is even implies \(n^2\) is even.
  Since \(n\) is even, by \cref{i:ac:4.4.2} there exists an \(m \in \N\) such that \(n = 2m\).
  By \cref{i:ac:2.3.1} we have \(m (2m) \in \N\).
  Since
  \begin{align*}
    n^2 & = (2m)^2                         \\
        & = (2m) (2m)   &  & \by{i:2.3.11} \\
        & = 2 (m (2m)), &  & \by{i:2.3.5}
  \end{align*}
  by \cref{i:ac:4.4.2} this means \(n^2\) is even.

  Now we show that \(n\) is odd implies \(n^2\) is odd.
  Since \(n\) is odd, by \cref{i:ac:4.4.2} there exists an \(m \in \N\) such that \(n = 2m + 1\).
  By \cref{i:ac:2.3.1} we know that \(m (2m) \in \N\), thus by \cref{i:ac:2.2.1} we have \(m (2m) + 2m \in \N\).
  Since
  \begin{align*}
    n^2 & = (2m + 1)^2                                   \\
        & = (2m)^2 + 2 (2m) + 1^2   &  & \by{i:ex:2.3.4} \\
        & = (2m) (2m) + 2 (2m) + 1  &  & \by{i:2.3.11}   \\
        & = 2 (m (2m)) + 2 (2m) + 1 &  & \by{i:2.3.5}    \\
        & = 2 (m (2m) + 2m) + 1,    &  & \by{i:2.3.4}
  \end{align*}
  by \cref{i:ac:4.4.2} this means \(n^2\) is odd.
\end{proof}

\begin{prop}\label{i:4.4.4}
  There does not exist any rational number \(x\) for which \(x^2 = 2\).
\end{prop}

\begin{proof}[\pf{i:4.4.4}]
  Suppose for sake of contradiction that we had a rational number \(x\) for which \(x^2 = 2\).
  Clearly, \(x\) is not zero by \cref{i:4.3.12}(c).
  We may assume that \(x\) is positive, for if \(x\) were negative then we could just replace \(x\) by \(-x\)
  (since \(x^2 = (-x)^2\)).
  Thus \(x = p / q\) for some positive integers \(p, q\), so \((p / q)^2 = 2\), which we can rearrange as \(p^2 = 2q^2\).
  By \cref{i:ac:4.4.2}, every natural number is either even or odd, but not both.
  By \cref{i:ac:4.4.3}, if \(p\) is odd, then \(p^2\) is also odd, which contradicts \(p^2 = 2q^2\).
  Thus \(p\) is even, i.e., \(p = 2k\) for some natural number \(k\).
  Since \(p\) is positive, \(k\) must also be positive.
  Inserting \(p = 2k\) into \(p^2 = 2q^2\) we obtain \(4k^2 = 2q^2\), so that \(q^2 = 2k^2\).

  To summarize, we started with a pair \((p, q)\) of positive integers such that \(p^2 = 2q^2\), and ended up with a pair \((q, k)\) of positive integers such that \(q^2 = 2k^2\).
  Now we claim that \(p^2 = 2q^2 \implies q < p\).
  We prove the claim by contradiction.
  Suppose for sake of contradiction that \(q \geq p\).
  \begin{itemize}
    \item If \(q = p\), then we have \(p^2 = q^2\).
          But \(p^2 = 2q^2 = q^2 + q^2\) implies \(q^2 = 0\), and by \cref{i:4.3.10}(b) this means \(q = 0\), which contradicts to \(q > 0\).
    \item If \(q > p\), then by \cref{i:4.3.10}(c) we have \(q^2 > p^2\).
          Since \(q > 0\), by \cref{i:4.3.10}(b) we have \(q^2 > 0\).
          But by \cref{i:2.2.12}(f) \(p^2 = 2q^2 = q^2 + q^2\) implies \(p^2 > q^2\), which contradicts to \(q^2 > p^2\).
  \end{itemize}
  From all cases above we derive contradiction.
  Thus we must have \(p^2 = 2q^2 \implies q < p\).
  If we rewrite \(p' \coloneqq q\) and \(q' \coloneqq k\), we thus can pass from one solution \((p, q)\) to the equation \(p^2 = 2q^2\) to a new solution \((p', q')\) to the same equation which has a smaller value of \(p\).
  But then we can repeat this procedure again and again, obtaining a sequence \((p'', q'')\), \((p''', q''')\), etc. of solutions to \(p^2 = 2q^2\), each one with a smaller value of \(p\) than the previous, and each one consisting of positive integers.
  But this contradicts the principle of infinite descent (see \cref{i:ex:4.4.2}).
  This contradiction shows that we could not have had a rational \(x\) for which \(x^2 = 2\).
\end{proof}

\begin{prop}\label{i:4.4.5}
  For every rational number \(\varepsilon > 0\), there exists a non-negative rational number \(x\) such that \(x^2 < 2 < (x + \varepsilon)^2\).
\end{prop}

\begin{proof}[\pf{i:4.4.5}]
  Let \(\varepsilon > 0\) be rational.
  Suppose for sake of contradiction that there is no non-negative rational number \(x\) for which \(x^2 < 2 < (x + \varepsilon)^2\).
  This means that whenever \(x\) is non-negative and \(x^2 < 2\), we must also have \((x + \varepsilon)^2 < 2\)
  (note that \((x + \varepsilon)^2\) cannot equal \(2\), by \cref{i:4.4.4}).
  Since \(0^2 < 2\), we thus have \(\varepsilon^2 < 2\), which then implies \((2\varepsilon)^2 < 2\), and indeed a simple induction shows that \((n\varepsilon)^2 < 2\) for every natural number \(n\).
  (Note that \(n\varepsilon\) is non-negative for every natural number \(n\) by \cref{i:ac:4.2.7})
  But, by \cref{i:4.4.1} we can find an integer \(n\) such that \(n > 2 / \varepsilon\), which implies that \(n\varepsilon > 2\), which implies that \((n\varepsilon)^2 > 4 > 2\), contradicting the claim that \((n\varepsilon)^2 < 2\) for all natural numbers \(n\).
  This contradiction gives the proof.
\end{proof}

\begin{note}
  \cref{i:4.4.5} indicates that, while the set \(\Q\) of rationals does not actually have \(\sqrt{2}\) as a member, we can get as close as we wish to \(\sqrt{2}\).
  For instance, the sequence of rationals
  \[
    1.4, 1.41, 1.414, 1.4142, 1.41421, \dots
  \]
  seem to get closer and closer to \(\sqrt{2}\), as their squares indicate:
  \[
    1.96, 1.9881, 1.99396, 1.99996164, 1.9999899241, \dots
  \]
  Thus it seems that we can create a square root of \(2\) by taking a ``limit'' of a sequence of rationals.
  This is how we shall construct the real numbers in \cref{i:ch:5}.
\end{note}

\begin{note}
  There is another way to construct the real numbers, using something called ``Dedekind cuts'', which we will not pursue here.
  One can also proceed using infinite decimal expansions, but there are some sticky issues when doing so, e.g., one has to make \(0.999\dots\) equal to \(1.000\dots\), and this approach, despite being the most familiar, is actually more complicated than other approaches.
\end{note}

\exercisesection

\begin{ex}\label{i:ex:4.4.1}
  Prove \cref{i:4.4.1}.
\end{ex}

\begin{proof}[\pf{i:ex:4.4.1}]
  See \cref{i:4.4.1}.
\end{proof}

\begin{ex}\label{i:ex:4.4.2}
  A definition: a sequence \(a_0, a_1, a_2, \dots\) of numbers (natural numbers, integers, rationals, or reals) is said to be in \emph{infinite descent} if we have \(a_n > a_{n + 1}\) for all natural numbers \(n\)
  (i.e., \(a_0 > a_1 > a_2 > \dots\)).
  \begin{enumerate}
    \item Prove the \emph{principle of infinite descent}:
          that it is not possible to have a sequence of \emph{natural numbers} which is in infinite descent.
    \item Does the principle of infinite descent work if the sequence \(a_1, a_2, a_3, \dots\) is allowed to take integer values instead of natural number values?
          What about if it is allowed to take positive rational values instead of natural numbers?
          Explain.
  \end{enumerate}
\end{ex}

\begin{proof}[\pf{i:ex:4.4.2}(a)]
  Suppose for sake of contradiction that there exists a sequence of natural numbers \(a_0, a_1, \dots\) is in infinite descent.
  Now we claim that if \(k \in \N\), then \(a_n \geq k\) for all \(n \in \N\).
  We induct on \(k\) to prove the claim.
  For \(k = 0\), by \cref{i:ac:2.2.4} we have \(a_n \geq 0\) for all \(n \in \N\), so the base case holds.
  Suppose inductively that for some \(k \in \N\) we have \(a_n \geq k\) for all \(n \in \N\).
  Then for \(k + 1\), we want to show that \(a_n \geq k + 1\) for all \(n \in \N\).
  By induction hypothesis we have \(a_n \geq k\) for all \(n \in \N\).
  Since the sequence \(a_0, a_1, \dots\) is in infinite descent, we know that \(a_n > a_{n + 1} \geq k\) for all \(n \in \N\).
  Thus we have
  \begin{align*}
             & \forall n \in \N, a_n > a_{n + 1} \geq k &  & \byIH            \\
    \implies & \forall n \in \N, a_n > k                &  & \by{i:2.2.12}[b] \\
    \implies & \forall n \in \N, a_n \geq k + 1.        &  & \by{i:2.2.12}[e]
  \end{align*}
  This closes the induction.

  Now we show that such sequence does not exist.
  From the proof above we see that \(a_1 \geq k\) for all \(k \in \N\).
  If we set \(k = a_0\), then we must have \(a_1 \geq a_0\).
  Since the sequence is in infinite descent, we must have \(a_0 > a_1\).
  But both \(a_1 \geq a_0\) and \(a_0 > a_1\) being true contradict to \cref{i:2.2.13}.
  So we cannot have a sequence of natural number which is in infinite descent.
\end{proof}

\begin{proof}[\pf{i:ex:4.4.2}(b)]
  By setting \(a_n = -n\) for all \(n \in \N\), we can always have \(a_n > a_{n + 1}\).
  So the principle of infinite descent does not work on integers.

  Similarly, by setting \(a_n = 1 / n\) for all \(n \in \N\), we can always have \(a_n > a_{n + 1}\).
  So the principle of infinite descent does not work on rationals.
\end{proof}

\begin{ex}\label{i:ex:4.4.3}
  Fill in the gaps marked (why?) in the proof of \cref{i:4.4.4}.
\end{ex}

\begin{proof}[\pf{i:ex:4.4.3}]
  See \cref{i:4.4.4}.
\end{proof}


\chapter{The real numbers}\label{ch:5}

\begin{note}
  We defined the natural numbers using the five Peano axioms, and postulated that such a number system existed;
  this is plausible, since the natural numbers correspond to the very intuitive and fundamental notion of \emph{sequential counting}.
\end{note}

\begin{note}
  The symbols \(\N\), \(\Q\), and \(\R\) stand for ``natural'', ``quotient'', and ``real'' respectively.
  \(\Z\) stands for ``Zahlen'', the German word for numbers.
  There is also the \emph{complex numbers} \(\C\), which obviously stands for ``complex''.
\end{note}

\begin{note}
  \emph{Formal} means ``having the form of'';
  at the beginning of our construction the expression \(a \text{---} b\) did not actually \emph{mean} the difference \(a - b\), since the symbol \text{---} was meaningless.
  It only had the \emph{form} of a difference.
  Later on we defined subtraction and verified that the formal difference was equal to the actual difference, so this eventually became a non-issue, and our symbol for formal differencing was discarded.
  Somewhat confusingly, this use of the term ``formal'' is unrelated to the notions of a formal argument and an informal argument.
\end{note}

\begin{note}
  There is a fundamental area of mathematics where the rational number system does not suffice - that of \emph{geometry}
  (the study of lengths, areas, etc.).
  For instance, a right-angled triangle with both sides equal to \(1\) gives a hypotenuse of \(\sqrt{2}\), which is an \emph{irrational} number, i.e., not a rational number;
  see \cref{4.4.4}.
  Things get even worse when one starts to deal with the sub-field of geometry known as \emph{trigonometry}, when one sees numbers such as \(\pi\) or \(\cos(1)\), which turn out to be in some sense ``even more'' irrational than \(\sqrt{2}\).
  (These numbers are known as \emph{transcendental numbers}, but to discuss this further would be far beyond the scope of this text.)
  Thus, in order to have a number system which can adequately describe geometry
  - or even something as simple as measuring lengths on a line
  - one needs to replace the rational number system with the real number system.
\end{note}

\begin{note}
  In the constructions of integers and rationals, the task was to introduce one more \emph{algebraic} operation to the number system
  - e.g., one can get integers from naturals by introducing subtraction, and get the rationals from the integers by introducing division.
  But to get the reals from the rationals is to pass from a ``discrete'' system to a ``continuous'' one, and requires the introduction of a somewhat different notion
  - that of a \emph{limit}.
\end{note}

\begin{note}
  The limit is a concept which on one level is quite intuitive, but to pin down rigorously turns out to be quite difficult.
  (Even such great mathematicians as Euler and Newton had difficulty with this concept.
  It was only in the nineteenth century that mathematicians such as Cauchy and Dedekind figured out how to deal with limits rigorously.)
\end{note}

\begin{note}
  The procedure we give here of obtaining the real numbers as the limit of sequences of rational numbers may seem rather complicated.
  However, it is in fact an instance of a very general and useful procedure, that of \emph{completing} one metric space to form another.
\end{note}

\section{Cauchy sequences}\label{i:sec:5.1}

\begin{defn}[Sequences]\label{i:5.1.1}
  Let \(m\) be an integer.
  A \emph{sequence \((a_n)_{n = m}^{\infty}\) of rational numbers} is any function from the set \(\Z_{\geq m}\) to \(\Q\), i.e., a mapping which assigns to each integer \(n\) greater than or equal to \(m\), a rational number \(a_n\).
  More informally, a sequence \((a_n)_{n = m}^{\infty}\) of rational numbers is a collection of rationals \(a_m, a_{m + 1}, a_{m + 2}, \dots\).
\end{defn}

\setcounter{thm}{2}
\begin{defn}[\(\varepsilon\)-steadiness]\label{i:5.1.3}
  Let \(\varepsilon \in \Q^+\).
  A sequence \((a_n)_{n = m}^{\infty}\) is said to be \emph{\(\varepsilon\)-steady} iff each pair \(a_j, a_k\) of sequence elements is \(\varepsilon\)-close for every natural number \(j, k \in \Z_{\geq m}\).
  In other words, the sequence \(a_m, a_{m + 1}, a_{m + 2}, \dots\) is \(\varepsilon\)-steady iff \(d(a_j, a_k) \leq \varepsilon\) for all \(j, k \in \Z_{\geq m}\).
\end{defn}

\begin{rmk}\label{i:5.1.4}
  \cref{i:5.1.3} is not standard in the literature;
  we will not need it outside of this section;
  similarly for the concept of ``eventual \(\varepsilon\)-steadiness'' below.
  We have defined \(\varepsilon\)-steadiness for sequences whose index starts at \(m\), but clearly we can make a similar notion for sequences whose indices start from any other number:
  a sequence \(a_N, a_{N + 1}, \dots\) is \(\varepsilon\)-steady if one has \(d(a_j, a_k) \leq \varepsilon\) for all \(j, k \in \Z_{\geq N}\).
\end{rmk}

\begin{note}
  The notion of \(\varepsilon\)-steadiness of a sequence is simple, but does not really capture the \emph{limiting} behavior of a sequence, because it is too sensitive to the initial members of the sequence.
  So we need a more robust notion of steadiness that does not care about the initial members of a sequence.
\end{note}

\setcounter{thm}{5}
\begin{defn}[Eventual \(\varepsilon\)-steadiness]\label{i:5.1.6}
  Let \(\varepsilon \in \Q^+\).
  A sequence \((a_n)_{n = m}^{\infty}\) is said to be \emph{eventually \(\varepsilon\)-steady} iff the sequence \(a_N, a_{N + 1}, a_{N + 2}, \dots\) is \(\varepsilon\)-steady for some integer \(N \geq m\).
  In other words, the sequence \(a_m, a_{m + 1}, a_{m + 2}, \dots\) is eventually \(\varepsilon\)-steady iff there exists an \(N \in \Z_{\geq m}\) such that \(\abs{a_j - a_k} \leq \varepsilon\) for all \(j, k \in \Z_{\geq N}\).
\end{defn}

\setcounter{thm}{7}
\begin{defn}[Cauchy sequences]\label{i:5.1.8}
  A sequence \((a_n)_{n = m}^{\infty}\) of rational numbers is said to be a \emph{Cauchy sequence} iff for every rational \(\varepsilon \in \Q^+\), the sequence \((a_n)_{n = m}^{\infty}\) is eventually \(\varepsilon\)-steady.
  In other words, the sequence \(a_m, a_{m + 1}, a_{m + 2}, \dots\) is a Cauchy sequence iff for every \(\varepsilon \in \Q^+\), there exists an \(N \in \Z_{\geq m}\) such that \(\abs{a_j - a_k} \leq \varepsilon\) for all \(j, k \in \Z_{\geq N}\).
\end{defn}

\begin{rmk}\label{i:5.1.9}
  At present, the parameter \(\varepsilon\) is restricted to be a positive rational;
  we cannot take \(\varepsilon\) to be an arbitrary positive real number, because the real numbers have not yet been constructed.
  However, once we do construct the real numbers, we shall see that \cref{i:5.1.8} will not change if we require \(\varepsilon\) to be real instead of rational (\cref{i:6.1.4}).
  In other words, we will eventually prove that a sequence is eventually \(\varepsilon\)-steady for every rational \(\varepsilon \in \Q^+\) iff it is eventually \(\varepsilon\)-steady for every real \(\varepsilon \in \Q^+\).
  This rather subtle distinction between a rational \(\varepsilon\) and a real \(\varepsilon\) turns out not to be very important in the long run, and the reader is advised not to pay too much attention as to what type of number \(\varepsilon\) should be.
\end{rmk}

\setcounter{thm}{10}
\begin{prop}\label{i:5.1.11}
  The sequence \((a_n)_{n = 1}^\infty\) defined by \(a_n \coloneqq 1 / n\) (i.e., the sequence \(1, 1 / 2, 1 / 3, \dots\)) is a Cauchy sequence.
\end{prop}

\begin{proof}[\pf{i:5.1.11}]
  We have to show that for every \(\varepsilon \in \Q^+\), the sequence \(a_1, a_2, \dots\) is eventually \(\varepsilon\)-steady.
  So let \(\varepsilon \in \Q^+\) be arbitrary.
  We now have to find a number \(N \in \Z_{\geq 1}\) such that the sequence \(a_N, a_{N + 1}, \dots\) is \(\varepsilon\)-steady.
  Let us see what this means.
  This means that \(d(a_j, a_k) \leq \varepsilon\) for every \(j, k \in \Z_{\geq N}\), i.e.
  \[
    \abs{\dfrac{1}{j} - \dfrac{1}{k}} \leq \varepsilon \text{ for every } j, k \in \Z_{\geq N}.
  \]
  Now since \(j, k \in \Z_{\geq N}\), by \cref{i:4.3.12}(b) we know that \(0 < 1 / j, 1 / k \leq 1 / N\), so that
  \begin{align*}
             & \begin{dcases}
                 \dfrac{-1}{N} < 0 < \dfrac{1}{j} \leq \dfrac{1}{N}  \\
                 \dfrac{-1}{N} \leq \dfrac{-1}{k} < 0 < \dfrac{1}{N} \\
               \end{dcases}                         &  & \by{i:ex:4.2.6}                                           \\
    \implies & \begin{dcases}
                 \dfrac{1}{j} - \dfrac{1}{k} \leq \dfrac{1}{N} - \dfrac{1}{k} < \dfrac{1}{N} \\
                 \dfrac{-1}{N} < \dfrac{1}{j} - \dfrac{1}{N} \leq \dfrac{1}{j} - \dfrac{1}{k}
               \end{dcases} &  & \by{i:4.2.9}[c,d]                         \\
    \implies & \dfrac{-1}{N} \leq \dfrac{1}{j} - \dfrac{1}{k} \leq \dfrac{1}{N}                                    \\
    \implies & \abs{\dfrac{1}{j} - \dfrac{1}{k}} \leq \dfrac{1}{N}.                           &  & \by{i:4.3.3}[c]
  \end{align*}
  So in order to force \(\abs{1 / j - 1 / k}\) to be less than or equal to \(\varepsilon\), it would be sufficient for \(1 / N\) to be less than \(\varepsilon\).
  So all we need to do is choose an \(N\) such that \(1 / N\) is less than \(\varepsilon\), or in other words that \(N\) is greater than \(1 / \varepsilon\).
  But this can be done thanks to \cref{i:4.4.1}.
\end{proof}

\begin{note}
  As you can see, verifying from first principles (i.e., without using any of the machinery of limits, etc.) that a sequence is a Cauchy sequence requires some effort, even for a sequence as simple as \(1 / n\).
  The part about selecting an \(N\) can be particularly difficult for beginners
  - one has to think in reverse, working out what conditions on \(N\) would suffice to force the sequence \(a_N, a_{N + 1}, a_{N + 2}, \dots\) to be \(\varepsilon\)-steady, and then finding an \(N\) which obeys those conditions.
  Later we will develop some limit laws which allow us to determine when a sequence is Cauchy more easily.
\end{note}

\begin{defn}[Bounded sequences]\label{i:5.1.12}
  Let \(M \in \Q_{\geq 0}\).
  A finite rational sequence \((a_n)_{n = m}^k\) is \emph{bounded by \(M\)} iff \(\abs{a_i} \leq M\) for all \(i \in \Z_{m \leq k}\).
  An infinite rational sequence \((a_n)_{n = m}^{\infty}\) is \emph{bounded by \(M\)} iff \(\abs{a_i} \leq M\) for all \(i \in \Z_{\geq m}\).
  A rational sequence is said to be \emph{bounded} iff it is bounded by \(M\) for some \(M \in \Q_{\geq 0}\).
\end{defn}

\setcounter{thm}{13}
\begin{lem}[Finite sequences are bounded]\label{i:5.1.14}
  Every finite rational sequence \((a_n)_{n = m}^k\) is bounded.
\end{lem}

\begin{proof}[\pf{i:5.1.14}]
  We induct on \(k\) and we start with \(k = m\).
  When \(k = m\) the rational sequence \((a_n)_{n = m}^m\) is clearly bounded, for if we choose \(M \coloneqq \abs{a_m}\) then clearly we have \(\abs{a_i} \leq M\) for all \(i \in \Z_{m \leq k}\).
  Now suppose that we have already proved the lemma for some \(k \geq m\);
  we now prove it for \(k + 1\), i.e., we prove every rational sequence \((a_n)_{n = m}^{k + 1}\) is bounded.
  By the induction hypothesis we know that \((a_n)_{n = m}^k\) is bounded by some \(M \in \Q_{\geq 0}\);
  in particular, it must be bounded by \(M + \abs{a_{k + 1}}\).
  On the other hand, \(a_{k + 1}\) is also bounded by \(M + \abs{a_{k + 1}}\).
  Thus, \((a_n)_{n = m}^{k + 1}\) is bounded by \(M + \abs{a_{k + 1}}\), and is hence bounded.
  This closes the induction.
\end{proof}

\begin{note}
  While \cref{i:5.1.14} shows that every finite rational sequence is bounded, no matter how long the finite sequence is, it does not say anything about whether an infinite rational sequence is bounded or not;
  infinity is not a natural number.
\end{note}

\begin{lem}[Cauchy sequences are bounded]\label{i:5.1.15}
  Every rational Cauchy sequence \((a_n)_{n = m}^{\infty}\) is bounded.
\end{lem}

\begin{proof}[\pf{i:5.1.15}]
  Since \((a_n)_{n = m}^{\infty}\) is a rational Cauchy sequence, by \cref{i:5.1.8} we know that \((a_n)_{n = m}^{\infty}\) is eventually \(\varepsilon\)-steady for all \(\varepsilon \in \Q^+\).
  In particular, \((a_n)_{n = m}^{\infty}\) is eventually \(1\)-steady.
  By \cref{i:5.1.6} there exists an \(N \in \Z_{\geq m}\) such that \((a_n)_{n = N}^{\infty}\) is \(1\)-steady.
  Fix such \(N\).
  Since \((a_n)_{n = N}^\infty\) is \(1\)-steady, we have
  \begin{align*}
             & \forall j \in \Z_{\geq N}, \abs{a_j - a_N} \leq 1                                                    &  & \by{i:5.1.3}    \\
    \implies & \forall j \in \Z_{\geq N}, \abs{a_j - a_N} + \abs{a_N} \leq 1 + \abs{a_N}                            &  & \by{i:4.2.9}[d] \\
    \implies & \forall j \in \Z_{\geq N}, \abs{a_j - a_N + a_N} \leq \abs{a_j - a_N} + \abs{a_N} \leq 1 + \abs{a_N} &  & \by{i:4.3.3}[b] \\
    \implies & \forall j \in \Z_{\geq N}, \abs{a_j} \leq 1 + \abs{a_N}.                                             &  & \by{i:4.2.4}
  \end{align*}
  Thus, by \cref{i:5.1.12} \((a_n)_{n = N}^\infty\) is bounded by \(1 + \abs{a_N}\).
  Now we split into two cases:
  \begin{itemize}
    \item If \(N = m\), then we see that \((a_n)_{n = m}^\infty\) is bounded by \(1 + \abs{a_N}\).
    \item If \(N \neq m\), then we must have \(m < N\).
          By \cref{i:5.1.14} we know that the finite rational sequence \((a_n)_{n = m}^{N - 1}\) is bounded by some \(M \in \Q_{\geq 0}\).
          So both \((a_n)_{n = m}^{N - 1}\) and \((a_n)_{n = N}^\infty\) are bounded by \(M + 1 + \abs{a_N}\).
          Thus, \((a_n)_{n = m}^\infty\) is bounded by \(M + 1 + \abs{a_N}\).
  \end{itemize}
  From all cases above we see that \((a_n)_{n = m}^\infty\) is bounded.
  Since \((a_n)_{n = m}^\infty\) was arbitrary, we conclude that every rational Cauchy sequences are bounded.
\end{proof}

\exercisesection

\begin{ex}\label{i:ex:5.1.1}
  Prove \cref{i:5.1.15}.
\end{ex}

\begin{proof}[\pf{i:ex:5.1.1}]
  See \cref{i:5.1.15}.
\end{proof}

\section{Equivalent Cauchy sequences}\label{sec:5.2}

\begin{defn}[\(\varepsilon\)-close sequences]\label{5.2.1}
  Let \((a_n)_{n = 0}^{\infty}\) and \((b_n)_{n = 0}^{\infty}\) be two sequences, and let \(\varepsilon > 0\).
  We say that the sequence \((a_n)_{n = 0}^{\infty}\) is \emph{\(\varepsilon\)-close} to \((b_n)_{n = 0}^{\infty}\) iff \(a_n\) is \(\varepsilon\)-close to \(b_n\) for each \(n \in \N\).
  In other words, the sequence \(a_0, a_1, a_2, \dots\) is \(\varepsilon\)-close to the sequence \(b_0, b_1, b_2, \dots\) iff \(\abs{a_n - b_n} \leq \varepsilon\) for all \(n = 0, 1, 2, \dots\).
\end{defn}

\setcounter{thm}{2}
\begin{defn}[\(Eventually \varepsilon\)-close sequences]\label{5.2.3}
  Let \((a_n)_{n = 0}^{\infty}\) and \((b_n)_{n = 0}^{\infty}\) be two sequences, and let \(\varepsilon > 0\).
  We say that the sequence \((a_n)_{n = 0}^{\infty}\) is \emph{eventually \(\varepsilon\)-close} to \((b_n)_{n = 0}^{\infty}\) iff there exists an \(N \geq 0\) such that the sequences \((a_n)_{n = N}^{\infty}\) and \((b_n)_{n = N}^{\infty}\) are \(\varepsilon\)-close.
  In other words, \(a_0, a_1, a_2, \dots\) is eventually \(\varepsilon\)-close to \(b_0, b_1, b_2, \dots\) iff there exists an \(N \geq 0\) such that \(\abs{a_n - b_n} \leq \varepsilon\) for all \(n \geq N\).
\end{defn}

\begin{rmk}\label{5.2.4}
  Again, the notations for \(\varepsilon\)-close sequences and eventually \(\varepsilon\)-close sequences are not standard in the literature, and we will not use them outside of this section.
\end{rmk}

\setcounter{thm}{5}
\begin{defn}[Equivalent sequences]\label{5.2.6}
  Two sequences \((a_n)_{n = 0}^{\infty}\) and \((b_n)_{n = 0}^{\infty}\) are \emph{equivalent} iff for each rational \(\varepsilon > 0\), the sequences \((a_n)_{n = 0}^{\infty}\) and \((b_n)_{n = 0}^{\infty}\) are eventually \(\varepsilon\)-close.
  In other words, \(a_0, a_1, a_2, \dots\) and \(b_0, b_1, b_2, \dots\) are equivalent iff for every rational \(\varepsilon > 0\), there exists an \(N \geq 0\) such that \(\abs{a_n - b_n} \leq \varepsilon\) for all \(n \geq N\).
\end{defn}

\begin{rmk}\label{5.2.7}
  As with \cref{5.1.8}, the quantity \(\varepsilon > 0\) is currently restricted to be a positive rational, rather than a positive real.
  However, we shall eventually see that it makes no difference whether \(\varepsilon\) ranges over the positive rationals or positive reals.
\end{rmk}

\begin{ac}\label{ac:5.2.1}
  Equivalence defined as \cref{5.2.6} is reflexive, symmetric and transitive.
\end{ac}

\begin{proof}
  Let \((a_n)_{n = m}^\infty\), \((b_n)_{n = m}^\infty\), \((c_n)_{n = m}^\infty\) be sequences of rationals.
  We have
  \begin{align*}
             & \forall \varepsilon \in \Q^+, \forall n \geq m, \abs{a_n - a_n} = \abs{0} = 0 \leq \varepsilon                 \\
    \implies & (a_n)_{n = m}^\infty = (a_n)_{n = m}^\infty                                                    &  & \by{5.2.6}
  \end{align*}
  and thus \cref{5.2.6} is reflexive.

  Now suppose that \((a_n)_{n = m}^\infty = (b_n)_{n = m}^\infty\).
  Then we have
  \begin{align*}
             & (a_n)_{n = m}^\infty = (b_n)_{n = m}^\infty                                       \\
    \implies & \forall \varepsilon \in \Q^+, \exists N \in \N \land N \geq m:                    \\
             & \forall n \geq N, \abs{a_n - b_n} \leq \varepsilon             &  & \by{5.2.6}    \\
    \implies & \forall \varepsilon \in \Q^+, \exists N \in \N \land N \geq m:                    \\
             & \forall n \geq N, \abs{b_n - a_n} \leq \varepsilon             &  & \by{4.3.3}[f] \\
    \implies & (b_n)_{n = m}^\infty = (a_n)_{n = m}^\infty                    &  & \by{5.2.6}
  \end{align*}
  and thus \cref{5.2.6} is symmetric.

  Finally suppose that \((a_n)_{n = m}^\infty = (b_n)_{n = m}^\infty\) and \((b_n)_{n = m}^\infty = (c_n)_{n = m}^\infty\).
  Then we have
  \begin{align*}
             & \big((a_n)_{n = m}^\infty = (b_n)_{n = m}^\infty\big) \land \big((b_n)_{n = m}^\infty = (c_n)_{n = m}^\infty\big)                        \\
    \implies & \forall \varepsilon \in \Q^+, \exists N_1, N_2 \in \N \land N_1, N_2 \geq m:                                        &  & \by{5.2.6}      \\
             & \begin{dcases}
                 \abs{a_n - b_n} \leq \dfrac{\varepsilon}{2} & \forall n \geq N_1 \\
                 \abs{b_n - c_n} \leq \dfrac{\varepsilon}{2} & \forall n \geq N_2 \\
               \end{dcases}                                                                         \\
    \implies & \forall \varepsilon \in \Q^+, \exists N = \max(N_1, N_2) \geq m:                                                    &  & \by{2.2.13}     \\
             & \forall n \geq N, (\abs{a_n - b_n} \leq \dfrac{\varepsilon}{2}) \land (\abs{b_n - c_n} \leq \dfrac{\varepsilon}{2})                      \\
    \implies & \forall \varepsilon \in \Q^+, \exists N = \max(N_1, N_2) \geq m:                                                                         \\
             & \forall n \geq N, \abs{a_n - b_n} + \abs{b_n - c_n} \leq \varepsilon                                                &  & \by{4.2.9}[c,d] \\
    \implies & \forall \varepsilon \in \Q^+, \exists N = \max(N_1, N_2) \geq m:                                                                         \\
             & \forall n \geq N, \abs{a_n - c_n} \leq \abs{a_n - b_n} + \abs{b_n - c_n} \leq \varepsilon                           &  & \by{4.3.3}[b]   \\
    \implies & (a_n)_{n = m}^\infty = (c_n)_{n = m}^\infty                                                                         &  & \by{5.2.6}
  \end{align*}
  and thus \cref{5.2.6} is transitive.
\end{proof}

\begin{prop}\label{5.2.8}
  Let \((a_n)_{n = 1}^{\infty}\) and \((b_n)_{n = 1}^{\infty}\) be the sequences \(a_n = 1 + 10^{-n}\) and \(b_n = 1 - 10^{-n}\).
  Then the sequences \(a_n, b_n\) are equivalent.
\end{prop}

\begin{proof}
  We need to prove that for every \(\varepsilon > 0\), the two sequences \((a_n)_{n = 1}^{\infty}\) and \((b_n)_{n = 1}^{\infty}\) are eventually \(\varepsilon\)-close to each other.
  So we fix an \(\varepsilon > 0\).
  We need to find an \(N > 0\) such that \((a_n)_{n = 1}^{\infty}\) and \((b_n)_{n = 1}^{\infty}\) are \(\varepsilon\)-close;
  in other words, we need to find an \(N > 0\) such that
  \[
    \abs{a_n - b_n} \leq \varepsilon \text{ for all } n \geq N.
  \]
  However, we have
  \[
    \abs{a_n - b_n} = \abs{(1 + 10^{-n}) - (1 - 10^{-n})} = 2 \times 10^{-n}.
  \]
  Since \(10^{-n}\) is a decreasing function of \(n\) (i.e., \(10^{-m} < 10^{-n}\) whenever \(m > n\);
  this is easily proven by induction), and \(n \geq N\), we have \(2 \times 10^{-n} \leq 2 \times 10^{-N}\).
  Thus we have
  \[
    \abs{a_n - b_n} \leq 2 \times 10^{-N} \text{ for all } n \geq N.
  \]
  Thus in order to obtain \(\abs{a_n - b_n} \leq \varepsilon\) for all \(n \geq N\), it will be sufficient to choose \(N\) so that \(2 \times 10^{-N} \leq \varepsilon\).
  This is easy to do using logarithms, but we have not yet developed logarithms yet, so we will use a cruder method.
  First, we observe \(10^N\) is always greater than \(N\) for any \(N \geq 1\) (see \cref{ex:4.3.5}).
  Thus \(10^{-N} \leq 1 / N\), and so \(2 \times 10^{-N} \leq 2 / N\).
  Thus to get \(2 \times 10^{-N} \leq \varepsilon\), it will suffice to choose \(N\) so that \(2 / N \leq \varepsilon\), or equivalently that \(N \geq 2 / \varepsilon\).
  But by \cref{4.4.1} we can always choose such an \(N\), and the claim follows.
\end{proof}

\begin{rmk}\label{5.2.9}
  \cref{5.2.8}, in decimal notation, asserts that
  \[
    1.0000 \dots = 0.9999 \dots.
  \]
\end{rmk}

\exercisesection

\begin{ex}\label{ex:5.2.1}
  Show that if \((a_n)_{n = 1}^{\infty}\) and \((b_n)_{n = 1}^{\infty}\) are equivalent sequences of rationals, then \((a_n)_{n = 1}^{\infty}\) is a Cauchy sequence iff \((b_n)_{n = 1}^{\infty}\) is a Cauchy sequence.
\end{ex}

\begin{proof}
  Let \(j, k \in \Z^+\).
  Since \((a_n)_{n = 1}^\infty = (b_n)_{n = 1}^\infty\), by \cref{5.2.6} we have
  \[
    \forall \varepsilon \in \Q^+, \exists N_1 \in \Z^+ : \forall n \geq N_1, \abs{a_n - b_n} \leq \dfrac{\varepsilon}{3}.
  \]
  Then we have
  \begin{align*}
             & (a_n)_{n = 1}^\infty \text{ is a Cauchy sequence}                                                                \\
    \implies & \exists N_2 \in \Z^+ : \forall j, k \geq N,                                                                      \\
             & \abs{a_j - a_k} \leq \dfrac{\varepsilon}{3}                                                 &  & \by{5.1.8}      \\
    \implies & \exists N = \max(N_1, N_2) \in \Z^+ : \forall j, k \geq N,                                  &  & \by{2.2.13}     \\
             & \abs{a_j - a_k} \leq \dfrac{\varepsilon}{3}                                                                      \\
    \implies & \exists N = \max(N_1, N_2) \in \Z^+ : \forall j, k \geq N,                                                       \\
             & \abs{a_j - a_k} + \abs{a_j - b_j} + \abs{a_k - b_k}                                                              \\
             & \leq \dfrac{\varepsilon}{3} + \dfrac{\varepsilon}{3} + \dfrac{\varepsilon}{3} = \varepsilon &  & \by{4.2.9}[c,d] \\
    \implies & \exists N = \max(N_1, N_2) \in \Z^+ : \forall j, k \geq N,                                                       \\
             & \abs{a_j - a_k} + \abs{b_j - a_j} + \abs{a_k - b_k} \leq \varepsilon                        &  & \by{4.3.3}[f]   \\
    \implies & \exists N = \max(N_1, N_2) \in \Z^+ : \forall j, k \geq N,                                                       \\
             & \abs{b_j - b_k} = \abs{a_j - a_k + b_j - a_j + a_k - b_k}                                                        \\
             & \leq \abs{a_j - a_k} + \abs{b_j - a_j} + \abs{a_k - b_k} \leq \varepsilon                   &  & \by{4.3.3}[b]   \\
    \implies & (b_n)_{n = 1}^\infty \text{ is a Cauchy sequence}.                                          &  & \by{5.1.8}
  \end{align*}
  Using similar arguments we can show that \((b_n)_{n = 1}^\infty\) is a Cauchy sequence implies \((a_n)_{n = 1}^\infty\) is a Cauchy sequence.
  Thus we conclude that \((a_n)_{n = 1}^\infty\) is a Cauchy sequence iff \((b_n)_{n = 1}^\infty\) is a Cauchy sequence.
\end{proof}

\begin{ex}\label{ex:5.2.2}
  Let \(\varepsilon > 0\).
  Show that if \((a_n)_{n = 1}^{\infty}\) and \((b_n)_{n = 1}^{\infty}\) are eventually \(\varepsilon\)-close, then \((a_n)_{n = 1}^{\infty}\) is bounded iff \((b_n)_{n = 1}^{\infty}\) is bounded.
\end{ex}

\begin{proof}
  Since \((a_n)_{n = 1}^{\infty}\) and \((b_n)_{n = 1}^{\infty}\) are eventually \(\varepsilon\)-close, by \cref{5.2.3} we have
  \[
    \exists N \in \Z^+ : \forall n \geq N, \abs{a_n - b_n} \leq \varepsilon.
  \]
  Then we have
  \begin{align*}
             & (a_n)_{n = 1}^\infty \text{ is bounded}                                                                          \\
    \implies & \exists M \in \Q \setminus \Q^- : \forall n \geq 1, \abs{a_n} \leq M                        &  & \by{5.1.12}     \\
    \implies & \exists M \in \Q \setminus \Q^- : \forall n \geq \max(1, N), \abs{a_n} \leq M               &  & \by{2.2.13}     \\
    \implies & \exists M \in \Q \setminus \Q^- : \forall n \geq \max(1, N), \abs{-a_n} \leq M              &  & \by{4.3.3}[d]   \\
    \implies & \exists M \in \Q \setminus \Q^- : \forall n \geq \max(1, N),                                                     \\
             & \abs{-a_n} + \abs{a_n - b_n} \leq M + \varepsilon                                           &  & \by{4.2.9}[c,d] \\
    \implies & \exists M \in \Q \setminus \Q^- : \forall n \geq \max(1, N),                                                     \\
             & \abs{-a_n + a_n - b_n} \leq \abs{-a_n} + \abs{a_n - b_n} \leq M + \varepsilon               &  & \by{4.3.3}[b]   \\
    \implies & \exists M \in \Q \setminus \Q^- : \forall n \geq \max(1, N),                                                     \\
             & \abs{-b_n} \leq M + \varepsilon                                                             &  & \by{4.2.4}      \\
    \implies & \exists M \in \Q \setminus \Q^- : \forall n \geq \max(1, N), \abs{b_n} \leq M + \varepsilon &  & \by{4.3.3}[d]   \\
    \implies & \exists M \in \Q \setminus \Q^- : \forall n \geq 1,                                                              \\
             & \abs{b_n} \leq M + \varepsilon + \max_{1 \leq n \leq N - 1}\set{\abs{b_n}}                  &  & \by{5.1.14}     \\
    \implies & (a_n)_{n = 1}^\infty \text{ is bounded}.                                                    &  & \by{5.1.12}
  \end{align*}
  Using similar arguments we can show that \((b_n)_{n = 1}^\infty\) is bounded implies \((a_n)_{n = 1}^\infty\) is bounded.
  Thus we conclude that \((a_n)_{n = 1}^\infty\) is bounded iff \((b_n)_{n = 1}^\infty\) is bounded.
\end{proof}

\section{The construction of the real numbers}\label{sec:5.3}

\begin{defn}[Real numbers]\label{5.3.1}
  A \emph{real number} is defined to be an object of the form \(\text{LIM}_{n \to \infty} a_n\), where \((a_n)_{n = 1}^{\infty}\) is a Cauchy sequence of rational numbers.
  Two real numbers \(\text{LIM}_{n \to \infty} a_n\) an and \(\text{LIM}_{n \to \infty} b_n\) are said to be equal iff \((a_n)_{n = 1}^{\infty}\) and \((b_n)_{n = 1}^{\infty}\) are equivalent Cauchy sequences.
  The set of all real numbers is denoted \(\R\).
\end{defn}

\begin{note}
  We will refer to \(\text{LIM}_{n \to \infty} a_n\) as the \emph{formal limit} of the sequence \((a_n)_{n = 1}^{\infty}\).
  Later on we will define a genuine notion of limit, and show that the formal limit of a Cauchy sequence is the same as the limit of that sequence;
  after that, we will not need formal limits ever again.
\end{note}

\setcounter{thm}{2}
\begin{prop}[Formal limits are well-defined]\label{5.3.3}
  Let
  \[
    x = \text{LIM}_{n \to \infty} a_n, y = \text{LIM}_{n \to \infty} b_n, z = \text{LIM}_{n \to \infty} c_n
  \]
  be real numbers.
  Then, with the above definition of equality for real numbers, we have \(x = x\).
  Also, if \(x = y\), then \(y = x\).
  Finally, if \(x = y\) and \(y = z\), then \(x = z\).
\end{prop}

\begin{proof}
  By \cref{ac:5.2.1} we know that the equality of sequence are well-defined.
  Since every Cauchy sequence is a sequence, we know that the Formal limits are well-defined.
\end{proof}

\begin{note}
  Because of \cref{5.3.3}, we know that our definition of equality between two real numbers is legitimate.
  Of course, when we define other operations on the reals, we have to check that they obey the axiom of substitution:
  two real number inputs which are equal should give equal outputs when applied to any operation on the real numbers.
\end{note}

\begin{defn}[Addition of reals]\label{5.3.4}
  Let \(x = \text{LIM}_{n \to \infty} a_n\) and \(y = \text{LIM}_{n \to \infty} b_n\) be real numbers.
  Then we define the sum \(x + y\) to be \(x + y \coloneqq \text{LIM}_{n \to \infty} (a_n + b_n)\).
\end{defn}

\setcounter{thm}{5}
\begin{lem}[Sum of Cauchy sequences is Cauchy]\label{5.3.6}
  Let \(x = \text{LIM}_{n \to \infty} a_n\) and \(y = \text{LIM}_{n \to \infty} b_n\) be real numbers.
  Then \(x + y\) is also a real number
  (i.e., \((a_n + b_n)_{n = 1}^{\infty}\) is a Cauchy sequence of rationals).
\end{lem}

\begin{proof}
  We need to show that for every \(\varepsilon > 0\), the sequence \((a_n + b_n)_{n = 1}^{\infty}\) is eventually \(\varepsilon\)-steady.
  Now from hypothesis we know that \((a_n)_{n = 1}^{\infty}\) is eventually \(\varepsilon\)-steady, and \((b_n)_{n = 1}^{\infty}\) is eventually \(\varepsilon\)-steady, but it turns out that this is not quite enough
  (this can be used to imply that \((a_n + b_n)_{n = 1}^{\infty}\) is eventually \(2\varepsilon\)-steady, but that's not what we want).
  So we need to do a little trick, which is to play with the value of \(\varepsilon\).

  We know that \((a_n)_{n = 1}^{\infty}\) is eventually \(\delta\)-steady for every value of \(\delta\).
  This implies not only that \((a_n)_{n = 1}^{\infty}\) is eventually \(\varepsilon\)-steady, but it is also eventually \(\varepsilon / 2\)-steady.
  Similarly, the sequence \((b_n)_{n = 1}^{\infty}\) is also eventually \(\varepsilon / 2\)-steady.
  This will turn out to be enough to conclude that \((a_n + b_n)_{n = 1}^{\infty}\) is eventually \(\varepsilon\)-steady.

  Since \((a_n)_{n = 1}^{\infty}\) is eventually \(\varepsilon / 2\)-steady, we know that there exists an \(N \geq 1\) such that \((a_n)_{n = N}^{\infty}\) is \(\varepsilon / 2\)-steady, i.e., \(a_n\) and \(a_m\) are \(\varepsilon / 2\)-close for every \(n, m \geq N\).
  Similarly there exists an \(M \geq 1\) such that \((b_n)_{n = M}^{\infty}\) is \(\varepsilon / 2\)-steady, i.e., \(b_n\) and \(b_m\) are \(\varepsilon / 2\)-close for every \(n, m \geq M\).

  Let \(\max(N, M)\) be the larger of \(N\) and \(M\)
  (we know from \cref{2.2.13} that one has to be greater than or equal to the other).
  If \(n, m \geq \max(N, M)\), then we know that \(a_n\) and \(a_m\) are \(\varepsilon / 2\)-close, and \(b_n\) and \(b_m\) are \(\varepsilon / 2\)-close, and so by \cref{4.3.7} we see that \(a_n + b_n\) and \(a_m + b_m\) are \(\varepsilon\)-close for every \(n, m \geq \max(N, M)\).
  This implies that the sequence \((a_n + b_n)_{n = 1}^{\infty}\) is eventually \(\varepsilon\)-steady, as desired.
\end{proof}

\begin{lem}[Sums of equivalent Cauchy sequences are equivalent]\label{5.3.7}
  Let
  \[
    x = \text{LIM}_{n \to \infty} a_n, y = \text{LIM}_{n \to \infty} b_n, x' = \text{LIM}_{n \to \infty} a'_n
  \]
  be real numbers.
  Suppose that \(x = x'\).
  Then we have \(x + y = x' + y\).
\end{lem}

\begin{proof}
  Since \(x\) and \(x'\) are equal, we know that the Cauchy sequences \((a_n)_{n = 1}^{\infty}\) and \((a'_n)_{n = 1}^{\infty}\) are equivalent, so in other words they are eventually \(\varepsilon\)-close for each \(\varepsilon > 0\).
  We need to show that the sequences \((a_n + b_n)_{n = 1}^{\infty}\) and \((a'_n + b_n)_{n = 1}^{\infty}\) are eventually \(\varepsilon\)-close for each \(\varepsilon > 0\).
  But we already know that there is an \(N \geq 1\) such that \((a_n)_{n = N}^{\infty}\) and \((a'_n)_{n = N}^{\infty}\) are \(\varepsilon\)-close, i.e., that \(a_n\) and \(a'_n\) are \(\varepsilon\)-close for each \(n \geq N\).
  Since \(b_n\) is of course \(0\)-close to \(b_n\) (where we extend the notion of \(\varepsilon\)-closeness to include \(\varepsilon = 0\) in the obvious fashion), we thus see from \cref{4.3.7} (extended in the obvious manner to the \(\delta = 0\) case) that \(a_n + b_n\) and \(a'_n + b_n\) are \(\varepsilon\)-close for each \(n \geq N\).
  This implies that \((a_n + b_n)_{n = 1}^{\infty}\) and \((a'_n + b_n)_{n = 1}^{\infty}\) are eventually \(\varepsilon\)-close for each \(\varepsilon > 0\), and we are done.
\end{proof}

\begin{rmk}\label{5.3.8}
  \cref{5.3.7} verifies the axiom of substitution for the ``x'' variable in \(x + y\), but one can similarly prove the axiom of substitution for the ``y'' variable.
  (A quick way is to observe from the definition of \(x + y\) that we certainly have \(x + y = y + x\), since \(a_n + b_n = b_n + a_n\).)
\end{rmk}

\begin{defn}[Multiplication of reals]\label{5.3.9}
  Let \(x = \text{LIM}_{n \to \infty} a_n\) and \(y = \text{LIM}_{n \to \infty} b_n\) be real numbers.
  Then we define the product \(xy\) to be \(xy \coloneqq \text{LIM}_{n \to \infty} a_n b_n\).
\end{defn}

\begin{prop}[Multiplication is well defined]\label{5.3.10}
  Let
  \[
    x = \text{LIM}_{n \to \infty} a_n, y = \text{LIM}_{n \to \infty} b_n, x' = \text{LIM}_{n \to \infty} a'_n
  \]
  be real numbers.
  Then \(xy\) is also a real number.
  Furthermore, if \(x = x'\), then \(xy = x'y\).
\end{prop}

\begin{proof}
  We first show that \(x, y \in \R \implies xy \in \R\).
  Let \(\varepsilon \in \Q^+\) and let \(j, k \in \Z^+\).
  Since \((a_n)_{n = 1}^\infty\) and \((b_n)_{n = 1}^\infty\) are Cauchy sequence, by \cref{5.1.15} we know that \((a_n)_{n = 1}^\infty\) and \((b_n)_{n = 1}^\infty\) are bounded by some \(M_1, M_2 \in \Q \setminus \Q^-\).
  Then by \cref{4.2.9}(a) we know that \((a_n)_{n = 1}^\infty\) and \((b_n)_{n = 1}^\infty\) are bounded by \(M = \max(M_1, M_2) + 1\).
  Since \(x = \text{LIM}_{n \to \infty} a_n\) and \(y = \text{LIM}_{n \to \infty} b_n\), by \cref{5.1.8} we have
  \begin{align*}
     & \exists N_1 \in \Z^+ : \forall j, k \geq N_1, \abs{a_j - a_k} \leq \varepsilon; \\
     & \exists N_2 \in \Z^+ : \forall j, k \geq N_2, \abs{b_j - b_k} \leq \varepsilon.
  \end{align*}
  In particular, we have
  \begin{align*}
     & \exists N_1 \in \Z^+ : \forall j, k \geq N_1, \abs{a_j - a_k} \leq \dfrac{\varepsilon}{2M}; \\
     & \exists N_2 \in \Z^+ : \forall j, k \geq N_2, \abs{b_j - b_k} \leq \dfrac{\varepsilon}{2M}.
  \end{align*}
  Let \(N = \max(N_1, N_2)\).
  Such \(N\) is well-defined since \cref{2.2.13}.
  Then \(\forall j, k \geq N\), we have
  \begin{align*}
    \abs{a_j b_j - a_k b_k} & = \abs{a_j b_j - a_j b_k + a_j b_k - a_k b_k}              &  & \by{4.2.4}                     \\
                            & \leq \abs{a_j b_j - a_j b_k} + \abs{a_j b_k - a_k b_k}     &  & \text{(by \cref{4.3.3}(b))}    \\
                            & = \abs{a_j} \abs{b_j - b_k} + \abs{b_k} \abs{a_j - a_k}    &  & \text{(by \cref{4.3.3}(d))}    \\
                            & \leq M \dfrac{\varepsilon}{2M} + M \dfrac{\varepsilon}{2M} &  & \text{(by \cref{4.2.9}(c)(e))} \\
                            & = \varepsilon.
  \end{align*}
  Thus by \cref{5.1.8} \((a_n b_n)_{n = 1}^\infty\) is a Cauchy sequence and by \cref{5.3.1} \(xy \in \R\).

  Now we show that \(x = x' \implies xy = xy'\).
  Since \((a_n)_{n = 1}^\infty\), \((a_n')_{n = 1}^\infty\), \((b_n)_{n = 1}^\infty\) are Cauchy sequence, by \cref{5.1.15} we know that \((a_n)_{n = 1}^\infty\) and \((a_n')_{n = 1}^\infty\) are bounded by some \(M_1, M_2 M_3 \in \Q \setminus \Q^-\).
  Then by \cref{4.2.9}(a) we know that \((a_n)_{n = 1}^\infty\), \((a_n')_{n = 1}^\infty\), \((b_n)_{n = 1}^\infty\) are bounded by \(M = \max(M_1, M_2, M_3) + 1\).
  Since \(x = x'\), by \cref{5.2.6} we know that
  \[
    \forall \varepsilon \in \Q^+, \exists N \in \Z^+ : \forall n \geq N, \abs{a_n - a_n'} \leq \varepsilon.
  \]
  In particular, we have
  \[
    \forall \varepsilon \in \Q^+, \exists N \in \Z^+ : \forall n \geq N, \abs{a_n - a_n'} \leq \dfrac{\varepsilon}{M}.
  \]
  Then we have
  \begin{align*}
             & \abs{a_n - a_n'} \leq \dfrac{\varepsilon}{M}                                                         \\
    \implies & \abs{b_n} \abs{a_n - a_n'} \leq \abs{b_n} \dfrac{\varepsilon}{M} &  & \text{(by \cref{4.2.9}(c)(e))} \\
    \implies & \abs{b_n} \abs{a_n - a_n'} \leq M \dfrac{\varepsilon}{M}         &  & \text{(by \cref{4.2.9}(c)(e))} \\
    \implies & \abs{b_n} \abs{a_n - a_n'} \leq \varepsilon                                                          \\
    \implies & \abs{b_n (a_n - a_n')} \leq \varepsilon                          &  & \text{(by \cref{4.3.3}(d))}    \\
    \implies & \abs{a_n b_n - a_n' b_n} \leq \varepsilon                        &  & \text{(by \cref{4.2.4}}
  \end{align*}
  and thus by \cref{5.2.6} we have \(xy = x'y\).
\end{proof}

\begin{note}
  Of course we can prove a similar substitution rule when \(y\) is replaced by a real number \(y'\) which is equal to \(y\).
\end{note}

\begin{note}
  At this point we embed the rationals back into the reals, by equating every rational number \(q\) with the real number \(\text{LIM}_{n \to \infty} q\).
  This embedding is consistent with our definitions of addition and multiplication, since for any rational numbers \(a, b\) we have
  \begin{align*}
    (\text{LIM}_{n \to \infty} a) + (\text{LIM}_{n \to \infty} b)      & = \text{LIM}_{n \to \infty} (a + b) \text{ and} \\
    (\text{LIM}_{n \to \infty} a) \times (\text{LIM}_{n \to \infty} b) & = \text{LIM}_{n \to \infty} (ab);
  \end{align*}
  this means that when one wants to add or multiply two rational numbers \(a, b\) it does not matter whether one thinks of these numbers as rationals or as the real numbers \(\text{LIM}_{n \to \infty} a, \text{LIM}_{n \to \infty} b\).
  Also, this identification of rational numbers and real numbers is consistent with our definitions of equality.
\end{note}

\begin{note}
  We can now easily define negation \(-x\) for real numbers \(x\) by the formula
  \[
    -x \coloneqq (-1) \times x,
  \]
  since \(-1\) is a rational number and is hence real.
  Note that this is clearly consistent with our negation for rational numbers since we have \(-q = (-1) \times q\) for all rational numbers \(q\).
  Also, from our definitions it is clear that
  \[
    -\text{LIM}_{n \to \infty} a_n = \text{LIM}_{n \to \infty} (-a_n).
  \]
  Once we have addition and negation, we can define subtraction as usual by
  \[
    x - y \coloneqq x + (-y),
  \]
  this implies
  \[
    \text{LIM}_{n \to \infty} a_n - \text{LIM}_{n \to \infty} b_n = \text{LIM}_{n \to \infty} (a_n - b_n).
  \]
\end{note}

\begin{prop}\label{5.3.11}
  All the laws of algebra from \cref{4.1.6} hold not only for the integers, but for the reals as well.
\end{prop}

\begin{proof}
  We illustrate this with one such rule: \(x(y + z) = xy + xz\).
  Let \(x = \text{LIM}_{n \to \infty} a_n\), \(y = \text{LIM}_{n \to \infty} b_n\), and \(z = \text{LIM}_{n \to \infty} c_n\) be real numbers.
  Then by definition, \(xy = \text{LIM}_{n \to \infty} a_n b_n\) and \(xz = \text{LIM}_{n \to \infty} a_n c_n\), and so \(xy + xz = \text{LIM}_{n \to \infty} (a_n b_n + a_n c_n)\).
  A similar line of reasoning shows that \(x(y + z) = \text{LIM}_{n \to \infty} a_n (b_n + c_n)\).
  But we already know that \(a_n (b_n + c_n)\) is equal to \(a_n b_n + a_n c_n\) for the rational numbers \(a_n, b_n, c_n\), and the claim follows.
  The other laws of algebra are proven similarly.
\end{proof}

\begin{defn}[Sequences bounded away from zero]\label{5.3.12}
  A sequence \(\text{LIM}_{n \to \infty} a_n\) of rational numbers is said to be \emph{bounded away from zero} iff there exists a rational number \(c > 0\) such that \(\abs{a_n} \geq c\) for all \(n \geq 1\).
\end{defn}

\setcounter{thm}{13}
\begin{lem}\label{5.3.14}
  Let \(x\) be a non-zero real number.
  Then \(x = \text{LIM}_{n \to \infty} a_n\) for some Cauchy sequence \((a_n)_{n = 1}^{\infty}\) which is bounded away from zero.
\end{lem}

\begin{proof}
  Since \(x\) is real, we know that \(x = \text{LIM}_{n \to \infty} b_n\) for some Cauchy sequence \((b_n)_{n = 1}^{\infty}\).
  But we are not yet done, because we do not know that \(b_n\) is bounded away from zero.
  On the other hand, we are given that \(x \neq 0 = \text{LIM}_{n \to \infty} 0\), which means that the sequence \((b_n)_{n = 1}^{\infty}\) is not equivalent to \((0)_{n = 1}^{\infty}\).
  Thus the sequence \((b_n)_{n = 1}^{\infty}\) cannot be eventually \(\varepsilon\)-close to \((0)_{n = 1}^{\infty}\) for every \(\varepsilon > 0\).
  Therefore we can find an \(\varepsilon > 0\) such that \((b_n)_{n = 1}^{\infty}\) is not eventually \(\varepsilon\)-close to \((0)_{n = 1}^{\infty}\).

  Let us fix this \(\varepsilon\).
  We know that \((b_n)_{n = 1}^{\infty}\) is a Cauchy sequence, so it is eventually \(\varepsilon\)-steady.
  Moreover, it is eventually \(\varepsilon / 2\)-steady, since \(\varepsilon / 2 > 0\).
  Thus there is an \(N \geq 1\) such that \(\abs{b_n - b_m} \leq \varepsilon / 2\) for all \(n, m \geq N\).

  On the other hand, we cannot have \(\abs{b_n} \leq \varepsilon\) for all \(n \geq N\), since this would imply that \((b_n)_{n = 1}^{\infty}\) is eventually \(\varepsilon\)-close to \((0)_{n = 1}^{\infty}\).
  Thus there must be some \(n_0 \geq N\) for which \(\abs{b_{n_0}} > \varepsilon\).
  Since we already know that \(\abs{b_{n_0} - b_n} \leq \varepsilon / 2\) for all \(n \geq N\), we have
  \begin{align*}
             & \abs{b_{n_0}} - \abs{b_{n_0} - b_n} \geq \varepsilon - \varepsilon / 2 = \varepsilon / 2 &  & \by{4.2.9}    \\
    \implies & \abs{b_{n_0}} - \abs{b_n - b_{n_0}} \geq \varepsilon / 2                                 &  & \by{4.3.3}    \\
    \implies & \abs{b_{n_0} + (b_n - b_{n_0})} \geq \varepsilon / 2                                     &  & \by{ac:4.3.1} \\
    \implies & \abs{b_n} \geq \varepsilon / 2.                                                          &  & \by{4.2.4}    \\
  \end{align*}
  Thus conclude from above that \(\abs{b_n} \geq \varepsilon / 2\) for all \(n \geq N\).

  This almost proves that \((b_n)_{n = 1}^{\infty}\) is bounded away from zero.
  Actually, what it does is show that \((b_n)_{n = 1}^{\infty}\) is \emph{eventually} bounded away from zero.
  But this is easily fixed, by defining a new sequence \(a_n\), by setting \(a_n \coloneqq \varepsilon / 2\) if \(n < N\) and \(a_n \coloneqq b_n\) if \(n \geq N\).
  Since \(b_n\) is a Cauchy sequence, it is not hard to verify that \(a_n\) is also a Cauchy sequence which is equivalent to \(b_n\) (because the two sequences are eventually the same), and so \(x = \text{LIM}_{n \to \infty} a_n\).
  And since \(\abs{b_n} \geq \varepsilon / 2\) for all \(n \geq N\), we know that \(\abs{a_n} \geq \varepsilon / 2\) for all \(n \geq 1\) (splitting into the two cases \(n \geq N\) and \(n < N\) separately).
  Thus we have a Cauchy sequence which is bounded away from zero (by \(\varepsilon / 2\) instead of \(\varepsilon\), but that's still OK since \(\varepsilon / 2 > 0\)), and which has \(x\) as a formal limit, and so we are done.
\end{proof}

\begin{lem}\label{5.3.15}
  Suppose that \((a_n)_{n = 1}^{\infty}\) is a Cauchy sequence which is bounded away from zero.
  Then the sequence \((a_n^{-1})_{n = 1}^{\infty}\) is also a Cauchy sequence.
\end{lem}

\begin{proof}
  Since \((a_n)_{n = 1}^{\infty}\) is bounded away from zero, we know that there is a \(c > 0\) such that \(\abs{a_n} \geq c\) for all \(n \geq 1\).
  Now we need to show that \((a_n^{-1})_{n = 1}^{\infty}\) is eventually \(\varepsilon\)-steady for each \(\varepsilon > 0\).
  Thus let us fix an \(\varepsilon > 0\);
  our task is now to find an \(N \geq 1\) such that \(\abs{a_n^{-1} - a_m^{-1}} \leq \varepsilon\) for all \(n, m \geq N\).
  But
  \[
    \abs{a_n^{-1} - a_m^{-1}} = \abs{\dfrac{a_m - a_n}{a_m a_n}} \leq \dfrac{\abs{a_m - a_n}}{c^2}
  \]
  (since \(\abs{a_m}, \abs{a_n} \geq c\)), and so to make \(\abs{a_n^{-1} - a_m^{-1}}\) less than or equal to \(\varepsilon\), it will suffice to make \(\abs{a_m - a_n}\) less than or equal to \(c^2 \varepsilon\).
  But since \((a_n)_{n = 1}^{\infty}\) is a Cauchy sequence, and \(c^2 \varepsilon > 0\), we can certainly find an \(N\) such that the sequence \((a_n)_{n = N}^{\infty}\) is \(c^2 \varepsilon\)-steady, i.e., \(\abs{a_m - a_n} \leq c^2 \varepsilon\) for all \(n, m \geq N\).
  By what we have said above, this shows that \(\abs{a_n^{-1} - a_m^{-1}} \leq \varepsilon\) for all \(m, n \geq N\), and hence the sequence \((a_n^{-1})_{n = 1}^{\infty}\) is eventually \(\varepsilon\)-steady.
  Since we have proven this for every \(\varepsilon\), we have that \((a_n^{-1})_{n = 1}^{\infty}\) is a Cauchy sequence, as desired.
\end{proof}

\begin{defn}[Reciprocals of real numbers]\label{5.3.16}
  Let \(x\) be a non-zero real number.
  Let \((a_n)_{n = 1}^{\infty}\) be a Cauchy sequence bounded away from zero such that \(x = \text{LIM}_{n \to \infty} a_n\) (such a sequence exists by \cref{5.3.14}).
  Then we define the reciprocal \(x^{-1}\) by the formula \(x^{-1} \coloneqq \text{LIM}_{n \to \infty} a_n^{-1}\).
  (From \cref{5.3.15} we know that \(x^{-1}\) is a real number.)
\end{defn}

\begin{lem}[Reciprocation is well defined]\label{5.3.17}
  Let \((a_n)_{n = 1}^{\infty}\) and \((b_n)_{n = 1}^{\infty}\) be two Cauchy sequences bounded away from zero such that \(\text{LIM}_{n \to \infty} a_n = \text{LIM}_{n \to \infty} b_n\) (i.e., the two sequences are equivalent).
  Then \(\text{LIM}_{n \to \infty} a_n^{-1} = \text{LIM}_{n \to \infty} b_n^{-1}\).
\end{lem}

\begin{proof}
  Consider the following product \(P\) of three real numbers:
  \[
    P \coloneqq (\text{LIM}_{n \to \infty} a_n^{-1}) \times (\text{LIM}_{n \to \infty} a_n) \times (\text{LIM}_{n \to \infty} b_n^{-1}).
  \]
  If we multiply this out, we obtain
  \[
    P = \text{LIM}_{n \to \infty} a_n^{-1} a_n b_n^{-1} = \text{LIM}_{n \to \infty} b_n^{-1}.
  \]
  On the other hand, since \(\text{LIM}_{n \to \infty} a_n = \text{LIM}_{n \to \infty} b_n\), we can write \(P\) in another way as
  \[
    P = (\text{LIM}_{n \to \infty} a_n^{-1}) \times (\text{LIM}_{n \to \infty} b_n) \times (\text{LIM}_{n \to \infty} b_n^{-1}).
  \]
  (cf. \cref{5.3.10}).
  Multiplying things out again, we get
  \[
    P = \text{LIM}_{n \to \infty} a_n^{-1} b_n b_n^{-1} = \text{LIM}_{n \to \infty} a_n^{-1}.
  \]
  Comparing our different formulae for \(P\) we see that \(\text{LIM}_{n \to \infty} a_n^{-1} = \text{LIM}_{n \to \infty} b_n^{-1}\), as desired.
\end{proof}

\begin{note}
  It is clear from the definition that \(xx^{-1} = x^{-1}x = 1\);
  thus all the field axioms (\cref{4.2.4}) apply to the reals as well as to the rationals.
  We of course cannot give \(0\) a reciprocal, since \(0\) multiplied by anything gives \(0\), not \(1\).
\end{note}

\begin{note}
  if \(q\) is a non-zero rational, and hence equal to the real number \(\text{LIM}_{n \to \infty} q\), then the reciprocal of \(\text{LIM}_{n \to \infty} q\) is \(\text{LIM}_{n \to \infty} q^{-1} = q^{-1}\);
  thus the operation of reciprocal on real numbers is consistent with the operation of reciprocal on rational numbers.
\end{note}

\begin{note}
  Once one has reciprocal, one can define division \(x / y\) of two real numbers \(x, y\), provided \(y\) is non-zero, by the formula
  \[
    x / y \coloneqq x \times y^{-1},
  \]
  just as we did with the rationals.
  In particular, we have the \emph{cancellation law}:
  if \(x, y, z\) are real numbers such that \(xz = yz\), and \(z\) is non-zero, then by dividing by \(z\) we conclude that \(x = y\).
  This cancellation law does not work when \(z\) is zero.
\end{note}

\exercisesection

\begin{ex}\label{ex:5.3.1}
  Prove \cref{5.3.3}.
\end{ex}

\begin{proof}
  See \cref{5.3.3}.
\end{proof}

\begin{ex}\label{ex:5.3.2}
  Prove \cref{5.3.10}.
\end{ex}

\begin{proof}
  See \cref{5.3.10}.
\end{proof}

\begin{ex}\label{ex:5.3.3}
  Let \(a, b\) be rational numbers.
  Show that \(a = b\) if and only if \(\text{LIM}_{n \to \infty} a = \text{LIM}_{n \to \infty} b\) (i.e., the Cauchy sequences \(a, a, a, a, \dots\) and \(b, b, b, b \dots\) equivalent if and only if \(a = b\)).
  This allows us to embed the rational numbers inside the real numbers in a well-defined manner.
\end{ex}

\begin{proof}
  Let \((a)_{n = 1}^{\infty}\) and \((b)_{n = 1}^{\infty}\) be two sequences where \(a, b \in \Q\).
  By \cref{5.2.6} \((a)_{n = 1}^\infty\) and \((b)_{n = 1}^\infty\) is Cauchy sequence since
  \[
    \forall \varepsilon \in \Q^+, \forall n \geq 1, \abs{a - a} = \abs{b - b} = 0 \leq \varepsilon.
  \]
  Then we have
  \begin{align*}
         & a = b                                                                                                         \\
    \iff & \forall \varepsilon \in \Q^+, \forall n \geq 1, \abs{a - b} \leq \varepsilon &  & \text{(by \cref{4.3.7}(a))} \\
    \iff & (a)_{n = 1}^\infty = (b)_{n = 1}^\infty                                      &  & \by{5.2.6}                  \\
    \iff & \text{LIM}_{n \to \infty} a = \text{LIM}_{n \to \infty} b.                   &  & \by{5.3.1}
  \end{align*}
\end{proof}

\begin{ex}\label{ex:5.3.4}
  Let \((a_n)_{n = 0}^{\infty}\) be a sequence of rational numbers which is bounded.
  Let \((b_n)_{n = 0}^{\infty}\) be another sequence of rational numbers which is equivalent to \((a_n)_{n = 0}^{\infty}\).
  Show that \((b_n)_{n = 0}^{\infty}\) is also bounded.
\end{ex}

\begin{proof}
  Since \((a_n)_{n = 0}^{\infty} = (b_n)_{n = 0}^{\infty}\), by \cref{5.2.6} we know that \((a_n)_{n = 0}^{\infty}\) and \((b_n)_{n = 0}^{\infty}\) are eventually \(\varepsilon\)-close for every \(\varepsilon \in \Q^+\).
  Thus by \cref{ex:5.2.2} \((a_n)_{n = 0}^{\infty}\) is bounded iff \((b_n)_{n = 0}^{\infty}\) is bounded.
\end{proof}

\begin{ex}\label{ex:5.3.5}
  Show that \(\text{LIM}_{n \to \infty} 1 / n = 0\).
\end{ex}

\begin{proof}
  By \cref{5.1.11} we know that the sequence \((\dfrac{1}{n})_{n = 1}^{\infty}\) is a Cauchy sequence.
  By \cref{4.4.1}, \(\forall \varepsilon \in \Q^+\), \(\exists N \in \Z^+\) such that
  \[
    \dfrac{1}{\varepsilon} < N \implies \dfrac{1}{N} < \varepsilon.
  \]
  Then we have
  \begin{align*}
    \forall n \geq N, \abs{\dfrac{1}{n} - 0} & = \dfrac{1}{n}    &  & \by{4.3.1}                   \\
                                             & \leq \dfrac{1}{N} &  & \text{(by \cref{4.3.12}(b))} \\
                                             & < \varepsilon
  \end{align*}
  and thus by \cref{5.2.6} \((\dfrac{1}{n})_{n = 1}^\infty = (0)_{n = 1}^\infty\).
  By \cref{5.3.1} and \cref{ex:5.3.3} we have
  \[
    \text{LIM}_{n \to \infty} \dfrac{1}{n} = \text{LIM}_{n \to \infty} 0 = 0.
  \]
\end{proof}
\section{Ordering the reals}\label{sec:5.4}

\begin{defn}\label{5.4.1}
  Let \((a_n)_{n = 1}^{\infty}\) be a sequence of rationals.
  We say that this sequence is \emph{positively bounded away from zero} iff we have a positive rational \(c > 0\) such that \(a_n \geq c\) for all \(n \geq 1\) (in particular, the sequence is entirely positive).
  The sequence is \emph{negatively bounded away from zero} iff we have a negative rational \(-c < 0\) such that \(a_n \leq -c\) for all \(n \geq 1\) (in particular, the sequence is entirely negative).
\end{defn}

\begin{note}
  It is clear that any sequence which is positively or negatively bounded away from zero, is bounded away from zero.
  Also, a sequence cannot be both positively bounded away from zero and negatively bounded away from zero at the same time.
\end{note}

\setcounter{thm}{2}
\begin{defn}\label{5.4.3}
  A real number \(x\) is said to be \emph{positive} iff it can be written as \(x = \text{LIM}_{n \to \infty} a_n\) for some Cauchy sequence \((a_n)_{n = 1}^{\infty}\) which is positively bounded away from zero.
  \(x\) is said to be \emph{negative} iff it can be written as \(x = \text{LIM}_{n \to \infty} a_n\) for some sequence \((a_n)_{n = 1}^{\infty}\) which is negatively bounded away from zero.
\end{defn}

\begin{prop}[Basic properties of positive reals]\label{5.4.4}
  For every real number \(x\), exactly one of the following three statements is true:
  \begin{enumerate}
    \item \(x\) is zero;
    \item \(x\) is positive;
    \item \(x\) is negative.
  \end{enumerate}
  A real number \(x\) is negative iff \(-x\) is positive.
  If \(x\) and \(y\) are positive, then so are \(x + y\) and \(xy\).
\end{prop}

\begin{proof}
  We first show that at least one of the three statements is true.
  Let \(x\) be the formal limit of some rational sequence \((a_n)_{n = 1}^{\infty}\), let \(\varepsilon, c \in \Q^+\) and let \(N, j, k \in \N\).
  Consider the following two cases:
  \begin{itemize}
    \item If \((a_n)_{n = 1}^{\infty}\) is eventually \(\varepsilon\)-close to \(0\) for all \(\varepsilon > 0\), then by \cref{5.2.6} we have \(x = 0\).
    \item If \((a_n)_{n = 1}^{\infty}\) is not eventually \(\varepsilon\)-close to \(0\) for all \(\varepsilon > 0\), then by \cref{5.2.6} we have \(x \neq 0\).
  \end{itemize}
  By \cref{5.3.14}, \(x \neq 0\) implies \(\exists c > 0\) such that \(\abs{a_n} \geq c > 0\) for every \(n \geq 1\).
  By \cref{4.3.3}(a) we have \(a_n \neq 0\).
  By \cref{4.2.9}(a) we now split into two cases:
  \begin{itemize}
    \item If \(a_n > 0\), then by \cref{4.3.1} we have \(a_n \geq c > 0\).
    \item If \(a_n < 0\), then by \cref{4.3.1} we have \(-a_n \geq c > 0\), and by \cref{ex:4.2.6} we have \(a_n \leq -c < 0\).
  \end{itemize}
  Since \((a_n)_{n = 1}^{\infty}\) is a Cauchy sequence, by \cref{5.1.8} we have
  \[
    \forall \varepsilon > 0, \exists N \geq 1 : \forall j, k \geq N, \abs{a_j - a_k} \leq \varepsilon.
  \]
  In particular,
  \[
    \exists N \geq 1 : \forall j, k \geq N, \abs{a_j - a_k} \leq c.
  \]
  So
  \begin{align*}
             & \abs{a_j - a_N} \leq c                          \\
    \implies & -c \leq a_j - a_N \leq c        &  & \by{4.3.3} \\
    \implies & -c + a_N \leq a_j \leq c + a_N. &  & \by{4.2.9} \\
  \end{align*}
  By \cref{4.2.9}(a) again we now split into two cases:
  \begin{itemize}
    \item If \(a_N > 0\), then we have
          \begin{align*}
                     & (-c + a_N \leq a_j \leq c + a_N) \land (0 < c \leq a_N)                                \\
            \implies & 0 \leq a_j \leq c + a_N                                 &            & \by{4.2.9}[c,d] \\
            \implies & c < a_j \leq c + a_N.                                   & (x \neq 0)
          \end{align*}
          Since this is true for all \(j \geq N\), by \cref{5.3.14} and \cref{5.4.1} we know that \((a_n)_{n = 1}^\infty\) is positively bounded away from zero.
          Thus by \cref{5.4.3} \(x\) is positive.
    \item If \(a_N < 0\), then we have
          \begin{align*}
                     & (-c + a_N \leq a_j \leq c + a_N) \land (a_N \leq -c < 0)                                \\
            \implies & -c + a_N \leq a_j \leq 0                                 &            & \by{4.2.9}[c,d] \\
            \implies & -c + a_N \leq a_j < -c.                                  & (x \neq 0)
          \end{align*}
          Since this is true for all \(j \geq N\), by \cref{5.3.14} and \cref{5.4.1} we know that \((a_n)_{n = 1}^\infty\) is negatively bounded away from zero.
          Thus by \cref{5.4.3} \(x\) is negative.
  \end{itemize}
  From all cases above we conclude that at least one of the three statements is true.

  Next we show that at most one of the three statements is true.
  Let \(x\) be the formal limit of some rational sequence \((a_n)_{n = 1}^{\infty}\) and let \(\varepsilon, c \in \Q^+\).
  \begin{itemize}
    \item If \(x = 0\) and \(x\) is positive, then we have \((a_n)\) eventually \(\varepsilon\)-close to \(0\) for all \(\varepsilon > 0\) and \(a_n \geq c\) for all \(n \geq 1\).
          But then we have \(\abs{a_n - 0} \leq c / 2\) and \(a_n \geq c > 0\), a contradiction.
    \item If \(x = 0\) and \(x\) is negative, then we have \((a_n)\) eventually \(\varepsilon\)-close to \(0\) for all \(\varepsilon > 0\) and \(a_n \leq -c\) for all \(n \geq 1\).
          But then we have \(\abs{a_n - 0} \leq c / 2\) and \(a_n \leq -c < 0\), a contradiction.
    \item If \(x\) is positive and \(x\) is negative, then we have \(a_n \geq c\) and \(a_n \leq -c\) for all \(n \geq 1\).
          But then we have \(a_n < 0\) and \(0 < a_n\), a contradiction.
  \end{itemize}
  From all cases above we conclude that at most one of the three statements is true.

  Next we show that \(x\) is negative iff \(-x\) is positive.
  Let \(x\) be the formal limit of some sequence \((a_n)_{n = 1}^{\infty}\).
  Then we have
  \begin{align*}
         & x \text{ is negative}                                                             \\
    \iff & (x = \text{LIM}_{n \to \infty} a_n)                                               \\
         & \land (\exists c \in \Q^+ : \forall n \geq 1, a_n \leq -c < 0) &  & \by{5.4.3}    \\
    \iff & (-x = \text{LIM}_{n \to \infty} -a_n)                          &  & \by{5.3.9}    \\
         & \land (\exists c \in \Q^+ : \forall n \geq 1, a_n \leq -c < 0)                    \\
    \iff & (-x = \text{LIM}_{n \to \infty} -a_n)                                             \\
         & \land (\exists c \in \Q^+ : \forall n \geq 1, -a_n \geq c > 0) &  & \by{ex:4.2.6} \\
    \iff & -x \text{ is positive}.                                        &  & \by{5.4.3}
  \end{align*}

  Next we show that \(x, y\) are positive implies \(x + y\) is also positive.
  Let \(x\) be the formal limit of some sequence \((a_n)_{n = 1}^{\infty}\) and let \(y\) be the formal limit of some sequence \((b_n)_{n = 1}^{\infty}\).
  Then we have
  \begin{align*}
             & (x \text{ is positive}) \land (y \text{ is positive})                                                \\
    \implies & (x = \text{LIM}_{n \to \infty} a_n) \land (y = \text{LIM}_{n \to \infty} b_n)                        \\
             & \land (\exists c_1 \in \Q^+ : \forall n \geq 1, 0 < c_1 < a_n)                                       \\
             & \land (\exists c_2 \in \Q^+ : \forall n \geq 1, 0 < c_2 < b_n)                  &  & \by{5.4.3}      \\
    \implies & (x + y = \text{LIM}_{n \to \infty} a_n + b_n)                                   &  & \by{5.3.4}      \\
             & \land (\exists c_1 \in \Q^+ : \forall n \geq 1, 0 < c_1 < a_n)                                       \\
             & \land (\exists c_2 \in \Q^+ : \forall n \geq 1, 0 < c_2 < b_n)                                       \\
    \implies & (x + y = \text{LIM}_{n \to \infty} a_n + b_n)                                                        \\
             & \land (\exists c_1, c_2 \in \Q^+ : \forall n \geq 1, 0 < c_1 + c_2 < a_n + b_n) &  & \by{4.2.9}[c,d] \\
    \implies & x + y \text{ is positive}.                                                      &  & \by{5.4.3}
  \end{align*}

  Finally we show that \(x, y\) are positive implies \(xy\) is also positive.
  Let \(x\) be the formal limit of some sequence \((a_n)_{n = 1}^{\infty}\) and let \(y\) be the formal limit of some sequence \((b_n)_{n = 1}^{\infty}\).
  Then we have
  \begin{align*}
             & (x \text{ is positive}) \land (y \text{ is positive})                                              \\
    \implies & (x = \text{LIM}_{n \to \infty} a_n) \land (y = \text{LIM}_{n \to \infty} b_n)                      \\
             & \land (\exists c_1 \in \Q^+ : \forall n \geq 1, 0 < c_1 < a_n)                                     \\
             & \land (\exists c_2 \in \Q^+ : \forall n \geq 1, 0 < c_2 < b_n)                &  & \by{5.4.3}      \\
    \implies & (xy = \text{LIM}_{n \to \infty} a_n b_n)                                      &  & \by{5.3.9}      \\
             & \land (\exists c_1 \in \Q^+ : \forall n \geq 1, 0 < c_1 < a_n)                                     \\
             & \land (\exists c_2 \in \Q^+ : \forall n \geq 1, 0 < c_2 < b_n)                                     \\
    \implies & (xy = \text{LIM}_{n \to \infty} a_n b_n)                                                           \\
             & \land (\exists c_1, c_2 \in \Q^+ : \forall n \geq 1, 0 < c_1 c_2 < a_n b_n)   &  & \by{4.2.9}[d,e] \\
    \implies & xy \text{ is positive}.                                                       &  & \by{5.4.3}
  \end{align*}
\end{proof}

\begin{note}
  If \(q\) is a positive rational number, then the Cauchy sequence \(q, q, q, \dots\) is positively bounded away from zero, and hence \(\text{LIM}_{n \to \infty} q = q\) is a positive real number.
  Thus the notion of positivity for rationals is consistent with that for reals.
  Similarly, the notion of negativity for rationals is consistent with that for reals.
\end{note}

\begin{defn}[Absolute value]\label{5.4.5}
  Let \(x\) be a real number.
  We define the \emph{absolute value} \(\abs{x}\) of \(x\) to equal \(x\) if \(x\) is positive, \(-x\) when \(x\) is negative, and \(0\) when \(x\) is zero.
\end{defn}

\begin{defn}[Ordering of the real numbers]\label{5.4.6}
  Let \(x\) and \(y\) be real numbers.
  We say that \(x\) is \emph{greater than} \(y\), and write \(x > y\), iff \(x - y\) is a positive real number, and \(x < y\) iff \(x - y\) is a negative real number.
  We define \(x \geq y\) iff \(x > y\) or \(x = y\), and similarly define \(x \leq y\).
\end{defn}

\begin{note}
  Comparing this with the definition of order on the rationals from \cref{4.2.8} we see that order on the reals is consistent with order on the rationals, i.e., if two rational numbers \(q, q'\) are such that \(q\) is less than \(q'\) in the rational number system, then \(q\) is still less than \(q'\) in the real number system, and similarly for ``greater than''.
  In the same way we see that the definition of absolute value given here is consistent with that in \cref{4.3.1}.
\end{note}

\begin{prop}\label{5.4.7}
  All the claims in \cref{4.2.9} which held for rationals, continue to hold for real numbers.
\end{prop}

\begin{proof}{(a)}
  We first show that at least one of the three statements is true.
  By \cref{5.4.4}, exactly one of the following three statements is true:
  \begin{itemize}
    \item \(x - y = 0\).
          Then by \cref{5.3.11} we have \(x = y\).
    \item \(x - y\) is positive.
          Then by \cref{5.4.6} we have \(x > y\).
    \item \(x - y\) is negative.
          Then by \cref{5.4.6} we have \(x < y\).
  \end{itemize}
  So at least one of the three statements is true.

  Now we show that at most one of the three statements is true.
  \begin{itemize}
    \item If \(x = y\) and \(x > y\) are true, then by \cref{5.3.11} we have \(x - y = 0\) and by \cref{5.4.6} we have \(x - y\) is positive.
          But this contradict to \cref{5.4.4}.
    \item If \(x = y\) and \(x < y\) are true, then by \cref{5.3.11} we have \(x - y = 0\) and by \cref{5.4.6} we have \(x - y\) is negative.
          But this contradict to \cref{5.4.4}.
    \item If \(x > y\) and \(x < y\) are true, then by \cref{5.4.6}, \(x - y\) is both positive and negative.
          But this contradict to \cref{5.4.4}.
  \end{itemize}
  From all cases above we conclude that at most one of the three statements is true.
\end{proof}

\begin{proof}{(b)}
  We have
  \begin{align*}
         & x < y                                         \\
    \iff & x - y \text{ is negative}    &  & \by{5.4.6}  \\
    \iff & -(x - y) \text{ is positive} &  & \by{5.4.4}  \\
    \iff & y - x \text{ is positive}    &  & \by{5.3.11} \\
    \iff & y > x.                       &  & \by{5.4.6}
  \end{align*}
\end{proof}

\begin{proof}{(c)}
  We have
  \begin{align*}
             & (x < y) \land (y < z)                                                                \\
    \implies & (x - y \text{ is negative}) \land (y - z \text{ is negative})       &  & \by{5.4.6}  \\
    \implies & (-(x - y) \text{ is positive}) \land (-(y - z) \text{ is positive}) &  & \by{5.4.4}  \\
    \implies & (y - x \text{ is positive}) \land (z - y \text{ is positive})       &  & \by{5.3.11} \\
    \implies & y - x + z - y \text{ is positive}                                   &  & \by{5.4.4}  \\
    \implies & - x + z \text{ is positive}                                         &  & \by{5.3.11} \\
    \implies & -(x - z) \text{ is positive}                                        &  & \by{5.3.11} \\
    \implies & x - z \text{ is negative}                                           &  & \by{5.4.4}  \\
    \implies & x < z.                                                              &  & \by{5.4.6}
  \end{align*}
\end{proof}

\begin{proof}{(d)}
  We have
  \begin{align*}
             & x < y                                                \\
    \implies & x - y \text{ is negative}           &  & \by{5.4.6}  \\
    \implies & x + z - z - y \text{ is negative}   &  & \by{5.3.11} \\
    \implies & x + z - (y + z) \text{ is negative} &  & \by{5.3.11} \\
    \implies & x + z < y + z.                      &  & \by{5.4.6}
  \end{align*}
\end{proof}

\begin{proof}{(e)}
  We have
  \begin{align*}
             & x < y                                           \\
    \implies & y > x                        &  & \by{5.4.7}[b] \\
    \implies & y - x \text{ is positive}    &  & \by{5.4.6}    \\
    \implies & (y - x)z \text{ is positive} &  & \by{5.4.4}    \\
    \implies & yz - xz \text{ is positive}  &  & \by{5.3.11}   \\
    \implies & yz > xz                      &  & \by{5.4.6}    \\
    \implies & xz < yz.                     &  & \by{5.4.7}[b]
  \end{align*}
\end{proof}

\begin{prop}\label{5.4.8}
  Let \(x\) be a positive real number.
  Then \(x^{-1}\) is also positive.
  Also, if \(y\) is another positive number and \(x > y\), then \(x^{-1} < y^{-1}\).
\end{prop}

\begin{proof}
  Let \(x\) be positive.
  Since \(xx^{-1} = 1\), the real number \(x^{-1}\) cannot be zero (since \(x0 = 0 \neq 1\)).
  Also, from \cref{5.4.4} it is easy to see that a positive number times a negative number is negative;
  this shows that \(x^{-1}\) cannot be negative, since this would imply that \(xx^{-1} = 1\) is negative, a contradiction.
  Thus, by \cref{5.4.4}, the only possibility left is that \(x^{-1}\) is positive.

  Now let \(y\) be positive as well, so \(x^{-1}\) and \(y^{-1}\) are also positive.
  If \(x^{-1} \geq y^{-1}\), then by \cref{5.4.7} we have \(xx^{-1} > yx^{-1} \geq yy^{-1}\), thus \(1 > 1\), which is a contradiction.
  Thus we must have \(x^{-1} < y^{-1}\).
\end{proof}

\begin{prop}[The non-negative reals are closed]\label{5.4.9}
  Let \(a_1, a_2, a_3, \dots\) be a Cauchy sequence of non-negative rational numbers.
  Then \(\text{LIM}_{n \to \infty} a_n\) is a non-negative real number.
\end{prop}

\begin{proof}
  We argue by contradiction, and suppose that the real number \(x \coloneqq \text{LIM}_{n \to \infty} a_n\) is a negative number.
  Then by definition of negative real number, we have \(x = \text{LIM}_{n \to \infty} b_n\) for some sequence \(b_n\) which is negatively bounded away from zero, i.e., there is a negative rational \(-c < 0\) such that \(b_n \leq -c\) for all \(n \geq 1\).
  On the other hand, we have \(a_n \geq 0\) for all \(n \geq 1\), by hypothesis.
  Thus the numbers \(a_n\) and \(b_n\) are never \(c / 2\)-close, since \(c / 2 < c\).
  Thus the sequences \((a_n)_{n = 1}^{\infty}\) and \((b_n)_{n = 1}^{\infty}\) are not eventually \(c / 2\)-close.
  Since \(c / 2 > 0\), this implies that \((a_n)_{n = 1}^{\infty}\) and \((b_n)_{n = 1}^{\infty}\) are not equivalent.
  But this contradicts the fact that both these sequences have \(x\) as their formal limit.
\end{proof}

\begin{note}
  Eventually, we will see a better explanation of \cref{5.4.9}:
  the set of non-negative reals is \emph{closed}, whereas the set of positive reals is \emph{open}.
\end{note}

\begin{cor}\label{5.4.10}
  Let \((a_n)_{n = 1}^{\infty}\) and \((b_n)_{n = 1}^{\infty}\) be Cauchy sequences of rationals such that \(a_n \geq b_n\) for all \(n \geq 1\).
  Then \(\text{LIM}_{n \to \infty} a_n \geq \text{LIM}_{n \to \infty} b_n\).
\end{cor}

\begin{proof}
  Apply \cref{5.4.9} to the sequence \(a_n - b_n\).
\end{proof}

\begin{rmk}\label{5.4.11}
  Note that \cref{5.4.10} does not work if the \(\geq\) signs are replaced by \(>\):
  for instance if \(a_n \coloneqq 1 + 1 / n\) and \(b_n \coloneqq 1 - 1 / n\), then \(a_n\) is always strictly greater than \(b_n\), but the formal limit of \(a_n\) is not greater than the formal limit of \(b_n\), instead they are equal.
\end{rmk}

\begin{note}
  We now define distance \(d(x, y) \coloneqq \abs{x - y}\) just as we did for the rationals.
  In fact, \cref{4.3.3,4.3.7} hold not only for the rationals, but for the reals;
  the proof is identical, since the real numbers obey all the laws of algebra and order that the rationals do.
\end{note}

\begin{prop}[Bounding of reals by rationals]\label{5.4.12}
  Let \(x\) be a positive real number.
  Then there exists a positive rational number \(q\) such that \(q \leq x\), and there exists a positive integer \(N\) such that \(x \leq N\).
\end{prop}

\begin{proof}
  Since \(x\) is a positive real, it is the formal limit of some Cauchy sequence \((a_n)_{n = 1}^{\infty}\) which is positively bounded away from zero.
  Also, by \cref{5.1.15}, this sequence is bounded.
  Thus we have rationals \(q > 0\) and \(r\) such that \(q \leq a_n \leq r\) for all \(n \geq 1\).
  But by \cref{4.4.1} we know that there is some integer \(N\) such that \(r \leq N\);
  since \(q\) is positive and \(q \leq r \leq N\), we see that \(N\) is positive.
  Thus \(q \leq a_n \leq N\) for all \(n \geq 1\).
  Applying \cref{5.4.10} we obtain that \(q \leq x \leq N\), as desired.
\end{proof}

\begin{cor}[Archimedean property]\label{5.4.13}
  Let \(x\) and \(\varepsilon\) be any positive real numbers.
  Then there exists a positive integer \(M\) such that \(M\varepsilon > x\).
\end{cor}

\begin{proof}
  The number \(x / \varepsilon\) is positive, and hence by \cref{5.4.12} there exists a positive integer \(N\) such that \(x / \varepsilon \leq N\).
  If we set \(M \coloneqq N + 1\), then \(x / \varepsilon < M\).
  Now multiply by \(\varepsilon\).
\end{proof}

\begin{note}
  This property (\cref{5.4.13}) is quite important;
  it says that no matter how large \(x\) is and how small \(\varepsilon\) is, if one keeps adding \(\varepsilon\) to itself, one will eventually overtake \(x\).
\end{note}

\begin{prop}\label{5.4.14}
  Given any two real numbers \(x < y\), we can find a rational number \(q\) such that \(x < q < y\).
\end{prop}

\begin{proof}
  We have
  \begin{align*}
             & x < y                                                 \\
    \implies & y > x                              &  & \by{5.4.7}    \\
    \implies & y - x \text{ is positive}          &  & \by{5.4.6}    \\
    \implies & \exists N \in \Z^+ : y - x > 1 / N &  & \by{ex:5.4.4} \\
    \implies & y > x + 1 / N.                     &  & \by{5.4.7}    \\
  \end{align*}
  Since \(x\) is a real number, by \cref{5.3.10} we know that \(Nx\) is also a real number.
  By \cref{ex:5.4.3}, \(\exists M \in \Z\) such that \(M \leq Nx < M + 1\).
  So we have
  \begin{align*}
             & M \leq Nx < M + 1                                                                     \\
    \implies & \dfrac{M}{N} \leq x < \dfrac{M + 1}{N}                                &  & \by{5.4.7} \\
    \implies & (\dfrac{M}{N} \leq x) \land (x < \dfrac{M + 1}{N})                    &  & \by{5.4.7} \\
    \implies & (\dfrac{M + 1}{N} \leq x + \dfrac{1}{N}) \land (x < \dfrac{M + 1}{N}) &  & \by{5.4.7} \\
    \implies & x < \dfrac{M + 1}{N} \leq x + \dfrac{1}{N}                            &  & \by{5.4.7} \\
    \implies & x < \dfrac{M + 1}{N} \leq x + \dfrac{1}{N} < y.                       &  & \by{5.4.7}
  \end{align*}
  Since \((M + 1) / N \in \Q\), we conclude that \(\exists q = (M + 1) / N \in \Q\) such that \(x < q < y\) for arbitrary real numbers \(x, y\).
\end{proof}

\begin{rmk}\label{5.4.15}
  Up until now, we have not addressed the fact that real numbers can be expressed using the decimal system.
  For instance, the formal limit of
  \[
    1.4, 1.41, 1.414, 1.4142, 1.41421, \dots
  \]
  is more conventionally represented as the decimal \(1.41421\dots\).
  There are some subtleties in the decimal system, for instance \(0.9999\dots\) and \(1.000\dots\) are in fact the same real number.
\end{rmk}

\begin{ac}\label{ac:5.4.1}
  Let \(X\) be an non-empty finite subset of \(\R\).
  Then \(X\) has exactly one maximum \(\max(X) \in X\) satisfying
  \[
    \forall x \in X, x \leq \max(X).
  \]
  Similarly, \(X\) has exactly one minimum \(\min(X) \in X\) satisfying
  \[
    \forall x \in X, x \geq \min(X).
  \]
\end{ac}

\begin{proof}
  Let \(n = \#(X)\).
  We use induction on \(n\) to show that \(\max(X), \min(X) \in X\) and we start with \(n = 1\) (since \(\#(\emptyset) = 0\) by \cref{ex:3.6.2}).
  For \(n = 1\), we have \(X = \set{x}\) for some \(x \in \R\).
  Then we have
  \begin{align*}
             & \forall y \in X, y = x                                        &  & \by{3.3}   \\
    \implies & (\forall y \in X, y \leq x) \land (\forall y \in X, y \geq x) &  & \by{5.4.6} \\
    \implies & \max(X) = \min(X) = x
  \end{align*}
  and the base case holds.
  Suppose inductively that for some \(n \geq 1\) we have \(\max(X) \in X\) and \(\min(X) \in X\).
  Then for \(n + 1\), we need to show that \(\max(X) \in X\) and \(\min(X) \in X\).
  Let \(x \in X\) and let \(X' = X \setminus \set{x}\).
  Then we have
  \begin{align*}
             & X = X' \cup \set{x}                                                                                                           \\
    \implies & \#(X') = n                                                                                &                  & \by{3.6.14}[a] \\
    \implies & (\max(X') \in X') \land (\min(X') \in X')                                                 &                  & \byIH          \\
    \implies & (\max(X') \in X) \land (\min(X') \in X)                                                   & (X' \subseteq X)                  \\
    \implies & \begin{dcases}
                 \max(X) = \begin{dcases}
                  \max(X') & \text{if } \max(X') > x \\
                  x        & \text{if } \max(X') < x
                \end{dcases} \\
                 \min(X) = \begin{dcases}
                  \min(X') & \text{if } \min(X') < x \\
                  x        & \text{if } \min(X') > x
                \end{dcases}
               \end{dcases}                                                                                          \\
    \implies & \big(\forall y \in X, x \leq \max(X)\big) \land \big(\forall y \in X, x \geq \min(X)\big)
  \end{align*}
  and this closes the induction.

  Now we show that at most one \(\max(X), \min(X) \in X\).
  Suppose that \(x_1 = \max(X)\) and \(x_2 = \max(X)\).
  Then by \cref{5.4.7}(a) we have
  \[
    (x_1 \leq x_2) \land (x_2 \leq x_1) \implies x_1 = x_2.
  \]
  Similarly suppose that \(x_1 = \min(X)\) and \(x_2 = \min(X)\).
  Then by \cref{5.4.7}(a) we have
  \[
    (x_1 \geq x_2) \land (x_2 \geq x_1) \implies x_1 = x_2.
  \]
\end{proof}

\exercisesection

\begin{ex}\label{ex:5.4.1}
  Prove \cref{5.4.4}.
\end{ex}

\begin{proof}
  See \cref{5.4.4}.
\end{proof}

\begin{ex}\label{ex:5.4.2}
  Prove the remaining claims in \cref{5.4.7}.
\end{ex}

\begin{proof}
  See \cref{5.4.7}.
\end{proof}

\begin{ex}\label{ex:5.4.3}
  Show that for every real number \(x\) there is exactly one integer \(N\) such that \(N \leq x < N + 1\).
  (This integer \(N\) is called the \emph{integer part} of \(x\), and is sometimes denoted \(N = \floor{x}\).)
\end{ex}

\begin{proof}
  We first prove the existence of the integer \(N\).
  By \cref{5.4.4}, exactly one of the following three statements is true:
  \begin{itemize}
    \item \(x = 0\).
          Then we choose \(N = 0\) so that \(0 \leq 0 < 1\).
    \item \(x\) is positive.
          Then by \cref{5.4.12} \(\exists q \in \Q^+\) and \(\exists N_1' \in \Z^+\) such that \(q \leq x \leq N_1'\).
          Let \(N_1 = N_1' + 1\).
          Then we have \(x < N_1\).
          By \cref{4.4.1} we know that \(\exists N_2 \in \Z\) such that \(N_2 \leq q\), thus by \cref{5.4.7}(c) we have \(N_2 \leq x < N_1\).
          Let \(X\) be the set
          \[
            X = \set{n \in \Z : N_2 \leq n < N_1}
          \]
          and let \(X'\) be the set
          \[
            X' = \set{n \in X : n \leq x < N_1}.
          \]
          We know that \(X, X'\) is finite since \(\#(X) = N_1 - N_2 + 1 \geq 1\) and \(X' \subseteq X\) (by \cref{3.6.14}(c)).
          We also know that \(X, X'\) is non-empty since \(N_2 \in X'X\).
          By \cref{ac:5.4.1} we know that \(\exists!\ \max(X') \in X'\).
          Let \(N = \max(X')\).
          By the definition of \(X'\) we know that \(N \leq x < N_1\).
          We must also have \(x < N + 1\), otherwise if \(N + 1 \leq x\) then \(\max(X') = N + 1\), a contradiction.
          Thus we have \(N \leq x < N + 1\).
    \item \(x\) is negative.
          Then we have
          \begin{align*}
                     & -x > 0                                  &  & \by{5.4.4}               \\
            \implies & \exists M \in \Z^+ : M1 = M > x         &  & \by{5.4.13}              \\
            \implies & x + M > 0                               &  & \by{5.4.7}[d]            \\
            \implies & \exists N \in \Z : N \leq x + M < N + 1 &  & \text{(from case above)} \\
            \implies & N - M \leq x < N - M + 1.
          \end{align*}
  \end{itemize}
  From all cases above we conclude that \(\exists N \in \Z : N \leq x < N + 1\).

  Now we prove the uniqueness of the integer \(N\).
  Suppose that \(\exists N_1, N_2 \in \Z\) such that \(N_1 \leq x < N_1 + 1\) and \(N_2 \leq x < N_2 + 1\).
  Then we have
  \begin{align*}
             & (N_1 \leq x < N_1 + 1) \land (N_2 \leq x < N_2 + 1)                    \\
    \implies & (N_1 < N_2 + 1) \land (N_2 < N_1 + 1)               &  & \by{5.4.7}[c] \\
    \implies & (N_1 + 1 \leq N_2 + 1) \land (N_2 + 1 \leq N_1 + 1) &  & \by{4.1.10}   \\
    \implies & (N_1 \leq N_2) \land (N_2 \leq N_1)                 &  & \by{4.1.10}   \\
    \implies & N_1 = N_2.                                          &  & \by{4.1.11}
  \end{align*}
  Thus we conclude that \(\exists!\ N \in \Z : N \leq x < N + 1\).
\end{proof}

\begin{ex}\label{ex:5.4.4}
  Show that for any positive real number \(x > 0\) there exists a positive integer \(N\) such that \(x > 1 / N > 0\).
\end{ex}

\begin{proof}
  We have
  \begin{align*}
             & x > 0                                                 \\
    \implies & x^{-1} > 0                           &  & \by{5.4.8}  \\
    \implies & \exists N \in \Z^+ : N1 = N > x^{-1} &  & \by{5.4.13} \\
    \implies & N^{-1} = \dfrac{1}{N} < x.           &  & \by{5.4.8}
  \end{align*}
\end{proof}

\begin{ex}\label{ex:5.4.5}
  Prove \cref{5.4.14}.
\end{ex}

\begin{proof}
  See \cref{5.4.14}.
\end{proof}

\begin{ex}\label{ex:5.4.6}
  Let \(x, y\) be real numbers and let \(\varepsilon > 0\) be a positive real.
  Show that \(\abs{x - y} < \varepsilon\) iff \(y - \varepsilon < x < y + \varepsilon\), and that \(\abs{x - y} \leq \varepsilon\) iff \(y - \varepsilon \leq x \leq y + \varepsilon\).
\end{ex}

\begin{proof}
  We first show that \(\abs{x - y} < \varepsilon \iff y - \varepsilon < x < y + \varepsilon\).
  \begin{align*}
         & \abs{x - y} < \varepsilon                                                                                     \\
    \iff & \big(-(x - y) \leq x - y < \varepsilon\big) \lor \big(x - y \leq -(x - y) < \varepsilon\big) &  & \by{5.4.5}  \\
    \iff & (x - y < \varepsilon) \land (-(x - y) < \varepsilon)                                                          \\
    \iff & (x - y < \varepsilon) \land (y - x < \varepsilon)                                            &  & \by{5.3.11} \\
    \iff & (x < y + \varepsilon) \land (y - \varepsilon < x)                                            &  & \by{5.4.7}  \\
    \iff & y - \varepsilon < x < y + \varepsilon.                                                       &  & \by{5.4.7}
  \end{align*}

  Now we show that \(\abs{x - y} \leq \varepsilon \iff y - \varepsilon \leq x \leq y + \varepsilon\).
  Since
  \begin{align*}
         & \abs{x - y} = \varepsilon                                            \\
    \iff & (x - y = \varepsilon) \lor (-(x - y) = \varepsilon) &  & \by{5.4.5}  \\
    \iff & (x = y + \varepsilon) \lor (x = y - \varepsilon),   &  & \by{5.3.11}
  \end{align*}
  we have
  \begin{align*}
         & \abs{x - y} \leq \varepsilon                                                                             \\
    \iff & (\abs{x - y} < \varepsilon) \land (\abs{x - y} = \varepsilon)                                            \\
    \iff & (y - \varepsilon < x < y + \varepsilon) \land \big((x = y + \varepsilon) \lor (x = y - \varepsilon)\big) \\
    \iff & (y - \varepsilon \leq x < y + \varepsilon) \lor (y - \varepsilon < x \leq y + \varepsilon)               \\
    \iff & y - \varepsilon \leq x \leq y + \varepsilon.
  \end{align*}
\end{proof}

\begin{ex}\label{ex:5.4.7}
  Let \(x\) and \(y\) be real numbers.
  Show that \(x \leq y + \varepsilon\) for all real numbers \(\varepsilon > 0\) iff \(x \leq y\).
  Show that \(\abs{x - y} \leq \varepsilon\) for all real numbers \(\varepsilon > 0\) iff \(x = y\).
\end{ex}

\begin{proof}
  We first show that \(x \leq y + \varepsilon\) for all real numbers \(\varepsilon > 0\) iff \(x \leq y\).
  \begin{align*}
         & \forall \varepsilon \in \R^+, x \leq y + \varepsilon                 \\
    \iff & \forall \varepsilon \in \R^+, x - y \leq \varepsilon &  & \by{5.4.7} \\
    \iff & \lnot (x - y > 0)                                                    \\
    \iff & x - y \leq 0                                         &  & \by{5.4.7} \\
    \iff & x \leq y.                                            &  & \by{5.4.6}
  \end{align*}

  Now we show that \(\abs{x - y} \leq \varepsilon\) for all real numbers \(\varepsilon > 0\) iff \(x = y\).
  \begin{align*}
         & \forall \varepsilon \in \R^+, \abs{x - y} \leq \varepsilon                                   \\
    \iff & \forall \varepsilon \in \R^+, y - \varepsilon \leq x \leq y + \varepsilon &  & \by{ex:5.4.6} \\
    \iff & (x \leq y) \land (y \leq x)                                                                  \\
    \iff & x = y.                                                                    &  & \by{5.3.11}
  \end{align*}
\end{proof}

\begin{ex}\label{ex:5.4.8}
  Let \((a_n)_{n = 1}^{\infty}\) be a Cauchy sequence of rationals, and let \(x\) be a real number.
  Show that if \(a_n \leq x\) for all \(n \geq 1\), then \(\text{LIM}_{n \to \infty} a_n \leq x\).
  Similarly, show that if \(a_n \geq x\) for all \(n \geq 1\), then \(\text{LIM}_{n \to \infty} a_n \geq x\).
\end{ex}

\begin{proof}
  We first show that if \(a_n \leq x\) for all \(n \geq 1\), then \(\text{LIM}_{n \to \infty} a_n \leq x\).
  Let \(a = \text{LIM}_{n \to \infty} a_n\).
  Suppose for sake of contradiction that \(a > x\).
  Then by \cref{5.4.14}, \(\exists q \in \Q\) such that \(a > q > x\).
  Since \(q > x\), we have \(a_n \leq x < q\) for all \(n \geq 1\).
  But by \cref{5.4.10} we have \(a = \text{LIM}_{n \to \infty} a_n \leq \text{LIM}_{n \to \infty} q = q\), which contradict to \(a > q\).
  Thus we must have \(a \leq x\).

  Now we show that if \(a_n \geq x\) for all \(n \geq 1\), then \(\text{LIM}_{n \to \infty} a_n \geq x\).
  We have
  \begin{align*}
             & a_n \geq x                                                            \\
    \implies & -a_n \leq -x                           &  & \by{5.4.7}                \\
    \implies & \text{LIM}_{n \to \infty} -a_n \leq -x &  & \text{(from proof above)} \\
    \implies & -\text{LIM}_{n \to \infty} a_n \leq -x &  & \by{5.3.10}               \\
    \implies & \text{LIM}_{n \to \infty} a_n \geq x.  &  & \by{5.4.7}
  \end{align*}
\end{proof}

\section{The least upper bound property}\label{i:sec:5.5}

\begin{defn}[Upper bound]\label{i:5.5.1}
  Let \(E\) be a subset of \(\R\), and let \(M\) be a real number.
  We say that \(M\) is an \emph{upper bound} for \(E\), iff we have \(x \leq M\) for every element \(x\) in \(E\).
\end{defn}

\setcounter{thm}{2}
\begin{eg}\label{i:5.5.3}
  Let \(\R^+\) be the set of positive reals: \(\R^+ \coloneqq \set{x \in \R : x > 0}\).
  Then \(\R^+\) does not have any upper bounds at all.
  (More precisely, \(\R^+\) has no upper bounds which are real numbers.)
\end{eg}

\begin{proof}[\pf{i:5.5.3}]
  Suppose for sake of contradiction that there exists an \(M \in \R\) such that \(M\) is an upper bound for \(\R^+\).
  Then we have \(x \leq M\) for all \(x \in \R^+\).
  Since \(x > 0\), by \cref{i:5.4.7} we have \(M > 0\), thus \(M \in \R^+\).
  But this means \(M + 1 \in \R^+\), and we must have \(M > M + 1\), a contradiction.
  Thus such \(M\) does not exist and \(\R^+\) has no upper bounds in \(\R\).
\end{proof}

\begin{ac}\label{i:ac:5.5.1}
  Let \(x, y \in \R\).
  We define the following eight subsets of \(\R\):
  \begin{align*}
    \R_{\leq x}   & \coloneqq \set{r \in \R : r \leq x};        & \R_{< x}   & \coloneqq \set{r \in \R : r < x};     & \R^+ & \coloneqq \R_{> 0}; \\
    \R_{\geq x}   & \coloneqq \set{r \in \R : r \geq x};        & \R_{> x}   & \coloneqq \set{r \in \R : r > x};     & \R^- & \coloneqq \R_{< 0}; \\
    \R_{x \leq y} & \coloneqq \set{r \in \R : x \leq r \leq y}; & \R_{x < y} & \coloneqq \set{r \in \R : x < r < y}. &      &
  \end{align*}
\end{ac}

\begin{eg}\label{i:5.5.4}
  Let \(\emptyset\) be the empty set.
  Then every number \(M\) is an upper bound for \(\emptyset\), because \(M\) is greater than every element of the empty set
  (this is a vacuously true statement, but still true).
\end{eg}

\begin{note}
  It is clear that if \(M\) is an upper bound of \(E\), then any larger number \(M' \geq M\) is also an upper bound of \(E\).
  On the other hand, it is not so clear whether it is also possible for any number smaller than \(M\) to also be an upper bound of \(E\).
  This motivates the \cref{i:5.5.5}.
\end{note}

\begin{defn}[Least upper bound]\label{i:5.5.5}
  Let \(E\) be a subset of \(\R\), and \(M\) be a real number.
  We say that \(M\) is a \emph{least upper bound} for \(E\) iff
  \begin{enumerate}
    \item \(M\) is an upper bound for \(E\), and also
    \item any other upper bound \(M'\) for \(E\) must be larger than or equal to \(M\).
  \end{enumerate}
\end{defn}

\setcounter{thm}{6}
\begin{eg}\label{i:5.5.7}
  The empty set does not have a least upper bound.
\end{eg}

\begin{proof}[\pf{i:5.5.7}]
  Suppose for sake of contradiction that there exists an \(M \in \R\) such that \(M\) is a least upper bound of \(\emptyset\).
  By \cref{i:5.5.5} we know that \(x \leq M\) for all \(x \in \emptyset\).
  But by \cref{i:5.5.4} we know that \(M - 1\) is also a upper bound of \(\emptyset\), so by \cref{i:5.5.5} we have \(M < M - 1\), a contradiction.
  Thus \(\emptyset\) does not have a least upper bound.
\end{proof}

\begin{prop}[Uniqueness of least upper bound]\label{i:5.5.8}
  Let \(E\) be a subset of \(\R\).
  Then \(E\) can have at most one least upper bound.
\end{prop}

\begin{proof}[\pf{i:5.5.8}]
  Let \(M_1\) and \(M_2\) be two least upper bounds of \(E\).
  Since \(M_1\) is a least upper bound and \(M_2\) is an upper bound, then by definition of least upper bound we have \(M_2 \geq M_1\).
  Since \(M_2\) is a least upper bound and \(M_1\) is an upper bound, we similarly have \(M_1 \geq M_2\).
  Thus \(M_1 = M_2\).
  Thus there is at most one least upper bound.
\end{proof}

\begin{thm}[Existence of least upper bound]\label{i:5.5.9}
  Let \(E\) be a non-empty subset of \(\R\).
  If \(E\) has an upper bound (i.e., \(E\) has some upper bound \(M\)), then it must have exactly one least upper bound.
\end{thm}

\begin{proof}[\pf{i:5.5.9}]
  Let \(E\) be a non-empty subset of \(\R\) with an upper bound \(M\).
  By \cref{i:5.5.8}, we know that \(E\) has at most one least upper bound;
  we have to show that \(E\) has at least one least upper bound.
  Since \(E\) is non-empty, we can choose some element \(x_0\) in \(E\).

  Let \(n \in \Z^+\).
  We know that \(E\) has an upper bound \(M\).
  By the Archimedean property (\cref{i:5.4.13}), we can find a \(K \in \Z^+\) such that \(K / n \geq M\), and hence \(K / n\) is also an upper bound for \(E\).
  (Note that \(K\) is positive, and \(M\) can be either zero or negative, but \(K / n\) is positive, so we are fine.)
  By the Archimedean property again, there exists another \(L \in \Z\) such that \(L / n < x_0\).
  (Note that if \(x_0 \geq 0\), then we can set \(L = -1\); if \(x_0 < 0\), then \(-x_0\) is positive, so by Archimedean property we have some \(-L \in \Z^+\) such that \(-L / n > -x_0\).)
  Since \(x_0\) lies in \(E\), we see that \(L / n\) is not an upper bound for \(E\).
  Since \(K / n\) is an upper bound but \(L / n\) is not, we see that \(K > L\).

  Since \(K / n\) is an upper bound for \(E\) and \(L / n\) is not, we can find an integer \(L < m_n \leq K\) with the property that \(m_n / n\) is an upper bound for \(E\), but \((m_n - 1) / n\) is not (see \cref{i:ex:5.5.2}).
  In fact, this integer \(m_n\) is unique (\cref{i:ex:5.5.3}).
  We subscript \(m_n\) by \(n\) to emphasize the fact that this integer \(m\) depends on the choice of \(n\).
  This gives a well-defined (and unique) sequence \(m_1, m_2, m_3, \dots\) of integers, with each of the \(m_n / n\) being upper bounds and each of the \((m_n - 1) / n\) not being upper bounds.

  Now let \(N \in \Z^+\), and let \(n, n' \in \Z_{\geq N}\).
  Since \(m_n / n\) is an upper bound for \(E\) and \((m_{n'} - 1) / n'\) is not, by \cref{i:5.5.1} we must have \(m_n / n > (m_{n'} - 1) / n'\).
  After a little algebra, this implies that
  \[
    \dfrac{m_n}{n} - \dfrac{m_{n'}}{n'} > -\dfrac{1}{n'} \geq -\dfrac{1}{N}.
  \]
  Similarly, since \(m_{n'} / n'\) is an upper bound for \(E\) and \((m_n - 1) / n\) is not, we have \(m_{n'} / n' > (m_n - 1) / n\), and hence
  \[
    \dfrac{m_n}{n} - \dfrac{m_{n'}}{n'} < \dfrac{1}{n} \leq \dfrac{1}{N}.
  \]
  Putting these two bounds together, we see that
  \[
    \abs{\dfrac{m_n}{n} - \dfrac{m_{n'}}{n'}} \leq \dfrac{1}{N} \text{ for all } n, n' \geq N \geq 1.
  \]
  This implies that \(\dfrac{m_n}{n}\) is a Cauchy sequence (\cref{i:ex:5.5.4}).
  Since the \(\dfrac{m_n}{n}\) are rational numbers, we can now define the real number \(S\) as
  \[
    S \coloneqq \LIM_{n \to \infty} \dfrac{m_n}{n}.
  \]
  From \cref{i:ex:5.3.5} we conclude that
  \[
    S = \LIM_{n \to \infty} \dfrac{m_n - 1}{n}.
  \]
  To finish the proof of the theorem, we need to show that \(S\) is the least upper bound for \(E\).
  First we show that it is an upper bound.
  Let \(x\) be any element of \(E\).
  Then, since \(m_n / n\) is an upper bound for \(E\), we have \(x \leq m_n / n\) for all \(n \in \Z^+\).
  Applying \cref{i:ex:5.4.8}, we conclude that \(x \leq \LIM_{n \to \infty} m_n / n = S\).
  Thus \(S\) is indeed an upper bound for \(E\).

  Now we show it is a least upper bound.
  Suppose \(y\) is an upper bound for \(E\).
  Since \((m_n - 1) / n\) is not an upper bound, we conclude that \(y \geq (m_n - 1) / n\) for all \(n \in \Z^+\).
  Applying \cref{i:ex:5.4.8}, we conclude that \(y \geq \LIM_{n \to \infty} (m_n - 1) / n = S\).
  Thus the upper bound \(S\) is less than or equal to every upper bound of \(E\), and \(S\) is thus a least upper bound of \(E\).
\end{proof}

\begin{defn}[Supremum]\label{i:5.5.10}
  Let \(E\) be a subset of the real numbers.
  If \(E\) is non-empty and has some upper bound, we define \(\sup(E)\) to be the least upper bound of \(E\)
  (this is well-defined by \cref{i:5.5.9}).
  We introduce two additional symbols, \(+\infty\) and \(-\infty\).
  If \(E\) is non-empty and has no upper bound, we set \(\sup(E) \coloneqq +\infty\);
  if \(E\) is empty, we set \(\sup(E) \coloneqq -\infty\).
  We refer to \(\sup(E)\) as the \emph{supremum} of \(E\), and also denote it by \(\sup E\).
\end{defn}

\begin{rmk}\label{i:5.5.11}
  At present, \(+\infty\) and \(-\infty\) are meaningless symbols;
  we have no operations on them at present, and none of our results involving real numbers apply to \(+\infty\) and \(-\infty\), because these are not real numbers.
  In \cref{i:sec:6.2} we add \(+\infty\) and \(-\infty\) to the reals to form the \emph{extended real number system}, but this system is not as convenient to work with as the real number system, because many of the laws of algebra break down.
  For instance, it is not a good idea to try to define \(+\infty + -\infty\);
  setting this equal to \(0\) causes some problems.
\end{rmk}

\begin{prop}\label{i:5.5.12}
  There exists a positive real number \(x\) such that \(x^2 = 2\).
\end{prop}

\begin{proof}[\pf{i:5.5.12}]
  Let \(E\) be the set \(\set{y \in \R : y \geq 0 \text{ and } y^2 < 2}\);
  thus \(E\) is the set of all non-negative real numbers whose square is less than \(2\).
  Observe that \(E\) has an upper bound of \(2\) (because if \(y > 2\), then \(y^2 > 4 > 2\) and hence \(y \notin E\)).
  Also, \(E\) is non-empty (for instance, \(1\) is an element of \(E\)).
  Thus by the least upper bound property (\cref{i:5.5.9}), we have a real number \(x \coloneqq \sup(E)\) which is the least upper bound of \(E\).
  Then \(x\) is greater than or equal to \(1\) (since \(1 \in E\)) and less than or equal to \(2\)
  (since \(2\) is an upper bound for \(E\)).
  So \(x\) is positive.
  Now we show that \(x^2 = 2\).

  We argue this by contradiction.
  We show that both \(x^2 < 2\) and \(x^2 > 2\) lead to contradictions.
  First suppose that \(x^2 < 2\).
  Let \(\varepsilon \in \Q_{0 < 1}\) be a small number;
  then we have
  \[
    (x + \varepsilon)^2 = x^2 + 2\varepsilon x + \varepsilon^2 \leq x^2 + 4\varepsilon + \varepsilon = x^2 + 5\varepsilon
  \]
  since \(x \leq 2\) and \(\varepsilon^2 \leq \varepsilon\).
  Since \(x^2 < 2\), we see that we can use the Archimedean property (\cref{i:5.4.13}) to choose an \(\varepsilon \in \Q_{0 < 1}\) such that \(x^2 + 5\varepsilon < 2\), thus \((x + \varepsilon)^2 < 2\).
  By construction of \(E\), this means that \(x + \varepsilon \in E\);
  but this contradicts the fact that \(x\) is an upper bound of \(E\).

  Now suppose that \(x^2 > 2\).
  Let \(\varepsilon \in \Q_{0 < 1}\) be a small number;
  then we have
  \[
    (x - \varepsilon)^2 = x^2 - 2\varepsilon x + \varepsilon^2 \geq x^2 - 2\varepsilon x \geq x^2 - 4\varepsilon
  \]
  since \(x \leq 2\) and \(\varepsilon^2 \geq 0\).
  Since \(x^2 > 2\), we can choose \(\varepsilon \in \Q_{0 < 1}\) such that \(x^2 - 4\varepsilon > 2\), and thus \((x - \varepsilon)^2 > 2\).
  But then this implies that \(x - \varepsilon \geq y\) for all \(y \in E\).
  (Why? If \(x - \varepsilon < y\) then \((x - \varepsilon)^2 < y^2 \leq 2\), a contradiction.)
  Thus \(x - \varepsilon\) is an upper bound for \(E\), which contradicts the fact that \(x\) is the \emph{least} upper bound of \(E\).
  From these two contradictions we see that \(x^2 = 2\), as desired.
\end{proof}

\begin{rmk}\label{i:5.5.13}
  Comparing \cref{i:5.5.12} with \cref{i:4.4.4}, we see that certain numbers are real but not rational.
  The proof of \cref{i:5.5.12} also shows that the rationals \(\Q\) do not obey the least upper bound property, otherwise one could use that property to construct a square root of \(2\), which by \cref{i:4.4.4} is not possible.
\end{rmk}

\begin{rmk}\label{i:5.5.14}
  In \cref{i:ch:6} we will use the least upper bound property to develop the theory of limits, which allows us to do many more things than just take square roots.
\end{rmk}

\begin{rmk}\label{i:5.5.15}
  We can of course talk about lower bounds, and greatest lower bounds, of sets \(E\);
  the greatest lower bound of a set \(E\) is also known as the \emph{infimum} of \(E\) and is denoted \(\inf(E)\) or \(\inf E\).
  Everything we say about suprema has a counterpart for infima;
  A precise relationship between the two notions is given by \cref{i:ex:5.5.1}.
  See also \cref{i:sec:6.2}.
\end{rmk}

\begin{note}
  Supremum means ``highest'' and infimum means ``lowest'', and the plurals are suprema and infima.
  Supremum is to superior, and infimum to inferior, as maximum is to major, and minimum to minor.
  The root words are ``super'', which means ``above'', and ``infer'', which means ``below''
  (this usage only survives in a few rare English words such as ``infernal'', with the Latin prefix ``sub'' having mostly replaced ``infer'' in English).
\end{note}

\exercisesection

\begin{ex}\label{i:ex:5.5.1}
  Let \(E\) be a subset of the real numbers \(\R\), and suppose that \(E\) has a least upper bound \(M\) which is a real number, i.e., \(M = \sup(E)\).
  Let \(-E\) be the set
  \[
    -E \coloneqq \set{-x : x \in E}.
  \]
  Show that \(-M\) is the greatest lower bound of \(-E\), i.e., \(-M = \inf(-E)\).
\end{ex}

\begin{proof}[\pf{i:ex:5.5.1}]
  We first show that \(-M\) is a lower bound for \(-E\).
  This is true since
  \begin{align*}
             & \forall x \in E, x \leq M           &  & \by{i:5.5.1}  \\
    \implies & \forall x \in E, -x \geq -M         &  & \by{i:5.4.7}  \\
    \implies & \forall -x \in -E, -x \geq -M                          \\
    \implies & -M \text{ is a lower bound of } -E. &  & \by{i:5.5.15}
  \end{align*}

  Next we show that \(-M\) is a greatest lower bound for \(-E\).
  Let \(L \in \R\) be any lower bound for \(-E\).
  Then we have
  \begin{align*}
             & \forall x \in E, L \leq -x                                      \\
    \implies & \forall x \in E, x \leq -L                   &  & \by{i:5.4.7}  \\
    \implies & -L \text{ is an upper bound of } E           &  & \by{i:5.5.1}  \\
    \implies & M \leq -L                                    &  & \by{i:5.5.5}  \\
    \implies & -M \geq L                                    &  & \by{i:5.4.7}  \\
    \implies & -M \text{ is a greatest lower bound of } -E. &  & \by{i:5.5.15}
  \end{align*}

  Now we show that the greatest lower bound is unique.
  Let \(M, M'\) be two greatest lower bounds of \(-E\).
  Then we have \(M \leq M'\) and \(M \geq M'\), which means \(M = M'\).
  So the greatest lower bound is unique.
\end{proof}

\begin{ex}\label{i:ex:5.5.2}
  Let \(E\) be a non-empty subset of \(\R\), let \(n \in \Z^+\), and let \(L < K\) be integers.
  Suppose that \(K / n\) is an upper bound for \(E\), but that \(L / n\) is not an upper bound for \(E\).
  Without using \cref{i:5.5.9}, show that there exists an integer \(L < m \leq K\) such that \(m / n\) is an upper bound for \(E\), but that \((m - 1) / n\) is not an upper bound for \(E\).
\end{ex}

\begin{proof}[\pf{i:ex:5.5.2}]
  Let \(d = K - L\), so \(d\) is positive by \cref{i:4.1.11}(a).
  Now we induct on \(d\) to show that for every \(d \in \Z^+\), there exists an \(m \in \Z_{> L} \cap \Z_{\leq K}\) such that \(m / n\) is an upper bound for \(E\), but that \((m - 1) / n\) is not.
  We start with \(d = 1\).
  For \(d = 1\), we have \(K - 1 = L\).
  Then let \(m = K\).
  So by hypothesis we have \(m = K \in \Z_{> L} \cap \Z_{\leq K}\), \(m / n = K / n\) is an upper bound for \(E\), and \((m - 1) / n = (K - 1) / n = L / n\) is not an upper bound for \(E\).
  Thus the base case holds.
  Suppose inductively that the statement holds for some \(d \in \Z^+\).
  Now we show that for \(d + 1\) the statement is also true.
  So suppose that for some \(K, L \in \Z\), we have \(K - L = d + 1\), \(K / n\) is an upper bound of \(E\), but \(L / n\) is not.
  Since \(K / n\) is an upper bound for \(E\), we can ask whether \((K - 1) / n\) is an upper bound for \(E\).
  \begin{itemize}
    \item If \((K - 1) / n\) is not an upper bound for \(E\), then we can choose \(m = K\) and we are done.
    \item If \((K - 1) / n\) is an upper bound for \(E\), then by \cref{i:5.5.1} we have \(L < K - 1\).
          Since \(K - 1 - L = d\), by induction hypothesis we know that there exists an \(m \in \Z_{> L} \cap \Z_{\leq K}\) such that \(m / n\) is an upper bound for \(E\), but \((m - 1) / n\) is not.
  \end{itemize}
  From all cases above we found an \(m \in \Z_{> L} \cap \Z_{\leq K}\) such that \(m / n\) is an upper bound for \(E\), but \((m - 1) / n\) is not.
  This closes the induction.
\end{proof}

\begin{ex}\label{i:ex:5.5.3}
  Let \(E\) be a non-empty subset of \(\R\), let \(n \in \Z^+\), and let \(m, m' \in \Z\) with the properties that \(m / n\) and \(m' / n\) are upper bounds for \(E\), but \((m - 1) / n\) and \((m' - 1) / n\) are not upper bounds for \(E\).
  Show that \(m = m'\).
  This shows that the integer \(m\) constructed in \cref{i:ex:5.5.2} is unique.
\end{ex}

\begin{proof}[\pf{i:ex:5.5.3}]
  Suppose for sake of contradiction that \(m \neq m'\).
  Then by \cref{i:4.1.11}(f) we have either \(m < m'\) or \(m > m'\).
  \begin{itemize}
    \item If \(m < m'\), then we have \(m \leq m' - 1\).
          By \cref{i:4.2.9}(e) we have \(m / n \leq (m' - 1) / n\), which means \((m' - 1) / n\) is an upper bound for \(E\).
          But this contradict to the hypothesis.
    \item If \(m > m'\), then by switching the row of \(m\) and \(m'\) in the previous case we can also derive contradiction.
  \end{itemize}
  From all cases above we derive contradictions.
  Thus we must have \(m = m'\).
\end{proof}

\begin{ex}\label{i:ex:5.5.4}
  Let \((q_n)_{n = 1}^\infty\) be a sequence of rational numbers with the property that \(\abs{q_n - q_{n'}} \leq \dfrac{1}{M}\) whenever \(M \in \Z^+\) and \(n, n' \in \Z_{\geq M}\).
  Show that \((q_n)_{n = 1}^\infty\) is a Cauchy sequence.
  Furthermore, if \(S \coloneqq \LIM_{n \to \infty} q_n\), show that \(\abs{q_M - S} \leq \dfrac{1}{M}\) for every \(M \in \Z^+\).
\end{ex}

\begin{proof}[\pf{i:ex:5.5.4}]
  We first show that \((q_n)_{n = 1}^\infty\) is a Cauchy sequence.
  Let \(\varepsilon \in \Q^+\).
  By Archimedean property (\cref{i:5.4.13}) we know that there exists an \(M \in \Z^+\) such that \(M \varepsilon > 1\).
  By \cref{i:4.2.9}(e) we have \(\varepsilon > 1 / M\).
  But by hypothesis we have \(\abs{q_n - q_{n'}} \leq \dfrac{1}{M} < \varepsilon\) for all \(n, n' \in \Z_{\geq M}\).
  Since \(\varepsilon\) was arbitrary, by \cref{i:5.1.8} this means \((q_n)_{n = 1}^{\infty}\) is a Cauchy sequence.

  Now we show that if \(S = \LIM_{n \to \infty} q_n\), then \(\abs{q_M - S} \leq \dfrac{1}{M}\) for every \(M \in \Z^+\).
  From the proof above we know that \((q_n)_{n = 1}^\infty\) is a Cauchy sequence, thus \(S = \LIM_{n \to \infty} q_n\) is well-defined.
  By hypothesis we have \(\abs{q_M - q_n} \leq 1 / M\) for all \(M \in \Z^+\) and for all \(n \geq M\).
  Then we have
  \begin{align*}
             & \forall M \in \Z^+, \forall n \in \Z_{\geq M}, \abs{q_M - q_n} \leq 1 / M                                  \\
    \implies & \forall M \in \Z^+, \forall n \in \Z_{\geq M}, \abs{q_n - q_M} \leq 1 / M             &  & \by{i:4.3.1}    \\
    \implies & \forall M \in \Z^+, \forall n \in \Z_{\geq M}, -1 / M \leq q_n - q_M \leq 1 / M       &  & \by{i:4.3.3}[c] \\
    \implies & \forall M \in \Z^+, \forall n \in \Z_{\geq M}, -1 / M + q_M \leq q_n \leq 1 / M + q_M &  & \by{i:4.2.9}[d] \\
    \implies & \forall M \in \Z^+, -1 / M + q_M \leq S \leq 1 / M + q_M                              &  & \by{i:ex:5.4.8} \\
    \implies & \forall M \in \Z^+, -1 / M \leq S - q_M \leq 1 / M                                    &  & \by{i:5.4.7}[d] \\
    \implies & \forall M \in \Z^+, \abs{S - q_M} \leq 1 / M                                          &  & \by{i:ex:5.4.6} \\
    \implies & \forall M \in \Z^+, \abs{q_M - S} \leq 1 / M.                                         &  & \by{i:5.4.5}
  \end{align*}
\end{proof}

\begin{ex}\label{i:ex:5.5.5}
  Establish an analogue of \cref{i:5.4.14}, in which ``rational'' is replaced by ``irrational.''
\end{ex}

\begin{proof}[\pf{i:ex:5.5.5}]
  Let \(x, y, z \in \R\) where \(x < y\) and \(z^2 = 2\).
  (\(z\) is well-defined thanks to \cref{i:5.5.12}.)
  So by \cref{i:5.4.7} we have \(x - z < y - z\).
  But by \cref{i:5.4.14} there exists a \(q \in \Q\) such that \(x - z < q < y - z\).
  So by \cref{i:5.4.7} again we have \(x < q + z < y\).
  Because \(z\) is irrational, \(q + z\) is also irrational
  (otherwised we have \(a = q + z \in \Q\) and \(z = a - q \in \Q\), contradicts to \cref{i:4.4.4}).
  So we have an irrational number in between any two real numbers \(x, y\) where \(x < y\).
\end{proof}

\section{Real exponentiation, part I}\label{i:sec:5.6}

\begin{defn}[Exponentiating a real by a natural number]\label{i:5.6.1}
  Let \(x\) be a real number.
  To raise \(x\) to the power \(0\), we define \(x^0 \coloneqq 1\).
  Now suppose recursively that \(x^n\) has been defined for some natural number \(n\), then we define \(x^{n + 1} \coloneqq x^n \times x\).
\end{defn}

\begin{defn}[Exponentiating a real by an integer]\label{i:5.6.2}
  Let \(x\) be a non-zero real number.
  Then for any negative integer \(-n\), we define \(x^{-n} \coloneqq 1 / x^n\).
\end{defn}

\begin{prop}\label{i:5.6.3}
  All the properties in \cref{i:4.3.10,i:4.3.12} remain valid if \(x\) and \(y\) are assumed to be real numbers instead of rational numbers.
\end{prop}

\begin{meta-proof}[\pf{i:5.6.3}]
If one inspects the proof of \cref{i:4.3.10,i:4.3.12} we see that they rely on the laws of algebra and the laws of order for the rationals (\cref{i:4.2.4,i:4.2.9}).
But by \cref{i:5.3.11,i:5.4.7}, and the identity \(xx^{-1} = x^{-1} x = 1\) we know that all these laws of algebra and order continue to hold for real numbers as well as rationals.
Thus we can modify the proof of \cref{i:4.3.10,i:4.3.12} to hold in the case when \(x\) and \(y\) are real.
\end{meta-proof}

\begin{note}
  Instead of giving an actual proof of \cref{i:5.6.3}, we shall give a meta-proof
  (an argument appealing to the nature of proofs, rather than the nature of real and rational numbers).
\end{note}

\begin{defn}\label{i:5.6.4}
  Let \(x \geq 0\) be a non-negative real, and let \(n \geq 1\) be a positive integer.
  We define \(x^{1 / n}\), also known as the \emph{\(n^{\opTh}\) root of \(x\)}, by the formula
  \[
    x^{1 / n} \coloneqq \sup\set{y \in \R : y \geq 0 \text{ and } y^n \leq x}.
  \]
  We often write \(\sqrt{x}\) for \(x^{1 / 2}\).
\end{defn}

\begin{note}
  we do not define the \(n^{\opTh}\) roots of a negative number.
  In fact, we will leave the \(n^{\opTh}\) roots of negative numbers undefined for the rest of the text
  (one can define these \(n^{\opTh}\) roots once one defines the complex numbers, but we shall refrain from doing so).
\end{note}

\begin{lem}[Existence of \(n^{\opTh}\) roots]\label{i:5.6.5}
  Let \(x \geq 0\) be a non-negative real, and let \(n \geq 1\) be a positive integer.
  Then the set \(E \coloneqq \set{y \in R : y \geq 0 \text{ and } y^n \leq x}\) is non-empty and is also bounded above.
  In particular, \(x^{1 / n}\) is a real number.
\end{lem}

\begin{proof}[\pf{i:5.6.5}]
  The set \(E\) contains \(0\), so it is certainly not empty.
  Now we show it has an upper bound.
  We divide into two cases: \(x \leq 1\) and \(x > 1\).
  First suppose that we are in the case where \(x \leq 1\).
  Then we claim that the set \(E\) is bounded above by \(1\).
  To see this, suppose for sake of contradiction that there was an element \(y \in E\) for which \(y > 1\).
  But then \(y^n > 1\), and hence \(y^n > x\), a contradiction.
  Thus \(E\) has an upper bound.
  Now suppose that we are in the case where \(x > 1\).
  Then we claim that the set \(E\) is bounded above by \(x\).
  To see this, suppose for contradiction that there was an element \(y \in E\) for which \(y > x\).
  Since \(x > 1\), we thus have \(y > 1\).
  Since \(y > x\) and \(y > 1\), we have \(y^n > x\), a contradiction.
  Thus in both cases \(E\) has an upper bound, and so \(x^{1 / n}\) is finite.
\end{proof}

\begin{lem}\label{i:5.6.6}
  Let \(x, y \geq 0\) be non-negative reals, and let \(n, m \geq 1\) be positive integers.
  \begin{enumerate}
    \item If \(y = x^{1 / n}\), then \(y^n = x\).
    \item Conversely, if \(y^n = x\), then \(y = x^{1 / n}\).
    \item \(x^{1 / n}\) is a non-negative real number, and is positive iff \(x\) is positive.
    \item We have \(x > y\) iff \(x^{1 / n} > y^{1 / n}\).
    \item Let \(k, l \in \Z^+\).
          If \(x > 1\), then \(x^{1 / k}\) is a decreasing (i.e., \(x^{1 / k} > x^{1 / l}\) whenever \(k < l\)) function of \(k\).
          If \(0 < x < 1\), then \(x^{1 / k}\) is an increasing (i.e., \(x^{1 / k} < x^{1 / l}\) whenever \(k < l\)) function of \(k\).
          If \(x = 1\), then \(x^{1 / k} = 1\) for all \(k\).
    \item We have \((xy)^{1 / n} = x^{1 / n} y^{1 / n}\).
    \item We have \((x^{1 / n})^{1 / m} = x^{1 / nm}\).
  \end{enumerate}
\end{lem}

\begin{proof}[\pf{i:5.6.6}(a)]
  Let \(E = \set{z \in \R : (z \geq 0) \land (z^n \leq x)}\) and let \(y = x^{1 / n} = \sup(E)\).
  Since \(0 \in E\), by \cref{i:5.5.5} we know that \(0 \leq y\).
  Suppose for sake of contradiction that \(y^n \neq x\).
  Then by \cref{i:5.4.7} exactly one of the following statements is true:
  \begin{itemize}
    \item \(y^n < x\).
          Now we show that \(\exists m \in \Z^+\) such that \((y + \dfrac{1}{m})^n < x\).
          Suppose for sake of contradiction that \(\forall m \in \Z^+\), we have \((y + \dfrac{1}{m})^n \geq x\).
          Let \((y + \dfrac{1}{m})_{m = 1}^\infty\) be a sequences.
          Then we have
          \begin{align*}
            \LIM_{m \to \infty} \bigg(y + \dfrac{1}{m}\bigg) & = \LIM_{m \to \infty} y + \LIM_{m \to \infty} \dfrac{1}{m} &  & \by{i:5.3.4}    \\
                                                             & = \LIM_{m \to \infty} y + 0                                &  & \by{i:ex:5.3.5} \\
                                                             & = y.                                                       &  & \by{i:6.1.15}
          \end{align*}
          Note that we use \cref{i:6.1.15} without circularity.
          But then we have
          \begin{align*}
                     & \forall m \in \Z^+, y^n < x \leq \bigg(y + \dfrac{1}{m}\bigg)^n                                  \\
            \implies & y^n < x \leq \LIM_{m \to \infty} \bigg(y + \dfrac{1}{m}\bigg)^n             &  & \by{i:ex:5.4.8} \\
            \implies & y^n < x \leq \Bigg(\LIM_{m \to \infty} \bigg(y + \dfrac{1}{m}\bigg)\Bigg)^n &  & \by{i:5.3.9}    \\
            \implies & y^n < x \leq y^n                                                                                 \\
            \implies & y^n < y^n,
          \end{align*}
          a contradiction.
          Thus we know that \(\exists m \in \Z^+\) such that \((y + \dfrac{1}{m})^n < x\).
          Since \(m \in \Z^+\), we know that \(y < y + \dfrac{1}{m}\), thus by \cref{i:5.6.3} we know that \(y^n < (y + \dfrac{1}{m})^n\).
          But this means \(y + \dfrac{1}{m} \in E\) and \(y + \dfrac{1}{m} \leq y\), a contradiction.
    \item \(y^n > x\).
          Now we show that \(\exists m \in \Z^+\) such that \((y - \dfrac{1}{m})^n > x\).
          Suppose for sake of contradiction that \(\forall m \in \Z^+\), we have \((y - \dfrac{1}{m})^n \leq x\).
          Let \((y - \dfrac{1}{m})_{m = 1}^\infty\) be a sequences.
          Then we have
          \begin{align*}
            \LIM_{m \to \infty} \bigg(y - \dfrac{1}{m}\bigg) & = \LIM_{m \to \infty} y - \LIM_{m \to \infty} \dfrac{1}{m} &  & \by{i:5.3.4}    \\
                                                             & = \LIM_{m \to \infty} y - 0                                &  & \by{i:ex:5.3.5} \\
                                                             & = y.                                                       &  & \by{i:6.1.15}
          \end{align*}
          Note that we use \cref{i:6.1.15} without circularity.
          But then we have
          \begin{align*}
                     & \forall m \in \Z^+, \bigg(y - \dfrac{1}{m}\bigg)^n \leq x < y^n                                  \\
            \implies & \LIM_{m \to \infty} \bigg(y - \dfrac{1}{m}\bigg)^n \leq x < y^n             &  & \by{i:ex:5.4.8} \\
            \implies & \Bigg(\LIM_{m \to \infty} \bigg(y - \dfrac{1}{m}\bigg)\Bigg)^n \leq x < y^n &  & \by{i:5.3.9}    \\
            \implies & y^n \leq x < y^n                                                                                 \\
            \implies & y^n < y^n,
          \end{align*}
          a contradiction.
          Thus we know that \(\exists m \in \Z^+\) such that \((y - \dfrac{1}{m})^n > x\).
          Since \(m \in \Z^+\), we know that \(y - \dfrac{1}{m} < y\), thus by \cref{i:5.6.3} we know that \((y - \dfrac{1}{m})^n < y^n\).
          But this means \(y - \dfrac{1}{m} \notin E\) and \(y - \dfrac{1}{m}\) and upper bound for \(E\) which is strictly less than \(y\), a contradiction.
  \end{itemize}
  From all cases above we get contradictions, so \(y = x^{1 / n} \implies y^n = x\).
\end{proof}

\begin{proof}[\pf{i:5.6.6}(b)]
  Let \(E = \set{z \in \R : (z \geq 0) \land (z^n \leq x)}\).
  By \cref{i:5.6.4} we have \(x^{1 / n} = \sup(E)\).
  Since \(0 \in E\), by \cref{i:5.5.5} we know that \(0 \leq x^{1 / n}\).
  Let \(y \in \R \setminus \R^-\) such that \(y^n = x\).
  Such \(y\) is well-defined since \cref{i:5.6.6}(a).
  By the definition of \(E\) we know that \(y \in E\).
  Suppose for sake of contradiction that \(y \neq x^{1 / n}\).
  Then by \cref{i:5.4.7} exactly one of the following statements is true:
  \begin{itemize}
    \item \(y < x^{1 / n}\).
          But then we have
          \begin{align*}
                     & y^n < (x^{1 / n})^n &  & \by{i:5.6.3}    \\
            \implies & x < (x^{1 / n})^n                        \\
            \implies & x < x,              &  & \by{i:5.6.6}[a]
          \end{align*}
          a contradiction.
    \item \(y > x^{1 / n}\).
          But then we have
          \begin{align*}
                     & y^n > (x^{1 / n})^n &  & \by{i:5.6.3}    \\
            \implies & x > (x^{1 / n})^n                        \\
            \implies & x > x,              &  & \by{i:5.6.6}[a]
          \end{align*}
          a contradiction.
  \end{itemize}
  From all cases above we get contradictions, so \(y^n = x \implies y = x^{1 / n}\).
\end{proof}

\begin{proof}[\pf{i:5.6.6}(c)]
  Let \(E = \set{z \in \R : (z \geq 0) \land (z^n \leq x)}\).
  By \cref{i:5.6.4} we have \(x^{1 / n} = \sup(E)\).
  Since \(0 \in E\), by \cref{i:5.5.5} we know that \(0 \leq x^{1 / n}\), thus \(x^{1 / n}\) is non-negative real number.

  Now suppose that \(x^{1 / n}\) is positive.
  Then we have
  \begin{align*}
             & x^{1 / n} > 0                          \\
    \implies & (x^{1 / n})^n > 0 &  & \by{i:5.6.3}    \\
    \implies & x > 0.            &  & \by{i:5.6.6}[a] \\
  \end{align*}

  Finally, suppose that \(x\) is positive.
  Suppose for sake of contradiction that \(x^{1 / n}\) is not positive.
  Then from proof above we know that \(x^{1 / n} = 0\).
  But then we have
  \begin{align*}
             & (x^{1 / n})^n = 0^n = 0                      \\
    \implies & x = 0,                  &  & \by{i:5.6.6}[a]
  \end{align*}
  a contradiction.
  Thus we must have \(x^{1 / n} > 0\).
  And we conclude that \(x^{1 / n}\) is positive iff \(x\) is positive.
\end{proof}

\begin{proof}[\pf{i:5.6.6}(d)]
  We first show that \(x^{1 / n} > y^{1 / n} \implies x > y\).
  \begin{align*}
             & x^{1 / n} > y^{1 / n}                              \\
    \implies & (x^{1 / n})^n > (y^{1 / n})^n &  & \by{i:5.6.3}    \\
    \implies & x > y.                        &  & \by{i:5.6.6}[a]
  \end{align*}

  Now we show that \(x > y \implies x^{1 / n} > y^{1 / n}\).
  Suppose for sake of contradiction that \(x^{1 / n} \leq y^{1 / n}\).
  But then we have
  \begin{align*}
             & x^{1 / n} \leq y^{1 / n}                              \\
    \implies & (x^{1 / n})^n \leq (y^{1 / n})^n &  & \by{i:5.6.3}    \\
    \implies & x \leq y,                        &  & \by{i:5.6.6}[a]
  \end{align*}
  a contradiction.
  Thus we must have \(x^{1 / n} > y^{1 / n}\).
  And we conclude that \(x > y \iff x^{1 / n} > y^{1 / n}\).
\end{proof}

\begin{proof}[\pf{i:5.6.6}(e)]
  If \(x = 0\), then by \cref{i:5.6.6}(c) we know that \(x^{1 / k} = 0\) for every \(k \in \Z^+\).
  Thus we only consider the case \(x \in \R^+\).

  We first show that if \(x > 1\), then \(x^{1 / k}\) is a decreasing function of \(k\).
  Let \(f : \Z^+ \to \R\) be a function such that \(f(k) = x^{1 / k}\).
  Such function \(f\) is well-defined since \cref{i:5.6.5}.
  Now we want to show that \(x^{1 / k} > x^{1 / (k + 1)}\).
  Suppose for sake of contradiction that \(x^{1 / k} \leq x^{1 / (k + 1)}\).
  But then we have
  \begin{align*}
             & x^{\dfrac{1}{k}} \leq x^{\dfrac{1}{k + 1}}                                                                                                              \\
    \implies & (x^{\dfrac{1}{k}})^k \leq (x^{\dfrac{1}{k + 1}})^k                                                                               &  & \by{i:5.6.3}      \\
    \implies & x \leq (x^{\dfrac{1}{k + 1}})^k                                                                                                  &  & \by{i:5.6.6}[a]   \\
    \implies & (x^{\dfrac{1}{k + 1}})^{k + 1} \leq (x^{\dfrac{1}{k + 1}})^k                                                                     &  & \by{i:5.6.6}[a,b] \\
    \implies & (x^{\dfrac{1}{k + 1}})^{k + 1} \cdot (x^{\dfrac{1}{k + 1}})^{-1} \leq (x^{\dfrac{1}{k + 1}})^k \cdot (x^{\dfrac{1}{k + 1}})^{-1} &  & \by{i:5.6.6}[c]   \\
    \implies & (x^{\dfrac{1}{k + 1}})^{k + 1} \cdot (x^{\dfrac{1}{k + 1}})^{-k} \leq (x^{\dfrac{1}{k + 1}})^k \cdot (x^{\dfrac{1}{k + 1}})^{-k} &  & \by{i:5.6.3}      \\
    \implies & x^{\dfrac{1}{k + 1}} \leq 1                                                                                                      &  & \by{i:5.6.3}      \\
    \implies & (x^{\dfrac{1}{k + 1}})^{k + 1} \leq 1^{k + 1} = 1                                                                                &  & \by{i:5.6.3}      \\
    \implies & x \leq 1,                                                                                                                        &  & \by{i:5.6.6}[a]
  \end{align*}
  a contradiction.
  Thus we must have \(f(k) = x^{1 / k} > x^{1 / (k + 1)} = f(k + 1)\), and \(f(k) = x^{1 / k}\) is a decreasing function of \(k\).

  Next we show that if \(x < 1\), then \(x^{1 / k}\) is a increasing function of \(k\).
  Let \(f : \Z^+ \to \R\) be a function such that \(f(k) = x^{1 / k}\).
  Such function \(f\) is well-defined since \cref{i:5.6.5}.
  Now we want to show that \(x^{1 / k} < x^{1 / (k + 1)}\).
  Suppose for sake of contradiction that \(x^{1 / k} \geq x^{1 / (k + 1)}\).
  But then we have
  \begin{align*}
             & x^{\dfrac{1}{k}} \geq x^{\dfrac{1}{k + 1}}                                                                                                              \\
    \implies & (x^{\dfrac{1}{k}})^k \geq (x^{\dfrac{1}{k + 1}})^k                                                                               &  & \by{i:5.6.3}      \\
    \implies & x \geq (x^{\dfrac{1}{k + 1}})^k                                                                                                  &  & \by{i:5.6.6}[a]   \\
    \implies & (x^{\dfrac{1}{k + 1}})^{k + 1} \geq (x^{\dfrac{1}{k + 1}})^k                                                                     &  & \by{i:5.6.6}[a,b] \\
    \implies & (x^{\dfrac{1}{k + 1}})^{k + 1} \cdot (x^{\dfrac{1}{k + 1}})^{-1} \geq (x^{\dfrac{1}{k + 1}})^k \cdot (x^{\dfrac{1}{k + 1}})^{-1} &  & \by{i:5.6.6}[c]   \\
    \implies & (x^{\dfrac{1}{k + 1}})^{k + 1} \cdot (x^{\dfrac{1}{k + 1}})^{-k} \geq (x^{\dfrac{1}{k + 1}})^k \cdot (x^{\dfrac{1}{k + 1}})^{-k} &  & \by{i:5.6.3}      \\
    \implies & x^{\dfrac{1}{k + 1}} \geq 1                                                                                                      &  & \by{i:5.6.3}      \\
    \implies & (x^{\dfrac{1}{k + 1}})^{k + 1} \geq 1^{k + 1} = 1                                                                                &  & \by{i:5.6.3}      \\
    \implies & x \geq 1,                                                                                                                        &  & \by{i:5.6.6}[a]
  \end{align*}
  a contradiction.
  Thus we must have \(f(k) = x^{1 / k} < x^{1 / (k + 1)} = f(k + 1)\), and \(f(k) = x^{1 / k}\) is a increasing function of \(k\).

  Finally we show that if \(x = 1\), then \(x^{1 / k} = 1\) for every \(k \in \Z^+\).
  Suppose for sake of contradiction that \(x^{1 / k} \neq 1\).
  Then by \cref{i:5.4.7} exactly one of the following two statements is true:
  \begin{itemize}
    \item \(x^{1 / k} > 1\).
          But then we have
          \begin{align*}
                     & (x^{1 / k})^k > 1^k = 1 &  & \by{i:5.6.3}    \\
            \implies & x > 1,                  &  & \by{i:5.6.6}[a]
          \end{align*}
          a contradiction.
    \item \(x^{1 / k} < 1\).
          But then we have
          \begin{align*}
                     & (x^{1 / k})^k < 1^k = 1 &  & \by{i:5.6.3}    \\
            \implies & x < 1,                  &  & \by{i:5.6.6}[a]
          \end{align*}
          a contradiction.
  \end{itemize}
  From all cases above we get contradictions.
  Thus we must have \(x^{1 / k} = 1\).
\end{proof}

\begin{proof}[\pf{i:5.6.6}(f)]
  We have
  \begin{align*}
    ((xy)^{1 / n})^n & = xy                          &  & \by{i:5.6.6}[a]   \\
                     & = (x^{1 / n})^n (y^{1 / n})^n &  & \by{i:5.6.6}[a,b] \\
                     & = (x^{1 / n} y^{1 / n})^n     &  & \by{i:5.6.3}
  \end{align*}
  and thus by \cref{i:5.6.6}(b) we have \((xy)^{1 / n} = x^{1 / n} y^{1 / n}\).
\end{proof}

\begin{proof}[\pf{i:5.6.6}(g)]
  We have
  \begin{align*}
    (x^{1 / nm})^{nm} & = x                                           &  & \by{i:5.6.6}[a]   \\
                      & = (x^{1 / n})^n                               &  & \by{i:5.6.6}[a,b] \\
                      & = \Big(\big((x^{1 / n})^{1 / m}\big)^m\Big)^n &  & \by{i:5.6.6}[a,b] \\
                      & = \big((x^{1 / n})^{1 / m}\big)^{nm}          &  & \by{i:5.6.3}
  \end{align*}
  and thus by \cref{i:5.6.6}(b) we have \(x^{1 / nm} = (x^{1 / n})^{1 / m}\).
\end{proof}

\begin{note}
  The observant reader may note that this definition of \(x^{1 / n}\) might possibly be inconsistent with our previous notion of \(x^n\) when \(n = 1\), but it is easy to check that \(x^{1 / 1} = x = x^1\) by using \cref{i:5.6.6}, so there is no inconsistency.
\end{note}

\begin{note}
  One consequence of \cref{i:5.6.6}(b) is another proof of the cancellation law from \cref{i:4.3.12}(c) and \cref{i:5.6.3}:
  if \(y\) and \(z\) are positive and \(y^n = z^n\), then \(y = z\).
  This only works when \(y\) and \(z\) are positive;
  for instance, \((-3)^2 = 3^2\), but we cannot conclude from this that \(-3 = 3\).
\end{note}

\begin{defn}\label{i:5.6.7}
  Let \(x > 0\) be a positive real number, and let \(q\) be a rational number.
  To define \(x^q\), we write \(q = a / b\) for some integer \(a\) and positive integer \(b\), and define
  \[
    x^q \coloneqq (x^{1 / b})^a.
  \]
\end{defn}

\begin{note}
  Every rational \(q\), whether positive, negative, or zero, can be written in the form \(a / b\) where \(a\) is an integer and \(b\) is positive.
  However, the rational number \(q\) can be expressed in the form \(a / b\) in more than one way, for instance \(1 / 2\) can also be expressed as \(2 / 4\) or \(3 / 6\).
  So to ensure that \cref{i:5.6.7} is well-defined, we need to check that different expressions \(a / b\) give the same formula for \(x^q\).
\end{note}

\begin{lem}\label{i:5.6.8}
  Let \(a, a'\) be integers and \(b, b'\) be positive integers such that \(a / b = a' / b'\), and let \(x\) be a positive real number.
  Then we have \((x^{1 / b'})^{a'} = (x^{1 / b})^a\).
\end{lem}

\begin{proof}[\pf{i:5.6.8}]
  There are three cases: \(a = 0, a > 0, a < 0\).
  If \(a = 0\), then we must have \(a' = 0\) and so both \((x^{1 / b'})^{a'}\) and \((x^{1 / b})^a\) are equal to 1, so we are done.

  Now suppose that \(a > 0\).
  Then \(a' > 0\), and \(ab' = ba'\).
  Write \(y \coloneqq x^{1 / (ab')} = x^{1 / (ba')}\).
  By \cref{i:5.6.6}(g) we have \(y = (x^{1 / b'})^{1 / a}\) and \(y = (x^{1 / b})^{1 / a'}\);
  by \cref{i:5.6.6}(a) we thus have \(y^{a'} = x^{1 / b}\) and \(y^a = x^{1 / b'}\).
  Thus we have
  \[
    (x^{1 / b'})^{a'} = (y^a)^{a'} = y^{aa'} = (y^{a'})^a = (x^{1 / b})^a
  \]
  as desired.

  Finally, suppose that \(a < 0\).
  Then we have \((-a) / b = (-a') / b'\).
  But \(-a\) is positive, so the previous case applies and we have \((x^{1 / b'})^{-a'} = (x^{1 / b})^{-a}\).
  Taking the reciprocal of both sides we obtain the result.
\end{proof}

\begin{note}
  Thus \(x^q\) is well-defined for every rational \(q\).
  \cref{i:5.6.7} is consistent with our old definition for \(x^{1 / n}\) (since \(x^{1 / n} = (x^{1 / n})^1\)) and is also consistent with our old definition for \(x^n\) (since \(x^n = (x^{1 / 1})^n\)).
\end{note}

\begin{lem}\label{i:5.6.9}
  Let \(x, y > 0\) be positive reals, and let \(q, r\) be rationals.
  \begin{enumerate}
    \item \(x^q\) is a positive real.
    \item \(x^{q + r} = x^q x^r\) and \((x^q)^r = x^{qr}\).
    \item \(x^{-q} = 1 / x^q\).
    \item If \(q > 0\), then \(x > y\) iff \(x^q > y^q\).
    \item If \(x > 1\), then \(x^q > x^r\) iff \(q > r\).
          If \(x < 1\), then \(x^q > x^r\) iff \(q < r\).
    \item \((xy)^q = x^q y^q\).
  \end{enumerate}
\end{lem}

\begin{proof}[\pf{i:5.6.9}(a)]
  Let \(q = a / b\) where \(a \in \Z\) and \(b \in \Z^+\).
  Then we have
  \begin{align*}
             & x \in \R^+                                  \\
    \implies & x^{1 / b} \in \R^+     &  & \by{i:5.6.6}[c] \\
    \implies & (x^{1 / b})^a \in \R^+ &  & \by{i:5.6.3}    \\
    \implies & x^q \in \R^+.          &  & \by{i:5.6.7}
  \end{align*}
\end{proof}

\begin{proof}[\pf{i:5.6.9}(b)]
  Let \(q = a / b\) and \(r = c / d\) where \(a, c \in \Z\) and \(b, d \in \Z^+\).
  Then we have
  \begin{align*}
    x^{q + r} & = x^{(ad + bc) / bd}                  &  & \by{i:4.2.2}                \\
              & = (x^{1 / bd})^{(ad + bc)}            &  & \by{i:5.6.7}                \\
              & = (x^{1 / bd})^{ad} (x^{1 / bd})^{bc} &  & \by{i:5.6.3}                \\
              & = x^{ad / bd} x^{bc / bd}             &  & \text{(by  \cref{i:5.6.7})} \\
              & = x^{a / b} x^{c / d}                 &  & \by{i:5.6.8}                \\
              & = x^q x^r
  \end{align*}
  and
  \begin{align*}
    x^{qr} & = x^{ac / bd}                                                        &  & \by{i:4.2.2}      \\
           & = (x^{1 / bd})^{ac}                                                  &  & \by{i:5.6.7}      \\
           & = \big((x^{1 / b})^{1 / d}\big)^{ac}                                 &  & \by{i:5.6.6}[g]   \\
           & = \bigg(\Big(\big((x^{1 / b})^a\big)^{1 / a}\Big)^{1 / d}\Bigg)^{ac} &  & \by{i:5.6.6}[a,b] \\
           & = \Big(\big((x^{a / b})^{1 / a}\big)^{1 / d}\Big)^{ac}               &  & \by{i:5.6.7}      \\
           & = \big((x^{a / b})^{1 / ad}\big)^{ac}                                &  & \by{i:5.6.6}[g]   \\
           & = (x^{a / b})^{ac / ad}                                              &  & \by{i:5.6.7}      \\
           & = (x^{a / b})^{c / d}                                                &  & \by{i:5.6.8}      \\
           & = (x^q)^r.
  \end{align*}
\end{proof}

\begin{proof}[\pf{i:5.6.9}(c)]
  Let \(q = a / b\) where \(a \in \Z\) and \(b \in \Z^+\).
  Then we have
  \begin{align*}
    x^{-q} & = x^{-a / b}                          \\
           & = (x^{1 / b})^{-a}  &  & \by{i:5.6.7} \\
           & = 1 / (x^{1 / b})^a &  & \by{i:5.6.3} \\
           & = 1 / x^q.          &  & \by{i:5.6.7}
  \end{align*}
\end{proof}

\begin{proof}[\pf{i:5.6.9}(d)]
  Let \(q = a / b\) where \(a \in \Z\) and \(b \in \Z^+\).
  Then we have
  \begin{align*}
             & x > y                                              \\
    \implies & x^{1 / b} > y^{1 / b}         &  & \by{i:5.6.6}[d] \\
    \implies & (x^{1 / b})^a > (y^{1 / b})^a &  & \by{i:5.6.3}    \\
    \implies & x^q > y^q                     &  & \by{i:5.6.7}
  \end{align*}
  and
  \begin{align*}
             & x^q > y^q                                                                                \\
    \implies & (x^{1 / b})^a > (y^{1 / b})^a                                     &  & \by{i:5.6.7}      \\
    \implies & \big((x^{1 / b})^a\big)^{1 / a} > \big((y^{1 / b})^a\big)^{1 / a} &  & \by{i:5.6.6}[d]   \\
    \implies & x^{1 / b} > y^{1 / b}                                             &  & \by{i:5.6.6}[a,b] \\
    \implies & x > y.                                                            &  & \by{i:5.6.6}[d]
  \end{align*}
  Thus we conclude that \(x > y \iff x^q > y^q\) when \(x, y \in \R^+\) and \(q \in \Q^+\).
\end{proof}

\begin{proof}[\pf{i:5.6.9}(e)]
  Let \(q = a / b\) and \(r = c / d\) where \(a, c \in \Z\) and \(b, d \in \Z^+\).
  First suppose that \(x > 1\) and \(x^q > x^r\).
  Then we have
  \begin{align*}
             & x^q > x^r                                                                  \\
    \implies & x^q x^{-r} > x^r x^{-r}                              &  & \by{i:5.6.9}[a]  \\
    \implies & x^{q - r} > x^{r - r} = x^0 = 1                      &  & \by{i:5.6.9}[b]  \\
    \implies & x^{(ad - bc) / bd} > 1                               &  & \by{i:4.2.2}     \\
    \implies & (x^{(ad - bc)})^{1 / bd} > 1                         &  & \by{i:5.6.9}[b]  \\
    \implies & \big((x^{(ad - bc)})^{1 / bd}\big)^{bd} > 1^{bd} = 1 &  & \by{i:5.6.3}     \\
    \implies & x^{(ad - bc)} > 1                                    &  & \by{i:5.6.6}[a]  \\
    \implies & ad - bc > 0                                          &  & \by{i:4.3.12}[b] \\
    \implies & ad > bc                                              &  & \by{i:4.1.11}[b] \\
    \implies & a / b > c / d                                        &  & \by{i:4.2.9}[e]  \\
    \implies & q > r.
  \end{align*}

  Now suppose that \(x > 1\) and \(q > r\).
  Then we have
  \begin{align*}
             & q > r                                                   \\
    \implies & q - r > 0                          &  & \by{i:4.2.9}    \\
    \implies & x^{q - r} > 1^{q - r}              &  & \by{i:5.6.9}[d] \\
    \implies & x^{q - r} > 1^{(ad - bc) / bd}     &  & \by{i:4.2.2}    \\
    \implies & x^{q - r} > (1^{1 / bd})^{ad - bc} &  & \by{i:5.6.7}    \\
    \implies & x^{q - r} > 1^{ad - bc} = 1        &  & \by{i:5.6.6}[e] \\
    \implies & x^{q - r} x^r > x^r                &  & \by{i:5.6.9}[a] \\
    \implies & x^{q - r + r} > x^r                &  & \by{i:5.6.9}[b] \\
    \implies & x^q > x^r.                         &  & \by{i:5.6.8}
  \end{align*}
  Thus we conclude that if \(x > 1\), then \(x^q > x^r \iff q > r\).

  Next suppose that \(x < 1\) and \(x^q > x^r\).
  Then we have
  \begin{align*}
             & x^q > x^r                                                                  \\
    \implies & x^q x^{-r} > x^r x^{-r}                              &  & \by{i:5.6.9}[a]  \\
    \implies & x^{q - r} > x^{r - r} = x^0 = 1                      &  & \by{i:5.6.9}[b]  \\
    \implies & x^{(ad - bc) / bd} > 1                               &  & \by{i:4.2.2}     \\
    \implies & (x^{(ad - bc)})^{1 / bd} > 1                         &  & \by{i:5.6.9}[b]  \\
    \implies & \big((x^{(ad - bc)})^{1 / bd}\big)^{bd} > 1^{bd} = 1 &  & \by{i:5.6.3}     \\
    \implies & x^{(ad - bc)} > 1                                    &  & \by{i:5.6.6}[a]  \\
    \implies & ad - bc < 0                                          &  & \by{i:4.3.12}[b] \\
    \implies & ad < bc                                              &  & \by{i:4.1.11}[b] \\
    \implies & a / b < c / d                                        &  & \by{i:4.2.9}[e]  \\
    \implies & q < r.
  \end{align*}

  Finally suppose that \(x < 1\) and \(q < r\).
  Then we have
  \begin{align*}
             & q < r                                                   \\
    \implies & r - q > 0                          &  & \by{i:4.2.9}    \\
    \implies & x^{r - q} < 1^{r - q}              &  & \by{i:5.6.9}[d] \\
    \implies & x^{r - q} < 1^{(bc - ad) / bd}     &  & \by{i:4.2.2}    \\
    \implies & x^{r - q} < (1^{1 / bd})^{bc - ad} &  & \by{i:5.6.7}    \\
    \implies & x^{r - q} < 1^{bc - ad} = 1        &  & \by{i:5.6.6}[e] \\
    \implies & x^{r - q} x^q < x^q                &  & \by{i:5.6.9}[a] \\
    \implies & x^{r - q + q} < x^q                &  & \by{i:5.6.9}[b] \\
    \implies & x^r < x^q.                         &  & \by{i:5.6.8}
  \end{align*}
  Thus we conclude that if \(x < 1\), then \(x^q > x^r \iff q < r\).
\end{proof}

\begin{proof}[\pf{i:5.6.9}(f)]
  Let \(q = a / b\) where \(a \in \Z\) and \(b \in \Z^+\).
  Then we have
  \begin{align*}
    (xy)^q & = \big((xy)^{1 / b}\big)^a    &  & \by{i:5.6.7}    \\
           & = (x^{1 / b} y^{1 / b})^a     &  & \by{i:5.6.6}[f] \\
           & = (x^{1 / b})^a (y^{1 / b})^a &  & \by{i:5.6.3}    \\
           & = x^q y^q.                    &  & \by{i:5.6.7}
  \end{align*}
\end{proof}

\exercisesection

\begin{ex}\label{i:ex:5.6.1}
  Prove \cref{i:5.6.6}.
\end{ex}

\begin{proof}[\pf{i:ex:5.6.1}]
  See \cref{i:5.6.6}.
\end{proof}

\begin{ex}\label{i:ex:5.6.2}
  Prove \cref{i:5.6.9}.
\end{ex}

\begin{proof}[\pf{i:ex:5.6.2}]
  See \cref{i:5.6.9}.
\end{proof}

\begin{ex}\label{i:ex:5.6.3}
  If \(x\) is a real number, show that \(\abs{x} = (x^2)^{1 / 2}\).
\end{ex}

\begin{proof}[\pf{i:ex:5.6.3}]
  By \cref{i:5.4.7} exactly one of the following three statements is true:
  \begin{enumerate}
    \item \(x > 0\).
          Then by \cref{i:5.4.5} we have \(\abs{x} = x\) and by \cref{i:5.6.6}(a)(b) we have \(\abs{x} = x = (x^2)^{1 / 2}\).
    \item \(x = 0\).
          Then by \cref{i:5.4.5} we have \(\abs{0} = 0\) and by \cref{i:5.6.6}(c) we have \(\abs{0} = 0 = (0^2)^{1 / 2}\).
    \item \(x < 0\).
          Then by \cref{i:5.4.5} we have \(\abs{x} = -x > 0\).
          By \cref{i:5.6.6}(b)(c) we have \(-x = ((-x)^2)^{1 / 2}\).
          But by \cref{i:5.3.11} we know that \((-x)^2 = (-x)(-x) = x^2\).
          Thus we have \(\abs{x} = -x = (x^2)^{1 / 2}\).
  \end{enumerate}
  From all cases above we conclude that \(\abs{x} = (x^2)^{1 / 2}\).
\end{proof}


\chapter{Limits of sequences}\label{ch:6}

\section{Convergence and limit laws}\label{i:sec:6.1}

\begin{defn}[Distance between two real numbers]\label{i:6.1.1}
  Given two real numbers \(x\) and \(y\), we define their distance \(d(x, y)\) to be \(d(x, y) \coloneqq \abs{x - y}\).
\end{defn}

\begin{note}
  Clearly \cref{i:6.1.1} is consistent with \cref{i:4.3.2}.
  Further, \cref{i:4.3.3} works just as well for real numbers as it does for rationals, because the real numbers obey all the rules of algebra that the rationals do.
\end{note}

\begin{defn}[\(\varepsilon\)-close real numbers]\label{i:6.1.2}
  Let \(\varepsilon \in \R_{\geq 0}\).
  We say that two real numbers \(x, y\) are \emph{\(\varepsilon\)-close} iff we have \(d(y, x) \leq \varepsilon\).
\end{defn}

\begin{note}
  Again, it is clear that \cref{i:6.1.2} is consistent with \cref{i:4.3.4}.
\end{note}

\begin{note}
  Now let \((a_n)_{n = m}^\infty\) be a sequence of \emph{real} numbers;
  i.e., we assign a real number \(a_n\) for every integer \(n \geq m\).
  The starting index \(m\) is some integer;
  usually this will be \(1\), but in some cases we will start from some index other than \(1\).
  (The choice of label used to index this sequence is unimportant; we could use for instance \((a_k)_{k = m}^{\infty}\) and this would represent exactly the same sequence as \((a_n)_{n = m}^{\infty}\).)
  We can define the notion of a Cauchy sequence in the same manner as before.
\end{note}

\begin{defn}[Cauchy sequences of reals]\label{i:6.1.3}
  Let \(\varepsilon \in \R^+\).
  A sequence \((a_n)_{n = N}^\infty\) of real numbers starting at some integer index \(N\) is said to be \emph{\(\varepsilon\)-steady} iff \(a_j\) and \(a_k\) are \(\varepsilon\)-close for every \(j, k \in \Z_{\geq N}\).
  A sequence \((a_n)_{n = m}^\infty\) starting at some integer index \(m\) is said to be \emph{eventually \(\varepsilon\)-steady} iff there exists an \(N \in \Z_{\geq m}\) such that \((a_n)_{n = N}^\infty\) is \(\varepsilon\)-steady.
  We say that \((a_n)_{n = m}^\infty\) is a \emph{Cauchy sequence} iff it is eventually \(\varepsilon\)-steady for every \(\varepsilon \in \R^+\).
\end{defn}

\begin{note}
  To put it another way, a sequence \((a_n)_{n = m}^\infty\) of real numbers is a Cauchy sequence if, for every \(\varepsilon \in \R^+\), there exists an \(N \in \Z_{\geq m}\) such that \(\abs{a_n - a_{n'}} \leq \varepsilon\) for all \(n, n' \in \Z_{\geq N}\).
  These definitions are consistent with the corresponding definitions for rational numbers (\cref{i:5.1.3,i:5.1.6,i:5.1.8}), although verifying consistency for Cauchy sequences takes a little bit of care.
\end{note}

\begin{prop}\label{i:6.1.4}
  Let \((a_n)_{n = m}^\infty\) be a sequence of rational numbers starting at some integer index \(m\).
  Then \((a_n)_{n = m}^\infty\) is a Cauchy sequence in the sense of \cref{i:5.1.8} iff it is a Cauchy sequence in the sense of \cref{i:6.1.3}.
\end{prop}

\begin{proof}[\pf{i:6.1.4}]
  Suppose first that \((a_n)_{n = m}^\infty\) is a Cauchy sequence in the sense of \cref{i:6.1.3};
  then it is eventually \(\varepsilon\)-steady for every \(\varepsilon \in \R^+\).
  In particular, it is eventually \(\varepsilon\)-steady for every \(\varepsilon \in \Q^+\), which makes it a Cauchy sequence in the sense of \cref{i:5.1.8}.

  Now suppose that \((a_n)_{n = m}^\infty\) is a Cauchy sequence in the sense of \cref{i:5.1.8};
  then it is eventually \(\varepsilon'\)-steady for every \(\varepsilon' \in \Q^+\).
  If \(\varepsilon \in \R^+\), then there exists an \(\varepsilon' \in \Q^+\) which is smaller than \(\varepsilon\), by \cref{i:5.4.12}.
  Since \(\varepsilon'\) is rational, we know that \((a_n)_{n = m}^\infty\) is eventually \(\varepsilon'\)-steady;
  since \(\varepsilon' < \varepsilon\), this implies that \((a_n)_{n = m}^\infty\) is eventually \(\varepsilon\)-steady.
  Since \(\varepsilon\) is an arbitrary positive real number, we thus see that \((a_n)_{n = m}^\infty\) is a Cauchy sequence in the sense of \cref{i:6.1.3}.
\end{proof}

\begin{note}
  Because of \cref{i:6.1.4}, we will no longer care about the distinction between \cref{i:5.1.8} and \cref{i:6.1.3}, and view the concept of a Cauchy sequence as a single unified concept.
\end{note}

\begin{defn}[Convergence of sequences]\label{i:6.1.5}
  Let \(\varepsilon \in \R^+\), and let \(L \in \R\).
  A sequence \((a_n)_{n = N}^\infty\) of real numbers is said to be \emph{\(\varepsilon\)-close to \(L\)} iff \(a_n\) is \(\varepsilon\)-close to \(L\) for every \(n \in \Z_{\geq N}\), i.e., we have \(\abs{a_n - L} \leq \varepsilon\) for every \(n \in \Z_{\geq N}\).
  We say that a sequence \((a_n)_{n = m}^\infty\) is \emph{eventually \(\varepsilon\)-close to \(L\)} iff there exists an \(N \in \Z_{\geq m}\) such that \((a_n)_{n = N}^\infty\) is \(\varepsilon\)-close to \(L\).
  We say that a sequence \((a_n)_{n = m}^\infty\) \emph{converges to \(L\)} iff it is eventually \(\varepsilon\)-close to \(L\) for every \(\varepsilon \in \R^+\).
\end{defn}

\setcounter{thm}{6}
\begin{prop}[Uniqueness of limits]\label{i:6.1.7}
  Let \((a_n)_{n = m}^\infty\) be a real sequence starting at some integer index \(m\), and let \(L \neq L'\) be two distinct real numbers.
  Then it is not possible for \((a_n)_{n = m}^\infty\) to converge to \(L\) while also converging to \(L'\).
\end{prop}

\begin{proof}[\pf{i:6.1.7}]
  Suppose for sake of contradiction that \((a_n)_{n = m}^\infty\) was converging to both \(L\) and \(L'\).
  Let \(\varepsilon = \abs{L - L'} / 3\).
  Note that \(\varepsilon\) is positive since \(L \neq L'\).
  Since \((a_n)_{n = m}^\infty\) converges to \(L\), we know that \((a_n)_{n = m}^\infty\) is eventually \(\varepsilon\)-close to \(L\);
  thus there is an \(N \in \Z_{\geq m}\) such that \(d(a_n, L) \leq \varepsilon\) for all \(n \in \Z_{\geq N}\).
  Similarly, there is an \(M \in \Z_{\geq m}\) such that \(d(a_n, L') \leq \varepsilon\) for all \(n \in \Z_{\geq M}\).
  In particular, if we set \(n \coloneqq \max(N, M)\), then we have \(d(a_n, L) \leq \varepsilon\) and \(d(a_n, L') \leq \varepsilon\), hence by the triangle inequality \(d(L, L') \leq 2\varepsilon = 2\abs{L - L'} / 3\).
  But then we have \(\abs{L - L'} \leq 2\abs{L - L'} / 3\), which contradicts the fact that \(\abs{L - L'} > 0\).
  Thus it is not possible to converge to both \(L\) and \(L'\).
\end{proof}

\begin{defn}[Limits of sequences]\label{i:6.1.8}
  If a sequence \((a_n)_{n = m}^\infty\) converges to some real number \(L\), we say that \((a_n)_{n = m}^\infty\) is \emph{convergent} and that its \emph{limit} is \(L\);
  we write
  \[
    L = \lim_{n \to \infty} a_n
  \]
  to denote this fact.
  If a sequence \((a_n)_{n = m}^\infty\) is not converging to any real number \(L\), we say that the sequence \((a_n)_{n = m}^\infty\) is \emph{divergent} and we leave \(\lim_{n \to \infty} a_n\) undefined.
\end{defn}

\begin{note}
  \cref{i:6.1.7} ensures that a sequence can have at most one limit.
  Thus, if the limit exists, it is a single real number, otherwise it is undefined.
\end{note}

\begin{rmk}\label{i:6.1.9}
  The notation \(\lim_{n \to \infty} a_n\) does not give any indication about the starting index \(m\) of the sequence, but the starting index is irrelevant (\cref{i:ex:6.1.3}).
  Thus in the rest of this discussion we shall not be too careful as to where these sequences start, as we shall be mostly focused on their limits.
\end{rmk}

\begin{note}
  We sometimes use the phrase ``\(a_n \to x\) as \(n \to \infty\)'' as an alternate way of writing the statement ``\((a_n)_{n = m}^\infty\) converges to \(x\)''.
  Bear in mind, though, that the individual statements \(a_n \to x\) and \(n \to \infty\) do not have any rigorous meaning;
  this phrase is just a convention, though of course a very suggestive one.
\end{note}

\begin{rmk}\label{i:6.1.10}
  The exact choice of letter used to denote the index (in this case \(n\)) is irrelevant:
  the phrase \(\lim_{n \to \infty} a_n\) has exactly the same meaning as \(\lim_{k \to \infty} a_k\), for instance.
  Sometimes it will be convenient to change the label of the index to avoid conflicts of notation;
  for instance, we might want to change \(n\) to \(k\) because \(n\) is simultaneously being used for some other purpose, and we want to reduce confusion.
  See \cref{i:ex:6.1.4}.
\end{rmk}

\begin{prop}\label{i:6.1.11}
  We have \(\lim_{n \to \infty} 1 / n = 0\).
\end{prop}

\begin{proof}[\pf{i:6.1.11}]
  We have to show that the sequence \((a_n)_{n = 1}^\infty\) converges to \(0\), where \(a_n \coloneqq 1 / n\).
  In other words, for every \(\varepsilon \in \R^+\), we need to show that the sequence \((a_n)_{n = 1}^\infty\) is eventually \(\varepsilon\)-close to \(0\).
  So, let \(\varepsilon \in \R^+\) be an arbitrary real number.
  We have to find an \(N \in \Z^+\) such that \(\abs{a_n - 0} \leq \varepsilon\) for every \(n \in \Z_{\geq N}\).
  But if \(n \in \Z_{\geq N}\), then
  \[
    \abs{a_n - 0} = \abs{1 / n - 0} = 1 / n \leq 1 / N.
  \]
  Thus, if we pick \(N > 1 / \varepsilon\) (which we can do by the Archimedean principle, \cref{i:5.4.13}), then \(1 / N < \varepsilon\), and so \((a_n)_{n = N}^\infty\) is \(\varepsilon\)-close to \(0\).
  Thus \((a_n)_{n = 1}^\infty\) is eventually \(\varepsilon\)-close to \(0\).
  Since \(\varepsilon\) was arbitrary, \((a_n)_{n = 1}^\infty\) converges to \(0\).
\end{proof}

\begin{prop}[Convergent sequences are Cauchy]\label{i:6.1.12}
  Suppose that \((a_n)_{n = m}^\infty\) is a convergent sequence of real numbers.
  Then \((a_n)_{n = m}^\infty\) is also a Cauchy sequence.
\end{prop}

\begin{proof}[\pf{i:6.1.12}]
  Suppose that \((a_n)_{n = m}^\infty\) converges to \(L \in \R\).
  Let \(\varepsilon \in \R^+\).
  Since \(\lim_{n \to \infty} a_n = L\), by \cref{i:6.1.5} there exists an \(N \in \Z_{\geq m}\) such that \((a_n)_{n = N}^\infty\) is \(\varepsilon / 2\)-close to \(L\).
  This means \(\abs{a_n - L} \leq \varepsilon / 2\) for all \(n \in \Z_{\geq N}\).
  Then we have
  \begin{align*}
    \forall j, k \in \Z_{\geq N}, \abs{a_j - a_k} & \leq \abs{a_j - L} + \abs{a_k - L}                                  &  & \by{i:ac:5.4.1}[f,g] \\
                                                  & \leq \dfrac{\varepsilon}{2} + \dfrac{\varepsilon}{2} = \varepsilon. &  & \by{i:5.4.7}[c,d]
  \end{align*}
  Thus \((a_n)_{n = N}^\infty\) is \(\varepsilon\)-steady and \((a_n)_{n = m}^\infty\) is eventually \(\varepsilon\)-steady.
  Since \(\varepsilon\) was arbitrary, by \cref{i:6.1.3} we see that \((a_n)_{n = m}^\infty\) is a Cauchy sequence of reals.
\end{proof}

\setcounter{thm}{13}
\begin{rmk}\label{i:6.1.14}
  For a converse to \cref{i:6.1.12}, see \cref{i:6.4.18} below.
\end{rmk}

\begin{prop}[Formal limits are genuine limits]\label{i:6.1.15}
  Suppose that \((a_n)_{n = m}^\infty\) is a Cauchy sequence of rational numbers.
  Then \((a_n)_{n = m}^\infty\) converges to \(\LIM_{n \to \infty} a_n\), i.e.
  \[
    \LIM_{n \to \infty} a_n = \lim_{n \to \infty} a_n.
  \]
\end{prop}

\begin{proof}[\pf{i:6.1.15}]
  Let \(L = \LIM_{n \to \infty} a_n\).
  By \cref{i:5.3.1} we know that \(L \in \R\).
  Thus by \cref{i:6.1.4,i:6.1.5} we can ask whether \((a_n)_{n = m}^\infty\) converges to \(L\).
  Suppose for sake of contradiction that \((a_n)_{n = m}^\infty\) does not converge to \(L\).
  Then there must exist an \(\varepsilon \in \R^+\) such that \((a_n)_{n = m}^\infty\) is not eventually \(\varepsilon\)-close to \(L\).
  Fix such \(\varepsilon\).

  Since \((a_n)_{n = m}^\infty\) is a Cauchy sequence of reals, we know that there exists an \(N \in \Z_{\geq m}\) such that \(\abs{a_j - a_k} \leq \dfrac{\varepsilon}{4}\) for all \(j, k \in \Z_{\geq N}\).
  Fix such \(N\).
  Since \((a_n)_{n = m}^\infty\) is not eventually \(\varepsilon\)-close to \(L\), we know that \((a_n)_{n = m}^\infty\) is not eventually \(\varepsilon'\)-close to \(L\) for every \(\varepsilon' \in \R_{0 < \varepsilon}\).
  So \((a_n)_{n = m}^\infty\) is not eventually \(\dfrac{2 \varepsilon}{3}\)-close to \(L\), and we can find a \(j \in \Z_{\geq N}\) such that \(\abs{a_j - L} > \dfrac{2 \varepsilon}{3}\).
  Similarly \((a_n)_{n = m}^\infty\) is not eventually \(\dfrac{\varepsilon}{3}\)-close to \(L\), and we can find a \(k \in \Z_{\geq N}\) such that \(\abs{a_k - L} > \dfrac{\varepsilon}{3}\).
  Fix such \(j\) and \(k\).
  But then we have
  \begin{align*}
    \dfrac{\varepsilon}{4} & \geq \abs{a_j - a_k}                                                             &  & \by{i:6.1.3}      \\
                           & \geq \abs{a_j - L} - \abs{a_k - L}                                               &  & \by{i:4.3.3}[f,g] \\
                           & \geq \dfrac{2 \varepsilon}{3} - \dfrac{\varepsilon}{3} = \dfrac{\varepsilon}{3},
  \end{align*}
  a contradiction.
  Thus such \(\varepsilon\) does not exist.
  Therefore we must have \(\lim_{n \to \infty} a_n = L = \LIM_{n \to \infty} a_n\).
\end{proof}

\begin{defn}[Bounded sequences]\label{i:6.1.16}
  A sequence \((a_n)_{n = m}^\infty\) of reals is \emph{bounded by} a real number \(M \in \R_{\geq 0}\) iff we have \(\abs{a_n} \leq M\) for all \(n \in \Z_{\geq m}\).
  We say that \((a_n)_{n = m}^\infty\) is bounded iff it is \emph{bounded} by \(M\) for some real number \(M \in \R_{\geq 0}\).
\end{defn}

\begin{note}
  \cref{i:6.1.16} is consistent with \cref{i:5.1.12}.
  See \cref{i:ex:6.1.7}.
\end{note}

\begin{cor}\label{i:6.1.17}
  Every convergent sequence of real numbers is bounded.
\end{cor}

\begin{proof}[\pf{i:6.1.17}]
  Recall from \cref{i:5.1.15} that every Cauchy sequence of rationals is bounded.
  An inspection of the proof of \cref{i:5.1.15} shows that the same argument works for reals;
  every Cauchy sequence of reals is bounded.
  From \cref{i:6.1.12} we see that every convergent sequence of reals is a Cauchy sequence.
  Thus every convergent sequence of reals is bounded.
\end{proof}

\begin{ac}\label{i:ac:6.1.1}
  Let \(x, y \in \R\).
  Then we have \(\min(x, y) = -\max(-x, -y)\).
  Similarly we have \(\max(x, y) = -\min(-x, -y)\).
\end{ac}

\begin{proof}[\pf{i:ac:6.1.1}]
  First we show that \(\min(x, y) = -\max(-x, -y)\).
  We split into two cases:
  \begin{itemize}
    \item If \(x \leq y\), then we have \(-x \geq -y\) by \cref{i:ex:4.2.6}.
          Thus,
          \begin{align*}
            \min(x, y) & = x                                   \\
                       & = -(-x)          &  & \by{i:ac:5.3.3} \\
                       & = -\max(-x, -y). &  & (-x \geq -y)
          \end{align*}
    \item If \(x > y\), then we have \(-x < -y\) by \cref{i:ex:4.2.6}.
          Thus,
          \begin{align*}
            \min(x, y) & = y                                   \\
                       & = -(-y)          &  & \by{i:ac:5.3.3} \\
                       & = -\max(-x, -y). &  & (-x < -y)
          \end{align*}
  \end{itemize}
  From all cases above we see that \(\min(x, y) = -\max(-x, -y)\).
  Thus we conclude that \(\min(x, y) = -\max(-x, -y)\).

  Now we show that \(\max(x, y) = -\min(-x, -y)\).
  This is true since
  \begin{align*}
    \max(x, y) & = -(-\max(-(-x), -(-y))) &  & \by{i:ac:5.3.3}               \\
               & = -\min(-x, -y).         &  & \text{(from the proof above)}
  \end{align*}
\end{proof}

\setcounter{thm}{18}
\begin{thm}[Limit Laws]\label{i:6.1.19}
  Let \((a_n)_{n = m}^\infty\) and \((b_n)_{n = m}^\infty\) be convergent sequences of real numbers, and let \(x, y\) be the real numbers \(x \coloneqq \lim_{n \to \infty} a_n\) and \(y \coloneqq \lim_{n \to \infty} b_n\).
  \begin{enumerate}
    \item The sequence \((a_n + b_n)_{n = m}^\infty\) converges to \(x + y\);
          in other words,
          \[
            \lim_{n \to \infty} (a_n + b_n) = \pa{\lim_{n \to \infty} a_n} + \pa{\lim_{n \to \infty} b_n}.
          \]
    \item The sequence \((a_n b_n)_{n = m}^\infty\) converges to \(xy\);
          in other words,
          \[
            \lim_{n \to \infty} (a_n b_n) = \pa{\lim_{n \to \infty} a_n} \pa{\lim_{n \to \infty} b_n}.
          \]
    \item For any real number \(c\), the sequence \((c a_n)_{n = m}^\infty\) converges to \(cx\);
          in other words,
          \[
            \lim_{n \to \infty} (c a_n) = c \pa{\lim_{n \to \infty} a_n}.
          \]
    \item The sequence \((a_n - b_n)_{n = m}^\infty\) converges to \(x - y\);
          in other words,
          \[
            \lim_{n \to \infty} (a_n - b_n) = \pa{\lim_{n \to \infty} a_n} - \pa{\lim_{n \to \infty} b_n}.
          \]
    \item Suppose that \(y \neq 0\), and that \(b_n \neq 0\) for all \(n \geq m\).
          Then the sequence \((b_n^{-1})_{n = m}^\infty\) converges to \(y^{-1}\);
          in other words,
          \[
            \lim_{n \to \infty} b_n^{-1} = \pa{\lim_{n \to \infty} b_n}^{-1}.
          \]
    \item Suppose that \(y \neq 0\), and that \(b_n \neq 0\) for all \(n \in \Z_{\geq m}\).
          Then the sequence \((a_n / b_n)_{n = m}^\infty\) converges to \(x / y\);
          in other words,
          \[
            \lim_{n \to \infty} \dfrac{a_n}{b_n} = \dfrac{\lim_{n \to \infty} a_n}{\lim_{n \to \infty} b_n}.
          \]
    \item The sequence \((\max(a_n, b_n))_{n = m}^\infty\) converges to \(\max(x, y)\);
          in other words,
          \[
            \lim_{n \to \infty} \max(a_n, b_n) = \max\pa{\lim_{n \to \infty} a_n, \lim_{n \to \infty} b_n}.
          \]
    \item The sequence \((\min(a_n, b_n))_{n = m}^\infty\) converges to \(\min(x, y)\);
          in other words,
          \[
            \lim_{n \to \infty} \min(a_n, b_n) = \min\pa{\lim_{n \to \infty} a_n, \lim_{n \to \infty} b_n}.
          \]
  \end{enumerate}
\end{thm}

\begin{proof}[\pf{i:6.1.19}(a)]
  Let \(\varepsilon \in \R^+\).
  By \cref{i:6.1.8}, there exists an \(N_a \in \Z_{\geq m}\) such that \(\abs{a_n - x} \leq \varepsilon / 2\) for every \(n \in \Z_{\geq N_a}\).
  Similarly, there exists an \(N_b \in \Z_{\geq m}\) such that \(\abs{b_n - y} \leq \varepsilon / 2\) for every \(n \in \Z_{\geq N_b}\).
  Now we fix both \(N_a\) and \(N_b\).
  Let \(N = \max(N_a, N_b)\).
  Then we have
  \begin{align*}
    \forall n \in \Z_{\geq N}, \abs{(a_n + b_n) - (x + y)} & = \abs{(a_n - x) + (b_n - y)}                         &  & \by{i:ac:5.3.3}   \\
                                                           & \leq \abs{a_n - x} + \abs{b_n - y}                    &  & \by{i:ac:5.4.1}   \\
                                                           & \leq \varepsilon / 2 + \varepsilon / 2 = \varepsilon. &  & \by{i:5.4.7}[c,d]
  \end{align*}
  Thus, by \cref{i:6.1.5}, \((a_n + b_n)_{n = N}^\infty\) is \(\varepsilon\)-close to \(x + y\), and \((a_n + b_n)_{n = m}^\infty\) is eventually \(\varepsilon\)-close to \(x + y\).
  Since \(\varepsilon\) was arbitrary, by \cref{i:6.1.5} again we know that \((a_n + b_n)_{n = m}^\infty\) converges to \(x + y\).
\end{proof}

\begin{proof}[\pf{i:6.1.19}(b)]
  Since \(x = \lim_{n \to \infty} a_n\) and \(y = \lim_{n \to \infty} b_n\), by \cref{i:6.1.17}, there exist some \(A, B \in \R_{\geq 0}\) such that \(\abs{a_n} \leq A\) and \(\abs{b_n} \leq B\) for all \(n \in \Z_{\geq m}\).
  Fix both \(A\) and \(B\).
  Clearly we have \(A < \abs{x} + A + 1\) and \(B < B + 1\), so we must have \(\abs{a_n} \leq \abs{x} + A + 1\) and \(\abs{b_n} \leq B + 1\) for all \(n \in \Z_{\geq m}\).

  Let \(\varepsilon \in \R^+\).
  Observe that \(\dfrac{\varepsilon}{2 (\abs{x} + A + 1)} \in \R^+\) and \(\dfrac{\varepsilon}{2 (B + 1)} \in \R^+\).
  Since \(x = \lim_{n \to \infty} a_n\), by \cref{i:6.1.8}, there exists an \(N_a \in \Z_{\geq m}\) such that \(\abs{a_n - x} \leq \dfrac{\varepsilon}{2(B + 1)}\) for all \(n \in \Z_{\geq N_a}\).
  Similarly, since \(y = \lim_{n \to \infty} b_n\), there exists an \(N_b \in \Z_{\geq m}\) such that \(\abs{b_n - y} \leq \dfrac{\varepsilon}{2 (\abs{x} + A + 1)}\) for all \(n \in \Z_{\geq N_b}\).
  Now we fix both \(N_a\) and \(N_b\).
  Let \(N = \max(N_a, N_b)\).
  Then we have
  \begin{align*}
    \forall n \in \Z_{\geq N}, \abs{a_n b_n - x y} & = \abs{a_n b_n - x y + x b_n - x b_n}                                                           &  & \by{i:ac:5.3.3}     \\
                                                   & = \abs{b_n(a_n - x) + x(b_n - y)}                                                               &  & \by{i:ac:5.3.3}     \\
                                                   & \leq \abs{b_n(a_n - x)} + \abs{x(b_n - y)}                                                      &  & \by{i:ac:5.4.1}     \\
                                                   & = \abs{b_n}\abs{a_n - x} + \abs{x}\abs{b_n - y}                                                 &  & \by{i:ac:5.4.1}     \\
                                                   & \leq (B + 1) \times \dfrac{\varepsilon}{2 (B + 1)} + \abs{x}\abs{b_n - y}                       &  & \by{i:5.4.7}[c,d,e] \\
                                                   & \leq \dfrac{\varepsilon}{2} + (\abs{x} + A + 1) \times \dfrac{\varepsilon}{2 (\abs{x} + A + 1)} &  & \by{i:5.4.7}[c,d,e] \\
                                                   & = \dfrac{\varepsilon}{2} + \dfrac{\varepsilon}{2} = \varepsilon.
  \end{align*}
  Thus, by \cref{i:6.1.5}, \((a_n b_n)_{n = N}^\infty\) is \(\varepsilon\)-close to \(xy\), and \((a_n b_n)_{n = m}^\infty\) is eventually \(\varepsilon\)-close to \(xy\).
  Since \(\varepsilon\) was arbitrary, by \cref{i:6.1.5} again we know that \((a_n b_n)_{n = m}^\infty\) converges to \(xy\).
\end{proof}

\begin{proof}[\pf{i:6.1.19}(c)]
  Let \((c_n)_{n = m}^\infty\) be a sequence of reals where \(c_n = c\) for all \(n \in \Z_{\geq m}\).
  Clearly we have \(\lim_{n \to \infty} c_n = \lim_{n \to \infty} c = c\).
  Then we have
  \begin{align*}
    \lim_{n \to \infty} (c a_n) & = \lim_{n \to \infty} (c_n a_n)                                                   \\
                                & = \pa{\lim_{n \to \infty} c_n} \pa{\lim_{n \to \infty} a_n} &  & \by{i:6.1.19}[b] \\
                                & = c \pa{\lim_{n \to \infty} a_n}.
  \end{align*}
\end{proof}

\begin{proof}[\pf{i:6.1.19}(d)]
  We have
  \begin{align*}
    \lim_{n \to \infty} (a_n - b_n) & = \lim_{n \to \infty} (a_n + (-1)(b_n))                                &  & \by{i:ac:5.3.2}  \\
                                    & = \pa{\lim_{n \to \infty} a_n} + \pa{\lim_{n \to \infty} ((-1)(b_n))}  &  & \by{i:6.1.19}[a] \\
                                    & = \pa{\lim_{n \to \infty} a_n} + \pa{(-1)\pa{\lim_{n \to \infty} b_n}} &  & \by{i:6.1.19}[c] \\
                                    & = \pa{\lim_{n \to \infty} a_n} - \pa{\lim_{n \to \infty} b_n}.         &  & \by{i:ac:5.3.2}
  \end{align*}
\end{proof}

\begin{proof}[\pf{i:6.1.19}(e)]
  First we show that \((b_n)_{n = m}^\infty\) is bounded away from zero.
  Since \(y \neq 0\), we know that \(\abs{y} > 0\).
  Since \(y = \lim_{n \to \infty} b_n\), we know that there exists an \(N \in \Z_{\geq m}\) such that \(\abs{b_n - y} \leq \dfrac{\abs{y}}{2}\) for all \(n \in \Z_{\geq N}\).
  Then we have
  \begin{align*}
             & \forall n \in \Z_{\geq N}, \dfrac{-\abs{y}}{2} \leq b_n - y \leq \dfrac{\abs{y}}{2} &  & \by{i:ac:5.4.1} \\
    \implies & \forall n \in \Z_{\geq N}, \begin{dcases}
                                            \dfrac{y}{2} \leq b_n \leq \dfrac{3y}{2} & \text{if } y \in \R^+ \\
                                            \dfrac{3y}{2} \leq b_n \leq \dfrac{y}{2} & \text{if } y \in \R^-
                                          \end{dcases}                 &  & \by{i:5.4.7}[c,d]              \\
    \implies & \forall n \in \Z_{\geq N}, \begin{dcases}
                                            \dfrac{y}{2} \leq b_n   & \text{if } y \in \R^+ \\
                                            \dfrac{-y}{2} \leq -b_n & \text{if } y \in \R^-
                                          \end{dcases}                                  &  & \by{i:ex:4.2.6}            \\
    \implies & \forall n \in \Z_{\geq N}, \abs{\dfrac{y}{2}} \leq \abs{b_n}.                       &  & \by{i:ac:5.4.1}
  \end{align*}
  Since \(y \neq 0\), we see that \(\abs{\dfrac{y}{2}} \in \R^+\).
  Thus \((b_n)_{n = N}^\infty\) is bounded away from zero.
  Since \(b_n \neq 0\) for all \(n \in \Z_{\geq m}\), we see that \((b_n)_{n = m}^{N - 1}\) is also bounded away from zero.
  Combining the results we see that \((b_n)_{n = m}^\infty\) is bounded away from zero.

  Now we show that \(\lim_{n \to \infty} b_n^{-1} = y^{-1}\).
  Let \(\varepsilon \in \R^+\).
  Since \((b_n)_{n = m}^\infty\) is bounded away from zero, there exists an \(M \in \R^+\) such that \(\abs{b_n} \geq M\) for all \(n \in \Z_{\geq m}\).
  Fix such \(M\).
  Clearly we have \(\varepsilon M \abs{y} \in \R^+\) and \(\dfrac{1}{\abs{b_n}} \leq \dfrac{1}{M}\) for all \(n \in \Z_{\geq m}\).
  Since \(y = \lim_{n \to \infty} b_n \neq 0\), by \cref{i:6.1.8}, there exists an \(N \in \Z_{\geq m}\) such that \(\abs{b_n - y} \leq \varepsilon M \abs{y}\) for all \(n \in \Z_{\geq N}\).
  Fix such \(N\).
  Then we have
  \begin{align*}
    \forall n \in \Z_{\geq N}, \abs{b_n^{-1} - y^{-1}} & = \abs{\dfrac{1}{b_n} - \dfrac{1}{y}}                       &  & \by{i:5.6.2}    \\
                                                       & = \abs{\dfrac{y - b_n}{b_n y}}                              &  & \by{i:ac:5.3.3} \\
                                                       & = \abs{y - b_n}\dfrac{1}{\abs{b_n}\abs{y}}                  &  & \by{i:ac:5.4.1} \\
                                                       & \leq \abs{y - b_n}\dfrac{1}{M\abs{y}}                       &  & \by{i:5.4.7}[e] \\
                                                       & \leq \varepsilon M\abs{y}\dfrac{1}{M\abs{y}} = \varepsilon. &  & \by{i:5.4.7}[e]
  \end{align*}
  Thus, by \cref{i:6.1.5}, \(\pa{b_n^{-1}}_{n = N}^\infty\) is \(\varepsilon\)-close to \(y^{-1}\), and \(\pa{b_n^{-1}}_{n = m}^\infty\) is eventually \(\varepsilon\)-close to \(y^{-1}\).
  Since \(\varepsilon\) was arbitrary, by \cref{i:6.1.5} again we know that \(\pa{b_n^{-1}}_{n = m}^\infty\) converges to \(y^{-1}\).
\end{proof}

\begin{proof}[\pf{i:6.1.19}(f)]
  We have
  \begin{align*}
    \lim_{n \to \infty} \dfrac{a_n}{b_n} & = \lim_{n \to \infty} \pa{a_n b_n^{-1}}                          &  & \by{i:ac:5.3.5}  \\
                                         & = \pa{\lim_{n \to \infty} a_n} \pa{\lim_{n \to \infty} b_n^{-1}} &  & \by{i:6.1.19}[b] \\
                                         & = \pa{\lim_{n \to \infty} a_n} \pa{\lim_{n \to \infty} b_n}^{-1} &  & \by{i:6.1.19}[e] \\
                                         & = \dfrac{\lim_{n \to \infty} a_n}{\lim_{n \to \infty} b_n}.      &  & \by{i:ac:5.3.5}
  \end{align*}
\end{proof}

\begin{proof}[\pf{i:6.1.19}(g)]
  First suppose that \(x = y\).
  Let \(\varepsilon \in \R^+\).
  By \cref{i:6.1.8}, there exists an \(N_a \in \Z_{\geq m}\) such that \(\abs{a_n - x} \leq \varepsilon\) for all \(n \in \Z_{\geq N_a}\).
  Similarly, there exists an \(N_b \in \Z_{\geq m}\) such that \(\abs{b_n - y} \leq \varepsilon\) for all \(n \in \Z_{\geq N_b}\).
  Fix both \(N_a\) and \(N_b\).
  Let \(N = \max(N_a, N_b)\).
  Then we have \(\abs{a_n - x} \leq \varepsilon\) and \(\abs{b_n - y} \leq \varepsilon\) for all \(n \in \Z_{\geq N}\).
  Since \(x = y\), we have \(\max(x, y) = x\).
  Thus,
  \begin{align*}
             & \forall n \in \Z_{\geq N}, \begin{dcases}
                                            \abs{a_n - \max(x, y)} = \abs{a_n - x} \leq \varepsilon \\
                                            \abs{b_n - \max(x, y)} = \abs{b_n - x} = \abs{b_n - y} \leq \varepsilon
                                          \end{dcases}                       \\
    \implies & \forall n \in \Z_{\geq N}, \abs{\max(a_n, b_n) - \max(x, y)} = \begin{dcases}
                                                                                \abs{a_n - x} \leq \varepsilon & \text{if } a_n \geq b_n \\
                                                                                \abs{b_n - x} \leq \varepsilon & \text{if } a_n < b_n
                                                                              \end{dcases} \\
    \implies & \forall n \in \Z_{\geq N}, \abs{\max(a_n, b_n) - \max(x, y)} \leq \varepsilon.
  \end{align*}
  By \cref{i:6.1.5}, \(\pa{\max(a_n, b_n)}_{n = N}^\infty\) is \(\varepsilon\)-close to \(\max(x, y)\), and \(\pa{\max(a_n, b_n)}_{n = m}^\infty\) is eventually \(\varepsilon\)-close to \(\max(x, y)\).
  Since \(\varepsilon\) was arbitrary, by \cref{i:6.1.5} again we know that \(\pa{\max(a_n, b_n)}_{n = m}^\infty\) converges to \(\max(x, y)\).

  Now suppose that \(x \neq y\).
  We have either \(x < y\) or \(x > y\), so without the loss of generality, suppose that \(x < y\).
  Then we have \(\dfrac{y - x}{2} \in \R^+\).
  Let \(\varepsilon \in \R^+\).
  Clearly we have \(\min\pa{\varepsilon, \dfrac{y - x}{2}} \in \R^+\).
  By \cref{i:6.1.8}, there exists an \(N_a \in \Z_{\geq m}\) such that \(\abs{a_n - x} \leq \min\pa{\varepsilon, \dfrac{y - x}{2}}\) for all \(n \in \Z_{\geq N_a}\).
  Similarly, there exists an \(N_b \in \Z_{\geq m}\) such that \(\abs{b_n - y} \leq \min\pa{\varepsilon, \dfrac{y - x}{2}}\) for all \(n \in \Z_{\geq N_b}\).
  Fix both \(N_a\) and \(N_b\).
  Let \(N = \max(N_a, N_b)\).
  Then we have
  \begin{align*}
             & \forall n \in \Z_{\geq N}, \begin{dcases}
                                            \abs{a_n - x} \leq \min\pa{\varepsilon, \dfrac{y - x}{2}} \leq \dfrac{y - x}{2} \\
                                            \abs{b_n - y} \leq \min\pa{\varepsilon, \dfrac{y - x}{2}} \leq \dfrac{y - x}{2}
                                          \end{dcases} &  & \by{i:5.4.7}[c] \\
    \implies & \forall n \in \Z_{\geq N}, \begin{dcases}
                                            -\dfrac{y - x}{2} \leq a_n - x \leq \dfrac{y - x}{2} \\
                                            -\dfrac{y - x}{2} \leq b_n - y \leq \dfrac{y - x}{2}
                                          \end{dcases}                                   &  & \by{i:ac:5.4.1}                    \\
    \implies & \forall n \in \Z_{\geq N}, \begin{dcases}
                                            a_n \leq \dfrac{y - x}{2} + x \\
                                            y - \dfrac{y - x}{2} \leq b_n
                                          \end{dcases}                                                 &  & \by{i:5.4.7}[d]      \\
    \implies & \forall n \in \Z_{\geq N}, \begin{dcases}
                                            a_n \leq \dfrac{x + y}{2} \\
                                            \dfrac{x + y}{2} \leq b_n
                                          \end{dcases}                                                 &  & \by{i:ac:5.3.3}      \\
    \implies & \forall n \in \Z_{\geq N}, a_n \leq \dfrac{x + y}{2} \leq b_n.                            &  & \by{i:5.4.7}[c]
  \end{align*}
  This means \(\max(a_n, b_n) = b_n\) for all \(n \in \Z_{\geq N}\).
  Thus,
  \begin{align*}
    \forall n \in \Z_{\geq N}, \abs{\max(a_n, b_n) - \max(x, y)} & = \abs{b_n - y}                                                  \\
                                                                 & \leq \min\pa{\varepsilon, \dfrac{y - x}{2}}                      \\
                                                                 & \leq \varepsilon.                           &  & \by{i:5.4.7}[c]
  \end{align*}
  By \cref{i:6.1.5}, \(\pa{\max(a_n, b_n)}_{n = N}^\infty\) is \(\varepsilon\)-close to \(\max(x, y)\), and \(\pa{\max(a_n, b_n)}_{n = m}^\infty\) is eventually \(\varepsilon\)-close to \(\max(x, y)\).
  Since \(\varepsilon\) was arbitrary, by \cref{i:6.1.5} again we know that \(\pa{\max(a_n, b_n)}_{n = m}^\infty\) converges to \(\max(x, y)\).
\end{proof}

\begin{proof}[\pf{i:6.1.19}(h)]
  We have
  \begin{align*}
    \lim_{n \to \infty} \min(a_n, b_n) & = \lim_{n \to \infty} -\max(-a_n, -b_n)                                  &  & \by{i:ac:6.1.1}  \\
                                       & = -\pa{\lim_{n \to \infty} \max(-a_n, -b_n)}                             &  & \by{i:6.1.19}[c] \\
                                       & = -\max\pa{\lim_{n \to \infty} -a_n, \lim_{n \to \infty} -b_n}           &  & \by{i:6.1.19}[g] \\
                                       & = -\max\pa{-\pa{\lim_{n \to \infty} a_n}, -\pa{\lim_{n \to \infty} b_n}} &  & \by{i:6.1.19}[c] \\
                                       & = \min\pa{\lim_{n \to \infty} a_n, \lim_{n \to \infty} b_n}.             &  & \by{i:ac:6.1.1}
  \end{align*}
\end{proof}

\exercisesection

\begin{ex}\label{i:ex:6.1.1}
  Let \((a_n)_{n = m}^\infty\) be a sequence of reals, such that \(a_{n + 1} > a_n\) for each \(n \in \Z_{\geq m}\).
  Prove that whenever \(j, k \in \Z_{\geq m}\) such that \(j > k\), then we have \(a_j > a_k\).
  (We refer to these sequences as \emph{increasing} sequences.)
\end{ex}

\begin{proof}[\pf{i:ex:6.1.1}]
  Let \(E = \set{z \in \Z_{\geq m} : j \leq z \leq k}\).
  Then \(E\) is finite (since \(\#(E) = k - j + 1\)) and non-empty (since \(j, k \in E\)).
  So \((a_n)_{n = j}^k\) is a finite sequence, and the elements in \((a_n)_{n = j}^k\) are \(\set{a_j, a_{j + 1}, \dots, a_{k - 1}, a_k}\).
  By hypothesis, we have \(a_{n + 1} > a_n\) for each \(n \in \Z_{\geq m}\).
  Thus, we have \(a_j < a_{j + 1} < \dots < a_{k - 1} < a_k\), and by \cref{i:5.4.7}(c) we have \(a_j < a_k\).
\end{proof}

\begin{ex}\label{i:ex:6.1.2}
  Let \((a_n)_{n = m}^\infty\) be a sequence of reals, and let \(L \in \R\).
  Show that \((a_n)_{n = m}^\infty\) converges to \(L\) iff, given any \(\varepsilon \in \R^+\), one can find an \(N \in \Z_{\geq m}\) such that \(\abs{a_n - L} \leq \varepsilon\) for all \(n \in \Z_{\geq N}\).
\end{ex}

\begin{proof}[\pf{i:ex:6.1.2}]
  We have
  \begin{align*}
         & (a_n)_{n = m}^\infty \text{ converges to } L                                                                                                \\
    \iff & \forall \varepsilon \in \R^+, (a_n)_{n = m}^\infty \text{ is eventually } \varepsilon\text{-close to } L                  &  & \by{i:6.1.5} \\
    \iff & \forall \varepsilon \in \R^+, \exists N \in \Z_{\geq m} : (a_n)_{n = N}^\infty \text{ is } \varepsilon\text{-close to } L &  & \by{i:6.1.5} \\
    \iff & \forall \varepsilon \in \R^+, \exists N \in \Z_{\geq m} : \forall n \in \Z_{\geq N}, \abs{a_n - L} \leq \varepsilon.      &  & \by{i:6.1.5}
  \end{align*}
\end{proof}

\begin{ex}\label{i:ex:6.1.3}
  Let \((a_n)_{n = m}^\infty\) be a sequence of reals, let \(c \in \R\), and let \(m' \in \Z_{\geq m}\).
  Show that \((a_n)_{n = m}^\infty\) converges to \(c\) iff \((a_n)_{n = m'}^\infty\) converges to \(c\).
\end{ex}

\begin{proof}[\pf{i:ex:6.1.3}]
  First suppose that \((a_n)_{n = m}^\infty\) converges to \(c\).
  Let \(\varepsilon \in \R^+\).
  By \cref{i:6.1.5}, there exists an \(N \in \Z_{\geq m}\) such that \(\abs{a_n - c} \leq \varepsilon\) for all \(n \in \Z_{\geq N}\).
  Fix such \(N\).
  Now we split into two cases:
  \begin{itemize}
    \item If \(N \geq m'\), then we have found an \(N \in \Z_{\geq m'}\) such that \(\abs{a_n - c} \leq \varepsilon\) for all \(n \in \Z_{\geq N}\).
    \item If \(N < m'\), then by setting \(N' = m'\) we see that there exists an \(N' \in \Z_{\geq m'}\) such that \(\abs{a_n - c} \leq \varepsilon\) for all \(n \in \Z_{\geq N'}\).
  \end{itemize}
  From all cases above we see that we can find an \(M \in \Z_{\geq m'}\) such that \(\abs{a_n - c} \leq \varepsilon\) for all \(n \in \Z_{\geq M}\).
  Thus, by \cref{i:6.1.5}, we see that \((a_n)_{n = m'}^\infty\) is eventually \(\varepsilon\)-close to \(c\).
  Since \(\varepsilon\) was arbitrary, by \cref{i:6.1.5} again we see that \((a_n)_{n = m'}^\infty\) converges to \(c\).

  Now suppose that \((a_n)_{n = m'}^\infty\) converges to \(c\).
  Let \(\varepsilon \in \R^+\).
  By \cref{i:6.1.5}, there exists an \(N \in \Z_{\geq m'}\) such that \(\abs{a_n - c} \leq \varepsilon\) for all \(n \in \Z_{\geq N}\).
  Fix such \(N\).
  Since \(m \leq m'\), we have found an \(N \in \Z_{\geq m}\) such that \(\abs{a_n - c} \leq \varepsilon\) for all \(n \in \Z_{\geq N}\).
  Thus, by \cref{i:6.1.5}, we see that \((a_n)_{n = m}^\infty\) is eventually \(\varepsilon\)-close to \(c\).
  Since \(\varepsilon\) was arbitrary, by \cref{i:6.1.5} again we see that \((a_n)_{n = m}^\infty\) converges to \(c\).
\end{proof}

\begin{ex}\label{i:ex:6.1.4}
  Let \((a_n)_{n = m}^\infty\) be a sequence of reals, let \(c \in \R\), and let \(k \in \Z_{\geq 0}\).
  Show that \((a_n)_{n = m}^\infty\) converges to \(c\) iff \((a_{n + k})_{n = m}^\infty\) converges to \(c\).
\end{ex}

\begin{proof}[\pf{i:ex:6.1.4}]
  First observe that \((a_{n + k})_{n = m}^\infty = (a_n)_{n = m + k}^\infty\).
  Since \(m + k \geq m\), by \cref{i:ex:6.1.3} we see that \((a_n)_{n = m}^\infty\) converges to \(c\) iff \((a_n)_{n = m + k}^\infty\) converges to \(c\).
  Thus \((a_n)_{n = m}^\infty\) converges to \(c\) iff \((a_{n + k})_{n = m}^\infty\) converges to \(c\).
\end{proof}

\begin{ex}\label{i:ex:6.1.5}
  Prove \cref{i:6.1.12}.
\end{ex}

\begin{proof}[\pf{i:ex:6.1.5}]
  See \cref{i:6.1.12}.
\end{proof}

\begin{ex}\label{i:ex:6.1.6}
  Prove \cref{i:6.1.15}.
\end{ex}

\begin{proof}[\pf{i:ex:6.1.6}]
  See \cref{i:6.1.15}.
\end{proof}

\begin{ex}\label{i:ex:6.1.7}
  Show that \cref{i:6.1.16} is consistent with \cref{i:5.1.12}
  (i.e., prove an analogue of \cref{i:6.1.4} for bounded sequences instead of Cauchy sequences).
\end{ex}

\begin{proof}[\pf{i:ex:6.1.7}]
  First suppose that \((a_n)_{n = m}^\infty\) is a sequence of reals which is bounded in the sense of \cref{i:6.1.16}.
  Then there exists an \(M \in \R_{\geq 0}\) such that \(\abs{a_n} \leq M\) for all \(n \in \Z_{\geq m}\).
  By \cref{i:5.4.12}, there exists an \(M' \in \Z^+\) such that \(M \leq M'\).
  Clearly \(M' \in \Z^+\) implies \(M' \in \Q_{\geq 0}\).
  Thus, by \cref{i:5.4.7}(c), we have \(\abs{a_n} \leq M'\) for all \(n \in \Z_{\geq m}\).
  This means that \((a_n)_{n = m}^\infty\) is a bounded sequence in the sense of \cref{i:5.1.12}.

  Now suppose that \((a_n)_{n = m}^\infty\) is a sequence of reals which is bounded in the sense of \cref{i:5.1.12}.
  Then there exists an \(M \in \Q_{\geq 0}\) such that \(\abs{a_n} \leq M\) for all \(n \in \Z_{\geq m}\).
  Since \(M\) is also a real number, we see that \((a_n)_{n = m}^\infty\) is a bounded sequence in the sense of \cref{i:6.1.16}.
\end{proof}

\begin{ex}\label{i:ex:6.1.8}
  Proof \cref{i:6.1.19}.
\end{ex}

\begin{proof}[\pf{i:ex:6.1.8}]
  See \cref{i:6.1.19}.
\end{proof}

\begin{ex}\label{i:ex:6.1.9}
  Explain why \cref{i:6.1.19}(f) fails when the limit of the denominator is \(0\).
  (To repair that problem requires \emph{L'Hôpital's rule}, see \cref{i:sec:10.5}.)
\end{ex}

\begin{proof}[\pf{i:ex:6.1.9}]
  Suppose for sake of contradiction that \cref{i:6.1.19}(f) works when denominator is \(0\).
  Let \((a_n)_{n = 1}^\infty = (1 / n)_{n = 1}^\infty\).
  Then we have
  \[
    \lim_{n \to \infty} \dfrac{a_n}{a_n} = \lim_{n \to \infty} \dfrac{1 / n}{1 / n} = \lim_{n \to \infty} 1 = 1.
  \]
  But by \cref{i:6.1.11} we also have
  \[
    \dfrac{\lim_{n \to \infty} a_n}{\lim_{n \to \infty} a_n} = \dfrac{0}{0}
  \]
  which is undefined.
  Thus \cref{i:6.1.19}(f) fails when denominator is \(0\).
\end{proof}

\begin{ex}\label{i:ex:6.1.10}
  Show that the concept of equivalent Cauchy sequence, as defined in \cref{i:5.2.6}, does not change if \(\varepsilon\) is required to be positive real instead of positive rational.
  More precisely, if \((a_n)_{n = m}^\infty\) and \((b_n)_{n = m}^\infty\) are sequences of reals, show that \((a_n)_{n = m}^\infty\) and \((b_n)_{n = m}^\infty\) are eventually \(\varepsilon\)-close for every \(\varepsilon \in \Q^+\) iff they are eventually \(\varepsilon\)-close for every \(\varepsilon \in \R^+\).
\end{ex}

\begin{proof}[\pf{i:ex:6.1.10}]
  Suppose first that \((a_n)_{n = m}^\infty\) and \((b_n)_{n = m}^\infty\) are eventually \(\varepsilon\)-close for all \(\varepsilon \in \Q^+\).
  Let \(\varepsilon' \in \R^+\).
  By \cref{i:5.4.12}, there exists an \(\varepsilon \in \Q^+\) such that \(\varepsilon \leq \varepsilon'\).
  Fix such \(\varepsilon\).
  Since \(\varepsilon \in \Q^+\), by hypothesis we know that \((a_n)_{n = m}^\infty\) and \((b_n)_{n = m}^\infty\) are eventually \(\varepsilon\)-close.
  This implies that \((a_n)_{n = m}^\infty\) and \((b_n)_{n = m}^\infty\) are eventually \(\varepsilon'\)-close.
  Since \(\varepsilon'\) was arbitrary, we see that \((a_n)_{n = m}^\infty\) and \((b_n)_{n = m}^\infty\) are eventually \(\varepsilon'\)-close for all \(\varepsilon' \in \R^+\).

  Now suppose that \((a_n)_{n = m}^\infty\) and \((b_n)_{n = m}^\infty\) are eventually \(\varepsilon'\)-close for all \(\varepsilon' \in \R^+\).
  This implies that \((a_n)_{n = m}^\infty\) and \((b_n)_{n = m}^\infty\) are eventually \(\varepsilon\)-close for all \(\varepsilon \in \Q^+\).
  Thus we conclude that \((a_n)_{n = m}^\infty\) and \((b_n)_{n = m}^\infty\) are eventually \(\varepsilon\)-close for all \(\varepsilon \in \Q^+\) iff they are eventually \(\varepsilon'\)-close for all \(\varepsilon' \in \R^+\).
\end{proof}

\section{The Extended real number system}\label{i:sec:6.2}

\begin{defn}[Extended real number system]\label{i:6.2.1}
  The \emph{extended real number system \(\R^*\)} is the real line \(\R\) with two additional elements attached, called \(+\infty\) and \(-\infty\).
  These elements are distinct from each other and also distinct from every real number.
  An extended real number \(x\) is called \emph{finite} iff it is a real number, and \emph{infinite} iff it is equal to \(+\infty\) or \(-\infty\).
  (This definition is not directly related to the notion of finite and infinite sets in \cref{i:sec:3.6}, though it is of course similar in spirit.)
\end{defn}

\begin{defn}[Negation of extended reals]\label{i:6.2.2}
  The operation of negation \(x \to -x\) on \(\R\), we now extend to \(\R^*\) by defining \(-(+\infty) \coloneqq -\infty\) and \(-(-\infty) \coloneqq +\infty\).
\end{defn}

\begin{note}
  Thus, every extended real number \(x\) has a negation, and \(-(-x)\) is always equal to \(x\).
\end{note}

\begin{defn}[Ordering of extended reals]\label{i:6.2.3}
  Let \(x\) and \(y\) be extended real numbers.
  We say that \(x \leq y\), i.e., \(x\) is less than or equal to \(y\), iff one of the following three statements is true:
  \begin{enumerate}
    \item \(x\) and \(y\) are real numbers, and \(x \leq y\) as real numbers.
    \item \(y = +\infty\).
    \item \(x = -\infty\).
  \end{enumerate}
  We say that \(x < y\) if we have \(x \leq y\) and \(x \neq y\).
  We sometimes write \(x < y\) as \(y > x\), and \(x \leq y\) as \(y \geq x\).
\end{defn}

\setcounter{thm}{4}
\begin{prop}\label{i:6.2.5}
  Let \(x, y, z\) be extended real numbers.
  Then the following statements are true:
  \begin{enumerate}
    \item (Reflexivity)
          We have \(x \leq x\).
    \item (Trichotomy)
          Exactly one of the statements \(x < y\), \(x = y\), or \(x > y\) is true.
    \item (Transitivity)
          If \(x \leq y\) and \(y \leq z\), then \(x \leq z\).
    \item (Negation reverses order) If \(x \leq y\), then \(-y \leq -x\).
  \end{enumerate}
\end{prop}

\begin{proof}{(a)}
  By \cref{i:5.4.7} we already have \(x \leq x\) when \(x \in \R\).
  So we only need to consider the cases \(x \in \set{+\infty, -\infty}\).
  By \cref{i:6.2.3} we have \(x \leq +\infty\) for every \(x \in \R^*\).
  So we have \(+\infty \leq +\infty\).
  Again by \cref{i:6.2.3} we have \(-\infty \leq x\) for every \(x \in \R^*\).
  So we have \(-\infty \leq -\infty\).
  Thus, we conclude that \(x \leq x\) for every \(x \in \R^*\).
\end{proof}

\begin{proof}{(b)}
  By \cref{i:5.4.7}, we already have exactly one of the statements \(x < y\), \(x = y\), or \(x > y\) is true when \(x, y \in \R\).
  So we only need to consider the cases \(x, y \in \set{+\infty, -\infty}\).
  \begin{itemize}
    \item If \(x = +\infty\), then by \cref{i:6.2.3} we have \(x \geq y\) for every \(y \in \R^*\).
          \begin{itemize}
            \item If \(y = +\infty\), then we have \(x = y\).
            \item If \(y \in \R\), then by \cref{i:6.2.1} \(x \neq y\).
                  Thus, by \cref{i:6.2.3} we have \(x > y\).
            \item If \(y = -\infty\), then by \cref{i:6.2.1} \(x \neq y\).
                  Thus, by \cref{i:6.2.3} we have \(x > y\).
          \end{itemize}
    \item If \(x = -\infty\), then by \cref{i:6.2.3} we have \(x \leq y\) for every \(y \in \R^*\).
          \begin{itemize}
            \item If \(y = +\infty\), then by \cref{i:6.2.1} \(x \neq y\).
                  Thus, by \cref{i:6.2.3} we have \(x < y\).
            \item If \(y \in \R\), then by \cref{i:6.2.1} \(x \neq y\).
                  Thus, by \cref{i:6.2.3} we have \(x < y\).
            \item If \(y = -\infty\), then we have \(x = y\).
          \end{itemize}
    \item If \(y = +\infty\), then by \cref{i:6.2.3} we have \(x \leq y\) for every \(x \in \R^*\).
          \begin{itemize}
            \item If \(x = +\infty\), then we have \(x = y\).
            \item If \(x \in \R\), then by \cref{i:6.2.1} \(x \neq y\).
                  Thus, by \cref{i:6.2.3} we have \(x < y\).
            \item If \(x = -\infty\), then by \cref{i:6.2.1} \(x \neq y\).
                  Thus, by \cref{i:6.2.3} we have \(x < y\).
          \end{itemize}
    \item If \(y = -\infty\), then by \cref{i:6.2.3} we have \(x \geq y\) for every \(x \in \R^*\).
          \begin{itemize}
            \item If \(x = +\infty\), then by \cref{i:6.2.1} \(x \neq y\).
                  Thus, by \cref{i:6.2.3} we have \(x > y\).
            \item If \(x \in \R\), then by \cref{i:6.2.1} \(x \neq y\).
                  Thus, by \cref{i:6.2.3} we have \(x > y\).
            \item If \(x = -\infty\), then we have \(x = y\).
          \end{itemize}
  \end{itemize}
  From all cases above we conclude that exactly one of the statements \(x < y\), \(x = y\), or \(x > y\) is true.
\end{proof}

\begin{proof}{(c)}
  By \cref{i:5.4.7}, we already have \((x \leq y) \land (y \leq z) \implies x \leq z\) when \(x, y, z \in \R\).
  So we only need to consider the cases \(x, y, z \in \set{+\infty, -\infty}\).
  \begin{itemize}
    \item If \(x = +\infty\), then by \cref{i:6.2.3} \((x = +\infty) \land (x \leq y) \implies y = +\infty\).
          Similarly, \((y = +\infty) \land (y \leq z) \implies z = +\infty\).
          Thus, we have \(x = +\infty = z\), and by \cref{i:6.2.5}(a) we have \(x \leq z\).
    \item If \(x = -\infty\), then by \cref{i:6.2.3} \(x \leq z \) for every \(z \in \R^*\).
    \item If \(y = +\infty\), then by \cref{i:6.2.3} \((y = +\infty) \land (y \leq z) \implies z = +\infty\).
          Again by \cref{i:6.2.3} we have \(x \leq +\infty = z\) for every \(x \in \R^*\).
    \item If \(y = -\infty\), then by \cref{i:6.2.3} \((y = -\infty) \land (x \leq y) \implies x = -\infty\).
          Again by \cref{i:6.2.3} we have \(x = -\infty \leq z\) for every \(z \in \R^*\).
    \item If \(z = +\infty\), then by \cref{i:6.2.3}, we have \(x \leq +\infty = z\) for every \(x \in \R^*\).
    \item If \(z = -\infty\), then by \cref{i:6.2.3} \((z = -\infty) \land (y \leq z) \implies y = -\infty\).
          Similarly, \((y = -\infty) \land (x \leq y) \implies x = -\infty\).
          Thus, we have \(x = -\infty = z\), and by \cref{i:6.2.5}(a) we have \(x \leq z\).
  \end{itemize}
  From all cases above we conclude that \((x \leq y) \land (y \leq z) \implies x \leq z\).
\end{proof}

\begin{proof}{(d)}
  By \cref{i:5.4.7} we already have \(x \leq y \implies -y \leq -x\) for every \(x, y \in \R\).
  So we only need to consider the cases \(x, y \in \set{+\infty, -\infty}\).
  \begin{itemize}
    \item If \(x = +\infty\), then by \cref{i:6.2.3} \((x = +\infty) \land (x \leq y) \implies y = +\infty\).
          And by \cref{i:6.2.2} we have \(-x = -\infty = -y\).
          Thus, by \cref{i:6.2.5}(a) we have \(-y \leq -x\).
    \item If \(x = -\infty\), then by \cref{i:6.2.2} \(-x = +\infty\) and by \cref{i:6.2.3} we have \(-y \leq -x\) for every \(-y \in \R^*\).
    \item If \(y = +\infty\), then by \cref{i:6.2.2} \(-y = -\infty\) and by \cref{i:6.2.3} we have \(-y \leq -x\) for every \(-x \in \R^*\).
    \item If \(y = -\infty\), then by \cref{i:6.2.3} \((y = -\infty) \land (x \leq y) \implies x = -\infty\).
          And by \cref{i:6.2.2} we have \(-x = +\infty = -y\).
          Thus, by \cref{i:6.2.5}(a) we have \(-y \leq -x\).
  \end{itemize}
  From all cases above we conclude that \(x \leq y \implies -y \leq -x\).
\end{proof}

\begin{note}
  One could also introduce other operations on the extended real number system, such as addition, multiplication, etc.
  However, this is somewhat dangerous as these operations will almost certainly fail to obey the familiar rules of algebra.
  For instance, to define addition it seems reasonable (given one's intuitive notion of infinity) to set \(+\infty + 5 = +\infty\) and \(+\infty + 3 = +\infty\), but then this implies that \(+\infty + 5 = +\infty + 3\), while \(5 \neq 3\).
  So things like the cancellation law begin to break down once we try to operate involving infinity.
  To avoid these issues we shall simply not define any arithmetic operations on the extended real number system other than negation and order.
\end{note}

\begin{defn}[Supremum of sets of extended reals]\label{i:6.2.6}
  Let \(E\) be a subset of \(\R^*\).
  Then we define the \emph{supremum} \(\sup(E)\) or \emph{least upper bound} of \(E\) by the following rule.
  \begin{enumerate}
    \item If \(E\) is contained in \(\R\) (i.e., \(+\infty\) and \(-\infty\) are not elements of \(E\)), then we let \(\sup(E)\) be as defined in \cref{i:5.5.10}.
    \item If \(E\) contains \(+\infty\), then we set \(\sup(E) \coloneqq +\infty\).
    \item If \(E\) does not contain \(+\infty\) but does contain \(-\infty\), then we set \(\sup(E) \coloneqq \sup(E \setminus \set{-\infty})\)
          (which is a subset of \(\R\) and thus falls under case (a)).
  \end{enumerate}
  We also define the \emph{infimum} \(\inf(E)\) of \(E\) (also known as the \emph{greatest lower bound} of \(E\)) by the formula
  \[
    \inf(E) \coloneqq -\sup(-E)
  \]
  where \(-E\) is the set \(-E \coloneqq \set{-x : x \in E}\).
\end{defn}

\setcounter{thm}{9}
\begin{eg}\label{i:6.2.10}
  Let \(E\) be the empty set.
  Then \(\sup(E) = -\infty\) and \(\inf(E) = +\infty\).
  This is the only case in which the supremum can be less than the infimum.
\end{eg}

\begin{proof}
  Since \(+\infty \notin \emptyset\) and \(-\infty \notin \emptyset\), by \cref{i:6.2.6} we know that \(\sup(\emptyset) = -\infty\).
  Since \(-\emptyset\) is also empty, by \cref{i:6.2.6} we know that \(\sup(-\emptyset) = -\infty\), thus by \cref{i:6.2.6,i:6.2.2} we have \(\inf(\emptyset) = -\sup(-\emptyset) = -(-\infty) = +\infty\).

  Now we show that the only case in which the supremum can be less than the infimum is when \(E = \emptyset\).
  Suppose for the sake of contradiction that there is a set \(E\) such that \(E \neq \emptyset\) and \(\sup(E) < \inf(E)\).
  Since \(E \neq \emptyset\), let \(x \in E\).
  Then we have \(\sup(E) < \inf(E) \leq x \leq \sup(E)\), a contradiction.
  Thus, \(E = \emptyset\).
\end{proof}

\begin{note}
  One can intuitively think of the supremum of \(E\) as follows.
  Imagine the real line with \(+\infty\) somehow on the far right, and \(-\infty\) on the far left.
  Imagine a piston at \(+\infty\) moving leftward until it is stopped by the presence of a set \(E\);
  the location where it stops is the supremum of \(E\).
  Similarly, if one imagines a piston at \(-\infty\) moving rightward until it is stopped by the presence of \(E\), the location where it stops is the infimum of \(E\).
  In the case when \(E\) is the empty set, the pistons pass through each other, the supremum landing at \(-\infty\) and the infimum landing at \(+\infty\).
\end{note}

\begin{thm}\label{i:6.2.11}
  Let \(E\) be a subset of \(\R^*\).
  Then the following statements are true.
  \begin{enumerate}
    \item For every \(x \in E\) we have \(x \leq \sup(E)\) and \(x \geq \inf(E)\).
    \item Suppose that \(M \in \R^*\) is an upper bound for \(E\), i.e., \(x \leq M\) for all \(x \in E\).
          Then we have \(\sup(E) \leq M\).
    \item Suppose that \(M \in \R^*\) is a lower bound for \(E\), i.e., \(x \geq M\) for all \(x \in E\).
          Then we have \(\inf(E) \geq M\).
  \end{enumerate}
\end{thm}

\begin{proof}{(a)}
  We first show that \(x \leq \sup(E)\) for every \(x \in E\).
  First suppose that \(E = \emptyset\).
  Then the statement ``\(x \leq \sup(\emptyset)\) for every \(x \in \emptyset\)'' is vacuously true.
  Now suppose that \(E \neq \emptyset\).
  We split into two cases:
  \begin{itemize}
    \item If \(+\infty \not\in E\), then we can further split into two cases:
          \begin{itemize}
            \item If \(-\infty \in E\), then by \cref{i:6.2.6} we know that \(\sup(E) = \sup(E \setminus \set{-\infty})\).
                  Let \(E' = E \setminus \set{-\infty}\).
                  Since \(E' \subseteq \R\), by \cref{i:5.5.9} we know that \(x \leq \sup(E)\) for every \(x \in E'\).
                  By \cref{i:6.2.3} we know that \(-\infty \leq \sup(E)\), thus we have \(x \leq \sup(E)\) for every \(x \in E\).
            \item If \(-\infty \notin E\), then \(E \subseteq \R\), thus by \cref{i:5.5.9} we know that \(x \leq \sup(E)\) for every \(x \in E\).
          \end{itemize}
    \item If \(+\infty \in E\), then by \cref{i:6.2.6} we have \(\sup(E) = +\infty\), and by \cref{i:6.2.3} we have \(x \leq \sup(E)\) for every \(x \in E\).
  \end{itemize}
  From all cases above we conclude that \(x \leq \sup(E)\) for every \(x \in E\).

  Now we show that \(x \geq \inf(E)\) for every \(x \in E\).
  First suppose that \(E = \emptyset\).
  Then the statement ``\(x \geq \inf(E)\) for every \(x \in E\)'' is vacuously true.
  Now suppose that \(E \neq \emptyset\).
  From the proof above we know that \(x \leq \sup(E)\) for every \(x \in E\).
  Then we have
  \begin{align*}
             & x \leq \sup(E)                           &  & \by{i:5.5.9} \\
    \implies & -x \geq -\sup(E)                                           \\
    \implies & \sup(-E) \geq -x \geq -\sup(E)           &  & \by{i:5.5.9} \\
    \implies & \inf(E) = -\sup(-E) \leq x \leq \sup(E). &  & \by{i:6.2.6}
  \end{align*}
  Thus, we conclude that \(x \geq \inf(E)\) for every \(x \in E\).
\end{proof}

\begin{proof}{(b)}
  First suppose that \(E = \emptyset\).
  Then by \cref{i:6.2.10} and \cref{i:6.2.3} we have \(\sup(E) = -\infty \leq M\).
  Now suppose that \(E \neq \emptyset\).
  We split into two cases:
  \begin{itemize}
    \item If \(+\infty \not\in E\), then we can further split into two cases:
          \begin{itemize}
            \item If \(-\infty \in E\), then by \cref{i:6.2.6} we know that \(\sup(E) = \sup(E \setminus \set{-\infty})\).
                  Let \(E' = E \setminus \set{-\infty}\).
                  Then by \cref{i:5.5.9} we know that \(\sup(E) \leq M\).
            \item If \(-\infty \notin E\), then \(E \subseteq \R\), thus by \cref{i:5.5.9} we know that \(\sup(E) \leq M\).
          \end{itemize}
    \item If \(+\infty \in E\), then by hypothesis we have \(+\infty \leq M\), and by \cref{i:6.2.3} we have \(M = +\infty\).
          Again by \cref{i:6.2.3} we have \(\sup(E) \leq M\).
  \end{itemize}
  From all cases above we conclude that \(\sup(E) \leq M\).
\end{proof}

\begin{proof}{(c)}
  We have
  \begin{align*}
             & \forall x \in E, x \geq M                       \\
    \implies & -x \leq -M                                      \\
    \implies & \sup(-E) \leq -M          &  & \by{i:6.2.11}[b] \\
    \implies & -\sup(-E) \geq M                                \\
    \implies & \inf(E) \geq M.           &  & \by{i:6.2.6}
  \end{align*}
\end{proof}

\exercisesection

\begin{ex}\label{i:ex:6.2.1}
  Prove \cref{i:6.2.5}.
\end{ex}

\begin{proof}
  See \cref{i:6.2.5}.
\end{proof}

\begin{ex}\label{i:ex:6.2.2}
  Prove \cref{i:6.2.11}.
\end{ex}

\begin{proof}
  See \cref{i:6.2.11}.
\end{proof}

\section{Suprema and Infima of sequences}\label{sec:6.3}

\begin{defn}[Sup and inf of sequences]\label{6.3.1}
  Let \((a_n)_{n = m}^\infty\) be a sequence of real numbers.
  Then we define \(\sup(a_n)_{n = m}^\infty\) to be the supremum of the set \(\set{a_n : n \geq m}\), and \(\inf(a_n)_{n = m}^\infty\) to the infimum of the same set \(\set{a_n : n \geq m}\).
\end{defn}

\begin{rmk}\label{6.3.2}
  The quantities \(\sup(a_n)_{n = m}^\infty\) and \(\inf(a_n)_{n = m}^\infty\) are sometimes written as \(\sup_{n \geq m} a_n\) and \(\inf_{n \geq m} a_n\) respectively.
\end{rmk}

\setcounter{thm}{3}
\begin{eg}\label{6.3.4}
  Let \(a_n \coloneqq 1 / n\);
  thus \((a_n)_{n = 1}^\infty\) is the sequence \(1, 1 / 2, 1 / 3, \dots\).
  Then the set \(\set{a_n : n \geq 1}\) is the countable set \(\set{1, 1 / 2, 1 / 3, 1 / 4, \dots}\).
  Thus \(\sup(a_n)_{n = 1}^\infty = 1\) and \(\inf(a_n)_{n = 1}^\infty = 0\).
\end{eg}

\begin{proof}
  We first show that \(\sup(a_n)_{n = 1}^\infty = 1\).
  By hypothesis \(\forall n \in \Z^+\) we have \(a_n \leq 1\).
  If \(x \in \R\) and \(x < 1\), then \(x < a_1\), which means \(x\) is not an upper bound of \((a_n)_{n = 1}^\infty\).
  Thus \(\sup(a_n)_{n = 1}^\infty = 1\).

  Now we show that \(\inf(a_n)_{n = 1}^\infty = 0\).
  \(\forall n \in \Z^+\), we have \(-a_n = -1 / n \leq 0\).
  So \(0\) is an upper bound of \(\set{-a_n : n \geq 1}\), and \(\sup(\set{-a_n : n \geq 1}) \leq 0\).
  Then we have
  \begin{align*}
    \inf(a_n)_{n = 1}^\infty & = \inf(\set{a_n : n \geq 1})   &  & \by{6.3.1} \\
                             & = -\sup(-\set{a_n : n \geq 1}) &  & \by{6.2.6} \\
                             & = -\sup(\set{-a_n : n \geq 1}) &  & \by{6.2.6} \\
                             & \geq 0.
  \end{align*}
  So \(0\) is a lower bound of \(\set{a_n : n \geq 1}\), i.e., \(0 \leq \inf(a_n)_{n = 1}^\infty\).
  Suppose for sake of contradiction that \(\exists x \in \R^+\) such that \(x = \inf(a_n)_{n = 1}^\infty\).
  Then by \cref{6.3.7} \(\forall n \in \Z^+\) we must have \(0 < x \leq a_n\).
  But by \cref{5.4.12} \(\exists q \in \Q^+\) such that \(q \leq x\).
  Let such \(q = a / b\) where \(a, b \in \Z^+\).
  Since \(b \in \Z^+\), we have \(1 / (b + 1) \in \set{a_n : n \geq 1}\).
  Since \(1 / (b + 1) < 1 / b \leq a / b\), we have \(1 / (b + 1) < x\), which contradict to \(x \leq a_n\) for every \(n \in \Z^+\).
  Thus \(\nexists x \in \R^+\) such that \(x = \inf(a_n)_{n = 1}^\infty\), therefore \(\inf(a_n)_{n = 1}^\infty = 0\).
\end{proof}

\begin{note}
  It is a little inaccurate to think of the supremum and infimum as the ``largest element of the sequence'' and ``smallest element of the sequence'' respectively.
\end{note}

\begin{note}
  It is possible for the supremum or infimum of a sequence to be \(+\infty\) or \(-\infty\).
  However, if a sequence \((a_n)_{n = m}^\infty\) is bounded, say bounded by \(M\), then all the elements \(a_n\) of the sequence lie between \(-M\) and \(M\), so that the set \(\set{a_n : n \geq m}\) has \(M\) as an upper bound and \(-M\) as a lower bound.
  Since this set is clearly non-empty, we can thus conclude that the supremum and infimum of a bounded sequence are real numbers (i.e., not \(+\infty\) and \(-\infty\)).
\end{note}

\setcounter{thm}{5}
\begin{prop}[Least upper bound property]\label{6.3.6}
  Let \((a_n)_{n = m}^\infty\) be a sequence of real numbers, and let \(x\) be the extended real number \(x \coloneqq \sup(a_n)_{n = m}^\infty\).
  Then we have \(a_n \leq x\) for all \(n \geq m\).
  Also, whenever \(M \in \R^*\) is an upper bound for \(a_n\) (i.e., \(a_n \leq M\) for all \(n \geq m\)), we have \(x \leq M\).
  Finally, for every extended real number \(y\) for which \(y < x\), there exists at least one \(n \geq m\) for which \(y < a_n \leq x\).
\end{prop}

\begin{proof}
  We first show that \(\forall n \geq m\) we have \(a_n \leq x\).
  By \cref{6.3.1} we have \(\sup(a_n)_{n = m}^\infty = \sup(\set{a_n : n \geq m})\).
  Thus by \cref{6.2.11}(a), \(\forall a_n \in \set{a_n : n \geq m}\) we have \(a_n \leq x\).

  Next we show that \(M \in \R^*\) is an upper bound of \((a_n)_{n = m}^\infty\) implies \(x \leq M\).
  By \cref{6.3.1} we have \(\sup(a_n)_{n = m}^\infty = \sup(\set{a_n : n \geq m})\).
  Thus by \cref{6.2.11}(b) we have \(x \leq M\).

  Finally we show that if \(y \in \R^*\) and \(y < x\), then \(\exists n \geq m\) such that \(y < a_n \leq x\).
  Suppose for sake of contradition that such \(n\) does not exist.
  Then \(\forall n \geq m\) we must have \(a_n \leq y < x\).
  But then \(y\) is the least upper bound, a contradiction.
  Thus \(\exists n \geq m\) such that \(y < a_n \leq x\).
\end{proof}

\begin{rmk}\label{6.3.7}
  Let \((a_n)_{n = m}^\infty\) be a sequence of real numbers, and let \(x\) be the extended real number \(x \coloneqq \inf(a_n)_{n = m}^\infty\).
  Then we have \(a_n \geq x\) for all \(n \geq m\).
  Also, whenever \(M \in \R^*\) is an lower bound for \(a_n\) (i.e., \(a_n \geq M\) for all \(n \geq m\)), we have \(x \geq M\).
  Finally, for every extended real number \(y\) for which \(y > x\), there exists at least one \(n \geq m\) for which \(y > a_n \geq x\).
  This is the corresponding Proposition for infima, but with all the references to order reversed, e.g., all upper bounds should now be lower bounds, etc.
  The proof is exactly the same.
\end{rmk}

\begin{note}
  In the previous section we saw that all convergent sequences are bounded.
  It is natural to ask whether the converse is true:
  are all bounded sequences convergent?
  The answer is no;
  for instance, the sequence \(1, -1, 1, -1, \dots\) is bounded, but not Cauchy and hence not convergent.
  However, if we make the sequence both bounded and \emph{monotone} (i.e., increasing or decreasing), then it is true that it must converge.
\end{note}

\begin{prop}[Monotone bounded sequences converge]\label{6.3.8}
  Let \((a_n)_{n = m}^\infty\) be a sequence of real numbers which has some finite upper bound \(M \in \R\), and which is also increasing (i.e., \(a_{n + 1} \geq a_n\) for all \(n \geq m\)).
  Then \((a_n)_{n = m}^\infty\) is convergent, and in fact
  \[
    \lim_{n \to \infty} a_n = \sup(a_n)_{n = m}^\infty \leq M.
  \]
\end{prop}

\begin{proof}
  Since \((a_n)_{n = m}^\infty\) have an upper bound \(M\), by \cref{6.3.1} the set \(E = \set{a_n : n \geq m}\) have an upper bound \(M\).
  By \cref{5.5.9} \(\sup(E)\) must exist and \(\sup(E) \leq M\).
  Now we want to show that \(\lim_{n \to \infty} a_n = \sup(E)\).
  By \cref{6.1.8,6.1.5}, we need to show that \(\forall \varepsilon \in \R^+\), \(\exists N \in \N\) and \(N \geq m\) such that \(\abs{a_n - \sup(E)} \leq \varepsilon\) for every \(n \geq N\).
  Since \(\forall n \geq m\) we have \(a_n \leq \sup(E)\), we must also have \(\abs{a_n - \sup(E)} = \sup(E) - a_n\).
  Thus
  \begin{align*}
             & \forall \varepsilon \in \R^+, -\varepsilon < 0                                                   \\
    \implies & \sup(E) - \varepsilon < \sup(E)                                                                  \\
    \implies & \exists N \geq m : \sup(E) - \varepsilon < a_N \leq \sup(E)          &  & \by{6.3.6}             \\
    \implies & \forall n \geq N : \sup(E) - \varepsilon < a_N \leq a_n \leq \sup(E) &  & \text{(by hypothesis)} \\
    \implies & \sup(E) - \varepsilon \leq a_n                                                                   \\
    \implies & \sup(E) - a_n \leq \varepsilon                                                                   \\
    \implies & \abs{a_n - \sup(E)} \leq \varepsilon
  \end{align*}
  and we conclude that \(\lim_{n \to \infty} a_n = \sup(E) = \sup(a_n)_{n = m}^\infty\).
\end{proof}

\begin{ac}\label{ac:6.3.1}
  Let \((a_n)_{n = m}^\infty\) be a sequence of real numbers which has some finite lower bound \(M \in \R\), and which is also decreasing (i.e., \(a_{n + 1} \leq a_n\) for all \(n \geq m\)).
  Then \((a_n)_{n = m}^\infty\) is convergent, and in fact
  \[
    \lim_{n \to \infty} a_n = \inf(a_n)_{n = m}^\infty \geq M.
  \]
\end{ac}

\begin{proof}
  Since \((a_n)_{n = m}^\infty\) is decreasing, we know that \((-a_n)_{n = m}^\infty\) is increasing since
  \[
    n \leq m \iff a_n \leq a_m \iff -a_n \geq -a_m.
  \]
  Thus we have
  \begin{align*}
             & \forall n \geq m, a_n \geq M                                                           \\
    \implies & -a_n \leq -M                                                                           \\
    \implies & \lim_{n \to \infty} -a_n = \sup(-a_n)_{n = m}^\infty &  & \by{6.3.8}                   \\
    \implies & -\lim_{n \to \infty} a_n = \sup(-a_n)_{n = m}^\infty &  & \text{(by \cref{6.1.19}(c))} \\
    \implies & \lim_{n \to \infty} a_n = -\sup(-a_n)_{n = m}^\infty                                   \\
    \implies & \lim_{n \to \infty} a_n = -\sup\set{-a_n : n \geq m} &  & \by{6.3.1}                   \\
    \implies & \lim_{n \to \infty} a_n = \inf\set{a_n : n \geq m}   &  & \by{6.2.6}                   \\
    \implies & \lim_{n \to \infty} a_n = \inf(a_n)_{n = m}^\infty.  &  & \by{6.3.1}
  \end{align*}
\end{proof}

\begin{note}
  A sequence is said to be \emph{monotone} if it is either increasing or decreasing.
  From \cref{6.3.8} and \cref{6.1.17} we see that a monotone sequence converges iff it is bounded.
\end{note}

\begin{eg}\label{6.3.9}
  The sequence
  \[
    3, 3.1, 3.14, 3.141, 3.1415, \dots
  \]
  is increasing, and is bounded above by \(4\).
  Hence by \cref{6.3.8} it must have a limit, which is a real number less than or equal to \(4\).
\end{eg}

\begin{note}
  \cref{6.3.8} asserts that the limit of a monotone sequence exists, but does not directly say what that limit is.
  Nevertheless, with a little extra work one can often find the limit once one is given that the limit does exist.
\end{note}

\begin{prop}\label{6.3.10}
  Let \(0 < x < 1\).
  Then we have \(\lim_{n \to \infty} x^n = 0\).
\end{prop}

\begin{proof}
  Since \(0 < x < 1\), one can show that the sequence \((x^n)_{n = 1}^\infty\) is decreasing
  (since \(x^{n + 1} / x^n = x < 1\), so \(x^{n + 1} < x^n\)).
  On the other hand, the sequence \((x^n)_{n = 1}^\infty\) has a lower bound of \(0\).
  Thus by \cref{ac:6.3.1} the sequence \((x^n)_{n = 1}^\infty\) converges to some limit \(L\).
  Since \(x^{n + 1} = x \times x^n\), we thus see from the limit laws (\cref{6.1.19}) that \((x^{n + 1})_{n = 1}^\infty\) converges to \(xL\).
  But the sequence \((x^{n + 1})_{n = 1}^\infty\) is just the sequence \((x^n)_{n = 2}^\infty\) shifted by one, and so they must have the same limits by \cref{ex:6.1.3,ex:6.1.4}.
  So \(xL = L\).
  Since \(x \neq 1\), we can solve for \(L\) to obtain \(L = 0\).
  Thus \((x^n)_{n = 1}^\infty\) converges to \(0\).
\end{proof}

\exercisesection

\begin{ex}\label{ex:6.3.1}
  Verify the claim in \cref{6.3.4}.
\end{ex}

\begin{proof}
  See \cref{6.3.4}.
\end{proof}

\begin{ex}\label{ex:6.3.2}
  Prove \cref{6.3.6}.
\end{ex}

\begin{proof}
  See \cref{6.3.6}.
\end{proof}

\begin{ex}\label{ex:6.3.3}
  Prove \cref{6.3.8}.
\end{ex}

\begin{proof}
  See \cref{6.3.8}.
\end{proof}

\begin{ex}\label{ex:6.3.4}
  Explain why \cref{6.3.10} fails when \(x > 1\).
  In fact, show that the sequence \((x_n)_{n = 1}^\infty\) diverges when \(x > 1\).
\end{ex}

\begin{proof}
  Since \(x = x^{n + 1} / x^n > 1\), we have \(x^{n + 1} > x^n\), which means \((x^n)_{n = 1}^\infty\) is increasing.
  Suppose for sake of contradiction that \((x^n)_{n = 1}^\infty\) has an upper bound of \(M\).
  Then by \cref{6.3.8} the sequence \((x^n)_{n = 1}^\infty\) converges to some limit \(L\).
  Since \((1 / x)^n x^n = 1\), we thus see from the limit laws (\cref{6.1.19}) that \(((1 / x)^n x^n)_{n = 1}^\infty\) converges to \(1\).
  But \(0 < (1 / x)^n < 1\), by \cref{6.3.10} we have \(((1 / x)^n)_{n = 1}^\infty\) converges to \(0\).
  So by \cref{6.1.19} we have \(0L = 1\), a contradiction.
  Thus \((x^n)_{n = 1}^\infty\) does not have an upper bound.
  This means \((x^n)_{n = 1}^\infty\) is diverge by \cref{6.1.17}.
\end{proof}

\section{Limsup, Liminf, and limit points}\label{i:sec:6.4}

\begin{defn}[Limit points]\label{i:6.4.1}
  Let \((a_n)_{n = m}^\infty\) be a sequence of real numbers, let \(x\) be a real number, and let \(\varepsilon > 0\) be a real number.
  We say that \(x\) is \emph{\(\varepsilon\)-adherent} to \((a_n)_{n = m}^\infty\) iff there exists an \(n \geq m\) such that \(a_n\) is \(\varepsilon\)-close to \(x\).
  We say that \(x\) is \emph{continually \(\varepsilon\)-adherent} to \((a_n)_{n = m}^\infty\) iff it is \(\varepsilon\)-adherent to \((a_n)_{n = N}^\infty\) for every \(N \geq m\).
  We say that \(x\) is a \emph{limit point} or \emph{adherent point} of \((a_n)_{n = m}^\infty\) iff it is continually \(\varepsilon\)-adherent to \((a_n)_{n = m}^\infty\) for every \(\varepsilon > 0\).
\end{defn}

\begin{rmk}\label{i:6.4.2}
  The verb ``to adhere'' means much the same as ``to stick to'';
  hence the term ``adhesive.''
\end{rmk}

\begin{note}
  Unwrapping all the definitions, we see that \(x\) is a limit point of \((a_n)_{n = m}^\infty\) if, for every \(\varepsilon > 0\) and every \(N \geq m\), there exists an \(n \geq N\) such that \(\abs{a_n - x} \leq \varepsilon\).
  Note the difference between a sequence being \(\varepsilon\)-close to \(L\)
  (which means that \emph{all} the elements of the sequence stay within a distance \(\varepsilon\) of \(L\))
  and \(L\) being \(\varepsilon\)-adherent to the sequence
  (which only needs a \emph{single} element of the sequence to stay within a distance \(\varepsilon\) of \(L\)).
  Also, for \(L\) to be continually \(\varepsilon\)-adherent to \((a_n)_{n = m}^\infty\), it has to be \(\varepsilon\)-adherent to \((a_n)_{n = N}^\infty\) for \emph{all} \(N \geq m\), whereas for \((a_n)_{n = m}^\infty\) to be eventually \(\varepsilon\)-close to \(L\), we only need \((a_n)_{n = m}^\infty\) to be \(\varepsilon\)-close to \(L\) for \emph{some} \(N \geq m\).
  Thus, there are some subtle differences in quantifiers between limits and limit points.
\end{note}

\setcounter{thm}{4}
\begin{prop}[Limits are limit points]\label{i:6.4.5}
  Let \((a_n)_{n = m}^\infty\) be a sequence which converges to a real number \(c\).
  Then \(c\) is a limit point of \((a_n)_{n = m}^\infty\), and in fact it is the only limit point of \((a_n)_{n = m}^\infty\).
\end{prop}

\begin{proof}
  We first show that \(c\) is a limit point of \((a_n)_{n = m}^\infty\).
  Let \(N_1, N_2 \in \N\).
  Since \(\lim_{n \to \infty} a_n = c\), we have \(\forall \varepsilon \in \R^+\), \(\exists N_1 \geq m\) such that \(\abs{a_n - c} \leq \varepsilon\) for every \(n \geq N_1\).
  This means \(\forall N_2 \geq m\), \(\exists n \geq \max(N_1, N_2)\) such that \(\abs{a_n - c} \leq \varepsilon\).
  Thus, by \cref{i:6.4.1} \(c\) is a limit point of \((a_n)_{n = m}^\infty\).

  Now we show that \(c\) is the only limit point of \((a_n)_{n = m}^\infty\).
  Let \(N \in \N\).
  Suppose for sake of contradiction that \(c'\) is also a limit point of \((a_n)_{n = m}^\infty\) and \(c' \neq c\).
  Since \(\lim_{n \to \infty} a_n = c\), we have \(\forall \varepsilon \in \R^+\), \(\exists N \geq m\) such that \(\abs{a_n - c} \leq \varepsilon\) for every \(n \geq N\).
  In particular, \(\abs{a_n - c} \leq \abs{c - c'} / 2\).
  So
  \begin{align*}
    \abs{c - c'} & = \abs{a_n - a_n + c - c'}          \\
                 & = \abs{c - a_n + a_n - c'}          \\
                 & \leq \abs{c - a_n} + \abs{a_n - c'} \\
                 & = \abs{a_n - c} + \abs{a_n - c'}.
  \end{align*}
  This means \(\abs{a_n - c'} \geq \abs{c - c'} - \abs{a_n - c}\).
  Then \(\forall n \geq N\),
  \begin{align*}
    \abs{a_n - c'} & \geq \abs{c - c'} - \abs{a_n - c}    \\
                   & \geq \abs{c - c'} - \abs{c - c'} / 2 \\
                   & = \abs{c - c'} / 2.
  \end{align*}
  Since \(\abs{c - c'} / 2 > 0\), this means we can not find an \(n \geq N\) where \(\abs{a_n - c'} < \abs{c - c'} / 2\), which contradict to the fact that \(c'\) is limit point.
  Thus, \(\lim_{n \to \infty} a_n = c\) is the only limit point of \((a_n)_{n = m}^\infty\).
\end{proof}

\begin{defn}[Limit superior and limit inferior]\label{i:6.4.6}
  Suppose that \((a_n)_{n = m}^\infty\) is a sequence.
  We define a new sequence \((a_N^+)_{N = m}^\infty\) by the formula
  \[
    a_N^+ \coloneqq \sup(a_n)_{n = N}^\infty.
  \]
  More informally, \(a_N^+\) is the supremum of all the elements in the sequence from \(a_N\) onwards.
  We then define the \emph{limit superior} of the sequence \((a_n)_{n = m}^\infty\) by the formula
  \[
    \limsup_{n \to \infty} a_n \coloneqq \inf(a_N^+)_{N = m}^\infty.
  \]
  Similarly, we can define
  \[
    a_N^- \coloneqq \inf(a_n)_{n = N}^\infty
  \]
  and define the \emph{limit inferior} of the sequence \((a_n)_{n = m}^\infty\) by the formula
  \[
    \liminf_{n \to \infty} a_n \coloneqq \sup(a_N^-)_{N = m}^\infty.
  \]
\end{defn}

\setcounter{thm}{10}
\begin{rmk}\label{i:6.4.11}
  Some authors use the notation \(\overline{\lim}_{n \to \infty} a_n\) instead of \(\limsup_{n \to \infty} a_n\), and \(\underline{\lim}_{n \to \infty} a_n\) instead of \(\liminf_{n \to \infty} a_n\).
  Note that the starting index \(m\) of the sequence is irrelevant.
\end{rmk}

\begin{note}
  Returning to the piston analogy, imagine a piston at \(+\infty\) moving leftward until it is stopped by the presence of the sequence \(a_1, a_2, \dots\).
  The place it will stop is the supremum of \(a_1, a_2, a_3, \dots\), which in our new notation is \(a_1^+\).
  Now let us remove the first element \(a_1\) from the sequence;
  this may cause our piston to slip leftward, to a new point \(a_2^+\)
  (though in many cases the piston will not move and \(a_2^+\) will just be the same as \(a_1^+\)).
  Then we remove the second element \(a_2\), causing the piston to slip a little more.
  If we keep doing this the piston will keep slipping, but there will be some point where it cannot go any further, and this is the limit superior of the sequence.
  A similar analogy can describe the limit inferior of the sequence.
\end{note}

\begin{prop}\label{i:6.4.12}
  Let \((a_n)_{n = m}^\infty\) be a sequence of real numbers, let \(L^+\) be the limit superior of this sequence, and let \(L^-\) be the limit inferior of this sequence
  (thus both \(L^+\) and \(L^-\) are extended real numbers).
  \begin{enumerate}
    \item For every \(x > L^+\), there exists an \(N \geq m\) such that \(a_n < x\) for all \(n \geq N\).
          (In other words, for every \(x > L^+\), the elements of the sequence \((a_n)_{n = m}^\infty\) are eventually less than \(x\).)
          Similarly, for every \(y < L^-\) there exists an \(N \geq m\) such that \(a_n > y\) for all \(n \geq N\).
    \item For every \(x < L^+\), and every \(N \geq m\), there exists an \(n \geq N\) such that \(a_n > x\).
          (In other words, for every \(x < L^+\), the elements of the sequence \((a_n)_{n = m}^\infty\) exceed \(x\) infinitely often.)
          Similarly, for every \(y > L^-\) and every \(N \geq m\), there exists an \(n \geq N\) such that \(a_n < y\).
    \item We have \(\inf(a_n)_{n = m}^\infty \leq L^- \leq L^+ \leq \sup(a_n)_{n = m}^\infty\).
    \item If \(c\) is any limit point of \((a_n)_{n = m}^\infty\), then we have \(L^- \leq c \leq L^+\).
    \item If \(L^+\) is finite, then it is a limit point of \((a_n)_{n = m}^\infty\).
          Similarly, if \(L^-\) is finite, then it is a limit point of \((a_n)_{n = m}^\infty\).
    \item Let \(c\) be a real number.
          If \((a_n)_{n = m}^\infty\) converges to \(c\), then we must have \(L^+ = L^- = c\).
          Conversely, if \(L^+ = L^- = c\), then \((a_n)_{n = m}^\infty\) converges to \(c\).
  \end{enumerate}
\end{prop}

\begin{proof}{(a)}
  Suppose first that \(x > L^+\).
  Then by definition of \(L^+\), we have \(x > \inf(a_N^+)_{N = m}^\infty\).
  By \cref{i:6.3.7}, there must then exist an integer \(N \geq m\) such that \(x > a_N^+\).
  By definition of \(a_N^+\), this means that \(x > \sup(a_n)_{n = N}^\infty\).
  Thus, by \cref{i:6.3.6}, we have \(x > a_n\) for all \(n \geq N\), as desired.

  Now suppose that \(y < L^-\).
  Then by definition of \(L^-\), we have \(y < \sup(a_N^-)_{N = m}^\infty\).
  By \cref{i:6.3.6}, there must then exist an integer \(N \geq m\) such that \(y < a_N^-\).
  By definition of \(a_N^-\), this means that \(y < \inf(a_n)_{n = N}^\infty\).
  Thus, by \cref{i:6.3.7}, we have \(y < a_n\) for all \(n \geq N\), as desired.
\end{proof}

\begin{proof}{(b)}
  Suppose that \(x < L^+\).
  Then we have \(x < \inf(a_N^+)_{N = m}^\infty\).
  If we fix any \(N \geq m\), then by \cref{i:6.3.7}, we thus have \(x < a_N^+\).
  By definition of \(a_N^+\), this means that \(x < \sup(a_n)_{n = N}^\infty\).
  By \cref{i:6.3.6}, there must thus exist \(n \geq N\) such that \(a_n > x\), as desired.

  Now suppose that \(y > L^-\).
  Then we have \(y > \sup(a_N^-)_{N = m}^\infty\).
  If we fix any \(N \geq m\), then by \cref{i:6.3.6}, we thus have \(y > a_N^-\).
  By definition of \(a_N^-\), this means that \(y > \inf(a_n)_{n = N}^\infty\).
  By \cref{i:6.3.7}, there must thus exist \(n \geq N\) such that \(a_n < y\), as desired.
\end{proof}

\begin{proof}{(c)}
  Let \(N \in \N\).
  We first show that \(\inf(a_n)_{n = m}^\infty \leq L^-\).
  \begin{align*}
    L^- & = \liminf_{n \to \infty} a_n &  & \by{i:6.4.6} \\
        & = \sup(a_N^-)_{N = m}^\infty &  & \by{i:6.4.6} \\
        & \geq a_m^-                   &  & \by{i:6.3.1} \\
        & = \inf(a_n)_{n = m}^\infty.  &  & \by{i:6.4.6}
  \end{align*}

  Next we show that \(L^+ \leq \sup(a_n)_{n = m}^\infty\).
  \begin{align*}
    L^+ & = \limsup_{n \to \infty} a_n &  & \by{i:6.4.6} \\
        & = \inf(a_N^+)_{N = m}^\infty &  & \by{i:6.4.6} \\
        & \leq a_m^+                   &  & \by{i:6.3.1} \\
        & = \sup(a_n)_{n = m}^\infty.  &  & \by{i:6.4.6}
  \end{align*}

  Finally we show that \(L^- \leq L^+\).
  Let \(N_1, N_2 \in \N\).
  Suppose for sake of contradiction that \(L^- > L^+\).
  Then we have
  \begin{align*}
             & L^+ < L^-                                                                     \\
    \implies & \inf(a_N^+)_{N = m}^\infty < \sup(a_N^-)_{N = m}^\infty     &  & \by{i:6.4.6} \\
    \implies & \exists N_1 \geq m : a_{N_1}^+ < \sup(a_N^-)_{N = m}^\infty &  & \by{i:6.3.7} \\
    \implies & \exists N_2 \geq m : a_{N_1}^+ < a_{N_2}^-                  &  & \by{i:6.3.6} \\
    \implies & \sup(a_n)_{n = N_1}^\infty < \inf(a_n)_{n = N_2}^\infty.    &  & \by{i:6.4.6}
  \end{align*}
  Let \(N = \max(N_1, N_2)\).
  Then we have
  \[
    \sup(a_n)_{n = N}^\infty \leq \sup(a_n)_{n = N_1}^\infty < \inf(a_n)_{n = N_2}^\infty \leq \inf(a_n)_{n = N}^\infty,
  \]
  a contradiction.
  Thus, \(L^- \leq L^+\), and we conclude that
  \[
    \inf(a_n)_{n = m}^\infty \leq L^- \leq L^+ \leq \sup(a_n)_{n = m}^\infty.
  \]
\end{proof}

\begin{proof}{(d)}
  Let \(N, n' \in \N\).
  Suppose that \(c\) is a limit point of \((a_n)_{n = m}^\infty\).
  We first show that \(L^- \leq c\).
  Suppose for sake of contradiction that \(L^- > c\).
  Let \(\varepsilon = (L^- - c) / 2\).
  Since \(c\) is a limit point, by \cref{i:6.4.1} \(\forall N \geq m\), \(\exists n \geq N\) such that \(\abs{a_n - c} \leq \varepsilon\).
  So we have \(-\varepsilon \leq a_n - c \leq \varepsilon\) and thus \(a_n \leq c + \varepsilon\).
  Since \(L^- - \varepsilon < L^-\), by \cref{i:6.4.12}(a) we have \(\exists N \geq m\) such that \(a_{n'} > L^- - \varepsilon\) for every \(n' \geq N\).
  Combine with the above result we get \(L^- - \varepsilon < c + \varepsilon\), but substituting \(\varepsilon\) with \((L^- - c) / 2\) we get \((L^- + c) / 2 < (L^- + c) / 2\), a contradiction.
  Thus, \(L^- \leq c\).

  Now we show that \(L^+ \geq c\).
  Suppose for sake of contradiction that \(L^+ < c\).
  Let \(\varepsilon = (c - L^+) / 2\).
  Since \(c\) is a limit point, by \cref{i:6.4.1} \(\forall N \geq m\), \(\exists n \geq N\) such that \(\abs{a_n - c} \leq \varepsilon\).
  So we have \(-\varepsilon \leq a_n - c \leq \varepsilon\) and thus \(c - \varepsilon \leq a_n\).
  Since \(L^+ < L^+ + \varepsilon\), by \cref{i:6.4.12}(a) we have \(\exists N \geq m\) such that \(a_{n'} < L^+ + \varepsilon\) for every \(n' \geq N\).
  Combine with the above result we get \(c - \varepsilon < L^+ + \varepsilon\), but substituting \(\varepsilon\) with \((c - L^+) / 2\) we get \((c + L^+) / 2 < (c + L^+) / 2\), a contradiction.
  Thus, \(L^+ \geq c\).
  And we conclude that \(L^- \leq c \leq L^+\).
\end{proof}

\begin{proof}{(e)}
  Let \(N, N' \in \N\).
  We first show that if \(L^+\) is finite, then \(L^+\) is a limit point of \((a_n)_{n = m}^\infty\).
  By \cref{i:6.4.1} we need to show that \(\forall \varepsilon \in \R^+\), \(\forall N \geq m\), \(\exists n \geq N\) such that \(\abs{a_n - L^+} \leq \varepsilon\).
  Since \(L^+ < L^+ + \varepsilon\), by \cref{i:6.4.12}(a) \(\exists N' \geq m\) such that \(a_n < L^+ + \varepsilon\) for every \(n \geq N'\).
  But this also means \(\forall N \geq m\), \(\exists n \geq N\) such that \(a_n < L^+ + \varepsilon\) as long as we always choose \(n \geq \max(N', N)\).
  Since \(L^+ - \varepsilon < L^+\), by \cref{i:6.4.12}(b), \(\forall N \geq m\), \(\exists n \geq N\) such that \(a_n > L^+ - \varepsilon\).
  Combine the two statements above we have \(\forall N \geq m\), \(\exists n \geq N\) such that \(L^+ - \varepsilon < a_n < L^+ + \varepsilon\).
  Thus, \(\abs{a_n - L^+} < \varepsilon\) and \(L^+\) is a limit point.

  Now we show that if \(L^-\) is finite, then \(L^-\) is a limit point of \((a_n)_{n = m}^\infty\).
  By \cref{i:6.4.1} we need to show that \(\forall \varepsilon \in \R^+\), \(\forall N \geq m\), \(\exists n \geq N\) such that \(\abs{a_n - L^-} \leq \varepsilon\).
  Since \(L^- - \varepsilon < L^-\), by \cref{i:6.4.12}(a) \(\exists N' \geq m\) such that \(a_n > L^- - \varepsilon\) for every \(n \geq N'\).
  But this also means \(\forall N \geq m\), \(\exists n \geq N\) such that \(a_n > L^- - \varepsilon\) as long as we always choose \(n \geq \max(N', N)\).
  Since \(L^- < L^- + \varepsilon\), by \cref{i:6.4.12}(b), \(\forall N \geq m\), \(\exists n \geq N\) such that \(a_n < L^- + \varepsilon\).
  Combine the two statements above we have \(\forall N \geq m\), \(\exists n \geq N\) such that \(L^- - \varepsilon < a_n < L^- + \varepsilon\).
  Thus, \(\abs{a_n - L^-} < \varepsilon\) and \(L^-\) is also a limit point.
\end{proof}

\begin{proof}{(f)}
  Let \(N, N_1, N_2, n_1, n_2 \in \N\).
  We first show that if \(\lim_{n \to \infty} a_n = c\), then \(L^+ = c\).
  By \cref{i:6.4.5}, \(c\) is the only limit point of \((a_n)_{n = m}^\infty\).
  And by \cref{i:6.4.12}(d), we have \(L^- \leq c \leq L^+\).
  Suppose for sake of contradiction that \(c \neq L^+\).
  Then we must have \(c < L^+\).
  Let \(\varepsilon = (L^+ - c) / 2\).
  Since \(L^+ - \varepsilon < L^+\), by \cref{i:6.4.12}(b), \(\forall N_1 \geq m\), \(\exists n_1 \geq N_1\) such that \(a_{n_1} > L^+ - \varepsilon\).
  Since \(\lim_{n \to \infty} a_n = c\), \(\exists N_2 \geq m\) such that \(\abs{a_{n_2} - c} \leq \varepsilon\) for every \(n_2 \geq N_2\), so \(-\varepsilon \leq a_{n_2} - c \leq \varepsilon\) and \(a_{n_2} \leq c + \varepsilon\).
  By setting \(N = \max(N_1, N_2)\) we have \(L^+ - \varepsilon < a_n \leq c + \varepsilon\), but substituting \(\varepsilon\) with \((L^+ - c) / 2\) we get \((L^+ + c) / 2 < (L^+ + c) / 2\), a contradiction.
  Thus, \(c = L^+\).

  Next we show that if \(\lim_{n \to \infty} a_n = c\), then \(L^- = c\).
  By \cref{i:6.4.5}, \(c\) is the only limit point of \((a_n)_{n = m}^\infty\).
  And by \cref{i:6.4.12}(d), we have \(L^- \leq c \leq L^+\).
  Suppose for sake of contradiction that \(c \neq L^-\).
  Then we must have \(L^- < c\).
  Let \(\varepsilon = (c - L^-) / 2\).
  Since \(L^- < L^- + \varepsilon\), by \cref{i:6.4.12}(b), \(\forall N_1 \geq m\), \(\exists n_1 \geq N_1\) such that \(a_{n_1} < L^- + \varepsilon\).
  Since \(\lim_{n \to \infty} a_n = c\), \(\exists N_2 \geq m\) such that \(\abs{a_{n_2} - c} \leq \varepsilon\) for every \(n_2 \geq N_2\), so \(-\varepsilon \leq a_{n_2} - c \leq \varepsilon\) and \(c - \varepsilon \leq a_{n_2}\).
  By setting \(N = \max(N_1, N_2)\) we have \(c - \varepsilon \leq a_n < L^- + \varepsilon\), but substituting \(\varepsilon\) with \((c - L^-) / 2\) we get \((L^- + c) / 2 < (L^- + c) / 2\), a contradiction.
  Thus, \(c = L^-\).
  We conclude that if \(\lim_{n \to \infty} a_n = c\), then \(L^+ = L^- = c\).

  Finally we show that if \(L^+ = L^- = c\), then \(\lim_{n \to \infty} a_n = c\).
  Let \(\varepsilon \in \R^+\).
  Then we have \(c - \varepsilon < c < c + \varepsilon\).
  Since \(c = L^+\), by \cref{i:6.4.11}(a) \(\exists N_1 \geq m\) such that \(a_{n_1} < c + \varepsilon\) for every \(n_1 \geq N_1\).
  Similarly since \(c = L^-\), by \cref{i:6.4.11}(a) \(\exists N_2 \geq m\) such that \(c - \varepsilon < a_{n_2}\) for every \(n_2 \geq N_2\).
  Let \(N = \max(N_1, N_2)\).
  Then \(\exists N \geq m\) such that \(c - \varepsilon < a_n < c + \varepsilon\) for every \(n \geq N\).
  But this means \(\abs{a_n - c} < \varepsilon\), and thus \(\lim_{n \to \infty} a_n = c\).
  We conclude that \(L^+ = L^- = c \iff \lim_{n \to \infty} a_n = c\).
\end{proof}

\begin{note}
  Parts (d) and (e) of \cref{i:6.4.12} say, in particular, that \(L^+\) is the largest limit point of \((a_n)_{n = m}^\infty\), and \(L^-\) is the smallest limit point
  (providing that \(L^+\) and \(L^-\) are finite).
  \cref{i:6.4.12} (f) then says that if \(L^+\) and \(L^-\) coincide (so there is only one limit point), then the sequence in fact converges.
  This gives a way to test if a sequence converges: compute its limit superior and limit inferior, and see if they are equal.
\end{note}

\begin{lem}[Comparison principle]\label{i:6.4.13}
  Suppose that \((a_n)_{n = m}^\infty\) and \((b_n)_{n = m}^\infty\) are two sequences of real numbers such that \(a_n \leq b_n\) for all \(n \geq m\).
  Then we have the inequalities
  \begin{align*}
    \sup(a_n)_{n = m}^\infty   & \leq \sup(b_n)_{n = m}^\infty   \\
    \inf(a_n)_{n = m}^\infty   & \leq \inf(b_n)_{n = m}^\infty   \\
    \limsup_{n \to \infty} a_n & \leq \limsup_{n \to \infty} b_n \\
    \liminf_{n \to \infty} a_n & \leq \liminf_{n \to \infty} b_n
  \end{align*}
\end{lem}

\begin{proof}
  Let \(n' \in \N\).
  We first show that \(\sup(a_n)_{n = m}^\infty \leq \sup(b_n)_{n = m}^\infty\).
  Suppose for sake of contradiction that \(\sup(a_n)_{n = m}^\infty > \sup(b_n)_{n = m}^\infty\).
  Then by \cref{i:6.3.6} \(\exists n' \geq m\) such that \(a_{n'} > \sup(b_n)_{n = m}^\infty\).
  Also by \cref{i:6.3.1} \(\forall n \geq m\) we have \(\sup(b_n)_{n = m}^\infty \geq b_n\).
  But this means \(a_{n'} > b_n\), in particular, \(a_{n'} > b_{n'}\), a contradiction.
  Thus, we must have \(\sup(a_n)_{n = m}^\infty \leq \sup(b_n)_{n = m}^\infty\).

  Next we show that \(\inf(a_n)_{n = m}^\infty \leq \inf(b_n)_{n = m}^\infty\).
  Suppose for sake of contradiction that \(\inf(a_n)_{n = m}^\infty > \inf(b_n)_{n = m}^\infty\).
  Then by \cref{i:6.3.7} \(\exists n' \geq m\) such that \(b_{n'} < \inf(a_n)_{n = m}^\infty\).
  Also by \cref{i:6.3.1} \(\forall n \geq m\) we have \(\inf(a_n)_{n = m}^\infty \leq a_n\).
  But this means \(b_{n'} < a_n\), in particular, \(b_{n'} < a_{n'}\), a contradiction.
  Thus, we must have \(\inf(a_n)_{n = m}^\infty \leq \inf(b_n)_{n = m}^\infty\).

  Next we show that \(\limsup_{n \to \infty} a_n \leq \limsup_{n \to \infty} b_n\).
  Suppose for sake of contradiction that \(\limsup_{n \to \infty} a_n > \limsup_{n \to \infty} b_n\).
  By \cref{i:6.4.6} we have \(\inf(a_n^+)_{n = m}^\infty > \inf(b_n^+)_{n = m}^\infty\).
  By \cref{i:6.3.7} \(\exists n' \geq m\) such that \(b_{n'}^+ < \inf(a_n^+)_{n = m}^\infty\).
  By \cref{i:6.3.1} \(\forall n \geq m\) we have \(\inf(a_n^+)_{n = m}^\infty \leq a_n^+\).
  But this means \(b_{n'}^+ < a_n^+\), in particular, \(b_{n'}^+ < a_{n'}^+\).
  Again by \cref{i:6.4.6} we have \(\sup(b_n)_{n = n'}^\infty < \sup(a_n)_{n = n'}^\infty\).
  But this contradict to the proof above that \(\sup(b_n)_{n = n'}^\infty \geq \sup(a_n)_{n = n'}^\infty\).
  Thus, we must have \(\limsup_{n \to \infty} a_n \leq \limsup_{n \to \infty} b_n\).

  Finally we show that \(\liminf_{n \to \infty} a_n \leq \liminf_{n \to \infty} b_n\).
  Suppose for sake of contradiction that \(\liminf_{n \to \infty} a_n > \liminf_{n \to \infty} b_n\).
  By \cref{i:6.4.6} we have \(\sup(a_n^-)_{n = m}^\infty > \sup(b_n^-)_{n = m}^\infty\).
  By \cref{i:6.3.6} \(\exists n' \geq m\) such that \(a_{n'}^- > \sup(b_n^-)_{n = m}^\infty\).
  By \cref{i:6.3.1} \(\forall n \geq m\) we have \(\sup(b_n^-)_{n = m}^\infty \geq b_n^-\).
  But this means \(a_{n'}^- > b_n^-\), in particular, \(a_{n'}^- > b_{n'}^-\).
  Again by \cref{i:6.4.6} we have \(\inf(a_n)_{n = n'}^\infty > \inf(b_n)_{n = n'}^\infty\).
  But this contradict to the proof above that \(\inf(a_n)_{n = n'}^\infty \leq \inf(b_n)_{n = n'}^\infty\).
  Thus, we must have \(\liminf_{n \to \infty} a_n \leq \liminf_{n \to \infty} b_n\).
\end{proof}

\begin{cor}[Squeeze test]\label{i:6.4.14}
  Let \((a_n)_{n = m}^\infty\), \((b_n)_{n = m}^\infty\), and \((c_n)_{n = m}^\infty\) be sequences of real numbers such that
  \[
    a_n \leq b_n \leq c_n
  \]
  for all \(n \geq m\).
  Suppose also that \((a_n)_{n = m}^\infty\) and \((c_n)_{n = m}^\infty\) both converge to the same limit \(L\).
  Then \((b_n)_{n = m}^\infty\) is also convergent to \(L\).
\end{cor}

\begin{proof}
  Since \(\lim_{n \to \infty} a_n = \lim_{n \to \infty} c_n = L\), by \cref{i:6.4.12}(f) we have
  \[
    \limsup_{n \to \infty} a_n = \liminf_{n \to \infty} a_n = L = \limsup_{n \to \infty} c_n = \liminf_{n \to \infty} c_n.
  \]
  Thus, we have
  \begin{align*}
             & a_n \leq b_n \leq c_n                                                                                            \\
    \implies & \limsup_{n \to \infty} a_n \leq \limsup_{n \to \infty} b_n \leq \limsup_{n \to \infty} c_n &  & \by{i:6.4.13}    \\
    \implies & L \leq \limsup_{n \to \infty} b_n \leq L                                                   &  & \by{i:6.4.12}[f] \\
    \implies & \limsup_{n \to \infty} b_n = L
  \end{align*}
  and
  \begin{align*}
             & a_n \leq b_n \leq c_n                                                                                            \\
    \implies & \liminf_{n \to \infty} a_n \leq \liminf_{n \to \infty} b_n \leq \liminf_{n \to \infty} c_n &  & \by{i:6.4.13}    \\
    \implies & L \leq \liminf_{n \to \infty} b_n \leq L                                                   &  & \by{i:6.4.12}[f] \\
    \implies & \liminf_{n \to \infty} b_n = L.
  \end{align*}
  Since
  \[
    \limsup_{n \to \infty} b_n = \liminf_{n \to \infty} b_n = L,
  \]
  by \cref{i:6.4.12}(f) we have \(\lim_{n \to \infty} b_n = L\).
\end{proof}

\setcounter{thm}{15}
\begin{rmk}\label{i:6.4.16}
  The squeeze test, combined with the limit laws and the principle that monotone bounded sequences always have limits, allows one to compute a large number of limits.
\end{rmk}

\begin{cor}[Zero test for sequences]\label{i:6.4.17}
  Let \((a_n)_{n = m}^\infty\) be a sequence of real numbers.
  Then the limit \(\lim_{n \to \infty} a_n\) exists and is equal to zero iff the limit \(\lim_{n \to \infty} \abs{a_n}\) exists and is equal to zero.
\end{cor}

\begin{proof}
  Let \(N \in \N\).
  We first show that \(\lim_{n \to \infty} a_n = 0\) implies \(\lim_{n \to \infty} \abs{a_n} = 0\).
  \begin{align*}
             & \lim_{n \to \infty} a_n = 0                                                                             \\
    \implies & \forall \varepsilon \in \R^+, \exists N \geq m : \forall n \geq m, \abs{a_n - 0} \leq \varepsilon       \\
    \implies & \forall \varepsilon \in \R^+, \exists N \geq m : \forall n \geq m, \abs{a_n} \leq \varepsilon           \\
    \implies & \forall \varepsilon \in \R^+, \exists N \geq m : \forall n \geq m, \abs{\abs{a_n} - 0} \leq \varepsilon \\
    \implies & \lim_{n \to \infty} \abs{a_n} = 0.
  \end{align*}

  Now we show that \(\lim_{n \to \infty} \abs{a_n} = 0\) implies \(\lim_{n \to \infty} a_n = 0\).
  Since \(-\abs{a_n} \leq a_n \leq \abs{a_n}\), we have
  \begin{align*}
             & \lim_{n \to \infty} \abs{a_n} = 0                        \\
    \implies & \lim_{n \to \infty} -\abs{a_n} = 0 &  & \by{i:6.1.19}[c] \\
    \implies & \lim_{n \to \infty} a_n = 0.       &  & \by{i:6.4.14}
  \end{align*}
  We conclude that \(\lim_{n \to \infty} a_n = 0 \iff \lim_{n \to \infty} \abs{a_n} = 0\).
\end{proof}

\begin{thm}[Completeness of the reals]\label{i:6.4.18}
  A sequence \((a_n)_{n = 1}^\infty\) of real numbers is a Cauchy sequence iff it is convergent.
\end{thm}

\begin{proof}
  \cref{i:6.1.12} already tells us that every convergent sequence is Cauchy, so it suffices to show that every Cauchy sequence is convergent.

  Let \((a_n)_{n = 1}^\infty\) be a Cauchy sequence.
  We know from \cref{i:5.1.15} (or more precisely, from the extension of this lemma to the real numbers, which is proven in exactly the same fashion) that the sequence \((a_n)_{n = 1}^\infty\) is bounded;
  by \cref{i:6.4.13} (or \cref{i:6.4.12}(c)) this implies that \(L^- \coloneqq \liminf_{n \to \infty} a_n\) and \(L^+ \coloneqq \limsup_{n \to \infty} a_n\) of the sequence are both finite.
  To show that the sequence converges, it will suffice by \cref{i:6.4.12}(f) to show that \(L^- = L^+\).

  Now let \(\varepsilon > 0\) be any real number.
  Since \((a_n)_{n = 1}^\infty\) is a Cauchy sequence, it must be eventually \(\varepsilon\)-steady.
  So, in particular, there exists an \(N \geq 1\) such that the sequence \((a_n)_{n = N}^\infty\) is \(\varepsilon\)-steady.
  In particular, we have \(a_N - \varepsilon \leq a_n \leq a_N + \varepsilon\) for all \(n \geq N\).
  By \cref{i:6.3.6} (or \cref{i:6.4.13}) this implies that
  \[
    a_N - \varepsilon \leq \inf(a_n)_{n = N}^\infty \leq \sup(a_n)_{n = N}^\infty \leq a_N + \varepsilon
  \]
  and hence by the definition of \(L^-\) and \(L^+\) (and \cref{i:6.3.6} again)
  \[
    a_N - \varepsilon \leq L^- \leq L^+ \leq a_N + \varepsilon.
  \]
  Thus, we have
  \[
    0 \leq L^+ - L^- \leq 2\varepsilon.
  \]
  But this is true for all \(\varepsilon > 0\), and \(L^+\) and \(L^-\) do not depend on \(\varepsilon\);
  so we must therefore have \(L^+ = L^-\).
  (If \(L^+ > L^-\) then we could set \(\varepsilon \coloneqq (L^+ - L^-) / 3\) and obtain a contradiction.)
  By \cref{i:6.4.12}(f) we thus see that the sequence converges.
\end{proof}

\begin{rmk}\label{i:6.4.19}
  While \cref{i:6.4.18} is very similar in spirit to \cref{i:6.1.15}, it is a bit more general, since \cref{i:6.1.15} refers to Cauchy sequences of rationals instead of real numbers.
\end{rmk}

\begin{rmk}\label{i:6.4.20}
  In the language of metric spaces, \cref{i:6.4.18} asserts that the real numbers are a \emph{complete} metric space
  - that they do not contain ``holes'' the same way the rationals do.
  (Certainly the rationals have lots of Cauchy sequences which do not converge to other rationals;
  take for instance the sequence \(1, 1.4, 1.41, 1.414, 1.4142, \dots\) which converges to the irrational \(\sqrt{2}\).)
  This property is closely related to the least upper bound property (\cref{i:5.5.9}), and is one of the principal characteristics which make the real numbers superior to the rational numbers for the purposes of doing analysis
  (taking limits, taking derivatives and integrals, finding zeroes of functions, that kind of thing).
\end{rmk}

\exercisesection

\begin{ex}\label{i:ex:6.4.1}
  Prove \cref{i:6.4.5}.
\end{ex}

\begin{proof}
  See \cref{i:6.4.5}.
\end{proof}

\begin{ex}\label{i:ex:6.4.2}
  Let \((a_n)_{n = m}^\infty\) be a sequence of real numbers, let \(c\) be a real number, let \(m' \geq m\) be an integer, and let \(k \geq 0\) be a non-negative integer.
  Show that
  \begin{enumerate}
    \item \(c\) is a limit point of \((a_n)_{n = m}^\infty\) iff \(c\) is a limit point of \((a_n)_{n = m'}^\infty\).
    \item \(c\) is the limit superior of \((a_n)_{n = m}^\infty\) iff \(c\) is the limit superior of \((a_n)_{n = m'}^\infty\).
    \item \(c\) is the limit inferior of \((a_n)_{n = m}^\infty\) iff \(c\) is the limit inferior of \((a_n)_{n = m'}^\infty\).
    \item \(c\) is a limit point of \((a_n)_{n = m}^\infty\) iff \(c\) is a limit point of \((a_{n + k})_{n = m}^\infty\).
    \item \(c\) is the limit superior of \((a_n)_{n = m}^\infty\) iff \(c\) is the limit superior of \((a_{n + k})_{n = m}^\infty\).
    \item \(c\) is the limit inferior of \((a_n)_{n = m}^\infty\) iff \(c\) is the limit inferior of \((a_{n + k})_{n = m}^\infty\).
  \end{enumerate}
\end{ex}

\begin{proof}{(a)}
  Let \(N, \in \N\).
  We first show that \(c\) is a limit point of \((a_n)_{n = m}^\infty\) implies \(c\) is a limit point of \((a_n)_{n = m'}^\infty\).
  Since \(c\) is a limit point of \((a_n)_{n = m}^\infty\), by \cref{i:6.4.1} \(\forall \varepsilon \in \R^+\), we have \(\forall N \geq m\), \(\exists n \geq N\) such that \(\abs{a_n - c} \leq \varepsilon\).
  Since \(m' \geq m\), we must also have \(\forall N \geq m'\), \(\exists n \geq N\) such that \(\abs{a_n - c} \leq \varepsilon\).
  Thus, by \cref{i:6.4.1} \(c\) is a limit point of \((a_n)_{n = m'}^\infty\).

  Now we show that \(c\) is a limit point of \((a_n)_{n = m'}^\infty\) implies \(c\) is a limit point of \((a_n)_{n = m}^\infty\).
  Since \(c\) is a limit point of \((a_n)_{n = m'}^\infty\), by \cref{i:6.4.1} \(\forall \varepsilon \in \R^+\), we have \(\forall N \geq m'\), \(\exists n \geq N\) such that \(\abs{a_n - c} \leq \varepsilon\).
  Since \(m' \geq m\), we only need to show that \(\forall N' \in \N\) and \(m \leq N' < m'\), \(\exists n \geq N'\) such that \(\abs{a_n - c} \leq \varepsilon\).
  Since \(N \geq m' > N'\), we can always find an \(n \geq N > N'\) such that \(\abs{a_n - c} \leq \varepsilon\).
  Thus, by \cref{i:6.4.1} \(c\) is a limit point of \((a_n)_{n = m}^\infty\).
  We conclude that \(c\) is a limit point of \((a_n)_{n = m}^\infty\) iff \(c\) is a limit point of \((a_n)_{n = m'}^\infty\).
\end{proof}

\begin{proof}{(b)}
  We have
  \begin{align*}
             & m \leq m'                                                                 \\
    \implies & \set{a_n : n \geq m'} \subseteq \set{a_n : n \geq m}                      \\
    \implies & \sup(a_n)_{n = m}^\infty \geq \sup(a_n)_{n = m'}^\infty                   \\
    \implies & a_m^+ \geq a_{m'}^+                                     &  & \by{i:6.4.6}
  \end{align*}
  and thus
  \begin{align*}
             & (m \leq m') \land (a_m^+ \geq a_{m'}^+)                                                                      \\
    \implies & (\set{a_N^+ : N \geq m'} \subseteq \set{a_N^+ : N \geq m}) \land (a_m^+ \geq a_{m'}^+)                       \\
    \implies & (\inf(a_N^+)_{N = m}^\infty \leq \inf(a_N^+)_{N = m'}^\infty) \land (a_m^+ \geq a_{m'}^+)                    \\
    \implies & \inf(a_N^+)_{N = m}^\infty = \inf(a_N^+)_{N = m'}^\infty                                                     \\
    \implies & c = \limsup_{n \to \infty} a_n = \inf(a_N^+)_{N = m}^\infty = \inf(a_N^+)_{N = m'}^\infty. &  & \by{i:6.4.6}
  \end{align*}
\end{proof}

\begin{proof}{(c)}
  We have
  \begin{align*}
             & m \leq m'                                                                 \\
    \implies & \set{a_n : n \geq m'} \subseteq \set{a_n : n \geq m}                      \\
    \implies & \inf(a_n)_{n = m}^\infty \leq \inf(a_n)_{n = m'}^\infty                   \\
    \implies & a_m^- \leq a_{m'}^-                                     &  & \by{i:6.4.6}
  \end{align*}
  and thus
  \begin{align*}
             & (m \leq m') \land (a_m^- \leq a_{m'}^-)                                                                          \\
    \implies & \big(\set{a_N^- : N \geq m'} \subseteq \set{a_N^- : N \geq m}\big) \land (a_m^- \leq a_{m'}^-)                   \\
    \implies & (\sup(a_N^-)_{N = m'}^\infty \leq \sup(a_N^-)_{N = m}^\infty) \land (a_m^- \leq a_{m'}^-)                        \\
    \implies & \sup(a_N^-)_{N = m}^\infty = \sup(a_N^-)_{N = m'}^\infty                                                         \\
    \implies & c = \liminf_{n \to \infty} a_n = \sup(a_N^-)_{N = m}^\infty = \sup(a_N^-)_{N = m'}^\infty.     &  & \by{i:6.4.6}
  \end{align*}
\end{proof}

\begin{proof}{(d)}
  Since \((a_{n + k})_{n = m}^\infty = (a_n)_{n = m + k}^\infty\), by \cref{i:ex:6.4.2}(a) we conclude that \(c\) is a limit point of \((a_n)_{n = m}^\infty\) iff \(c\) is a limit point of \((a_{n + k})_{n = m}^\infty\).
\end{proof}

\begin{proof}{(e)}
  Since \((a_{n + k})_{n = m}^\infty = (a_n)_{n = m + k}^\infty\), by \cref{i:ex:6.4.2}(b) we conclude that \(c\) is the limit superior of \((a_n)_{n = m}^\infty\) iff \(c\) is the limit superior of \((a_{n + k})_{n = m}^\infty\).
\end{proof}

\begin{proof}{(f)}
  Since \((a_{n + k})_{n = m}^\infty = (a_n)_{n = m + k}^\infty\), by \cref{i:ex:6.4.2}(c) we conclude that \(c\) is the limit inferior of \((a_n)_{n = m}^\infty\) iff \(c\) is the limit inferior of \((a_{n + k})_{n = m}^\infty\).
\end{proof}

\begin{ex}\label{i:ex:6.4.3}
  Prove \cref{i:6.4.12}.
\end{ex}

\begin{proof}
  See \cref{i:6.4.12}.
\end{proof}

\begin{ex}\label{i:ex:6.4.4}
  Prove \cref{i:6.4.13}.
\end{ex}

\begin{proof}
  See \cref{i:6.4.13}.
\end{proof}

\begin{ex}\label{i:ex:6.4.5}
  Use \cref{i:6.4.13} to prove \cref{i:6.4.14}.
\end{ex}

\begin{proof}
  See \cref{i:6.4.14}.
\end{proof}

\begin{ex}\label{i:ex:6.4.6}
  Give an example of two bounded sequences \((a_n)_{n = 1}^\infty\) and \((b_n)_{n = 1}^\infty\) such that \(a_n < b_n\) for all \(n \geq 1\), but that \(\sup(a_n)_{n = 1}^\infty \not< \sup(b_n)_{n = 1}^\infty\).
  Explain why this does not contradict \cref{i:6.4.13}.
\end{ex}

\begin{proof}
  Let \(a_n = -1 / (n + 1)\) and \(b_n = -1 / (n + 1)^2\).
  Then we have \(a_n < b_n\) for every \(n \geq 1\).
  But \(\sup(a_n)_{n = 1}^\infty = \sup(b_n)_{n = 1}^\infty = 0\).
  This does not contradict to \cref{i:6.4.13} since \(\sup(a_n)_{n = 1}^\infty \leq \sup(b_n)_{n = 1}^\infty\) does not enforce strictly order, i.e., \(\sup(a_n)_{n = 1}^\infty < \sup(b_n)_{n = 1}^\infty\).
\end{proof}

\begin{ex}\label{i:ex:6.4.7}
  Prove \cref{i:6.4.17}.
  Is the corollary still true if we replace zero in the statement of this Corollary by some other number?
\end{ex}

\begin{proof}
  See \cref{i:6.4.17}.

  Now we show that \cref{i:6.4.17} is not true if we replace zero by some other number.
  Let \(N \in \N\) and let \((a_n)_{n = m}^\infty\) be a sequence of reals.
  Suppose that \(\lim_{n \to \infty} \abs{a_n} = c\) for some \(c \in \R^-\).
  Then we have
  \begin{align*}
             & \lim_{n \to \infty} \abs{a_n} = c                                                                       \\
    \implies & \forall \varepsilon \in \R^+, \exists N \geq m : \forall n \geq N, \abs{\abs{a_n} - c} \leq \varepsilon \\
    \implies & -\varepsilon \leq \abs{a_n} - c \leq \varepsilon                                                        \\
    \implies & -\varepsilon + c \leq \abs{a_n} \leq \varepsilon + c.
  \end{align*}
  But by setting \(\varepsilon = -c / 2\) we see that \(\abs{a_n} \leq c / 2 < 0\), a contradiction.

  Now suppose that \(\lim_{n \to \infty} \abs{a_n} = c\) for some \(c \in \R^+\).
  By setting \(a_n = (-1)^n c\) we see that
  \begin{align*}
             & \forall n \geq m, \abs{a_n} = c    \\
    \implies & \lim_{n \to \infty} \abs{a_n} = c.
  \end{align*}
  But we know that \((a_n)_{n = m}^\infty\) is not eventually \(2c\)-close, thus \(\lim_{n \to \infty} a_n\) does not exist.
  We conclude that \cref{i:6.4.17} is not true if we replace zero by some other number.
\end{proof}

\begin{ex}\label{i:ex:6.4.8}
  Let us say that a sequence \((a_n)_{n = m}^\infty\) of real numbers has \(+\infty\) as a limit point iff it has no finite upper bound, and that it has \(-\infty\) as a limit point iff it has no finite lower bound.
  With this definition, show that \(\limsup_{n \to \infty} a_n\) is a limit point of \((a_n)_{n = m}^\infty\), and furthermore that it is larger than all the other limit points of \((a_n)_{n = m}^\infty\);
  in other words, the limit superior is the largest limit point of a sequence.
  Similarly, show that the limit inferior is the smallest limit point of a sequence.
\end{ex}

\begin{proof}
  By \cref{i:6.4.12}(e) we already know the cases where \((a_n)_{n = m}^\infty\) has finite upper bound or lower bound.
  So we only consider the cases where \((a_n)_{n = m}^\infty\) has no finite upper bound or lower bound.
  By \cref{i:6.4.12}(d) we know that \(L^- \leq c \leq L^+\) if \(c\) is a limit point of \((a_n)_{n = m}^\infty\).
  By definition \((a_n)_{n = m}^\infty\) has no finite upper bound iff \(+\infty\) is a limit point.
  This means \(+\infty \leq L^+\), and thus \(L^+ = +\infty\) is the largest limit point.
  Similarly, by definition \((a_n)_{n = m}^\infty\) has no finite lower bound iff \(-\infty\) is a limit point.
  This means \(L^- \leq -\infty\), and thus \(L^- = \infty\) is the smallest limit point.
\end{proof}

\begin{ex}\label{i:ex:6.4.9}
  Using the definition in \cref{i:ex:6.4.8}, construct a sequence \((a_n)_{n = 1}^\infty\) which has exactly three limit points, at \(-\infty\), \(0\) and \(+\infty\).
\end{ex}

\begin{proof}
  Let \((a_n)_{n = 1}^\infty\) be the sequence \(1, 1/2, -3, -1/4, 5, 1/6, -7, -1/8, \dots\).
  Then \((a_n)_{n = 1}^\infty\) has no finite upper bound and lower bound, and has \(0\) as limit point.
\end{proof}

\begin{ex}\label{i:ex:6.4.10}
  Let \((a_n)_{n = N}^\infty\) be a sequence of real numbers, and let \((b_m)_{m = M}^\infty\) be another sequence of real numbers such that each \(b_m\) is a limit point of \((a_n)_{n = N}^\infty\).
  Let \(c\) be a limit point of \((b_m)_{m = M}^\infty\).
  Prove that \(c\) is also a limit point of \((a_n)_{n = N}^\infty\).
  (In other words, limit points of limit points are themselves limit points of the original sequence.)
\end{ex}

\begin{proof}
  Let \(\varepsilon \in \R^+\), and let \(i, j \in \N\).
  Since \(c\) is a limit point of \((b_m)_{m = M}^\infty\), by \cref{i:6.4.1} we have \(\forall i \geq M\), \(\exists m \geq i\) such that \(\abs{b_m - c} \leq \varepsilon / 2\).
  Now we fix such \(m\).
  Since \(b_m\) is a limit point of \((a_n)_{n = N}^\infty\), by \cref{i:6.4.1} we have \(\forall j \geq N\), \(\exists n \geq j\) such that \(\abs{a_n - b_m} \leq \varepsilon / 2\).
  Then we have
  \begin{align*}
    \abs{a_n - c} & = \abs{a_n - c + b_m - b_m}            \\
                  & = \abs{(a_n - b_m) + (b_m - c)}        \\
                  & \leq \abs{a_n - b_m} + \abs{b_m - c}   \\
                  & \leq \varepsilon / 2 + \varepsilon / 2 \\
                  & = \varepsilon
  \end{align*}
  This means \(\forall i \geq N\), \(\exists n \geq i\) such that \(\abs{a_n - c} \leq \varepsilon\).
  Since \(\varepsilon\) was arbitrary, by \cref{i:6.4.1} we know that \(c\) is a limit point of \((a_n)_{n = N}^\infty\).
\end{proof}

\section{Some standard limits}\label{i:sec:6.5}

\begin{ac}\label{i:ac:6.5.1}
  We have
  \[
    \lim_{n \to \infty} c = c
  \]
  for any constant \(c\).
\end{ac}

\begin{proof}
  Let \((a_n)_{n = 1}^\infty\) be a constant sequence where \(a_n = c\) for all \(n \geq 1\), and let \(N \in \N\).
  Then \(\forall \varepsilon \in \R^+\), \(\exists N \geq 1\) such that for every \(n \geq N\),
  \[
    \abs{a_n - c} = \abs{c - c} = 0 \leq \varepsilon.
  \]
  Thus, by \cref{i:6.1.8} we have \(\lim_{n \to \infty} a_n = \lim_{n \to \infty} c = c\).
\end{proof}

\begin{cor}\label{i:6.5.1}
  We have \(\lim_{n \to \infty} 1 / n^{1 / k} = 0\) for every integer \(k \geq 1\).
\end{cor}

\begin{proof}
  From \cref{i:5.6.6} we know that \(1 / n^{1 / k}\) is a decreasing function of \(n\), while being bounded below by \(0\).
  By \cref{i:ac:6.3.1} (for decreasing sequences instead of increasing sequences) we thus know that this sequence converges to some limit \(L \geq 0\):
  \[
    L = \lim_{n \to \infty} 1 / n^{1 / k}.
  \]
  Raising this to the \(k^{th}\) power and using the limit laws (or more precisely, \cref{i:6.1.19}(b) and induction), we obtain
  \[
    L^k = \lim_{n \to \infty} 1 / n.
  \]
  By \cref{i:6.1.11} we thus have \(L^k = 0\);
  but this means that \(L\) cannot be positive (else \(L^k\) would be positive), so \(L = 0\), and we are done.
\end{proof}

\begin{lem}\label{i:6.5.2}
  Let \(x\) be a real number.
  Then the limit \(\lim_{n \to \infty} x^n\) exists and is equal to zero when \(\abs{x} < 1\), exists and is equal to \(1\) when \(x = 1\), and diverges when \(x = -1\) or when \(\abs{x} > 1\).
\end{lem}

\begin{proof}
  We first show that if \(\abs{x} < 1\), then \(\lim_{n \to \infty} x^n = 0\).
  Since \(0 \leq \abs{x} < 1\), we have
  \begin{align*}
             & 0 \leq \abs{x} < 1                                                       \\
    \implies & \lim_{n \to \infty} \abs{x}^n = 0                     &  & \by{i:6.3.10} \\
    \implies & (-1)\lim_{n \to \infty} \abs{x}^n = (-1) \times 0 = 0                    \\
    \implies & \lim_{n \to \infty} -\abs{x}^n = 0.                   &  & \by{i:6.1.19}
  \end{align*}
  So \(\lim_{n \to \infty} -\abs{x}^n = \lim_{n \to \infty} \abs{x}^n\).
  And because \(-\abs{x}^n \leq x^n \leq \abs{x}^n\), by Squeeze test (\cref{i:6.4.14}) we have \(\lim_{n \to \infty} x^n = 0\).

  Next we show that if \(x = 1\), then \(\lim_{n \to \infty} x^n = 1\).
  This is done by \cref{i:ac:6.5.1}.

  Finally we show that if \(x = -1\) or \(\abs{x} > 1\), then \(\lim_{n \to \infty} x^n\) does not exist.
  If \(x = -1\), then we have sequence \(-1, 1, -1, 1, \dots\), which is not eventually \(1\)-steady for any \(n \geq 1\) and thus does not converge to any value.
  If \(\abs{x} > 1\), then we can divide into two cases:
  \begin{itemize}
    \item If \(x > 1\), then by \cref{i:ex:6.3.4} \(\lim_{n \to \infty} x^n\) does not exist.
    \item If \(x < -1\), then \(\forall n \geq 1\) we have
          \begin{align*}
            \abs{x^{n + 1} - x^n} & = \abs{x^n(x - 1)}     \\
                                  & = \abs{x^n}\abs{x - 1} \\
                                  & > \abs{x^n}\abs{-2}    \\
                                  & = 2\abs{x^n}           \\
                                  & > 2.
          \end{align*}
          This means \(x\) is not a Cauchy sequence, so by \cref{i:6.4.18} \(\lim_{n \to \infty} x^n\) does not exist.
  \end{itemize}
  From all cases above we conclude that if \(\abs{x} > 1\) then \(\lim_{n \to \infty} x^n\) does not exist.
\end{proof}

\begin{lem}\label{i:6.5.3}
  For any \(x > 0\), we have \(\lim_{n \to \infty} x^{1 / n} = 1\).
\end{lem}

\begin{proof}
  We first show that \(\forall \varepsilon, M \in \R^+\), \(\exists n \geq 1\) such that \(M^{1 / n} \leq 1 + \varepsilon\).
  Since \(1 / (1 + \varepsilon) < 1\), by \cref{i:6.5.2} we have \(\lim_{n \to \infty} 1 / (1 + \varepsilon)^n = 0\).
  Let \(a_n = 1 / (1 + \varepsilon)^n\).
  Then \(\inf(a_n)_{n = 1}^\infty = 0\).
  Since \(1 / M > 0\), we have \(1 / M > \inf(a_n)_{n = 1}^\infty\).
  By \cref{i:6.3.7} \(\exists n \geq 1\) such that \(a_n = 1 / (1 + \varepsilon)^n < 1 / M\).
  Thus, we have \(M < (1 + \varepsilon)^n\), and by \cref{i:5.6.9}(d) \(M^{1 / n} < 1 + \varepsilon\).

  Now we show that \(\lim_{n \to \infty} x^{1 / n} = 1\).
  We split into two cases:
  \begin{itemize}
    \item If \(x \geq 1\), then by the proof above we have \(\forall \varepsilon \in \R^+\), \(\exists n \geq 1\) such that \(x^{1 / n} < 1 + \varepsilon\).
          Thus
          \begin{align*}
                     & x \geq 1                                                                 \\
            \implies & x^{1 / n} \geq 1^{1 / n} = 1                      &  & \by{i:5.6.6}[d,e] \\
            \implies & \abs{x^{1 / n} - 1} = x^{1 / n} - 1 < \varepsilon
          \end{align*}
          and by \cref{i:6.1.8} \(\lim_{n \to \infty} x^{1 / n} = 1\).
    \item If \(x < 1\), then \(1 / x > 1\).
          So from the proof above we have \(\lim_{n \to \infty} x^{-1 / n} = 1\).
          By \cref{i:6.1.19}(e) we have \(\lim_{n \to \infty} x^{-1 / n} = (\lim_{n \to \infty} x^{1 / n})^{-1} = 1^{-1} = 1\).
  \end{itemize}
  From all cases above we conclude that \(\lim_{n \to \infty} x^{1 / n} = 1\).
\end{proof}

\exercisesection

\begin{ex}\label{i:ex:6.5.1}
  Show that \(\lim_{n \to \infty} 1 / n^q = 0\) for any rational \(q > 0\).
  Conclude that the limit \(\lim_{n \to \infty} n^q\) does not exist.
\end{ex}

\begin{proof}
  We firs show that \(\lim_{n \to \infty} 1 / n^q = 0\) for every \(q \in \Q^+\).
  Let \(q = a / b\) where \(a, b \in \Z^+\).
  Then we have
  \begin{align*}
             & \dfrac{1}{n^q} = \dfrac{1}{n^{a / b}} = \bigg(\dfrac{1}{n^{1 / b}}\bigg)^a                       \\
    \implies & \lim_{n \to \infty} 1 / n^{1 / b} = 0                                      &  & \by{i:6.5.1}     \\
    \implies & \lim_{n \to \infty} (1 / n^{1 / b})^a = 0                                  &  & \by{i:6.1.19}[b] \\
    \implies & \lim_{n \to \infty} 1 / n^q = 0.                                           &  & \by{i:5.6.7}
  \end{align*}

  Now we show that \(\lim_{n \to \infty} n^q\) does not exist.
  Suppose for sake of contradiction that \(\lim_{n \to \infty} n^q\) exists and equals to \(y\).
  Since \(n \geq 1\), we have \(n^q \geq 1\), so \(y \geq 1\).
  Then we have
  \begin{align*}
             & (\lim_{n \to \infty} n^q)^{-1} = y^{-1}     &  & \by{i:6.1.19}[e]        \\
    \implies & \lim_{n \to \infty} (n^q)^{-1} = y^{-1}     &  & \by{i:6.1.19}[e]        \\
    \implies & \lim_{n \to \infty} \dfrac{1}{n^q} = y^{-1}                              \\
    \implies & y^{-1} = 0.                                 &  & \text{(by proof above)}
  \end{align*}
  But this means \(y = 1 / 0\), which means such \(y\) does not exist, a contradiction.
  Thus, \(\lim_{n \to \infty} n^q\) does not exist.
\end{proof}

\begin{ex}\label{i:ex:6.5.2}
  Prove \cref{i:6.5.2}.
\end{ex}

\begin{proof}
  See \cref{i:6.5.2}.
\end{proof}

\begin{ex}\label{i:ex:6.5.3}
  Prove \cref{i:6.5.3}.
\end{ex}

\begin{proof}
  See \cref{i:6.5.3}.
\end{proof}

\section{Subsequences}\label{i:sec:6.6}

\begin{defn}[Subsequences]\label{i:6.6.1}
  Let \((a_n)_{n = 0}^\infty\) and \((b_n)_{n = 0}^\infty\) be sequences of real numbers.
  We say that \((b_n)_{n = 0}^\infty\) is a \emph{subsequence} of \((a_n)_{n = 0}^\infty\) iff there exists a function \(f : \N \to \N\) which is strictly increasing (i.e., \(f(n + 1) > f(n)\) for all \(n \in \N\)) such that
  \[
    b_n = a_{f(n)} \text{ for all } n \in \N.
  \]
\end{defn}

\setcounter{thm}{3}
\begin{lem}\label{i:6.6.4}
  Let \((a_n)_{n = 0}^\infty\) and \((b_n)_{n = 0}^\infty\) be sequences of real numbers.
  Then \((a_n)_{n = 0}^\infty\) is a subsequence of \((a_n)_{n = 0}^\infty\).
  Furthermore, if \((b_n)_{n = 0}^\infty\) is a subsequence of \((a_n)_{n = 0}^\infty\), and \((c_n)_{n = 0}^\infty\) is a subsequence of \((b_n)_{n = 0}^\infty\), then \((c_n)_{n = 0}^\infty\) is a subsequence of \((a_n)_{n = 0}^\infty\).
\end{lem}

\begin{proof}
  We first show that \((a_n)_{n = 0}^\infty\) is a subsequence of \((a_n)_{n = 0}^\infty\).
  Let \(f : \N \to \N\) be a function where \(f(n) = n\).
  Since \(f\) is strictly increasing, by \cref{i:6.6.1} we know that \((a_n)_{n = 0}^\infty\) is a subsequence of \((a_n)_{n = 0}^\infty\).

  Now we show that if \((b_n)_{n = 0}^\infty\) is a subsequence of \((a_n)_{n = 0}^\infty\), and \((c_n)_{n = 0}^\infty\) is a subsequence of \((b_n)_{n = 0}^\infty\), then \((c_n)_{n = 0}^\infty\) is a subsequence of \((a_n)_{n = 0}^\infty\).
  Since \((b_n)_{n = 0}^\infty\) is a subsequence of \((a_n)_{n = 0}^\infty\), by \cref{i:6.6.1} \(\exists f : \N \to \N\) such that \(f\) is strictly increasing and \(b_n = a_{f(n)}\).
  Since \((c_n)_{n = 0}^\infty\) is a subsequence of \((b_n)_{n = 0}^\infty\), by \cref{i:6.6.1} \(\exists g : \N \to \N\) such that \(g\) is strictly increasing and \(c_n = b_{g(n)}\).
  Let \(h = g \circ f\).
  Since \(f\) is strictly increasing, \(\forall n_1, n_2 \in \N\), we have \(n_1 < n_2 \implies f(n_1) < f(n_2)\).
  Since \(g\) is strictly increasing, we have \(f(n_1) < f(n_2) \implies g(f(n_1)) < g(f(n_2))\).
  Thus \(h\) is also strictly increasing, and by \cref{i:6.6.1} \((c_n)_{n = 0}^\infty\) is a subsequence of \((a_n)_{n = 0}^\infty\) where \(c_n = a_{g(f(n))}\).
\end{proof}

\begin{prop}[Subsequences related to limits]\label{i:6.6.5}
  Let \((a_n)_{n = 0}^\infty\) be a sequence of real numbers, and let \(L\) be a real number.
  Then the following two statements are logically equivalent (each one implies the other):
  \begin{enumerate}
    \item The sequence \((a_n)_{n = 0}^\infty\) converges to \(L\).
    \item Every subsequence of \((a_n)_{n = 0}^\infty\) converges to \(L\).
  \end{enumerate}
\end{prop}

\begin{proof}
  We first show that \(\lim_{n \to \infty} a_n = L\) implies every subsequence of \((a_n)_{n = 0}^\infty\) converges to \(L\).
  Let \((b_n)_{n = 0}^\infty\) be a subsequence of \((a_n)_{n = 0}^\infty\), and let \(f : \N \to \N\) be a function where \(b_n = a_{f(n)}\).
  Since \((a_n)_{n = 0}^\infty\) converges to \(L\), we have \(\forall \varepsilon \in \R^+\), \(\exists N \in \N\) such that \(\abs{a_n - L} \leq \varepsilon\) for every \(n \geq N\).
  Since \(f(n) \in \N\), we know that \(\forall f(n) \geq N\), \(\abs{a_{f(n)} - L} = \abs{b_n - L} \leq \varepsilon\).
  Thus \((b_n)_{n = 0}^\infty\) also converges to \(L\).
  Since \((b_n)_{n = 0}^\infty\) was arbitrary, we conclude that \(\lim_{n \to \infty} a_n = L\) implies every subsequence of \((a_n)_{n = 0}^\infty\) converges to \(L\).

  Now we show that every subsequence of \((a_n)_{n = 0}^\infty\) converges to \(L\) implies \(\lim_{n \to \infty} a_n = L\).
  By \cref{i:6.6.4} we know that \((a_n)_{n = 0}^\infty\) is a subsequence of \((a_n)_{n = 0}^\infty\), thus \(\lim_{n \to \infty} a_n = L\).
  We conclude that every subsequence of \((a_n)_{n = 0}^\infty\) converges to \(L\) iff \(\lim_{n \to \infty} a_n = L\).
\end{proof}

\begin{prop}[Subsequences related to limit points]\label{i:6.6.6}
  Let \((a_n)_{n = 0}^\infty\) be a sequence of real numbers, and let \(L\) be a real number.
  Then the following two statements are logically equivalent.
  \begin{enumerate}
    \item \(L\) is a limit point of \((a_n)_{n = 0}^\infty\).
    \item There exists a subsequence of \((a_n)_{n = 0}^\infty\) which converges to \(L\).
  \end{enumerate}
\end{prop}

\begin{proof}
  We first show that \(L\) is a limit point of \((a_n)_{n = 0}^\infty\) implies there exists a subsequence of \((a_n)_{n = 0}^\infty\) which converges to \(L\).
  Since \(L\) is a limit point of \((a_n)_{n = 0}^\infty\), by \cref{i:6.4.1} \(\forall \varepsilon \in \R^+\), \(\forall n_j > 0\), \(\exists n \geq n_j\) such that \(\abs{a_n - L} \leq \varepsilon\).
  In particular, \(\abs{a_n - L} \leq 1 / j\) for every \(j \in \Z^+\).
  Let \(f : \N \to \N\) be a function where \(f(0) = 0\) and \(f(n_j) = \min\set{n > n_{j - 1} : \abs{a_n - L} \leq 1 / j}\).
  Thus such \(f\) is well-defined and \((a_{f(n)})_{n = 1}^\infty\) is a subsequence of \((a_n)_{n = 0}^\infty\).
  Now we show that \(\lim_{n \to \infty} a_{f(n)} = L\).
  By the definition of \(f\) we know that \(\forall j \geq 1\), \(\exists N \in \Z^+\) such that \(\abs{a_{f(n)} - L} \leq 1 / j\) for every \(n \geq N\).
  By Archimedian property (\cref{i:5.4.13}) we know that \(\forall \varepsilon \in \R^+\), \(\exists j \in \Z^+\) such that \(j \varepsilon > 1\).
  Thus \(\forall \varepsilon \in \R^+\), \(\exists N \in \Z^+\) such that \(\abs{a_{f(n)} - L} \leq 1 / j < \varepsilon\) for every \(n \geq N\).
  This means \(\lim_{n \to \infty} a_{f(n)} = L\).

  Now we show that a subsequence of \((a_n)_{n = 0}^\infty\) converges to \(L\) implies \(L\) is a limit point of \((a_n)_{n = 0}^\infty\).
  Let \((b_n)_{n = 0}^\infty\) be a subsequence of \((a_n)_{n = 0}^\infty\) and \(\lim_{n \to \infty} b_n = L\).
  Let \(f : \N \to \N\) be a function where \(b_n = a_{f(n)}\).
  Since \(\lim_{n \to \infty} b_n = L\), \(\forall \varepsilon \in \R^+\), \(\exists n \in \N\) such that \(\abs{b_n - L} \leq \varepsilon\).
  This means \(\forall N \in \N\), \(\exists n \geq N\) such that \(\abs{b_n - L} = \abs{a_{f(n)} - L} \leq \varepsilon\).
  Thus by \cref{i:6.4.1} \(L\) is a limit point of \((a_n)_{n = 0}^\infty\).
\end{proof}

\begin{rmk}\label{i:6.6.7}
  \cref{i:6.6.5,i:6.6.6} give a sharp contrast between the notion of a limit, and that of a limit point.
  When a sequence has a limit \(L\), then \emph{all} subsequences also converge to \(L\).
  But when a sequence has \(L\) as a limit point, then only \emph{some} subsequences converge to \(L\).
\end{rmk}

\begin{note}
  We can now prove an important theorem in real analysis, due to Bernard Bolzano (1781--1848) and Karl Weierstrass (1815--1897):
  every bounded sequence has a convergent subsequence.
\end{note}

\begin{thm}[Bolzano-Weierstrass theorem]\label{i:6.6.8}
  Let \((a_n)_{n = 0}^\infty\) be a bounded sequence
  (i.e., there exists a real number \(M > 0\) such that \(\abs{a_n} \leq M\) for all \(n \in \N\)).
  Then there is at least one subsequence of \((a_n)_{n = 0}^\infty\) which converges.
\end{thm}

\begin{proof}
  Let \(L\) be the limit superior of the sequence \((a_n)_{n = 0}^\infty\).
  Since we have \(-M \leq a_n \leq M\) for all natural numbers \(n\), it follows from the comparison principle (\cref{i:6.4.13}) that \(-M \leq L \leq M\).
  In particular, \(L\) is a real number (not \(+\infty\) or \(-\infty\)).
  By \cref{i:6.4.12}(e), \(L\) is thus a limit point of \((a_n)_{n = 0}^\infty\).
  Thus by \cref{i:6.6.6}, there exists a subsequence of \((a_n)_{n = 0}^\infty\) which converges
  (in fact, it converges to \(L\)).
\end{proof}

\begin{note}
  we could as well have used the limit inferior instead of the limit superior in the argument of \cref{i:6.6.8}.
\end{note}

\begin{rmk}\label{i:6.6.9}
  The Bolzano-Weierstrass theorem says that if a sequence is bounded, then eventually it has no choice but to converge in some places;
  it has ``no room'' to spread out and stop itself from acquiring limit points.
  It is not true for unbounded sequences;
  In the language of topology, this means that the interval \(\set{x \in \R : -M \leq x \leq M}\) is \emph{compact}, whereas an unbounded set such as the real line \(\R\) is not compact.
\end{rmk}

\exercisesection

\begin{ex}\label{i:ex:6.6.1}
  Prove \cref{i:6.6.4}.
\end{ex}

\begin{proof}
  See \cref{i:6.6.4}.
\end{proof}

\begin{ex}\label{i:ex:6.6.2}
  Can you find two sequences \((a_n)_{n = 0}^\infty\) and \((b_n)_{n = 0}^\infty\) which are not the same sequence, but such that each is a subsequence of the other?
\end{ex}

\begin{proof}
  Let \((a_n)_{n = 0}^\infty = \set{0, 1, 0, 1, \dots}\) and \((b_n)_{n = 0}^\infty = \set{1, 0, 1, 0, \dots}\).
\end{proof}

\begin{ex}\label{i:ex:6.6.3}
  Let \((a_n)_{n = 0}^\infty\) be a sequence which is not bounded.
  Show that there exists a subsequence \((b_n)_{n = 0}^\infty\) of \((a_n)_{n = 0}^\infty\) such that \(\lim_{n \to \infty} 1 / b_n\) exists and is equal to zero.
\end{ex}

\begin{proof}
  Let \(j \in \N\), and let \(f(n) : \N \to \N\) be a function defined as follow:
  \begin{align*}
    f(0)   & = \min\set{n \in \N : \abs{a_n} \geq 0}                         \\
    f(n_j) & = \min\set{n \in \N : (\abs{a_n} \geq j) \land (n > n_{j - 1})}
  \end{align*}
  Since \((a_n)_{n = 0}^\infty\) is not bounded, we know that \(\exists n \in \N\) such that \(\abs{a_n} \geq j\) for every \(j \in \N\).
  Thus by \cref{i:5.5.9} such \(f\) is well-defined.
  Let \((b_n)_{n = 1}^\infty\) be a subsequence of \((a_n)_{n = 0}^\infty\) where \(b_n = a_{f(n)}\).
  Then \(0 \leq \abs{1 / b_n} \leq 1 / n\).
  Since \(\lim_{n \to \infty} 0 = 0\) and \(\lim_{n \to \infty} 1 / n = 0\), by squeeze test (\cref{i:6.4.14}) we have \(\lim_{n \to \infty} \abs{1 / b_n} = 0\).
  Since \(\lim_{n \to \infty} \abs{1 / b_n} = 0\), by zero test (\cref{i:6.4.17}) we have \(\lim_{n \to \infty} 1 / b_n = 0\).
  Thus there exists a subsequence \((b_n)_{n = 0}^\infty\) of \((a_n)_{n = 0}^\infty\) such that \(\lim_{n \to \infty} 1 / b_n = 0\)
\end{proof}

\begin{ex}\label{i:ex:6.6.4}
  Prove \cref{i:6.6.5}.
\end{ex}

\begin{proof}
  See \cref{i:6.6.5}.
\end{proof}

\begin{ex}\label{i:ex:6.6.5}
  Prove \cref{i:6.6.6}.
\end{ex}

\begin{proof}
  See \cref{i:6.6.6}.
\end{proof}

\section{Real exponentiation, part II}\label{i:sec:6.7}

\begin{lem}[Continuity of exponentiation]\label{i:6.7.1}
  Let \(x > 0\), and let \(\alpha\) be a real number.
  Let \((q_n)_{n = 1}^\infty\) be any sequence of rational numbers converging to \(\alpha\).
  Then \((x^{q_n})_{n = 1}^\infty\) is also a convergent sequence.
  Furthermore, if \((q_n')_{n = 1}^\infty\) is any other sequence of rational numbers converging to \(\alpha\), then \((x^{q_n'})_{n = 1}^\infty\) has the same limit as \((x^{q_n})_{n = 1}^\infty\):
  \[
    \lim_{n \to \infty} x^{q_n} = \lim_{n \to \infty} x^{q_n'}.
  \]
\end{lem}

\begin{proof}
  There are three cases: \(x < 1\), \(x = 1\), and \(x > 1\).
  The case \(x = 1\) is rather easy (because then \(x^q = 1\) for all rational \(q\)).

  We first show that if \(x > 1\) then \((x^{q_n})_{n = 1}^\infty\) converges.
  By \cref{i:6.4.18} it is enough to show that \((x^{q_n})_{n = 1}^\infty\) is a Cauchy sequence.

  To do this, we need to estimate the distance between \(x^{q_n}\) and \(x^{q_m}\);
  let us say for the time being that \(q_n \geq q_m\), so that \(x^{q_n} \geq x^{q_m}\) (since \(x > 1\)).
  We have
  \[
    d(x^{q_n}, x^{q_m}) = x^{q_n} - x^{q_m} = x^{q_m} (x^{q_n - q_m} - 1).
  \]
  Since \((q_n)_{n = 1}^\infty\) is a convergent sequence, it has some upper bound \(M\);
  since \(x > 1\), we have \(x^{q_m} \leq x^M\).
  Thus
  \[
    d(x^{q_n}, x^{q_m}) = \abs{x^{q_n} - x^{q_m}} \leq x^M (x^{q_n - q_m} - 1).
  \]
  Now let \(\varepsilon > 0\).
  We know by \cref{i:6.5.3} that the sequence \((x^{1 / k})_{k = 1}^\infty\) is eventually \(\varepsilon x^{-M}\)-close to \(1\).
  Thus, there exists some \(K \geq 1\) such that
  \[
    \abs{x^{1 / K} - 1} \leq \varepsilon x^{-M}.
  \]
  Now since \((q_n)_{n = 1}^\infty\) is convergent, it is a Cauchy sequence, and so there is an \(N \geq 1\) such that \(q_n\) and \(q_m\) are \(1 / K\)-close for all \(n, m \geq N\).
  Thus, we have
  \[
    d(x^{q_n}, x^{q_m}) \leq x^M (x^{q_n - q_m} - 1) \leq x^M (x^{1 / K} - 1) \leq x^M \varepsilon x^{-M} = \varepsilon.
  \]
  for every \(n, m \geq N\) such that \(q_n \geq q_m\).
  By symmetry we also have this bound when \(n, m \geq N\) and \(q_n \leq q_m\).
  Thus, the sequence \((x^{q_n})_{n = 1}^\infty\) is \(\varepsilon\)-steady.
  Thus, the sequence \((x^{q_n})_{n = 1}^\infty\) is eventually \(\varepsilon\)-steady for every \(\varepsilon > 0\), and is thus a Cauchy sequence as desired.
  This proves the convergence of \((x^{q_n})_{n = 1}^\infty\) when \(x > 1\).

  Next we show that if \(x < 1\) then \((x^{q_n})_{n = 1}^\infty\) also converges.
  By \cref{i:6.4.18} it is enough to show that \((x^{q_n})_{n = 1}^\infty\) is a Cauchy sequence.

  To do this, we need to estimate the distance between \(x^{q_n}\) and \(x^{q_m}\);
  let us say for the time being that \(q_n \leq q_m\), so that \(x^{q_n} \geq x^{q_m}\) (since \(x < 1\)).
  We have
  \[
    d(x^{q_n}, x^{q_m}) = x^{q_n} - x^{q_m} = x^{q_m} (x^{q_n - q_m} - 1).
  \]
  Since \((q_n)_{n = 1}^\infty\) is a convergent sequence, it has some lower bound \(M\);
  since \(x < 1\), we have \(x^{q_m} \leq x^M\).
  Thus
  \[
    d(x^{q_n}, x^{q_m}) = \abs{x^{q_n} - x^{q_m}} \leq x^M (x^{q_n - q_m} - 1).
  \]
  Now let \(\varepsilon > 0\).
  We know by \cref{i:6.5.3} that the sequence \((x^{1 / k})_{k = 1}^\infty\) is eventually \(\varepsilon x^{-M}\)-close to \(1\).
  Thus, there exists some \(K \geq 1\) such that
  \[
    \abs{x^{1 / K} - 1} \leq \varepsilon x^{-M}.
  \]
  Now since \((q_n)_{n = 1}^\infty\) is convergent, it is a Cauchy sequence, and so there is an \(N \geq 1\) such that \(q_n\) and \(q_m\) are \(1 / K\)-close for all \(n, m \geq N\).
  Thus, we have
  \[
    d(x^{q_n}, x^{q_m}) \leq x^M (x^{q_n - q_m} - 1) \leq x^M (x^{1 / K} - 1) \leq x^M \varepsilon x^{-M} = \varepsilon.
  \]
  for every \(n, m \geq N\) such that \(q_n \leq q_m\).
  By symmetry we also have this bound when \(n, m \geq N\) and \(q_n \geq q_m\).
  Thus, the sequence \((x^{q_n})_{n = 1}^\infty\) is \(\varepsilon\)-steady.
  Thus, the sequence \((x^{q_n})_{n = 1}^\infty\) is eventually \(\varepsilon\)-steady for every \(\varepsilon > 0\), and is thus a Cauchy sequence as desired.
  This proves the convergence of \((x^{q_n})_{n = 1}^\infty\) when \(x < 1\).

  Now we prove the second claim.
  It will suffice to show that
  \[
    \lim_{n \to \infty} x^{q_n - q_n'} = 1,
  \]
  since the claim would then follow from limit laws
  (since \(x^{q_n} = x^{q_n - q_n'} x^{q_n'}\)).

  Write \(r_n \coloneqq q_n - q_n'\);
  by limit laws we know that \((r_n)_{n = 1}^\infty\) converges to \(0\).
  We have to show that for every \(\varepsilon > 0\), the sequence \((x^{r_n})_{n = 1}^\infty\) is eventually \(\varepsilon\)-close to \(1\).
  But from \cref{i:6.5.3} we know that the sequence \((x^{1 / k})_{k = 1}^\infty\) is eventually \(\varepsilon\)-close to \(1\).
  Since \(\lim_{k \to \infty} x^{-1 / k}\) is also equal to \(1\) by \cref{i:6.5.3}, we know that \((x^{-1 / k})_{k = 1}^\infty\) is also eventually \(\varepsilon\)-close to \(1\).
  Thus, we can find a \(K\) such that \(x^{1 / K}\) and \(x^{-1 / K}\) are both \(\varepsilon\)-close to \(1\).
  But since \((r_n)_{n = 1}^\infty\) is convergent to \(0\), it is eventually \(1 / K\)-close to \(0\), so that eventually \(-1 / K \leq r_n \leq 1 / K\), and thus when \(x > 1\) we have \(x^{-1 / K} \leq x^{r_n} \leq x^{1 / K}\), when \(x < 1\) we have \(x^{1 / K} \leq x^{r_n} \leq x^{-1 / K}\).
  In particular, \(x^{r_n}\) is also eventually \(\varepsilon\)-close to \(1\) (see \cref{i:4.3.7}(f)), as desired.
\end{proof}

\begin{defn}[Exponentiation to a real exponent]\label{i:6.7.2}
  Let \(x > 0\) be real, and let \(\alpha\) be a real number.
  We define the quantity \(x^\alpha\) by the formula \(x^\alpha = \lim_{n \to \infty} x^{q_n}\), where \((q_n)_{n = 1}^\infty\) is any sequence of rational numbers converging to \(\alpha\).
\end{defn}

\begin{note}
  Let us check that \cref{i:6.7.2} is well-defined.
  First of all, given any real number \(\alpha\) we always have at least one sequence \((q_n)_{n = 1}^\infty\) of rational numbers converging to \(\alpha\), by the definition of real numbers (and \cref{i:6.1.15}).
  Secondly, given any such sequence \((q_n)_{n = 1}^\infty\), the limit \(\lim_{n \to \infty} x^{q_n}\) exists by \cref{i:6.7.1}.
  Finally, even though there can be multiple choices for the sequence \((q_n)_{n = 1}^\infty\), they all give the same limit by \cref{i:6.7.1}.
  Thus, \cref{i:6.7.2} is well-defined.
\end{note}

\begin{note}
  If \(\alpha\) is not just real but rational, i.e., \(\alpha = q\) for some rational \(q\), then \cref{i:6.7.2} could in principle be inconsistent with our earlier definition of exponentiation in \cref{i:sec:5.6}.
  But in this case \(\alpha\) is clearly the limit of the sequence \((q)_{n = 1}^\infty\), so by definition \(x^\alpha = \lim_{n \to \infty} x^q = x^q\).
  Thus, the new definition of exponentiation is consistent with the old one.
\end{note}

\begin{prop}\label{i:6.7.3}
  All the results of \cref{i:5.6.9}, which held for rational numbers \(q\) and \(r\), continue to hold for real numbers \(q\) and \(r\).
\end{prop}

\begin{proof}{(a)}
  Let \(r\) be a real number.
  Then we can write \(r = \lim_{n \to \infty} r_n\) for some sequences \((r_n)_{n = 1}^\infty\) of rationals, by the definition of real numbers (and \cref{i:6.1.15}).
  Since \((r_n)_{n = 1}^\infty\) is a Cauchy sequence, it is bounded by some \(M \in \Q^+\), i.e, \(-M \leq r_n \leq M\) for every \(n \geq 1\).
  By \cref{i:5.6.9}, both \(x^M\) and \(x^{-M}\) are positive real numbers.
  If \(0 < x < 1\), then \(x^M \leq x^{r_n} \leq x^{-M}\).
  If \(x \geq 1\), then \(x^{-M} \leq x^{r_n} \leq x^M\).
  By \cref{i:6.1.19}(h) we have
  \begin{align*}
     & \lim_{n \to \infty} \min(x^{-M}, x^M, x^{r_n})                                           \\
     & = \min(\lim_{n \to \infty} x^{-M}, \lim_{n \to \infty} x^M, \lim_{n \to \infty} x^{r_n}) \\
     & = \min(x^M, x^{-M}).
  \end{align*}
  Since \(\min(x^M, x^{-M})\) is positive real number, we know that \(x^r\) must also be a positive real number.
\end{proof}

\begin{proof}{(b)}
  Let \(q\) and \(r\) be real numbers.
  Then we can write \(q = \lim_{n \to \infty} q_n\) and \(r = \lim_{n \to \infty} r_n\) for some sequences \((q_n)_{n = 1}^\infty\) and \((r_n)_{n = 1}^\infty\) of rationals, by the definition of real numbers (and \cref{i:6.1.15}).
  Then by the limit laws, \(q + r\) is the limit of \((q_n + r_n)_{n = 1}^\infty\).
  By definition of real exponentiation, we have
  \[
    x^{q + r} = \lim_{n \to \infty} x^{q_n + r_n} ; x^q = \lim_{n \to \infty} x^{q_n} ;  x^r = \lim_{n \to \infty} x^{r_n}.
  \]
  But by \cref{i:5.6.9}(b) (applied to \emph{rational} exponents) we have \(x^{q_n + r_n} = x^{q_n} x^{r_n}\).
  Thus, by limit laws we have \(x^{q + r} = x^q x^r\), as desired.

  Now we show that \((x^q)^r = \lim_{n \to \infty} (x^{q_n})^{r_n}\).
  By \cref{i:6.1.19}(c) we know that \(q r_n = \lim_{m \to \infty} q_m r_n\).
  Thus, we have
  \begin{align*}
    (x^q)^r & = \lim_{n \to \infty} (x^q)^{r_n}                         &  & \by{i:6.7.2}    \\
            & = \lim_{n \to \infty} (\lim_{m \to \infty} x^{q_m})^{r_n} &  & \by{i:6.7.2}    \\
            & = \lim_{n \to \infty} (\lim_{m \to \infty} x^{q_m r_n})   &  & \by{i:5.6.9}[b] \\
            & = \lim_{n \to \infty} x^{q r_n}                           &  & \by{i:6.7.2}    \\
            & = x^{qr}.                                                 &  & \by{i:6.7.2}
  \end{align*}
\end{proof}

\begin{proof}{(c)}
  Let \(r \in \R\) where \(r = \lim_{n \to \infty} r_n\) for some sequences \((r_n)_{n = 1}^\infty\) of rationals.
  By \cref{i:6.1.15} \(r\) is well-defined and by \cref{i:6.1.19}(c) we know that \(-r\) is the limit of \((-r_n)_{n = 1}^\infty\).
  By \cref{i:6.7.3}(a) we know that \(x^{-r} > 0\) and by \cref{i:5.6.9}(a) we know that \(x^{r_n} > 0\) for every \(n \in \Z^+\).
  Thus, we have
  \begin{align*}
    x^{-r} & = \lim_{n \to \infty} x^{-r_n}    &  & \by{i:6.7.2}     \\
           & = 1 / \lim_{n \to \infty} x^{r_n} &  & \by{i:6.1.19}[e] \\
           & = 1 / x^r.                        &  & \by{i:6.7.2}
  \end{align*}
\end{proof}

\begin{proof}{(d)}
  Let \(x, y, r \in \R^+\) where \(r = \lim_{n \to \infty} r_n\) for some sequences \((r_n)_{n = 1}^\infty\) of rationals.
  By \cref{i:6.1.15} \(r\) is well-defined.
  Since \(r \in \R^+\), by \cref{i:5.4.3} we know that \(\exists c \in \R^+\) such that \(r_n \geq c\) for every \(n \in \Z^+\).

  We first show that \(x > y \implies x^r > y^r\).
  \begin{align*}
             & x > y                                                                             \\
    \implies & \forall n \in \Z^+, x^{r_n} > y^{r_n}                        &  & \by{i:5.6.9}[d] \\
    \implies & \lim_{n \to \infty} x^{r_n} \geq \lim_{n \to \infty} y^{r_n} &  & \by{i:6.4.13}   \\
    \implies & x^r \geq y^r.                                                &  & \by{i:6.7.2}
  \end{align*}
  Now we show that \(x^r \neq y^r\).
  Suppose for the sake of contradiction that \(x^r = y^r\).
  Then we have
  \begin{align*}
             & x^r = y^r                                                                                                                        \\
    \implies & \lim_{n \to \infty} x^{r_n} = \lim_{n \to \infty} y^{r_n}                                                      &  & \by{i:6.7.2} \\
    \implies & \forall \varepsilon \in \R^+, \exists N \in \Z^+ : \forall n \geq N, \abs{x^{r_n} - y^{r_n}} \leq \varepsilon. &  & \by{i:5.3.1}
  \end{align*}
  But by \cref{i:5.6.9}(d) we know that \(x > y \implies x^{r_n} > y^{r_n}\), thus by setting \(\varepsilon = (x^{r_n} - y^{r_n}) / 2\) we get
  \[
    \abs{x^{r_n} - y^{r_n}} = x^{r_n} - y^{r_n} \leq \dfrac{x^{r_n} - y^{r_n}}{2},
  \]
  a contradiction.
  Thus, we must have \(x^r > y^r\).

  Finally we show that \(x^r > y^r \implies x > y\).
  Suppose for the sake of contradiction that \(x \leq y\).
  Then from the proof above we know that \(x^r \leq y^r\), a contradiction.
  Thus, we must have \(x > y\).
  We conclude that if \(r > 0\), then \(x > y \iff x^r > y^r\).
\end{proof}

\begin{proof}{(e)}
  Let \(x, q, r \in \R^+\) where \(q = \lim_{n \to \infty} q_n\) and \(r = \lim_{n \to \infty} r_n\) for some sequences \((q_n)_{n = 1}^\infty\), \((r_n)_{n = 1}^\infty\) of rationals.
  By \cref{i:6.1.15} \(q, r\) are well-defined.
  Since \(q, r \in \R^+\), by \cref{i:5.4.3} we know that \(\exists c_1, c_2 \in \R^+\) such that \(q_n \geq c_1\) and \(r_n \geq c_2\) for every \(n \in \Z^+\).
  Thus, by \cref{i:5.6.9}(a) we know that \(x^{q_n}, x^{r_n} > 0\) for every \(n \in \Z^+\).

  We first show that if \(x > 1\), then \(x^q > x^r \implies q > r\).
  We have
  \begin{align*}
             & x^q > x^r > 0                          &  & \by{i:6.7.3}[a]   \\
    \implies & x^{q - r} > 1                          &  & \by{i:6.7.3}[b,c] \\
    \implies & \lim_{n \to \infty} x^{q_n - r_n} > 1. &  & \by{i:6.7.2}      \\
  \end{align*}
  Now we show that \(\exists N \in \Z^+\) such that \(q_n - r_n > 0\) for every \(n \geq N\).
  Suppose for the sake of contradiction that \(\forall N \in \Z^+\), \(\exists n \geq N\) such that \(q_n - r_n \leq 0\).
  Then we have
  \begin{align*}
             & q_n - r_n \leq 0                                               \\
    \implies & r_n - q_n \geq 0                                               \\
    \implies & x^{r_n - q_n} \geq 1^{r_n - q_n} = 1      &  & \by{i:6.7.3}[d] \\
    \implies & x^{q_n - r_n} \leq 1                      &  & \by{i:6.7.3}[c] \\
    \implies & \lim_{n \to \infty} x^{q_n - r_n} \leq 1, &  & \by{i:6.4.13}
  \end{align*}
  which contradict to \(\lim_{n \to \infty} x^{q_n - r_n} > 1\).
  Thus, \(\exists N \in \Z^+\) such that \(q_n - r_n > 0\) for every \(n \geq N\).
  This means \(q_n > r_n\) for every \(n \geq N\), and by \cref{i:6.4.13} we know that \(q = \lim_{n \to \infty} q_n \geq \lim_{n \to \infty} r_n = r\).

  Next we show that if \(x > 1\), then \(q > r \implies x^q > x^r\).
  \begin{align*}
             & q > r                                                                                          \\
    \implies & q - r > 0                                                                                      \\
    \implies & x^{q - r} > 1^{q - r}                                                     &  & \by{i:6.7.3}[d] \\
    \implies & x^{q - r} > \lim_{n \to \infty} 1^{q_n - r_n} = \lim_{n \to \infty} 1 = 1 &  & \by{i:6.7.2}    \\
    \implies & x^{q - r} x^r > x^r                                                       &  & \by{i:6.7.3}[a] \\
    \implies & x^{q - r + r} > x^r                                                       &  & \by{i:6.7.3}[b] \\
    \implies & x^q > x^r.                                                                &  & \by{i:5.6.8}
  \end{align*}
  Thus, we conclude that if \(x > 1\), then \(x^q > x^r \iff q > r\).

  Finally we show that if \(x < 1\), then \(x^q > x^r \iff q < r\).
  \begin{align*}
             & x < 1                                                                           \\
    \implies & x^{-1} > 1                                                                      \\
    \implies & \big((x^{-1})^q < (x^{-1})^r \iff q < r\big) &  & \text{(from the proof above)} \\
    \implies & (x^{-q} < x^{-r} \iff q < r)                 &  & \by{i:6.7.3}[b]               \\
    \implies & (x^q > x^r \iff q < r).                      &  & \by{i:6.7.3}[a,c]
  \end{align*}
\end{proof}

\begin{proof}{(f)}
  Let \(x, y, r \in \R^+\) where \(r = \lim_{n \to \infty} r_n\) for some sequences \((r_n)_{n = 1}^\infty\) of rationals.
  By \cref{i:6.1.15} \(r\) is well-defined.
  Then we have
  \begin{align*}
    (xy)^r & = \lim_{n \to \infty} (xy)^{r_n}                             &  & \by{i:6.7.2}     \\
           & = \lim_{n \to \infty} x^{r_n} y^{r_n}                        &  & \by{i:5.6.9}[f]  \\
           & = (\lim_{n \to \infty} x^{r_n})(\lim_{n \to \infty} y^{r_n}) &  & \by{i:6.1.19}[b] \\
           & = x^r y^r.                                                   &  & \by{i:6.7.2}
  \end{align*}
\end{proof}

\exercisesection

\begin{ex}\label{i:ex:6.7.1}
  Prove the remaining components of \cref{i:6.7.3}.
\end{ex}

\begin{proof}
  See \cref{i:6.7.3}.
\end{proof}

\chapter{Series}\label{i:ch:7}

\section{Finite series}\label{sec:7.1}

\begin{defn}[Finite series]\label{7.1.1}
  Let \(m, n\) be integers, and let \((a_i)_{i = m}^n\) be a finite sequence of real numbers, assigning a real number \(a_i\) to each integer \(i\) between \(m\) and \(n\) inclusive (i.e., \(m \leq i \leq n\)).
  Then we define the finite sum (or finite series) \(\sum_{i = m}^n a_i\) by the recursive formula
  \begin{align*}
     & \sum_{i = m}^n a_i \coloneqq 0 \text{ whenever } n < m ;                                                      \\
     & \sum_{i = m}^{n + 1} a_i \coloneqq \Bigg(\sum_{i = m}^n a_i\Bigg) + a_{n + 1} \text{ whenever } n \geq m - 1.
  \end{align*}
\end{defn}

\begin{note}
  we sometimes express \(\sum_{i = m}^n a_i\) less formally as
  \[
    \sum_{i = m}^n a_i = a_m + a_{m + 1} + \dots + a_n.
  \]
\end{note}

\begin{rmk}\label{7.1.2}
  The difference between ``sum'' and ``series'' is a subtle linguistic one.
  Strictly speaking, a series is an \emph{expression} of the form \(\sum_{i = m}^n a_i\);
  this series is mathematically (but not semantically) equal to a real number, which is then the \emph{sum} of that series.
  For instance, \(1 + 2 + 3 + 4 + 5\) is a series, whose sum is \(15\);
  if one were to be very picky about semantics, one would not consider \(15\) a series and one would not consider \(1 + 2 + 3 + 4 + 5\) a sum, despite the two expressions having the same value.
  However, we will not be very careful about this distinction as it is purely linguistic and has no bearing on the mathematics;
  the expressions \(1 + 2 + 3 + 4 + 5\) and \(15\) are the same number, and thus \emph{mathematically} interchangeable, in the sense of the axiom of substitution, even if they are not semantically interchangeable.
\end{rmk}

\begin{rmk}\label{7.1.3}
  Note that the variable \(i\) (sometimes called the \emph{index of summation}) is a \emph{bound variable} (sometimes called a \emph{dummy variable});
  the expression \(\sum_{i = m}^n a_i\) does not actually depend on any quantity named \(i\).
  In particular, one can replace the index of summation \(i\) with any other symbol, and obtain the same sum:
  \[
    \sum_{i = m}^n a_i = \sum_{j = m}^n a_j.
  \]
\end{rmk}

\begin{lem}\label{7.1.4}
  \mbox{}
  \begin{enumerate}
    \item Let \(m \leq n < p\) be integers, and let \(a_i\) be a real number assigned to each integer \(m \leq i \leq p\).
          Then we have
          \[
            \sum_{i = m}^n a_i + \sum_{i = n + 1}^p a_i = \sum_{i = m}^p a_i.
          \]
    \item Let \(m \leq n\) be integers, \(k\) be another integer, and let \(a_i\) be a real number assigned to each integer \(m \leq i \leq n\).
          Then we have
          \[
            \sum_{i = m}^n a_i = \sum_{j = m + k}^{n + k} a_{j - k}.
          \]
    \item Let \(m \leq n\) be integers, and let \(a_i, b_i\) be real numbers assigned to each integer \(m \leq i \leq n\).
          Then we have
          \[
            \sum_{i = m}^n (a_i + b_i) = \Bigg(\sum_{i = m}^n a_i\Bigg) + \Bigg(\sum_{i = m}^n b_i\Bigg).
          \]
    \item Let \(m \leq n\) be integers, and let \(a_i\) be a real number assigned to each integer \(m \leq i \leq n\), and let \(c\) be another real number.
          Then we have
          \[
            \sum_{i = m}^n (ca_i) = c\Bigg(\sum_{i = m}^n a_i\Bigg).
          \]
    \item (Triangle inequality for finite series)
          Let \(m \leq n\) be integers, and let \(a_i\) be a real number assigned to each integer \(m \leq i \leq n\).
          Then we have
          \[
            \abs{\sum_{i = m}^n a_i} \leq \sum_{i = m}^n \abs{a_i}.
          \]
    \item (Comparison test for finite series) Let \(m \leq n\) be integers, and let \(a_i\), \(b_i\) be real numbers assigned to each integer \(m \leq i \leq n\).
          Suppose that \(a_i \leq b_i\) for all \(m \leq i \leq n\).
          Then we have
          \[
            \sum_{i = m}^n a_i \leq \sum_{i = m}^n b_i
          \]
  \end{enumerate}
\end{lem}

\begin{proof}{(a)}
  Let \(k = p - m\).
  By hypothesis we know that \(k > 0\).
  Now we use induction on \(k\) to show that \cref{7.1.4}(a) is true and we start with \(k = 1\).
  For \(k = 1\), we have \(p = m + 1\) and by \cref{7.1.1} we have
  \[
    \sum_{i = m}^n a_i + \sum_{i = n + 1}^p a_i = \sum_{i = m}^m a_i + \sum_{i = m + 1}^p a_i = a_m + a_{m + 1} = \sum_{i = m}^p a_i.
  \]
  Thus the base case holds.
  Suppose inductively that for some \(k \geq 1\) \cref{7.1.4}(a) is true.
  Then for \(k + 1 = p - m\), we have \(p - 1 = k + m\) and
  \begin{align*}
    \sum_{i = m}^n a_i + \sum_{i = n + 1}^p a_i & = \Bigg(\sum_{i = m}^n a_i\Bigg) + \Bigg(\sum_{i = n + 1}^{p - 1} a_i\Bigg) + a_p &  & \by{7.1.1} \\
                                                & = \Bigg(\sum_{i = m}^{p - 1} a_i\Bigg) + a_p                                      &  & \byIH      \\
                                                & = \sum_{i = m}^p a_i.                                                             &  & \by{7.1.1}
  \end{align*}
  This closes the induction.
\end{proof}

\begin{proof}{(b)}
  Let \(p = n - m\).
  By hypothesis we know that \(p \geq 0\).
  Now we use induction on \(p\) to show that \cref{7.1.4}(b) is true.
  For \(p = 0\), we have \(n = m\) and
  \begin{align*}
    \sum_{j = m + k}^{m + k} a_{j - k} & = \Bigg(\sum_{j = m + k}^{m + k - 1} a_{j - k}\Bigg) + a_{m + k - k} &  & \by{7.1.1} \\
                                       & = 0 + a_{m + k - k}                                                  &  & \by{7.1.1} \\
                                       & = 0 + a_m                                                                            \\
                                       & = \Bigg(\sum_{i = m}^{m - 1} a_i\Bigg) + a_m                         &  & \by{7.1.1} \\
                                       & = \sum_{i = m}^m a_i.                                                &  & \by{7.1.1}
  \end{align*}
  So the base case holds.
  Suppose inductively that for some \(p \geq 0\) \cref{7.1.4}(b) is true.
  Then for \(p + 1 = n - m\), we have \(p = n - m - 1\) and
  \begin{align*}
    \sum_{j = m + k}^{n + k} a_{j - k} & = \Bigg(\sum_{j = m + k}^{n + k - 1} a_{j - k}\Bigg) + a_{n + k - k} &  & \by{7.1.1} \\
                                       & = \Bigg(\sum_{j = m + k}^{n + k - 1} a_{j - k}\Bigg) + a_n                           \\
                                       & = \Bigg(\sum_{i = m}^{n - 1} a_i\Bigg) + a_n                         &  & \byIH      \\
                                       & = \sum_{i = m}^n a_i.                                                &  & \by{7.1.1}
  \end{align*}
  This closes the induction.
\end{proof}

\begin{proof}{(c)}
  Let \(p = n - m\).
  By hypothesis we know that \(p \geq 0\).
  Now we use induction on \(p\) to show that \cref{7.1.4}(c) is true.
  For \(p = 0\), we have \(n = m\) and
  \begin{align*}
    \sum_{i = m}^m (a_i + b_i) & = \Bigg(\sum_{i = m}^{m - 1} (a_i + b_i)\Bigg) + a_m + b_m                                &  & \by{7.1.1} \\
                               & = 0 + a_m + b_m                                                                           &  & \by{7.1.1} \\
                               & = \Bigg(\sum_{i = m}^{m - 1} a_i\Bigg) + \Bigg(\sum_{i = m}^{m - 1} b_i\Bigg) + a_m + b_m &  & \by{7.1.1} \\
                               & = \Bigg(\sum_{i = m}^m a_i\Bigg) + \Bigg(\sum_{i = m}^m b_i\Bigg).                        &  & \by{7.1.1}
  \end{align*}
  So the base case holds.
  Suppose inductively that for some \(p \geq 0\) \cref{7.1.4}(c) is true.
  Then for \(p + 1 = n - m\), we have \(p = n - m - 1\) and
  \begin{align*}
    \sum_{i = m}^n (a_i + b_i) & = \Bigg(\sum_{i = m}^{n - 1} (a_i + b_i)\Bigg) + a_n + b_n                                &  & \by{7.1.1} \\
                               & = \Bigg(\sum_{i = m}^{n - 1} a_i\Bigg) + \Bigg(\sum_{i = m}^{n - 1} b_i\Bigg) + a_n + b_n &  & \byIH      \\
                               & = \Bigg(\sum_{i = m}^n a_i\Bigg) + \Bigg(\sum_{i = m}^n b_i\Bigg).                        &  & \by{7.1.1}
  \end{align*}
  This closes the induction.
\end{proof}

\begin{proof}{(d)}
  Let \(p = n - m\).
  By hypothesis we know that \(p \geq 0\).
  Now we use induction on \(p\) to show that \cref{7.1.4}(d) is true.
  For \(p = 0\), we have \(n = m\) and
  \begin{align*}
    \sum_{i = m}^m ca_i & = \Bigg(\sum_{i = m}^{m - 1} ca_i\Bigg) + ca_m             &  & \by{7.1.1} \\
                        & = 0 + ca_m                                                 &  & \by{7.1.1} \\
                        & = c \times 0 + ca_m                                                        \\
                        & = c \Bigg(\sum_{i = m}^{m - 1} a_i\Bigg) + ca_m            &  & \by{7.1.1} \\
                        & = c \Bigg(\Bigg(\sum_{i = m}^{m - 1} a_i\Bigg) + a_m\Bigg)                 \\
                        & = c \Bigg(\sum_{i = m}^m a_i\Bigg).                        &  & \by{7.1.1}
  \end{align*}
  So the base case holds.
  Suppose inductively that for some \(p \geq 0\) \cref{7.1.4}(d) is true.
  Then for \(p + 1 = n - m\), we have \(p = n - m - 1\) and
  \begin{align*}
    \sum_{i = m}^n ca_i & = \Bigg(\sum_{i = m}^{n - 1} ca_i\Bigg) + ca_n             &  & \by{7.1.1} \\
                        & = c \Bigg(\sum_{i = m}^{n - 1} a_i\Bigg) + ca_n            &  & \byIH      \\
                        & = c \Bigg(\Bigg(\sum_{i = m}^{n - 1} a_i\Bigg) + a_n\Bigg)                 \\
                        & = c \Bigg(\sum_{i = m}^n a_i\Bigg).                        &  & \by{7.1.1}
  \end{align*}
  This closes the induction.
\end{proof}

\begin{proof}{(e)}
  Let \(p = n - m\).
  By hypothesis we know that \(p \geq 0\).
  Now we use induction on \(p\) to show that \cref{7.1.4}(e) is true.
  For \(p = 0\), we have \(n = m\) and
  \begin{align*}
    \abs{\sum_{i = m}^m a_i} & = \abs{\Bigg(\sum_{i = m}^{m - 1} a_i\Bigg) + a_m}       &  & \by{7.1.1} \\
                             & = \abs{0 + a_m}                                          &  & \by{7.1.1} \\
                             & = 0 + \abs{a_m}                                                          \\
                             & = \Bigg(\sum_{i = m}^{m - 1} \abs{a_i}\Bigg) + \abs{a_m} &  & \by{7.1.1} \\
                             & = \sum_{i = m}^m \abs{a_i}.                              &  & \by{7.1.1}
  \end{align*}
  So the base case holds.
  Suppose inductively that for some \(p \geq 0\) \cref{7.1.4}(e) is true.
  Then for \(p + 1 = n - m\), we have \(p = n - m - 1\) and
  \begin{align*}
    \abs{\sum_{i = m}^n a_i} & = \abs{\Bigg(\sum_{i = m}^{n - 1} a_i\Bigg) + a_n} &  & \by{7.1.1} \\
                             & \leq \abs{\sum_{i = m}^{n - 1} a_i} + \abs{a_n}                    \\
                             & \leq \sum_{i = m}^{n - 1} \abs{a_i} + \abs{a_n}    &  & \byIH      \\
                             & = \sum_{i = m}^n \abs{a_i}.                        &  & \by{7.1.1}
  \end{align*}
  This closes the induction.
\end{proof}

\begin{proof}{(f)}
  Let \(p = n - m\).
  By hypothesis we know that \(p \geq 0\).
  Now we use induction on \(p\) to show that \cref{7.1.4}(f) is true.
  For \(p = 0\), we have \(n = m\) and
  \begin{align*}
    \sum_{i = m}^m a_i & = \Bigg(\sum_{i = m}^{m - 1} a_i\Bigg) + a_m &  & \by{7.1.1}             \\
                       & = 0 + a_m                                    &  & \by{7.1.1}             \\
                       & \leq 0 + b_m                                 &  & \text{(by hypothesis)} \\
                       & = \Bigg(\sum_{i = m}^{m - 1} b_i\Bigg) + b_m &  & \by{7.1.1}             \\
                       & = \sum_{i = m}^m b_i.                        &  & \by{7.1.1}             \\
  \end{align*}
  So the base case holds.
  Suppose inductively that for some \(p \geq 0\) \cref{7.1.4}(f) is true.
  Then for \(p + 1 = n - m\), we have \(p = n - m - 1\) and
  \begin{align*}
    \sum_{i = m}^n a_i & = \Bigg(\sum_{i = m}^{n - 1} a_i\Bigg) + a_n    &  & \by{7.1.1}             \\
                       & \leq \Bigg(\sum_{i = m}^{n - 1} b_i\Bigg) + a_n &  & \byIH                  \\
                       & \leq \Bigg(\sum_{i = m}^{n - 1} b_i\Bigg) + b_n &  & \text{(by hypothesis)} \\
                       & = \sum_{i = m}^n b_i.                           &  & \by{7.1.1}             \\
  \end{align*}
  This closes the induction.
\end{proof}

\begin{rmk}\label{7.1.5}
  In the future we may omit some of the parentheses in series expressions, for instance we may write \(\sum_{i = m}^n (a_i + b_i)\) simply as \(\sum_{i = m}^n a_i + b_i\).
  This is reasonably safe from being mis-interpreted, because the alternative interpretation \((\sum_{i = m}^n a_i) + b_i\) does not make any sense
  (the index \(i\) in \(b_i\) is meaningless outside of the summation, since \(i\) is only a dummy variable).
\end{rmk}

\begin{defn}[Summations over finite sets]\label{7.1.6}
  Let \(X\) be a finite set with \(n\) elements (where \(n \in \N\)), and let \(f : X \to \R\) be a function from \(X\) to the real numbers
  (i.e., \(f\) assigns a real number \(f(x)\) to each element \(x\) of \(X\)).
  Then we can define the finite sum \(\sum_{x \in X} f(x)\) as follows.
  We first select any bijection \(g\) from \(\set{i \in \N : 1 \leq i \leq n}\) to \(X\);
  such a bijection exists since \(X\) is assumed to have \(n\) elements.
  We then define
  \[
    \sum_{x \in X} f(x) \coloneqq \sum_{i = 1}^n f(g(i)).
  \]
  In some cases we would like to define the sum \(\sum_{x \in X} f(x)\) when \(f : Y \to \R\) is defined on a larger set \(Y\) than \(X\).
  In such cases we use exactly the same definition as is given above.
\end{defn}

\setcounter{thm}{7}
\begin{prop}[Finite summations are well-defined]\label{7.1.8}
  Let \(X\) be a finite set with \(n\) elements (where \(n \in \N\)), let \(f : X \to \R\) be a function, and let \(g : \set{i \in \N : 1 \leq i \leq n} \to X\) and \(h : \set{i \in \N : 1 \leq i \leq n} \to X\) be bijections.
  Then we have
  \[
    \sum_{i = 1}^n f(g(i)) = \sum_{i = 1}^n f(h(i)).
  \]
\end{prop}

\begin{proof}
  We use induction on \(n\);
  more precisely, we let \(P(n)\) be the assertion that ``For any set \(X\) of \(n\) elements, any function \(f : X \to \R\), and any two bijections \(g, h\) from \(\set{i \in \N : 1 \leq i \leq n}\) to \(X\), we have \(\sum_{i = 1}^n f(g(i)) = \sum_{i = 1}^n f(h(i))\)''.
  (More informally, \(P(n)\) is the assertion that \cref{7.1.8} is true for that value of \(n\).)
  We want to prove that \(P(n)\) is true for all natural numbers \(n\).

  We first check the base case \(P(0)\).
  In this case \(\sum_{i = 1}^0 f(g(i))\) and \(\sum_{i = 1}^0 f(h(i))\) both equal to \(0\), by definition of finite series (\cref{7.1.1}), so we are done.

  Now suppose inductively that \(P(n)\) is true;
  we now prove that \(P(n + 1)\) is true.
  Thus, let \(X\) be a set with \(n + 1\) elements, let \(f : X \to \R\) be a function, and let \(g\) and \(h\) be bijections from \(\set{i \in N : 1 \leq i \leq n + 1}\) to \(X\).
  We have to prove that
  \[
    \sum_{i = 1}^{n + 1} f(g(i)) = \sum_{i = 1}^{n + 1} f(h(i)). \tag{7.1}\label{eq 7.1}
  \]
  Let \(x \coloneqq g(n + 1)\);
  thus \(x\) is an element of \(X\).
  By definition of finite series (\cref{7.1.1}), we can expand the left-hand side of \eqref{eq 7.1} as
  \[
    \sum_{i = 1}^{n + 1} f(g(i)) = \Bigg(\sum_{i = 1}^n f(g(i))\Bigg) + f(x).
  \]
  Now let us look at the right-hand side of \eqref{eq 7.1}.
  Ideally we would like to have \(h(n + 1)\) also equal to \(x\)
  - this would allow us to use the inductive hypothesis \(P(n)\) much more easily
  - but we cannot assume this.
  However, since \(h\) is a bijection, we do know that there is \emph{some} index \(j\), with \(1 \leq j \leq n + 1\), for which \(h(j) = x\).
  We now use \cref{7.1.4} and the definition of finite series (\cref{7.1.1}) to write
  \begin{align*}
    \sum_{i = 1}^{n + 1} f(h(i)) & = \Bigg(\sum_{i = 1}^j f(h(i))\Bigg) + \Bigg(\sum_{i = j + 1}^{n + 1} f(h(i))\Bigg)                 \\
                                 & = \Bigg(\sum_{i = 1}^{j - 1} f(h(i))\Bigg) + f(h(j)) + \Bigg(\sum_{i = j + 1}^{n + 1} f(h(i))\Bigg) \\
                                 & = \Bigg(\sum_{i = 1}^{j - 1} f(h(i))\Bigg) + f(x) + \Bigg(\sum_{i = j}^n f(h(i + 1))\Bigg).
  \end{align*}
  We now define the function \(\tilde{h} : \set{i \in \N : 1 \leq i \leq n} \to X - \set{x}\) by setting \(\tilde{h}(i) \coloneqq h(i)\) when \(i < j\) and \(\tilde{h}(i) \coloneqq h(i + 1)\) when \(i \geq j\).
  We can thus write the right-hand side of \eqref{eq 7.1} as
  \[
    = \Bigg(\sum_{i = 1}^{j - 1} f(\tilde{h}(i))\Bigg) + f(x) + \Bigg(\sum_{i = j}^n f(\tilde{h}(i))\Bigg) = \Bigg(\sum_{i = 1}^n f(\tilde{h}(i))\Bigg) + f(x)
  \]
  where we have used \cref{7.1.4} once again.
  Thus to finish the proof of \eqref{eq 7.1} we have to show that
  \[
    \sum_{i = 1}^n f(g(i)) = \sum_{i = 1}^n f(\tilde{h}(i)). \tag{7.2}\label{eq 7.2}
  \]
  But the function \(g\) (when restricted to \(\set{i \in \N : 1 \leq i \leq n}\)) is a bijection from \(\set{i \in \N : 1 \leq i \leq n} \to X - \set{x}\).
  The function \(\tilde{h}\) is also a bijection from \(\set{i \in \N : 1 \leq i \leq n} \to X - \set{x}\) (cf. \cref{3.6.9}).
  Since \(X - \set{x}\) has \(n\) elements (by \cref{3.6.9}), the claim \eqref{eq 7.2} then follows directly from the induction hypothesis \(P(n)\).
\end{proof}

\begin{rmk}\label{7.1.9}
  The issue is somewhat more complicated when summing over infinite sets;
  See \cref{sec:8.2}.
\end{rmk}

\begin{rmk}\label{7.1.10}
  Suppose that \(X\) is a set, that \(P(x)\) is a property pertaining to an element \(x\) of \(X\), and \(f : \set{y \in X : P(y) \text{ is true}} \to \R\) is a function.
  Then we will often abbreviate
  \[
    \sum_{x \in \set{y \in X : P(y) \text{ is true}}} f(x)
  \]
  as \(\sum_{x \in X : P(x) \text{ is true}} f(x)\) or even as \(\sum_{P(x) \text{ is true}} f(x)\) when there is no change of confusion.
\end{rmk}

\begin{prop}[Basic properties of summation over finite sets]\label{7.1.11}
  \mbox{}
  \begin{enumerate}
    \item If \(X\) is empty, and \(f : X \to \R\) is a function (i.e., \(f\) is the empty function), we have
          \[
            \sum_{x \in X} f(x) = 0.
          \]
    \item If \(X\) consists of a single element, \(X = \set{x_0}\), and \(f : X \to \R\) is a function, we have
          \[
            \sum_{x \in X} f(x) = f(x_0).
          \]
    \item (Substitution, part I) If \(X\) is a finite set, \(f : X \to \R\) is a function, and \(g : Y \to X\) is a bijection, then
          \[
            \sum_{x \in X} f(x) = \sum_{y \in Y} f(g(y)).
          \]
    \item (Substitution, part II) Let \(n \leq m\) be integers, and let \(X\) be the set \(X \coloneqq \set{i \in \Z : n \leq i \leq m}\).
          If \(a_i\) is a real number assigned to each integer \(i \in X\), then we have
          \[
            \sum_{i = n}^m a_i = \sum_{i \in X} a_i.
          \]
    \item Let \(X, Y\) be disjoint finite sets (so \(X \cap Y = \emptyset\)), and \(f : X \cup Y \to \R\) is a function.
          Then we have
          \[
            \sum_{z \in X \cup Y} f(z) = \Bigg(\sum_{x \in X} f(x)\Bigg) + \Bigg(\sum_{y \in Y} f(y)\Bigg).
          \]
    \item (Linearity, part I) Let \(X\) be a finite set, and let \(f : X \to \R\) and \(g : X \to \R\) be functions.
          Then
          \[
            \sum_{x \in X} (f(x) + g(x)) = \sum_{x \in X} f(x) + \sum_{x \in X} g(x).
          \]
    \item (Linearity, part II) Let \(X\) be a finite set, let \(f : X \to \R\) be a function, and let \(c\) be a real number.
          Then
          \[
            \sum_{x \in X} cf(x) = c \sum_{x \in X} f(x).
          \]
    \item (Monotonicity) Let \(X\) be a finite set, and let \(f : X \to \R\) and \(g : X \to \R\) be functions such that \(f(x) \leq g(x)\) for all \(x \in \mathbf{X}\).
          Then we have
          \[
            \sum_{x \in X} f(x) \leq \sum_{x \in X} g(x).
          \]
    \item (Triangle inequality) Let \(X\) be a finite set, and let \(f : X \to \R\) be a function, then
          \[
            \abs{\sum_{x \in X} f(x)} \leq \sum_{x \in X} \abs{f(x)}.
          \]
  \end{enumerate}
\end{prop}

\begin{proof}{(a)}
  Let \(g : \set{i \in \N : 1 \leq i \leq 0} \to \emptyset\) be a function.
  Then \(g\) is a bijection and
  \begin{align*}
    \sum_{x \in X} f(x) & = \sum_{i = 1}^0 f(g(i)) &  & \by{7.1.6} \\
                        & = 0.                     &  & \by{7.1.1}
  \end{align*}
\end{proof}

\begin{proof}{(b)}
  Let \(g : \set{1} \to \set{x_0}\) be a function.
  Then \(g\) is a bijection and
  \begin{align*}
    \sum_{x \in X} f(x) & = \sum_{i = 1}^1 f(g(i))                       &  & \by{7.1.6} \\
                        & = \bigg(\sum_{i = 1}^0 f(g(i))\bigg) + f(g(1)) &  & \by{7.1.1} \\
                        & = 0 + f(g(1))                                  &  & \by{7.1.1} \\
                        & = f(x_0).
  \end{align*}
\end{proof}

\begin{proof}{(c)}
  Let \(h : \set{i \in \N : 1 \leq i \leq \#(Y)} \to Y\) be a bijection.
  Since \(X\) is finite and \(g\) is a bijection between \(X\) and \(Y\), we know that \(Y\) is finite and thus such \(h\) is well-defined.
  Then we know that \(g \circ h : \set{i \in \N : 1 \leq i \leq \#(Y)} \to X\) is also a bijection and
  \begin{align*}
    \sum_{x \in X} f(x) & = \sum_{i = 1}^{\#(Y)} f((g \circ h)(i)) &  & \by{7.1.6} \\
                        & = \sum_{i = 1}^{\#(Y)} f(g(h(i)))                        \\
                        & = \sum_{i = 1}^{\#(Y)} (f \circ g)(h(i))                 \\
                        & = \sum_{y \in Y} (f \circ g)(y)          &  & \by{7.1.6} \\
                        & = \sum_{y \in Y} f(g(y)).
  \end{align*}
\end{proof}

\begin{proof}{(d)}
  Let \(f : X \to \set{a_i \in \R : n \leq i \leq m}\) be a function where \(f = i \mapsto a_i\).
  Let \(g : \set{i \in \N : 1 \leq i \leq m - n + 1} \to X\) be a function where \(g = i \mapsto i + n - 1\).
  Then \(g\) is a bijection and
  \begin{align*}
    \sum_{i \in X} a_i & = \sum_{i \in X} f(i)                                                                             \\
                       & = \sum_{i = 1}^{m - n + 1} f(g(i))                               &  & \by{7.1.6}                  \\
                       & = \sum_{i = 1}^{m - n + 1} f(i + n - 1)                                                           \\
                       & = \sum_{i = 1}^{m - n + 1} a_{i + n - 1}                                                          \\
                       & = \sum_{i = 1 + n - 1}^{m - n + 1 + n - 1} a_{i + n - 1 - n + 1} &  & \text{(by \cref{7.1.4}(b))} \\
                       & = \sum_{i = n}^m a_i.
  \end{align*}
\end{proof}

\begin{proof}{(e)}
  Let \(g : \set{i \in \N : 1 \leq i \leq \#(X)} \to X\) and \(h : \set{i \in \N : 1 \leq i \leq \#(Y)} \to Y\) be bijections.
  Since \(X, Y\) are finite, we know that \(g, h\) are well-defined and \(X \cup Y\) is finite.
  Let \(k : \set{i \in \N : 1 \leq i \leq \#(X \cup Y)} \to X \cup Y\) be a bijection where
  \[
    k(i) = \begin{dcases}
      g(i)         & \text{if } 1 \leq i \leq \#(X)                  \\
      h(i - \#(X)) & \text{if } \#(X) + 1 \leq i \leq \#(X) + \#(Y).
    \end{dcases}
  \]
  Since \(X \cup Y\) is finite, we know that \(k\) is well-defined and \(\#(X \cup Y) = \#(X) + \#(Y)\).
  Then we have
  \begin{align*}
    \sum_{z \in X \cup Y} f(z) & = \sum_{i = 1}^{\#(X \cup Y)} f(k(i))                                                &  & \by{7.1.6}                  \\
                               & = \sum_{i = 1}^{\#(X)} f(k(i)) + \sum_{i = \#(X) + 1}^{\#(X \cup Y)} f(k(i))         &  & \text{(by \cref{7.1.4}(a))} \\
                               & = \sum_{i = 1}^{\#(X)} f(g(i)) + \sum_{i = \#(X) + 1}^{\#(X \cup Y)} f(h(i - \#(X)))                                  \\
                               & = \sum_{i = 1}^{\#(X)} f(g(i)) + \sum_{i = 1}^{\#(Y)} f(h(i))                        &  & \text{(by \cref{7.1.4}(b))} \\
                               & = \sum_{x \in X} f(x) + \sum_{y \in Y} f(y).                                         &  & \by{7.1.6}
  \end{align*}
\end{proof}

\begin{proof}{(f)}
  Let \(h : \set{i \in \N : 1 \leq i \leq \#(X)} \to X\) be a bijection.
  Since \(X\) is finite, we know that \(h\) is well-defined and
  \begin{align*}
    \sum_{x \in X} (f(x) + g(x)) & = \sum_{x \in X} (f + g)(x)                                                                    \\
                                 & = \sum_{i = 1}^{\#(X)} (f + g)(h(i))                          &  & \by{7.1.6}                  \\
                                 & = \sum_{i = 1}^{\#(X)} (f(h(i)) + g(h(i)))                                                     \\
                                 & = \sum_{i = 1}^{\#(X)} f(h(i)) + \sum_{i = 1}^{\#(X)} g(h(i)) &  & \text{(by \cref{7.1.4}(c))} \\
                                 & = \sum_{x \in X} f(x) + \sum_{x \in X} g(x).                  &  & \by{7.1.6}
  \end{align*}
\end{proof}

\begin{proof}{(g)}
  Let \(g : \set{i \in \N : 1 \leq i \leq \#(X)} \to X\) be a bijection.
  Since \(X\) is finite, we know that \(g\) is well-defined and
  \begin{align*}
    \sum_{x \in X} cf(x) & = \sum_{x \in X} (cf)(x)                                           \\
                         & = \sum_{i = 1}^{\#(X)} (cf)(g(i)) &  & \by{7.1.6}                  \\
                         & = \sum_{i = 1}^{\#(X)} cf(g(i))                                    \\
                         & = c \sum_{i = 1}^{\#(X)} f(g(i))  &  & \text{(by \cref{7.1.4}(d))} \\
                         & = c \sum_{x \in X} f(x).          &  & \by{7.1.6}
  \end{align*}
\end{proof}

\begin{proof}{(h)}
  Let \(h : \set{i \in \N : 1 \leq i \leq \#(X)} \to X\) be a bijection.
  Since \(X\) is finite, we know that \(h\) is well-defined and
  \begin{align*}
    \sum_{x \in X} f(x) & = \sum_{i = 1}^{\#(X)} f(h(i))    &  & \by{7.1.6}                  \\
                        & \leq \sum_{i = 1}^{\#(X)} g(h(i)) &  & \text{(by \cref{7.1.4}(f))} \\
                        & = \sum_{x \in X} g(x).            &  & \by{7.1.6}
  \end{align*}
\end{proof}

\begin{proof}{(i)}
  Let \(g : \set{i \in \N : 1 \leq i \leq \#(X)} \to X\) be a bijection.
  Since \(X\) is finite, we know that \(g\) is well-defined and
  \begin{align*}
    \abs{\sum_{x \in X} f(x)} & = \abs{\sum_{i = 1}^{\#(X)} f(g(i))}    &  & \by{7.1.6}                  \\
                              & \leq \sum_{i = 1}^{\#(X)} \abs{f(g(i))} &  & \text{(by \cref{7.1.4}(e))} \\
                              & = \sum_{x \in X} \abs{f(x)}.            &  & \by{7.1.6}
  \end{align*}
\end{proof}

\begin{rmk}\label{7.1.12}
  The substitution rule in \cref{7.1.11}(c) can be thought of as making the substitution \(x \coloneqq g(y)\) (hence the name).
  Note that the assumption that \(g\) is a bijection is essential.
  From \cref{7.1.11}(c) and (d) we see that
  \[
    \sum_{i = n}^m a_i = \sum_{i = n}^m a_{f(i)}
  \]
  for any bijection \(f\) from the set \(\set{i \in \Z : n \leq i \leq m}\) to itself.
  Informally, this means that we can rearrange the elements of a finite sequence at will and still obtain the same value.
\end{rmk}

\begin{lem}\label{7.1.13}
  Let \(X, Y\) be finite sets, and let \(f : X \times Y \to \R\) be a function.
  Then
  \[
    \sum_{x \in X} \bigg(\sum_{y \in Y} f(x, y)\bigg) = \sum_{(x, y) \in X \times Y} f(x, y).
  \]
\end{lem}

\begin{proof}
  Let \(n\) be the number of elements in \(X\).
  We will use induction on \(n\) (cf. \cref{7.1.8});
  i.e., we let \(P(n)\) be the assertion that \cref{7.1.13} is true for any set \(X\) with \(n\) elements, and any finite set \(Y\) and any function \(f : X \times Y \to \R\).
  We wish to prove \(P(n)\) for all natural numbers \(n\).

  The base case \(P(0)\) is easy, following from \cref{7.1.11}(a).
  Now suppose that \(P(n)\) is true;
  we now show that \(P(n + 1)\) is true.
  Let \(X\) be a set with \(n + 1\) elements.
  In particular, by \cref{3.6.9}, we can write \(X = X' \cup \set{x_0}\), where \(x_0\) is an element of \(X\) and \(X' \coloneqq X - \set{x_0}\) has \(n\) elements.
  Then by \cref{7.1.11}(e) we have
  \[
    \sum_{x \in X} \bigg(\sum_{y \in Y} f(x, y)\bigg) = \sum_{x \in X'} \bigg(\sum_{y \in Y} f(x, y)\bigg) + \bigg(\sum_{y \in Y} f(x_0, y)\bigg);
  \]
  by the induction hypothesis this is equal to
  \[
    \sum_{(x, y) \in X' \times Y} f(x, y) + \bigg(\sum_{y \in Y} f(x_0, y)\bigg).
  \]
  By \cref{7.1.11}(c) this is equal to
  \[
    \sum_{(x, y) \in X' \times Y} f(x, y) + \bigg(\sum_{(x, y) \in \set{x_0} \times Y} f(x, y)\bigg).
  \]
  By \cref{7.1.11}(e) this is equal to
  \[
    \sum_{(x, y) \in X \times Y} f(x, y)
  \]
  as desired.
\end{proof}

\begin{cor}[Fubini's theorem for finite series]\label{7.1.14}
  Let \(X, Y\) be finite sets, and let \(f : X \times Y \to \R\) be a function.
  Then
  \begin{align*}
    \sum_{x \in X} \bigg(\sum_{y \in Y} f(x, y)\bigg) & = \sum_{(x, y) \in X \times Y} f(x, y)               \\
                                                      & = \sum_{(y, x) \in Y \times X} f(x, y)               \\
                                                      & = \sum_{y \in Y} \bigg(\sum_{x \in X} f(x, y)\bigg).
  \end{align*}
\end{cor}

\begin{proof}
  In light of \cref{7.1.13}, it suffices to show that
  \[
    \sum_{(x, y) \in X \times Y} f(x, y) = \sum_{(y, x) \in Y \times X} f(x, y).
  \]
  But this follows from \cref{7.1.11}(c) by applying the bijection \(h : Y \times X \to X \times Y\) defined by \(h(y, x) \coloneqq (x, y)\).
\end{proof}

\begin{rmk}\label{7.1.15}
  We anticipate something interesting to happen when we move from finite sums to infinite sums.
  However, see \cref{8.2.2}.
\end{rmk}

\begin{ac}[Products over finite sets]\label{ac:7.1.1}
  Let \(m, n\) be integers, and let \((a_i)_{i = m}^n\) be a finite sequence of real numbers, assigning a real number \(a_i\) to each integer \(i\) between \(m\) and \(n\) inclusive (i.e., \(m \leq i \leq n\)).
  Then we define the finite product \(\prod_{i = m}^n a_i\) by the recursive formula
  \begin{align*}
     & \prod_{i = m}^n a_i \coloneqq 1 \text{ whenever } n < m;                                                             \\
     & \prod_{i = m}^{n + 1} a_i \coloneqq \Bigg(\prod_{i = m}^n a_i\Bigg) \times a_{n + 1} \text{ whenever } n \geq m - 1.
  \end{align*}
\end{ac}

\begin{ac}\label{ac:7.1.2}
  \mbox{}
  \begin{enumerate}
    \item Let \(m \leq n < p\) be integers, and let \(a_i\) be a real number assigned to each integer \(m \leq i \leq p\).
          Then we have
          \[
            \prod_{i = m}^n a_i \times \prod_{i = n + 1}^p a_i = \prod_{i = m}^p a_i.
          \]
    \item Let \(m \leq n\) be integers, \(k\) be another integer, and let \(a_i\) be a real number assigned to each integer \(m \leq i \leq n\).
          Then we have
          \[
            \prod_{i = m}^n a_i = \prod_{j = m + k}^{n + k} a_{j - k}.
          \]
    \item Let \(m \leq n\) be integers, and let \(a_i, b_i\) be real numbers assigned to each integer \(m \leq i \leq n\).
          Then we have
          \[
            \prod_{i = m}^n (a_i \times b_i) = \Bigg(\prod_{i = m}^n a_i\Bigg) \times \Bigg(\prod_{i = m}^n b_i\Bigg).
          \]
    \item Let \(m \leq n\) be integers, and let \(a_i\) be a real number assigned to each integer \(m \leq i \leq n\), and let \(c\) be another real number.
          Then we have
          \[
            \prod_{i = m}^n (ca_i) = c^{n - m + 1} \Bigg(\prod_{i = m}^n a_i\Bigg).
          \]
    \item Let \(m \leq n\) be integers, and let \(a_i\) be a real number assigned to each integer \(m \leq i \leq n\).
          Then we have
          \[
            \abs{\prod_{i = m}^n a_i} = \prod_{i = m}^n \abs{a_i}.
          \]
  \end{enumerate}
\end{ac}

\begin{proof}{(a)}
  Let \(k = p - m\).
  By hypothesis we know that \(k > 0\).
  Now we use induction on \(k\) to show that \cref{ac:7.1.2}(a) is true and we start with \(k = 1\).
  For \(k = 1\), we have \(p = m + 1\) and by \cref{ac:7.1.1} we have
  \[
    \prod_{i = m}^n a_i \times \prod_{i = n + 1}^p a_i = \prod_{i = m}^m a_i \times \prod_{i = m + 1}^p a_i = a_m \times a_{m + 1} = \prod_{i = m}^p a_i.
  \]
  Thus the base case holds.
  Suppose inductively that for some \(k \geq 1\) \cref{ac:7.1.2}(a) is true.
  Then for \(k + 1 = p - m\), we have \(p - 1 = k + m\) and
  \begin{align*}
     & \prod_{i = m}^n a_i \times \prod_{i = n + 1}^p a_i                                                               \\
     & = \Bigg(\prod_{i = m}^n a_i\Bigg) \times \Bigg(\prod_{i = n + 1}^{p - 1} a_i\Bigg) \times a_p &  & \by{ac:7.1.1} \\
     & = \Bigg(\prod_{i = m}^{p - 1} a_i\Bigg) \times a_p                                            &  & \byIH         \\
     & = \prod_{i = m}^p a_i.                                                                        &  & \by{ac:7.1.1}
  \end{align*}
  This closes the induction.
\end{proof}

\begin{proof}{(b)}
  Let \(p = n - m\).
  By hypothesis we know that \(p \geq 0\).
  Now we use induction on \(p\) to show that \cref{ac:7.1.2}(b) is true.
  For \(p = 0\), we have \(n = m\) and
  \begin{align*}
    \prod_{j = m + k}^{m + k} a_{j - k} & = \Bigg(\prod_{j = m + k}^{m + k - 1} a_{j - k}\Bigg) \times a_{m + k - k} &  & \by{ac:7.1.1} \\
                                        & = 1 \times a_{m + k - k}                                                   &  & \by{ac:7.1.1} \\
                                        & = 1 \times a_m                                                                                \\
                                        & = \Bigg(\prod_{i = m}^{m - 1} a_i\Bigg) \times a_m                         &  & \by{ac:7.1.1} \\
                                        & = \prod_{i = m}^m a_i.                                                     &  & \by{ac:7.1.1}
  \end{align*}
  So the base case holds.
  Suppose inductively that for some \(p \geq 0\) \cref{ac:7.1.2}(b) is true.
  Then for \(p + 1 = n - m\), we have \(p = n - m - 1\) and
  \begin{align*}
    \prod_{j = m + k}^{n + k} a_{j - k} & = \Bigg(\prod_{j = m + k}^{n + k - 1} a_{j - k}\Bigg) \times a_{n + k - k} &  & \by{ac:7.1.1} \\
                                        & = \Bigg(\prod_{j = m + k}^{n + k - 1} a_{j - k}\Bigg) \times a_n                              \\
                                        & = \Bigg(\prod_{i = m}^{n - 1} a_i\Bigg) \times a_n                         &  & \byIH         \\
                                        & = \prod_{i = m}^n a_i.                                                     &  & \by{ac:7.1.1}
  \end{align*}
  This closes the induction.
\end{proof}

\begin{proof}{(c)}
  Let \(p = n - m\).
  By hypothesis we know that \(p \geq 0\).
  Now we use induction on \(p\) to show that \cref{ac:7.1.2}(c) is true.
  For \(p = 0\), we have \(n = m\) and
  \begin{align*}
    \prod_{i = m}^m (a_i \times b_i) & = \Bigg(\prod_{i = m}^{m - 1} (a_i \times b_i)\Bigg) \times a_m \times b_m                                 &  & \by{ac:7.1.1} \\
                                     & = 1 \times a_m \times b_m                                                                                  &  & \by{ac:7.1.1} \\
                                     & = \Bigg(\prod_{i = m}^{m - 1} a_i\Bigg) \times \Bigg(\prod_{i = m}^{m - 1} b_i\Bigg) \times a_m \times b_m &  & \by{ac:7.1.1} \\
                                     & = \Bigg(\prod_{i = m}^m a_i\Bigg) \times \Bigg(\prod_{i = m}^m b_i\Bigg).                                  &  & \by{ac:7.1.1}
  \end{align*}
  So the base case holds.
  Suppose inductively that for some \(p \geq 0\) \cref{ac:7.1.2}(c) is true.
  Then for \(p + 1 = n - m\), we have \(p = n - m - 1\) and
  \begin{align*}
    \prod_{i = m}^n (a_i \times b_i) & = \Bigg(\prod_{i = m}^{n - 1} (a_i \times b_i)\Bigg) \times a_n \times b_n                                 &  & \by{ac:7.1.1} \\
                                     & = \Bigg(\prod_{i = m}^{n - 1} a_i\Bigg) \times \Bigg(\prod_{i = m}^{n - 1} b_i\Bigg) \times a_n \times b_n &  & \byIH         \\
                                     & = \Bigg(\prod_{i = m}^n a_i\Bigg) \times \Bigg(\prod_{i = m}^n b_i\Bigg).                                  &  & \by{ac:7.1.1}
  \end{align*}
  This closes the induction.
\end{proof}

\begin{proof}{(d)}
  Let \(p = n - m\).
  By hypothesis we know that \(p \geq 0\).
  Now we use induction on \(p\) to show that \cref{ac:7.1.2}(d) is true.
  For \(p = 0\), we have \(n = m\) and
  \begin{align*}
    \prod_{i = m}^m ca_i & = \Bigg(\prod_{i = m}^{m - 1} ca_i\Bigg) \times ca_m &  & \by{ac:7.1.1} \\
                         & = 1 \times ca_m                                      &  & \by{ac:7.1.1} \\
                         & = c \times a_m                                                          \\
                         & = c \Bigg(\prod_{i = m}^m a_i\Bigg)                  &  & \by{ac:7.1.1} \\
                         & = c^{m - m + 1} \Bigg(\prod_{i = m}^m a_i\Bigg).
  \end{align*}
  So the base case holds.
  Suppose inductively that for some \(p \geq 0\) \cref{ac:7.1.2}(d) is true.
  Then for \(p + 1 = n - m\), we have \(p = n - m - 1\) and
  \begin{align*}
    \prod_{i = m}^n ca_i & = \Bigg(\prod_{i = m}^{n - 1} ca_i\Bigg) \times ca_n                         &  & \by{ac:7.1.1} \\
                         & = c^{n - 1 - m + 1} \Bigg(\prod_{i = m}^{n - 1} a_i\Bigg) \times ca_n        &  & \byIH         \\
                         & = c^{n - m + 1} \Bigg(\Bigg(\prod_{i = m}^{n - 1} a_i\Bigg) \times a_n\Bigg)                    \\
                         & = c^{n - m + 1} \Bigg(\prod_{i = m}^n a_i\Bigg).                             &  & \by{ac:7.1.1}
  \end{align*}
  This closes the induction.
\end{proof}

\begin{proof}{(e)}
  Let \(p = n - m\).
  By hypothesis we know that \(p \geq 0\).
  Now we use induction on \(p\) to show that \cref{ac:7.1.2}(e) is true.
  For \(p = 0\), we have \(n = m\) and
  \begin{align*}
    \abs{\prod_{i = m}^m a_i} & = \abs{\Bigg(\prod_{i = m}^{m - 1} a_i\Bigg) \times a_m}       &  & \by{ac:7.1.1} \\
                              & = \abs{1a_m}                                                   &  & \by{ac:7.1.1} \\
                              & = \abs{1}\abs{a_m}                                                                \\
                              & = \Bigg(\prod_{i = m}^{m - 1} \abs{a_i}\Bigg) \times \abs{a_m} &  & \by{ac:7.1.1} \\
                              & = \prod_{i = m}^m \abs{a_i}.                                   &  & \by{ac:7.1.1}
  \end{align*}
  So the base case holds.
  Suppose inductively that for some \(p \geq 0\) \cref{ac:7.1.2}(e) is true.
  Then for \(p + 1 = n - m\), we have \(p = n - m - 1\) and
  \begin{align*}
    \abs{\prod_{i = m}^n a_i} & = \abs{\Bigg(\prod_{i = m}^{n - 1} a_i\Bigg) \times a_n}       &  & \by{ac:7.1.1} \\
                              & = \abs{\prod_{i = m}^{n - 1} a_i} \times \abs{a_n}                                \\
                              & = \Bigg(\prod_{i = m}^{n - 1} \abs{a_i}\Bigg) \times \abs{a_n} &  & \byIH         \\
                              & = \prod_{i = m}^n \abs{a_i}.                                   &  & \by{ac:7.1.1}
  \end{align*}
  This closes the induction.
\end{proof}

\begin{ac}\label{ac:7.1.3}
  Let \(X\) be a finite set with \(n\) elements (where \(n \in \N\)), and let \(f : X \to \R\) be a function from \(X\) to the real numbers
  (i.e., \(f\) assigns a real number \(f(x)\) to each element \(x\) of \(X\)).
  Then we can define the finite product \(\prod_{x \in X} f(x)\) as follows.
  We first select any bijection \(g\) from \(\set{i \in \N : 1 \leq i \leq n}\) to \(X\);
  such a bijection exists since \(X\) is assumed to have \(n\) elements.
  We then define
  \[
    \prod_{x \in X} f(x) \coloneqq \prod_{i = 1}^n f(g(i))
  \]
\end{ac}

\begin{ac}[Finite products are well-defined]\label{ac:7.1.4}
  Let \(X\) be a finite set with \(n\) elements (where \(n \in \N\)), let \(f : X \to \R\) be a function, and let \(g : \set{i \in \N : 1 \leq i \leq n} \to X\) and \(h : \set{i \in \N : 1 \leq i \leq n} \to X\) be bijections.
  Then we have
  \[
    \prod_{i = 1}^n f(g(i)) = \prod_{i = 1}^n f(h(i)).
  \]
\end{ac}

\begin{proof}
  Let \(P(n)\) be the assertion that ``For any set \(X\) of \(n\) elements, any function \(f : X \to \R\), and any two bijections \(g, h\) from \(\set{i \in \N : 1 \leq i \leq n}\) to \(X\), we have \(\prod_{i = 1}^n f(g(i)) = \prod_{i = 1}^n f(h(i))\)''.
  (More informally, \(P(n)\) is the assertion that \cref{ac:7.1.4} is true for that value of \(n\).)
  We use induction on \(n\);

  We first check the base case \(P(0)\).
  In this case \(\prod_{i = 1}^0 f(g(i))\) and \(\prod_{i = 1}^0 f(h(i))\) both equal to \(1\), by \cref{ac:7.1.1}, so we are done.

  Now suppose inductively that \(P(n)\) is true;
  we now prove that \(P(n + 1)\) is true.
  Thus, let \(X\) be a set with \(n + 1\) elements, let \(f : X \to \R\) be a function, and let \(g\) and \(h\) be bijections from \(\set{i \in N : 1 \leq i \leq n + 1}\) to \(X\).
  We have to prove that
  \[
    \prod_{i = 1}^{n + 1} f(g(i)) = \prod_{i = 1}^{n + 1} f(h(i)). \tag{ac 7.1}\label{eq ac 7.1}
  \]
  Let \(x \coloneqq g(n + 1)\);
  thus \(x\) is an element of \(X\).
  By \cref{ac:7.1.1}, we can expand the left-hand side of \eqref{eq ac 7.1} as
  \[
    \prod_{i = 1}^{n + 1} f(g(i)) = \Bigg(\prod_{i = 1}^n f(g(i))\Bigg) \times f(x).
  \]
  Now let us look at the right-hand side of \eqref{eq ac 7.1}.
  Since \(h\) is a bijection, we do know that there is \emph{some} index \(j\), with \(1 \leq j \leq n + 1\), for which \(h(j) = x\).
  We now use \cref{ac:7.1.1,ac:7.1.2} to write
  \begin{align*}
    \prod_{i = 1}^{n + 1} f(h(i)) & = \Bigg(\prod_{i = 1}^j f(h(i))\Bigg) \times \Bigg(\prod_{i = j + 1}^{n + 1} f(h(i))\Bigg)                      \\
                                  & = \Bigg(\prod_{i = 1}^{j - 1} f(h(i))\Bigg) \times f(h(j)) \times \Bigg(\prod_{i = j + 1}^{n + 1} f(h(i))\Bigg) \\
                                  & = \Bigg(\prod_{i = 1}^{j - 1} f(h(i))\Bigg) \times f(x) \times \Bigg(\prod_{i = j}^n f(h(i + 1))\Bigg).
  \end{align*}
  We now define the function \(\tilde{h} : \set{i \in \N : 1 \leq i \leq n} \to X - \set{x}\) by setting \(\tilde{h}(i) \coloneqq h(i)\) when \(i < j\) and \(\tilde{h}(i) \coloneqq h(i + 1)\) when \(i \geq j\).
  We can thus write the right-hand side of \eqref{eq ac 7.1} as
  \[
    = \Bigg(\prod_{i = 1}^{j - 1} f(\tilde{h}(i))\Bigg) \times f(x) \times \Bigg(\prod_{i = j}^n f(\tilde{h}(i))\Bigg) = \Bigg(\prod_{i = 1}^n f(\tilde{h}(i))\Bigg) \times f(x)
  \]
  where we have used \cref{ac:7.1.2} once again.
  Thus to finish the proof of \eqref{eq ac 7.1} we have to show that
  \[
    \prod_{i = 1}^n f(g(i)) = \prod_{i = 1}^n f(\tilde{h}(i)). \tag{ac 7.2}\label{eq ac 7.2}
  \]
  But the function \(g\) (when restricted to \(\set{i \in \N : 1 \leq i \leq n}\)) is a bijection from \(\set{i \in \N : 1 \leq i \leq n} \to X - \set{x}\).
  The function \(\tilde{h}\) is also a bijection from \(\set{i \in \N : 1 \leq i \leq n} \to X - \set{x}\) (cf. \cref{3.6.9}).
  Since \(X - \set{x}\) has \(n\) elements (by \cref{3.6.9}), the claim \eqref{eq ac 7.2} then follows directly from the induction hypothesis \(P(n)\).
\end{proof}

\begin{ac}[Basic properties of product over finite sets]\label{ac:7.1.5}
  \mbox{}
  \begin{enumerate}
    \item If \(X\) is empty, and \(f : X \to \R\) is a function (i.e., \(f\) is the empty function), we have
          \[
            \prod_{x \in X} f(x) = 1.
          \]
    \item If \(X\) consists of a single element, \(X = \set{x_0}\), and \(f : X \to \R\) is a function, we have
          \[
            \prod_{x \in X} f(x) = f(x_0).
          \]
    \item (Substitution, part I) If \(X\) is a finite set, \(f : X \to \R\) is a function, and \(g : Y \to X\) is a bijection, then
          \[
            \prod_{x \in X} f(x) = \prod_{y \in Y} f(g(y)).
          \]
    \item (Substitution, part II) Let \(n \leq m\) be integers, and let \(X\) be the set \(X \coloneqq \set{i \in \Z : n \leq i \leq m}\).
          If \(a_i\) is a real number assigned to each integer \(i \in X\), then we have
          \[
            \prod_{i = n}^m a_i = \prod_{i \in X} a_i.
          \]
    \item Let \(X, Y\) be disjoint finite sets (so \(X \cap Y = \emptyset\)), and \(f : X \cup Y \to \R\) is a function.
          Then we have
          \[
            \prod_{z \in X \cup Y} f(z) = \Bigg(\prod_{x \in X} f(x)\Bigg) \times \Bigg(\prod_{y \in Y} f(y)\Bigg).
          \]
    \item Let \(X\) be a finite set, and let \(f : X \to \R\) and \(g : X \to \R\) be functions.
          Then
          \[
            \prod_{x \in X} (f(x) \times g(x)) = \prod_{x \in X} f(x) \times \prod_{x \in X} g(x).
          \]
    \item Let \(X\) be a finite set, let \(f : X \to \R\) be a function, and let \(c\) be a real number.
          Then
          \[
            \prod_{x \in X} cf(x) = c^{\#(X)} \prod_{x \in X} f(x).
          \]
    \item Let \(X\) be a finite set, and let \(f : X \to \R\) be a function, then
          \[
            \abs{\prod_{x \in X} f(x)} = \prod_{x \in X} \abs{f(x)}.
          \]
  \end{enumerate}
\end{ac}

\begin{proof}{(a)}
  Let \(g : \set{i \in \N : 1 \leq i \leq 0} \to \emptyset\) be a function.
  Then \(g\) is a bijection and
  \begin{align*}
    \prod_{x \in X} f(x) & = \prod_{i = 1}^0 f(g(i)) &  & \by{ac:7.1.3} \\
                         & = 1.                      &  & \by{ac:7.1.1}
  \end{align*}
\end{proof}

\begin{proof}{(b)}
  Let \(g : \set{1} \to \set{x_0}\) be a function.
  Then \(g\) is a bijection and
  \begin{align*}
    \prod_{x \in X} f(x) & = \prod_{i = 1}^1 f(g(i))                            &  & \by{ac:7.1.3} \\
                         & = \bigg(\prod_{i = 1}^0 f(g(i))\bigg) \times f(g(1)) &  & \by{ac:7.1.1} \\
                         & = 1 \times f(g(1))                                   &  & \by{ac:7.1.1} \\
                         & = f(x_0).
  \end{align*}
\end{proof}

\begin{proof}{(c)}
  Let \(h : \set{i \in \N : 1 \leq i \leq \#(Y)} \to Y\) be a bijection.
  Since \(X\) is finite and \(g\) is a bijection between \(X\) and \(Y\), we know that \(Y\) is finite and thus such \(h\) is well-defined.
  Then we know that \(g \circ h : \set{i \in \N : 1 \leq i \leq \#(Y)} \to X\) is also a bijection and
  \begin{align*}
    \prod_{x \in X} f(x) & = \prod_{i = 1}^{\#(Y)} f((g \circ h)(i)) &  & \by{ac:7.1.3} \\
                         & = \prod_{i = 1}^{\#(Y)} f(g(h(i)))                           \\
                         & = \prod_{i = 1}^{\#(Y)} (f \circ g)(h(i))                    \\
                         & = \prod_{y \in Y} (f \circ g)(y)          &  & \by{ac:7.1.3} \\
                         & = \prod_{y \in Y} f(g(y)).
  \end{align*}
\end{proof}

\begin{proof}{(d)}
  Let \(f : X \to \set{a_i \in \R : n \leq i \leq m}\) be a function where \(f = i \mapsto a_i\).
  Let \(g : \set{i \in \N : 1 \leq i \leq m - n + 1} \to X\) be a function where \(g = i \mapsto i + n - 1\).
  Then \(g\) is a bijection and
  \begin{align*}
    \prod_{i \in X} a_i & = \prod_{i \in X} f(i)                                                                                \\
                        & = \prod_{i = 1}^{m - n + 1} f(g(i))                               &  & \by{ac:7.1.3}                  \\
                        & = \prod_{i = 1}^{m - n + 1} f(i + n - 1)                                                              \\
                        & = \prod_{i = 1}^{m - n + 1} a_{i + n - 1}                                                             \\
                        & = \prod_{i = 1 + n - 1}^{m - n + 1 + n - 1} a_{i + n - 1 - n + 1} &  & \text{(by \cref{ac:7.1.2}(b))} \\
                        & = \prod_{i = n}^m a_i.
  \end{align*}
\end{proof}

\begin{proof}{(e)}
  Let \(g : \set{i \in \N : 1 \leq i \leq \#(X)} \to X\) and \(h : \set{i \in \N : 1 \leq i \leq \#(Y)} \to Y\) be bijections.
  Since \(X, Y\) are finite, we know that \(g, h\) are well-defined and \(X \cup Y\) is finite.
  Let \(k : \set{i \in \N : 1 \leq i \leq \#(X \cup Y)} \to X \cup Y\) be a bijection where
  \[
    k(i) = \begin{dcases}
      g(i)         & \text{if } 1 \leq i \leq \#(X)                  \\
      h(i - \#(X)) & \text{if } \#(X) + 1 \leq i \leq \#(X) + \#(Y).
    \end{dcases}
  \]
  Since \(X \cup Y\) is finite, we know that \(k\) is well-defined and \(\#(X \cup Y) = \#(X) + \#(Y)\).
  Then we have
  \begin{align*}
     & \prod_{z \in X \cup Y} f(z)                                                                                                     \\
     & = \prod_{i = 1}^{\#(X \cup Y)} f(k(i))                                                      &  & \by{ac:7.1.3}                  \\
     & = \prod_{i = 1}^{\#(X)} f(k(i)) \times \prod_{i = \#(X) + 1}^{\#(X \cup Y)} f(k(i))         &  & \text{(by \cref{ac:7.1.2}(a))} \\
     & = \prod_{i = 1}^{\#(X)} f(g(i)) \times \prod_{i = \#(X) + 1}^{\#(X \cup Y)} f(h(i - \#(X)))                                     \\
     & = \prod_{i = 1}^{\#(X)} f(g(i)) \times \prod_{i = 1}^{\#(Y)} f(h(i))                        &  & \text{(by \cref{ac:7.1.2}(b))} \\
     & = \prod_{x \in X} f(x) \times \prod_{y \in Y} f(y).                                         &  & \by{ac:7.1.3}
  \end{align*}
\end{proof}

\begin{proof}{(f)}
  Let \(h : \set{i \in \N : 1 \leq i \leq \#(X)} \to X\) be a bijection.
  Since \(X\) is finite, we know that \(h\) is well-defined and
  \begin{align*}
     & \prod_{x \in X} (f(x) \times g(x))                                                                       \\
     & = \prod_{x \in X} (f \times g)(x)                                                                        \\
     & = \prod_{i = 1}^{\#(X)} (f \times g)(h(i))                           &  & \by{ac:7.1.3}                  \\
     & = \prod_{i = 1}^{\#(X)} (f(h(i)) \times g(h(i)))                                                         \\
     & = \prod_{i = 1}^{\#(X)} f(h(i)) \times \prod_{i = 1}^{\#(X)} g(h(i)) &  & \text{(by \cref{ac:7.1.2}(c))} \\
     & = \prod_{x \in X} f(x) \times \prod_{x \in X} g(x).                  &  & \by{ac:7.1.3}
  \end{align*}
\end{proof}

\begin{proof}{(g)}
  Let \(g : \set{i \in \N : 1 \leq i \leq \#(X)} \to X\) be a bijection.
  Since \(X\) is finite, we know that \(g\) is well-defined and
  \begin{align*}
    \prod_{x \in X} cf(x) & = \prod_{x \in X} (cf)(x)                                                     \\
                          & = \prod_{i = 1}^{\#(X)} (cf)(g(i))        &  & \by{ac:7.1.3}                  \\
                          & = \prod_{i = 1}^{\#(X)} cf(g(i))                                              \\
                          & = c^{\#(X)} \prod_{i = 1}^{\#(X)} f(g(i)) &  & \text{(by \cref{ac:7.1.2}(d))} \\
                          & = c^{\#(X)} \prod_{x \in X} f(x).         &  & \by{ac:7.1.3}
  \end{align*}
\end{proof}

\begin{proof}{(h)}
  Let \(g : \set{i \in \N : 1 \leq i \leq \#(X)} \to X\) be a bijection.
  Since \(X\) is finite, we know that \(g\) is well-defined and
  \begin{align*}
    \abs{\prod_{x \in X} f(x)} & = \abs{\prod_{i = 1}^{\#(X)} f(g(i))} &  & \by{ac:7.1.3}                  \\
                               & = \prod_{i = 1}^{\#(X)} \abs{f(g(i))} &  & \text{(by \cref{ac:7.1.2}(e))} \\
                               & = \prod_{x \in X} \abs{f(x)}.         &  & \by{ac:7.1.3}
  \end{align*}
\end{proof}

\begin{ac}\label{ac:7.1.6}
  Let \(X, Y\) be finite sets, and let \(f : X \times Y \to \R\) be a function.
  Then
  \[
    \prod_{x \in X} \bigg(\prod_{y \in Y} f(x, y)\bigg) = \prod_{(x, y) \in X \times Y} f(x, y).
  \]
\end{ac}

\begin{proof}
  Let \(n\) be the number of elements in \(X\).
  We will use induction on \(n\) (cf. \cref{ac:7.1.4});
  i.e., we let \(P(n)\) be the assertion that \cref{ac:7.1.6} is true for any set \(X\) with \(n\) elements, and any finite set \(Y\) and any function \(f : X \times Y \to \R\).
  We wish to prove \(P(n)\) for all natural numbers \(n\).

  The base case \(P(0)\) is easy, following from \cref{ac:7.1.5}(a).
  Now suppose that \(P(n)\) is true;
  we now show that \(P(n + 1)\) is true.
  Let \(X\) be a set with \(n + 1\) elements.
  In particular, by \cref{3.6.9}, we can write \(X = X' \cup \set{x_0}\), where \(x_0\) is an element of \(X\) and \(X' \coloneqq X - \set{x_0}\) has \(n\) elements.
  Then by \cref{7.1.5}(e) we have
  \[
    \prod_{x \in X} \bigg(\prod_{y \in Y} f(x, y)\bigg) = \prod_{x \in X'} \bigg(\prod_{y \in Y} f(x, y)\bigg) \times \bigg(\prod_{y \in Y} f(x_0, y)\bigg);
  \]
  by the induction hypothesis this is equal to
  \[
    \prod_{(x, y) \in X' \times Y} f(x, y) \times \bigg(\prod_{y \in Y} f(x_0, y)\bigg).
  \]
  By \cref{7.1.11}(c) this is equal to
  \[
    \prod_{(x, y) \in X' \times Y} f(x, y) \times \bigg(\prod_{(x, y) \in \set{x_0} \times Y} f(x, y)\bigg).
  \]
  By \cref{7.1.11}(e) this is equal to
  \[
    \prod_{(x, y) \in X \times Y} f(x, y)
  \]
  as desired.
\end{proof}

\begin{ac}\label{ac:7.1.7}
  Let \(X, Y\) be finite sets, and let \(f : X \times Y \to \R\) be a function.
  Then
  \begin{align*}
    \prod_{x \in X} \bigg(\prod_{y \in Y} f(x, y)\bigg) & = \prod_{(x, y) \in X \times Y} f(x, y)                \\
                                                        & = \prod_{(y, x) \in Y \times X} f(x, y)                \\
                                                        & = \prod_{y \in Y} \bigg(\prod_{x \in X} f(x, y)\bigg).
  \end{align*}
\end{ac}

\begin{proof}
  In light of \cref{ac:7.1.6}, it suffices to show that
  \[
    \prod_{(x, y) \in X \times Y} f(x, y) = \prod_{(y, x) \in Y \times X} f(x, y).
  \]
  But this follows from \cref{ac:7.1.5}(c) by applying the bijection \(h : Y \times X \to X \times Y\) defined by \(h(y, x) \coloneqq (x, y)\).
\end{proof}

\exercisesection

\begin{ex}\label{ex:7.1.1}
  Prove \cref{7.1.4}.
\end{ex}

\begin{proof}
  See \cref{7.1.4}.
\end{proof}

\begin{ex}\label{ex:7.1.2}
  Prove \cref{7.1.11}.
\end{ex}

\begin{proof}
  See \cref{7.1.11}.
\end{proof}

\begin{ex}\label{ex:7.1.3}
  Form a definition for the finite products \(\prod_{i = 1}^n a_i\) and \(\prod_{x \in X} f(x)\).
  Which of the above result for finite series have analoges for finite products?
\end{ex}

\begin{proof}
  See \crefrange{ac:7.1.1}{ac:7.1.7}.
\end{proof}

\begin{ex}\label{ex:7.1.4}
  Define the \emph{factorial function} \(n!\) for natural numbers \(n\) by the recursive definition \(0! \coloneqq 1\) and \((n + 1)! \coloneqq n! \times (n + 1)\).
  If \(x\) and \(y\) are real numbers, prove the \emph{binomial formula}
  \[
    (x + y)^n = \sum_{j = 0}^n \dfrac{n!}{j!(n - j)!} x^j y^{n - j}
  \]
  for all natural numbers \(n\).
\end{ex}

\begin{proof}
  We use induction on \(n\).
  For \(n = 0\), we have
  \begin{align*}
    (x + y)^0 & = 1                                                                                                                         \\
              & = \dfrac{0!}{0!(0 - 0)!} x^0 y^{0 - 0}                                                          &  & \text{(by definition)} \\
              & = \sum_{j = 0}^{-1} \dfrac{0!}{j!(0 - j)!} x^j y^{0 - j} + \dfrac{0!}{0!(0 - 0)!} x^0 y^{0 - 0} &  & \by{7.1.1}             \\
              & = \sum_{j = 0}^0 \dfrac{0!}{j!(0 - j)!} x^j y^{0 - j}                                           &  & \by{7.1.1}
  \end{align*}
  So the base case holds.
  Suppose inductively that for some \(n \geq 0\) the statement holds.
  Then for \(n + 1\), we have
  \begin{align*}
    (x + y)^{n + 1} & = (x + y)^n \times (x + y)                                                                                                   \\
                    & = \bigg(\sum_{j = 0}^n \dfrac{n!}{j!(n - j)!} x^j y^{n - j}\bigg) \times (x + y)            &  & \byIH                       \\
                    & = \bigg(\sum_{j = 0}^n \dfrac{n!}{j!(n - j)!} x^{j + 1} y^{n - j}\bigg)                                                      \\
                    & \quad + \bigg(\sum_{j = 0}^n \dfrac{n!}{j!(n - j)!} x^j y^{n + 1 - j}\bigg)                                                  \\
                    & = \bigg(\sum_{j = 0}^{n - 1} \dfrac{n!}{j!(n - j)!} x^{j + 1} y^{n - j}\bigg)               &  & \by{7.1.1}                  \\
                    & \quad + \bigg(\dfrac{n!}{n!0!} x^{n + 1} y^0\bigg)                                                                           \\
                    & \quad + \bigg(\sum_{j = 1}^n \dfrac{n!}{j!(n - j)!} x^j y^{n + 1 - j}\bigg)                                                  \\
                    & \quad + \bigg(\dfrac{n!}{0!n!} x^0 y^{n + 1}\bigg)                                                                           \\
                    & = \bigg(\sum_{j = 0}^{n - 1} \dfrac{n!}{j!(n - j)!} x^{j + 1} y^{n - j}\bigg) + x^{n + 1}   &  & \text{(by definition)}      \\
                    & \quad + \bigg(\sum_{j = 1}^n \dfrac{n!}{j!(n - j)!} x^j y^{n + 1 - j}\bigg) + y^{n + 1}                                      \\
                    & = \bigg(\sum_{j = 1}^n \dfrac{n!}{(j - 1)!(n + 1 - j)!} x^j y^{n + 1 - j}\bigg) + x^{n + 1} &  & \text{(by \cref{7.1.4}(b))} \\
                    & \quad + \bigg(\sum_{j = 1}^n \dfrac{n!}{j!(n - j)!} x^j y^{n + 1 - j}\bigg) + y^{n + 1}
  \end{align*}
  and
  \begin{align*}
     & \bigg(\sum_{j = 1}^n \dfrac{n!}{(j - 1)!(n + 1 - j)!} x^j y^{n + 1 - j}\bigg)                                                                                                         \\
     & \quad + \bigg(\sum_{j = 1}^n \dfrac{n!}{j!(n - j)!} x^j y^{n + 1 - j}\bigg)                                                                                                           \\
     & = \sum_{j = 1}^n \bigg(\dfrac{n!}{(j - 1)!(n + 1 - j)!} x^j y^{n + 1 - j} + \dfrac{n!}{j!(n - j)!} x^j y^{n + 1 - j}\bigg)                           &  & \text{(by \cref{7.1.4}(c))} \\
     & = \sum_{j = 1}^n \bigg(\dfrac{j \times n!}{j!(n + 1 - j)!} x^j y^{n + 1 - j} + \dfrac{(n + 1 - j) \times n!}{j!(n + 1 - j)!} x^j y^{n + 1 - j}\bigg)                                  \\
     & = \sum_{j = 1}^n \bigg(\dfrac{j \times n! + (n + 1 - j) \times n!}{j!(n + 1 - j)!} x^j y^{n + 1 - j}\bigg)                                                                            \\
     & = \sum_{j = 1}^n \bigg(\dfrac{(n + 1)!}{j!(n + 1 - j)!} x^j y^{n + 1 - j}\bigg).
  \end{align*}
  We also have
  \begin{align*}
     & \sum_{j = 0}^{n + 1} \dfrac{(n + 1)!}{j!(n + 1 - j)!} x^j y^{n + 1 - j}                                                                                        \\
     & = \dfrac{(n + 1)!}{(n + 1)! 0!} x^{n + 1} y^0 + \bigg(\sum_{j = 0}^n \dfrac{(n + 1)!}{j!(n + 1 - j)!} x^j y^{n + 1 - j}\bigg) &  & \by{7.1.1}                  \\
     & = x^{n + 1} + \bigg(\sum_{j = 0}^n \dfrac{(n + 1)!}{j!(n + 1 - j)!} x^j y^{n + 1 - j}\bigg)                                   &  & \text{(by definition)}      \\
     & = x^{n + 1} + \bigg(\sum_{j = 0}^0 \dfrac{(n + 1)!}{j!(n + 1 - j)!} x^j y^{n + 1 - j}\bigg)                                   &  & \text{(by \cref{7.1.4}(a))} \\
     & \quad + \bigg(\sum_{j = 1}^n \dfrac{(n + 1)!}{j!(n + 1 - j)!} x^j y^{n + 1 - j}\bigg)                                                                          \\
     & = x^{n + 1} + \dfrac{(n + 1)!}{0! (n + 1)!} x^0 y^{n + 1}                                                                     &  & \by{7.1.1}                  \\
     & \quad + \bigg(\sum_{j = 1}^n \dfrac{(n + 1)!}{j!(n + 1 - j)!} x^j y^{n + 1 - j}\bigg)                                                                          \\
     & = x^{n + 1} + y^{n + 1} + \bigg(\sum_{j = 1}^n \dfrac{(n + 1)!}{j!(n + 1 - j)!} x^j y^{n + 1 - j}\bigg).                      &  & \text{(by definition)}
  \end{align*}
  Thus we have
  \[
    (x + y)^{n + 1} = \sum_{j = 0}^{n + 1} \dfrac{(n + 1)!}{j!(n + 1 - j)!} x^j y^{n + 1 - j}.
  \]
  and this closes the induction.
\end{proof}

\begin{ex}\label{ex:7.1.5}
  Let \(X\) be a finite set, let \(m\) be an integer, and for each \(x \in X\) let \((a_n(x))_{n = m}^\infty\) be a convergent sequence of real numbers.
  Show that the sequence \((\sum_{x \in X} a_n(x))_{n = m}^\infty\) is convergent, and
  \[
    \lim_{n \to \infty} \sum_{x \in X} a_n(x) = \sum_{x \in X} \lim_{n \to \infty} a_n(x).
  \]
  Thus we may always interchange finite sums with convergent limits.
  Things however get trickier with infinite sums.
\end{ex}

\begin{proof}
  Let \(k = \#(X)\).
  We use induction on \(k\).
  For \(k = 0\), we have \(X = \emptyset\).
  So
  \begin{align*}
    \lim_{n \to \infty} \sum_{x \in X} a_n(x) & = \lim_{n \to \infty} 0                      &  & \by{7.1.11} \\
                                              & = 0                                                           \\
                                              & = \sum_{x \in X} \lim_{n \to \infty} a_n(x). &  & \by{7.1.11}
  \end{align*}
  Thus the base case holds.
  Suppose inductively that for some \(k \geq 0\) the statement is true.
  Then for \(k + 1\), we have to show that the statement is also true.
  Let \(x_0 \in X\) and \(X' = X \setminus \set{x_0}\).
  So \(\#(X') = \#(X) - 1 = n\), and we have
  \begin{align*}
     & \lim_{n \to \infty} \sum_{x \in X} a_n(x)                                                                                                                  \\
     & = \lim_{n \to \infty} \sum_{x \in \set{x_0} \cup X'} a_n(x)                                                                                                \\
     & = \lim_{n \to \infty} \bigg(\sum_{x \in \set{x_0}} a_n(x) + \sum_{x \in X'} a_n(x)\bigg)                                 &  & \text{(by \cref{7.1.11}(e))} \\
     & = \bigg(\lim_{n \to \infty} \sum_{x \in \set{x_0}} a_n(x)\bigg) + \bigg(\lim_{n \to \infty} \sum_{x \in X'} a_n(x)\bigg) &  & \text{(by \cref{6.1.19}(a))} \\
     & = \bigg(\lim_{n \to \infty} a_n(x_0)\bigg) + \bigg(\lim_{n \to \infty} \sum_{x \in X'} a_n(x)\bigg)                      &  & \text{(by \cref{7.1.11}(b))} \\
     & = \bigg(\sum_{x \in \set{x_0}} \lim_{n \to \infty} a_n(x)\bigg) + \bigg(\lim_{n \to \infty} \sum_{x \in X'} a_n(x)\bigg) &  & \text{(by \cref{7.1.11}(b))} \\
     & = \bigg(\sum_{x \in \set{x_0}} \lim_{n \to \infty} a_n(x)\bigg) + \bigg(\sum_{x \in X'} \lim_{n \to \infty} a_n(x)\bigg) &  & \byIH                        \\
     & = \bigg(\sum_{x \in \set{x_0} \cup X'} \lim_{n \to \infty} a_n(x)\bigg)                                                  &  & \text{(by \cref{7.1.11}(e))} \\
     & = \sum_{x \in X} \lim_{n \to \infty} a_n(x).
  \end{align*}
  This closes the induction.
\end{proof}
\section{Infinite series}\label{sec:7.2}

\begin{defn}[Formal infinite series]\label{7.2.1}
  A (formal) infinite series is any expression of the form
  \[
    \sum_{n = m}^\infty a_n,
  \]
  where \(m\) is an integer, and \(a_n\) is a real number for any integer \(n \geq m\).
\end{defn}

\begin{note}
  We sometimes write this series as
  \[
    a_m + a_{m + 1} + a_{m + 2} + \dots.
  \]
\end{note}

\begin{note}
  At present, this series is only defined \emph{formally};
  we have not set this sum equal to any real number;
  the notation \(a_m + a_{m + 1} + a_{m + 2} + \dots\) is of course designed to look very suggestively like a sum, but is not actually a finite sum because of the ``\(\dots\)'' symbol.
  To rigorously define what the series actually sums to, we need another definition.
\end{note}

\begin{defn}[Convergence of series]\label{7.2.2}
  Let \(\sum_{n = m}^\infty a_n\) be a formal infinite series.
  For any integer \(N \geq m\), we define the \emph{\(N^{\text{th}}\) partial sum} \(S_N\) of this series to be \(S_N \coloneqq \sum_{n = m}^N a_n\);
  of course, \(S_N\) is a real number.
  If the sequence \((S_N)_{N = m}^\infty\) converges to some limit \(L\) as \(N \to \infty\), then we say that the infinite series \(\sum_{n = m}^\infty a_n\) is \emph{convergent}, and \emph{converges to \(L\)};
  we also write \(L = \sum_{n = m}^\infty a_n\), and say that \(L\) is the \emph{sum} of the infinite series \(\sum_{n = m}^\infty a_n\).
  If the partial sums \(S_N\) diverge, then we say that the infinite series \(\sum_{n = m}^\infty a_n\) is \emph{divergent}, and we do not assign any real number value to that series.
\end{defn}

\begin{rmk}\label{7.2.3}
  Note that \cref{6.1.7} shows that if a series converges, then it has a unique sum, so it is safe to talk about \emph{the} sum \(L = \sum_{n = m}^\infty a_n\) of a convergent series.
\end{rmk}

\setcounter{thm}{4}
\begin{prop}\label{7.2.5}
  Let \(\sum_{n = m}^\infty a_n\) be a formal series of real numbers.
  Then \(\sum_{n = m}^\infty a_n\) converges if and only if, for every real number \(\varepsilon > 0\), there exists an integer \(N \geq m\) such that
  \[
    \abs{\sum_{n = p + 1}^q a_n} \leq \varepsilon \text{ for all } p, q \geq N.
  \]
\end{prop}

\begin{proof}
  Let \(N, p, q \in \N\).
  We first show that if \(\sum_{n = m}^\infty a_n\) converges, then \(\forall \varepsilon \in \R^+\), \(\exists\ N \geq m\) such that \(\abs{\sum_{n = p + 1}^q a_n} \leq \varepsilon\) for every \(p, q \geq N\).
  Let \(k \in \N\) and let \(S_k = \sum_{n = m}^k a_n\) be the \(k^{\text{th}}\) partial sum of \((a_n)_{n = m}^\infty\).
  Since \((S_k)_{k = m}^\infty\) converges, by \cref{6.4.18} \((S_k)_{k = m}^\infty\) is a Cauchy sequence.
  Then we have \(\forall \varepsilon \in \R^+\), \(\exists\ N \geq m\) such that \(\abs{S_q - S_p} \leq \varepsilon\) for every \(p, q \geq N\).
  We now split into two cases:
  \begin{itemize}
    \item If \(p \geq q\), then \(p + 1 > q\).
          By \cref{7.1.1} we have \(\abs{\sum_{n = p + 1}^q a_n} = \abs{0} = 0 \leq \varepsilon\).
    \item If \(p < q\), then
          \begin{align*}
                     & \abs{S_q - S_p} \leq \varepsilon                                                                                                                          \\
            \implies & \abs{\bigg(\sum_{n = m}^q a_n\bigg) - \bigg(\sum_{n = m}^p a_n\bigg)} \leq \varepsilon                                                                    \\
            \implies & \abs{\bigg(\sum_{n = m}^p a_n\bigg) + \bigg(\sum_{n = p + 1}^q a_n\bigg) - \bigg(\sum_{n = m}^p a_n\bigg)} \leq \varepsilon & \text{(by \cref{7.1.4}(a))} \\
            \implies & \abs{\sum_{n = p + 1}^q a_n} \leq \varepsilon.
          \end{align*}
  \end{itemize}
  From all cases above we conclude that \(\abs{\sum_{n = p + 1}^q a_n} \leq \varepsilon\) for every \(p, q \geq N\).

  Now we show that if \(\forall \varepsilon \in \R^+\), \(\exists\ N \geq m\) such that \(\abs{\sum_{n = p + 1}^q a_n} \leq \varepsilon\) for every \(p, q \geq N\), then \(\sum_{n = m}^\infty a_n\) converges.
  Since \(\abs{\sum_{n = p + 1}^q a_n} \leq \varepsilon\) for every \(p, q \geq N\), we can choose some \(p \leq q\) such that
  \begin{align*}
             & \abs{\sum_{n = p + 1}^q a_n} \leq \varepsilon                                                                      \\
    \implies & \abs{\sum_{n = p + 1}^q a_n + \sum_{n = m}^p a_n - \sum_{n = m}^p a_n} \leq \varepsilon                            \\
    \implies & \abs{\sum_{n = m}^q a_n - \sum_{n = m}^p a_n} \leq \varepsilon                          & \text{(by \cref{7.1.4})} \\
    \implies & \abs{S_q - S_p} \leq \varepsilon.
  \end{align*}
  This means \((S_k)_{k = m}^\infty\) is a Cauchy Sequence, so by \cref{6.4.18} \((S_k)_{k = m}^\infty\) converges, and by \cref{7.2.2} \(\sum_{n = m}^\infty a_n\) converges.
  We conclude that \(\sum_{n = m}^\infty a_n\) converges iff \(\forall \varepsilon \in \R^+\), \(\exists\ N \geq m\) such that \(\abs{\sum_{n = p + 1}^q a_n} \leq \varepsilon\) for every \(p, q \geq N\).
\end{proof}

\begin{cor}[Zero test]\label{7.2.6}
  Let \(\sum_{n = m}^\infty a_n\) be a convergent series of real numbers.
  Then we must have \(\lim_{n \to \infty} a_n = 0\).
  To put this another way, if \(\lim_{n \to \infty} a_n\) is non-zero or divergent, then the series \(\sum_{n = m}^\infty a_n\) is divergent.
\end{cor}

\begin{proof}
  Let \(N, p, q \in \N\).
  Then we have
  \begin{align*}
             & \sum_{n = m}^\infty a_n \text{ converges}                                                                                                          \\
    \implies & \forall \varepsilon \in \R^+, \exists\ N \geq m : \forall p, q \geq N, \abs{\sum_{n = p + 1}^q a_n} \leq \varepsilon    & \text{(by \cref{7.2.5})} \\
    \implies & \forall \varepsilon \in \R^+, \exists\ N \geq m : \forall p \geq N, \abs{\sum_{n = p + 1}^{p + 1} a_n} \leq \varepsilon                            \\
    \implies & \forall \varepsilon \in \R^+, \exists\ N \geq m : \forall p \geq N, \abs{a_{p + 1}} \leq \varepsilon                    & \text{(by \cref{7.1.1})} \\
    \implies & \forall \varepsilon \in \R^+, \exists\ N \geq m : \forall p \geq N, \abs{a_{p + 1} - 0} \leq \varepsilon                                           \\
    \implies & \lim_{n \to \infty} a_n = 0.                                                                                            & \text{(by \cref{6.1.8})}
  \end{align*}
\end{proof}

\begin{note}
  If a sequence \((a_n)_{n = m}^\infty\) \emph{does} converge to zero, then the series \(\sum_{n = m}^\infty a_n\) may or may not be convergent;
  it depends on the series.
\end{note}

\setcounter{thm}{7}
\begin{defn}[Absolute convergence]\label{7.2.8}
  Let \(\sum_{n = m}^\infty a_n\) be a formal series of real numbers.
  We say that this series is \emph{absolutely convergent} iff the series \(\sum_{n = m}^\infty \abs{a_n}\) is convergent.
\end{defn}

\begin{note}
  In order to distinguish convergence from absolute convergence, we sometimes refer to the former as \emph{conditional convergence}.
\end{note}

\begin{prop}[Absolute convergence test]\label{7.2.9}
  Let \(\sum_{n = m}^\infty a_n\) be a formal series of real numbers.
  If this series is absolutely convergent, then it is also conditionally convergent.
  Furthermore, in this case we have the triangle inequality
  \[
    \abs{\sum_{n = m}^\infty a_n} \leq \sum_{n = m}^\infty \abs{a_n}.
  \]
\end{prop}

\begin{proof}
  We first show that if \(\sum_{n = m}^\infty a_n\) is absolutely convergent, then it is also conditionally convergent.
  Let \(N, p, q \in \N\).
  Then we have
  \begin{align*}
             & \sum_{n = m}^\infty \abs{a_n} \text{ converge}                                                                                                            \\
    \implies & \forall \varepsilon \in \R^+, \exists\ N \geq m : \forall p, q \geq N, \abs{\sum_{n = p + 1}^q \abs{a_n}} \leq \varepsilon, & \text{(by \cref{7.2.5})}    \\
    \implies & \forall \varepsilon \in \R^+, \exists\ N \geq m : \forall p, q \geq N, \sum_{n = p + 1}^q \abs{a_n} \leq \varepsilon                                      \\
    \implies & \forall \varepsilon \in \R^+, \exists\ N \geq m : \forall p, q \geq N,                                                                                    \\
             & \abs{\sum_{n = p + 1}^q a_n} \leq \sum_{n = p + 1}^q \abs{a_n} \leq \varepsilon                                             & \text{(by \cref{7.1.4}(e))} \\
    \implies & \sum_{n = m}^\infty a_n \text{ converge}.                                                                                   & \text{(by \cref{7.2.5})}
  \end{align*}

  Now we show that the triangle inequality is true.
  From the proof above we know that \(\sum_{n = m}^\infty a_n\) converges, thus \(\abs{\sum_{n = m}^\infty a_n}\) exists and
  \begin{align*}
             & \forall N \geq m, \abs{\sum_{n = m}^N a_n} \leq \sum_{n = m}^N \abs{a_n}                                                                & \text{(by \cref{7.1.4}(e))}  \\
    \implies & \lim_{N \to \infty} \abs{\sum_{n = m}^N a_n} \leq \lim_{N \to \infty} \sum_{n = m}^N \abs{a_n}                                          & \text{(by \cref{6.4.13})}    \\
    \implies & \lim_{N \to \infty} \max(\sum_{n = m}^N a_n, -\sum_{n = m}^N a_n) \leq \lim_{N \to \infty} \sum_{n = m}^N \abs{a_n}                                                    \\
    \implies & \max(\lim_{N \to \infty} \sum_{n = m}^N a_n, \lim_{N \to \infty} -\sum_{n = m}^N a_n) \leq \lim_{N \to \infty} \sum_{n = m}^N \abs{a_n} & \text{(by \cref{6.1.19}(g))} \\
    \implies & \abs{\lim_{N \to \infty} \sum_{n = m}^N a_n} \leq \lim_{N \to \infty} \sum_{n = m}^N \abs{a_n}                                                                         \\
    \implies & \abs{\sum_{n = m}^\infty a_n} \leq \sum_{n = m}^\infty \abs{a_n}.                                                                       & \text{(by \cref{7.2.2})}
  \end{align*}
\end{proof}

\begin{rmk}\label{7.2.10}
  The converse to this proposition is not true;
  there exist series which are conditionally convergent but not absolutely convergent.
\end{rmk}

\begin{rmk}\label{7.2.11}
  We consider the class of conditionally convergent series to include the class of absolutely convergent series as a subclass.
  Thus when we say a statement such as ``\(\sum_{n = m}^\infty a_n\) is conditionally convergent'', this does not automatically mean that \(\sum_{n = m}^\infty a_n\) is not absolutely convergent.
  If we wish to say that a series is conditionally convergent but not absolutely convergent, then we will instead use a phrasing such as ``\(\sum_{n = m}^\infty a_n\) is \emph{only} conditionally convergent'', or ``\(\sum_{n = m}^\infty a_n\) converges conditionally, but not absolutely''.
  We caution however that in most other texts, the terminology ``conditional convergence'' is meant in this latter sense
  (that is, of a series that converges but does not converge absolutely).
\end{rmk}

\begin{prop}[Alternating series test]\label{7.2.12}
  Let \((a_n)_{n = m}^\infty\) be a sequence of real numbers which are non-negative and decreasing, thus \(a_n \geq 0\) and \(a_n \geq a_{n + 1}\) for every \(n \geq m\).
  Then the series \(\sum_{n = m}^\infty (-1)^n a_n\) is convergent if and only if the sequence \(a_n\) converges to \(0\) as \(n \to \infty\).
\end{prop}

\begin{proof}
  From the zero test (\cref{7.2.6}), we know that if \(\sum_{n = m}^\infty (-1)^n a_n\) is a convergent series, then the sequence \(((-1)^n a_n)_{n = m}^\infty\) converges to \(0\), which implies that \(a_n\) also converges to \(0\), since \((-1)^n a_n\) and \(a_n\) have the same distance from \(0\).
  \begin{align*}
     & \forall \varepsilon \in \R^+, \exists\ N \in \N \land N \geq m : \forall n \geq N,   \\
     & \abs{a_n - 0} = \abs{a_n} = \abs{(-1)^n a_n} = \abs{(-1)^n a_n - 0} \leq \varepsilon
  \end{align*}

  Now suppose conversely that \(a_n\) converges to \(0\).
  For each \(N\), let \(S_N\) be the partial sum \(S_N \coloneqq \sum_{n = m}^N (-1)^n a_n\);
  our job is to show that \(S_N\) converges.
  Observe that
  \begin{align*}
    S_{N + 2} & = S_N + (-1)^{N + 1} a_{N + 1} + (-1)^{N + 2} a_{N + 2} \\
              & = S_N + (-1)^{N + 1} (a_{N + 1} - a_{N + 2}).
  \end{align*}
  But by hypothesis, \((a_{N + 1} - a_{N + 2})\) is non-negative.
  Thus we have \(S_{N + 2} \geq S_N\) when \(N\) is odd and \(S_{N + 2} \leq S_N\) if \(N\) is even.

  Now suppose that \(N\) is even.
  From the above discussion and induction we see that \(S_{N + 2k} \leq S_N\) for all natural numbers \(k\).
  Also we have \(S_{N + 2k + 1} \geq S_{N + 1} = S_N - a_{N + 1}\).
  Finally, we have \(S_{N + 2k + 1} = S_{N + 2k} - a_{N + 2k + 1} \leq S_{N + 2k}\).
  Thus we have
  \[
    S_N - a_{N + 1} \leq S_{N + 2k + 1} \leq S_{N + 2k} \leq S_N
  \]
  for all \(k\).
  In particular, we have
  \[
    S_N - a_{N + 1} \leq S_n \leq S_N \text{ for all } n \geq N.
  \]
  In particular, the sequence \((S_n)_{n = m}^\infty\) is eventually \(a_{N + 1}\)-steady
  \[
    S_p - S_n \leq S_N - S_n \leq a_{N + 1} \text{ for all } p, n \geq N.
  \]
  But the sequence \((a_N)_{N = m}^\infty\) converges to \(0\) as \(N \to \infty\), thus this implies that \((S_n)_{n = m}^\infty\) is eventually \(\varepsilon\)-steady for every \(\varepsilon > 0\).
  Thus \((S_n)_{n = m}^\infty\) converges, and so the series \(\sum_{n = m}^\infty (-1)^n a_n\) is convergent.
\end{proof}

\begin{note}
  Lack of absolute convergence does not imply lack of conditional convergence, even though absolute convergence implies conditional convergence.
\end{note}

\setcounter{thm}{13}
\begin{prop}[Series law]\label{7.2.14}
  \mbox{}
  \begin{enumerate}
    \item If \(\sum_{n = m}^\infty a_n\) is a series of real numbers converging to \(x\), and \(\sum_{n = m}^\infty b_n\) is a series of real numbers converging to \(y\), then \(\sum_{n = m}^\infty (a_n + b_n)\) is also a convergent series, and converges to \(x + y\).
          In particular, we have
          \[
            \sum_{n = m}^\infty (a_n + b_n) = \sum_{n = m}^\infty a_n + \sum_{n = m}^\infty b_n.
          \]
    \item If \(\sum_{n = m}^\infty a_n\) is a series of real numbers converging to \(x\), and \(c\) is a real number, then \(\sum_{n = m}^\infty (c a_n)\) is also a convergent series, and converges to \(cx\).
          In particular, we have
          \[
            \sum_{n = m}^\infty (c a_n) = c \sum_{n = m}^\infty a_n.
          \]
    \item Let \(\sum_{n = m}^\infty a_n\) be a series of real numbers, and let \(k \geq 0\) be an integer.
          If one of the two series \(\sum_{n = m}^\infty a_n\) and \(\sum_{n = m + k}^\infty a_n\) are convergent, then the other one is also, and we have the identity
          \[
            \sum_{n = m}^\infty a_n = \sum_{n = m}^{m + k - 1} a_n + \sum_{n = m + k}^\infty a_n.
          \]
    \item Let \(\sum_{n = m}^\infty a_n\) be a series of real numbers converging to \(x\), and let \(k\) be an integer.
          Then \(\sum_{n = m + k}^\infty a_{n - k}\) also converges to \(x\).
  \end{enumerate}
\end{prop}

\begin{proof}{(a)}
  We have
  \begin{align*}
    x + y & = \sum_{n = m}^\infty a_n + \sum_{n = m}^\infty b_n                                                              \\
          & = \lim_{N \to \infty} \sum_{n = m}^N a_n + \lim_{N \to \infty} \sum_{n = m}^N b_n & \text{(by \cref{7.2.2})}     \\
          & = \lim_{N \to \infty} \Bigg(\sum_{n = m}^N a_n + \sum_{n = m}^N b_n\Bigg)         & \text{(by \cref{6.1.19}(a))} \\
          & = \lim_{N \to \infty} \sum_{n = m}^N (a_n + b_n)                                  & \text{(by \cref{7.1.4}(c))}  \\
          & = \sum_{n = m}^\infty (a_n + b_n).                                                & \text{(by \cref{7.2.2})}
  \end{align*}
\end{proof}

\begin{proof}{(b)}
  We have
  \begin{align*}
    cx & = c \sum_{n = m}^\infty a_n                                                           \\
       & = c \lim_{N \to \infty} \sum_{n = m}^N a_n             & \text{(by \cref{7.2.2})}     \\
       & = \lim_{N \to \infty} \Bigg(c \sum_{n = m}^N a_n\Bigg) & \text{(by \cref{6.1.19}(c))} \\
       & = \lim_{N \to \infty} \sum_{n = m}^N (c a_n)           & \text{(by \cref{7.1.4}(d))}  \\
       & = \sum_{n = m}^\infty (c a_n).                         & \text{(by \cref{7.2.2})}
  \end{align*}
\end{proof}

\begin{proof}{(c)}
  Let \(M \in \N\).
  Then we have
  \begin{align*}
         & L = \sum_{n = m}^\infty a_n = \lim_{N \to \infty} \sum_{n = m}^N a_n                                                  & \text{(by \cref{7.2.2})}    \\
    \iff & \forall \varepsilon \in \R^+, \exists\ M \geq m : \forall N \geq M, \abs{\sum_{n = m}^N a_N - L} \leq \varepsilon                                   \\
    \iff & \forall \varepsilon \in \R^+, \exists\ M \geq m + k : \forall N \geq M, \abs{\sum_{n = m}^N a_N - L} \leq \varepsilon                               \\
    \iff & \forall \varepsilon \in \R^+, \exists\ M \geq m + k : \forall N \geq M,                                                                             \\
         & \abs{\sum_{n = m}^{m + k - 1} a_N + \sum_{n = m + k}^N a_N - L} \leq \varepsilon                                      & \text{(by \cref{7.1.4}(a))} \\
    \iff & L = \lim_{N \to \infty} \Bigg(\sum_{n = m}^{m + k - 1} a_N + \sum_{n = m + k}^N a_n\Bigg)                                                           \\
    \iff & L = \sum_{n = m}^{m + k - 1} a_N + \lim_{N \to \infty} \sum_{n = m + k}^N a_n                                                                       \\
    \iff & L = \sum_{n = m}^{m + k - 1} a_N + \sum_{n = m + k}^\infty a_n.                                                       & \text{(by \cref{7.2.2})}
  \end{align*}
\end{proof}

\begin{proof}{(d)}
  Let \(M \in \N\).
  Then we have
  \begin{align*}
             & x = \sum_{n = m}^\infty a_n = \lim_{N \to \infty} \sum_{n = m}^N a_n                                                  & \text{(by \cref{7.2.2})}    \\
    \implies & \forall \varepsilon \in \R^+, \exists\ M \geq m : \forall N \geq M, \abs{\sum_{n = m}^N a_n - x} \leq \varepsilon                                   \\
    \implies & \forall \varepsilon \in \R^+, \exists\ M \geq m : \forall N \geq M, \abs{\sum_{n = m}^N a_n - x} \leq \varepsilon                                   \\
    \implies & \forall \varepsilon \in \R^+, \exists\ M \geq m - k : \forall N \geq M, \abs{\sum_{n = m}^N a_n - x} \leq \varepsilon                               \\
    \implies & \forall \varepsilon \in \R^+, \exists\ M \geq m - k : \forall N \geq M,                                                                             \\
             & \abs{\sum_{n = m + k}^{N + k} a_{n - k} - x} \leq \varepsilon.                                                        & \text{(by \cref{7.1.4}(b))}
  \end{align*}
  Now let \(M' = M + k\) and \(N' = N + k\).
  Then we have
  \begin{align*}
             & \forall \varepsilon \in \R^+, \exists\ M \geq m - k : \forall N \geq M,                                                  \\
             & \abs{\sum_{n = m + k}^{N + k} a_{n - k} - x} \leq \varepsilon                                                            \\
    \implies & \forall \varepsilon \in \R^+, \exists\ M + k \geq m : \forall N + k \geq M + k,                                          \\
             & \abs{\sum_{n = m + k}^{N + k} a_{n - k} - x} \leq \varepsilon                                                            \\
    \implies & \forall \varepsilon \in \R^+, \exists\ M' \geq m : \forall N' \geq M',                                                   \\
             & \abs{\sum_{n = m + k}^{N'} a_{n - k} - x} \leq \varepsilon                                                               \\
    \implies & x = \lim_{N' \to \infty} \sum_{n = m + k}^{N'} a_{n - k} = \sum_{n = m + k}^\infty a_{n - k}. & \text{(by \cref{7.2.2})}
  \end{align*}
\end{proof}

\begin{note}
  From \cref{7.2.14}(c) we see that the convergence of a series does not depend on the first few elements of the series
  (though of course those elements do influence which value the series converges to).
  Because of this, we will usually not pay much attention as to what the initial index \(m\) of the series is.
\end{note}

\begin{lem}[Telescoping series]\label{7.2.15}
  Let \((a_n)_{n = 0}^\infty\) be a sequence of real numbers which converge to \(0\), i.e., \(\lim_{n \to \infty} a_n = 0\).
  Then the series \(\sum_{n = 0}^\infty (a_n - a_{n + 1})\) converges to \(a_0\).
  If \(\lim_{n \to \infty} a_n = L\), then the series \(\sum_{n = 0}^\infty (a_n - a_{n + 1})\) converges to \(a_0 + L\).
\end{lem}

\begin{proof}
  Let \(S_N = \sum_{n = 0}^N (a_n - a_{n + 1})\) be the \(N^{\text{th}}\) partial sum of \(\sum_{n = m}^\infty (a_n - a_{n + 1})\).
  We first show that \(S_N = a_0 - a_{N + 1}\) for every \(N \in \N\).
  We use induction on \(N\).
  For \(N = 0\), by \cref{7.1.1} we have
  \[
    S_0 = \sum_{n = 0}^0 a_n - a_{n + 1} = a_0 - a_1,
  \]
  so the base case holds.
  Suppose inductively that for some \(N \geq 0\) the statement holds.
  Then for \(N + 1\), we have
  \begin{align*}
    S_{N + 1} & = \sum_{n = 0}^{N + 1} (a_n - a_{n + 1})                                                      \\
              & = \sum_{n = 0}^N (a_n - a_{n + 1}) + a_{N + 1} - a_{N + 2} & \text{(by \cref{7.1.1})}         \\
              & = a_0 - a_{N + 1} + a_{N + 1} - a_{N + 2}                  & \text{(by induction hypothesis)} \\
              & = a_0 - a_{N + 2}.
  \end{align*}
  This closes the induction.

  Now we show that if \(\lim_{n \to \infty} a_n = L\), then \(\sum_{n = 0}^\infty (a_n - a_{n + 1}) = a_0 + L\).
  \begin{align*}
    \sum_{n = 0}^\infty (a_n - a_{n + 1}) & = \lim_{N \to \infty} S_N       & \text{(by \cref{7.2.2})}     \\
                                          & = \lim_{N \to \infty} a_0 + a_N & \text{(from claim above)}    \\
                                          & = a_0 + \lim_{N \to \infty} a_N & \text{(by \cref{6.1.19}(a))} \\
                                          & = a_0 + L.
  \end{align*}
  In particular, we have \(\sum_{n = 0}^\infty (a_n - a_{n + 1}) = a_0\) if \(\lim_{n \to \infty} a_n = 0\).
\end{proof}

\exercisesection

\begin{ex}\label{ex:7.2.1}
  Is the series \(\sum_{n = 1}^\infty (-1)^n\) convergent or divergent?
  Justify your answer.
\end{ex}

\begin{proof}
  Let \((a_n)_{n = 1}^\infty\) be a sequence where \(a_n = 1\) for every \(n \geq 1\).
  Since \(\lim_{n \to \infty} a_n = 1\), by alternating series test (\cref{7.2.12}) we know that \(\sum_{n = m}^\infty (-1)^n a_n\) diverges.
\end{proof}

\begin{ex}\label{ex:7.2.2}
  Prove \cref{7.2.5}.
\end{ex}

\begin{proof}
  See \cref{7.2.5}.
\end{proof}

\begin{ex}\label{ex:7.2.3}
  Use \cref{7.2.5} to prove \cref{7.2.6}.
\end{ex}

\begin{proof}
  See \cref{7.2.6}.
\end{proof}

\begin{ex}\label{ex:7.2.4}
  Prove \cref{7.2.9}.
\end{ex}

\begin{proof}
  See \cref{7.2.9}.
\end{proof}

\begin{ex}\label{ex:7.2.5}
  Prove \cref{7.2.14}.
\end{ex}

\begin{proof}
  See \cref{7.2.14}.
\end{proof}

\begin{ex}\label{ex:7.2.6}
  Prove \cref{7.2.15}.
  How does the proposition change if we assume that an does not converge to zero, but instead converges to some other real number \(L\)?
\end{ex}

\begin{proof}
  See \cref{7.2.15}.
\end{proof}
\section{Sums of non-negative numbers}\label{i:sec:7.3}

\begin{note}
  When all the terms in a series are non-negative, there is no distinction between conditional convergence and absolute convergence.
\end{note}

\begin{prop}\label{i:7.3.1}
  Let \(\sum_{n = m}^\infty a_n\) be a formal series of non-negative real numbers.
  Then this series is convergent iff there is a real number \(M\) such that
  \[
    \sum_{n = m}^N a_n \leq M \text{ for all integers } N \geq m.
  \]
\end{prop}

\begin{proof}
  Suppose \(\sum_{n = m}^\infty a_n\) is a series of non-negative numbers.
  Then the partial sums \(S_N \coloneqq \sum_{n = m}^N a_n\) are increasing, i.e., \(S_{N + 1} \geq S_N\) for all \(N \geq m\).
  From \cref{i:6.3.8} and \cref{i:6.1.17}, we thus see that the sequence \((S_N)_{n = m}^\infty\) is convergent iff it has an upper bound \(M\).
\end{proof}

\begin{cor}[Comparison test]\label{i:7.3.2}
  Let \(\sum_{n = m}^\infty a_n\) and \(\sum_{n = m}^\infty b_n\) be two formal series of real numbers, and suppose that \(\abs{a_n} \leq b_n\) for all \(n \geq m\).
  Then if \(\sum_{n = m}^\infty b_n\) is convergent, then \(\sum_{n = m}^\infty a_n\) is absolutely convergent, and in fact
  \[
    \abs{\sum_{n = m}^\infty a_n} \leq \sum_{n = m}^\infty \abs{a_n} \leq \sum_{n = m}^\infty b_n.
  \]
\end{cor}

\begin{proof}
  Let \(N \in \N\).
  Then we have
  \begin{align*}
             & \sum_{n = m}^\infty b_n \text{ converges}                                                                                  \\
    \implies & \exists M \in \R : \forall N \geq m, \sum_{n = m}^N b_n \leq M                                 &  & \by{i:7.3.1}           \\
    \implies & \exists M \in \R : \forall N \geq m, \sum_{n = m}^N \abs{a_n} \leq \sum_{n = m}^N b_n \leq M   &  & \text{(by hypothesis)} \\
    \implies & \sum_{n = m}^\infty \abs{a_n} \text{ converges}                                                &  & \by{i:7.3.1}           \\
    \implies & \sum_{n = m}^\infty \abs{a_n} \leq \sum_{n = m}^\infty b_n                                     &  & \by{i:6.4.13}          \\
    \implies & \abs{\sum_{n = m}^\infty a_n} \leq \sum_{n = m}^\infty \abs{a_n} \leq \sum_{n = m}^\infty b_n. &  & \by{i:7.2.9}
  \end{align*}
\end{proof}

\begin{note}
  We can also run the comparison test in the contrapositive:
  if we have \(\abs{a_n} \leq b_n\) for all \(n \geq m\), and \(\sum_{n = m}^\infty a_n\) is not absolutely convergent, then \(\sum_{n = m}^\infty b_n\) is not conditionally convergent.
\end{note}

\begin{lem}[Geometric series]\label{i:7.3.3}
  Let \(x\) be a real number.
  If \(\abs{x} \geq 1\), then the series \(\sum_{n = 0}^\infty x^n\) is divergent.
  If however \(\abs{x} < 1\), then the series is absolutely convergent and
  \[
    \sum_{n = 0}^\infty x^n = 1 / (1 - x).
  \]
\end{lem}

\begin{proof}
  We first show that if \(\abs{x} \geq 1\), then \(\sum_{n = 0}^\infty x^n\) is divergent.
  We split into two cases:
  \begin{enumerate}
    \item If \(x = 1\), then \(\lim_{n \to \infty} x^n = 1\). By zero test (\cref{i:7.2.6}) \(\sum_{n = 0}^\infty x^n\) diverges.
    \item If \(x = -1\) or \(\abs{x} > 1\), then by \cref{i:6.5.2} \(\lim_{n \to \infty} x^n\) diverges.
          Thus by zero test (\cref{i:7.2.6}) \(\sum_{n = 0}^\infty x^n\) diverges.
  \end{enumerate}
  From all cases above we conclude that if \(\abs{x} \geq 1\), then \(\sum_{n = 0}^\infty x^n\) diverges.

  Next we show that \(\sum_{n = 0}^N x^n = (1 - x^{N + 1}) / (1 - x)\).
  We use induction on \(N\).
  For \(N = 0\), by \cref{i:7.1.1} we have
  \[
    \sum_{n = 0}^0 x^n = x^0 = 1 = \dfrac{1 - x}{1 - x} = \dfrac{1 - x^1}{1 - x}.
  \]
  So the base case holds.
  Suppose inductivly that for some \(N \geq 0\) we have \(\sum_{n = 0}^N x^n = (1 - x^{N + 1}) / (1 - x)\).
  Then for \(N + 1\), we have
  \begin{align*}
    \sum_{n = 0}^{N + 1} x^n & = \sum_{n = 0}^N x^n + x^{N + 1}                                      &  & \by{i:7.1.1} \\
                             & = \dfrac{1 - x^{N + 1}}{1 - x} + x^{N + 1}                            &  & \byIH        \\
                             & = \dfrac{1 - x^{N + 1}}{1 - x} + \dfrac{(1 - x) x^{N + 1}}{1 - x}                       \\
                             & = \dfrac{1 - x^{N + 1}}{1 - x} + \dfrac{x^{N + 1} - x^{N + 2}}{1 - x}                   \\
                             & = \dfrac{1 - x^{N + 2}}{1 - x}.
  \end{align*}
  This closes the induction.
  Using similar argument we can show that
  \[
    \sum_{n = 0}^N \abs{x^n} = \dfrac{1 - \abs{x^{N + 1}}}{1 - \abs{x}}.
  \]

  Now we use the induction result to show that if \(\abs{x} < 1\), then \(\sum_{n = 0}^\infty x^n\) is absolutely convergent and \(\sum_{n = 0}^\infty x^n = 1 / (1 - x)\).
  \begin{align*}
    \sum_{n = 0}^\infty x^n & = \lim_{N \to \infty} \sum_{n = 0}^N x^n                               &  & \by{i:7.2.2}              \\
                            & = \lim_{N \to \infty} \dfrac{1 - x^{N + 1}}{1 - x}                     &  & \text{(from claim above)} \\
                            & = \dfrac{\lim_{N \to \infty} 1 - x^{N + 1}}{1 - x}                     &  & \by{i:6.1.19}[f]          \\
                            & = \dfrac{\lim_{N \to \infty} 1 - \lim_{N \to \infty} x^{N + 1}}{1 - x} &  & \by{i:6.1.19}[d]          \\
                            & = \dfrac{1 - (\lim_{N \to \infty} x^{N + 1})}{1 - x}                   &  & \by{i:ac:6.5.1}           \\
                            & = \dfrac{1 - 0}{1 - x}                                                 &  & \by{i:6.5.2}              \\
                            & = \dfrac{1}{1 - x}.
  \end{align*}
  Using similar argument we can show that \(\sum_{n = 0}^\infty \abs{x^n} = 1 / (1 - \abs{x})\).
  Thus we conclude that if \(\abs{x} < 1\), then \(\sum_{n = 0}^\infty x^n\) is absolutely convergent and \(\sum_{n = 0}^\infty x^n = 1 / (1 - x)\).
\end{proof}

\begin{prop}[Cauchy criterion]\label{i:7.3.4}
  Let \((a_n)_{n = 1}^\infty\) be a decreasing sequence of non-negative real numbers
  (so \(a_n \geq 0\) and \(a_{n + 1} \leq a_n\) for all \(n \geq 1\)).
  Then the series \(\sum_{n = 1}^\infty a_n\) is convergent iff the series
  \[
    \sum_{k = 0}^\infty 2^k a_{2^k} = a_1 + 2a_2 + 4a_4 + 8a_8 + \dots
  \]
  is convergent.
\end{prop}

\begin{proof}
  Let \(S_N \coloneqq \sum_{n = 1}^N a_n\) be the partial sums of \(\sum_{n = 1}^\infty a_n\), and let \(T_K \coloneqq \sum_{k = 0}^K 2^k a_{2^k}\) be the partial sums of \(\sum_{k = 0}^\infty 2^k a_{2^k}\).
  In light of \cref{i:7.3.1}, our task is to show that the sequence \((S_N)_{N = 1}^\infty\) is bounded iff the sequence \((T_K)_{K = 0}^\infty\) is bounded.
  From \cref{i:7.3.6} we see that if \((S_N)_{N = 1}^\infty\) is bounded, then \((S_{2^K})_{K = 0}^\infty\) is bounded, and hence \((T_K)_{K = 0}^\infty\) is bounded.
  Conversely, if \((T_K)_{K = 0}^\infty\) is bounded, then \cref{i:7.3.6} implies that \((S_{2^{K + 1} - 1})_{K = 0}^\infty\) is bounded, i.e., there is an \(M\) such that \(S_{2^{K + 1} - 1} \leq M\) for all natural numbers \(K\).
  But one can easily show (using induction) that \(2^{K + 1} - 1 \geq K + 1\), and hence that \(S_{K + 1} \leq M\) for all natural numbers \(K\), hence \((S_N)_{N = 1}^\infty\) is bounded.
\end{proof}

\begin{rmk}\label{i:7.3.5}
  An interesting feature of this criterion is that it only uses a small number of elements of the sequence \(a_n\)
  (namely, those elements whose index \(n\) is a power of \(2\), \(n = 2^k\))
  in order to determine whether the whole series is convergent or not.
\end{rmk}

\begin{lem}\label{i:7.3.6}
  For any natural number \(K\), we have \(S_{2^{K + 1} - 1} \leq T_K \leq 2S_{2^K}\).
\end{lem}

\begin{proof}
  We use induction on \(K\).
  First we prove the claim when \(K = 0\), i.e.
  \[
    S_1 \leq T_0 \leq 2S_1.
  \]
  This becomes
  \[
    a_1 \leq a_1 \leq 2a_1
  \]
  which is clearly true, since \(a_1\) is non-negative.
  Now suppose the claim has been proven for \(K\), and now we try to prove it for \(K + 1\):
  \[
    S_{2^{K + 2} - 1} \leq T_{K + 1} \leq 2S_{2^{K + 1}}.
  \]
  Clearly we have
  \[
    T_{K + 1} = T_K + 2^{K + 1} a_{2^{K + 1}}.
  \]
  Also, we have
  (using \cref{i:7.1.4}(a) and (f), and the hypothesis that the \(a_n\) are decreasing)
  \[
    S_{2^{K + 1}} = S_{2^K} + \sum_{n = 2^K + 1}^{2^{K + 1}} a_n \geq S_{2^K} + \sum_{n = 2^K + 1}^{2^{K + 1}} a_{2^{K + 1}} = S_{2^K} + 2^K a_{2^{K + 1}}
  \]
  and hence
  \[
    2S_{2^{K + 1}} \geq 2S_{2^K} + 2^{K + 1} a_{2^{K + 1}}.
  \]
  Similarly we have
  \begin{align*}
    S_{2^{K + 2} - 1} & = S_{2^{K + 1} - 1} + \sum_{n = 2^{K + 1}}^{2^{K + 2} - 1} a_n              \\
                      & \leq S_{2^{K + 1} - 1} + \sum_{n = 2^{K + 1}}^{2^{K + 2} - 1} a_{2^{K + 1}} \\
                      & = S_{2^{K + 1} - 1} + 2^{K + 1} a_{2^{K + 1}}.
  \end{align*}
  Combining these inequalities with the induction hypothesis
  \[
    S_{2^{K + 1} - 1} \leq T_K \leq 2S_{2^K}
  \]
  we obtain
  \[
    S_{2^{K + 2} - 1} \leq T_{K + 1} \leq 2S_{2^{K + 1}}
  \]
  as desired.
  This proves the claim.
\end{proof}

\begin{cor}\label{i:7.3.7}
  Let \(q > 0\) be a real number.
  Then the series \(\sum_{n = 1}^\infty 1 / n^q\) is convergent when \(q > 1\) and divergent when \(q \leq 1\).
\end{cor}

\begin{proof}
  The sequence \((1 / n^q)_{n = 1}^\infty\) is non-negative and decreasing (by \cref{i:6.7.3}), and so the Cauchy criterion (\cref{i:7.3.4}) applies.
  Thus this series is convergent iff
  \[
    \sum_{k = 0}^\infty 2^k \dfrac{1}{(2^k)^q}
  \]
  is convergent.
  But by the laws of exponentiation (\cref{i:6.7.3}) we can rewrite this as the geometric series
  \[
    \sum_{k = 0}^\infty (2^{1 - q})^k.
  \]
  By \cref{i:7.3.3}, the geometric series \(\sum_{k = 0}^\infty x^k\) converges iff \(\abs{x} < 1\).
  Thus the series \(\sum_{n = 1}^\infty 1 / n^q\) will converge iff \(\abs{2^{1 - q}} < 1\), which happens iff \(q > 1\).
\end{proof}

\begin{note}
  In particular, the series \(\sum_{n = 1}^\infty 1 / n\) (also known as the \emph{harmonic series}) is divergent, as claimed earlier.
  However, the series is \(\sum_{n = 1}^\infty 1 / n^2\) convergent.
\end{note}

\begin{rmk}\label{i:7.3.8}
  The quantity \(\sum_{n = 1}^\infty 1 / n^q\), when it converges, is called \(\zeta(q)\), the \\
  \emph{Riemann-zeta function of \(q\)}.
  This function is very important in number theory, and in particular in the distribution of the primes;
  there is a very famous unsolved problem regarding this function, called the \emph{Riemann hypothesis}, but to discuss it further is far beyond the scope of this text.
  I will mention however that there is a US\$ \(1\) million prize
  - and instant fame among all mathematicians -
  attached to the solution to this problem.
\end{rmk}

\exercisesection

\begin{ex}\label{i:ex:7.3.1}
  Use \cref{i:7.3.1} to prove \cref{i:7.3.2}.
\end{ex}

\begin{proof}
  See \cref{i:7.3.2}.
\end{proof}

\begin{ex}\label{i:ex:7.3.2}
  Prove \cref{i:7.3.3}.
\end{ex}

\begin{proof}
  See \cref{i:7.3.3}.
\end{proof}

\begin{ex}\label{i:ex:7.3.3}
  Let \(\sum_{n = 0}^\infty a_n\) be an absolutely convergent series of real numbers such that \(\sum_{n = 0}^\infty \abs{a_n} = 0\).
  Show that \(a_n = 0\) for every natural number \(n\).
\end{ex}

\begin{proof}
  Let \(N, k \in \N\).
  Then we have \(\forall 0 \leq k \leq N\),
  \begin{align*}
             & 0 \leq \abs{a_k} \leq \sum_{n = 0}^N \abs{a_n}                                                                                \\
    \implies & \lim_{N \to \infty} 0 \leq \lim_{N \to \infty} \abs{a_k} \leq \lim_{N \to \infty} \sum_{n = 0}^N \abs{a_n} &  & \by{i:6.4.13} \\
    \implies & 0 \leq \abs{a_k} \leq \sum_{n = 0}^\infty \abs{a_n} = 0                                                    &  & \by{i:7.2.2}  \\
    \implies & \abs{a_k} = 0 = a_k.
  \end{align*}
  Since \(N\) was arbitrary, we have \(a_k = 0\) for every \(k \geq 0\).
\end{proof}

\section{Rearrangement of series}\label{sec:7.4}

\begin{note}
  One feature of finite sums is that no matter how one rearranges the terms in a sequence, the total sum is the same.
  A more rigorous statement of this, involving bijections, has already appeared earlier, see \cref{7.1.12}.
\end{note}

\begin{prop}\label{7.4.1}
  Let \(\sum_{n = 0}^\infty a_n\) be a convergent series of non-negative real numbers, and let \(f : \N \to \N\) be a bijection.
  Then \(\sum_{m = 0}^\infty a_{f(m)}\) is also convergent, and has the same sum:
  \[
    \sum_{n = 0}^\infty a_n = \sum_{m = 0}^\infty a_{f(m)}.
  \]
\end{prop}

\begin{proof}
  We introduce the partial sums \(S_N \coloneqq \sum_{n = 0}^N a_n\) and \(T_M \coloneqq \sum_{m = 0}^M a_{f(m)}\).
  We know that the sequences \((S_N)_{N = 0}^\infty\) and \((T_M)_{M = 0}^\infty\) are increasing.
  Write \(L \coloneqq \sup(S_N)_{N = 0}^\infty\) and \(L' \coloneqq \sup(T_M)_{M = 0}^\infty\).
  By \cref{6.3.8} we know that \(L\) is finite, and in fact \(L = \sum_{n = 0}^\infty a_n\);
  by \cref{6.3.8} again we see that we will thus be done as soon as we can show that \(L' = L\).

  Fix \(M\), and let \(Y\) be the set \(Y \coloneqq \{m \in \N : m \leq M\}\).
  Note that \(f\) is a bijection between \(Y\) and \(f(Y)\).
  By \cref{7.1.11}, we have
  \[
    T_M = \sum_{m = 0}^M a_{f(m)} = \sum_{m \in Y} a_{f(m)} = \sum_{n \in f(Y)} a_n.
  \]
  The sequence \((f(m))_{m = 0}^M\) is finite, hence bounded, i.e., there exists an \(N\) such that \(f(m) \leq N\) for all \(m \leq M\).
  In particular \(f(Y)\) is a subset of \(\{n \in \N : n \leq N\}\), and so by \cref{7.1.11} again (and the assumption that all the \(a_n\) are non-negative)
  \[
    T_M = \sum_{n \in f(Y)} a_n \leq \sum_{n \in \{n \in \N : n \leq N\}} a_n = \sum_{n = 0}^N a_n = S_N.
  \]
  But since \((S_N)_{N = 0}^\infty\) has a supremum of \(L\), we thus see that \(S_N \leq L\), and hence that \(T_M \leq L\) for all \(M\).
  Since \(L'\) is the least upper bound of \((T_M)_{M = 0}^\infty\), this implies that \(L' \leq L\).

  Now we fix \(N\), and let \(X\) be the set \(X \coloneqq \{n \in \N : n \leq N\}\).
  Note that \(f^{-1}\) is a bijection between \(X\) and \(f^{-1}(X)\).
  By \cref{7.1.11}, we have
  \[
    S_N = \sum_{n = 0}^N a_n = \sum_{n \in X} a_n = \sum_{m \in f^{-1}(X)} a_{f(m)}.
  \]
  The sequence \((f^{-1}(n))_{n = 0}^N\) is finite, hence bounded, i.e., there exists an \(M\) such that \(f^{-1}(n) \leq M\) for all \(n \leq N\).
  In particular \(f^{-1}(X)\) is a subset of \(\{m \in \N : m \leq M\}\), and so by \cref{7.1.11} again (and the assumption that all the \(a_n\) are non-negative)
  \[
    S_N = \sum_{m \in f^{-1}(X)} a_{f(m)} \leq \sum_{m \in \{m \in \N : m \leq M\}} a_{f(m)} = \sum_{m = 0}^M a_{f(m)} = T_M.
  \]
  But since \((T_M)_{M = 0}^\infty\) has a supremum of \(L'\), we thus see that \(T_M \leq L'\), and hence that \(S_N \leq L'\) for all \(N\).
  Since \(L\) is the least upper bound of \((S_N)_{N = 0}^\infty\), this implies that \(L \leq L'\).

  Combining these two inequalities we obtain \(L = L'\), as desired.
\end{proof}

\setcounter{thm}{2}
\begin{prop}[Rearrangement of series]\label{7.4.3}
  Let \(\sum_{n = 0}^\infty a_n\) be an absolutely convergent series of real numbers, and let \(f : \N \to \N\) be a bijection.
  Then \(\sum_{m = 0}^\infty a_{f(m)}\) is also absolutely convergent, and has the same sum:
  \[
    \sum_{n = 0}^\infty a_n = \sum_{m = 0}^\infty a_{f(m)}.
  \]
\end{prop}

\begin{proof}
  We apply \cref{7.4.1} to the infinite series \(\sum_{n = 0}^\infty \abs{a_n}\), which by hypothesis is a convergent series of non-negative numbers.
  If we write \(L \coloneqq \sum_{n = 0}^\infty \abs{a_n}\), then by \cref{7.4.1} we know that \(\sum_{m = 0}^\infty \abs{a_{f(m)}}\) also converges to \(L\).

  Now write \(L' \coloneqq \sum_{n = 0}^\infty a_n\).
  We have to show that \(\sum_{m = 0}^\infty a_{f(m)}\) also converges to \(L'\).
  In other words, given any \(\varepsilon > 0\), we have to find an \(M\) such that \(\sum_{m = 0}^{M'} a_{f(m)}\) is \(\varepsilon\)-close to \(L'\) for every \(M' \geq M\).

  Since \(\sum_{n = 0}^\infty \abs{a_n}\) is convergent, we can use \cref{7.2.5} and find an \(N_1\) such that \(\sum_{n = p + 1}^q \abs{a_n} \leq \varepsilon / 2\) for all \(p, q \geq N_1\).
  Since \(\sum_{n = 0}^\infty a_n\) converges to \(L'\), the partial sums \(\sum_{n = 0}^N a_n\) also converge to \(L'\), and so there exists \(N \geq N_1\) such that \(\sum_{n = 0}^N a_n\) is \(\varepsilon / 2\)-close to \(L'\).

  Now the sequence \((f^{-1}(n))_{n = 0}^N\) is finite, hence bounded, so there exists an \(M\) such that \(f^{-1}(n) \leq M\) for all \(0 \leq n \leq N\).
  In particular, for any \(M' \geq M\), the set \(\{f(m) : m \in \N; m \leq M'\}\) contains \(\{n \in \N : n \leq N\}\).
  So by \cref{7.1.11}, for any \(M' \geq M\),
  \[
    \sum_{m = 0}^{M'} a_{f(m)} = \sum_{n \in \{f(m) : m \in \N; m \leq M'\}} a_n = \sum_{n = 0}^N a_n + \sum_{n \in X} a_n
  \]
  where \(X\) is the set
  \[
    X = \{f(m) : m \in \N; m \leq M'\} \setminus \{n \in \N : n \leq N\}.
  \]
  The set \(X\) is finite, and is therefore bounded by some natural number \(q\);
  we must therefore have
  \[
    X \subseteq \{n \in \N : N + 1 \leq n \leq q\}.
  \]
  Thus
  \[
    \abs{\sum_{m = 0}^{M'} a_{f(m)} - \sum_{n = 0}^N a_n} = \abs{\sum_{n \in X} a_n} \leq \sum_{n \in X} \abs{a_n} \leq \sum_{n = N + 1}^q \abs{a_n} \leq \varepsilon / 2
  \]
  by our choice of \(N\).
  Thus \(\sum_{m = 0}^{M'} a_{f(m)}\) is \(\varepsilon / 2\)-close to \(\sum_{n = 0}^N a_n\), which as mentioned before is \(\varepsilon / 2\)-close to \(L'\).
  Thus \(\sum_{m = 0}^{M'} a_{f(m)}\) is \(\varepsilon\)-close to \(L'\) for all \(M' \geq M\), as desired.
\end{proof}

\begin{note}
  There is in fact a surprising result of Riemann, which shows that a series which is conditionally convergent but not absolutely convergent can in fact be rearranged to converge to \emph{any} value
  (or rearranged to diverge).
\end{note}

\begin{note}
  To summarize, rearranging series is safe when the series is absolutely convergent, but is somewhat dangerous otherwise.
  (This is not to say that rearranging a series that is not absolutely convergent necessarily gives you the wrong answer
  - for instance, in theoretical physics one often performs similar maneuvres, and one still (usually) obtains a correct answer at the end
  - but doing so is risky, unless it is backed by a rigorous result such as \cref{7.4.3}.)
\end{note}

\exercisesection

\begin{ex}\label{ex:7.4.1}
  Let \(\sum_{n = 0}^\infty a_n\) be an absolutely convergent series of real numbers.
  Let \(f : \N \to \N\) be an increasing function (i.e., \(f(n + 1) > f(n)\) for all \(n \in \N\)).
  Show that \(\sum_{n = 0}^\infty a_{f(n)}\) is also an absolutely convergent series.
  What happens if we assume \(f\) is merely one-to-one, rather than increasing?
\end{ex}

\begin{proof}
  Since \(f\) is bijective implies \(f\) is one-to-one, we only need to proof the case for \(f\) being one-to-one.

  Let \(S_N = \sum_{n = 0}^N \abs{a_n}\) and \(T_N = \sum_{n = 0}^N \abs{a_{f(n)}}\).
  Since \(\sum_{n = 0}^\infty a_n\) is absolutely convergent and \(S_N\) is an increasing sequence, by \cref{6.3.8} we have \(\lim_{N \to \infty} S_N = \sup(S_N)_{N = 0}^\infty\).
  Since \((f(n))_{n = 0}^N\) is a finite sequence with \(N + 1\) unique elements (\(f\) is assumed to be one-to-one), it is bounded by some \(M \in \N\), thus \(\{f(n) : n \in \N \land n \leq N\} \subseteq \{n \in \N : n \leq M\}\).
  Now we have
  \[
    T_N = \sum_{n = 0}^N \abs{a_{f(n)}} \leq \sum_{n \in \N : n \leq M} \abs{a_n} = S_M \leq \sup(S_M)_{M = 0}^\infty = \lim_{M \to \infty} S_M,
  \]
  which means \(T_N\) is bounded.
  Since \((T_N)_{N = 0}^\infty\) is an increasing sequence and is bounded, by \cref{6.3.8} \((T_N)_{N = 0}^\infty\) converges, and thus \(\sum_{n = 0}^\infty a_{f(n)}\) is absolutely convergent.
\end{proof}

\begin{ex}\label{ex:7.4.2}
  Obtain an alternate proof of \cref{7.4.3} using \cref{7.4.1}, \cref{7.2.14}, and expressing \(a_n\) as the difference of \(a_n + \abs{a_n}\) and \(\abs{a_n}\).
  (This proof is due to Will Ballard.)
\end{ex}

\begin{proof}
  From hypothesis we know that \(\sum_{n = 0}^\infty \abs{a_n}\) converges, thus by \cref{7.2.9} we know that \(\sum_{n = 0}^\infty a_n\) converges.
  Since \(\sum_{n = 0}^\infty \abs{a_n}\) converges, by \cref{7.2.14}(b) we know that \(\sum_{n = 0}^\infty 2 \abs{a_n}\) converges and
  \begin{align*}
             & \forall n \geq 0, -\abs{a_n} \leq a_n \leq \abs{a_n}                                        \\
    \implies & 0 \leq a_n + \abs{a_n} \leq 2\abs{a_n}                                                      \\
    \implies & \sum_{n = 0}^\infty \big(a_n + \abs{a_n}\big) \text{ converges}. & \text{(by \cref{7.3.2})}
  \end{align*}
  Now we write \(a_n = a_n + \abs{a_n} - \abs{a_n}\).
  Since \(0 \leq a_n + \abs{a_n}\), we have
  \begin{align*}
             & \sum_{n = 0}^\infty (a_n + \abs{a_n}) \text{ and } \sum_{n = 0}^\infty \abs{a_n} \text{ converges}                                            \\
    \implies & \Bigg(\sum_{n = 0}^\infty (a_n + \abs{a_n}) = \sum_{m = 0}^\infty \big(a_{f(m)} + \abs{a_{f(m)}}\big)\Bigg)                                   \\
             & \land \Bigg(\sum_{n = 0}^\infty \abs{a_n} = \sum_{m = 0}^\infty \abs{a_{f(m)}}\Bigg)                        & \text{(by \cref{7.4.1})}        \\
    \implies & \sum_{n = 0}^\infty \big(a_n + \abs{a_n}\big) - \sum_{n = 0}^\infty \abs{a_n}                                                                 \\
             & = \sum_{m = 0}^\infty \big(a_{f(m)} + \abs{a_{f(m)}}\big) - \sum_{m = 0}^\infty \abs{a_{f(m)}}                                                \\
    \implies & \sum_{n = 0}^\infty a_n = \sum_{m = 0}^\infty a_{f(m)}.                                                     & \text{(by \cref{7.2.14}(a)(b))}
  \end{align*}
\end{proof}
\section{The root and ratio tests}\label{sec:7.5}

\begin{thm}[Root test]\label{7.5.1}
  Let \(\sum_{n = m}^\infty a_n\) be a series of real numbers, and let \(\alpha \coloneqq \limsup_{n \to \infty} \abs{a_n}^{1 / n}\).
  \begin{itemize}
    \item If \(\alpha < 1\), then the series \(\sum_{n = m}^\infty a_n\) is absolutely convergent
          (and hence conditionally convergent).
    \item If \(\alpha > 1\), then the series \(\sum_{n = m}^\infty a_n\) is not conditionally convergent
          (and hence cannot be absolutely convergent either).
    \item If \(\alpha = 1\), we cannot assert any conclusion.
  \end{itemize}
\end{thm}

\begin{proof}
  By \cref{7.2.14}(c), we may assume without loss of generality that \(m \geq 1\)
  (in particular \(\abs{a_n}^{1 / n}\) is well-defined for any \(n \geq m\)).

  First suppose that \(\alpha < 1\).
  Note that we must have \(\alpha \geq 0\), since by \cref{5.6.6}(c) \(\abs{a_n}^{1 / n} \geq 0\) for every \(n\).
  Then we can find an \(\varepsilon > 0\) such that \(0 < \alpha + \varepsilon < 1\) (for instance, we can set \(\varepsilon \coloneqq (1 - \alpha) / 2\)).
  By \cref{6.4.12}(a), there exists an \(N \geq m\) such that \(\abs{a_n}^{1 / n} \leq \alpha + \varepsilon\) for all \(n \geq N\).
  In other words, we have \(\abs{a_n} \leq (\alpha + \varepsilon)^n\) for all \(n \geq N\).
  But from the geometric series (\cref{7.3.3}) we have that \(\sum_{n = N}^\infty (\alpha + \varepsilon)^n\) is absolutely convergent, since \(0 < \alpha + \varepsilon < 1\)
  (note that the fact that we start from \(N\) is irrelevant by \cref{7.2.14}(c)).
  Thus by the comparison test (\cref{7.3.2}), we see that \(\sum_{n = N}^\infty a_n\) is absolutely convergent, and thus \(\sum_{n = m}^\infty a_n\) is absolutely convergent, by \cref{7.2.14}(c) again.

  Now suppose that \(\alpha > 1\).
  Then by \cref{6.4.12}(b), we see that for every \(N \geq m\) there exists an \(n \geq N\) such that \(\abs{a_n}^{1 / n} \geq 1\), and hence that \(\abs{a_n} \geq 1\).
  In particular, \((a_n)_{n = N}^\infty\) is not \(1\)-close to \(0\) for any \(N\), and hence \((a_n)_{n = m}^\infty\) is not eventually \(1\)-close to \(0\).
  In particular, \((a_n)_{n = m}^\infty\) does not converge to zero.
  Thus by the zero test (\cref{7.2.6}), \(\sum_{n = m}^\infty a_n\) is not conditionally convergent.

  For \(\alpha = 1\), we show two sequences \((a_n)_{n = m}^\infty\) and \((b_n)_{n = m}^\infty\) where
  \[
    \limsup_{n \to \infty} \abs{a_n}^{1 / n} = \limsup_{n \to \infty} \abs{b_n}^{1 / n} = 1
  \]
  and
  \[
    \limsup_{n \to \infty} \dfrac{\abs{a_{n + 1}}}{\abs{a_n}} = \limsup_{n \to \infty} \dfrac{\abs{b_{n + 1}}}{\abs{b_n}} = 1
  \]
  but \(\sum_{n = m}^\infty a_n\) converges and \(\sum_{n = m}^\infty b_n\) diverges.
  Let \(a_n = 1 / n\) and \(b_n = (-1)^n / n\).
  Then by \cref{7.3.7} \(\sum_{n = m}^\infty a_n\) diverges and by alternating series test (\cref{7.2.12}) \(\sum_{n = m}^\infty b_n\) converges.

  Since
  \begin{align*}
    \limsup_{n \to \infty} \abs{\dfrac{1}{n}}^{1 / n} & \leq \limsup_{n \to \infty} \dfrac{\abs{n}}{\abs{n + 1}}  &  & \by{7.5.2}                \\
                                                      & = \limsup_{n \to \infty} \bigg(1 - \dfrac{1}{n + 1}\bigg)                                \\
                                                      & = 1                                                       &  & \text{(by \cref{6.1.11})}
  \end{align*}
  and
  \begin{align*}
    \liminf_{n \to \infty} \abs{\dfrac{1}{n}}^{1 / n} & \geq \liminf_{n \to \infty} \dfrac{\abs{n}}{\abs{n + 1}}  &  & \by{7.5.2}                \\
                                                      & = \liminf_{n \to \infty} \bigg(1 - \dfrac{1}{n + 1}\bigg)                                \\
                                                      & = 1,                                                      &  & \text{(by \cref{6.1.11})}
  \end{align*}
  by \cref{6.4.12}(f) we have
  \[
    \limsup_{n \to \infty} \dfrac{\abs{n}}{\abs{n + 1}} = \liminf_{n \to \infty} \dfrac{\abs{n}}{\abs{n + 1}} = 1 = \lim_{n \to \infty} \dfrac{\abs{n}}{\abs{n + 1}}
  \]
  and
  \[
    \limsup_{n \to \infty} \abs{\dfrac{1}{n}}^{1 / n} = \liminf_{n \to \infty} \abs{\dfrac{1}{n}}^{1 / n} = 1 = \lim_{n \to \infty} \abs{\dfrac{1}{n}}^{1 / n}.
  \]
  Thus we have \(\limsup_{n \to \infty} \abs{a_n}^{1 / n} = \limsup_{n \to \infty} \dfrac{\abs{a_{n + 1}}}{\abs{a_n}} = 1\) but \(\sum_{n = m}^\infty a_n\) diverges.

  Since
  \begin{align*}
    \limsup_{n \to \infty} \abs{\dfrac{(-1)^n}{n}}^{1 / n} & \leq \limsup_{n \to \infty} \dfrac{\abs{(-1)^{n + 1} n}}{\abs{(-1)^n (n + 1)}} &  & \by{7.5.2}                \\
                                                           & = \limsup_{n \to \infty} \dfrac{n}{n + 1}                                                                     \\
                                                           & = 1                                                                            &  & \text{(by \cref{6.1.11})}
  \end{align*}
  and
  \begin{align*}
    \liminf_{n \to \infty} \abs{\dfrac{(-1)^n}{n}}^{1 / n} & \geq \liminf_{n \to \infty} \dfrac{\abs{(-1)^{n + 1} n}}{\abs{(-1)^n (n + 1)}} &  & \by{7.5.2}                \\
                                                           & = \liminf_{n \to \infty} \dfrac{n}{n + 1}                                                                     \\
                                                           & = 1,                                                                           &  & \text{(by \cref{6.1.11})}
  \end{align*}
  By \cref{6.4.12}(f) we have
  \[
    \limsup_{n \to \infty} \dfrac{\abs{(-1)^{n + 1} n}}{\abs{(-1)^n (n + 1)}} = \liminf_{n \to \infty} \dfrac{\abs{(-1)^{n + 1} n}}{\abs{(-1)^n (n + 1)}} = 1 = \lim_{n \to \infty} \dfrac{\abs{(-1)^{n + 1} n}}{\abs{(-1)^n (n + 1)}}
  \]
  and
  \[
    \limsup_{n \to \infty} \abs{\dfrac{(-1)^n}{n}}^{1 / n} = \liminf_{n \to \infty} \abs{\dfrac{(-1)^n}{n}}^{1 / n} = \lim_{n \to \infty} \abs{\dfrac{(-1)^n}{n}}^{1 / n} = 1.
  \]
  Thus we have \(\limsup_{n \to \infty} \abs{b_n}^{1 / n} = \limsup_{n \to \infty} \dfrac{\abs{b_{n + 1}}}{\abs{b_n}} = 1\) but \(\sum_{n = m}^\infty b_n\) converges.
  We conclude that when \(\limsup_{n \to \infty} \abs{a_n}^{1 / n} = \limsup_{n \to \infty} \dfrac{\abs{a_{n + 1}}}{\abs{a_n}} = 1\) we cannot assert any conclusion.
\end{proof}

\begin{note}
  The root test is phrased using the limit superior, but of course if \(\lim_{n \to \infty} \abs{a_n}^{1 / n}\) converges then the limit is the same as the limit superior.
  Thus one can phrase the root test using the limit instead of the limit superior, but \emph{only when the limit exists}.
\end{note}

\begin{lem}\label{7.5.2}
  Let \((c_n)_{n = m}^\infty\) be a sequence of positive numbers.
  Then we have
  \[
    \liminf_{n \to \infty} \dfrac{c_{n + 1}}{c_n} \leq \liminf_{n \to \infty} c_n^{1 / n} \leq \limsup_{n \to \infty} c_n^{1 / n} \leq \limsup_{n \to \infty} \dfrac{c_{n + 1}}{c_n}.
  \]
\end{lem}

\begin{proof}
  There are three inequalities to prove here.
  The middle inequality follows from \cref{6.4.12}(c).

  Next we show that \(\limsup_{n \to \infty} c_n^{1 / n} \leq \limsup_{n \to \infty} \dfrac{c_{n + 1}}{c_n}\).
  Write \(L \coloneqq \limsup_{n \to \infty} \dfrac{c_{n + 1}}{c_n}\).
  If \(L = +\infty\) then there is nothing to prove (since \(x \leq +\infty\) for every extended real number \(x\)), so we may assume that \(L\) is a finite real number.
  (Note that \(L\) cannot equal \(-\infty\)).
  Since \(\dfrac{c_{n + 1}}{c_n}\) is always positive, we know that \(L \geq 0\).

  Let \(\varepsilon > 0\).
  By \cref{6.4.12}(a), we know that there exists an \(N \geq m\) such that \(\dfrac{c_{n + 1}}{c_n} \leq L + \varepsilon\) for all \(n \geq N\)
  (without loss of generality we may assume that \(N \geq 1\)).
  This implies that \(c_{n + 1} \leq c_n (L + \varepsilon)\) for all \(n \geq N\).
  By induction this implies that
  \[
    c_n \leq c_N (L + \varepsilon)^{n - N} \text{ for all } n \geq N.
  \]
  If we write \(A \coloneqq c_N (L + \varepsilon)^{-N}\), then we have
  \[
    c_n \leq A(L + \varepsilon)^n
  \]
  and thus
  \[
    c_n^{1 / n} \leq A^{1 / n} (L + \varepsilon)
  \]
  for all \(n \geq N\).
  But we have
  \[
    \lim_{n \to \infty} A^{1 / n} (L + \varepsilon) = L + \varepsilon
  \]
  by the limit laws (\cref{6.1.19}) and \cref{6.5.3}.
  Thus by the comparison principle (\cref{6.4.13}) we have
  \[
    \limsup_{n \to \infty} c_n^{1 / n} \leq L + \varepsilon.
  \]
  But this is true for all \(\varepsilon > 0\), so this must imply that
  \[
    \limsup_{n \to \infty} c_n^{1 / n} \leq L.
  \]
  (If \(\limsup_{n \to \infty} c_n^{1 / n} > L\), then when \(\varepsilon = (\limsup_{n \to \infty} c_n^{1 / n} - L) / 2\) we have
  \[
    \limsup_{n \to \infty} c_n^{1 / n} \leq \dfrac{\limsup_{n \to \infty} c_n^{1 / n} + L}{2},
  \]
  a contradiction.), as desired.

  Finally we show that \(\liminf_{n \to \infty} \dfrac{c_{n + 1}}{c_n} \leq \liminf_{n \to \infty} c_n^{1 / n}\).
  Write \(L \coloneqq \liminf_{n \to \infty} \dfrac{c_{n + 1}}{c_n}\).
  Since \(\dfrac{c_{n + 1}}{c_n}\) is always positive, we know that \(L \geq 0\).

  Let \(\varepsilon > 0\).
  By \cref{6.4.12}(a), we know that there exists an \(N \geq m\) such that \(\dfrac{c_{n + 1}}{c_n} \geq L - \varepsilon\) for all \(n \geq N\)
  (without loss of generality we may assume that \(N \geq 1\)).
  This implies that \(c_{n + 1} \geq c_n (L - \varepsilon)\) for all \(n \geq N\).
  By induction this implies that
  \[
    c_n \geq c_N (L - \varepsilon)^{n - N} \text{ for all } n \geq N.
  \]
  If we write \(A \coloneqq c_N (L - \varepsilon)^{-N}\), then we have
  \[
    c_n \geq A(L - \varepsilon)^n
  \]
  and thus
  \[
    c_n^{1 / n} \geq A^{1 / n} (L - \varepsilon)
  \]
  for all \(n \geq N\).
  But we have
  \[
    \lim_{n \to \infty} A^{1 / n} (L - \varepsilon) = L - \varepsilon
  \]
  by the limit laws (\cref{6.1.19}) and \cref{6.5.3}.
  Thus by the comparison principle (\cref{6.4.13}) we have
  \[
    \liminf_{n \to \infty} c_n^{1 / n} \geq L - \varepsilon.
  \]
  But this is true for all \(\varepsilon > 0\), so this must imply that
  \[
    \liminf_{n \to \infty} c_n^{1 / n} \geq L.
  \]
  (If \(\liminf_{n \to \infty} c_n^{1 / n} < L\), then when \(\varepsilon = (L - \liminf_{n \to \infty} c_n^{1 / n}) / 2\) we have
  \[
    \liminf_{n \to \infty} c_n^{1 / n} \geq \dfrac{\liminf_{n \to \infty} c_n^{1 / n} + L}{2},
  \]
  a contradiction.), as desired.
\end{proof}

\begin{cor}[Ratio test]\label{7.5.3}
  Let \(\sum_{n = m}^\infty a_n\) be a series of non-zero numbers.
  (The non-zero hypothesis is required so that the ratios \(\abs{a_{n + 1}} / \abs{a_n}\) appearing below are well-defined.)
  \begin{itemize}
    \item If \(\limsup_{n \to \infty} \dfrac{\abs{a_{n + 1}}}{\abs{a_n}} < 1\), then the series \(\sum_{n = m}^\infty a_n\) is absolutely convergent (hence conditionally convergent).
    \item If \(\liminf_{n \to \infty} \dfrac{\abs{a_{n + 1}}}{\abs{a_n}} > 1\), then the series \(\sum_{n = m}^\infty a_n\) is not conditionally convergent (and thus cannot be absolutely convergent).
    \item In the remaining cases, we cannot assert any conclusion.
  \end{itemize}
\end{cor}

\begin{proof}
  We first show that if \(\limsup_{n \to \infty} \dfrac{\abs{a_{n + 1}}}{\abs{a_n}} < 1\), then the series \(\sum_{n = m}^\infty a_n\) is absolutely convergent.
  \begin{align*}
             & \limsup_{n \to \infty} \dfrac{\abs{a_{n + 1}}}{\abs{a_n}} < 1                                                               \\
    \implies & \limsup_{n \to \infty} \abs{a_n}^{1 / n} \leq \limsup_{n \to \infty} \dfrac{\abs{a_{n + 1}}}{\abs{a_n}} < 1 &  & \by{7.5.2} \\
    \implies & \sum_{n = m}^\infty a_n \text{ is absolutely convergent}.                                                   &  & \by{7.5.1}
  \end{align*}

  Next we show that if \(\liminf_{n \to \infty} \dfrac{\abs{a_{n + 1}}}{\abs{a_n}} > 1\), then the series \(\sum_{n = m}^\infty a_n\) is not conditionally convergent.
  \begin{align*}
             & \liminf_{n \to \infty} \dfrac{\abs{a_{n + 1}}}{\abs{a_n}} > 1                                                               \\
    \implies & \limsup_{n \to \infty} \abs{a_n}^{1 / n} \geq \liminf_{n \to \infty} \dfrac{\abs{a_{n + 1}}}{\abs{a_n}} > 1 &  & \by{7.5.2} \\
    \implies & \sum_{n = m}^\infty a_n \text{ is not conditionally convergent}.                                            &  & \by{7.5.1}
  \end{align*}

  Finally we show that if \(\liminf_{n \to \infty} \dfrac{\abs{a_{n + 1}}}{\abs{a_n}} \leq 1\) or \(\limsup_{n \to \infty} \dfrac{\abs{a_{n + 1}}}{\abs{a_n}} \geq 1\), then we cannot assert any conclusion.
  See \cref{7.5.1}.
\end{proof}

\begin{prop}\label{7.5.4}
  We have \(\lim_{n \to \infty} n^{1 / n} = 1\).
\end{prop}

\begin{proof}
  By \cref{7.5.2} we have
  \[
    \limsup_{n \to \infty} n^{1 / n} \leq \limsup_{n \to \infty} (n + 1) / n = \limsup_{n \to \infty} 1 + 1 / n = 1
  \]
  by \cref{6.1.11} and limit laws (\cref{6.1.19}).
  Similarly we have
  \[
    \liminf_{n \to \infty} n^{1 / n} \geq \liminf_{n \to \infty} (n + 1) / n = \liminf_{n \to \infty} 1 + 1 / n = 1.
  \]
  The claim then follows from \cref{6.4.12}(c) and (f).
\end{proof}

\begin{rmk}\label{7.5.5}
  In addition to the ratio and root tests, another very useful convergence test is the \emph{integral test}, which we will cover in \cref{11.6.4}.
\end{rmk}

\exercisesection

\begin{ex}\label{ex:7.5.1}
  Prove the first inequality in \cref{7.5.2}.
\end{ex}

\begin{proof}
  See \cref{7.5.2}.
\end{proof}

\begin{ex}\label{ex:7.5.2}
  Let \(x\) be a real number with \(\abs{x} < 1\), and \(q\) be a real number.
  Show that the series \(\sum_{n = 1}^\infty n^q x^n\) is absolutely convergent, and that \(\lim_{n \to \infty} n^q x^n = 0\).
\end{ex}

\begin{proof}
  Let \(N \in \N\).
  Since \(q \in \R\), by \cref{5.4.12} \(\exists\ N \geq q\).
  Then we have
  \begin{align*}
             & \forall n \geq 1, \abs{n^q x^n} \leq \abs{n^N x^n}                                         &  & \by{5.6.9}                \\
    \implies & \forall n \geq 1, \abs{n^q x^n}^{1 / n} \leq \abs{n^N x^n}^{1 / n}                         &  & \by{5.6.9}                \\
    \implies & \forall n \geq 1, n^{q / n} \abs{x} \leq n^{N / n} \abs{x}                                 &  & \by{5.6.3}                \\
    \implies & \limsup_{n \to \infty} n^{q / n} \abs{x} \leq \limsup_{n \to \infty} n^{N / n} \abs{x}     &  & \text{(by \cref{6.4.13})} \\
    \implies & \limsup_{n \to \infty} n^{q / n} \abs{x} \leq (\limsup_{n \to \infty} n^{1 / n})^N \abs{x} &  & \text{(by \cref{6.1.19})} \\
    \implies & \limsup_{n \to \infty} n^{q / n} \abs{x} \leq 1^N \abs{x}                                  &  & \by{7.5.4}                \\
    \implies & \limsup_{n \to \infty} n^{q / n} \abs{x} \leq \abs{x} < 1                                  &  & \text{(by hypothesis)}    \\
    \implies & \sum_{n = 1}^\infty n^q x^n \text{ is absolutely convergent}                               &  & \by{7.5.1}                \\
    \implies & \sum_{n = 1}^\infty n^q x^n \text{ converges}                                              &  & \by{7.2.9}                \\
    \implies & \lim_{n \to \infty} n^q x^n = 0                                                            &  & \by{7.2.6}
  \end{align*}
\end{proof}

\begin{ex}\label{ex:7.5.3}
  Give an example of a divergent series \(\sum_{n = 1}^\infty a_n\) of positive numbers \(a_n\) such that \(\lim_{n \to \infty} a_{n + 1} / a_n = \lim_{n \to \infty} a_n^{1 / n} = 1\), and give an example of a convergent series \(\sum_{n = 1}^\infty b_n\) of positive numbers \(b_n\) such that \(\lim_{n \to \infty} b_{n + 1} / b_n = \lim_{n \to \infty} b_n^{1 / n} = 1\).
  This shows that the ratio and root tests can be inconclusive even when the summands are positive and all the limits converge.
\end{ex}

\begin{proof}
  See \cref{7.5.1}.
\end{proof}

\chapter{Infinite sets}\label{i:ch:8}

\section{Countability}\label{i:sec:8.1}

\begin{note}
  From \cref{i:3.6.12} we know that the set \(\N\) of natural numbers is infinite.
  The set \(\N - \set{0}\) is also infinite, thanks to \cref{i:3.6.14}(a), and is a proper subset of \(\N\).
  However, the set \(\N - \set{0}\), despite being ``smaller'' than \(\N\), still has the same cardinality as \(\N\), because the function \(f : \N \to \N - \set{0}\) defined by \(f(n) \coloneqq n + 1\), is a bijection from \(\N\) to \(\N - \set{0}\).
  This is one characteristic of infinite sets.
\end{note}

\begin{defn}[Countable sets]\label{i:8.1.1}
  A set \(X\) is said to be \emph{countably infinite} (or just \emph{countable}) iff it has equal cardinality with the natural numbers \(\N\).
  A set \(X\) is said to be \emph{at most countable} iff it is either countable or finite.
  We say that a set is \emph{uncountable} if it is infinite but not countable.
\end{defn}

\begin{rmk}\label{i:8.1.2}
  Countably infinite sets are also called \emph{denumerable} sets.
\end{rmk}

\begin{eg}\label{i:8.1.3}
  The even natural numbers \(\set{2n : n \in \N}\), since the function \(f(n) \coloneqq 2n\) provides a bijection between \(\N\) and the even natural numbers.
\end{eg}

\begin{note}
  Let \(X\) be a countable set.
  Then, by definition, we know that there exists a bijection \(f : \N \to X\).
  Thus, every element of \(X\) can be written in the form \(f(n)\) for exactly one natural number \(n\).
  Informally, we thus have
  \[
    X = \set{f(0), f(1), f(2), f(3), \dots}.
  \]
  Thus, a countable set can be arranged in a sequence, so that we have a zeroth element \(f(0)\), followed by a first element \(f(1)\), then a second element \(f(2)\), and so forth, in such a way that all these elements \(f(0), f(1), f(2), \dots\) are all distinct, and together they fill out all of \(X\).
  (This is why these sets are called \emph{countable};
  because we can literally count them one by one, starting from \(f(0)\), then \(f(1)\), and so forth.)
\end{note}

\begin{prop}[Well ordering principle]\label{i:8.1.4}
  Let \(X\) be a non-empty subset of the natural numbers \(\N\).
  Then there exists exactly one element \(n \in X\) such that \(n \leq m\) for all \(m \in X\).
  In other words, every non-empty set of natural numbers has a minimum element.
\end{prop}

\begin{proof}
  Suppose for sake of contradiction that \(X\) has no minimum element.
  Let \(n \in \N\) and let \(P(n)\) be the statement ``\(\forall m \in X\), we have \(n \leq m\) and \(n \notin X\).''
  We now use induction to show that \(P(n)\) is true \(\forall n \in \N\).
  For \(n = 0\), we have
  \begin{align*}
             & X \subseteq \N                                                       \\
    \implies & \forall m \in X, m \in \N &  & \by{i:3.1.15}                         \\
    \implies & \forall m \in X, 0 \leq m &  & \by{i:2.3}                            \\
    \implies & 0 \notin X.               &  & \text{(\(X\) has no minimum element)}
  \end{align*}
  Thus, the base case holds.
  Suppose inductively that \(P(n)\) is true for some \(n \geq 0\).
  Then for \(n + 1\), we have
  \begin{align*}
             & \forall m \in X, n \leq m \land n \notin X &  & \byIH                                 \\
    \implies & \forall m \in X, n < m                     &  & \by{i:2.2.11}                         \\
    \implies & \forall m \in X, n + 1 \leq m              &  & \by{i:2.2.12}[e]                      \\
    \implies & n + 1 \notin X.                            &  & \text{(\(X\) has no minimum element)}
  \end{align*}
  This closes the induction.

  By hypothesis we know that \(X \subseteq \N\) and \(X \neq \emptyset\).
  So let \(n \in X\).
  But \(P(n)\) is true, we must have \(n \notin X\), a contradiction.
  Thus \(X\) must have a minimum element \(\min(X) \in X\).

  Now we show that such \(\min(X)\) is unique.
  Suppose that \(\exists n, n' \in X\) such that \(\forall m \in X\), we have \(n \leq m \land n' \leq m\).
  Since \(n, n' \in X\), we have \(n \leq n' \land n' \leq n\).
  Thus \(n = n'\).
\end{proof}

\begin{note}
  We will refer to the element \(n\) given by the well-ordering principle as the \emph{minimum} of \(X\), and write it as \(\min(X)\).
  This minimum is clearly the same as the infimum of \(X\), as defined in \cref{i:5.5.10}.
\end{note}

\begin{prop}\label{i:8.1.5}
  Let \(X\) be an infinite subset of the natural numbers \(\N\).
  Then there exists a unique bijection \(f : \N \to X\) which is increasing, in the sense that \(f(n + 1) > f(n)\) for all \(n \in N\).
  In particular, \(X\) has equal cardinality with \(\N\) and is hence countable.
\end{prop}

\begin{proof}
  We now define a sequence \(a_0, a_1, a_2, \dots\) of natural numbers recursively by the formula
  \[
    a_n \coloneqq \min\set{x \in X : x \neq a_m \text{ for all } m < n}.
  \]
  Intuitively speaking, \(a_0\) is the smallest element of \(X\);
  \(a_1\) is the second smallest element of \(X\), i.e., the smallest element of \(X\) once \(a_0\) is removed;
  \(a_2\) is the third smallest element of \(X\);
  and so forth.
  Observe that in order to define \(a_n\), one only needs to know the values of \(a_m\) for all \(m < n\), so this definition is recursive.
  Also, since \(X\) is infinite, the set \(\set{x \in X : x \neq a_m \text{ for all } m < n}\) is infinite, hence non-empty.
  (If it is finite, then its union with the set \(\set{a_0, \dots, a_{n - 1}}\) is also finite, but the union is \(X\), which contradict to \(X\) is infinite.)
  Thus by the well-ordering principle (\cref{i:8.1.5}), the minimum, \(\min\set{x \in X : x \neq a_m \text{ for all } m < n}\) is always well-defined.

  Since \(a_{n + 1} = \min\set{x \in X : x \neq a_m \text{ for all } m < n + 1}\), we know that \(a_n < a_{n + 1}\).
  Since \(n\) was arbitrary, we see that \(a_n\) is an increasing sequence, i.e.
  \[
    a_0 < a_1 < a_2 < \dots
  \]
  and in particular that \(a_n \neq a_m\) for all \(n \neq m\).
  Also, we have \(a_n \in X\) for each natural number \(n\) (by \cref{i:8.1.4}).

  Now define the function \(f : \N \to X\) by \(f(n) \coloneqq a_n\).
  From the previous paragraph we know that \(f\) is one-to-one.
  Now we show that \(f\) is onto.
  In other words, we claim that for every \(y \in X\), there exists an \(n\) such that \(a_n = y\).

  Let \(y \in X\).
  Suppose for sake of contradiction that \(a_n \neq y\) for every natural number \(n\).
  Then this implies that \(y\) is an element of the set \(\set{x \in X : x \neq a_m \text{ for all } m < n}\) for all \(n\).
  By definition of \(a_n\), this implies that \(y > a_n\) for every natural number \(n\).
  (If \(y < a_n\), then \(y = \min\set{x \in X : a \neq a_m \text{ for all } m < n}\) instead of \(a_n\), a contradiction)
  However, since \(a_n\) is an increasing sequence, we have \(a_n \geq n\), and hence \(y \geq n\) for every natural number \(n\).
  In particular we have \(y \geq y + 1\), which is a contradiction.
  Thus we must have \(a_n = y\) for some natural number \(n\), and hence \(f\) is onto.

  Since \(f : \N \to X\) is both one-to-one and onto, it is a bijection.
  We have thus found at least one increasing bijection \(f\) from \(\N\) to \(X\).
  Now suppose for sake of contradiction that there was at least one other increasing bijection \(g\) from \(\N\) to \(X\) which was not equal to \(f\).
  Then the set \(\set{n \in \N : g(n) \neq f(n)}\) is non-empty, and define \(m \coloneqq \min\set{n \in \N : g(n) \neq f(n)}\), thus in particular \(g(m) \neq f(m) = a_m\), and \(g(n) = f(n) = a_n\) for all \(n < m\).
  But we then must have
  \[
    g(m) = \min\set{x \in X : x \neq a_t \text{ for all } t < m} = a_m,
  \]
  a contradiction.
  Thus there is no other increasing bijection from \(\N\) to \(X\) other than \(f\).
\end{proof}

\begin{cor}\label{i:8.1.6}
  All subsets of the natural numbers are at most countable.
\end{cor}

\begin{proof}
  Since finite sets are at most countable by definition, combine with \cref{i:8.1.5} we thus have all subsets of the natural numbers are at most countable.
\end{proof}

\begin{cor}\label{i:8.1.7}
  If \(X\) is an at most countable set, and \(Y\) is a subset of \(X\), then \(Y\) is at most countable.
\end{cor}

\begin{proof}
  If \(X\) is finite then this follows from \cref{i:3.6.14}(c), so assume \(X\) is countable.
  Then there is a bijection \(f : X \to \N\) between \(X\) and \(\N\).
  Since \(Y\) is a subset of \(X\), and \(f\) is a bijection from \(X\) and \(\N\), then when we restrict \(f\) to \(Y\), we obtain a bijection between \(Y\) and \(f(Y)\).
  Thus \(f(Y)\) has equal cardinality with \(Y\).
  But \(f(Y)\) is a subset of \(\N\), and hence at most countable by \cref{i:8.1.6}.
  Hence \(Y\) is also at most countable.
\end{proof}

\begin{prop}\label{i:8.1.8}
  Let \(Y\) be a set, and let \(f : \N \to Y\) be a function.
  Then \(f(\N)\) is at most countable.
\end{prop}

\begin{proof}
  If \(f(\N)\) is finite then by \cref{i:8.1.1} it is at most countable.
  So assume that \(f(\N)\) is infinite.
  Let \(A\) be the set
  \[
    A = \set{n \in \N : f(m) \neq f(n) \text{ for all } 0 \leq m < n}.
  \]
  So \(A \subseteq \N\) and \(A\) is infinite.
  We now show that \(f|_A : A \to f(A)\) is a bijection.

  Let \(p, q \in A\) and \(p \neq q\).
  By the definition of \(A\) we know that \(f|_A(p) \neq f|_A(q)\) and thus \(f|_A\) is injective.
  By \cref{i:3.4.1} we also know that \(f|_A\) is surjective, thus \(f|_A\) is bijective.

  Now we show that \(\forall y \in f(\N)\), \(\exists p \in A\) such that \(f|_A(p) = y\).
  Suppose for sake of contradiction that \(\nexists p \in A\) such that \(f|_A(p) = y\).
  Then we have \(y \neq f|_A(p)\) for every \(p \in A\).
  Since \(y \in f(\N)\), we know that \(\exists q \in \N\) such that \(f(q) = y\) and \(q \notin A\).
  Since \(q \notin A\), by the definition of \(A\) we know that \(\exists 0 \leq m < q\) such that \(f(m) = f(q) = y\).
  Now we let \(E\) be the set
  \[
    E = \set{m \in \N : f(m) = f(q) = y}.
  \]
  Since \(E \subseteq \N\) and \(E \neq \emptyset\), by well ordering principle (\cref{i:8.1.4}) we know that \(\min(E)\) exists.
  This means \(\exists p \in E\) such that \(\forall 0 \leq m < p\), we have \(f(m) \neq f(p) = f(q)\).
  But then we must have \(p \in A\), a contradiction.
  Thus \(\forall y \in f(\N)\), \(\exists p \in A\) such that \(f|_A(p) = y\).
  This means \(f(\N) \subseteq f(A)\), thus we have \(f(\N) = f(A)\).

  Since \(A \subseteq \N\) and \(A\) is infinite, by \cref{i:8.1.5} \(\exists g : \N \to A\) where \(g\) is bijective.
  This means \(f|_A \circ g\) is bijective and we have
  \[
    (f|_A \circ g)(\N) = f|_A\big(g(\N)\big) = f|_A(A) = f(A) = f(\N).
  \]
  Thus by \cref{i:8.1.1} \(f(\N)\) is countable, and thus at most countable.
\end{proof}

\begin{cor}\label{i:8.1.9}
  Let \(X\) be a countable set, and let \(f : X \to Y\) be a function.
  Then \(f(X)\) is at most countable.
\end{cor}

\begin{proof}
  By \cref{i:8.1.1} \(\exists g : \N \to X\) such that \(g\) is a bijection.
  Then we have \(f \circ g : \N \to Y\) and by \cref{i:8.1.8} \((f \circ g)(\N)\) is at most countable.
  But
  \[
    (f \circ g)(\N) = f(g(\N)) = f(X).
  \]
  Thus \(f(X)\) is at most countable.
\end{proof}

\begin{prop}\label{i:8.1.10}
  Let \(X\) be a countable set, and let \(Y\) be a countable set.
  Then \(X \cup Y\) is a countable set.
\end{prop}

\begin{proof}
  By \cref{i:8.1.1} \(\exists f : \N \to X\) and \(g : \N \to Y\) such that \(f\) and \(g\) are bijections.
  Let \(h : \N \to X \cup Y\) by setting \(h(2n) = f(n)\) and \(h(2n + 1) = g(n)\) for every natural number \(n\).
  We now show that \(h(\N) = X \cup Y\).
  \begin{align*}
         & z \in h(\N)                                                            \\
    \iff & \exists k \in \N : h(k) = z                                            \\
    \iff & (\exists k \in \N : h(k) = z)                                          \\
         & \land (\exists n \in \N : k = 2n \lor k = 2n + 1) &  & \by{i:ac:4.4.2} \\
    \iff & \exists n \in \N : z = h(2n) \lor z = h(2n + 1)                        \\
    \iff & z = f(n) \lor z = g(n)                                                 \\
    \iff & z \in X \lor z \in Y                                                   \\
    \iff & z \in X \cup Y.
  \end{align*}
  Then by \cref{i:8.1.9} we have \(h(\N) = X \cup Y\) is at most countable.
  But since \(X\) and \(Y\) are infinite sets, \(X \cup Y\) can not be finite, thus \(X \cup Y\) is countable.
\end{proof}

\begin{note}
  To summarize, any subset or image of a countable set is at most countable, and any finite union of countable sets is still countable.
\end{note}

\begin{ac}\label{i:ac:8.1.1}
  Let \(X, Y\) be at most countable sets.
  Then \(X \cup Y\) is at most countable.
\end{ac}

\begin{proof}
  We split into following three cases:
  \begin{itemize}
    \item \(X, Y\) are countable.
          Then by \cref{i:8.1.10} we know that \(X \cup Y\) is countable, thus at most countable.
    \item \(X, Y\) are finite.
          Then by \cref{i:3.6.14}(b) we know that \(X \cup Y\) is finite, thus at most countable.
    \item \(X, Y\) consist of one finite set and one countable set.
          Without the loss of generality, suppose that \(X\) is finite and \(Y\) is countable.
          Since \(X\) is finite, there exists a function \(f : \set{i \in \N : 1 \leq i \leq \#(X)} \to X\) such that \(f\) is bijective.
          Since \(Y\) is countable, by \cref{i:8.1.1} there exists a function \(g : \N \to Y\) such that \(g\) is bijective.
          Now we define a functiton \(h : \N \to X \cup Y\) as follow:
          \[
            \forall n \in \N, h(n) = \begin{dcases}
              f(n + 1)     & \text{ if } n < \#(X)    \\
              g(n - \#(X)) & \text{ if } n \geq \#(X)
            \end{dcases}
          \]
          We need to show that \(h(\N) = X \cup Y\).
          Since \(h(\N) \subseteq X \cup Y\), it suffices to show that \(X \cup Y \subseteq h(\N)\).
          \begin{align*}
                     & \forall z \in X \cup Y                                                  \\
            \implies & (z \in X) \lor (z \in Y)                                                \\
            \implies & \big(\exists n \in \set{i \in \N : 1 \leq i \leq \#(X)} : f(n) = z\big) \\
                     & \land \big(\exists n \in \N : g(n) = z\big)                             \\
            \implies & \big(h(n - 1) = f(n) = z\big) \land \big(h(n + \#(X)) = g(n) = z\big)   \\
            \implies & z \in h(\N),
          \end{align*}
          Thus we have \(X \cup Y \subseteq h(\N)\).
          By \cref{i:8.1.8} \(X \cup Y\) is at most countable.
  \end{itemize}
  From all cases above we conclude that \(X \cup Y\) is at most countable.
\end{proof}

\begin{cor}\label{i:8.1.11}
  The integers \(\Z\) are countable.
\end{cor}

\begin{proof}
  We already know that the set \(\N = \set{0, 1, 2, 3, \dots}\) of natural numbers are countable.
  The set \(-\N\) defined by
  \[
    -\N \coloneqq \set{-n : n \in \N} = \set{0, -1, -2, -3, \dots}
  \]
  is also countable, since the map \(f(n) \coloneqq -n\) is a bijection between \(\N\) and this set.
  Since the integers are the union of \(\N\) and \(-\N\), the claim follows from \cref{i:8.1.10}.
\end{proof}

\begin{note}
  To establish countability of the rationals, we need to relate countability with Cartesian products.
  In particular, we need to show that the set \(\N \times \N\) is countable.
\end{note}

\begin{lem}\label{i:8.1.12}
  The set
  \[
    A \coloneqq \set{(n, m) \in \N \times \N : 0 \leq m \leq n}
  \]
  is countable.
\end{lem}

\begin{proof}
  Define the sequence \(a_0, a_1, a_2, \dots\) recursively by setting \(a_0 \coloneqq 0\), and \(a_{n + 1} \coloneqq a_n + n + 1\) for all natural numbers \(n\).
  Thus
  \[
    a_0 = 0; a_1 = 0 + 1; a_2 = 0 + 1 + 2; a_3 = 0 + 1 + 2 + 3; \dots
  \]
  By induction one can show that \(a_n\) is increasing, i.e., that \(a_n > a_m\) whenever \(n > m\).

  Now define the function \(f : A \to \N\) by
  \[
    f(n, m) \coloneqq a_n + m.
  \]
  We claim that \(f\) is one-to-one.
  In other words, if \((n, m)\) and \((n', m')\) are any two distinct elements of \(A\), then we claim that \(f(n, m) \neq f(n', m')\).

  To prove this claim, let \((n, m)\) and \((n', m')\) be two distinct elements of \(A\).
  There are three cases: \(n' = n\), \(n' > n\), and \(n' < n\).
  First suppose that \(n' = n\).
  Then we must have \(m \neq m'\), otherwise \((n, m)\) and \((n', m')\) would not be distinct.
  Thus \(a_n + m \neq a_n + m'\), and hence \(f(n, m) \neq f(n', m')\), as desired.

  Now suppose that \(n' > n\).
  Then \(n' \geq n + 1\), and hence
  \[
    f(n', m') = a_{n'} + m' \geq a_{n'} \geq a_{n + 1} = a_n + n + 1.
  \]
  But since \((n, m) \in A\), we have \(m \leq n < n + 1\), and hence
  \[
    f(n', m') \geq a_n + n + 1 > a_n + m = f(n, m),
  \]
  and thus \(f(n', m') \neq f(n, m)\).

  The case \(n' < n\) is proven similarly, by switching the roles of \(n\) and \(n'\) in the previous argument.
  Thus we have shown that \(f\) is one-to-one.
  Thus \(f\) is a bijection from \(A\) to \(f(A)\), and so \(A\) has equal cardinality with \(f(A)\).
  But \(f(A)\) is a subset of \(\N\), and hence by \cref{i:8.1.6} \(f(A)\) is at most countable.
  Therefore \(A\) is at most countable.
  But, \(A\) is clearly not finite.
  (if \(A\) was finite, then every subset of \(A\) would be finite, and in particular \(\set{(n, 0) : n \in \N}\) would be finite, but this is clearly countably infinite, a contradiction.)
  Thus, \(A\) must be countable.
\end{proof}

\begin{cor}\label{i:8.1.13}
  The set \(\N \times \N\) is countable.
\end{cor}

\begin{proof}
  We already know that the set
  \[
    A \coloneqq \set{(n, m) \in \N \times \N : 0 \leq m \leq n}
  \]
  is countable.
  This implies that the set
  \[
    B \coloneqq \set{(n, m) \in \N \times \N : 0 \leq n \leq m}
  \]
  is also countable, since the map \(f : A \to B\) given by \(f(n, m) \coloneqq (m, n)\) is a bijection from \(A\) to \(B\).
  We prove \(f\) is bijective by showing that \(f\) is both injective and surjective.
  \begin{itemize}
    \item To prove that \(f\) is injective, suppose that \((n, m), (n', m') \in A\) and \(f(n, m) = f(n', m')\).
          Then we have
          \begin{align*}
                     & f(n, m) = f(n', m')                                          \\
            \implies & (m, n) = (m', n')   &  & \text{(by the definition of \(f\))} \\
            \implies & n = n' \land m = m' &  & \by{i:3.5.1}                        \\
            \implies & (n, m) = (n', m').  &  & \by{i:3.5.1}
          \end{align*}
          Thus \(f\) is injective.
    \item Since \(\forall (n, m) \in B\), we have \(n \leq m\), thus \((m, n) \in A\) and \(f(m, n) = (n, m)\).
          So \(f\) is surjective.
  \end{itemize}

  We now show that \(\N \times \N = A \cup B\).
  By \cref{i:3.1.18} we need to show that \(\N \times \N \subseteq A \cup B\) and \(A \cup B \subseteq \N \times \N\).
  It is clearly that \(A \cup B \subseteq \N \times \N\).
  So we only need to show that \(\N \times \N \subseteq A \cup B\).
  \begin{align*}
             & \forall (a, b) \in \N \times \N                                                      \\
    \implies & (a < b) \lor (a = b) \lor (a > b) &  & \by{i:2.2.13}                                 \\
    \implies & (a, b) \in A \lor (a, b) \in B    &  & \text{(by the definition of \(A\) and \(B\))} \\
    \implies & (a, b) \in A \cup B.              &  & \by{i:3.4}
  \end{align*}
  Thus by \cref{i:3.1.15} we have \(\N \times \N \subseteq A \cup B\).

  Since \(\N \times \N\) is the union of \(A\) and \(B\), the claim then follows from \cref{i:8.1.10}.
\end{proof}

\begin{cor}\label{i:8.1.14}
  If \(X\) and \(Y\) are countable, then \(X \times Y\) is countable.
\end{cor}

\begin{proof}
  By \cref{i:8.1.1} \(\exists f : \N \to X\) and \(g : \N \to Y\) such that \(f\) and \(g\) are bijections.
  Let \(h : \N \times \N \to X \times Y\) by setting \(h(x, y) = (f(x), g(y))\).
  If \(n, n', m, m' \in \N\) and \((n, m) \neq (n', m')\), then
  \[
    h(n, m) = (f(n), g(m)) \neq (f(n'), g(m')) = h(n', m')
  \]
  since \(f, g\) are bijections, so \(h\) is injective.
  Again since \(f, g\) are bijections, \(\forall x \in X \land \forall y \in Y\), \(\exists n, m \in \N\) such that \(x = f(n) \land y = g(m)\).
  So \(h\) is surjective, and thus \(h\) is bijective.

  Since \(h\) is bijective, \(\N \times \N\) and \(X \times Y\) has the same cardinality.
  But by \cref{i:8.1.13} we know that \(\N \times \N\) is countable.
  Thus by \cref{i:8.1.1} \(X \times Y\) is countable.
\end{proof}

\begin{cor}\label{i:8.1.15}
  The rationals \(\Q\) are countable.
\end{cor}

\begin{proof}
  We already know that the integers \(\Z\) are countable, which implies that the non-zero integers \(\Z - \set{0}\) are countable.
  (since \(\Z - \set{0} \subseteq \Z\), by \cref{i:8.1.7} we know that \(\Z - \set{0}\) is at most countable, and clearly \(\Z - \set{0}\) is not finite.)
  By \cref{i:8.1.14}, the set
  \[
    \Z \times (\Z - \set{0}) = \set{(a, b) : a, b \in \Z, b \neq 0}
  \]
  is thus countable.
  If one lets \(f : \Z \times (\Z - \set{0}) \to \Q\) be the function \(f(a, b) \coloneqq a / b\)
  (note that \(f\) is well-defined since we prohibit \(b\) from being equal to \(0\)), we see from \cref{i:8.1.9} that \(f(\Z \times (\Z - \set{0}))\) is at most countable.
  But we have \(f(\Z \times (\Z - \set{0})) = \Q\)
  (This is basically the definition of the rationals \(\Q\)).
  Thus \(\Q\) is at most countable.
  However, \(\Q\) cannot be finite, since it contains the infinite set \(\N\).
  Thus \(\Q\) is countable.
\end{proof}

\begin{rmk}\label{i:8.1.16}
  Because the rationals are countable, we know \emph{in principle} that it is possible to arrange the rational numbers as a sequence:
  \[
    \Q = \set{a_0, a_1, a_2, a_3, \dots}
  \]
  such that every element of the sequence is different from every other element, and that the elements of the sequence exhaust \(\Q\)
  (i.e., every rational number turns up as one of the elements \(a_n\) of the sequence).
  However, it is quite difficult (though not impossible) to actually try and come up with an explicit sequence \(a_0, a_1, \dots\) which does this.
\end{rmk}

\exercisesection

\begin{ex}[Dedekind-infinite set]\label{i:ex:8.1.1}
  Let \(X\) be a set.
  Show that \(X\) is infinite iff there exists a proper subset \(Y \subsetneq X\) of \(X\) which has the same cardinality as \(X\).
\end{ex}

\begin{proof}
  We first show that \(X\) is infinite implies \(\exists Y \subsetneq X\) such that \(Y\) has the same cardinality as \(X\).
  Suppose that \(X\) is an infinite set.
  Then we have \(X \neq \emptyset\) since by \cref{i:ex:3.6.2} \(\#(\emptyset) = 0\).

  Let \(n \in \N\) and let \(P(n)\) be the statement ``\(\exists A_n \subseteq X\) such that \(\#(A_n) = n\).''
  We induct on \(n\) to show that \(\forall n \in \N\), \(P(n)\) is true.
  For \(n = 0\), we have \(\emptyset \subseteq X\) by \cref{i:3.1.17}, Thus, the base case holds.
  Suppose inductively that \(P(n)\) is true for some \(n \geq 0\).
  Then we need to show that \(P(n + 1)\) is true.
  By induction hypothesis, \(\exists A_n \subseteq X : \#(A_n) = n\).
  Then by \cref{i:3.6.14}(b) we know that \(X \setminus A_n\) is infinite.
  Since \(X \setminus A_n\) is infinite, we know that \(X \setminus A_n \neq \emptyset\).
  Let \(x \in X \setminus A_n\).
  Then we define \(A_{n + 1} = A_n \cup \set{x}\), and this closes the induction.

  By axiom of choice (\cref{i:8.1}) the set \(\prod_{n \in \Z^+} A_n\) is non-empty since \(\forall n \in \Z^+\), \(P(n)\) is true.
  We can now choose an element \((x_n)_{n \in \Z^+}\) from \(\prod_{n \in \Z^+} A_n\).
  In particular, we want to choose a \((x_n)_{n \in \Z^+}\) where \(x_i \neq x_j\) for every \(i, j \in \Z^+\) and \(i \neq j\).
  This can be done since \(\#(A_i) \neq \#(A_j)\) for every \(i, j \in \Z^+\) and \(i \neq j\).
  We collect \(x_i\) as a set \(A = \set{x_i : i \in \Z^+}\).
  By axiom of choice (\cref{i:8.1}) \(A\) can be construct as the image of \((x_n)_{n \in \Z^+}\).
  Now we define a function \(f : X \to X \setminus \set{x_1}\) as follow:
  \[
    f(x) = \begin{dcases}
      x_{n + 1} & \text{if } x = x_n \text{ for some } x_n \in A, \\
      x         & \text{if } x \notin A.
    \end{dcases}
  \]
  We show that such \(f\) is bijective.
  We start by showing \(f\) is injective.
  Let \(x, x' \in X\) and \(x \neq x'\).
  We split into four cases:
  \begin{itemize}
    \item If \(x \in A \land x' \in A\), then \(\exists n, n' \in \Z^+\) such that \(x = x_n \land x' = x_{n'}\).
          By the definition of \(x_n\) and \(x_{n'}\), we must have \(x_n \neq x_{n'} \implies x_{n + 1} \neq x_{n' + 1}\).
          Thus we have \(x_{n + 1} = f(x) \neq f(x') = x_{n' + 1}\).
    \item If \(x \in A \land x' \notin A\), then \(f(x) \in A \land f(x') = x' \notin A\) and thus \(f(x) \neq f(x')\).
    \item If \(x \notin A \land x' \in A\), then \(f(x) = x \notin A \land f(x') \in A\) and thus \(f(x) \neq f(x')\).
    \item If \(x \notin A \land x' \notin A\), then \(f(x) = x \neq x' = f(x')\).
  \end{itemize}
  From all cases above we conclude that \(x \neq x' \implies f(x) \neq f(x')\), thus \(f\) is injective.
  Now we show that \(f\) is surjective.
  Let \(x \in X \setminus \set{x_1}\).
  We split into two cases:
  \begin{itemize}
    \item If \(x \in A\), then \(x \neq x_1\) and \(\exists n \in \Z^+ \setminus \set{1}\) such that \(x = x_n\).
          Since \(n \geq 2\), we have \(n - 1 \geq 1\).
          Thus by the definition of \(A\) we have \(x_{n - 1} \in A\) and \(f(x_{n - 1}) = x_n\).
    \item If \(x \notin A\), then we have \(f(x) = x\).
  \end{itemize}
  Since \(x\) was arbitrary, we know that \(f\) is surjective.
  Since \(f\) is both injective and surjective, we know that \(f\) is bijective, and by \cref{i:3.6.1} \(X\) and \(X \setminus \set{x_1}\) have the same cardinality.
  But \(x_1 \in X \land x_1 \notin X \setminus \set{x_1}\), we have \(X \neq X \setminus \set{x_1}\).
  Thus by \cref{i:3.1.15} \(X \setminus \set{x_1} \subsetneq X\).

  Now we show that if \(\exists Y \subsetneq X\) where \(X\) and \(Y\) have the same cardinality, then \(X\) is infinite.
  We prove this by contradiction.
  Suppose for sake of contradiction that \(X\) is finite.
  Then by \cref{i:3.6.14}(c) we have \(\#(Y) < \#(X)\), a contradiction.
  Thus \(X\) is infinite.
\end{proof}

\begin{ex}\label{i:ex:8.1.2}
  Prove \cref{i:8.1.4}.
\end{ex}

\begin{proof}
  See \cref{i:8.1.4}.
\end{proof}

\begin{ex}\label{i:ex:8.1.3}
  Fill in the gaps marked in \cref{i:8.1.5}.
\end{ex}

\begin{proof}
  See \cref{i:8.1.5}.
\end{proof}

\begin{ex}\label{i:ex:8.1.4}
  Prove \cref{i:8.1.8}.
\end{ex}

\begin{proof}
  See \cref{i:8.1.8}.
\end{proof}

\begin{ex}\label{i:ex:8.1.5}
  Use \cref{i:8.1.8} to prove \cref{i:8.1.9}.
\end{ex}

\begin{proof}
  See \cref{i:8.1.9}.
\end{proof}

\begin{ex}\label{i:ex:8.1.6}
  Let \(A\) be a set.
  Show that \(A\) is at most countable iff there exists an injective map \(f : A \to \N\) from \(A\) to \(\N\).
\end{ex}

\begin{proof}
  We first show that if \(A\) is at most countable, then there exists an injective map \(f : A \to \N\).
  Suppose that \(A\) is at most countable.
  By \cref{i:8.1.1} \(A\) is either finite or countable.
  \begin{itemize}
    \item If \(A\) is finite, then by \cref{i:3.6.10} \(\exists g : A \to \set{i \in \N : 1 \leq i \leq \#(A)}\) such that \(g\) is bijective.
          Now let \(f : A \to \N\) be the function \(f(x) = g(x)\) for every \(x \in A\).
          Since \(g\) is a bijection and \(\set{i \in \N : 1 \leq i \leq \#(A)} \subseteq \N\), we have \(f : A \to \N\) is injective.
    \item If \(A\) is countable, then by \cref{i:8.1.1} \(\exists f : A \to \N\) such that \(f\) is a bijection, and hence \(f\) is injective.
  \end{itemize}
  From all cases above we conclude that if \(A\) is at most countable then there exists an injective map \(f : A \to \N\).

  Now we show that if there exists an injective map \(f : A \to \N\), then \(A\) is at most countable.
  Suppose that \(f : A \to \N\) is injective.
  Since \(f(A) \subseteq \N\), by \cref{i:8.1.6} \(f(A)\) is at most countable.
  Since \(f\) is bijective from \(A\) to \(f(A)\), we know that \(A\) and \(f(A)\) have equal cardinality, and thus \(A\) is at most countable.
\end{proof}

\begin{ex}\label{i:ex:8.1.7}
  Prove \cref{i:8.1.10}.
\end{ex}

\begin{proof}
  See \cref{i:8.1.10}.
\end{proof}

\begin{ex}\label{i:ex:8.1.8}
  Use \cref{i:8.1.13} to prove \cref{i:8.1.14}.
\end{ex}

\begin{proof}
  See \cref{i:8.1.14}.
\end{proof}

\begin{ex}\label{i:ex:8.1.9}
  Suppose that \(I\) is an at most countable set, and for each \(\alpha \in I\), let \(A_{\alpha}\) be an at most countable set.
  Show that the set \(\bigcup_{\alpha \in I} A_{\alpha}\) is also at most countable.
  In particular, countable unions of countable sets are countable.
\end{ex}

\begin{proof}
  Suppose that \(I\) be an at most countable set and \(\forall \alpha \in I\) we have \(A_{\alpha}\) is an at most countable set.
  By \cref{i:8.1.1} \(I\) is either finite or countable.

  We first show that if \(I\) is finite, then \(\bigcup_{\alpha \in I} A_{\alpha}\) is at most countable.
  Since \(I\) is finite, by \cref{i:3.6.5} \(\exists n \in \N\) such that \(\#(I) = n\).
  Let \(P(n)\) be the statement ``\(\#(I) = n\) and \(\bigcup_{\alpha \in I} A_{\alpha}\) is at most countable.''
  We induct on \(n\) to show that \(P(n)\) is true for every \(n \in \N\).
  For \(n = 0\), we have \(\#(\emptyset) = 0\) and \(\bigcup_{\alpha \in \emptyset} A_{\alpha} = \emptyset\).
  Thus, the base case holds.
  Suppose inductively that \(P(n)\) is true for some \(n \geq 0\).
  Then we need to show that \(P(n + 1)\) is also true.
  Since \(\#(I) = n + 1 > 0\), we know that \(I \neq \emptyset\).
  Let \(i \in I\).
  Since \(\#(I \setminus \set{i}) = n\), by induction hypothesis we know that the set \(\bigcup_{\alpha \in I \setminus \set{i}} A_{\alpha}\) is at most countable.
  By \cref{i:3.11} we have \(\bigcup_{\alpha \in I} A_{\alpha} = (\bigcup_{\alpha \in I \setminus \set{i}} A_{\alpha}) \cup A_i\).
  Then by \cref{i:ac:8.1.1} we know that \(\bigcup_{\alpha \in I} A_{\alpha}\) is at most countable.
  This closes the induction.
  We conclude that finite union of at most countable sets is at most countable.

  Now we show the case where \(I\) is countable.
  Let \(J = \set{\alpha \in I : A_{\alpha} \neq \emptyset}\).
  Since \(J \subseteq I\), by \cref{i:8.1.7} we know that \(J\) is at most countable.
  If \(J\) is finite (including the case where \(J = \emptyset\)), then we already show that finite union of at most countable sets is at most countable.
  So suppose that \(J\) is countable.
  Then we have
  \begin{align*}
         & \forall x \in \bigcup_{\alpha \in I} A_{\alpha} \\
    \iff & \exists \alpha' \in I : x \in A_{\alpha'}'      \\
    \iff & A_{\alpha'}' \neq \emptyset                     \\
    \iff & \alpha' \in J                                   \\
    \iff & \exists \alpha' \in J : x \in A_{\alpha'}'      \\
    \iff & x \in \bigcup_{\alpha \in J} A_{\alpha}.
  \end{align*}
  Thus by \cref{i:3.1.4} we have \(\bigcup_{\alpha \in I} A_{\alpha} = \bigcup_{\alpha \in J} A_{\alpha}\).
  To show that \(\bigcup_{\alpha \in I} A_{\alpha}\) is at most countable, it suffices to show that \(\bigcup_{\alpha \in J} A_{\alpha}\) is at most countable.

  Since \(\forall \alpha \in J\), \(A_{\alpha}\) is at most countable.
  We split into two cases:
  \begin{itemize}
    \item If \(A_{\alpha}\) is finite, then by \cref{i:3.6.5} \(\exists f_{\alpha}' : \set{n \in \N : 1 \leq n \leq \#(A_{\alpha})} \to A_{\alpha}\) such that \(f_{\alpha}'\) is bijective.
          We now define a function \(f_{\alpha} : \N \to A_{\alpha}\) as follow:
          \[
            \forall n \in \N : f_{\alpha}(n) = \begin{dcases}
              f_{\alpha}'(n) & \text{if } 1 \leq n \leq \#(A_{\alpha}),  \\
              f_{\alpha}'(1) & \text{if } n = 0 \lor n > \#(A_{\alpha}).
            \end{dcases}
          \]
          Thus \(f_{\alpha}\) is surjective.
          We can define \(F_{\alpha}\) to be a set of functions
          \[
            F_{\alpha} = \set{f_{\alpha} : \N \to A_{\alpha} | f_{\alpha} \text{ follows the definition above}}
          \]
          and \(F_{\alpha} \neq \emptyset\).
    \item If \(A_{\alpha}\) is countable, then we define \(F_{\alpha}\) to be a set of bijections
          \[
            F_{\alpha} = \set{f_{\alpha} : \N \to A_{\alpha} | f_{\alpha} \text{ is bijective}}.
          \]
          Since \(A_{\alpha}\) is countable, we know that \(F_{\alpha} \neq \emptyset\).
  \end{itemize}
  Since \(\forall \alpha \in J\), \(F_{\alpha} \neq \emptyset\), by axiom of choice (\cref{i:8.1}) the set \(\prod_{\alpha \in J} F_{\alpha} \neq \emptyset\).
  This means we can choose a function \((f_{\alpha})_{\alpha \in J}\) from \(\prod_{\alpha \in J} F_{\alpha}\) which maps \(\alpha \in J\) to a function \(f_{\alpha} : \N \to A_{\alpha}\).

  We now use axiom of choice (\cref{i:8.1}) to choose a function \((f_{\alpha})_{\alpha \in J}\) and fix such function.
  Since \(J\) is countable, \(\exists g : \N \to J\) such that \(g\) is bijective.
  We now define another function \(h : \N \times \N \to \bigcup_{\alpha \in J} A_{\alpha}\) as follow:
  \[
    \forall (n, m) \in \N \times \N : h(n, m) = f_{g(n)}(m).
  \]
  By \cref{i:8.1.9} we now that \(h(\N \times \N)\) is at most countable.
  If we can show that \(h\) is surjective, then we can show that \(\bigcup_{\alpha \in J} A_{\alpha}\) is at most countable.
  Let \(x \in \bigcup_{\alpha \in J} A_{\alpha}\).
  We know that \(\exists \beta \in J\) such that \(x \in A_{\beta}\).
  By the definition of \(f_{\beta}\) we know that \(f_{\beta}\) is surjective.
  Since \(f_{\beta}\) is surjective, \(\exists m \in \N\) such that \(f_{\beta}(m) = x\).
  Since \(g\) is bijective, \(\exists n \in \N\) such that \(g(n) = \beta\).
  Then we have
  \[
    (n, m) \in \N \times \N \implies h(n, m) = f_{g(n)}(m) = f_{\beta}(m) = x.
  \]
  Since \(x\) was arbitrary, we thus know that \(h\) is surjective.
  We conclude that countable union of at most countable set is at most countable.

  Finally we show that countable union of countable set is countable.
  Let \(I\) be a countable set and \(\forall \alpha \in I\) let \(A_{\alpha}\) be countable set.
  From the proof above we know that \(\bigcup_{\alpha \in I} A_{\alpha}\) is at most countable.
  Suppose for sake of contradiction that \(\bigcup_{\alpha \in I} A_{\alpha}\) is finite.
  Let \(\beta \in I\).
  By hypothesis we know that \(A_{\beta}\) is countable, and we have
  \[
    A_{\beta} \subseteq \bigcup_{\alpha \in I} A_{\alpha}.
  \]
  But \(\bigcup_{\alpha \in I} A_{\alpha}\) is finite, thus by \cref{i:3.6.14}(c) we know that \(A_{\beta}\) is finite, a contradiction.
  We conclude that countable union of countable set is countable.
\end{proof}

\begin{ex}\label{i:ex:8.1.10}
  Find a bijection \(f : \N \to \Q\) from the natural numbers to the rationals.
\end{ex}

\begin{proof}
  Helped needed.
\end{proof}

\section{Summation on infinite sets}\label{sec:8.2}

\begin{defn}[Series on countable sets]\label{8.2.1}
  Let \(X\) be a countable set, and let \(f : X \to \R\) be a function.
  We say that the series \(\sum_{x \in X} f(x)\) is absolutely convergent iff for some bijection \(g : \N \to X\), the sum \(\sum_{n = 0}^\infty f(g(n))\) is absolutely convergent.
  We then define the sum of \(\sum_{x \in X} f(x)\) by the formula
  \[
    \sum_{x \in X} f(x) = \sum_{n = 0}^\infty f(g(n)).
  \]
\end{defn}

\begin{note}
  From \cref{7.4.3}, one can show that these definitions do not depend on the choice of \(g\), and so are well defined.
\end{note}

\begin{note}
  For finite sets \(X\) we adopt the convention that series \(\sum_{x \in X} f(x)\) are automatically considered to be absolutely convergent.
\end{note}

\begin{ac}\label{ac:8.2.1}
  Let \(X\) be an at most countable set, and let \(f : X \to \R\) and \(g : X \to \R\) be functions such that the series \(\sum_{x \in X} f(x)\) and \(\sum_{x \in X} g(x)\) are both absolutely convergent.
  \begin{enumerate}
    \item The series \(\sum_{x \in X} (f(x) + g(x))\) is absolutely convergent, and
          \[
            \sum_{x \in X} (f(x) + g(x)) = \sum_{x \in X} f(x) + \sum_{x \in X} g(x).
          \]
    \item If \(c\) is a real number, then \(\sum_{x \in X} cf(x)\) is absolutely convergent, and
          \[
            \sum_{x \in X} cf(x) = c \sum_{x \in X} f(x).
          \]
    \item If \(X = X_1 \cup X_2\) for some disjoint sets \(X_1\) and \(X_2\), then \(\sum_{x \in X_1} f(x)\) and \\
          \(\sum_{x \in X_2} f(x)\) are absolutely convergent, and
          \[
            \sum_{x \in X_1 \cup X_2} f(x) = \sum_{x \in X_1} f(x) + \sum_{x \in X_2} f(x).
          \]
          Conversely, if \(h : X \to \R\) is such that \(\sum_{x \in X_1} h(x)\) and \(\sum_{x \in X_2} h(x)\) are absolutely convergent, then \(\sum_{x \in X_1 \cup X_2} h(x)\) is also absolutely convergent, and
          \[
            \sum_{x \in X_1 \cup X_2} h(x) = \sum_{x \in X_1} h(x) + \sum_{x \in X_2} h(x).
          \]
    \item If \(Y\) is another set, and \(\phi : Y \to X\) is a bijection, then \(\sum_{y \in Y} f(\phi(y))\) is absolutely convergent, and
          \[
            \sum_{y \in Y} f(\phi(y)) = \sum_{x \in X} f(x).
          \]
  \end{enumerate}
\end{ac}

\begin{proof}{(a)}
  Since \(X\) is at most countable, by \cref{8.1.1} we know that \(X\) is either finite or countable.
  If \(X\) is finite, then the statement follows from \cref{7.1.11}(f).
  So suppose that \(X\) is countable.
  By \cref{8.2.1} we know that there exists a bijection \(p : \N \to X\) such that \(\sum_{n = 0}^\infty f\big(p(n)\big)\) converges.
  Similarly, there exists a bijection \(q : \N \to X\) such that \(\sum_{n = 0}^\infty g\big(q(n)\big)\) converges.
  Since \(p\) is bijective, by \cref{7.4.3} we know that
  \[
    \sum_{x \in X} g(x) = \sum_{n = 0}^\infty g(q(n)) = \sum_{n = 0}^\infty g(p(n)).
  \]
  Thus we have
  \begin{align*}
     & \sum_{x \in X} \abs{f(x)} + \sum_{x \in X} \abs{g(x)}                                                                          \\
     & = \sum_{n = 0}^\infty \abs{f\big(p(n)\big)} + \sum_{n = 0}^\infty \abs{g\big(p(n)\big)}      &  & \by{8.2.1}                   \\
     & = \sum_{n = 0}^\infty \Big(\abs{f\big(p(n)\big)} + \abs{g\big(p(n)\big)}\Big)                &  & \text{(by \cref{7.2.14}(a))} \\
     & = \lim_{N \to \infty} \sum_{n = 0}^N \Big(\abs{f\big(p(n)\big)} + \abs{g\big(p(n)\big)}\Big) &  & \by{7.2.2}                   \\
     & \geq \lim_{N \to \infty} \sum_{n = 0}^N \abs{f\big(p(n)\big) + g\big(p(n)\big)}              &  & \text{(by \cref{6.1.19}(h))} \\
     & = \sum_{n = 0}^\infty \abs{f\big(p(n)\big) + g\big(p(n)\big)}                                &  & \by{6.3.8}                   \\
     & = \sum_{x \in X} \abs{f(x) + g(x)}                                                           &  & \by{8.2.1}
  \end{align*}
  and \(\sum_{x \in X} f(x) + g(x)\) is absolutely convergent.
  This implies
  \begin{align*}
     & \sum_{x \in X} f(x) + \sum_{x \in X} g(x)                                                                     \\
     & = \sum_{n = 0}^\infty f\big(p(n)\big) + \sum_{n = 0}^\infty g\big(p(n)\big) &  & \by{8.2.1}                   \\
     & = \sum_{n = 0}^\infty \Big(f\big(p(n)\big) + g\big(p(n)\big)\Big)           &  & \text{(by \cref{7.2.14}(a))} \\
     & = \sum_{x \in X} \big(f(x) + g(x)\big).                                     &  & \by{8.2.1}
  \end{align*}
\end{proof}

\begin{proof}{(b)}
  Since \(X\) is at most countable, by \cref{8.1.1} we know that \(X\) is either finite or countable.
  If \(X\) is finite, then the statement follows from \cref{7.1.11}(g).
  So suppose that \(X\) is countable.
  By \cref{8.2.1} we know that there exists a bijection \(p : \N \to X\) such that \(\sum_{n = 0}^\infty f\big(p(n)\big)\) converges.
  Then we have
  \begin{align*}
    \abs{c} \sum_{x \in X} \abs{f(x)} & = \abs{c} \sum_{n = 0}^\infty \abs{f\big(p(n)\big)} &  & \by{8.2.1}                   \\
                                      & = \sum_{n = 0}^\infty \abs{c} \abs{f\big(p(n)\big)} &  & \text{(by \cref{7.2.14}(b))} \\
                                      & = \sum_{n = 0}^\infty \abs{c f\big(p(n)\big)}                                         \\
                                      & = \sum_{x \in X} \abs{c f(x)}                       &  & \by{8.2.1}
  \end{align*}
  and thus \(\sum_{x \in X} \abs{c f(x)}\) is absolutely convergent.
  This implies
  \begin{align*}
    c \sum_{x \in X} f(x) & = c \sum_{n = 0}^\infty f\big(p(n)\big) &  & \by{8.2.1}                   \\
                          & = \sum_{n = 0}^\infty c f\big(p(n)\big) &  & \text{(by \cref{7.2.14}(b))} \\
                          & = \sum_{x \in X} f(x).                  &  & \by{8.2.1}
  \end{align*}
\end{proof}

\begin{proof}{(c)}
  We first show that if \(X = X_1 \cup X_2\), \(X_1 \cap X_2 = \emptyset\), then \(\sum_{x \in X_1} f(x)\) and \(\sum_{x \in X_2} f(x)\) is absolutely convergent.
  Since \(X\) is at most countable, by \cref{8.1.1} we know that \(X\) is either finite or countable.
  If \(X\) is finite, then the statement follows from \cref{7.1.11}(e).
  So suppose that \(X\) is countable.
  Since \(X = X_1 \cup X_2\), we know that \(X_1\) and \(X_2\) cannot both be finite.
  Now we split into two cases:
  \begin{itemize}
    \item One of \(X_1, X_2\) is finite and one is countable.
          Without the loss of generality suppose that \(X_1\) is finite.
          Since \(X_1\) is finite, we know that \(\exists q_1 : \{i \in \N : 1 \leq i \leq \#(X_1)\} \to X_1\) such that \(q_1\) is bijective.
          Since \(X_2\) is countable, by \cref{8.1.1} we know that \(\exists q_2 : \N \to X_2\) such that \(q_2\) is bijective.
          Then we define a function \(q : \N \to X\) as follow:
          \[
            \forall n \in \N, q(n) = \begin{dcases}
              q_1(n + 1)       & \text{if } n < \#(X_1)    \\
              q_2(n - \#(X_1)) & \text{if } n \geq \#(X_1)
            \end{dcases}
          \]
          Such \(q\) is bijective since \(X_1 \cap X_2 = \emptyset\) and \(q_1, q_2\) are bijective.
          Then we have
          \begin{align*}
            \sum_{x \in X} \abs{f(x)} & = \sum_{n = 0}^\infty \abs{f\big(q(n)\big)}                                                          &  & \by{8.2.1}                   \\
                                      & = \sum_{n = 0}^{\#(X_1) - 1} \abs{f\big(q(n)\big)} + \sum_{n = \#(X_1)}^\infty \abs{f\big(q(n)\big)} &  & \text{(by \cref{7.2.14}(c))} \\
                                      & = \sum_{n = 0}^{\#(X_1) - 1} \abs{f\big(q_1(n + 1)\big)}                                                                               \\
                                      & \quad + \sum_{n = \#(X_1)}^\infty \abs{f\Big(q_2\big(n - \#(X_1)\big)\Big)}                                                            \\
                                      & = \sum_{n = 1}^{\#(X_1)} \abs{f\big(q_1(n)\big)}                                                     &  & \text{(by \cref{7.1.4}(b))}  \\
                                      & \quad + \sum_{n = 0}^\infty \abs{f\big(q_2(n)\big)}                                                  &  & \text{(by \cref{7.2.14}(d))} \\
                                      & = \sum_{x \in X_1} \abs{f(x)}                                                                        &  & \by{7.1.6}                   \\
                                      & \quad + \sum_{x \in X_2} \abs{f\big(q(n)\big)}                                                       &  & \by{8.2.1}
          \end{align*}
          and thus both \(\sum_{x \in X_1} f(x)\) and \(\sum_{x \in X_2} f(x)\) are absolutely convergent.
          This implies
          \begin{align*}
            \sum_{x \in X} f(x) & = \sum_{n = 0}^\infty f(q(x))                                                            &  & \by{8.2.1}                   \\
                                & = \sum_{n = 0}^{\#(X_1) - 1} f\big(q(n)\big) + \sum_{n = \#(X_1)}^\infty f\big(q(n)\big) &  & \text{(by \cref{7.2.14}(c))} \\
                                & = \sum_{n = 0}^{\#(X_1) - 1} f\big(q_1(n + 1)\big)                                                                         \\
                                & \quad + \sum_{n = \#(X_1)}^\infty f\Big(q_2\big(n - \#(X_1)\big)\Big)                                                      \\
                                & = \sum_{n = 1}^{\#(X_1)} f\big(q_1(n)\big)                                               &  & \text{(by \cref{7.1.4}(b))}  \\
                                & \quad + \sum_{n = 0}^\infty f\big(q_2(n)\big)                                            &  & \text{(by \cref{7.2.14}(c))} \\
                                & = \sum_{x \in X_1} f(x)                                                                  &  & \by{7.1.6}                   \\
                                & \quad + \sum_{x \in X_2} f\big(q(n)\big).                                                &  & \by{8.2.1}
          \end{align*}
    \item Both \(X_1, X_2\) are countable.
          Since \(X_1\) is countable, by \cref{8.1.1} we know that \(\exists q_1 : \N \to X_1\) such that \(q_1\) is bijective.
          Similarly, \(\exists q_2 : \N \to X_2\) such that \(q_2\) is bijective.
          Then we define a function \(q : \N \to X\) as follow:
          \[
            \forall n \in \N, q(n) = \begin{dcases}
              q_1(\dfrac{n}{2})     & \text{if } n \text{ is even} \\
              q_2(\dfrac{n - 1}{2}) & \text{if } n \text{ is odd}
            \end{dcases}
          \]
          Such \(q\) is bijective since \(X_1 \cap X_2 = \emptyset\) and \(q_1, q_2\) are bijective.
          Then we have
          \begin{align*}
             & \sum_{x \in X} \abs{f(x)}                                                                                                                                   \\
             & = \sum_{n = 0}^\infty \abs{f\big(q(n)\big)}                                                                               &  & \by{8.2.1}                   \\
             & = \lim_{N \to \infty} \sum_{n = 0}^{2N} \abs{f\big(q(n)\big)}                                                             &  & \by{7.2.2}                   \\
             & = \lim_{N \to \infty} \sum_{n \leq 2N} \abs{f\big(q(n)\big)}                                                              &  & \by{7.1.6}                   \\
             & = \lim_{N \to \infty} \Bigg(\sum_{n \leq 2N \land n \text{ is even}} \abs{f\big(q(n)\big)}                                                                  \\
             & \quad + \sum_{n \leq 2N \land n \text{ is odd}} \abs{f\big(q(n)\big)}\Bigg)                                               &  & \text{(by \cref{7.1.11}(e))} \\
             & = \lim_{N \to \infty} \Bigg(\sum_{n \leq 2N \land n \text{ is even}} \abs{f\big(q_1(\dfrac{n}{2})\big)}                                                     \\
             & \quad + \sum_{n \leq 2N \land n \text{ is odd}} \abs{f\big(q_2(\dfrac{n - 1}{2})\big)}\Bigg)                                                                \\
             & = \lim_{N \to \infty} \sum_{n \leq 2N \land n \text{ is even}} \abs{f\big(q_1(\dfrac{n}{2})\big)}                                                           \\
             & \quad + \lim_{N \to \infty} \sum_{n \leq 2N \land n \text{ is odd}} \abs{f\big(q_2(\dfrac{n - 1}{2})\big)}                &  & \text{(by \cref{6.1.19}(a))} \\
             & = \lim_{N \to \infty} \sum_{n = 0}^N \abs{f\big(q_1(n)\big)} + \lim_{N \to \infty} \sum_{n = 0}^N \abs{f\big(q_2(n)\big)} &  & \by{7.1.6}                   \\
             & = \sum_{n = 0}^\infty \abs{f\big(q_1(n)\big)} + \sum_{n = 0}^\infty \abs{f\big(q_2(n)\big)}                               &  & \by{7.2.2}                   \\
             & = \sum_{x \in X_1} \abs{f(x)} + \sum_{x \in X_2} \abs{f(x)}                                                               &  & \by{8.2.1}
          \end{align*}
          and thus both \(\sum_{x \in X_1} f(x)\) and \(\sum_{x \in X_2} f(x)\) are absolutely convergent.
          This implies
          \begin{align*}
             & \sum_{x \in X} f(x)                                                                                                                             \\
             & = \sum_{n = 0}^\infty f\big(q(n)\big)                                                                         &  & \by{8.2.1}                   \\
             & = \lim_{N \to \infty} \sum_{n = 0}^{2N} f\big(q(n)\big)                                                       &  & \by{7.2.2}                   \\
             & = \lim_{N \to \infty} \sum_{n \leq 2N} f\big(q(n)\big)                                                        &  & \by{7.1.6}                   \\
             & = \lim_{N \to \infty} \Bigg(\sum_{n \leq 2N \land n \text{ is even}} f\big(q(n)\big)                                                            \\
             & \quad + \sum_{n \leq 2N \land n \text{ is odd}} f\big(q(n)\big)\Bigg)                                         &  & \text{(by \cref{7.1.11}(e))} \\
             & = \lim_{N \to \infty} \Bigg(\sum_{n \leq 2N \land n \text{ is even}} f\big(q_1(\dfrac{n}{2})\big)                                               \\
             & \quad + \sum_{n \leq 2N \land n \text{ is odd}} f\big(q_2(\dfrac{n - 1}{2})\big)\Bigg)                                                          \\
             & = \lim_{N \to \infty} \sum_{n \leq 2N \land n \text{ is even}} f\big(q_1(\dfrac{n}{2})\big)                                                     \\
             & \quad + \lim_{N \to \infty} \sum_{n \leq 2N \land n \text{ is odd}} f\big(q_2(\dfrac{n - 1}{2})\big)          &  & \text{(by \cref{6.1.19}(a))} \\
             & = \lim_{N \to \infty} \sum_{n = 0}^N f\big(q_1(n)\big) + \lim_{N \to \infty} \sum_{n = 0}^N f\big(q_2(n)\big) &  & \by{7.1.6}                   \\
             & = \sum_{n = 0}^\infty f\big(q_1(n)\big) + \sum_{n = 0}^\infty f\big(q_2(n)\big)                               &  & \by{7.2.2}                   \\
             & = \sum_{x \in X_1} f(x) + \sum_{x \in X_2} f(x).                                                              &  & \by{8.2.1}
          \end{align*}
  \end{itemize}
  From all cases above we conclude that both \(\sum_{x \in X_1} f(x)\) and \(\sum_{x \in X_2} f(x)\) are absolutely convergent, and we have
  \[
    \sum_{x \in X} f(x) = \sum_{x \in X_1} f(x) + \sum_{x \in X_2} f(x).
  \]

  Now we show that if \(X_1 \cup X_2 \subseteq X\), \(X_1 \cap X_2 = \emptyset\), \(\sum_{x \in X_1} h(x)\) and \(\sum_{x \in X_2} h(x)\) are absolutely convergent, then \(\sum_{x \in X_1 \cup X_2} h(x)\) is absolutely convergent.
  Since \(X\) is at most countable, by \cref{8.1.7} we know that \(X_1 \cup X_2\) is at most countable.
  By \cref{8.1.1} we know that \(X_1 \cup X_2\) is either finite or countable.
  If \(X_1 \cup X_2\) is finite, then the statement follows from \cref{7.1.11}(e).
  So suppose that \(X_1 \cup X_2\) is countable.
  We know that \(X_1\) and \(X_2\) cannot both be finite.
  Now we split into two cases:
  \begin{itemize}
    \item One of \(X_1, X_2\) is finite and one is countable.
          Without the loss of generality suppose that \(X_1\) is finite.
          Since \(X_1\) is finite, we know that \(\exists q_1 : \{i \in \N : 1 \leq i \leq \#(X_1)\} \to X_1\) such that \(q_1\) is bijective.
          Since \(X_2\) is countable, by \cref{8.1.1} we know that \(\exists q_2 : \N \to X_2\) such that \(q_2\) is bijective.
          Then we define a function \(q : \N \to X_1 \cup X_2\) as follow:
          \[
            \forall n \in \N, q(n) = \begin{dcases}
              q_1(n + 1)       & \text{if } n < \#(X_1)    \\
              q_2(n - \#(X_1)) & \text{if } n \geq \#(X_1)
            \end{dcases}
          \]
          Such \(q\) is bijective since \(X_1 \cap X_2 = \emptyset\) and \(q_1, q_2\) are bijective.
          Then we have
          \begin{align*}
             & \sum_{x \in X_1} \abs{h(x)} + \sum_{x \in X_2} \abs{h(x)}                                                                              \\
             & = \sum_{n = 1}^{\#(X_1)} \abs{h\big(q_1(n)\big)}                                                     &  & \by{7.1.6}                   \\
             & \quad + \sum_{n = 0}^\infty \abs{h\big(q_2(n)\big)}                                                  &  & \by{8.2.1}                   \\
             & = \sum_{n = 0}^{\#(X_1) - 1} \abs{h\big(q_1(n + 1)\big)}                                             &  & \text{(by \cref{7.1.4}(b))}  \\
             & \quad + \sum_{n = \#(X_1)}^\infty \abs{h\Big(q_2\big(n - \#(X_1)\big)\Big)}                          &  & \text{(by \cref{7.2.14}(d))} \\
             & = \sum_{n = 0}^{\#(X_1) - 1} \abs{h\big(q(n)\big)} + \sum_{n = \#(X_1)}^\infty \abs{h\big(q(n)\big)}                                   \\
             & = \sum_{n = 0}^\infty \abs{h\big(q(n)\big)}                                                          &  & \text{(by \cref{7.2.14}(c))} \\
             & = \sum_{x \in X_1 \cup X_2} \abs{h(x)}                                                               &  & \by{8.2.1}
          \end{align*}
          and thus \(\sum_{x \in X_1 \cup X_2} h(x)\) is absolutely convergent.
    \item Both \(X_1, X_2\) are countable.
          Since \(X_1\) is countable, by \cref{8.1.1} we know that \(\exists q_1 : \N \to X_1\) such that \(q_1\) is bijective.
          Similarly, \(\exists q_2 : \N \to X_2\) such that \(q_2\) is bijective.
          Then we define a function \(q : \N \to X_1 \cup X_2\) as follow:
          \[
            \forall n \in \N, q(n) = \begin{dcases}
              q_1(\dfrac{n}{2})     & \text{if } n \text{ is even} \\
              q_2(\dfrac{n - 1}{2}) & \text{if } n \text{ is odd}
            \end{dcases}
          \]
          Such \(q\) is bijective since \(X_1 \cap X_2 = \emptyset\) and \(q_1, q_2\) are bijective.
          Then we have
          \begin{align*}
             & \sum_{x \in X_1} \abs{h(x)} + \sum_{x \in X_2} \abs{h(x)}                                                                                             \\
             & = \sum_{n = 0}^\infty \abs{h\big(q_1(n)\big)} + \sum_{n = 0}^\infty \abs{h\big(q_2(n)\big)}                         &  & \by{8.2.1}                   \\
             & = \sum_{n = 0}^\infty \Big(\abs{h\big(q_1(n)\big)} + \abs{h\big(q_2(n)\big)}\Big)                                   &  & \text{(by \cref{7.2.14}(a))} \\
             & = \lim_{N \to \infty} \Bigg(\sum_{n = 0}^N \abs{h\big(q_1(n)\big)} + \sum_{n = 0}^N \abs{h\big(q_2(n)\big)}\Bigg)   &  & \by{7.2.2}                   \\
             & = \lim_{N \to \infty} \Bigg(\sum_{n = 0}^N \abs{h\big(q(2n)\big)} + \sum_{n = 0}^N \abs{h\big(q(2n + 1)\big)}\Bigg)                                   \\
             & = \lim_{N \to \infty} \Bigg(\sum_{n \leq 2N : n \text{ is even}} \abs{h\big(q(n)\big)}                                                                \\
             & \quad + \sum_{n \leq 2N : n \text{ is odd}} \abs{h\big(q(n)\big)}\Bigg)                                             &  & \by{7.1.6}                   \\
             & = \lim_{N \to \infty} \sum_{n \leq 2N} \abs{h\big(q(n)\big)}                                                        &  & \text{(by \cref{7.1.11}(e))} \\
             & = \lim_{N \to \infty} \sum_{n = 0}^{2N} \abs{h\big(q(n)\big)}                                                       &  & \by{7.1.6}                   \\
             & = \sum_{n = 0}^\infty \abs{h\big(q(n)\big)}                                                                         &  & \by{7.2.2}                   \\
             & = \sum_{x \in X_1 \cup X_2}^\infty \abs{h(x)}                                                                       &  & \by{8.2.1}
          \end{align*}
          and thus \(\sum_{x \in X_1 \cup X_2} h(x)\) is absolutely convergent.
  \end{itemize}
  From all cases above we conclude that \(\sum_{x \in X_1 \cup X_2} h(x)\) is absolutely convergent.
  Since \(\sum_{x \in X_1 \cup X_2} h(x)\) is absolutely convergent, from the proof above we have
  \[
    \sum_{x \in X_1 \cup X_2} h(x) = \sum_{x \in X_1} h(x) + \sum_{x \in X_2} h(x).
  \]
\end{proof}

\begin{proof}{(d)}
  Since \(X\) is at most countable, by \cref{8.1.1} we know that \(X\) is either finite or countable.
  If \(X\) is finite, then the statement follows from \cref{7.1.11}(c).
  So suppose that \(X\) is countable.
  By \cref{8.2.1} we know that there exists a bijection \(p : \N \to X\) such that \(\sum_{n = 0}^\infty f\big(p(n)\big)\) converges.
  Since \(\phi\) is bijective, we know that \(Y\) is also countable and by \cref{8.1.1} \(\exists q : \N \to Y\) such that \(q\) is bijective.
  Then we have \(\phi \circ q : \N \to X\) is bijective and
  \begin{align*}
    \sum_{x \in X} f(x) & = \sum_{n = 0}^\infty f\big(p(n)\big)               &  & \by{8.2.1} \\
                        & = \sum_{n = 0}^\infty f\big((\phi \circ q)(n)\big)  &  & \by{7.4.3} \\
                        & = \sum_{n = 0}^\infty f\Big(\phi\big(q(n)\big)\Big)                 \\
                        & = \sum_{y \in Y} f\big(\phi(y)\big).                &  & \by{8.2.1}
  \end{align*}
  Thus \(\sum_{y \in Y} f\big(\phi(y)\big)\) is absolutely convergent.
\end{proof}

\begin{thm}[Fubini's theorem for infinite sums]\label{8.2.2}
  Let \(f : \N \times \N \to \R\) be a function such that \(\sum_{(n, m) \in \N \times \N} f(n, m)\) is absolutely convergent.
  Then we have
  \begin{align*}
    \sum_{n = 0}^\infty \bigg(\sum_{m = 0}^\infty f(n, m)\bigg) & = \sum_{(n, m) \in \N \times \N} f(n, m)                       \\
                                                                & = \sum_{(m, n) \in \N \times \N} f(n, m)                       \\
                                                                & = \sum_{m = 0}^\infty \bigg(\sum_{n = 0}^\infty f(n, m)\bigg).
  \end{align*}
  In other words, we can switch the order of infinite sums \emph{provided that the entire sum is absolutely convergent}.
\end{thm}

\begin{proof}
  The second equality follows easily from \cref{7.4.3} (and \cref{3.6.4}).

  Let us first consider the case when \(f(n, m)\) is always non-negative (we will deal with the general case later).
  Write
  \[
    L \coloneqq \sum_{(n, m) \in \N \times \N} f(n, m);
  \]
  our task is to show that the series \(\sum_{n = 0}^\infty (\sum_{m = 0}^\infty f(n, m))\) converges to \(L\).

  One can easily show that \(\sum_{(n, m) \in X} f(n, m) \leq L\) for all finite sets \(X \subseteq \N \times \N\).
  (Use a bijection \(g\) between \(\N \times \N\) and \(\N\), and then use the fact that \(g(X)\) is finite, hence bounded.)
  In particular, for every \(n \in \N\) and \(M \in \N\) we have \(\sum_{m = 0}^M f(n, m) \leq L\), which implies by \cref{6.3.8} that \(\sum_{m = 0}^\infty f(n, m)\) is convergent for each \(n\).
  Similarly, for any \(N \in \N\) and \(M \in \N\) we have (by \cref{7.1.14})
  \[
    \sum_{n = 0}^N \sum_{m = 0}^M f(n, m) = \sum_{(n, m) \in X} f(n, m) \leq L
  \]
  where \(X\) is the set \(\{(n,m) \in \N \times \N : n \leq N, m \leq M\}\) which is finite by \cref{3.6.14}.
  Taking limits of this as \(M \to \infty\) we have (by \cref{ex:7.1.5} and either \cref{6.3.8} or \cref{6.4.13})
  \[
    \sum_{n = 0}^N \sum_{m = 0}^\infty f(n, m) \leq L.
  \]
  By \cref{6.3.8}, this implies that \(\sum_{n = 0}^\infty \sum_{m = 0}^\infty f(n, m)\) converges, and
  \[
    \sum_{n = 0}^\infty \sum_{m = 0}^\infty f(n, m) \leq L.
  \]
  To finish the proof, it will suffice to show that
  \[
    \sum_{n = 0}^\infty \sum_{m = 0}^\infty f(n, m) \geq L - \varepsilon
  \]
  for every \(\varepsilon > 0\).
  \begin{align*}
             & L \geq \sum_{n = 0}^\infty \sum_{m = 0}^\infty f(n, m) \geq L - \varepsilon               \\
    \implies & L + \varepsilon \geq \sum_{n = 0}^\infty \sum_{m = 0}^\infty f(n, m) \geq L - \varepsilon \\
    \implies & \varepsilon \geq \sum_{n = 0}^\infty \sum_{m = 0}^\infty f(n, m) - L \geq -\varepsilon    \\
    \implies & \abs{\sum_{n = 0}^\infty \sum_{m = 0}^\infty f(n, m) - L} \leq \varepsilon                \\
  \end{align*}
  So, let \(\varepsilon > 0\).
  By definition of \(L\), we can then find a finite set \(X \subseteq \N \times \N\) such that \(\sum_{(n, m) \in X} f(n, m) \geq L - \varepsilon\).
  (Since \(\N \times \N\) is countable by \cref{8.1.13}, we can find a bijection \(g : \N \to \N \times \N\) such that \(\sum_{i = 0}^\infty f(g(i)) = L\), which means \(\forall \varepsilon > 0\), \(\exists H \in \N\) such that \(\abs{\sum_{i = 0}^h f(g(i)) - L} \leq \varepsilon\) for all \(h \geq H\).
  Now we can choose \(X = \{g(i) : 0 \leq i \leq H\}\))
  This set, being finite, must be contained in some set of the form \(Y \coloneqq \{(n,m) \in \N \times \N : n \leq N; m \leq M \}\).
  Thus by \cref{7.1.14}
  \[
    \sum_{n = 0}^N \sum_{m = 0}^M f(n, m) = \sum_{(n, m) \in Y} f(n, m) \geq \sum_{(n, m) \in X} f(n, m) \geq L - \varepsilon
  \]
  and hence
  \[
    \sum_{n = 0}^\infty \sum_{m = 0}^\infty f(n, m) \geq \sum_{n = 0}^N \sum_{m = 0}^\infty f(n, m) \geq \sum_{n = 0}^N \sum_{m = 0}^M f(n, m) \geq L - \varepsilon
  \]
  as desired.

  This proves the claim when the \(f(n, m)\) are all non-negative.
  A similar argument works when the \(f(n, m)\) are all non-positive
  (in fact, one can simply apply the result just obtained to the function \(-f(n, m)\), and then use limit laws to remove the \(-\).
  For the general case, note that any function \(f(n, m)\) can be written as \(f_+(n, m) + f_-(n, m)\), where \(f_+(n, m)\) is the positive part of \(f(n, m)\)
  (i.e., it equals \(f(n, m)\) when \(f(n, m)\) is positive, and \(0\) otherwise),
  and \(f_-\) is the negative part of \(f(n, m)\)
  (it equals \(f(n, m)\) when \(f(n, m)\) is negative, and \(0\) otherwise).
  It is easy to show that if \(\sum_{(n, m) \in \N \times \N} f(n, m)\) is absolutely convergent, then so are \(\sum_{(n, m) \in \N \times \N} f_+(n, m)\) and \(\sum_{(n, m) \in \N \times \N} f_-(n, m)\).
  (We can construct a bijection \(g : \N \to \N \times \N\) and then since \(\forall n \in \N\) we have \(f_+(g(n)) \leq \abs{f(g(n))}\) and \(\abs{f_-(g(n))} \leq \abs{f(g(n))}\), we know that \((f_+(g(n)))_{n = 0}^\infty\) and \((f_-(g(n)))_{n = 0}^\infty\) are absolutely convergent by comparison test (\cref{7.3.2}.))
  So now one applies the results just obtained to \(f_+\) and to \(f_-\) and adds them together using limit laws to obtain the result for a general \(f\).
\end{proof}

\begin{lem}\label{8.2.3}
  Let \(X\) be a countable set, and let \(f : X \to \R\) be a function.
  Then the series \(\sum_{x \in X} f(x)\) is absolutely convergent iff
  \[
    \sup\Bigg\{\sum_{x \in A} \abs{f(x)} : A \subseteq X, A \text{ finite}\Bigg\} < \infty.
  \]
\end{lem}

\begin{proof}
  Let \(P(X, f)\) be the statement
  \[
    \sup\Bigg\{\sum_{x \in A} \abs{f(x)} : A \subseteq X, A \text{ finite}\Bigg\} < \infty.
  \]
  We first show that if \(\sum_{x \in X} f(x)\) is absolutely convergent, then \(P(X, f)\) is true.
  Let \(L = \sum_{x \in X} f(x)\).
  Since \(\sum_{x \in X} f(x)\) is absolutely convergent, by \cref{8.2.1} \(\exists g : \N \to X\) where \(g\) is a bijection such that
  \[
    L = \sum_{x \in X} \abs{f(x)} = \sum_{n = 0}^\infty \abs{f\big(g(n)\big)}.
  \]
  Let \(A \subseteq X\) be a finite set.
  Since \(g\) is a bijection, we have
  \[
    \sum_{x \in A} \abs{f(x)} = \sum_{n \in g^{-1}(A)} \abs{f(g(n))}
  \]
  Since \(A\) is finite, by \cref{ex:3.6.3} \(\exists M \in \N\) such that \(g^{-1}(A)\) is bounded by \(M\).
  So we have
  \[
    \sum_{x \in A} \abs{f(x)} = \sum_{n \in g^{-1}(A)} \abs{f(g(n))} \leq \sum_{n = 0}^M \abs{f(g(n))} \leq L
  \]
  This is true for any finite subset of \(X\).
  Thus by \cref{5.5.9} \(P(X, f)\) is true.

  Now we show that if \(P(X, f)\) is true, then \(\sum_{x \in X} f(x)\) is absolutely convergent.
  Let \(L\) be the supremum described by \(P(X, f)\).
  Since \(X\) is countable, \(\exists g : \N \to X\) where \(g\) is a bijection.
  So we have
  \begin{align*}
             & \forall n \in \N : \sum_{x \in g(\{i \in \N : 0 \leq i \leq n\})} \abs{f(x)} \leq L & (P(X, f) \text{ is true})              \\
    \implies & \forall n \in \N : \sum_{i = 0}^n \abs{f(g(i))} \leq L                              &                           & \by{7.1.6} \\
    \implies & \sum_{i = 0}^\infty \abs{f(g(i))} \text{ converges}                                 &                           & \by{7.3.1} \\
    \implies & \sum_{x \in X} \abs{f(x)} \text{ converges}.                                        &                           & \by{8.2.1}
  \end{align*}
\end{proof}

\begin{note}
  Inspired by \cref{8.2.3}, we may now define the concept of an absolutely convergent series even when the set \(X\) could be uncountable.
\end{note}

\begin{defn}\label{8.2.4}
  Let \(X\) be a set (which could be uncountable), and let \(f : X \to \R\) be a function.
  We say that the series \(\sum_{x \in X} f(x)\) is absolutely convergent iff
  \[
    \sup\Bigg\{\sum_{x \in A} \abs{f(x)} : A \subseteq X, A \text{ finite}\Bigg\} < \infty.
  \]
\end{defn}

\begin{lem}\label{8.2.5}
  Let \(X\) be a set (which could be uncountable), and let \(f : X \to \R\) be a function such that the series \(\sum_{x \in X} f(x)\) is absolutely convergent.
  Then the set \(\{x \in X : f(x) \neq 0\}\) is at most countable.
\end{lem}

\begin{proof}
  Suppose that \(X\) is a set and \(f : X \to \R\) is a function such that \(\sum_{x \in X} f(x)\) is absolutely convergent.
  Since \(\sum_{x \in X} f(x)\) is absolutely convergent, by \cref{8.2.4} we have
  \[
    M = \sup\Bigg\{\sum_{x \in A} \abs{f(x)} : A \subseteq X, A \text{ finite}\Bigg\} < \infty.
  \]
  We first show that \(\forall n \in \Z^+\), the set \(S_n = \{x \in X : \abs{f(x)} > 1 / n\}\) is finite and \(\#(S_n) \leq Mn\).

  Suppose for sake of contradiction that \(S_n\) is infinite.
  Then we can have a finite set \(S \subseteq S_n\) where \(\#(S) > (M + 1)n\).
  Since \(S\) is finite, we have \(\sum_{x \in S} \abs{f(x)} \leq M\).
  Since \(S \subseteq S_n\), we have \(\abs{f(x)} > 1 / n\) for every \(x \in S\).
  But now we have
  \[
    M \geq \sum_{x \in S} \abs{f(x)} > \dfrac{(M + 1)n}{n} = M + 1,
  \]
  a contradiction.
  Thus \(S_n\) must be finite.

  Now suppose for sake of contradiction \(\#(S_n) > Mn\).
  Again we have
  \[
    M \geq \sum_{x \in S_n} \abs{f(x)} > \dfrac{Mn}{n} = M,
  \]
  a contradiction.
  Thus \(\#(S_n) \leq Mn\).

  Let \(x \in X\) where \(f(x) \neq 0\).
  If \(x\) does not exist, then we have \(\{x \in X : f(x) \neq 0\} = \emptyset\) which is at most countable.
  So suppose that such \(x\) exists.
  Since \(\abs{f(x)} \in \R^+\), by \cref{5.4.12} we have
  \begin{align*}
             & \exists N \in \Z^+ : \dfrac{1}{\abs{f(x)}} < N                                                           \\
    \implies & \abs{f(x)} > \dfrac{1}{N}                                                                                \\
    \implies & x \in S_N                                                     &  & \text{(by the definition of \(S_N\))} \\
    \implies & x \in \bigcup_{n \in \Z^+} S_n                                &  & \text{(by \cref{3.11})}               \\
    \implies & \{x \in X : f(x) \neq 0\} \subseteq \bigcup_{n \in \Z^+} S_n. &  & \by{3.1.15}
  \end{align*}
  Since \(\forall n \in \Z^+\), \(S_n\) is finite, by \cref{8.1.9} we know that \(\bigcup_{n \in \Z^+} S_n\) is at most countable.
  Since \(\{x \in X : f(x) \neq 0\} \subseteq \bigcup_{n \in \Z^+} S_n\), by \cref{8.1.7} we know that \(\{x \in X : f(x) \neq 0\}\) is at most countable.
\end{proof}

\begin{note}
  Because of \cref{8.2.5}, we can define the value of \(\sum_{x \in X} f(x)\) for any absolutely convergent series on an uncountable set \(X\) by the formula
  \[
    \sum_{x \in X} \coloneqq \sum_{x \in X : f(x) \neq 0} f(x),
  \]
  since we have replaced a sum on an uncountable set \(X\) by a sum on the at most countable set \(\{x \in X : f(x) \neq 0\}\).
  (If the former sum is absolutely convergent, then the latter one is also.)
  \cref{8.2.4} is consistent with the definitions we already have for series on countable sets (\cref{8.2.1}).
\end{note}

\begin{prop}[Absolutely convergent series laws]\label{8.2.6}
  Let \(X\) be an arbitrary set (possibly uncountable), and let \(f : X \to \R\) and \(g : X \to \R\) be functions such that the series \(\sum_{x \in X} f(x)\) and \(\sum_{x \in X} g(x)\) are both absolutely convergent.
  \begin{enumerate}
    \item The series \(\sum_{x \in X} (f(x) + g(x))\) is absolutely convergent, and
          \[
            \sum_{x \in X} (f(x) + g(x)) = \sum_{x \in X} f(x) + \sum_{x \in X} g(x).
          \]
    \item If \(c\) is a real number, then \(\sum_{x \in X} cf(x)\) is absolutely convergent, and
          \[
            \sum_{x \in X} cf(x) = c \sum_{x \in X} f(x).
          \]
    \item If \(X = X_1 \cup X_2\) for some disjoint sets \(X_1\) and \(X_2\), then \(\sum_{x \in X_1} f(x)\) and \\
          \(\sum_{x \in X_2} f(x)\) are absolutely convergent, and
          \[
            \sum_{x \in X_1 \cup X_2} f(x) = \sum_{x \in X_1} f(x) + \sum_{x \in X_2} f(x).
          \]
          Conversely, if \(h : X \to \R\) is such that \(\sum_{x \in X_1} h(x)\) and \(\sum_{x \in X_2} h(x)\) are absolutely convergent, then \(\sum_{x \in X_1 \cup X_2} h(x)\) is also absolutely convergent, and
          \[
            \sum_{x \in X_1 \cup X_2} h(x) = \sum_{x \in X_1} h(x) + \sum_{x \in X_2} h(x).
          \]
    \item If \(Y\) is another set, and \(\phi : Y \to X\) is a bijection, then \(\sum_{y \in Y} f(\phi(y))\) is absolutely convergent, and
          \[
            \sum_{y \in Y} f(\phi(y)) = \sum_{x \in X} f(x).
          \]
  \end{enumerate}
\end{prop}

\begin{proof}{(a)}
  Suppose that \(X\) is a set and \(f : X \to \R, g : X \to \R\) are functions such that \(\sum_{x \in X} f(x)\) and \(\sum_{x \in X} g(x)\) are both absolutely convergent.
  By \cref{7.1.11}(f) we already show that the statement is true when \(X\) is finite.
  So suppose that \(X\) is infinite.

  We first show that \(\sum_{x \in X} (f(x) + g(x))\) is absolutely convergent.
  Since \(\sum_{x \in X} f(x)\) and \(\sum_{x \in X} g(x)\) are both absolutely convergent, by \cref{8.2.4} \(\exists N, M \in \R\) such that
  \[
    N = \sup\Bigg\{\sum_{x \in A} \abs{f(x)} : A \subseteq X, A \text{ finite}\Bigg\} < \infty
  \]
  and
  \[
    M = \sup\Bigg\{\sum_{x \in A} \abs{g(x)} : A \subseteq X, A \text{ finite}\Bigg\} < \infty.
  \]
  Let \(A \subseteq X\) be a finite set.
  Then we have
  \begin{align*}
    \sum_{x \in A} \abs{f(x) + g(x)} & \leq \sum_{x \in A} (\abs{f(x)} + \abs{g(x)})                                             \\
                                     & = \sum_{x \in A} \abs{f(x)} + \sum_{x \in A} \abs{g(x)} &  & \text{(by \cref{7.1.11}(f))} \\
                                     & \leq N + M.
  \end{align*}
  Since \(A\) is arbitrary, we have
  \[
    \sup\Bigg\{\sum_{x \in A} \abs{f(x) + g(x)} : A \subseteq X, A \text{ finite}\Bigg\} \leq N + M < \infty.
  \]
  Thus by \cref{8.2.4} \(\sum_{x \in X} \big(f(x) + g(x)\big)\) is absolutely convergent.

  Now we show that \(\sum_{x \in X} (f(x) + g(x)) = \sum_{x \in X} f(x) + \sum_{x \in X} g(x)\).
  If \(X\) is at most countable, then the statement follows by \cref{ac:8.2.1}(a).
  So suppose that \(X\) is uncountable.
  Let \(X_f = \{x \in X : f(x) \neq 0\}\), \(X_g = \{x \in X : g(x) \neq 0\}\) and \(X_h = \{x \in X : f(x) + g(x) \neq 0\}\) be sets.
  Then by \cref{8.2.5} we know that \(X_f\), \(X_g\) and \(X_h\) are at most countable.
  Since
  \begin{align*}
             & \forall x \in X_f \setminus X_g \\
    \implies & f(x) \neq 0 \land g(x) = 0      \\
    \implies & f(x) + g(x) \neq 0              \\
    \implies & x \in X_h,
  \end{align*}
  we know that \(X_f \setminus X_g \subseteq X_h\).
  Similarly we have \(X_g \setminus X_f \subseteq X_h\).
  Then we have
  \begin{align*}
             & \forall x \in X_h                                                                                  \\
    \implies & f(x) + g(x) \neq 0                                                                                 \\
    \implies & f(x) \neq -g(x)                                                                                    \\
    \implies & \big(f(x) \neq 0 \land g(x) = 0\big) \lor \big(f(x) = 0 \land g(x) \neq 0\big)                     \\
             & \lor \big(f(x) \neq -g(x) \land f(x) \neq 0 \land g(x) \neq 0\big)                                 \\
    \implies & (x \in X_f \setminus X_g) \lor (x \in X_g \setminus X_f) \lor (x \in X_f \cap X_g \land x \in X_h) \\
    \implies & \big(x \in (X_f \setminus X_g) \cup (X_g \setminus X_f) \cup (X_f \cap X_g)\big)                   \\
             & \land \big(x \in (X_f \setminus X_g) \cup (X_g \setminus X_f) \cup X_h\big)                        \\
    \implies & (x \in X_f \cup X_g) \land (x \in X_h)                                                             \\
    \implies & x \in X_f \cup X_g
  \end{align*}
  and \(X_h \subseteq X_f \cup X_g\).
  By \cref{ac:8.1.1} we know that \(X_f \cup X_g\) is at most countable.
  By \cref{8.1.7} we know that \(X_f \cup X_g \setminus X_f\), \(X_f \cup X_g \setminus X_g\), \(X_f \cup X_g \setminus X_h\) are at most countable.
  Thus we have
  \begin{align*}
     & \sum_{x \in X} \big(f(x) + g(x)\big)                                                                                                                                       \\
     & = \sum_{x \in X_h} \big(f(x) + g(x)\big)                                       &                                                          & \by{8.2.5}                     \\
     & = \sum_{x \in X_h} \big(f(x) + g(x)\big)                                                                                                                                   \\
     & \quad + \sum_{x \in (X_f \cup X_g) \setminus X_h} \big(f(x) + g(x)\big)        & (x \notin X_h \iff f(x) + g(x) = 0)                                                       \\
     & = \sum_{x \in X_f \cup X_g} \big(f(x) + g(x)\big)                              &                                                          & \text{(by \cref{ac:8.2.1}(c))} \\
     & = \sum_{x \in X_f \cup X_g} f(x) + \sum_{x \in X_f \cup X_g} g(x)              &                                                          & \text{(by \cref{ac:8.2.1}(a))} \\
     & = \sum_{x \in X_f} f(x) + \sum_{x \in (X_f \cup X_g) \setminus X_f} f(x)       &                                                          & \text{(by \cref{ac:8.2.1}(c))} \\
     & \quad + \sum_{x \in X_g} g(x) + \sum_{x \in (X_f \cup X_g) \setminus X_g} g(x) &                                                          & \text{(by \cref{ac:8.2.1}(c))} \\
     & = \sum_{x \in X_f} f(x) + \sum_{x \in X_g} g(x)                                & (x \notin X_f \iff f(x) = 0, x \notin X_g \iff g(x) = 0)                                  \\
     & = \sum_{x \in X} f(x) + \sum_{x \in X} g(x).                                   &                                                          & \by{8.2.5}
  \end{align*}
\end{proof}

\begin{proof}{(b)}
  Suppose that \(c \in \R\), \(X\) is a set and \(f : X \to \R\) is a function such that \(\sum_{x \in X} f(x)\) is absolutely convergent.
  Since \(\sum_{x \in X} f(x)\) is absolutely convergent, by \cref{8.2.4} \(\exists N \in \R\) such that
  \[
    N = \sup\Bigg\{\sum_{x \in A} \abs{f(x)} : A \subseteq X, A \text{ finite}\Bigg\} < \infty.
  \]
  Let \(A \subseteq X\) be a finite set.
  Then we have
  \begin{align*}
    \sum_{x \in A} \abs{cf(x)} & = \sum_{x \in A} \abs{c}\abs{f(x)}                                   \\
                               & = \abs{c}\sum_{x \in A} \abs{f(x)} &  & \text{(by \cref{7.1.11}(g))} \\
                               & \leq \abs{c} N.
  \end{align*}
  Since \(A\) is arbitrary, we have
  \[
    \sup\Bigg\{\sum_{x \in A} \abs{cf(x)} : A \subseteq X, A \text{ finite}\Bigg\} \leq \abs{c}N < \infty.
  \]
  Thus by \cref{8.2.4} \(\sum_{x \in X} cf(x)\) is absolutely convergent.

  Now we show that \(\sum_{x \in X} cf(x) = c \sum_{x \in X} f(x)\).
  If \(X\) is at most countable, then the statement follows by \cref{ac:8.2.1}(b).
  So suppose that \(X\) is uncountable.
  Let \(X_f = \{x \in X : f(x) \neq 0\}\) and \(X_h = \{x \in X : cf(x) \neq 0\}\) be two sets.
  By \cref{8.2.5} we know that both \(X_f\) and \(X_h\) are at most countable.
  \begin{itemize}
    \item If \(c = 0\), then \(X_h = \emptyset\) and we have
          \begin{align*}
            \sum_{x \in X} 0f(x) & = \sum_{x \in X_h} 0f(x) &  & \by{8.2.5}                   \\
                                 & = 0                      &  & \text{(by \cref{7.1.11}(a))} \\
                                 & = 0 \sum_{x \in X} f(x).
          \end{align*}
    \item If \(c \neq 0\), then we have \(X_f = X_h\) since
          \[
            \forall x, x \in X_f \iff f(x) \neq 0 \iff cf(x) \neq 0 \iff x \in X_h.
          \]
          Thus
          \begin{align*}
            \sum_{x \in X} cf(x) & = \sum_{x \in X_h} cf(x)  &  & \by{8.2.5}                     \\
                                 & = c \sum_{x \in X_h} f(x) &  & \text{(by \cref{ac:8.2.1}(b))} \\
                                 & = c \sum_{x \in X_f} f(x)                                     \\
                                 & = c \sum_{x \in X} f(x).  &  & \by{8.2.5}
          \end{align*}
  \end{itemize}
  From all cases above we conclude that \(\sum_{x \in X} cf(x) = c \sum_{x \in X} f(x)\).
\end{proof}

\begin{proof}{(c)}
  If \(X\) is at most countable, then the statements follow by \cref{ac:8.2.1}(c).
  So suppose that \(X\) is uncountable.

  We first show that if \(X = X_1 \cup X_2\), \(X_1 \cap X_2 = \emptyset\), then \(\sum_{x \in X_1} f(x)\) and \(\sum_{x \in X_2} f(x)\) is absolutely convergent.
  Since \(\sum_{x \in X} f(x)\) is absolutely convergent, by \cref{8.2.4} \(\exists N \in \R\) such that
  \[
    N = \sup\Bigg\{\sum_{x \in A} \abs{f(x)} : A \subseteq X, A \text{ finite}\Bigg\} < \infty.
  \]
  Let \(A_1 \subseteq X_1, A_2 \subseteq X_2\) and both \(A_1, A_2\) are finite.
  Then we have
  \begin{align*}
     & \sum_{x \in A_1} \abs{f(x)} \leq N, \\
     & \sum_{x \in A_2} \abs{f(x)} \leq N.
  \end{align*}
  Since \(A_1, A_2\) are arbitrary, we have
  \[
    \sup\Bigg\{\sum_{x \in A} \abs{f(x)} : A \subseteq X_1, A \text{ finite}\Bigg\} \leq N < \infty
  \]
  and
  \[
    \sup\Bigg\{\sum_{x \in A} \abs{f(x)} : A \subseteq X_2, A \text{ finite}\Bigg\} \leq N < \infty.
  \]
  Thus by \cref{8.2.4} both \(\sum_{x \in X_1} f(x)\) and \(\sum_{x \in X_2} f(x)\) are absolutely convergent.

  Next we show that if \(X = X_1 \cup X_2\), \(X_1 \cap X_2 = \emptyset\), then
  \[
    \sum_{x \in X_1 \cup X_2} f(x) = \sum_{x \in X_1} f(x) + \sum_{x \in X_2} f(x).
  \]
  Let \(X_f = \{x \in X : f(x) \neq 0\}, X_{f_1} = \{x \in X_1 : f(x) \neq 0\}, X_{f_2} = \{x \in X_2 : f(x) \neq 0\}\) be sets.
  Clearly we have \(X_{f_1} \cup X_{f_2} = X_f\) and \(X_{f_1} \cap X_{f_2} = \emptyset\).
  By \cref{8.2.5} we know that \(X_f, X_{f_1}, X_{f_2}\) are at most countable.
  Then we have
  \begin{align*}
    \sum_{x \in X} f(x) & = \sum_{x \in X_f} f(x)                                 &  & \by{8.2.5}                     \\
                        & = \sum_{x \in X_{f_1} \cup X_{f_2}} f(x)                                                    \\
                        & = \sum_{x \in X_{f_1}} f(x) + \sum_{x \in X_{f_2}} f(x) &  & \text{(by \cref{ac:8.2.1}(c))} \\
                        & = \sum_{x \in X_1} f(x) + \sum_{x \in X_2} f(x).        &  & \by{8.2.5}
  \end{align*}

  Finally we show that if \(X_1 \cup X_2 \subseteq X\), \(X_1 \cap X_2 = \emptyset\), \(h : X \to \R\) is a function such that \(\sum_{x \in X_1} h(x)\) and \(\sum_{x \in X_2} h(x)\) are absolutely convergent, then \(\sum_{x \in X_1 \cup X_2} h(x)\) is also absolutely convergent.
  Since \(\sum_{x \in X_1} h(x)\) and \(\sum_{x \in X_2} h(x)\) are absolutely convergent, by \cref{8.2.4} \(\exists N, M \in \R\) such that
  \[
    N = \sup\Bigg\{\sum_{x \in A} \abs{h(x)} : A \subseteq X_1, A \text{ finite}\Bigg\} < \infty
  \]
  and
  \[
    M = \sup\Bigg\{\sum_{x \in A} \abs{h(x)} : A \subseteq X_2, A \text{ finite}\Bigg\} < \infty.
  \]
  Let \(A \subseteq X_1 \cup X_2\) be a finite set.
  Let \(A_1 = A \cap X_1\) and \(A_2 = A \cap X_2\).
  Clearly we have
  \begin{align*}
     & A_1 \cap A_2 = \emptyset, \\
     & A_1 \cup A_2 = A,         \\
     & A_1 \subseteq X_1,        \\
     & A_2 \subseteq X_2,        \\
     & A_1 \text{ is finite},    \\
     & A_2 \text{ is finite}.
  \end{align*}
  Then we have
  \begin{align*}
    \sum_{x \in A} \abs{h(x)} & = \sum_{x \in A_1 \cup A_2} \abs{h(x)}                      \\
                              & = \sum_{x \in A_1} \abs{h(x)} + \sum_{x \in A_2} \abs{h(x)} \\
                              & \leq N + M.
  \end{align*}
  Since \(A\) is arbitrary, we have
  \[
    \sup\Bigg\{\sum_{x \in A} \abs{h(x)} : A \subseteq X_1 \cup X_2, A \text{ finite}\Bigg\} \leq N + M < \infty
  \]
  Thus by \cref{8.2.4} \(\sum_{x \in X_1 \cup X_2} h(x)\) is absolutely convergent.
  Since \(\sum_{x \in X_1 \cup X_2} h(x)\) is absolutely convergent, from the proof above we have
  \[
    \sum_{x \in X_1 \cup X_2} h(x) = \sum_{x \in X_1} h(x) + \sum_{x \in X_2} h(x).
  \]
\end{proof}

\begin{proof}{(d)}
  If \(X\) is at most countable, then the statements follow by \cref{ac:8.2.1}(d).
  So suppose that \(X\) is uncountable.

  We first show that \(\sum_{y \in Y} f\big(\phi(y)\big)\) is absolutely convergent.
  Since \(\sum_{x \in X} f(x)\) is absolutely convergent, by \cref{8.2.4}, \(\exists N \in \R\) such that
  \[
    N = \sup\Bigg\{\sum_{x \in A} \abs{f(x)} : A \subseteq X, A \text{ finite}\Bigg\} < \infty
  \]
  Let \(A \subseteq Y\) be a finite set.
  Then we have
  \begin{align*}
    \sum_{y \in A} \abs{f\big(\phi(y)\big)} & = \sum_{x \in \phi(A)} \abs{f(x)} &  & \text{(by \cref{7.1.11}(c))}   \\
                                            & \leq N.                           &  & \text{(\(\phi(A)\) is finite)}
  \end{align*}
  Since \(A\) is arbitrary, we have
  \[
    \sup\Bigg\{\sum_{y \in A} \abs{f\big(\phi(x)\big)} : A \subseteq Y, A \text{ finite}\Bigg\} \leq N < \infty.
  \]
  Thus by \cref{8.2.4} \(\sum_{y \in Y} f\big(\phi(y)\big)\) is absolutely convergent.

  Now we show that \(\sum_{y \in Y} f\big(\phi(y)\big) = \sum_{x \in X} f(x)\).
  Let \(X_f = \{x \in X : f(x) \neq 0\}\) and \(Y_f = \{y \in Y : f\big(\phi(y)\big) \neq 0\}\) be sets.
  By \cref{8.2.5} we know that \(X_f\) and \(Y_f\) are at most countable.
  Clearly \(\phi\) is a bijective between \(X_f\) and \(Y_f\).
  Thus we have
  \begin{align*}
    \sum_{y \in Y} f\big(\phi(y)\big) & = \sum_{y \in Y_f} f\big(\phi(y)\big) &  & \by{8.2.5}                     \\
                                      & = \sum_{x \in X_f} f(x)               &  & \text{(by \cref{ac:8.2.1}(d))} \\
                                      & = \sum_{x \in X} f(x).                &  & \by{8.2.5}
  \end{align*}
\end{proof}

\begin{lem}\label{8.2.7}
  Let \(\sum_{n = 0}^\infty a_n\) be a series of real numbers which is conditionally convergent, but not absolutely convergent.
  Define the sets \(A_+ \coloneqq \{n \in \N : a_n \geq 0\}\) and \(A_- \coloneqq \{n \in \N : a_n < 0\}\), thus \(A_+ \cup A_- = \N\) and \(A_+ \cap A_- = \emptyset\).
  Then both of the series \(\sum_{n \in A_+} a_n\) and \(\sum_{n \in A_-} a_n\) are not absolutely convergent.
\end{lem}

\begin{proof}
  Suppose for sake of contradiction that at least one of the series \(\sum_{n \in A_+} a_n\) and \(\sum_{n \in A_-} a_n\) is absolutely convergent.
  Let \(b_n = \max(a_n, 0)\) and \(c_n = -\min(a_n, 0)\).
  Then we have \(a_n = b_n - c_n\) and
  \begin{align*}
    \sum_{n = 0}^\infty a_n & = \sum_{n = 0}^\infty b_n - c_n                                                                         \\
                            & = \sum_{n = 0}^\infty b_n - \sum_{n = 0}^\infty c_n                   &  & \text{(by \cref{7.2.14}(a))} \\
                            & = \sum_{n = 0}^\infty \max(a_n, 0) + \sum_{n = 0}^\infty \min(a_n, 0)                                   \\
                            & = \sum_{n \in A} \max(a_n, 0) + \sum_{n \in A} \min(a_n, 0)           &  & \by{8.2.1}                   \\
                            & = \sum_{n \in A_+} a_n + \sum_{n \in A_-} a_n.                        &  & \by{8.2.5}
  \end{align*}
  Thus \(\sum_{n \in A_+} a_n\) and \(\sum_{n \in A_-} a_n\) converges.
  Since \(\sum_{n \in A_+} a_n\) converges and
  \[
    \sum_{n \in A_+} \abs{a_n} = \sum_{n \in A_+} a_n,
  \]
  we know that \(\sum_{n \in A_+} a_n\) is absolutely converges.
  Since
  \begin{align*}
    \sum_{n \in A_-} \abs{a_n} & = \sum_{n = 0}^\infty \abs{\min(a_n, 0)} &  & \by{8.2.5}                   \\
                               & = \sum_{n = 0}^\infty -\min(a_n, 0)                                        \\
                               & = -\sum_{n = 0}^\infty \min(a_n, 0)      &  & \text{(by \cref{7.2.14}(b))} \\
                               & = -\sum_{n \in A_-} a_n,                 &  & \by{8.2.5}
  \end{align*}
  we know that \(\sum_{n \in A_-} \abs{a_n}\) is absolutely convergent.
  But by \cref{8.2.6}(c) we have
  \[
    \sum_{n \in A_+} a_n + \sum_{n \in A_-} a_n = \sum_{n \in A} a_n
  \]
  and \(\sum_{n \in A} a_n\) is absolutely convergent, a contradiction.
  Thus both \(\sum_{n \in A_+} a_n\) and \(\sum_{n \in A_-} a_n\) are not absolutely convergent.
\end{proof}

\begin{note}
  \cref{8.2.8} is done by Georg Riemann (1826 -- 1866), which asserts that a series which converges conditionally but not absolutely can be rearranged to converge to any value one pleases!
\end{note}

\begin{thm}\label{8.2.8}
  Let \(\sum_{n = 0}^\infty a_n\) be a series which is conditionally convergent, but not absolutely convergent, and let \(L\) be any real number.
  Then there exists a bijection \(f : \N \to \N\) such that \(\sum_{m = 0}^\infty a_{f(m)}\) converges conditionally to \(L\).
\end{thm}

\begin{proof}
  Let \(A_+\) and \(A_-\) be the sets in \cref{8.2.7};
  from \cref{8.2.7} we know that \(\sum_{n \in A_+} a_n\) and \(\sum_{n \in A_-} a_n\) both fail to be absolutely convergent.
  In particular \(A_+\) and \(A_-\) are infinite.
  (If \(A_-\) is finite, then \(\sum_{n \in A_-} a_n\) is absolutely convergent.
  If \(A_+\) is finite, then \(\sum_{n \in A_+} a_n\) is also absolutely convergent.)
  By \cref{8.1.5} we can then find increasing bijections \(f_+ : \N \to A_+\) and \(f_- : \N \to A_-\).
  Thus the sums \(\sum_{m = 0}^\infty a_{f_+(m)}\) and \(\sum_{m = 0}^\infty a_{f_-(m)}\) both fail to be absolutely convergent (first by \cref{8.2.1} then by \cref{8.2.7}).
  The plan shall be to select terms from the divergent series \(\sum_{m = 0}^\infty a_{f_+(m)}\) and \(\sum_{m = 0}^\infty a_{f_-(m)}\) in a well-chosen order in order to keep their difference converging towards \(L\).

  We define the sequence \(n_0, n_1, n_2, \dots\) of natural numbers recursively as follows.
  Suppose that \(j\) is a natural number, and that \(n_i\) has already been defined for all \(i < j\) (this is vacuously true if \(j = 0\)).
  We then define \(n_j\) by the following rule:
  \begin{enumerate}[label=(\Roman*)]
    \item If \(\sum_{0 \leq i < j} a_{n_i} < L\), then we set
          \[
            n_j \coloneqq \min\{n \in A_+ : n \neq n_i \text{ for all } i < j\}.
          \]
    \item If instead \(\sum_{0 \leq i < j} a_{n_i} \geq L\), then we set
          \[
            n_j \coloneqq \min\{n \in A_- : n \neq n_i \text{ for all } i < j\}.
          \]
  \end{enumerate}
  Note that this recursive definition is well-defined because \(A_+\) and \(A_-\) are infinite, and so the sets \(\{n \in A_+ : n \neq n_i \text{ for all } i < j\}\) and \(\{n \in A_- : n \neq n_i \text{ for all } i < j\}\) are never empty.
  (Intuitively, we add a non-negative number to the series whenever the partial sum is too low, and add a negative number when the sum is too high.)
  One can then verify the following claims:
  \begin{itemize}
    \item The map \(j \mapsto n_j\) is injective.
          This is true since
          \begin{align*}
                     & \forall j_1, j_2 \in \N, j_1 \neq j_2 \\
            \implies & j_1 < j_2 \lor j_1 > j_2              \\
            \implies & n_{j_1} \neq n_{j_2}.
          \end{align*}
    \item Case I occurs an infinite number of times, and Case II also occurs an infinite number of times.
          We prove this by contradiction.
          Suppose for sake of contradiction that case I occurs only finite number of times.
          Then we have
          \[
            \Bigg(\sum_{0 \leq i < j} a_{n_i} < L\Bigg) \land \Bigg(\sum_{0 \leq i < j} a_{n_i} + a_{n_j} + \sum_{i > j} a_{n_i} \geq L\Bigg)
          \]
          where \(j\) is the last time case I occurs.
          Since \(\sum_{0 \leq i < j} a_{n_i} + a_{n_j}\) is finite, \(\sum_{i > j} a_{n_i}\) have a lower bound.
          Since all \(i > j\) are cases II, \(\sum_{i > j} a_{n_i}\) is decreasing.
          Since \(\sum_{i > j} a_{n_i}\) is decreasing and has lower bound, by \cref{ac:6.3.1} \(\sum_{i > j} a_{n_i}\) is convergent.
          But this means \(\sum_{n \in A_-} a_n\) is absolutely convergent, a contradiction.
          Thus case I occurs infinite number of times.
          Similar proof show that case II also occurs infinite number of times.
    \item The map \(j \mapsto n_j\) is surjective.
          We know that \(\forall n \in \N\), either \(n \in A_+\) or \(n \in A_-\).
          If \(n \in A_+\) and there is no \(j \mapsto n\), then \(\forall n' > n\) there must also have no \(j \mapsto n'\), otherwise by definition we must have \(n = \min\{n \in A_+ : n \neq n_i \text{ for all } i < j\}\).
          But then case I only occur finite number of times, a contradiction.
          Thus \(\exists j \mapsto n\).
          Similar argument show that if \(n \in A_-\) then \(\exists j \mapsto n\).
          Thus \(j \mapsto n_j\) is surjective.
    \item We have \(\lim_{j \to \infty} a_{n_j} = 0\).
          By \cref{7.2.6} we have \(\lim_{j \to \infty} a_j = 0\).
          This means
          \[
            \forall \varepsilon \in \R^+, \exists N \in \N : \forall j \geq N, \abs{a_j - 0} \leq \varepsilon.
          \]
          Thus the set \(E = \{j \in \N : a_j - 0 > \varepsilon\}\) is finite.
          Since \(j \to n_j\) is bijective, we know that the set \(E' = \{n_j \in \N : j \in E\}\) is also finite.
          Let \(M = \max(E')\).
          Then we have
          \[
            \forall n_j \geq M, \abs{a_{n_j} - 0} \leq \varepsilon.
          \]
          Thus \(\lim_{j \to \infty} a_{n_j} = 0\).
    \item We have \(\lim_{j \to \infty} \sum_{0 \leq i \leq j} a_{n_i} = L\).
          Since \(\lim_{j \to \infty} a_{n_j} = 0\), we have
          \[
            \forall \varepsilon \in \R^+, \exists N \in \N : \forall j \geq N, \abs{a_{n_j} - 0} \leq \varepsilon.
          \]
          Let \(K\) be the set
          \[
            K = \Bigg\{k \in \N : (k \geq j) \land \Bigg(\sum_{i = 0}^k a_{n_i} < L\Bigg) \land \Bigg(\sum_{i = 0}^{k + 1} a_{n_i} \geq L\Bigg)\Bigg\}.
          \]
          We know that \(K \neq \emptyset\) since case II occurs infinite number of times.
          Let \(k = \min(K)\).
          Such \(k\) is well-defined by well ordering principle (\cref{8.1.4}).
          Now we show that for every \(p \in \N\), we have
          \[
            L - \varepsilon \leq \sum_{i = 0}^{k + p} a_{n_i} \leq L + \varepsilon.
          \]
          We use induction on \(p\).
          For \(p = 0\), we have
          \begin{align*}
                     & \sum_{i = 0}^{k + p} a_{n_i} = \sum_{i = 0}^k a_{n_i}                                                                                                \\
            \implies & \sum_{i = 0}^{k + 1} a_{n_i} \geq L                                   &                                        & \text{(by the definition of \(k\))} \\
            \implies & \sum_{i = 0}^k a_{n_i} + a_{n_{k + 1}} \geq L                                                                                                        \\
            \implies & \sum_{i = 0}^k a_{n_i} \geq L - a_{n_{k + 1}} \geq L - \varepsilon    & (\abs{a_{n_{k + 1}}} \leq \varepsilon)                                       \\
            \implies & L + \varepsilon \geq L > \sum_{i = 0}^k a_{n_i} \geq L - \varepsilon. &                                        & \text{(by the definition of \(k\))}
          \end{align*}
          Thus the base case holds.
          Suppose inductively that for some \(p \geq 0\) the statement is true.
          Then we need to show that for \(p + 1\) the statement is also true.
          We split into two cases:
          \begin{itemize}
            \item If \(a_{n_{p + 1}} \geq 0\), then this means case I happened.
                  Thus we have
                  \begin{align*}
                             & (0 \leq a_{n_{p + 1}} \leq \varepsilon) \land \Bigg(\sum_{i = 0}^{k + p} a_{n_i} < L\Bigg)                      &  & \text{(case I)} \\
                    \implies & (0 \leq a_{n_{p + 1}} \leq \varepsilon) \land \Bigg(L - \varepsilon \leq \sum_{i = 0}^{k + p} a_{n_i} < L\Bigg) &  & \byIH           \\
                    \implies & L - \varepsilon \leq \sum_{i = 0}^{k + p} a_{n_i} + a_{n_{p + 1}} \leq L + \varepsilon                                               \\
                    \implies & L - \varepsilon \leq \sum_{i = 0}^{k + p + 1} a_{n_i} \leq L + \varepsilon.
                  \end{align*}
            \item If \(a_{n_{p + 1}} < 0\), then this means case II happened.
                  Thus we have
                  \begin{align*}
                             & (-\varepsilon \leq a_{n_{p + 1}} < 0) \land \Bigg(L \leq \sum_{i = 0}^{k + p} a_{n_i}\Bigg) &  & \text{(case II)} \\
                    \implies & (-\varepsilon \leq a_{n_{p + 1}} < 0)                                                                             \\
                             & \land \Bigg(L \leq \sum_{i = 0}^{k + p} a_{n_i} \leq L + \varepsilon\Bigg)                  &  & \byIH            \\
                    \implies & L - \varepsilon \leq \sum_{i = 0}^{k + p} a_{n_i} + a_{n_{p + 1}} \leq L + \varepsilon                            \\
                    \implies & L - \varepsilon \leq \sum_{i = 0}^{k + p + 1} a_{n_i} \leq L + \varepsilon.
                  \end{align*}
          \end{itemize}
          From all cases above we conclude that \(L - \varepsilon \leq \sum_{i = 0}^{k + p + 1} a_{n_i} \leq L + \varepsilon\).
          This closes the induction.
          This means \(\forall p \geq 0\), we have
          \[
            L - \varepsilon \leq \sum_{i = 0}^{k + p} \leq L + \varepsilon \iff \abs{\sum_{i = 0}^{k + p} - L} \leq \varepsilon.
          \]
          Rewritting with \(q = k + p\), we have
          \[
            \forall \varepsilon \in \R^+, \exists k \geq 0 : \forall q \geq k, \abs{\sum_{i = 0}^q - L} \leq \varepsilon.
          \]
          Thus \(\lim_{q \to \infty} \sum_{i = 0}^q a_{n_i} = L\).
  \end{itemize}
  The claim then follows by setting \(f(i) \coloneqq n_i\) for all \(i \in \N\).
\end{proof}

\exercisesection

\begin{ex}\label{ex:8.2.1}
  Prove \cref{8.2.3}.
\end{ex}

\begin{proof}
  See \cref{8.2.3}.
\end{proof}

\begin{ex}\label{ex:8.2.2}
  Prove \cref{8.2.5}.
\end{ex}

\begin{proof}
  See \cref{8.2.5}.
\end{proof}

\begin{ex}\label{ex:8.2.3}
  Prove \cref{8.2.6}.
\end{ex}

\begin{proof}
  See \cref{8.2.6}.
\end{proof}

\begin{ex}\label{ex:8.2.4}
  Prove \cref{8.2.7}.
\end{ex}

\begin{proof}
  See \cref{8.2.7}.
\end{proof}

\begin{ex}\label{ex:8.2.5}
  Explain the gaps marked (why?) in the proof of \cref{8.2.8}.
\end{ex}

\begin{proof}
  See \cref{8.2.8}.
\end{proof}

\begin{ex}\label{ex:8.2.6}
  Let \(\sum_{n = 0}^\infty a_n\) be a series which is conditionally convergent, but not absolutely convergent.
  Show that there exists a bijection \(f : \N \to \N\) such that \(\sum_{m = 0}^\infty a_{f(m)}\) diverges to \(+\infty\), or more precisely that
  \[
    \liminf_{N \to \infty} \sum_{m = 0}^N a_{f(m)} = \limsup_{N \to \infty} \sum_{m = 0}^N a_{f(m)} = +\infty.
  \]
  (Of course, a similar statement holds with \(+\infty\) replaced by \(-\infty\).)
\end{ex}

\begin{proof}
  Let \(A_+\) and \(A_-\) defined as \cref{8.2.7}.
  In \cref{8.2.8} we know that both \(A_+\) and \(A_-\) are countable, and there exist two increasing bijections \(f_+ : \N \to A_+\) and \(f_- : \N \to A_-\).
  We know that both \(\sum_{m = 0}^\infty a_{f_+(m)}\) and \(\sum_{m = 0}^\infty a_{f_-(m)}\) fail to be absolutely convergent.

  We first show that there exists a bijection \(f : \N \to \N\) such that \(\sum_{m = 0}^\infty a_{f(m)}\) diverges to \(+\infty\).
  Let \(L_0 = 0\).
  Suppose that \(j \in \N\), and \(n_i\) has been defined for all \(i < j\)
  (this is vacuously true if \(j = 0\)).
  We define \(n_j\) by the following rule:
  \begin{enumerate}[label=(\Roman*)]
    \item If \(\sum_{0 \leq i < j} a_{n_i} < L_j\), then we set
          \begin{align*}
            n_j       & = \min\{n \in A_+ : n \neq n_i \text{ for all } i < j\}; \\
            L_{j + 1} & = L_j.
          \end{align*}
    \item If \(\sum_{0 \leq i < j} a_{n_i} \geq L_j\), then we set
          \begin{align*}
            n_j       & = \min\{n \in A_- : n \neq n_i \text{ for all } i < j\}; \\
            L_{j + 1} & = L_j + 1.
          \end{align*}
  \end{enumerate}
  Now we verify the following claims:
  \begin{itemize}
    \item The map \(j \mapsto n_j\) is injective.
          Suppose that \(i, j \in \N\) and \(i \neq j\).
          Then by the definition of \(n_i, n_j\) we know that \(n_i \neq n_j\).
          Thus \(j \mapsto n_j\) is injective.
    \item Both Case I and II occur infinite number of times.
          Obviously, at least one case must occur infinite number of times.
          Suppose for sake of contradiction that Case I only occurs finite number of times.
          Let \(j\) be the largest number such that Case I occurs, i.e.,
          \[
            \Bigg(\sum_{0 \leq i < j} a_{n_i} < L_j\Bigg) \land \Bigg(\sum_{0 \leq i \leq j} a_{n_i} \geq L_j\Bigg).
          \]
          Then \(\forall k \in \N\) and \(k > j\), Case II occurs, i.e.,
          \[
            S_k = \sum_{i = 0}^k a_{n_i} \geq L_k.
          \]
          Since Case II occurs, we know that \(S_k\) is decreasing and \(L_k\) is increasing.
          Thus
          \begin{align*}
                     & S_k > S_{k + 1} \geq L_{k + 1} > L_k \geq 0                                           \\
            \implies & (\lim_{k \to \infty} S_k \text{ converges})         &  & \by{ac:6.3.1}                \\
                     & \land (S_k - L_k \geq S_{k + 1} - L_{k + 1} \geq 0)                                   \\
            \implies & \lim_{k \to \infty} S_k - L_k \text{ converges}     &  & \by{ac:6.3.1}                \\
            \implies & \lim_{k \to \infty} L_k \text{ converges}.          &  & \text{(by \cref{6.1.19}(a))}
          \end{align*}
          But since Case II occurs infinite number of times, we know that \(\lim_{k \to \infty} L_k\) diverges to \(\infty\), a contradiction.
          Thus case I must occurs infinite number of times.

          Now Suppose for sake of contradiction that Case II only occurs finite number of times.
          Let \(j\) be the largest number such that Case II occurs, i.e.,
          \[
            \Bigg(\sum_{0 \leq i < j} a_{n_i} \geq L_j\Bigg) \land \Bigg(\sum_{0 \leq i \leq j} a_{n_i} < L_j\Bigg).
          \]
          Then \(\forall k \in \N\) and \(k > j\), Case I occurs, i.e.,
          \[
            S_k = \sum_{i = 0}^k a_{n_i} < L_k.
          \]
          Since Case I occurs, we know that \(S_k\) is increasing.
          Thus
          \begin{align*}
                     & S_k < S_{k + 1} < L_{k + 1} = L_k                                                                                       \\
            \implies & \lim_{k \to \infty} S_k \text{ converges}               &                                 & \by{6.3.8}                  \\
            \implies & \sum_{k = j + 1}^\infty a_{n_k} \text{ converges}       &                                 & \by{7.2.2}                  \\
            \implies & \sum_{k = j + 1}^\infty \abs{a_{n_k}} \text{ converges} & (\forall k > j, a_{n_k} \geq 0)                               \\
            \implies & \sum_{k \in A_+} \abs{a_k} \text{ converges}            &                                 & \text{(by \cref{8.2.6}(c))}
          \end{align*}
          But we know that \(\sum_{k \in A_+} \abs{a_k}\) is not absolutely convergent, a contradiction.
          Thus case II must occurs infinite number of times.
          We conclude that both Case I and II occur infinite number of times.
    \item The map \(j \mapsto n_j\) is surjective.
          We know that \(\forall n \in \N\), either \(n \in A_+\) or \(n \in A_-\).
          If \(n \in A_+\) and there is no \(j \mapsto n\), then \(\forall n' > n\) there must also have no \(j \mapsto n'\), otherwise by definition we must have \(n = \min\{n \in A_+ : n \neq n_i \text{ for all } i < j\}\).
          But then case I only occur finite number of times, a contradiction.
          Thus \(\exists j \mapsto n\).
          Similar argument show that if \(n \in A_-\) then \(\exists j \mapsto n\).
          Thus \(j \mapsto n_j\) is surjective.
    \item We have \(\limsup_{j \to \infty} \sum_{i = 0}^j a_{n_j}\) diverges to \(\infty\).
          Since Case II occurs infinite number of times, we know that for every \(M \in \R^+\), \(\exists j \geq 0\) such that \(M \leq L_i\).
          This means
          \[
            \exists k \in \N \land k > j : M \leq L_i \leq \sum_{i = 0}^k a_{n_i}.
          \]
          Since this is true for every \(M \in \R^+\), by \cref{ex:6.4.8} we thus have
          \[
            \limsup_{j \to \infty} \sum_{i = 0}^j a_{n_j} = \infty.
          \]
  \end{itemize}
  The claim then follows by setting \(f(i) \coloneqq n_i\) for all \(i \in \N\).
  Similar proof can be used to show the case \(-\infty\).
\end{proof}
\section{Uncountable sets}\label{sec:8.3}

\begin{note}
  It was great shock when Georg Cantor (1845 -- 1918) showed in 1873 that certain sets
  - including the real numbers \(\R\) are in fact uncountable -
  no matter how hard you try, you cannot arrange the real numbers \(\R\) as a sequence \(a_0, a_1, a_2, \dots\).
  (Of course, the real numbers \(\R\) can contain many infinite sequences, e.g., the sequence \(0, 1, 2, 3, 4, \dots\).
  However, what Cantor proved is that no such sequence can ever exhaust the real numbers;
  no matter what sequence of real numbers you choose, there will always be some real numbers that are not covered by that sequence.)
\end{note}

\begin{thm}[Cantor's theorem]\label{8.3.1}
  Let \(X\) be an arbitrary set (finite or infinite).
  Then the sets \(X\) and \(2^X\) cannot have equal cardinality.
\end{thm}

\begin{proof}
  Suppose for sake of contradiction that the sets \(X\) and \(2^X\) had equal cardinality.
  Then there exists a bijection \(f : X \to 2^X\) between \(X\) and the power set of \(X\).
  Now consider the set
  \[
    A \coloneqq \{x \in X : x \notin f(x)\}.
  \]
  Note that this set is well-defined since \(f(x)\) is an element of \(2^X\) and is hence a subset of \(X\).
  Clearly \(A\) is a subset of \(X\), hence is an element of \(2^X\).
  Since \(f\) is a bijection, there must therefore exist \(x \in X\) such that \(f(x) = A\).
  There are now two cases, depending on whether \(x \in A\) or \(x \notin A\).
  If \(x \in A\), then by definition of \(A\) we have \(x \notin f(x)\), hence \(x \notin A\), a contradiction.
  But if \(x \notin A\), then \(x \notin f(x)\), hence by definition of \(A\) we have \(x \in A\), a contradiction.
  Thus in either case we have a contradiction.
\end{proof}

\begin{rmk}\label{8.3.2}
  The reader should compare the proof of Cantor's theorem with the statement of Russell's paradox (\cref{sec:3.2}).
  The point is that a bijection between \(X\) and \(2^X\) would come dangerously close to the concept of a set \(X\) ``containing itself''.
\end{rmk}

\begin{cor}\label{8.3.3}
  \(2^{\N}\) is uncountable.
\end{cor}

\begin{proof}
  By \cref{8.3.1}, \(2^{\N}\) cannot have equal cardinality with \(\N\), hence is either uncountable or finite.
  However, \(2^{\N}\) contains as a subset the set of singletons \(\{\{n\} : n \in \N\}\), which is clearly bijective to \(\N\) and hence countably infinite.
  Thus \(2^{\N}\) cannot be finite (by \cref{3.6.14}), and is hence uncountable.
\end{proof}

\begin{cor}\label{8.3.4}
  \(\R\) is uncountable.
\end{cor}

\begin{proof}
  Let us define the map \(f : 2^{\N} \to \R\) by the formula
  \[
    f(A) \coloneqq \sum_{n \in A} 10^{-n}.
  \]
  Observe that since \(\sum_{n = 0}^\infty 10^{-n}\) is an absolutely convergent series (by \cref{7.3.3}), the series \(\sum_{n \in A} 10^{-n}\) is also absolutely convergent (by \cref{8.2.6}(c)).
  Thus the map \(f\)  is well defined.
  We now claim that \(f\) is injective.
  Suppose for sake of contradiction that there were two distinct sets \(A, B \in 2^{\N}\) such that \(f(A) = f(B)\).
  Since \(A \neq B\), the set \((A \setminus B) \cup (B \setminus A)\) is a non-empty subset of \(\N\).
  By the well-ordering principle (\cref{8.1.4}), we can then define the minimum of this set, say \(n_0 \coloneqq \min(A \setminus B) \cup (B \setminus A)\).
  Thus \(n_0\) either lies in \(A \setminus B\) or \(B \setminus A\).
  By symmetry we may assume it lies in \(A \setminus B\).
  Then \(n_0 \in A\), \(n_0 \notin B\), and for all \(n < n_0\) we either have \(n \in A, B\) or \(n \notin A, B\).
  Thus
  \begin{align*}
    0 & = f(A) - f(B)                                                                                 \\
      & = \sum_{n \in A} 10^{-n} - \sum_{n \in B} 10^{-n}                                             \\
      & = \Bigg(\sum_{n < n_0 : n \in A} 10^{-n} + 10^{-n_0} + \sum_{n > n_0 : n \in A} 10^{-n}\Bigg) \\
      & \quad - \Bigg(\sum_{n < n_0 : n \in B} 10^{-n} + \sum_{n > n_0 : n \in B} 10^{-n}\Bigg)       \\
      & = 10^{-n_0} + \sum_{n > n_0 : n \in A} 10^{-n} - \sum_{n > n_0 : n \in B} 10^{-n}             \\
      & \geq 10^{-n_0} + 0 - \sum_{n > n_0} 10^{-n}                                                   \\
      & \geq 10^{-n_0} - \dfrac{1}{9} 10^{-n_0}                                                       \\
      & > 0,
  \end{align*}
  a contradiction, where we have used the geometric series lemma (\cref{7.3.3}) to sum
  \[
    \sum_{n > n_0} 10^{-n} = \sum_{m = 0}^\infty 10^{-(n_0 + 1 + m)} = 10^{-n_0 - 1} \sum_{m = 0}^\infty 10^{-m} = \dfrac{1}{9} 10^{-n_0}.
  \]
  Thus \(f\) is injective, which means that \(f(2^{\N})\) has the same cardinality as \(2^{\N}\) and is thus uncountable.
  Since \(f(2^{\N})\) is a subset of \(\R\), this forces \(\R\) to be uncountable also (otherwise this would contradict \cref{8.1.7}), and we are done.
\end{proof}

\setcounter{thm}{5}
\begin{rmk}\label{8.3.6}
  \cref{8.3.4} shows that the reals have strictly larger cardinality than the natural numbers (in the sense of \cref{ex:3.6.7}).
  One could ask whether there exist any sets which have strictly larger cardinality than the natural numbers, but strictly smaller cardinality than the reals.
  The \emph{Continuum Hypothesis} asserts that no such sets exist.
  Interestingly, it was shown in separate works of Kurt Gödel (1906 -- 1978) and Paul Cohen (1934 -- 2007) that this hypothesis is independent of the other axioms of set theory;
  it can neither be proved nor disproved in that set of axioms
  (unless those axioms are inconsistent, which is highly unlikely).
\end{rmk}

\exercisesection

\begin{ex}\label{ex:8.3.1}
  Let \(X\) be a finite set of cardinality \(n\).
  Show that \(2^X\) is a finite set of cardinality \(2^n\).
\end{ex}

\begin{proof}
  We use induction on \(n\).
  For \(n = 0\), \(X = \emptyset\) and \(2^{\emptyset} = \{\emptyset\}\).
  Clearly \(\#(2^{\emptyset}) = 1\).
  So the base case holds.
  Suppose inductively that \(\#(2^X) = 2^n\) for some \(n \geq 0\).
  We need to show that for \(n + 1\), \(\#(2^X) = 2^{n + 1}\).
  Since \(\#(X) = n + 1\), we have \(X \neq \emptyset\).
  Let \(x \in X\).
  Then we have
  \begin{align*}
    \#(2^X) & = \#(2^{(X \setminus \{x\}) \cup \{x\}})                                          \\
            & = \#(2^{X \setminus \{x\}}) \times \#(2^{\{x\}}) &  & \text{(by \cref{ex:3.6.6})} \\
            & = 2^n \times \#(2^{\{x\}})                       &  & \byIH                       \\
            & = 2^n \times \#(\{\emptyset, \{x\}\})                                             \\
            & = 2^n \times 2                                                                    \\
            & = 2^{n + 1}.
  \end{align*}
  This closes the induction.
\end{proof}

\begin{ex}\label{ex:8.3.2}
  Let \(A, B, C\) be sets such that \(A \subseteq B \subseteq C\), and suppose that there is a injection \(f : C \to A\).
  Define the sets \(D_0, D_1, D_2, \dots\) recursively by setting \(D_0 \coloneqq B \setminus A\), and then \(D_{n + 1} \coloneqq f(D_n)\) for all natural numbers \(n\).
  Prove that the sets \(D_0, D_1, \dots\) are all disjoint from each other
  (i.e., \(D_n \cap D_m = \emptyset\) whenever \(n \neq m\)).
  Also show that if \(g : A \to B\) is the function defined by setting \(g(x) \coloneqq f^{-1}(x)\) when \(x \in \bigcup_{n = 1}^\infty D_n\), and \(g(x) \coloneqq x\) when \(x \notin \bigcup_{n = 1}^\infty D_n\), then \(g\) does indeed map \(A\) to \(B\) and is a bijection between the two.
  In particular, \(A\) and \(B\) have the same cardinality.
\end{ex}

\begin{proof}
  We first show that \(\forall n, m \in \N\), \(n \neq m \implies D_n \cap D_m = \emptyset\).
  Let \(P(n)\) be the statement ``\(\forall k \in \Z^+ : D_n \cap D_{n + k} = \emptyset\)''.
  We use induction on \(n\) to show that \(P(n)\) is true.
  For \(n = 0\), we have \(D_0 = B \setminus A\) and \(D_k = f(D_{k - 1}) \subseteq A\).
  Thus we have \(D_0 \cap D_k = \emptyset\) and the base case holds.
  Suppose inductively that \(P(n)\) is true for some \(n \geq 0\).
  Then for \(n + 1\), we need to show that \(P(n + 1)\) is also true.
  Suppose for sake of contradiction that \(D_{n + 1} \cap D_{n + k + 1} \neq \emptyset\).
  Then we have
  \begin{align*}
             & \exists\ x : (x \in D_{n + 1} \cap D_{n + k + 1})                                                               \\
    \implies & \exists\ x : \big(x \in f(D_n) \cap f(D_{n + k})\big)                                                           \\
    \implies & \exists\ z, z' : \big(z \in D_n \land z' \in D_{n + k} \land f(z) = f(z')\big)                                  \\
    \implies & z = z'.                                                                        &  & \text{(\(f\) is injective)}
  \end{align*}
  But by induction hypothesis we know that \(D_n \cap D_{n + k} = \emptyset\), a contradiction.
  Thus \(D_{n + 1} \cap D_{n + k + 1} = \emptyset\).
  This closes the induction.

  Now we use \(P(n)\) to show that \(\forall n, m \in \N\), \(n \neq m \implies D_n \cap D_m = \emptyset\).
  Since \(m \neq n\), we have
  \[
    \begin{dcases}
      n = m + k & \text{if } m < n, \\
      m = n + k & \text{if } n < m,
    \end{dcases}
  \]
  where \(k \in \Z^+\).
  Using \(P(n)\) we can thus have \(D_n \cap D_m = \emptyset\).

  Finally we show that if \(g : A \to B\) is a function where
  \[
    \forall x \in A : g(x) = \begin{dcases}
      f^{-1}(x) & \text{if } x \in \bigcup_{n = 1}^\infty D_n, \\
      x         & \text{otherwise},
    \end{dcases}
  \]
  then \(g\) is bijective.
  Since \(f\) is injective, \(f\) is thus bijective from \(\bigcup_{n = 0}^\infty D_n\) to \(f(\bigcup_{n = 0}^\infty D_n)\).
  By definition we have \(f(\bigcup_{n = 0}^\infty D_n) = \bigcup_{n = 1}^\infty D_n\).
  Thus \(g\) is well-defined.

  We now show that \(g\) is bijective.
  We start by showing that \(g\) is injective.
  Let \(x, x' \in A\) such that \(g(x) = g(x')\).
  We split into four cases:
  \begin{itemize}
    \item If \(x, x' \in \bigcup_{n = 1}^\infty D_n\), then since \(f\) is bijective from \(\bigcup_{n = 0}^\infty D_n\) to \(\bigcup_{n = 1}^\infty D_n\), we have \(f^{-1}(x) = f^{-1}(x') \implies x = x'\).
    \item If \(x, x' \in A \setminus \bigcup_{n = 1}^\infty D_n\), then we have \(g(x) = x = x' = g(x')\).
    \item One of \(x, x'\) is in \(A \setminus \bigcup_{n = 1}^\infty D_n\) and the other is in \(\bigcup_{n = 1}^\infty D_n\).
          Without the loss of generality suppose that \(x \in \bigcup_{n = 1}^\infty D_n\) and \(x' \in A \setminus \bigcup_{n = 1}^\infty D_n\).
          Then we show that this case is impossible.
          Since \(x \in \bigcup_{n = 1}^\infty D_n\), we have
          \[
            f^{-1}(x) \in \bigcup_{n = 0}^\infty D_n = D_0 \cup (\bigcup_{n = 1}^\infty D_n) = (B \setminus A) \cup (\bigcup_{n = 1}^\infty D_n).
          \]
          But \(g(x) = f^{-1}(x) = x' = g(x')\) implies \(x' \in (B \setminus A) \cup (\bigcup_{n = 1}^\infty D_n)\), a contradiction.
          Thus this case is impossible.
  \end{itemize}
  From all possible cases above we conclude that \(x = x'\).
  Thus \(g\) is injective.

  Next we show that \(g\) is surjective.
  Let \(y \in B\).
  We split into two cases:
  \begin{itemize}
    \item If \(y \in \bigcup_{n = 0}^\infty D_n\), then \(\exists\ x \in \bigcup_{n = 1}^\infty D_n\) such that \(f^{-1}(x) = y\).
          This is true since \(f\) is bijective from \(\bigcup_{n = 0}^\infty D_n\) to \(\bigcup_{n = 1}^\infty D_n\).
    \item If \(y \in B \setminus (\bigcup_{n = 0}^\infty D_n)\), then we know that \(y \in A\) since
          \begin{align*}
            B \setminus \bigcup_{n = 0}^\infty D_n & = B \setminus \bigg((B \setminus A) \cup (\bigcup_{n = 1}^\infty D_n)\bigg)                                                        \\
                                                   & = \bigg(B \setminus (B \setminus A)\bigg) \cap \bigg(B \setminus (\bigcup_{n = 1}^\infty D_n)\bigg) &  & \text{(by \cref{3.1.28})} \\
                                                   & = (A \cap B) \cap \bigg(B \setminus (\bigcup_{n = 1}^\infty D_n)\bigg)                                                             \\
                                                   & \subseteq A.
          \end{align*}
          Since \(y \notin \bigcup_{n = 0}^\infty D_n\), we have \(g(y) = y\).
  \end{itemize}
  From all cases above we can find a \(x \in A\) such that \(g(x) = y\).
  Thus \(g\) is surjective.
  Since \(g\) is both injective and surjective, we know that \(g\) is bijective.
\end{proof}

\begin{ex}\label{ex:8.3.3}
  Recall from \cref{ex:3.6.7} that a set \(A\) is said to have lesser or equal cardinality than a set \(B\) iff there is an injective map \(f : A \to B\) from \(A\) to \(B\).
  Using \cref{ex:8.3.2}, show that if \(A, B\) are sets such that \(A\) has lesser or equal cardinality to \(B\) and \(B\) has lesser or equal cardinality to \(A\), then \(A\) and \(B\) have equal cardinality.
  (This is known as the \emph{Schröder-Bernstein theorem}, after Ernst Schröder (1841 -- 1902) and Felix Bernstein (1878 -- 1956).)
\end{ex}

\begin{proof}
  Suppose that \(A, B\) are sets, \(A\) has lesser or equal cardinality to \(B\) and \(B\) has lesser or equal cardinality to \(A\).
  Then by \cref{ex:3.6.7} we know that \(\exists\ f : A \to B, g : B \to A\) such that both \(f, g\) are injective.
  By \cref{ex:3.3.2} we have \(f \circ g : B \to B\) is injective.
  By \cref{3.4.1} we know that \(f \circ g : B \to f\big(g(B)\big)\) is bijective and
  \[
    f\big(g(B)\big) \subseteq f(A) \subseteq B.
  \]
  Define the sets \(D_n\) recursively by setting
  \begin{align*}
    D_0       & = f(A) \setminus f\big(g(B)\big), \\
    D_{n + 1} & = f\big(g(D_n)\big),
  \end{align*}
  where \(n \in \N\).
  By \cref{ex:8.3.2} we know that \(\forall i, j \in \N\), \(i \neq j \implies D_i \cap D_j = \emptyset\).
  Now let \(h : f\big(g(B)\big) \to f(A)\) be a function as follow:
  \[
    \forall x \in f\big(g(B)\big), h(x) = \begin{dcases}
      (g^{-1} \circ f^{-1})(x) & \text{if } x \in \bigcup_{n = 1}^\infty D_n,    \\
      x                        & \text{if } x \notin \bigcup_{n = 1}^\infty D_n.
    \end{dcases}
  \]
  By \cref{ex:8.3.2} we know that \(h\) is bijective.
  Thus by \cref{3.6.1} we know \(f(A)\) and \(f\big(g(B)\big)\) have the same cardinality.
  Since \(f \circ g\) is bijective, we know that \(B, f\big(g(B)\big)\) have the same cardinality, thus by \cref{3.6.4} \(f(A), B\) have the same cardinality.
  But since \(f\) is injective, we know that \(f\) is bijective from \(A\) to \(f(A)\), which means \(A, f(A)\) have the same cardinality.
  Thus by \cref{3.6.4} \(A, B\) have the same cardinality.
\end{proof}

\begin{ex}\label{ex:8.3.4}
  Let us say that a set \(A\) has \emph{strictly lesser cardinality} than a set \(B\) if \(A\) has lesser than or equal cardinality to \(B\) (in the sense of \cref{ex:3.6.7}) but \(A\) does not have equal cardinality to \(B\).
  Show that for any set \(X\), that \(X\) has strictly lesser cardinality than \(2^X\).
  Also, show that if \(A\) has strictly lesser cardinality than \(B\), and \(B\) has strictly lesser cardinality than \(C\), then \(A\) has strictly lesser cardinality than \(C\).
\end{ex}

\begin{proof}
  We first show that \(X\) has strictly lesser cardinality than \(2^X\).
  Suppose that \(X\) is a set.
  Since \(2^X\) has a subset \(S = \{\{x\} : x \in X\}\), we have a bijection \(f : X \to S\) which maps \(x \mapsto \{x\}\) for every \(x \in X\).
  Now we define \(g : X \to 2^X\) where \(\forall x \in X : g(x) = f(x)\).
  By \cref{8.3.1} we know that \(g\) is not bijective.
  Since \(f\) is bijective, we know that \(g\) is injective.
  Thus by definition \(X\) has strictly lesser cardinality than \(2^X\).

  Now we show that if \(A\) has strictly lesser cardinality than \(B\), and \(B\) has strictly lesser cardinality than \(C\), then \(A\) has strictly lesser cardinality than \(C\).
  By \cref{ex:3.6.7} \(\exists\ f : A \to B, g : B \to C\) such that both \(f, g\) are injective.
  By \cref{ex:3.3.2} we know that \(g \circ f : A \to C\) is injective.
  Thus to show that \(A\) has strictly lesser cardinality than \(B\), by definition it suffices to show that \(A\) does not have equal cardinality to \(C\).

  Suppose for sake of contradiction that \(A, C\) have the same cardinality.
  Then by \cref{3.6.1} \(\exists\ h : C \to A\) such that \(h\) is bijective.
  Since \(g\) is injective, by \cref{ex:3.3.2} we know that \(h \circ g : B \to A\) is injective.
  But \(f\) is also injective, by \cref{ex:8.3.3} we know that \(A, B\) have the same cardinality, a contradiction.
  Thus \(A, C\) does not have the same cardinality, therefore \(A\) has strictly lesser cardinality than \(C\).
\end{proof}

\begin{ex}\label{ex:8.3.5}
  Show that no power set (i.e., a set of the form \(2^X\) for some set \(X\)) can be countably infinite.
\end{ex}

\begin{proof}
  Suppose for sake of contradiction that there exists a set \(X\) such that \(2^X\) is countable.
  By \cref{ex:8.3.4} \(X\) has strictly lesser cardinality than \(2^X\).
  Since \(2^X\) is countable, by \cref{8.1.1} we know that \(X\) has strictly lesser cardinality than \(\N\) and \(X\) can not be countable.
  Since \(X\) has strictly lesser cardinality than \(\N\), by \cref{ex:3.6.7}, \(\exists\ f : X \to \N\) such that \(f\) is injective.
  Since \(f\) is injective, we know that \(f\) is bijective from \(X\) to \(f(X)\).
  Since \(f(X) \subseteq \N\), by \cref{8.1.6} \(f(X)\) is at most countable.
  Since \(X, f(X)\) have the same cardinality and \(X\) is not countable, \(X\) must be finite.
  But by \cref{ex:8.3.1} we know that \(X\) is finite implies \(2^X\) is finite, a contradiction.
  Thus such \(X\) does not exists.
\end{proof}
\section{The axiom of choice}\label{sec:8.4}

\begin{note}
  We now discuss the final axiom of the standard Zermelo-Fraenkel-Choice system of set theory, namely the \emph{axiom of choice}.
  We have delayed introducing this axiom for a while now, to demonstrate that a large portion of the foundations of analysis can be constructed without appealing to this axiom.
  However, in many further developments of the theory, it is very convenient (and in some cases even essential) to employ this powerful axiom.
  On the other hand, the axiom of choice can lead to a number of unintuitive consequences (for instance the \emph{Banach-Tarski paradox}), and can lead to proofs that are philosophically somewhat unsatisfying.
  Nevertheless, the axiom is almost universally accepted by mathematicians.
  One reason for this confidence is a theorem due to the great logician Kurt Gödel, who showed that a result proven using the axiom of choice will never contradict a result proven without the axiom of choice
  (unless all the other axioms of set theory are themselves inconsistent, which is highly unlikely).
  More precisely, Gödel demonstrated that the axiom of choice is \emph{undecidable};
  it can neither be proved nor disproved from the other axioms of set theory, so long as those axioms are themselves consistent.
  (From a set of inconsistent axioms one can prove that every statement is both true and false.)
  In practice, this means that any ``real-life'' application of analysis
  (more precisely, any application involving only ``decidable'' questions)
  which can be rigorously supported using the axiom of choice, can also be rigorously supported without the axiom of choice, though in many cases it would take a much more complicated and lengthier argument to do so if one were not allowed to use the axiom of choice.
  Thus one can view the axiom of choice as a convenient and safe labour-saving device in analysis.
  In other disciplines of mathematics, notably in set theory in which many of the questions are not decidable, the issue of whether to accept the axiom of choice is more open to debate, and involves some philosophical concerns as well as mathematical and logical ones.
\end{note}

\begin{defn}[Infinite Cartesian products]\label{8.4.1}
  Let \(I\) be a set (possibly infinite), and for each \(\alpha \in I\) let \(X_{\alpha}\) be a set.
  We then define the Cartesian product \(\prod_{\alpha \in I} X_{\alpha}\) to be the set
  \[
    \prod_{\alpha \in I} X_{\alpha} = \set{(x_{\alpha})_{\alpha \in I} \in (\bigcup_{\beta \in I} X_{\beta})^I : x_{\alpha} \in X_{\alpha} \ \forall \alpha \in I},
  \]
  where we recall (from \cref{3.10}) that \((\bigcup_{\alpha \in I} X_{\alpha})^I\) is the set of all functions \((x_{\alpha})_{\alpha \in I}\) which assign an element \(x_{\alpha} \in \bigcup_{\beta \in I} X_{\beta}\) to each \(\alpha \in I\).
  Thus \(\prod_{\alpha \in I} X_{\alpha}\) is a subset of that set of functions, consisting instead of those functions \((x_{\alpha})_{\alpha \in I}\) which assign an element \(x_{\alpha} \in X_{\alpha}\) to each \(\alpha \in I\).
\end{defn}

\begin{eg}\label{8.4.2}
  For any sets \(I\) and X, we have \(\prod_{\alpha \in I} X = X^I\).
  If \(I\) is a set of the form \(I \coloneqq \set{i \in \N : 1 \leq i \leq n}\), then \(\prod_{\alpha \in I} X_{\alpha}\) is the same set as the set \(\prod_{1 \leq i \leq N} X_i\) defined in \cref{3.5.7}.
\end{eg}

\begin{note}
  Recall from \cref{3.5.12} that if \(X_1, \dots, X_n\) were any finite collection of non-empty sets, then the finite Cartesian product \(\prod_{1 \leq i \leq n} X_i\) was also non-empty.
  The axiom of choice asserts that this statement is also true fo infinite Cartesian products.
\end{note}

\begin{ax}[Choice]\label{8.1}
  Let \(I\) be a set, and for each \(\alpha \in I\), let \(X_{\alpha}\) be a non-empty set.
  Then \(\prod_{\alpha \in I} X_{\alpha}\) is also non-empty.
  In other words, there exists a function \((x_{\alpha})_{\alpha \in I}\) which assigns to each \(\alpha \in I\) an element \(x_{\alpha} \in X_{\alpha}\).
\end{ax}

\begin{rmk}\label{8.4.3}
  The intuition behind this axiom is that given a (possibly infinite) collection of non-empty sets \(X_{\alpha}\), one should be able to choose a single element \(x_{\alpha}\) from each one, and then form the possibly infinite tuple \((x_{\alpha})_{\alpha \in I}\) from all the choices one has made.
  On one hand, this is a very intuitively appealing axiom;
  in some sense one is just applying \cref{3.1.6} over and over again.
  On the other hand, the fact that one is making an infinite number of arbitrary choices, with no explicit rule as to \emph{how} to make these choices, is a little disconcerting.
  Indeed, there are many theorems proven using the axiom of choice which assert the abstract existence of some object \(x\) with certain properties, without saying at all \emph{what} that object is, or how to construct it.
  Thus the axiom of choice can lead to proofs which are \emph{non-constructive} - demonstrating existence of an object without actually constructing the object explicitly.
  This problem is not unique to the axiom of choice - it already appears for instance in \cref{3.1.6} - but the objects shown to exist using the axiom of choice tend to be rather extreme in their level of non-constructiveness.
  However, as long as one is aware of the distinction between a non-constructive existence statement, and a constructive existence statement (with the latter being preferable, but not strictly necessary in many cases), there is no difficulty here, except perhaps on a philosophical level.
\end{rmk}

\begin{rmk}\label{8.4.4}
  There are many equivalent formulations of the axiom of choice.
\end{rmk}

\begin{note}
  In analysis one often does not need the full power of the axiom of choice.
  Instead, one often only needs the \emph{axiom of countable choice}, which is the same as the axiom of choice but with the index set \(I\) restricted to be at most countable.
\end{note}

\begin{lem}\label{8.4.5}
  Let \(E\) be a non-empty subset of the real line with \(\sup(E) < \infty\)
  (i.e., \(E\) is bounded from above).
  Then there exists a sequence \((a_n)_{n = 1}^\infty\) whose elements \(a_n\) all lie in \(E\), such that \(\lim_{n \to \infty} a_n = \sup(E)\).
\end{lem}

\begin{proof}
  For each positive natural number \(n\), let \(X_n\) denote the set
  \[
    X_n \coloneqq \set{x \in E : \sup(E) - 1 / n \leq x \leq \sup(E)}.
  \]
  Since \(\sup(E)\) is the least upper bound for \(E\), then \(\sup(E) - 1 / n\) cannot be an upper bound for \(E\), and so \(X_n\) is non-empty for each \(n\).
  Using the axiom of choice (\cref{8.1}, or the axiom of countable choice), we can then find a sequence \((a_n)_{n = 1}^\infty\) such that \(a_n \in X_n\) for all \(n \geq 1\).
  In particular \(a_n \in E\) for all \(n\), and \(\sup(E) - 1 / n \leq a_n \leq \sup(E)\) for all \(n\).
  But then we have \(\lim_{n \to \infty} a_n = \sup(E)\) by the squeeze test (\cref{6.4.14}).
\end{proof}

\begin{rmk}\label{8.4.6}
  In many special cases, one can obtain the conclusion of \cref{8.4.5} without using the axiom of choice.
  For instance, if \(E\) is a closed set then one can define \(a_n\) without choice by the formula \(a_n \coloneqq \inf(X_n)\);
  the extra hypothesis that \(E\) is closed will ensure that \(a_n\) lies in \(E\).
\end{rmk}

\begin{prop}\label{8.4.7}
  Let \(X\) and \(Y\) be sets, and let \(P(x, y)\) be a property pertaining to an object \(x \in X\) and an object \(y \in Y\) such that for every \(x \in X\) there is at least one \(y \in Y\) such that \(P(x, y)\) is true.
  Then there exists a function \(f : X \to Y\) such that \(P(x, f(x))\) is true for all \(x \in X\).
\end{prop}

\begin{proof}
  We first show that axiom of choice (\cref{8.1}) implies \(\exists f : X \to Y\) such that \(P(x, f(x))\) is true for all \(x \in X\).
  Define
  \[
    Y_x \coloneqq \set{y \in Y : P(x, y) \text{ is true}}
  \]
  for each \(x \in X\).
  Such a set exist by \cref{3.5} and is non-empty by the hypothesis.
  By \cref{3.10}, we know that the set
  \[
    \prod_{x \in X} Y_x = \set{(y_x)_{x \in X} \in (\bigcup_{x \in X} Y_x)^X : y_x \in Y_x \text{ for all } x \in X}
  \]
  exists.
  By axiom of choice (\cref{8.1}) we also know that the set \(\prod_{x \in X} Y_x\) is non-empty.
  Now we can choose an element \(f \in \prod_{x \in X} Y_x\).
  We know that \(f\) is a function with domain \(X\) and range \(Y\).
  Also, \(\forall x \in X\), we have an unique \(f(x) \in Y_x\).
  By the definition of \(Y_x\) we know that \(P(x, f(x))\) is true.
  Thus axiom of choice (\cref{8.1}) implies \(\exists f : X \to Y\) such that \(P(x, f(x))\) is true for all \(x \in X\).

  Now we show that if \(\forall x \in X\), \(\exists y \in Y\) such that \(P(x, y)\) is true and \(\exists f : X \to Y\) such that \(P(x, f(x))\) is true for all \(x \in X\), then axiom of choice (\cref{8.1}) is true.
  Using the definition of \(Y_x\) again we know that \(f(x) \in Y_x\), so \(Y_x \neq \emptyset\).
  By \cref{3.10} we can have a set \(\prod_{x \in X} Y_x\).
  We want to show that \(\prod_{x \in X} Y_x \neq \emptyset\).
  But this is true since \(f \in \prod_{x \in X} Y_x\).
  Thus \cref{8.4.7} implies axiom of choice (\cref{8.1}).
\end{proof}

\exercisesection

\begin{ex}\label{ex:8.4.1}
  Show that the axiom of choice implies \cref{8.4.7}.
  Conversely, show that if \cref{8.4.7} is true, then the axiom of choice is also true.
\end{ex}

\begin{proof}
  See \cref{8.4.7}.
\end{proof}

\begin{ex}\label{ex:8.4.2}
  Let \(I\) be a set, and for each \(\alpha \in I\) let \(X_{\alpha}\) be a non-empty set.
  Suppose that all the sets \(X_{\alpha}\) are disjoint from each other, i.e., \(X_{\alpha} \cap X_{\beta} = \emptyset\) for all distinct \(\alpha, \beta \in I\).
  Using the axiom of choice, show that there exists a set \(Y\) such that \(\#(Y \cap X_{\alpha}) = 1\) for all \(\alpha \in I\) (i.e., \(Y\) intersects each \(X_{\alpha}\) in exactly one element).
  Conversely, show that if the above statement was true for an arbitrary choice of sets \(I\) and non-empty disjoint sets \(X_{\alpha}\), then the axiom of choice is true.
\end{ex}

\begin{proof}
  We first show that axiom of choice (\cref{8.1}) implies there exists a set \(Y\) such that \(\#(Y \cap X_{\alpha}) = 1\) for all \(\alpha \in I\).
  By \cref{8.1}, the set \(\prod_{\alpha \in I} X_{\alpha}\) is non-empty.
  Let \(f \in \prod_{\alpha \in I} X_{\alpha}\) and let \(Y = f(I)\).
  Then we have
  \begin{align*}
     & \forall \alpha \in I, Y \cap X_{\alpha}                                                                                                             \\
     & = f(I) \cap X_{\alpha}                                                                                                                              \\
     & = \Bigg(\bigcup_{\beta \in I} \set{f(\beta)}\Bigg) \cap X_{\alpha}                                                                                  \\
     & = \Bigg(\set{f(\alpha)} \cup \bigg(\bigcup_{\beta \in I : \beta \neq \alpha} \set{f(\beta)}\bigg)\Bigg) \cap X_{\alpha}                             \\
     & = \set{f(\alpha)} \cap X_{\alpha}                                                                                       &  & \text{(by hypothesis)} \\
     & = \set{f(\alpha)}.
  \end{align*}
  Thus \(\#(Y \cap X_{\alpha}) = 1\) for every \(\alpha \in I\).
  We conclude that axiom of choice (\cref{8.1}) implies there exists a set \(Y\) such that \(\#(Y \cap X_{\alpha}) = 1\) for all \(\alpha \in I\).

  Now we show that if there exists a set \(Y\) such that \(\#(Y \cap X_{\alpha}) = 1\) for all \(\alpha \in I\), then axiom of choice (\cref{8.1}) is true.
  Since we know that \(\exists Y : \#(Y \cap X_{\alpha}) = 1\) for all \(\alpha \in I\), we can let \(Y \cap X_{\alpha} = \set{x_{\alpha}}\) for some \(x_{\alpha} \in X_{\alpha}\).
  Thus by \cref{3.6} we have a function \(f : I \to \bigcup_{\alpha \in I} X_{\alpha}\) such that \(f(\alpha) = x_{\alpha} \in X_{\alpha}\) for all \(\alpha \in I\).
  But this means \(f \in \prod_{\alpha \in I} X_{\alpha}\), so \(\prod_{\alpha \in I} X_{\alpha} \neq \emptyset\), and axiom of choice (\cref{8.1}) is true.
  Thus if there exists a set \(Y\) such that \(\#(Y \cap X_{\alpha}) = 1\) for all \(\alpha \in I\), then axiom of choice (\cref{8.1}) is true.
\end{proof}

\begin{ex}\label{ex:8.4.3}
  Let \(A\) and \(B\) be sets such that there exists a surjection \(g : B \to A\).
  Using the axiom of choice, show that there exists an injection \(f: A \to B\) with \(g \circ f : A \to A\) the identity map;
  in particular \(A\) has lesser or equal cardinality to \(B\) in the sense of \cref{ex:3.6.7}.
  Compare this with \cref{ex:3.6.8}.
  Conversely, show that if the above statement is true for arbitrary sets \(A, B\) and surjections \(g : B \to A\), then the axiom of choice is true.
\end{ex}

\begin{proof}
  We first show that if \(g : B \to A\) is a surjection, then there exists an injection \(f : A \to B\).
  Let \(g^{-1}(A)\) be the inverse image of \(A\).
  Since \(g\) is surjective, we know that \(g^{-1}(A) = \bigcup_{a \in A} g^{-1}(\set{a}) = B\).
  By axiom of choice (\cref{8.1}), we know that the set \(\prod_{a \in A} g^{-1}(\set{a}) \neq \emptyset\).
  Let \(f \in \prod_{a \in A} g^{-1}(\set{a})\).
  We know that \(f\) has domain \(A\) and range \(\bigcup_{a \in A} g^{-1}(\set{a}) = B\).
  We now show that \(f\) is injective.
  \(\forall a_1, a_2 \in A\), if \(a_1 \neq a_2\), then we must have \(g^{-1}(\set{a_1}) \cap g^{-1}(\set{a_2}) = \emptyset\).
  Otherwise by \cref{3.4.4} \(\exists b \in B\) such that \(g(b) = a_1 \land g(b) = a_2\), a contradiction.
  By the definition of \(f\), we know that \(f(a_1) \in g^{-1}(\set{a_1}) \land f(a_2) \in g^{-1}(\set{a_2})\).
  Since \(g^{-1}(\set{a_1}) \cap g^{-1}(\set{a_2}) = \emptyset\), we know that \(f(a_1) \neq f(a_2)\), thus \(f\) is injective.

  Next we show that \(g \circ f : A \to A\) is an identity map.
  \begin{align*}
             & \forall a \in A, f(a) \in g^{-1}(\set{a})                 \\
    \implies & g(f(a)) = a.                              &  & \by{3.4.4}
  \end{align*}
  Thus \(g \circ f : A \to A\) is an identity map.

  Now we show that if \(g : B \to A\) is a surjection and there exists an injection \(f : A \to B\) where \(g \circ f : A \to A\) is an identity map, then the axiom of choice is true.
  Let \(A_a = \set{a \in A : g(f(a)) = a}\).
  Since \(g \circ f\) is an identity map, \(\forall a_1, a_2 \in A\), we have \(a_1 \neq a_2 \implies g(f(a_1)) \neq g(f(a_2))\).
  Thus \(\#(A_a) = 1\) and \(\forall a_1, a_2 \in A\), \(a_1 \neq a_2 \implies A_{a_1} \cap A_{a_2} = \emptyset\).
  Also we have \(A \cap A_a = A_a\) for every \(a \in A\), and thus \(\#(A \cap A_a) = 1\).
  So by \cref{ex:8.4.2} we know that axiom of choice (\cref{8.1}) is true.
\end{proof}

\section{Ordered sets}\label{i:sec:8.5}

\begin{defn}[Partially ordered sets]\label{i:8.5.1}
  A \emph{partially ordered set} (or \emph{poset}) is a set \(X\), together with a relation \(\leq_X\) on \(X\)
  (thus for any two objects \(x, y \in X\), the statement \(x \leq_X y\) is either a true statement or a false statement).
  Furthermore, this relation is assumed to obey the following three properties:
  \begin{itemize}
    \item (Reflexivity) For any \(x \in X\), we have \(x \leq_X x\).
    \item (Anti-symmetry) If \(x, y \in X\) are such that \(x \leq_X y\) and \(y \leq_X x\), then \(x = y\).
    \item (Transitivity) If \(x, y, z \in X\) are such that \(x \leq_X y\) and \(y \leq_X z\), then \(x \leq_X z\).
  \end{itemize}
  We refer to \(\leq_X\) as the \emph{ordering relation}.
  In most situations it is understood what the set \(X\) is from context, and in those cases we shall simply write \(\leq\) instead of \(\leq_X\).
  We write \(x <_X y\) (or \(x < y\) for short) if \(x \leq_X y\) and \(x \neq y\).
\end{defn}

\begin{note}
  Strictly speaking, a partially ordered set is not a set \(X\), but rather a pair \((X, \leq_X)\).
  But in many cases the ordering \(\leq_X\) will be clear from context, and so we shall refer to \(X\) itself as the partially ordered set even though this is technically incorrect.
\end{note}

\begin{eg}\label{i:8.5.2}
  The natural numbers \(\N\) together with the usual less-than-or-equal-to relation \(\leq\) (as defined in \cref{i:2.2.11}) forms a partially ordered set, by \cref{i:2.2.12}.
  Similar arguments (using the appropriate definitions and propositions) show that the integers \(\Z\), the rationals \(\Q\), the reals \(\R\), and the extended reals \(\R^*\) are also partially ordered sets.
  Meanwhile, if \(X\) is any collection of sets, and one uses the relation of is-a-subset-of \(\subseteq\) (as defined in \cref{i:3.1.15}) for the ordering relation \(\leq_X\), then \(X\) is also partially ordered (\cref{i:3.1.18}).
  Note that it is certainly possible to give these sets a different partial ordering than the standard one.
\end{eg}

\begin{defn}[Totally ordered set]\label{i:8.5.3}
  Let \(X\) be a partially ordered set with some order relation \(\leq_X\).
  A subset \(Y\) of \(X\) is said to be \emph{totally ordered} if, given any two \(y, y' \in Y\), we either have \(y \leq_X y'\) or \(y' \leq_X y\) (or both).
  If \(X\) itself is totally ordered, we say that \(X\) is a \emph{totally ordered set} (or \emph{chain}) with order relation \(\leq_X\).
\end{defn}

\begin{eg}\label{i:8.5.4}
  The natural numbers \(\N\), the integers \(\Z\), the rationals \(\Q\), reals \(\R\), and the extended reals \(\R^*\), all with the usual ordering relation \(\leq\), are totally ordered
  (by \cref{i:2.2.13}, \cref{i:4.1.11}, \cref{i:4.2.9}, \cref{i:5.4.7}, and \cref{i:6.2.5} respectively).
  Also, any subset of a totally ordered set is again totally ordered. On the other hand, a collection of sets with the \(\subseteq\) relation is usually not totally ordered.
\end{eg}

\begin{defn}[Maximal and minimal elements]\label{i:8.5.5}
  Let \(X\) be a partially ordered set, and let \(Y\) be a subset of \(X\).
  We say that \(y\) is a \emph{minimal element} of \(Y\) if \(y \in Y\) and there is no element \(y' \in Y\) such that \(y' < y\).
  We say that \(y\) is a \emph{maximal element} of \(Y\) if \(y \in Y\) and there is no element \(y' \in Y\) such that \(y < y'\).
\end{defn}

\setcounter{thm}{6}
\begin{eg}\label{i:8.5.7}
  The natural numbers \(\N\) (ordered by \(\leq\)) has a minimal element, namely \(0\), but no maximal element.
  The set of integers \(\Z\) has no maximal and no minimal element.
\end{eg}

\begin{defn}[Well-ordered sets]\label{i:8.5.8}
  Let \(X\) be a partially ordered set, and let \(Y\) be a totally ordered subset of \(X\).
  We say that \(Y\) is \emph{well-ordered} if every non-empty subset \(Z\) of \(Y\) has a minimal element \(\min(Z)\).
\end{defn}

\begin{eg}\label{i:8.5.9}
  The natural numbers \(\N\) are well-ordered by \cref{i:8.1.4}.
  However, the integers \(\Z\), the rationals \(\Q\), and the real numbers \(\R\) are not (see \cref{i:ex:8.1.2}).
  Every subset of a well-ordered set is again well-ordered.
\end{eg}

\begin{prop}[Principle of strong induction]\label{i:8.5.10}
  Let \(X\) be a well-ordered set with an ordering relation \(\leq_X\), and let \(P(n)\) be a property pertaining to an element \(n \in X\)
  (i.e., for each \(n \in X\), \(P(n)\) is either a true statement or a false statement).
  Suppose that for every \(n \in X\), we have the following implication:
  if \(P(m)\) is true for all \(m \in X\) with \(m <_X n\), then \(P(n)\) is also true.
  Then \(P(n)\) is true for all \(n \in X\).
\end{prop}

\begin{proof}
  Since \((X, \leq_X)\) is well-ordered, by \cref{i:8.5.8} we know that \(X \neq \emptyset\) and \(\min\big((X, \leq_X)\big)\) exists.
  By \cref{i:8.5.5} we know that the statement ``\(\forall m \in X\), \(m <_X \min\big((X, \leq_X)\big) \implies P(m)\) is true'' is vacuously true since there is no \(m <_X n\).
  Thus, by hypothesis we know that \(P\Big(\min\big((X, \leq_X)\big)\Big)\) is vacuously true.

  Now let \(Y\) be the set
  \[
    Y = \set{m \in X : P(m) \text{ is false}}.
  \]
  Suppose for the sake of contradiction that \(Y \neq \emptyset\).
  Since \(X\) is well-ordered and \(Y \subseteq X\), by \cref{i:8.5.8} we know that \(\min\big((Y, \leq_X)\big)\) exists.
  From previous claim we know that \(\min\big((Y, \leq_X)\big) \neq \min\big((X, \leq_X)\big)\), thus the set
  \[
    Y' = \set{m \in X : m <_X \min\big((Y, \leq_X)\big)}
  \]
  is not empty.
  Also by \cref{i:8.5.5} we know that \(\forall m \in Y'\), \(P(m)\) is true, otherwise contradict to the definition of \(\min\big((Y, \leq_X)\big)\).
  But this means
  \[
    \forall m \in X, m <_X \min\big((Y, \leq_X)\big) \implies P\Big(\min\big((Y, \leq_X)\big)\Big)
  \]
  is false, which contradict to the hypothesis.
  Thus, we must have \(P(n)\) is true for all \(n \in X\).
\end{proof}

\begin{rmk}\label{i:8.5.11}
  It may seem strange that there is no ``base'' case in strong induction, corresponding to the hypothesis \(P(0)\) in \cref{i:2.5}.
  However, such a base case is automatically included in the strong induction hypothesis.
  Indeed, if \(0\) is the minimal element of \(X\), then by specializing the hypothesis ``if \(P(m)\) is true for all \(m \in X\) with \(m <_X n\), then \(P(n)\) is also true'' to the \(n = 0\) case, we automatically obtain that \(P(0)\) is true.
  (Since there is no element \(m <_X n\), such statement ``if \(P(m)\) is true for all \(m \in X\) with \(m <_X n\)'' is false, thus the implication holds vacuously.)
\end{rmk}

\begin{defn}[Upper bounds and strict upper bounds]\label{i:8.5.12}
  Let \(X\) be a partially ordered set with ordering relation \(\leq\), and let \(Y\) be a subset of \(X\).
  If \(x \in X\), we say that \(x\) is an \emph{upper bound} for \(Y\) iff \(y \leq x\) for all \(y \in Y\).
  If in addition \(x \notin Y\), we say that \(x\) is a \emph{strict upper bound} for \(Y\).
  Equivalently, \(x\) is a strict upper bound for \(Y\) iff \(y < x\) for all \(y \in Y\).
\end{defn}

\setcounter{thm}{13}
\begin{lem}\label{i:8.5.14}
  Let \(X\) be a partially ordered set with ordering relation \(\leq\), and let \(x_0\) be an element of \(X\).
  Then there is a well-ordered subset \(Y\) of \(X\) which has \(x_0\) as its minimal element, and which has no strict upper bound.
\end{lem}

\begin{proof}
  The intuition behind this lemma is that one is trying to perform the following algorithm:
  we initialize \(Y \coloneqq \set{x_0}\).
  If \(Y\) has no strict upper bound, then we are done;
  otherwise, we choose a strict upper bound and add it to \(Y\).
  Then we look again to see if \(Y\) has a strict upper bound or not.
  If not, we are done;
  otherwise we choose another strict upper bound and add it to \(Y\).
  We continue this algorithm ``infinitely often'' until we exhaust all the strict upper bounds;
  the axiom of choice comes in because infinitely many choices are involved.
  This is however not a rigorous proof because it is quite difficult to precisely pin down what it means to perform an algorithm ``infinitely often.''
  Instead, what we will do is that we will isolate a collection of ``partially completed'' sets \(Y\), which we shall call \emph{good sets}, and then take the union of all these good sets to obtain a ``completed'' object \(Y_{\infty}\) which will indeed have no strict upper bound.

  We now begin the rigorous proof.
  Suppose for the sake of contradiction that every well-ordered subset \(Y\) of \(X\) which has \(x_0\) as its minimal element has at least one strict upper bound.
  Using the axiom of choice (in the form of \cref{i:8.4.7}), we can thus assign a strict upper bound \(s(Y) \in X\) to each well-ordered subset \(Y\) of \(X\) which has \(x_0\) as its minimal element.

  Henceforth we fix a single such strict upper bound function \(s\).
  Let us define a special class of subsets \(Y\) of \(X\).
  We say that a subset \(Y\) of \(X\) is \emph{good} iff it is well-ordered, contains \(x_0\) as its minimal element, and obeys the property that
  \[
    x = s(\set{y \in Y : y < x}) \text{ for all } x \in Y \setminus \set{x_0}.
  \]
  Note that if \(x \in Y \setminus \set{x_0}\) then the set \(\set{y \in Y : y < x}\) is a subset of \(X\) which is well-ordered and contains \(x_0\) as its minimal element.
  Let \(\Omega \coloneqq \set{Y \subseteq X : Y \text{ is good}}\) be the collection of all good subsets of \(X\).
  This collection is not empty, since the subset \(\set{x_0}\) of \(X\) is clearly good
  (which is vacuously true).

  We make the following important observation:
  if \(Y\) and \(Y'\) are two good subsets of \(X\), then every element of \(Y' \setminus Y\) is a strict upper bound for \(Y\), and every element of \(Y \setminus Y'\) is a strict upper bound for \(Y'\).
  (\cref{i:ex:8.5.13}).
  In particular, given any two good sets \(Y\) and \(Y'\), at least one of \(Y' \setminus Y\) and \(Y \setminus Y'\) must be empty
  (since they are both strict upper bounds of each other).
  In other words, \(\Omega\) is totally ordered by set inclusion:
  given any two good sets \(Y\) and \(Y'\), either \(Y \subseteq Y'\) or \(Y' \subseteq Y\).

  Let \(Y_{\infty} = \bigcup \Omega\), i.e., \(Y_{\infty}\) is the set of all elements of \(X\) which belong to at least one good subset of \(X\).
  Clearly, \(x_0 \in Y_{\infty}\).
  Also, since each good subset of \(X\) has \(x_0\) as its minimal element, the set \(Y_{\infty}\) also has \(x_0\) as its minimal element.

  Next, we show that \(Y_{\infty}\) is totally ordered.
  Let \(x, x'\) be two elements of \(Y_{\infty}\).
  By definition of \(Y_{\infty}\), we know that \(x\) lies in some good set \(Y\) and \(x'\) lies in some good set \(Y'\).
  But since \(\Omega\) is totally ordered, one of these good sets contains the other.
  Thus, \(x, x'\) are contained in a single good set (either \(Y\) or \(Y'\));
  since good sets are totally ordered, we thus see that either \(x \leq x'\) or \(x' \leq x\) as desired.

  Next, we show that \(Y_{\infty}\) is well-ordered.
  Let \(A\) be any non-empty subset of \(Y_{\infty}\).
  Then we can pick an element \(a \in A\), which then lies in \(Y_{\infty}\).
  Therefore there is a good set \(Y\) such that \(a \in Y\).
  Then \(A \cap Y\) is a non-empty subset of \(Y\);
  since \(Y\) is well-ordered, the set \(A \cap Y\) thus has a minimal element, call it \(b\).
  Now recall that for any other good set \(Y'\), every element of \(Y' \setminus Y\) is a strict upper bound for \(Y\), and in particular, is larger than \(b\).
  Since \(b\) is a minimal element of \(A \cap Y\), this implies that \(b\) is also a minimal element of \(A \cap Y'\) for any good set \(Y'\) with \(A \cap Y' \neq \emptyset\).
  This is true since
  \begin{align*}
             & \forall y_1 \in Y, y_1 < \min(Y' \setminus Y)                                                    \\
    \implies & y_1 \in Y \cap Y'                                                                                \\
    \implies & \forall y_2 \in A \cap Y, \big(y_2 < \min(Y' \setminus Y)\big) \land \big(y_2 \in Y \cap Y'\big) \\
    \implies & y_2 \in A \cap Y'                                                                                \\
    \implies & b = \min(A \cap Y) = \min(A \cap Y').
  \end{align*}
  Since every element of \(A\) belongs to \(Y_{\infty}\) and hence belongs to at least one good set \(Y'\), we thus see that \(b\) is a minimal element of \(A\).
  Thus, \(Y_{\infty}\) is well-ordered as claimed.

  Since \(Y_{\infty}\) is well-ordered with \(x_0\) as its minimal element, it has a strict upper bound \(s(Y_{\infty})\).
  But then \(Y_{\infty} \cup \set{s(Y_{\infty})}\) is well-ordered (by \cref{i:ex:8.5.11}) and has \(x_0\) as its minimal element.
  We now claim that \(Y_{\infty} \cup \set{s(Y_{\infty})}\) is good.
  By the preceding discussion, it suffices to show that \(x = s\big(\set{y \in Y_{\infty} \cup \set{s(Y_{\infty})} : y < x}\big)\) when \(x \in \big(Y_{\infty} \cup \set{s(Y_{\infty})}\big) \setminus \set{x_0}\).
  If \(x = s(Y_{\infty})\) this is clear since \(\set{y \in Y_{\infty} \cup \set{s(Y_{\infty})} : y < x} = Y_{\infty}\) in this case.
  If instead \(x \in Y_{\infty}\), then \(x \in Y\) for some good \(Y\).
  Then the set \(\set{y \in Y_{\infty} \cup \set{s(Y_{\infty})}: y < x}\) is equal to \(\set{y \in Y : y < x}\)
  (why? use the previous observation that every element of \(Y' \setminus Y\) is an upper bound for \(x\) for every good \(Y'\)), and the claim then follows since \(Y\) is good.
  By definition of \(Y_{\infty}\), we conclude that the good set \(Y_{\infty} \cup \set{s(Y_{\infty})}\) is contained in \(Y_{\infty}\).
  But this is a contradiction since \(s(Y_{\infty})\) is a strict upper bound for \(Y_{\infty}\).
  Thus, we have constructed a set with no strict upper bound, as desired.
\end{proof}

\begin{lem}[Zorn's lemma]\label{i:8.5.15}
  Let \(X\) be a non-empty partially ordered set, with the property that every totally ordered subset \(Y\) of \(X\) has an upper bound.
  Then \(X\) contains at least one maximal element.
\end{lem}

\begin{proof}
  Let \((X, \leq)\) be partially ordered such that \(X \neq \emptyset\).
  Suppose for the sake of contradiction that \(X\) has no maximal element.
  We show that any subset \(Y \subseteq X\) which has an upper bound also has a strict upper bound.
  Let \(s\) be an upper bound of \(Y\).
  By \cref{i:8.5.12}, we know that \(s \in X\) and \(\forall y \in Y \implies y \leq s\).
  Since \(X\) has no maximal element, we know \(\exists s' \in X\) such that \(s < s'\), otherwise by \cref{i:8.5.5} we have \(s = \max((X, \leq))\), a contradiction.
  Since \(\forall y \in Y\), \(y < s'\), we know that \(s' \notin Y\), and thus by \cref{i:8.5.12} \(s'\) is a strict upper bound of \(Y\).

  Since \(X \neq \emptyset\), let \(x_0 \in X\).
  By \cref{i:8.5.14}, \(\exists Y \subseteq X\) such that \((Y, \leq)\) is well-ordered, \(\min\big((Y, \leq)\big) = x_0\) and \(Y\) has no strict upper bound.
  But by hypothesis we know that \(Y\) has an upper bound and thus has a strict upper bound, a contradiction.
  Thus, \(X\) must has at least one maximal element.
\end{proof}

\begin{note}
  Zorn's lemma is also called the \emph{principle of transfinite induction}.
\end{note}

\exercisesection

\begin{ex}\label{i:ex:8.5.1}
  Consider the empty set \(\emptyset\) with the empty order relation \(\leq_\emptyset\)
  (this relation is vacuous because the empty set has no elements).
  Is this set partially ordered? totally ordered? well-ordered? Explain.
\end{ex}

\begin{proof}
  Since
  \[
    \forall x \in \emptyset, x \leq_{\emptyset} x
  \]
  is vacuously true, we know that \((\emptyset, \leq_{\emptyset})\) is reflexive.
  Since
  \[
    \forall x, y \in \emptyset, (x \leq_{\emptyset} y) \land (y \leq_{\emptyset} x) \implies x = y
  \]
  is vacuously true, we know that \((\emptyset, \leq_{\emptyset})\) is anti-symmetric.
  Since
  \[
    \forall x, y, z \in \emptyset, (x \leq_{\emptyset} y) \land (y \leq_{\emptyset} z) \implies x \leq_{\emptyset} z
  \]
  is vacuously true, we know that \((\emptyset, \leq_{\emptyset})\) is transitive.
  Since \((\emptyset, \leq_{\emptyset})\) is reflexive, anti-symmetric and transitive, by \cref{i:8.5.1} \((\emptyset, \leq_{\emptyset})\) is partially ordered.
  Since
  \[
    \forall x, y \in \emptyset, (x \leq_{\emptyset} y) \lor (y \leq_{\emptyset} x)
  \]
  is vacuously true, by \cref{i:8.5.3} we know that \((\emptyset, \leq_{\emptyset})\) is totally ordered.
  Since
  \[
    \forall X \subseteq \emptyset, X \neq \emptyset \implies \exists \min\big((X, \leq_{\emptyset})\big) \in X
  \]
  is vacuously true, by \cref{i:8.5.8} we know that \((\emptyset, \leq_{\emptyset})\) is well-ordered.
\end{proof}

\begin{ex}\label{i:ex:8.5.2}
  Give examples of a set \(X\) and a relation \(\leq_X\) such that
  \begin{enumerate}
    \item The relation \(\leq_X\) is reflexive and anti-symmetric, but not transitive;
    \item The relation \(\leq_X\) is reflexive and transitive, but not anti-symmetric;
    \item The relation \(\leq_X\) is anti-symmetric and transitive, but not reflexive.
  \end{enumerate}
\end{ex}

\begin{proof}
  \begin{enumerate}
    \item Let \(X = \set{1, 2, 4}\) be a set and let \(\leq_X\) be the relation
          \[
            \forall a, b \in X, a \leq_X b \iff (a = b) \lor (2a = b).
          \]
          Since
          \[
            \forall a \in X, a = a \implies a \leq_X a,
          \]
          we know that \((X, \leq_X)\) is reflexive.
          Since
          \[
            (1 \leq_X 1) \land (1 \leq_X 2) \land (2 \leq_X 2) \land (2 \leq_X 4) \land (4 \leq_X 4),
          \]
          we know that when pairs of \(a, b \in X\) satisfying \(a \leq_X b \land b \leq_X a\) we must have \(a = b\), thus \((X, \leq_X)\) is anti-symmetric.
          Since we have \((1 \leq_X 2) \land (2 \leq_X 4)\) but not \(1 \leq_X 4\), we know that \((X, \leq_X)\) is not transitive.
    \item Let \(X = \Z\) be a set and let \(\leq_X\) be the relation
          \[
            \forall a, b \in \Z, a \leq_X b \iff \abs{a} \leq \abs{b}.
          \]
          Since
          \[
            \forall a \in \Z, \abs{a} \leq \abs{a} \implies a \leq_X a,
          \]
          we know that \((\Z, \leq_X)\) is reflexive.
          Since
          \[
            (\abs{1} \leq \abs{-1}) \land (\abs{-1} \leq \abs{1}) \implies (1 \leq_X -1) \land (-1 \leq_X 1)
          \]
          but \(1 \neq -1\), we know that \((\Z, \leq_X)\) is not anti-symmetric.
          Since
          \begin{align*}
                     & \forall a, b, c \in \Z, (a \leq_X b) \land (b \leq_X c) \\
            \implies & (\abs{a} \leq \abs{b}) \land (\abs{b} \leq \abs{c})     \\
            \implies & \abs{a} \leq \abs{c}                                    \\
            \implies & a \leq_X c,
          \end{align*}
          we know that \((\Z, \leq_X)\) is transitive.
    \item Let \(X = \set{0}\) be a set and let \(\leq_X\) be the relation
          \[
            \forall a, b \in X, a \leq_X b \iff a \leq a + 1.
          \]
          Since
          \[
            \forall a \in X, a + 1 \not\leq a,
          \]
          we know that \((X, \leq_X)\) is not reflexive.
          Since
          \[
            \forall a, b \in X, (a \leq_X b) \land (b \leq_X a) \implies a = b
          \]
          is vacuously true, we know that \((X, \leq_X)\) is anti-symmetric.
          Since
          \[
            \forall a, b \in X, (a \leq_X b) \land (b \leq_X c) \implies a \leq_X b
          \]
          is vacuously true, we know that \((X, \leq_X)\) is transitive.
  \end{enumerate}
\end{proof}

\begin{ex}\label{i:ex:8.5.3}
  Given two positive integers \(n, m \in \N \setminus \set{0}\), we say that \emph{\(n\) divides \(m\)}, and write \(n | m\), if there exists a positive integer \(a\) such that \(m = na\).
  Show that the set \(\N \setminus \set{0}\) with the ordering relation \(|\) is a partially ordered set but not a totally ordered one.
  Note that this is a different ordering relation from the usual \(\leq\) ordering of \(\N \setminus \set{0}\).
\end{ex}

\begin{proof}
  Since
  \[
    \forall n \in \N \setminus \set{0}, n = 1n \implies n | n,
  \]
  we know that \((X, |)\) is reflexive.
  Since
  \begin{align*}
             & \forall n, m \in \N \setminus \set{0}, (n | m) \land (m | n)   \\
    \implies & \exists a, b \in \N \setminus \set{0}, (m = na) \land (n = bm) \\
    \implies & n = abn                                                        \\
    \implies & ab = 1                                                         \\
    \implies & (a = 1) \land (b = 1)                                          \\
    \implies & n = m,
  \end{align*}
  we know that \((X, |)\) is anti-symmetric.
  Since
  \begin{align*}
             & \forall n, m, p \in \N \setminus \set{0}, (n | m) \land (m | p) \\
    \implies & \exists a, b \in \N \setminus \set{0}, (m = na) \land (p = bm)  \\
    \implies & (p = abn) \land (ab > 0)                                        \\
    \implies & n | p,
  \end{align*}
  we know that \((X, |)\) is transitive.
  Since \((X, |)\) is reflexive, anti-symmetric and transitive, by \cref{i:8.5.1} \((X, |)\) is partially ordered.
  Since \((2 | 3) \lor (3 | 2)\) is false, by \cref{i:8.5.3} \((X, |)\) is not totally ordered.
\end{proof}

\begin{ex}\label{i:ex:8.5.4}
  Show that the set of positive reals \(\R^+ \coloneqq \set{x \in \R : x > 0}\) have no minimal element.
\end{ex}

\begin{proof}
  Suppose for the sake of contradiction that \(\exists x \in \R^+\) such that \(x = \min(\R^+)\).
  Since \(x > 0\), we know that \(x / 2 > 0\) and \(x / 2 \in \R^+\).
  But \(x = \min(\R^+)\) implies we have \(x < x / 2\), a contradiction.
  Thus, \(\nexists x \in \R^+\) such that \(x = \min(\R^+)\).
\end{proof}

\begin{ex}\label{i:ex:8.5.5}
  Let \(f : X \to Y\) be a function from one set \(X\) to another set \(Y\).
  Suppose that \(Y\) is partially ordered with some ordering relation \(\leq_Y\).
  Define a relation \(\leq_X\) on \(X\) by defining \(x \leq_X x'\) iff \(f(x) <_Y f(x')\) or \(x = x'\).
  Show that this relation \(\leq_X\) turns \(X\) into a partially ordered set.
  If we know in addition that the relation \(\leq_Y\) makes \(Y\) totally ordered, does this mean that the relation \(\leq_X\) makes \(X\) totally ordered also?
  If not, what additional assumption needs to be made on \(f\) in order to ensure that \(\leq_X\) makes \(X\) totally ordered?
\end{ex}

\begin{proof}
  We first show that \((X, \leq_X)\) is partially ordered.
  Since
  \[
    \forall x \in X, x = x \implies x \leq_X x,
  \]
  we know that \((X, \leq_X)\) is reflexive.
  Since
  \begin{align*}
             & \forall x, x' \in X, (x \leq_X x') \land (x' \leq_X x)                                                                  \\
    \implies & \Big(\big(f(x) <_Y f(x')\big) \lor \big(x = x'\big)\Big) \land \Big(\big(f(x') <_Y f(x)\big) \lor \big(x = x'\big)\Big) \\
    \implies & x = x',
  \end{align*}
  we know that \((X, \leq_X)\) is anti-symmetric.
  Since
  \begin{align*}
             & \forall x_1, x_2, x_3 \in X, (x_1 \leq_X x_2) \land (x_2 \leq_X x_3)                                                                \\
    \implies & \Big(\big(f(x_1) <_Y f(x_2)\big) \lor \big(x_1 = x_2\big)\Big) \land \Big(\big(f(x_2) <_Y f(x_3)\big) \lor \big(x_2 = x_3\big)\Big) \\
    \implies & \bigg(\Big(\big(f(x_1) <_Y f(x_2)\big) \lor \big(x_1 = x_2\big)\Big) \land \big(f(x_2) <_Y f(x_3)\big)\bigg)                        \\
             & \lor \bigg(\Big(\big(f(x_1) <_Y f(x_2)\big) \lor \big(x_1 = x_2\big)\Big) \land \big(x_2 = x_3\big)\bigg)                           \\
    \implies & \big(f(x_1) <_Y f(x_2) <_Y f(x_3)\big)                                                                                              \\
             & \lor \Big(\big(x_1 = x_2\big) \land \big(f(x_1) <_Y f(x_3)\big)\Big)                                                                \\
             & \lor \Big(\big(f(x_1) <_Y f(x_2)\big) \land \big(x_2 = x_3\big)\Big)                                                                \\
             & \lor \big(x_1 = x_2 = x_3\big)                                                                                                      \\
    \implies & \big(f(x_1) <_Y f(x_3)\big)                                                                                                         \\
             & \lor \big(f(x_1) <_Y f(x_3)\big)                                                                                                    \\
             & \lor \big(f(x_1) <_Y f(x_3)\big)                                                                                                    \\
             & \lor \big(x_1 = x_3\big)                                                                                                            \\
    \implies & x_1 \leq_X x_3,
  \end{align*}
  we know that \((X, \leq_X)\) is transitive.
  Since \((X, \leq_X)\) is reflexive, anti-symmetric and transitive, by \cref{i:8.5.1} \((X, \leq_X)\) is partially ordered.

  Next we show that \((X, \leq_X)\) may not be totally ordered when \((Y, \leq_Y)\) is totally ordered.
  If \(\exists x, x' \in X : x \neq x' \implies f(x) = f(x')\), then we do not have the ordering relation \(x \leq_X x'\) and \(x' \leq_X x\).
  Thus, by \cref{i:8.5.3} \((X, \leq_X)\) is not totally ordered.

  Next we show that if \(f : X \to Y\) is injective and \((Y, \leq_Y)\) is totally ordered, then \((X, \leq_X)\) is totally ordered.
  Since \(f\) is injective, we know that \(\forall x, x' \in X, x \neq x' \implies f(x) \neq f(x')\).
  Since \(Y\) is totally ordered, we know that \(\big(f(x) \leq_Y f(x')\big) \lor \big(f(x') \leq_Y f(x)\big)\) is true.
  Since \(f(x) \neq f(x')\), we know that exactly one of \(\big(f(x) <_Y f(x')\big) \lor \big(f(x') <_Y f(x)\big)\) is true.
  In either cases we get exactly one of \((x \leq_X x') \lor (x' \leq_X x)\) is true.
  Thus, by \cref{i:8.5.3} \((X, \leq_X)\) is totally ordered.
\end{proof}

\begin{ex}\label{i:ex:8.5.6}
  Let \(X\) be a partially ordered set.
  For any \(x\) in \(X\), define the \emph{order ideal} \((x) \subseteq X\) to be the set \((x) \coloneqq \set{y \in X : y \leq_X x}\).
  Let \((X) \coloneqq \set{(x) : x \in X}\) be the set of all order ideals, and let \(f : X \to (X)\) be the map \(f(x) \coloneqq (x)\) that sends every element of \(x\) to its order ideal.
  Show that \(f\) is a bijection, and that given any \(x, y \in X\), that \(x \leq_X y\) iff \(f(x) \subseteq f(y)\).
  This exercise shows that any partially ordered set can be \emph{represented} by a collection of sets whose ordering relation is given by set inclusion.
\end{ex}

\begin{proof}
  We first show that \(f\) is bijective.
  We start by showing \(f\) is injective.
  Since \((X, \leq_X)\) is partially ordered, by \cref{i:8.5.1} we know that
  \begin{align*}
             & \forall x, x' \in X, f(x) = f(x')                                                               \\
    \implies & (x) = (x')                                                                                      \\
    \implies & \big(x \in (x')\big) \land \big(x' \in (x)\big) &  & \text{(\((X, \leq_X)\) is reflexive)}      \\
    \implies & (x \leq_X x') \land (x' \leq_X x')                                                              \\
    \implies & x = x'.                                         &  & \text{(\((X, \leq_X)\) is anti-symmetric)}
  \end{align*}
  Thus, \(f\) is injective.
  Next we show that \(f\) is surjective.
  \begin{align*}
             & \forall (x) \in (X), x \in X \\
    \implies & f(x) = (x).
  \end{align*}
  Thus, \(f\) is surjective.
  Since \(f\) is both injective and surjective, we know that \(f\) is bijective.

  Finally we show that \(\forall x, y \in X, x \leq_X y \iff f(x) \subseteq f(y)\).
  \begin{align*}
         & \forall x, y \in X, x \leq_X y                                                                \\
    \iff & \forall z \in X, (z \leq_X x \implies z \leq_X y) &  & \text{(\((X, \leq_X)\) is transitive)} \\
    \iff & (x) \subseteq (y)                                                                             \\
    \iff & f(x) \subseteq f(y).
  \end{align*}
\end{proof}

\begin{ex}\label{i:ex:8.5.7}
  Let \(X\) be a partially ordered set with ordering relation \(\leq_X\), and let \(Y\) be a totally ordered subset of \(X\).
  Show that \(Y\) can have at most one maximum and at most one minimum.
\end{ex}

\begin{proof}
  Suppose for the sake of contradiction that \(\exists y, y' \in Y\) such that \(y \neq y'\) and both \(y, y'\) are maximum of \(Y\).
  Then by \cref{i:8.5.5} we have \((y \leq_X y') \land (y' \leq_X y)\), so \(y = y'\), a contradiction.
  Thus, \(Y\) can have at most one maximum.
  Similar arguments show that \(Y\) can have at most one minimum.
\end{proof}

\begin{ex}\label{i:ex:8.5.8}
  Show that every finite non-empty subset of a totally ordered set has a minimum and a maximum.
  Conclude in particular that every finite totally ordered set is well-ordered.
\end{ex}

\begin{proof}
  Let \((X, \leq_X)\) be totally ordered such that \(X \neq \emptyset\), let \(Y \subseteq X\) be a finite set where \(Y \neq \emptyset\) and let \(n = \#(Y)\).
  We induct on \(n\) to show that \(\min\big((Y, \leq_X)\big)\) and \(\max\big((Y, \leq_X)\big)\) exist.
  For \(n = 1\), let \(y \in Y\).
  Then we have
  \begin{align*}
             & \forall y' \in Y, y' = y                                                                                   \\
    \implies & y' = y \leq_X y = y'                                      &  & \text{(\((Y, \leq_X)\) is totally ordered)} \\
    \implies & y = \min\big((Y, \leq_X)\big) = \max\big((Y, \leq_X)\big) &  & \by{i:8.5.5}
  \end{align*}
  and the base case holds.
  Suppose inductively that for some \(n \geq 1\) we know that \(\min\big((Y, \leq_X)\big)\) and \(\max\big((Y, \leq_X)\big)\) exist.
  Then for \(n + 1\), let \(y \in Y\).
  By the induction hypothesis we know that \(y_{\min} = \min\big((Y \setminus \set{y}, \leq_X)\big)\) and \(\max\big((Y \setminus \set{y}, \leq_X)\big)\) exist.
  Since \(y \notin Y \setminus \set{y}\), we know that \(y \neq y_{\min}\).
  Since \(Y \subseteq X\), we know that \(Y\) is totally ordered, and thus we have \((y <_X y_{\min}) \lor (y_{\min} <_X y)\).
  Now we split into two cases:
  \begin{itemize}
    \item If \(y <_X y_{\min}\), then \(y = \min\big((Y, \leq_X)\big)\) since
          \begin{align*}
                     & \forall y' \in Y                                                                       \\
            \implies & (y' = y) \lor (y' \in Y \setminus \set{y})                                             \\
            \implies & (y' = y) \lor (y_{\min} <_X y')            &  & \by{i:8.5.5}                           \\
            \implies & (y' = y) \lor (y <_X y')                   &  & \text{(\((X, \leq_X)\) is transitive)} \\
            \implies & y \leq_X y'.
          \end{align*}
    \item If \(y_{\min} <_X y\), then \(y_{\min} = \min\big((Y, \leq_X)\big)\) since \(\forall y' \in Y, y_{\min} \leq_X y'\).
  \end{itemize}
  From all cases above, we conclude that \(\min\big((Y, \leq_X)\big)\) exists.
  Similar arguments show that \(\max\big((Y, \leq_X)\big)\) exists.
  This closes the induction.
\end{proof}

\begin{ex}\label{i:ex:8.5.9}
  Let \(X\) be a totally ordered set such that every non-empty subset of \(X\) has both a minimum and a maximum.
  Show that \(X\) is finite.
\end{ex}

\begin{proof}
  Let \((X, \leq_X)\) be totally ordered such that \(\forall Y \subseteq X\), \(Y \neq \emptyset\) implies \(\min\big((Y, \leq_X)\big)\) and \(\max\big((Y, \leq_X)\big)\) exist.
  Suppose for the sake of contradiction that \(X\) is infinite.
  Since \(X \subseteq X\), by hypothesis \(\min\big((X, \leq_X)\big)\) and \(\max\big((X, \leq_X)\big)\) exist.
  Define \(x_n\) recursively as follow
  \[
    \forall n \in \N, x_n = \begin{dcases}
      \min\big((X, \leq_X)\big)                                             & \text{if } n = 0 \\
      \min\big((X \setminus \bigcup_{m = 0}^{n - 1} \set{x_m}, \leq_X)\big) & \text{if } n > 0
    \end{dcases}
  \]
  Then we have an strictly increasing sequence \((x_n)_{n = 0}^\infty\), i.e., \(x_0 \leq_X x_1 \leq_X x_2 \dots\).
  Let \(X_n = \set{x_n : n \in \N}\) be the set of all elements in sequence \((x_n)_{n = 0}^\infty\).
  We know that \(X_n \subseteq X\) and \(X_n \neq \emptyset\).
  By hypothesis we know that \(\exists \max\big((X_n, \leq_X)\big) \in X_n\).
  Let \(m \in \N\) be the index of the maximum in \(X_n\), i.e., \(x_m = \min\big((X_n, \leq_X)\big)\).
  But \((x_n)_{n = 0}^\infty\) is an strictly increasing sequence, so we have \(x_m <_X x_{m + 1}\) and \(x_m\) is not a maximum, a contradiction.
  Thus, \(X\) must be finite.
\end{proof}

\begin{ex}\label{i:ex:8.5.10}
  Prove \cref{i:8.5.10}, without using the axiom of choice.
\end{ex}

\begin{proof}
  See \cref{i:8.5.10}.
\end{proof}

\begin{ex}\label{i:ex:8.5.11}
  Let \(X\) be a partially ordered set, and let \(Y\) and \(Y'\) be well-ordered subsets of \(X\).
  Show that \(Y \cup Y'\) is well-ordered iff it is totally ordered.
\end{ex}

\begin{proof}
  Suppose that \((X, \leq_X)\) is partially ordered.
  Let \(Y \subseteq X\), \(Y' \subseteq X\) and both \((Y, \leq_X), (Y', \leq_X)\) are well-ordered.
  By \cref{i:8.5.8} we know that if \((Y \cup Y', \leq_X)\) is well-ordered, then \((Y \cup Y', \leq_X)\) is totally ordered.
  So we only need to show that if \((Y \cup Y', \leq_X)\) is totally ordered, then \((Y \cup Y', \leq_X)\) is well-ordered.
  Suppose that \((Y \cup Y', \leq_X)\) is totally ordered.
  Let \(Z \subseteq Y \cup Y'\) and \(Z \neq \emptyset\).
  Let \(Z_Y = Z \cap Y\) and let \(Z_{Y'} = Z \cap Y'\).
  We know that
  \begin{align*}
             & (Z_Y = \emptyset) \lor (Z_{Y'} = \emptyset)                   \\
    \implies & (Z \subseteq Y') \lor (Z \subseteq Y)                         \\
    \implies & Z \text{ is well-ordered}.                  &  & \by{i:8.5.8}
  \end{align*}
  So suppose that \(Z_Y \neq \emptyset \land Z_{Y'} \neq \emptyset\).
  Since \(Z_Y \subseteq Y\) and \(Z_{Y'} \subseteq Y'\), by \cref{i:8.5.8} \(z_Y = \min\big((Z_Y, \leq_X)\big)\) and \(z_{Y'} = \min\big((Z_{Y'}, \leq_X)\big)\) exist.
  Since \((Y \cup Y', \leq_X)\) is totally order, we know that \((z_Y \leq_X z_{Y'}) \lor (z_{Y'} \leq_X z_Y)\).
  Without the loss of generality suppose that \(z_Y \leq_X z_{Y'}\).
  Then we have
  \begin{align*}
             & \forall z \in Z                                                    \\
    \implies & z \in Y \cup Y'                                                    \\
    \implies & z \in Y \lor z \in Y'                                              \\
    \implies & (z_Y \leq_X z) \lor (z_{Y'} \leq_X z)            &  & \by{i:8.5.5} \\
    \implies & (z_Y \leq_X z) \lor (z_Y \leq_X z_{Y'} \leq_X z) &  & \by{i:8.5.1} \\
    \implies & z_Y \leq_X z                                                       \\
    \implies & z_Y = \min\big((Z, \leq_X)\big).                 &  & \by{i:8.5.5}
  \end{align*}
  This means for any \(Z \subseteq Y \cup Y'\), \(\min\big((Z, \leq_X)\big)\) must exists.
  Thus, by \cref{i:8.5.8} \((Y \cup Y', \leq_X)\) is well-ordered.
\end{proof}

\begin{ex}\label{i:ex:8.5.12}
  Let \(X\) and \(Y\) be partially ordered sets with ordering relations \(\leq_X\) and \(\leq_Y\) respectively.
  Define a relation \(\leq_{X \times Y}\) on the Cartesian product \(X \times Y\) by defining \((x, y) \leq_{X \times Y} (x', y')\) if \(x <_X x'\), or if \(x = x'\) and \(y \leq_Y y'\).
  (This is called the \emph{lexicographical ordering} on \(X \times Y\), and is similar to the alphabetical ordering of words;
  a word \(w\) appears earlier in a dictionary than another word \(w'\) if the first letter of \(w\) is earlier in the alphabet than the first letter of \(w'\), or if the first letters match and the second letter of \(w\) is earlier than the second letter of \(w'\), and so forth.)
  Show that \(\leq_{X \times Y}\) defines a partial ordering on \(X \times Y\).
  Furthermore, show that if \(X\) and \(Y\) are totally ordered, then so is \(X \times Y\), and if \(X\) and \(Y\) are well-ordered, then so is \(X \times Y\).
\end{ex}

\begin{proof}
  We first show that \((X \times Y, \leq_{X \times Y})\) is partially ordered.
  If \(X = \emptyset \lor Y = \emptyset\), then by \cref{i:ex:3.5.8} \(X \times Y = \emptyset\) and by \cref{i:ex:8.5.1} we know that \(\emptyset\) is partially ordered.
  So suppose that \(X \neq \emptyset \land Y \neq \emptyset\).
  Since
  \[
    \forall (x, y) \in X \times Y, x = x \implies (x, y) \leq_{X \times Y} (x, y),
  \]
  we know that \((X \times Y, \leq_{X \times Y})\) is reflexive.
  Since
  \begin{align*}
             & \forall (x, y), (x', y') \in X \times Y,                                                                        \\
             & \big((x, y) \leq_{X \times Y} (x', y')\big) \land \big((x', y') \leq_{X \times Y} (x, y)\big)                   \\
    \implies & \Big((x <_X x') \lor \big((x = x') \land (y \leq_Y y')\big)\Big)                                                \\
             & \land \Big((x' <_X x) \lor \big((x = x') \land (y' \leq_Y y)\big)\Big)                                          \\
    \implies & \Big((x \leq_X x') \land \big((x <_X x') \lor (y \leq_Y y')\big)\Big)                                           \\
             & \land \Big((x' \leq_X x) \land \big((x' <_X x) \lor (y' \leq_Y y)\big)\Big)                                     \\
    \implies & (x = x') \land (y \leq_Y y') \land (y' \leq_Y y)                                                                \\
    \implies & (x = x') \land (y = y')                                                                       &  & \by{i:8.5.1} \\
    \implies & (x, y) = (x', y'),                                                                            &  & \by{i:3.5.1}
  \end{align*}
  we know that \((X \times Y, \leq_{X \times Y})\) is anti-symmetric.
  Since
  \begin{align*}
             & \forall (x_1, y_1), (x_2, y_2), (x_3, y_3) \in X \times Y :                                                                 \\
             & \big((x_1, y_1) \leq_{X \times Y} (x_2, y_2)\big) \land \big((x_2, y_2) \leq_{X \times Y} (x_3, y_3)\big)                   \\
    \implies & \Big((x_1 <_X x_2) \lor \big((x_1 = x_2) \land (y_1 \leq_Y y_2)\big)\Big)                                                   \\
             & \land \Big((x_2 <_X x_3) \lor \big((x_2 = x_3) \land (y_2 \leq_Y y_3)\big)\Big)                                             \\
    \implies & (x_1 <_X x_2 <_X x_3)                                                                                     &  & \by{i:8.5.1} \\
             & \lor \big((x_1 = x_2 <_X x_3) \land (y_1 \leq_Y y_2)\big)                                                                   \\
             & \lor \big((x_1 <_X x_2 = x_3) \land (y_2 \leq_Y y_3)\big)                                                                   \\
             & \lor \big((x_1 = x_2 = x_3) \land (y_1 \leq_Y y_2 \leq_Y y_3)\big)                                        &  & \by{i:8.5.1} \\
    \implies & (x_1, y_1) \leq_{X \times Y} (x_3, y_3),
  \end{align*}
  we know that \((X \times Y, \leq_{X \times Y})\) is transitive.
  Since \((X \times Y, \leq_{X \times Y})\) is reflexive, anti-symmetric and transitive, by \cref{i:8.5.1} \((X \times Y, \leq_{X \times Y})\) is partially ordered.

  Now we show that if \((X, \leq_X), (Y, \leq_Y)\) are totally ordered, then \((X \times Y, \leq_{X \times Y})\) is totally ordered.
  If \(X = \emptyset \lor Y = \emptyset\), then by \cref{i:ex:3.5.8} \(X \times Y = \emptyset\) and by \cref{i:ex:8.5.1} we know that \(\emptyset\) is totally ordered.
  So suppose that \(X \neq \emptyset \land Y \neq \emptyset\).
  From the proof above we know that \((X \times Y, \leq_{X \times Y})\) is partially ordered.
  Since
  \begin{align*}
             & \forall (x, y), (x', y') \in X \times Y, (x, y) \neq (x', y')                   \\
    \implies & (x \neq x') \lor \big((x = x') \land (y \neq y')\big),        &  & \by{i:3.5.1}
  \end{align*}
  we split into two cases:
  \begin{itemize}
    \item If \(x \neq x'\), then \((x <_X x') \lor (x' <_X x)\) since \((X, \leq_X)\) is totally ordered.
          Thus, we have \(\big((x, y) \leq_{X \times Y} (x', y')\big) \lor \big((x, y) \leq_{X \times Y} (x', y')\big)\).
    \item If \((x = x') \land (y \neq y')\), then \((y <_Y y') \lor (y' <_Y y)\) since \((Y, \leq_Y)\) is totally ordered.
          Thus, we have \(\big((x, y) \leq_{X \times Y} (x', y')\big) \lor \big((x, y) \leq_{X \times Y} (x', y')\big)\).
  \end{itemize}
  From all cases above, we conclude that \(\big((x, y) \leq_{X \times Y} (x', y')\big) \lor \big((x, y) \leq_{X \times Y} (x', y')\big)\).
  Thus, \((X \times Y, \leq_{X \times Y})\) is totally ordered.

  Now we show that if \((X, \leq_X), (Y, \leq_Y)\) are well-ordered, then \((X \times Y, \leq_{X \times Y})\) is well-ordered.
  If \(X = \emptyset \lor Y = \emptyset\), then by \cref{i:ex:3.5.8} \(X \times Y = \emptyset\) and by \cref{i:ex:8.5.1} we know that \((\emptyset, \leq_{X \times Y})\) is well-ordered.
  So suppose that \(X \neq \emptyset \land Y \neq \emptyset\).
  From the proof above we know that \((X \times Y, \leq_{X \times Y})\) is totally ordered.
  Let \(Z \subseteq X \times Y\) and \(Z \neq \emptyset\).
  We know that the set \(Z_X = \set{x \in X | \exists y \in Y : (x, y) \in Z} \neq \emptyset\).
  Since \((X, \leq_X)\) is well-ordered and \(Z_X \subseteq X\), by \cref{i:8.5.8} we know that \(z_x = \min\big((Z_X, \leq_X)\big)\) exists.
  Since \(z_x \in Z_X\), we know that the set \(Z_Y = \set{y \in Y : (z_x, y) \in Z} \neq \emptyset\).
  Since \((Y, \leq_Y)\) is well-ordered and \(Z_Y \subseteq Y\), by \cref{i:8.5.8} we know that \(z_y = \min\big((Z_Y, \leq_Y)\big)\) exists.
  Then we have
  \begin{align*}
             & \forall (x, y) \in Z                                                                        \\
    \implies & (z_x <_X x) \lor (x = z_x)                                                                  \\
    \implies & \big((z_x, z_y) \leq_{X \times Y} (x, y)\big) \lor (x = z_x)                                \\
    \implies & \big((z_x, z_y) \leq_{X \times Y} (x, y)\big) \lor \big((x = z_x) \land (y \in Z_y)\big)    \\
    \implies & \big((z_x, z_y) \leq_{X \times Y} (x, y)\big) \lor \big((x = z_x) \land (z_y \leq_Y y)\big) \\
    \implies & (z_x, z_y) \leq_{X \times Y} (x, y).
  \end{align*}
  Thus, \(\min\big((Z, \leq_{X \times Y})\big) = (z_x, z_y)\).
  Since \(Z\) was arbitrary, by \cref{i:8.5.8} we know that \((X \times Y, \leq_{X \times Y})\) is well-ordered.
\end{proof}

\begin{ex}\label{i:ex:8.5.13}
  Prove the claim in the proof of \cref{i:8.5.14}, namely that every element of \(Y' \setminus Y\) is an strict upper bound for \(Y\) and vice versa.
\end{ex}

\begin{proof}
  Since \(Y, Y'\) are good, we know that \(x_0 \in Y \cap Y'\) and thus \(Y \cap Y' \neq \emptyset\).
  Let \(n \in Y \cap Y'\) and let \(P(n)\) be the statement define as follow:
  \[
    \set{y \in Y : y \leq n} = \set{y \in Y' : y \leq n} = \set{y \in Y \cap Y' : y \leq n}.
  \]
  Let \(Q(n)\) be the statement ``if \(P(m)\) is true for every \(m \in Y \cap Y'\) and \(m < n\), then \(P(n)\) is true.''
  Note that if we can show that \(Q(n)\) is true for every \(n \in Y \cap Y'\), then by principle of strong induction (\cref{i:8.5.10}) we know that \(P(n)\) is true for every \(n \in Y \cap Y'\).

  Since \(\min(Y) = \min(Y') = x_0\), we know that \(\min(Y \cap Y') = x_0\) and
  \[
    \set{y \in Y : y \leq x_0} = \set{y \in Y' : y \leq x_0} = \set{y \in Y \cap Y' : y \leq x_0} = \set{x_0}.
  \]
  Thus, \(Q(x_0)\) is vacuously true (there is no element \(z \in Y \cap Y'\) which satisfy \(z < x_0\)).
  Suppose inductively that \(Q(m)\) is true for some \(m \geq x_0\).
  We need to show that \(Q(n)\) is also true for the next smallest item \(n \in Y \cap Y'\).
  By smallest we mean that
  \[
    n = \min\big(\set{n \in Y \cap Y' : Q(m) \text{ is true and } x_0 \leq m < n \text { for every } m \in Y \cap Y'}\big).
  \]
  Such \(n\) is well-defined since in the proof of \cref{i:8.5.14} we suppose for the sake of contradiction that every well-ordered subset \(Y\) of \(X\) which has \(x_0\) as its minimal element has at least one strict upper bound.
  We know that the set \(\set{y \in Y : y \leq n}\) is not empty since it contains \(x_0\), so let \(x \in \set{y \in Y : y \leq n}\).
  Now we split into two cases:
  \begin{itemize}
    \item If \(x = n\), then by the definition of \(n\) we know that \(x \in \set{y \in Y \cap Y' : y \leq n}\).
    \item If \(x < n\), then by the definition of \(n\) we know that by the induction hypothesis \(Q(x)\) is true.
          Then we have
          \begin{align*}
                     & P(x) \text{ is true}                                        \\
            \implies & \set{y \in Y : y \leq x} = \set{y \in Y \cap Y' : y \leq x} \\
            \implies & x \in \set{y \in Y \cap Y' : y \leq x < n}.
          \end{align*}
  \end{itemize}
  From all cases above, we conclude that \(\set{y \in Y : y \leq n} \subseteq \set{y \in Y \cap Y' : y \leq n}\).
  We also have \(\set{y \in Y \cap Y' : y \leq n} \subseteq \set{y \in Y : y \leq n}\) since
  \begin{align*}
             & \forall x \in \set{y \in Y \cap Y' : y \leq n} \\
    \implies & (x \in Y \cap Y') \land (x \leq n)             \\
    \implies & (x \in Y) \land (x \leq n)                     \\
    \implies & x \in \set{y \in Y : y \leq n}.
  \end{align*}
  Thus, we have \(\set{y \in Y : y \leq n} = \set{y \in Y \cap Y' : y \leq n}\).
  Using similar arguments above, we have \(\set{y \in Y' : y \leq n} = \set{y \in Y \cap Y' : y \leq n}\).
  Thus, we conclude that \(P(n)\) is also true, i.e.,
  \[
    \set{y \in Y : y \leq n} = \set{y \in Y' : y \leq n} = \set{y \in Y \cap Y' : y \leq n}.
  \]
  By the induction hypothesis we know that \(Q(m)\) is true for every \(m \in Y \cap Y'\) and \(m < n\).
  Thus, we conclude that \(Q(n)\) is also true, this closes the induction.

  By principle of strong induction (\cref{i:8.5.10}) we know that \(P(n)\) is true for all \(n \in Y \cap Y'\).
  Thus, we have
  \begin{align*}
             & \forall x \in (Y \cap Y') \setminus \set{x_0}, P(x) \text{ is true}                                                         \\
    \implies & \set{y \in Y \cap Y' : y \leq x} = \set{y \in Y : y \leq x}                                                                 \\
    \implies & \set{y \in Y \cap Y' : y \leq x} \setminus \set{x} = \set{y \in Y : y \leq x} \setminus \set{x}                             \\
    \implies & \set{y \in Y \cap Y' : y < x} = \set{y \in Y : y < x}                                                                       \\
    \implies & s(\set{y \in Y \cap Y' : y < x}) = s(\set{y \in Y : y < x}) = x                                 &  & \text{(\(Y\) is good)}
  \end{align*}
  and \(Y \cap Y'\) is good.

  Now we show that if \(Y \setminus Y' \neq \emptyset\), then \(s(Y \cap Y') = \min(Y \setminus Y')\).
  Since \(Y \cap Y'\) is good, \(s(Y \cap Y')\) is well-defined.
  Since \(Y\) is well-ordered and \(Y \setminus Y' \subseteq Y\), \(Y \setminus Y'\) is also well-ordered and thus \(\min(Y \setminus Y')\) is well-defined.
  We know that \(\forall y_1 \in Y\), \(y_1 < \min(Y \setminus Y') \implies y_1 \notin Y \setminus Y'\), otherwise \(y_1 = \min(Y \setminus Y')\), a contradiction.
  Thus, \(y_1 \in Y \cap Y'\) and \(\set{y \in Y : y < \min(Y \setminus Y')} \subseteq Y \cap Y\).
  Similarly, we know that \(\forall y_2 \in Y \cap Y'\), \(y_2 \in Y\) and \(y_2 < \min(Y \setminus Y')\), so \(y_2 \in \set{y \in Y : y < \min(Y \setminus Y')}\) and \(Y \cap Y' \subseteq \set{y \in Y : y < \min(Y \setminus Y')}\).
  We now conclude that \(\set{y \in Y : y < \min(Y \setminus Y')} = Y \cap Y'\) and
  \begin{align*}
             & \min(Y \setminus Y') \in Y \setminus \set{x_0}                                                         \\
    \implies & \min(Y \setminus Y') = s\big(\set{y \in Y : y < \min(Y \setminus Y')}\big) &  & \text{(\(Y\) is good)} \\
    \implies & \min(Y \setminus Y') = s(Y \cap Y').
  \end{align*}
  Similar arguments show that if \(Y' \setminus Y \neq \emptyset\), then \(s(Y \cap Y') = \min(Y' \setminus Y)\).

  Finally we show that every element of \(Y' \setminus Y\) is an upper bound for \(Y\) and vice versa.
  Since \((Y \setminus Y') \cap (Y' \setminus Y) = \emptyset\), we know that at least one of \(Y \setminus Y'\) or \(Y' \setminus Y\) is empty.
  Otherwise we have \(s(Y \cap Y') = \min(Y \setminus Y') = \min(Y' \setminus Y)\), which means \(\big(s(Y \cap Y') \in Y \setminus Y'\big) \land \big(s(Y \cap Y') \in Y' \setminus Y\big)\), a contradiction.
  Now we split into two cases:
  \begin{itemize}
    \item If \(Y \setminus Y' = \emptyset\), then it is vacuously true that every element of \(Y \setminus Y'\) is an strict upper bound of \(Y'\).
    \item If \(Y \setminus Y' \neq \emptyset\), then \(Y' \setminus Y = \emptyset\) and \(Y' \subseteq Y\).
          Since \(Y \cap Y' = Y'\) and \(s(Y \cap Y') = s(Y') = \min(Y \setminus Y')\), by \cref{i:8.5.5} we know that every element of \(Y \setminus Y'\) is an strict upper bound of \(Y'\).
  \end{itemize}
  From all cases above, we conclude that every element of \(Y \setminus Y'\) is an strict upper bound of \(Y'\).
  Using similar arguments, we can show that every element of \(Y' \setminus Y\) is an strict upper bound of \(Y\).
\end{proof}

\begin{ex}\label{i:ex:8.5.14}
  Use \cref{i:8.5.14} to prove \cref{i:8.5.15}.
\end{ex}

\begin{proof}
  See \cref{i:8.5.15}
\end{proof}

\begin{ex}\label{i:ex:8.5.15}
  Let \(A\) and \(B\) be two non-empty sets such that \(A\) does not have lesser or equal cardinality to \(B\).
  Using Zorn's lemma, prove that \(B\) has lesser or equal cardinality to \(A\).
  This exercise (combined with \cref{i:ex:8.3.3}) shows that the cardinality of any two sets is comparable, as long as one assumes the axiom of choice.
\end{ex}

\begin{proof}
  For every subset \(X \subseteq B\), let \(P(X)\) denote the property that there exists an injective map from \(X \to A\).
  Let \(S = \set{X \subseteq B : P(X)}\) be a set.
  Since \(B \neq \emptyset\), let \(b \in B\) and let \(f_{\set{b}} : \set{b} \to A\).
  Clearly, \(f_{\set{b}}\) is injective, so we know that \(S \neq \emptyset\).
  Let \(\leq_S\) be a ordering relation of \(S\) by setting
  \begin{align*}
         & \forall X, Y \in S, X \leq_S Y                                           \\
    \iff & (X \subseteq Y)                                                          \\
         & \land \big(\exists f_X : X \to A \text{ where \(f_X\) is injective}\big) \\
         & \land \big(\exists f_Y : Y \to A \text{ where \(f_Y\) is injective}\big) \\
         & \land \big(\forall x \in X, f_X(x) = f_Y(x)\big).
  \end{align*}
  Since
  \[
    \forall X \in S, (X \subseteq X) \land \big(\forall x \in X, f_X(x) = f_X(x)\big) \implies X \leq_S X,
  \]
  we know that \((S, \leq_S)\) is reflexive.
  Since
  \begin{align*}
             & \forall X, Y \in S, (X \leq_S Y) \land (Y \leq_S X)                    \\
    \implies & (X \subseteq Y) \land (Y \subseteq X)                                  \\
    \implies & X = Y,                                              &  & \by{i:3.1.18}
  \end{align*}
  we know that \((S, \leq_S)\) is anti-symmetric.
  Since
  \begin{align*}
             & \forall X, Y, Z \in S, (X \leq_S Y) \land (Y \leq_S Z)                                                                                  \\
    \implies & (X \subseteq Y) \land (Y \subseteq Z) \land \big(\forall x \in X, f_X(x) = f_Y(x)\big) \land \big(\forall y \in Y, f_Y(y) = f_Z(y)\big) \\
    \implies & (X \subseteq Z) \land \big(\forall x \in X, f_X(x) = f_Z(x)\big)                                                                        \\
    \implies & X \leq_S Z,
  \end{align*}
  we know that \((S, \leq_S)\) is transitive.
  Since \((S, \leq_S)\) is reflexive, anti-symmetric and transitive, by \cref{i:8.5.1} we know that \((S, \leq_S)\) is partially ordered.

  Next we show that \(\forall S' \subseteq S\), if \((S', \leq_S)\) is totally ordered, then \(\bigcup S' \in S\).
  Clearly, we have \(\bigcup S' \subseteq B\).
  So we only need to show that there exists a function \(f : \bigcup S' \to A\) such that \(f\) is injective.
  For each \(X \in S'\), let \(F_X = \set{f_X : X \to A \text{ is injective}}\) be a set.
  Since \(P(X)\) is true, we know that \(F_X \neq \emptyset\).
  By axiom of choice (\cref{i:8.1}) we know that \(\prod_{X \in S'} F_X \neq \emptyset\).
  Let \((f_X)_{X \in S'} \in \prod_{X \in S'} F_X\).
  Fix such \((f_X)_{X \in S'}\).
  We define a function \(f : \bigcup S' \to A\) as follow:
  \[
    \forall x_0 \in \bigcup S', f(x_0) = f_Y(x_0)
  \]
  for some \(Y \in S'\), \(x_0 \in Y\) and \(f_Y = (f_X)_{X \in S'}(Y)\).
  To show that such \(f\) is well-defined, we need to show that \(Y\) is unique for each \(x_0 \in \bigcup S'\).
  Since \((S', \leq_S)\) is totally ordered, we know that \(\forall Z \in S'\), we have \(Z \leq_S Y\) or \(Y \leq_S Z\).
  Let \(f_Z = (f_X)_{X \in S'}(Z)\).
  We split into two cases:
  \begin{itemize}
    \item If \(Z \leq_S Y\), then by the definition of \(\leq_S\) we have \(Z \subseteq Y\) and \(\forall z \in Z\), \(f_Z(z) = f_Y(z)\).
          Now we further split into two cases:
          \begin{itemize}
            \item If \(x_0 \in Z\), then \(f_Z(x_0) = f_Y(x_0)\).
            \item If \(x_0 \notin Z\), then \(\forall z \in Z\), we have \(f_Z(z) = f_Y(z) \neq f_Y(x_0)\) since \(f_Y, f_Z\) are injective.
          \end{itemize}
    \item If \(Y \leq_S Z\), then by the definition of \(\leq_S\) we have \(Y \subseteq Z\) and \(\forall y \in Y\), \(f_Y(y) = f_Z(y)\).
          Thus, \(x_0 \in Z\) and \(f_Y(x_0) = f_Z(x_0)\).
  \end{itemize}
  From all cases above, we conclude that \(f\) is well-defined.
  Now we show that \(f\) is injective.
  Let \(y, z \in \bigcup S'\) and \(y \neq z\).
  Then \(\exists Y, Z \in S'\) such that \(y \in Y\) and \(z \in Z\).
  Again we have either \(Y \leq_S Z\) or \(Z \leq_S Y\).
  Without the loss of generality suppose that \(Y \leq_S Z\).
  Then we have
  \[
    f_Y(y) = f_Z(y) \neq f_Z(z)
  \]
  where \(f_Y = (f_X)_{X \in S'}(Y)\) and \(f_Z = (f_X)_{X \in S'}(Z)\).
  Thus, \(f\) is injective and \(\bigcup S' \in S\).

  Now we show that \(\forall S' \subseteq S\), if \((S', \leq_S)\) is totally ordered, then there exists an upper bound of \(S'\).
  Let \(S' \subseteq S\) such that \((S', \leq_S)\) is totally ordered.
  Since
  \begin{align*}
             & \forall Y \in S'                                                                                                 \\
    \implies & \Big(Y \subseteq \bigcup S' \in S\Big)                                            &  & \text{(from claim above)} \\
             & \land \Big(\forall y \in Y, f_Y(x) = \big((f_X)_{X \in S'}(Y)\big)(x) = f(y)\Big)                                \\
    \implies & Y \leq_S \bigcup S',
  \end{align*}
  we know that \(\bigcup S'\) is an upper bound of \(S'\).
  Since \(S'\) was arbitrary, we conclude that every totally ordered subset of \(S\) with ordering relation \(\leq_S\) has an upper bound.

  By Zorn's lemma (\cref{i:8.5.15}) we know that there exists at least one maximal element of \(S\).
  Suppose for the sake of contradiction that \(B \neq \max\big((S, \leq_S)\big)\).
  Let \(X = \max\big((S, \leq_S)\big)\).
  So \(B \setminus X \neq \emptyset\).
  Then we know that \(P(X)\) is true, i.e., \(\exists f_X : X \to A\) such that \(f_X\) is injective.
  We must have \(f_X(X) = A\), otherwise \(f_X(X) \subseteq A\), we know that \(\exists a \in A \setminus f_X(X)\), and we can let \(b \in B \setminus X\) map to \(a\), i.e., \(f_X(b) = a\).
  This cause \((X \subseteq X \cup \set{b}) \land \big(\forall x \in X, f_X(x) = f_{X \cup \set{b}}(x)\big)\), which means \(X \leq_S X \cup \set{b}\) and contradicts to \(X = \max\big((S, \leq_S)\big)\).
  So we have \(f_X(X) = A\).
  But this also means \(f_X\) is a bijection from \(X\) to \(A\).
  So we can set \(g : A \to B\) as \(g(a) = f_X^{-1}(a)\), which is injective.
  By hypothesis we know that we cannot have a injection from \(A\) to \(B\) (this is the definition of \(A\) having lesser or equal cardinality than \(B\), see \cref{i:ex:3.6.7}).
  Thus, we derived a contradiction.
  So \(B = \max\big((S, \leq_S)\big) \in S\) and \(\exists f_B : B \to A\) such that \(f_B\) is injective.
  By \cref{i:ex:8.3.3} \(B\) has lesser or equal cardinality than \(A\).
\end{proof}

\begin{ex}\label{i:ex:8.5.16}
  Let \(X\) be a set, and let \(P\) be the set of all partial orderings of \(X\).
  (For instance, if \(X \coloneqq \N \setminus \set{0}\), then both the usual partial ordering \(\leq\), and the partial ordering in \cref{i:ex:8.5.3}, are elements of \(P\).)
  We say that one partial ordering \(\leq \in P\) is \emph{coarser} than another partial ordering \(\leq' \in P\) if for any \(x, y \in X\), we have the implication \((x \leq y) \implies (x \leq' y)\).
  Thus, for instance the partial ordering in \cref{i:ex:8.5.3} is coarser than the usual ordering \(\leq\).
  Let us write \(\leq \preceq \leq'\) if \(\leq\) is coarser than \(\leq'\).
  Show that \(\preceq\) turns \(P\) into a partially ordered set;
  thus the set of partial orderings on \(X\) is itself partially ordered.
  There is exactly one minimal element of \(P\);
  what is it?
  Show that the maximal elements of \(P\) are precisely the total orderings of \(P\).
  Using Zorn's lemma (\cref{i:8.5.15}), show that given any partial ordering \(\leq\) of \(X\) there exists a total ordering \(\leq'\) such that \(\leq\) is coarser than \(\leq'\).
\end{ex}

\begin{proof}
  Since
  \begin{align*}
             & \forall (X, \leq) \in P                          \\
    \implies & (\forall x, y \in X, x \leq y \implies x \leq y) \\
    \implies & \leq \preceq \leq,
  \end{align*}
  we know that \((P, \preceq)\) is reflexive.
  Since
  \begin{align*}
             & \forall (X, \leq_1), (X, \leq_2) \in P, (\leq_1 \preceq \leq_2) \land (\leq_2 \preceq \leq_1)          \\
    \implies & \big(\forall x, y \in X, (x \leq_1 y \implies x \leq_1 y') \land (x \leq_2 y \implies x \leq_1 y)\big) \\
    \implies & \big(\forall x, y \in X, (x \leq_1 y \iff x \leq_2 y)\big)                                             \\
    \implies & \leq_1 = \leq_2,
  \end{align*}
  we know that \((P, \preceq)\) is anti-symmetric.
  Since
  \begin{align*}
             & \forall (X, \leq_1), (X, \leq_2), (X, \leq_3) \in P, (\leq_1 \preceq \leq_2) \land (\leq_2 \preceq \leq_3) \\
    \implies & \big(\forall x, y \in X, (x \leq_1 y \implies x \leq_2 y) \land (x \leq_2 y \implies x \leq_3 y)\big)      \\
    \implies & \big(\forall x, y \in X, (x \leq_1 y \implies x \leq_3 y)\big)                                             \\
    \implies & \leq_1 \preceq \leq_3,
  \end{align*}
  we know that \((P, \preceq)\) is transitive.
  Since \((P, \preceq)\) is reflexive, anti-symmetric and transitive, by \cref{i:8.5.1} we know that \((P, \preceq)\) is partially ordered.

  Next we show that \((P, \preceq)\) has exactly one minimal element.
  We claim that the equivalent relation on \(X\), denote as \(=_X\), is the minimal element of \((P, \preceq)\).
  To show that \((X, =_X)\) is partially ordered, since equivalent relation is reflexive and transitive, we only need to show that \((X, =_X)\) is anti-symmetric.
  Since
  \[
    \forall x, y \in X, (x =_X y) \land (y =_X x) \implies x =_X y,
  \]
  we know that \((X, =_X)\) is anti-symmetric.
  Thus, by \cref{i:8.5.1} \((X, =_X)\) is partially ordered and \((X, =_X) \in P\).
  Since
  \begin{align*}
             & \forall (X, \leq) \in P                                           \\
    \implies & (\forall x, y \in X, x =_X y \implies x \leq y) &  & \by{i:8.5.1} \\
    \implies & =_X \preceq \leq,
  \end{align*}
  by \cref{i:8.5.5} we have \((X, =_X) = \min\big((P, \preceq)\big)\).

  Next we show that if \(S \subseteq P\), \(S \neq \emptyset\) and \((S, \preceq)\) is totally ordered, then the relation \((X, \leq_S)\) defined as follow is a element of \(P\):
  \[
    \forall x, y \in X, x \leq_S y \iff \exists \leq \in S : x \leq y.
  \]
  Since
  \begin{align*}
             & \forall x \in X, \forall \leq \in S, x \leq x &  & \text{(\((X, \leq)\) is partially ordered)} \\
    \implies & x \leq_S x,
  \end{align*}
  we know that \((X, \leq_S)\) is reflexive.
  Since
  \begin{align*}
             & \forall x, y \in X, (x \leq_S y) \land (y \leq_S x)                                                                                        \\
    \implies & \exists \leq_1, \leq_2 \in S : (x \leq_1 y) \land (y \leq_2 x)                           &  & \text{(\((X, \leq)\) is partially ordered)}  \\
    \implies & (\leq_2 \preceq \leq_1) \lor (\leq_1 \preceq \leq_2)                                     &  & \text{(\((S, \preceq)\) is totally ordered)} \\
    \implies & \big((x \leq_1 y) \land (y \leq_1 x)\big) \lor \big((x \leq_2 y) \land (y \leq_2 x)\big)                                                   \\
    \implies & x =_X y,
  \end{align*}
  we know that \((X, \leq_S)\) is anti-symmetric.
  Since
  \begin{align*}
             & \forall x, y, z \in X, (x \leq_S y) \land (y \leq_S z)                                                                                     \\
    \implies & \exists \leq_1, \leq_2 \in S : (x \leq_1 y) \land (y \leq_2 z)                           &  & \text{(\((X, \leq)\) is partially ordered)}  \\
    \implies & (\leq_2 \preceq \leq_1) \lor (\leq_1 \preceq \leq_2)                                     &  & \text{(\((S, \preceq)\) is totally ordered)} \\
    \implies & \big((x \leq_1 y) \land (y \leq_1 z)\big) \lor \big((x \leq_2 y) \land (y \leq_2 z)\big)                                                   \\
    \implies & (x \leq_1 z) \lor (x \leq_2 z)                                                                                                             \\
    \implies & x \leq_S z,
  \end{align*}
  we know that \((X, \leq_S)\) is transitive.
  Since \((X, \leq_S)\) is reflexive, anti-symmetric and transitive, by \cref{i:8.5.1} we know that \((X, \leq_S)\) is partially ordered.
  Thus, \((X, \leq_S) \in P\).

  Next we show that every totally ordered subset of \(P\) has an upper bound.
  Let \(S \subseteq P\) such that \((S, \preceq)\) is totally ordered.
  If \(S = \emptyset\), by \cref{i:ex:8.5.1} we know that \((\emptyset, \preceq)\) is totally ordered, and the statement
  \[
    \forall (X, \leq) \in \emptyset, \leq \preceq \leq'
  \]
  is vacuously true for every \((X, \leq') \in P\).
  Thus, \((\emptyset, \preceq)\) has an upper bound in \(P\).
  So suppose that \(S \neq \emptyset\).
  Then the relation \((S, \leq_S)\) defined as above is an upper bound of \(S\) since for every \((X, \leq) \in S\), we have
  \begin{align*}
             & \forall x, y \in X, x \leq y                                                    \\
    \implies & x \leq_S y                   &  & \text{(by the definition of \((X, \leq_S)\))} \\
    \implies & \leq \preceq \leq_S.
  \end{align*}
  Thus, every totally ordered subset of \(P\) has an upper bound.

  By Zorn's lemma (\cref{i:8.5.15}) we know that \((X, \leq_{\max}) = \max\big((P, \preceq)\big)\) exists.
  We show that \((X, \leq_{\max})\) is totally ordered.
  Suppose for the sake of contradiction that \((X, \leq_{\max})\) is not totally ordered.
  Then \(\exists a, b \in X\) such that \(a \not\leq_{\max} b\) and \(b \not\leq_{\max} a\).
  Note that we must have \(a \neq_X b\) since \(\leq_{\max}\) is reflexive.
  Now we define a relation \((X, \leq_p)\) as follow:
  \[
    \forall x, y \in X, x \leq_p y \iff (x, y) = (a, b) \lor (x =_X y)
  \]
  We show that \((X, \leq_p)\) is partially ordered.
  Since
  \[
    \forall x \in X, x =_X x \implies x \leq_p x,
  \]
  we know that \((X, \leq_p)\) is reflexive.
  Since
  \begin{align*}
         & \forall x, y \in X, (x \leq_p y) \land (y \leq_p x)                                                           \\
    \iff & \begin{dcases}
             \big((x, y) = (a, b)\big) \land \big((y, x) = (a, b)\big) & \iff \text{false} \\
             \big((x, y) = (a, b)\big) \land (y =_X x)                 & \iff \text{false} \\
             (x =_X y) \land \big((y, x) = (a, b)\big)                 & \iff \text{false} \\
             (x =_X y) \land (y =_X x)                                 & \iff (x =_X y)
           \end{dcases}                                 \\
    \iff & (\text{false}) \lor (x =_X y)                                                                                 \\
    \iff & (x =_X y),                                                                       &  & \text{(vacuously true)}
  \end{align*}
  we know that \((X, \leq_p)\) is anti-symmetric.
  Since
  \begin{align*}
         & \forall x, y, z \in X, (x \leq_p y) \land (y \leq_p z)                                                           \\
    \iff & \begin{dcases}
             \big((x, y) = (a, b)\big) \land \big((y, z) = (a, b)\big) & \iff \text{false}    \\
             \big((x, y) = (a, b)\big) \land (y =_X z)                 & \iff (x, z) = (a, b) \\
             (x =_X y) \land \big((y, z) = (a, b)\big)                 & \iff (x, z) = (a, b) \\
             (x =_X y) \land (y =_X z)                                 & \iff x =_X y =_X z
           \end{dcases}                                 \\
    \iff & (\text{false}) \lor \big((x, z) = (a, b)\big) \lor (x =_X z)                                                     \\
    \iff & x \leq_p z,                                                                         &  & \text{(vacuously true)}
  \end{align*}
  we know that \((X, \leq_p)\) is transitive.
  Since \((X, \leq_p)\) is reflexive, anti-symmetric and transitive, by \cref{i:8.5.1} we know that \((X, \leq_p)\) is partially ordered.
  Thus, \((X, \leq_p) \in P\).
  Then we must have \(\leq_p \preceq \leq_{\max}\).
  But we know that \(a \leq_p b\) does not implies \(a \leq_{\max} b\), a contradiction.
  Thus, \((X, \leq_{\max})\) is totally ordered.

  Finally we show that any partial order \((X, \leq)\), there exists a total order \((X, \leq')\) such that \(\leq \preceq \leq'\).
  Since \(\leq \preceq \leq_{\max}\) and \((X, \leq_{\max})\) is totally ordered, the claim follows.
\end{proof}

\begin{ex}\label{i:ex:8.5.17}
  Use Zorn's lemma (\cref{i:8.5.15}) to give another proof of the claim in \cref{i:ex:8.4.2}.
  Deduce that Zorn's lemma (\cref{i:8.5.15}) and the axiom of choice (\cref{i:8.1}) are in fact logically equivalent
  (i.e., they can be deduced from each other).
\end{ex}

\begin{proof}
  Let \(I\) be a set, and for each \(\alpha \in I\) let \(X_{\alpha}\) be a non-empty set.
  Suppose that all the sets \(X_{\alpha}\) are disjoint from each other, i.e., \(X_{\alpha} \cap X_{\beta} = \emptyset\) for all distinct \(\alpha, \beta \in I\).
  We want to show that there exists a set \(Y\) such that \(\#(Y \cap X_{\alpha}) = 1\) for all \(\alpha \in I\).

  We define a set \(\Omega\) as follow:
  \[
    \Omega = \set{Y \subseteq \bigcup_{\alpha \in I} X_{\alpha} : \#(Y \cap X_{\alpha}) \leq 1 \text{ for all } \alpha \in I}.
  \]
  Clearly, \(\emptyset \in \Omega\), thus \(\Omega \neq \emptyset\).
  By \cref{i:8.5.2} we know that \((\Omega, \subseteq)\) is partially ordered.

  Let \(S \subseteq \Omega\) such that \((S, \subseteq)\) is totally ordered.
  We show that \(\bigcup S \in \Omega\).
  If \(S = \emptyset\) or \(S = \set{\emptyset}\), then \(\bigcup S = \emptyset\) and \(\emptyset \in \Omega\).
  So suppose that \(S \neq \emptyset\) and \(S \neq \set{\emptyset}\).
  Since
  \begin{align*}
             & \forall \beta \in I, \forall x_1, x_2 \in \Big(X_{\beta} \cap \bigcup S\Big)                                                                 \\
    \implies & \exists Y_1, Y_2 \in S : (x_1 \in X_{\beta} \cap Y_1) \land (x_2 \in X_{\beta} \cap Y_2)                                                     \\
             & \land (Y_1 \subseteq Y_2 \lor Y_2 \subseteq Y_1)                                         &  & \text{(\((S, \subseteq)\) is totally ordered)} \\
    \implies & (x_1, x_2 \in X_{\beta} \cap Y_1) \lor (x_1, x_2 \in X_{\beta} \cap Y_2)                                                                     \\
    \implies & \big(\#(X_{\beta} \cap Y_1) = \#(X_{\beta} \cap Y_2) = 1\big) \land (x_1 = x_2)                                                              \\
    \implies & \#\Big(X_{\beta} \cap \bigcup S\Big) = 1,
  \end{align*}
  we know that \(\bigcup S \in \Omega\).
  Thus, we conclude that if \(S \subseteq \Omega\) and \((S, \subseteq)\) is totally ordered, then \(\bigcup S \in \Omega\).

  Now we show that every totally ordered subset of \(\Omega\) has an upper bound.
  Let \(S \subseteq \Omega\) such that \((S, \subseteq)\) is totally ordered.
  If \(S = \emptyset\), then \(\emptyset \subseteq \emptyset\) and the claim follows vacuously.
  So suppose that \(S \neq \emptyset\).
  Since \(\bigcup S \in \Omega\) and
  \[
    \forall Y \in S, Y \subseteq \bigcup S,
  \]
  by \cref{i:8.5.12} we know that \(\bigcup S\) is an upper bound of \(S\) (under the relation \(\subseteq\)).
  Thus, we conclude that every totally ordered subset of \(\Omega\) has an upper bound.
  By Zorn's lemma (\cref{i:8.5.15}) we know that \(M = \max\big((\Omega, \subseteq)\big)\) exists.
  We claim that \(\#(M \cap X_{\alpha}) = 1\) for every \(\alpha \in I\).
  Suppose for the sake of contradiction that \(\exists \beta \in I\) such that \(\#(M \cap X_{\beta}) = \#(\emptyset) = 0\).
  Let \(x_{\beta} \in X_{\beta}\).
  Now we define a set \(Y_{\beta} = \set{x_{\beta}}\).
  Clearly, \(\#(Y_{\beta} \cap X_{\alpha}) = \#(\emptyset) = 0\) for every \(\alpha \in I \setminus \set{\beta}\), and \(\#(Y_{\beta} \cap X_{\beta}) = \#(\set{x_{\beta}}) = 1\).
  Thus, we have \(Y_{\beta} \in \Omega\).
  But then we have
  \begin{align*}
             & Y_{\beta} \subseteq M           \\
    \implies & x_{\beta} \in M                 \\
    \implies & M \cap X_{\beta} \neq \emptyset \\
    \implies & \#(M \cap X_{\beta}) \neq 0,
  \end{align*}
  a contradiction.
  Thus, we have \(\#(M \cap X_{\alpha}) = 1\) for every \(\alpha \in I\).
  By \cref{i:ex:8.4.2} this means Zorn's lemma and the axiom of choice are in fact logically equivalent.
\end{proof}

\begin{ex}\label{i:ex:8.5.18}
  Using Zorn's lemma (\cref{i:8.5.15}), prove \emph{Hausdorff's maximality principle}:
  if \(X\) is a partially ordered set, then there exists a totally ordered subset \(Y\) of \(X\) which is maximal with respect to set inclusion
  (i.e. there is no other totally ordered subset \(Y'\) of \(X\) which contains \(Y\)).
  Conversely, show that if Hausdorff's maximality principle is true, then Zorn's lemma (\cref{i:8.5.15}) is true.
  Thus, by \cref{i:ex:8.5.17}, these two statements are logically equivalent to the axiom of choice.
\end{ex}

\begin{proof}
  We first show that Zorn's lemma (\cref{i:8.5.15}) implies Hausdorff's maximality principle.
  Suppose that \((X, \leq)\) is partially ordered.
  Let \(X_T = \set{Y \subseteq X : (Y, \leq) \text{ is totally ordered}}\).
  By \cref{i:ex:8.5.1} we know that \((\emptyset, \leq)\) is totally ordered, thus \(\emptyset \in X_T\) and \(X_T \neq \emptyset\).
  By \cref{i:8.5.2} we know that \((X_T, \subseteq)\) is partially ordered.

  Let \(S \subseteq X_T\) such that \((S, \subseteq)\) is totally ordered.
  We claim that \(\bigcup S \in X_T\).
  Clearly, \(\bigcup S \subseteq X\).
  If \(S = \emptyset\), then \(\bigcup S = \emptyset\) and \(\emptyset \in X_T\).
  So suppose that \(S \neq \emptyset\).
  Since
  \[
    \forall x \in \bigcup S, \exists Y \in S : x \in Y \implies x \leq x,
  \]
  we know that \((\bigcup S, \leq)\) is reflexive.
  Since
  \begin{align*}
             & \forall x_1, x_2 \in \bigcup S, (x_1 \leq x_2) \land (x_2 \leq x_1)                                                     \\
    \implies & \exists Y_1, Y_2 \in S : (x_1 \in Y_1) \land (x_2 \in Y_2)                                                              \\
             & \land (Y_1 \subseteq Y_2 \lor Y_2 \subseteq Y_1)                    &  & \text{(\((S, \subseteq)\) is totally ordered)} \\
    \implies & (x_1, x_2 \in Y_1) \lor (x_1, x_2 \in Y_2)                                                                              \\
    \implies & x_1 = x_2,
  \end{align*}
  (the last implication follows since \((Y_1, \leq)\), \((Y_2, \leq)\) are partially ordered)
  we know that \((\bigcup S, \leq)\) is anti-symmetric.
  Since
  \begin{align*}
             & \forall x_1, x_2, x_3 \in \bigcup S, (x_1 \leq x_2) \land (x_2 \leq x_3)                                                              \\
    \implies & \exists Y_1, Y_2, Y_3 \in S :                                                                                                         \\
             & (x_1 \in Y_1) \land (x_2 \in Y_2) \land (x_3 \in Y_3)                                                                                 \\
             & \land (Y_1 \subseteq Y_2 \lor Y_2 \subseteq Y_1)                                  &  & \text{(\((S, \subseteq)\) is totally ordered)} \\
             & \land (Y_1 \subseteq Y_3 \lor Y_3 \subseteq Y_1)                                                                                      \\
             & \land (Y_2 \subseteq Y_3 \lor Y_3 \subseteq Y_2)                                                                                      \\
    \implies & (x_1, x_2, x_3 \in Y_1) \lor (x_1, x_2, x_3 \in Y_2) \lor (x_1, x_2, x_3 \in Y_3)                                                     \\
    \implies & x_1 \leq x_3,
  \end{align*}
  (the last implication follows since \((Y_1, \leq)\), \((Y_2, \leq)\), \((Y_3, \leq)\) are partially ordered)
  we know that \((\bigcup S, \leq)\) is transitive.
  Since \((\bigcup S, \leq)\) is reflexive, anti-symmetric and transitive, by \cref{i:8.5.1} we know that \((\bigcup S, \leq)\) is partially ordered.
  Since
  \begin{align*}
             & \forall x_1, x_2 \in \bigcup S, \exists Y_1, Y_2 \in S :                                                                    \\
             & (x_1 \in Y_1) \land (x_2 \in Y_2)                                                                                           \\
             & \land (Y_1 \subseteq Y_2 \lor Y_2 \subseteq Y_1)         &  & \text{(\((S, \subseteq)\) is totally ordered)}                \\
    \implies & (x_1, x_2 \in Y_1) \lor (x_1, x_2 \in Y_2)                                                                                  \\
    \implies & (x_1 \leq x_2) \lor (x_2 \leq x_1),                      &  & \text{(\((Y_1, \leq)\), \((Y_2, \leq)\) are totally ordered)}
  \end{align*}
  by \cref{i:8.5.3} we know that \((\bigcup S, \leq)\) is totally ordered.
  Thus, \(\bigcup S \in X_T\).

  Next we show that every totally ordered subset of \(X_T\) has an upper bound.
  Let \(S \subseteq X_T\) such that \((S, \subseteq)\) is totally ordered.
  Since \(\bigcup S \in X_T\) and
  \[
    \forall Y \in S, Y \subseteq \bigcup S,
  \]
  by \cref{i:8.5.12} we know that \(\bigcup S\) is an upper bound of \(S\) (under the relation \(\subseteq\)).
  Thus, the claim follows.

  By Zorn's lemma (\cref{i:8.5.15}) we know that \(\max\big((X_T, \subseteq)\big)\) exists.
  By \cref{i:8.5.5} we know that
  \[
    \forall Y \in X_T, Y \subseteq \max\big((X_T, \subseteq)\big).
  \]
  Thus, we conclude that Zorn's lemma implies Hausdorff's maximality principle.

  Finally we show that Hausdorff's maximality principle implies Zorn's lemma.
  Suppose that \((X, \leq)\) is partially ordered and every totally ordered subset of \(X\) has an upper bound.
  Let \(X_T = \set{Y \subseteq X : (Y, \leq) \text{ is totally ordered}}\).
  By hypothesis we know that \(Y_{\max} = \max\big((X_T, \subseteq)\big)\) exists.
  We want to show that \(\max\big((X, \leq)\big)\) exists.

  Suppose for the sake of contradiction that \(\max\big((X, \leq)\big)\) does not exists.
  Then every totally ordered subset of \(X\) must have a strictly upper bound.
  To be precise, let \(Y \subseteq X\) such that \((Y, \leq)\) is totally ordered.
  Let \(u(Y)\) be the upper bound of \(Y\).
  Then \(\exists s(Y) \in X\) such that \(s(Y)\) is a strict upper bound of \(Y\), i.e., \(u(Y) < s(Y)\).
  If such strict upper bound \(s(Y)\) does not exists, then by \cref{i:8.5.5} we must have \(u(Y) = \max\big((X, \leq)\big)\), a contradiction.
  Since \((Y_{\max}, \leq)\) is totally ordered, we know that \(\exists s(Y_{\max}) \in X\) such that \(u(Y_{\max}) < s(Y_{\max})\).
  But then \(\big(Y_{\max} \cup \set{s(Y_{\max})}, \leq\big)\) is totally ordered and \(Y_{\max} \subseteq Y_{\max} \cup \set{s(Y_{\max})}\), a contradiction.
  Thus, \(\max\big((X, \leq)\big)\) must exists.
  We conclude that Zorn's lemma, Hausdorff's maximality principle and axiom of choice are logically equivalent.
\end{proof}

\begin{ex}\label{i:ex:8.5.19}
  Let \(X\) be a set, and let \(\Omega\) be the space of all pairs \((Y, \leq)\), where \(Y\) is a subset of \(X\) and \(\leq\) is a well-ordering of \(Y\).
  If \((Y, \leq)\) and \((Y', \leq')\) are elements of \(\Omega\), we say that \((Y, \leq)\) is an \emph{initial segment} of \((Y', \leq')\) if there exists an \(x \in Y'\) such that \(Y \coloneqq \set{y \in Y' : y <' x}\) (so, in particular, \(Y \subsetneq Y'\)), and for any \(y, y' \in Y\), \(y \leq y'\) iff \(y \leq' y'\).
  Define a relation \(\preceq\) on \(\Omega\) by defining \((Y, \leq) \preceq (Y', \leq')\) if either \((Y, \leq) = (Y', \leq')\), or if \((Y, \leq)\) is an initial segment of \((Y', \leq')\).
  Show that \(\preceq\) is a partial ordering of \(\Omega\).
  There is exactly one minimal element of \(\Omega\);
  what is it?
  Show that the maximal elements of \(\Omega\) are precisely the well-orderings \((X, \leq)\) of \(X\).
  Using Zorn's lemma (\cref{i:8.5.15}), conclude the \emph{well-ordering principle}:
  every set \(X\) has at least one well-ordering.
  Conversely, use the well-ordering principle to prove the axiom of choice, \cref{i:8.1}.
  We thus see that the axiom of choice, Zorn's lemma (\cref{i:8.5.15}), and the well-ordering principle are all logically equivalent to each other.
\end{ex}

\begin{proof}
  We first show that \((\Omega, \preceq)\) is partially ordered.
  Since
  \[
    \forall (Y, \leq), \in \Omega, (Y, \leq) = (Y, \leq) \implies (Y, \leq) \preceq (Y, \leq),
  \]
  we know that \((\Omega, \preceq)\) is reflexive.
  Since
  \begin{align*}
         & \forall (Y_1, \leq_1), (Y_2, \leq_2) \in \Omega, \big((Y_1, \leq_1) \preceq (Y_2, \leq_2)\big) \land \big((Y_2, \leq_2) \preceq (Y_1, \leq_1)\big) \\
    \iff & \begin{dcases}
             \big((Y_1, \leq_1) = (Y_2, \leq_2)\big) \land \big((Y_2, \leq_2) = (Y_1, \leq_1)\big)     \\
             \big((Y_1, \leq_1) = (Y_2, \leq_2)\big) \land (Y_2 \text{ is an initial segment of } Y_1) \\
             (Y_1 \text{ is an initial segment of } Y_2) \land \big((Y_2, \leq_2) = (Y_1, \leq_1)\big) \\
             (Y_1 \text{ is an initial segment of } Y_2) \land (Y_2 \text{ is an initial segment of } Y_1)
           \end{dcases}                                                       \\
    \iff & \begin{dcases}
             (Y_1, \leq_1) = (Y_2, \leq_2)          \\
             (Y_1 = Y_2) \land (Y_2 \subsetneq Y_1) \\
             (Y_1 \subsetneq Y_2) \land (Y_1 = Y_2) \\
             (Y_1 \subsetneq Y_2) \land (Y_2 \subsetneq Y_1)
           \end{dcases}                                                                                                     \\
    \iff & \big((Y_1, \leq_1) = (Y_2, \leq_2)\big) \lor (\text{false})                                                                                        \\
    \iff & (Y_1, \leq_1) = (Y_2, \leq_2),
  \end{align*}
  we know that \((\Omega, \preceq)\) is anti-symmetric.
  Since
  \begin{align*}
         & \forall (Y_1, \leq_1), (Y_2, \leq_2), (Y_3, \leq_3) \in \Omega, \big((Y_1, \leq_1) \preceq (Y_2, \leq_2)\big) \land \big((Y_2, \leq_2) \preceq (Y_3, \leq_3)\big) \\
    \iff & \begin{dcases}
             \big((Y_1, \leq_1) = (Y_2, \leq_2)\big) \land \big((Y_2, \leq_2) = (Y_3, \leq_3)\big)     \\
             \big((Y_1, \leq_1) = (Y_2, \leq_2)\big) \land (Y_2 \text{ is an initial segment of } Y_3) \\
             (Y_1 \text{ is an initial segment of } Y_2) \land \big((Y_2, \leq_2) = (Y_3, \leq_3)\big) \\
             (Y_1 \text{ is an initial segment of } Y_2) \land (Y_2 \text{ is an initial segment of } Y_3)
           \end{dcases}                                                                      \\
    \iff & \begin{dcases}
             (Y_1, \leq_1) = (Y_3, \leq_3)                                                 \\
             Y_1 \text{ is an initial segment of } Y_3                                     \\
             Y_1 \text{ is an initial segment of } Y_3                                     \\
             \big((\exists y_2 \in Y_2 : Y_1 = \set{y \in Y_2 : y <_2 y_2})                \\
             \quad \land (\forall y_1, y_1' \in Y_1, y_1 \leq_1 y_1' \iff y_1 \leq_2 y_1') \\
             \quad \land (\exists y_3 \in Y_3 : Y_2 = \set{y \in Y_3 : y <_3 y_3})         \\
             \quad \land (\forall y_2, y_2' \in Y_2, y_2 \leq_2 y_2' \iff y_2 \leq_3 y_2')\big)
           \end{dcases}                                                                                 \\
    \iff & \begin{dcases}
             (Y_1, \leq_1) = (Y_3, \leq_3)                                  \\
             Y_1 \text{ is an initial segment of } Y_3                      \\
             Y_1 \text{ is an initial segment of } Y_3                      \\
             \big((\exists y_3 \in Y_3 : Y_1 = \set{y \in Y_3 : y <_2 y_3}) \\
             \quad \land (\forall y_1, y_1' \in Y_1, y_1 \leq_1 y_1' \iff y_1 \leq_3 y_1')\big)
           \end{dcases}                                                                                 \\
    \iff & \big((Y_1, \leq_1) = (Y_3, \leq_3)\big) \lor (Y_1 \text{ is an initial segment of } Y_3)                                                                          \\
    \iff & (Y_1, \leq_1) \preceq (Y_3, \leq_3),
  \end{align*}
  we know that \((\Omega, \preceq)\) is transitive.
  Since \((\Omega, \preceq)\) is reflexive, anti-symmetric and transitive, by \cref{i:8.5.1} we know that \((\Omega, \preceq)\) is partially ordered.

  Next we show that \(\min\big((\Omega, \preceq)\big)\) exists.
  We claim that \((\emptyset, \leq_{\emptyset}) = \min\big((\Omega, \preceq)\big)\).
  By \cref{i:ex:8.5.1} we know that \((\emptyset, \leq_{\emptyset})\) is well-ordered.
  Let \((Y, \leq) \in \Omega\).
  Now we split into two cases:
  \begin{itemize}
    \item If \(Y = \emptyset\), then we know that
          \[
            \forall y_1, y_2 \in \emptyset, y_1 \leq_{\emptyset} y_2 \iff y_1 \leq y_2
          \]
          is vacuously true.
          Thus, we have \((\emptyset, \leq_{\emptyset}) = (\emptyset, \leq)\) and \((\emptyset, \leq_{\emptyset}) \preceq (\emptyset, \leq)\).
    \item If \(Y \neq \emptyset\), then by \cref{i:8.5.8} we know that \(y_{\min} = \min\big((Y, \leq)\big)\) exists.
          By \cref{i:8.5.5} we know that
          \[
            \emptyset = \set{y \in Y : y < y_{\min}}.
          \]
          Since
          \[
            \forall y_1, y_2 \in \emptyset, y_1 \leq_{\emptyset} y_2 \iff y_1 \leq y_2
          \]
          is vacuously true, we know that \(\emptyset\) is an initial segment of \((Y, \leq)\).
          Thus, \((\emptyset, \leq_{\emptyset}) \preceq (\emptyset, \leq)\).
  \end{itemize}
  From all cases above, we conclude that \((\emptyset, \leq_{\emptyset}) \preceq (\emptyset, \leq)\).
  Since \((Y, \leq)\) was arbitrary, we know that \((\emptyset, \leq_{\emptyset}) = \min\big((\Omega, \preceq)\big)\).

  Next we show that if \(S \subseteq \Omega\) and \((S, \preceq)\) is well-ordered, then there exists a well-ordered relation \(\leq_S\) on \(\bigcup S\) and \((\bigcup S, \leq_S) \in \Omega\).
  If \(S = \emptyset \lor S = \set{\emptyset}\), then \(\bigcup S = \emptyset\).
  By \cref{i:ex:8.5.1} we know that \((\emptyset, \leq_{\emptyset})\) is well-ordered, thus \((\emptyset, \leq_{\emptyset}) \in \Omega\).
  So suppose that \(S \neq \emptyset \land S \neq \set{\emptyset}\).
  Observe that
  \begin{align*}
             & \forall y_1, y_2 \in \bigcup S                                                                                                                  \\
    \implies & \exists (Y_1, \leq_1), (Y_2, \leq_2) \in S : (y_1 \in Y_1) \land (y_2 \in Y_2)                                                                  \\
    \implies & \big((Y_1, \leq_1) \preceq (Y_2, \leq_2)\big) \lor \big((Y_2, \leq_2) \preceq (Y_1, \leq_1)\big) &  & \text{(\((S, \preceq)\) is well-ordered)} \\
    \implies & \begin{dcases}
                 (Y_1, \leq_1) = (Y_2, \leq_2)             \\
                 Y_1 \text{ is an initial segment of } Y_2 \\
                 Y_2 \text{ is an initial segment of } Y_1
               \end{dcases}                                                                                                       \\
    \implies & (Y_1 \subseteq Y_2) \lor (Y_2 \subseteq Y_1)                                                                                                    \\
    \implies & (y_1 \leq_1 y_2) \lor (y_1 \leq_2 y_2).
  \end{align*}
  We know that \(\exists (Y_S, \leq_S) \in S\) such that \(\forall y \in \bigcup S \implies y \in Y_S\).
  If not, then we would have some \(y_1, y_2 \in \bigcup S\) such that
  \[
    (y_1 \in Y_S) \land (y_2 \in Y_S') \land (y_1 \notin Y_S' \lor y_2 \notin Y_S) \text{ for some } (Y_S, \leq_S), (Y_S', \leq_S') \in S.
  \]
  But this contradicts \(Y_S \subseteq Y_S' \lor Y_S' \subseteq Y_S\).
  So such \((Y_S, \leq_S)\) must exists.
  Since \((Y_S, \leq_S)\) is well-ordered and \(\bigcup S \subseteq Y_S\), by \cref{i:8.5.8} \((\bigcup S, \leq_S) \in \Omega\).

  Next we show that if \(S \subseteq \Omega\) and \((S, \preceq)\) is totally ordered, then there exists an upper bound of \((S, \preceq)\) in \(\Omega\).
  Since \((\bigcup S, \leq_S) \in \Omega\) and
  \begin{align*}
             & \forall (Y, \leq) \in S, Y \subseteq \bigcup S \\
    \implies & \begin{dcases}
                 Y = \bigcup S \\
                 Y \subsetneq \bigcup S
               \end{dcases}                          \\
    \implies & \begin{dcases}
                 (Y, \leq) = (\bigcup S, \leq_S) \\
                 Y \text{ is an initial segment of } \bigcup S
               \end{dcases}   \\
    \implies & (Y, \leq) \preceq (\bigcup S, \leq_S),
  \end{align*}
  we know that \((\bigcup S, \leq_S)\) is an upper bound of \((S, \preceq)\).
  Since \(S\) was arbitrary, we conclude that every totally ordered subset of \(\Omega\) has an upper bound.

  By Zorn's lemma (\cref{i:8.5.15}) we know that \((Y_{\max}, \leq_{\max}) = \max\big((\Omega, \preceq)\big)\) exists.
  By the definition of \(\Omega\) we know that \((Y_{\max}, \leq_{\max})\) is well-ordered.
  Now we show that \(X = Y_{\max}\).
  By the definition of \(\Omega\) we know that \(Y_{\max} \subseteq X\).
  So we only need to show that \(X \subseteq Y_{\max}\).
  Suppose for the sake of contradiction that \(\exists y \in X\) such that \(y \notin Y_{\max}\).
  We claim that \((\set{y}, =)\) is well-ordered.
  Since
  \[
    \forall y_1, y_2 \in \set{y}, (y_1 = y_2) \land (y_2 = y_1) \implies y_1 = y_2 = y,
  \]
  we know that \((\set{y}, =)\) is anti-symmetric.
  Since \((\set{y}, =)\) is reflexive, anti-symmetric and transitive, by \cref{i:8.5.1} we know that \((\set{y}, =)\) is partially ordered.
  Since
  \[
    \forall y_1, y_2 \in \set{y}, y_1 = y_2 = y,
  \]
  by \cref{i:8.5.3} we know that \((\set{y}, =)\) is totally ordered.
  Since every non-emptyset subset of \(\set{y} = \set{y}\), by \cref{i:8.5.5} \(\min\big((\set{y}, =)\big) = y\) and by \cref{i:8.5.8} \((\set{y}, =)\) is well-ordered.
  Then we have \((\set{y}, =) \in \Omega\) and \((\set{y}, =) \preceq (Y_{\max}, \leq_{\max})\), which means \(y \in Y_{\max}\), a contradiction.
  Thus, \(X \subseteq Y_{\max}\), \(X = Y_{\max}\) and \((X, \leq_{\max})\) is well-ordered.
  We conclude that the well-ordering principle is true, i.e., every \(X\) has at least one well-ordering \(\max\big((\Omega, \preceq)\big)\).

  Finally we use well-ordering principle to prove axiom of choice (\cref{i:8.1}).
  Suppose that every set has at least one well-ordering.
  Let \(I\) be a set and let every \(\alpha \in I\) map to a set \(X_{\alpha}\).
  Since \(\bigcup_{\alpha \in I} X_{\alpha}\) is a set, we know that there exists a well-ordering \(\leq\) on \(\bigcup_{\alpha \in I} X_{\alpha}\).
  Since \(X_{\beta} \subseteq \bigcup_{\alpha \in I} X_{\alpha}\) for every \(\beta \in I\), by \cref{i:8.5.8} we know that \(\min\big((X_{\beta}, \leq)\big)\) exists.
  Now we define \(f : I \to \bigcup_{\alpha \in I} X_{\alpha}\) as follow:
  \[
    \forall \alpha \in I, f(\alpha) = \min\big((X_{\alpha}, \leq)\big)
  \]
  Then we know that \(f \in \prod_{\alpha \in I} X_{\alpha}\), so \(\prod_{\alpha \in I} X_{\alpha} \neq \emptyset\) and axiom of choice is true.
\end{proof}

\begin{ex}\label{i:ex:8.5.20}
  Let \(X\) be a set, and let \(\Omega \subseteq 2^X\) be a collection of subsets of \(X\).
  Assume that \(\Omega\) does not contain the empty set \(\emptyset\).
  Using Zorn's lemma, show that there is a subcollection \(\Omega' \subseteq \Omega\) such that all the elements of \(\Omega'\) are disjoint from each other (i.e., \(A \cap B = \emptyset\) whenever \(A, B\) are distinct elements of \(\Omega'\)), but that all the elements of \(\Omega\) intersect at least one element of \(\Omega'\) (i.e., for all \(C \in \Omega\) there exists \(A \in \Omega'\) such that \(C \cap A \neq \emptyset\)).
  Conversely, if the above claim is true, show that it implies the claim in \cref{i:ex:8.4.2}, and thus this is yet another claim which is logically equivalent to the axiom of choice.
\end{ex}

\begin{proof}
  Let \(S\) be the set
  \[
    S = \set{A \subseteq \Omega : \forall A_1, A_2 \in A, A_1 \neq A_2 \implies A_1 \cap A_2 = \emptyset}.
  \]
  By \cref{i:8.5.2} we know that \((S, \subseteq)\) is partially ordered.
  Let \(S_T \subseteq S\) such that \((S_T, \subseteq)\) is totally ordered.
  We claim that \(\bigcup S_T \in S\) and \(\bigcup S_T\) is an upper bound of \((S_T, \subseteq)\).
  Clearly, \(\bigcup S_T \subseteq \Omega\).
  Since
  \begin{align*}
             & \forall A_1, A_2 \in \bigcup S_T, A_1 \neq A_2                                                                     \\
    \implies & \exists S_1, S_2 \in S_T : (A_1 \in S_1) \land (A_2 \in S_2)                                                       \\
    \implies & (S_1 \subseteq S_2) \lor (S_2 \subseteq S_1)                 &  & \text{(\((S_T, \subseteq)\) is totally ordered)} \\
    \implies & (A_1, A_2 \in S_1) \lor (A_1, A_2 \in S_2)                                                                         \\
    \implies & A_1 \cap A_2 = \emptyset,
  \end{align*}
  we know that \(\bigcup S_T \in S\).
  Since \(\bigcup S_T \in S\) and
  \[
    \forall A \in S_T, A \subseteq \bigcup S_T,
  \]
  we know that \(\bigcup S_T\) is an upper bound of \((S_T, \subseteq)\).
  Thus, we conclude that every totally ordered subset of \(S\) has an upper bounded.

  By Zorn's lemma (\cref{i:8.5.15}) we know that \(\Omega' = \max\big((S, \subseteq)\big)\) exists.
  Clearly, \(\Omega' \subseteq \Omega\).
  Since \(\Omega' \in S\), we know that all the elements of \(\Omega'\) are disjoint from each other.
  Now we show that \(\forall C \in \Omega\), \(\exists A \in \Omega'\) such that \(C \cap A \neq \emptyset\).
  Suppose for the sake of contradiction that \(\exists C \in \Omega\) such that \(\forall A \in \Omega'\), \(C \cap A = \emptyset\).
  By hypothesis we know that \(C \neq \emptyset\).
  Then we have \(\set{C} \cup \Omega' \in S\) and \(\Omega' \subsetneq \set{C} \cup \Omega'\), a contradiction.
  Thus, the claim is true.

  Now we show that if for any set \(X\), \(\Omega \subseteq 2^X\) and \(\exists \Omega' \subseteq \Omega\) such that
  \[
    \forall A_1, A_2 \in \Omega', A_1 \neq A_2 \implies A_1 \cap A_2
  \]
  and
  \[
    \forall C \in \Omega, \exists A \in \Omega' : C \cap A \neq \emptyset,
  \]
  then \cref{i:ex:8.4.2} is true.
  Suppose that \(I\) is a set and for each \(\alpha \in I\), \(X_{\alpha}\) is a set and \(X_{\alpha} \neq \emptyset\).
  Suppose that \(\forall \alpha, \beta \in I\), \(X_{\alpha} \cap X_{\beta} = \emptyset\).
  We want to show that \(\exists Y\) such that \(\#(Y \cap X_{\alpha}) = 1\) for all \(\alpha \in I\).
  Let \(X\) be the set
  \[
    X = (\set{0} \times I) \cup (\set{1} \times \bigcup_{\alpha \in I} X_{\alpha}),
  \]
  let \(\Omega \subseteq 2^X\) be the set
  \[
    \Omega = \set{\set{(0, \alpha), (1, x_{\alpha})} : (\alpha \in I) \land (x_{\alpha} \in X_{\alpha})}.
  \]
  We know that \(\exists \Omega' \subseteq \Omega\) satisfying the hypothesis.
  Let \(Y\) be the set
  \[
    Y = \set{x_{\beta} \in \bigcup_{\alpha \in I} X_{\alpha} | \exists \beta \in I : \set{(0, \beta), (1, x_{\beta})} \in \Omega'}.
  \]
  We claim that \(\#(Y \cap X_{\alpha}) = 1\) for all \(\alpha \in I\).
  Since
  \begin{align*}
             & \forall \alpha \in I, X_{\alpha} \neq \emptyset                                                                                                                            \\
    \implies & \exists x_{\alpha} \in X_{\alpha}                                                                                                                                          \\
    \implies & \exists \set{(0, \alpha), (1, x_{\alpha})} \in \Omega                                                                                                                      \\
    \implies & \exists \set{(0, \beta), (1, x_{\beta})} \in \Omega' : \set{(0, \alpha), (1, x_{\alpha})} \cap \set{(0, \beta), (1, x_{\beta})} \neq \emptyset &  & \text{(by hypothesis)} \\
    \implies & (\alpha \neq \beta \implies X_{\alpha} \cap X_{\beta} = \emptyset \implies x_{\alpha} \neq x_{\beta})                                          &  & \text{(by hypothesis)} \\
    \implies & (\alpha = \beta) \land (x_{\alpha}, x_{\beta} \in X_{\alpha} = X_{\beta})                                                                                                  \\
    \implies & (x_\alpha \in Y) \land (Y \cap X_{\alpha} \neq \emptyset)
  \end{align*}
  and
  \begin{align*}
             & \forall x_{\alpha}, x_{\beta} \in Y                                                                                                        \\
    \implies & \exists \alpha, \beta \in I : \set{(0, \alpha), (1, x_{\alpha})}, \set{(0, \beta), (1, x_{\beta})} \in \Omega'                             \\
    \implies & \set{(0, \alpha), (1, x_{\alpha})} \cap \set{(0, \beta), (1, x_{\beta})} = \emptyset                           &  & \text{(by hypothesis)} \\
    \implies & (\alpha \neq \beta) \land (x_{\alpha} \neq x_{\beta})                                                                                      \\
    \implies & \#(Y \cap X_{\alpha}) = \#(Y \cap X_{\beta}) = 1,
  \end{align*}
  we conclude that \cref{i:ex:8.4.2} is true.
\end{proof}


\chapter{Continuous functions on R}\label{i:ch:9}

\begin{note}
  Roughly speaking a set is discrete if each element is separated from the rest of the set by some non-zero distance, whereas a set is a \emph{continuum} if it is connected and contains no ``holes''.
\end{note}

\section{Subsets of the real line}\label{sec:9.1}

\begin{defn}[Intervals]\label{9.1.1}
  Let \(a, b \in \R^*\) be extended real numbers.
  We define the \emph{closed interval} \([a, b]\) by
  \[
    [a, b] \coloneqq \{x \in \R^* : a \leq x \leq b\},
  \]
  the \emph{half-open intervals} \([a, b)\) and \((a, b]\) by
  \[
    [a, b) \coloneqq \{x \in \R^* : a \leq x < b\}; (a, b] \coloneqq \{x \in \R^* : a < x \leq b\},
  \]
  and the \emph{open interval} \((a, b)\) by
  \[
    (a, b) \coloneqq \{x \in \R^* : a < x < b\}.
  \]
  We call \(a\) the \emph{left endpoint} of these intervals, and \(b\) the \emph{right endpoint}.
\end{defn}

\begin{rmk}\label{9.1.2}
  Once again, we are overloading the parenthesis notation;
  for instance, we are now using \((2, 3)\) to denote both an open interval from \(2\) to \(3\), as well as an ordered pair in the Cartesian plane \(\R^2 \coloneqq \R \times \R\).
  This can cause some genuine ambiguity, but the reader should still be able to resolve which meaning of the parentheses is intended from context.
  In some texts, this issue is resolved by using reversed brackets instead of parenthesis, thus for instance \([a, b)\) would now be \([a, b[\), \((a, b]\) would be \(]a, b]\), and \((a, b)\) would be \(]a, b[\).
\end{rmk}

\begin{eg}\label{9.1.3}
  The positive real axis \(\{x \in \R : x > 0\}\) is the open interval \((0, +\infty)\), while the non-negative real axis \(\{x \in \R : x \geq 0\}\) is the half-open interval \([0, +\infty)\).
      Similarly, the negative real axis \(\{x \in \R : x < 0\}\) is \((-\infty, 0)\), and the non-positive real axis \(\{x \in \R : x \leq 0\}\) is \((-\infty, 0]\).
  Finally, the real line \(\R\) itself is the open interval \((-\infty, +\infty)\), while the extended real line \(\R^*\) is the closed interval \([-\infty, +\infty]\).
  We sometimes refer to an interval in which one endpoint is infinite (either \(+\infty\) or \(-\infty\)) as \emph{half-infinite} intervals, and intervals in which both endpoints are infinite as \emph{doubly-infinite} intervals;
  all other intervals are \emph{bounded intervals}.
  Thus the positive and negative real axes are half-infinite intervals, and \(\R\) and \(\R^*\) are infinite intervals.
\end{eg}

\begin{eg}\label{9.1.4}
  If \(a > b\) then all four of the intervals \([a, b], [a, b), (a, b]\), and \((a, b)\) are the empty set (by trichotomy, see \cref{5.4.7}(a)).
  If \(a = b\), then the three intervals \([a, b), (a, b]\), and \((a, b)\) are the empty set, while \([a, b]\) is just the singleton set \(\{a\}\).
  Because of this, we call these intervals \emph{degenerate};
  most (but not all) of our analysis will be restricted to non-degenerate intervals.
\end{eg}

\begin{defn}[\(\varepsilon\)-adherent points]\label{9.1.5}
  Let \(X\) be a subset of \(\R\), let \(\varepsilon > 0\), and let \(x \in \R\).
  We say that \(x\) is \emph{\(\varepsilon\)-adherent to \(X\)} iff there exists a \(y \in X\) which is \(\varepsilon\)-close to \(x\)
  (i.e., \(\abs{x - y} \leq \varepsilon\)).
\end{defn}

\begin{rmk}\label{9.1.6}
  The terminology ``\(\varepsilon\)-adherent'' is not standard in the literature.
  However, we shall shortly use it to define the notion of an adherent point, which is standard.
\end{rmk}

\setcounter{thm}{7}
\begin{defn}[Adherent points]\label{9.1.8}
  Let \(X\) be a subset of \(\R\), and let \(x \in \R\).
  We say that \(x\) is an \emph{adherent point} of \(X\) iff it is \(\varepsilon\)-adherent to \(X\) for every \(\varepsilon > 0\).
\end{defn}

\setcounter{thm}{9}
\begin{defn}[Closure]\label{9.1.10}
  Let \(X\) be a subset of \(\R\).
  The \emph{closure} of \(X\), sometimes denoted \(\overline{X}\) is defined to be the set of all the adherent points of \(X\).
\end{defn}

\begin{lem}[Elementary properties of closures]\label{9.1.11}
  Let \(X\) and \(Y\) be arbitrary subsets of \(\R\).
  Then \(X \subseteq \overline{X}\), \(\overline{X \cup Y} = \overline{X} \cup \overline{Y}\), and \(\overline{X \cap Y} \subseteq \overline{X} \cap \overline{Y}\).
  If \(X \subseteq Y\), then \(\overline{X} \subseteq \overline{Y}\).
\end{lem}

\begin{proof}
  We first show that \(X \subseteq \overline{X}\).
  Since
  \begin{align*}
             & \forall x \in X, \abs{x - x} = 0                                                          \\
    \implies & \forall \varepsilon \in \R^+, \abs{x - x} \leq \varepsilon                                \\
    \implies & x \in \overline{X},                                        &  & \text{(by \cref{9.1.10})}
  \end{align*}
  by \cref{3.1.15} we have \(X \subseteq \overline{X}\).

  Next we show that \(\overline{X \cup Y} = \overline{X} \cup \overline{Y}\).
  Since
  \begin{align*}
             & \forall x \in \overline{X} \cup \overline{Y}                                                                                            \\
    \implies & x \in \overline{X} \lor x \in \overline{Y}                                                               &  & \text{(by \cref{3.4})}    \\
    \implies & (\exists\ a \in X : \abs{x - a} \leq \varepsilon) \lor (\exists\ a \in Y : \abs{x - a} \leq \varepsilon) &  & \text{(by \cref{9.1.10})} \\
    \implies & \exists\ a \in X \cup Y : \abs{x - a} \leq \varepsilon                                                   &  & \text{(by \cref{3.4})}    \\
    \implies & x \in \overline{X \cup Y}                                                                                &  & \text{(by \cref{9.1.10})}
  \end{align*}
  and
  \begin{align*}
             & \forall \varepsilon \in \R^+, \forall x \in \overline{X \cup Y}, \exists\ a \in X \cup Y : \abs{x - a} \leq \varepsilon &  & \text{(by \cref{9.1.10})} \\
    \implies & \begin{dcases}
                 \exists\ \varepsilon' \in \R^+ : \forall b \in X, \abs{x - b} > \varepsilon' & \text{if } x \notin \overline{X} \\
                 \exists\ \varepsilon' \in \R^+ : \forall b \in Y, \abs{x - b} > \varepsilon' & \text{if } x \notin \overline{Y}
               \end{dcases}      &  & \by{9.1.8}                                        \\
    \implies & \begin{dcases}
                 (a \in Y) \land (\abs{x - a} < \varepsilon') & \text{if } x \notin \overline{X} \\
                 (a \in X) \land (\abs{x - a} < \varepsilon') & \text{if } x \notin \overline{Y}
               \end{dcases}                                                                        \\
    \implies & \begin{dcases}
                 x \in \overline{Y} & \text{if } x \notin \overline{X} \\
                 x \in \overline{X} & \text{if } x \notin \overline{Y}
               \end{dcases}                                                                &  & \text{(by \cref{9.1.10})}                                             \\
    \implies & x \in \overline{X} \cup \overline{Y},                                                                                   &  & \text{(by \cref{3.4})}
  \end{align*}
  by \cref{3.1.18} we have \(\overline{X \cup Y} = \overline{X} \cup \overline{Y}\).

  Next we show that \(\overline{X \cap Y} \subseteq \overline{X} \cap \overline{Y}\).
  Since
  \begin{align*}
             & \forall \varepsilon \in \R^+, \forall x \in \overline{X \cap Y}, \exists\ y \in X \cap Y : \abs{x - y} \leq \varepsilon &  & \text{(by \cref{9.1.10})} \\
    \implies & (y \in X) \land (y \in Y)                                                                                               &  & \text{(by \cref{3.1.23})} \\
    \implies & (x \in \overline{X}) \land (x \in \overline{Y})                                                                         &  & \text{(by \cref{9.1.10})} \\
    \implies & x \in \overline{X} \cap \overline{Y},                                                                                   &  & \text{(by \cref{3.1.23})}
  \end{align*}
  by \cref{3.1.15} we have \(\overline{X \cap Y} \subseteq \overline{X} \cap \overline{Y}\).

  Finally we show that \(X \subseteq Y \implies \overline{X} \subseteq \overline{Y}\).
  Suppose that \(X \subseteq Y\).
  Then we have
  \begin{align*}
             & \forall \varepsilon \in \R^+, \forall x \in \overline{X}, \exists\ y \in X : \abs{x - y} \leq \varepsilon &                 & \text{(by \cref{9.1.10})} \\
    \implies & y \in Y                                                                                                   & (X \subseteq Y)                             \\
    \implies & x \in \overline{Y}.                                                                                       &                 & \text{(by \cref{9.1.10})}
  \end{align*}
  Thus by \cref{3.1.15} we have \(\overline{X} \subseteq \overline{Y}\).
\end{proof}

\begin{lem}[Closures of intervals]\label{9.1.12}
  Let \(a < b\) be real numbers, and let \(I\) be any one of the four intervals \((a, b)\), \((a, b]\), \([a, b)\), or \([a, b]\).
  Then the closure of \(I\) is \([a, b]\).
  Similarly, the closure of \((a, \infty)\) or \([a, \infty)\) is \([a, \infty)\), while the closure of \((-\infty, a)\) or \((-\infty, a]\) is \((-\infty, a]\).
  Finally, the closure of \((-\infty, \infty)\) is \((-\infty, \infty)\).
\end{lem}

\begin{proof}
  First let us show that every element of \([a, b]\) is adherent to \((a, b)\).
  Let \(x \in [a, b]\).
  If \(x \in (a, b)\) then it is definitely adherent to \((a, b)\).
  This is true since \(\forall \varepsilon \in \R^+\) we have \(\abs{x - x} \leq \varepsilon\).
  If \(x = b\) then \(x\) is also adherent to \((a, b)\).
  Otherwise \(\exists\ \varepsilon \in \R^+\) such that
  \[
    \forall y \in (a, b), \abs{b - y} > \varepsilon.
  \]
  But this means
  \begin{align*}
             & \abs{b - y} = b - y > \varepsilon                        & (y \in (a, b))                                         \\
    \implies & b - \varepsilon > y                                                                                               \\
    \implies & b > b - \varepsilon > y > a                              & (\varepsilon \in \R^+ \land y \in (a, b))              \\
    \implies & b - \varepsilon \in (a, b)                               &                                           & \by{9.1.1} \\
    \implies & \varepsilon < \abs{b - (b - \varepsilon)} = \varepsilon,
  \end{align*}
  a contradiction.
  Thus \(b\) is adherent to \((a, b)\).
  Similarly when \(x = a\).
  Thus every point in \([a, b]\) is adherent to \((a, b)\).

  Now we show that every point \(x\) that is adherent to \((a, b)\) lies in \([a, b]\).
  Suppose for sake of contradiction that \(x\) does not lie in \([a, b]\), then either \(x > b\) or \(x < a\).
  If \(x > b\) then \(x\) is not \((x - b)\)-adherent to \((a, b)\), and is hence not an adherent point to \((a, b)\)
  (by setting \(\varepsilon = x - b\) we have \(\forall y \in (a, b)\), \(\abs{x - y} = x - y > x - b = \varepsilon\)).
  Similarly, if \(x < a\), then \(x\) is not \((a - x)\)-adherent to \((a, b)\), and is hence not an adherent point to \((a, b)\).
  This contradiction shows that \(x\) is in fact in \([a, b]\) as claimed.
  Using similar arguments we can show that \(\overline{(a, b]} = \overline{[a, b)} = \overline{[a, b]} = [a, b]\).

  Now we show that \(\overline{(a, \infty)} = [a, \infty)\).
  By \cref{9.1.11} we know that \((a, \infty) \subseteq \overline{(a, \infty)}\).
  We also know that \(a\) is an adherent point of \((a, \infty)\).
  If not, then \(\exists\ \varepsilon \in \R^+\) such that
  \[
    \forall y \in (a, \infty), \abs{a - y} > \varepsilon.
  \]
  But this means
  \begin{align}
             & \abs{a - y} = y - a > \varepsilon                        & (y \in (a, \infty))                                    \\
    \implies & y > a + \varepsilon                                                                                               \\
    \implies & b > y > a + \varepsilon > a                              & (\varepsilon \in \R^+ \land y \in (a, b))              \\
    \implies & a + \varepsilon \in (a, b)                               &                                           & \by{9.1.1} \\
    \implies & \varepsilon < \abs{a - (a - \varepsilon)} = \varepsilon,
  \end{align}
  a contradiction.
  Thus \(a\) is an adherent point of \((a, \infty)\) and \([a, \infty) \subseteq \overline{(a, \infty)}\).
  Suppose for sake of contradiction that \(\exists\ x \in \overline{(a, \infty)}\) such that \(x \notin [a, \infty)\).
  Then \(x < a\) (by \cref{9.1.8} \(x \in \R\) so \(x \neq \infty\)).
  But we know \(x\) is not \((a - x)\)-adherent to \((a, \infty)\), and is hence not an adherent point to \((a, b)\), a contradiction.
  Thus \(\overline{(a, \infty)} = [a, \infty)\).
  Using similar arguments we can show that \(\overline{[a, \infty)} = [a, \infty)\) and \(\overline{(-\infty, b)} = \overline{(-\infty, b]} = (-\infty, b]\).

  Finally we show that \(\overline{(-\infty, \infty)} = (-\infty, \infty)\).
  By \cref{9.1.11} we know that \((-\infty, \infty) \subseteq \overline{(-\infty, \infty)}\).
  By \cref{9.1.8} we know that \(\overline{(-\infty, \infty)} \subseteq \R = (-\infty, \infty)\).
  Thus \(\overline{(-\infty, \infty)} = (-\infty, \infty)\).
\end{proof}

\begin{lem}\label{9.1.13}
  The closure of \(\N\) is \(\N\).
  The closure of \(\Z\) is \(\Z\).
  The closure of \(\Q\) is \(\R\), and the closure of \(\R\) is \(\R\).
  The closure of the empty set \(\emptyset\) is \(\emptyset\).
\end{lem}

\begin{proof}
  We first show that \(\overline{\N} = \N\).
  Let \(\overline{\N}\) be the closure of \(\N\).
  By \cref{9.1.8} we have \(\overline{\N} \subseteq \R\).
  By \cref{9.1.11} we have \(\N \subseteq \overline{\N}\).
  Now we show that \(\overline{\N} \subseteq \N\).
  Suppose for sake of contradiction that \(\exists\ x \in \overline{\N}\) such that \(x \notin \N\).
  Since
  \begin{align*}
             & \N \subseteq [0, \infty)                       &  & \by{9.1.1}                \\
    \implies & \overline{\N} \subseteq \overline{[0, \infty)} &  & \text{(by \cref{9.1.11})} \\
    \implies & \overline{\N} \subseteq [0, \infty),           &  & \text{(by \cref{9.1.12})}
  \end{align*}
  we know that \(x > 0\).
  By \cref{5.4.12} \(\exists\ n \in \N\) such that \(n < x < n + 1\).
  Let \(\varepsilon = \min(x - n, n + 1 - x) / 2\).
  By \cref{9.1.10}, \(\exists\ m \in \N\) such that \(\abs{x - m} \leq \varepsilon\).
  We split into two cases:
  \begin{itemize}
    \item If \(m \leq n\), then we have \(x - m \geq x - n \geq \min(x - n, n + 1 - x) > \varepsilon\), a contradiction.
    \item If \(m > n\), then we have \(m \geq n + 1\) and \(m - x \geq n + 1 - x \geq \min(x - n, n + 1 - x) > \varepsilon\), a contradiction.
  \end{itemize}
  From all cases above we derived contradictions.
  Thus such \(m\) does not exists and by \cref{9.1.10} \(x \notin \overline{\N}\).
  So we have \(\overline{\N} \subseteq \N\).
  Since \(\N \subseteq \overline{\N} \land \overline{\N} \subseteq \N\), by \cref{3.1.18} we have \(\N = \overline{\N}\).

  Next we show that \(\overline{\Z} = \Z\).
  Let \(\overline{\Z}\) be the closure of \(\Z\) and let \(\Z^- = \{z \in \Z : z < 0\}\).
  Then we have
  \begin{align*}
    \overline{\Z} & = \overline{\N \cup \Z^-}                                           \\
                  & = \overline{\N} \cup \overline{\Z^-} &  & \text{(by \cref{9.1.11})} \\
                  & = \N \cup \overline{\Z^-}.           &  & \text{(from proof above)}
  \end{align*}
  Thus to show that \(\Z = \overline{\Z}\), it suffices to show that \(\Z^- = \overline{\Z^-}\).
  By \cref{9.1.11} we have \(\Z^- \subseteq \overline{\Z^-}\).
  We need to show that \(\overline{\Z^-} \subseteq \Z^-\).
  Suppose for sake of contradiction that \(\exists\ x \in \overline{\Z^-}\) such that \(x \notin \Z^-\).
  Since
  \begin{align*}
             & \Z^- \subseteq (-\infty, 0)                       &  & \by{9.1.1}                \\
    \implies & \overline{\Z^-} \subseteq \overline{(-\infty, 0)} &  & \text{(by \cref{9.1.11})} \\
    \implies & \overline{\Z^-} \subseteq (-\infty, 0],           &  & \text{(by \cref{9.1.12})}
  \end{align*}
  we know that \(x < 0\).
  By \cref{5.4.12} \(\exists\ n \in \Z^-\) such that \(n < x < n + 1\).
  Let \(\varepsilon = \min(x - n, n + 1 - x) / 2\).
  By \cref{9.1.10}, \(\exists\ m \in \Z^-\) such that \(\abs{x - m} \leq \varepsilon\).
  We split into two cases:
  \begin{itemize}
    \item If \(m \leq n\), then we have \(x - m \geq x - n \geq \min(x - n, n + 1 - x) > \varepsilon\), a contradiction.
    \item If \(m > n\), then we have \(m \geq n + 1\) and \(m - x \geq n + 1 - x \geq \min(x - n, n + 1 - x) > \varepsilon\), a contradiction.
  \end{itemize}
  From all cases above we derived contradictions.
  Thus such \(m\) does not exists and by \cref{9.1.10} \(x \notin \overline{\Z^-}\).
  So we have \(\overline{\Z^-} \subseteq \Z^-\).
  Since \(\Z^- \subseteq \overline{\Z^-} \land \overline{\Z^-} \subseteq \Z^-\), by \cref{3.1.18} we have \(\Z^- = \overline{\Z^-}\), and thus \(\Z = \overline{\Z}\).

  Next we show that \(\overline{\Q} = \R\).
  Let \(\overline{\Q}\) be the closure of \(\Q\).
  We have
  \begin{align*}
             & \Q \subseteq \R = (-\infty, \infty)                  &  & \by{9.1.1}                \\
    \implies & \overline{\Q} \subseteq \overline{(-\infty, \infty)} &  & \text{(by \cref{9.1.11})} \\
    \implies & \overline{\Q} \subseteq (-\infty, \infty) = \R.      &  & \text{(by \cref{9.1.12})}
  \end{align*}
  Since
  \begin{align*}
             & \forall x \in \R, \forall \varepsilon \in \R^+, x - \varepsilon < x < x + \varepsilon                                \\
    \implies & \exists\ q \in \Q : x - \varepsilon < q < x + \varepsilon                             &  & \text{(by \cref{5.4.14})} \\
    \implies & \abs{x - q} < \varepsilon                                                                                            \\
    \implies & x \in \overline{\Q},                                                                  &  & \text{(by \cref{9.1.10})}
  \end{align*}
  by \cref{3.1.15} we have \(\R \subseteq \overline{\Q}\).
  Since \(\R \subseteq \overline{\Q} \land \overline{\Q} \subseteq \R\), by \cref{3.1.18} we have \(\R = \overline{\Q}\).

  Next we show that \(\overline{\R} = \R\).
  Since
  \begin{align*}
         & \R = (-\infty, \infty)                                            &  & \by{9.1.1}                \\
    \iff & \overline{\R} = \overline{(-\infty, \infty)} = (-\infty, \infty), &  & \text{(by \cref{9.1.12})}
  \end{align*}
  we know that \(\overline{\R} = \R\).

  Finally we show that \(\overline{\emptyset} = \emptyset\).
  Suppose for sake of contradiction that \(\overline{\emptyset} \neq \emptyset\).
  Let \(x \in \overline{\emptyset}\)
  Then by \cref{9.1.10} \(\forall \varepsilon \in \R^+\), \(\exists\ y \in \emptyset\) such that \(\abs{x - y} \leq \varepsilon\), a contradiction.
  Thus \(\overline{\emptyset} = \emptyset\).
\end{proof}

\begin{lem}\label{9.1.14}
  Let \(X\) be a subset of \(\R\), and let \(x \in \R\).
  Then \(x\) is an adherent point of \(X\) if and only if there exists a sequence \((a_n)_{n = 0}^\infty\), consisting entirely of elements in \(X\), which converges to \(x\).
\end{lem}

\begin{proof}
  We first show that if \(x\) is an adherent point of \(X\), then there exists a sequence \((a_n)_{n = 0}^\infty\) such that \(\forall n \in \N\), \(a_n \in X\) and \(\lim_{n \to \infty} a_n = x\).
  For each \(n \in \N\) let \(A_n\) be the set
  \[
    A_n = \{y \in X : \abs{x - y} \leq \dfrac{1}{n}\}.
  \]
  We know by \cref{9.1.10} that \(A_n \neq \emptyset\).
  By axiom of choice (\cref{8.1}) we know \(\prod_{n \in \N} A_n \neq \emptyset\).
  Let \(f \in \prod_{n \in \N} A_n\).
  We can define a sequence \((a_n)_{n = 0}^\infty\) by setting \(a_n = f(n)\).
  Then we have
  \begin{align*}
             & \forall n \in \N, a_n \in A_n                                         \\
    \implies & 0 \leq \abs{x - a_n} \leq \dfrac{1}{n}                                \\
    \implies & \lim_{n \to \infty} \abs{x - a_n} = 0  &  & \text{(by \cref{6.4.14})} \\
    \implies & \lim_{n \to \infty} x - a_n = 0        &  & \text{(by \cref{6.4.17})} \\
    \implies & x = \lim_{n \to \infty} a_n.           &  & \text{(by \cref{6.1.19})}
  \end{align*}

  Now we show that if there exists a sequence \((a_n)_{n = 0}^\infty\) such that \(\forall n \in \N\), \(a_n \in X\) and \(\lim_{n \to \infty} a_n = x\), then \(x\) is an adherent point of \(X\).
  Since \(\lim_{n \to \infty} a_n = x\), by \cref{6.4.5} \(x\) is the only limit point of \((a_n)_{n = m}^\infty\).
  So we have
  \begin{align*}
             & \forall \varepsilon \in \R^+, \exists\ n \in \N : \abs{x - a_n} \leq \varepsilon  &  & \by{6.4.1}                \\
    \implies & \forall \varepsilon \in \R^+, \exists\ a_n \in X : \abs{x - a_n} \leq \varepsilon                                \\
    \implies & x \in \overline{X}.                                                               &  & \text{(by \cref{9.1.10})}
  \end{align*}
  We conclude that \(x\) is an adherent point of \(X\) iff there exists a sequence \((a_n)_{n = 0}^\infty\) such that \(\forall n \in \N\), \(a_n \in X\) and \(\lim_{n \to \infty} a_n = x\).
\end{proof}

\begin{defn}\label{9.1.15}
  A subset \(E \subseteq \R\) is said to be \emph{closed} if \(\overline{E} = E\), or in other words that \(E\) contains all of its adherent points.
\end{defn}

\begin{eg}\label{9.1.16}
  From \cref{9.1.12} we see that if \(a < b\) are real numbers, then \([a, b]\), \([a, +\infty)\), \((-\infty, a]\), and \((-\infty, +\infty)\) are closed, while \((a, b)\), \((a, b]\), \([a, b)\), \((a, +\infty)\), and \((-\infty, a)\) are not.
  From \cref{9.1.13} we see that \(\N\), \(\Z\), \(\R\), \(\emptyset\) are closed, while \(\Q\) is not.
\end{eg}

\begin{cor}\label{9.1.17}
  Let \(X\) be a subset of \(\R\).
  If \(X\) is closed, and \((a_n)_{n = 0}^\infty\) is a convergent sequence consisting of elements in \(X\), then \(\lim_{n \to \infty} a_n\) also lies in \(X\).
  Conversely, if it is true that every convergent sequence \((a_n)_{n = 0}^\infty\) of elements in \(X\) has its limit in \(X\) as well, then \(X\) is necessarily closed.
\end{cor}

\begin{proof}
  We first show that if \(X\) is closed, and \((a_n)_{n = 0}^\infty\) is a convergent sequence consisting of elements in \(X\), then \(\lim_{n \to \infty} a_n\) also lies in \(X\).
  Let \(x = \lim_{n \to \infty} a_n\).
  Then we have
  \begin{align*}
             & \forall \varepsilon \in \R^+, \exists\ n \in \N : \forall n' \in \N \land n' \geq n, \abs{x - a_{n'}} \leq \varepsilon                                \\
    \implies & \forall \varepsilon \in \R^+, \exists\ a_n \in X : \abs{x - a_n} \leq \varepsilon                                                                     \\
    \implies & x \in \overline{X}                                                                                                     &  & \text{(by \cref{9.1.10})} \\
    \implies & x \in X.                                                                                                               &  & \text{(by \cref{9.1.15})}
  \end{align*}

  Now we show that if every convergent sequence \((a_n)_{n = 0}^\infty\) of elements in \(X\) has its limit in \(X\) as well, then \(X\) is closed.
  By \cref{9.1.11} we have \(X \subseteq \overline{X}\).
  Since
  \begin{align*}
             & \forall x \in \overline{X}, \exists\ (a_n)_{n = 0}^\infty : (\forall n \in \N, a_n \in X) \land (\lim_{n \to \infty} a_n = x) &  & \text{(by \cref{9.1.14})} \\
    \implies & x \in X,                                                                                                                      &  & \text{(by hypothesis)}
  \end{align*}
  by \cref{3.1.15} we have \(\overline{X} \subseteq X\).
  Since \(X \subseteq \overline{X} \land \overline{X} \subseteq X\), by \cref{3.1.8} we have \(X = \overline{X}\), and thus by \cref{9.1.15} \(X\) is closed.
\end{proof}

\begin{defn}[Limit points]\label{9.1.18}
  Let \(X\) be a subset of the real line.
  We say that \(x\) is a \emph{limit point} (or a \emph{cluster point}) of \(X\) iff it is an adherent point of \(X \setminus \{x\}\).
  We say that \(x\) is an \emph{isolated point} of \(X\) if \(x \in X\) and there exists some \(\varepsilon > 0\) such that \(\abs{x - y} > \varepsilon\) for all \(y \in X \setminus \{x\}\).
\end{defn}

\setcounter{thm}{19}
\begin{rmk}\label{9.1.20}
  From \cref{9.1.14} we see that \(x\) is a limit point of \(X\) iff there exists a sequence \((a_n)_{n = 0}^\infty\), consisting entirely of elements in \(X\) that are distinct from \(x\), and such that \((a_n)_{n = 0}^\infty\) converges to \(x\).
  It turns out that the set of adherent points splits into the set of limit points and the set of isolated points.
\end{rmk}

\begin{lem}\label{9.1.21}
  Let \(I\) be an interval (possibly infinite), i.e., \(I\) is a set of the form \((a, b)\), \((a, b]\), \([a, b)\), \([a, b]\), \((a, +\infty)\), \([a, +\infty)\), \((-\infty, a)\), or \((-\infty, a]\), with \(a < b\) in the first four cases.
  Then every element of \(I\) is a limit point of \(I\).
\end{lem}

\begin{proof}
  We show this for the case \(I = [a, b]\);
  the other cases are similar.
  Let \(x \in I\);
  we have to show that \(x\) is a limit point of \(I\).
  There are three cases: \(x = a\), \(a < x < b\), and \(x = b\).
  If \(x = a\), then consider the sequence \((x + \dfrac{1}{n})_{n = N}^\infty\).
  This sequence converges to \(x\), and will lie inside \(I \setminus \{a\} = (a, b]\) if \(N\) is chosen large enough (by \cref{5.4.14}).
  Thus by \cref{9.1.20} we see that \(x = a\) is a limit point of \([a, b]\).
  A similar argument works when \(a < x < b\).
  When \(x = b\) one has to use the sequence \((x - \dfrac{1}{n})_{n = N}^\infty\) instead.
  This sequence converges to \(x\), and will lie inside \(I \setminus \{b\} = [a, b)\) if \(N\) is chosen large enough (by \cref{5.4.14}).
  Thus by \cref{9.1.20} we see that \(x = b\) is a limit point of \([a, b]\).
\end{proof}

\begin{defn}[Bounded sets]\label{9.1.22}
  A subset \(X\) of the real line is said to be \emph{bounded} if we have \(X \subseteq [-M, M]\) for some real number \(M > 0\).
\end{defn}

\begin{eg}\label{9.1.23}
  For any real numbers \(a, b\), the interval \([a, b]\) is bounded, because it is contained inside \([-M, M]\), where \(M \coloneqq \max(\abs{a}, \abs{b})\).
  However, the half-infinite interval \([0, +\infty)\) is unbounded.
  In fact, no half-infinite interval or doubly infinite interval can be bounded.
  The sets \(\N\), \(\Z\), \(\Q\), and \(\R\) are all unbounded.
\end{eg}

\begin{thm}[Heine-Borel theorem for the line]\label{9.1.24}
  Let \(X\) be a subset of \(\R\).
  Then the following two statements are equivalent:
  \begin{enumerate}
    \item \(X\) is closed and bounded.
    \item Given any sequence \((a_n)_{n = 0}^\infty\) of real numbers which takes values in \(X\) (i.e., \(a_n \in X\) for all \(n\)), there exists a subsequence \((a_{n_j})_{j = 0}^\infty\) of the original sequence, which converges to some number \(L\) in \(X\).
  \end{enumerate}
\end{thm}

\begin{proof}
  We first show that statement (a) implies statement (b).
  Suppose that \(X\) is a set such that \(X\) is closed and bounded.
  Let \((a_n)_{n = 0}^\infty\) be a sequence where \(\forall n \in \N\), \(a_n \in X\).
  Since \(X\) is bounded, by \cref{9.1.22} \(\exists\ M \in \R^+\) such that \(X \subseteq [-M, M]\), thus \((a_n)_{n = 0}^\infty\) is also bounded by \(M\), i.e., \(\forall n \in \N\), \(\abs{a_n} \leq M\).
  By Bolzano-Weierstrass theorem (\cref{6.6.8}) we know that there exists a subsequence \((a_{n_j})_{j = 0}^\infty\) of \((a_n)_{n = 0}^\infty\) such that \((a_{n_j})_{j = 0}^\infty\) converges.
  Since \(X\) is closed, by \cref{9.1.17} we know that \(\lim_{j \to \infty} a_{n_j} \in X\).

  Now we show that statement (b) implies statement (a).
  Since given any sequence \((a_n)_{n = 0}^\infty\) we can always find a subsequence \((a_{n_j})_{j = 0}^\infty\) such that \(\lim_{j \to \infty} a_{n_j} \in X\), we know that if \((a_n)_{n = 0}^\infty\) converges then \(\lim_{n \to \infty} a_n \in X\).
  Thus every convergent sequence \((a_n)_{n = 0}^\infty\) have its limit in \(X\), and by \cref{9.1.17} we know that \(X\) is closed.
  Suppose for sake of contradiction that \(X\) is unbounded.
  Then \(\nexists M \in \R^+\) such that \(X \subseteq [-M, M]\).
  Now we define \(X_n = \{x \in X : \abs{x} > n\}\) for every \(n \in \N\).
  We know that \(X_n \neq \emptyset\) since \(X\) is unbounded.
  By axiom of choice (\cref{8.1}) we know that \(\prod_{n \in \N} X_n \neq \emptyset\).
  Let \(f \in \prod_{n \in \N} X_n\).
  We can define a sequence \((a_n)_{n = 0}^\infty\) by setting \(a_n = f(n)\).
  By hypothesis we know that there exists a subsequence \((a_{n_j})_{j = 0}^\infty\) such that \(L = \lim_{j \to \infty} a_{n_j} \in X\).
  We know that \((a_{n_j})_{j = 0}^\infty\) is unbounded since \(\abs{a_{n_j}} > n_j\) for every \(n_j \in \N\).
  But by \cref{6.4.18} \((a_{n_j})_{j = 0}^\infty\) is Cauchy sequence and by \cref{6.1.17} \((a_{n_j})_{j = 0}^\infty\) is bounded, a contradiction.
  Thus \(X\) is closed and bounded.
\end{proof}

\begin{rmk}\label{9.1.25}
  This theorem shall play a key role in subsequent sections of \cref{ch:9}.
  In the language of metric space topology, it asserts that every subset of the real line which is closed and bounded, is also compact
  A more general version of this theorem, due to Eduard Heine (1821 -- 1881) and Emile Borel (1871 -- 1956), can be found is Analysis II, Theorem 1.5.7.
\end{rmk}

\exercisesection

\begin{ex}\label{ex:9.1.1}
  Let \(X\) be any subset of the real line, and let \(Y\) be a set such that \(X \subseteq Y \subseteq \overline{X}\).
  Show that \(\overline{Y} = \overline{X}\).
\end{ex}

\begin{proof}
  By \cref{9.1.11} we have \(X \subseteq Y \implies \overline{X} \subseteq \overline{Y}\).
  Since
  \begin{align*}
             & \forall y \in \overline{Y}, \forall \varepsilon \in \R^+, \exists\ x \in Y : \abs{y - x} \leq \dfrac{\varepsilon}{2} &                            & \text{(by \cref{9.1.10})} \\
    \implies & \exists\ x \in \overline{X} : \abs{y - x} \leq \dfrac{\varepsilon}{2}                                                & (Y \subseteq \overline{X})                             \\
    \implies & \exists\ z \in X : \abs{x - z} \leq \dfrac{\varepsilon}{2}                                                           &                            & \text{(by \cref{9.1.10})} \\
    \implies & \abs{y - x} + \abs{x - z} \leq \dfrac{\varepsilon}{2} + \dfrac{\varepsilon}{2}                                                                                                \\
    \implies & \abs{y - x + x - z} \leq \abs{y - x} + \abs{x - z} \leq \varepsilon                                                                                                           \\
    \implies & \abs{y - z} \leq \varepsilon                                                                                                                                                  \\
    \implies & y \in \overline{X},                                                                                                  &                            & \text{(by \cref{9.1.10})}
  \end{align*}
  by \cref{3.1.15} we know that \(\overline{Y} \subseteq \overline{X}\).
  Since \(\overline{Y} \subseteq \overline{X} \land \overline{X} \subseteq \overline{Y}\), by \cref{3.1.18} we have \(\overline{Y} = \overline{X}\).
\end{proof}

\begin{ex}\label{ex:9.1.2}
  Prove \cref{9.1.11}.
\end{ex}

\begin{proof}
  See \cref{9.1.11}.
\end{proof}

\begin{ex}\label{ex:9.1.3}
  Prove \cref{9.1.13}.
\end{ex}

\begin{proof}
  See \cref{9.1.13}.
\end{proof}

\begin{ex}\label{ex:9.1.4}
  Give an example of two subsets \(X, Y\) of the real line such that \(\overline{X \cap Y} \neq \overline{X} \cap \overline{Y}\).
\end{ex}

\begin{proof}
  Let \(X = [0, 0.5)\) and \(Y = (0.5, 1]\).
  By \cref{9.1.12} we have \(\overline{X} = [0, 0.5]\) and \(\overline{Y} = [0.5, 1]\), so \(\overline{X} \cap \overline{Y} = \{0.5\}\).
  By \cref{9.1.13} we have \(\overline{X \cap Y} = \overline{\emptyset} = \emptyset\).
  Thus \(\overline{X \cap Y} \neq \overline{X} \cap \overline{Y}\).
\end{proof}

\begin{ex}\label{ex:9.1.5}
  Prove \cref{9.1.14}.
\end{ex}

\begin{proof}
  See \cref{9.1.14}.
\end{proof}

\begin{ex}\label{ex:9.1.6}
  Let \(X\) be a subset of \(\R\).
  Show that \(\overline{X}\) is closed (i.e., \(\overline{\overline{X}} = \overline{X}\)).
  Furthermore, show that if \(Y\) is any closed set that contains \(X\), then \(Y\) also contains \(\overline{X}\).
  Thus the closure \(\overline{X}\) of \(X\) is the smallest closed set which contains \(X\).
\end{ex}

\begin{proof}
  We first show that \(X \subseteq \R \implies \overline{\overline{X}} = \overline{X}\).
  We have
  \begin{align*}
             & X \subseteq \R                                                                 \\
    \implies & \overline{X} \subseteq \overline{\R}            &  & \text{(by \cref{9.1.11})} \\
    \implies & \overline{X} \subseteq \R                       &  & \text{(by \cref{9.1.13})} \\
    \implies & \overline{X} \subseteq \overline{\overline{X}}. &  & \text{(by \cref{9.1.11})}
  \end{align*}
  Since
  \begin{align*}
             & \forall x \in \overline{\overline{X}}, \forall \varepsilon \in \R^+, \exists\ y \in \overline{X} : \abs{x - y} \leq \dfrac{\varepsilon}{2} &  & \text{(by \cref{9.1.10})} \\
    \implies & \exists\ z \in X : \abs{y - z} \leq \dfrac{\varepsilon}{2}                                                                                 &  & \text{(by \cref{9.1.10})} \\
    \implies & \abs{x - y} + \abs{y - z} \leq \dfrac{\varepsilon}{2} + \dfrac{\varepsilon}{2}                                                                                            \\
    \implies & \abs{x - y + y - z} \leq \abs{x - y} + \abs{y - z} \leq \varepsilon                                                                                                       \\
    \implies & \abs{x - z} \leq \varepsilon                                                                                                                                              \\
    \implies & x \in \overline{X},                                                                                                                        &  & \text{(by \cref{9.1.10})}
  \end{align*}
  by \cref{3.1.15} we know that \(\overline{\overline{X}} \subseteq \overline{X}\).
  Since \(\overline{\overline{X}} \subseteq \overline{X} \land \overline{X} \subseteq \overline{\overline{X}}\), by \cref{3.1.18} we have \(\overline{\overline{X}} = \overline{X}\).
  By \cref{9.1.15} \(\overline{X}\) is closed.

  Now we show that if \(Y\) is any closed set that contains \(X\), then \(Y\) also contains \(\overline{X}\).
  \begin{align*}
             & (X \subseteq Y) \land (Y = \overline{Y})                       &  & \text{(by \cref{9.1.15})} \\
    \implies & (\overline{X} \subseteq \overline{Y}) \land (Y = \overline{Y}) &  & \text{(by \cref{9.1.11})} \\
    \implies & \overline{X} \subseteq Y.
  \end{align*}
\end{proof}

\begin{ex}\label{ex:9.1.7}
  Let \(n \geq 1\) be a positive integer, and let \(X_1, \dots, X_n\) be closed subsets of \(\R\).
  Show that \(X_1 \cup X_2 \cup \dots \cup X_n\) is also closed.
\end{ex}

\begin{proof}
  Suppose that \(\forall m \in \N\) we have \(X_m\) is a closed subset of \(\R\).
  We use induction on \(n\) to show that \(X_1 \cup \dots \cup X_n\) is closed and we start with \(n = 1\).
  For \(n = 1\), by the given hypothesis we have \(X_1\) is closed.
  So the base case holds.
  Suppose inductively that for some \(n \geq 1\) we have \(X_1 \cup \dots \cup X_n\) is closed.
  Then for \(n + 1\), we have
  \begin{align*}
      & \overline{X_1 \cup \dots \cup X_n \cup X_{n + 1}}                                                \\
    = & \overline{(X_1 \cup \dots \cup X_n) \cup X_{n + 1}}            &  & \text{(by \cref{3.1.28}(e))} \\
    = & \overline{(X_1 \cup \dots \cup X_n)} \cup \overline{X_{n + 1}} &  & \text{(by \cref{9.1.11})}    \\
    = & (X_1 \cup \dots \cup X_n) \cup \overline{X_{n + 1}}            &  & \byIH                        \\
    = & (X_1 \cup \dots \cup X_n) \cup X_{n + 1}                       &  & \text{(by hypothesis)}       \\
    = & X_1 \cup \dots \cup X_n \cup X_{n + 1}.                        &  & \text{(by \cref{3.1.28}(e))}
  \end{align*}
  This closes the induction.
  Thus \(\forall n \in \N\), if \(X_1, \dots, X_n\) are closed subset of \(\R\), then \(X_1 \cup \dots \cup X_n\) is also closed.
\end{proof}

\begin{ex}\label{ex:9.1.8}
  Let \(I\) be a set (possibly infinite), and for each \(\alpha \in I\) let \(X_{\alpha}\) be a closed subset of \(\R\).
  Show that the intersection \(\bigcap_{\alpha \in I} X_{\alpha}\) is also closed.
\end{ex}

\begin{proof}
  By \cref{9.1.11} we have
  \[
    \bigcap_{\alpha \in I} X_{\alpha} \subseteq \R \implies \bigcap_{\alpha \in I} X_{\alpha} \subseteq \overline{\bigcap_{\alpha \in I} X_{\alpha}}.
  \]
  Since
  \begin{align*}
             & \forall x \in \overline{\bigcap_{\alpha \in I} X_{\alpha}}, \forall \varepsilon \in \R^+, \exists\ y \in \bigcap_{\alpha \in I} X_{\alpha} : \abs{x - y} \leq \varepsilon &  & \text{(by \cref{9.1.10})} \\
    \implies & \forall \alpha \in I, y \in X_{\alpha}                                                                                                                                                                   \\
    \implies & \forall \alpha \in I, x \in \overline{X_{\alpha}}                                                                                                                         &  & \text{(by \cref{9.1.10})} \\
    \implies & \forall \alpha \in I, x \in X_{\alpha}                                                                                                                                    &  & \text{(by hypothesis)}    \\
    \implies & x \in \bigcap_{\alpha \in I} X_{\alpha},
  \end{align*}
  by \cref{3.1.15} we have \(\overline{\bigcap_{\alpha \in I} X_{\alpha}} \subseteq \bigcap_{\alpha \in I} X_{\alpha}\).
  By \cref{3.1.18} we have
  \[
    \bigg(\overline{\bigcap_{\alpha \in I} X_{\alpha}} \subseteq \bigcap_{\alpha \in I} X_{\alpha}\bigg) \land \bigg(\bigcap_{\alpha \in I} X_{\alpha} \subseteq \overline{\bigcap_{\alpha \in I} X_{\alpha}}\bigg) \iff \overline{\bigcap_{\alpha \in I} X_{\alpha}} = \bigcap_{\alpha \in I} X_{\alpha},
  \]
  and thus by \cref{9.1.15} \(\bigcap_{\alpha \in I} X_{\alpha}\) is closed.
\end{proof}

\begin{ex}\label{ex:9.1.9}
  Let \(X\) be a subset of the real line.
  Show that every adherent point of \(X\) is either a limit point or an isolated point of \(X\), but cannot be both.
  Conversely, show that every limit point and every isolated point of \(X\) is an adherent point of \(X\).
\end{ex}

\begin{proof}
  Let \(X \subseteq \R\).
  We first show that every adherent point of \(X\) is either a limit point or an isolated point of \(X\), but cannot be not both.
  Let \(x \in \overline{X}\).
  Observe that
  \begin{align*}
             & x \in \overline{X}                                                                                 \\
    \implies & x \in \overline{(X \setminus \{x\}) \cup \{x\}}                                                    \\
    \implies & x \in \overline{X \setminus \{x\}} \cup \overline{\{x\}}            &  & \text{(by \cref{9.1.11})} \\
    \implies & (x \in \overline{X \setminus \{x\}}) \lor (x \in \overline{\{x\}}).
  \end{align*}
  By \cref{9.1.10} we know that \(x \in \overline{\{x\}}\).
  Thus we have either \(x \in \overline{X \setminus \{x\}}\) or \(x \notin \overline{X \setminus \{x\}}\).
  \begin{itemize}
    \item If \(x \in \overline{X \setminus \{x\}}\), then by \cref{9.1.18} \(x\) is a limit point of \(X\).
    \item If \(x \notin \overline{X \setminus \{x\}}\), then by \cref{9.1.10} \(\exists\ \varepsilon \in \R^+\) such that \(\forall y \in X \setminus \{x\}\), \(\abs{x - y} > \varepsilon\).
          Since \(x \in \overline{X}\), we must have \(x \in X\), otherwise \(\nexists y \in X \setminus \{x\}\) such that \(\abs{x - y} \leq \varepsilon\).
          Thus by \cref{9.1.18} \(x\) is a isolated point of \(X\).
  \end{itemize}
  From all cases above we conclude that \(x\) is either a limit point or an isolated point of \(X\).
  Since \(x \in \overline{X \setminus \{x\}}\) and \(x \notin \overline{X \setminus \{x\}}\) cannot be true at the same time, we know that \(x\) is either a limit point or an isolated point of \(X\), but cannot be both.

  Now we show that every limit point and every isolated point of \(X\) is an adherent point of \(X\).
  Suppose that \(x\) is a limit point of \(X\).
  Then we have
  \begin{align*}
             & x \in \overline{X \setminus \{x\}}                                                            &  & \text{(by \cref{9.1.18})} \\
    \implies & \forall \varepsilon \in \R^+, \exists\ y \in X \setminus \{x\} : \abs{x - y} \leq \varepsilon &  & \text{(by \cref{9.1.10})} \\
    \implies & \forall \varepsilon \in \R^+, \exists\ y \in X : \abs{x - y} \leq \varepsilon                                                \\
    \implies & x \in \overline{X}.                                                                           &  & \text{(by \cref{9.1.10})}
  \end{align*}
  Now suppose that \(x\) is a isolated point of \(X\).
  Then we have
  \begin{align*}
             & (x \in X) \land (X \subseteq \overline{X}) &  & \text{(by \cref{9.1.18})} \\
    \implies & x \in \overline{X}.                        &  & \text{(by \cref{9.1.10})}
  \end{align*}
  Since \(x\) is arbitrary limit point or isolated point of \(X\), we conclude that every limit point and every isolated point of \(X\) is an adherent point of \(X\).
\end{proof}

\begin{ex}\label{ex:9.1.10}
  If \(X\) is a non-empty subset of \(\R\), show that \(X\) is bounded if and only if \(\inf(X)\) and \(\sup(X)\) are finite.
\end{ex}

\begin{proof}
  Suppose that \(X\) is a set, \(X \subseteq \R\) and \(X \neq \emptyset\).
  We first show that if \(X\) is bounded then \(\inf(X)\) and \(\sup(X)\) are finite.
  Since
  \begin{align*}
             & \exists\ M \in \R^+ : X \subseteq [-M, M]                     &  & \text{(by \cref{9.1.22})} \\
    \implies & \forall x \in X : -M \leq x \leq M                            &  & \by{9.1.1}                \\
    \implies & \forall x \in X : -M \leq \inf(X) \leq x \leq \sup(X) \leq M, &  & \text{(by \cref{6.2.11})}
  \end{align*}
  we know that \(\inf(X)\) and \(\sup(X)\) are finite.

  Now we show that if \(\inf(X)\) and \(\sup(X)\) are finite then \(X\) is bounded.
  Let \(M = \max\big(\abs{\inf(X)}, \abs{\sup(X)}\big)\).
  Then we have
  \begin{align*}
             & M = \max\big(\abs{\inf(X)}, \abs{\sup(X)}\big)                                                                  \\
    \implies & \big(\abs{\inf(X)} \leq M\big) \land \big(\sup(X) \leq \abs{\sup(X)} \leq M\big)                                \\
    \implies & \big(-M \leq \inf(X) \leq M\big) \land \big(\sup(X) \leq M\big)                                                 \\
    \implies & -M \leq \inf(X) \leq \sup(X) \leq M                                              &  & \text{(by \cref{6.2.11})} \\
    \implies & \forall x \in X, -M \leq \inf(X) \leq x \leq \sup(X) \leq M                      &  & \text{(by \cref{6.2.11})} \\
    \implies & X \subseteq [-M, M].                                                             &  & \by{9.1.1}
  \end{align*}
  Thus by \cref{9.1.22} \(X\) is bounded.
\end{proof}

\begin{ex}\label{ex:9.1.11}
  Show that if \(X\) is a bounded subset of \(\R\), then the closure \(X\) is also bounded.
\end{ex}

\begin{proof}
  Since \(X\) is bounded, by \cref{9.1.22} we know that \(\exists\ M \in \R^+ : X \subseteq [-M, M]\).
  Let \(\varepsilon \in \R^+\).
  Then we have
  \begin{align*}
             & \forall x \in \overline{X}                                                                                                              \\
    \implies & \exists\ y \in X : \abs{x - y} \leq \varepsilon                                        &                    & \text{(by \cref{9.1.10})} \\
    \implies & -\varepsilon \leq x - y \leq \varepsilon                                                                                                \\
    \implies & y - \varepsilon \leq x \leq y + \varepsilon                                                                                             \\
    \implies & -M - \varepsilon \leq y - \varepsilon \leq x \leq y + \varepsilon \leq M + \varepsilon & (-M \leq y \leq M)                             \\
    \implies & x \in [-M - \varepsilon, M + \varepsilon].                                             &                    & \by{9.1.1}
  \end{align*}
  By \cref{3.1.15} we have \(\overline{X} \subseteq [-M - \varepsilon, M + \varepsilon]\), and thus by \cref{9.1.22} \(\overline{X}\) is bounded.
\end{proof}

\begin{ex}\label{ex:9.1.12}
  Show that the union of any finite collection of bounded subsets of \(\R\) is still a bounded set.
  Is this conclusion still true if one takes an infinite collection of bounded subsets of \(\R\)?
\end{ex}

\begin{proof}
  Let \(n \in \N\), let \(I_n = \{i \in \N : 1 \leq i \leq n\}\) and let \(X_1, \dots, X_n\) be bounded subsets of \(\R\).
  We want to show that \(\bigcup_{i \in I_n} X_i\) is bounded.
  If \(n = 0\), then \(I_n = \emptyset \implies \bigcup_{i \in I_n} X_i = \emptyset\) and by \cref{5.1.14} \(\emptyset\) is bounded.
  So suppose that \(n \neq 0\).
  Since \(X_1, \dots, X_n\) are bounded, \(\exists\ M_1, \dots, M_n \in \R^+\) such that \(\forall i \in I_n\), \(X_i \subseteq [-M_i, M_i]\).
  Let \(S = \{M_i : i \in I_n\}\).
  Clearly \(S\) is finite.
  By \cref{5.1.14} we know that \(S\) is bounded by some \(M \in \R\).
  Then we have
  \begin{align*}
             & \forall x \in \bigcup_{i \in I_n} X_i, \exists\ i \in I_n : x \in X_i                                                           \\
    \implies & -M_i \leq x \leq M_i                                                  & (X_i \subseteq [-M_i, M_i])                             \\
    \implies & -M \leq -M_i \leq x \leq M_i \leq M                                   &                             & \text{(by \cref{5.1.14})} \\
    \implies & x \in [-M, M].
  \end{align*}
  By \cref{3.1.15} we have \(\bigcup_{i \in I_n} X_i \subseteq [-M, M]\), thus by \cref{9.1.22} \(\bigcup_{i \in I_n} X_i\) is bounded.

  Now we show that the union of an infinite collection of bounded subsets of \(\R\) may not be bounded.
  Let \(n \in \N\) and let \(X_n = \{n\}\).
  Clearly \(\forall n \in \N\), \(X_n \subseteq [-n, n]\) and thus by \cref{9.1.22} \(X_n\) is bounded.
  Then we have \(\bigcup_{n \in \N} X_n = \N\) and by \cref{3.6.12} \(\N\) is unbounded.
\end{proof}

\begin{ex}\label{ex:9.1.13}
  Prove \cref{9.1.24}.
\end{ex}

\begin{proof}
  See \cref{9.1.24}.
\end{proof}

\begin{ex}\label{ex:9.1.14}
  Show that any finite subset of \(\R\) is closed and bounded.
\end{ex}

\begin{proof}
  Let \(n \in \N\) and let \(x_i \in \R\) for every \(i \in \N \land i \leq n\).
  Given any sequence \((a_m)_{m = 0}^\infty\) of real numbers which taks values in the singleton set \(\{x_i\}\), we know that \(\forall m \in \N\), \(a_m = x_i\).
  Thus by \cref{6.1.19} we have \(\lim_{m \to \infty} a_m = x_i\).
  Since \(x_i \in \{x_i\}\), by Heine-Borel theorem (\cref{9.1.24}) we know that \(\{x_i\}\) is closed and bounded.
  Let \(S = \bigcup_{i \in \N : i \leq n} \{x_i\}\).
  By \cref{ex:9.1.7} we know that \(S\) is closed, and by \cref{ex:9.1.12} we know that \(S\) is bounded.
  Since \(n\) is arbitrary natural number, we conclude that every finite subset of \(\R\) is closed and bounded.
\end{proof}

\begin{ex}\label{ex:9.1.15}
  Let \(E\) be a non-empty bounded subset of \(\R\), and let \(S \coloneqq \sup(E)\) be the least upper bound of \(E\).
  (Note from the least upper bound principle, \cref{5.5.9}, that \(S\) is a real number.)
  Show that \(S\) is an adherent point of \(E\), and is also an adherent point of \(\R \setminus E\).
\end{ex}

\begin{proof}
  We first show that \(S\) is an adherent point of \(E\).
  \(\forall \varepsilon \in \R^+\), we can always find \(x \in E\) such that \(S - \varepsilon < x \leq S\).
  If not, then \(\exists\ \varepsilon \in \R^+\) such that \(S - \varepsilon < S\) and
  \[
    \forall x \in E \implies x \leq S - \varepsilon < S \implies \sup(E) \neq S,
  \]
  a contradiction.
  Then we have
  \begin{align*}
             & \forall \varepsilon \in \R^+, \exists\ x \in E : S - \varepsilon < x \leq S                                           \\
    \implies & \forall \varepsilon \in \R^+, \exists\ x \in E : S - \varepsilon < x < S + \varepsilon                                \\
    \implies & \forall \varepsilon \in \R^+, \exists\ x \in E : -\varepsilon < x - S < \varepsilon                                   \\
    \implies & \forall \varepsilon \in \R^+, \exists\ x \in E : \abs{x - S} < \varepsilon                                            \\
    \implies & S \in \overline{E}.                                                                    &  & \text{(by \cref{9.1.10})}
  \end{align*}

  Now we show that \(S\) is an adherent point of \(\R \setminus E\).
  Let \(\varepsilon \in \R^+\).
  Then we have
  \begin{align*}
             & \forall x \in E, x \leq S                                           &  & \by{5.5.5}                \\
    \implies & x \leq S < S + \varepsilon                                                                         \\
    \implies & x \notin (S, S + \varepsilon)                                       &  & \by{9.1.1}                \\
    \implies & E \cap (S, S + \varepsilon) = \emptyset                                                            \\
    \implies & (S, S + \varepsilon) \subseteq \R \setminus E                                                      \\
    \implies & \overline{(S, S + \varepsilon)} \subseteq \overline{\R \setminus E} &  & \text{(by \cref{9.1.11})} \\
    \implies & [S, S + \varepsilon] \subseteq \overline{\R \setminus E}            &  & \text{(by \cref{9.1.12})} \\
    \implies & S \in \overline{\R \setminus E}.                                    &  & \by{9.1.1}
  \end{align*}
\end{proof}
\section{The algebra of real-valued functions}\label{i:sec:9.2}

\begin{note}
  We can take any one of the previous functions \(f : \R \to \R\) defined on all of \(\R\), and restrict the domain to a smaller set \(X \subseteq \R\), creating a new function, sometimes called \(f|_X\), from \(X\) to \(\R\).
  This is the same function as the original function \(f\), but is only defined on a smaller domain.
  (Thus \(f|_X(x) \coloneqq f(x)\) when \(x \in X\), and \(f|_X(x)\) is undefined when \(x \notin X\).)
\end{note}

\begin{note}
  If \(X\) is a subset of \(\R\), and \(f : X \to \R\) is a function, we can form the \emph{graph} \(\set{(x, f(x)) : x \in X}\) of the function \(f\);
  this is a subset of \(X \times \R\), and hence a subset of the Euclidean plane \(\R^2 = \R \times \R\).
  One can certainly study a function through its graph, by using the geometry of the plane \(\R^2\)
  (e.g., employing such concepts as tangent lines, area, and so forth).
  We however will pursue a more ``analytic'' approach, in which we rely instead on the properties of the real numbers to analyze these functions.
  The two approaches are complementary;
  the geometric approach offers more visual intuition, while the analytic approach offers rigour and precision.
  Both the geometric intuition and the analytic formalism become useful when extending analysis of functions of one variable to functions of many variables
  (or possibly even infinitely many variables).
\end{note}

\begin{defn}[Arithmetic operations on functions]\label{i:9.2.1}
  Given two functions \(f : X \to \R\) and \(g : X \to \R\), we can define their sum \(f + g : X \to \R\) by the formula
  \[
    (f + g)(x) \coloneqq f(x) + g(x),
  \]
  their difference \(f - g : X \to \R\) by the formula
  \[
    (f - g)(x) \coloneqq f(x) - g(x),
  \]
  their maximum \(\max(f, g) : X \to \R\) by
  \[
    \max(f, g)(x) \coloneqq \max(f(x), g(x)),
  \]
  their minimum \(\min(f, g) : X \to \R\) by
  \[
    \min(f, g)(x) \coloneqq \min(f(x), g(x)),
  \]
  their product \(fg : X \to \R\) (or \(f \cdot g : X \to \R\)) by the formula
  \[
    (fg)(x) \coloneqq f(x)g(x),
  \]
  and (provided that \(g(x) = 0\) for all \(x \in X\)) the quotient \(f / g : X \to \R\) by the formula
  \[
    (f / g)(x) \coloneqq f(x) / g(x).
  \]
  Finally, if \(c\) is a real number, we can define the function \(cf : X \to \R\) (or \(c \cdot f : X \to \R\)) by the formula
  \[
    (cf)(x) \coloneqq c \times f(x).
  \]
\end{defn}

\exercisesection

\begin{ex}\label{i:ex:9.2.1}
  Let \(f : \R \to \R\), \(g : \R \to \R\), \(h : \R \to \R\).
  Which of the following identities are true, and which ones are false?
  In the former case, give a proof;
  in the latter case, give a counterexample.
  \begin{align*}
    (f + g) \circ h & = (f \circ h) + (g \circ h) \\
    f \circ (g + h) & = (f \circ g) + (f \circ h) \\
    (f + g) \cdot h & = (f \cdot h) + (g \cdot h) \\
    f \cdot (g + h) & = (f \cdot g) + (f \cdot h)
  \end{align*}
\end{ex}

\begin{proof}
  We first show that \((f + g) \circ h = (f \circ h) + (g \circ h)\).
  Since
  \begin{align*}
             & f + g \text{ has domain } \R \text{ and codomain } \R                                   &  & \by{i:9.2.1}  \\
    \implies & (f + g) \circ h, f \circ h, g \circ h  \text{ have domain } \R \text{ and codomain } \R &  & \by{i:3.3.10} \\
    \implies & (f \circ h) + (g \circ h)  \text{ has domain } \R \text{ and codomain } \R              &  & \by{i:9.2.1}  \\
    \implies & (f + g) \circ h \text{ and }                                                                               \\
             & (f \circ h) + (g \circ h) \text{ have same domain and codomain}
  \end{align*}
  and
  \begin{align*}
    \forall x \in \R, \big((f + g) \circ h\big)(x) & = (f + g)\big(h(x)\big)                   &  & \by{i:3.3.10} \\
                                                   & = f\big(h(x)\big) + g\big(h(x)\big)       &  & \by{i:9.2.1}  \\
                                                   & = (f \circ h)(x) + (g \circ h)(x)         &  & \by{i:3.3.10} \\
                                                   & = \big((f \circ h) + (g \circ h)\big)(x), &  & \by{i:9.2.1}
  \end{align*}
  by \cref{i:3.3.7} we have \((f + g) \circ h = (f \circ h) + (g \circ h)\).

  Next we show that \(f \circ (g + h) = (f \circ g) + (f \circ h)\) may not be true.
  Let \(f(x) = 2^x\), \(g(x) = h(x) = x\).
  Then we have
  \begin{align*}
    \big(f \circ (g + h)\big)(x) & = f\big((g + h)(x)\big)  &  & \by{i:3.3.10} \\
                                 & = f\big(g(x) + h(x)\big) &  & \by{i:9.2.1}  \\
                                 & = f(x + x)                                  \\
                                 & = f(2x)                                     \\
                                 & = 2^{2x}
  \end{align*}
  and
  \begin{align*}
    \big((f \circ g) + (f \circ h)\big)(x) & = (f \circ g)(x) + (f \circ h)(x)   &  & \by{i:9.2.1}  \\
                                           & = f\big(g(x)\big) + f\big(h(x)\big) &  & \by{i:3.3.10} \\
                                           & = f(x) + f(x)                                          \\
                                           & = 2^x + 2^x                                            \\
                                           & = 2 \cdot 2^x.
  \end{align*}
  When \(x = 2\) we have \(2^{2x} = 16\) and \(2 \cdot 2^x = 8\).
  Thus \(f \circ (g + h) = (f \circ g) + (f \circ h)\) may not be true.

  Next we show that \((f + g) \cdot h = (f \cdot h) + (g \cdot h)\).
  Since
  \begin{align*}
             & f + g \text{ has domain } \R \text{ and codomain } \R                                   &  & \by{i:9.2.1}  \\
    \implies & (f + g) \cdot h, f \cdot h, g \cdot h  \text{ have domain } \R \text{ and codomain } \R &  & \by{i:3.3.10} \\
    \implies & (f \cdot h) + (g \cdot h)  \text{ has domain } \R \text{ and codomain } \R              &  & \by{i:9.2.1}  \\
    \implies & (f + g) \cdot h \text{ and }                                                                               \\
             & (f \cdot h) + (g \cdot h) \text{ have same domain and codomain}
  \end{align*}
  and
  \begin{align*}
    \forall x \in \R, \big((f + g) \cdot h\big)(x) & = (f + g)(x) \cdot h(x)                   &  & \by{i:9.2.1} \\
                                                   & = \big(f(x) + g(x)\big) \cdot h(x)        &  & \by{i:9.2.1} \\
                                                   & = f(x)h(x) + g(x)h(x)                                       \\
                                                   & = (f \cdot h)(x) + (g \cdot h)(x)         &  & \by{i:9.2.1} \\
                                                   & = \big((f \cdot h) + (g \cdot h)\big)(x), &  & \by{i:9.2.1}
  \end{align*}
  by \cref{i:3.3.7} we have \((f + g) \cdot h = (f \cdot h) + (g \cdot h)\).

  Finally we show that \(f \cdot (g + h) = (f \cdot g) + (f \cdot h)\).
  Since
  \begin{align*}
             & g + h \text{ has domain } \R \text{ and codomain } \R                                   &  & \by{i:9.2.1}  \\
    \implies & f \cdot (g + h), f \cdot g, f \cdot h  \text{ have domain } \R \text{ and codomain } \R &  & \by{i:3.3.10} \\
    \implies & (f \cdot g) + (f \cdot h)  \text{ has domain } \R \text{ and codomain } \R              &  & \by{i:9.2.1}  \\
    \implies & f \cdot (g + h) \text{ and }                                                                               \\
             & (f \cdot g) + (f \cdot h) \text{ have same domain and codomain}
  \end{align*}
  and
  \begin{align*}
    \forall x \in \R, \big(f \cdot (g + h)\big)(x) & = f(x) \cdot (g + h)(x)                   &  & \by{i:9.2.1} \\
                                                   & = f(x) \cdot \big(g(x) + h(x)\big)        &  & \by{i:9.2.1} \\
                                                   & = f(x)g(x) + f(x)h(x)                                       \\
                                                   & = (f \cdot g)(x) + (f \cdot h)(x)         &  & \by{i:9.2.1} \\
                                                   & = \big((f \cdot g) + (f \cdot h)\big)(x), &  & \by{i:9.2.1}
  \end{align*}
  by \cref{i:3.3.7} we have \(f \cdot (g + h) = (f \cdot g) + (f \cdot h)\).
\end{proof}

\section{Limiting values of functions}\label{sec:9.3}

\begin{defn}[\(\varepsilon\)-closeness]\label{9.3.1}
  Let \(X\) be a subset of \(\R\), let \(f : X \to \R\) be a function, let \(L\) be a real number, and let \(\varepsilon > 0\) be a real number.
  We say that the function \(f\) is \emph{\(\varepsilon\)-close to \(L\)} iff \(f(x)\) is \(\varepsilon\)-close to \(L\) for every \(x \in X\).
\end{defn}

\setcounter{thm}{2}
\begin{defn}[Local \(\varepsilon\)-closeness]\label{9.3.3}
  Let \(X\) be a subset of \(\R\), let \(f : X \to \R\) be a function, let \(L\) be a real number, \(x_0\) be an adherent point of \(X\), and \(\varepsilon > 0\) be a real number.
  We say that \(f\) is \emph{\(\varepsilon\)-close to \(L\) near \(x_0\)} iff there exists a \(\delta > 0\) such that \(f\) becomes \(\varepsilon\)-close to \(L\) when restricted to the set \(\set{x \in X : \abs{x - x_0} < \delta}\).
\end{defn}

\setcounter{thm}{5}
\begin{defn}[Convergence of functions at a point]\label{9.3.6}
  Let \(X\) be a subset of \(\R\), let \(f : X \to \R\) be a function, let \(E\) be a subset of \(X\), \(x_0\) be an adherent point of \(E\), and let \(L\) be a real number.
  We say that \emph{\(f\) converges to \(L\) at \(x_0\) in \(E\)}, and write \(\lim_{x \to x_0 ; x \in E} f(x) = L\), iff \(f\), after restricting to \(E\), is \(\varepsilon\)-close to \(L\) near \(x_0\) for every \(\varepsilon > 0\).
  If \(f\) does not converge to any number \(L\) at \(x_0\), we say that \emph{\(f\) diverges at \(x_0\)}, and leave \(\lim_{x \to x_0 ; x \in E} f(x)\) undefined.
\end{defn}

\begin{note}
  In other words, we have \(\lim_{x \to x_0 ; x \in E} f(x) = L\) iff for every \(\varepsilon > 0\), there exists a \(\delta > 0\) such that \(\abs{f(x) - L} \leq \varepsilon\) for all \(x \in E\) such that \(\abs{x - x_0} < \delta\).
\end{note}

\begin{rmk}\label{9.3.7}
  In many cases we will omit the set \(E\) from the above notation (i.e., we will just say that \(f\) converges to \(L\) at \(x_0\), or that \(\lim_{x \to x_0} f(x) = L\)), although this is slightly dangerous.
  For instance, it sometimes makes a difference whether \(E\) actually contains \(x_0\) or not.
  To give an example, if \(f : \R \to \R\) is the function defined by setting \(f(x) = 1\) when \(x = 0\) and \(f(x) = 0\) when \(x \neq 0\), then one has \(\lim_{x \to 0 ; x \in \R \setminus \set{0}} f(x) = 0\), but \(\lim_{x \to 0 ; x \in \R} f(x)\) is undefined.
  Some authors only define the limit \(\lim_{x \to x_0 ; x \in E} f(x)\) when \(E\) does not contain \(x_0\) (so that \(x_0\) is now a limit point of \(E\) rather than an adherent point), or would use \(\lim_{x \to x_0 ; x \in E} f(x)\) to denote what we would call \(\lim_{x \to x_0 ; x \in E \setminus \set{x_0}} f(x)\), but we have chosen a slightly more general notation, which allows the possibility that \(E\) contains \(x_0\).
\end{rmk}

\setcounter{thm}{8}
\begin{prop}\label{9.3.9}
  Let \(X\) be a subset of \(\R\), let \(f : X \to \R\) be a function, let \(E\) be a subset of \(X\), let \(x_0\) be an adherent point of \(E\), and let \(L\) be a real number.
  Then the following two statements are logically equivalent:
  \begin{enumerate}
    \item \(f\) converges to \(L\) at \(x_0\) in \(E\).
    \item For every sequence \((a_n)_{n = 0}^\infty\) which consists entirely of elements of \(E\) and converges to \(x_0\), the sequence \((f(a_n))_{n = 0}^\infty\) converges to \(L\).
  \end{enumerate}
\end{prop}

\begin{proof}
  We first show that statement (a) implies statement (b).
  Since \(f\) converges to \(L\) at \(x_0\) in \(E\), by \cref{9.3.6} we have
  \[
    \forall \varepsilon \in \R^+, \exists \delta \in \R^+ : \big(\forall x \in E, \abs{x - x_0} < \delta \implies \abs{f(x) - L} \leq \varepsilon\big).
  \]
  Now we fix \(\varepsilon\), and we have some \(\delta\) satisfying the statement above, we also fix such \(\delta\).
  Let \((a_n)_{n = 0}^\infty\) be a sequence which consists entirely of elements of \(E\) and \(\lim_{n \to \infty} a_n = x_0\).
  Such sequence exists since \cref{9.1.14}.
  By \cref{6.1.5} we have
  \[
    \forall \varepsilon' \in \R^+, \exists N \in \N : \forall n \geq N, \abs{a_n - x_0} \leq \varepsilon'.
  \]
  In particular, we have
  \[
    \exists N \in \N : \forall n \geq N, \abs{a_n - x_0} \leq \dfrac{\delta}{2} < \delta.
  \]
  Since \((a_n)_{n = 0}^\infty\) consists entirely of elements of \(E\), we have
  \[
    \abs{a_n - x_0} < \delta \implies \abs{f(a_n) - L} \leq \varepsilon.
  \]
  Since \(\varepsilon\) is arbitrary, we have
  \[
    \forall \varepsilon \in \R^+, \exists N \in \N : \forall n \geq N, \abs{f(a_n) - L} \leq \varepsilon
  \]
  and by \cref{6.1.5} we have \(\lim_{n \to \infty} f(a_n) = L\).

  Now we show that statement (b) implies statement (a).
  Suppose for sake of contradiction that \(f\) does not converge to \(L\) at \(x_0\) in \(E\).
  Then by \cref{9.3.6} we have
  \[
    \exists \varepsilon \in \R^+ : \forall \delta \in \R^+, \Big(\forall x \in E, (\abs{x - x_0} < \delta) \land \big(\abs{f(x) - L} > \varepsilon\big)\Big).
  \]
  Let \((a_n)_{n = 0}^\infty\) be a sequence which consists entirely of elements of \(E\) and \(\lim_{n \to \infty} a_n = x_0\).
  By hypothesis we have \(\lim_{n \to \infty} f(a_n) = L\).
  By \cref{6.1.5} the following two statements are true:
  \begin{align*}
     & \exists N_1 \in \N : \forall n_1 \geq N_1, \abs{a_{n_1} - x_0} \leq \dfrac{\delta}{2} < \delta \\
     & \exists N_2 \in \N : \forall n_2 \geq N_2, \abs{f(a_{n_2}) - L} \leq \varepsilon
  \end{align*}
  Let \(N = \max(N_1, N_2)\).
  Then we have
  \[
    \forall n \geq N, (\abs{a_n - x_0} < \delta) \land \big(\abs{f(a_n) - L} \leq \varepsilon\big).
  \]
  But \(a_n \in E\), so this contradict to \(\abs{f(a_n) - L} > \varepsilon\).
  Thus \(\lim_{x \to x_0 ; x \in E} f(x) = L\).
\end{proof}

\begin{note}
  In view of \cref{9.3.9}, we will sometimes write ``\(f(x) \to L\) as \(x \to x_0\) in \(E\)'' or ``\(f\) has a limit \(L\) at \(x_0\) in \(E\)'' instead of ``\(f\) converges to \(L\) at \(x_0\)'', or ``\(\lim_{x \to x_0} f(x) = L\)''.
\end{note}

\begin{rmk}\label{9.3.10}
  With the notation of \cref{9.3.9}, we have the following corollary:
  if \(\lim_{x \to x_0 ; x \in E} f(x) = L\), and \(\lim_{n \to \infty} a_n = x_0\), then \(\lim_{n \to \infty} f(a_n) = L\).
\end{rmk}

\begin{rmk}\label{9.3.11}
  We only consider limits of a function \(f\) at \(x_0\) in the case when \(x_0\) is an adherent point of \(E\).
  When \(x_0\) is not an adherent point then it is not worth it to define the concept of a limit.
\end{rmk}

\begin{rmk}\label{9.3.12}
  The variable \(x\) used to denote a limit is a dummy variable;
  we could replace it by any other variable and obtain exactly the same limit.
  For instance, if \(\lim_{x \to x_0 ; x \in E} f(x) = L\), then \(\lim_{y \to x_0 ; y \in E} f(y) = L\), and conversely.
  (Since \(x \in \R\).)
\end{rmk}

\begin{cor}\label{9.3.13}
  Let \(X\) be a subset of \(\R\), let \(E\) be a subset of \(X\), let \(x_0\) be an adherent point of \(E\), and let \(f : X \to \R\) be a function.
  Then \(f\) can have at most one limit at \(x_0\) in \(E\).
\end{cor}

\begin{proof}
  Suppose for sake of contradiction that there are two distinct numbers \(L\) and \(L'\) such that \(f\) has a limit \(L\) at \(x_0\) in \(E\), and such that \(f\) also has a limit \(L'\) at \(x_0\) in \(E\).
  Since \(x_0\) is an adherent point of \(E\), we know by \cref{9.1.14} that there is a sequence \((a_n)_{n = 0}^\infty\) consisting of elements in \(E\) which converges to \(x_0\).
  Since \(f\) has a limit \(L\) at \(x_0\) in \(E\), we thus see by \cref{9.3.9}, that \((f(a_n))_{n = 0}^\infty\) converges to \(L\).
  But since \(f\) also has a limit \(L'\) at \(x_0\) in \(E\), we see that \((f(a_n))_{n = 0}^\infty\) also converges to \(L'\).
  But this contradicts the uniqueness of limits of sequences (\cref{6.1.7}).
\end{proof}

\begin{prop}[Limit laws for functions]\label{9.3.14}
  Let \(X\) be a subset of \(R\), let \(E\) be a subset of \(X\), let \(x_0\) be an adherent point of \(E\), and let \(f : X \to \R\) and \(g : X \to \R\) be functions.
  Suppose that \(f\) has a limit \(L\) at \(x_0\) in \(E\), and \(g\) has a limit \(M\) at \(x_0\) in \(E\).
  Then \(f + g\) has a limit \(L + M\) at \(x_0\) in \(E\), \(f - g\) has a limit \(L - M\) at \(x_0\) in \(E\), \(\max(f, g\)) has a limit \(\max(L, M)\) at \(x_0\) in \(E\), \(\min(f, g)\) has a limit \(\min(L, M)\) at \(x_0\) in \(E\) and \(fg\) has a limit \(LM\) at \(x_0\) in \(E\).
  If \(c\) is a real number, then \(cf\) has a limit \(cL\) at \(x_0\) in \(E\).
  Finally, if \(g\) is non-zero on \(E\) (i.e., \(g(x) \neq 0\) for all \(x \in E\)) and \(M\) is non-zero, then \(f / g\) has a limit \(L / M\) at \(x_0\) in \(E\).
\end{prop}

\begin{proof}
  Since \(x_0\) is an adherent point of \(E\), we know by \cref{9.1.14} that there is a sequence \((a_n)_{n = 0}^\infty\) consisting of elements in \(E\), which converges to \(x_0\).
  Since \(f\) has a limit \(L\) at \(x_0\) in \(E\), we thus see by \cref{9.3.9}, that \((f(a_n))_{n = 0}^\infty\) converges to \(L\).
  Similarly \((g(a_n))_{n = 0}^\infty\) converges to \(M\).

  By the limit laws for sequences (\cref{6.1.19}) we conclude that
  \begin{align*}
             & (\lim_{n \to \infty} f(a_n) = L) \land (\lim_{n \to \infty} g(a_n) = M) \\
    \implies & \begin{dcases}
                 \lim_{n \to \infty} f(a_n) + g(a_n) = L + M                   \\
                 \lim_{n \to \infty} f(a_n) - g(a_n) = L - M                   \\
                 \lim_{n \to \infty} \max\big(f(a_n), g(a_n)\big) = \max(L, M) \\
                 \lim_{n \to \infty} \min\big(f(a_n), g(a_n)\big) = \min(L, M) \\
                 \lim_{n \to \infty} f(a_n) g(a_n) = LM                        \\
                 \lim_{n \to \infty} cf(a_n) = cL
               \end{dcases}           \\
    \implies & \begin{dcases}
                 \lim_{n \to \infty} (f + g)(a_n) = L + M         \\
                 \lim_{n \to \infty} (f - g)(a_n) = L - M         \\
                 \lim_{n \to \infty} \max(f, g)(a_n) = \max(L, M) \\
                 \lim_{n \to \infty} \min(f, g)(a_n) = \min(L, M) \\
                 \lim_{n \to \infty} (fg)(a_n) = LM               \\
                 \lim_{n \to \infty} (cf)(a_n) = cL
               \end{dcases}                     &  & \by{9.2.1}
  \end{align*}
  If \(\forall x \in E, g(x) \neq 0\) and \(M \neq 0\), then by the limit laws for sequences (\cref{6.1.19}) and \cref{9.2.1} we have
  \[
    \lim_{n \to \infty} \big(f(a_n) / g(a_n)\big) = \lim_{n \to \infty} \big((f / g)(a_n)\big)_{n = 0}^\infty = L / M.
  \]
  By \cref{9.3.9} again, this implies
  \[
    \begin{dcases}
      \lim_{x \to x_0 ; x \in E} (f + g)(x) = L + M                                                                 \\
      \lim_{x \to x_0 ; x \in E} (f - g)(x) = L - M                                                                 \\
      \lim_{x \to x_0 ; x \in E} \max(f, g)(x) = \max(L, M)                                                         \\
      \lim_{x \to x_0 ; x \in E} \min(f, g)(x) = \min(L, M)                                                         \\
      \lim_{x \to x_0 ; x \in E} (fg)(x) = LM                                                                       \\
      \lim_{x \to x_0 ; x \in E} (cf)(x) = cL                                                                       \\
      \lim_{x \to x_0 ; x \in E} (f / g)(x) = L / M & \text{if } \forall x \in E, g(x) \neq 0 \text{ and } M \neq 0
    \end{dcases}
  \]
  (since \((a_n)_{n = 0}^\infty\) was an arbitrary sequence in \(E\) converging to \(x_0\)).
\end{proof}

\begin{rmk}\label{9.3.15}
  One can phrase \cref{9.3.14} more informally as saying that
  \begin{align*}
    \lim_{x \to x_0} (f \pm g)(x)  & = \lim_{x \to x_0} f(x) \pm \lim_{x \to x_0} g(x)              \\
    \lim_{x \to x_0} \max(f, g)(x) & = \max\bigg(\lim_{x \to x_0} f(x), \lim_{x \to x_0} g(x)\bigg) \\
    \lim_{x \to x_0} \min(f, g)(x) & = \min\bigg(\lim_{x \to x_0} f(x), \lim_{x \to x_0} g(x)\bigg) \\
    \lim_{x \to x_0} (fg)(x)       & = \lim_{x \to x_0} f(x) \lim_{x \to x_0} g(x)                  \\
    \lim_{x \to x_0} (f / g)(x)    & = \dfrac{\lim_{x \to x_0} f(x)}{\lim_{x \to x_0} g(x)}
  \end{align*}
  (where we have dropped the restriction \(x \in E\) for brevity)
  but bear in mind that these identities are only true when the right-hand side makes sense, and furthermore for the final identity we need \(g\) to be non-zero, and also \(\lim_{x \to x_0} g(x)\) to be non-zero.
\end{rmk}

\begin{note}
  If \(f\) converges to \(L\) at \(x_0\) in \(X\), and \(Y\) is any subset of \(X\) such that \(x_0\) is still an adherent point of \(Y\), then \(f\) will also converge to \(L\) at \(x_0\) in \(Y\)
  (since \(Y \subseteq X\) and \(x \in \overline{Y}\)).
  Thus convergence on a large set implies convergence on a smaller set.
  The converse, however, is not true.
\end{note}

\begin{eg}\label{9.3.16}
  Consider the \emph{signum function} \(\text{sgn} : \R \to \R\), defined by
  \[
    \text{sgn}(x) \coloneqq \begin{dcases}
      1  & \text{if } x > 0 \\
      0  & \text{if } x = 0 \\
      -1 & \text{if } x < 0
    \end{dcases}
  \]
  Then \(\lim_{x \to 0 ; x \in (0, \infty)} \text{sgn}(x) = 1\), whereas \(\lim_{x \to 0 ; x \in (-\infty, 0)} = -1\) and \(\lim_{x \to 0 ; x \in \R} \text{sgn}(x)\) is undefined.
  Thus it is sometimes dangerous to drop the set \(E\) from the notation of limit.
  However, in many cases it is safe to do so.
\end{eg}

\begin{eg}\label{9.3.17}
  Let \(f(x)\) be the function
  \[
    f(x) \coloneqq \begin{dcases}
      1 & \text{if } x = 0     \\
      0 & \text{if } x \neq 0.
    \end{dcases}
  \]
  Then \(\lim_{x \to 0 ; x \in \R \setminus \set{0}} f(x) = 0\), but \(\lim_{x \to 0 ; x \in \R} f(x)\) is undefined.
  (When this happens, we say that \(f\) has a ``removable singularity'' or ``removable discontinuity'' at \(0\).
  Because of such singularities, it is sometimes the convention when writing \(\lim_{x \to x_0} f(x)\) to automatically exclude \(x_0\) from the set;
  for instance, in some textbook, \(\lim_{x \to x_0} f(x)\) is used as shorthand for \(\lim_{x \to x_0 ; x \in X \setminus \set{x_0}} f(x)\).)
\end{eg}

\begin{note}
  On the other hand, the limit at \(x_0\) should only depend on the values of the function near \(x_0\);
  the values away from \(x_0\) are not relevant.
\end{note}

\begin{prop}[Limits are local]\label{9.3.18}
  Let \(X\) be a subset of \(\R\), let \(E\) be a subset of \(X\), let \(x_0\) be an adherent point of \(E\), let \(f : X \to \R\) be a function, and let \(L\) be a real number.
  Let \(\delta > 0\).
  Then we have
  \[
    \lim_{x \to x_0 ; x \in E} f(x) = L
  \]
  iff
  \[
    \lim_{x \to x_0 ; x \in E \cap (x_0 - \delta, x_0 + \delta)} f(x) = L.
  \]
\end{prop}

\begin{proof}
  We know that \(\lim_{x \to x_0 ; x \in E} f(x) = L \implies \lim_{x \to x_0 ; x \in E \cap (x_0 - \delta, x_0 + \delta)} f(x) = L\) since \(E \cap (x_0 - \delta, x_0 + \delta) \subseteq E\).
  We also know that \(\lim_{x \to x_0 ; x \in E \cap (x_0 - \delta, x_0 + \delta)} f(x) = L \implies \lim_{x \to x_0 ; x \in E} f(x) = L\) since
  \begin{align*}
             & \lim_{x \to x_0 ; x \in E \cap (x_0 - \delta, x_0 + \delta)} f(x) = L                                                                 \\
    \implies & \forall \varepsilon \in \R^+, \exists \delta' \in \R^+ :                                                                              \\
             & \big(\forall x \in E \cap (x_0 - \delta, x_0 + \delta), \abs{x - x_0} < \delta' \implies \abs{f(x) - L} \leq \varepsilon\big)         \\
    \implies & \forall \varepsilon \in \R^+, \exists \delta' \in \R^+ :                                                                              \\
             & \big(\forall x \in E, (x_0 - \delta < x < x_0 + \delta) \land (\abs{x - x_0} < \delta') \implies \abs{f(x) - L} \leq \varepsilon\big) \\
    \implies & \forall \varepsilon \in \R^+, \exists \delta' \in \R^+ :                                                                              \\
             & \big(\forall x \in E, (-\delta < x - x_0 < \delta) \land (\abs{x - x_0} < \delta') \implies \abs{f(x) - L} \leq \varepsilon\big)      \\
    \implies & \forall \varepsilon \in \R^+, \exists \delta' \in \R^+ :                                                                              \\
             & \big(\forall x \in E, (\abs{x - x_0} < \delta) \land (\abs{x - x_0} < \delta') \implies \abs{f(x) - L} \leq \varepsilon\big)          \\
    \implies & \forall \varepsilon \in \R^+, \exists \delta' \in \R^+ :                                                                              \\
             & \big(\forall x \in E, \abs{x - x_0} < \min(\delta, \delta') \implies \abs{f(x) - L} \leq \varepsilon\big)                             \\
    \implies & \lim_{x \to x_0 ; x \in E} f(x) = L.
  \end{align*}
  Thus we conclude that \(\lim_{x \to x_0 ; x \in E} f(x) = \lim_{x \to x_0 ; x \in E \cap (x_0 - \delta, x_0 + \delta)} f(x) = L\).
\end{proof}

\begin{note}
  Informally, \cref{9.3.18} asserts that
  \[
    \lim_{x \to x_0 ; x \in E} f(x) = \lim_{x \to x_0 ; x \in E \cap (x_0 - \delta, x_0 + \delta)} f(x).
  \]
  Thus the limit of a function at \(x_0\), if it exists, only depends on the values of \(f\) near \(x_0\);
  the values far away do not actually influence the limit.
\end{note}

\begin{ac}[Limit superior and limi inferior]\label{ac:9.3.1}
  Let \(X\) be a subset of \(\R\), let \(f : X \to \R\) be a function, let \(E\) be a subset of \(X\), and let \(x_0\) be an adherent point of \(E\).
  We define \emph{limit superior at \(x_0\) in \(E\)} as
  \[
    \limsup_{x \to x_0 ; x \in E} f(x) = \inf\set{\sup\set{f(x) : x \in E \land \abs{x - x_0} < \delta} : \delta \in \R^+}
  \]
  and define \emph{limit inferior at \(x_0\) in \(E\)} as
  \[
    \liminf_{x \to x_0 ; x \in E} f(x) = \sup\set{\inf\set{f(x) : x \in E \land \abs{x - x_0} < \delta} : \delta \in \R^+}.
  \]
\end{ac}

\begin{ac}\label{ac:9.3.2}
  Let \(X\) be a subset of \(\R\), let \(f : X \to \R\) be a function, let \(E\) be a subset of \(X\), and let \(x_0\) be an adherent point of \(E\).
  Let \(L \in \R\).
  We claim that the following statements are equivalent:
  \begin{enumerate}
    \item \(\limsup_{x \to x_0 ; x \in E} f(x) = L\)
    \item For every sequence \((a_n)_{n = 1}^\infty\) which consists entirely of elements of \(E\) and converges to \(x_0\), the sequence \((f(a_n))_{n = 1}^\infty\) has limit superior \(\limsup_{n \to \infty} f(a_n) \leq L\).
          There exists a sequence \((b_n)_{n = 1}^\infty\) which consists entirely of elements of \(E\) and converges to \(x_0\), and \(\limsup_{n \to \infty} f(b_n) = L\).
  \end{enumerate}
  Similarly, the following statements are equivalent:
  \begin{enumerate}
    \item \(\liminf_{x \to x_0 ; x \in E} f(x) = L\).
    \item For every sequence \((a_n)_{n = 1}^\infty\) which consists entirely of elements of \(E\) and converges to \(x_0\), the sequence \((f(a_n))_{n = 1}^\infty\) has limit inferior \(\liminf_{n \to \infty} f(a_n) \geq L\).
          There exists a sequence \((b_n)_{n = 1}^\infty\) which consists entirely of elements of \(E\) and converges to \(x_0\), and \(\liminf_{n \to \infty} f(b_n) = L\).
  \end{enumerate}
\end{ac}

\begin{proof}
  We only prove for limit superior.
  The proof for limit inferior are similar.
  For each \(\delta \in \R^+\), define \(X_{\delta}\) as follow:
  \[
    X_{\delta} = \set{x \in E : \abs{x - x_0} \leq \delta}.
  \]
  We know \(X_{\delta} \neq \emptyset\) for each \(\delta \in \R^+\) since \(\delta \in \R^+\) and \(x_0\) is an adherent point.
  By axiom of choice (\cref{8.1}) we know that \(\prod_{n \in \N} X_{\dfrac{1}{n}} \neq \emptyset\).
  If we choose some \(g \in \prod_{n \in \N} X_{\dfrac{1}{n}}\) and define \((b_n)_{n = 0}^\infty\) by setting \(b_n = g(n)\) for each \(n \in \N\), then we have
  \begin{align*}
             & \forall n \in \N, b_n \in X_{\dfrac{1}{n}}                                  \\
    \implies & \forall n \in \N, 0 \leq \abs{b_n - x_0} \leq \dfrac{1}{n}                  \\
    \implies & \lim_{n \to \infty} \abs{b_n - x_0} = 0                    &  & \by{6.4.14} \\
    \implies & \lim_{n \to \infty} b_n - x_0 = 0                          &  & \by{6.4.17} \\
    \implies & \lim_{n \to \infty} b_n = x_0.                             &  & \by{6.1.19}
  \end{align*}
  Let \(S_{\delta} = \set{f(x) : x \in X_{\delta}}\).
  We can rewrite limit superior as follow:
  \[
    \limsup_{x \to x_0 ; x \in E} f(x) = \inf\set{\sup(S_\delta) : \delta \in \R^+}.
  \]
  We will use \(X_{\delta}\) and \(S_{\delta}\) throughout the proof of this corollary.

  We first show that statement (a) implies statement (b).
  Suppose that
  \[
    L = \limsup_{x \to x_0 ; x \in E} f(x) = \inf\set{\sup(S_\delta) : \delta \in \R^+}.
  \]
  Let \((a_n)_{n = 1}^\infty\) be a sequence which consists entirely of elements of \(E\) and \(\lim_{n \to \infty} a_n = x_0\).
  Such sequence exists since \cref{9.1.14}.
  Then we have
  \begin{align*}
             & \lim_{n \to \infty} a_n = x_0                                                                                                    \\
    \implies & \forall \delta \in \R^+, \exists N \in \Z^+ :                                                                                    \\
             & \forall n \geq N, \abs{a_n - x_0} \leq \delta                                                                                    \\
    \implies & \forall \delta \in \R^+, \exists N \in \Z^+ :                                                                                    \\
             & \forall n \geq N, \big(f(a_n) \in S_{\delta}\big) \land \big(f(a_n) \leq \sup(S_{\delta})\big) &  & \by{5.5.5}                   \\
    \implies & \forall \delta \in \R^+, \exists N \in \Z^+ :                                                                                    \\
             & \sup\big(f(a_n)\big)_{n = N}^\infty \leq \sup(S_{\delta})                                      &  & \by{5.5.5}                   \\
    \implies & \forall \delta \in \R^+, \limsup_{n \to \infty} f(a_n) \leq \sup(S_{\delta})                   &  & \text{(by \cref{6.4.12}(c))} \\
    \implies & \limsup_{n \to \infty} f(a_n) \leq \inf\set{\sup(S_{\delta}) : \delta \in \R^+} = L.           &  & \by{5.5.15}
  \end{align*}
  For each \(n \in \Z^+\), define \(Y_n\) as follow:
  \[
    Y_n = \set{x \in X_{\dfrac{1}{n}} : L - \dfrac{1}{n} < f(x) \leq L}.
  \]
  We must have \(Y_n \neq \emptyset\) for each \(n \in \N\), otherwise
  \begin{align*}
             & Y_n = \emptyset                                                                                     \\
    \implies & \forall x \in X_{\dfrac{1}{n}}, f(x) \leq L - \dfrac{1}{n}                                          \\
    \implies & \sup(S_{\dfrac{1}{n}}) \leq L - \dfrac{1}{n}                                                        \\
    \implies & L = \inf\set{\sup(S_{\delta}) : \delta \in \R^+} \leq \sup(S_{\dfrac{1}{n}}) \leq L - \dfrac{1}{n},
  \end{align*}
  a contradiction.
  By axiom of choice (\cref{8.1}) we know that \(\prod_{n \in \Z^+} Y_n \neq \emptyset\).
  Let \(g \in \prod_{n \in \Z^+} Y_n\) and define \((b_n)_{n = 1}^\infty\) by setting \(b_n = g(n)\) for each \(n \in \Z^+\).
  Then we have \(\lim_{n \to \infty} b_n = x_0\) and \(L - \dfrac{1}{n} < f(b_n) \leq L\) for each \(n \in \Z^+\).
  By squeeze test (\cref{6.4.14}) we have \(\lim_{n \to \infty} f(b_n) = L\).
  By \cref{6.4.12}(f) we have \(\limsup_{n \to \infty} f(b_n) = \lim_{n \to \infty} f(b_n) = L\).

  Now we show that statement (b) implies statement (a).
  Suppose that for every sequence \((a_n)_{n = 1}^\infty\) in \(E\), \(\lim_{n \to \infty} a_n = x_0 \implies \limsup_{n \to \infty} f(a_n) \leq L\).
  Suppose also that there exists a sequence \((b_n)_{n = 1}^\infty\) in \(E\) such that \(\lim_{n \to \infty} b_n = x_0 \implies \limsup_{n \to \infty} f(b_n) = L\).
  Observe that
  \begin{align*}
             & \lim_{n \to \infty} b_n = x_0                                                                                                    \\
    \implies & \forall \delta \in \R^+, \exists N \in \Z^+ :                                                                                    \\
             & \forall n \geq N, \abs{b_n - x_0} \leq \delta                                                                                    \\
    \implies & \forall \delta \in \R^+, \exists N \in \Z^+ :                                                                                    \\
             & \forall n \geq N, \big(f(b_n) \in S_{\delta}\big) \land \big(f(b_n) \leq \sup(S_{\delta})\big) &  & \by{5.5.5}                   \\
    \implies & \forall \delta \in \R^+, \exists N \in \Z^+ :                                                                                    \\
             & \sup\big(f(b_n)\big)_{n = N}^\infty \leq \sup(S_{\delta})                                      &  & \by{5.5.5}                   \\
    \implies & \forall \delta \in \R^+, L = \limsup_{n \to \infty} f(b_n) \leq \sup(S_{\delta})               &  & \text{(by \cref{6.4.12}(c))} \\
    \implies & L \leq \inf\set{\sup(S_{\delta}) : \delta \in \R^+}.                                           &  & \by{5.5.15}
  \end{align*}
  Suppose for sake of contradiction that \(L < \inf\set{\sup(S_\delta) : \delta \in \R^+}\).
  Then we know that
  \[
    \inf\set{\sup(S_\delta) : \delta \in \R^+} = L + \varepsilon
  \]
  for some \(\varepsilon \in \R^+\).
  Since statement (a) implies statement (b), we know that there exist a sequence \((c_n)_{n = 1}^\infty\) in \(E\) such that \(\lim_{n \to \infty} c_n = x_0\) and \(\lim_{n \to \infty} f(c_n) = L + \varepsilon\).
  But this contradict to the hypothesis that \(\lim_{n \to \infty} f(c_n) \leq L\).
  Thus we must have \(L = \inf\set{\sup(S_{\delta}) : \delta \in \R^+}\).
\end{proof}

\exercisesection

\begin{ex}\label{ex:9.3.1}
  Prove \cref{9.3.9}.
\end{ex}

\begin{proof}
  See \cref{9.3.9}.
\end{proof}

\begin{ex}\label{ex:9.3.2}
  Prove the remaining claims in \cref{9.3.14}.
\end{ex}

\begin{proof}
  See \cref{9.3.14}.
\end{proof}

\begin{ex}\label{ex:9.3.3}
  Prove \cref{9.3.18}.
\end{ex}

\begin{proof}
  See \cref{9.3.18}.
\end{proof}

\begin{ex}\label{ex:9.3.4}
  Propose a definition for limit superior \(\limsup_{x \to x_0 ; x \in E} f(x)\) and limit inferior \(\liminf_{x \to x_0 ; x \in E} f(x)\), and then propose an analogue of \cref{9.3.9} for your definition.
  (For an additional challenge: prove that analogue.)
\end{ex}

\begin{proof}
  See Addtional \cref{ac:9.3.1} and Addtional \cref{ac:9.3.2}.
\end{proof}

\begin{ex}[Continuous version of squeeze test]\label{ex:9.3.5}
  Let \(X\) be a subset of \(\R\), let \(E\) be a subset of \(X\), let \(x_0\) be an adherent point of \(E\), and let \(f : X \to \R\), \(g : X \to \R\), \(h : X \to \R\) be functions such that \(f(x) \leq g(x) \leq h(x)\) for all \(x \in E\).
  If we have \(\lim_{x \to x_0 ; x \in E} f(x) = \lim_{x \to x_0 ; x \in E} h(x) = L\) for some real number \(L\), show that \(\lim_{x \to x_0 ; x \in E} g(x) = L\).
\end{ex}

\begin{proof}
  Since \(\lim_{x \to x_0 ; x \in E} f(x) = L\), by \cref{9.3.6} we have
  \[
    \forall \varepsilon \in \R^+, \exists \delta_1 \in \R^+ : \big(\forall x \in E, \abs{x - x_0} < \delta_1 \implies \abs{f(x) - L} \leq \varepsilon\big).
  \]
  Similarly we have
  \[
    \forall \varepsilon \in \R^+, \exists \delta_2 \in \R^+ : \big(\forall x \in E, \abs{x - x_0} < \delta_2 \implies \abs{h(x) - L} \leq \varepsilon\big).
  \]
  Let \(\delta = \min(\delta_1, \delta_2)\).
  Then we have
  \begin{align*}
     & \forall \varepsilon \in \R^+, \exists \delta \in \R^+ :                                                                                     \\
     & \forall x \in E, \abs{x - x_0} < \delta \implies \big(\abs{f(x) - L} \leq \varepsilon\big) \land \big(\abs{h(x) - L} \leq \varepsilon\big).
  \end{align*}
  Since \(f(x) \leq g(x) \leq h(x)\), we have
  \begin{align*}
             & \big(x \in E\big) \land \big(\abs{x - x_0} < \delta\big)                                                                     \\
    \implies & \big(f(x) \leq g(x) \leq h(x)\big) \land \big(\abs{f(x) - L} < \varepsilon\big) \land \big(\abs{h(x) - L} < \varepsilon\big) \\
    \implies & -\varepsilon \leq f(x) - L \leq g(x) - L \leq h(x) - L \leq \varepsilon                                                      \\
    \implies & \abs{g(x) - L} \leq \varepsilon.
  \end{align*}
  But this means
  \[
    \forall \varepsilon \in \R^+, \exists \delta \in \R^+ : \big(\forall x \in E, \abs{x - x_0} < \delta \implies \abs{g(x) - L} \leq \varepsilon\big)
  \]
  and thus by \cref{9.3.6} \(\lim_{x \to x_0 ; x \in E} g(x) = L\).
\end{proof}

\section{Continuous functions}\label{sec:9.4}

\begin{defn}[Continuity]\label{9.4.1}
  Let \(X\) be a subset of \(\R\), and let \(f : X \to \R\) be a function.
  Let \(x_0\) be an element of \(X\).
  We say that \(f\) is \emph{continuous at \(x_0\)} iff we have
  \[
    \lim_{x \to x_0 ; x \in X} f(x) = f(x_0);
  \]
  in other words, the limit of \(f(x)\) as \(x\) converges to \(x_0\) in \(X\) exists and is equal to \(f(x_0)\).
  We say that \(f\) is \emph{continuous on \(X\)} (or simply \emph{continuous}) iff \(f\) is continuous at \(x_0\) for every \(x_0 \in X\).
  We say that \(f\) is \emph{discontinuous at \(x_0\)} iff it is not continuous at \(x_0\).
  We also extend these notions to functions \(f : X \to Y\) that take values in a subset \(Y\) of \(\R\), by identifying such functions (by abuse of notation) with the function \(\tilde{f} : X \to \R\) that agrees everywhere with \(f\) (so \(\tilde{f(x)} = f(x)\) for all \(x \in X\)) but where the codomain has been enlarged from \(Y\) to \(\R\).
\end{defn}

\begin{note}
  Restricting the domain of a function can make a discontinuous function continuous again.
\end{note}

\setcounter{thm}{6}
\begin{prop}[Equivalent formulations of continuity]\label{9.4.7}
  Let \(X\) be a subset of \(\R\), let \(f : X \to \R\) be a function, and let \(x_0\) be an element of \(X\).
  Then the following four statements are logically equivalent:
  \begin{enumerate}
    \item \(f\) is continuous at \(x_0\).
    \item For every sequence \((a_n)_{n = 0}^\infty\) consisting of elements of \(X\) with \(\lim_{n \to \infty} a_n = x_0\), we have \(\lim_{n \to \infty} f(a_n) = f(x_0)\).
    \item For every \(\varepsilon > 0\), there exists a \(\delta > 0\) such that \(\abs{f(x) - f(x_0)} < \varepsilon\) for all \(x \in X\) with \(\abs{x - x_0} < \delta\).
    \item For every \(\varepsilon > 0\), there exists a \(\delta > 0\) such that \(\abs{f(x) - f(x_0)} \leq \varepsilon\) for all \(x \in X\) with \(\abs{x - x_0} \leq \delta\).
  \end{enumerate}
\end{prop}

\begin{proof}
  We first show that the statement (a) and the statement (b) are equivalent.
  By \cref{9.4.1}, \(f\) is continuous at \(x_0\) iff \(f\) converges to \(f(x_0)\) at \(x_0\) in \(X\).
  Thus by \cref{9.3.9} we know that the statement (a) and the statement (b) are equivalent.

  Next we show that the statement (a) and the statement (c) are equivalent.
  By \cref{9.4.1}, \(f\) is continuous at \(x_0\) iff \(f\) converges to \(f(x_0)\) at \(x_0\) in \(X\).
  By \cref{9.3.6} this is equivalent to the statement
  \[
    \forall \varepsilon \in \R^+, \exists\ \delta \in \R^+ : \big(\forall x \in X, \abs{x - x_0} < \delta \implies \abs{f(x) - f(x_0)} \leq \dfrac{\varepsilon}{2} < \varepsilon\big).
  \]
  Thus the statement (a) and the statement (c) are equivalent.

  Finally we show that the statement (a) and the statement (d) are equivalent.
  By \cref{9.4.1}, \(f\) is continuous at \(x_0\) iff \(f\) converges to \(f(x_0)\) at \(x_0\) in \(X\).
  By \cref{9.3.6} this is equivalent to the statement
  \[
    \forall \varepsilon \in \R^+, \exists\ \delta' \in \R^+ : \big(\forall x \in X, \abs{x - x_0} < \delta' \implies \abs{f(x) - f(x_0)} \leq \varepsilon\big).
  \]
  Let \(\delta \in \R^+\) and \(\abs{x - x_0} \leq \delta < \delta'\).
  By \cref{5.4.14} we know such \(\delta\) exists.
  Then we have
  \[
    \forall \varepsilon \in \R^+, \exists\ \delta \in \R^+ : \big(\forall x \in X, \abs{x - x_0} \leq \delta \implies \abs{f(x) - f(x_0)} \leq \varepsilon\big).
  \]
  Thus the statement (a) and the statement (d) are equivalent.
\end{proof}

\begin{rmk}\label{9.4.8}
  A particularly useful consequence of \cref{9.4.7} is the following:
  if \(f\) is continuous at \(x_0\), and \(a_n \to x_0\) as \(n \to \infty\), then \(f(a_n) \to f(x_0)\) as \(n \to \infty\)
  (provided that all the elements of the sequence \((a_n)_{n = 0}^\infty\) lie in the domain of \(f\), of course).
  Thus continuous functions are very useful in computing limits.
\end{rmk}

\begin{prop}[Arithmetic preserves continuity]\label{9.4.9}
  Let \(X\) be a subset of \(\R\), and let \(f : X \to \R\) and \(g : X \to \R\) be functions.
  Let \(x_0 \in X\).
  Then if \(f\) and \(g\) are both continuous at \(x_0\), then the functions \(f + g\), \(f - g\), \(\max(f, g)\), \(\min(f, g)\) and \(fg\) are also continuous at \(x_0\).
  If \(g\) is non-zero on \(X\), then \(f / g\) is also continuous at \(x_0\).
\end{prop}

\begin{proof}
  By \cref{9.3.14}, we have
  \begin{align*}
    \lim_{x \to x_0 ; x \in X} f(x) + g(x)      & = \lim_{x \to x_0 ; x \in X} f(x) + \lim_{x \to x_0 ; x \in X} g(x)                                                               \\
                                                & = f(x_0) + g(x_0);                                                                       &  & \by{9.4.1}                          \\
    \lim_{x \to x_0 ; x \in X} f(x) - g(x)      & = \lim_{x \to x_0 ; x \in X} f(x) - \lim_{x \to x_0 ; x \in X} g(x)                                                               \\
                                                & = f(x_0) - g(x_0);                                                                       &  & \by{9.4.1}                          \\
    \lim_{x \to x_0 ; x \in X} \max(f(x), g(x)) & = \max(\lim_{x \to x_0 ; x \in X} f(x), \lim_{x \to x_0 ; x \in X} g(x))                                                          \\
                                                & = \max(f(x_0), g(x_0));                                                                  &  & \by{9.4.1}                          \\
    \lim_{x \to x_0 ; x \in X} \min(f(x), g(x)) & = \min(\lim_{x \to x_0 ; x \in X} f(x), \lim_{x \to x_0 ; x \in X} g(x))                                                          \\
                                                & = \min(f(x_0), g(x_0));                                                                  &  & \by{9.4.1}                          \\
    \lim_{x \to x_0 ; x \in X} f(x) g(x)        & = \bigg(\lim_{x \to x_0 ; x \in X} f(x)\bigg)\bigg(\lim_{x \to x_0 ; x \in X} g(x)\bigg)                                          \\
                                                & = f(x_0) g(x_0);                                                                         &  & \by{9.4.1}                          \\
    \lim_{x \to x_0 ; x \in X} f(x) / g(x)      & = \lim_{x \to x_0 ; x \in X} f(x) / \lim_{x \to x_0 ; x \in X} g(x)                      &  & \text{(\(g\) is non-zero on \(X\))} \\
                                                & = f(x_0) / g(x_0).                                                                       &  & \by{9.4.1}                          \\
  \end{align*}
\end{proof}

\begin{prop}[Exponentiation is continuous, I]\label{9.4.10}
  Let \(a > 0\) be a positive real number.
  Then the function \(f : \R \to \R\) defined by \(f(x) \coloneqq a^x\) is continuous.
\end{prop}

\begin{proof}
  Let \(x_0 \in \R\).
  By \cref{9.1.13} \(x_0\) is an adherent point.
  By \cref{6.7.3} we know that \(a^{x_0} > 0\).
  By \cref{6.5.3} we know that \(\lim_{n \to \infty} a^{\dfrac{1}{n}} = 1\), thus
  \begin{align*}
    a^{x_0} & = a^{x_0} \lim_{n \to \infty} a^{\pm \dfrac{1}{n}}   &  & \by{6.7.3}                \\
            & = \lim_{n \to \infty} (a^{x_0} a^{\pm \dfrac{1}{n}}) &  & \text{(by \cref{6.1.19})} \\
            & = \lim_{n \to \infty} a^{x_0 \pm \dfrac{1}{n}}.
  \end{align*}
  Observe that
  \begin{align*}
             & \lim_{n \to \infty} a^{x_0 \pm \dfrac{1}{n}} = a^{x_0}                                                                                        \\
    \implies & \forall \varepsilon \in \R^+, \exists\ N \in \N : \forall n \geq N, \abs{a^{x_0 \pm \dfrac{1}{n}} - a^{x_0}} \leq \varepsilon                 \\
    \implies & \forall \varepsilon \in \R^+, \exists\ N \in \N : \abs{a^{x_0 \pm \dfrac{1}{N}} - a^{x_0}} \leq \varepsilon                                   \\
    \implies & \forall \varepsilon \in \R^+, \exists\ N \in \N : -\varepsilon \leq a^{x_0 \pm \dfrac{1}{N}} - a^{x_0} \leq \varepsilon.      &  & \by{6.7.3}
  \end{align*}
  Now fix \(N\) for each \(\varepsilon \in \R^+\).
  By \cref{6.7.3} we have
  \begin{align*}
             & \forall x \in \R, \abs{x - x_0} < \dfrac{1}{N}                                                                                                             \\
    \implies & \dfrac{-1}{N} < x - x_0 < \dfrac{1}{N}                                                                                                                     \\
    \implies & x_0 - \dfrac{1}{N} < x < x_0 + \dfrac{1}{N}                                                                                                                \\
    \implies & \begin{dcases}
                 a^{x_0 - \dfrac{1}{N}} < a^{x} < a^{x_0 + \dfrac{1}{N}} & \text{if } a \geq 1 \\
                 a^{x_0 + \dfrac{1}{N}} < a^{x} < a^{x_0 - \dfrac{1}{N}} & \text{if } a < 1
               \end{dcases}                                                          \\
    \implies & \begin{dcases}
                 -\varepsilon < a^{x_0 - \dfrac{1}{N}} - a^{x_0} < a^x - a^{x_0} < a^{x_0 + \dfrac{1}{N}} - a^{x_0} < \varepsilon & \text{if } a \geq 1 \\
                 -\varepsilon < a^{x_0 + \dfrac{1}{N}} - a^{x_0} < a^x - a^{x_0} < a^{x_0 - \dfrac{1}{N}} - a^{x_0} < \varepsilon & \text{if } a < 1
               \end{dcases} \\
    \implies & \abs{a^x - a^{x_0}} < \varepsilon.
  \end{align*}
  By setting \(\delta = \dfrac{1}{N}\) we have
  \[
    \forall \varepsilon \in \R^+, \exists\ \delta \in \R^+ : \big(\forall x \in \R, \abs{x - x_0} < \delta \implies \abs{a^x - a^{x_0}} < \varepsilon\big).
  \]
  Thus by \cref{9.3.6} we have \(\lim_{x \to x_0 ; x \in \R} a^x = a^{x_0}\).
  Since \(x_0\) is arbitrary, by \cref{9.4.1} \(a^x\) is continuous on \(\R\).
\end{proof}

\begin{prop}[Exponentiation is continuous, II]\label{9.4.11}
  Let \(p\) be a real number.
  Then the function \(f : (0, \infty) \to \R\) defined by \(f(x) \coloneqq x^p\) is continuous.
\end{prop}

\begin{proof}
  By \cref{9.3.14} we know that
  \[
    \forall n \in \N, \lim_{x \to 1 ; x \in (0, \infty)} x^n = 1.
  \]
  Again by \cref{9.3.14} we know that
  \[
    \forall n \in \N, \lim_{x \to 1 ; x \in (0, \infty)} x^{-n} = \lim_{x \to 1 ; x \in (0, \infty)} 1 / x^n = 1.
  \]
  By \cref{ex:5.4.3} we have
  \[
    \forall p \in \R, \exists\ N \in \Z : N \leq p < N + 1.
  \]
  By \cref{6.7.3} we have
  \[
    \forall x \in (0, \infty), \begin{dcases}
      x^N \leq x^p < x^{N + 1}   & \text{if } x \in (0, 1)      \\
      x^{N + 1} < x^p \leq x^{N} & \text{if } x \in [1, \infty)
    \end{dcases}
  \]
  Thus by squeeze test (\cref{ex:9.3.5}) we have \(\lim_{x \to 1 ; x \in (0, \infty)} x^p = 1\).

  Let \(x_0 \in (0, \infty)\).
  By \cref{9.1.12} \(x_0\) is an adherent point.
  By \cref{6.7.3} \(x_0^p > 0\).
  Since \(\lim_{x \to 1 ; x \in (0, \infty)} x^p = 1\), by \cref{9.3.6} we have
  \[
    \forall \varepsilon \in \R^+, \exists\ \delta \in \R^+ : \big(\forall x \in (0, \infty), \abs{x - 1} < \delta \implies \abs{x^p - 1} \leq \varepsilon\big).
  \]
  Let \(y = x \cdot x_0\).
  Then \(x = y / x_0\) and
  \[
    \forall \varepsilon \in \R^+, \exists\ \delta \in \R^+ : \big(\forall y \in (0, \infty), \abs{y / x_0 - 1} < \delta \implies \abs{(\dfrac{y}{x_0})^p - 1} \leq \varepsilon\big).
  \]
  Since \(\delta / x_0 < \delta\), we have
  \[
    \forall \varepsilon \in \R^+, \exists\ \delta \in \R^+ : \big(\forall y \in (0, \infty), \abs{y / x_0 - 1} < \delta / x_0 \implies \abs{(\dfrac{y}{x_0})^p - 1} \leq \varepsilon\big).
  \]
  This means
  \[
    \forall \varepsilon \in \R^+, \exists\ \delta \in \R^+ : \big(\forall y \in (0, \infty), \abs{y - x_0} < \delta \implies \abs{(\dfrac{y}{x_0})^p - 1} \leq \varepsilon\big).
  \]
  In particular, we have
  \[
    \forall \varepsilon \in \R^+, \exists\ \delta \in \R^+ : \big(\forall y \in (0, \infty), \abs{y - x_0} < \delta \implies \abs{(\dfrac{y}{x_0})^p - 1} \leq \dfrac{\varepsilon}{x_0^p}\big).
  \]
  This means
  \[
    \forall \varepsilon \in \R^+, \exists\ \delta \in \R^+ : \big(\forall y \in (0, \infty), \abs{y - x_0} < \delta \implies \abs{y^p - x_0^p} \leq \varepsilon\big).
  \]
  Thus by \cref{9.3.6} we have \(\lim_{y \to x_0 ; y \in (0, \infty)} y^p = x_0^p\).
  Since \(x_0\) is arbitrary, by \cref{9.4.1} \(x^p\) is continuous on \((0, \infty)\).
\end{proof}

\begin{prop}[Absolute value is continuous]\label{9.4.12}
  The function \(f : \R \to \R\) defined by \(f(x) \coloneqq \abs{x}\) is continuous.
\end{prop}

\begin{proof}
  This follows since \(\abs{x} = \max(x, -x)\) and the functions \(x, -x\) are already continuous.
\end{proof}

\begin{prop}[Composition preserves continuity]\label{9.4.13}
  Let \(X\) and \(Y\) be subsets of \(\R\), and let \(f : X \to Y\) and \(g : Y \to \R\) be functions.
  Let \(x_0\) be a point in \(X\).
  If \(f\) is continuous at \(x_0\), and \(g\) is continuous at \(f(x_0)\), then the composition \(g \circ f : X \to \R\) is continuous at \(x_0\).
\end{prop}

\begin{proof}
  Since \(\lim_{y \to f(x_0) ; y \in Y} g(x) = g\big(f(x_0)\big)\), by \cref{9.3.6} we have
  \[
    \forall \varepsilon \in \R^+, \exists\ \delta' \in \R^+ : \Big(\forall y \in Y, \abs{y - f(x_0)} < \delta' \implies \abs{g(y) - g\big(f(x_0)\big)} < \varepsilon\Big).
  \]
  Since \(\lim_{x \to x_0 ; x \in X} f(x) = f(x_0)\), by \cref{9.3.6} we have
  \[
    \forall \varepsilon' \in \R^+, \exists\ \delta \in \R^+ : \Big(\forall x \in X, \abs{x - x_0} < \delta \implies \abs{f(x) - f(x_0)} < \varepsilon'\Big).
  \]
  In particular, we have
  \[
    \exists\ \delta \in \R^+ : \Big(\forall x \in X, \abs{x - x_0} < \delta \implies \abs{f(x) - f(x_0)} < \delta'\Big).
  \]
  Since \(f(x) \in Y\) and \(\abs{f(x) - f(x_0)} < \delta'\) implies \(\abs{g\big(f(x)\big) - g\big(f(x_0)\big)} < \varepsilon\), we have
  \[
    \forall \varepsilon \in \R^+, \exists\ \delta \in \R^+ : \Big(\forall x \in X, \abs{x - x_0} < \delta \implies \abs{g\big(f(x)\big) - g\big(f(x_0)\big)} < \varepsilon\Big).
  \]
  Thus by \cref{9.3.6} we have \(\lim_{x \to x_0 ; x \in X} g\big(f(x)\big) = g\big(f(x_0)\big)\).
  By \cref{9.4.1} \(g \circ f\) is continuous at \(x_0\).
\end{proof}

\exercisesection

\begin{ex}\label{ex:9.4.1}
  Prove \cref{9.4.7}.
\end{ex}

\begin{proof}
  See \cref{9.4.7}
\end{proof}

\begin{ex}\label{ex:9.4.2}
  Let \(X\) be a subset of \(\R\), and let \(c \in \R\).
  Show that the constant function \(f : X \to \R\) defined by \(f(x) \coloneqq c\) is continuous, and show that the identity function \(g : X \to \R\) defined by \(g(x) \coloneqq x\) is also continuous.
\end{ex}

\begin{proof}
  We first show that the constant functions \(f\) is continuous.
  Let \(x_0 \in X\).
  By \cref{9.1.11} we know that \(X \subseteq \overline{X}\), thus \(x_0\) is an adherent point of \(X\).
  Since
  \[
    \forall \varepsilon \in \R^+, \forall \delta \in \R^+, \abs{x - x_0} < \delta \implies \abs{f(x) - c} = \abs{c - c} = 0 < \varepsilon,
  \]
  we know that \(\lim_{x \to x_0 ; x \in X} f(x) = c\).
  Since \(f(x_0) = c\) and \(x_0\) is arbitrary, by \cref{9.4.1} we know that the constant functions \(f\) is continuous on \(X\).

  Now we show that the identity function \(g\) is continuous.
  Let \(x_0 \in X\).
  By \cref{9.1.11} we know that \(X \subseteq \overline{X}\), thus \(x_0\) is an adherent point of \(X\).
  Since
  \[
    \forall \varepsilon \in \R^+, \abs{x - x_0} < \varepsilon \implies \abs{g(x) - x_0} = \abs{x - x_0} < \varepsilon,
  \]
  we know that \(\lim_{x \to x_0 ; x \in X} g(x) = x_0\).
  Since \(g(x_0) = x_0\) and \(x_0\) is arbitrary, by \cref{9.4.1} we know that the constant functions \(g\) is continuous on \(X\).
\end{proof}

\begin{ex}\label{ex:9.4.3}
  Prove \cref{9.4.10}.
\end{ex}

\begin{proof}
  See \cref{9.4.10}.
\end{proof}

\begin{ex}\label{ex:9.4.4}
  Prove \cref{9.4.11}.
\end{ex}

\begin{proof}
  See \cref{9.4.11}.
\end{proof}

\begin{ex}\label{ex:9.4.5}
  Prove \cref{9.4.13}.
\end{ex}

\begin{proof}
  See \cref{9.4.13}.
\end{proof}

\begin{ex}\label{ex:9.4.6}
  Let \(X\) be a subset of \(\R\), and let \(f : X \to \R\) be a continuous function.
  If \(Y\) is a subset of \(X\), show that the restriction \(f|_Y : Y \to \R\) of \(f\) to \(Y\) is also a continuous function.
\end{ex}

\begin{proof}
  By \cref{9.4.1}, \(\forall x_0 \in X\), we have \(\lim_{x \to x_0 ; x \in X} f(x) = f(x_0)\).
  Since \(Y \subseteq X\), we have \(\forall y \in Y \implies y \in X\).
  Thus \(\forall y_0 \in Y\) we have \(\lim_{y \to y_0 ; y \in Y} f(y) = f(y_0)\).
  By \cref{9.4.1}, \(f|_Y\) is continuous on \(Y\).
\end{proof}

\begin{ex}\label{ex:9.4.7}
  Let \(n \geq 0\) be an integer, and for each \(0 \leq i \leq n\) let \(c_i\) be a real number.
  Let \(P : \R \to \R\) be the function
  \[
    P(x) \coloneqq \sum_{i = 0}^n c_i x^i;
  \]
  Such a function is known as a \emph{polynomial of one variable};
  Show that \(P\) is continuous.
\end{ex}

\begin{proof}
  Let \(F_n = \{\text{all polynomial function with the highest order being \(n\)}\}\).
  Let \(Q(n)\) be the statement ``all function \(f \in F_n\) are continuous''.
  We use induction on \(n\) to show that \(\forall n \in \N\), \(Q(n)\) is true.
  For \(n = 0\), we have
  \[
    \sum_{n = 0}^0 c_i x^i = c_0 x^0 = c_0.
  \]
  By \cref{ex:9.4.2} we know that constant functions are continuous, and thus the base case holds.
  Suppose inductively that \(Q(n)\) is true for some \(n \geq 0\).
  To show that \(Q(n + 1)\) is true, observe that every function \(f \in F_{n + 1}\) are in the form
  \begin{align*}
    f(x) & = \sum_{i = 0}^{n + 1} c_i x^i                      \\
         & = \sum_{i = 0}^n c_i x^i + c_{n + 1} x^{n + 1}      \\
         & = \sum_{i = 0}^n c_i x^i + c_{n + 1} (x^n \cdot x).
  \end{align*}
  By induction hypothesis we know that \(\sum_{i = 0}^n c_i x^i\) and \(x^n\) are continuous.
  By \cref{ex:9.4.2} we know that \(c_{n + 1}\) and \(x\) are continuous.
  Thus by \cref{9.4.9} we know that \(f\) is continuous.
  This closes the induction.
\end{proof}
\section{Left and right limits}\label{sec:9.5}

\begin{defn}[Left and right limits]\label{9.5.1}
  Let \(X\) be a subset of \(\R\), \(f : X \to \R\) be a function, and let \(x_0\) be a real number.
  If \(x_0\) is an adherent point of \(X \cap (x_0, \infty)\), then we define the \emph{right limit} \(f(x_0+)\) of \(f\) at \(x_0\) by the formula
  \[
    f(x_0+) \coloneqq \lim_{x \to x_0 ; x \in X \cap (x_0, \infty)} f(x),
  \]
  provided of course that this limit exists.
  Similarly, if \(x_0\) is an adherent point of \(X \cap (-\infty, x_0)\), then we define the \emph{left limit} \(f(x_0-)\) of \(f\) at \(x_0\) by the formula
  \[
    f(x_0-) \coloneqq \lim_{x \to x_0 ; x \in X \cap (-\infty, x_0)} f(x),
  \]
  again provided that the limit exists.
  (Thus in many cases \(f(x_0+)\) and \(f(x_0-)\) will not be defined.)
  Sometimes we use the shorthand notations
  \begin{align*}
    \lim_{x \to x_0+} f(x) & \coloneqq \lim_{x \to x_0 ; x \in X \cap (x_0, \infty)} f(x); \\
    \lim_{x \to x_0-} f(x) & \coloneqq \lim_{x \to x_0 ; x \in X \cap (-\infty, x_0)} f(x)
  \end{align*}
  when the domain \(X\) of \(f\) is clear from context.
\end{defn}

\begin{note}
  From \cref{9.3.9} we see that if the right limit \(f(x_0+)\) exists, and \((a_n)_{n = 0}^\infty\) is a sequence in \(X\) converging to \(x_0\) from the right (i.e., \(a_n > x_0\) for all \(n \in \N\)), then \(\lim_{n \to \infty} f(a_n) = f(x_0+)\).
  Similarly, if \((b_n)_{n = 0}^\infty\) is a sequence converging to \(x_0\) from the left (i.e., \(a_n < x_0\) for all \(n \in \N\)) then \(\lim_{n \to \infty} f(a_n) = f(x_0-)\).
\end{note}

\begin{ac}\label{ac:9.5.1}
  Let \(x_0\) be an adherent point of both \(X \cap (x_0, \infty)\) and \(X \cap (-\infty, x_0)\).
  If \(f\) is continuous at \(x_0\), then \(f(x_0+)\) and \(f(x_0-)\) both exists and are equal to \(f(x_0)\).
\end{ac}

\begin{proof}
  Since \(f\) is continuous at \(x_0\), by \cref{9.4.1} we know that
  \[
    \forall \varepsilon \in \R^+, \exists \delta \in \R^+ : \big(\forall x \in X, \abs{x - x_0} < \delta \implies \abs{f(x) - f(x_0)} < \varepsilon\big).
  \]
  Since \(X \cap (-\infty, x_0) \subseteq X\), we must have
  \[
    \forall \varepsilon \in \R^+, \exists \delta \in \R^+ : \big(\forall x \in X \cap (-\infty, x_0), \abs{x - x_0} < \delta \implies \abs{f(x) - f(x_0)} < \varepsilon\big)
  \]
  and by \cref{9.5.1} we have \(f(x_0+) = f(x_0)\).
  Similarly we have \(f(x_0-) = f(x_0)\).
\end{proof}

\setcounter{thm}{2}
\begin{prop}\label{9.5.3}
  Let \(X\) be a subset of \(\R\) containing a real number \(x_0\), and suppose that \(x_0\) is an adherent point of both \(X \cap (x_0, \infty)\) and \(X \cap (-\infty, x_0)\).
  Let \(f : X \to \R\) be a function.
  If \(f(x_0+)\) and \(f(x_0-)\) both exist and are both equal to \(f(x_0)\), then \(f\) is continuous at \(x_0\).
\end{prop}

\begin{proof}
  Let us write \(L \coloneqq f(x_0)\).
  Then by hypothesis we have
  \[
    \lim_{x \to x_0 ; x \in X \cap (x_0, \infty)} f(x) = L
  \]
  and
  \[
    \lim_{x \to x_0 ; x \in X \cap (-\infty, x_0)} f(x) = L.
  \]
  Let \(\varepsilon > 0\) be given.
  From the first statement above and \cref{9.4.7} (applied to the restriction of \(f\) to \(X \cap (x_0, +\infty)\)), we know that there exists a \(\delta_+ > 0\) such that \(\abs{f(x) - L} < \varepsilon\) for all \(x \in X \cap(x_0, \infty)\) for which \(\abs{x - x_0} < \delta_+\).
  From the second statement above we similarly know that there exists a \(\delta_- > 0\) such that \(\abs{f(x) - L} < \varepsilon\) for all \(x \in X \cap (-\infty, x_0)\) for which \(\abs{x - x_0} < \delta_-\).
  Now let \(\delta \coloneqq \min(\delta_-, \delta_+)\);
  then \(\delta > 0\), and suppose that \(x \in X\) is such that \(\abs{x - x_0} < \delta\).
  Then there are three cases:
  \(x > x_0\), \(x = x_0\), and \(x < x_0\), but in all three cases we know that \(\abs{f(x) - L} < \varepsilon\) since
  \begin{itemize}
    \item If \(x > x_0\), then \(x \in X \cap (x_0, \infty)\) and \(\abs{x - x_0} < \delta \leq \delta_+ \implies \abs{f(x) - L} < \varepsilon\).
    \item If \(x < x_0\), then \(x \in X \cap (-\infty, x_0)\) and \(\abs{x - x_0} < \delta \leq \delta_- \implies \abs{f(x) - L} < \varepsilon\).
    \item If \(x = x_0\), then we have \(\abs{x_0 - x_0} = 0 < \delta\) and \(\abs{f(x_0) - f(x_0)} = 0 < \varepsilon\).
  \end{itemize}
  By \cref{9.4.7} we thus have that \(f\) is continuous at \(x_0\), as desired.
\end{proof}

\begin{note}
  When both \(f(x_0+), f(x_0-)\) exist and \(f(x_0+) \neq f(x_0-)\), we say that \(f\) has a \emph{jump discontinuity} at \(x_0\).
  When both \(f(x_0+), f(x_0-)\) exist and \(f(x_0+) = f(x_0-) \neq f(x_0)\), we say that \(f\) has a \emph{removable discontinuity} (or \emph{removable singularity}) at \(x_0\).
\end{note}

\begin{rmk}\label{9.5.4}
  Jump discontinuities and removable discontinuities are not the only way a function can be discontinuous.
  Another way is for a function to go to infinity at the discontinuity:
  for instance, the function \(f : \R \setminus \set{0} \to \R\) defined by \(f(x) \coloneqq 1 / x\) has a discontinuity at \(0\) which is neither a jump discontinuity or a removable singularity;
  informally, \(f(x)\) converges to \(+\infty\) when \(x\) approaches \(0\) from the right, and converges to \(-\infty\) when \(x\) approaches \(0\) from the left.
  These types of singularities are sometimes known as \emph{asymptotic discontinuities}.
  There are also \emph{oscillatory discontinuities}, where the function remains bounded but still does not have a limit near \(x_0\).
  For instance, the function \(f : \R \to \R\) defined by
  \[
    f(x) \coloneqq \begin{dcases}
      1 & \text{if } x \in \Q    \\
      0 & \text{if } x \notin \Q
    \end{dcases}
  \]
  has an oscillatory discontinuity at \(0\) (and in fact at any other real number also).
  This is because the function does not have left or right limits at \(0\), despite the fact that the function is bounded.
\end{rmk}

\begin{note}
  The study of discontinuities is also called \emph{singularities}.
\end{note}

\begin{ac}\label{ac:9.5.2}
  We define \(\lim_{x \to x_0 ; x \in E} f(x) = +\infty\) iff
  \[
    \forall \varepsilon \in \R^+, \exists \delta \in \R^+ : \big(\forall x \in E, \abs{x - x_0} < \delta \implies f(x) > \varepsilon\big).
  \]
  And define \(\lim_{x \to x_0 ; x \in E} f(x) = -\infty\) iff
  \[
    \forall \varepsilon \in \R^+, \exists \delta \in \R^+ : \big(\forall x \in E, \abs{x - x_0} < \delta \implies f(x) < -\varepsilon\big).
  \]
  Show that \(\lim_{x \to 0 ; x \in \R \cap (0, \infty)} 1 / x = +\infty\) and \(\lim_{x \to 0 ; x \in \R \cap (-\infty, 0)} 1 / x = -\infty\).
\end{ac}

\begin{proof}
  We first show that \(\lim_{x \to 0 ; x \in \R \cap (0, +\infty)} 1 / x = +\infty\) and \(\lim_{x \to 0 ; x \in \R \cap (-\infty, 0)} 1 / x = -\infty\).
  Let \(\varepsilon \in \R^+\).
  \(\forall x \in \R \cap (0, \infty)\), we have
  \[
    x = \abs{x} = \abs{x - 0} < 1 / \varepsilon \implies 1 / x < \varepsilon.
  \]
  By letting \(\delta = 1 / \varepsilon\) we have
  \[
    \forall \varepsilon \in \R^+, \exists \delta \in \R^+ : \big(\forall x \in \R \cap (0, \infty), \abs{x - 0} < \delta \implies 1 / x > \varepsilon\big).
  \]
  Thus by definition we have \(\lim_{x \to 0 ; x \in \R \cap (0, +\infty)} 1 / x = +\infty\).
  Similarly, \(\forall x \in \R \cap (-\infty, 0)\), we have
  \[
    -x = \abs{x} = \abs{x - 0} < 1 / \varepsilon \implies 1 / x < -\varepsilon.
  \]
  By letting \(\delta = 1 / \varepsilon\) we have
  \[
    \forall \varepsilon \in \R^+, \exists \delta \in \R^+ : \big(\forall x \in \R \cap (-\infty, 0), \abs{x - 0} < \delta \implies 1 / x < -\varepsilon\big).
  \]
  Thus by definition we have \(\lim_{x \to 0 ; x \in \R \cap (-\infty, 0)} 1 / x = -\infty\).
\end{proof}

\begin{ac}\label{ac:9.5.3}
  Let \(X \subseteq \R\), let \(E \subseteq X\), let \(x_0 \in \overline{E}\).
  Show that the following two statements are equivalent:
  \begin{enumerate}
    \item \(\lim_{x \to x_0 ; x \in E} f(x) = +\infty\).
    \item For every sequence \((a_n)_{n = 0}^\infty\) which consists entirely of elements of \(E\) and converges to \(x_0\), the sequence \((f(a_n))_{n = 0}^\infty\) diverges to \(+\infty\).
  \end{enumerate}
\end{ac}

\begin{proof}
  We first show that statement (a) implies statement (b).
  By \cref{ac:9.5.2} we have
  \begin{align*}
         & \lim_{x \to x_0 ; x \in E} f(x) = +\infty                                                                                              \\
    \iff & \forall \varepsilon \in \R^+, \exists \delta \in \R^+ : \big(\forall x \in E, \abs{x - x_0} < \delta \implies f(x) > \varepsilon\big).
  \end{align*}
  Let \((a_n)_{n = 0}^\infty\) be a sequence which consists entirely of elements of \(E\) and \(\lim_{n \to \infty} a_n = x_0\).
  Such sequence exists since \cref{9.1.14}.
  Then we have
  \begin{align*}
             & \lim_{n \to \infty} a_n = x_0                                                                      \\
    \implies & \exists N \in \N : \forall n \geq N, \abs{a_n - x_0} \leq \dfrac{\delta}{2} < \delta               \\
    \implies & \exists N \in \N : \forall n \geq N, f(a_n) > \varepsilon                            & (a_n \in E) \\
    \implies & \lim_{n \to \infty} f(a_n) = +\infty.
  \end{align*}
  Since \((a_n)_{n = 0}^\infty\) is arbitrary, we conclude that statement (a) implies statement (b).

  Now we show that statement (b) implies statement (a).
  Suppose for sake of contradiction that \(\lim_{x \to x_0 ; x \in E} f(x) \neq +\infty\).
  Then we must have
  \[
    \exists \varepsilon \in \R^+ : \forall \delta \in \R^+, (\abs{x - x_0} < \delta) \land \big(f(x) \leq \varepsilon\big).
  \]
  Let \((a_n)_{n = 0}^\infty\) be a sequence in \(E\) such that \(\lim_{n \to \infty} a_n = x_0\).
  Such sequence exists since \cref{9.1.14}.
  Then we have
  \begin{align*}
             & (\lim_{n \to \infty} a_n = x_0) \land \big(\lim_{n \to \infty} f(a_n) = +\infty\big)                                 &  & \text{(by hypothesis)} \\
    \implies & \begin{dcases}
                 \forall \delta \in \R^+, \exists N_1 \in \N : \forall n \geq N_1, \abs{a_n - x_0} \leq \dfrac{\delta}{2} < \delta \\
                 \forall \varepsilon \in \R^+, \exists N_2 \in \N : \forall n \geq N_2, f(a_n) > \varepsilon
               \end{dcases}                                \\
    \implies & \exists N = \max(N_1, N_2) : \forall n \geq N, (\abs{a_n - x_0} < \delta) \land \big(f(a_n) > \varepsilon\big).
  \end{align*}
  But this contradict to \((\abs{x - x_0} < \delta) \land \big(f(x) \leq \varepsilon\big)\).
  Thus \(\lim_{x \to x_0 ; x \in E} f(x) = +\infty\).
\end{proof}

\begin{ac}\label{ac:9.5.4}
  Let \(X \subseteq \R\), let \(E \subseteq X\), let \(x_0 \in \overline{E}\).
  Show that the following two statements are equivalent:
  \begin{enumerate}
    \item \(\lim_{x \to x_0 ; x \in E} f(x) = -\infty\).
    \item For every sequence \((a_n)_{n = 0}^\infty\) which consists entirely of elements of \(E\) and converges to \(x_0\), the sequence \((f(a_n))_{n = 0}^\infty\) diverges to \(-\infty\).
  \end{enumerate}
\end{ac}

\begin{proof}
  We first show that statement (a) implies statement (b).
  By \cref{ac:9.5.2} we have
  \begin{align*}
         & \lim_{x \to x_0 ; x \in E} f(x) = -\infty                                                                                               \\
    \iff & \forall \varepsilon \in \R^+, \exists \delta \in \R^+ : \big(\forall x \in E, \abs{x - x_0} < \delta \implies f(x) < -\varepsilon\big).
  \end{align*}
  Let \((a_n)_{n = 0}^\infty\) be a sequence which consists entirely of elements of \(E\) and \(\lim_{n \to \infty} a_n = x_0\).
  Such sequence exists since \cref{9.1.14}.
  Then we have
  \begin{align*}
             & \lim_{n \to \infty} a_n = x_0                                                                      \\
    \implies & \exists N \in \N : \forall n \geq N, \abs{a_n - x_0} \leq \dfrac{\delta}{2} < \delta               \\
    \implies & \exists N \in \N : \forall n \geq N, f(a_n) < -\varepsilon                           & (a_n \in E) \\
    \implies & \lim_{n \to \infty} f(a_n) = -\infty.
  \end{align*}
  Since \((a_n)_{n = 0}^\infty\) is arbitrary, we conclude that statement (a) implies statement (b).

  Now we show that statement (b) implies statement (a).
  Suppose for sake of contradiction that \(\lim_{x \to x_0 ; x \in E} f(x) \neq -\infty\).
  Then we must have
  \[
    \exists \varepsilon \in \R^+ : \forall \delta \in \R^+, (\abs{x - x_0} < \delta) \land \big(f(x) \geq -\varepsilon\big).
  \]
  Let \((a_n)_{n = 0}^\infty\) be a sequence in \(E\) such that \(\lim_{n \to \infty} a_n = x_0\).
  Such sequence exists since \cref{9.1.14}.
  Then we have
  \begin{align*}
             & (\lim_{n \to \infty} a_n = x_0) \land \big(\lim_{n \to \infty} f(a_n) = -\infty\big)                                 &  & \text{(by hypothesis)} \\
    \implies & \begin{dcases}
                 \forall \delta \in \R^+, \exists N_1 \in \N : \forall n \geq N_1, \abs{a_n - x_0} \leq \dfrac{\delta}{2} < \delta \\
                 \forall \varepsilon \in \R^+, \exists N_2 \in \N : \forall n \geq N_2, f(a_n) < -\varepsilon
               \end{dcases}                                \\
    \implies & \exists N = \max(N_1, N_2) : \forall n \geq N, (\abs{a_n - x_0} < \delta) \land \big(f(a_n) < -\varepsilon\big).
  \end{align*}
  But this contradict to \((\abs{x - x_0} < \delta) \land \big(f(x) \geq -\varepsilon\big)\).
  Thus \(\lim_{x \to x_0 ; x \in E} f(x) = -\infty\).
\end{proof}

\exercisesection

\begin{ex}\label{ex:9.5.1}
  Let \(E\) be a subset of \(\R\), let \(f : E \to \R\) be a function, and let \(x_0\) be an adherent point of \(E\).
  Write down a definition of what it would mean for the limit \(\lim_{x \to x_0 ; x \in E} f(x)\) to exist and equal \(+\infty\) or \(-\infty\).
  If \(f : \R \setminus \set{0} \to \R\) is the function \(f(x) \coloneqq 1 / x\), use your definition to conclude \(f(0+) = +\infty\) and \(f(0-) = -\infty\).
  Also, state and prove some analogue of \cref{9.3.9} when \(L = +\infty\) or \(L = -\infty\).
\end{ex}

\begin{proof}
  See \cref{ac:9.5.2}, \cref{ac:9.5.3} and \cref{ac:9.5.4}.
\end{proof}
\section{The maximum principle}\label{i:sec:9.6}

\begin{defn}\label{i:9.6.1}
  Let \(X\) be a subset of \(\R\), and let \(f : X \to \R\) be a function.
  We say that \(f\) is \emph{bounded from above} iff there exists a real number \(M\) such that \(f(x) \leq M\) for all \(x \in X\).
  We say that \(f\) is \emph{bounded from below} iff there exists a real number \(M\) such that \(f(x) \geq -M\) for all \(x \in X\).
  We say that \(f\) is \emph{bounded} iff there exists a real number \(M\) such that \(\abs{f(x)} \leq M\) for all \(x \in X\).
\end{defn}

\begin{rmk}\label{i:9.6.2}
  A function is bounded iff it is bounded both from above and below.
  Also, a function \(f : X \to \R\) is bounded iff its image \(f(X)\) is a bounded set in the sense of \cref{i:9.1.22}.
\end{rmk}

\begin{lem}\label{i:9.6.3}
  Let \(a < b\) be real numbers, and let \(f : [a, b] \to \R\) be a function continuous on \([a, b]\).
  Then \(f\) is a bounded function.
\end{lem}

\begin{proof}
  Suppose for sake of contradiction that \(f\) is not bounded.
  Thus for every real number \(M\) there exists an element \(x \in [a, b]\) such that \(\abs{f(x)} \geq M\).

  In particular, for every natural number \(n\), the set \(\set{x \in [a, b] : \abs{f(x)} \geq n}\) is non-empty.
  We can thus choose a sequence \((x_n)_{n = 0}^\infty\) in \([a, b]\) such that \(\abs{f(x_n)} \geq n\) for all \(n\).
  This sequence lies in \([a, b]\), and so by \cref{i:9.1.24} there exists a subsequence \((x_{n_j})_{j = 0}^\infty\) which converges to some limit \(L \in [a, b]\), where \(n_0 < n_1 < n_2 < \dots\) is an increasing sequence of natural numbers.
  In particular, we see that \(n_j \geq j\) for all \(j \in \N\) (use induction).

  Since \(f\) is continuous on \([a, b]\), it is continuous at \(L\), and in particular we see that
  \[
    \lim_{j \to \infty} f(x_{n_j}) = f(L).
  \]
  Thus the sequence \((f(x_{n_j}))_{j = 0}^\infty\) is convergent, and hence it is bounded.
  On the other hand, we know from the construction that \(\abs{f(x_{n_j})} \geq n_j \geq j\) for all \(j\), and hence the sequence \((f(x_{n_j}))_{j = 0}^\infty\) is not bounded, a contradiction.
\end{proof}

\begin{rmk}\label{i:9.6.4}
  There are two things about the proof of \cref{i:9.6.3} that are worth noting.
  Firstly, it shows how useful the Heine-Borel theorem (\cref{i:9.1.24}) is.
  Secondly, it is an indirect proof;
  it doesn't say \emph{how} to find the bound for \(f\), but it shows that having \(f\) unbounded leads to a contradiction.
\end{rmk}

\begin{defn}[Maxima and minima]\label{i:9.6.5}
  Let \(X\) be a subset of \(\R\), and let \(f : X \to \R\) be a function, and let \(x_0 \in X\).
  We say that \emph{\(f\) attains its maximum at \(x_0\)} if we have \(f(x_0) \geq f(x)\) for all \(x \in X\)
  (i.e., the value of \(f\) at the point \(x_0\) is larger than or equal to the value of \(f\) at any other point in \(X\)).
  We say that \emph{\(f\) attains its minimum at \(x_0\)} if we have \(f(x_0) \leq f(x)\) for all \(x \in X\).
\end{defn}

\begin{rmk}\label{i:9.6.6}
  If a function attains its maximum somewhere, then it must be bounded from above.
  Similarly if it attains its minimum somewhere, then it must be bounded from below.
  These notions of maxima and minima are \emph{global}.
\end{rmk}

\begin{prop}[Maximum principle]\label{i:9.6.7}
  Let \(a < b\) be real numbers, and let \(f : [a, b] \to \R\) be a function continuous on \([a, b]\).
  Then \(f\) attains its maximum at some point \(x_{\max} \in [a, b]\), and also attains its minimum at some point \(x_{\min} \in [a, b]\).
\end{prop}

\begin{proof}
  From \cref{i:9.6.3} we know that \(f\) is bounded, thus there exists an \(M\) such that \(-M \leq f(x) \leq M\) for each \(x \in [a, b]\).
  Now let \(E\) denote the
  set
  \[
    E \coloneqq \set{f(x) : x \in [a, b]}.
  \]
  (In other words, \(E \coloneqq f([a, b])\).)
  By what we just said, this set is a subset of \([-M, M]\).
  It is also non-empty, since it contains for instance the point \(f(a)\).
  Hence by the least upper bound principle, it has a supremum \(\sup(E)\) which is a real number.

  Write \(m \coloneqq \sup(E)\).
  By definition of supremum, we know that \(y \leq m\) for all \(y \in E\);
  by definition of \(E\), this means that \(f(x) \leq m\) for all \(x \in [a, b]\).
  Thus to show that \(f\) attains its maximum somewhere, it will suffice to find an \(x_{\max} \in [a, b]\) such that \(f(x_{\max}) = m\).

  Let \(n \geq 1\) be any integer.
  Then \(m - \dfrac{1}{n} < m = \sup(E)\).
  As \(\sup(E)\) is the least upper bound for \(E\), \(m - \dfrac{1}{n}\) cannot be an upper bound for \(E\), thus there exists a \(y \in E\) such that \(m - \dfrac{1}{n} < y\).
  By definition of \(E\), this implies that there exists an \(x \in [a, b]\) such that \(m - \dfrac{1}{n} < f(x)\).

  We now choose a sequence \((x_n)_{n = 1}^\infty\) by choosing, for each \(n\), \(x_n\) to be an element of \([a, b]\) such that \(m - \dfrac{1}{n} < f(x_n)\).
  (Again, this requires the axiom of choice;
  however it is possible to prove this principle without the axiom of choice.
  For instance, you will see a better proof of this proposition using the notion of \emph{compactness})
  This is a sequence in \([a, b]\);
  by the Heine-Borel theorem (\cref{i:9.1.24}), we can thus find a subsequence \((x_{n_j})_{j = 1}^\infty\), where \(n_1 < n_2 < \dots\), which converges to some limit \(x_{\max} \in [a, b]\).
  Since \((x_{n_j})_{j = 1}^\infty\) converges to \(x_{\max}\), and \(f\) is continuous at \(x_{\max}\), we have as before that
  \[
    \lim_{j \to \infty} f(x_{n_j}) = f(x_{\max})
  \]
  On the other hand, by construction we know that
  \[
    f(x_{n_j}) > m - \dfrac{1}{n_j} \geq m - \dfrac{1}{j},
  \]
  and so by taking limits of both sides we see that
  \[
    f(x_{\max}) = \lim_{j \to \infty} f(x_{n_j}) \geq \lim_{j \to \infty} m - \dfrac{1}{j} = m.
  \]
  On the other hand, we know that \(f(x) \leq m\) for all \(x \in [a, b]\), so in particular \(f(x_{\max}) \leq m\).
  Combining these two inequalities we see that \(f(x_{\max}) = m\) as desired.

  By the least upper bound principle again, \(E\) has a infimum \(\inf(E)\) which is a real number.
  Write \(m \coloneqq \inf(E)\).
  By definition of infimum, we know that \(y \geq m\) for all \(y \in E\);
  by definition of \(E\), this means that \(f(x) \geq m\) for all \(x \in [a, b]\).
  Thus to show that \(f\) attains its minimum somewhere, it will suffice to find an \(x_{\min} \in [a, b]\) such that \(f(x_{\min}) = m\).

  Let \(n \geq 1\) be any integer.
  Then \(m + \dfrac{1}{n} > m = \inf(E)\).
  As \(\inf(E)\) is the greatest lower bound for \(E\), \(m + \dfrac{1}{n}\) cannot be an lower bound for \(E\), thus there exists a \(y \in E\) such that \(m + \dfrac{1}{n} > y\).
  By definition of \(E\), this implies that there exists an \(x \in [a, b]\) such that \(m + \dfrac{1}{n} > f(x)\).

  We now choose a sequence \((x_n)_{n = 1}^\infty\) by choosing, for each \(n\), \(x_n\) to be an element of \([a, b]\) such that \(m + \dfrac{1}{n} > f(x_n)\).
  (Again, this requires the axiom of choice)
  This is a sequence in \([a, b]\);
  by the Heine-Borel theorem (\cref{i:9.1.24}), we can thus find a subsequence \((x_{n_j})_{j = 1}^\infty\), where \(n_1 < n_2 < \dots\), which converges to some limit \(x_{\min} \in [a, b]\).
  Since \((x_{n_j})_{j = 1}^\infty\) converges to \(x_{\min}\), and \(f\) is continuous at \(x_{\min}\), we have as before that
  \[
    \lim_{j \to \infty} f(x_{n_j}) = f(x_{\min})
  \]
  On the other hand, by construction we know that
  \[
    f(x_{n_j}) < m + \dfrac{1}{n_j} \leq m + \dfrac{1}{j},
  \]
  and so by taking limits of both sides we see that
  \[
    f(x_{\min}) = \lim_{j \to \infty} f(x_{n_j}) \leq \lim_{j \to \infty} m + \dfrac{1}{j} = m.
  \]
  On the other hand, we know that \(f(x) \geq m\) for all \(x \in [a, b]\), so in particular \(f(x_{\min}) \geq m\).
  Combining these two inequalities we see that \(f(x_{\min}) = m\) as desired.
\end{proof}

\begin{rmk}\label{i:9.6.8}
  Strictly speaking, ``maximum principle'' is a misnomer, since the principle also concerns the minimum.
  Perhaps a more precise name would have been ``extremum principle'';
  the word ``extremum'' is used to denote either a maximum or a minimum.
\end{rmk}

\begin{note}
  The maximum principle (\cref{i:9.6.7}) does not prevent a function from attaining its maximum or minimum at more than one point.
\end{note}

\begin{note}
  Let us write \(\sup_{x \in [a, b]} f(x)\) as short-hand for \(\sup\set{f(x) : x \in [a, b]}\), and similarly define \(\inf_{x \in [a, b]} f(x)\).
  The maximum principle (\cref{i:9.6.7}) thus asserts that \(m \coloneqq \sup_{x \in [a, b]} f(x)\) is a real number and is the maximum value of \(f\) on \([a, b]\), i.e., there is at least one point \(x_{\max}\) in \([a, b]\) for which \(f(x_{\max}) = m\), and for every other \(x \in [a, b]\), \(f(x)\) is less than or equal to \(m\).
  Similarly \(\inf_{x \in [a, b]} f(x)\) is the minimum value of \(f\) on \([a, b]\).
\end{note}

\begin{rmk}\label{i:9.6.9}
  You may encounter a rather different ``maximum principle'' in complex analysis or partial differential equations, involving analytic functions and harmonic functions respectively, instead of continuous functions.
  Those maximum principles are not directly related to this one
  (though they are also concerned with whether maxima exist, and where the maxima are located).
\end{rmk}

\exercisesection

\begin{ex}\label{i:ex:9.6.1}
  Give example of
  \begin{enumerate}
    \item a function \(f : (1, 2) \to \R\) which is continuous and bounded, attains its minimum somewhere, but does not attain its maximum anywhere;
    \item a function \(f : [0, \infty) \to \R\) which is continuous, bounded, attains its maximum somewhere, but does not attain its minimum anywhere;
    \item a function \(f : [-1, 1] \to \R\) which is bounded but does not attain its minimum anywhere or its maximum anywhere.
    \item a function \(f : [-1, 1] \to \R\) which has no upper bound and no lower bound.
  \end{enumerate}
  Explain why none of the examples you construct violate the maximum principle.
\end{ex}

\begin{proof}
  Let \(f_a : (1, 2) \to \R\) be a function where \(f_a(x) = \abs{x - 1.5}\).
  Then by \cref{i:9.4.12} and \cref{i:9.4.13} \(f_a\) is continuous.
  Since \(f_a\big((1, 2)\big) = [0, 0.5)\), by \cref{i:9.6.1} \(f_a\) is bounded.
  Since \(f_a\big((1, 2)\big) = [0, 0.5)\), by \cref{i:9.6.5} \(f_a\) attains its minimum at \(x = 1.5\).
  Since \(0.5 \notin f_a\big((1, 2)\big)\), by \cref{i:9.6.5} \(f_a\) does not attain its maximum anywhere.
  Since the domain of \(f_a\) is not closed, \(f_a\) does not violate the maximum principle (\cref{i:9.6.7}).

  Let \(f_b : [0, \infty) \to \R\) be a function where \(f_b(x) = 0.5^x\).
  Then by \cref{i:9.4.10} \(f_b\) is continuous.
  Since \(f_b\big([0, \infty)\big) = (0, 1]\), by \cref{i:9.6.1} \(f_b\) is bounded.
  Since \(f_b\big([0, \infty)\big) = (0, 1]\), by \cref{i:9.6.5} \(f_b\) attains its maximum at \(x = 0\).
  Since \(0 \notin f_b\big([0, \infty)\big)\), by \cref{i:9.6.5} \(f_b\) does not attain its minimum anywhere.
  Since the domain of \(f_b\) is not closed, \(f_b\) does not violate the maximum principle (\cref{i:9.6.7}).

  Let \(f_c : [-1, 1] \to \R\) be a function where
  \[
    f(x) = \begin{dcases}
      0 & \text{if } x = -1;        \\
      x & \text{if } x \in (-1, 1); \\
      0 & \text{if } x = 1.
    \end{dcases}
  \]
  Since \(f_c\big([-1, 1]\big) = (-1, 1)\), by \cref{i:9.6.1} \(f_c\) is bounded.
  Since \(-1 \notin f_c\big([-1, 1]\big)\), by \cref{i:9.6.5} \(f_c\) does not attain its minimum anywhere.
  Since \(1 \notin f_c\big([-1, 1]\big)\), by \cref{i:9.6.5} \(f_c\) does not attain its maximum anywhere.
  Since \(f_c\) is not continuous on its domain, \(f_c\) does not violate the maximum principle (\cref{i:9.6.7}).

  Let \(f_d : [-1, 1] \to \R\) be a function where
  \[
    f(x) = \begin{dcases}
      1 / x & \text{if } x \in [-1, 0) \cap (0, 1]; \\
      0     & \text{if } x = 0.
    \end{dcases}
  \]
  Since \(f_d\big([-1, 1]\big) = (-\infty, -1) \cup \set{0} \cup (1, \infty)\), by \cref{i:9.6.1} \(f_d\) is unbounded.
  Since \(f_d\) is not bounded, \(f_d\) does not violate the maximum principle (\cref{i:9.6.7}).
\end{proof}

\begin{ex}\label{i:ex:9.6.2}
  Let \(X \subseteq \R\).
  If \(f, g: X \to \R\) are bounded functions, show that \(f + g\), \(f - g\), and \(f \cdot g\) are also bounded functions.
  If we furthermore assume that \(g(x) \neq 0\) for all \(x \in X\), is it true that \(f / g\) is bounded?
  Prove this or give a counterexample.
\end{ex}

\begin{proof}
  Suppose that \(f\) is bounded by \(M \in \R^+\) and \(g\) is bounded by \(N \in \R^+\).
  Then we have
  \begin{align*}
             & \forall x \in X, \big(\abs{f(x)} \leq M\big) \land \big(\abs{g(x)} \leq N\big) &  & \by{i:9.6.1} \\
    \implies & \big(-M \leq f(x) \leq M\big) \land \big(-N \leq g(x) \leq N\big)                                \\
    \implies & \big(-M \leq f(x) \leq M\big) \land \big(N \geq -g(x) \geq -N\big)                               \\
    \implies & \begin{dcases}
                 -(M + N) \leq f(x) + g(x) \leq M + N \\
                 -(M + N) \leq f(x) - g(x) \leq M + N \\
                 -MN \leq f(x) g(x) \leq MN
               \end{dcases}                                                             \\
    \implies & \begin{dcases}
                 \abs{f(x) + g(x)} \leq M + N \\
                 \abs{f(x) - g(x)} \leq M + N \\
                 \abs{f(x) g(x)} \leq MN
               \end{dcases}
  \end{align*}
  Thus by \cref{i:9.6.1} \(f + g\), \(f - g\), \(f \cdot g\) are bounded.

  Now we show an counterexample when \(g(x) \neq 0\) for all \(x \in X\).
  Let \(X = (0, \infty)\), let \(f(x) = 1\) and let \(g(x) = 1 / x\).
  Then \(g(x) \neq 0\) for all \(x \in X\) and is bounded by \(1\), but \((f / g)(X) = (0, \infty)\) is unbounded.
\end{proof}

\section{The intermediate value theorem}\label{sec:9.7}

\begin{thm}[Intermediate value theorem]\label{9.7.1}
  Let \(a < b\), and let \(f : [a, b] \to \R\) be a continuous function on \([a, b]\).
  Let \(y\) be a real number between \(f(a)\) and \(f(b)\), i.e., either \(f(a) \leq y \leq f(b)\) or \(f(a) \geq y \geq f(b)\).
  Then there exists \(c \in [a, b]\) such that \(f(c) = y\).
\end{thm}

\begin{proof}
  We have two cases: \(f(a) \leq y \leq f(b)\) or \(f(a) \geq y \geq f(b)\).
  We will assume the former, that \(f(a) \leq y \leq f(b)\);
  the latter is proven similarly.

  If \(y = f(a)\) or \(y = f(b)\) then the claim is easy, as one can simply set \(c = a\) or \(c = b\), so we will assume that \(f(a) < y < f(b)\).
  Let \(E\) denote the set
  \[
    E \coloneqq \set{x \in [a, b] : f(x) < y}.
  \]
  Clearly \(E\) is a subset of \([a, b]\), and is hence bounded.
  Also, since \(f(a) < y\), we see that \(a\) is an element of \(E\), so \(E\) is non-empty.
  By the least upper bound principle, the supremum
  \[
    c \coloneqq \sup(E)
  \]
  is thus finite.
  Since \(E\) is bounded by \(b\), we know that \(c \leq b\);
  since \(E\) contains \(a\), we know that \(c \geq a\).
  Thus we have \(c \in [a, b]\).
  To complete the proof we now show that \(f(c) = y\).
  The idea is to work from the left of \(c\) to show that \(f(c) \leq y\), and to work from the right of \(c\) to show that \(f(c) \geq y\).

  Let \(n \geq 1\) be an integer.
  The number \(c - \dfrac{1}{n}\) is less than \(c = \sup(E)\) and hence cannot be an upper bound for \(E\).
  Thus there exists a point, call it \(x_n\), which lies in \(E\) and which is greater than \(c - \dfrac{1}{n}\).
  Also \(x_n \leq c\) since \(c\) is an upper bound for \(E\).
  Thus
  \[
    c - \dfrac{1}{n} \leq x_n \leq c.
  \]
  By the squeeze test (\cref{6.4.14}) we thus have \(\lim_{n \to \infty} x_n = c\).
  Since \(f\) is continuous at \(c\), this implies that \(\lim_{n \to \infty} f(x_n) = f(c)\).
  But since \(x_n\) lies in \(E\) for every \(n\), we have \(f(x_n) < y\) for every \(n\).
  By the comparison principle (\cref{6.4.13}) we thus have \(f(c) \leq y\).
  Since \(f(b) > f(c)\), we conclude \(c \neq b\).

  Since \(c \neq b\) and \(c \in [a, b]\), we must have \(c < b\).
  In particular there is an \(N > 0\) such that \(c + \dfrac{1}{n} < b\) for all \(n > N\)
  (since \(c + \dfrac{1}{n}\) converges to \(c\) as \(n \to \infty\)).
  Since \(c\) is the supremum of \(E\) and \(c + \dfrac{1}{n} > c\), we thus have \(c + \dfrac{1}{n} \notin E\) for all \(n > N\).
  Since \(c + \dfrac{1}{n} \in [a, b]\), we thus have \(f(c + \dfrac{1}{n}) \geq y\) for all \(n \geq N\).
  But \(c + \dfrac{1}{n}\) converges to \(c\), and \(f\) is continuous at \(c\), thus \(f(c) \geq y\).
  But we already knew that \(f(c) \leq y\), thus \(f(c) = y\), as desired.
\end{proof}

\begin{note}
  The intermediate value theorem says that if \(f\) takes the values \(f(a)\) and \(f(b)\), then it must also take all the values in between.
  If \(f\) is not assumed to be continuous, then the intermediate value theorem no longer applies.
  If a function is discontinuous, it can ``jump'' past intermediate values;
  however continuous functions cannot do so.
\end{note}

\begin{rmk}\label{9.7.2}
  A continuous function may take an intermediate value multiple times.
\end{rmk}

\begin{rmk}\label{9.7.3}
  The intermediate value theorem gives another way to show that one can take \(n^{\text{th}}\) roots of a number.
  For instance, to construct the square root of \(2\), consider the function \(f : [0, 2] \to \R\) defined by \(f(x) = x^2\).
  This function is continuous, with \(f(0) = 0\) and \(f(2) = 4\).
  Thus there exists a \(c \in [0, 2]\) such that \(f(c) = 2\), i.e., \(c^2 = 2\).
  (This argument does not show that there is just one square root of \(2\), but it does prove that there is \emph{at least} one square root of \(2\).)
\end{rmk}

\begin{cor}[Images of continuous functions]\label{9.7.4}
  Let \(a < b\), and let \(f : [a, b] \to \R\) be a continuous function on \([a, b]\).
  Let \(M \coloneqq \sup_{x \in [a, b]} f(x)\) be the maximum value of \(f\), and let \(m \coloneqq \inf_{x \in [a, b]} f(x)\) be the minimum value.
  Let \(y\) be a real number between \(m\) and \(M\) (i.e., \(m \leq y \leq M\)).
  Then there exists a \(c \in [a, b]\) such that \(f(c) = y\).
  Furthermore, we have \(f([a, b]) = [m, M]\).
\end{cor}

\begin{proof}
  We first show that \(\exists c \in [a, b]\) such that \(f(c) = y\).
  By maximum principle (\cref{9.6.7}) we know that \(\exists x_M, x_m \in [a, b]\) such that \(f(x_M) = M\) and \(f(x_m) = m\).
  We have either \(x_m \leq x_M\) or \(x_m \geq x_M\).
  Without the loss of generality suppose that \(x_m \leq x_M\).
  Then we have \([x_m, x_M] \subseteq [a, b]\) and by \cref{ex:9.4.6} we know that \(f\) is continuous on \([x_m, x_M]\).
  Since \(m \leq y \leq M\), by \cref{9.7.1} \(\exists c \in [x_m, x_M]\) such that \(f(c) = y\).
  Since \([x_m, x_M] \subseteq [a, b]\), we have \(c \in [a, b]\).

  Now we show that \(f([a, b]) = [m, M]\).
  Since
  \begin{align*}
             & \big(M = \sup_{x \in [a, b]} f(x)\big) \land \big(m = \inf_{x \in [a, b]} f(x)\big)                  \\
    \implies & \forall y \in f([a, b], m \leq y \leq M                                             &  & \by{9.6.5}  \\
    \implies & \forall y \in f([a, b], y \in [m, M]                                                &  & \by{9.1.1}  \\
    \implies & f([a, b]) \subseteq [m, M]                                                          &  & \by{3.1.15}
  \end{align*}
  and
  \begin{align*}
             & \forall y \in [m, M], \exists c \in [a, b] : f(c) = y &  & \text{(by proof above)} \\
    \implies & \forall y \in [m, M], y \in f([a, b])                                              \\
    \implies & [m, M] \subseteq f([a, b]),                           &  & \by{3.1.15}
  \end{align*}
  by \cref{3.1.18} we know that \(f([a, b]) = [m ,M]\).
\end{proof}

\exercisesection

\begin{ex}\label{ex:9.7.1}
  Prove \cref{9.7.4}.
\end{ex}

\begin{proof}
  See \cref{9.7.4}.
\end{proof}

\begin{ex}\label{ex:9.7.2}
  Let \(f : [0, 1] \to [0, 1]\) be a continuous function.
  Show that there exists a real number \(x\) in \([0, 1]\) such that \(f(x) = x\).
  This point \(x\) is known as a \emph{fixed point} of \(f\), and this result is a basic example of a \emph{fixed point theorem}, which play an important role in certain types of analysis.
\end{ex}

\begin{proof}
  Let \(F : [0, 1] \to \R\) be a function where \(F(x) = f(x) - x\).
  Since \(x \mapsto x\) is continuous on \([0, 1]\), by \cref{9.4.9} \(F\) is continuous on \([0, 1]\).
  Since \(f([0, 1]) \subseteq [0, 1]\), we know that \(0 \leq f(0)\) and \(f(1) \leq 1\).
  Thus
  \begin{align*}
             & \big(0 \leq f(0)\big) \land \big(f(1) \leq 1\big)                                       \\
    \implies & \big(0 \leq F(0) = f(0) - 0\big) \land \big(F(1) = f(1) - 1 \leq 0\big)                 \\
    \implies & F(1) \leq 0 \leq F(0)                                                                   \\
    \implies & \exists x \in [0, 1] : F(x) = 0                                         &  & \by{9.7.1} \\
    \implies & \exists x \in [0, 1] : f(x) - x = 0                                                     \\
    \implies & \exists x \in [0, 1] : f(x) = x.
  \end{align*}
\end{proof}

\section{Monotonic functions}\label{i:sec:9.8}

\begin{defn}[Monotonic functions]\label{i:9.8.1}
  Let \(X\) be a subset of \(\R\), and let \(f : X \to \R\) be a function.
  We say that \(f\) is \emph{monotone increasing} iff \(f(y) \geq f(x)\) whenever \(x, y \in X\) and \(y > x\).
  We say that \(f\) is \emph{strictly monotone increasing} iff \(f(y) > f(x)\) whenever \(x, y \in X\) and \(y > x\).
  Similarly, we say \(f\) is \emph{monotone decreasing} iff \(f(y) \leq f(x)\) whenever \(x, y \in X\) and \(y > x\), and \emph{strictly monotone decreasing} iff \(f(y) < f(x)\) whenever \(x, y \in X\) and \(y > x\).
  We say that \(f\) is \emph{monotone} if it is monotone increasing or monotone decreasing, and \emph{strictly monotone} if it is strictly monotone increasing or strictly monotone decreasing.
\end{defn}

\begin{note}
  If a function is strictly monotone on a domain \(X\), it is automatically monotone as well on the same domain \(X\).
  Constant functions, when restricted to an arbitrary domain \(X \subseteq \R\), are both monotone increasing and monotone decreasing, but is not strictly monotone
  (unless \(X\) consists of at most one point).
\end{note}

\begin{note}
  Continuous functions are not necessarily monotone, and monotone functions are not necessarily continuous.
\end{note}

\setcounter{thm}{2}
\begin{prop}\label{i:9.8.3}
  Let \(a < b\) be real numbers, and let \(f : [a, b] \to \R\) be a function which is both continuous and strictly monotone increasing.
  Then \(f\) is a bijection from \([a, b]\) to \([f(a), f(b)]\), and the inverse \(f^{-1} : [f(a), f(b)] \to [a, b]\) is also continuous and strictly monotone increasing.
\end{prop}

\begin{proof}
  We first show that \(f\) is bijective from \([a, b]\) to \([f(a), f(b)]\).
  Since \(a < b\) and \(f\) is strictly monotone increasing, by \cref{i:9.8.1} we know that \(f(a) < f(b)\).
  In particular, \(\forall c \in (a, b)\), we have \(a < c < b\) and \(f(a) < f(c) < f(b)\).
  By \cref{i:9.6.5} this means \(f\) attains its minimum at \(a\) and attains its maximum at \(b\).
  By \cref{i:9.7.4} we know that \(f([a, b]) = [f(a), f(b)]\), thus \(f\) is surjective from \([a, b]\) to \([f(a), f(b)]\).
  Since \(f\) is strictly monotone increasing, by \cref{i:9.8.1} \(x \neq y \implies f(x) \neq f(y)\), thus \(f\) is injective from \([a, b]\) to \([f(a), f(b)]\).
  Since \(f\) is both injective and surjective from \([a, b]\) to \([f(a), f(b)]\), we know that \(f\) is bijective from \([a, b]\) to \([f(a), f(b)]\).

  Next we show that \(f^{-1}\) is continuous.
  Let \(y_0 \in [f(a), f(b)]\).
  By \cref{i:9.1.12} we know that \([f(a), f(b)]\) is closed, so \(y_0\) is an adherent point of \([f(a), f(b)]\).
  Since \(f\) is bijective from \([a, b]\) to \([f(a), f(b)]\), we know that \(\exists!\ x_0 \in [a, b]\) such that \(f(x_0) = y_0\) and thus \(f^{-1}(y_0) = x_0\).
  Again by \cref{i:9.1.12} we know that \([a, b]\) is closed, so \(x_0\) is an adherent point of \([a, b]\).
  To show that \(f^{-1}\) is continuous at \(y_0\), by \cref{i:9.4.1} we need to show that
  \[
    \lim_{y \to y_0 ; y \in [f(a), f(b)]} f^{-1}(y) = f^{-1}(y_0) = x_0.
  \]
  By \cref{i:9.3.6}, it suffice to show that
  \[
    \forall \varepsilon \in \R^+, \exists \delta \in \R^+ : \big(\forall y \in [f(a), f(b)], \abs{y - y_0} < \delta \implies \abs{f^{-1}(y) - x_0} \leq \varepsilon\big).
  \]
  Now fix \(\varepsilon\).
  Let \(x_L = \max(x_0 - \varepsilon, a)\) and \(x_H = \min(x_0 + \varepsilon, b)\).
  Then \(x_L, x_H \in [a, b]\) and \(f(x_L), f(x_H)\) are well-defined.
  Since \(x_L \leq x_0 \leq x_H\) and \(f\) is strictly monotone increasing, we have \(f(x_L) < y_0 < f(x_H)\) and thus \(f(x_L) - y_0 < 0 < f(x_H) - y_0\).
  Let \(\delta = \min\big(y_0 - f(x_L), f(x_H) - y_0\big)\).
  Then we have
  \begin{align*}
             & \forall y \in [f(a), f(b)], \abs{y - y_0} < \delta                                                                 \\
    \implies & -\delta < y - y_0 < \delta                                                                                         \\
    \implies & f(x_L) - y_0 \leq -\delta < y - y_0 < \delta \leq f(x_H) - y_0                                                     \\
    \implies & f(x_L) \leq y \leq f(x_H)                                                                                          \\
    \implies & \exists x \in [x_L, x_H] : \big(f(x) = y\big) \land (x_L \leq x \leq x_H)                        &  & \by{i:9.7.1} \\
    \implies & \exists x \in [x_L, x_H] : \big(f(x) = y\big)                                                                      \\
             & \land (x_0 - \varepsilon \leq x_L \leq x \leq x_H \leq x_0 + \varepsilon)                                          \\
    \implies & \exists x \in [x_L, x_H] : \big(f(x) = y\big) \land (-\varepsilon \leq x - x_0 \leq \varepsilon)                   \\
    \implies & \exists x \in [x_L, x_H] : \big(f(x) = y\big) \land (\abs{x - x_0} \leq \varepsilon)                               \\
    \implies & \exists x \in [x_L, x_H] : \big(f(x) = y\big) \land (\abs{f^{-1}(y) - x_0} \leq \varepsilon).
  \end{align*}
  Since \(\varepsilon\) is arbitrary, \(f^{-1}\) is continuous at \(y_0\).
  Since \(y_0\) is arbitrary, \(f^{-1}\) is continuous on \([f(a), f(b)]\).

  Next we show that \(f^{-1}\) is strictly monotone increasing.
  Let \(y_1, y_2 \in [f(a), f(b)]\) and \(y_1 < y_2\).
  We want to show that \(f^{-1}(y_1) < f^{-1}(y_2)\).
  Suppose for sake of contradiction that \(f^{-1}(y_1) \geq f^{-1}(y_2)\).
  Since \(f\) is strictly monotone increasing, we know that
  \begin{align*}
             & f^{-1}(y_1) \geq f^{-1}(y_2)                                         \\
    \implies & f\big(f^{-1}(y_1)\big) \geq f\big(f^{-1}(y_2)\big) &  & \by{i:9.8.1} \\
    \implies & y_1 \geq y_2.
  \end{align*}
  But this contradict to \(y_1 < y_2\), thus \(f^{-1}(y_1) < f^{-1}(y_2)\).

  Next we show that if the continuity assumption is dropped, then the proposition is false.
  Let \(f : [0, 1] \to \R\) be a function
  \[
    f(x) = \begin{dcases}
      x     & \text{if } x \in [0, 0.5)  \\
      x + 1 & \text{if } x \in [0.5, 1].
    \end{dcases}
  \]
  Then \(f\) is not continuous and \(f([0, 1]) = [0, 0.5) \cup [1.5, 2] \neq [0, 2]\).

  Finally we show that if strict monotonicity is replaced just by monotonicity, then the proposition is false.
  Let \(f : [0, 1] \to \R\) be a function where \(f(x) = 1\).
  Then \(f\) is monotone but not bijective.
\end{proof}

\begin{ac}\label{i:ac:9.8.1}
  Let \(a < b\) be real numbers, and let \(f : [a, b] \to \R\) be a function which is both continuous and strictly monotone decreasing.
  Then \(f\) is a bijection from \([a, b]\) to \([f(b), f(a)]\), and the inverse \(f^{-1} : [f(b), f(a)] \to [a, b]\) is also continuous and strictly monotone decreasing.
\end{ac}

\begin{proof}
  Let \(h : \R \to \R\) be the function \(h(x) = -x\).
  We define a function \(g : [a, b] \to \R\) by setting \(g = -f\).
  Then we have \(g = h \circ f\) and \(f = h \circ g\).
  By \cref{i:9.4.9} we know that \(h\) is continuous on \(\R\), thus by \cref{i:9.4.13} we know that \(g\) is continuous.
  Since \(f\) is strictly monotone decreasing, we have
  \begin{align*}
             & \forall x_1, x_2 \in [a, b], x_1 < x_2                   \\
    \implies & f(x_1) > f(x_2)                        &  & \by{i:9.8.1} \\
    \implies & -f(x_1) < -f(x_2)                                        \\
    \implies & g(x_1) < g(x_2)
  \end{align*}
  and \(g\) is strictly monotone increasing.
  Since \(g\) is monotone increasing and continuous, by \cref{i:9.8.3} we know that \(g : [a, b] \to [g(a), g(b)]\) is bijective and \(g^{-1} : [g(a), g(b)] \to [a, b]\) is continuous and strictly monotone increasing.
  Since \(h\) is bijective, by \cref{i:ex:3.3.7} we know that \(f = h \circ g\) is bijective from \([a, b]\) to \([-g(b), -g(a)] = [f(b), f(a)]\).
  Again by \cref{i:ex:3.3.7} we know that
  \[
    f^{-1} = g^{-1} \circ h^{-1}|_{[f(a), f(b)]}.
  \]
  Since \(h^{-1} = h\) is continuous, by \cref{i:9.4.13} we know that \(f^{-1}\) is continuous.
  Since \(g^{-1}\) is strictly monotone increasing, we have
  \begin{align*}
             & \forall y_1, y_2 \in [f(b), f(a)], y_1 < y_2                                                              \\
    \implies & \exists x_1, x_2 \in [a, b] :                                                                             \\
             & \big(x_1 = f^{-1}(y_1)\big) \land \big(x_2 = f^{-1}(y_2)\big) \land \big(y_1 < y_2\big)                   \\
    \implies & \exists x_1, x_2 \in [a, b] :                                                                             \\
             & \big(g(x_1) = -y_1\big) \land \big(g(x_2) = -y_2\big) \land \big(y_1 < y_2\big)                           \\
    \implies & \exists x_1, x_2 \in [a, b] :                                                                             \\
             & \big(g(x_1) = -y_1\big) \land \big(g(x_2) = -y_2\big) \land \big(g(x_1) > g(x_2)\big)                     \\
    \implies & \exists x_1, x_2 \in [a, b] :                                                                             \\
             & \big(g(x_1) = -y_1\big) \land \big(g(x_2) = -y_2\big)                                                     \\
             & \land \Big(x_1 = g^{-1}\big(g(x_1)\big) > g^{-1}\big(g(x_2)\big) = x_2\Big)             &  & \by{i:9.8.1} \\
    \implies & f^{-1}(y_1) > f^{-1}(y_2)
  \end{align*}
  and thus by \cref{i:9.8.1} \(f^{-1}\) is strictly monotone decreasing.
\end{proof}

\begin{eg}\label{i:9.8.4}
  Let \(n\) be a positive integer and \(R > 0\).
  Since the function \(f(x) \coloneqq x^n\) is strictly increasing on the interval \([0, R]\), we see from \cref{i:9.8.3} that this function is a bijection from \([0, R]\) to \([0, R^n]\), and hence there is an inverse from \([0, R^n]\) to \([0, R]\).
  This can be used to give an alternate means to construct the \(n^{\opTh}\) root \(x^{1 / n}\) of a number \(x \in [0, R]\) than what was done in \cref{i:5.6.5}.
\end{eg}

\exercisesection

\begin{ex}\label{i:ex:9.8.1}
  Explain why the maximum principle remains true if the hypothesis that \(f\) is continuous is replaced with \(f\) being monotone, or with \(f\) being strictly monotone.
\end{ex}

\begin{proof}
  Let \(a < b\) be real numbers, and let \(f : [a, b] \to \R\) be a function.
  Suppose that \(f\) is monotone.
  Then we have
  \begin{align*}
             & \forall c \in [a, b]                                                                         \\
    \implies & a \leq c \leq b                                                            &  & \by{i:9.1.1} \\
    \implies & \big(f(a) \leq f(c) \leq f(b)\big) \lor \big(f(a) \geq f(c) \geq f(b)\big) &  & \by{i:9.8.1} \\
  \end{align*}
  Thus \(f\) attains its maximum at \(f(b)\) and attains its minimum at \(f(a)\) when \(f\) is monotone increasing;
  \(f\) attains its maximum at \(f(a)\) and attains its minimum at \(f(b)\) when \(f\) is monotone decreasing.
  The same argument holds when \(f\) is strictly monotone.
\end{proof}

\begin{ex}\label{i:ex:9.8.2}
  Give an example to show that the intermediate value theorem becomes false if the hypothesis that \(f\) is continuous is replaced with \(f\) being monotone, or with \(f\) being strictly monotone.
\end{ex}

\begin{proof}
  Let \(f : [0, 1] \to \R\) be a function where
  \[
    f(x) = \begin{dcases}
      x     & \text{if } x \in [0, 0.5); \\
      x + 1 & \text{if } x \in [0.5, 1].
    \end{dcases}
  \]
  Then \(f\) is strictly monotone and thus monotone.
  Since \(\forall y \in [0.5, 1.5)\), \(\nexists x \in \N\) such that \(f(x) = y\), the intermediate value theorem (\cref{i:9.7.1}) does not hold.
\end{proof}

\begin{ex}\label{i:ex:9.8.3}
  Let \(a < b\) be real numbers, and let \(f : [a, b] \to \R\) be a function which is both continuous and one-to-one.
  Show that \(f\) is strictly monotone.
\end{ex}

\begin{proof}
  Since \(a < b\) and \(f\) is injective, we cannot have \(f(a) = f(b)\).
  So we have two cases:
  \begin{itemize}
    \item \(f(a) < f(b)\).
          Suppose for sake of contradiction that \(f\) is not strictly monotone.
          Then \(\exists c \in (a, b)\) such that
          \[
            \big(f(c) \leq f(a) < f(b)\big) \lor \big(f(a) < f(b) \leq f(c)\big).
          \]
          Since \(f\) is injective, we have \(\big(f(c) \neq f(a)\big) \land \big(f(c) \neq f(b)\big)\).
          But then we have
          \begin{align*}
                     & \begin{dcases}
                         f(c) < f(a) < f(b) \\
                         f(a) < f(b) < f(c)
                       \end{dcases}                                                        \\
            \implies & \begin{dcases}
                         \exists x_1 \in (c, b) : f(x_1) = f(a) \\
                         \exists x_2 \in (a, c) : f(x_2) = f(b) \\
                       \end{dcases} &  & \by{i:9.7.1}                                    \\
            \implies & \begin{dcases}
                         x_1 = a \\
                         x_2 = b
                       \end{dcases},                            &  & \text{(\(f\) is injective)}
          \end{align*}
          a contradiction.
          Thus \(f\) is strictly monotone.
          Since \(f(a) < f(b)\), by \cref{i:9.8.1} \(f\) is strictly monotone increasing.
    \item \(f(a) > f(b)\).
          Suppose for sake of contradiction that \(f\) is not strictly monotone.
          Then \(\exists c \in (a, b)\) such that
          \[
            \big(f(c) \leq f(b) < f(a)\big) \lor \big(f(b) < f(a) \leq f(c)\big).
          \]
          Since \(f\) is injective, we have \(\big(f(c) \neq f(a)\big) \land \big(f(c) \neq f(b)\big)\).
          But then we have
          \begin{align*}
                     & \begin{dcases}
                         f(c) < f(b) < f(a) \\
                         f(b) < f(a) < f(c)
                       \end{dcases}                                                        \\
            \implies & \begin{dcases}
                         \exists x_1 \in (a, c) : f(x_1) = f(b) \\
                         \exists x_2 \in (c, b) : f(x_2) = f(a) \\
                       \end{dcases} &  & \by{i:9.7.1}                                    \\
            \implies & \begin{dcases}
                         x_1 = b \\
                         x_2 = a
                       \end{dcases},                            &  & \text{(\(f\) is injective)}
          \end{align*}
          a contradiction.
          Thus \(f\) is strictly monotone.
          Since \(f(a) > f(b)\), by \cref{i:9.8.1} \(f\) is strictly monotone decreasing.
  \end{itemize}
  From all cases above we conclude that \(f\) is strictly monotone.
\end{proof}

\begin{ex}\label{i:ex:9.8.4}
  Prove \cref{i:9.8.3}.
  Is the proposition still true if the continuity assumption is dropped, or if strict monotonicity is replaced just by monotonicity?
  How should one modify the proposition to deal with strictly monotone decreasing functions instead of strictly monotone increasing functions?
\end{ex}

\begin{proof}
  See \cref{i:9.8.3} and \cref{i:ac:9.8.1}.
\end{proof}

\begin{ex}\label{i:ex:9.8.5}
  In this exercise we give an example of a function which has a discontinuity at every rational point, but is continuous at every irrational.
  Since the rationals are countable, we can write them as \(\Q = \set{q(0), q(1), q(2), \dots}\), where \(q : \N \to \Q\) is a bijection from \(\N\) to \(\Q\).
  Now define a function \(g : \Q \to \R\) by setting \(g(q(n)) \coloneqq 2^{-n}\) for each natural number \(n\);
  thus \(g\) maps \(q(0)\) to \(1\), \(q(1)\) to \(2^{-1}\), etc.
  Since \(\sum_{n = 0}^\infty 2^{-n}\) is absolutely convergent, we see that \(\sum_{r \in \Q} g(r)\) is also absolutely convergent.
  Now define the function \(f : \R \to \R\) by
  \[
    f(x) \coloneqq \sum_{r \in \Q : r < x} g(r).
  \]
  Since \(\sum_{r \in \Q} g(r)\) is absolutely convergent, we know that \(f(x)\) is well-defined for every real number \(x\).
  \begin{enumerate}
    \item Show that \(f\) is strictly monotone increasing.
    \item Show that for every rational number \(r\), \(f\) is discontinuous at \(r\).
    \item Show that for every irrational number \(x\), \(f\) is continuous at \(x\).
  \end{enumerate}
\end{ex}

\begin{proof}{(a)}
  Let \(a, b \in \R\) and \(a < b\).
  By \cref{i:5.4.14}, \(\exists c \in \Q\) such that \(a < c < b\).
  Then we have
  \begin{align*}
    f(b) & = \sum_{r \in \Q : r < b} g(r)                                                                                                  \\
         & = \sum_{r \in \Q : r < a} g(r) + \sum_{r \in \Q : a \leq r < c} g(r) + \sum_{r \in \Q : c \leq r < b} g(r) &  & \by{i:8.2.6}[c] \\
         & = f(a) + \sum_{r \in \Q : a \leq r < c} g(r) + g(c) + \sum_{r \in \Q : c < r < b} g(r)                     &  & \by{i:8.2.6}[c] \\
         & > f(a).
  \end{align*}
  and by \cref{i:9.8.1} \(f\) is strictly monotone increasing.
\end{proof}

\begin{proof}{(b)}
  Let \(\gamma \in \Q\).
  Since \(q\) is bijective, \(\exists!\ n \in \N\) such that \(q(n) = \gamma\).
  Then \(\forall x \in (\gamma, \infty)\), we have
  \begin{align*}
    f(x) & = \sum_{r \in \Q : r < x} g(r)                                                                               \\
         & = \sum_{r \in \Q : r < \gamma} g(r) + g(\gamma) + \sum_{r \in \Q : \gamma < r < x} g(r) &  & \by{i:8.2.6}[c] \\
         & = f(\gamma) + g(\gamma) + \sum_{r \in \Q : \gamma < r < x} g(r)                                              \\
         & > f(\gamma) + g(\gamma)                                                                                      \\
         & = f(\gamma) + 2^{-n}.
  \end{align*}
  But this means
  \begin{align*}
             & \forall x \in (\gamma, \infty), f(x) - f(\gamma) > 2^{-n}                            \\
    \implies & \forall x \in (\gamma, \infty), \abs{f(x) - f(\gamma)} > 2^{-n}                      \\
    \implies & f(\gamma^+) \text{ does not exist}                              &  & \by{i:9.5.1}    \\
    \implies & f \text{ is not continuous at } \gamma.                         &  & \by{i:ac:9.5.1}
  \end{align*}
  Since \(\gamma\) is arbitrary, we conclude that \(f\) is discontinuous at each \(\gamma \in \Q\).
\end{proof}

\begin{proof}{(c)}
  Let \(n \in \N\), let \(E_n\) be a set where
  \[
    E_n = \set{r \in \Q : g(r) \geq 2^{-n}},
  \]
  and let \(f_n : \R \to \R\) be a function
  \[
    f_n(x) = \sum_{r \in E_n : r < x} g(r) = \sum_{r \in \Q : r < x, g(r) \geq 2^{-n}} g(r).
  \]
  Since \(q\) is bijective, there are at most \(n + 1\) rationals satisfying \((r \in \Q) \land \big(g(r) \geq 2^{-n}\big)\), thus \(E_n\) is finite and non-empty.
  Since \(\R \setminus \Q \subseteq \R\), by \cref{i:9.1.11} \(\forall x_0 \in \R \setminus \Q\), \(x_0\) is an adherent point of \(\R \setminus \Q\).
  Let \(\varepsilon \in \R^+\) and let \(\delta = \min\set{\abs{r - x_0} : r \in E_n}\).
  Since \(x_0 \notin \Q\), we have \(\delta > 0\).
  Then we have
  \begin{align*}
             & \forall x \in \R \setminus \Q, \abs{x - x_0} < \delta               \\
    \implies & \forall r \in E_n, \abs{x - x_0} < \delta \leq \abs{r - x_0}        \\
    \implies & \forall r \in E_n, \abs{x - x_0} < \abs{r - x_0}                    \\
    \implies & \forall r \in E_n, -\abs{r - x_0} < x - x_0 < \abs{r - x_0}         \\
    \implies & \forall r \in E_n, \begin{dcases}
                                    r - x_0 < x - x_0 < x_0 - r & \text{if } r < x_0 \\
                                    x_0 - r < x - x_0 < r - x_0 & \text{if } r > x_0
                                  \end{dcases} \\
    \implies & \forall r \in E_n, \begin{dcases}
                                    r < x & \text{if } r < x_0 \\
                                    r > x & \text{if } r > x_0
                                  \end{dcases}                       \\
    \implies & \set{r \in E_n : r < x} = \set{r \in E_n : r < x_0}                 \\
    \implies & f_n(x) = f_n(x_0)                                                   \\
    \implies & 0 = \abs{f_n(x) - f_n(x_0)} < \varepsilon.
  \end{align*}
  Since \(\varepsilon\) is arbitrary, by \cref{i:9.3.6} we have \(\lim_{x \to x_0 ; x \in \R \setminus \Q} f_n(x) = f_n(x_0)\), and by \cref{i:9.4.1} \(f_n\) is continuous at \(x_0\).

  Now we show that \(f\) is continuous at \(x_0\).
  We have
  \begin{align*}
     & \forall x \in \R \setminus \Q, \abs{f(x) - f_n(x)}                                                         \\
     & = \abs{\sum_{r \in \Q : r < x} g(r) - \sum_{r \in E_n : r < x} g(r)}                                       \\
     & = \abs{\sum_{r \in \Q : r < x} g(r) - \sum_{r \in \Q : r < x, g(r) \geq 2^{-n}} g(r)}                      \\
     & = \abs{\sum_{r \in \Q : r < x, g(r) < 2^{-n}} g(r)}                                   &  & \by{i:8.2.6}[c] \\
     & = \sum_{r \in \Q : r < x, g(r) < 2^{-n}} g(r)                                                              \\
     & = \sum_{r \in \Q : r < x, g(r) \leq 2^{-(n + 1)}} g(r)                                                     \\
     & \leq \sum_{r \in \Q : g(r) \leq 2^{-(n + 1)}} g(r)                                                         \\
     & \leq \sum_{k = n + 1}^\infty 2^{-k}                                                                        \\
     & = 2^{-(n + 1)} \bigg(\sum_{k = 0}^\infty 2^{-k}\bigg)                                                      \\
     & = 2^{-n}.                                                                             &  & \by{i:7.3.3}
  \end{align*}
  By \cref{i:5.4.14}, \(\forall \varepsilon \in \R^+\), \(\exists n \in \N\) such that \(2^{-n} < \varepsilon / 2\).
  From the proof above we also have
  \[
    \forall x \in \R \setminus \Q, \abs{x - x_0} < \delta \implies f_n(x) = f_n(x_0) \\
  \]
  Combine the results above we have
  \begin{align*}
     & \abs{f(x) - f(x_0)}                              \\
     & = \abs{f(x) - f_n(x) + f_n(x) - f(x_0)}          \\
     & \leq \abs{f(x) - f_n(x)} + \abs{f_n(x) - f(x_0)} \\
     & = \abs{f(x) - f_n(x)} + \abs{f_n(x_0) - f(x_0)}  \\
     & \leq 2^{-n} + 2^{-n}                             \\
     & < \varepsilon / 2 + \varepsilon / 2              \\
     & = \varepsilon.
  \end{align*}
  Since \(\varepsilon\) is arbitrary, by \cref{i:9.3.6} we have \(\lim_{x \to x_0 ; x \in \R \setminus \Q} f(x) = f(x_0)\), and by \cref{i:9.4.1} \(f\) is continuous at \(x_0\).
\end{proof}

\section{Uniform continuity}\label{sec:9.9}

\setcounter{thm}{1}
\begin{defn}[Uniform continuity]\label{9.9.2}
  Let \(X\) be a subset of \(\R\), and let \(f : X \to \R\) be a function.
  We say that \(f\) is \emph{uniformly continuous} on \(X\) if, for every \(\varepsilon > 0\), there exists a \(\delta > 0\) such that \(f(x)\) and \(f(x_0)\) are \(\varepsilon\)-close whenever \(x, x_0 \in X\) are two points in \(X\) which are \(\delta\)-close.
\end{defn}

\begin{rmk}\label{9.9.3}
  This definition should be compared with the notion of continuity.
  From \cref{9.4.7}(c), we know that a function \(f\) is \emph{continuous} if for every \(\varepsilon > 0\), and every \(x_0 \in X\), there is a \(\delta > 0\) such that \(f(x)\) and \(f(x_0)\) are \(\varepsilon\)-close whenever \(x \in X\) is \(\delta\)-close to \(x_0\).
  The difference between uniform continuity and continuity is that in uniform continuity one can take a single \(\delta\) which works for all \(x_0 \in X\);
  for ordinary continuity, each \(x_0 \in X\) might use a different \(\delta\).
  Thus every uniformly continuous function is continuous, but not conversely.
\end{rmk}

\setcounter{thm}{4}
\begin{defn}[Equivalent sequences]\label{9.9.5}
  Let \(m\) be an integer, let \((a_n)_{n = m}^\infty\) and \((b_n)_{n = m}^\infty\) be two sequences of real numbers, and let \(\varepsilon > 0\) be given.
  We say that \((a_n)_{n = m}^\infty\) is \emph{\(\varepsilon\)-close} to \((b_n)_{n = m}^\infty\) iff \(a_n\) is \(\varepsilon\)-close to \(b_n\) for each \(n \geq m\).
  We say that \((a_n)_{n = m}^\infty\) is \emph{eventually \(\varepsilon\)-close} to \((b_n)_{n = m}^\infty\) iff there exists an \(N \geq m\) such that the sequences \((a_n)_{n = N}^\infty\) and \((b_n)_{n = N}^\infty\) are \(\varepsilon\)-close.
  Two sequences \((a_n)_{n = m}^\infty\) and \((b_n)_{n = m}^\infty\) are \emph{equivalent} iff for each \(\varepsilon > 0\), the sequences \((a_n)_{n = m}^\infty\) and \((b_n)_{n = m}^\infty\) are eventually \(\varepsilon\)-close.
\end{defn}

\begin{rmk}\label{9.9.6}
  One could debate whether \(\varepsilon\) should be assumed to be rational or real, but a minor modification of \cref{6.1.4} shows that this does not make any difference to the above definitions.
\end{rmk}

\begin{lem}\label{9.9.7}
  Let \((a_n)_{n = 1}^\infty\) and \((b_n)_{n = 1}^\infty\) be sequences of real numbers
  (not necessarily bounded or convergent).
  Then \((a_n)_{n = 1}^\infty\) and \((b_n)_{n = 1}^\infty\) are equivalent if and only if \(\lim_{n \to \infty} (a_n - b_n) = 0\).
\end{lem}

\begin{proof}
  \begin{align*}
         & (a_n)_{n = 1}^\infty = (b_n)_{n = 1}^\infty                                                                                             \\
    \iff & \forall \varepsilon \in \R^+, \exists\ N \in \Z^+ : \forall n \geq N, \abs{a_n - b_n} \leq \varepsilon       & \text{(by \cref{9.9.5})} \\
    \iff & \forall \varepsilon \in \R^+, \exists\ N \in \Z^+ : \forall n \geq N, \abs{(a_n - b_n) - 0} \leq \varepsilon                            \\
    \iff & \lim_{n \to \infty} (a_n - b_n) = 0.                                                                         & \text{(by \cref{6.1.5})}
  \end{align*}
\end{proof}

\begin{prop}\label{9.9.8}
  Let \(X\) be a subset of \(\R\), and let \(f : X \to \R\) be a function.
  Then the following two statements are logically equivalent:
  \begin{enumerate}
    \item \(f\) is uniformly continuous on \(X\).
    \item Whenever \((x_n)_{n = 0}^\infty\) and \((y_n)_{n = 0}^\infty\) are two equivalent sequences consisting of elements of \(X\), the sequences \((f(x_n))_{n = 0}^\infty\) and \((f(y_n))_{n = 0}^\infty\) are also equivalent.
  \end{enumerate}
\end{prop}

\begin{proof}
  We first show that statement (a) implies statement (b).
  By \cref{9.9.2}, \(f\) is uniformly continuous on \(X\) iff
  \[
    \forall \varepsilon \in \R^+, \exists\ \delta \in \R^+ : \big(\forall x, y \in X, \abs{x - y} < \delta \implies \abs{f(x) - f(y)} \leq \varepsilon\big).
  \]
  By \cref{9.9.5}, \((x_n)_{n = 0}^\infty = (y_n)_{n = 0}^\infty\) iff
  \[
    \forall \varepsilon \in \R^+, \exists\ N \in \N : \forall n \geq N, \abs{x_n - y_n} \leq \varepsilon.
  \]
  In particular, we have
  \[
    \exists\ N \in \N : \forall n \geq N, \abs{x_n - y_n} \leq \delta / 2 < \delta.
  \]
  Since \(\forall n \geq 0\) we have \(x_n, y_n \in X\), by replacing \(x, y\) with \(x_n, y_n\) we see that
  \[
    \forall \varepsilon \in \R^+, \exists\ \delta \in \R^+ : \big(\forall x_n, y_n \in X, \abs{x_n - y_n} < \delta \implies \abs{f(x_n) - f(y_n)} \leq \varepsilon\big).
  \]
  Thus we have
  \[
    \forall \varepsilon \in \R^+, \exists\ N \in \N : \forall n \geq N, \abs{f(x_n) - f(y_n)} \leq \varepsilon
  \]
  and by \cref{9.9.5} \(\big(f(x_n)\big)_{n = 0}^\infty = \big(f(y_n)\big)_{n = 0}^\infty\).

  Now we show that statement (b) implies statement (a).
  By hypothesis we know that if \((x_n)_{n = 0}^\infty, (y_n)_{n = 0}^\infty\) in \(X\), then \((x_n)_{n = 0}^\infty = (y_n)_{n = 0}^\infty \implies \big(f(x_n)\big)_{n = 0}^\infty = \big(f(y_n)\big)_{n = 0}^\infty\).
  Suppose for sake of contradiction that \(f\) is not uniformly continuous on \(X\).
  Then we have
  \[
    \exists\ \varepsilon \in \R^+ : \forall \delta \in \R^+, \Big(\forall x, y \in X, (\abs{x - y} < \delta) \land \big(\abs{f(x) - f(y)} > \varepsilon\big)\Big).
  \]
  By replacing \(x, y\) with \(x_n, y_n\) we have
  \[
    \exists\ \varepsilon \in \R^+ : \forall \delta \in \R^+, \Big(\forall x_n, y_n \in X, (\abs{x_n - y_n} < \delta) \land \big(\abs{f(x_n) - f(y_n)} > \varepsilon\big)\Big).
  \]
  Since \((x_n)_{n = 0}^\infty = (y_n)_{n = 0}^\infty\), we have
  \[
    \exists\ N_1 \in \N : \forall n \geq N_1, \abs{x_n - y_n} \leq \delta / 2 < \delta.
  \]
  Since \(\big(f(x_n)\big)_{n = 0}^\infty = \big(f(y_n)\big)_{n = 0}^\infty\), we have
  \[
    \exists\ N_2 \in \N : \forall n \geq N_2, \abs{f(x_n) - f(y_n)} \leq \varepsilon.
  \]
  Let \(N = \max(N_1, N_2)\).
  Then we have
  \[
    \forall n \in \N \land n \geq N, (\abs{x_n - y_n} < \delta) \land \big(\abs{f(x_n) - f(y_n)} \leq \varepsilon\big),
  \]
  which contradict to \((\abs{x_n - y_n} < \delta) \land \big(\abs{f(x_n) - f(y_n)} > \varepsilon\big)\).
  Thus \(f\) is uniformly continuous on \(X\).
\end{proof}

\begin{rmk}\label{9.9.9}
  The reader should compare \cref{9.9.8} with \cref{9.3.9}.
  \cref{9.3.9} asserted that if \(f\) was continuous, then \(f\) maps convergent sequences to convergent sequences.
  In contrast, \cref{9.9.8} asserts that if \(f\) is \emph{uniformly} continuous, then \(f\) maps \emph{equivalent} pairs of sequences to equivalent pairs of sequences.
  To see how the two Propositions are connected, observe from \cref{9.9.7} that \((x_n)_{n = 0}^\infty\) will converge to \(x_*\) if and only if the sequences \((x_n)_{n = 0}^\infty\) and \((x_*)_{n = 0}^\infty\) are equivalent.
\end{rmk}

\setcounter{thm}{11}
\begin{prop}\label{9.9.12}
  Let \(X\) be a subset of \(\R\), and let \(f : X \to \R\) be a uniformly continuous function.
  Let \((x_n)_{n = 0}^\infty\) be a Cauchy sequence consisting entirely of elements in \(X\).
  Then \((f(x_n))_{n = 0}^\infty\) is also a Cauchy sequence.
\end{prop}

\begin{proof}
  Since \(f\) is uniformly continuous, by \cref{9.9.2} we have
  \[
    \forall \varepsilon \in \R^+, \exists\ \delta \in \R^+ : \big(\forall x, y \in X, \abs{x - y} < \delta \implies \abs{f(x) - f(y)} \leq \varepsilon\big).
  \]
  Let \(i, j \in \N\).
  Since \((x_n)_{n = 0}^\infty\) is a Cauchy sequence, by \cref{6.1.3} we have
  \[
    \forall \varepsilon \in \R^+, \exists\ N \in \N : \forall i, j \geq N, \abs{x_i - x_j} \leq \varepsilon.
  \]
  In particular, we have
  \[
    \exists\ N \in \N : \forall i, j \geq N, \abs{x_i - x_j} \leq \delta / 2 < \delta.
  \]
  Since \(x_i, x_j \in X\), we have
  \[
    \abs{x_i - x_j} < \delta \implies \abs{f(x_i) - f(x_j)} \leq \varepsilon.
  \]
  Thus we have
  \[
    \forall \varepsilon \in \R^+, \exists\ N \in \N : \forall i, j \geq N, \abs{f(x_i) - f(x_j)} \leq \varepsilon.
  \]
  and by \cref{6.1.3} \(\big(f(x_n)\big)_{n = 0}^\infty\) is a Cauchy sequence.
\end{proof}

\setcounter{thm}{13}
\begin{cor}\label{9.9.14}
  Let \(X\) be a subset of \(\R\), let \(f : X \to \R\) be a uniformly continuous function, and let \(x_0\) be an adherent point of \(X\).
  Then the limit \(\lim_{x \to x_0 ; x \in X} f(x)\) exists
  (in particular, it is a real number ).
\end{cor}

\begin{proof}
  Since \(x_0\) is an adherent point of \(X\), by \cref{9.1.14} there exists a sequence \((a_n)_{n = 0}^\infty\), consisting entirely of elements in \(X\), which converges to \(x_0\).
  Since \(\lim_{n \to \infty} a_n = x_0\), by \cref{6.1.12} \((a_n)_{n = 0}^\infty\) is a Cauchy sequence.
  Since \(f\) is uniformly continuous, by \cref{9.9.12} \(\big(f(a_n)\big)_{n = 0}^\infty\) is also a Cauchy sequence.
  Let \((b_n)_{n = 0}^\infty\) in \(X\) such that \(\lim_{n \to \infty} b_n = x_0\).
  Since
  \begin{align*}
             & \lim_{n \to \infty} b_n = x_0                                                                 \\
    \implies & \lim_{n \to \infty} a_n - b_n = 0                                 & \text{(by \cref{6.1.19})} \\
    \implies & (a_n)_{n = 0}^\infty = (b_n)_{n = 0}^\infty                       & \text{(by \cref{9.9.7})}  \\
    \implies & \big(f(a_n)\big)_{n = 0}^\infty = \big(f(b_n)\big)_{n = 0}^\infty & \text{(by \cref{9.9.8})}  \\
    \implies & \lim_{n \to \infty} f(a_n) = \lim_{n \to \infty} f(b_n)           & \text{(by \cref{9.9.12})}
  \end{align*}
  and \((b_n)_{n = 0}^\infty\) is arbitrary, by \cref{9.3.9} we know that \(\lim_{x \to x_0 ; x \in X} f(x)\) exists.

  Now we show that \(f : (0, 2) \to \R\) defined by \(f(x) = 1 / x\) is not uniformly continuous.
  Suppose for sake of contradiction that \(f\) is uniformly continuous.
  Since \(0\) is an adherent point of \((0, 2)\), we know that \(\lim_{x \to 0 ; x \in (0, 2)} f(x)\) exists.
  Since \(f\) is continuous, \(f(0) = 1 / 0\) must exist, a contradiction.
  Thus \(f\) is not uniformly continuous.
\end{proof}

\begin{prop}\label{9.9.15}
  Let \(X\) be a subset of \(\R\), and let \(f : X \to \R\) be a uniformly continuous function.
  Suppose that \(E\) is a bounded subset of \(X\).
  Then \(f(E)\) is also bounded.
\end{prop}

\begin{proof}
  Suppose for sake of contradiction that \(f(E)\) is unbounded.
  Thus for every real number \(M\) there exists an element \(x \in E\) such that \(\abs{f(x)} \geq M\).

  In particular, for every natural number \(n\), the set \(\{x \in E : \abs{f(x)} \geq n\}\) is non-empty.
  We can thus choose a sequence \((x_n)_{n = 0}^\infty\) in \(E\) such that \(\abs{f(x_n)} \geq n\) for all \(n\).
  Since \((x_n)_{n = 0}^\infty\) in \(E\), by Bolzano-Weierstrass theorem (\cref{6.6.8}) there exists a subsequence \((x_{n_j})_{j = 0}^\infty\) which converges, where \(n_0 < n_1 < n_2 < \dots\) is an increasing sequence of natural numbers.
  In particular, we see that \(n_j \geq j\) for all \(j \in \N\) (use induction).

  Let \(\lim_{n \to \infty} x_n = x_*\).
  Since \((x_n)_{n = 0}^\infty\) in \(E\), by \cref{9.1.14} we know that \(x_*\) is an adherent point of \(E\).
  Since \(f\) is continuous on \(X\), by \cref{ex:9.4.6} \(f\) is continuous on \(E\).
  In particular, \(f\) is uniformly continuous on \(E\).
  Thus by \cref{9.9.14} we know that \(\lim_{x \to x_* ; x \in E} f(x)\) exists.
  By \cref{9.3.9} we see that
  \[
    \lim_{j \to \infty} f(x_{n_j}) = \lim_{x \to x_* ; x \in E} f(x).
  \]
  Thus the sequence \(\big(f(x_{n_j})\big)_{j = 0}^\infty\) is convergent, and hence it is bounded.
  On the other hand, we know from the construction that \(\abs{f(x_{n_j})} \geq n_j \geq j\) for all \(j\), and hence the sequence \(\big(f(x_{n_j})\big)_{j = 0}^\infty\) is unbounded, a contradiction.
\end{proof}

\begin{thm}\label{9.9.16}
  Let \(a < b\) be real numbers, and let \(f : [a, b] \to \R\) be a function which is continuous on \([a, b]\).
  Then \(f\) is also uniformly continuous.
\end{thm}

\begin{proof}
  Suppose for sake of contradiction that \(f\) is not uniformly continuous.
  By \cref{9.9.8}, there must therefore exist two equivalent sequences \((x_n)_{n = 0}^\infty\) and \((y_n)_{n = 0}^\infty\) in \([a, b]\) such that the sequences \(\big(f(x_n)\big)_{n = 0}^\infty\) and \(\big(f(y_n)\big)_{n = 0}^\infty\) are not equivalent.
  In particular, we can find an \(\varepsilon > 0\) such that \(\big(f(x_n)\big)_{n = 0}^\infty\) and \(\big(f(y_n)\big)_{n = 0}^\infty\) are not eventually \(\varepsilon\)-close.

  Fix this value of \(\varepsilon\), and let \(E\) be the set
  \[
    E \coloneqq \{n \in \N : f(x_n) \text{ and } f(y_n) \text{ are not \(\varepsilon\)-close}\}.
  \]
  We must have \(E\) infinite, since if \(E\) were finite then \(\big(f(x_n)\big)_{n = 0}^\infty\) and \(\big(f(y_n)\big)_{n = 0}^\infty\) would be eventually \(\varepsilon\)-close.
  By \cref{8.1.5}, \(E\) is countable;
  in fact from the proof of that proposition we see that we can find an infinite sequence
  \[
    n_0 < n_1 < n_2 < \dots
  \]
  consisting entirely of elements in \(E\).
  In particular, we have
  \[
    \abs{f(x_{n_j}) - f(y_{n_j})} > \varepsilon \text{ for all } j \in \N. \tag{9.3}\label{eq 9.3}
  \]
  On the other hand, the sequence \((x_{n_j})_{j = 0}^\infty\) is a sequence in \([a, b]\), and so by the Heine-Borel theorem (\cref{9.1.24}) there must be a subsequence \((x_{n_{j_k}})_{k = 0}^\infty\) which converges to some limit \(L\) in \([a, b]\).
  In particular, \(f\) is continuous at \(L\), and so by \cref{9.4.7},
  \[
    \lim_{k \to \infty} f(x_{n_{j_k}}) = f(L). \tag{9.4}\label{eq 9.4}
  \]
  Note that \((x_{n_{j_k}})_{k = 0}^\infty\) is a subsequence of \((x_n)_{n = 0}^\infty\), and \((y_{n_{j_k}})_{k = 0}^\infty\) is a subsequence of \((y_n)_{n = 0}^\infty\), by \cref{6.6.4}.
  On the other hand, from \cref{9.9.7} we have
  \[
    \lim_{n \to \infty} (x_n - y_n) = 0.
  \]
  By \cref{6.6.5}, we thus have
  \[
    \lim_{k \to \infty} (x_{n_{j_k}} - y_{n_{j_k}}) = 0.
  \]
  Since \(x_{n_{j_k}}\) converges to \(L\) as \(k \to \infty\), we thus have by limit laws
  \[
    \lim_{k \to \infty} y_{n_{j_k}} = L.
  \]
  and hence by continuity of \(f\) at \(L\)
  \[
    \lim_{k \to \infty} f(y_{n_{j_k}}) = f(L).
  \]
  Subtracting this from (9.4) using limit laws, we obtain
  \[
    \lim_{k \to \infty} (f(x_{n_{j_k}}) - f(y_{n_{j_k}})) = 0.
  \]
  But this contradicts (9.3) (by \cref{9.9.7}).
  From this contradiction we conclude that \(f\) is in fact uniformly continuous.
\end{proof}

\begin{rmk}\label{9.9.17}
  One should compare \cref{9.6.3}, \cref{9.9.15}, and \cref{9.9.16} with each other.
  Note in particular that \cref{9.6.3} follows from combining \cref{9.9.15} and \cref{9.9.16}.
\end{rmk}

\exercisesection

\begin{ex}\label{ex:9.9.1}
  Prove \cref{9.9.7}.
\end{ex}

\begin{proof}
  See \cref{9.9.7}.
\end{proof}

\begin{ex}\label{ex:9.9.2}
  Prove \cref{9.9.8}.
\end{ex}

\begin{proof}
  See \cref{9.9.8}.
\end{proof}

\begin{ex}\label{ex:9.9.3}
  Prove \cref{9.9.12}.
\end{ex}

\begin{proof}
  See \cref{9.9.12}.
\end{proof}

\begin{ex}\label{ex:9.9.4}
  Use \cref{9.9.12} to prove \cref{9.9.14}.
  Use this corollary to give an alternate demonstration of the results in Example 9.9.10.
\end{ex}

\begin{proof}
  See \cref{9.9.14}.
\end{proof}

\begin{ex}\label{ex:9.9.5}
  Prove \cref{9.9.15}.
\end{ex}

\begin{proof}
  See \cref{9.9.15}.
\end{proof}

\begin{ex}\label{ex:9.9.6}
  Let \(X, Y, Z\) be subsets of \(\R\).
  Let \(f : X \to Y\) be a function which is uniformly continuous on \(X\), and let \(g : Y \to Z\) be a function which is uniformly continuous on \(Y\).
  Show that the function \(g \circ f : X \to Z\) is uniformly continuous on \(X\).
\end{ex}

\begin{proof}
  Since \(f\) is continuous on \(X\) and \(g\) is continuous on \(Y\), by \cref{9.4.13} we know that \(g \circ f\) is continuous on \(X\).
  Since \(g\) is uniformly continuous, by \cref{9.9.2} we have
  \[
    \forall \varepsilon \in \R^+, \exists\ \delta' \in \R^+ : \big(\forall y_1, y_2 \in Y, \abs{y_1 - y_2} < \delta' \implies \abs{g(y_1) - g(y_2)} \leq \varepsilon\big).
  \]
  Similarly, since \(f\) is uniformly continuous, by \cref{9.9.2} we have
  \[
    \forall \varepsilon' \in \R^+, \exists\ \delta \in \R^+ : \big(\forall x_1, x_2 \in X, \abs{x_1 - x_2} < \delta \implies \abs{f(x_1) - f(x_2)} \leq \varepsilon'\big).
  \]
  In particular, we have
  \[
    \exists\ \delta \in \R^+ : \big(\forall x_1, x_2 \in X, \abs{x_1 - x_2} < \delta \implies \abs{f(x_1) - f(x_2)} \leq \delta' / 2 < \delta'\big).
  \]
  Since \(f(x_1), f(x_2) \in Y\), we have
  \[
    \abs{f(x_1) - f(x_2)} < \delta' \implies \abs{g\big(f(x_1)\big) - g\big(f(x_2)\big)} \leq \varepsilon.
  \]
  Thus we have showed that
  \[
    \forall \varepsilon \in \R^+, \exists\ \delta \in \R^+ : \Big(\forall x_1, x_2 \in X, \abs{x_1 - x_2} < \delta \implies \abs{g\big(f(x_1)\big) - g\big(f(x_2)\big)} \leq \varepsilon\Big).
  \]
  and by \cref{9.9.2} \(g \circ f\) is uniformly continuous on \(X\).
\end{proof}
\section{Limits at infinity}\label{i:sec:9.10}

\begin{defn}[Infinite adherent points]\label{i:9.10.1}
  Let \(X\) be a subset of \(\R\).
  We say that \(+\infty\) is \emph{adherent} to \(X\) iff for every \(M \in \R\) there exists an \(x \in X\) such that \(x > M\);
  we say that \(-\infty\) is \emph{adherent} to \(X\) iff for every \(M \in \R\) there exists an \(x \in X\) such that \(x < M\).
\end{defn}

\begin{note}
  In other words, \(+\infty\) is adherent to \(X\) iff \(X\) has no upper bound, or equivalently iff \(\sup(X) = +\infty\).
  Similarly \(-\infty\) is adherent to \(X\) iff \(X\) has no lower bound, or iff \(\inf(X) = -\infty\).
  Thus a set is bounded iff \(+\infty\) and \(-\infty\) are not adherent points.
\end{note}

\begin{rmk}\label{i:9.10.2}
  \cref{i:9.10.1} may seem rather different from \cref{i:9.1.8}, but can be unified using the topological structure of the extended real line \(\R^*\).
\end{rmk}

\begin{defn}[Limits at infinity]\label{i:9.10.3}
  Let \(X\) be a subset of \(\R\) with \(+\infty\) as an adherent point, and let \(f : X \to \R\) be a function.
  We say that \emph{\(f(x)\) converges to \(L\)} as \(x \to +\infty\) in \(X\), and write \(\lim_{x \to +\infty ; x \in X} f(x) = L\), iff for every \(\varepsilon > 0\) there exists an \(M\) such that \(f\) is \(\varepsilon\)-close to \(L\) on \(X \cap (M, +\infty)\)
  (i.e., \(\abs{f(x) - L} \leq \varepsilon\) for all \(x \in X\) such that \(x > M\)).
  Similarly we say that \emph{\(f(x)\) converges to \(L\)} as \(x \to -\infty\) iff for every \(\varepsilon > 0\) there exists an \(M\) such that \(f\) is \(\varepsilon\)-close to \(L\) on \(X \cap (-\infty, M)\).
\end{defn}

\begin{note}
  One can do many of the same things with these limits at infinity as we have been doing with limits at other points \(x_0\);
  for instance, it turns out that all of the limit laws continue to hold.
  However, as we will not be using these limits much in this text, we will not devote much attention to these matters.
  We will note though that this definition is consistent with the notion of a limit \(\lim_{n \to \infty} a_n\) of a sequence.
\end{note}

\exercisesection

\begin{ex}\label{i:ex:9.10.1}
  Let \((a_n)_{n = 0}^\infty\) be a sequence of real numbers, then \(a_n\) can also be thought of as a function from \(\N\) to \(\R\), which takes each natural number \(n\) to a real number \(a_n\).
  Show that
  \[
    \lim_{n \to +\infty ; n \in \N} a_n = \lim_{n \to \infty} a_n
  \]
  where the left-hand limit is defined by \cref{i:9.10.3} and the right-hand limit is defined by \cref{i:6.1.8}.
  More precisely, show that if one of the above two limits exists then so does the other, and then they both have the same value.
  Thus the two notions of limit here are compatible.
\end{ex}

\begin{proof}
  We first show that \(\lim_{n \to +\infty ; n \in \N} a_n = L\) implies \(\lim_{n \to \infty} a_n = L\).
  By \cref{i:9.10.3}, we have
  \[
    \forall \varepsilon \in \R^+, \exists M \in \R : \big(\forall n \in \N, n > M \implies \abs{a_n - L} \leq \varepsilon\big).
  \]
  Since \(M \in \R\), by \cref{i:5.4.12} \(\exists N \in \N\) such that \(M \leq N\).
  Then we have
  \[
    \forall n \in \N, n > \N \implies n > M \implies \abs{a_n - L} \leq \varepsilon.
  \]
  Thus by \cref{i:6.1.5} we have \(\lim_{n \to \infty} a_n = L\).

  Now we show that \(\lim_{n \to \infty} a_n = L\) implies \(\lim_{n \to +\infty ; n \in \N} a_n = L\).
  By \cref{i:6.1.5} we have
  \[
    \forall \varepsilon \in \R^+, \exists N \in \N : \forall n \geq N, \abs{a_n - L} \leq \varepsilon.
  \]
  Since \(\N \subseteq \R\), we have \(N \in \R\) and thus by \cref{i:9.10.3} we have \(\lim_{n \to +\infty, n \in \N} a_n = L\).
\end{proof}


\chapter{Differentiation of functions}\label{ch:10}

\section{Basic definitions}\label{sec:10}

\begin{note}
  We can now define derivatives analyti cally, using limits, in contrast to the geometric definition of derivatives, which uses tangents.
  The advantage of working analytically is that
  (a) we do not need to know the axioms of geometry, and
  (b) these definitions can be modified to handle functions of several variables, or functions whose values are vectors instead of scalar.
  Furthermore, one's geometric intuition becomes difficult to rely on once one has more than three dimensions in play.
  (Conversely, one can use one's experience in analytic rigour to extend one's geometric intuition to such abstract settings;
  as mentioned earlier, the two viewpoints complement rather than oppose each other.)
\end{note}

\begin{defn}[Differentiability at a point]\label{10.1.1}
  Let \(X\) be a subset of \(\R\), and let \(x_0 \in X\) be an element of \(X\) which is also a limit point of \(X\).
  Let \(f : X \to \R\) be a function.
  If the limit
  \[
    \lim_{x \to x_0 ; x \in X \setminus \{x_0\}} \dfrac{f(x) - f(x_0)}{x - x_0}
  \]
  converges to some real number \(L\), then we say that \(f\) is \emph{differentiable at \(x_0\) on \(X\) with derivative \(L\)}, and write \(f'(x_0) \coloneqq L\).
  If the limit does not exist, or if \(x_0\) is not an element of \(X\) or not a limit point of \(X\), we leave \(f'(x_0)\) undefined, and say that \(f\) is \emph{not differentiable at \(x_0\) on \(X\)}.
\end{defn}

\begin{rmk}\label{10.1.2}
  Note that we need \(x_0\) to be a limit point in order for \(x_0\) to be adherent to \(X \setminus \{x_0\}\), otherwise the limit
  \[
    \lim_{x \to x_0 ; x \in X \setminus \{x_0\}} \dfrac{f(x) - f(x_0)}{x - x_0}
  \]
  would automatically be undefined.
  In particular, we do not define the derivative of a function at an isolated point;
  In practice, the domain \(X\) will almost always be an interval, and so by \cref{9.1.21} all elements \(x_0\) of \(X\) will automatically be limit points and we will not have to care much about these issues.
\end{rmk}

\setcounter{thm}{3}
\begin{rmk}\label{10.1.4}
  This point is trivial, but it is worth mentioning:
  if \(f : X \to \R\) is differentiable at \(x_0\), and \(g : X \to \R\) is equal to \(f\) (i.e., \(g(x) = f(x)\) for all \(x \in X\)), then \(g\) is also differentiable at \(x_0\) and \(g'(x_0) = f'(x_0)\).
  However, if two functions \(f\) and \(g\) merely have the same value at \(x_0\), i.e., \(g(x_0) = f(x_0)\), this does not imply that \(g'(x_0) = f'(x_0)\).
  Thus there is a big difference between two functions being equal on their whole domain, and merely being equal at one point.
\end{rmk}

\begin{rmk}\label{10.1.5}
  One sometimes writes \(\dfrac{df}{dx}\) instead of \(f'\).
  This notation is of course very familiar and convenient, but one has to be a little careful, because it is only safe to use as long as \(x\) is the only variable used to represent the input for \(f\);
  otherwise one can get into all sorts of trouble.
  Because of this possible source of confusion, we will refrain from using the notation \(\dfrac{df}{dx}\) whenever it could possibly lead to confusion.
  (This confusion becomes even worse in the calculus of several variables, and the standard notation of \(\dfrac{\partial f}{\partial x}\) can lead to some serious ambiguities.
  There are ways to resolve these ambiguities, most notably by introducing the notion of differentiation along vector fields, but this is beyond the scope of this text.)
\end{rmk}

\begin{eg}\label{10.1.6}
  Let \(f : \R \to \R\) be the function \(f(x) \coloneqq \abs{x}\), and let \(x_0 = 0\).
  To see whether \(f\) is differentiable at \(0\) on \(\R\), we compute the limit
  \[
    \lim_{x \to 0 ; x \in \R \setminus \{0\}} \dfrac{f(x) - f(0)}{x - 0} = \lim_{x \to 0 ; x \in \R \setminus \{0\}} \dfrac{\abs{x}}{x}.
  \]
  Now we take left limits and right limits.
  The right limit is
  \[
    \lim_{x \to 0 ; x \in (0, \infty)} \dfrac{\abs{x}}{x} = \lim_{x \to 0 ; x \in (0, \infty)} \dfrac{x}{x} = \lim_{x \to 0 ; x \in (0, \infty)} 1 = 1,
  \]
  while the left limit is
  \[
    \lim_{x \to 0 ; x \in (-\infty, 0)} \dfrac{\abs{x}}{x} = \lim_{x \to 0 ; x \in (-\infty, 0)} \dfrac{-x}{x} = \lim_{x \to 0 ; x \in (-\infty, 0)} -1 = -1,
  \]
  and these limits do not match.
  Thus \(\lim_{x \to 0 ; x \in (0, \infty)} \dfrac{\abs{x}}{x}\) does not exist, and \(f\) is not differentiable at \(0\) on \(\R\).
  However, if one restricts \(f\) to \([0, \infty)\), then the restricted function \(f|_{[0, \infty)}\) \emph{is} differentiable at \(0\) on \([0, \infty)\), with derivative \(1\):
  \[
  \lim_{x \to 0 ; x \in [0, \infty) \setminus \{0\}} \dfrac{f(x) - f(0)}{x - 0} = \lim_{x \to 0 ; x \in (0, \infty)} \dfrac{\abs{x}}{x} = 1.
    \]
    Similarly, when one restricts \(f\) to \((-\infty, 0]\), the restricted function \(f|_{(-\infty, 0]}\) is differentiable at \(0\) on \((-\infty, 0]\), with derivative \(-1\).
  Thus even when a function is not differentiable, it is sometimes possible to restore the differentiability by restricting the domain of the function.
\end{eg}

\begin{prop}[Newton's approximation]\label{10.1.7}
  Let \(X\) be a subset of \(\R\), let \(x_0 \in X\) be a limit point of \(X\), let \(f : X \to \R\) be a function, and let \(L\) be a real number.
  Then the following statements are logically equivalent:
  \begin{enumerate}
    \item \(f\) is differentiable at \(x_0\) on \(X\) with derivative \(L\).
    \item For every \(\varepsilon > 0\), there exists a \(\delta > 0\) such that \(f(x)\) is \(\varepsilon \abs{x - x_0}\)-close to \(f(x_0) + L(x - x_0)\) whenever \(x \in X\) is \(\delta\)-close to \(x_0\), i.e., we have
          \[
            \abs{f(x) - (f(x_0) + L(x - x_0))} \leq \varepsilon \abs{x - x_0}
          \]
          whenever \(x \in X\) and \(\abs{x - x_0} \leq \delta\).
  \end{enumerate}
\end{prop}

\begin{proof}
  We first show that the first statement implies the second statement.
  Since \(f\) is differentiable at \(x_0\) on \(X\) with derivative \(L\), by \cref{10.1.1} we have
  \[
    \lim_{x \to x_0 ; x \in X \setminus \{x_0\}} \dfrac{f(x) - f(x_0)}{x - x_0} = L.
  \]
  By \cref{9.3.6} this means
  \[
    \forall \varepsilon \in \R^+, \exists\ \delta \in \R^+ : \bigg(\forall x \in X \setminus \{x_0\}, \abs{x - x_0} < \delta \implies \abs{\dfrac{f(x) - f(x_0)}{x - x_0} - L} \leq \varepsilon\bigg).
  \]
  Thus we have
  \begin{align*}
             & \forall x \in X \setminus \{x_0\}, \abs{x - x_0} \leq \delta / 2 < \delta             \\
    \implies & \abs{\dfrac{f(x) - f(x_0)}{x - x_0} - L} \leq \varepsilon                             \\
    \implies & \abs{\dfrac{f(x) - f(x_0)}{x - x_0} - L} \abs{x - x_0} \leq \varepsilon \abs{x - x_0} \\
    \implies & \abs{(f(x) - f(x_0)) - L(x - x_0)} \leq \varepsilon \abs{x - x_0}                     \\
    \implies & \abs{f(x) - \big(f(x_0) + L(x - x_0)\big)} \leq \varepsilon \abs{x - x_0}.
  \end{align*}
  If \(x = x_0\), then we have
  \begin{align*}
             & 0 = \abs{x_0 - x_0} \leq \delta / 2 < \delta                                             \\
    \implies & 0 = \abs{f(x_0) - \big(f(x_0) + L(x_0 - x_0)\big)} \leq \varepsilon \abs{x_0 - x_0} = 0.
  \end{align*}
  Thus we have \(\forall \varepsilon \in \R^+\), \(\exists\ \delta \in \R^+\) such that
  \[
    \forall x \in X, \abs{x - x_0} \leq \delta / 2 \implies \abs{f(x) - \big(f(x_0) + L(x - x_0)\big)} \leq \varepsilon \abs{x - x_0}.
  \]

  Now we show that the second statement implies the first statement.
  By hypothesis we have \(\forall \varepsilon \in \R^+\), \(\exists\ \delta \in \R^+\) such that
  \[
    \forall x \in X, \abs{x - x_0} \leq \delta \implies \abs{f(x) - \big(f(x_0) + L(x - x_0)\big)} \leq \varepsilon \abs{x - x_0}.
  \]
  In particular, we have
  \[
    \forall x \in X \setminus \{x_0\}, \abs{x - x_0} \leq \delta \implies \abs{f(x) - \big(f(x_0) + L(x - x_0)\big)} \leq \varepsilon \abs{x - x_0}.
  \]
  Thus we have
  \begin{align*}
             & \forall x \in X \setminus \{x_0\}, \abs{x - x_0} \leq \delta                       \\
    \implies & \abs{f(x) - \big(f(x_0) + L(x - x_0)\big)} \leq \varepsilon \abs{x - x_0}          \\
    \implies & \dfrac{\abs{f(x) - \big(f(x_0) + L(x - x_0)\big)}}{\abs{x - x_0}} \leq \varepsilon \\
    \implies & \abs{\dfrac{f(x) - f(x_0)}{x - x_0} - L} \leq \varepsilon.
  \end{align*}
  By \cref{9.3.6} this means
  \[
    \lim_{x \to x_0 ; x \in X \setminus \{x_0\}} \dfrac{f(x) - f(x_0)}{x - x_0} = L
  \]
  and by \cref{10.1.1} we know that \(f\) is differentiable at \(x_0\) on \(X\) with derivative \(L\).
\end{proof}

\begin{rmk}\label{10.1.8}
  Newton's approximation is of course named after the great scientist and mathematician Isaac Newton (1642 -- 1727), one of the founders of differential and integral calculus.
\end{rmk}

\begin{rmk}\label{10.1.9}
  We can phrase \cref{10.1.7} in a more informal way:
  if \(f\) is differentiable at \(x_0\), then one has the approximation \(f(x) \approx f(x_0) + f'(x_0)(x - x_0)\), and conversely.
\end{rmk}

\begin{prop}[Differentiability implies continuity]\label{10.1.10}
  Let \(X\) be a subset of \(\R\), let \(x_0 \in X\) be a limit point of \(X\), and let \(f : X \to \R\) be a function.
  If \(f\) is differentiable at \(x_0\), then \(f\) is also continuous at \(x_0\).
\end{prop}

\begin{proof}
  Since \(f\) is differentiable at \(x_0\), by \cref{10.1.1} we have
  \[
    L = \lim_{x \to x_0 ; x \in X \setminus \{x_0\}} \dfrac{f(x) - f(x_0)}{x - x_0}
  \]
  for some \(L \in \R\).
  By \cref{10.1.7}, we have \(\forall \varepsilon \in \R^+\), \(\exists\ \delta \in \R^+\) such that
  \begin{align*}
             & \forall x \in X, \abs{x - x_0} \leq \delta                                                                      \\
    \implies & \abs{f(x) - \big(f(x_0) + L(x - x_0)\big)} \leq \varepsilon \abs{x - x_0}                                       \\
    \implies & \abs{f(x) - \big(f(x_0) + L(x - x_0)\big)} + \abs{L(x - x_0)} \leq \varepsilon \abs{x - x_0} + \abs{L(x - x_0)} \\
    \implies & \abs{f(x) - \big(f(x_0) + L(x - x_0)\big)} + \abs{L(x - x_0)} \leq (\varepsilon + \abs{L}) \abs{x - x_0}        \\
    \implies & \abs{f(x) - \big(f(x_0) + L(x - x_0)\big) + L(x - x_0)}                                                         \\
             & \leq \abs{f(x) - \big(f(x_0) + L(x - x_0)\big)} + \abs{L(x - x_0)}                                              \\
             & \leq (\varepsilon + \abs{L}) \abs{x - x_0}                                                                      \\
    \implies & \abs{f(x) - f(x_0)} \leq (\varepsilon + \abs{L}) \abs{x - x_0}.
  \end{align*}
  Let \(\delta' = \min(\delta, \varepsilon / (\varepsilon + \abs{L}))\).
  Then we have
  \begin{align*}
             & \forall x \in X, \abs{x - x_0} < \delta' \leq \delta                                                                                                                        \\
    \implies & \abs{f(x) - f(x_0)} \leq (\varepsilon + \abs{L}) \abs{x - x_0}                                                                                                              \\
    \implies & \abs{f(x) - f(x_0)} \leq (\varepsilon + \abs{L}) \abs{x - x_0} \leq (\varepsilon + \abs{L}) \delta' \leq (\varepsilon + \abs{L}) \dfrac{\varepsilon}{\varepsilon + \abs{L}} \\
    \implies & \abs{f(x) - f(x_0)} \leq \varepsilon.
  \end{align*}
  Thus by \cref{9.3.6} we have \(\lim_{x \to x_0 ; x \in X} f(x) = f(x_0)\), and by \cref{9.4.1} \(f\) is continuous at \(x_0\).
\end{proof}

\begin{defn}[Differentiability on a domain]\label{10.1.11}
  Let \(X\) be a subset of \(\R\), and let \(f : X \to \R\) be a function.
  We say that \(f\) is \emph{differentiable on} \(X\) if, for every limit point \(x_0 \in X\), the function \(f\) is differentiable at \(x_0\) on \(X\).
\end{defn}

\begin{cor}\label{10.1.12}
  Let \(X\) be a subset of \(\R\), and let \(f : X \to \R\) be a function which is differentiable on \(X\).
  Then \(f\) is also continuous on \(X\).
\end{cor}

\begin{proof}
  By \cref{9.1.11} we know that \(\forall x_0 \in X\), \(x_0\) is an adherent point.
  By \cref{ex:9.1.9} we know that \(x_0\) is either a limit point or an isolated point.
  By \cref{10.1.10} and \cref{10.1.11} we know that if \(x_0\) is a limit point then \(f\) is continuous at \(x_0\).
  So we only need to show that if \(x_0\) is an isolated point, then \(f\) is also continuous at \(x_0\).
  Suppose that \(x_0\) is an isolated point of \(X\).
  By \cref{9.1.18} we know that \(\exists\ \varepsilon' \in \R^+\) such that \(\abs{x - x_0} > \varepsilon'\) for all \(x \in X \setminus \{x_0\}\).
  To show that \(f\) is continuous at \(x_0\), by \cref{9.4.1} and \cref{9.3.6} we need to show that
  \[
    \forall \varepsilon \in \R^+, \exists\ \delta \in \R^+ : \big(\forall x \in X, \abs{x - x_0} < \delta \implies \abs{f(x) - f(x_0)} \leq \varepsilon\big).
  \]
  Let \(\delta = \varepsilon'\).
  Since \(x_0\) is an isolated point, the only \(x \in X\) satisfying \(\abs{x - x_0} < \varepsilon'\) is \(x_0\).
  Thus we have \(0 = \abs{f(x_0) - f(x_0)} \leq \varepsilon\) and \(\lim_{x \to x_0 ; x \in X} f(x) = f(x_0)\).
\end{proof}

\begin{thm}[Differential calculus]\label{10.1.13}
  Let \(X\) be a subset of \(\R\), let \(x_0 \in X\) be a limit point of \(X\), and let \(f : X \to \R\) and \(g : X \to \R\) be functions.
  \begin{enumerate}
    \item If \(f\) is a constant function, i.e., there exists a real number \(c\) such that \(f(x) = c\) for all \(x \in X\), then \(f\) is differentiable at \(x_0\) and \(f'(x_0) = 0\).
    \item If \(f\) is the identity function, i.e., \(f(x) = x\) for all \(x \in X\), then \(f\) is differentiable at \(x_0\) and \(f'(x_0) = 1\).
    \item (Sum rule)
          If \(f\) and \(g\) are differentiable at \(x_0\), then \(f + g\) is also differentiable at \(x_0\), and \((f + g)'(x_0) = f'(x_0) + g'(x_0)\).
    \item (Product rule)
          If \(f\) and \(g\) are differentiable at \(x_0\), then \(fg\) is also differentiable at \(x_0\), and \((fg)'(x_0) = f'(x_0)g(x_0) + f(x_0)g'(x_0)\).
    \item If \(f\) is differentiable at \(x_0\) and \(c\) is a real number, then \(cf\) is also differentiable at \(x_0\), and \((cf)'(x_0) = cf'(x_0)\).
    \item (Difference rule)
          If \(f\) and \(g\) are differentiable at \(x_0\), then \(f - g\) is also differentiable at \(x_0\), and \((f - g)'(x_0) = f'(x_0) - g'(x_0)\).
    \item If \(g\) is differentiable at \(x_0\), and \(g\) is non-zero on \(X\) (i.e., \(g(x) \neq 0\) for all \(x \in X\)), then \(1 / g\) is also differentiable at \(x_0\), and \((\dfrac{1}{g})'(x_0) = -\dfrac{g'(x_0)}{g(x_0)^2}\).
    \item (Quotient rule)
          If \(f\) and \(g\) are differentiable at \(x_0\), and \(g\) is non-zero on \(X\), then \(f / g\) is also differentiable at \(x_0\), and
          \[
            (\dfrac{f}{g})'(x_0) = \dfrac{f'(x_0) g(x_0) - f(x_0) g'(x_0)}{g(x_0)^2}.
          \]
  \end{enumerate}
\end{thm}

\begin{proof}{(a)}
  We have \(\forall \varepsilon \in \R^+\), \(\forall \delta \in \R^+\) such that
  \[
    \forall x \in X \setminus \{x_0\}, \abs{x - x_0} < \delta \implies \abs{\dfrac{f(x) - f(x_0)}{x - x_0} - 0} = \abs{\dfrac{c - c}{x - x_0} - 0} = 0 \leq \varepsilon.
  \]
  Thus by \cref{9.3.6} we have
  \[
    \lim_{x \to x_0 ; x \in X \setminus \{x_0\}} \dfrac{f(x) - f(x_0)}{x - x_0} = 0
  \]
  and by \cref{10.1.1} we have \(f'(x_0) = 0\).
\end{proof}

\begin{proof}{(b)}
  We have \(\forall \varepsilon \in \R^+\), \(\forall \delta \in \R^+\) such that
  \[
    \forall x \in X \setminus \{x_0\}, \abs{x - x_0} < \delta \implies \abs{\dfrac{f(x) - f(x_0)}{x - x_0} - 1} = \abs{\dfrac{x - x_0}{x - x_0} - 1} = 0 \leq \varepsilon.
  \]
  Thus by \cref{9.3.6} we have
  \[
    \lim_{x \to x_0 ; x \in X \setminus \{x_0\}} \dfrac{f(x) - f(x_0)}{x - x_0} = 1
  \]
  and by \cref{10.1.1} we have \(f'(x_0) = 1\).
\end{proof}

\begin{proof}{(c)}
  By \cref{10.1.1} and \cref{9.3.14} we have
  \begin{align*}
      & f'(x_0) + g'(x_0)                                                                                                                                         \\
    = & \lim_{x \to x_0 ; x \in X \setminus \{x_0\}} \dfrac{f(x) - f(x_0)}{x - x_0} + \lim_{x \to x_0 ; x \in X \setminus \{x_0\}} \dfrac{g(x) - g(x_0)}{x - x_0} \\
    = & \lim_{x \to x_0 ; x \in X \setminus \{x_0\}} \dfrac{f(x) - f(x_0)}{x - x_0} + \dfrac{g(x) - g(x_0)}{x - x_0}                                              \\
    = & \lim_{x \to x_0 ; x \in X \setminus \{x_0\}} \dfrac{f(x) - f(x_0) + g(x) - g(x_0)}{x - x_0}                                                               \\
    = & \lim_{x \to x_0 ; x \in X \setminus \{x_0\}} \dfrac{f(x) + g(x) - (f(x_0) + g(x_0))}{x - x_0}                                                             \\
    = & \lim_{x \to x_0 ; x \in X \setminus \{x_0\}} \dfrac{(f + g)(x) - (f + g)(x_0)}{x - x_0}                                                                   \\
    = & (f + g)'(x_0).
  \end{align*}
  Thus \(f + g\) is differentiable at \(x_0\) and \((f + g)'(x_0) = f'(x_0) + g'(x_0)\).
\end{proof}

\begin{proof}{(d)}
  By \cref{10.1.1} and \cref{9.3.14} we have
  \begin{align*}
      & f'(x_0) g(x_0) + f(x_0) g'(x_0)                                                                                                                                                                                   \\
    = & \bigg(\lim_{x \to x_0 ; x \in X \setminus \{x_0\}} \dfrac{f(x) - f(x_0)}{x - x_0}\bigg) \bigg(\lim_{x \to x_0 ; x \in X \setminus \{x_0\}} g(x)\bigg)                                                             \\
      & + f(x_0) \bigg(\lim_{x \to x_0 ; x \in X \setminus \{x_0\}} \dfrac{g(x) - g(x_0)}{x - x_0}\bigg)                                                                                                                  \\
    = & \bigg(\lim_{x \to x_0 ; x \in X \setminus \{x_0\}} \dfrac{\big(f(x) - f(x_0)\big) g(x)}{x - x_0}\bigg) + \bigg(\lim_{x \to x_0 ; x \in X \setminus \{x_0\}} \dfrac{f(x_0) \big(g(x) - g(x_0)\big)}{x - x_0}\bigg) \\
    = & \bigg(\lim_{x \to x_0 ; x \in X \setminus \{x_0\}} \dfrac{f(x) g(x) - f(x_0) g(x)}{x - x_0}\bigg) + \bigg(\lim_{x \to x_0 ; x \in X \setminus \{x_0\}} \dfrac{f(x_0) g(x) - f(x_0) g(x_0)}{x - x_0}\bigg)         \\
    = & \lim_{x \to x_0 ; x \in X \setminus \{x_0\}} \bigg(\dfrac{f(x) g(x) - f(x_0) g(x)}{x - x_0} + \dfrac{f(x_0) g(x) - f(x_0) g(x_0)}{x - x_0}\bigg)                                                                  \\
    = & \lim_{x \to x_0 ; x \in X \setminus \{x_0\}} \dfrac{f(x) g(x) - f(x_0) g(x) + f(x_0) g(x) - f(x_0) g(x_0)}{x - x_0}                                                                                               \\
    = & \lim_{x \to x_0 ; x \in X \setminus \{x_0\}} \dfrac{f(x) g(x) - f(x_0) g(x_0)}{x - x_0}                                                                                                                           \\
    = & \lim_{x \to x_0 ; x \in X \setminus \{x_0\}} \dfrac{(fg)(x) - (fg)(x_0)}{x - x_0}                                                                                                                                 \\
    = & (fg)'(x_0).
  \end{align*}
  Thus \(fg\) is differentiable at \(x_0\) and \((fg)'(x_0) = f'(x_0) g(x_0) + f(x_0) g'(x_0)\).
\end{proof}

\begin{proof}{(e)}
  By \cref{10.1.1} and \cref{9.3.14} we have
  \begin{align*}
      & cf'(x_0)                                                                                  \\
    = & c \bigg(\lim_{x \to x_0 ; x \in X \setminus \{x_0\}} \dfrac{f(x) - f(x_0)}{x - x_0}\bigg) \\
    = & \lim_{x \to x_0 ; x \in X \setminus \{x_0\}} \bigg(c \dfrac{f(x) - f(x_0)}{x - x_0}\bigg) \\
    = & \lim_{x \to x_0 ; x \in X \setminus \{x_0\}} \dfrac{cf(x) - cf(x_0)}{x - x_0}             \\
    = & \lim_{x \to x_0 ; x \in X \setminus \{x_0\}} \dfrac{(cf)(x) - (cf)(x_0)}{x - x_0}         \\
    = & (cf)'(x_0).
  \end{align*}
  Thus \(cf\) is differentiable at \(x_0\) and \((cf)'(x_0) = cf'(x_0)\).
\end{proof}

\begin{proof}{(f)}
  \begin{align*}
      & f'(x_0) - g'(x_0)                                                 \\
    = & f'(x_0) + \big(-g'(x_0)\big)                                      \\
    = & f'(x_0) + \big((-g)'(x_0)\big) &  & \text{(by \cref{10.1.13}(e))} \\
    = & \big(f + (-g)\big)'(x_0)       &  & \text{(by \cref{10.1.13}(c))} \\
    = & (f - g)'(x_0).                 &  & \by{9.2.1}
  \end{align*}
  Thus \(f - g\) is differentiable at \(x_0\) and \((f - g)'(x_0) = f'(x_0) - g'(x_0)\).
\end{proof}

\begin{proof}{(g)}
  By \cref{10.1.1} and \cref{9.3.14} we have
  \begin{align*}
      & -\dfrac{g'(x_0)}{g(x_0)^2}                                                                                                                                                      \\
    = & \bigg(\lim_{x \to x_0 ; x \in X \setminus \{x_0\}} \dfrac{g(x) - g(x_0)}{x - x_0}\bigg) \bigg(\dfrac{-1}{g(x_0)^2}\bigg)                                                        \\
    = & \bigg(\lim_{x \to x_0 ; x \in X \setminus \{x_0\}} \dfrac{g(x) - g(x_0)}{x - x_0}\bigg) \bigg(\dfrac{-g(x_0)}{g(x_0) g(x_0)^2}\bigg)                                            \\
    = & \bigg(\lim_{x \to x_0 ; x \in X \setminus \{x_0\}} \dfrac{g(x) - g(x_0)}{x - x_0}\bigg) \bigg(\lim_{x \to x_0 ; x \in X \setminus \{x_0\}} \dfrac{-g(x_0)}{g(x) g(x_0)^2}\bigg) \\
    = & \lim_{x \to x_0 ; x \in X \setminus \{x_0\}} \Bigg(\bigg(\dfrac{g(x) - g(x_0)}{x - x_0}\bigg) \bigg(\dfrac{-g(x_0)}{g(x) g(x_0)^2}\bigg)\Bigg)                                  \\
    = & \lim_{x \to x_0 ; x \in X \setminus \{x_0\}} \dfrac{\dfrac{g(x_0)(g(x_0) - g(x))}{g(x) g(x_0)^2}}{x - x_0}                                                                      \\
    = & \lim_{x \to x_0 ; x \in X \setminus \{x_0\}} \dfrac{\dfrac{g(x_0) - g(x)}{g(x) g(x_0)}}{x - x_0}                                                                                \\
    = & \lim_{x \to x_0 ; x \in X \setminus \{x_0\}} \dfrac{\dfrac{1}{g(x)} - \dfrac{1}{g(x_0)}}{x - x_0}                                                                               \\
    = & (\dfrac{1}{g})'(x_0).
  \end{align*}
  Thus \(1 / g\) is differentiable at \(x_0\) and \((1 / g)'(x_0) = -\dfrac{g'(x_0)}{g(x_0)^2}\).
\end{proof}

\begin{proof}{(h)}
  \begin{align*}
      & (\dfrac{f}{g})'(x_0)                                                                                   \\
    = & (f \cdot \dfrac{1}{g})'(x_0)                                        &  & \by{9.2.1}                    \\
    = & f'(x_0) \dfrac{1}{g}(x_0) + f(x_0) (\dfrac{1}{g})'(x_0)             &  & \text{(by \cref{10.1.13}(d))} \\
    = & \dfrac{f'(x_0)}{g(x_0)} + f(x_0) (\dfrac{1}{g})'(x_0)               &  & \by{9.2.1}                    \\
    = & \dfrac{f'(x_0)}{g(x_0)} + f(x_0) \dfrac{-g'(x_0)}{g(x_0)^2}         &  & \text{(by \cref{10.1.13}(g))} \\
    = & \dfrac{f'(x_0) g(x_0)}{g(x_0)^2} - \dfrac{f(x_0) g'(x_0)}{g(x_0)^2}                                    \\
    = & \dfrac{f'(x_0) g(x_0) - f(x_0) g'(x_0)}{g(x_0)^2}.                                                     \\
  \end{align*}
  Thus \(f / g\) is differentiable at \(x_0\) and \((f / g)'(x_0) = \dfrac{f'(x_0) g(x_0) - f(x_0) g'(x_0)}{g(x_0)^2}\).
\end{proof}

\begin{rmk}\label{10.1.14}
  The product rule is also known as the \emph{Leibniz rule}, after Gottfried Leibniz (1646 -- 1716), who was the other founder of differential and integral calculus besides Newton.
\end{rmk}

\begin{note}
  The trick of adding and subtracting an intermediate term is sometimes known as the ``middle-man trick'' and is very useful in analysis.
\end{note}

\begin{thm}[Chain rule]\label{10.1.15}
  Let \(X, Y\) be subsets of \(\R\), let \(x_0 \in X\) be a limit point of \(X\), and let \(y_0 \in Y\) be a limit point of \(Y\).
  Let \(f : X \to Y\) be a function such that \(f(x_0) = y_0\), and such that \(f\) is differentiable at \(x_0\).
  Suppose that \(g : Y \to \R\) is a function which is differentiable at \(y_0\).
  Then the function \(g \circ f : X \to \R\) is differentiable at \(x_0\), and
  \[
    (g \circ f)'(x_0) = g'(y_0) f'(x_0)
  \]
\end{thm}

\begin{proof}
  By \cref{10.1.7} we want to show that
  \begin{align*}
             & \forall \varepsilon \in \R^+, \exists\ \delta \in \R^+ : \forall x \in X, \abs{x - x_0} \leq \delta \\
    \implies & \abs{g\big(f(x)\big) - g(y_0) - g'(y_0) f'(x_0) (x - x_0)} \leq \varepsilon \abs{x - x_0}.
  \end{align*}
  Since \(g'(y_0)\) exists, by \cref{10.1.7} we have
  \begin{align*}
             & \forall \varepsilon \in \R^+, \exists\ \delta_g \in \R^+ : \forall y \in Y, \abs{y - y_0} \leq \delta_g \\
    \implies & \abs{g(y) - g(y_0) - g'(y_0) (y - y_0)} \leq \varepsilon_g \abs{y - y_0}
  \end{align*}
  where
  \[
    \varepsilon_g = \dfrac{-\big(\abs{f'(x_0)} + \abs{g'(y_0)}\big) + \sqrt{\big(\abs{f'(x_0)} + \abs{g'(y_0)}\big)^2 + 4 \varepsilon}}{2}.
  \]
  Note that \(\varepsilon_g \in \R^+\) since
  \[
    \sqrt{\big(\abs{f'(x_0)} + \abs{g'(y_0)}\big)^2 + 4 \varepsilon} > \sqrt{\big(\abs{f'(x_0)} + \abs{g'(y_0)}\big)^2} = \abs{f'(x_0)} + \abs{g'(y_0)}.
  \]
  Now fix such \(\varepsilon\) (and \(\varepsilon_g\)) and \(\delta_g\).
  Since \(f'(x_0)\) exists, by \cref{10.1.12} \(f\) is continuous at \(x_0\) and by \cref{9.4.1} \(\lim_{x \to x_0 ; x \in X} f(x) = f(x_0)\).
  This means
  \begin{align*}
             & \exists\ \delta \in \R^+ : \forall x \in X, \abs{x - x_0} \leq \delta                \\
    \implies & \begin{dcases}
                 \abs{f(x) - y_0} \leq \delta_g \\
                 \abs{f(x) - y_0 - f'(x_0) (x - x_0)} \leq \varepsilon_g \abs{x - x_0}
               \end{dcases}                 &  & \text{(by \cref{9.4.7}(d) and \cref{10.1.7})}      \\
    \implies & \begin{dcases}
                 \abs{f(x) - y_0} \leq \delta_g                                        \\
                 \abs{f(x) - y_0 - f'(x_0) (x - x_0)} \leq \varepsilon_g \abs{x - x_0} \\
                 \abs{f(x) - y_0} \leq \varepsilon_g \abs{x - x_0} + \abs{f'(x_0) (x - x_0)}
               \end{dcases}           \\
    \implies & \begin{dcases}
                 \abs{f(x) - y_0} \leq \delta_g                                        \\
                 \abs{f(x) - y_0 - f'(x_0) (x - x_0)} \leq \varepsilon_g \abs{x - x_0} \\
                 \abs{f(x) - y_0} \leq \big(\varepsilon_g + \abs{f'(x_0)}\big) \abs{x - x_0}
               \end{dcases}
  \end{align*}
  We know that
  \begin{align*}
             & \exists\ \delta \in \R^+ : \forall x \in X, \abs{x - x_0} \leq \delta                                                \\
    \implies & \abs{f(x) - y_0} \leq \delta_g                                                                                       \\
    \implies & \abs{g\big(f(x)\big) - g(y_0) - g'(y_0) \big(f(x_0) - y_0\big)} \leq \varepsilon_g \abs{f(x) - y_0}                  \\
    \implies & \abs{g\big(f(x)\big) - g(y_0) - g'(y_0) f'(x_0) (x - x_0) - g'(y_0) \big(f(x) - y_0 - f'(x_0) (x - x_0)\big)}        \\
             & \leq \varepsilon_g \abs{f(x) - y_0}                                                                                  \\
    \implies & \abs{g\big(f(x)\big) - g(y_0) - g'(y_0) f'(x_0) (x - x_0)}                                                           \\
             & \leq \varepsilon_g \abs{f(x) - y_0} + \abs{g'(y_0) \big(f(x) - y_0 - f'(x_0) (x - x_0)\big)}                         \\
    \implies & \abs{g\big(f(x)\big) - g(y_0) - g'(y_0) f'(x_0) (x - x_0)}                                                           \\
             & \leq \varepsilon_g \abs{f(x) - y_0} + \abs{g'(y_0)} \abs{f(x_0) - y_0 - f'(x_0) (x - x_0)}                           \\
    \implies & \abs{g\big(f(x)\big) - g(y_0) - g'(y_0) f'(x_0) (x - x_0)}                                                           \\
             & \leq \varepsilon_g \abs{f(x) - y_0} + \varepsilon_g \abs{g'(y_0)} \abs{x - x_0}                                      \\
    \implies & \abs{g\big(f(x)\big) - g(y_0) - g'(y_0) f'(x_0) (x - x_0)}                                                           \\
             & \leq \varepsilon_g \big(\varepsilon_g + \abs{f'(x_0)}\big) \abs{x - x_0} + \varepsilon_g \abs{g'(y_0)} \abs{x - x_0} \\
    \implies & \abs{g\big(f(x)\big) - g(y_0) - g'(y_0) f'(x_0) (x - x_0)}                                                           \\
             & \leq \varepsilon_g \big(\varepsilon_g + \abs{f'(x_0)} + \abs{g'(y_0)}\big) \abs{x - x_0}.
  \end{align*}
  Expanding \(\varepsilon_g\) we have
  \begin{align*}
     & \varepsilon_g \big(\varepsilon_g + \abs{f'(x_0)} + \abs{g'(y_0)}\big)                                                              \\
     & = \dfrac{-\big(\abs{f'(x_0)} + \abs{g'(y_0)}\big) + \sqrt{\big(\abs{f'(x_0)} + \abs{g'(y_0)}\big)^2 + 4 \varepsilon}}{2}           \\
     & \quad \times \dfrac{\big(\abs{f'(x_0)} + \abs{g'(y_0)}\big) + \sqrt{\big(\abs{f'(x_0)} + \abs{g'(y_0)}\big)^2 + 4 \varepsilon}}{2} \\
     & = \dfrac{\big(\abs{f'(x_0)} + \abs{g'(y_0)}\big)^2 + 4 \varepsilon - \big(\abs{f'(x_0)} + \abs{g'(y_0)}\big)^2}{4}                 \\
     & = \varepsilon.
  \end{align*}
  Thus we conclude that
  \begin{align*}
             & \forall \varepsilon \in \R^+, \exists\ \delta \in \R^+ : \forall x \in X, \abs{x - x_0} \leq \delta \\
    \implies & \abs{g\big(f(x)\big) - g(y_0) - g'(y_0) f'(x_0) (x - x_0)} \leq \varepsilon \abs{x - x_0}
  \end{align*}
  and by \cref{10.1.7} we have \((g \circ f)'(x_0) = g'(y_0) f'(x_0)\).
\end{proof}

\setcounter{thm}{16}
\begin{rmk}\label{10.1.17}
  If one writes \(y\) for \(f(x)\), and \(z\) for \(g(y)\), then the chain rule can be written in the more visually appealing manner \(\dfrac{dz}{dx} = \dfrac{dz}{dy} \dfrac{dy}{dx}\).
  However, this notation can be misleading (for instance it blurs the distinction between dependent variable and independent variable, especially for \(y\)), and leads one to believe that the quantities \(dz, dy, dx\) can be manipulated like real numbers.
  However, these quantities are not real numbers (in fact, we have not assigned any meaning to them at all), and treating them as such can lead to problems in the future.
  For instance, if \(f\) depends on \(x_1\) and \(x_2\), which depend on \(t\), then chain rule for several variables asserts that \(\dfrac{df}{dt} = \dfrac{\partial f}{\partial x_1} \dfrac{dx_1}{dt} + \dfrac{\partial f}{\partial x_2} \dfrac{dx_2}{dt}\), but this rule might seem suspect if one treated \(df, dt\), etc. as real numbers.
  It is possible to think of \(dy, dx\), etc. as ``infinitesimal real numbers'' if one knows what one is doing, but for those just starting out in analysis, I would not recommend this approach, especially if one wishes to work rigorously.
  (There is a way to make all of this rigorous, even for the calculus of several variables, but it requires the notion of a tangent vector, and the derivative map, both of which are beyond the scope of this text.)
\end{rmk}

\exercisesection

\begin{ex}\label{ex:10.1.1}
  Suppose that \(X\) is a subset of \(\R\), \(x_0\) is a limit point of \(X\), and \(f : X \to \R\) is a function which is differentiable at \(x_0\).
  Let \(Y \subseteq X\) be such that \(x_0 \in Y\), and \(x_0\) is also a limit point of \(Y\).
  Prove that the restricted function \(f|_Y : Y \to \R\) is also differentiable at \(x_0\), and has the same derivative as \(f\) at \(x_0\).
  Explain why this does not contradict the discussion in \cref{10.1.2}.
\end{ex}

\begin{proof}
  Since \(f\) is differentiable at \(x_0\), by Newton's approximation (\cref{10.1.7}) we have \(\forall \varepsilon \in \R^+\), \(\exists\ \delta \in \R^+\) such that
  \[
    \forall x \in X, \abs{x - x_0} \leq \delta \implies \abs{f(x) - (f(x_0) + f'(x_0)(x - x_0))} \leq \varepsilon \abs{x - x_0}.
  \]
  Since \(Y \subseteq X\), we have
  \begin{align*}
             & \forall x \in Y, \abs{x - x_0} \leq \delta                                             \\
    \implies & (x \in X) \land (\abs{x - x_0} \leq \delta)                                            \\
    \implies & \abs{f(x) - \big(f(x_0) + f'(x_0)(x - x_0)\big)} \leq \varepsilon \abs{x - x_0}        \\
    \implies & \abs{f|_Y(x) - \big(f|_Y(x_0) + f'(x_0)(x - x_0)\big)} \leq \varepsilon \abs{x - x_0}.
  \end{align*}
  Thus by Newton's approximation (\cref{10.1.7}) we know that \(f|_Y'(x_0) = f'(x_0)\).
  This does not contradict to \cref{10.1.2} since \(x_0\) is a limit point of \(Y\) implies \(x_0\) is also a limit point of \(X\).
\end{proof}

\begin{ex}\label{ex:10.1.2}
  Prove \cref{10.1.7}.
\end{ex}

\begin{proof}
  See \cref{10.1.7}.
\end{proof}

\begin{ex}\label{ex:10.1.3}
  Prove \cref{10.1.10}.
\end{ex}

\begin{proof}
  See \cref{10.1.10}.
\end{proof}

\begin{ex}\label{ex:10.1.4}
  Prove \cref{10.1.13}.
\end{ex}

\begin{proof}
  See \cref{10.1.13}.
\end{proof}

\begin{ex}\label{ex:10.1.5}
  Let \(n\) be a natural number, and let \(f : \R \to \R\) be the function \(f(x) \coloneqq x^n\).
  Show that \(f\) is differentiable on \(\R\) and \(f'(x) = n x^{n - 1}\) for all \(x \in \R\) with the convention that \(n x^{n - 1} = 0\) when \(n = 0\).
\end{ex}

\begin{proof}
  We use induction on \(n\) to show that \(\forall n \in \N\), \(f_n(x) = x^n\) is differentiable on \(\R\) and \(f_n'(x) = n x^{n - 1}\).
  For \(n = 0\), we have \(f_0(x) = x^0 = 1\).
  By \cref{10.1.13}(a) we know that \(f_0\) is differentiable on \(\R\) and \(f_0'(x) = 0\) for every \(x \in X\).
  Thus (by convention) the base case holds.
  Suppose inductively that for some \(n \geq 0\) we have \(f_n(x) = x^n\) is differentiable on \(\R\) and \(f_n'(x) = n x^{n - 1}\).
  Then for \(n + 1\), we have \(f_{n + 1}(x) = x^{n + 1} = x^n \cdot x^1 = f_n(x) f_1(x) = (f_n \cdot f_1)(x)\) and
  \begin{align*}
      & (f_n \cdot f_1)'(x)                                                    \\
    = & f_n'(x) f_1(x) + f_n(x) f_1'(x)     &  & \text{(by \cref{10.1.13}(d))} \\
    = & (n x^{n - 1})(x^1) + f_n(x) f_1'(x) &  & \byIH                         \\
    = & (n x^{n - 1})(x^1) + (x^n)(1 x^0)   &  & \text{(by \cref{10.1.13}(b))} \\
    = & n x^n + x^n                                                            \\
    = & (n + 1) x^n.
  \end{align*}
  This closes the induction.
\end{proof}

\begin{ex}\label{ex:10.1.6}
  Let \(n\) be a \emph{negative} integer, and let \(f : \R \setminus \{0\} \to \R\) be the function \(f(x) \coloneqq x^n\).
  Show that \(f\) is differentiable on \(\R \setminus \{0\}\) and \(f'(x) = n x^{n - 1}\) for all \(x \in \R \setminus \{0\}\).
\end{ex}

\begin{proof}
  Let \(x \in \R \setminus \{0\}\).
  Since \(n \in \Z^-\), \(-n \in \Z^+\).
  Then we have \((1 / f)(x) = 1 / x^n = x^{-n}\) and \(f(x) = \big(1 / (1 / f)\big)(x)\).
  Thus \(f\) is differentiable at \(x\) and
  \begin{align*}
    f'(x) & = \big(1 / (1 / f)\big)'(x)                                                       \\
          & = -\dfrac{(1 / f)'(x)}{\big((1 / f)(x)\big)^2} &  & \text{(by \cref{10.1.13}(g))} \\
          & = -\dfrac{(-n) x^{-n - 1}}{x^{-2n}}            &  & \text{(by \cref{ex:10.1.5})}  \\
          & = n x^{n - 1}.
  \end{align*}
\end{proof}

\begin{ex}\label{ex:10.1.7}
  Prove \cref{10.1.15}.
\end{ex}

\begin{proof}
  See \cref{10.1.15}.
\end{proof}
\section{Local maxima, local minima, and derivatives}\label{sec:10.2}

\begin{defn}[Local maxima and minima]\label{10.2.1}
  Let \(X\) be a subset of \(\R\), and let \(f : X \to \R\) be a function, and let \(x_0 \in X\).
  We say that \(f\) attains a \emph{local maximum} at \(x_0\) iff there exists a \(\delta > 0\) such that the restriction \(f|_{X \cap (x_0 - \delta, x_0 + \delta)}\) of \(f\) to \(X \cap (x_0 - \delta, x_0 + \delta)\) attains a maximum at \(x_0\).
  We say that \(f\) attains a \emph{local minimum} at \(x_0\) iff there exists a \(\delta > 0\) such that the restriction \(f|_{X \cap (x_0 - \delta, x_0 + \delta)}\) of \(f\) to \(X \cap (x_0 - \delta, x_0 + \delta)\) attains a minimum at \(x_0\).
\end{defn}

\begin{rmk}\label{10.2.2}
  If \(f\) attains a maximum at \(x_0\), we sometimes say that \(f\) attains a \emph{global} maximum at \(x_0\), in order to distinguish it from the local maxima defined in \cref{10.2.1}.
  Note that if \(f\) attains a global maximum at \(x_0\), then it certainly also attains a local maximum at this \(x_0\), and similarly for minima.
\end{rmk}

\setcounter{thm}{4}
\begin{rmk}\label{10.2.5}
  If \(f : X \to \R\) attains a local maximum at a point \(x_0\) in \(X\), and \(Y \subseteq X\) is a subset of \(X\) which contains \(x_0\), then the restriction \(f|_Y : Y \to \R\) also attains a local maximum at \(x_0\).
  Similarly for minima.
\end{rmk}

\begin{prop}[Local extrema are stationary]\label{10.2.6}
  Let \(a < b\) be real numbers, and let \(f : (a, b) \to \R\) be a function.
  If \(x_0 \in (a, b)\), \(f\) is differentiable at \(x_0\), and \(f\) attains either a local maximum or local minimum at \(x_0\), then \(f'(x_0) = 0\).
\end{prop}

\begin{proof}
  Suppose \(f\) attains local maximum at \(x_0\).
  Then by \cref{10.2.1} we know that
  \[
    \exists\ \delta \in \R^+ : \forall x \in (a, b) \cap (x_0 - \delta, x_0 + \delta), f(x) \leq f(x_0).
  \]
  Since \(f\) is differentiable at \(x_0\), by \cref{10.1.1} we know that
  \[
    \lim_{x \to x_0 ; x \in (a, b) \setminus \{x_0\}} \frac{f(x) - f(x_0)}{x - x_0} = f'(x_0)
  \]
  By \cref{9.3.6} we know that \(\forall \varepsilon \in \R^+\), \(\exists\ \delta' \in \R^+\) such that
  \[
    \forall x \in (a, b) \setminus \{x_0\}, \abs{x - x_0} < \delta' \implies \abs{\frac{f(x) - f(x_0)}{x - x_0} - f'(x_0)} \leq \varepsilon.
  \]
  In particular, we have
  \[
    \forall x \in (a, b) \cap (x_0, x_0 + \delta), \abs{x - x_0} < \delta' \implies \abs{\frac{f(x) - f(x_0)}{x - x_0} - f'(x_0)} \leq \varepsilon
  \]
  and
  \[
    \forall x \in (a, b) \cap (x_0 - \delta, x_0), \abs{x - x_0} < \delta' \implies \abs{\frac{f(x) - f(x_0)}{x - x_0} - f'(x_0)} \leq \varepsilon.
  \]
  Thus by \cref{9.3.6} we must have
  \[
    \lim_{x \to x_0 ; x \in (a, b) \cap (x_0, x_0 + \delta)} \frac{f(x) - f(x_0)}{x - x_0} = \lim_{x \to x_0 ; x \in (a, b) \cap (x_0 - \delta, x_0)} \frac{f(x) - f(x_0)}{x - x_0} = f'(x_0).
  \]
  Since
  \begin{align*}
             & \forall x \in (a, b) \cap (x_0, x_0 + \delta)                                                                             \\
    \implies & f(x) \leq f(x_0)                                                                                                          \\
    \implies & f(x) - f(x_0) \leq 0                                                                                                      \\
    \implies & \frac{f(x) - f(x_0)}{x - x_0} \leq 0                                                                                      \\
    \implies & \lim_{x \to x_0 ; x \in (a, b) \cap (x_0, x_0 + \delta)} \frac{f(x) - f(x_0)}{x - x_0} \leq 0 & \text{(by \cref{9.3.14})} \\
    \implies & f'(x_0) \leq 0
  \end{align*}
  and
  \begin{align*}
             & \forall x \in (a, b) \cap (x_0 - \delta, x_0)                                                                              \\
    \implies & f(x) \leq f(x_0)                                                                                                           \\
    \implies & f(x) - f(x_0) \leq 0                                                                                                       \\
    \implies & \frac{f(x) - f(x_0)}{x - x_0} \geq 0                                                                                       \\
    \implies & \lim_{x \to x_0 ; x \in (a, b) \cap (x_0 - \delta, x_0)} \frac{f(x) - f(x_0)}{x - x_0} \geq 0, & \text{(by \cref{9.3.14})} \\
    \implies & f'(x_0) \geq 0,
  \end{align*}
  we must have \(f'(x_0) = 0\).
  Similar arguments work for the case \(f\) attains local minimum at \(x_0\).
\end{proof}

\begin{note}
  \(f\) must be differentiable for \cref{10.2.6} to work.
  Also, \cref{10.2.6} does not work if the open interval \((a, b)\) is replaced by a closed interval \([a, b]\).
  For instance, the function \(f : [1, 2] \to \R\) defined by \(f(x) \coloneqq x\) has a local maximum at \(x_0 = 2\) and a local minimum \(x_0 = 1\) (in fact, these local extrema are global extrema), but at both points the derivative is \(f'(x_0) = 1\), not \(f'(x_0) = 0\).
  Thus the endpoints of an interval can be local maxima or minima even if the derivative is not zero there.
  Finally, the converse of this proposition is false.
\end{note}

\begin{thm}[Rolle's theorem]\label{10.2.7}
  Let \(a < b\) be real numbers, and let \(g : [a, b] \to \R\) be a continuous function which is differentiable on \((a, b)\).
  Suppose also that \(g(a) = g(b)\).
  Then there exists an \(x \in (a, b)\) such that \(g'(x) = 0\).
\end{thm}

\begin{proof}
  Since \(g\) is continuous on \([a, b]\), by \cref{9.6.7} \(g\) attains its maximum at some point \(x_{\max} \in [a, b]\), and also attains its minimum at some point \(x_{\min} \in [a, b]\).
  If \((x_{\min} \in \{a, b\}) \land (x_{\max} \in \{a, b\})\) is true, then by \cref{9.6.5} we have \(g(x) = g(a) = g(b)\) for every \(x \in [a, b]\), and by \cref{10.1.13}(a) we know that \(g'(x) = 0\).
  So suppose that at least one of \(x_{\min}, x_{\max} \notin \{a, b\}\), i.e., \(\big(x_{\min} \in (a, b)\big) \lor \big(x_{\max} \in (a, b)\big)\) is true.
  If \(x_{\min} \in (a, b)\), then by \cref{10.2.6} we know that \(f'(x_{\min}) = 0\).
  Similarly, if \(x_{\max} \in (a, b)\), then by \cref{10.2.6} we know that \(f'(x_{\max}) = 0\).
  Thus there exists an \(x \in (a, b)\) such that \(g'(x) = 0\).
\end{proof}

\begin{rmk}\label{10.2.8}
  We only assume \(f\) is differentiable on the open interval \((a, b)\), though of course \cref{10.2.7} also holds if we assume \(f\) is differentiable on the closed interval \([a, b]\), since this is larger than \((a, b)\).
\end{rmk}

\begin{cor}[Mean value theorem]\label{10.2.9}
  Let \(a < b\) be real numbers, and let \(f : [a, b] \to \R\) be a function which is continuous on \([a, b]\) and differentiable on \((a, b)\).
  Then there exists an \(x \in (a, b)\) such that \(f'(x) = \frac{f(b) - f(a)}{b - a}\).
\end{cor}

\begin{proof}
  Let \(g : [a, b] \to \R\) be a function where \(g(x) = f(x) - \frac{f(a) - f(b)}{a - b} x\).
  Since \(a < b\), we know that \(g\) is well-defined.
  Since \(f\) is differentiable on \((a, b)\), we know that
  \begin{align*}
             & x \text{ is differentiable on } (a, b)                                  & \text{(by \cref{10.1.13}(b))} \\
    \implies & \frac{f(a) - f(b)}{a - b} x \text{ is differentiable on } (a, b)        & \text{(by \cref{10.1.13}(e))} \\
    \implies & f(x) - \frac{f(a) - f(b)}{a - b} x \text{ is differentiable on } (a, b) & \text{(by \cref{10.1.13}(f))} \\
    \implies & g(x) \text{ is differentiable on } (a, b)                                                               \\
    \implies & g'(x) = f'(x) - \frac{f(a) - f(b)}{a - b}.
  \end{align*}
  Since
  \[
    g(a) = f(a) - \frac{f(a) - f(b)}{a - b} a = \frac{af(a) - bf(a) - af(a) + af(b)}{a - b} = \frac{af(b) - bf(a)}{a - b}
  \]
  and
  \[
    g(b) = f(b) - \frac{f(a) - f(b)}{a - b} b = \frac{af(b) - bf(b) - bf(a) + bf(b)}{a - b} = \frac{af(b) - bf(a)}{a - b},
  \]
  we have \(g(a) = g(b)\) and by \cref{10.2.7} \(\exists\ x_0 \in (a, b)\) such that \(g'(x_0) = 0\).
  Thus
  \begin{align*}
             & g'(x_0) = 0                             \\
    \implies & f'(x_0) - \frac{f(a) - f(b)}{a - b} = 0 \\
    \implies & f'(x_0) = \frac{f(a) - f(b)}{a - b}.
  \end{align*}
\end{proof}

\exercisesection

\begin{ex}\label{ex:10.2.1}
  Prove \cref{10.2.6}.
\end{ex}

\begin{proof}
  See \cref{10.2.6}.
\end{proof}

\begin{ex}\label{ex:10.2.2}
  Give an example of a function \(f : (-1, 1) \to \R\) which is continuous and attains a global maximum at \(0\), but which is not differentiable at \(0\).
  Explain why this does not contradict \cref{10.2.6}.
\end{ex}

\begin{proof}
  Let \(f(x) = -\abs{x}\).
  Then \(f\) is continuous and attains a global maximum at \(0\), but which is not differentiable at \(0\).
  The fact that \(f\) is not differentiable at \(0\) does not contradict to \cref{10.2.6}.
\end{proof}

\begin{ex}\label{ex:10.2.3}
  Give an example of a function \(f : (-1, 1) \to \R\) which is differentiable, and whose derivative equals \(0\) at \(0\), but such that \(0\) is neither a local minimum nor a local maximum.
  Explain why this does not contradict \cref{10.2.6}.
\end{ex}

\begin{proof}
  Let \(f(x) = x^3\).
  Then by \cref{ex:10.1.5} we know that \(f'(x) = 3x^2\).
  Then we have \(f(0) = 0\) and \(f'(0) = 0\).
  But \(f(x) < f(0)\) for every \(x \in (-1, 0)\) and \(f(x) > f(0)\) for every \(x \in (0, 1)\).
  Thus \(0\) is neither a local minimum nor a local maximum.
  This does not contradict to \cref{10.2.6} since \(0\) is not given to be a local minimum or local maximum.
\end{proof}

\begin{ex}\label{ex:10.2.4}
  Prove \cref{10.2.7}.
\end{ex}

\begin{proof}
  See \cref{10.2.7}.
\end{proof}

\begin{ex}\label{ex:10.2.5}
  Use \cref{10.2.7} to prove \cref{10.2.9}.
\end{ex}

\begin{proof}
  See \cref{10.2.9}.
\end{proof}

\begin{ex}\label{ex:10.2.6}
  Let \(M > 0\), and let \(f : [a, b] \to \R\) be a function which is continuous on \([a, b]\) and differentiable on \((a, b)\), and such that \(\abs{f'(x)} \leq M\) for all \(x \in (a, b)\) (i.e., the derivative of \(f\) is bounded).
  Show that for any \(x, y \in [a, b]\) we have the inequality \(\abs{f(x) - f(y)} \leq M \abs{x - y}\).
  Functions which obey the bound \(\abs{f(x) - f(y)} \leq M \abs{x - y}\) are known as \emph{Lipschitz continuous functions} with \emph{Lipschitz constant} \(M\);
  thus this exercise shows that functions with bounded derivative are Lipschitz continuous.
\end{ex}

\begin{proof}
  Let \(x, y \in [a, b]\).
  If \(x = y\), then we have \(0 = \abs{f(x) - f(y)} \leq M \abs{x - y} = 0\).
  So suppose that \(x \neq y\).
  We have either \(x < y\) or \(x > y\).
  Without the loss of generality suppose that \(x < y\).
  Then we have \([x, y] \subseteq [a, b]\) and \((x, y) \subseteq (a, b)\).
  By \cref{ex:9.4.6} we know that \(f|_{[x, y]}\) is continuous on \([x, y]\).
  By \cref{ex:10.1.1} we know that \(f|_{[x, y]}\) is differentiable on \((x, y)\).
  By mean value theorem (\cref{10.2.9}) we know that \(\exists\ c \in (x, y)\) such that
  \[
    f'(c) = \frac{f(y) - f(x)}{y - x}.
  \]
  Since \(c \in (x, y)\), we have \(c \in (a, b)\).
  By hypothesis we have
  \begin{align*}
             & \abs{f'(c)} = \abs{\frac{f(y) - f(x)}{y - x}} \leq M \\
    \implies & \abs{f(y) - f(x)} \leq M \abs{y - x}                 \\
    \implies & \abs{f(x) - f(y)} \leq M \abs{x - y}.
  \end{align*}
  Thus we conclude that \(\forall x, y \in [a, b]\), we have \(\abs{f(x) - f(y)} \leq M \abs{x - y}\).
\end{proof}

\begin{ex}\label{ex:10.2.7}
  Let \(f : \R \to \R\) be a differentiable function such that \(f'\) is bounded.
  Show that \(f\) is uniformly continuous.
\end{ex}

\begin{proof}
  Since \(f'\) is bounded, by \cref{9.6.1} we know that \(\exists\ M \in \R^+\) such that \(\abs{f'(x)} \leq M\) for every \(x \in \R\).
  By \cref{ex:10.2.6} we know that \(\abs{f(x) - f(y)} \leq M \abs{x - y}\) for every \(x, y \in \R\).
  Then we have
  \begin{align*}
             & \forall \varepsilon \in \R^+, \exists\ \delta = \varepsilon / M : \forall x, y \in \R, \abs{x - y} \leq \delta \\
    \implies & \abs{f(x) - f(y)} \leq M \abs{x - y} \leq M \delta = M \frac{\varepsilon}{M} = \varepsilon
  \end{align*}
  and by \cref{9.9.2} \(f\) is uniformly continuous.
\end{proof}
\section{Monotone functions and derivatives}\label{i:sec:10.3}

\begin{prop}\label{i:10.3.1}
  Let \(X\) be a subset of \(\R\), let \(x_0 \in X\) be a limit point of \(X\), and let \(f : X \to \R\) be a function.
  If \(f\) is monotone increasing and \(f\) is differentiable at \(x_0\), then \(f'(x_0) \geq 0\).
  If f is monotone decreasing and \(f\) is differentiable at \(x_0\), then \(f'(x_0) \leq 0\).
\end{prop}

\begin{proof}
  First suppose that \(f\) is monotone increasing.
  Since
  \begin{align*}
             & \forall x \in X \setminus \set{x_0}, x \neq x_0    \\
    \implies & \begin{dcases}
                 x < x_0 \\
                 x > x_0 \\
               \end{dcases}                                      \\
    \implies & \begin{dcases}
                 (x - x_0 < 0) \land \big(f(x) - f(x_0) \leq 0\big) \\
                 (x - x_0 > 0) \land \big(f(x) - f(x_0) \geq 0\big) \\
               \end{dcases} &  & \by{i:9.8.1} \\
    \implies & \dfrac{f(x) - f(x_0)}{x - x_0} \geq 0,
  \end{align*}
  by \cref{i:9.3.14} we have \(f'(x_0) \geq 0\).

  Now suppose that \(f\) is monotone decreasing.
  Since
  \begin{align*}
             & \forall x \in X \setminus \set{x_0}, x \neq x_0    \\
    \implies & \begin{dcases}
                 x < x_0 \\
                 x > x_0 \\
               \end{dcases}                                      \\
    \implies & \begin{dcases}
                 (x - x_0 < 0) \land \big(f(x) - f(x_0) \geq 0\big) \\
                 (x - x_0 > 0) \land \big(f(x) - f(x_0) \leq 0\big) \\
               \end{dcases} &  & \by{i:9.8.1} \\
    \implies & \dfrac{f(x) - f(x_0)}{x - x_0} \leq 0,
  \end{align*}
  by \cref{i:9.3.14} we have \(f'(x_0) \leq 0\).
\end{proof}

\begin{rmk}\label{i:10.3.2}
  We have to assume that \(f\) is differentiable at \(x_0\);
  There exist monotone functions which are not always differentiable, and of course if \(f\) is not differentiable at \(x_0\) we cannot possibly conclude that \(f'(x_0) \geq 0\) or \(f'(x_0) \leq 0\).
\end{rmk}

\begin{note}
  One might naively guess that if \(f\) were strictly monotone increasing, and \(f\) was differentiable at \(x_0\), then the derivative \(f'(x_0)\) would be strictly positive instead of merely non-negative.
  Unfortunately, this is not always the case.
\end{note}

\begin{prop}\label{i:10.3.3}
  Let \(a < b\), and let \(f : [a, b] \to \R\) be a differentiable function.
  If \(f'(x) > 0\) for all \(x \in [a, b]\), then \(f\) is strictly monotone increasing.
  If \(f'(x) < 0\) for all \(x \in [a, b]\), then \(f\) is strictly monotone decreasing.
  If \(f'(x) = 0\) for all \(x \in [a, b]\), then \(f\) is a constant function.
\end{prop}

\begin{proof}
  We first show that if \(f'(x) > 0\) for all \(x \in [a, b]\), then \(f\) is strictly monotone increasing.
  Let \(x_1, x_2 \in [a, b]\) and \(x_1 < x_2\).
  Since \(f\) is differentiable on \([a, b]\), by \cref{i:ex:10.1.1} we know that \(f\) is differentiable on \((x_1, x_2)\), and by \cref{i:10.1.12} \(f\) is continuous on \([x_1, x_2]\).
  By mean value theorem (\cref{i:10.2.9}) \(\exists c \in (x_1, x_2)\) such that \(f'(c) = \dfrac{f(x_2) - f(x_1)}{x_2 - x_1}\).
  Since \(c \in (x_1, x_2)\), we have \(c \in [a, b]\).
  Now we split into three cases:
  \begin{itemize}
    \item If \(f'(x) > 0\) for all \(x \in [a, b]\), then we have
          \begin{align*}
                     & f'(c) > 0                                            \\
            \implies & \dfrac{f(x_2) - f(x_1)}{x_2 - x_1} > 0               \\
            \implies & f(x_2) - f(x_1) > 0                    & (x_2 > x_1) \\
            \implies & f(x_2) > f(x_1).
          \end{align*}
          Since \(x_1, x_2\) is arbitrary, by \cref{i:9.8.1} we conclude that \(f\) is strictly monotone increasing.
    \item If \(f'(x) < 0\) for all \(x \in [a, b]\), then we have
          \begin{align*}
                     & f'(c) < 0                                            \\
            \implies & \dfrac{f(x_2) - f(x_1)}{x_2 - x_1} < 0               \\
            \implies & f(x_2) - f(x_1) < 0                    & (x_2 > x_1) \\
            \implies & f(x_2) < f(x_1).
          \end{align*}
          Since \(x_1, x_2\) is arbitrary, by \cref{i:9.8.1} we conclude that \(f\) is strictly monotone decreasing.
    \item If \(f'(x) = 0\) for all \(x \in [a, b]\), then we have
          \begin{align*}
                     & f'(c) = 0                                            \\
            \implies & \dfrac{f(x_2) - f(x_1)}{x_2 - x_1} = 0               \\
            \implies & f(x_2) - f(x_1) = 0                    & (x_2 > x_1) \\
            \implies & f(x_2) = f(x_1).
          \end{align*}
          Since \(x_1, x_2\) is arbitrary, we conclude that \(f\) is a constant function.
  \end{itemize}
\end{proof}

\exercisesection

\begin{ex}\label{i:ex:10.3.1}
  Prove \cref{i:10.3.1}.
\end{ex}

\begin{proof}
  See \cref{i:10.3.1}.
\end{proof}

\begin{ex}\label{i:ex:10.3.2}
  Give an example of a function \(f : (-1, 1) \to \R\) which is continuous and monotone increasing, but which is not differentiable at \(0\).
  Explain why this does not contradict \cref{i:10.3.1}.
\end{ex}

\begin{proof}
  Define \(f\) as follow
  \[
    \forall x \in (-1, 1), f(x) = \begin{dcases}
      x  & \text{if } x \in (-1, 0), \\
      2x & \text{if } x \in [0, 1).
    \end{dcases}
  \]
  Then \(f\) is monotone increasing, \(f(0+) \geq 0\) and \(f(0-) < 0\).
  Since \(f(0+) \neq f(0-)\), by \cref{i:ac:9.5.1} \(f\) is not continuous at \(0\), and by \cref{i:10.1.10} \(f\) is not differentiable at \(0\).
  This does not contradict to \cref{i:10.3.1} since \(0\) is not given to be differentiable.
\end{proof}

\begin{ex}\label{i:ex:10.3.3}
  Give an example of a function \(f : \R \to \R\) which is strictly monotone increasing and differentiable, but whose derivative at \(0\) is zero.
  Explain why this does not contradict \cref{i:10.3.1} or \cref{i:10.3.3}.
\end{ex}

\begin{proof}
  Let \(f(x) = x^3\).
  By \cref{i:ex:10.1.5} \(f\) is differentiable on \(\R\) and \(f'(x) = 3x^2\), thus \(f'(0) = 0\).
  If \(x, y \in \R\) and \(x < y\), then \(x^3 < y^3\), thus by \cref{i:9.8.1} \(f\) is strictly monotone increasing.
  This does not contradict to \cref{i:10.3.1} since \(f'(0) = 0 \geq 0\).
  This does not contradict to \cref{i:10.3.3} since \(\forall x \in \R\), \(3x^2 \geq 0\).
\end{proof}

\begin{ex}\label{i:ex:10.3.4}
  Prove \cref{i:10.3.3}.
\end{ex}

\begin{proof}
  See \cref{i:10.3.3}.
\end{proof}

\begin{ex}\label{i:ex:10.3.5}
  Give an example of a subset \(X \subseteq \R\) and a function \(f : X \to \R\) which is differentiable on \(X\), is such that \(f'(x) > 0\) for all \(x \in X\), but \(f\) is not strictly monotone increasing.
\end{ex}

\begin{proof}
  Let \(X = [0, 0.5] \cup [1, 2]\) and let \(f : X \to \R\) be the following function
  \[
    \forall x \in X, f(x) = \begin{dcases}
      2x & \text{if } x \in [0, 0.5], \\
      x  & \text{if } x \in [1, 2].
    \end{dcases}
  \]
  By \cref{i:ex:10.1.5} we know that \(x\) and \(2x\) are differentiable and
  \[
    \forall x \in X, f'(x) = \begin{dcases}
      2 & \text{if } x \in [0, 0.5], \\
      1 & \text{if } x \in [1, 2].
    \end{dcases}
  \]
  So we have \(f'(x) > 0\) for every \(x \in X\).
  Since \(0.5 < 1\) and \(f(0.5) = f(1) = 1\), by \cref{i:9.8.1} \(f\) is not strictly monotone increasing.
\end{proof}

\section{Inverse functions and derivatives}\label{sec:10.4}

\begin{lem}\label{10.4.1}
  Let \(f : X \to Y\) be an invertible function, with inverse \(f^{-1} : Y \to X\).
  Suppose that \(x_0 \in X\) and \(y_0 \in Y\) are such that \(y_0 = f(x_0)\)
  (which also implies that \(x_0 = f^{-1}(y_0)\)).
  If \(f\) is differentiable at \(x_0\), and \(f^{-1}\) is differentiable at \(y_0\), then
  \[
    (f^{-1})'(y_0) = \frac{1}{f'(x_0)}.
  \]
\end{lem}

\begin{proof}
  From the chain rule (\cref{10.1.15}) we have
  \[
    (f^{-1} \circ f)'(x_0) = (f^{-1})'(y_0) f'(x_0).
  \]
  But \(f^{-1} \circ f\) is the identity function on \(X\), and hence by \cref{10.1.13}(b) \((f^{-1} \circ f)'(x_0) = 1\).
  The claim follows.
\end{proof}

\begin{note}
  As a particular corollary of \cref{10.4.1}, we see that if \(f\) is differentiable at \(x_0\) with \(f'(x_0) = 0\), then \(f^{-1}\) cannot be differentiable at \(y_0 = f(x_0)\), since \(1 / f'(x_0)\) is undefined in that case.
\end{note}

\begin{note}
  If one writes \(y = f(x)\), so that \(x = f^{-1}(y)\), then one can write the conclusion of \cref{10.4.1} in the more appealing form \(dx / dy = 1 / (dy / dx)\).
  However, as mentioned before, this way of writing things, while very convenient and easy to remember, can be misleading and cause errors if applied too carelessly (especially when one begins to work in the calculus of several variables).
\end{note}

\begin{note}
  \cref{10.4.1} seems to answer the question of how to differentiate the inverse of a function, however it has one significant drawback:
  the lemma only works if one assumes a \emph{priori} that \(f^{-1}\) is differentiable.
  Thus, if one does not already know that \(f^{-1}\) is differentiable, one cannot use \cref{10.4.1} to compute the derivative of \(f^{-1}\).
\end{note}

\begin{thm}[Inverse function theorem]\label{10.4.2}
  Let \(f : X \to Y\) be an invertible function, with inverse \(f^{-1} : Y \to X\).
  Suppose that \(x_0 \in X\) and \(y_0 \in Y\) are such that \(f(x_0) = y_0\).
  If \(f\) is differentiable at \(x_0\), \(f^{-1}\) is continuous at \(y_0\), and \(f'(x_0 ) \neq 0\), then \(f^{-1}\) is differentiable at \(y_0\) and
  \[
    (f^{-1})'(y_0) = \frac{1}{f'(x_0)}.
  \]
\end{thm}

\begin{proof}
  We have to show that
  \[
    \lim_{y \to y_0 ; y \in Y \setminus \{y_0\}} \frac{f^{-1}(y) - f^{-1}(y_0)}{y - y_0} = \frac{1}{f'(x_0)}.
  \]
  By \cref{9.3.9}, it suffices to show that
  \[
    \lim_{n \to \infty} \frac{f^{-1}(y_n) - f^{-1}(y_0)}{y_n - y_0} = \frac{1}{f'(x_0)}
  \]
  for any sequence \((y_n)_{n = 1}^\infty\) of elements in \(Y \setminus \{y_0\}\) which converge to \(y_0\).

  To prove this, we set \(x_n \coloneqq f^{-1}(y_n)\).
  Then \((x_n)_{n = 1}^\infty\) is a sequence of elements in \(X \setminus \{x_0\}\).
  (Note that \(f^{-1}\) is a bijection)
  Since \(f^{-1}\) is continuous by assumption, we know that \(x_n = f^{-1}(y_n)\) converges to \(f^{-1}(y_0) = x_0\) as \(n \to \infty\).
  Thus, since \(f\) is differentiable at \(x_0\), we have (by \cref{9.3.9} again)
  \[
    \lim_{n \to \infty} \frac{f(x_n) - f(x_0)}{x_n - x_0} = f'(x_0).
  \]
  But since \(x_n \neq x_0\) and \(f\) is a bijection, the fraction \(\frac{f(x_n) - f(x_0)}{x_n - x_0}\) is non-zero.
  Also, by hypothesis \(f(x_0)\) is non-zero.
  So by limit laws
  \[
    \lim_{n \to \infty} \frac{x_n - x_0}{f(x_n) - f(x_0)} = \frac{1}{f'(x_0)}.
  \]
  But since \(x_n = f^{-1}(y_n)\) and \(x_0 = f^{-1}(y_0)\), we thus have
  \[
    \lim_{n \to \infty} \frac{f^{-1}(y_n) - f^{-1}(y_0)}{y_n - y_0} = \frac{1}{f'(x_0)}.
  \]
  as desired.
\end{proof}

\exercisesection

\begin{ex}\label{ex:10.4.1}
  Let \(n \geq 1\) be a natural number, and let \(g : (0, \infty) \to (0, \infty)\) be the function \(g(x) \coloneqq x^{1 / n}\).
  \begin{enumerate}
    \item Show that \(g\) is continuous on \((0, \infty)\).
    \item Show that \(g\) is differentiable on \((0, \infty)\), and that \(g'(x) = \frac{1}{n} x^{\frac{1}{n} - 1}\) for all \(x \in (0, \infty)\).
  \end{enumerate}
\end{ex}

\begin{proof}
  We first show that \(g\) is continuous on \((0, \infty)\).
  Let \(f : (0, \infty) \to (0, \infty)\) be a function where \(f(x) = x^n\).
  Then \(g \circ f : (0, \infty) \to (0, \infty) = x\) and thus \(g = f^{-1}\).
  By \cref{ex:10.1.5} we know that \(x^n\) is differentiable on \((0, \infty)\), thus by \cref{10.1.12} \(f\) is continuous on \((0, \infty)\).
  By \cref{9.8.4} \(f\) is strictly monotone increasing.
  Then by \cref{9.8.3} we know that \(g\) is also continuous and strictly monotone increasing.

  Now we show that \(g\) is differentiable on \((0, \infty)\), and that \(g'(x) = \frac{1}{n} x^{\frac{1}{n} - 1}\) for all \(x \in (0, \infty)\).
  Since \(g\) is continuous, \(g = f^{-1}\), and \(f(x) \neq 0\) for every \(x \in (0, \infty)\), we have
  \begin{align*}
             & g(x) = f(x^{1 / n})                                                                                       \\
    \implies & g'(x) = \frac{1}{f'(x^{1 / n})}                                            & \text{(by \cref{10.4.2})}    \\
    \implies & g'(x) = \frac{1}{n (x^{1 / n})^{n - 1}} = \frac{1}{n} x^{\frac{1}{n} - 1}. & \text{(by \cref{ex:10.1.5})}
  \end{align*}
\end{proof}

\begin{ex}\label{ex:10.4.2}
  Let \(q\) be a rational number, and let \(f : (0, \infty) \to \R\) be the function \(f(x) = x^q\).
  \begin{enumerate}
    \item Show that \(f\) is differentiable on \((0, \infty)\) and that \(f'(x) = q x^{q - 1}\).
    \item Show that \(\lim_{x \to 1 ; x \in (0, \infty) \setminus \{1\}} \frac{x^q - 1}{x - 1} = q\) for every rational number \(q\).
  \end{enumerate}
\end{ex}

\begin{proof}
  We first show that \(f\) is differentiable on \((0, \infty)\) and that \(f'(x) = q x^{q - 1}\).
  Let \(q = a / b\) where \(a, b \in \Z\) and \(b > 0\).
  Then we have
  \[
    x^q = x^{a / b} = (x^a)^{1 / b}.
  \]
  By \cref{ex:10.1.5} and \cref{ex:10.1.6} we know that \(x^a\) is differentiable on \((0, \infty)\).
  By \cref{ex:10.4.1}(b) we know that \(x^{1 / b}\) is differentiable on \((0, \infty)\).
  Thus by chain rule (\cref{10.1.15}) we know that \(x^q = x^{a / b}\) is differentiable and
  \begin{align*}
    (x^q)' & = (x^{a / b})'                                                                     \\
           & = (\frac{1}{b} (x^a)^{\frac{1}{b} - 1}) (a x^{a - 1}) & \text{(by \cref{10.1.15})} \\
           & = \frac{a}{b} x^{\frac{a}{b} - 1}                                                  \\
           & = q x^{q - 1}.
  \end{align*}

  Now we show that \(\lim_{x \to 1 ; x \in (0, \infty) \setminus \{1\}} \frac{x^q - 1}{x - 1} = q\) for every \(q \in \Q\).
  Since \(x^q\) is differentiable on \((0, \infty)\), we know that \(x^q\) is differentiable at \(1\).
  Thus by \cref{10.1.1} we have
  \[
    \lim_{x \to 1 ; x \in (0, \infty) \setminus \{1\}} \frac{x^q - 1}{x - 1} = f'(1) = q \cdot 1^{q - 1} = q.
  \]
\end{proof}

\begin{ex}\label{ex:10.4.3}
  Let \(\alpha\) be a real number, and let \(f : (0, \infty) \to \R\) be the function \(f(x) = x^{\alpha}\).
  \begin{enumerate}
    \item Show that \(\lim_{x \to 1 ; x \in (0, \infty) \setminus \{1\}} \frac{f(x) - f(1)}{x - 1} = \alpha\).
    \item Show that \(f\) is differentiable on \((0, \infty)\) and that \(f'(x) = \alpha x^{\alpha - 1}\).
  \end{enumerate}
\end{ex}

\begin{proof}
  We first show that \(\lim_{x \to 1 ; x \in (0, \infty) \setminus \{1\}} \frac{f(x) - f(1)}{x - 1} = \alpha\).
  By \cref{6.7.2} we know that \(1^{\alpha} = 1\).
  Then we have
  \begin{align*}
             & \forall \varepsilon \in \R^+, \alpha - \varepsilon < \alpha < \alpha + \varepsilon                                                                                                            \\
    \implies & \exists\ q_1, q_2 \in \Q : \alpha - \varepsilon < q_1 < \alpha < q_2 < \alpha + \varepsilon                                                     & \text{(by \cref{5.4.14})}                   \\
    \implies & \forall x \in (0, \infty) \setminus \{1\},                                                                                                                                                    \\
             & \frac{x^{q_1} - 1}{x - 1} < \frac{x^{\alpha} - 1}{x - 1} < \frac{x^{q_2} - 1}{x - 1}                                                            & \text{(by \cref{6.7.3})}                    \\
    \implies & q_1 = \lim_{x \to 1 ; x \in (0, \infty) \setminus \{1\}} \frac{x^{q_1} - 1}{x - 1}                                                              & \text{(by \cref{ex:10.4.2}(b))}             \\
             & \leq \lim_{x \to 1 ; x \in (0, \infty) \setminus \{1\}} \frac{x^{\alpha} - 1}{x - 1}                                                            & \text{(by \cref{9.3.14})}                   \\
             & \leq \lim_{x \to 1 ; x \in (0, \infty) \setminus \{1\}} \frac{x^{q_2} - 1}{x - 1} = q_2                                                         & \text{(by \cref{ex:10.4.2}(b))}             \\
    \implies & \alpha - \varepsilon < q_1 \leq \lim_{x \to 1 ; x \in (0, \infty) \setminus \{1\}} \frac{x^{\alpha} - 1}{x - 1} \leq q_2 < \alpha + \varepsilon                                               \\
    \implies & \abs{\lim_{x \to 1 ; x \in (0, \infty) \setminus \{1\}} \frac{x^{\alpha} - 1}{x - 1} - \alpha} \leq \varepsilon                                                                               \\
    \implies & \lim_{x \to 1 ; x \in (0, \infty) \setminus \{1\}} \frac{x^{\alpha} - 1}{x - 1} = \alpha.                                                       & \text{(Since \(\varepsilon\) is arbitrary)}
  \end{align*}

  Now we show that \(f\) is differentiable on \((0, \infty)\) and that \(f'(x) = \alpha x^{\alpha - 1}\).
  From proof above we know that \(f\) is differentiable at \(1\).
  Let \(x_0 \in (0, \infty)\).
  By \cref{10.1.13}(b)(e) we know that \(g(x) = x / x_0\) is differentiable at \(x_0\).
  Then by chain rule (\cref{10.1.15}) we know that \(f \circ g\) is differentiable at \(x_0\).
  Thus we have
  \begin{align*}
    (f \circ g)'(x_0) & = f'(1) g'(x_0)                                                                                                                       & \text{(by \cref{10.1.15})}       \\
                      & = \frac{\alpha}{x_0}                                                                                                                  & \text{(by \cref{10.1.13}(b)(e))} \\
                      & = \lim_{x \to x_0 ; x \in (0, \infty) \setminus \{x_0\}} \frac{(f \circ g)(x) - (f \circ g)(x_0)}{x - x_0}                            & \text{(by \cref{10.1.1})}        \\
                      & = \lim_{x \to x_0 ; x \in (0, \infty) \setminus \{x_0\}} \frac{(x / x_0)^{\alpha} - (x_0 / x_0)^{\alpha}}{x - x_0}                                                       \\
                      & = \frac{1}{x_0^{\alpha}} \bigg(\lim_{x \to x_0 ; x \in (0, \infty) \setminus \{x_0\}} \frac{x^{\alpha} - x_0^{\alpha}}{x - x_0}\bigg) & \text{(by \cref{10.1.13}(e))}
  \end{align*}
  and
  \[
    \lim_{x \to x_0 ; x \in (0, \infty) \setminus \{x_0\}} \frac{x^{\alpha} - x_0^{\alpha}}{x - x_0} = \alpha x_0^{\alpha - 1}.
  \]
  By \cref{10.1.1} \(f\) is differentiable at \(x_0\).
  Since \(x_0\) is arbitrary, we know that \(f\) is differentiable on \((0, \infty)\).
\end{proof}
\input{10-5-L-Hôpital-s-rule.tex}

\chapter{The Riemann integral}\label{i:ch:11}

\begin{note}
  In \cref{i:ch:10} we reviewed \emph{differentiation} - one of the two pillars of single variable calculus.
  The other pillar is, of course, \emph{integration}, which is the focus of the current chapter.
  More precisely, we will turn to the \emph{definite integral}, the integral of a function on a fixed interval, as opposed to the \emph{indefinite integral}, otherwise known as the \emph{antiderivative}.
  These two are of course linked by the \emph{Fundamental theorem of calculus}.
\end{note}

\begin{note}
  To actually \emph{define} this integral \(\int_I f\) is somewhat delicate (especially if one does not want to assume any axioms concerning geometric notions such as area), and not all functions \(f\) are integrable.
  It turns out that there are at least two ways to define this integral:
  the \emph{Riemann integral}, named after Georg Riemann (1826--1866), which suffices for most applications, and the \emph{Lebesgue integral}, named after Henri Lebesgue (1875--1941), which supercedes the Riemann integral and works for a much larger class of functions.
  There is also the \emph{Riemann-Stieltjes integral} \(\int_I f(x) d \alpha(x)\), a generalization of the Riemann integral due to Thomas Stieltjes (1856--1894).
\end{note}

\section{Partitions}\label{sec:11.1}

\begin{defn}\label{11.1.1}
  Let \(X\) be a subset of \(\R\).
  We say that \(X\) is \emph{connected} iff \(X\) is nonempty and the following property is true:
  whenever \(x, y\) are elements in \(X\) such that \(x < y\), the bounded interval \([x, y]\) is a subset of \(X\)
  (i.e., every number between \(x\) and \(y\) is also in \(X\)).
\end{defn}

\setcounter{thm}{3}
\begin{lem}\label{11.1.4}
  Let \(X\) be a subset of the real line.
  Then the following two statements are logically equivalent:
  \begin{enumerate}
    \item \(X\) is bounded and either connected or empty.
    \item \(X\) is a bounded interval.
  \end{enumerate}
\end{lem}

\begin{proof}
  Both statements are logically equivalent when \(X = \emptyset\) (which is vacuously true).
  So suppose that \(X \neq \emptyset\).

  We first show that \(X\) is bounded and connected implies \(X\) is a bounded interval.
  Since \(X\) is bounded, by \cref{5.5.9} we know that \(\inf(X), \sup(X) \in \R\).
  Thus \(X \subseteq [\inf(X), \sup(X)]\).
  Now we split into four cases:
  \begin{itemize}
    \item If \(\sup(X) \in X\) and \(\inf(X) \in X\), then by \cref{11.1.1} \(X\) is connected implies \([\inf(X), \sup(X)] \subseteq X\).
          Thus by \cref{3.1.18} we have \(X = [\inf(X), \sup(X)]\).
    \item If \(\sup(X) \in X\) and \(\inf(X) \notin X\), then we claim that \(\big(\inf(X), \sup(X)] \subseteq X\).
          This is true since \(X\) is connected and by \cref{11.1.1} we have \(\big(a, \sup(X)] \subseteq X\) for every \(a \in X\).
    \item If \(\sup(X) \notin X\) and \(\inf(X) \in X\), then we claim that \([\inf(X), \sup(X)\big) \subseteq X\).
          This is true since \(X\) is connected and by \cref{11.1.1} we have \([\inf(X), b\big) \subseteq X\) for every \(b \in X\).
    \item If \(\sup(X) \notin X\) and \(\inf(X) \notin X\), then we claim that \(\big(\inf(X), \sup(X)\big) \subseteq X\).
          This is true since \(X\) is connected and by \cref{11.1.1} we have \((a, b) \subseteq X\) for every \(a, b \in X\) and \(a < b\).
  \end{itemize}
  From all cases above we conclude that \(X\) is a bounded interval.

  Now we show that \(X\) is a bounded interval implies \(X\) is bounded and connected.
  Obviously \(X\) is bounded.
  Let \(a, b \in \R\).
  Then \(X\) can be one of \((a, b), [a, b], (a, b], [a, b)\), and by \cref{11.1.1} all of which are connected.
\end{proof}

\begin{rmk}\label{11.1.5}
  Recall that intervals are allowed to be singleton points, or even the empty set.
\end{rmk}

\begin{cor}\label{11.1.6}
  If \(I\) and \(J\) are bounded intervals, then the intersection \(I \cap J\) is also a bounded interval.
\end{cor}

\begin{proof}
  If \(I \cap J = \emptyset\), then \(I \cap J\) is bounded interval.
  So suppose that \(I \cap J \neq \emptyset\).
  Since \(I, J\) are bounded intervals, by \cref{11.1.4} we know that \(I, J\) are bounded and connected.
  Since \(I, J\) are bounded, \(\exists M_1, M_2 \in \R\) such that \(I \subseteq [-M_1, M_1]\) and \(J \subseteq [-M_2, M_2]\).
  Let \(M = \min(M_1, M_2)\).
  Then we have \(I \cap J \subseteq [-M, M]\) and thus \(I \cap J\) is bounded.
  Let \(x, y \in I \cap J\) and \(x < y\).
  Since \(I\) is connected and \(I \cap J \subseteq I\), we have \([x, y] \subseteq I\).
  Similarly since \(J\) is connected and \(I \cap J \subseteq J\), we have \([x, y] \subseteq J\).
  Thus \([x, y] \subseteq I \cap J\) and by \cref{11.1.1} \(I \cap J\) is connected.
  Since \(I \cap J\) is bounded and connected, by \cref{11.1.4} \(I \cap J\) is bounded interval.
\end{proof}

\setcounter{thm}{7}
\begin{defn}[Length of intervals]\label{11.1.8}
  If \(I\) is a bounded interval, we define the \emph{length} of \(I\), denoted \(\abs{I}\) as follows.
  If \(I\) is one of the intervals \([a, b]\), \((a, b)\), \([a, b)\), or \((a, b]\) for some real numbers \(a < b\), then we define \(\abs{I} \coloneqq b - a\).
  Otherwise, if \(I\) is a point or the empty set, we define \(\abs{I} = 0\).
\end{defn}

\setcounter{thm}{9}
\begin{defn}[Partitions]\label{11.1.10}
  Let \(I\) be a bounded interval.
  A \emph{partition} of \(I\) is a finite set \(\mathbf{P}\) of bounded intervals contained in \(I\), such that every \(x\) in \(I\) lies in exactly one of the bounded intervals \(J\) in \(\mathbf{P}\).
\end{defn}

\begin{rmk}\label{11.1.11}
  Note that a partition is a set of intervals, while each interval is itself a set of real numbers.
  Thus a partition is a set consisting of other sets.
\end{rmk}

\setcounter{thm}{12}
\begin{thm}[Length is finitely additive]\label{11.1.13}
  Let \(I\) be a bounded interval, \(n\) be a natural number, and let \(\mathbf{P}\) be a partition of \(I\) of cardinality \(n\).
  Then
  \[
    \abs{I} = \sum_{J \in \mathbf{P}} \abs{J}.
  \]
\end{thm}

\begin{proof}
  We prove this by induction on \(n\).
  More precisely, we let \(P(n)\) be the property that whenever \(I\) is a bounded interval, and whenever \(\mathbf{P}\) is a partition of \(I\) with cardinality \(n\), that \(\abs{I} = \sum_{J \in \mathbf{P}} \abs{J}\).

  The base case \(P(0)\) is trivial;
  the only way that \(I\) can be partitioned into an empty partition is if \(I\) is itself empty, at which point the claim is easy.
  The case \(P(1)\) is also very easy;
  the only way that \(I\) can be partitioned into a singleton set \(\{J\}\) is if \(J = I\), at which point the claim is again very easy.

  Now suppose inductively that \(P(n)\) is true for some \(n \geq 1\), and now we prove \(P(n + 1)\).
  Let \(I\) be a bounded interval, and let \(\mathbf{P}\) be a partition of \(I\) of cardinality \(n + 1\).

  If \(I\) is the empty set or a point, then all the intervals in \(\mathbf{P}\) must also be either the empty set or a point, and so every interval has length zero and the claim is trivial.
  Thus we will assume that \(I\) is an interval of the form \((a, b)\), \((a, b]\), \([a, b)\), or \([a, b]\).

      Let us first suppose that \(b \in I\), i.e., \(I\) is either \((a, b]\) or \([a, b]\).
  Since \(b \in I\), we know that one of the intervals \(K\) in \(\mathbf{P}\) contains \(b\).
  Since \(K\) is contained in \(I\), it must therefore be of the form \((c, b]\), \([c, b]\), or \(\{b\}\) for some real number \(c\), with \(a \leq c \leq b\) (in the latter case of \(K = \{b\}\), we set \(c \coloneqq b\)).
  In particular, this means that the set \(I \setminus K\) is also an interval of the form \([a, c]\), \((a, c)\), \((a, c]\), \([a, c)\) when \(c > a\), or a point or empty set when \(a = c\).
  Either way, we easily see that
  \[
    \abs{I} = \abs{K} + \abs{I \setminus K}.
  \]
  On the other hand, since \(\mathbf{P}\) forms a partition of \(I\), we see that \(\mathbf{P} \setminus \{K\}\) forms a partition of \(I \setminus K\).
  By the induction hypothesis, we thus have
  \[
    \abs{I \setminus K} = \sum_{J \in \mathbf{P} \setminus \{K\}} \abs{J}.
  \]
  Combining these two identities (and using the laws of addition for finite sets, see \cref{7.1.11}(e)) we obtain
  \[
    \abs{I} = \sum_{J \in \mathbf{P}} \abs{J}
  \]
  as desired.

  Now suppose that \(b \notin I\), i.e., \(I\) is either \((a, b)\) or \([a, b)\).
  Then one of the intervals \(K\) also is of the form \((c, b)\) or \([c, b)\) (see \cref{ex:11.1.3}).
      In particular, this means that the set \(I \setminus K\) is also an interval of the form \([a, c]\), \((a, c)\), \((a, c]\), \([a, c)\) when \(c > a\), or a point or empty set when \(a = c\).
  The rest of the argument then proceeds as above.
\end{proof}

\begin{defn}[Finer and coarser partitions]\label{11.1.14}
  Let \(I\) be a bounded interval, and let \(\mathbf{P}\) and \(\mathbf{P}'\) be two partitions of \(I\).
  We say that \(\mathbf{P}'\) is \emph{finer} than \(\mathbf{P}\) (or equivalently, that \(\mathbf{P}\) is \emph{coarser} than \(\mathbf{P}'\)) if for every \(J\) in \(\mathbf{P}'\), there exists a \(K\) in \(\mathbf{P}\) such that \(J \subseteq K\).
\end{defn}

\begin{note}
  There is no such thing as a ``finest'' partition of some interval \(I\).
  (recall all partitions are assumed to be finite.)
  We do not compare partitions of different intervals.
\end{note}

\setcounter{thm}{15}
\begin{defn}[Common refinement]\label{11.1.16}
  Let \(I\) be a bounded interval, and let \(\mathbf{P}\) and \(\mathbf{P}'\) be two partitions of \(I\).
  We define the \emph{common refinement} \(\mathbf{P} \# \mathbf{P}'\) of \(\mathbf{P}\) and \(\mathbf{P}'\) to be the set
  \[
    \mathbf{P} \# \mathbf{P}' \coloneqq \{K \cap J : K \in \mathbf{P} \text{ and } J \in \mathbf{P}'\}.
  \]
\end{defn}

\begin{ac}\label{ac:11.1.1}
  Let \(I\) be a bounded interval, and let \(\mathbf{P}, \mathbf{P}'\) be two partitions of \(I\).
  Then we have \(I = \bigcup (\mathbf{P} \# \mathbf{P}')\).
\end{ac}

\begin{proof}
  Let \(x \in I\).
  By \cref{11.1.10} we know that \(\exists!\ K \in \mathbf{P}\) such that \(x \in K\).
  Similarly \(\exists!\ K' \in \mathbf{P}'\) such that \(x \in K'\), thus \(x \in K \cap K'\).
  By \cref{11.1.16} we know that \(K \cap K' \in \mathbf{P} \# \mathbf{P}'\), thus \(x \in \bigcup (\mathbf{P} \# \mathbf{P}')\).
  Since \(x\) is arbitrary, we have
  \[
    I \subseteq \bigcup \big(\mathbf{P} \# \mathbf{P}'\big).
  \]

  Let \(S \in \mathbf{P} \# \mathbf{P}'\).
  By \cref{11.1.16} we know that \(\exists J \in \mathbf{P}\) and \(\exists J' \in \mathbf{P}'\) such that \(S = J \cap J'\).
  Since \(S = J \cap J'\), we have \(S \subseteq I\).
  Since \(S\) is arbitrary, we have
  \[
    \bigcup \big(\mathbf{P} \# \mathbf{P}'\big) \subseteq I.
  \]
  Thus by \cref{3.1.18} we have
  \[
    I = \bigcup \big(\mathbf{P} \# \mathbf{P}'\big).
  \]
\end{proof}

\begin{ac}\label{ac:11.1.2}
  Let \(I\) be a bounded interval, and let \(\mathbf{P}, \mathbf{P}'\) be two partitions of \(I\).
  Then every element \(x \in I\) contains in exactly one of the element \(\mathbf{P} \# \mathbf{P}'\).
  In other words, \(\exists!\ S \in \mathbf{P} \# \mathbf{P}\) such that \(x \in S\).
\end{ac}

\begin{proof}
  By \cref{ac:11.1.1} we know that at least one element in \(\mathbf{P} \# \mathbf{P}'\) contains \(x\).
  Suppose for sake of contradiction that \(\exists S_1, S_2 \in \mathbf{P} \# \mathbf{P}'\) such that \(x \in S_1\) and \(x \in S_2\) but \(S_1 \neq S_2\).
  By \cref{11.1.16} we know that \(S_1 = K \cap K'\) for some \(K \in \mathbf{P}\) and \(K' \in \mathbf{P}'\).
  Similarly \(S_2 = J \cap J'\) for some \(J \in \mathbf{P}\) and \(J' \in \mathbf{P}'\).
  We know that \(x \in S_1\) implies \(x \in K\).
  Similarly \(x \in S_2\) implies \(x \in J\).
  But by \cref{11.1.10} we know that \(K = J\), similar argument holds for \(K' = J'\).
  Thus we must have \(S_1 = S_2\), a contradiction.
\end{proof}

\begin{ac}\label{ac:11.1.3}
  Let \(I\) be a bounded interval, and let \(\mathbf{P}, \mathbf{P}'\) be two partitions of \(I\).
  Then \(\mathbf{P} \# \mathbf{P}'\) is finite and every element in \(\mathbf{P} \# \mathbf{P}'\) is a bounded interval.
\end{ac}

\begin{proof}
  Let \(f : \mathbf{P} \times \mathbf{P}' \to \mathbf{P} \# \mathbf{P}'\) be a function where
  \[
    f(K, K') = K \cap K' \text{ for every } (K, K') \in \mathbf{P} \times \mathbf{P}'.
  \]
  By \cref{11.1.16} we see that \(f\) is surjective.
  By \cref{11.1.10} we know that both \(\#(\mathbf{P}), \#(\mathbf{P}')\) are finite.
  Thus by \cref{3.6.14}(e) and \cref{ex:8.4.3} we have
  \[
    \#(\mathbf{P} \times \mathbf{P}') = \#(\mathbf{P}) \times \#(\mathbf{P}') \geq \#(\mathbf{P} \# \mathbf{P}').
  \]
  This means \(\mathbf{P} \# \mathbf{P}'\) is finite.

  By \cref{11.1.16} we know that for every \(S \in \mathbf{P} \# \mathbf{P}'\), \(S = K \cap K'\) for some \(K \in \mathbf{P}\) and \(K' \in \mathbf{P}'\).
  By \cref{11.1.10} we know that both \(K, K'\) are bounded interval, thus by \cref{11.1.6} we know that \(S\) is also a bounded interval.
  Since \(S\) is arbitrary, we conclude that every element in \(\mathbf{P} \# \mathbf{P}'\) is a bounded interval.
\end{proof}

\setcounter{thm}{17}
\begin{lem}\label{11.1.18}
  Let \(I\) be a bounded interval, and let \(\mathbf{P}\) and \(\mathbf{P}'\) be two partitions of \(I\).
  Then \(\mathbf{P} \# \mathbf{P}'\) is also a partition of \(I\), and is both finer than \(\mathbf{P}\) and finer than \(\mathbf{P}'\).
\end{lem}

\begin{proof}
  By \cref{ac:11.1.1} we know that \(I = \bigcup (\mathbf{P} \# \mathbf{P}')\).
  By \cref{ac:11.1.2} we know that every element in \(I\) contains in exactly one of the element \(\mathbf{P} \# \mathbf{P}'\).
  By \cref{ac:11.1.3} we know that \(\mathbf{P} \# \mathbf{P}'\) is finite and every element in  \(\mathbf{P} \# \mathbf{P}'\) is a bounded interval.
  Thus by \cref{11.1.10} \(\mathbf{P} \# \mathbf{P}'\) is a partition of \(I\).

  By \cref{11.1.16} we know that for every \(S \in \mathbf{P} \# \mathbf{P}'\), \(S = K \cap K'\) for some \(K \in \mathbf{P}\) and \(K' \in \mathbf{P}'\).
  This means \(S \subseteq K\) and \(S \subseteq K'\), thus by \cref{11.1.14} \(\mathbf{P} \# \mathbf{P}'\) is both finer than \(\mathbf{P}\) and finer than \(\mathbf{P}'\)
\end{proof}

\begin{ac}\label{ac:11.1.4}
  Let \(I\) be a bounded interval, and let \(\mathbf{P}, \mathbf{P}'\) be two partitions of \(I\) such that \(\mathbf{P}'\) is finer than \(\mathbf{P}\).
  For each \(K \in \mathbf{P}\), we define \(\mathbf{P}_K\) as follow:
  \[
    \mathbf{P}_K = \{K' \in \mathbf{P}' : K' \subseteq K\}.
  \]
  Then \(\mathbf{P}_K\) is a partition of \(K\) for every \(K \in \mathbf{P}\), and \(\bigcup_{K \in \mathbf{P}} \mathbf{P}_K = \mathbf{P}'\).
\end{ac}

\begin{proof}
  Since \(\mathbf{P}_K \subseteq \mathbf{P}'\) and \(\mathbf{P}'\) is a partition of \(I\), by \cref{11.1.10} we know the following facts:
  \begin{itemize}
    \item \(\mathbf{P}_K\) is finite.
    \item All distinct elements in \(\mathbf{P}_K\) are disjoint.
    \item All elements in \(\mathbf{P}_K\) are bounded interval.
  \end{itemize}
  To show that \(\mathbf{P}_K\) is a partition of \(K\), by \cref{11.1.10} it suffices to show that \(K = \bigcup \mathbf{P}_K\).

  Let \(x \in K\).
  By \cref{11.1.10} we know that \(x \in I\), thus \(\exists!\ K' \in \mathbf{P}'\) such that \(x \in K'\).
  Since \(\mathbf{P}'\) is finer than \(\mathbf{P}\), we must have \(K' \subseteq K\).
  If not, then we have some \(J \in \mathbf{P}\) such that \(K' \subseteq J\), but \(x \in J\) implies \(J = K\), a contradiction.
  Since \(K' \in \mathbf{P}'\) and \(K' \subseteq K\), we have \(K' \in \mathbf{P}_K\).
  Since \(x\) is arbitrary, we have \(K \subseteq \bigcup \mathbf{P}_K\).
  By the definition of \(\mathbf{P}_K\) we know that \(\bigcup \mathbf{P}_K \subseteq K\), thus by \cref{3.1.18} we have \(K = \bigcup \mathbf{P}_K\).

  Now we show that \(\bigcup_{K \in \mathbf{P}} \mathbf{P}_K = \mathbf{P}'\).
  We know that \(\bigcup_{K \in \mathbf{P}} \mathbf{P}_K \subseteq \mathbf{P}'\).
  Let \(K' \in \mathbf{P}'\).
  By \cref{11.1.18} we know that \(\mathbf{P}'\) is finer than \(\mathbf{P}\).
  By \cref{11.1.14} we know that \(K' \subseteq K\) for some \(K \in \mathbf{P}\).
  Thus we have \(K' \in \mathbf{P}_K\).
  Since \(K'\) is arbitrary, we have \(\mathbf{P}' \subseteq \bigcup_{K \in \mathbf{P}} \mathbf{P}_K\).
  Thus by \cref{3.1.18} we have \(\bigcup_{K \in \mathbf{P}} \mathbf{P}_K = \mathbf{P}'\).
\end{proof}

\begin{ac}\label{ac:11.1.5}
  Let \(I, J\) be bounded intervals such that \(I \neq \emptyset\) and \(I \subseteq J\), and let \(\mathbf{P}\) be a partition of \(I\).
  Let \(I_1, I_2\) be the sets
  \[
    I_1 = \Big\{x \in J : \big(x \leq \inf(I)\big) \land (x \notin I)\Big\}
  \]
  and
  \[
    I_2 = \Big\{x \in J : \big(x \geq \sup(I)\big) \land (x \notin I)\Big\}.
  \]
  Then \(\mathbf{P} \cup \{I_1, I_2\}\) is a partion of \(J\).
\end{ac}

\begin{proof}
  First we claim that \(I_1\) is a bounded interval.
  If \(I_1 = \emptyset\), then \(I_1\) is a bounded interval.
  So suppose that \(I_1 \neq \emptyset\).
  We know that \(\inf(I) \in J\) since if \(\inf(I) \notin J\), then by definition we would have \(I_1 = \emptyset\), a contradiction.
  We must have \(\inf(I_1) = \inf(J)\).
  If not, then we have \(\inf(J) < \inf(I_1) \leq \inf(I)\).
  Since \(J\) is a bounded interval, we have \(\inf(J) < x < \inf(I_1) \leq \inf(I)\) for some \(x \in J\).
  But \(x \in J\) and \(x < \inf(I)\) implies \(x \in I_1\), which contradict to \(\inf(I_1) \leq x\).
  So we have \(\inf(I_1) = \inf(J)\).
  Now we split into four cases:
  \begin{itemize}
    \item If \(\inf(J) \in J\) and \(\inf(I) \in I\), then \(I_1 = [\inf(J), \inf(I)\big)\).
    \item If \(\inf(J) \in J\) and \(\inf(I) \notin I\), then \(I_1 = [\inf(J), \inf(I)]\).
    \item If \(\inf(J) \notin J\) and \(\inf(I) \in I\), then \(I_1 = \big(\inf(J), \inf(I)\big)\).
    \item If \(\inf(J) \notin J\) and \(\inf(I) \notin I\), then \(I_1 = \big(\inf(J), \inf(I)]\).
  \end{itemize}
  From all cases above we conclude that \(I_1\) is a bounded interval.

  Next we claim that \(I_2\) is a bounded interval.
  If \(I_2 = \emptyset\), then \(I_2\) is a bounded interval.
  So suppose that \(I_2 \neq \emptyset\).
  We know that \(\sup(I) \in J\) since if \(\sup(I) \notin J\), then by definition we would have \(I_2 = \emptyset\), a contradiction.
  We must have \(\sup(I_2) = \sup(J)\).
  If not, then we have \(\sup(J) > \sup(I_2) \geq \sup(I)\).
  Since \(J\) is a bounded interval, we have \(\sup(J) > x > \sup(I_2) \geq \sup(I)\) for some \(x \in J\).
  But \(x \in J\) and \(x > \sup(I)\) implies \(x \in I_2\), which contradict to \(\sup(I_2) \geq x\).
  So we have \(\sup(I_2) = \sup(J)\).
  Now we split into four cases:
  \begin{itemize}
    \item If \(\sup(J) \in J\) and \(\sup(I) \in I\), then \(I_2 = \big(\sup(I), \sup(J)]\).
    \item If \(\sup(J) \in J\) and \(\sup(I) \notin I\), then \(I_2 = [\sup(I), \sup(J)]\).
    \item If \(\sup(J) \notin J\) and \(\sup(I) \in I\), then \(I_2 = \big(\sup(I), \sup(J)\big)\).
    \item If \(\sup(J) \notin J\) and \(\sup(I) \notin I\), then \(I_2 = [\sup(I), \sup(J)\big)\).
  \end{itemize}
  From all cases above we conclude that \(I_2\) is a bounded interval.

  Next we show that \(I \cap I_1 = I \cap I_2 = I_1 \cap I_2 = \emptyset\).
  By definition we know that \(I \cap I_1 = I \cap I_2 = \emptyset\).
  So we only need to show that \(I_1 \cap I_2 = \emptyset\).
  If \((I_1 = \emptyset) \lor (I_2 = \emptyset)\), then we have \(I_1 \cap I_2 = \emptyset\).
  So suppose that \((I_1 \neq \emptyset) \land (I_2 \neq \emptyset)\).
  Suppose for sake of contradiction that \(I_1 \cap I_2 \neq \emptyset\).
  Let \(x \in I_1 \cap I_2\).
  Then we have \(x \leq \inf(I) \leq \sup(I) \leq x\).
  Now we split into two cases:
  \begin{itemize}
    \item If \(\inf(I) = \sup(I)\), then \(I = \{a\}\) for some \(a \in \R\).
          But \(x \leq a \leq x\) implies \(x = a\) and \(x \in I\), which contradict to \(x \notin I\).
    \item If \(\inf(I) < \sup(I)\), then we have \(x < x\), a contradiction.
  \end{itemize}
  From all cases above we conclude that \(I_1 \cap I_2 = \emptyset\).

  Let \(\mathbf{P}_J = \mathbf{P} \cup \{I_1, I_2\}\).
  By definition we know that \(\bigcup \mathbf{P}_J \subseteq J\).
  Let \(x \in J\).
  Now we split into two cases:
  \begin{itemize}
    \item If \(x \in I\), then we have \(x \in \bigcup \mathbf{P}\).
    \item If \(x \notin I\), then we have \(\big(x \leq \inf(I)\big) \lor \big(x \geq \sup(I)\big)\).
          Thus \((x \in I_1) \lor (x \in I_2)\) and \(x \in \bigcup \mathbf{P}\).
  \end{itemize}
  From all cases above we conclude that \(x \in \bigcup \mathbf{P}_J\).
  Since \(x\) is arbitrary, we have \(J \subseteq \bigcup \mathbf{P}_J\).
  By \cref{3.1.18} we have \(J = \bigcup \mathbf{P}_J\).

  From proofs above we have showed that \(J = \bigcup \mathbf{P}_J\), all distinct element in \(\mathbf{P}_J\) are disjoint, and all elements in \(\mathbf{P}_J\) are bounded interval.
  Since \(\mathbf{P}_J\) is finite (\(\#(\mathbf{P}_J) = 3\)), by \cref{11.1.10} \(\mathbf{P}_J\) is a partition of \(J\).
\end{proof}

\exercisesection

\begin{ex}\label{ex:11.1.1}
  Prove \cref{11.1.4}.
\end{ex}

\begin{proof}
  See \cref{11.1.4}.
\end{proof}

\begin{ex}\label{ex:11.1.2}
  Prove \cref{11.1.6}.
\end{ex}

\begin{proof}
  Prove \cref{11.1.6}.
\end{proof}

\begin{ex}\label{ex:11.1.3}
  Let \(I\) be a bounded interval of the form \(I = (a, b)\) or \(I = [a, b)\) for some real numbers \(a < b\).
  Let \(I_1, \dots, I_n\) be a partition of \(I\).
  Prove that one of the intervals \(I_j\) in this partition is of the form \(I_j = (c, b)\) or \(I_j = [c, b)\) for some \(a \leq c \leq b\).
\end{ex}

\begin{proof}
  Let \(\mathbf{P} = \{I_1, \dots, I_n\}\).
  If \(c = b\), then \((c, b) = \emptyset\), and thus by \cref{11.1.10} \(\mathbf{P} \cup \{\emptyset\}\) is a partition of \(I\).
  So we only need to proof the cases where \(a \leq c < b\).
  Suppose for sake of contradiction that every interval \(I_j\) in the partition \(\mathbf{P}\) is not of the form \((c, b)\) or \([c, b)\).
  By \cref{11.1.10} this means for every \(j \in \{1, \dots, n\}\), \(x \in I_j\) implies \(x \geq b\) or \(x < c\).
  Since \(I = (a, b)\) or \(I = [a, b)\), we cannot have \(x \geq b\), thus we must have \(x < c\).
  This means \(\sup(I_j) \leq c < b\) for every \(j \in \{1, \dots, n\}\).
  But then we have \(\sup(I) = b > \max\big\{\sup(I_j) : j \in \{1, \dots, n\}\big\}\), a contradiction.
  Thus we must have one interval \(I_j \in \mathbf{P}\) such that \(I_j = (c, b)\) for some \(a \leq c < b\).
\end{proof}

\begin{ex}\label{ex:11.1.4}
  Prove \cref{11.1.18}.
\end{ex}

\begin{proof}
  Prove \cref{11.1.18}.
\end{proof}
\section{Piecewise constant functions}\label{i:sec:11.2}

\begin{defn}[Constant functions]\label{i:11.2.1}
  Let \(X\) be a subset of \(\R\), and let \(f : X \to \R\) be a function.
  We say that \(f\) is \emph{constant} iff there exists a real number \(c\) such that \(f(x) = c\) for all \(x \in X\).
  If \(E\) is a subset of \(X\), we say that \(f\) is \emph{constant on} \(E\) if the restriction \(f|_E\) of \(f\) to \(E\) is constant, in other words there exists a real number \(c\) such that \(f(x) = c\) for all \(x \in E\).
  We refer to \(c\) as the \emph{constant value} of \(f\) on \(E\).
\end{defn}

\begin{rmk}\label{i:11.2.2}
  If \(E\) is a non-empty set, then a function \(f\) which is constant on \(E\) can have only one constant value;
  However, if \(E\) is empty, every real number \(c\) is a constant value for \(f\) on \(E\).
\end{rmk}

\begin{defn}[Piecewise constant functions I]\label{i:11.2.3}
  Let \(I\) be a bounded interval, let \(f : I \to \R\) be a function, and let \(\mathbf{P}\) be a partition of \(I\).
  We say that \(f\) is \emph{piecewise constant with respect to \(\mathbf{P}\)} if for every \(J \in \mathbf{P}\), \(f\) is constant on \(J\).
\end{defn}

\setcounter{thm}{4}
\begin{defn}[Piecewise constant functions II]\label{i:11.2.5}
  Let \(I\) be a bounded interval, and let \(f : I \to \R\) be a function.
  We say that \(f\) is \emph{piecewise constant on \(I\)} if there exists a partition \(\mathbf{P}\) of \(I\) such that \(f\) is piecewise constant with respect to \(\mathbf{P}\).
\end{defn}

\setcounter{thm}{6}
\begin{lem}\label{i:11.2.7}
  Let \(I\) be a bounded interval, let \(\mathbf{P}\) be a partition of \(I\), and let \(f : I \to \R\) be a function which is piecewise constant with respect to \(\mathbf{P}\).
  Let \(\mathbf{P}'\) be a partition of \(I\) which is finer than \(\mathbf{P}\).
  Then \(f\) is also piecewise constant with respect to \(\mathbf{P}'\).
\end{lem}

\begin{proof}
  Let \(K' \in \mathbf{P}'\).
  Since \(\mathbf{P}'\) is finer than \(\mathbf{P}\), by \cref{i:11.1.14} \(\exists K \in \mathbf{P}\) such that \(K' \subseteq K\).
  Since \(f\) is piecewise constant with respect to \(\mathbf{P}\), by \cref{i:11.2.3} we know that \(\forall x \in K\), \(f(x)\) is constant.
  Thus for every \(x \in K'\), \(x \in K\) and \(f(x)\) is constant.
  Since \(K'\) was arbitrary, by \cref{i:11.2.3} \(f\) is piecewise constant with respect to \(\mathbf{P}'\).
\end{proof}

\begin{lem}\label{i:11.2.8}
  Let \(I\) be a bounded interval, and let \(f : I \to \R\) and \(g : I \to \R\) be piecewise constant functions on \(I\).
  Then the functions \(f + g\), \(f - g\), \(\max(f, g)\), \(\min(f, g)\) and \(fg\) are also piecewise constant functions on \(I\).
  Here of course \(\max(f, g) : I \to \R\) is the function \(\max(f, g)(x) \coloneqq \max(f(x), g(x))\).
  If \(g\) does not vanish anywhere on \(I\) (i.e., \(g(x) \neq 0\) for all \(x \in I\)) then \(f / g\) is also a piecewise constant function on \(I\).
\end{lem}

\begin{proof}
  Since \(f\) is piecewise constant function on \(I\), by \cref{i:11.2.5} \(\exists \mathbf{P}\) such that \(\mathbf{P}\) is a partition of \(I\) and \(f\) is piecewise constant with respect to \(\mathbf{P}\).
  Similarly \(\exists \mathbf{P}'\) such that \(\mathbf{P}'\) is a partition of \(I\) and \(g\) is piecewise constant with respect to \(\mathbf{P}'\).
  By \cref{i:11.1.18} we know that \(\mathbf{P} \# \mathbf{P}'\) is also a partition of \(I\) and \(\mathbf{P} \# \mathbf{P}'\) is both finer than \(\mathbf{P}\) and finer than \(\mathbf{P}'\).
  By \cref{i:11.2.7} we know that both \(f\) and \(g\) are piecewise constant with respect to \(\mathbf{P} \# \mathbf{P}'\).

  Now we show that \(f, g\) remain piecewise constant functions on \(I\) after algebraic operation.
  For every \(J \in \mathbf{P} \# \mathbf{P}'\), we have \(f(x)\) is constant and \(g(x)\) is constant for every \(x \in J\).
  Thus we know that \(f(x) + g(x)\), \(f(x) - g(x)\), \(\max\big(f(x), g(x)\big)\), \(\min\big(f(x), g(x)\big)\) and \(f(x) g(x)\) are constant.
  If \(g(x) \neq 0\), then we also have \(f(x) / g(x)\) is constant.
  Thus by \cref{i:11.2.3} \(f + g\), \(f - g\), \(\max(f, g)\), \(\min(f, g)\), \(fg\) is piecewise constant with respect to \(\mathbf{P} \# \mathbf{P}'\), and when \(g(x) \neq 0\) we have \(f / g\) is piecewise constant with respect to \(\mathbf{P} \# \mathbf{P}'\).
  By \cref{i:11.2.5} \(f + g\), \(f - g\), \(\max(f, g)\), \(\min(f, g)\), \(fg\) is piecewise constant on \(I\), and when \(g(x) \neq 0\) we have \(f / g\) is piecewise constant on \(I\).
\end{proof}

\begin{defn}[Piecewise constant integral I]\label{i:11.2.9}
  Let \(I\) be a bounded interval, let \(\mathbf{P}\) be a partition of \(I\).
  Let \(f : I \to \R\) be a function which is piecewise constant with respect to \(\mathbf{P}\).
  Then we define the \emph{piecewise constant integral} \(p.c. \int_{[\mathbf{P}]} f\) of \(f\) with respect to the partition \(\mathbf{P}\) by the formula
  \[
    p.c. \int_{[\mathbf{P}]} f \coloneqq \sum_{J \in \mathbf{P}} c_J \abs{J},
  \]
  where for each \(J\) in \(\mathbf{P}\), we let \(c_J\) be the constant value of \(f\) on \(J\).
\end{defn}

\begin{rmk}\label{i:11.2.10}
  This definition seems like it could be ill-defined, because if \(J\) is empty then every number \(c_J\) can be the constant value of \(f\) on \(J\), but fortunately in such cases \(\abs{J}\) is zero and so the choice of \(c_J\) is irrelevant.
  The notation \(p.c. \int_{[\mathbf{P}]} f\) is rather artificial, but we shall only need it temporarily, en route to a more useful definition.
  Note that since \(\mathbf{P}\) is finite, the sum \(\sum_{J \in \mathbf{P}} c_J \abs{J}\) is always well-defined
  (it is never divergent or infinite).
\end{rmk}

\begin{rmk}\label{i:11.2.11}
  The piecewise constant integral corresponds intuitively to one's notion of area, given that the area of a rectangle ought to be the product of the lengths of the sides.
  (Of course, if \(f\) is negative somewhere, then the ``area'' \(c_J \abs{J}\) would also be negative.)
\end{rmk}

\setcounter{thm}{12}
\begin{prop}[Piecewise constant integral is independent of partition]\label{i:11.2.13}
  Let \(I\) be a bounded interval, and let \(f : I \to \R\) be a function.
  Suppose that \(\mathbf{P}\) and \(\mathbf{P}'\) are partitions of \(I\) such that \(f\) is piecewise constant both with respect to \(\mathbf{P}\) and with respect to \(\mathbf{P}'\).
  Then \(p.c. \int_{[\mathbf{P}]} f = p.c. \int_{[\mathbf{P}']} f\).
\end{prop}

\begin{proof}
  By \cref{i:11.1.18} we know that \(\mathbf{P} \# \mathbf{P}'\) is a partition of \(I\) and is both finer than \(\mathbf{P}\) and finer than \(\mathbf{P}'\), thus by \cref{i:11.2.9} we have
  \[
    p.c. \int_{[\mathbf{P} \# \mathbf{P}']} f = \sum_{J \in \mathbf{P} \# \mathbf{P}'} c_J \abs{J}.
  \]
  By \cref{i:11.1.13}, we know that
  \[
    \abs{I} = \sum_{J \in \mathbf{P}} \abs{J} = \sum_{J \in \mathbf{P} \# \mathbf{P}'} \abs{J}.
  \]
  For each \(K \in \mathbf{P}\), let \(\mathbf{P}_K\) be the set
  \[
    \mathbf{P}_K = \set{S \in \mathbf{P} \# \mathbf{P}' : S \subseteq K}.
  \]
  Since \(\mathbf{P} \# \mathbf{P}'\) is finer than \(\mathbf{P}\), by \cref{i:ac:11.1.4} we know that \(\mathbf{P}_K\) is a partition of \(K\), and \(\bigcup_{K \in \mathbf{P}} \mathbf{P}_K = \mathbf{P} \# \mathbf{P}'\).
  Since \(f\) is piecewise constant with respect to \(\mathbf{P}\), by \cref{i:11.2.7} we know that \(f\) is piecewise constant with respect to \(\mathbf{P} \# \mathbf{P}'\).
  So we have
  \begin{align*}
    p.c. \int_{[\mathbf{P} \# \mathbf{P}']} f & = \sum_{J \in \mathbf{P} \# \mathbf{P}'} c_J \abs{J}                        &                 & \by{i:11.2.9}    \\
                                              & = \sum_{J \in \bigcup_{K \in \mathbf{P}} \mathbf{P}_K} c_J \abs{J}                                               \\
                                              & = \sum_{K \in \mathbf{P}} \sum_{J \in \mathbf{P}_K} c_J \abs{J}             &                 & \by{i:7.1.11}[e] \\
                                              & = \sum_{K \in \mathbf{P}} \sum_{J \in \mathbf{P}_K} c_K \abs{J}             & (J \subseteq K)                    \\
                                              & = \sum_{K \in \mathbf{P}} c_K \bigg(\sum_{J \in \mathbf{P}_K} \abs{J}\bigg)                                      \\
                                              & = \sum_{K \in \mathbf{P}} c_K \abs{K}                                       &                 & \by{i:11.1.13}   \\
                                              & = p.c. \int_{[\mathbf{P}]} f.                                               &                 & \by{i:11.2.9}
  \end{align*}
  Using similar arguments we can show that \(p.c. \int_{[\mathbf{P}']} f = p.c. \int_{[\mathbf{P} \# \mathbf{P}']} f\).
  Thus we have \(p.c. \int_{[\mathbf{P}]} f = p.c. \int_{[\mathbf{P}']} f\).
\end{proof}

\begin{defn}[Piecewise constant integral II]\label{i:11.2.14}
  Let \(I\) be a bounded interval, and let \(f : I \to \R\) be a piecewise constant function on \(I\).
  We define the \emph{piecewise constant integral} \(p.c. \int_I f\) by the formula
  \[
    p.c. \int_I f \coloneqq p.c. \int_{[\mathbf{P}]} f,
  \]
  where \(\mathbf{P}\) is any partition of \(I\) with respect to which \(f\) is piecewise constant.
  (Note that \cref{i:11.2.13} tells us that the precise choice of this partition is irrelevant.)
\end{defn}

\setcounter{thm}{15}
\begin{thm}[Laws of integration]\label{i:11.2.16}
  Let \(I\) be a bounded interval, and let \(f : I \to \R\) and \(g : I \to \R\) be piecewise constant functions on \(I\).
  \begin{enumerate}
    \item We have \(p.c. \int_I (f + g) = p.c. \int_I f + p.c. \int_I g\).
    \item For any real number \(c\), we have \(p.c. \int_I (cf) = c (p.c. \int_I f)\).
    \item We have \(p.c. \int_I (f - g) = p.c. \int_I f - p.c. \int_I g\).
    \item If \(f(x) \geq 0\) for all \(x \in I\), then \(p.c. \int_I f \geq 0\).
    \item If \(f(x) \geq g(x)\) for all \(x \in I\), then \(p.c. \int_I f \geq p.c. \int_I g\).
    \item If \(f\) is the constant function \(f(x) = c\) for all \(x \in I\), then \(p.c. \int_I f = c \abs{I}\).
    \item Let \(J\) be a bounded interval containing \(I\) (i.e., \(I \subseteq J\)), and let \(F : J \to \R\) be the function
          \[
            F(x) \coloneqq \begin{dcases}
              f(x) & \text{if } x \in I    \\
              0    & \text{if } x \notin I
            \end{dcases}
          \]
          Then \(F\) is piecewise constant on \(J\), and \(p.c. \int_J F = p.c. \int_I f\).
    \item Suppose that \(\set{J, K}\) is a partition of \(I\) into two intervals \(J\) and \(K\).
          Then the function \(f|_J : J \to \R\) and \(f|_K : K \to \R\) are piecewise constant on \(J\) and \(K\) respectively, and we have
          \[
            p.c. \int_I f = p.c. \int_J f|_J + p.c. \int_K f|_K.
          \]
  \end{enumerate}
\end{thm}

\begin{proof}{(a)}
  Since \(f, g\) are both piecewise constant on \(I\), by \cref{i:11.2.3} \(f\) is piecewise constant with respect to \(\mathbf{P}_f\) and \(g\) is piecewise constant with respect to \(\mathbf{P}_g\) for some partitions \(\mathbf{P}_f, \mathbf{P}_g\) of \(I\).
  Let \(\mathbf{P} = (\mathbf{P}_f \# \mathbf{P}_g) \setminus \set{\emptyset}\).
  Then by \cref{i:11.1.18} we know that \(\mathbf{P}\) is a partition of \(I\) and by \cref{i:11.2.7} \(f, g\) are piecewise constant with respect to \(\mathbf{P}\).
  For each \(J \in \mathbf{P}\), we define \(c_{f|_J}, c_{g|_J} \in \R\) to be the constant value of \(f|_J, g|_J\), respectively.
  Then by \cref{i:11.2.1} \(c_{f|_J} + c_{g|_J}\) is the constant value of \((f + g)|_J\) for each \(J \in \mathbf{P}\).
  Thus \(f + g\) is piecewise constant with respect to \(\mathbf{P}\) and
  \begin{align*}
    p.c. \int_I f + p.c. \int_I g & = p.c. \int_{[\mathbf{P}]} f + p.c. \int_{[\mathbf{P}]} g                             &  & \by{i:11.2.14}   \\
                                  & = \sum_{J \in \mathbf{P}} c_{f|_J} \abs{J} + \sum_{J \in \mathbf{P}} c_{g|_J} \abs{J} &  & \by{i:11.2.9}    \\
                                  & = \sum_{J \in \mathbf{P}} (c_{f|_J} + c_{g|_J}) \abs{J}                               &  & \by{i:7.1.11}[f] \\
                                  & = p.c. \int_{[\mathbf{P}]} (f + g)                                                    &  & \by{i:11.2.9}    \\
                                  & = p.c. \int_I (f + g).                                                                &  & \by{i:11.2.14}
  \end{align*}
\end{proof}

\begin{proof}{(b)}
  By \cref{i:11.2.3} \(f\) is piecewise constant with respect to \(\mathbf{P}\) for some partition \(\mathbf{P}\) of \(I\).
  Without the loss of generality suppose that \(\emptyset \notin \mathbf{P}\).
  For each \(J \in \mathbf{P}\), we define \(c_J \in \R\) to be the constant value of \(f|_J\).
  Then by \cref{i:11.2.1} \(c \cdot c_J\) is the constant value of \((cf)|_J\).
  Thus \(cf\) is piecewise constant with respect to \(\mathbf{P}\) and
  \begin{align*}
    c \bigg(p.c. \int_I f\bigg) & = c \bigg(p.c. \int_{[\mathbf{P}]} f\bigg)          &  & \by{i:11.2.14}   \\
                                & = c \bigg(\sum_{J \in \mathbf{P}} c_J \abs{J}\bigg) &  & \by{i:11.2.9}    \\
                                & = \sum_{J \in \mathbf{P}} c \cdot c_J \abs{J}       &  & \by{i:7.1.11}[g] \\
                                & = p.c. \int_{[\mathbf{P}]} (c f)                    &  & \by{i:11.2.9}    \\
                                & = p.c. \int_I (c f).                                &  & \by{i:11.2.14}
  \end{align*}
\end{proof}

\begin{proof}{(c)}
  We have
  \begin{align*}
    p.c. \int_I f - p.c. \int_I g & = p.c. \int_I f + (-1) p.c. \int_I g                        \\
                                  & = p.c. \int_I f + p.c. \int_I (-g)   &  & \by{i:11.2.16}[b] \\
                                  & = p.c. \int_I \big(f + (-g)\big)     &  & \by{i:11.2.16}[a] \\
                                  & = p.c. \int_I (f - g).               &  & \by{i:9.2.1}
  \end{align*}
\end{proof}

\begin{proof}{(d)}
  By \cref{i:11.2.3} \(f\) is piecewise constant with respect to \(\mathbf{P}\) for some partition \(\mathbf{P}\) of \(I\).
  Without the loss of generality suppose that \(\emptyset \notin \mathbf{P}\).
  For each \(J \in \mathbf{P}\), we define \(c_J \in \R\) to be the constant value of \(f|_J\).
  Since \(f(x) \geq 0\) for every \(x \in I\), we have \(c_J \geq 0\) and \(c_J \abs{J} \geq 0\) for every \(J \in \mathbf{P}\).
  Thus
  \begin{align*}
    p.c. \int_I f & = p.c. \int_{[\mathbf{P}]} f          &  & \by{i:11.2.14}   \\
                  & = \sum_{J \in \mathbf{P}} c_J \abs{J} &  & \by{i:11.2.9}    \\
                  & \geq \sum_{J \in \mathbf{P}} 0        &  & \by{i:7.1.11}[h] \\
                  & = 0.
  \end{align*}
\end{proof}

\begin{proof}{(e)}
  Since \(f(x) \geq g(x)\) for all \(x \in I\), we have \(f(x) - g(x) \geq 0\) for all \(x \in I\) and
  \begin{align*}
    p.c. \int_I f - p.c. \int_I g & = p.c. \int_I (f - g) &  & \by{i:11.2.16}[c] \\
                                  & \geq 0.               &  & \by{i:11.2.16}[d]
  \end{align*}
  Thus
  \[
    p.c. \int_I f \geq p.c. \int_I g.
  \]
\end{proof}

\begin{proof}{(f)}
  Since \(\set{I}\) is a partition of \(I\), we have
  \begin{align*}
    p.c. \int_I f & = p.c. \int_{[I]} f        &  & \by{i:11.2.14}   \\
                  & = \sum_{J \in I} c \abs{J} &  & \by{i:11.2.9}    \\
                  & = c \sum_{J \in I} \abs{J} &  & \by{i:7.1.11}[g] \\
                  & = c \abs{I}.               &  & \by{i:11.1.13}
  \end{align*}
\end{proof}

\begin{proof}{(g)}
  If \(I = \emptyset\), then by \cref{i:11.2.3} \(F\) is piecewise constant with respect to \(\set{J}\), and by \cref{i:11.2.16}(f) we have
  \[
    p.c. \int_J F = 0 \abs{J} = 0 = p.c \int_I f.
  \]
  So suppose that \(I \neq \emptyset\).
  By \cref{i:11.2.3}, \(f\) is piecewise constant with respect to \(\mathbf{P}\) for some partition \(\mathbf{P}\) of \(I\).
  Let \(I_1, I_2\) be the sets
  \[
    I_1 = \set{x \in J, \big(x \leq \inf(I)\big) \land (x \notin I)}
  \]
  and
  \[
    I_2 = \set{x \in J, \big(x \geq \sup(I)\big) \land (x \notin I)}.
  \]
  By \cref{i:ac:11.1.5} we know that \(\mathbf{P} \cup \set{I_1, I_2}\) is a partition of \(J\).
  By hypothesis we know that
  \[
    \forall x \in J, F(x) = \begin{dcases}
      f(x) & \text{if } x \in K \text{ for some } K \in \mathbf{P} \\
      0    & \text{if } x \in I_1 \text{ or } x \in I_2
    \end{dcases}
  \]
  Thus by \cref{i:11.2.5} \(F\) is piecewise constant on \(J\).
  For each \(K \in \mathbf{P} \cup \set{I_1, I_2}\), we define \(c_K \in \R\) to be the constant value of \(F|_K\).
  Then we have
  \begin{align*}
    p.c. \int_J F & = p.c. \int_{[\mathbf{P} \cup \set{I_1, I_2}]} F                              &  & \by{i:11.2.14}         \\
                  & = \sum_{K \in \mathbf{P} \cup \set{I_1, I_2}} c_K \abs{K}                     &  & \by{i:11.2.9}          \\
                  & = c_{I_1} \abs{I_1} + \sum_{K \in \mathbf{P}} c_K \abs{K} + c_{I_2} \abs{I_2} &  & \by{i:7.1.11}[e]       \\
                  & = 0 \abs{I_1} + \sum_{K \in \mathbf{P}} c_K \abs{K} + 0 \abs{I_2}             &  & \text{(by hypothesis)} \\
                  & = \sum_{K \in \mathbf{P}} c_K \abs{K}                                                                     \\
                  & = p.c. \int_{[\mathbf{P}]} f                                                  &  & \by{i:11.2.9}          \\
                  & = p.c. \int_I f.                                                              &  & \by{i:11.2.14}
  \end{align*}
\end{proof}

\begin{proof}{(h)}
  Let \(\mathbf{P} = \set{J, K}\).
  By \cref{i:11.2.3} \(f\) is piecewise constant with respect to \(\mathbf{P}'\) for some partition \(\mathbf{P}'\) of \(I\).
  Now we define \(\mathbf{P}_J\) as
  \[
    \mathbf{P}_J = \set{S \in \mathbf{P} \# \mathbf{P}' : S \subseteq J}
  \]
  and define \(\mathbf{P}_K\) as
  \[
    \mathbf{P}_K = \set{S \in \mathbf{P} \# \mathbf{P}' : S \subseteq K}.
  \]
  By \cref{i:11.1.8} we know that \(\mathbf{P} \# \mathbf{P}'\) is a partition of \(I\) and is finer than \(\mathbf{P}\).
  Since \(\mathbf{P} \# \mathbf{P}'\) is finer than \(\mathbf{P}\), by \cref{i:ac:11.1.4} we know that \(\mathbf{P}_J, \mathbf{P}_K\) are partitions of \(J, K\), respectively.
  Again by \cref{i:ac:11.1.4} we know that \(\mathbf{P}_J \cup \mathbf{P}_K\) is a partition of \(I\).
  Then by \cref{i:11.2.7} \(f\) is piecewise constant with respect to \(\mathbf{P}_J \cup \mathbf{P}_K\).
  Without the loss of generality suppose that \(\emptyset \notin \mathbf{P}_J \cup \mathbf{P}_K\).
  For each \(S \in \mathbf{P}_J\), we define \(c_S \in \R\) to be the constant value of \(f|_J\).
  Similarly, for each \(S \in \mathbf{P}_K\), we define \(c_S \in \R\) to be the constant value of \(f|_K\).
  Then we have
  \begin{align*}
    p.c. \int_J f|_J + p.c. \int_K f|_K & = p.c. \int_{[\mathbf{P}_J]} f|_J + p.c. \int_{[\mathbf{P}_K]} f|_K             &  & \by{i:11.2.14}   \\
                                        & = \sum_{S \in \mathbf{P}_J} c_S \abs{S} + \sum_{S \in \mathbf{P}_K} c_S \abs{S} &  & \by{i:7.1.11}[e] \\
                                        & = \sum_{S \in \mathbf{P}_J \cup \mathbf{P}_K} c_S \abs{S}                       &  & \by{i:11.2.9}    \\
                                        & = \sum_{S \in \mathbf{P}} c_S \abs{S}                                                                 \\
                                        & = p.c. \int_{[\mathbf{P}]} f                                                    &  & \by{i:11.2.9}    \\
                                        & = p.c. \int_I f.                                                                &  & \by{i:11.2.14}
  \end{align*}
\end{proof}

\exercisesection

\begin{ex}\label{i:ex:11.2.1}
  Prove \cref{i:11.2.7}.
\end{ex}

\begin{proof}
  See \cref{i:11.2.7}.
\end{proof}

\begin{ex}\label{i:ex:11.2.2}
  Prove \cref{i:11.2.8}.
\end{ex}

\begin{proof}
  See \cref{i:11.2.8}.
\end{proof}

\begin{ex}\label{i:ex:11.2.3}
  Prove \cref{i:11.2.13}.
\end{ex}

\begin{proof}
  See \cref{i:11.2.13}.
\end{proof}

\begin{ex}\label{i:ex:11.2.4}
  Prove \cref{i:11.2.16}.
\end{ex}

\begin{proof}
  See \cref{i:11.2.16}.
\end{proof}

\section{Upper and lower Riemann integrals}\label{sec:11.3}

\begin{defn}[Majorization of functions]\label{11.3.1}
  Let \(f : I \to \R\) and \(g : I \to \R\).
  We say that \(g\) \emph{majorizes} \(f\) on \(I\) if we have \(g(x) \geq f(x)\) for all \(x \in I\), and that \(g\) \emph{minorizes} \(f\) on \(I\) if \(g(x) \leq f(x)\) for all \(x \in I\).
\end{defn}

\begin{defn}[Upper and lower Riemann integrals]\label{11.3.2}
  Let \(f : I \to \R\) be a bounded function defined on a bounded interval \(I\).
  We define the \emph{upper Riemann integral} \(\overline{\int}_I f\) by the formula
  \[
    \overline{\int}_I f \coloneqq \inf\bigg\{p.c. \int_I g : g \text{ is a piecewise constant function on \(I\) which majorizes } f\bigg\}
  \]
  and the \emph{lower Riemann integral} \(\underline{\int}_I f\) by the formula
  \[
    \underline{\int}_I f \coloneqq \sup\bigg\{p.c. \int_I g : g \text{ is a piecewise constant function on \(I\) which minorizes } f\bigg\}.
  \]
\end{defn}

\begin{lem}\label{11.3.3}
  Let \(f : I \to \R\) be a function on a bounded interval \(I\) which is bounded by some real number \(M\), i.e., \(-M \leq f(x) \leq M\) for all \(x \in I\).
  Then we have
  \[
    -M \abs{I} \leq \underline{\int}_I f \leq \overline{\int}_I f \leq M \abs{I}.
  \]
  in particular, both the lower and upper Riemann integrals are real numbers (i.e., they are not infinite).
\end{lem}

\begin{proof}
  The function \(g : I \to \R\) defined by \(g(x) = M\) is constant, hence piecewise constant, and majorizes \(f\);
  thus \(\overline{\int}_I f \leq p.c. \int_I g = M \abs{I}\) by definition of the upper Riemann integral.
  A similar argument gives \(-M \abs{I} \leq \underline{\int}_I f\).
  Finally, we have to show that \(\underline{\int}_I f \leq \overline{\int}_I f\).
  Let \(g\) be any piecewise constant function majorizing \(f\), and let \(h\) be any piecewise constant function minorizing \(f\).
  Then \(g\) majorizes \(h\), and hence \(p.c. \int_I h \leq p.c. \int_I g\).
  Taking suprema in \(h\), we obtain that \(\underline{\int}_I f \leq p.c. \int_I g\).
  Taking infima in \(g\), we thus obtain \(\underline{\int}_I f \leq \overline{\int}_I f\), as desired.
\end{proof}

\begin{defn}[Riemann integral]\label{11.3.4}
  Let \(f : I \to \R\) be a bounded function on a bounded interval \(I\).
  If \(\underline{\int}_I f = \overline{\int}_I f\), then we say that \(f\) is \emph{Riemann integrable on \(I\)} and define
  \[
    \int_I f \coloneqq \underline{\int}_I f = \overline{\int}_I f.
  \]
  If the upper and lower Riemann integrals are unequal, we say that \(f\) is not Riemann integrable.
\end{defn}

\begin{rmk}\label{11.3.5}
  Compare this definition to the relationship between the \(\limsup\), \(liminf\), and limit of a sequence \(a_n\) that was established in \cref{6.4.12}(f);
  the \(\limsup\) is always greater than or equal to the \(\liminf\), but they are only equal when the sequence converges, and in this case they are both equal to the limit of the sequence.
  The definition given above may differ from the definition you may have encountered in your calculus courses, based on Riemann sums.
  However, the two definitions turn out to be equivalent.
\end{rmk}

\begin{rmk}\label{11.3.6}
  Note that we do not consider unbounded functions to be Riemann integrable;
  an integral involving such functions is known as an \emph{improper integral}.
  It is possible to still evaluate such integrals using more sophisticated integration methods (such as the Lebesgue integral).
\end{rmk}

\begin{lem}\label{11.3.7}
  Let \(f : I \to \R\) be a piecewise constant function on a bounded interval \(I\).
  Then \(f\) is Riemann integrable, and \(\int_I f = p.c. \int_I f\).
\end{lem}

\begin{proof}
  Since \(f(x) \leq f(x)\) for every \(x \in I\), by \cref{11.3.2} we have
  \[
    \overline{\int}_I f \leq p.c. \int_I f
  \]
  and
  \[
    p.c. \int_I f \leq \underline{\int}_I f.
  \]
  By \cref{11.3.3} we know that
  \[
    p.c. \int_I f \leq \underline{\int}_I f \leq \overline{\int}_I f \leq p.c. \int_I f.
  \]
  Thus by \cref{11.3.4} we have
  \[
    \int_I f = \underline{\int}_I f = \overline{\int}_I f = p.c. \int_I f.
  \]
\end{proof}

\begin{rmk}\label{11.3.8}
  Because of \cref{11.3.7}, we will not refer to the piecewise constant integral \(p.c. \int_I\) again, and just use the Riemann integral \(\int_I\) throughout
  (until this integral is itself superceded by the Lebesgue integral).
  We observe one special case of \cref{11.3.7}:
  if \(I\) is a point or the empty set, then \(\int_I f = 0\) for all functions \(f : I \to \R\).
  (Note that all such functions are automatically constant.)
\end{rmk}

\begin{defn}[Riemann sums]\label{11.3.9}
  Let \(f : I \to \R\) be a bounded function on a bounded interval \(I\), and let \(\mathbf{P}\) be a partition of \(I\).
  We define the \emph{upper Riemann sum} \(U(f, \mathbf{P})\) and the \emph{lower Riemann sum} \(L(f, \mathbf{P})\) by
  \[
    U(f, \mathbf{P}) \coloneqq \sum_{J \in \mathbf{P} : J \neq \emptyset} \big(\sup_{x \in J} f(x)\big) \abs{J}
  \]
  and
  \[
    L(f, \mathbf{P}) \coloneqq \sum_{J \in \mathbf{P} : J \neq \emptyset} \big(\inf_{x \in J} f(x)\big) \abs{J}.
  \]
\end{defn}

\begin{rmk}\label{11.3.10}
  The restriction \(J \neq \emptyset\) is required because the quantities \(\inf_{x \in J} f(x)\) and \(\sup_{x \in J} f(x)\) are infinite (or negative infinite) if \(J\) is empty.
\end{rmk}

\begin{lem}\label{11.3.11}
  Let \(f : I \to \R\) be a bounded function on a bounded interval \(I\), and let \(g\) be a function which majorizes \(f\) and which is piecewise constant with respect to some partition \(\mathbf{P}\) of \(I\).
  Then
  \[
    p.c. \int_I g \geq U(f, \mathbf{P}).
  \]
  Similarly, if \(h\) is a function which minorizes \(f\) and is piecewise constant with respect to \(\mathbf{P}\), then
  \[
    p.c. \int_I h \leq L(f, \mathbf{P}).
  \]
\end{lem}

\begin{proof}
  Since \(g\) majorizes \(f\) and \(h\) minorizes \(f\), by \cref{11.3.1} we have \(h(x) \leq f(x) \leq g(x)\) for every \(x \in I\).
  Since \(\mathbf{P}\) is a partition of \(I\), by \cref{11.1.10} for every \(J \in \mathbf{P}\), we have \(h(x) \leq f(x) \leq g(x)\) for all \(x \in J\).
  In particular, when \(J \neq \emptyset\) we have
  \[
    h(x) \leq \inf_{x \in J} f(x) \leq f(x) \leq \sup_{x \in J} f(x) \leq g(x)
  \]
  for every \(x \in J\).
  Let \(c_{g|_J}, c_{h|_J}\) be constant values of \(g|_J, h|_J\), respectively.
  Then we have
  \begin{align*}
    U(f, \mathbf{P}) & = \sum_{J \in \mathbf{P} : J \neq \emptyset} \big(\sup_{x \in J} f(x)\big) \abs{J} &  & \by{11.3.9}                     \\
                     & \leq \sum_{J \in \mathbf{P} : J \neq \emptyset} c_{g|_J} \abs{J}                   &  & \text{(by \cref{7.1.11}(h))}    \\
                     & = \sum_{J \in \mathbf{P}} c_{g|_J} \abs{J}                                         &  & \text{(by \cref{7.1.11}(a)(e))} \\
                     & = p.c. \int_{[\mathbf{P}]} g                                                       &  & \by{11.2.9}                     \\
                     & = p.c. \int_I g                                                                    &  & \by{11.2.14}
  \end{align*}
  and
  \begin{align*}
    L(f, \mathbf{P}) & = \sum_{J \in \mathbf{P} : J \neq \emptyset} \big(\inf_{x \in J} f(x)\big) \abs{J} &  & \by{11.3.9}                     \\
                     & \geq \sum_{J \in \mathbf{P} : J \neq \emptyset} c_{h|_J} \abs{J}                   &  & \text{(by \cref{7.1.11}(h))}    \\
                     & = \sum_{J \in \mathbf{P}} c_{h|_J} \abs{J}                                         &  & \text{(by \cref{7.1.11}(a)(e))} \\
                     & = p.c. \int_{[\mathbf{P}]} h                                                       &  & \by{11.2.9}                     \\
                     & = p.c. \int_I h.                                                                   &  & \by{11.2.14}
  \end{align*}
\end{proof}

\begin{prop}\label{11.3.12}
  Let \(f : I \to \R\) be a bounded function on a bounded interval \(I\).
  Then
  \[
    \overline{\int}_I f = \inf\{U(f, \mathbf{P}) : \mathbf{P} \text{ is a partition of } I\}
  \]
  and
  \[
    \underline{\int}_I f = \sup\{L(f, \mathbf{P}) : \mathbf{P} \text{ is a partition of } I\}.
  \]
\end{prop}

\begin{proof}
  Let \(g\) be a function which majorizes \(f\) and which is piecewise constant with respect to some partition \(\mathbf{P}_g\) of \(I\).
  Let \(h\) be a function which minorizes \(f\) and which is piecewise constant with respect to some partition \(\mathbf{P}_h\) of \(I\).
  Both functions are well defined since \(f\) is bounded function on a bounded interval \(I\).
  By \cref{11.3.11} we have
  \[
    \inf\big\{U(f, \mathbf{P}) : \mathbf{P} \text{ is a partition of } I\big\} \leq U(f, \mathbf{P}_g) \leq p.c. \int_I g
  \]
  and
  \[
    \sup\big\{L(f, \mathbf{P}) : \mathbf{P} \text{ is a partition of } I\big\} \geq L(f, \mathbf{P}_h) \geq p.c. \int_I h.
  \]
  Since \(g, h\) are arbitrary, by \cref{11.3.2} we have
  \[
    \inf\big\{U(f, \mathbf{P}) : \mathbf{P} \text{ is a partition of } I\big\} \leq \overline{\int}_I f
  \]
  and
  \[
    \sup\big\{L(f, \mathbf{P}) : \mathbf{P} \text{ is a partition of } I\big\} \geq \underline{\int}_I f.
  \]

  Let \(\mathbf{P}\) be a partition of \(I\).
  Let \(G : I \to \R\) be a function where \(G(x) = \sup_{x \in J} f(x)\) for all \(J \in \mathbf{P}\).
  Let \(H : I \to \R\) be a function where \(H(x) = \inf_{x \in J} f(x)\) for all \(J \in \mathbf{P}\).
  By \cref{11.2.3} we know that \(G, H\) are piecewise constant functions with respect to \(\mathbf{P}\).
  Thus we have
  \begin{align*}
    U(f, \mathbf{P}) & = \sum_{J \in \mathbf{P} : J \neq \emptyset} \big(\sup_{x \in J} f(x)\big) \abs{J} &  & \by{11.3.9}                     \\
                     & = \sum_{J \in \mathbf{P}} \big(\sup_{x \in J} f(x)\big) \abs{J}                    &  & \text{(by \cref{7.1.11}(a)(e))} \\
                     & = p.c. \int_{[\mathbf{P}]} G                                                       &  & \by{11.2.9}                     \\
                     & = p.c. \int_I G                                                                    &  & \by{11.2.14}
  \end{align*}
  and
  \begin{align*}
    L(f, \mathbf{P}) & = \sum_{J \in \mathbf{P} : J \neq \emptyset} \big(\inf_{x \in J} f(x)\big) \abs{J} &  & \by{11.3.9}                     \\
                     & = \sum_{J \in \mathbf{P}} \big(\inf_{x \in J} f(x)\big) \abs{J}                    &  & \text{(by \cref{7.1.11}(a)(e))} \\
                     & = p.c. \int_{[\mathbf{P}]} H                                                       &  & \by{11.2.9}                     \\
                     & = p.c. \int_I H.                                                                   &  & \by{11.2.14}
  \end{align*}
  By \cref{11.3.2} we have
  \[
    \overline{\int}_I f \leq p.c. \int_I G = U(f, \mathbf{P})
  \]
  and
  \[
    \underline{\int}_I f \geq p.c. \int_I H = L(f, \mathbf{P}).
  \]
  Since \(\mathbf{P}\) is arbitrary, we have
  \[
    \overline{\int}_I f \leq \inf\big\{U(f, \mathbf{P}) : \mathbf{P} \text{ is a partition of } I\big\} \leq U(f, \mathbf{P})
  \]
  and
  \[
    \underline{\int}_I f \geq \sup\big\{L(f, \mathbf{P}) : \mathbf{P} \text{ is a partition of } I\big\} \leq L(f, \mathbf{P}).
  \]
  Combine all results above we have
  \[
    \overline{\int}_I f = \inf\big\{U(f, \mathbf{P}) : \mathbf{P} \text{ is a partition of } I\big\}
  \]
  and
  \[
    \underline{\int}_I f = \sup\big\{L(f, \mathbf{P}) : \mathbf{P} \text{ is a partition of } I\big\}.
  \]
\end{proof}

\exercisesection

\begin{ex}\label{ex:11.3.1}
  Let \(f : I \to \R\), \(g : I \to \R\), and \(h : I \to \R\) be functions.
  Show that if \(f\) majorizes \(g\) and \(g\) majorizes \(h\), then \(f\) majorizes \(h\).
  Show that if \(f\) and \(g\) majorize each other, then they must be equal.
\end{ex}

\begin{proof}
  We first show that if \(f\) majorizes \(g\) and \(g\) majorizes \(h\), then \(f\) majorizes \(h\).
  Since
  \begin{align*}
             & \forall x \in I, f(x) \geq g(x) \geq h(x) &  & \by{11.3.1} \\
    \implies & f(x) \geq h(x),
  \end{align*}
  by \cref{11.3.1} we know that \(f\) majorize \(h\).

  Now we show that if \(f\) and \(g\) majorize each other, then they must be equal.
  Since
  \begin{align*}
             & \forall x \in I, \big(f(x) \geq g(x)\big) \land \big(g(x) \geq f(x)\big) &  & \by{11.3.1} \\
    \implies & f(x) = g(x),
  \end{align*}
  by \cref{3.3.7} we know that \(f = g\).
\end{proof}

\begin{ex}\label{ex:11.3.2}
  Let \(f : I \to \R\), \(g : I \to \R\), and \(h : I \to \R\) be functions.
  If \(f\) majorizes \(g\), is it true that \(f + h\) majorizes \(g + h\)?
  Is it true that \(f \cdot h\) majorizes \(g \cdot h\)?
  If \(c\) is a real number, is it true that \(cf\) majorizes \(cg\)?
\end{ex}

\begin{proof}
  We first show that if \(f\) majorizes \(g\), then \(f + h\) majorizes \(g + h\).
  Since
  \begin{align*}
             & \forall x \in I, f(x) \geq g(x) &  & \by{11.3.1} \\
    \implies & f(x) + h(x) \geq g(x) + h(x)                     \\
    \implies & (f + h)(x) \geq (g + h)(x),     &  & \by{9.2.1}
  \end{align*}
  by \cref{11.3.1} we know that \(f + h\) majorizes \(g + h\).

  Now we show that \(f \cdot h\) may not majorized \(g \cdot h\) and \(cf\) may not majorize \(cg\).
  Let \(c = h(x) = -1\).
  Then we have
  \begin{align*}
             & \forall x \in I, f(x) \geq g(x)                         &  & \by{11.3.1} \\
    \implies & cf(x) = f(x) h(x) \leq cg(x) = g(x) h(x)                                 \\
    \implies & (cf)(x) = (f \cdot h)(x) \leq (cg)(x) = (g \cdot h)(x). &  & \by{9.2.1}
  \end{align*}
  In this case \(f \cdot h\) does not majorized \(g \cdot h\) and \(cf\) does not majorized \(cg\).
\end{proof}

\begin{ex}\label{ex:11.3.3}
  Prove \cref{11.3.7}.
\end{ex}

\begin{proof}
  See \cref{11.3.7}.
\end{proof}

\begin{ex}\label{ex:11.3.4}
  Prove \cref{11.3.11}.
\end{ex}

\begin{proof}
  See \cref{11.3.11}.
\end{proof}

\begin{ex}\label{ex:11.3.5}
  Prove \cref{11.3.12}.
\end{ex}

\begin{prop}
  See \cref{11.3.12}.
\end{prop}
\section{Basic properties of the Riemann integral}\label{i:sec:11.4}

\begin{thm}[Laws of Riemann integration]\label{i:11.4.1}
  Let \(I\) be a bounded interval, and let \(f : I \to \R\) and \(g : I \to \R\) be Riemann integrable functions on \(I\).
  \begin{enumerate}
    \item The function \(f + g\) is Riemann integrable, and we have \(\int_I (f + g) = \int_I f + \int_I g\).
    \item For any real number \(c\), the function \(cf\) is Riemann integrable, and we have \(\int_I (cf) = c(\int_I f)\).
    \item The function \(f - g\) is Riemann integrable, and we have \(\int_I (f - g) = \int_I f - \int_I g\).
    \item If \(f(x) \geq 0\) for all \(x \in I\), then \(\int_I f \geq 0\).
    \item If \(f(x) \geq g(x)\) for all \(x \in I\), then \(\int_I f \geq \int_I g\).
    \item If \(f\) is the constant function \(f(x) = c\) for all \(x \in I\), then \(\int_I f = c \abs{I}\).
    \item Let \(J\) be a bounded interval containing \(I\) (i.e., \(I \subseteq J\)), and let \(F : J \to \R\) be the function
          \[
            F(x) \coloneqq \begin{dcases}
              f(x) & \text{if } x \in I    \\
              0    & \text{if } x \notin I \\
            \end{dcases}
          \]
          Then \(F\) is Riemann integrable on \(J\), and \(\int_J F = \int_I f\).
    \item Suppose that \(\set{J, K}\) is a partition of \(I\) into two intervals \(J\) and \(K\).
          Then the functions \(f|_J : J \to \R\) and \(f|_K : K \to \R\) are Riemann integrable on \(J\) and \(K\) respectively, and we have
          \[
            \int_I f = \int_J f|_J + \int_K f|_K.
          \]
  \end{enumerate}
\end{thm}

\begin{proof}{(a)}
  Since \(f, g\) are Riemann integrable on \(I\), by \cref{i:11.3.4} we have
  \[
    \int_I f = \overline{\int}_I f = \underline{\int}_I f
  \]
  and
  \[
    \int_I g = \overline{\int}_I g = \underline{\int}_I g.
  \]
  Let \(f_U : I \to \R\) and \(g_U : I \to \R\) be piecewise constant functions on \(I\) which majorizes \(f\) and \(g\), respectively.
  Let \(f_L : I \to \R\) and \(g_L : I \to \R\) be piecewise constant functions on \(I\) which minorizes \(f\) and \(g\), respectively.
  \(f_U, g_U, f_L, g_L\) are well-defined since by \cref{i:11.3.4} \(f, g\) are bounded functions on a bounded interval \(I\).
  By \cref{i:11.3.2} we have
  \[
    p.c. \int_I f_L \leq \underline{\int}_I f = \int_I f = \overline{\int}_I f \leq p.c. \int_I f_U
  \]
  and
  \[
    p.c. \int_I g_L \leq \underline{\int}_I g = \int_I g = \overline{\int}_I g \leq p.c. \int_I g_U.
  \]
  By \cref{i:11.3.4} both \(f, g\) are bounded functions, so \(f + g\) is bounded function, and \(\underline{\int}_I (f + g), \overline{\int}_I (f + g)\) are well-defined (by \cref{i:11.3.2}).
  By \cref{i:ex:11.3.2} we know that \(f_U + g_U\) majorizes \(f + g_U\) and \(f + g_U\) majorizes \(f + g\), thus \(f_U + g_U\) majorizes \(f + g\).
  Similarly, \(f_L + g_L\) minorizes \(f + g\).
  Then we have
  \begin{align*}
             & \overline{\int}_I (f + g) \leq p.c. \int_I (f_U + g_U)                   &   & \by{i:11.3.2}                            \\
    \implies & \overline{\int}_I (f + g) \leq p.c. \int_I f_U + p.c. \int_I g_U         &   & \by{i:11.2.16}[a]                        \\
    \implies & \overline{\int}_I (f + g) - p.c. \int_I g_U \leq p.c. \int_I f_U         &   & \text{(note that \(f_U\) was arbitrary)} \\
    \implies & \overline{\int}_I (f + g) - p.c. \int_I g_U \leq \overline{\int}_I f     &   & \by{i:11.3.2}                            \\
    \implies & \overline{\int}_I (f + g) - \overline{\int}_I f \leq p.c. \int_I g_U     &   & \text{(note that \(g_U\) was arbitrary)} \\
    \implies & \overline{\int}_I (f + g) - \overline{\int}_I f \leq \overline{\int}_I g &   & \by{i:11.3.2}                            \\
    \implies & \overline{\int}_I (f + g) \leq \overline{\int}_I f + \overline{\int}_I g &                                              \\
    \implies & \overline{\int}_I (f + g) \leq \int_I f + \int_I g                       &   & \by{i:11.3.4}
  \end{align*}
  and
  \begin{align*}
             & \underline{\int}_I (f + g) \geq p.c. \int_I (f_L + g_L)                     &   & \by{i:11.3.2}                            \\
    \implies & \underline{\int}_I (f + g) \geq p.c. \int_I f_L + p.c. \int_I g_L           &   & \by{i:11.2.16}[a]                        \\
    \implies & \underline{\int}_I (f + g) - p.c. \int_I g_L \geq p.c. \int_I f_L           &   & \text{(note that \(f_L\) was arbitrary)} \\
    \implies & \underline{\int}_I (f + g) - p.c. \int_I g_L \geq \underline{\int}_I f      &   & \by{i:11.3.2}                            \\
    \implies & \underline{\int}_I (f + g) - \underline{\int}_I f \geq p.c. \int_I g_L      &   & \text{(note that \(g_L\) was arbitrary)} \\
    \implies & \underline{\int}_I (f + g) - \underline{\int}_I f \geq \underline{\int}_I g &   & \by{i:11.3.2}                            \\
    \implies & \underline{\int}_I (f + g) \geq \underline{\int}_I f + \underline{\int}_I g &                                              \\
    \implies & \underline{\int}_I (f + g) \geq \int_I f + \int_I g.                        &   & \by{i:11.3.4}
  \end{align*}
  By \cref{i:11.3.3} we have
  \[
    \int_I f + \int_I g \leq \underline{\int}_I (f + g) \leq \overline{\int}_I (f + g) \leq \int_I f + \int_I g
  \]
  and thus by \cref{i:11.3.4} we have
  \[
    \int_I (f + g) = \underline{\int}_I (f + g) = \overline{\int}_I (f + g) = \int_I f + \int_I g.
  \]
\end{proof}

\begin{proof}{(b)}
  Since \(f\) is Riemann integrable on \(I\), by \cref{i:11.3.4} we have
  \[
    \int_I f = \overline{\int}_I f = \underline{\int}_I f.
  \]
  First suppose that \(c = 0\).
  Then we have \((cf)(x) = 0\) for all \(x \in 0\), thus we have
  \begin{align*}
    \int_I (cf) & = p.c. \int_I (cf) &  & \by{i:11.3.7} \\
                & = 0                                   \\
                & = c \int_I f.
  \end{align*}

  Next suppose that \(c > 0\).
  Let \(f_U : I \to \R\) be a piecewise constant function on \(I\) which majorizes \(f\).
  Let \(f_L : I \to \R\) be a piecewise constant function on \(I\) which minorizes \(f\).
  \(f_U, f_L\) are well-defined since by \cref{i:11.3.4} \(f\) is a bounded function on a bounded interval \(I\).
  Then by \cref{i:11.3.2} we have
  \[
    p.c. \int_I f_L \leq \underline{\int}_I f = \int_I f = \overline{\int}_I f \leq p.c. \int_I f_U.
  \]
  Since \(f\) is a bounded function, \(cf\) is also a bounded function, by \cref{i:11.3.2} both \(\overline{\int}_I (cf), \underline{\int}_I (cf)\) are well-defined.
  Since \(c > 0\), by \cref{i:11.3.1} we know that \(c f_U\) majorizes \(c f\) and \(c f_L\) minorizes \(c f\).
  Then we have
  \begin{align*}
             & \overline{\int}_I (cf) \leq p.c. \int_I (c f_U)                          &  & \by{i:11.3.2}                            \\
    \implies & \overline{\int}_I (cf) \leq c \bigg(p.c. \int_I f_U\bigg)                &  & \by{i:11.2.16}[b]                        \\
    \implies & \dfrac{1}{c} \bigg(\overline{\int}_I (cf)\bigg) \leq p.c. \int_I f_U     &  & \text{(note that \(f_U\) was arbitrary)} \\
    \implies & \dfrac{1}{c} \bigg(\overline{\int}_I (cf)\bigg) \leq \overline{\int}_I f &  & \by{i:11.3.2}                            \\
    \implies & \overline{\int}_I (cf) \leq c\bigg(\overline{\int}_I f\bigg)                                                           \\
    \implies & \overline{\int}_I (cf) \leq c\bigg(\int_I f\bigg)                        &  & \by{i:11.3.4}
  \end{align*}
  and
  \begin{align*}
             & \underline{\int}_I (cf) \geq p.c. \int_I (c f_L)                           &  & \by{i:11.3.2}                            \\
    \implies & \underline{\int}_I (cf) \geq c \bigg(p.c. \int_I f_L\bigg)                 &  & \by{i:11.2.16}[b]                        \\
    \implies & \dfrac{1}{c} \bigg(\underline{\int}_I (cf)\bigg) \geq p.c. \int_I f_L      &  & \text{(note that \(f_L\) was arbitrary)} \\
    \implies & \dfrac{1}{c} \bigg(\underline{\int}_I (cf)\bigg) \geq \underline{\int}_I f &  & \by{i:11.3.2}                            \\
    \implies & \underline{\int}_I (cf) \geq c\bigg(\underline{\int}_I f\bigg)                                                           \\
    \implies & \underline{\int}_I (cf) \geq c\bigg(\int_I f\bigg).                        &  & \by{i:11.3.4}
  \end{align*}
  By \cref{i:11.3.3} we have
  \[
    c\bigg(\int_I f\bigg) \leq \underline{\int}_I (cf) \leq \overline{\int}_I (cf) \leq c\bigg(\int_I f\bigg)
  \]
  and thus by \cref{i:11.3.4} we have
  \[
    \int_I (cf) = \underline{\int}_I (cf) = \overline{\int}_I (cf) = c\bigg(\int_I f\bigg).
  \]

  Finally suppose that \(c < 0\).
  Using the same definition of \(f_U, f_L\) we have
  \begin{align*}
             & \overline{\int}_I (cf) \leq p.c. \int_I (c f_U)                                               &  & \by{i:11.3.2}     \\
    \implies & \overline{\int}_I (cf) \leq c \bigg(p.c. \int_I f_U\bigg)                                     &  & \by{i:11.2.16}[b] \\
    \implies & \dfrac{1}{c} \bigg(\overline{\int}_I (cf)\bigg) \geq p.c. \int_I f_U                                                 \\
    \implies & \dfrac{1}{c} \bigg(\overline{\int}_I (cf)\bigg) \geq p.c. \int_I f_U \geq \overline{\int}_I f &  & \by{i:11.3.2}     \\
    \implies & \overline{\int}_I (cf) \leq c\bigg(\overline{\int}_I f\bigg)                                                         \\
    \implies & \overline{\int}_I (cf) \leq c\bigg(\int_I f\bigg)                                             &  & \by{i:11.3.4}
  \end{align*}
  and
  \begin{align*}
             & \underline{\int}_I (cf) \geq p.c. \int_I (c f_L)                                                &  & \by{i:11.3.2}     \\
    \implies & \underline{\int}_I (cf) \geq c \bigg(p.c. \int_I f_L\bigg)                                      &  & \by{i:11.2.16}[b] \\
    \implies & \dfrac{1}{c} \bigg(\underline{\int}_I (cf)\bigg) \leq p.c. \int_I f_L                                                  \\
    \implies & \dfrac{1}{c} \bigg(\underline{\int}_I (cf)\bigg) \leq p.c. \int_I f_L \leq \underline{\int}_I f &  & \by{i:11.3.2}     \\
    \implies & \underline{\int}_I (cf) \geq c\bigg(\underline{\int}_I f\bigg)                                                         \\
    \implies & \underline{\int}_I (cf) \geq c\bigg(\int_I f\bigg).                                             &  & \by{i:11.3.4}
  \end{align*}
  By \cref{i:11.3.3} we have
  \[
    c\bigg(\int_I f\bigg) \leq \underline{\int}_I (cf) \leq \overline{\int}_I (cf) \leq c\bigg(\int_I f\bigg)
  \]
  and thus by \cref{i:11.3.4} we have
  \[
    \int_I (cf) = \underline{\int}_I (cf) = \overline{\int}_I (cf) = c\bigg(\int_I f\bigg).
  \]
  We conclude that \(\forall c \in \R\), \(\int_I (cf) = c (\int_I f)\).
\end{proof}

\begin{proof}{(c)}
  We have
  \begin{align*}
    \int_I f - \int_I g & = \int_I f + \int_I (-g)    &  & \by{i:11.4.1}[b] \\
                        & = \int_I \big(f + (-g)\big) &  & \by{i:11.4.1}[a] \\
                        & = \int_I (f - g).           &  & \by{i:9.2.1}
  \end{align*}
\end{proof}

\begin{proof}{(d)}
  Let \(f_U : I \to \R\) be a piecewise constant function on \(I\) which majorizes \(f\).
  \(f_U\) is well-defined since by \cref{i:11.3.4} \(f\) is a bounded function on a bounded interval \(I\).
  Since \(0 \leq f(x) \leq f_U(x)\) for every \(x \in I\), we have
  \begin{align*}
             & 0 \leq p.c. \int_I f_U     &  & \by{i:11.2.16}[d] \\
    \implies & 0 \leq \overline{\int}_I f &  & \by{i:11.3.2}     \\
    \implies & 0 \leq \int_I f.           &  & \by{i:11.3.4}
  \end{align*}
\end{proof}

\begin{proof}{(e)}
  We have \(f(x) - g(x) \geq 0\) for every \(x \in I\) and by \cref{i:11.4.1}(c) \(f - g\) is Riemann integrable on \(I\).
  Thus
  \begin{align*}
             & \int_I (f - g) \geq 0      &  & \by{i:11.4.1}[d] \\
    \implies & \int_I f - \int_I g \geq 0 &  & \by{i:11.4.1}[c] \\
    \implies & \int_I f \geq \int_I g.
  \end{align*}
\end{proof}

\begin{proof}{(f)}
  We have
  \begin{align*}
    \int_I f & = p.c. \int_I f &  & \by{i:11.3.7}     \\
             & = c \abs{I}.    &  & \by{i:11.2.16}[f]
  \end{align*}
\end{proof}

\begin{proof}{(g)}
  Let \(f_U : I \to \R\) be a piecewise constant function on \(I\) which majorizes \(f\).
  Let \(f_L : I \to \R\) be a piecewise constant function on \(I\) which minorizes \(f\).
  \(f_U, f_L\) are well-defined since by \cref{i:11.3.4} \(f\) is a bounded function on a bounded interval \(I\).
  Then by \cref{i:11.3.2} we have
  \[
    p.c. \int_I f_L \leq \underline{\int}_I f = \int_I f = \overline{\int}_I f \leq p.c. \int_I f_U.
  \]
  Let \(F_U : J \to \R\) be the function
  \[
    F_U(x) = \begin{dcases}
      f_U(x) & \text{if } x \in I    \\
      0      & \text{if } x \notin I
    \end{dcases}
  \]
  and let \(F_L : J \to \R\) be the function
  \[
    F_L(x) = \begin{dcases}
      f_L(x) & \text{if } x \in I     \\
      0      & \text{if } x \notin I.
    \end{dcases}
  \]
  We know that \(F_U\) majorizes \(F\) and \(F_L\) minorizes \(F\), and by \cref{i:11.2.16}(g) we have \(p.c. \int_J F_U = p.c. \int_I f_U\) and \(p.c. \int_J F_L = p.c. \int_I f_L\).
  Thus, \(F\) is a bounded function on a bounded interval \(I\), and we have
  \begin{align*}
             & \overline{\int}_J F \leq p.c. \int_J F_U     &  & \by{i:11.3.2}                                         \\
    \implies & \overline{\int}_J F \leq p.c. \int_I f_U     &  & \by{i:11.2.16}[g]                                     \\
    \implies & \overline{\int}_J F \leq \overline{\int}_I f &  & \text{(by \cref{i:11.3.2} and \(f_U\) was arbitrary)} \\
    \implies & \overline{\int}_J F \leq \int_I f            &  & \by{i:11.3.4}
  \end{align*}
  and
  \begin{align*}
             & \underline{\int}_J F \geq p.c. \int_J F_L      &  & \by{i:11.3.2}                                         \\
    \implies & \underline{\int}_J F \geq p.c. \int_I f_L      &  & \by{i:11.2.16}[g]                                     \\
    \implies & \underline{\int}_J F \geq \underline{\int}_I f &  & \text{(by \cref{i:11.3.2} and \(f_L\) was arbitrary)} \\
    \implies & \underline{\int}_J F \geq \int_I f.            &  & \by{i:11.3.4}
  \end{align*}
  By \cref{i:11.3.3} we have
  \[
    \int_I f \leq \underline{\int}_J F \leq \overline{\int}_J F \leq \int_I f
  \]
  and thus by \cref{i:11.3.4} we have
  \[
    \int_J F = \underline{\int}_J F = \overline{\int}_J F = \int_I f.
  \]
\end{proof}

\begin{proof}{(h)}
  Let \(f_U : I \to \R\) be a piecewise constant function on \(I\) which majorizes \(f\).
  Let \(f_L : I \to \R\) be a piecewise constant function on \(I\) which minorizes \(f\).
  \(f_U, f_L\) are well-defined since by \cref{i:11.3.4} \(f\) is a bounded function on a bounded interval \(I\).
  Then by \cref{i:11.3.2} we have
  \[
    p.c. \int_I f_L \leq \underline{\int}_I f = \int_I f = \overline{\int}_I f \leq p.c. \int_I f_U.
  \]
  By \cref{i:11.2.16}(h) we know that \(f_U|_J : J \to \R, f_L|_J : J \to \R\) are piecewise constant function on \(J\) and \(f_U|_K : K \to \R\), \(f_L|_K : K \to \R\) are piecewise constant functions on \(K\).
  By \cref{i:11.3.1} we know that \(f_U|_J\) majorizes \(f|_J\) and \(f_L|_J\) minorizes \(f|_J\), similarly \(f_U|_K\) majorizes \(f|_K\) and \(f_L|_K\) minorizes \(f|_K\).
  Thus, \(f|_J, f|_K\) are bounded functions on bounded intervals \(J, K\), respectively.
  So \(\overline{\int}_J f|_J\), \(\overline{\int}_K f|_K\), \(\underline{\int}_J f|_J\), \(\underline{\int}_K f|_K\) are well-defined.
  Then we have
  \begin{align*}
             & \overline{\int}_J f|_J + \overline{\int}_K f|_K \leq p.c. \int_J f_U|_J + p.c. \int_K f_U|_K &  & \by{i:11.3.2}     \\
    \implies & \overline{\int}_J f|_J + \overline{\int}_K f|_K \leq p.c. \int_I f_U                         &  & \by{i:11.2.16}[h] \\
    \implies & \overline{\int}_J f|_J + \overline{\int}_K f|_K \leq \overline{\int}_I f                     &  & \by{i:11.3.2}     \\
    \implies & \overline{\int}_J f|_J + \overline{\int}_K f|_K \leq \int_I f                                &  & \by{i:11.3.4}
  \end{align*}
  and
  \begin{align*}
             & \underline{\int}_J f|_J + \underline{\int}_K f|_K \geq p.c. \int_J f_L|_J + p.c. \int_K f_L|_K &  & \by{i:11.3.2}     \\
    \implies & \underline{\int}_J f|_J + \underline{\int}_K f|_K \geq p.c. \int_I f_L                         &  & \by{i:11.2.16}[h] \\
    \implies & \underline{\int}_J f|_J + \underline{\int}_K f|_K \geq \underline{\int}_I f                    &  & \by{i:11.3.2}     \\
    \implies & \underline{\int}_J f|_J + \underline{\int}_K f|_K \geq \int_I f.                               &  & \by{i:11.3.4}
  \end{align*}
  By \cref{i:11.3.3} we have
  \[
    \int_I f \leq \underline{\int}_J f|_J + \underline{\int}_K f|_K \leq \overline{\int}_J f|_J + \overline{\int}_K f|_K \leq \int_I f
  \]
  and thus we have
  \[
    \underline{\int}_J f|_J + \underline{\int}_K f|_K = \overline{\int}_J f|_J + \overline{\int}_J f|_K = \int_I f.
  \]
  Since
  \begin{align*}
             & \underline{\int}_J f|_J + \underline{\int}_K f|_K = \overline{\int}_J f|_J + \overline{\int}_J f|_K                                  \\
    \implies & 0 \geq \underline{\int}_J f|_J - \overline{\int}_J f|_J = \overline{\int}_J f|_K - \underline{\int}_K f|_K \geq 0 &  & \by{i:11.3.3} \\
    \implies & \underline{\int}_J f|_J - \overline{\int}_J f|_J = \overline{\int}_J f|_K - \underline{\int}_K f|_K = 0,
  \end{align*}
  by \cref{i:11.3.4} we have
  \begin{align*}
     & \int_J f|_J = \underline{\int}_J f|_J = \overline{\int}_J f|_J, \\
     & \int_K f|_K = \underline{\int}_K f|_K = \overline{\int}_K f|_K, \\
     & \int_J f|_J + \int_K f|_K = \int_I f.
  \end{align*}
\end{proof}

\begin{rmk}\label{i:11.4.2}
  We often abbreviate \(\int_J f|_J\) as \(\int_J f\) even though \(f\) is really defined on a larger domain than just \(J\).
  We also observe from \cref{i:11.4.1}(h) and \cref{i:11.3.8} that if \(f : [a, b] \to \R\) is Riemann integrable on a closed interval \([a, b]\), then \(\int_{[a, b]} f = \int_{(a, b]} f = \int_{[a, b)} f = \int_{(a, b)} f\).
\end{rmk}

\begin{thm} and min preserve integrability]\label{i:11.4.3}
  Let \(I\) be a bounded interval, and let \(f : I \to \R\) and \(g : I \to \R\) be a Riemann integrable function.
  Then the functions \(\max(f, g) : I \to \R\) and \(\min(f, g) : I \to \R\) defined by \(\max(f, g)(x) \coloneqq \max\big(f(x), g(x)\big)\) and \(\min(f, g)(x) \coloneqq \min\big(f(x), g(x)\big)\) are also Riemann integrable.
\end{thm}

\begin{proof}
  We shall just prove the claim for \(\max(f, g)\), the case of \(\min(f, g)\) being similar.
  First note that since \(f\) and \(g\) are bounded, then \(\max(f, g)\) is also bounded.

  Let \(\varepsilon > 0\).
  Since \(\int_I f = \underline{\int}_I f\), there exists a piecewise constant function \(\underline{f} : I \to \R\) which minorizes \(f\) on \(I\) such that
  \[
    \int_I \underline{f} \geq \int_I f - \varepsilon.
  \]
  Similarly, we can find a piecewise constant \(g : I \to \R\) which minorizes \(g\) on \(I\) such that
  \[
    \int_I \underline{g} \geq \int_I g - \varepsilon,
  \]
  and we can find piecewise functions \(\overline{f}, \overline{g}\) which majorize \(f, g\) respectively on \(I\) such that
  \[
    \int_I \overline{f} \leq \int_I f + \varepsilon
  \]
  and
  \[
    \int_I \overline{g} \leq \int_I g + \varepsilon.
  \]
  In particular, if \(h : I \to \R\) denotes the function
  \[
    h \coloneqq (\overline{f} - \underline{f}) + (\overline{g} - \underline{g})
  \]
  we have
  \[
    \int_I h \leq 4 \varepsilon.
  \]
  On the other hand, \(\max(\underline{f}, \underline{g})\) is a piecewise constant function on \(I\) which minorizes \(\max(f, g)\), while \(\max(\overline{f}, \overline{g})\) is similarly a piecewise constant function on \(I\) which majorizes \(\max(f, g)\).
  Thus
  \[
    \int_I \max(\underline{f}, \underline{g}) \leq \underline{\int}_I \max(f, g) \leq \overline{\int}_I \max(f, g) \leq \int_I \max(\overline{f}, \overline{g}),
  \]
  and so
  \[
    0 \leq \overline{\int}_I \max(f, g) - \underline{\int}_I \max(f, g) \leq \int_I \max(\overline{f}, \overline{g}) - \max(\underline{f}, \underline{g}).
  \]
  But we have
  \[
    \overline{f}(x) = \underline{f}(x) + (\overline{f} - \underline{f})(x) \leq \underline{f}(x) + h(x)
  \]
  and similarly
  \[
    \overline{g}(x) = \underline{g}(x) + (\overline{g} - \underline{g})(x) \leq \underline{g}(x) + h(x)
  \]
  and thus
  \[
    \max\big(\overline{f}(x), \overline{g}(x)\big) \leq \max\big(\underline{f}(x), \underline{g}(x)\big) + h(x).
  \]
  Inserting this into the previous inequality, we obtain
  \[
    0 \leq \overline{\int}_I \max(f, g) - \underline{\int}_I \max(f, g) \leq \int_I h \leq 4 \varepsilon.
  \]
  To summarize, we have shown that
  \[
    0 \leq \overline{\int}_I \max(f, g) - \underline{\int}_I \max(f, g) \leq 4 \varepsilon
  \]
  for every \(\varepsilon\).
  Since \(\overline{\int}_I \max(f, g) - \underline{\int}_I \max(f, g)\) does not depend on \(\varepsilon\), we thus see that
  \[
    \overline{\int}_I \max(f, g) - \underline{\int}_I \max(f, g) = 0
  \]
  and hence that \(\max(f, g)\) is Riemann integrable.
\end{proof}

\begin{cor}[Absolute values preserve Riemann integrability]\label{i:11.4.4}
  \quad
  Let \(I\) be a bounded interval.
  If \(f : I \to \R\) is a Riemann integrable function, then the positive part \(f_+ \coloneqq \max(f, 0)\) and the negative part \(f_- \coloneqq \min(f, 0)\) are also Riemann integrable on \(I\).
  Also, the absolute value \(\abs{f}\), defined by \(\abs{f}(x) = \abs{f(x)}\) is also Riemann integrable on \(I\).
  (observe that \(\abs{f} = f_+ - f_-\))
\end{cor}

\begin{proof}
  By \cref{i:11.4.3} we know that \(f_+, f_-\) are Riemann integrable.
  Since \(\abs{f} = f_+ - f_-\), by \cref{i:11.4.1}(a) we know that \(\abs{f}\) is Riemann integrable.
\end{proof}

\begin{thm}[products preserve Riemann integrability]\label{i:11.4.5}
  Let \(I\) be a bounded interval.
  If \(f : I \to \R\) and \(g : I \to \R\) are Riemann integrable, then \(fg : I \to \R\) is also Riemann integrable.
\end{thm}

\begin{proof}
  We split \(f = f_+ + f_-\) and \(g = g_+ + g_-\) into positive and negative parts;
  by \cref{i:11.4.4}, the functions \(f_+\), \(f_-\), \(g_+\), \(g_-\) are Riemann integrable.
  Since
  \[
    fg = f_+ g_+ + f_+ g_- + f_- g_+ + f_- g_-
  \]
  then it suffices to show that the functions \(f_+ g_+\), \(f_+ g_-\), \(f_- g_+\), \(f_- g_-\) are individually Riemann integrable.
  We will just show this for \(f_+ g_+\);
  the other three are similar.

  Since \(f_+\) and \(g_+\) are bounded and positive, there are \(M_1, M_2 > 0\) such that
  \[
    0 \leq f_+(x) \leq M_1 \text{ and } 0 \leq g_+(x) \leq M_2
  \]
  for all \(x \in I\).
  Now let \(\varepsilon > 0\) be arbitrary.
  Then, as in the proof of \cref{i:11.4.3}, we can find a piecewise constant function \(\underline{f_+}\) minorizing \(f_+\) on \(I\), and a piecewise constant function \(\overline{f_+}\) majorizing \(f_+\) on \(I\), such that
  \[
    \int_I \overline{f_+} \leq \int_I f_+ + \varepsilon
  \]
  and
  \[
    \int_I \underline{f_+} \geq \int_I f_+ - \varepsilon.
  \]
  Note that \(\underline{f_+}\) may be negative at places, but we can fix this by replacing \(\underline{f_+}\) by \(\max(\underline{f_+}, 0)\), since this still minorizes \(f_+\) and still has integral greater than or equal to \(\int_I f_+ - \varepsilon\).
  So without loss of generality we may assume that \(\underline{f_+}(x) \geq 0\) for all \(x \in I\).
  Similarly, we may assume that \(\overline{f_+}(x) \leq M_1\) for all \(x \in I\);
  thus
  \[
    0 \leq \underline{f_+}(x) \leq f_+(x) \leq \overline{f_+}(x) \leq M_1
  \]
  for all \(x \in I\).

  Similar reasoning allows us to find piecewise constant \(\underline{g_+}\) minorizing \(g_+\), and \(\overline{g_+}\) majorizing \(g_+\), such that
  \[
    \int_I \overline{g_+} \leq \int_I g_+ + \varepsilon
  \]
  and
  \[
    \int_I \underline{g_+} \geq \int_I g_+ - \varepsilon,
  \]
  and
  \[
    0 \leq \underline{g_+}(x) \leq g_+(x) \leq \overline{g_+}(x) \leq M_2
  \]
  for all \(x \in I\).

  Notice that \(\underline{f_+} \underline{g_+}\) is piecewise constant and minorizes \(f_+ g_+\), while \(\overline{f_+} \overline{g_+}\) is piecewise constant and majorizes \(f_+ g_+\).
  Thus
  \[
    0 \leq \overline{\int}_I f_+ g_+ - \underline{\int}_I f_+ g_+ \leq \int_I \overline{f}_+ \overline{g_+} - \underline{f_+} \underline{g_+}.
  \]
  However, we have
  \begin{align*}
    \overline{f_+}(x) \overline{g_+}(x) - \underline{f_+}(x) \underline{g_+}(x) & = \overline{f_+}(x) (\overline{g_+} - \underline{g_+})(x) + \underline{g_+}(x) (\overline{f_+} - \underline{f_+})(x) \\
                                                                                & \leq M_1 (\overline{g_+} - \underline{g_+})(x) + M_2 (\overline{f_+} - \underline{f_+})(x)
  \end{align*}
  for all \(x \in I\), and thus
  \begin{align*}
    0 \leq \overline{\int}_I f_+ g_+ - \underline{\int}_I f_+ g_+ & \leq M_1 \int_I (\overline{g_+} - \underline{g_+}) + M_2 \int_I (\overline{f_+} - \underline{f_+}) \\
                                                                  & \leq M_1 (2\varepsilon) + M_2 (2\varepsilon).
  \end{align*}
  Again, since \(\varepsilon\) was arbitrary, we can conclude that \(f_+ g_+\) is Riemann integrable, as before.
  Similar arguments show that \(f_+ g_-\), \(f_- g_+\), \(f_- g_-\) are Riemann integrable;
  combining them we obtain that \(fg\) is Riemann integrable.
\end{proof}

\exercisesection

\begin{ex}\label{i:ex:11.4.1}
  Prove \cref{i:11.4.1}.
\end{ex}

\begin{proof}
  See \cref{i:11.4.1}.
\end{proof}

\begin{ex}\label{i:ex:11.4.2}
  Let \(a < b\) be real numbers, and let \(f : [a, b] \to \R\) be a continuous, non-negative function
  (so \(f(x) \geq 0\) for all \(x \in [a, b]\)).
  Suppose that \(\int_{[a, b]} f = 0\).
  Show that \(f(x) = 0\) for all \(x \in [a, b]\).
\end{ex}

\begin{proof}
  Suppose for the sake of contradiction that \(\exists x_0 \in [a, b]\) such that \(f(x_0) > 0\).
  Since \(f\) is continuous, by \cref{i:9.4.7} we have
  \[
    \forall \varepsilon \in \R^+, \exists \delta \in \R^+ : \big(\forall x \in [a, b], \abs{x - x_0} < \delta \implies \abs{f(x) - f(x_0)} \leq \varepsilon\big),
  \]
  or equivalently
  \[
    \forall \varepsilon \in \R^+, \exists \delta \in \R^+ : \big(\forall x \in [a, b] \cap (x_0 - \delta, x_0 + \delta) \implies \abs{f(x) - f(x_0)} \leq \varepsilon\big).
  \]
  In particular, we have
  \[
    \exists \delta \in \R^+ : \bigg(\forall x \in [a, b] \cap (x_0 - \delta, x_0 + \delta) \implies \abs{f(x) - f(x_0)} \leq \dfrac{f(x_0)}{2}\bigg),
  \]
  or equivalently
  \[
    \exists \delta \in \R^+ : \bigg(\forall x \in [a, b] \cap (x_0 - \delta, x_0 + \delta) \implies \dfrac{f(x_0)}{2} \leq f(x) \leq \dfrac{3 f(x_0)}{2}\bigg).
  \]
  Since \(\delta \neq 0\), we know that \([a, b] \cap (x_0 - \delta, x_0 + \delta) \neq \emptyset\).
  Since \(a \neq b\), we know that
  \[
    \sup\big([a, b] \cap (x_0 - \delta, x_0 + \delta)\big) \neq \inf\big([a, b] \cap (x_0 - \delta, x_0 + \delta)\big).
  \]
  Thus, by \cref{i:11.1.8} we have \(\abs{[a, b] \cap (x_0 - \delta, x_0 + \delta)} > 0\).
  By \cref{i:11.1.6} we know that \([a, b] \cap (x_0 - \delta, x_0 + \delta)\) is a bounded interval.
  Let \(f_L : [a, b] \to \R\) be the function
  \[
    f_L(x) = \begin{dcases}
      \dfrac{f(x_0)}{2} & \text{if } x \in [a, b] \cap (x_0 - \delta, x_0 + \delta)    \\
      0                 & \text{if } x \notin [a, b] \cap (x_0 - \delta, x_0 + \delta)
    \end{dcases}
  \]
  Since \(f(x) \geq 0\) for all \(x \in [a, b]\), we know that \(f_L\) minorizes \(f\).
  By \cref{i:11.2.16}(g) we know that \(f_L\) is a piecewise constant function.
  By \cref{i:11.3.7} we have
  \[
    \int_{[a, b]} f_L = p.c. \int_{[a, b]} f_L = \dfrac{f(x_0)}{2}\abs{[a, b] \cap (x_0 - \delta, x_0 + \delta)} > 0.
  \]
  But by \cref{i:11.3.2} and \cref{i:11.3.4} we have
  \[
    0 < \int_{[a, b]} f_L \leq \underline{\int}_{[a, b]} f = \int_{[a, b]} f = 0,
  \]
  a contradiction.
  Thus, we must have \(f(x) = 0\) for all \(x \in [a, b]\).
\end{proof}

\begin{ex}\label{i:ex:11.4.3}
  Let \(I\) be a bounded interval, let \(f : I \to \R\) be a Riemann integrable function, and let \(\mathbf{P}\) be a partition of \(I\).
  Show that
  \[
    \int_I f = \sum_{J \in \mathbf{P}} \int_J f|_J.
  \]
\end{ex}

\begin{proof}
  Let \(P(n)\) be the statement ``\(\#(\mathbf{P}) = n\) and \(\int_I f = \sum_{J \in \mathbf{P}} \int_J f|_J\).''
  We induct on \(n\) to show that \(P(n)\) is true \(\forall n \in \N\).
  For \(n = 0\), we have \(\mathbf{P} = \emptyset\) and \(I = \emptyset\).
  Thus
  \begin{align*}
    p.c. \int_{[\emptyset]} f & = \sum_{J \in \emptyset} c_J \abs{J} &  & \by{i:11.2.9}    \\
                              & = 0                                  &  & \by{i:7.1.11}[a] \\
                              & = p.c. \int_{\emptyset} f            &  & \by{i:11.2.14}   \\
                              & = \int_{\emptyset} f                 &  & \by{i:11.3.7}    \\
                              & = \sum_{J \in \emptyset} \int_J f|_J &  & \by{i:7.1.11}[a]
  \end{align*}
  and the base case holds.
  Suppose inductively that \(P(n)\) is true for some \(n \geq 0\).
  Then we need to show that \(P(n + 1)\) is true.
  Let \(K \in \mathbf{P}\) such that \(x < y\) for every \(x \in K\) and \(y \in I \setminus K\).
  Then \(\set{K, \bigcup (\mathbf{P} \setminus \set{K})}\) is a partition of \(I\), and
  \begin{align*}
    \int_I f & = \int_K f|_K + \int_{\bigcup (\mathbf{P} \setminus \set{K})} f|_{\bigcup (\mathbf{P} \setminus \set{K})} &  & \by{i:11.4.1}[h] \\
             & = \int_K f|_K + \sum_{J \in \mathbf{P} \setminus \set{K}} \int_J f|_J                                     &  & \byIH            \\
             & = \sum_{J \in \mathbf{P}} \int_J f|_J.                                                                    &  & \by{i:7.1.11}[e]
  \end{align*}
  This closes the induction.
\end{proof}

\begin{ex}\label{i:ex:11.4.4}
  Without repeating all the computations in the above proofs, give a short explanation as to why the remaining cases of \cref{i:11.4.3} and \cref{i:11.4.5} follow automatically from the cases presented in the text.
\end{ex}

\begin{proof}
  We first show that the remaining case of \cref{i:11.4.3} is true.
  By \cref{i:11.4.1}(b) \(-f\) and \(-g\) are Riemann integrable on \(I\).
  Since \(\max(-f, -g)\) is Riemann integrable and \(\min(f, g) = -\max(-f, -g)\), by \cref{i:11.4.1}(b) we know that \(\min(f, g)\) is Riemann integrable.

  Now we show that the remaining cases of \cref{i:11.4.5} are true.
  By \cref{i:11.4.4} \((-f)_+\) and \((-g)_+\) are Riemann integrable on \(I\).
  Since for any Riemann integrable functions \(p\) and \(q\), \(p_+ q_+\) are Riemann integrable (which is showed in the proof of \cref{i:11.4.5}), we have
  \begin{align*}
    f_+ g_- & = f_+ \cdot \big(\min(g, 0)\big)                      &  & \by{i:11.4.4} \\
            & = f_+ \cdot \big(-\max(-g, 0)\big)                                       \\
            & = f_+ \cdot \big(-(-g)_+\big)                         &  & \by{i:11.4.4} \\
            & = -\big(f_+ \cdot (-g)_+\big)                         &  & \by{i:9.2.1}  \\
    f_- g_+ & = \big(\min(f, 0)\big) \cdot g_+                      &  & \by{i:11.4.4} \\
            & = \big(-\max(-f, 0)\big) \cdot g_+                                       \\
            & = \big(-(-f)_+\big) \cdot g_+                         &  & \by{i:11.4.4} \\
            & = -\big((-f)_+ \cdot g_+\big)                         &  & \by{i:9.2.1}  \\
    f_- g_- & = \big(\min(f, 0)\big) \cdot \big(\min(g, 0)\big)     &  & \by{i:11.4.4} \\
            & = \big(-\max(-f, 0)\big) \cdot \big(-\max(-g, 0)\big)                    \\
            & = \big(-(-f)_+\big) \big(-(-g)_+\big)                 &  & \by{i:11.4.4} \\
            & = (-f)_+ \cdot (-g)_+                                 &  & \by{i:9.2.1}
  \end{align*}
  and thus \(f_+ g_-\), \(f_- g_+\), \(f_- g_-\) are Riemann integrable.
\end{proof}

\section{Riemann integrability of continuous functions}\label{sec:11.5}

\begin{thm}\label{11.5.1}
  Let \(I\) be a bounded interval, and let \(f\) be a function which is uniformly continuous on \(I\).
  Then \(f\) is Riemann integrable.
\end{thm}

\begin{proof}
  From \cref{9.9.15} we see that \(f\) is bounded.
  Now we have to show that \(\underline{\int}_I f = \overline{\int}_I f\).

  If \(I\) is a point or the empty set then the theorem is trivial, so let us assume that \(I\) is one of the four intervals \([a, b]\), \((a, b)\), \((a, b]\), or \([a, b)\) for some real numbers \(a < b\).

      Let \(\varepsilon > 0\) be arbitrary.
      By uniform continuity, there exists a \(\delta > 0\) such that \(\abs{f(x) - f(y)} < \varepsilon\) whenever \(x, y \in I\) are such that \(\abs{x - y} < \delta\).
      By the Archimedean principle, there exists an integer \(N > 0\) such that \((b - a) / N < \delta\).

      Note that we can partition \(I\) into \(N\) intervals \(J_1, \dots, J_N\), each of length \((b - a) / N\).
      (How? One has to treat each of the cases \([a, b]\), \((a, b)\), \((a, b]\), \([a, b)\) slightly differently.)
  By \cref{11.3.12}, we thus have
  \[
    \overline{\int}_I f \leq \sum_{k = 1}^N \big(\sup_{x \in J_k} f(x)\big) \abs{J_k}
  \]
  and
  \[
    \underline{\int}_I f \geq \sum_{k = 1}^N \big(\inf_{x \in J_k} f(x)\big) \abs{J_k}
  \]
  so in particular
  \[
    \overline{\int}_I f - \underline{\int}_I f \leq \sum_{k = 1}^N \big(\sup_{x \in J_k} f(x) - \inf_{x \in J_k} f(x)\big) \abs{J_k}.
  \]
  However, we have \(\abs{f(x) - f(y)} < \varepsilon\) for all \(x, y \in J_k\), since \(\abs{J_k} = (b - a) / N < \delta\).
  In particular we have
  \[
    f(x) < f(y) + \varepsilon \text{ for all } x, y \in J_k.
  \]
  Taking suprema in \(x\), we obtain
  \[
    \sup_{x \in J_k} f(x) \leq f(y) + \varepsilon \text{ for all } y \in J_k,
  \]
  and then taking infima in \(y\) we obtain
  \[
    \sup_{x \in J_k} f(x) \leq \inf_{y \in J_k} f(y) + \varepsilon.
  \]
  Inserting this bound into our previous inequality, we obtain
  \[
    \overline{\int}_I f - \underline{\int}_I f \leq \sum_{k = 1}^N \varepsilon \abs{J_k},
  \]
  but by \cref{11.1.13} we thus have
  \[
    \overline{\int}_I f - \underline{\int}_I f \leq \varepsilon (b - a).
  \]
  But \(\varepsilon > 0\) was arbitrary, while \((b - a)\) is fixed.
  Thus \(\overline{\int}_I f - \underline{\int}_I f\) cannot be positive.
  By \cref{11.3.3} and the definition of Riemann integrability we thus have that \(f\) is Riemann integrable.
\end{proof}

\begin{cor}\label{11.5.2}
  Let \([a, b]\) be a closed interval, and let \(f : [a, b] \to \R\) be continuous.
  Then \(f\) is Riemann integrable.
\end{cor}

\begin{proof}
  Combining \cref{11.5.1} with \cref{9.9.16} we are done.
\end{proof}

\begin{note}
  Note that \cref{11.5.2} is not true if \([a, b]\) is replaced by any other sort of interval, since it is not even guaranteed then that continuous functions are bounded.
  For instance, the function \(f : (0, 1) \to \R\) defined by \(f(x) \coloneqq 1 / x\) is continuous but not Riemann integrable.
  However, if we assume that a function is both continuous \emph{and} bounded, we can recover Riemann integrability (see \cref{11.5.3}).
\end{note}

\begin{prop}\label{11.5.3}
  Let \(I\) be a bounded interval, and let \(f : I \to \R\) be both continuous and bounded.
  Then \(f\) is Riemann integrable on \(I\).
\end{prop}

\begin{proof}
  If \(I\) is a point or an empty set then the claim is trivial;
  if \(I\) is a closed interval the claim follows from \cref{11.5.2}.
  So let us assume that \(I\) is of the form \((a, b]\), \((a, b)\), or \([a, b)\) for some \(a < b\).

  We have a bound \(M\) for \(f\), so that \(-M \leq f(x) \leq M\) for all \(x \in I\).
  Now let \(0 < \varepsilon < (b - a) / 2\) be a small number.
  The function \(f\) when restricted to the interval \([a + \varepsilon, b - \varepsilon]\) is continuous, and hence Riemann integrable by \cref{11.5.2}.
  In particular, we can find a piecewise constant function \(h : [a + \varepsilon, b - \varepsilon] \to \R\) which majorizes \(f\) on \([a + \varepsilon, b - \varepsilon]\) such that
  \[
    \int_{[a + \varepsilon, b - \varepsilon]} h \leq \int_{[a + \varepsilon, b - \varepsilon]} f + \varepsilon.
  \]
  Define \(\tilde{h} : I \to \R\) by
  \[
    \tilde{h}(x) \coloneqq \begin{dcases}
      h(x) & \text{if } x \in [a + \varepsilon, b - \varepsilon]             \\
      M    & \text{if } x \in I \setminus [a + \varepsilon, b - \varepsilon]
    \end{dcases}
  \]
  Clearly \(\tilde{h}\) is piecewise constant on \(I\) and majorizes \(f\);
  by \cref{11.2.16} we have
  \[
    \int_I \tilde{h} = \varepsilon M + \int_{[a + \varepsilon, b - \varepsilon]} h + \varepsilon M \leq \int_{[a + \varepsilon, b - \varepsilon]} f + (2M + 1) \varepsilon.
  \]
  In particular we have
  \[
    \overline{\int}_I f \leq \int_{[a + \varepsilon, b - \varepsilon]} f + (2M + 1) \varepsilon.
  \]
  This is true since \(\tilde{h}\) majorize \(f\).
  A similar argument gives
  \[
    \underline{\int}_I f \geq \int_{[a + \varepsilon, b - \varepsilon]} f - (2M + 1) \varepsilon.
  \]
  and hence
  \[
    \overline{\int}_I f - \underline{\int}_I f \leq (4M + 2) \varepsilon.
  \]
  But \(\varepsilon\) is arbitrary, and so we can argue as in the proof of \cref{11.5.1} to conclude Riemann integrability.
\end{proof}

\begin{note}
  From \cref{11.5.1}, \cref{11.5.2} and \cref{11.5.3} we see that if we can show a function \(f\) being \emph{uniformly continuous} (not just continuous) on some bounded interval \(I\), then \(f\) is Riemann integrable on \(I\).
\end{note}

\begin{defn}\label{11.5.4}
  Let \(I\) be a bounded interval, and let \(f : I \to \R\).
  We say that \(f\) is \emph{piecewise continuous on \(I\)} iff there exists a partition \(\mathbf{P}\) of \(I\) such that \(f|_J\) is continuous on \(J\) for all \(J \in \mathbf{P}\).
\end{defn}

\setcounter{thm}{5}
\begin{prop}\label{11.5.6}
  Let \(I\) be a bounded interval, and let \(f : I \to \R\) be both piecewise continuous and bounded.
  Then \(f\) is Riemann integrable.
\end{prop}

\begin{proof}
  Since \(f\) is piecewise continuous on \(I\), by \cref{11.5.4} \(\exists \mathbf{P}\) such that \(\mathbf{P}\) is a partition of \(I\) and \(f|_J\) is continuous on \(J\) for all \(J \in \mathbf{P}\).
  Since \(f\) is bounded, we know that \(f|_J\) is bounded for all \(J \in \mathbf{P}\).
  Thus by \cref{11.5.3} \(f|_J\) is Riemann integrable on \(J\) for all \(J \in \mathbf{P}\).
  For each \(J \in \mathbf{P}\), we define \(F_J : I \to \R\) to be the function
  \[
    F_J(x) = \begin{dcases}
      f|_J(x) & \text{if } x \in J    \\
      0       & \text{if } x \notin J
    \end{dcases}
  \]
  Then by \cref{11.4.1}(g) \(F|_J\) is Riemann integrable for all \(J \in \mathbf{P}\) and
  \begin{align*}
    \sum_{J \in \mathbf{P}} \int_I F_J & = \sum_{J \in \mathbf{P}} \int_J f|_J &  & \text{(by \cref{11.4.1}(g))} \\
                                       & = \int_I f.                           &  & \by{ex:11.4.3}
  \end{align*}
  Thus \(f\) is Riemann integrable on \(I\).
\end{proof}

\exercisesection

\begin{ex}\label{ex:11.5.1}
  Prove \cref{11.5.6}.
\end{ex}

\begin{proof}
  See \cref{11.5.6}.
\end{proof}

\section{Riemann integrability of monotone functions}\label{i:sec:11.6}

\begin{prop}\label{i:11.6.1}
  Let \([a, b]\) be a closed and bounded interval and let \(f : [a, b] \to \R\) be a monotone function.
  Then \(f\) is Riemann integrable on \([a, b]\).
\end{prop}

\begin{proof}
  Without loss of generality we may take \(f\) to be monotone increasing (instead of monotone decreasing).
  From \cref{i:ex:9.8.1} we know that \(f\) is bounded.
  Now let \(N > 0\) be an integer, and partition \([a, b]\) into \(N\) half-open intervals
  \[
    \set{\big[a + \dfrac{b - a}{N} j, a + \dfrac{b - a}{N} (j + 1)\big) : 0 \leq j \leq N - 1}
  \]
  of length \((b - a) / N\), together with the point \(\set{b}\).
  Then by \cref{i:11.3.12} we have
  \[
    \overline{\int}_I f \leq \sum_{j = 0}^{N - 1} \Bigg(\sup_{x \in \big[a + \dfrac{b - a}{N} j, a + \dfrac{b - a}{N} (j + 1)\big)} f(x)\Bigg) \dfrac{b - a}{N},
  \]
  (the point \(\set{b}\) clearly giving only a zero contribution).
  Since \(f\) is monotone increasing, we thus have
  \[
    \overline{\int}_I f \leq \sum_{j = 0}^{N - 1} f\bigg(a + \dfrac{b - a}{N} (j + 1)\bigg) \dfrac{b - a}{N}.
  \]
  Similarly we have
  \[
    \underline{\int}_I f \geq \sum_{j = 0}^{N - 1} f\bigg(a + \dfrac{b - a}{N} j\bigg) \dfrac{b - a}{N}.
  \]
  Thus we have
  \[
    \overline{\int}_I f - \underline{\int}_I f \leq \sum_{j = 0}^{N - 1} \Bigg(f\bigg(a + \dfrac{b - a}{N} (j + 1)\bigg) - f\bigg(a + \dfrac{b - a}{N} j\bigg)\Bigg) \dfrac{b - a}{N}.
  \]
  Using telescoping series (\cref{i:7.2.15}) we thus have
  \begin{align*}
    \overline{\int}_I f - \underline{\int}_I f & \leq \Bigg(f\bigg(a + \dfrac{b - a}{N} N\bigg) - f\bigg(a + \dfrac{b - a}{N} 0\bigg)\Bigg) \dfrac{b - a}{N} \\
                                               & = \big(f(b) - f(a)\big) \dfrac{b - a}{N}.
  \end{align*}
  But \(N\) was arbitrary, so we can conclude as in the proof of \cref{i:11.5.1} that \(f\) is Riemann integrable.
\end{proof}

\begin{rmk}\label{i:11.6.2}
  From \cref{i:ex:9.8.5} we know that there exist monotone functions which are not piecewise continuous, so \cref{i:11.6.1} is not subsumed by \cref{i:11.5.6}.
\end{rmk}

\begin{cor}\label{i:11.6.3}
  Let \(I\) be a bounded interval, and let \(f : I \to \R\) be both monotone and bounded.
  Then \(f\) is Riemann integrable on \(I\).
\end{cor}

\begin{proof}
  Without loss of generality we may take \(f\) to be monotone increasing (instead of monotone decreasing).
  If \(I\) is a point or an empty set then the claim is trivial;
  if \(I\) is a closed interval the claim follows from \cref{i:11.6.1}.
  So let us assume that \(I\) is of the form \((a, b]\), \((a, b)\), or \([a, b)\) for some \(a < b\).

  We have a bound \(M\) for \(f\), so that \(-M \leq f(x) \leq M\) for all \(x \in I\).
  Now let \(0 < \varepsilon < (b - a) / 2\) be a small number.
  The function \(f\) when restricted to the interval \([a + \varepsilon, b - \varepsilon]\) is monotone, and hence Riemann integrable by \cref{i:11.6.1}.
  In particular, we can find a piecewise constant function \(h : [a + \varepsilon, b - \varepsilon] \to \R\) which majorizes \(f\) on \([a + \varepsilon, b - \varepsilon]\) such that
  \[
    \int_{[a + \varepsilon, b - \varepsilon]} h \leq \int_{[a + \varepsilon, b - \varepsilon]} f + \varepsilon.
  \]
  Define \(\tilde{h} : I \to \R\) by
  \[
    \tilde{h}(x) \coloneqq \begin{dcases}
      h(x) & \text{if } x \in [a + \varepsilon, b - \varepsilon]             \\
      M    & \text{if } x \in I \setminus [a + \varepsilon, b - \varepsilon]
    \end{dcases}
  \]
  Clearly, \(\tilde{h}\) is piecewise constant on \(I\) and majorizes \(f\);
  by \cref{i:11.2.16} we have
  \[
    \int_I \tilde{h} = \varepsilon M + \int_{[a + \varepsilon, b - \varepsilon]} h + \varepsilon M \leq \int_{[a + \varepsilon, b - \varepsilon]} f + (2M + 1) \varepsilon.
  \]
  In particular we have
  \[
    \overline{\int}_I f \leq \int_{[a + \varepsilon, b - \varepsilon]} f + (2M + 1) \varepsilon
  \]
  This is true since \(\tilde{h}\) majorize \(f\).
  A similar argument gives
  \[
    \underline{\int}_I f \geq \int_{[a + \varepsilon, b - \varepsilon]} f - (2M + 1) \varepsilon.
  \]
  and hence
  \[
    \overline{\int}_I f - \underline{\int}_I f \leq (4M + 2) \varepsilon.
  \]
  But \(\varepsilon\) was arbitrary, and so we can argue as in the proof of \cref{i:11.5.1} to conclude Riemann integrability.
\end{proof}

\begin{prop}[Integral test]\label{i:11.6.4}
  Let \(f : [0, \infty) \to \R\) be a monotone decreasing function which is non-negative
  (i.e., \(f(x) \geq 0\) for all \(x \geq 0\)).
  Then the sum \(\sum_{n = 0}^\infty f(n)\) is convergent iff \(\sup_{N > 0} \int_{[0, N]} f\) is finite.
\end{prop}

\begin{proof}
  Let \(N \in \Z^+\).
  Since \(f\) is monotone decreasing, by \cref{i:11.6.1} we know that \(f\) is Riemann integrable on both \([0, N]\) and every interval \([a, b] \subseteq [0, N]\).
  Then we have
  \begin{align*}
    \int_{[0, N]} f & = \sum_{n = 0}^{N - 1} \int_{[n, n + 1)} f|_{[n, n + 1)} + \int_{[N, N]} f|_{[N, N]} &                                          & \by{i:ex:11.4.3} \\
                    & = \sum_{n = 0}^{N - 1} \int_{[n, n + 1)} f|_{[n, n + 1)}                             &                                          & \by{i:11.1.8}    \\
                    & \leq \sum_{n = 0}^{N - 1} \int_{[n, n + 1)} f(n)                                     &                                          & \by{i:11.4.1}[e] \\
                    & = \sum_{n = 0}^{N - 1} f(n) \abs{n + 1 - n}                                          &                                          & \by{i:11.2.9}    \\
                    & = \sum_{n = 0}^{N - 1} f(n)                                                                                                                        \\
                    & \leq \sum_{n = 0}^N f(n)                                                             & (\forall x \in [0, \infty), f(x) \geq 0)
  \end{align*}
  and
  \begin{align*}
    \int_{[0, N]} f & = \sum_{n = 0}^{N - 1} \int_{[n, n + 1)} f|_{[n, n + 1)} + \int_{[N, N]} f|_{[N, N]} &  & \by{i:ex:11.4.3} \\
                    & = \sum_{n = 0}^{N - 1} \int_{[n, n + 1)} f|_{[n, n + 1)}                             &  & \by{i:11.1.8}    \\
                    & \geq \sum_{n = 0}^{N - 1} \int_{[n, n + 1)} f(n + 1)                                 &  & \by{i:11.4.1}[e] \\
                    & = \sum_{n = 0}^{N - 1} f(n + 1) \abs{n + 1 - n}                                      &  & \by{i:11.2.9}    \\
                    & = \sum_{n = 0}^{N - 1} f(n + 1)                                                                            \\
                    & = \sum_{n = 1}^N f(n).                                                               &  & \by{i:7.1.4}[b]
  \end{align*}

  Next we show that if \(\sum_{n = 0}^\infty f(n)\) is convergent, then \(\sup_{N > 0} \int_{[0, N]} f\) is finite.
  Suppose that \(\sum_{n = 0}^\infty f(n)\) is convergent.
  Then by \cref{i:7.2.2} we know that
  \[
    \sum_{n = 0}^\infty f(n) = \lim_{m \to \infty} \sum_{n = 0}^m f(n)
  \]
  and by \cref{i:6.1.12} \(\big(\sum_{n = 0}^m f(n)\big)_{m = 0}^\infty\) is a Cauchy sequence.
  By \cref{i:5.1.15} we know that \(\big(\sum_{n = 0}^m f(n)\big)_{m = 0}^\infty\) is bounded by some \(M \in \R\).
  By comparison principle (\cref{i:6.4.13}) we have
  \[
    \int_{[0, N]} f \leq \sum_{n = 0}^N f(n) \implies \sup\bigg(\int_{[0, N]} f\bigg)_{N = 1}^\infty \leq \sup\bigg(\sum_{n = 0}^N f(n)\bigg)_{N = 1}^\infty \leq M
  \]
  and thus \(\sup_{N > 0} \int_{[0, N]} f\) is finite.

  Now we show that if \(\sup_{N > 0} \int_{[0, N]} f\) is finite, then \(\sum_{n = 0}^\infty f(n)\) is convergent.
  Suppose that \(\sup_{N > 0} \int_{[0, N]} f\) is finite.
  By comparison principle (\cref{i:6.4.13}) we have
  \[
    \sum_{n = 1}^N f(n) \leq \int_{[0, N]} f \implies \sup\bigg(\sum_{n = 1}^N f(n)\bigg)_{N = 1}^\infty \leq \sup\bigg(\int_{[0, N]} f\bigg)_{N = 1}^\infty
  \]
  Thus by \cref{i:7.3.1} \(\sum_{n = 0}^\infty f(n)\) is convergent.
\end{proof}

\begin{cor}\label{i:11.6.5}
  Let \(p\) be a real number.
  Then \(\sum_{n = 1}^\infty \dfrac{1}{n^p}\) converges absolutely when \(p > 1\) and diverges when \(p \leq 1\).
\end{cor}

\begin{proof}
  Let \(f : [1, \infty) \to \R\) be the function \(f(x) = \dfrac{1}{x^p}\).
  By \cref{i:6.7.3}(a)(d) we know that \(f\) is positive and
  \[
    \begin{dcases}
      f \text{ is monotone decreasing if } p > 1; \\
      f \text{ is monotone increasing if } p < 1; \\
      f \text{ is both monotone increasing and decreasing if } p = 1.
    \end{dcases}
  \]
  By \cref{i:11.6.1} \(f\) is Riemann integrable on \([1, N]\) for every \(N \in \R\) and \(N \geq 1\).
  If \(p \neq 1\), then we have
  \begin{align*}
    \int_{[1, N]} f & = \dfrac{1}{1 - p} (N^{1 - p} - 1^{1 - p}) &  & \by{i:11.9.4} \\
                    & = \dfrac{1}{1 - p} (N^{1 - p} - 1).
  \end{align*}
  If \(p = 1\), then we have
  \[
    \int_{[1, N]} f = \ln N - \ln 1 = \ln N.
  \]
  Note that we use \cref{i:11.9.4} and logarithm without circularity.

  First suppose that \(p > 1\).
  Since
  \begin{align*}
    \int_{[1, N]} f & = \dfrac{1}{1 - p} (N^{1 - p} - 1)                      \\
                    & = \dfrac{1}{p - 1} (1 - N^{1 - p})                      \\
                    & \leq \dfrac{1}{p - 1}              & (N^{1 - p} \leq 1)
  \end{align*}
  and \(N\) was arbitrary, we know that \(\sup_{N > 1} \int_{[1, N]} f \leq \dfrac{1}{p - 1}\).
  Thus \(\sup_{N > 1} \int_{[1, N]} f\) is finite and by \cref{i:11.6.4} \(\sum_{n = 1}^\infty \dfrac{1}{n^p}\) is convergent.

  Next suppose that \(p = 1\).
  Since \(\int_{[1, N]} f = \ln N\) and \(\ln N\) is unbounded, we know that \(\sup_{N > 1} \int_{[1, N]} f = +\infty\) and by \cref{i:11.6.4} \(\sum_{n = 1}^\infty \dfrac{1}{n^p}\) is divergent.

  Next suppose that \(0 < p < 1\).
  Since \(\int_{[1, N]} f = \dfrac{1}{1 - p} (N^{1 - p} - 1)\) and \(\set{N^{1 - p} : N \in \R^+}\) is unbounded, we know that \(\sup_{N > 1} \int_{[1, N]} f = +\infty\) and by \cref{i:11.6.4} \(\sum_{n = 1}^\infty \dfrac{1}{n^p}\) is divergent.

  Finally suppose that \(p \leq 0\).
  By \cref{i:6.7.3}(e) we know that \(1 = x^0 \geq x^p\) for all \(x \in [1, \infty)\), thus \(1 = \dfrac{1}{x^0} \leq \dfrac{1}{x^p}\).
  By zero test (\cref{i:7.2.6}) we know that \(\lim_{n \to \infty} 1 \neq 0\) implies \(\sum_{n = 0}^\infty \dfrac{1}{x^0}\) diverges.
  Thus by comparison test (\cref{i:7.3.1}) \(\sum_{n = 1}^N \dfrac{1}{x^p}\) is divergent.
\end{proof}

\exercisesection

\begin{ex}\label{i:ex:11.6.1}
  Use \cref{i:11.6.1} to prove \cref{i:11.6.3}.
\end{ex}

\begin{proof}
  See \cref{i:11.6.3}.
\end{proof}

\begin{ex}\label{i:ex:11.6.2}
  Formulate a reasonable notion of a piecewise monotone function, and then show that all bounded piecewise monotone functions are Riemann integrable.
\end{ex}

\begin{proof}
  Let \(I\) be a bounded interval, and let \(f : I \to \R\).
  We say that \(f\) is \emph{piecewise monotone on \(I\)} iff there exists a partition \(\mathbf{P}\) of \(I\) such that \(f|_J\) is monotone on \(J\) for all \(J \in \mathbf{P}\).

  Now we show that all bounded piecewise monotone functions are Riemann integrable.
  Suppose that \(f : I \to \R\) is a bounded piecewise monotone function.
  Then by definition \(\exists \mathbf{P}\) such that \(\mathbf{P}\) is a partition of \(I\) and \(f|_J\) is monotone on \(J\) for all \(J \in \mathbf{P}\).
  Since \(f\) is bounded, \(f|_J\) is also bounded, by \cref{i:11.6.3} we know that \(f|_J\) is Riemann integrable on \(J\).
  Let \(F_J : I \to \R\) be the function
  \[
    F_J(x) = \begin{dcases}
      f|_J(x) & \text{if } x \in J    \\
      0       & \text{if } x \notin J
    \end{dcases}
  \]
  Then by \cref{i:11.4.1}(g) we know that \(F_J\) is Riemann integrable and
  \begin{align*}
    \sum_{J \in \mathbf{P}} \int_I F_J & = \sum_{J \in \mathbf{P}} \int_J f|_J &  & \by{i:11.4.1}[g] \\
                                       & = \int_I f.                           &  & \by{i:ex:11.4.3}
  \end{align*}
  Thus \(f\) is Riemann integrable on \(I\).
\end{proof}

\begin{ex}\label{i:ex:11.6.3}
  Prove \cref{i:11.6.4}.
\end{ex}

\begin{proof}
  See \cref{i:11.6.4}.
\end{proof}

\begin{ex}\label{i:ex:11.6.4}
  Give examples to show that both directions of the integral test break down if \(f\) is not assumed to be monotone decreasing.
\end{ex}

\begin{proof}
  Let \(f_1 : [0, \infty) \to \R\) be the function
  \[
    f_1(x) = \begin{dcases}
      1 & \text{if } x \in \N    \\
      0 & \text{if } x \notin \N
    \end{dcases}
  \]
  Then we know that \(f_1\) is not monotone decreasing and \(\sum_{n = 0}^\infty f_1(n)\) diverges.
  But \(\int_{[0, N]} f_1 = 0\) for all \(N \in \Z^+\), thus \(\sup_{N > 0} \int_{[0, N]} f_1\) is finite.

  Let \(f_2 : [0, \infty) \to \R\) be the function
  \[
    f_2(x) = \begin{dcases}
      \dfrac{1}{x^2} & \text{if } x \in \N    \\
      \dfrac{1}{x}   & \text{if } x \notin \N
    \end{dcases}
  \]
  Then we know that \(f_2\) is not monotone decreasing.
  By \cref{i:11.6.5} we know that \(\sup_{N > 0} \int_{[0, N]} \dfrac{1}{x}\) is not finite, and since \(\int_{[0, N]} f_2 = \int_{[0, N]} \dfrac{1}{x}\) we also have \(\sup_{N > 0} \int_{[0, N]} f_2\) is not finite.
  But by \cref{i:11.6.5} we know that \(\sum_{n = 0}^\infty \dfrac{1}{x^2}\) converges.
\end{proof}

\begin{ex}\label{i:ex:11.6.5}
  Use \cref{i:11.6.4} to prove \cref{i:11.6.5}.
\end{ex}

\begin{proof}
  See \cref{i:11.6.5}.
\end{proof}

\section{A non-Riemann integrable function}\label{i:sec:11.7}

\begin{prop}\label{i:11.7.1}
  Let \(f : [0, 1] \to \R\) be the discontinuous function
  \[
    f(x) \coloneqq \begin{dcases}
      1 & \text{if } x \in \Q    \\
      0 & \text{if } x \notin \Q
    \end{dcases}
  \]
  Then \(f\) is bounded but not Riemann integrable.
\end{prop}

\begin{proof}
  It is clear that \(f\) is bounded, so let us show that it is not Riemann integrable.

  Let \(\mathbf{P}\) be any partition of \([0, 1]\).
  For any \(J \in \mathbf{P}\), observe that if \(J\) is not a point or the empty set, then
  \[
    \sup_{x \in J} f(x) = 1
  \]
  (by \cref{i:5.4.14}).
  In particular we have
  \[
    \bigg(\sup_{x \in J} f(x)\bigg) \abs{J} = \abs{J}.
  \]
  (Note this is also true when \(J\) is a point, since both sides are zero.)
  In particular we see that
  \[
    U(f, \mathbf{P}) = \sum_{J \in \mathbf{P} : J \neq \emptyset} \abs{J} = \abs{[0, 1]} = 1
  \]
  by \cref{i:11.1.13};
  note that the empty set does not contribute anything to the total length.
  In particular we have \(\overline{\int}_{[0, 1]} f = 1\), by \cref{i:11.3.12}.

  A similar argument gives that
  \[
    \inf_{x \in J} f(x) = 0
  \]
  for all \(J\) (other than points or the empty set), and so
  \[
    L(f, \mathbf{P}) = \sum_{J \in \mathbf{P} : J \neq \emptyset} 0 = 0.
  \]
  In particular we have \(\underline{\int}_{[0, 1]} f = 0\), by \cref{i:11.3.12}.
  Thus the upper and lower Riemann integrals do not match, and so this function is not Riemann integrable.
\end{proof}

\begin{rmk}\label{i:11.7.2}
  It is only rather ``artificial'' bounded functions which are not Riemann integrable.
  Because of this, the Riemann integral is good enough for a large majority of cases.
  There are ways to generalize or improve this integral, though.
  One of these is the \emph{Lebesgue integral}.
  Another is the \emph{Riemann-Stieltjes integral} \(\int_I f d\alpha\), where \(\alpha : I \to \R\) is a monotone increasing function.
\end{rmk}

\section{The Riemann-Stieltjes integral}\label{sec:11.8}

\begin{ac}\label{ac:11.8.1}
  Let \(I\) be a bounded interval, and let \(f : X \to \R\) be a monotone increasing function defined on some interval \(X\) which contains \(I\).
  Then we have
  \[
    f(x_0+) = \lim_{x \to x_0^+ ; x \in X} f(x) = \inf_{x \in X \cap (x_0, +\infty)} f(x)
  \]
  and
  \[
    f(x_0-) = \lim_{x \to x_0^- ; x \in X} f(x) = \sup_{x \in X \cap (-\infty, x_0)} f(x)
  \]
  for every \(x_0 \in I\) and \(x_0\) is not an endpoint of \(X\).
\end{ac}

\begin{proof}
  If \(I = \emptyset\), then the statement are vacuously true.
  So suppose that \(I \neq \emptyset\).
  Define
  \begin{align*}
    U & = \inf_{x \in X \cap (x_0, +\infty)} f(x); \\
    L & = \sup_{x \in X \cap (-\infty, x_0)} f(x).
  \end{align*}
  Since \(x_0 \in X\) and \(x_0\) is not an endpoint of \(X\), we know that \(X \cap (x_0, +\infty) \neq \emptyset\) and \(X \cap (-\infty, x_0) \neq \emptyset\).
  Since \(f\) is monotone increasing, we have
  \[
    U = \inf_{x \in X \cap (x_0, +\infty)} f(x) \geq f(x_0)
  \]
  and
  \[
    L = \sup_{x \in X \cap (-\infty, x_0)} f(x) \leq f(x_0).
  \]
  Thus \(U, L \in \R\).

  First we show that \(f(x_0+) = U\).
  By the definition of \(U\) we know that
  \[
    \forall \varepsilon \in \R^+, \exists x \in X \cap (x_0, +\infty) : 0 \leq f(x) - U \leq \varepsilon.
  \]
  Now fix one pair of \(\varepsilon\) and \(x\).
  Since \(f\) is monotone increasing, we know that
  \[
    \forall y \in X \cap (x_0, +\infty), y < x \implies 0 \leq f(y) - U \leq f(x) - U \leq \varepsilon.
  \]
  Thus by setting \(\delta = x - x_0\) we have
  \[
    \forall y \in X \cap (x_0, +\infty), \abs{y - x_0} < \delta \implies \abs{f(y) - U} \leq \varepsilon.
  \]
  Since \(\varepsilon\) is arbitrary, by \cref{9.3.6} and \cref{9.5.1} we have \(f(x_0+) = U\).

  Now we show that \(f(x_0-) = L\).
  By the definition of \(L\) we know that
  \[
    \forall \varepsilon \in \R^+, \exists x \in X \cap (-\infty, x_0) : 0 \leq L - f(x) \leq \varepsilon.
  \]
  Now fix one pair of \(\varepsilon\) and \(x\).
  Since \(f\) is monotone increasing, we know that
  \[
    \forall y \in X \cap (-\infty, x_0), y > x \implies 0 \leq L - f(y) \leq L - f(x) \leq \varepsilon.
  \]
  Thus by setting \(\delta = x_0 - x\) we have
  \[
    \forall y \in X \cap (-\infty, x_0), \abs{y - x_0} < \delta \implies \abs{f(y) - L} \leq \varepsilon.
  \]
  Since \(\varepsilon\) is arbitrary, by \cref{9.3.6} and \cref{9.5.1} we have \(f(x_0-) = L\).
\end{proof}

\begin{ac}\label{ac:11.8.2}
  Let \(I\) be a bounded interval, and let \(f : X \to \R\) be a monotone decreasing function defined on some interval \(X\) which contains \(I\).
  Then we have
  \[
    f(x_0+) = \lim_{x \to x_0^+ ; x \in X} f(x) = \sup_{x \in X \cap (x_0, \infty)} f(x)
  \]
  and
  \[
    f(x_0-) = \lim_{x \to x_0^- ; x \in X} f(x) = \inf_{x \in X \cap (-\infty, x_0)} f(x)
  \]
  for every \(x_0 \in I\) and \(x_0\) is not an endpoint of \(X\).
\end{ac}

\begin{proof}
  If \(I = \emptyset\), then the statement are vacuously true.
  So suppose that \(I \neq \emptyset\).
  Define
  \begin{align*}
    U & = \sup_{x \in X \cap (x_0, +\infty)} f(x); \\
    L & = \inf_{x \in X \cap (-\infty, x_0)} f(x).
  \end{align*}
  Since \(x_0 \in X\) and \(x_0\) is not an endpoint of \(X\), we know that \(X \cap (x_0, +\infty) \neq \emptyset\) and \(X \cap (-\infty, x_0) \neq \emptyset\).
  Since \(f\) is monotone decreasing, we have
  \[
    U = \sup_{x \in X \cap (x_0, +\infty)} f(x) \leq f(x_0)
  \]
  and
  \[
    L = \inf_{x \in X \cap (-\infty, x_0)} f(x) \geq f(x_0).
  \]
  Thus \(U, L \in \R\).

  First we show that \(f(x_0+) = U\).
  By the definition of \(U\) we know that
  \[
    \forall \varepsilon \in \R^+, \exists x \in X \cap (x_0, +\infty) : 0 \leq U - f(x) \leq \varepsilon.
  \]
  Now fix one pair of \(\varepsilon\) and \(x\).
  Since \(f\) is monotone decreasing, we know that
  \[
    \forall y \in X \cap (x_0, +\infty), y < x \implies 0 \leq U - f(y) \leq U - f(x) \leq \varepsilon.
  \]
  Thus by setting \(\delta = x - x_0\) we have
  \[
    \forall y \in X \cap (x_0, +\infty), \abs{y - x_0} < \delta \implies \abs{f(y) - U} \leq \varepsilon.
  \]
  Since \(\varepsilon\) is arbitrary, by \cref{9.3.6} and \cref{9.5.1} we have \(f(x_0+) = U\).

  Now we show that \(f(x_0-) = L\).
  By the definition of \(L\) we know that
  \[
    \forall \varepsilon \in \R^+, \exists x \in X \cap (-\infty, x_0) : 0 \leq f(x) - L \leq \varepsilon.
  \]
  Now fix one pair of \(\varepsilon\) and \(x\).
  Since \(f\) is monotone decreasing, we know that
  \[
    \forall y \in X \cap (-\infty, x_0), y > x \implies 0 \leq f(y) - L \leq f(x) - L \leq \varepsilon.
  \]
  Thus by setting \(\delta = x_0 - x\) we have
  \[
    \forall y \in X \cap (-\infty, x_0), \abs{y - x_0} < \delta \implies \abs{f(y) - L} \leq \varepsilon.
  \]
  Since \(\varepsilon\) is arbitrary, by \cref{9.3.6} and \cref{9.5.1} we have \(f(x_0-) = L\).
\end{proof}

\begin{defn}[\(\alpha\)-length]\label{11.8.1}
  Let \(I\) be a bounded interval, and let \(\alpha : X \to \R\) be a monotone increasing function defined on some interval \(X\) which contains \(I\).
  Then we define the \emph{\(\alpha\)-length} \(\alpha[I]\) of \(I\) as follows.
  \begin{itemize}
    \item If \(I\) is the empty set, we set
          \[
            \alpha[\emptyset] \coloneqq 0.
          \]
    \item If \(I\) is a point of the form \(\set{a}\) for some real number \(a\), we set
          \[
            \alpha\big[\set{a}\big] \coloneqq \lim_{x \to a^+ ; x \in X} \alpha(x) - \lim_{x \to a^- ; x \in X} \alpha(x),
          \]
          with the convention that \(\lim_{x \to a^+ ; x \in X} \alpha(x)\) (resp. \(\lim_{x \to a^- ; x \in X} \alpha(x)\)) is \(\alpha(a)\) when \(a\) is the right (resp. left) endpoint of \(X\).
    \item If \(I\) is an interval of the form \((a, b)\) for some real numbers \(b > a\), set
          \[
            \alpha\big[(a, b)\big] \coloneqq \lim_{x \to b^- ; x \in X} \alpha(x) - \lim_{x \to a^+ ; x \in X} \alpha(x).
          \]
    \item If \(I\) is an interval of the form \([a, b)\), \((a, b]\), or \([a, b]\) for some real numbers \(b > a\), then we set
          \[
            \alpha[I] = \begin{dcases}
              \alpha\big[\set{a}\big] + \alpha\big[(a, b)\big]                           & \text{if } I = [a, b) \\
              \alpha\big[(a, b)\big] + \alpha\big[\set{b}\big]                           & \text{if } I = (a, b] \\
              \alpha\big[\set{a}\big] + \alpha\big[(a, b)\big] + \alpha\big[\set{b}\big] & \text{if } I = [a, b]
            \end{dcases}
          \]
  \end{itemize}
\end{defn}

\begin{note}
  In the special case when \(\alpha\) is continuous, the definition of \(\alpha[I]\) where \(I\) is of the form \((a, b)\), \([a, b)\), \((a, b]\), or \([a, b]\) simplifies to \(\alpha[I] = \alpha(b) - \alpha(a)\).
\end{note}

\begin{note}
  We sometimes write \(\alpha\big|_a^b\) or \(\alpha(x)\big|_{x = a}^{x = b}\) instead of \(\alpha\big[[a, b]\big]\).
\end{note}

\begin{note}
  \cref{11.8.1} is well-defined, thanks to \cref{ac:11.8.1}.
  \cref{11.8.1} is can also be applied when \(\alpha\) is monotone decreasing, thanks to \cref{ac:11.8.2}.
\end{note}

\setcounter{thm}{3}
\begin{lem}\label{11.8.4}
  Let \(I\) be a bounded interval, let \(\alpha : X \to \R\) be a monotone increasing function defined on some interval \(X\) which contains \(I\), and let \(\mathbf{P}\) be a partition of \(I\).
  Then we have
  \[
    \alpha[I] = \sum_{J \in \mathbf{P}} \alpha[J].
  \]
\end{lem}

\begin{proof}
  We prove this by induction on \(n\).
  More precisely, we let \(P(n)\) be the property that whenever \(I\) is a bounded interval, and whenever \(\mathbf{P}\) is a partition of \(I\) with cardinality \(n\), that \(\alpha[I] = \sum_{J \in \mathbf{P}} \alpha[J]\).

  The base case \(P(0)\) is trivial;
  the only way that \(I\) can be partitioned into an empty partition is if \(I\) is itself empty, so by \cref{11.8.1} \(\alpha[I] = 0\).
  The case \(P(1)\) is also very easy;
  the only way that \(I\) can be partitioned into a singleton set \(\set{J}\) is if \(J = I\), at which point the claim is again very easy.

  Now suppose inductively that \(P(n)\) is true for some \(n \geq 1\), and now we prove \(P(n + 1)\).
  Let \(I\) be a bounded interval, and let \(\mathbf{P}\) be a partition of \(I\) of cardinality \(n + 1\).

  If \(I\) is the empty set or a point, then all the intervals in \(\mathbf{P}\) must also be either the empty set or a point, and by \cref{11.8.1} every interval either has \(\alpha\)-length zero or
  \[
    \alpha[\set{a}] = \lim_{x \to a^+ ; x \in X} \alpha(x) - \lim_{x \to a^- ; x \in X} \alpha(x),
  \]
  and the claim is trivial.
  Thus we will assume that \(I\) is an interval of the form \((a, b)\), \((a, b]\), \([a, b)\), or \([a, b]\).

      Let us first suppose that \(b \in I\), i.e., \(I\) is either \((a, b]\) or \([a, b]\).
  Since \(b \in I\), we know that one of the intervals \(K\) in \(\mathbf{P}\) contains \(b\).
  Since \(K\) is contained in \(I\), it must therefore be of the form \((c, b]\), \([c, b]\), or \(\set{b}\) for some real number \(c\), with \(a \leq c \leq b\) (in the latter case of \(K = \set{b}\), we set \(c \coloneqq b\)).
  In particular, this means that the set \(I \setminus K\) is also an interval of the form \([a, c]\), \((a, c)\), \((a, c]\), \([a, c)\) when \(c > a\), or a point or empty set when \(a = c\).
  Either way, by \cref{11.8.1} we see that
  \begin{align*}
    \alpha\big[(a, b]\big] & = \alpha\big[(a, b)\big] + \alpha\big[\set{b}\big]                                                            \\
                           & = \lim_{x \to b^- ; x \in X} \alpha(x) - \lim_{x \to a^+ ; x \in X} \alpha(x) + \alpha\big[\set{b}\big]       \\
                           & = \lim_{x \to b^- ; x \in X} \alpha(x) - \lim_{x \to c^+ ; x \in X} \alpha(x)                                 \\
                           & \quad + \lim_{x \to c^+ ; x \in X} \alpha(x) - \lim_{x \to c^- ; x \in X} \alpha(x)                           \\
                           & \quad + \lim_{x \to c^- ; x \in X} \alpha(x) - \lim_{x \to a^+ ; x \in X} \alpha(x) + \alpha\big[\set{b}\big] \\
                           & = \alpha\big[(c, b)\big] + \alpha\big[\set{c}\big] + \alpha\big[(a, c)\big] + \alpha\big[\set{b}\big]         \\
                           & = \begin{dcases}
                                 \alpha\big[(a, c)\big] + \alpha\big[[c, b]\big] \\
                                 \alpha\big[(a, c]\big] + \alpha\big[(c, b]\big]
                               \end{dcases}                             \\
                           & = \alpha[K] + \alpha[I \setminus K]
  \end{align*}
  and
  \begin{align*}
    \alpha\big[[a, b]\big] & = \alpha\big[\set{a}\big] + \alpha\big[(a, b]\big]                                                                         \\
                           & = \begin{dcases}
                                 \alpha\big[\set{a}\big] + \alpha\big[(a, c)\big] + \alpha\big[[c, b]\big] \\
                                 \alpha\big[\set{a}\big] + \alpha\big[(a, c]\big] + \alpha\big[(c, b]\big]
                               \end{dcases} \\
                           & = \begin{dcases}
                                 \alpha\big[[a, c)\big] + \alpha\big[[c, b]\big] \\
                                   \alpha\big[[a, c]\big] + \alpha\big[(c, b]\big]
                               \end{dcases}                                         \\
                           & = \alpha[K] + \alpha[I \setminus K].
  \end{align*}
  On the other hand, since \(\mathbf{P}\) forms a partition of \(I\), we see that \(\mathbf{P} \setminus \set{K}\) forms a partition of \(I \setminus K\).
  By the induction hypothesis, we thus have
  \[
    \alpha[I \setminus K] = \sum_{J \in \mathbf{P} \setminus \set{K}} \alpha[J].
  \]
  Combining these two identities (and using the laws of addition for finite sets, see \cref{7.1.11}(e)) we obtain
  \[
    \alpha[I] = \sum_{J \in \mathbf{P}} \alpha[J]
  \]
  as desired.

  Now suppose that \(b \notin I\), i.e., \(I\) is either \((a, b)\) or \([a, b)\).
  Then one of the intervals \(K\) also is of the form \((c, b)\) or \([c, b)\) (see \cref{ex:11.1.3}).
      In particular, this means that the set \(I \setminus K\) is also an interval of the form \([a, c]\), \((a, c)\), \((a, c]\), \([a, c)\) when \(c > a\), or a point or empty set when \(a = c\).
  By \cref{11.8.1} we see that
  \begin{align*}
    \alpha\big[(a, b)\big] & = \lim_{x \to b^- ; x \in X} \alpha(x) - \lim_{x \to a^+ ; x \in X} \alpha(x)       \\
                           & = \lim_{x \to b^- ; x \in X} \alpha(x) - \lim_{x \to c^+ ; x \in X} \alpha(x)       \\
                           & \quad + \lim_{x \to c^+ ; x \in X} \alpha(x) - \lim_{x \to c^- ; x \in X} \alpha(x) \\
                           & \quad + \lim_{x \to c^- ; x \in X} \alpha(x) - \lim_{x \to a^+ ; x \in X} \alpha(x) \\
                           & = \alpha\big[(c, b)\big] + \alpha\big[\set{c}\big] + \alpha\big[(a, c)\big]         \\
                           & = \begin{dcases}
                                 \alpha\big[(a, c)\big] + \alpha\big[[c, b)\big] \\
                                   \alpha\big[(a, c]\big] + \alpha\big[(c, b)\big]
                               \end{dcases}        \\
                           & = \alpha[K] + \alpha[I \setminus K]
  \end{align*}
  and
  \begin{align*}
    \alpha\big[[a, b)\big] & = \alpha\big[\set{a}\big] + \alpha\big[(a, b)\big]                                                                     \\
                           & = \begin{dcases}
                                 \alpha\big[\set{a}\big] + \alpha\big[(a, c)\big] + \alpha\big[[c, b)\big] \\
                                   \alpha\big[\set{a}\big] + \alpha\big[(a, c]\big] + \alpha\big[(c, b)\big]
                               \end{dcases} \\
                           & = \begin{dcases}
                                 \alpha\big[[a, c)\big] + \alpha\big[[c, b)\big] \\
                                 \alpha\big[[a, c]\big] + \alpha\big[(c, b)\big]
                               \end{dcases}                                      \\
                           & = \alpha[K] + \alpha[I \setminus K].
  \end{align*}
  The rest of the argument then proceeds as above.
\end{proof}

\begin{ac}\label{ac:11.8.3}
  Let \(I\) be a bounded interval, let \(\alpha : X \to \R\) be a monotone decreasing function defined on some interval \(X\) which contains \(I\), and let \(\mathbf{P}\) be a partition of \(I\).
  Then we have
  \[
    \alpha[I] = \sum_{J \in \mathbf{P}} \alpha[J].
  \]
\end{ac}

\begin{proof}
  Since \(\alpha\) is monotone decreasing, we know that \(-\alpha\) is monotone increasing.
  Thus by \cref{11.8.4} we have
  \begin{align*}
             & (-\alpha)[I] = \sum_{J \in \mathbf{P}} (-\alpha)[J]                                                                                  \\
    \implies & -\big(\alpha[I]\big) = \sum_{J \in \mathbf{P}} -\big(\alpha[J]\big) = -\sum_{J \in \mathbf{P}} \alpha[J] &  & \text{(by limit laws)} \\
    \implies & \alpha[I] = \sum_{J \in \mathbf{P}} \alpha[J].
  \end{align*}
\end{proof}

\begin{defn}[piecewise constant Riemann-Stieltjes integral]\label{11.8.5}
  Let \(I\) be a bounded interval, and let \(\mathbf{P}\) be a partition of \(I\).
  Let \(\alpha : X \to \R\) be a monotone increasing function defined on some interval \(X\) which contains \(I\), and let \(f : I \to \R\) be a function which is piecewise constant with respect to \(\mathbf{P}\).
  Then we define
  \[
    p.c. \int_{[\mathbf{P}]} f \; d \alpha \coloneqq \sum_{J \in \mathbf{P}} c_J \alpha[J]
  \]
  where \(c_J\) is the constant value of \(f\) on \(J\).
\end{defn}

\begin{note}
  When \(\alpha\) is monotone decreasing, by \cref{11.8.5} we have
  \[
    p.c. \int_{[\mathbf{P}]} f \; d (-\alpha) = \sum_{J \in \mathbf{P}} c_J (-\alpha)[J] = - \sum_{J \in \mathbf{P}} c_J \alpha[J].
  \]
\end{note}

\setcounter{thm}{6}
\begin{eg}\label{11.8.7}
  Let \(\alpha : \R \to \R\) be the identity function \(\alpha(x) \coloneqq x\).
  Then for any bounded interval \(I\), any partition \(\mathbf{P}\) of \(I\), and any function \(f\) that is piecewise constant with respect to \(P\), we have \(p.c. \int_{[\mathbf{P}]} f \; d \alpha = p.c. \int_{[\mathbf{P}]} f\).
\end{eg}

\begin{ac}\label{ac:11.8.4}
  Let \(I\) be a bounded interval, let \(\alpha : X \to \R\) be a monotone increasing function defined on some interval \(X\) which contains \(I\), and let \(f : I \to \R\) be a function.
  Suppose that \(\mathbf{P}\) and \(\mathbf{P}'\) are partitions of \(I\) such that \(f\) is piecewise constant both with respect to \(\mathbf{P}\) and with respect to \(\mathbf{P}'\).
  Also suppose that both \(p.c. \int_{[\mathbf{P}]} f \; d \alpha\) and \(p.c. \int_{[\mathbf{P}']} f \; d \alpha\) are well-defined.
  Then \(p.c. \int_{[\mathbf{P}]} f \; d \alpha = p.c. \int_{[\mathbf{P}']} f \; d \alpha\).
\end{ac}

\begin{proof}
  By \cref{11.1.18} we know that \(\mathbf{P} \# \mathbf{P}'\) is a partition of \(I\) and is both finer than \(\mathbf{P}\) and finer than \(\mathbf{P}'\), thus by \cref{11.8.5} we have
  \[
    p.c. \int_{[\mathbf{P} \# \mathbf{P}']} f \; d \alpha = \sum_{J \in \mathbf{P} \# \mathbf{P}'} c_J \alpha[J].
  \]
  By \cref{11.8.4} we know that
  \[
    \alpha[I] = \sum_{J \in \mathbf{P}} \alpha[J] = \sum_{J \in \mathbf{P} \# \mathbf{P}'} \alpha[J].
  \]
  For each \(K \in \mathbf{P}\), let \(\mathbf{P}_K\) be the set
  \[
    \mathbf{P}_K = \set{S \in \mathbf{P} \# \mathbf{P}' : S \subseteq K}.
  \]
  Since \(\mathbf{P} \# \mathbf{P}'\) is finer than \(\mathbf{P}\), by \cref{ac:11.1.4} we know that \(\mathbf{P}_K\) is a partition of \(K\), and \(\bigcup_{K \in \mathbf{P}} \mathbf{P}_K = \mathbf{P} \# \mathbf{P}'\).
  Since \(f\) is piecewise constant with respect to \(\mathbf{P}\), by \cref{11.2.7} we know that \(f\) is piecewise constant with respect to \(\mathbf{P} \# \mathbf{P}'\).
  So we have
  \begin{align*}
    p.c. \int_{[\mathbf{P} \# \mathbf{P}']} f \; d \alpha & = \sum_{J \in \mathbf{P} \# \mathbf{P}'} c_J \alpha[J]                        &                 & \by{11.8.5}    \\
                                                          & = \sum_{J \in \bigcup_{K \in \mathbf{P}} \mathbf{P}_K} c_J \alpha[J]                                             \\
                                                          & = \sum_{K \in \mathbf{P}} \sum_{J \in \mathbf{P}_K} c_J \alpha[J]             &                 & \by{7.1.11}[e] \\
                                                          & = \sum_{K \in \mathbf{P}} \sum_{J \in \mathbf{P}_K} c_K \alpha[J]             & (J \subseteq K)                  \\
                                                          & = \sum_{K \in \mathbf{P}} c_K \bigg(\sum_{J \in \mathbf{P}_K} \alpha[J]\bigg)                                    \\
                                                          & = \sum_{K \in \mathbf{P}} c_K \alpha[K]                                       &                 & \by{11.8.4}    \\
                                                          & = p.c. \int_{[\mathbf{P}]} f \; d \alpha.                                     &                 & \by{11.8.5}
  \end{align*}
  Using similar arguments we can show that \(p.c. \int_{[\mathbf{P}']} f \; d \alpha = p.c. \int_{[\mathbf{P} \# \mathbf{P}']} f \; d \alpha\).
  Thus we have \(p.c. \int_{[\mathbf{P}]} f \; d \alpha = p.c. \int_{[\mathbf{P}']} f \; d \alpha\).
\end{proof}

\begin{ac}\label{ac:11.8.5}
  Let \(I\) be a bounded interval, let \(\alpha : X \to \R\) be a monotone increasing function defined on some interval \(X\) which contains \(I\), and let \(f : I \to \R\) be a piecewise constant function on \(I\).
  Then we define
  \[
    p.c. \int_I f \; d \alpha \coloneqq p.c. \int_{[\mathbf{P}]} f \; d \alpha,
  \]
  where \(\mathbf{P}\) is any partition of \(I\) with respect to which \(f\) is piecewise constant.
  (Note that \cref{ac:11.8.4} tells us that the precise choice of this partition is irrelevant.)
\end{ac}

\begin{ac}\label{ac:11.8.6}
  Let \(I\) be a bounded interval, let \(\alpha : X \to \R\) be a monotone function defined on some interval \(X\) which contains \(I\).
  \begin{itemize}
    \item If \(\alpha\) is monotone increasing, then \(\alpha[I] \geq 0\).
    \item If \(\alpha\) is monotone decreasing, then \(\alpha[I] \leq 0\).
  \end{itemize}
\end{ac}

\begin{proof}
  We split into four cases:
  \begin{itemize}
    \item \(I = \emptyset\).
          Then by \cref{11.8.1} we have \(\alpha[\emptyset] = 0\).
    \item \(I = \set{x_0}\) for some \(x_0 \in \R\).
          If \(\alpha\) is monotone increasing, then we have
          \begin{align*}
            \alpha\big[\set{x_0}\big] & = \lim_{x \to x_0^+} \alpha(x) - \lim_{x \to x_0^-} \alpha(x)                      &  & \by{11.8.1}                                      \\
                                      & = \inf_{x \in X \cap (x_0, \infty)} f(x) - \sup_{x \in X \cap (-\infty, x_0)} f(x) &  & \by{ac:11.8.1}                                   \\
                                      & \geq f(x_0) - f(x_0)                                                               &  & \text{(since \(\alpha\) is monotone increasing)} \\
                                      & = 0.
          \end{align*}
          If \(\alpha\) is monotone decreasing, then we have
          \begin{align*}
            \alpha\big[\set{x_0}\big] & = \lim_{x \to x_0^+} \alpha(x) - \lim_{x \to x_0^-} \alpha(x)                      &  & \by{11.8.1}                                      \\
                                      & = \sup_{x \in X \cap (x_0, \infty)} f(x) - \inf_{x \in X \cap (-\infty, x_0)} f(x) &  & \by{ac:11.8.2}                                   \\
                                      & \leq f(x_0) - f(x_0)                                                               &  & \text{(since \(\alpha\) is monotone decreasing)} \\
                                      & = 0.
          \end{align*}
    \item \(I = (a, b)\) for some \(a, b \in \R\) and \(a < b\).
          If \(\alpha\) is monotone increasing, then we have
          \begin{align*}
             & \alpha\big[(a, b)\big]                                                                                                 \\
             & = \lim_{x \to b^- ; x \in (a, b)} \alpha(x) - \lim_{x \to a^+ ; x \in (a, b)} \alpha(x)            &  & \by{11.8.1}    \\
             & = \sup_{x \in (a, b) \cap (-\infty, b)} \alpha(x) - \inf_{x \in (a, b) \cap (a, \infty)} \alpha(x) &  & \by{ac:11.8.1} \\
             & = \sup_{x \in (a, b)} \alpha(x) - \inf_{x \in (a, b)} \alpha(x)                                                        \\
             & \geq 0.
          \end{align*}
          If \(\alpha\) is monotone decreasing, then we have
          \begin{align*}
             & \alpha\big[(a, b)\big]                                                                                                 \\
             & = \lim_{x \to b^- ; x \in (a, b)} \alpha(x) - \lim_{x \to a^+ ; x \in (a, b)} \alpha(x)            &  & \by{11.8.1}    \\
             & = \inf_{x \in (a, b) \cap (-\infty, b)} \alpha(x) - \sup_{x \in (a, b) \cap (a, \infty)} \alpha(x) &  & \by{ac:11.8.2} \\
             & = \inf_{x \in (a, b)} \alpha(x) - \sup_{x \in (a, b)} \alpha(x)                                                        \\
             & \leq 0.
          \end{align*}
    \item \(I\) is one of \([a, b), (a, b], [a, b]\).
          If \(\alpha\) is monotone increasing, then from the proof above we have
          \begin{align*}
            \alpha\big[[a, b)\big] & = \alpha\big[\set{a}\big] + \alpha\big[(a, b)\big] \geq 0 \\
            \alpha\big[(a, b]\big] & = \alpha\big[(a, b)\big] + \alpha\big[\set{b}\big] \geq 0 \\
            \alpha\big[[a, b]\big] & = \alpha\big[\set{a}\big] + \alpha\big[(a, b]\big] \geq 0
          \end{align*}
          If \(\alpha\) is monotone decreasing, then from the proof above we have
          \begin{align*}
            \alpha\big[[a, b)\big] & = \alpha\big[\set{a}\big] + \alpha\big[(a, b)\big] \leq 0 \\
            \alpha\big[(a, b]\big] & = \alpha\big[(a, b)\big] + \alpha\big[\set{b}\big] \leq 0 \\
            \alpha\big[[a, b]\big] & = \alpha\big[\set{a}\big] + \alpha\big[(a, b]\big] \leq 0
          \end{align*}
  \end{itemize}
  From all cases above we conclude that \(\alpha[I] \geq 0\) if \(\alpha\) is monotone increasing and \(\alpha[I] \leq 0\) if \(\alpha\) is monotone decreasing.
\end{proof}

\begin{ac}\label{ac:11.8.7}
  Let \(I\) be a bounded interval, let \(\alpha : X \to \R\) be a monotone increasing function defined on some interval \(X\) which contains \(I\), and let \(f : I \to \R\) and \(g : I \to \R\) be piecewise constant functions on \(I\) such that both \(p.c. \int_I f \; d \alpha\) and \(p.c. \int_I g \; d \alpha\) are well-defined.
  \begin{enumerate}
    \item We have \(p.c. \int_I (f + g) \; d \alpha = p.c. \int_I f \; d \alpha + p.c. \int_I g \; d \alpha\).
    \item For any real number \(c\), we have \(p.c. \int_I (cf) \; d \alpha = c (p.c. \int_I f \; d \alpha)\).
    \item We have \(p.c. \int_I (f - g) \; d \alpha = p.c. \int_I f \; d \alpha - p.c. \int_I g \; d \alpha\).
    \item If \(f(x) \geq 0\) for all \(x \in I\), then \(p.c. \int_I f \; d \alpha \geq 0\).
    \item If \(f(x) \geq g(x)\) for all \(x \in I\), then \(p.c. \int_I f \; d \alpha \geq p.c. \int_I g \; d \alpha\).
    \item If \(f\) is the constant function \(f(x) = c\) for all \(x \in I\), then \(p.c. \int_I f \; d \alpha = c \alpha[I]\).
    \item Let \(J\) be a bounded interval containing \(I\) (i.e., \(I \subseteq J\)), and let \(F : J \to \R\) be the function
          \[
            F(x) \coloneqq \begin{dcases}
              f(x) & \text{if } x \in I    \\
              0    & \text{if } x \notin I
            \end{dcases}
          \]
          Then \(F\) is piecewise constant on \(J\), and \(p.c. \int_J F \; d \alpha = p.c. \int_I f \; d \alpha\).
    \item Suppose that \(\set{J, K}\) is a partition of \(I\) into two intervals \(J\) and \(K\).
          Then the function \(f|_J : J \to \R\) and \(f|_K : K \to \R\) are piecewise constant on \(J\) and \(K\) respectively, and we have
          \[
            p.c. \int_I f \; d \alpha = p.c. \int_J f|_J \; d \alpha + p.c. \int_K f|_K \; d \alpha.
          \]
  \end{enumerate}
\end{ac}

\begin{proof}{(a)}
  Let \(\mathbf{P}\) be a partition of \(I\).
  By \cref{11.2.16}(a) we know that \(f + g\) is piecewise constant with respect to \(\mathbf{P}\).
  For each \(J \in \mathbf{P}\), we define \(c_{f|_J}, c_{g|_J} \in \R\) to be the constant value of \(f|_J, g|_J\), respectively.
  Then by \cref{11.2.1} \(c_{f|_J} + c_{g|_J}\) is the constant value of \((f + g)|_J\) for each \(J \in P\).
  Thus we have
  \begin{align*}
     & p.c. \int_I f \; d \alpha + p.c. \int_I g \; d \alpha                                                 \\
     & = p.c. \int_{[\mathbf{P}]} f \; d \alpha + p.c. \int_{[\mathbf{P}]} g \; d \alpha &  & \by{ac:11.8.5} \\
     & = \sum_{J \in \mathbf{P}} f_J \alpha[J] + \sum_{J \in \mathbf{P}} g_J \alpha[J]   &  & \by{11.8.5}    \\
     & = \sum_{J \in \mathbf{P}} (f_J + g_J) \alpha[J]                                   &  & \by{7.1.11}[f] \\
     & = p.c. \int_{[\mathbf{P}]} (f_J + g_J) \; d \alpha                                &  & \by{11.8.5}    \\
     & = p.c. \int_I (f_J + g_J) \; d \alpha.                                            &  & \by{ac:11.8.5}
  \end{align*}
\end{proof}

\begin{proof}{(b)}
  Let \(\mathbf{P}\) be a partition of \(I\).
  By \cref{11.2.16}(b) we know that \(cf\) is piecewise constant with respect to \(\mathbf{P}\).
  For each \(J \in \mathbf{P}\), we define \(c_J \in \R\) to be the constant value of \(f|_J\).
  Then by \cref{11.2.1} \(c \cdot c_J\) is the constant value of \((cf)|_J\).
  Thus we have
  \begin{align*}
    c \bigg(p.c. \int_I f \; d \alpha\bigg) & = c \bigg(p.c. \int_{[\mathbf{P}]} f \; d \alpha\bigg) &  & \by{ac:11.8.5} \\
                                            & = c \bigg(\sum_{J \in \mathbf{P}} c_J \alpha[J]\bigg)  &  & \by{11.8.5}    \\
                                            & = \sum_{J \in \mathbf{P}} c \cdot c_J \alpha[J]        &  & \by{7.1.11}[g] \\
                                            & = p.c. \int_{[\mathbf{P}]} (c f) \; d \alpha           &  & \by{11.8.5}    \\
                                            & = p.c. \int_I (c f) \; d \alpha.                       &  & \by{ac:11.8.5}
  \end{align*}
\end{proof}

\begin{proof}{(c)}
  We have
  \begin{align*}
     & p.c. \int_I f \; d \alpha - p.c. \int_I g \; d \alpha                               \\
     & = p.c. \int_I f \; d \alpha + (-1) p.c. \int_I g \; d \alpha                        \\
     & = p.c. \int_I f \; d \alpha + p.c. \int_I (-g) \; d \alpha   &  & \by{ac:11.8.7}[b] \\
     & = p.c. \int_I \big(f + (-g)\big) \; d \alpha                 &  & \by{ac:11.8.7}[a] \\
     & = p.c. \int_I (f - g) \; d \alpha.                           &  & \by{9.2.1}
  \end{align*}
\end{proof}

\begin{proof}{(d)}
  By \cref{ac:11.8.5} \(f\) is piecewise constant with respect to \(\mathbf{P}\) for some partition \(\mathbf{P}\) of \(I\).
  Let \(J \in \mathbf{P}\) and let \(c_J \in \R\) be the constant value of \(f|_J\).
  By \cref{ac:11.8.6} we know that \(\alpha[J] \geq 0\) for all \(J \in \mathbf{P}\).
  Since \(f(x) \geq 0\) for all \(x \in I\), we have \(c_J \geq 0\) and \(c_J \alpha[J] \geq 0\) for all \(J \in \mathbf{P}\).
  Thus
  \begin{align*}
    p.c. \int_I f \; d \alpha & = p.c. \int_{[\mathbf{P}]} f \; d \alpha &  & \by{ac:11.8.5} \\
                              & = \sum_{J \in \mathbf{P}} c_J \alpha[J]  &  & \by{11.8.5}    \\
                              & \geq \sum_{J \in \mathbf{P}} 0           &  & \by{7.1.11}[h] \\
                              & = 0.
  \end{align*}
\end{proof}

\begin{proof}{(e)}
  Since \(f(x) \geq g(x)\) for all \(x \in I\), we have \(f(x) - g(x) \geq 0\).
  By \cref{ac:11.8.7}(c) we have
  \[
    p.c. \int_I f \; d \alpha - p.c. \int_I g \; d \alpha = p.c. \int_I (f - g) \; d \alpha.
  \]
  Then by \cref{ac:11.8.7}(d) we have
  \[
    p.c. \int_I (f - g) \; d \alpha \geq 0 \implies p.c. \int_I f \; d \alpha \geq p.c. \int_I g \; d \alpha.
  \]
\end{proof}

\begin{proof}{(f)}
  Since \(I\) is a partition of \(I\), we have
  \begin{align*}
    p.c. \int_I f \; d \alpha & = p.c. \int_{[I]} f \; d \alpha &  & \by{ac:11.8.5} \\
                              & = \sum_{J \in I} c \alpha[J]    &  & \by{11.8.5}    \\
                              & = c \sum_{J \in I} \alpha[J]    &  & \by{7.1.11}[g] \\
                              & = c \alpha[I].                  &  & \by{11.8.4}
  \end{align*}
\end{proof}

\begin{proof}{(g)}
  If \(I = \emptyset\), then by \cref{11.2.3} \(F\) is piecewise constant with respect to \(\set{J}\), and by \cref{ac:11.8.7}(f) we have
  \[
    p.c. \int_J F \; d \alpha = 0 \alpha[J] = 0 = p.c \int_I f \; d \alpha.
  \]
  So suppose that \(I \neq \emptyset\).
  By \cref{11.2.3}, \(f\) is piecewise constant with respect to \(\mathbf{P}\) for some partition \(\mathbf{P}\) of \(I\).
  Let \(I_1, I_2\) be the sets
  \[
    I_1 = \set{x \in J, \big(x \leq \inf(I)\big) \land (x \notin I)}
  \]
  and
  \[
    I_2 = \set{x \in J, \big(x \geq \sup(I)\big) \land (x \notin I)}.
  \]
  By \cref{ac:11.1.5} we know that \(\mathbf{P} \cup \set{I_1, I_2}\) is a partition of \(J\).
  By hypothesis we know that
  \[
    \forall x \in J, F(x) = \begin{dcases}
      f(x) & \text{if } x \in K \text{ for some } K \in \mathbf{P} \\
      0    & \text{if } x \in I_1 \text{ or } x \in I_2
    \end{dcases}
  \]
  Thus by \cref{11.2.5} \(F\) is piecewise constant on \(J\).
  For each \(K \in \mathbf{P} \cup \set{I_1, I_2}\), we define \(c_K \in \R\) to be the constant value of \(F|_K\).
  Then we have
  \begin{align*}
    p.c. \int_J F \; d \alpha & = p.c. \int_{[\mathbf{P} \cup \set{I_1, I_2}]} F \; d \alpha                        &  & \by{ac:11.8.5}         \\
                              & = \sum_{K \in \mathbf{P}} c_K \alpha[K]                                             &  & \by{11.8.5}            \\
                              & = c_{I_1} \alpha[I_1] + \sum_{K \in \mathbf{P}} c_K \alpha[K] + c_{I_2} \alpha[I_2] &  & \by{7.1.11}[e]         \\
                              & = 0 \alpha[I_1] + \sum_{K \in \mathbf{P}} c_K \alpha[K] + 0 \alpha[I_2]             &  & \text{(by hypothesis)} \\
                              & = \sum_{K \in \mathbf{P}} c_K \alpha[K]                                                                         \\
                              & = p.c. \int_{[\mathbf{P}]} f \; d \alpha                                            &  & \by{11.8.5}            \\
                              & = p.c. \int_I f \; d \alpha.                                                        &  & \by{ac:11.8.5}
  \end{align*}
\end{proof}

\begin{proof}{(h)}
  Let \(\mathbf{P} = \set{J, K}\).
  By \cref{11.2.3} \(f\) is piecewise constant with respect to \(\mathbf{P}'\) for some partition \(\mathbf{P}'\) of \(I\).
  Now we define \(\mathbf{P}_J\) as
  \[
    \mathbf{P}_J = \set{S \in \mathbf{P} \# \mathbf{P}' : S \subseteq J}
  \]
  and define \(\mathbf{P}_K\) as
  \[
    \mathbf{P}_K = \set{S \in \mathbf{P} \# \mathbf{P}' : S \subseteq K}.
  \]
  By \cref{11.1.8} we know that \(\mathbf{P} \# \mathbf{P}'\) is a partition of \(I\) and is finer than \(\mathbf{P}\).
  Since \(\mathbf{P} \# \mathbf{P}'\) is finer than \(\mathbf{P}\), by \cref{ac:11.1.4} we know that \(\mathbf{P}_J, \mathbf{P}_K\) are partitions of \(J, K\), respectively.
  Again by \cref{ac:11.1.4} we know that \(\mathbf{P}_J \cup \mathbf{P}_K\) is a partition of \(I\).
  Then by \cref{11.2.7} \(f\) is piecewise constant with respect to \(\mathbf{P}_J \cup \mathbf{P}_K\).
  Without the loss of generality suppose that \(\emptyset \notin \mathbf{P}_J \cup \mathbf{P}_K\).
  For each \(S \in \mathbf{P}_J\), we define \(c_S \in \R\) to be the constant value of \(f|_J\).
  Similarly, for each \(S \in \mathbf{P}_K\), we define \(c_S \in \R\) to be the constant value of \(f|_K\).
  Then we have
  \begin{align*}
      & p.c. \int_J f|_J \; d \alpha + p.c. \int_K f|_K \; d \alpha                                                   \\
    = & p.c. \int_{[\mathbf{P}_J]} f|_J \; d \alpha + p.c. \int_{[\mathbf{P}_K]} f|_K \; d \alpha &  & \by{ac:11.8.5} \\
    = & \sum_{S \in \mathbf{P}_J} c_S \alpha[S] + \sum_{S \in \mathbf{P}_K} c_S \alpha[S]         &  & \by{7.1.11}[e] \\
    = & \sum_{S \in \mathbf{P}_J \cup \mathbf{P}_K} c_S \alpha[S]                                 &  & \by{11.8.5}    \\
    = & \sum_{S \in \mathbf{P}} c_S \alpha[S]                                                                         \\
    = & p.c. \int_{[\mathbf{P}]} f \; d \alpha                                                    &  & \by{11.8.5}    \\
    = & p.c. \int_I f \; d \alpha.                                                                &  & \by{ac:11.8.5}
  \end{align*}
\end{proof}

\begin{ac}\label{ac:11.8.8}
  Let \(I\) be a bounded interval, let \(\alpha : X \to \R\) be a monotone increasing function defined on some interval \(X\) which contains \(I\), and let \(f : I \to \R\) be a bounded function.
  We define the \emph{upper Riemann-Stieltjes integral} \(\overline{\int}_I f \; d \alpha\) by the formula
  \[
    \overline{\int}_I f \; d \alpha \coloneqq \inf\set{p.c. \int_I g \; d \alpha : g \text{ is a p.c. function on \(I\) which majorizes } f}
  \]
  and the \emph{lower Riemann-Stieltjes integral} \(\underline{\int}_I f \; d \alpha\) by the formula
  \[
    \underline{\int}_I f \; d \alpha \coloneqq \sup\set{p.c. \int_I g \; d \alpha : g \text{ is a p.c. function on \(I\) which minorizes } f}.
  \]
  If \(\underline{\int}_I f \; d \alpha = \overline{\int}_I f \; d \alpha\), then we say that \(f\) is \emph{Riemann-Stieltjes integrable on \(I\) with respect to \(\alpha\)} and define
  \[
    \int_I f \; d \alpha \coloneqq \underline{\int}_I f \; d \alpha = \overline{\int}_I f \; d \alpha.
  \]
  If the upper and lower Riemann-Stieltjes integrals are unequal, we say that \(f\) is not Riemann-Stieltjes integrable on \(I\) with respect to \(\alpha\).
\end{ac}

\begin{ac}\label{ac:11.8.9}
  Let \(I\) be a bounded interval, let \(\alpha : X \to \R\) be a monotone increasing function defined on some interval \(X\) which contains \(I\).
  Let \(f : I \to \R\) be a function which is bounded by some real number \(M\), i.e., \(-M \leq f(x) \leq M\) for all \(x \in I\).
  Then we have
  \[
    -M \alpha[I] \leq \underline{\int}_I f \; d \alpha \leq \overline{\int}_I f \; d \alpha \leq M \alpha[I].
  \]
  in particular, both the lower and upper Riemann-Stieltjes integrals are real numbers (i.e., they are not infinite).
\end{ac}

\begin{proof}
  The function \(g : I \to \R\) defined by \(g(x) = M\) is constant, hence piecewise constant, and majorizes \(f\);
  thus \(\overline{\int}_I f \; d \alpha \leq p.c. \int_I g \; d \alpha = M \alpha[I]\) by definition of the upper Riemann-Stieltjes integral.
  A similar argument gives \(-M \alpha[I] \leq \underline{\int}_I f \; d \alpha\).
  Finally, we have to show that \(\underline{\int}_I f \; d \alpha \leq \overline{\int}_I f \; d \alpha\).
  Let \(g\) be any piecewise constant function majorizing \(f\), and let \(h\) be any piecewise constant function minorizing \(f\).
  Then \(g\) majorizes \(h\), and hence \(p.c. \int_I h \; d \alpha \leq p.c. \int_I g \; d \alpha\).
  Taking suprema in \(h\), we obtain that \(\underline{\int}_I f \; d \alpha \leq p.c. \int_I g \; d \alpha\).
  Taking infima in \(g\), we thus obtain \(\underline{\int}_I f \; d \alpha \leq \overline{\int}_I f \; d \alpha\), as desired.
\end{proof}

\begin{note}
  When \(\alpha\) is the identity function \(\alpha(x) \coloneqq x\) then the Riemann-Stieltjes integral is identical to the Riemann integral;
  thus the Riemann-Stieltjes integral is a generalization of the Riemann integral.
  We sometimes write \(\int_I f\) as \(\int_I f \; dx\) or \(\int_I f(x) \; dx\).
\end{note}

\begin{ac}\label{ac:11.8.10}
  Let \(I\) be a bounded interval, let \(\alpha : X \to \R\) be a monotone increasing function defined on some interval \(X\) which contains \(I\).
  Let \(f : I \to \R\) be a piecewise constant function.
  Then \(f\) is Riemann-Stieltjes integrable on \(I\) with respect to \(\alpha\), and \(\int_I f \; d \alpha = p.c. \int_I f \; d \alpha\).
\end{ac}

\begin{proof}
  Since \(f(x) \leq f(x)\) for every \(x \in I\), by \cref{ac:11.8.8} and \cref{ac:11.8.9} we have
  \[
    p.c. \int_I f \; d \alpha \leq \underline{\int}_I f \; d \alpha \leq \overline{\int}_I f \; d \alpha \leq p.c. \int_I f \; d \alpha
  \]
  Thus by \cref{ac:11.8.8} we have
  \[
    \int_I f \; d \alpha = \underline{\int}_I f \; d \alpha = \overline{\int}_I f \; d \alpha = p.c. \int_I f \; d \alpha.
  \]
\end{proof}

\begin{ac}[Laws of Riemann-Stieltjes integration]\label{ac:11.8.11}
  Let \(I\) be a bounded interval, let \(\alpha : X \to \R\) be a monotone increasing function defined on some interval \(X\) which contains \(I\).
  Let \(f : I \to \R\) and \(g : I \to \R\) be Riemann-Stieltjes integrable functions on \(I\) with respect to \(\alpha\).
  \begin{enumerate}
    \item The function \(f + g\) is Riemann-Stieltjes integrable, and we have \(\int_I (f + g) \; d \alpha = \int_I f \; d \alpha + \int_I g \; d \alpha\).
    \item For any real number \(c\), the function \(cf\) is Riemann-Stieltjes integrable, and we have \(\int_I (cf) \; d \alpha = c(\int_I f \; d \alpha)\).
    \item The function \(f - g\) is Riemann-Stieltjes integrable, and we have \(\int_I (f - g) \; d \alpha = \int_I f \; d \alpha - \int_I g \; d \alpha\).
    \item If \(f(x) \geq 0\) for all \(x \in I\), then \(\int_I f \; d \alpha \geq 0\).
    \item If \(f(x) \geq g(x)\) for all \(x \in I\), then \(\int_I f \; d \alpha \geq \int_I g \; d \alpha\).
    \item If \(f\) is the constant function \(f(x) = c\) for all \(x \in I\), then \(\int_I f \; d \alpha = c \alpha[I]\).
    \item Suppose that \(\set{J, K}\) is a partition of \(I\) into two intervals \(J\) and \(K\).
          Then the functions \(f|_J : J \to \R\) and \(f|_K : K \to \R\) are Riemann-Stieltjes integrable on \(J\) and \(K\) respectively, and we have
          \[
            \int_I f \; d \alpha = \int_J f|_J \; d \alpha + \int_K f|_K \; d \alpha.
          \]
  \end{enumerate}
\end{ac}

\begin{proof}{(a)}
  Let \(f_U : I \to \R\) and \(g_U : I \to \R\) be piecewise constant functions on \(I\) which majorizes \(f\) and \(g\), respectively.
  Let \(f_L : I \to \R\) and \(g_L : I \to \R\) be piecewise constant functions on \(I\) which minorizes \(f\) and \(g\), respectively.
  \(f_U, g_U, f_L, g_L\) are well-defined since by \cref{ac:11.8.8} \(f, g\) are bounded functions on a bounded interval \(I\).
  Then we have
  \[
    p.c. \int_I f_L \; d \alpha \leq \underline{\int}_I f \; d \alpha = \int_I f \; d \alpha = \overline{\int}_I f \; d \alpha \leq p.c. \int_I f_U \; d \alpha
  \]
  and
  \[
    p.c. \int_I g_L \; d \alpha \leq \underline{\int}_I g \; d \alpha = \int_I g \; d \alpha = \overline{\int}_I g \; d \alpha \leq p.c. \int_I g_U \; d \alpha.
  \]
  By \cref{ac:11.8.8} both \(f, g\) are bounded functions, so \(f + g\) is bounded function, and \(\underline{\int}_I (f + g) \; d \alpha, \overline{\int}_I (f + g) \; d \alpha\) are well-defined (by \cref{ac:11.8.8}).
  By \cref{ex:11.3.2} we know that \(f_U + g_U\) majorizes \(f + g_U\) and \(f + g_U\) majorizes \(f + g\), thus \(f_U + g_U\) majorizes \(f + g\).
  Similarly \(f_L + g_L\) minorizes \(f + g\).
  Then we have
  \begin{align*}
             & \overline{\int}_I (f + g) \; d \alpha \leq p.c. \int_I (f_U + g_U) \; d \alpha                               &   & \by{ac:11.8.8}                          \\
    \implies & \overline{\int}_I (f + g) \; d \alpha                                                                                                                      \\
             & \quad \leq p.c. \int_I f_U \; d \alpha + p.c. \int_I g_U \; d \alpha                                         &   & \by{ac:11.8.7}[a]                       \\
    \implies & \overline{\int}_I (f + g) \; d \alpha - p.c. \int_I g_U \; d \alpha                                                                                        \\
             & \quad \leq p.c. \int_I f_U \; d \alpha                                                                       &   & \text{(note that \(f_U\) is arbitrary)} \\
    \implies & \overline{\int}_I (f + g) \; d \alpha - p.c. \int_I g_U \; d \alpha \leq \overline{\int}_I f \; d \alpha     &   & \by{ac:11.8.8}                          \\
    \implies & \overline{\int}_I (f + g) \; d \alpha - \overline{\int}_I f \; d \alpha \leq p.c. \int_I g_U \; d \alpha     &   & \text{(note that \(g_U\) is arbitrary)} \\
    \implies & \overline{\int}_I (f + g) \; d \alpha - \overline{\int}_I f \; d \alpha \leq \overline{\int}_I g \; d \alpha &   & \by{ac:11.8.8}                          \\
    \implies & \overline{\int}_I (f + g) \; d \alpha \leq \overline{\int}_I f \; d \alpha + \overline{\int}_I g \; d \alpha &                                             \\
    \implies & \overline{\int}_I (f + g) \; d \alpha \leq \int_I f \; d \alpha + \int_I g \; d \alpha                       &   & \by{ac:11.8.8}
  \end{align*}
  and
  \begin{align*}
             & \underline{\int}_I (f + g) \; d \alpha \geq p.c. \int_I (f_L + g_L) \; d \alpha                                 &   & \by{ac:11.8.8}                          \\
    \implies & \underline{\int}_I (f + g) \; d \alpha                                                                                                                        \\
             & \quad \geq p.c. \int_I f_L \; d \alpha + p.c. \int_I g_L \; d \alpha                                            &   & \by{ac:11.8.7}[a]                       \\
    \implies & \underline{\int}_I (f + g) \; d \alpha - p.c. \int_I g_L \; d \alpha                                                                                          \\
             & \quad \geq p.c. \int_I f_L \; d \alpha                                                                          &   & \text{(note that \(f_L\) is arbitrary)} \\
    \implies & \underline{\int}_I (f + g) \; d \alpha - p.c. \int_I g_L \; d \alpha \geq \underline{\int}_I f \; d \alpha      &   & \by{ac:11.8.8}                          \\
    \implies & \underline{\int}_I (f + g) \; d \alpha - \underline{\int}_I f \; d \alpha \geq p.c. \int_I g_L \; d \alpha      &   & \text{(note that \(g_L\) is arbitrary)} \\
    \implies & \underline{\int}_I (f + g) \; d \alpha - \underline{\int}_I f \; d \alpha \geq \underline{\int}_I g \; d \alpha &   & \by{ac:11.8.8}                          \\
    \implies & \underline{\int}_I (f + g) \; d \alpha \geq \underline{\int}_I f \; d \alpha + \underline{\int}_I g \; d \alpha &                                             \\
    \implies & \underline{\int}_I (f + g) \; d \alpha \geq \int_I f \; d \alpha + \int_I g \; d \alpha.                        &   & \by{ac:11.8.8}
  \end{align*}
  By \cref{ac:11.8.9} we have
  \[
    \int_I f \; d \alpha + \int_I g \; d \alpha \leq \underline{\int}_I (f + g) \; d \alpha \leq \overline{\int}_I (f + g) \; d \alpha \leq \int_I f \; d \alpha + \int_I g \; d \alpha
  \]
  and thus by \cref{ac:11.8.8} we have
  \[
    \int_I (f + g) \; d \alpha = \underline{\int}_I (f + g) \; d \alpha = \overline{\int}_I (f + g) \; d \alpha = \int_I f \; d \alpha + \int_I g \; d \alpha.
  \]
\end{proof}

\begin{proof}{(b)}
  Since \(f\) is Riemann-Stieltjes integrable on \(I\) with respect to \(\alpha\), by \cref{ac:11.8.8} we have
  \[
    \int_I f \; d \alpha = \overline{\int}_I f \; d \alpha = \underline{\int}_I f \; d \alpha.
  \]
  First suppose that \(c = 0\).
  Then we have \((cf)(x) = 0\) for all \(x \in 0\), thus we have
  \begin{align*}
    \int_I (cf) \; d \alpha & = p.c. \int_I (cf) \; d \alpha &  & \by{ac:11.8.10} \\
                            & = 0                                                 \\
                            & = c \int_I f \; d \alpha.
  \end{align*}

  Next suppose that \(c > 0\).
  Let \(f_U : I \to \R\) be a piecewise constant function on \(I\) which majorizes \(f\).
  Let \(f_L : I \to \R\) be a piecewise constant function on \(I\) which minorizes \(f\).
  \(f_U, f_L\) are well-defined since by \cref{ac:11.8.8} \(f\) is a bounded function on a bounded interval \(I\).
  Then by \cref{ac:11.8.8} we have
  \[
    p.c. \int_I f_L \; d \alpha \leq \underline{\int}_I f \; d \alpha = \int_I f \; d \alpha = \overline{\int}_I f \; d \alpha \leq p.c. \int_I f_U \; d \alpha.
  \]
  Since \(f\) is a bounded function, \(cf\) is also a bounded function, by \cref{ac:11.8.8} both \(\overline{\int}_I (cf) \; d \alpha, \underline{\int}_I (cf) \; d \alpha\) are well-defined.
  Since \(c > 0\), by \cref{11.3.1} we know that \(c f_U\) majorizes \(c f\) and \(c f_L\) minorizes \(c f\).
  Then we have
  \begin{align*}
             & \overline{\int}_I (cf) \; d \alpha \leq p.c. \int_I (c f_U) \; d \alpha                          &  & \by{ac:11.8.8}                          \\
    \implies & \overline{\int}_I (cf) \; d \alpha \leq c \bigg(p.c. \int_I f_U \; d \alpha\bigg)                &  & \by{ac:11.8.7}[b]                       \\
    \implies & \dfrac{1}{c} \bigg(\overline{\int}_I (cf) \; d \alpha\bigg) \leq p.c. \int_I f_U \; d \alpha     &  & \text{(note that \(f_U\) is arbitrary)} \\
    \implies & \dfrac{1}{c} \bigg(\overline{\int}_I (cf) \; d \alpha\bigg) \leq \overline{\int}_I f \; d \alpha &  & \by{ac:11.8.8}                          \\
    \implies & \overline{\int}_I (cf) \; d \alpha \leq c\bigg(\overline{\int}_I f \; d \alpha\bigg)                                                          \\
    \implies & \overline{\int}_I (cf) \; d \alpha \leq c\bigg(\int_I f \; d \alpha\bigg)                        &  & \by{ac:11.8.8}
  \end{align*}
  and
  \begin{align*}
             & \underline{\int}_I (cf) \; d \alpha \geq p.c. \int_I (c f_L) \; d \alpha                           &  & \by{ac:11.8.8}                          \\
    \implies & \underline{\int}_I (cf) \; d \alpha \geq c \bigg(p.c. \int_I f_L \; d \alpha\bigg)                 &  & \by{ac:11.8.7}[b]                       \\
    \implies & \dfrac{1}{c} \bigg(\underline{\int}_I (cf) \; d \alpha\bigg) \geq p.c. \int_I f_L \; d \alpha      &  & \text{(note that \(f_L\) is arbitrary)} \\
    \implies & \dfrac{1}{c} \bigg(\underline{\int}_I (cf) \; d \alpha\bigg) \geq \underline{\int}_I f \; d \alpha &  & \by{ac:11.8.8}                          \\
    \implies & \underline{\int}_I (cf) \; d \alpha \geq c\bigg(\underline{\int}_I f \; d \alpha\bigg)                                                          \\
    \implies & \underline{\int}_I (cf) \; d \alpha \geq c\bigg(\int_I f \; d \alpha\bigg).                        &  & \by{ac:11.8.8}
  \end{align*}
  By \cref{ac:11.8.9} we have
  \[
    c\bigg(\int_I f \; d \alpha\bigg) \leq \underline{\int}_I (cf) \; d \alpha \leq \overline{\int}_I (cf) \; d \alpha \leq c\bigg(\int_I f \; d \alpha\bigg)
  \]
  and thus by \cref{ac:11.8.8} we have
  \[
    \int_I (cf) \; d \alpha = \underline{\int}_I (cf) \; d \alpha = \overline{\int}_I (cf) \; d \alpha = c\bigg(\int_I f \; d \alpha\bigg).
  \]

  Finally suppose that \(c < 0\).
  Using the same definition of \(f_U, f_L\) we have
  \begin{align*}
             & \overline{\int}_I (cf \; d \alpha) \leq p.c. \int_I (c f_U \; d \alpha)                                                           &  & \by{ac:11.8.8}    \\
    \implies & \overline{\int}_I (cf \; d \alpha) \leq c \bigg(p.c. \int_I f_U \; d \alpha\bigg)                                                 &  & \by{ac:11.8.7}[b] \\
    \implies & \dfrac{1}{c} \bigg(\overline{\int}_I (cf) \; d \alpha\bigg) \geq p.c. \int_I f_U \; d \alpha                                                             \\
    \implies & \dfrac{1}{c} \bigg(\overline{\int}_I (cf) \; d \alpha\bigg) \geq p.c. \int_I f_U \; d \alpha \geq \overline{\int}_I f \; d \alpha &  & \by{ac:11.8.8}    \\
    \implies & \overline{\int}_I (cf) \; d \alpha \leq c\bigg(\overline{\int}_I f \; d \alpha\bigg)                                                                     \\
    \implies & \overline{\int}_I (cf) \; d \alpha \leq c\bigg(\int_I f \; d \alpha\bigg)                                                         &  & \by{ac:11.8.8}
  \end{align*}
  and
  \begin{align*}
             & \underline{\int}_I (cf) \; d \alpha \geq p.c. \int_I (c f_L) \; d \alpha                                                            &  & \by{ac:11.8.8}    \\
    \implies & \underline{\int}_I (cf) \; d \alpha \geq c \bigg(p.c. \int_I f_L \; d \alpha\bigg)                                                  &  & \by{ac:11.8.7}[b] \\
    \implies & \dfrac{1}{c} \bigg(\underline{\int}_I (cf) \; d \alpha\bigg) \leq p.c. \int_I f_L \; d \alpha                                                              \\
    \implies & \dfrac{1}{c} \bigg(\underline{\int}_I (cf) \; d \alpha\bigg) \leq p.c. \int_I f_L \; d \alpha \leq \underline{\int}_I f \; d \alpha &  & \by{ac:11.8.8}    \\
    \implies & \underline{\int}_I (cf) \; d \alpha \geq c\bigg(\underline{\int}_I f \; d \alpha\bigg)                                                                     \\
    \implies & \underline{\int}_I (cf) \; d \alpha \geq c\bigg(\int_I f \; d \alpha\bigg).                                                         &  & \by{ac:11.8.8}
  \end{align*}
  By \cref{ac:11.8.9} we have
  \[
    c\bigg(\int_I f \; d \alpha\bigg) \leq \underline{\int}_I (cf) \; d \alpha \leq \overline{\int}_I (cf) \; d \alpha \leq c\bigg(\int_I f \; d \alpha\bigg)
  \]
  and thus by \cref{ac:11.8.8} we have
  \[
    \int_I (cf) \; d \alpha = \underline{\int}_I (cf) \; d \alpha = \overline{\int}_I (cf) \; d \alpha = c\bigg(\int_I f \; d \alpha\bigg).
  \]
  We conclude that \(\forall c \in \R\), \(\int_I (cf) \; d \alpha = c (\int_I f \; d \alpha)\).
\end{proof}

\begin{proof}{(c)}
  We have
  \begin{align*}
    \int_I f \; d \alpha - \int_I g \; d \alpha & = \int_I f \; d \alpha + \int_I (-g) \; d \alpha &  & \by{ac:11.8.11}[b] \\
                                                & = \int_I \big(f + (-g) \; d \alpha\big)          &  & \by{ac:11.8.11}[a] \\
                                                & = \int_I (f - g) \; d \alpha.                    &  & \by{9.2.1}
  \end{align*}
\end{proof}

\begin{proof}{(d)}
  Let \(f_U : I \to \R\) be a piecewise constant function on \(I\) which majorizes \(f\).
  \(f_U\) is well-defined since by \cref{ac:11.8.8} \(f\) is a bounded function on a bounded interval \(I\).
  Since \(0 \leq f(x) \leq f_U(x)\) for every \(x \in I\), we have
  \begin{align*}
             & 0 \leq p.c. \int_I f_U \; d \alpha     &  & \by{ac:11.8.7}[d] \\
    \implies & 0 \leq \overline{\int}_I f \; d \alpha &  & \by{ac:11.8.8}    \\
    \implies & 0 \leq \int_I f \; d \alpha.           &  & \by{ac:11.8.8}
  \end{align*}
\end{proof}

\begin{proof}{(e)}
  We have \(f(x) - g(x) \geq 0\) for every \(x \in I\) and by \cref{ac:11.8.11}(c) \(f - g\) is Riemann-Stieltjes integrable on \(I\) with respect to \(\alpha\).
  Thus
  \begin{align*}
             & \int_I (f - g) \; d \alpha \geq 0                  &  & \by{ac:11.8.11}[d] \\
    \implies & \int_I f \; d \alpha - \int_I g \; d \alpha \geq 0 &  & \by{ac:11.8.11}[c] \\
    \implies & \int_I f \; d \alpha \geq \int_I g \; d \alpha.
  \end{align*}
\end{proof}

\begin{proof}{(f)}
  We have
  \begin{align*}
    \int_I f \; d \alpha & = p.c. \int_I f \; d \alpha &  & \by{ac:11.8.10}   \\
                         & = c \alpha[I].              &  & \by{ac:11.8.7}[f]
  \end{align*}
\end{proof}

\begin{proof}{(g)}
  Let \(f_U : I \to \R\) be a piecewise constant function on \(I\) which majorizes \(f\).
  Let \(f_L : I \to \R\) be a piecewise constant function on \(I\) which minorizes \(f\).
  \(f_U, f_L\) are well-defined since by \cref{ac:11.8.8} \(f\) is a bounded function on a bounded interval \(I\).
  Then we have
  \[
    p.c. \int_I f_L \leq \underline{\int}_I f = \int_I f = \overline{\int}_I f \leq p.c. \int_I f_U.
  \]
  By \cref{ac:11.8.7}(h) we know that \(f_U|_J : J \to \R\), \(f_L|_J : J \to \R\) are piecewise constant function on \(J\) and \(f_U|_K : K \to \R\), \(f_L|_K : K \to \R\) are piecewise constant functions on \(K\).
  By \cref{11.3.1} we know that \(f_U|_J\) majorizes \(f|_J\) and \(f_L|_J\) minorizes \(f|_J\), similarly \(f_U|_K\) majorizes \(f|_K\) and \(f_L|_K\) minorizes \(f|_K\).
  Thus \(f|_J\), \(f|_K\) are bounded functions on bounded intervals \(J, K\), respectively.
  So \(\overline{\int}_J f|_J \; d \alpha\), \(\overline{\int}_K f|_K \; d \alpha\), \(\underline{\int}_J f|_J \; d \alpha\), \(\underline{\int}_K f|_K \; d \alpha\) are well-defined.
  Then we have
  \begin{align*}
             & \overline{\int}_J f|_J \; d \alpha + \overline{\int}_K f|_K \; d \alpha                                                             \\
             & \quad \leq p.c. \int_J f_U|_J \; d \alpha + p.c. \int_K f_U|_K \; d \alpha                                   &  & \by{ac:11.8.8}    \\
    \implies & \overline{\int}_J f|_J \; d \alpha + \overline{\int}_K f|_K \; d \alpha \leq p.c. \int_I f_U \; d \alpha     &  & \by{ac:11.8.7}[h] \\
    \implies & \overline{\int}_J f|_J \; d \alpha + \overline{\int}_K f|_K \; d \alpha \leq \overline{\int}_I f \; d \alpha &  & \by{ac:11.8.8}    \\
    \implies & \overline{\int}_J f|_J \; d \alpha + \overline{\int}_K f|_K \; d \alpha \leq \int_I f \; d \alpha            &  & \by{ac:11.8.8}
  \end{align*}
  and
  \begin{align*}
             & \underline{\int}_J f|_J \; d \alpha + \underline{\int}_K f|_K \; d \alpha                                                              \\
             & \quad \geq p.c. \int_J f_L|_J \; d \alpha + p.c. \int_K f_L|_K \; d \alpha                                      &  & \by{ac:11.8.8}    \\
    \implies & \underline{\int}_J f|_J \; d \alpha + \underline{\int}_K f|_K \; d \alpha \geq p.c. \int_I f_L \; d \alpha      &  & \by{ac:11.8.7}[h] \\
    \implies & \underline{\int}_J f|_J \; d \alpha + \underline{\int}_K f|_K \; d \alpha \geq \underline{\int}_I f \; d \alpha &  & \by{ac:11.8.8}    \\
    \implies & \underline{\int}_J f|_J \; d \alpha + \underline{\int}_K f|_K \; d \alpha \geq \int_I f \; d \alpha.            &  & \by{ac:11.8.8}
  \end{align*}
  By \cref{ac:11.8.9} we have
  \[
    \int_I f \; d \alpha \leq \underline{\int}_J f|_J \; d \alpha + \underline{\int}_K f|_K \; d \alpha \leq \overline{\int}_J f|_J \; d \alpha + \overline{\int}_K f|_K \; d \alpha \leq \int_I f \; d \alpha
  \]
  and thus we have
  \[
    \underline{\int}_J f|_J \; d \alpha + \underline{\int}_K f|_K \; d \alpha = \overline{\int}_J f|_J \; d \alpha + \overline{\int}_J f|_K \; d \alpha = \int_I f \; d \alpha.
  \]
  Since
  \begin{align*}
             & \underline{\int}_J f|_J \; d \alpha + \underline{\int}_K f|_K \; d \alpha                                   \\
             & \quad = \overline{\int}_J f|_J \; d \alpha + \overline{\int}_J f|_K \; d \alpha                             \\
    \implies & 0 \geq \underline{\int}_J f|_J \; d \alpha - \overline{\int}_J f|_J \; d \alpha                             \\
             & \quad = \overline{\int}_J f|_K \; d \alpha - \underline{\int}_K f|_K \; d \alpha \geq 0 &  & \by{ac:11.8.9} \\
    \implies & \underline{\int}_J f|_J \; d \alpha - \overline{\int}_J f|_J \; d \alpha                                    \\
             & \quad = \overline{\int}_J f|_K \; d \alpha - \underline{\int}_K f|_K \; d \alpha = 0,
  \end{align*}
  by \cref{ac:11.8.8} we have
  \begin{align*}
     & \int_J f|_J \; d \alpha = \underline{\int}_J f|_J \; d \alpha = \overline{\int}_J f|_J \; d \alpha, \\
     & \int_K f|_K \; d \alpha = \underline{\int}_K f|_K \; d \alpha = \overline{\int}_K f|_K \; d \alpha, \\
     & \int_J f|_J \; d \alpha + \int_K f|_K \; d \alpha = \int_I f \; d \alpha.
  \end{align*}
\end{proof}

\begin{ac}[Laws of Riemann-Stieltjes integration]\label{ac:11.8.12}
  Let \(I\) be a bounded interval, let \(\alpha : X \to \R\) be a monotone increasing function defined on some interval \(X\) which contains \(I\).
  Let \(f : I \to \R\) be a bounded function, and let \(\mathbf{P}\) be a partition of \(I\).
  We define the \emph{upper Riemann-Stieltjes sum} \(U(f, \alpha, \mathbf{P})\) and the \emph{lower Riemann-Stieltjes sum} \(L(f, \alpha, \mathbf{P})\) by
  \[
    U(f, \alpha, \mathbf{P}) \coloneqq \sum_{J \in \mathbf{P} : J \neq \emptyset} \big(\sup_{x \in J} f(x)\big) \alpha[J]
  \]
  and
  \[
    L(f, \alpha, \mathbf{P}) \coloneqq \sum_{J \in \mathbf{P} : J \neq \emptyset} \big(\inf_{x \in J} f(x)\big) \alpha[J].
  \]
\end{ac}

\begin{ac}\label{ac:11.8.13}
  Let \(I\) be a bounded interval, let \(\alpha : X \to \R\) be a monotone increasing function defined on some interval \(X\) which contains \(I\).
  Let \(f : I \to \R\) be a bounded function, and let \(g\) be a function which majorizes \(f\) and which is piecewise constant with respect to some partition \(\mathbf{P}\) of \(I\).
  Then
  \[
    p.c. \int_I g \; d \alpha \geq U(f, \alpha, \mathbf{P}).
  \]
  Similarly, if \(h\) is a function which minorizes \(f\) and is piecewise constant with respect to \(\mathbf{P}\), then
  \[
    p.c. \int_I h \; d \alpha \leq L(f, \alpha, \mathbf{P}).
  \]
\end{ac}

\begin{proof}
  Since \(g\) majorizes \(f\) and \(h\) minorizes \(f\), by \cref{11.3.1} we have \(h(x) \leq f(x) \leq g(x)\) for every \(x \in I\).
  Since \(\mathbf{P}\) is a partition of \(I\), by \cref{11.1.10} for every \(J \in \mathbf{P}\), we have \(h(x) \leq f(x) \leq g(x)\) for all \(x \in J\).
  In particular, when \(J \neq \emptyset\) we have
  \[
    h(x) \leq \inf_{x \in J} f(x) \leq f(x) \leq \sup_{x \in J} f(x) \leq g(x)
  \]
  for every \(x \in J\).
  Let \(c_{g|_J}, c_{h|_J}\) be constant values of \(g|_J, h|_J\), respectively.
  Then we have
  \begin{align*}
    U(f, \alpha, \mathbf{P}) & = \sum_{J \in \mathbf{P} : J \neq \emptyset} \big(\sup_{x \in J} f(x)\big) \alpha[J] &  & \by{ac:11.8.12}  \\
                             & \leq \sum_{J \in \mathbf{P} : J \neq \emptyset} c_{g|_J} \alpha[J]                   &  & \by{7.1.11}[h]   \\
                             & = \sum_{J \in \mathbf{P}} c_{g|_J} \alpha[J]                                         &  & \by{7.1.11}[a,e] \\
                             & = p.c. \int_{[\mathbf{P}]} g \; d \alpha                                             &  & \by{11.8.5}      \\
                             & = p.c. \int_I g \; d \alpha                                                          &  & \by{ac:11.8.5}
  \end{align*}
  and
  \begin{align*}
    L(f, \alpha, \mathbf{P}) & = \sum_{J \in \mathbf{P} : J \neq \emptyset} \big(\inf_{x \in J} f(x)\big) \alpha[J] &  & \by{ac:11.8.12}  \\
                             & \geq \sum_{J \in \mathbf{P} : J \neq \emptyset} c_{h|_J} \alpha[J]                   &  & \by{7.1.11}[h]   \\
                             & = \sum_{J \in \mathbf{P}} c_{h|_J} \alpha[J]                                         &  & \by{7.1.11}[a,e] \\
                             & = p.c. \int_{[\mathbf{P}]} h \; d \alpha                                             &  & \by{11.8.5}      \\
                             & = p.c. \int_I h \; d \alpha.                                                         &  & \by{ac:11.8.5}
  \end{align*}
\end{proof}

\begin{ac}\label{ac:11.8.14}
  Let \(I\) be a bounded interval, let \(\alpha : X \to \R\) be a monotone increasing function defined on some interval \(X\) which contains \(I\).
  Let \(f : I \to \R\) be a bounded function.
  Then
  \[
    \overline{\int}_I f \; d \alpha = \inf\set{U(f, \alpha, \mathbf{P}) : \mathbf{P} \text{ is a partition of } I}
  \]
  and
  \[
    \underline{\int}_I f \; d \alpha = \sup\set{L(f, \alpha, \mathbf{P}) : \mathbf{P} \text{ is a partition of } I}.
  \]
\end{ac}

\begin{proof}
  Let \(g\) be a function which majorizes \(f\) and which is piecewise constant with respect to some partition \(\mathbf{P}_g\) of \(I\).
  Let \(h\) be a function which minorizes \(f\) and which is piecewise constant with respect to some partition \(\mathbf{P}_h\) of \(I\).
  Both functions are well defined since \(f\) is bounded function on a bounded interval \(I\).
  By \cref{ac:11.8.13} we have
  \[
    \inf\set{U(f, \alpha, \mathbf{P}) : \mathbf{P} \text{ is a partition of } I} \leq U(f, \alpha, \mathbf{P}_g) \leq p.c. \int_I g \; d \alpha
  \]
  and
  \[
    \sup\set{L(f, \alpha, \mathbf{P}) : \mathbf{P} \text{ is a partition of } I} \geq L(f, \alpha, \mathbf{P}_h) \geq p.c. \int_I h \; d \alpha.
  \]
  Since \(g, h\) are arbitrary, by \cref{ac:11.8.8} we have
  \[
    \inf\set{U(f, \alpha, \mathbf{P}) : \mathbf{P} \text{ is a partition of } I} \leq \overline{\int}_I f \; d \alpha
  \]
  and
  \[
    \sup\set{L(f, \alpha, \mathbf{P}) : \mathbf{P} \text{ is a partition of } I} \geq \underline{\int}_I f \; d \alpha.
  \]

  Let \(\mathbf{P}\) be a partition of \(I\).
  Let \(G : I \to \R\) be a function where \(G(x) = \sup_{x \in J} f(x)\) for all \(J \in \mathbf{P}\).
  Let \(H : I \to \R\) be a function where \(H(x) = \inf_{x \in J} f(x)\) for all \(J \in \mathbf{P}\).
  By \cref{11.2.3} we know that \(G, H\) are piecewise constant functions with respect to \(\mathbf{P}\).
  Thus we have
  \begin{align*}
    U(f, \alpha, \mathbf{P}) & = \sum_{J \in \mathbf{P} : J \neq \emptyset} \big(\sup_{x \in J} f(x)\big) \alpha[J] &  & \by{ac:11.8.12}  \\
                             & = \sum_{J \in \mathbf{P}} \big(\sup_{x \in J} f(x)\big) \alpha[J]                    &  & \by{7.1.11}[a,e] \\
                             & = p.c. \int_{[\mathbf{P}]} G \; d \alpha                                             &  & \by{11.8.5}      \\
                             & = p.c. \int_I G \; d \alpha                                                          &  & \by{ac:11.8.5}
  \end{align*}
  and
  \begin{align*}
    L(f, \alpha, \mathbf{P}) & = \sum_{J \in \mathbf{P} : J \neq \emptyset} \big(\inf_{x \in J} f(x)\big) \alpha[J] &  & \by{ac:11.8.12}  \\
                             & = \sum_{J \in \mathbf{P}} \big(\inf_{x \in J} f(x)\big) \alpha[J]                    &  & \by{7.1.11}[a,e] \\
                             & = p.c. \int_{[\mathbf{P}]} H \; d \alpha                                             &  & \by{11.8.5}      \\
                             & = p.c. \int_I H \; d \alpha.                                                         &  & \by{ac:11.8.5}
  \end{align*}
  By \cref{ac:11.8.8} we have
  \[
    \overline{\int}_I f \; d \alpha \leq p.c. \int_I G \; d \alpha = U(f, \alpha, \mathbf{P})
  \]
  and
  \[
    \underline{\int}_I f \; d \alpha \geq p.c. \int_I H \; d \alpha = L(f, \alpha, \mathbf{P}).
  \]
  Since \(\mathbf{P}\) is arbitrary, we have
  \[
    \overline{\int}_I f \; d \alpha \leq \inf\set{U(f, \alpha, \mathbf{P}) : \mathbf{P} \text{ is a partition of } I} \leq U(f, \alpha, \mathbf{P})
  \]
  and
  \[
    \underline{\int}_I f \; d \alpha \geq \sup\set{L(f, \alpha, \mathbf{P}) : \mathbf{P} \text{ is a partition of } I} \leq L(f, \alpha, \mathbf{P}).
  \]
  Combine all results above we have
  \[
    \overline{\int}_I f \; d \alpha = \inf\set{U(f, \alpha, \mathbf{P}) : \mathbf{P} \text{ is a partition of } I}
  \]
  and
  \[
    \underline{\int}_I f \; d \alpha = \sup\set{L(f, \alpha, \mathbf{P}) : \mathbf{P} \text{ is a partition of } I}.
  \]
\end{proof}

\begin{ac}\label{ac:11.8.15}
  Let \(I\) be a bounded interval, let \(\alpha : X \to \R\) be a monotone increasing function defined on some interval \(X\) which contains \(I\).
  Let \(f\) be a function which is uniformly continuous on \(I\).
  Then \(f\) is Riemann-Stieltjes integrable on \(I\) with respect to \(\alpha\).
\end{ac}

\begin{proof}
  From \cref{9.9.15} we see that \(f\) is bounded.
  By \cref{ac:11.8.8} we have to show that \(\underline{\int}_I f \; d \alpha = \overline{\int}_I f \; d \alpha\).

  If \(I\) is a point or the empty set then the theorem is trivial, so let us assume that \(I\) is one of the four intervals \([a, b]\), \((a, b)\), \((a, b]\), or \([a, b)\) for some real numbers \(a < b\).

  Let \(\varepsilon > 0\) be arbitrary.
  By uniform continuity, there exists a \(\delta > 0\) such that \(\abs{f(x) - f(y)} < \varepsilon\) whenever \(x, y \in I\) are such that \(\abs{x - y} < \delta\).
  By the Archimedean principle, there exists an integer \(N > 0\) such that \((b - a) / N < \delta\) and
  \[
    \dfrac{\lim_{x \to b^- ; x \in X} \alpha(x) - \lim_{x \to a^+ ; x \in X} \alpha(x)}{N} < \delta.
  \]

  Note that we can partition \(I\) into \(N\) intervals \(J_1, \dots, J_N\), each of length \((b - a) / N\).
  By \cref{ac:11.8.14}, we thus have
  \[
    \overline{\int}_I f \; d \alpha \leq \sum_{k = 1}^N \big(\sup_{x \in J_k} f(x)\big) \alpha[J_k]
  \]
  and
  \[
    \underline{\int}_I f \; d \alpha \geq \sum_{k = 1}^N \big(\inf_{x \in J_k} f(x)\big) \alpha[J_k]
  \]
  so in particular
  \[
    \overline{\int}_I f \; d \alpha - \underline{\int}_I f \; d \alpha \leq \sum_{k = 1}^N \big(\sup_{x \in J_k} f(x) - \inf_{x \in J_k} f(x)\big) \alpha[J_k].
  \]
  However, we have \(\abs{f(x) - f(y)} < \varepsilon\) for all \(x, y \in J_k\), since
  \begin{align*}
    \alpha[J_k] & = \dfrac{\lim_{x \to \sup(J_K)^- ; x \in X} \alpha(x) - \lim_{x \to \inf(J_K)^+ ; x \in X} \alpha(x)}{N}                                                 \\
                & \leq \dfrac{\lim_{x \to b^- ; x \in X} \alpha(x) - \lim_{x \to a^+ ; x \in X} \alpha(x)}{N}              &  & \text{(\(\alpha\) is monotone increasing)} \\
                & < \delta.
  \end{align*}
  In particular we have
  \[
    f(x) < f(y) + \varepsilon \text{ for all } x, y \in J_k.
  \]
  Taking suprema in \(x\), we obtain
  \[
    \sup_{x \in J_k} f(x) \leq f(y) + \varepsilon \text{ for all } y \in J_k,
  \]
  and then taking infima in \(y\) we obtain
  \[
    \sup_{x \in J_k} f(x) \leq \inf_{y \in J_k} f(y) + \varepsilon.
  \]
  Inserting this bound into our previous inequality, we obtain
  \[
    \overline{\int}_I f \; d \alpha - \underline{\int}_I f \; d \alpha \leq \sum_{k = 1}^N \varepsilon \alpha[J_k],
  \]
  but by \cref{11.8.4} we thus have
  \[
    \overline{\int}_I f \; d \alpha - \underline{\int}_I f \; d \alpha \leq \varepsilon \alpha[I].
  \]
  But \(\varepsilon > 0\) was arbitrary, while \(\alpha[I]\) is fixed.
  Thus \(\overline{\int}_I f \; d \alpha - \underline{\int}_I f \; d \alpha\) cannot be positive.
  By \cref{ac:11.8.8} we thus have that \(f\) is Riemann-Stieltjes integrable on \(I\) with respect to \(\alpha\).
\end{proof}

\exercisesection

\begin{ex}\label{ex:11.8.1}
  Prove \cref{11.8.4}.
\end{ex}

\begin{proof}
  See \cref{11.8.4}.
\end{proof}

\begin{ex}\label{ex:11.8.2}
  State and prove a version of \cref{11.2.13} for the Riemann-Stieltjes integral.
\end{ex}

\begin{proof}
  See \cref{ac:11.8.4}.
\end{proof}

\begin{ex}\label{ex:11.8.3}
  State and prove a version of \cref{11.2.16} for the Riemann-Stieltjes integral.
\end{ex}

\begin{proof}
  See \cref{ac:11.8.7}.
\end{proof}

\begin{ex}\label{ex:11.8.4}
  State and prove a version of \cref{11.5.1} for the Riemann-Stieltjes integral.
\end{ex}

\begin{proof}
  See \cref{ac:11.8.15}.
\end{proof}

\begin{ex}\label{ex:11.8.5}
  Let \(\text{sgn} : \R \to \R\) be the signum function
  \[
    \text{sgn}(x) = \begin{dcases}
      1  & \text{when } x > 0  \\
      0  & \text{when } x = 0  \\
      -1 & \text{when } x < 0.
    \end{dcases}
  \]
  Let \(f : [-1, 1] \to \R\) be a continuous function.
  Show that \(f\) is Riemann-Stieltjes integrable with respect to \(\text{sgn}\), and that
  \[
    \int_{[-1, 1]} f \; d \, \text{sgn} = 2f(0).
  \]
\end{ex}

\begin{proof}
  We first show that \(f\) is Riemann-Stieltjes integrable on \([-1, 1]\) with respect to \(\text{sgn}\).
  By \cref{9.9.16} \(f\) is uniformly continuous, and thus by \cref{9.9.15} \(f\) is bounded.
  Since \(\text{sgn}\) is monotone increasing, by \cref{ac:11.8.15} we know that \(f\) is Riemann-Stieltjes integrable on \([-1, 1]\) with respect to \(\text{sgn}\).

  Now we show that \(\int_{[-1, 1]} f \; d \, \text{sgn} = 2f(0)\).
  Since \(f\) is continuous, by \cref{9.4.7} we have
  \begin{align*}
             & \forall \varepsilon \in \R^+, \exists \delta \in \R^+ : \forall x \in [-1, 1], \abs{x - 0} \leq \delta \\
    \implies & \abs{f(x) - f(0)} \leq \varepsilon                                                                     \\
    \implies & f(0) - \varepsilon \leq f(x) \leq f(0) + \varepsilon.
  \end{align*}
  In particular, we can choose some \(\delta \leq 1\) such that
  \begin{align*}
     & \forall \varepsilon \in \R^+, \exists \delta \in \R^+ :                                                              \\
     & \forall x \in [-1, 1], \abs{x - 0} \leq \delta \leq 1 \implies f(0) - \varepsilon \leq f(x) \leq f(0) + \varepsilon.
  \end{align*}
  Since \(f\) is bounded, \(\exists M \in \R^+\) such that \(\abs{f(x)} \leq M\) for all \(x \in [-1, 1]\).
  Let \(f_U : [-1, 1] \to \R\) be the function
  \[
    f_U(x) = \begin{dcases}
      f(0) + \varepsilon & \text{if } x \in [-\delta, \delta]                   \\
      M                  & \text{if } x \in [-1, 1] \setminus [-\delta, \delta]
    \end{dcases}
  \]
  and let \(f_L : [-1, 1] \to \R\) be the function
  \[
    f_L(x) = \begin{dcases}
      f(0) - \varepsilon & \text{if } x \in [-\delta, \delta]                   \\
      -M                 & \text{if } x \in [-1, 1] \setminus [-\delta, \delta]
    \end{dcases}
  \]
  Clearly \(f_U, f_L\) are piecewise constant on \([-1, 1]\), \(f_U\) majorizes \(f\) and \(f_L\) minorizes \(f\).
  Then we have
  \begin{align*}
    \overline{\int}_{[-1, 1]} f \; d \, \text{sgn} & \leq p.c. \int f_U \; d \, \text{sgn}                                   &  & \by{ac:11.8.8} \\
                                                   & = M \big(\text{sgn}(-\delta) - \text{sgn}(-1)\big)                      &  & \by{11.8.5}    \\
                                                   & \quad + (f(0) + \varepsilon) (\text{sgn}(\delta) - \text{sgn}(-\delta))                     \\
                                                   & \quad + M \big(\text{sgn}(1) - \text{sgn}(\delta)\big)                                      \\
                                                   & = 2\big(f(0) + \varepsilon\big)
  \end{align*}
  and
  \begin{align*}
    \underline{\int}_{[-1, 1]} f \; d \, \text{sgn} & \geq p.c. \int f_L \; d \, \text{sgn}                                   &  & \by{ac:11.8.8} \\
                                                    & = M \big(\text{sgn}(-\delta) - \text{sgn}(-1)\big)                      &  & \by{11.8.5}    \\
                                                    & \quad + (f(0) - \varepsilon) (\text{sgn}(\delta) - \text{sgn}(-\delta))                     \\
                                                    & \quad + M \big(\text{sgn}(1) - \text{sgn}(\delta)\big)                                      \\
                                                    & = 2\big(f(0) - \varepsilon\big).
  \end{align*}
  Combining results above we have
  \[
    2(f(0) - \varepsilon) \leq \underline{\int}_{[-1, 1]} f \; d \text{sgn} = \int_{[-1, 1]} f \; d \text{sgn} = \overline{\int}_{[-1, 1]} f \; d \text{sgn} \leq 2(f(0) + \varepsilon).
  \]
  Since \(\varepsilon\) is arbitrary, we thus have \(\int_{[-1, 1]} f \; d \text{sgn} = 2f(0)\).
\end{proof}

\section{The two fundamental theorems of calculus}\label{i:sec:11.9}

\begin{thm}[First Fundamental Theorem of Calculus]\label{i:11.9.1}
  Let \(a < b\) be real numbers, and let \(f : [a, b] \to \R\) be a Riemann integrable function.
  Let \(F : [a, b] \to \R\) be the function
  \[
    F(x) \coloneqq \int_{[a, x]} f.
  \]
  Then \(F\) is continuous.
  Furthermore, if \(x_0 \in [a, b]\) and \(f\) is continuous at \(x_0\), then \(F\) is differentiable at \(x_0\), and \(F'(x_0) = f(x_0)\).
\end{thm}

\begin{proof}
  Since \(f\) is Riemann integrable, it is bounded (by \cref{i:11.3.4}).
  Thus we have some real number \(M\) such that \(-M \leq f(x) \leq M\) for all \(x \in [a, b]\).

  Now let \(x < y\) be two elements of \([a, b]\).
  Then notice that
  \[
    F(y) - F(x) = \int_{[a, y]} f - \int_{[a, x]} f = \int_{[x, y]} f
  \]
  by \cref{i:11.4.1}(h).
  By \cref{i:11.4.1}(e) we thus have
  \[
    \int_{[x, y]} f \leq \int_{[x, y]} M = p.c. \int_{[x, y]} M = M(y - x)
  \]
  and
  \[
    \int_{[x, y]} f \geq \int_{[x, y]} -M = p.c. \int_{[x, y]} -M = -M(y - x)
  \]
  and thus
  \[
    \abs{F(y) - F(x)} \leq M(y - x).
  \]
  This is for \(y > x\).
  By interchanging \(x\) and \(y\) we thus see that
  \[
    \abs{F(y) - F(x)} \leq M(x - y)
  \]
  when \(x > y\).
  Also, we have \(F(y) - F(x) = 0\) when \(x = y\).
  Thus in all
  three cases we have
  \[
    \abs{F(y) - F(x)} \leq M \abs{x - y}.
  \]
  Now let \(z \in [a, b]\), and let \((z_n)_{n = 0}^\infty\) be any sequence in \([a, b]\) converging to \(z\).
  Then we have
  \[
    -M \abs{z_n - z} \leq F(z_n) - F(z) \leq M \abs{z_n - z}
  \]
  for each \(n\).
  But \(-M \abs{z_n - z}\) and \(M \abs{z_n - z}\) both converge to \(0\) as \(n \to \infty\), so by the squeeze test \(F(z_n) - F(z)\) converges to \(0\) as \(n \to \infty\), and thus \(\lim_{n \to \infty} F(z_n) = F(z)\).
  Since this is true for all sequences \(z_n \in [a, b]\) converging to \(z\), we thus see that \(F\) is continuous at \(z\) (by \cref{i:9.4.7}).
  Since \(z\) was an arbitrary element of \([a, b]\), we thus see that \(F\) is continuous
  (The above proof also show that when \(F\) is Lipschitz continuous, \(F\) is also continuous, see \cref{i:ex:10.2.6}).

  Now suppose that \(x_0 \in [a, b]\), and \(f\) is continuous at \(x_0\).
  Choose any \(\varepsilon > 0\).
  Then by continuity, we can find a \(\delta > 0\) such that \(\abs{f(x) - f(x_0)} \leq \varepsilon\) for all \(x\) in the interval \(I \coloneqq [x_0 - \delta, x_0 + \delta] \cap [a, b]\), or in other words
  \[
    f(x_0) - \varepsilon \leq f(x) \leq f(x_0) + \varepsilon \text{ for all } x \in I.
  \]
  We now show that
  \[
    \abs{F(y) - F(x_0) - f(x_0)(y - x_0)} \leq \varepsilon \abs{y - x_0}
  \]
  for all \(y \in I\), since \cref{i:10.1.7} will then imply that \(F\) is differentiable at \(x_0\) with derivative \(F'(x_0) = f(x_0)\) as desired.

  Now fix \(y \in I\).
  There are three cases.
  If \(y = x_0\), then \(F(y) - F(x_0) - f(x_0)(y - x_0) = 0\) and so the claim is obvious.
  If \(y > x_0\), then
  \[
    F(y) - F(x_0) = \int_{[x_0, y]} f.
  \]
  Since \(x_0\), \(y \in I\), and \(I\) is a connected set (by \cref{i:11.1.6}), then \([x_0, y]\) is a subset of \(I\), and thus we have
  \[
    f(x_0) - \varepsilon \leq f(x) \leq f(x_0) + \varepsilon \text{ for all } x \in [x_0, y],
  \]
  and thus by \cref{i:11.4.1}(e)
  \[
    \big(f(x_0) - \varepsilon\big) (y - x_0) \leq \int_{[x_0, y]} f \leq \big(f(x_0) + \varepsilon\big) (y - x_0)
  \]
  and so in particular
  \[
    \abs{F(y) - F(x_0) - f(x_0)(y - x_0)} \leq \varepsilon \abs{y - x_0}
  \]
  as desired.
  If \(y < x_0\), then
  \[
    F(y) - F(x_0) = -\big(F(x_0) - F(y)\big) = -\int_{[y, x_0]} f.
  \]
  Since \(x_0\), \(y \in I\), and \(I\) is a connected set (by \cref{i:11.1.6}), then \([y, x_0]\) is a subset of \(I\), and thus we have
  \[
    f(x_0) - \varepsilon \leq f(x) \leq f(x_0) + \varepsilon \text{ for all } x \in [y, x_0],
  \]
  and thus by \cref{i:11.4.1}(e)
  \begin{align*}
             & \big(f(x_0) - \varepsilon\big) (x_0 - y) \leq \int_{[y, x_0]} f \leq \big(f(x_0) + \varepsilon\big) (x_0 - y)                  \\
    \implies & \big(f(x_0) - \varepsilon\big) (y - x_0) \geq -\int_{[y, x_0]} f = F(y) - F(x_0) \geq \big(f(x_0) + \varepsilon\big) (y - x_0) \\
    \implies & -\varepsilon (y - x_0) \geq F(y) - F(x_0) - f(x_0)(y - x_0) \geq \varepsilon (y - x_0)                                         \\
    \implies & \abs{F(y) - F(x_0) - f(x_0)(y - x_0)} \leq \varepsilon \abs{y - x_0}
  \end{align*}
  as desired.
\end{proof}

\begin{note}
  Informally, the first fundamental theorem of calculus asserts that
  \[
    \bigg(\int_{[a, x]} f\bigg)'(x) = f(x)
  \]
  given a certain number of assumptions on \(f\).
  Roughly, this means that the derivative of an integral recovers the original function.
\end{note}

\setcounter{thm}{2}
\begin{defn}[Antiderivatives]\label{i:11.9.3}
  Let \(I\) be a bounded interval, and let \(f : I \to \R\) be a function.
  We say that a function \(F : I \to \R\) is an \emph{antiderivative} of \(f\) if \(F\) is differentiable on \(I\) and \(F'(x) = f(x)\) for all limit points \(x\) of \(I\).
\end{defn}

\begin{thm}ond Fundamental Theorem of Calculus]\label{i:11.9.4}
  Let \(a < b\) be real numbers, and let \(f : [a, b] \to \R\) be a Riemann integrable function.
  If \(F : [a, b] \to \R\) is an antiderivative of \(f\), then
  \[
    \int_{[a, b]} f = F(b) - F(a).
  \]
\end{thm}

\begin{proof}
  The claim is trivial when \(b = a\), so assume \(b > a\), so in particular all points of \([a, b]\) are limit points.
  We will use Riemann sums.
  The idea is to show that
  \[
    U(f, \mathbf{P}) \geq F(b) - F(a) \geq L(f, \mathbf{P})
  \]
  for every partition \(\mathbf{P}\) of \([a, b]\).
  The left inequality asserts that \(F(b) - F(a)\) is a lower bound for \(\set{U(f, \mathbf{P}) : \mathbf{P} \text{ is a partition of } [a, b]}\), while the right inequality asserts that \(F(b) - F(a)\) is an upper bound for \(\set{L(f, \mathbf{P}) : \mathbf{P} \text{ is a partition of } [a, b]}\).
  But by \cref{i:11.3.12}, this means that
  \[
    \overline{\int}_{[a, b]} f \geq F(b) - F(a) \geq \underline{\int}_{[a, b]} f,
  \]
  but since \(f\) is assumed to be Riemann integrable, both the upper and lower Riemann integral equal \(\int_{[a, b]} f\).
  The claim follows.

  We have to show the bound \(U(f, \mathbf{P}) \geq F(b) - F(a) \geq L(f, \mathbf{P})\).
  We shall just show the first inequality \(U(f, \mathbf{P}) \geq F(b) - F(a)\);
  the other inequality is similar.

  Let \(\mathbf{P}\) be a partition of \([a, b]\).
  From \cref{i:11.8.4} we have
  \[
    F(b) - F(a) = \sum_{J \in \mathbf{P}} F[J] = \sum_{J \in \mathbf{P} : J \neq \emptyset} F[J],
  \]
  while from definition we have
  \[
    U(f, \mathbf{P}) = \sum_{J \in \mathbf{P} : J \neq \emptyset} \sup_{x \in J} f(x) \abs{J}.
  \]
  Thus it will suffice to show that
  \[
    F[J] \leq \sup_{x \in J} f(x) \abs{J}
  \]
  for all \(J \in \mathbf{P}\)
  (other than the empty set).

  When \(J\) is a point then the claim is clear, since both sides are zero.
  Now suppose that \(J = [c, d], (c, d], [c, d)\), or \((c, d)\) for some \(c < d\).
  Then the left-hand side is \(F[J] = F(d) - F(c)\).
  Note that \(F\), being differentiable, is continuous, so we may use the simplified formula for the \(F\)-length as opposed to the more complicated one in \cref{i:11.8.1}.
  By the mean-value theorem (\cref{i:10.2.9}), this is equal to \((d - c) F'(e)\) for some \(e \in J\).
  But since \(F'(e) = f(e)\), we thus have
  \[
    F[J] = (d - c) f(e) = f(e) \abs{J} \leq \sup_{x \in J} f(x) \abs{J}
  \]
  as desired.
\end{proof}

\begin{note}
  One can use the second fundamental theorem of calculus to compute integrals relatively easily provided that you can find an anti-derivative of the integrand \(f\).
  The first fundamental theorem of calculus ensures that every \emph{continuous} Riemann integrable function has an anti-derivative.
  For discontinuous functions, the situation is more complicated.
  Also, not every function with an anti-derivative is Riemann integrable.
\end{note}

\begin{lem}\label{i:11.9.5}
  Let \(I\) be a bounded interval, and let \(f : I \to \R\) be a function.
  Let \(F : I \to \R\) and \(G : I \to \R\) be two antiderivatives of \(f\).
  Then there exists a real number \(C\) such that \(F(x) = G(x) + C\) for all \(x \in I\).
\end{lem}

\begin{proof}
  If \(I = \emptyset\), then the claim is trivially true.
  If \(I = \set{a}\) for some \(a \in \R\), then we can simply set \(C = F(a) - G(a)\).
  So suppose that \(I\) is one of \((a, b), [a, b), (a, b], [a, b]\) for some \(a, b \in \R\) and \(a < b\).
  Since \(I\) is a bounded interval, for all \(x \in I\) we know that \(x\) is a limit point.
  Since \(F, G\) are antiderivatives of \(f\), by \cref{i:11.9.3} we know that \(F, G\) are differentiable on \(I\).
  By \cref{i:10.1.13}(f) we know that \(F - G\) is differentiable on \(I\), and thus by \cref{i:10.1.12} we know that \(F - G\) are continuous on \(I\).
  Let \(x, y \in I\) and \(x < y\).
  Since \(I\) is a bounded interval, by \cref{i:11.1.10} we know that \(I\) is connected, and thus by \cref{i:11.1.1} \([x, y] \subseteq I\).
  By the mean-value theorem (\cref{i:10.2.9}) we know that
  \[
    \exists c \in I : \dfrac{(F - G)(x) - (F - G)(y)}{x - y} = (F - G)'(c).
  \]
  Thus we have
  \begin{align*}
             & \dfrac{(F - G)(x) - (F - G)(y)}{x - y} = (F - G)'(c)                               \\
    \implies & \dfrac{(F - G)(x) - (F - G)(y)}{x - y} = F'(c) - G'(c) &  & \by{i:10.1.13}[f]      \\
    \implies & \dfrac{(F - G)(x) - (F - G)(y)}{x - y} = 0             &  & \text{(by hypothesis)} \\
    \implies & (F - G)(x) = (F - G)(y)                                                            \\
    \implies & F(x) - G(x) = F(y) - G(y)                              &  & \by{i:9.2.1}           \\
    \implies & F(x) = G(x) + F(y) - G(y).
  \end{align*}
  By setting \(C = F(y) - G(y)\) we are done.
\end{proof}

\exercisesection

\begin{ex}\label{i:ex:11.9.1}
  Let \(f : [0, 1] \to \R\) be the function in \cref{i:ex:9.8.5}.
  Show that for every rational number \(q \in \Q \cap (0, 1)\), the function \(F : [0, 1] \to \R\) defined by the formula \(F(x) \coloneqq \int_{[0, x]} f\) is not differentiable at \(q\).
\end{ex}

\begin{proof}
  By \cref{i:ex:9.8.5} we know that \(f\) is strictly monotone increasing, thus by \cref{i:11.6.1} we know that \(f\) is Riemann integrable and \(F\) is well-defined.
  By \cref{i:ex:9.8.5} \(f\) is not continuous at \(q\), thus by \cref{i:ex:11.9.3} we know that \(F\) is not differentiable at \(q\).
\end{proof}

\begin{ex}\label{i:ex:11.9.2}
  Prove \cref{i:11.9.5}.
\end{ex}

\begin{proof}
  See \cref{i:11.9.5}.
\end{proof}

\begin{ex}\label{i:ex:11.9.3}
  Let \(a < b\) be real numbers, and let \(f : [a, b] \to \R\) be a monotone increasing function.
  Let \(F : [a, b] \to \R\) be the function \(F(x) \coloneqq \int_{[a, x]} f\).
  Let \(x_0\) be an element of \((a, b)\).
  Show that \(F\) is differentiable at \(x_0\) iff \(f\) is continuous at \(x_0\).
\end{ex}

\begin{proof}
  Since \(f\) is monotone increasing, by \cref{i:11.6.1} we know that \(f\) is Riemann integrable and \(F\) is well-defined.
  If \(f\) is continuous at \(x_0\), then by \cref{i:11.9.1} we know that \(F\) is differentiable at \(x_0\).
  So we only need to show that if \(F\) is differentiable at \(x_0\), then \(f\) is continuous at \(x_0\).

  Suppose that \(F\) is differentiable at \(x_0\).
  Suppose for sake of contradiction that \(f\) is not continuous at \(x_0\).
  Since \(f\) is monotone increasing, by \cref{i:ac:11.8.1} we know that both \(f(x_0+)\) and \(f(x_0-)\) exist and \(f(x_0-) \leq f(x_0+)\).
  Since \(f\) is not continuous at \(x_0\), by \cref{i:9.5.3} we have \(f(x_0-) < f(x_0+)\).

  If \(x \in [a, b] \cap (-\infty, x_0)\), then we have \(f(x) \leq f(x_0-)\) and
  \[
    \dfrac{F(x) - F(x_0)}{x - x_0} = \dfrac{-\int_{[x, x_0]} f}{x - x_0} = \dfrac{\int_{[x, x_0]} f}{x_0 - x} \leq \dfrac{p.c. \int_{[x, x_0]} f(x_0-)}{x_0 - x} = f(x_0-).
  \]
  If \(x \in [a, b] \cap (x_0, \infty)\), then we have \(f(x) \geq f(x_0+)\) and
  \[
    \dfrac{F(x) - F(x_0)}{x - x_0} = \dfrac{\int_{[x_0, x]} f}{x - x_0} \geq \dfrac{p.c. \int_{[x_0, x]} f(x_0+)}{x - x_0} = f(x_0+).
  \]
  Thus by \cref{i:9.3.14} we have
  \[
    F'(x_0-) \leq f(x_0-) < f(x_0+) \leq F'(x_0+).
  \]
  But by \cref{i:9.3.6} \(F'(x_0-) \neq F'(x_0+)\) implies \(F'(x_0)\) does not exist, a contradiction.
  So we conclude that \(f\) is continuous at \(x_0\).
\end{proof}

\section{Consequences of the fundamental theorems}\label{sec:11.10}

\begin{prop}[Integration by parts formula]\label{11.10.1}
  Let \(I = [a, b]\), and let \(F : [a, b] \to \R\) and \(G : [a, b] \to \R\) be differentiable functions on \([a, b]\) such that \(F'\) and \(G'\) are Riemann integrable on \(I\).
  Then we have
  \[
    \int_{[a, b]} F G' = F(b) G(b) - F(a) G(a) - \int_{[a, b]} F' G.
  \]
\end{prop}

\begin{proof}
  Since \(F\) is an antiderivative of \(F'\) and \(F'\) is Riemann integrable on \([a, b]\), by \cref{11.9.1} we know that \(F\) is continuous on \([a, b]\).
  Similarly \(G\) is continuous on \([a, b]\).
  By \cref{11.5.2} we know that \(F\) and \(G\) are Riemann integrable on \([a, b]\).
  By \cref{11.4.5} we know that \(F G'\) and \(F' G\) are Riemann integrable on \([a, b]\).
  By \cref{10.1.13}(d) we have \((FG)' = F' G + F G'\).
  Thus by \cref{11.4.1}(a) \((FG)'\) is Riemann integrable on \([a, b]\) and
  \begin{align*}
    \int_{[a, b]} (F G') & = \int_{[a, b]} \big((FG)' - F' G\big)                                                   \\
                         & = \int_{[a, b]} \big((FG)'\big) - \int_{[a, b]} (F' G) &  & \text{(by \cref{11.4.1}(c))} \\
                         & = F(b) G(b) - F(a) G(a) - \int_{[a, b]} (F' G).        &  & \by{11.9.4}
  \end{align*}
\end{proof}

\begin{thm}\label{11.10.2}
  Let \(\alpha : [a, b] \to \R\) be a monotone increasing function, and suppose that \(\alpha\) is also differentiable on \([a, b]\), with \(\alpha'\) being Riemann integrable.
  Let \(f : [a, b] \to \R\) be a piecewise constant function on \([a, b]\).
  Then \(f \alpha'\) is Riemann integrable on \([a, b]\), and
  \[
    \int_{[a, b]} f \; d \alpha = \int_{[a, b]} f \alpha'.
  \]
\end{thm}

\begin{proof}
  Since \(f\) is piecewise constant, it is Riemann integrable, and since \(\alpha'\) is also Riemann integrable, then \(f \alpha'\) is Riemann integrable by \cref{11.4.5}.

  Suppose that \(f\) is piecewise constant with respect to some partition \(\mathbf{P}\) of \([a, b]\);
  without loss of generality we may assume that \(\mathbf{P}\) does not contain the empty set.
  Then we have
  \[
    \int_{[a, b]} f \; d \alpha = p.c. \int_{[\mathbf{P}]} f \; d \alpha = \sum_{J \in \mathbf{P}} c_J \alpha[J]
  \]
  where \(c_J\) is the constant value of \(f\) on \(J\).
  On the other hand, from \cref{11.4.1}(h) (and \cref{ex:11.4.3}) we have
  \[
    \int_{[a, b]} f \alpha' = \sum_{J \in \mathbf{P}} \int_J f \alpha' = \sum_{J \in \mathbf{P}} \int_J c_J \alpha' = \sum_{J \in \mathbf{P}} c_J \int_J \alpha'.
  \]
  But by the second fundamental theorem of calculus (\cref{11.9.4}), \(\int_J \alpha' = \alpha[J]\), and the claim follows.
\end{proof}

\begin{cor}\label{11.10.3}
  Let \(\alpha : [a, b] \to \R\) be a monotone increasing function, and suppose that \(\alpha\) is also differentiable on \([a, b]\), with \(\alpha'\) being Riemann integrable.
  Let \(f : [a, b] \to \R\) be a function which is Riemann-Stieltjes integrable with respect to \(\alpha\) on \([a, b]\).
  Then \(f \alpha'\) is Riemann integrable on \([a, b]\), and
  \[
    \int_{[a, b]} f \; d \alpha = \int_{[a, b]} f \alpha'.
  \]
\end{cor}

\begin{proof}
  Note that since \(f\) and \(\alpha'\) are bounded, then \(f \alpha'\) must also be bounded.
  Also, since \(\alpha\) is monotone increasing and differentable, \(\alpha'\) is non-negative (by \cref{10.3.1}).

  Let \(\varepsilon > 0\).
  Then, we can find a piecewise constant function \(\overline{f}\) majorizing \(f\) on \([a, b]\), and a piecewise constant function \(\underline{f}\) minorizing \(f\) on \([a, b]\), such that
  \[
    \int_{[a, b]} f \; d \alpha - \varepsilon \leq \int_{[a, b]} \underline{f} \; d \alpha \leq \int_{[a, b]} \overline{f} \; d \alpha \leq \int_{[a, b]} f \; d \alpha + \varepsilon.
  \]
  Applying \cref{11.10.2}, we obtain
  \[
    \int_{[a, b]} f \; d \alpha - \varepsilon \leq \int_{[a, b]} \underline{f} \alpha' \leq \int_{[a, b]} \overline{f} \alpha' \leq \int_{[a, b]} f \; d \alpha + \varepsilon.
  \]

  Since \(\alpha'\) is non-negative and \(\underline{f}\) minorizes \(f\), then \(\underline{f} \alpha'\) minorizes \(f \alpha'\).
  Thus \(\int_{[a, b]} \underline{f} \alpha' \leq \underline{\int}_{[a, b]} f \alpha'\).
  Thus
  \[
    \int_{[a, b]} f \; d \alpha - \varepsilon \leq \underline{\int}_{[a, b]} f \alpha'.
  \]
  Similarly we have
  \[
    \overline{\int}_{[a, b]} f \alpha' \leq \int_{[a, b]} f \; d \alpha + \varepsilon.
  \]
  Since these statements are true for any \(\varepsilon > 0\), we must have
  \[
    \int_{[a, b]} f \; d \alpha \leq \underline{\int}_{[a, b]} f \alpha' \leq \overline{\int}_{[a, b]} f \alpha' \leq \int_{[a, b]} f \; d \alpha
  \]
  and the claim follows.
\end{proof}

\begin{rmk}\label{11.10.4}
  Informally, \cref{11.10.3} asserts that \(f \; d \alpha\) is essentially equivalent to \(f \dfrac{d \alpha}{dx} dx\), when \(\alpha\) is differentiable.
  However, the advantage of the Riemann-Stieltjes integral is that it still makes sense even when \(\alpha\) is not differentiable.
\end{rmk}

\begin{lem}[Change of variables formula I]\label{11.10.5}
  Let \([a, b]\) be a closed interval, and let \(\phi : [a, b] \to [\phi(a), \phi(b)]\) be a continuous monotone increasing function.
  Let \(f : [\phi(a), \phi(b)] \to \R\) be a piecewise constant function on \([\phi(a), \phi(b)]\).
  Then \(f \circ \phi : [a, b] \to \R\) is also piecewise constant on \([a, b]\), and
  \[
    \int_{[a, b]} f \circ \phi \; d \phi = \int_{[\phi(a), \phi(b)]} f.
  \]
\end{lem}

\begin{proof}
  Let \(\mathbf{P}\) be a partition of \([\phi(a), \phi(b)]\) such that \(f\) is piecewise constant with respect to \(\mathbf{P}\);
  we may assume that \(\mathbf{P}\) does not contain the empty set.
  For each \(J \in \mathbf{P}\), let \(c_J\) be the constant value of \(f\) on \(J\), thus
  \[
    \int_{[\phi(a), \phi(b)]} f = \sum_{J \in \mathbf{P}} c_J \abs{J}.
  \]
  For each interval \(J\), let \(\phi^{-1}(J)\) be the set \(\phi^{-1}(J) \coloneqq \set{x \in [a, b] : \phi(x) \in J}\).
  Then \(\phi^{-1}(J)\) is connected (by \cref{9.8.3} and \cref{11.1.4}), and is thus an interval.
  Furthermore, \(c_J\) is the constant value of \(f \circ \phi\) on \(\phi^{-1}(J)\) (since \((f \circ \phi)\big(\phi^{-1}(J)\big) = f(J)\)).
  Thus, if we define \(\mathbf{S} \coloneqq \set{\phi^{-1} (J) : J \in \mathbf{P}}\), then \(\mathbf{S}\) partitions \([a, b]\)
  (\(\mathbf{S}\) is finite since \(\mathbf{P}\) is finite;
  \(\phi^{-1}(J)\) is an interval and \(\phi\) is a bijection from \([a, b]\) to \([\phi(a), \phi(b)]\)),
  and \(f \circ \phi\) is piecewise constant with respect to \(\mathbf{S}\) (for every \(\phi^{-1}(J) \in \mathbf{S}\), \(f\) is constant on \(\phi^{-1}(J)\)).
  Thus
  \[
    \int_{[a, b]} f \circ \phi \; d \phi = p.c. \int_{[\mathbf{S}]} f \circ \phi \; d \phi = \sum_{J \in \mathbf{P}} c_J \phi[\phi^{-1}(J)].
  \]
  But \(\phi[\phi^{-1}(J)] = \abs{J}\) (since \(\phi(\phi^{-1}(J)) = J\) and \(\phi\) is continuous), and the claim follows.
\end{proof}

\begin{prop}[Change of variables formula II]\label{11.10.6}
  Let \([a, b]\) be a closed interval, and let \(\phi : [a, b] \to [\phi(a), \phi(b)]\) be a continuous monotone increasing function.
  Let \(f : [\phi(a), \phi(b)] \to \R\) be a Riemann integrable function on \([\phi(a), \phi(b)]\).
  Then \(f \circ \phi : [a, b] \to \R\) is Riemann-Stieltjes integrable with respect to \(\phi\) on \([a, b]\), and
  \[
    \int_{[a, b]} f \circ \phi \; d \phi = \int_{[\phi(a), \phi(b)]} f.
  \]
\end{prop}

\begin{proof}
  This will be obtained from \cref{11.10.5} in a similar manner to how \cref{11.10.3} was obtained from \cref{11.10.2}.
  First observe that since \(f\) is Riemann integrable, it is bounded, and then \(f \circ \phi\) must also be bounded (by \cref{9.8.3} \(\phi\) is a bijection).

  Let \(\varepsilon > 0\).
  Then, we can find a piecewise constant function \(\overline{f}\) majorizing \(f\) on \([\phi(a), \phi(b)]\), and a piecewise constant function \(\underline{f}\) minorizing \(f\) on \([\phi(a), \phi(b)]\), such that
  \[
    \int_{[\phi(a), \phi(b)]} f - \varepsilon \leq \int_{[\phi(a), \phi(b)]} \underline{f} \leq \int_{[\phi(a), \phi(b)]} \overline{f} \leq \int_{[\phi(a), \phi(b)]} f + \varepsilon.
  \]
  Applying \cref{11.10.5}, we obtain
  \[
    \int_{[\phi(a), \phi(b)]} f - \varepsilon \leq \int_{[a, b]} \underline{f} \circ \phi \; d \phi \leq \int_{[a, b]} \overline{f} \circ \; d \phi \leq \int_{[\phi(a), \phi(b)]} f + \varepsilon.
  \]
  Since \(\underline{f} \circ \phi\) is piecewise constant and minorizes \(f \circ \phi\), we have
  \[
    \int_{[a, b]} \underline{f} \circ \phi \; d \phi \leq \underline{\int}_{[a, b]} f \circ \phi \; d \phi
  \]
  while similarly we have
  \[
    \int_{[a, b]} \overline{f} \circ \phi \; d \phi \geq \overline{\int}_{[a, b]} f \circ \phi \; d \phi.
  \]
  Thus
  \[
    \int_{[\phi(a), \phi(b)]} f - \varepsilon \leq \underline{\int}_{[a, b]} f \circ \phi \; d \phi \leq \overline{\int}_{[a, b]} f \circ \; d \phi \leq \int_{[\phi(a), \phi(b)]} f + \varepsilon.
  \]
  Since \(\varepsilon > 0\) was arbitrary, this implies that
  \[
    \int_{[\phi(a), \phi(b)]} f \leq \underline{\int}_{[a, b]} f \circ \phi \; d \phi \leq \overline{\int}_{[a, b]} f \circ \; d \phi \leq \int_{[\phi(a), \phi(b)]} f
  \]
  and the claim follows.
\end{proof}

\begin{prop}[Change of variables formula III]\label{11.10.7}
  Let \([a, b]\) be a closed interval, and let \(\phi : [a, b] \to [\phi(a), \phi(b)]\) be a differentiable monotone increasing function such that \(\phi'\) is Riemann integrable.
  Let \(f : [\phi(a), \phi(b)] \to \R\) be a Riemann integrable function on \([\phi(a), \phi(b)]\).
  Then \((f \circ \phi) \phi' : [a, b] \to \R\) is Riemann integrable on \([a, b]\), and
  \[
    \int_{[a, b]} (f \circ \phi) \phi' = \int_{[\phi(a), \phi(b)]} f.
  \]
\end{prop}

\begin{proof}
  Since \(\phi\) is differentable on \([a, b]\), by \cref{10.1.12} we know that \(\phi\) is continuous on \([a, b]\).
  By \cref{11.10.6} we know that \(f \circ \phi\) is Riemann-Stieltjes integrable with respect to \(\phi\) on \([a, b]\).
  By \cref{11.10.3} we know that \((f \circ \phi) \phi'\) is Riemann integrable on \([a, b]\), and
  \begin{align*}
    \int_{[a, b]} (f \circ \phi) \phi' & = \int_{[a, b]} (f \circ \phi) \; d \phi &  & \by{11.10.3} \\
                                       & = \int_{[\phi(a), \phi(b)]} f.           &  & \by{11.10.6}
  \end{align*}
\end{proof}

\begin{ac}[Change of variables formula IV]\label{ac:11.10.1}
  Let \([a, b]\) be a closed interval, and let \(\phi : [a, b] \to [\phi(a), \phi(b)]\) be a differentiable monotone increasing function such that \(\phi'\) is Riemann integrable.
  Let \(f : [\phi(a), \phi(b)] \to \R\) be a continuous function on \([\phi(a), \phi(b)]\).
  Then \((f \circ \phi) \phi' : [a, b] \to \R\) is Riemann integrable on \([a, b]\), and
  \[
    \int_{[a, b]} (f \circ \phi) \phi' = \int_{[\phi(a), \phi(b)]} f.
  \]
\end{ac}

\begin{proof}
  Since \(\phi\) is differentiable on \([a, b]\), by \cref{10.1.12} we know that \(\phi\) is continuous on \([a, b]\).
  Since \(f\) is continuous on \([\phi(a), \phi(b)]\) and \(\phi\) is continuous on \([a, b]\), by \cref{9.4.13} we know that \(f \circ \phi\) is continuous on \([a, b]\).
  By \cref{11.5.2} we know that \(f \circ \phi\) is Riemann integrable on \([a, b]\).
  Since \(\phi'\) is Riemann integrable on \([a, b]\), by \cref{11.4.5} we know that \((f \circ \phi) \phi'\) is Riemann integrable on \([a, b]\).
  Thus
  \[
    \int_{[a, b]} (f \circ \phi) \phi'
  \]
  is well-defined.
  Since \(f\) is continuous on \([\phi(a), \phi(b)]\), by \cref{11.5.2} we know that \(f\) is Riemann integrable on \([\phi(a), \phi(b)]\).
  Thus
  \[
    \int_{[\phi(a), \phi(b)]} f
  \]
  is well-defined.
  Let \(F : [\phi(a), \phi(b)] \to \R\) be the function
  \[
    F(x) = \int_{[\phi(a), x]} f
  \]
  Since \(f\) is continuous on \([\phi(a), \phi(b)]\), by \cref{11.9.1} we know that \(F'(x) = f(x)\) for each \(x \in [\phi(a), \phi(b)]\).
  Then by \cref{10.1.15} we have
  \[
    \forall x \in [a, b], (F \circ \phi)'(x) = F'\big(\phi(x)\big) \phi'(x) = f\big(\phi(x)\big) \phi'(x) = (f \circ \phi)(x) \cdot \phi(x)
  \]
  and \((F \circ \phi)' = (f \circ \phi) \phi'\).
  Thus
  \begin{align*}
    \int_{[a, b]} (f \circ \phi) \phi' & = \int_{[a, b]} (F \circ \phi)'                                              \\
                                       & = (F \circ \phi)(b) - (F \circ \phi)(a)                     &  & \by{11.9.4} \\
                                       & = F\big(\phi(b)\big) - F\big(\phi(a)\big)                                    \\
                                       & = \int_{[\phi(a), \phi(b)]} f - \int_{[\phi(a), \phi(a)]} f                  \\
                                       & = \int_{[\phi(a), \phi(b)]} f.
  \end{align*}
\end{proof}

\exercisesection

\begin{ex}\label{ex:11.10.1}
  Prove \cref{11.10.1}.
\end{ex}

\begin{proof}
  See \cref{11.10.1}.
\end{proof}

\begin{ex}\label{ex:11.10.2}
  Fill in the gaps marked (why?) in the proof of \cref{11.10.5}.
\end{ex}

\begin{proof}
  See \cref{11.10.5}.
\end{proof}

\begin{ex}\label{ex:11.10.3}
  Let \(a < b\) be real numbers, and let \(f : [a, b] \to \R\) be a Riemann integrable function.
  Let \(g : [-b, -a] \to \R\) be defined by \(g(x) \coloneqq f(-x)\).
  Show that \(g\) is also Riemann integrable, and \(\int_{[-b, -a]} g = \int_{[a, b]} f\).
\end{ex}

\begin{proof}
  Let \(\varepsilon > 0\).
  Then, we can find a piecewise constant function \(\overline{f}\) majorizing \(f\) on \([a, b]\), and a piecewise constant function \(\underline{f}\) minorizing \(f\) on \([a, b]\), such that
  \[
    \int_{[a, b]} f - \varepsilon \leq \int_{[a, b]} \underline{f} = \int_{[a, b]} \overline{f} \leq \int_{[a, b]} f + \varepsilon.
  \]
  Let \(\overline{g} : [-b, -a] \to \R\) be the function \(\overline{g}(x) = \overline{f}(-x)\).
  Since \(\overline{f}\) majorizes \(f\) on \([a, b]\), we know that \(\overline{g}\) majorizes \(g\) on \([-b, -a]\) and
  \[
    \int_{[-b, -a]} \overline{g} = \int_{[a, b]} \overline{f} \leq \int_{[a, b]} f + \varepsilon.
  \]
  Let \(\underline{g} : [-b, -a] \to \R\) be the function \(\underline{g}(x) = \underline{f}(-x)\).
  Since \(\underline{f}\) minorizes \(f\) on \([a, b]\), we know that \(\underline{g}\) minorizes \(g\) on \([-b, -a]\) and
  \[
    \int_{[-b, -a]} \underline{g} = \int_{[a, b]} \underline{f} \geq \int_{[a, b]} f - \varepsilon.
  \]
  By \cref{11.3.2} and \cref{11.3.3} we have
  \[
    \int_{[a, b]} f - \varepsilon \leq \int_{[-b, -a]} \underline{g} \leq \underline{\int}_{[-b, -a]} g \leq \overline{\int}_{[-b, -a]} g \leq \int_{[-b, -a]} \overline{g} \leq \int_{[a, b]} f + \varepsilon.
  \]
  Since these statements are true for any \(\varepsilon > 0\), we must have
  \[
    \int_{[a, b]} f \leq \underline{\int}_{[-b, -a]} g \leq \overline{\int}_{[-b, -a]} g \leq \int_{[a, b]} f
  \]
  and the claim follows.
\end{proof}

\begin{ex}\label{ex:11.10.4}
  What is the analogue of \cref{11.10.7} when \(\phi\) is monotone decreasing instead of monotone increasing?
  (When \(\phi\) is neither monotone increasing or monotone decreasing, the situation becomes significantly more complicated.)
\end{ex}

\begin{proof}
  Let \([a, b]\) be a closed interval, and let \(\phi : [a, b] \to \big[\phi(b), \phi(a)\big]\) be a differentiable monotone decreasing function such that \(\phi'\) is Riemann integrable.
  Let \(f : \big[\phi(b), \phi(a)\big] \to \R\) be a Riemann integrable function on \(\big[\phi(b), \phi(a)\big]\).
  Then \((f \circ \phi) \phi' : [a, b] \to \R\) is Riemann integrable on \([a, b]\), and
  \[
    \int_{[a, b]} (f \circ \phi) \phi' = -\int_{[\phi(b), \phi(a)]} f.
  \]

  Now we proof the statement.
  Let \(\eta : [-b, -a] \to [a, b]\) be the function \(\eta = x \mapsto -x\).
  Let \(\gamma : [-b, -a] \to [\phi(b), \phi(a)]\) be the function
  \[
    \forall x \in [-b, -a], \gamma(x) = \phi(-x) = (\phi \circ \eta)(x).
  \]
  Since \(\eta\) is differentiable on \([-b, -a]\), by chain rule (\cref{10.1.15}) we know that
  \[
    \forall x \in [-b, -a], \gamma'(x) = (\phi \circ \eta)'(x) = \phi'\big(\eta(x)\big) \eta'(x) = \phi'\big(\eta(x)\big) (-1) = - (\phi' \circ \eta)(x).
  \]
  Observe that
  \begin{align*}
             & \forall x, y \in [-b, -a], x \leq y                                               \\
    \implies & -x \geq -y                                                                        \\
    \implies & \phi(-x) \leq \phi(-y)              &  & \text{(\(\phi\) is monotone decreasing)} \\
    \implies & \gamma(x) \leq \gamma(y).
  \end{align*}
  Thus \(\gamma\) is monotone increasing and by \cref{11.10.7} we have
  \[
    \int_{[-b, -a]} (f \circ \gamma) \gamma' = \int_{[\gamma(-b), \gamma(-a)]} f.
  \]
  Since
  \begin{align*}
     & \int_{[-b, -a]} (f \circ \gamma) \gamma'                                                                           \\
     & = \int_{[-b, -a]} (f \circ \phi \circ \eta) \cdot (\phi \circ \eta)'                                               \\
     & = \int_{[-b, -a]} (f \circ \phi \circ \eta) \cdot \big(-(\phi' \circ \eta)\big) &  & \text{(from the proof above)} \\
     & = -\int_{[-b, -a]} (f \circ \phi \circ \eta) \cdot (\phi' \circ \eta)           &  & \text{(by \cref{11.4.1}(b))}
  \end{align*}
  and
  \[
    \int_{[\gamma(-b), \gamma(-a)]} f = \int_{\big[\phi(b), \phi(a)\big]} f,
  \]
  we know that
  \[
    \int_{[-b, -a]} (f \circ \phi \circ \eta) \cdot (\phi' \circ \eta) = -\int_{\big[\phi(b), \phi(a)\big]} f.
  \]
  Since
  \begin{align*}
    \forall x \in [a, b], & \big((f \circ \phi) \cdot \phi'\big)(x)                                              \\
                          & = (f \circ \phi)(x) \cdot \phi'(x)                                  &  & \by{9.2.1}  \\
                          & = (f \circ \phi \circ \eta)(-x) \cdot (\phi' \circ \eta)(-x)        &  & \by{3.3.10} \\
                          & = \big((f \circ \phi \circ \eta) \cdot (\phi' \circ \eta)\big)(-x), &  & \by{9.2.1}
  \end{align*}
  By \cref{ex:11.10.3} we know that
  \[
    \int_{[a, b]} (f \circ \phi) \phi' = \int_{[-b, -a]} (f \circ \phi \circ \eta) \cdot (\phi' \circ \eta).
  \]
\end{proof}


%------------------------------------------------------------------------------
% Back matters.
%------------------------------------------------------------------------------

\backmatter

\end{document}
