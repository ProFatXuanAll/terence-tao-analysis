\section{The construction of the real numbers}

\begin{definition}[Real numbers]\label{5.3.1}
A \emph{real number} is defined to be an object of the form \(\text{LIM}_{n \to \infty} a_n\), where \((a_n)_{n = 1}^{\infty}\) is a Cauchy sequence of rational numbers.
Two real numbers \(\text{LIM}_{n \to \infty} a_n\) an and \(\text{LIM}_{n \to \infty} b_n\) are said to be equal iff \((a_n)_{n = 1}^{\infty}\) and \((b_n)_{n = 1}^{\infty}\) are equivalent Cauchy sequences.
The set of all real numbers is denoted \(\mathds{R}\).
\end{definition}

\begin{note}
We will refer to \(\text{LIM}_{n \to \infty} a_n\) as the \emph{formal limit} of the sequence \((a_n)_{n = 1}^{\infty}\).
Later on we will define a genuine notion of limit, and show that the formal limit of a Cauchy sequence is the same as the limit of that sequence;
after that, we will not need formal limits ever again.
\end{note}

\setcounter{theorem}{2}
\begin{proposition}[Formal limits are well-defined]\label{5.3.3}
Let \(x = \text{LIM}_{n \to \infty} a_n\), \(y = \text{LIM}_{n \to \infty} b_n\), and \(z = \text{LIM}_{n \to \infty} c_n\) be real numbers.
Then, with the above definition of equality for real numbers, we have \(x = x\).
Also, if \(x = y\), then \(y = x\).
Finally, if \(x = y\) and \(y = z\), then \(x = z\).
\end{proposition}

\begin{proof}
We first prove the reflexivity.
Because \(x = \text{LIM}_{n \to \infty} a_n\), by Definition \ref{5.3.1}, \((a_n)_{n = 1}^{\infty}\) is a Cauchy sequence of rational numbers.
By Definition \ref{5.1.8}, \(\forall\ \varepsilon > 0\) and \(\varepsilon \in \mathds{Q}\), \(\exists\ N \geq 1\) and \(N \in \mathds{N}\) such that
\[
    |a_j - a_k| \leq \varepsilon \ \forall\ j, k \geq N,
\]
where \(j, k \in \mathds{N}\).
In particular, we have
\[
    |a_j - a_j| \leq \varepsilon \ \forall\ j \geq N.
\]
By Definition \ref{5.2.6}, \((a_n)_{n = 1}^{\infty}\) and \((a_n)_{n = 1}^{\infty}\) are equivalent sequences.
Since \((a_n)_{n = 1}^{\infty}\) is a Cauchy sequence, by Definition \ref{5.3.1}, \(x = \text{LIM}_{n \to \infty} a_n = \text{LIM}_{n \to \infty} a_n = x\).

Next we prove the symmetry.
By Definition \ref{5.3.1}, \(x = y\) implies \((a_n)_{n = 1}^{\infty}\) and \((b_n)_{n = 1}^{\infty}\) are equivalent Cauchy sequences.
Then by Definition \ref{5.2.6}, \(\forall\ \varepsilon > 0\) and \(\varepsilon \in \mathds{Q}\), \(\exists\ N \geq 1\) and \(N \in \mathds{N}\) such that \(a_n\) is \(\varepsilon\)-close to \(b_n\) for all \(n \geq N\).
But by Proposition \ref{4.3.7}, \(a_n\) is \(\varepsilon\)-close to \(b_n\) implies that \(b_n\) is \(\varepsilon\)-close to \(a_n\).
So we have \(b_n\) is \(\varepsilon\)-close to \(a_n\), \(\forall\ \varepsilon > 0\) and \(\forall\ n \geq N\), which means \((b_n)_{n = 1}^{\infty}\) and \((a_n)_{n = 1}^{\infty}\) are equivalent sequences by Definition \ref{5.2.6}.
Since \((a_n)_{n = 1}^{\infty}\) and \((b_n)_{n = 1}^{\infty}\) are Cauchy sequences, by Definition \ref{5.3.1}, \(y = \text{LIM}_{n \to \infty} b_n = \text{LIM}_{n \to \infty} a_n = x\).

Finally we prove the transitivity.
By Definition \ref{5.3.1}, \(x = y\) implies that \((a_n)_{n = 1}^{\infty}\) and \((b_n)_{n = 1}^{\infty}\) are equivalent Cauchy sequences.
Then by Definition \ref{5.2.6}, \(\forall\ \varepsilon > 0\) and \(\varepsilon \in \mathds{Q}\), \(\exists\ N_1 \geq 1\) and \(N_1 \in \mathds{N}\) such that \(a_n\) is \(\varepsilon\)-close to \(b_n\) for all \(n \geq N_1\).
Since \(\varepsilon > 0\), by Additional Corollary \ref{ac 4.2.5}, \(\varepsilon / 2 > 0\), so \(a_n\) is also \((\varepsilon / 2)\)-close to \(b_n\).
Similarly, \(y = z\) implies that \(\forall\ \varepsilon > 0\) and \(\varepsilon \in \mathds{Q}\), \(\exists\ N_2 \geq 1\) and \(N_2 \in \mathds{N}\) such that \(b_n\) is \((\varepsilon / 2)\)-close to \(c_n\) for all \(n \geq N_2\).
Let \(N = N_1 + N_2\).
Then by Additional Corollary \ref{ac 2.2.1}, \(N \in \mathds{N}\).
And by Definition \ref{2.2.11}, \(N \geq N_1\) and \(N \geq N_2\).
So we have \(a_n\) is \((\varepsilon / 2)\)-close to \(b_n\) and \(b_n\) is \((\varepsilon / 2)\)-close to \(c_n\), \(\forall\ \varepsilon > 0\) and \(\forall\ n \geq N\).
Then by Proposition \ref{4.3.7}, \(a_n\) is \(\varepsilon\)-close to \(c_n\), \(\forall\ \varepsilon > 0\) and \(\forall\ n \geq N\).
Thus by Definition \ref{5.2.6}, \((a_n)_{n = 1}^{\infty}\) and \((c_n)_{n = 1}^{\infty}\) are equivalent sequences.
Since \((a_n)_{n = 1}^{\infty}\) and \((c_n)_{n = 1}^{\infty}\) are Cauchy sequences, by Definition \ref{5.3.1}, \(x = \text{LIM}_{n \to \infty} a_n = \text{LIM}_{n \to \infty} c_n = z\).
\end{proof}

\exercisesection

\begin{exercise}\label{ex 5.3.1}
Prove Proposition \ref{5.3.3}.
\end{exercise}

\begin{proof}
See Proposition \ref{5.3.3}.
\end{proof}