\section{Ordering the reals}

\begin{definition}\label{5.4.1}
Let \((a_n)_{n = 1}^{\infty}\) be a sequence of rationals.
We say that this sequence is \emph{positively bounded away from zero} iff we have a positive rational \(c > 0\) such that \(a_n \geq c\) for all \(n \geq 1\) (in particular, the sequence is entirely positive).
The sequence is \emph{negatively bounded away from zero} iff we have a negative rational \(-c < 0\) such that \(a_n \leq -c\) for all \(n \geq 1\) (in particular, the sequence is entirely negative).
\end{definition}

\begin{note}
It is clear that any sequence which is positively or negatively bounded away from zero, is bounded away from zero.
Also, a sequence cannot be both positively bounded away from zero and negatively bounded away from zero at the same time.
\end{note}

\setcounter{theorem}{2}
\begin{definition}\label{5.4.3}
A real number \(x\) is said to be \emph{positive} iff it can be written as \(x = \text{LIM}_{n \to \infty} a_n\) for some Cauchy sequence \((a_n)_{n = 1}^{\infty}\) which is positively bounded away from zero.
\(x\) is said to be \emph{negative} iff it can be written as \(x = \text{LIM}_{n \to \infty} a_n\) for some sequence \((a_n)_{n = 1}^{\infty}\) which is negatively bounded away from zero.
\end{definition}

\begin{proposition}[Basic properties of positive reals]\label{5.4.4}
For every real number \(x\), exactly one of the following three statements is true:
\begin{enumerate*}
    \item \(x\) is zero;
    \item \(x\) is positive;
    \item \(x\) is negative.
\end{enumerate*}
A real number \(x\) is negative if and only if \(-x\) is positive.
If \(x\) and \(y\) are positive, then so are \(x + y\) and \(xy\).
\end{proposition}

\begin{proof}
We first show that at least one of the three statements is true.
Let \(x\) be the formal limit of some sequence \((a_n)_{n = 1}^{\infty}\), \(\varepsilon, c \in \mathds{Q}\) and \(N, j, k \in \mathds{N}\).
If \((a_n)_{n = 1}^{\infty}\) is eventually \(\varepsilon\)-close to \(0\) for all \(\varepsilon > 0\), then \(x = 0\) by Definition \ref{5.3.1}.
If \((a_n)_{n = 1}^{\infty}\) is not eventually \(\varepsilon\)-close to \(0\) for all \(\varepsilon > 0\), then \(x \neq 0\) by Definition \ref{5.3.1}.
By Lemma \ref{5.3.14}, \(x \neq 0\) implies \(\exists\ c > 0\) such that \(\abs*{a_n} \geq c \ \forall\ n \geq 1\).
Since \(c > 0\), we have \(a_n \neq 0\), otherwised \(0 \geq c\), a contradiction.
If \(a_n > 0\), then \(a_n \geq c > 0\).
If \(a_n < 0\), then \(-a_n \geq c > 0\), which implies \(a_n \leq -c < 0\) by Exercise \ref{ex 4.2.6}.
Since \((a_n)_{n = 1}^{\infty}\) is a Cauchy sequence, \(\forall\ \varepsilon > 0\), \(\exists\ N \geq 1\) such that \(\abs*{a_j - a_k} \leq \varepsilon \ \forall\ j, k \geq N\).
In particular, \(\abs*{a_j - a_k} \leq c\).
So
\begin{align*}
& \abs*{a_j - a_k} \leq c \\
\implies & -c \leq a_j - a_k \leq c & \text{(by Proposition \ref{4.3.3})} \\
\implies & -c + a_k \leq a_j \leq c + a_k. & \text{(by Proposition \ref{4.2.9})} \\
\end{align*}
\begin{enumerate}[label=(\roman*)]
    \item If \(a_k > 0\), then we have \(0 \leq a_j \leq 2c\).
    Since \(a_j \neq 0\), we have \(a_j > 0\) for all \(j \geq N\).
    Then we can define a sequence \((b_n)_{n = 1}^{\infty}\) by setting \(b_n = a_n\) if \(a_n > 0\) and \(b_n = -a_n\) if \(a_n < 0\) for all \(n \geq 1\).
    Because \(\forall\ \varepsilon > 0\), \(\abs*{b_n - a_n} = 0 \leq \varepsilon \ \forall\ n \geq N\), so \((a_n)_{n = 1}^{\infty}\) and \((b_n)_{n = 1}^{\infty}\) are equivalent Cauchy sequence.
    And because \((b_n)_{n = 1}^{\infty}\) is positively bounded away from zero, \(x\) is positive.
    \item If \(a_k < 0\), then we have \(-2c \leq a_j \leq 0\).
    Since \(a_j \neq 0\), we have \(a_j < 0\) for all \(j \geq N\).
    Then we can define a sequence \((b_n)_{n = 1}^{\infty}\) by setting \(b_n = a_n\) if \(a_n < 0\) and \(b_n = -a_n\) if \(a_n > 0\) for all \(n \geq 1\).
    Because \(\forall\ \varepsilon > 0\), \(\abs*{b_n - a_n} = 0 \leq \varepsilon \ \forall\ n \geq N\), so \((a_n)_{n = 1}^{\infty}\) and \((b_n)_{n = 1}^{\infty}\) are equivalent Cauchy sequence.
    And because \((b_n)_{n = 1}^{\infty}\) is negatively bounded away from zero, \(x\) is negative.
\end{enumerate}
Since \(x\) can be either \(0\), positive or negative, we conclude that at least one of the three statements is true.

Now we show that at most one of the three statements is true.
Let \(x\) be the formal limit of some sequence \((a_n)_{n = 1}^{\infty}\) and \(\varepsilon, c \in \mathds{Q}\).
If \(x = 0\), then \(\exists\ N \geq 1\) such that \(\abs*{a_n - 0} = \abs*{a_n} \leq \varepsilon \ \forall\ n \geq N\).
If \(x\) is positive, then \(\exists\ c_1 > 0\) such that \(a_n > c_1 \ \forall\ n \geq 1\).
If \(x\) is negative, then \(\exists\ c_2 > 0\) such that \(a_n < -c_2 \ \forall\ n \geq 1\).
Suppose for sake of contradiction that \(x\) is both \(0\) and positive.
Then we have \(\abs*{a_n} \leq c_1\) and \(\abs*{a_n} > c_1\) for all \(n \geq N\) at the same time, which is a contradiction by Proposition \ref{4.2.9}.
Similarly suppose for sake of contradiction that \(x\) is both \(0\) and negative.
Then we also have \(\abs*{a_n} \leq c_2\) and \(\abs*{a_n} > c_2\) for all \(n \geq N\) at the same time, which is a contradiction by Proposition \ref{4.2.9}.
Now suppose for sake of contradiction that \(x\) is both positive and negative.
Then we have \(a_n \geq c_1 > 0\) and \(a_n \leq -c_2 < 0\) for all \(n \geq 1\) at the same time, which is a contradiction by Proposition \ref{4.2.9}.
So at most one of the three statements is true.
Combine with the previous proof, we conclude that exactly one of the three statements is true.

Now we show that \(x\) is negative iff \(-x\) is positive.
Let \(x\) be the formal limit of some sequence \((a_n)_{n = 1}^{\infty}\).
Then by Definition \ref{5.4.3}, \(\exists\ c > 0\) such that \(a_n \leq -c < 0 \ \forall\ n \geq 1\).
So \(\forall\ n \geq 1\),
\begin{align*}
& x \text{ is negative} \\
\iff & (x = \text{LIM}_{n \to \infty} a_n) \land (a_n \leq -c < 0) & \text{(by Definition \ref{5.4.3})} \\
\iff & (-x = \text{LIM}_{n \to \infty} -a_n) \land (a_n \leq -c < 0) & \text{(by Definition \ref{5.3.9})} \\
\iff & (-x = \text{LIM}_{n \to \infty} -a_n) \land (-a_n \geq c > 0) & \text{(by Exercise \ref{ex 4.2.6})} \\
\iff & -x \text{ is positive}.
\end{align*}
Thus \(x\) is negative iff \(-x\) is positive.

Now we show that \(x, y\) are positive implies \(x + y\) is also positive.
Let \(x = (a_n)_{n = 1}^{\infty}\) and \(y = (b_n)_{n = 1}^{\infty}\).
Then by Definition \ref{5.4.3}, \(\exists\ c_1 > 0\) such that \(a_n \geq c_1 > 0 \ \forall\ n \geq 1\) and \(\exists\ c_2 > 0\) such that \(b_n \geq c_2 > 0 \ \forall\ n \geq 1\).
But by Definition \ref{5.3.4}, \(x + y = \text{LIM}_{n \to \infty} a_n + b_n\), and by Proposition \ref{4.2.9}, \(a_n + b_n \geq c_1 + c_2 > 0 \ \forall\ n \geq 1\).
So \(x + y\) is positive by Definition \ref{5.4.3}.

Finally we show that \(x, y\) are positive implies \(xy\) is also positive.
Let \(x = (a_n)_{n = 1}^{\infty}\) and \(y = (b_n)_{n = 1}^{\infty}\).
Then by Definition \ref{5.4.3}, \(\exists\ c_1 > 0\) such that \(a_n \geq c_1 > 0 \ \forall\ n \geq 1\) and \(\exists\ c_2 > 0\) such that \(b_n \geq c_2 > 0 \ \forall\ n \geq 1\).
But by Definition \ref{5.3.9}, \(xy = \text{LIM}_{n \to \infty} a_n b_n\), and by Proposition \ref{4.2.9}, \(a_n b_n \geq c_1 c_2 > 0 \ \forall\ n \geq 1\).
So \(xy\) is positive by Definition \ref{5.4.3}.
\end{proof}

\exercisesection

\begin{exercise}\label{ex 5.4.1}
Prove Proposition \ref{5.4.4}.
\end{exercise}

\begin{proof}
See Proposition \ref{5.4.4}.
\end{proof}