\section{Left and right limits}\label{sec 9.5}

\begin{definition}[Left and right limits]\label{9.5.1}
    Let \(X\) be a subset of \(\mathbf{R}\), \(f : X \to \mathbf{R}\) be a function, and let \(x_0\) be a real number.
    If \(x_0\) is an adherent point of \(X \cap (x_0, \infty)\), then we define the \emph{right limit} \(f(x_0+)\) of \(f\) at \(x_0\) by the formula
    \[
        f(x_0+) \coloneqq \lim_{x \to x_0 ; x \in X \cap (x_0, \infty)} f(x),
    \]
    provided of course that this limit exists.
    Similarly, if \(x_0\) is an adherent point of \(X \cap (-\infty, x_0)\), then we define the \emph{left limit} \(f(x_0-)\) of \(f\) at \(x_0\) by the formula
    \[
        f(x_0-) \coloneqq \lim_{x \to x_0 ; x \in X \cap (-\infty, x_0)} f(x),
    \]
    again provided that the limit exists.
    (Thus in many cases \(f(x_0+)\) and \(f(x_0-)\) will not be defined.)
    Sometimes we use the shorthand notations
    \begin{align*}
        \lim_{x \to x_0+} f(x) & \coloneqq \lim_{x \to x_0 ; x \in X \cap (x_0, \infty)} f(x); \\
        \lim_{x \to x_0-} f(x) & \coloneqq \lim_{x \to x_0 ; x \in X \cap (-\infty, x_0)} f(x)
    \end{align*}
    when the domain \(X\) of \(f\) is clear from context.
\end{definition}

\begin{note}
    From Proposition \ref{9.4.7} we see that if the right limit \(f(x_0+)\) exists, and \((a_n)_{n = 0}^\infty\) is a sequence in \(X\) converging to \(x_0\) from the right (i.e., \(a_n > x_0\) for all \(n \in \mathbf{N}\)), then \(\lim_{n \to \infty} f(a_n) = f(x_0+)\).
    Similarly, if \((b_n)_{n = 0}^\infty\) is a sequence converging to \(x_0\) from the left (i.e., \(a_n < x_0\) for all \(n \in \mathbf{N}\)) then \(\lim_{n \to \infty} f(a_n) = f(x_0-)\).
\end{note}

\begin{additional corollary}\label{ac 9.5.1}
    Let \(x_0\) be an adherent point of both \(X \cap (x_0, \infty)\) and \(X \cap (-\infty, x_0)\).
    If \(f\) is continuous at \(x_0\), then \(f(x_0+)\) and \(f(x_0-)\) both exists and are equal to \(f(x_0)\).
\end{additional corollary}

\begin{proof}
    Since \(f\) is continuous at \(x_0\), by Definition \ref{9.4.1} we know that
    \[
        \forall\ \varepsilon \in \mathbf{R}^+, \exists\ \delta \in \mathbf{R}^+ : \bigg(\forall\ x \in X, \abs*{x - x_0} < \delta \implies \abs*{f(x) - f(x_0)} < \varepsilon\bigg).
    \]
    Since \(X \cap (-\infty, x_0) \subseteq X\) and \(X \cap (x_0, \infty) \subseteq X\), we know that the following two statements are true:
    \begin{align*}
        &\forall\ \varepsilon \in \mathbf{R}^+, \exists\ \delta \in \mathbf{R}^+ : \\
        &\bigg(\forall\ x \in X \cap (-\infty, x_0), \abs*{x - x_0} < \delta \implies \abs*{f(x) - f(x_0)} < \varepsilon\bigg) \\
        \implies & \lim_{x \to x_0 ; x \in X \cap (-\infty, x_0)} f(x) = f(x_0). \\
        &\forall\ \varepsilon \in \mathbf{R}^+, \exists\ \delta \in \mathbf{R}^+ : \\
        &\bigg(\forall\ x \in X \cap (x_0, \infty), \abs*{x - x_0} < \delta \implies \abs*{f(x) - f(x_0)} < \varepsilon\bigg) \\
        \implies & \lim_{x \to x_0 ; x \in X \cap (x_0, \infty)} f(x) = f(x_0).
    \end{align*}
    Thus by Definition \ref{9.5.1} we have \(f(x_0+) = f(x_0-) = f(x_0)\).
\end{proof}