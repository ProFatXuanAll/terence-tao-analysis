\section{Left and right limits}\label{sec 9.5}

\begin{definition}[Left and right limits]\label{9.5.1}
    Let \(X\) be a subset of \(\mathbf{R}\), \(f : X \to \mathbf{R}\) be a function, and let \(x_0\) be a real number.
    If \(x_0\) is an adherent point of \(X \cap (x_0, \infty)\), then we define the \emph{right limit} \(f(x_0+)\) of \(f\) at \(x_0\) by the formula
    \[
        f(x_0+) \coloneqq \lim_{x \to x_0 ; x \in X \cap (x_0, \infty)} f(x),
    \]
    provided of course that this limit exists.
    Similarly, if \(x_0\) is an adherent point of \(X \cap (-\infty, x_0)\), then we define the \emph{left limit} \(f(x_0-)\) of \(f\) at \(x_0\) by the formula
    \[
        f(x_0-) \coloneqq \lim_{x \to x_0 ; x \in X \cap (-\infty, x_0)} f(x),
    \]
    again provided that the limit exists.
    (Thus in many cases \(f(x_0+)\) and \(f(x_0-)\) will not be defined.)
    Sometimes we use the shorthand notations
    \begin{align*}
        \lim_{x \to x_0+} f(x) & \coloneqq \lim_{x \to x_0 ; x \in X \cap (x_0, \infty)} f(x); \\
        \lim_{x \to x_0-} f(x) & \coloneqq \lim_{x \to x_0 ; x \in X \cap (-\infty, x_0)} f(x)
    \end{align*}
    when the domain \(X\) of \(f\) is clear from context.
\end{definition}