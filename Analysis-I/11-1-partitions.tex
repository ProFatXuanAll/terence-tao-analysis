\section{Partitions}\label{sec 11.1}

\begin{definition}\label{11.1.1}
    Let \(X\) be a subset of \(\mathbf{R}\).
    We say that \(X\) is \emph{connected} iff the following property is true:
    whenever \(x, y\) are elements in \(X\) such that \(x < y\), the bounded interval \([x, y]\) is a subset of \(X\)
    (i.e., every number between \(x\) and \(y\) is also in \(X\)).
\end{definition}

\setcounter{theorem}{3}
\begin{lemma}\label{11.1.4}
    Let \(X\) be a subset of the real line.
    Then the following two statements are logically equivalent:
    \begin{enumerate}
        \item \(X\) is bounded and connected.
        \item \(X\) is a bounded interval.
    \end{enumerate}
\end{lemma}

\begin{proof}
    Both statements are logically equivalent when \(X = \emptyset\) (which is vacuously true).
    So suppose that \(X \neq \emptyset\).

    We first show that \(X\) is bounded and connected implies \(X\) is a bounded interval.
    Since \(X\) is bounded, by Theorem \ref{5.5.9} we know that \(\inf(X), \sup(X) \in \mathbf{R}\).
    Thus \(X \subseteq [\inf(X), \sup(X)]\).
    Let \(x \in X\).
    Then we must have \(\exists\ a, b \in X\) such that
    \[
        \inf(X) < a \leq x \leq b < \sup(X),
    \]
    otherwise contradict to the existence of \(\inf(X)\) and \(\sup(X)\).
    Since \(X\) is connected, by Definition \ref{11.1.1} we know that \([a, b] \subseteq X\).
    Since \(\inf(X) < a\) and \(b < \sup(X)\), we have \((\inf(X), \sup(X)) \subseteq [a, b] \subseteq X\).
    Thus \(X\) is a bounded interval.

    Now we show that \(X\) is a bounded interval implies \(X\) is bounded and connected.
    Obviously \(X\) is bounded.
    Let \(a, b \in \mathbf{R}\).
    Then \(X\) can be one of \((a, b), [a, b], (a, b], [a, b)\), and by Definition \ref{11.1.1} all of which are connected.
\end{proof}

\begin{remark}\label{11.1.5}
    Recall that intervals are allowed to be singleton points, or even the empty set.
\end{remark}