\section{The Riemann-Stieltjes integral}\label{i:sec:11.8}

\begin{ac}\label{i:ac:11.8.1}
  Let \(I\) be a bounded interval, and let \(f : X \to \R\) be a monotone increasing function defined on some interval \(X\) which contains \(I\).
  Then we have
  \[
    f(x_0+) = \lim_{x \to x_0^+ ; x \in X} f(x) = \inf_{x \in X \cap (x_0, +\infty)} f(x)
  \]
  and
  \[
    f(x_0-) = \lim_{x \to x_0^- ; x \in X} f(x) = \sup_{x \in X \cap (-\infty, x_0)} f(x)
  \]
  for every \(x_0 \in I\) and \(x_0\) is not an endpoint of \(X\).
\end{ac}

\begin{proof}
  If \(I = \emptyset\), then the statement are vacuously true.
  So suppose that \(I \neq \emptyset\).
  Define
  \begin{align*}
    U & = \inf_{x \in X \cap (x_0, +\infty)} f(x); \\
    L & = \sup_{x \in X \cap (-\infty, x_0)} f(x).
  \end{align*}
  Since \(x_0 \in X\) and \(x_0\) is not an endpoint of \(X\), we know that \(X \cap (x_0, +\infty) \neq \emptyset\) and \(X \cap (-\infty, x_0) \neq \emptyset\).
  Since \(f\) is monotone increasing, we have
  \[
    U = \inf_{x \in X \cap (x_0, +\infty)} f(x) \geq f(x_0)
  \]
  and
  \[
    L = \sup_{x \in X \cap (-\infty, x_0)} f(x) \leq f(x_0).
  \]
  Thus \(U, L \in \R\).

  First we show that \(f(x_0+) = U\).
  By the definition of \(U\) we know that
  \[
    \forall \varepsilon \in \R^+, \exists x \in X \cap (x_0, +\infty) : 0 \leq f(x) - U \leq \varepsilon.
  \]
  Now fix one pair of \(\varepsilon\) and \(x\).
  Since \(f\) is monotone increasing, we know that
  \[
    \forall y \in X \cap (x_0, +\infty), y < x \implies 0 \leq f(y) - U \leq f(x) - U \leq \varepsilon.
  \]
  Thus by setting \(\delta = x - x_0\) we have
  \[
    \forall y \in X \cap (x_0, +\infty), \abs{y - x_0} < \delta \implies \abs{f(y) - U} \leq \varepsilon.
  \]
  Since \(\varepsilon\) was arbitrary, by \cref{i:9.3.6} and \cref{i:9.5.1} we have \(f(x_0+) = U\).

  Now we show that \(f(x_0-) = L\).
  By the definition of \(L\) we know that
  \[
    \forall \varepsilon \in \R^+, \exists x \in X \cap (-\infty, x_0) : 0 \leq L - f(x) \leq \varepsilon.
  \]
  Now fix one pair of \(\varepsilon\) and \(x\).
  Since \(f\) is monotone increasing, we know that
  \[
    \forall y \in X \cap (-\infty, x_0), y > x \implies 0 \leq L - f(y) \leq L - f(x) \leq \varepsilon.
  \]
  Thus by setting \(\delta = x_0 - x\) we have
  \[
    \forall y \in X \cap (-\infty, x_0), \abs{y - x_0} < \delta \implies \abs{f(y) - L} \leq \varepsilon.
  \]
  Since \(\varepsilon\) was arbitrary, by \cref{i:9.3.6} and \cref{i:9.5.1} we have \(f(x_0-) = L\).
\end{proof}

\begin{ac}\label{i:ac:11.8.2}
  Let \(I\) be a bounded interval, and let \(f : X \to \R\) be a monotone decreasing function defined on some interval \(X\) which contains \(I\).
  Then we have
  \[
    f(x_0+) = \lim_{x \to x_0^+ ; x \in X} f(x) = \sup_{x \in X \cap (x_0, \infty)} f(x)
  \]
  and
  \[
    f(x_0-) = \lim_{x \to x_0^- ; x \in X} f(x) = \inf_{x \in X \cap (-\infty, x_0)} f(x)
  \]
  for every \(x_0 \in I\) and \(x_0\) is not an endpoint of \(X\).
\end{ac}

\begin{proof}
  If \(I = \emptyset\), then the statement are vacuously true.
  So suppose that \(I \neq \emptyset\).
  Define
  \begin{align*}
    U & = \sup_{x \in X \cap (x_0, +\infty)} f(x); \\
    L & = \inf_{x \in X \cap (-\infty, x_0)} f(x).
  \end{align*}
  Since \(x_0 \in X\) and \(x_0\) is not an endpoint of \(X\), we know that \(X \cap (x_0, +\infty) \neq \emptyset\) and \(X \cap (-\infty, x_0) \neq \emptyset\).
  Since \(f\) is monotone decreasing, we have
  \[
    U = \sup_{x \in X \cap (x_0, +\infty)} f(x) \leq f(x_0)
  \]
  and
  \[
    L = \inf_{x \in X \cap (-\infty, x_0)} f(x) \geq f(x_0).
  \]
  Thus \(U, L \in \R\).

  First we show that \(f(x_0+) = U\).
  By the definition of \(U\) we know that
  \[
    \forall \varepsilon \in \R^+, \exists x \in X \cap (x_0, +\infty) : 0 \leq U - f(x) \leq \varepsilon.
  \]
  Now fix one pair of \(\varepsilon\) and \(x\).
  Since \(f\) is monotone decreasing, we know that
  \[
    \forall y \in X \cap (x_0, +\infty), y < x \implies 0 \leq U - f(y) \leq U - f(x) \leq \varepsilon.
  \]
  Thus by setting \(\delta = x - x_0\) we have
  \[
    \forall y \in X \cap (x_0, +\infty), \abs{y - x_0} < \delta \implies \abs{f(y) - U} \leq \varepsilon.
  \]
  Since \(\varepsilon\) was arbitrary, by \cref{i:9.3.6} and \cref{i:9.5.1} we have \(f(x_0+) = U\).

  Now we show that \(f(x_0-) = L\).
  By the definition of \(L\) we know that
  \[
    \forall \varepsilon \in \R^+, \exists x \in X \cap (-\infty, x_0) : 0 \leq f(x) - L \leq \varepsilon.
  \]
  Now fix one pair of \(\varepsilon\) and \(x\).
  Since \(f\) is monotone decreasing, we know that
  \[
    \forall y \in X \cap (-\infty, x_0), y > x \implies 0 \leq f(y) - L \leq f(x) - L \leq \varepsilon.
  \]
  Thus by setting \(\delta = x_0 - x\) we have
  \[
    \forall y \in X \cap (-\infty, x_0), \abs{y - x_0} < \delta \implies \abs{f(y) - L} \leq \varepsilon.
  \]
  Since \(\varepsilon\) was arbitrary, by \cref{i:9.3.6} and \cref{i:9.5.1} we have \(f(x_0-) = L\).
\end{proof}

\begin{defn}[\(\alpha\)-length]\label{i:11.8.1}
  Let \(I\) be a bounded interval, and let \(\alpha : X \to \R\) be a monotone increasing function defined on some interval \(X\) which contains \(I\).
  Then we define the \emph{\(\alpha\)-length} \(\alpha[I]\) of \(I\) as follows.
  \begin{itemize}
    \item If \(I\) is the empty set, we set
          \[
            \alpha[\emptyset] \coloneqq 0.
          \]
    \item If \(I\) is a point of the form \(\set{a}\) for some real number \(a\), we set
          \[
            \alpha\big[\set{a}\big] \coloneqq \lim_{x \to a^+ ; x \in X} \alpha(x) - \lim_{x \to a^- ; x \in X} \alpha(x),
          \]
          with the convention that \(\lim_{x \to a^+ ; x \in X} \alpha(x)\) (resp. \(\lim_{x \to a^- ; x \in X} \alpha(x)\)) is \(\alpha(a)\) when \(a\) is the right (resp. left) endpoint of \(X\).
    \item If \(I\) is an interval of the form \((a, b)\) for some real numbers \(b > a\), set
          \[
            \alpha\big[(a, b)\big] \coloneqq \lim_{x \to b^- ; x \in X} \alpha(x) - \lim_{x \to a^+ ; x \in X} \alpha(x).
          \]
    \item If \(I\) is an interval of the form \([a, b)\), \((a, b]\), or \([a, b]\) for some real numbers \(b > a\), then we set
          \[
            \alpha[I] = \begin{dcases}
              \alpha\big[\set{a}\big] + \alpha\big[(a, b)\big]                           & \text{if } I = [a, b) \\
              \alpha\big[(a, b)\big] + \alpha\big[\set{b}\big]                           & \text{if } I = (a, b] \\
              \alpha\big[\set{a}\big] + \alpha\big[(a, b)\big] + \alpha\big[\set{b}\big] & \text{if } I = [a, b]
            \end{dcases}
          \]
  \end{itemize}
\end{defn}

\begin{note}
  In the special case when \(\alpha\) is continuous, the definition of \(\alpha[I]\) where \(I\) is of the form \((a, b)\), \([a, b)\), \((a, b]\), or \([a, b]\) simplifies to \(\alpha[I] = \alpha(b) - \alpha(a)\).
\end{note}

\begin{note}
  We sometimes write \(\alpha\big|_a^b\) or \(\alpha(x)\big|_{x = a}^{x = b}\) instead of \(\alpha\big[[a, b]\big]\).
\end{note}

\begin{note}
  \cref{i:11.8.1} is well-defined, thanks to \cref{i:ac:11.8.1}.
  \cref{i:11.8.1} is can also be applied when \(\alpha\) is monotone decreasing, thanks to \cref{i:ac:11.8.2}.
\end{note}

\setcounter{thm}{3}
\begin{lem}\label{i:11.8.4}
  Let \(I\) be a bounded interval, let \(\alpha : X \to \R\) be a monotone increasing function defined on some interval \(X\) which contains \(I\), and let \(\mathbf{P}\) be a partition of \(I\).
  Then we have
  \[
    \alpha[I] = \sum_{J \in \mathbf{P}} \alpha[J].
  \]
\end{lem}

\begin{proof}
  We prove this by induction on \(n\).
  More precisely, we let \(P(n)\) be the property that whenever \(I\) is a bounded interval, and whenever \(\mathbf{P}\) is a partition of \(I\) with cardinality \(n\), that \(\alpha[I] = \sum_{J \in \mathbf{P}} \alpha[J]\).

  The base case \(P(0)\) is trivial;
  the only way that \(I\) can be partitioned into an empty partition is if \(I\) is itself empty, so by \cref{i:11.8.1} \(\alpha[I] = 0\).
  The case \(P(1)\) is also very easy;
  the only way that \(I\) can be partitioned into a singleton set \(\set{J}\) is if \(J = I\), at which point the claim is again very easy.

  Now suppose inductively that \(P(n)\) is true for some \(n \geq 1\), and now we prove \(P(n + 1)\).
  Let \(I\) be a bounded interval, and let \(\mathbf{P}\) be a partition of \(I\) of cardinality \(n + 1\).

  If \(I\) is the empty set or a point, then all the intervals in \(\mathbf{P}\) must also be either the empty set or a point, and by \cref{i:11.8.1} every interval either has \(\alpha\)-length zero or
  \[
    \alpha[\set{a}] = \lim_{x \to a^+ ; x \in X} \alpha(x) - \lim_{x \to a^- ; x \in X} \alpha(x),
  \]
  and the claim is trivial.
  Thus we will assume that \(I\) is an interval of the form \((a, b)\), \((a, b]\), \([a, b)\), or \([a, b]\).

      Let us first suppose that \(b \in I\), i.e., \(I\) is either \((a, b]\) or \([a, b]\).
  Since \(b \in I\), we know that one of the intervals \(K\) in \(\mathbf{P}\) contains \(b\).
  Since \(K\) is contained in \(I\), it must therefore be of the form \((c, b]\), \([c, b]\), or \(\set{b}\) for some real number \(c\), with \(a \leq c \leq b\) (in the latter case of \(K = \set{b}\), we set \(c \coloneqq b\)).
  In particular, this means that the set \(I \setminus K\) is also an interval of the form \([a, c], (a, c), (a, c], [a, c)\) when \(c > a\), or a point or empty set when \(a = c\).
  Either way, by \cref{i:11.8.1} we see that
  \begin{align*}
    \alpha\big[(a, b]\big] & = \alpha\big[(a, b)\big] + \alpha\big[\set{b}\big]                                                            \\
                           & = \lim_{x \to b^- ; x \in X} \alpha(x) - \lim_{x \to a^+ ; x \in X} \alpha(x) + \alpha\big[\set{b}\big]       \\
                           & = \lim_{x \to b^- ; x \in X} \alpha(x) - \lim_{x \to c^+ ; x \in X} \alpha(x)                                 \\
                           & \quad + \lim_{x \to c^+ ; x \in X} \alpha(x) - \lim_{x \to c^- ; x \in X} \alpha(x)                           \\
                           & \quad + \lim_{x \to c^- ; x \in X} \alpha(x) - \lim_{x \to a^+ ; x \in X} \alpha(x) + \alpha\big[\set{b}\big] \\
                           & = \alpha\big[(c, b)\big] + \alpha\big[\set{c}\big] + \alpha\big[(a, c)\big] + \alpha\big[\set{b}\big]         \\
                           & = \begin{dcases}
                                 \alpha\big[(a, c)\big] + \alpha\big[[c, b]\big] \\
                                 \alpha\big[(a, c]\big] + \alpha\big[(c, b]\big]
                               \end{dcases}                             \\
                           & = \alpha[K] + \alpha[I \setminus K]
  \end{align*}
  and
  \begin{align*}
    \alpha\big[[a, b]\big] & = \alpha\big[\set{a}\big] + \alpha\big[(a, b]\big]                                                                         \\
                           & = \begin{dcases}
                                 \alpha\big[\set{a}\big] + \alpha\big[(a, c)\big] + \alpha\big[[c, b]\big] \\
                                 \alpha\big[\set{a}\big] + \alpha\big[(a, c]\big] + \alpha\big[(c, b]\big]
                               \end{dcases} \\
                           & = \begin{dcases}
                                 \alpha\big[[a, c)\big] + \alpha\big[[c, b]\big] \\
                                   \alpha\big[[a, c]\big] + \alpha\big[(c, b]\big]
                               \end{dcases}                                         \\
                           & = \alpha[K] + \alpha[I \setminus K].
  \end{align*}
  On the other hand, since \(\mathbf{P}\) forms a partition of \(I\), we see that \(\mathbf{P} \setminus \set{K}\) forms a partition of \(I \setminus K\).
  By the induction hypothesis, we thus have
  \[
    \alpha[I \setminus K] = \sum_{J \in \mathbf{P} \setminus \set{K}} \alpha[J].
  \]
  Combining these two identities (and using the laws of addition for finite sets, see \cref{i:7.1.11}(e)) we obtain
  \[
    \alpha[I] = \sum_{J \in \mathbf{P}} \alpha[J]
  \]
  as desired.

  Now suppose that \(b \notin I\), i.e., \(I\) is either \((a, b)\) or \([a, b)\).
  Then one of the intervals \(K\) also is of the form \((c, b)\) or \([c, b)\) (see \cref{i:ex:11.1.3}).
      In particular, this means that the set \(I \setminus K\) is also an interval of the form \([a, c], (a, c), (a, c], [a, c)\) when \(c > a\), or a point or empty set when \(a = c\).
  By \cref{i:11.8.1} we see that
  \begin{align*}
    \alpha\big[(a, b)\big] & = \lim_{x \to b^- ; x \in X} \alpha(x) - \lim_{x \to a^+ ; x \in X} \alpha(x)       \\
                           & = \lim_{x \to b^- ; x \in X} \alpha(x) - \lim_{x \to c^+ ; x \in X} \alpha(x)       \\
                           & \quad + \lim_{x \to c^+ ; x \in X} \alpha(x) - \lim_{x \to c^- ; x \in X} \alpha(x) \\
                           & \quad + \lim_{x \to c^- ; x \in X} \alpha(x) - \lim_{x \to a^+ ; x \in X} \alpha(x) \\
                           & = \alpha\big[(c, b)\big] + \alpha\big[\set{c}\big] + \alpha\big[(a, c)\big]         \\
                           & = \begin{dcases}
                                 \alpha\big[(a, c)\big] + \alpha\big[[c, b)\big] \\
                                   \alpha\big[(a, c]\big] + \alpha\big[(c, b)\big]
                               \end{dcases}        \\
                           & = \alpha[K] + \alpha[I \setminus K]
  \end{align*}
  and
  \begin{align*}
    \alpha\big[[a, b)\big] & = \alpha\big[\set{a}\big] + \alpha\big[(a, b)\big]                                                                     \\
                           & = \begin{dcases}
                                 \alpha\big[\set{a}\big] + \alpha\big[(a, c)\big] + \alpha\big[[c, b)\big] \\
                                   \alpha\big[\set{a}\big] + \alpha\big[(a, c]\big] + \alpha\big[(c, b)\big]
                               \end{dcases} \\
                           & = \begin{dcases}
                                 \alpha\big[[a, c)\big] + \alpha\big[[c, b)\big] \\
                                 \alpha\big[[a, c]\big] + \alpha\big[(c, b)\big]
                               \end{dcases}                                      \\
                           & = \alpha[K] + \alpha[I \setminus K].
  \end{align*}
  The rest of the argument then proceeds as above.
\end{proof}

\begin{ac}\label{i:ac:11.8.3}
  Let \(I\) be a bounded interval, let \(\alpha : X \to \R\) be a monotone decreasing function defined on some interval \(X\) which contains \(I\), and let \(\mathbf{P}\) be a partition of \(I\).
  Then we have
  \[
    \alpha[I] = \sum_{J \in \mathbf{P}} \alpha[J].
  \]
\end{ac}

\begin{proof}
  Since \(\alpha\) is monotone decreasing, we know that \(-\alpha\) is monotone increasing.
  Thus by \cref{i:11.8.4} we have
  \begin{align*}
             & (-\alpha)[I] = \sum_{J \in \mathbf{P}} (-\alpha)[J]                                                                                  \\
    \implies & -\big(\alpha[I]\big) = \sum_{J \in \mathbf{P}} -\big(\alpha[J]\big) = -\sum_{J \in \mathbf{P}} \alpha[J] &  & \text{(by limit laws)} \\
    \implies & \alpha[I] = \sum_{J \in \mathbf{P}} \alpha[J].
  \end{align*}
\end{proof}

\begin{defn}[piecewise constant Riemann-Stieltjes integral]\label{i:11.8.5}
  Let \(I\) be a bounded interval, and let \(\mathbf{P}\) be a partition of \(I\).
  Let \(\alpha : X \to \R\) be a monotone increasing function defined on some interval \(X\) which contains \(I\), and let \(f : I \to \R\) be a function which is piecewise constant with respect to \(\mathbf{P}\).
  Then we define
  \[
    p.c. \int_{[\mathbf{P}]} f \; d \alpha \coloneqq \sum_{J \in \mathbf{P}} c_J \alpha[J]
  \]
  where \(c_J\) is the constant value of \(f\) on \(J\).
\end{defn}

\begin{note}
  When \(\alpha\) is monotone decreasing, by \cref{i:11.8.5} we have
  \[
    p.c. \int_{[\mathbf{P}]} f \; d (-\alpha) = \sum_{J \in \mathbf{P}} c_J (-\alpha)[J] = - \sum_{J \in \mathbf{P}} c_J \alpha[J].
  \]
\end{note}

\setcounter{thm}{6}
\begin{eg}\label{i:11.8.7}
  Let \(\alpha : \R \to \R\) be the identity function \(\alpha(x) \coloneqq x\).
  Then for any bounded interval \(I\), any partition \(\mathbf{P}\) of \(I\), and any function \(f\) that is piecewise constant with respect to \(P\), we have \(p.c. \int_{[\mathbf{P}]} f \; d \alpha = p.c. \int_{[\mathbf{P}]} f\).
\end{eg}

\begin{ac}\label{i:ac:11.8.4}
  Let \(I\) be a bounded interval, let \(\alpha : X \to \R\) be a monotone increasing function defined on some interval \(X\) which contains \(I\), and let \(f : I \to \R\) be a function.
  Suppose that \(\mathbf{P}\) and \(\mathbf{P}'\) are partitions of \(I\) such that \(f\) is piecewise constant both with respect to \(\mathbf{P}\) and with respect to \(\mathbf{P}'\).
  Also suppose that both \(p.c. \int_{[\mathbf{P}]} f \; d \alpha\) and \(p.c. \int_{[\mathbf{P}']} f \; d \alpha\) are well-defined.
  Then \(p.c. \int_{[\mathbf{P}]} f \; d \alpha = p.c. \int_{[\mathbf{P}']} f \; d \alpha\).
\end{ac}

\begin{proof}
  By \cref{i:11.1.18} we know that \(\mathbf{P} \# \mathbf{P}'\) is a partition of \(I\) and is both finer than \(\mathbf{P}\) and finer than \(\mathbf{P}'\), thus by \cref{i:11.8.5} we have
  \[
    p.c. \int_{[\mathbf{P} \# \mathbf{P}']} f \; d \alpha = \sum_{J \in \mathbf{P} \# \mathbf{P}'} c_J \alpha[J].
  \]
  By \cref{i:11.8.4} we know that
  \[
    \alpha[I] = \sum_{J \in \mathbf{P}} \alpha[J] = \sum_{J \in \mathbf{P} \# \mathbf{P}'} \alpha[J].
  \]
  For each \(K \in \mathbf{P}\), let \(\mathbf{P}_K\) be the set
  \[
    \mathbf{P}_K = \set{S \in \mathbf{P} \# \mathbf{P}' : S \subseteq K}.
  \]
  Since \(\mathbf{P} \# \mathbf{P}'\) is finer than \(\mathbf{P}\), by \cref{i:ac:11.1.4} we know that \(\mathbf{P}_K\) is a partition of \(K\), and \(\bigcup_{K \in \mathbf{P}} \mathbf{P}_K = \mathbf{P} \# \mathbf{P}'\).
  Since \(f\) is piecewise constant with respect to \(\mathbf{P}\), by \cref{i:11.2.7} we know that \(f\) is piecewise constant with respect to \(\mathbf{P} \# \mathbf{P}'\).
  So we have
  \begin{align*}
    p.c. \int_{[\mathbf{P} \# \mathbf{P}']} f \; d \alpha & = \sum_{J \in \mathbf{P} \# \mathbf{P}'} c_J \alpha[J]                        &                 & \by{i:11.8.5}    \\
                                                          & = \sum_{J \in \bigcup_{K \in \mathbf{P}} \mathbf{P}_K} c_J \alpha[J]                                               \\
                                                          & = \sum_{K \in \mathbf{P}} \sum_{J \in \mathbf{P}_K} c_J \alpha[J]             &                 & \by{i:7.1.11}[e] \\
                                                          & = \sum_{K \in \mathbf{P}} \sum_{J \in \mathbf{P}_K} c_K \alpha[J]             & (J \subseteq K)                    \\
                                                          & = \sum_{K \in \mathbf{P}} c_K \bigg(\sum_{J \in \mathbf{P}_K} \alpha[J]\bigg)                                      \\
                                                          & = \sum_{K \in \mathbf{P}} c_K \alpha[K]                                       &                 & \by{i:11.8.4}    \\
                                                          & = p.c. \int_{[\mathbf{P}]} f \; d \alpha.                                     &                 & \by{i:11.8.5}
  \end{align*}
  Using similar arguments we can show that \(p.c. \int_{[\mathbf{P}']} f \; d \alpha = p.c. \int_{[\mathbf{P} \# \mathbf{P}']} f \; d \alpha\).
  Thus we have \(p.c. \int_{[\mathbf{P}]} f \; d \alpha = p.c. \int_{[\mathbf{P}']} f \; d \alpha\).
\end{proof}

\begin{ac}\label{i:ac:11.8.5}
  Let \(I\) be a bounded interval, let \(\alpha : X \to \R\) be a monotone increasing function defined on some interval \(X\) which contains \(I\), and let \(f : I \to \R\) be a piecewise constant function on \(I\).
  Then we define
  \[
    p.c. \int_I f \; d \alpha \coloneqq p.c. \int_{[\mathbf{P}]} f \; d \alpha,
  \]
  where \(\mathbf{P}\) is any partition of \(I\) with respect to which \(f\) is piecewise constant.
  (Note that \cref{i:ac:11.8.4} tells us that the precise choice of this partition is irrelevant.)
\end{ac}

\begin{ac}\label{i:ac:11.8.6}
  Let \(I\) be a bounded interval, let \(\alpha : X \to \R\) be a monotone function defined on some interval \(X\) which contains \(I\).
  \begin{itemize}
    \item If \(\alpha\) is monotone increasing, then \(\alpha[I] \geq 0\).
    \item If \(\alpha\) is monotone decreasing, then \(\alpha[I] \leq 0\).
  \end{itemize}
\end{ac}

\begin{proof}
  We split into four cases:
  \begin{itemize}
    \item \(I = \emptyset\).
          Then by \cref{i:11.8.1} we have \(\alpha[\emptyset] = 0\).
    \item \(I = \set{x_0}\) for some \(x_0 \in \R\).
          If \(\alpha\) is monotone increasing, then we have
          \begin{align*}
            \alpha\big[\set{x_0}\big] & = \lim_{x \to x_0^+} \alpha(x) - \lim_{x \to x_0^-} \alpha(x)                      &  & \by{i:11.8.1}                                    \\
                                      & = \inf_{x \in X \cap (x_0, \infty)} f(x) - \sup_{x \in X \cap (-\infty, x_0)} f(x) &  & \by{i:ac:11.8.1}                                 \\
                                      & \geq f(x_0) - f(x_0)                                                               &  & \text{(since \(\alpha\) is monotone increasing)} \\
                                      & = 0.
          \end{align*}
          If \(\alpha\) is monotone decreasing, then we have
          \begin{align*}
            \alpha\big[\set{x_0}\big] & = \lim_{x \to x_0^+} \alpha(x) - \lim_{x \to x_0^-} \alpha(x)                      &  & \by{i:11.8.1}                                    \\
                                      & = \sup_{x \in X \cap (x_0, \infty)} f(x) - \inf_{x \in X \cap (-\infty, x_0)} f(x) &  & \by{i:ac:11.8.2}                                 \\
                                      & \leq f(x_0) - f(x_0)                                                               &  & \text{(since \(\alpha\) is monotone decreasing)} \\
                                      & = 0.
          \end{align*}
    \item \(I = (a, b)\) for some \(a, b \in \R\) and \(a < b\).
          If \(\alpha\) is monotone increasing, then we have
          \begin{align*}
             & \alpha\big[(a, b)\big]                                                                                                   \\
             & = \lim_{x \to b^- ; x \in (a, b)} \alpha(x) - \lim_{x \to a^+ ; x \in (a, b)} \alpha(x)            &  & \by{i:11.8.1}    \\
             & = \sup_{x \in (a, b) \cap (-\infty, b)} \alpha(x) - \inf_{x \in (a, b) \cap (a, \infty)} \alpha(x) &  & \by{i:ac:11.8.1} \\
             & = \sup_{x \in (a, b)} \alpha(x) - \inf_{x \in (a, b)} \alpha(x)                                                          \\
             & \geq 0.
          \end{align*}
          If \(\alpha\) is monotone decreasing, then we have
          \begin{align*}
             & \alpha\big[(a, b)\big]                                                                                                   \\
             & = \lim_{x \to b^- ; x \in (a, b)} \alpha(x) - \lim_{x \to a^+ ; x \in (a, b)} \alpha(x)            &  & \by{i:11.8.1}    \\
             & = \inf_{x \in (a, b) \cap (-\infty, b)} \alpha(x) - \sup_{x \in (a, b) \cap (a, \infty)} \alpha(x) &  & \by{i:ac:11.8.2} \\
             & = \inf_{x \in (a, b)} \alpha(x) - \sup_{x \in (a, b)} \alpha(x)                                                          \\
             & \leq 0.
          \end{align*}
    \item \(I\) is one of \([a, b), (a, b], [a, b]\).
          If \(\alpha\) is monotone increasing, then from the proof above we have
          \begin{align*}
            \alpha\big[[a, b)\big] & = \alpha\big[\set{a}\big] + \alpha\big[(a, b)\big] \geq 0 \\
            \alpha\big[(a, b]\big] & = \alpha\big[(a, b)\big] + \alpha\big[\set{b}\big] \geq 0 \\
            \alpha\big[[a, b]\big] & = \alpha\big[\set{a}\big] + \alpha\big[(a, b]\big] \geq 0
          \end{align*}
          If \(\alpha\) is monotone decreasing, then from the proof above we have
          \begin{align*}
            \alpha\big[[a, b)\big] & = \alpha\big[\set{a}\big] + \alpha\big[(a, b)\big] \leq 0 \\
            \alpha\big[(a, b]\big] & = \alpha\big[(a, b)\big] + \alpha\big[\set{b}\big] \leq 0 \\
            \alpha\big[[a, b]\big] & = \alpha\big[\set{a}\big] + \alpha\big[(a, b]\big] \leq 0
          \end{align*}
  \end{itemize}
  From all cases above we conclude that \(\alpha[I] \geq 0\) if \(\alpha\) is monotone increasing and \(\alpha[I] \leq 0\) if \(\alpha\) is monotone decreasing.
\end{proof}

\begin{ac}\label{i:ac:11.8.7}
  Let \(I\) be a bounded interval, let \(\alpha : X \to \R\) be a monotone increasing function defined on some interval \(X\) which contains \(I\), and let \(f : I \to \R\) and \(g : I \to \R\) be piecewise constant functions on \(I\) such that both \(p.c. \int_I f \; d \alpha\) and \(p.c. \int_I g \; d \alpha\) are well-defined.
  \begin{enumerate}
    \item We have \(p.c. \int_I (f + g) \; d \alpha = p.c. \int_I f \; d \alpha + p.c. \int_I g \; d \alpha\).
    \item For any real number \(c\), we have \(p.c. \int_I (cf) \; d \alpha = c (p.c. \int_I f \; d \alpha)\).
    \item We have \(p.c. \int_I (f - g) \; d \alpha = p.c. \int_I f \; d \alpha - p.c. \int_I g \; d \alpha\).
    \item If \(f(x) \geq 0\) for all \(x \in I\), then \(p.c. \int_I f \; d \alpha \geq 0\).
    \item If \(f(x) \geq g(x)\) for all \(x \in I\), then \(p.c. \int_I f \; d \alpha \geq p.c. \int_I g \; d \alpha\).
    \item If \(f\) is the constant function \(f(x) = c\) for all \(x \in I\), then \(p.c. \int_I f \; d \alpha = c \alpha[I]\).
    \item Let \(J\) be a bounded interval containing \(I\) (i.e., \(I \subseteq J\)), and let \(F : J \to \R\) be the function
          \[
            F(x) \coloneqq \begin{dcases}
              f(x) & \text{if } x \in I    \\
              0    & \text{if } x \notin I
            \end{dcases}
          \]
          Then \(F\) is piecewise constant on \(J\), and \(p.c. \int_J F \; d \alpha = p.c. \int_I f \; d \alpha\).
    \item Suppose that \(\set{J, K}\) is a partition of \(I\) into two intervals \(J\) and \(K\).
          Then the function \(f|_J : J \to \R\) and \(f|_K : K \to \R\) are piecewise constant on \(J\) and \(K\) respectively, and we have
          \[
            p.c. \int_I f \; d \alpha = p.c. \int_J f|_J \; d \alpha + p.c. \int_K f|_K \; d \alpha.
          \]
  \end{enumerate}
\end{ac}

\begin{proof}{(a)}
  Let \(\mathbf{P}\) be a partition of \(I\).
  By \cref{i:11.2.16}(a) we know that \(f + g\) is piecewise constant with respect to \(\mathbf{P}\).
  For each \(J \in \mathbf{P}\), we define \(c_{f|_J}, c_{g|_J} \in \R\) to be the constant value of \(f|_J, g|_J\), respectively.
  Then by \cref{i:11.2.1} \(c_{f|_J} + c_{g|_J}\) is the constant value of \((f + g)|_J\) for each \(J \in P\).
  Thus we have
  \begin{align*}
     & p.c. \int_I f \; d \alpha + p.c. \int_I g \; d \alpha                                                   \\
     & = p.c. \int_{[\mathbf{P}]} f \; d \alpha + p.c. \int_{[\mathbf{P}]} g \; d \alpha &  & \by{i:ac:11.8.5} \\
     & = \sum_{J \in \mathbf{P}} f_J \alpha[J] + \sum_{J \in \mathbf{P}} g_J \alpha[J]   &  & \by{i:11.8.5}    \\
     & = \sum_{J \in \mathbf{P}} (f_J + g_J) \alpha[J]                                   &  & \by{i:7.1.11}[f] \\
     & = p.c. \int_{[\mathbf{P}]} (f_J + g_J) \; d \alpha                                &  & \by{i:11.8.5}    \\
     & = p.c. \int_I (f_J + g_J) \; d \alpha.                                            &  & \by{i:ac:11.8.5}
  \end{align*}
\end{proof}

\begin{proof}{(b)}
  Let \(\mathbf{P}\) be a partition of \(I\).
  By \cref{i:11.2.16}(b) we know that \(cf\) is piecewise constant with respect to \(\mathbf{P}\).
  For each \(J \in \mathbf{P}\), we define \(c_J \in \R\) to be the constant value of \(f|_J\).
  Then by \cref{i:11.2.1} \(c \cdot c_J\) is the constant value of \((cf)|_J\).
  Thus we have
  \begin{align*}
    c \bigg(p.c. \int_I f \; d \alpha\bigg) & = c \bigg(p.c. \int_{[\mathbf{P}]} f \; d \alpha\bigg) &  & \by{i:ac:11.8.5} \\
                                            & = c \bigg(\sum_{J \in \mathbf{P}} c_J \alpha[J]\bigg)  &  & \by{i:11.8.5}    \\
                                            & = \sum_{J \in \mathbf{P}} c \cdot c_J \alpha[J]        &  & \by{i:7.1.11}[g] \\
                                            & = p.c. \int_{[\mathbf{P}]} (c f) \; d \alpha           &  & \by{i:11.8.5}    \\
                                            & = p.c. \int_I (c f) \; d \alpha.                       &  & \by{i:ac:11.8.5}
  \end{align*}
\end{proof}

\begin{proof}{(c)}
  We have
  \begin{align*}
     & p.c. \int_I f \; d \alpha - p.c. \int_I g \; d \alpha                                 \\
     & = p.c. \int_I f \; d \alpha + (-1) p.c. \int_I g \; d \alpha                          \\
     & = p.c. \int_I f \; d \alpha + p.c. \int_I (-g) \; d \alpha   &  & \by{i:ac:11.8.7}[b] \\
     & = p.c. \int_I \big(f + (-g)\big) \; d \alpha                 &  & \by{i:ac:11.8.7}[a] \\
     & = p.c. \int_I (f - g) \; d \alpha.                           &  & \by{i:9.2.1}
  \end{align*}
\end{proof}

\begin{proof}{(d)}
  By \cref{i:ac:11.8.5} \(f\) is piecewise constant with respect to \(\mathbf{P}\) for some partition \(\mathbf{P}\) of \(I\).
  Let \(J \in \mathbf{P}\) and let \(c_J \in \R\) be the constant value of \(f|_J\).
  By \cref{i:ac:11.8.6} we know that \(\alpha[J] \geq 0\) for all \(J \in \mathbf{P}\).
  Since \(f(x) \geq 0\) for all \(x \in I\), we have \(c_J \geq 0\) and \(c_J \alpha[J] \geq 0\) for all \(J \in \mathbf{P}\).
  Thus
  \begin{align*}
    p.c. \int_I f \; d \alpha & = p.c. \int_{[\mathbf{P}]} f \; d \alpha &  & \by{i:ac:11.8.5} \\
                              & = \sum_{J \in \mathbf{P}} c_J \alpha[J]  &  & \by{i:11.8.5}    \\
                              & \geq \sum_{J \in \mathbf{P}} 0           &  & \by{i:7.1.11}[h] \\
                              & = 0.
  \end{align*}
\end{proof}

\begin{proof}{(e)}
  Since \(f(x) \geq g(x)\) for all \(x \in I\), we have \(f(x) - g(x) \geq 0\).
  By \cref{i:ac:11.8.7}(c) we have
  \[
    p.c. \int_I f \; d \alpha - p.c. \int_I g \; d \alpha = p.c. \int_I (f - g) \; d \alpha.
  \]
  Then by \cref{i:ac:11.8.7}(d) we have
  \[
    p.c. \int_I (f - g) \; d \alpha \geq 0 \implies p.c. \int_I f \; d \alpha \geq p.c. \int_I g \; d \alpha.
  \]
\end{proof}

\begin{proof}{(f)}
  Since \(I\) is a partition of \(I\), we have
  \begin{align*}
    p.c. \int_I f \; d \alpha & = p.c. \int_{[I]} f \; d \alpha &  & \by{i:ac:11.8.5} \\
                              & = \sum_{J \in I} c \alpha[J]    &  & \by{i:11.8.5}    \\
                              & = c \sum_{J \in I} \alpha[J]    &  & \by{i:7.1.11}[g] \\
                              & = c \alpha[I].                  &  & \by{i:11.8.4}
  \end{align*}
\end{proof}

\begin{proof}{(g)}
  If \(I = \emptyset\), then by \cref{i:11.2.3} \(F\) is piecewise constant with respect to \(\set{J}\), and by \cref{i:ac:11.8.7}(f) we have
  \[
    p.c. \int_J F \; d \alpha = 0 \alpha[J] = 0 = p.c \int_I f \; d \alpha.
  \]
  So suppose that \(I \neq \emptyset\).
  By \cref{i:11.2.3}, \(f\) is piecewise constant with respect to \(\mathbf{P}\) for some partition \(\mathbf{P}\) of \(I\).
  Let \(I_1, I_2\) be the sets
  \[
    I_1 = \set{x \in J, \big(x \leq \inf(I)\big) \land (x \notin I)}
  \]
  and
  \[
    I_2 = \set{x \in J, \big(x \geq \sup(I)\big) \land (x \notin I)}.
  \]
  By \cref{i:ac:11.1.5} we know that \(\mathbf{P} \cup \set{I_1, I_2}\) is a partition of \(J\).
  By hypothesis we know that
  \[
    \forall x \in J, F(x) = \begin{dcases}
      f(x) & \text{if } x \in K \text{ for some } K \in \mathbf{P} \\
      0    & \text{if } x \in I_1 \text{ or } x \in I_2
    \end{dcases}
  \]
  Thus by \cref{i:11.2.5} \(F\) is piecewise constant on \(J\).
  For each \(K \in \mathbf{P} \cup \set{I_1, I_2}\), we define \(c_K \in \R\) to be the constant value of \(F|_K\).
  Then we have
  \begin{align*}
    p.c. \int_J F \; d \alpha & = p.c. \int_{[\mathbf{P} \cup \set{I_1, I_2}]} F \; d \alpha                        &  & \by{i:ac:11.8.5}       \\
                              & = \sum_{K \in \mathbf{P}} c_K \alpha[K]                                             &  & \by{i:11.8.5}          \\
                              & = c_{I_1} \alpha[I_1] + \sum_{K \in \mathbf{P}} c_K \alpha[K] + c_{I_2} \alpha[I_2] &  & \by{i:7.1.11}[e]       \\
                              & = 0 \alpha[I_1] + \sum_{K \in \mathbf{P}} c_K \alpha[K] + 0 \alpha[I_2]             &  & \text{(by hypothesis)} \\
                              & = \sum_{K \in \mathbf{P}} c_K \alpha[K]                                                                         \\
                              & = p.c. \int_{[\mathbf{P}]} f \; d \alpha                                            &  & \by{i:11.8.5}          \\
                              & = p.c. \int_I f \; d \alpha.                                                        &  & \by{i:ac:11.8.5}
  \end{align*}
\end{proof}

\begin{proof}{(h)}
  Let \(\mathbf{P} = \set{J, K}\).
  By \cref{i:11.2.3} \(f\) is piecewise constant with respect to \(\mathbf{P}'\) for some partition \(\mathbf{P}'\) of \(I\).
  Now we define \(\mathbf{P}_J\) as
  \[
    \mathbf{P}_J = \set{S \in \mathbf{P} \# \mathbf{P}' : S \subseteq J}
  \]
  and define \(\mathbf{P}_K\) as
  \[
    \mathbf{P}_K = \set{S \in \mathbf{P} \# \mathbf{P}' : S \subseteq K}.
  \]
  By \cref{i:11.1.8} we know that \(\mathbf{P} \# \mathbf{P}'\) is a partition of \(I\) and is finer than \(\mathbf{P}\).
  Since \(\mathbf{P} \# \mathbf{P}'\) is finer than \(\mathbf{P}\), by \cref{i:ac:11.1.4} we know that \(\mathbf{P}_J, \mathbf{P}_K\) are partitions of \(J, K\), respectively.
  Again by \cref{i:ac:11.1.4} we know that \(\mathbf{P}_J \cup \mathbf{P}_K\) is a partition of \(I\).
  Then by \cref{i:11.2.7} \(f\) is piecewise constant with respect to \(\mathbf{P}_J \cup \mathbf{P}_K\).
  Without the loss of generality suppose that \(\emptyset \notin \mathbf{P}_J \cup \mathbf{P}_K\).
  For each \(S \in \mathbf{P}_J\), we define \(c_S \in \R\) to be the constant value of \(f|_J\).
  Similarly, for each \(S \in \mathbf{P}_K\), we define \(c_S \in \R\) to be the constant value of \(f|_K\).
  Then we have
  \begin{align*}
      & p.c. \int_J f|_J \; d \alpha + p.c. \int_K f|_K \; d \alpha                                                     \\
    = & p.c. \int_{[\mathbf{P}_J]} f|_J \; d \alpha + p.c. \int_{[\mathbf{P}_K]} f|_K \; d \alpha &  & \by{i:ac:11.8.5} \\
    = & \sum_{S \in \mathbf{P}_J} c_S \alpha[S] + \sum_{S \in \mathbf{P}_K} c_S \alpha[S]         &  & \by{i:7.1.11}[e] \\
    = & \sum_{S \in \mathbf{P}_J \cup \mathbf{P}_K} c_S \alpha[S]                                 &  & \by{i:11.8.5}    \\
    = & \sum_{S \in \mathbf{P}} c_S \alpha[S]                                                                           \\
    = & p.c. \int_{[\mathbf{P}]} f \; d \alpha                                                    &  & \by{i:11.8.5}    \\
    = & p.c. \int_I f \; d \alpha.                                                                &  & \by{i:ac:11.8.5}
  \end{align*}
\end{proof}

\begin{ac}\label{i:ac:11.8.8}
  Let \(I\) be a bounded interval, let \(\alpha : X \to \R\) be a monotone increasing function defined on some interval \(X\) which contains \(I\), and let \(f : I \to \R\) be a bounded function.
  We define the \emph{upper Riemann-Stieltjes integral} \(\overline{\int}_I f \; d \alpha\) by the formula
  \[
    \overline{\int}_I f \; d \alpha \coloneqq \inf\set{p.c. \int_I g \; d \alpha : g \text{ is a p.c. function on \(I\) which majorizes } f}
  \]
  and the \emph{lower Riemann-Stieltjes integral} \(\underline{\int}_I f \; d \alpha\) by the formula
  \[
    \underline{\int}_I f \; d \alpha \coloneqq \sup\set{p.c. \int_I g \; d \alpha : g \text{ is a p.c. function on \(I\) which minorizes } f}.
  \]
  If \(\underline{\int}_I f \; d \alpha = \overline{\int}_I f \; d \alpha\), then we say that \(f\) is \emph{Riemann-Stieltjes integrable on \(I\) with respect to \(\alpha\)} and define
  \[
    \int_I f \; d \alpha \coloneqq \underline{\int}_I f \; d \alpha = \overline{\int}_I f \; d \alpha.
  \]
  If the upper and lower Riemann-Stieltjes integrals are unequal, we say that \(f\) is not Riemann-Stieltjes integrable on \(I\) with respect to \(\alpha\).
\end{ac}

\begin{ac}\label{i:ac:11.8.9}
  Let \(I\) be a bounded interval, let \(\alpha : X \to \R\) be a monotone increasing function defined on some interval \(X\) which contains \(I\).
  Let \(f : I \to \R\) be a function which is bounded by some real number \(M\), i.e., \(-M \leq f(x) \leq M\) for all \(x \in I\).
  Then we have
  \[
    -M \alpha[I] \leq \underline{\int}_I f \; d \alpha \leq \overline{\int}_I f \; d \alpha \leq M \alpha[I].
  \]
  in particular, both the lower and upper Riemann-Stieltjes integrals are real numbers (i.e., they are not infinite).
\end{ac}

\begin{proof}
  The function \(g : I \to \R\) defined by \(g(x) = M\) is constant, hence piecewise constant, and majorizes \(f\);
  thus \(\overline{\int}_I f \; d \alpha \leq p.c. \int_I g \; d \alpha = M \alpha[I]\) by definition of the upper Riemann-Stieltjes integral.
  A similar argument gives \(-M \alpha[I] \leq \underline{\int}_I f \; d \alpha\).
  Finally, we have to show that \(\underline{\int}_I f \; d \alpha \leq \overline{\int}_I f \; d \alpha\).
  Let \(g\) be any piecewise constant function majorizing \(f\), and let \(h\) be any piecewise constant function minorizing \(f\).
  Then \(g\) majorizes \(h\), and hence \(p.c. \int_I h \; d \alpha \leq p.c. \int_I g \; d \alpha\).
  Taking suprema in \(h\), we obtain that \(\underline{\int}_I f \; d \alpha \leq p.c. \int_I g \; d \alpha\).
  Taking infima in \(g\), we thus obtain \(\underline{\int}_I f \; d \alpha \leq \overline{\int}_I f \; d \alpha\), as desired.
\end{proof}

\begin{note}
  When \(\alpha\) is the identity function \(\alpha(x) \coloneqq x\) then the Riemann-Stieltjes integral is identical to the Riemann integral;
  thus the Riemann-Stieltjes integral is a generalization of the Riemann integral.
  We sometimes write \(\int_I f\) as \(\int_I f \; dx\) or \(\int_I f(x) \; dx\).
\end{note}

\begin{ac}\label{i:ac:11.8.10}
  Let \(I\) be a bounded interval, let \(\alpha : X \to \R\) be a monotone increasing function defined on some interval \(X\) which contains \(I\).
  Let \(f : I \to \R\) be a piecewise constant function.
  Then \(f\) is Riemann-Stieltjes integrable on \(I\) with respect to \(\alpha\), and \(\int_I f \; d \alpha = p.c. \int_I f \; d \alpha\).
\end{ac}

\begin{proof}
  Since \(f(x) \leq f(x)\) for every \(x \in I\), by \cref{i:ac:11.8.8} and \cref{i:ac:11.8.9} we have
  \[
    p.c. \int_I f \; d \alpha \leq \underline{\int}_I f \; d \alpha \leq \overline{\int}_I f \; d \alpha \leq p.c. \int_I f \; d \alpha
  \]
  Thus by \cref{i:ac:11.8.8} we have
  \[
    \int_I f \; d \alpha = \underline{\int}_I f \; d \alpha = \overline{\int}_I f \; d \alpha = p.c. \int_I f \; d \alpha.
  \]
\end{proof}

\begin{ac}[Laws of Riemann-Stieltjes integration]\label{i:ac:11.8.11}
  Let \(I\) be a bounded interval, let \(\alpha : X \to \R\) be a monotone increasing function defined on some interval \(X\) which contains \(I\).
  Let \(f : I \to \R\) and \(g : I \to \R\) be Riemann-Stieltjes integrable functions on \(I\) with respect to \(\alpha\).
  \begin{enumerate}
    \item The function \(f + g\) is Riemann-Stieltjes integrable, and we have \(\int_I (f + g) \; d \alpha = \int_I f \; d \alpha + \int_I g \; d \alpha\).
    \item For any real number \(c\), the function \(cf\) is Riemann-Stieltjes integrable, and we have \(\int_I (cf) \; d \alpha = c(\int_I f \; d \alpha)\).
    \item The function \(f - g\) is Riemann-Stieltjes integrable, and we have \(\int_I (f - g) \; d \alpha = \int_I f \; d \alpha - \int_I g \; d \alpha\).
    \item If \(f(x) \geq 0\) for all \(x \in I\), then \(\int_I f \; d \alpha \geq 0\).
    \item If \(f(x) \geq g(x)\) for all \(x \in I\), then \(\int_I f \; d \alpha \geq \int_I g \; d \alpha\).
    \item If \(f\) is the constant function \(f(x) = c\) for all \(x \in I\), then \(\int_I f \; d \alpha = c \alpha[I]\).
    \item Suppose that \(\set{J, K}\) is a partition of \(I\) into two intervals \(J\) and \(K\).
          Then the functions \(f|_J : J \to \R\) and \(f|_K : K \to \R\) are Riemann-Stieltjes integrable on \(J\) and \(K\) respectively, and we have
          \[
            \int_I f \; d \alpha = \int_J f|_J \; d \alpha + \int_K f|_K \; d \alpha.
          \]
  \end{enumerate}
\end{ac}

\begin{proof}{(a)}
  Let \(f_U : I \to \R\) and \(g_U : I \to \R\) be piecewise constant functions on \(I\) which majorizes \(f\) and \(g\), respectively.
  Let \(f_L : I \to \R\) and \(g_L : I \to \R\) be piecewise constant functions on \(I\) which minorizes \(f\) and \(g\), respectively.
  \(f_U, g_U, f_L, g_L\) are well-defined since by \cref{i:ac:11.8.8} \(f, g\) are bounded functions on a bounded interval \(I\).
  Then we have
  \[
    p.c. \int_I f_L \; d \alpha \leq \underline{\int}_I f \; d \alpha = \int_I f \; d \alpha = \overline{\int}_I f \; d \alpha \leq p.c. \int_I f_U \; d \alpha
  \]
  and
  \[
    p.c. \int_I g_L \; d \alpha \leq \underline{\int}_I g \; d \alpha = \int_I g \; d \alpha = \overline{\int}_I g \; d \alpha \leq p.c. \int_I g_U \; d \alpha.
  \]
  By \cref{i:ac:11.8.8} both \(f, g\) are bounded functions, so \(f + g\) is bounded function, and \(\underline{\int}_I (f + g) \; d \alpha, \overline{\int}_I (f + g) \; d \alpha\) are well-defined (by \cref{i:ac:11.8.8}).
  By \cref{i:ex:11.3.2} we know that \(f_U + g_U\) majorizes \(f + g_U\) and \(f + g_U\) majorizes \(f + g\), thus \(f_U + g_U\) majorizes \(f + g\).
  Similarly \(f_L + g_L\) minorizes \(f + g\).
  Then we have
  \begin{align*}
             & \overline{\int}_I (f + g) \; d \alpha \leq p.c. \int_I (f_U + g_U) \; d \alpha                               &   & \by{i:ac:11.8.8}                         \\
    \implies & \overline{\int}_I (f + g) \; d \alpha                                                                                                                       \\
             & \quad \leq p.c. \int_I f_U \; d \alpha + p.c. \int_I g_U \; d \alpha                                         &   & \by{i:ac:11.8.7}[a]                      \\
    \implies & \overline{\int}_I (f + g) \; d \alpha - p.c. \int_I g_U \; d \alpha                                                                                         \\
             & \quad \leq p.c. \int_I f_U \; d \alpha                                                                       &   & \text{(note that \(f_U\) was arbitrary)} \\
    \implies & \overline{\int}_I (f + g) \; d \alpha - p.c. \int_I g_U \; d \alpha \leq \overline{\int}_I f \; d \alpha     &   & \by{i:ac:11.8.8}                         \\
    \implies & \overline{\int}_I (f + g) \; d \alpha - \overline{\int}_I f \; d \alpha \leq p.c. \int_I g_U \; d \alpha     &   & \text{(note that \(g_U\) was arbitrary)} \\
    \implies & \overline{\int}_I (f + g) \; d \alpha - \overline{\int}_I f \; d \alpha \leq \overline{\int}_I g \; d \alpha &   & \by{i:ac:11.8.8}                         \\
    \implies & \overline{\int}_I (f + g) \; d \alpha \leq \overline{\int}_I f \; d \alpha + \overline{\int}_I g \; d \alpha &                                              \\
    \implies & \overline{\int}_I (f + g) \; d \alpha \leq \int_I f \; d \alpha + \int_I g \; d \alpha                       &   & \by{i:ac:11.8.8}
  \end{align*}
  and
  \begin{align*}
             & \underline{\int}_I (f + g) \; d \alpha \geq p.c. \int_I (f_L + g_L) \; d \alpha                                 &   & \by{i:ac:11.8.8}                         \\
    \implies & \underline{\int}_I (f + g) \; d \alpha                                                                                                                         \\
             & \quad \geq p.c. \int_I f_L \; d \alpha + p.c. \int_I g_L \; d \alpha                                            &   & \by{i:ac:11.8.7}[a]                      \\
    \implies & \underline{\int}_I (f + g) \; d \alpha - p.c. \int_I g_L \; d \alpha                                                                                           \\
             & \quad \geq p.c. \int_I f_L \; d \alpha                                                                          &   & \text{(note that \(f_L\) was arbitrary)} \\
    \implies & \underline{\int}_I (f + g) \; d \alpha - p.c. \int_I g_L \; d \alpha \geq \underline{\int}_I f \; d \alpha      &   & \by{i:ac:11.8.8}                         \\
    \implies & \underline{\int}_I (f + g) \; d \alpha - \underline{\int}_I f \; d \alpha \geq p.c. \int_I g_L \; d \alpha      &   & \text{(note that \(g_L\) was arbitrary)} \\
    \implies & \underline{\int}_I (f + g) \; d \alpha - \underline{\int}_I f \; d \alpha \geq \underline{\int}_I g \; d \alpha &   & \by{i:ac:11.8.8}                         \\
    \implies & \underline{\int}_I (f + g) \; d \alpha \geq \underline{\int}_I f \; d \alpha + \underline{\int}_I g \; d \alpha &                                              \\
    \implies & \underline{\int}_I (f + g) \; d \alpha \geq \int_I f \; d \alpha + \int_I g \; d \alpha.                        &   & \by{i:ac:11.8.8}
  \end{align*}
  By \cref{i:ac:11.8.9} we have
  \[
    \int_I f \; d \alpha + \int_I g \; d \alpha \leq \underline{\int}_I (f + g) \; d \alpha \leq \overline{\int}_I (f + g) \; d \alpha \leq \int_I f \; d \alpha + \int_I g \; d \alpha
  \]
  and thus by \cref{i:ac:11.8.8} we have
  \[
    \int_I (f + g) \; d \alpha = \underline{\int}_I (f + g) \; d \alpha = \overline{\int}_I (f + g) \; d \alpha = \int_I f \; d \alpha + \int_I g \; d \alpha.
  \]
\end{proof}

\begin{proof}{(b)}
  Since \(f\) is Riemann-Stieltjes integrable on \(I\) with respect to \(\alpha\), by \cref{i:ac:11.8.8} we have
  \[
    \int_I f \; d \alpha = \overline{\int}_I f \; d \alpha = \underline{\int}_I f \; d \alpha.
  \]
  First suppose that \(c = 0\).
  Then we have \((cf)(x) = 0\) for all \(x \in 0\), thus we have
  \begin{align*}
    \int_I (cf) \; d \alpha & = p.c. \int_I (cf) \; d \alpha &  & \by{i:ac:11.8.10} \\
                            & = 0                                                   \\
                            & = c \int_I f \; d \alpha.
  \end{align*}

  Next suppose that \(c > 0\).
  Let \(f_U : I \to \R\) be a piecewise constant function on \(I\) which majorizes \(f\).
  Let \(f_L : I \to \R\) be a piecewise constant function on \(I\) which minorizes \(f\).
  \(f_U, f_L\) are well-defined since by \cref{i:ac:11.8.8} \(f\) is a bounded function on a bounded interval \(I\).
  Then by \cref{i:ac:11.8.8} we have
  \[
    p.c. \int_I f_L \; d \alpha \leq \underline{\int}_I f \; d \alpha = \int_I f \; d \alpha = \overline{\int}_I f \; d \alpha \leq p.c. \int_I f_U \; d \alpha.
  \]
  Since \(f\) is a bounded function, \(cf\) is also a bounded function, by \cref{i:ac:11.8.8} both \(\overline{\int}_I (cf) \; d \alpha, \underline{\int}_I (cf) \; d \alpha\) are well-defined.
  Since \(c > 0\), by \cref{i:11.3.1} we know that \(c f_U\) majorizes \(c f\) and \(c f_L\) minorizes \(c f\).
  Then we have
  \begin{align*}
             & \overline{\int}_I (cf) \; d \alpha \leq p.c. \int_I (c f_U) \; d \alpha                          &  & \by{i:ac:11.8.8}                         \\
    \implies & \overline{\int}_I (cf) \; d \alpha \leq c \bigg(p.c. \int_I f_U \; d \alpha\bigg)                &  & \by{i:ac:11.8.7}[b]                      \\
    \implies & \dfrac{1}{c} \bigg(\overline{\int}_I (cf) \; d \alpha\bigg) \leq p.c. \int_I f_U \; d \alpha     &  & \text{(note that \(f_U\) was arbitrary)} \\
    \implies & \dfrac{1}{c} \bigg(\overline{\int}_I (cf) \; d \alpha\bigg) \leq \overline{\int}_I f \; d \alpha &  & \by{i:ac:11.8.8}                         \\
    \implies & \overline{\int}_I (cf) \; d \alpha \leq c\bigg(\overline{\int}_I f \; d \alpha\bigg)                                                           \\
    \implies & \overline{\int}_I (cf) \; d \alpha \leq c\bigg(\int_I f \; d \alpha\bigg)                        &  & \by{i:ac:11.8.8}
  \end{align*}
  and
  \begin{align*}
             & \underline{\int}_I (cf) \; d \alpha \geq p.c. \int_I (c f_L) \; d \alpha                           &  & \by{i:ac:11.8.8}                         \\
    \implies & \underline{\int}_I (cf) \; d \alpha \geq c \bigg(p.c. \int_I f_L \; d \alpha\bigg)                 &  & \by{i:ac:11.8.7}[b]                      \\
    \implies & \dfrac{1}{c} \bigg(\underline{\int}_I (cf) \; d \alpha\bigg) \geq p.c. \int_I f_L \; d \alpha      &  & \text{(note that \(f_L\) was arbitrary)} \\
    \implies & \dfrac{1}{c} \bigg(\underline{\int}_I (cf) \; d \alpha\bigg) \geq \underline{\int}_I f \; d \alpha &  & \by{i:ac:11.8.8}                         \\
    \implies & \underline{\int}_I (cf) \; d \alpha \geq c\bigg(\underline{\int}_I f \; d \alpha\bigg)                                                           \\
    \implies & \underline{\int}_I (cf) \; d \alpha \geq c\bigg(\int_I f \; d \alpha\bigg).                        &  & \by{i:ac:11.8.8}
  \end{align*}
  By \cref{i:ac:11.8.9} we have
  \[
    c\bigg(\int_I f \; d \alpha\bigg) \leq \underline{\int}_I (cf) \; d \alpha \leq \overline{\int}_I (cf) \; d \alpha \leq c\bigg(\int_I f \; d \alpha\bigg)
  \]
  and thus by \cref{i:ac:11.8.8} we have
  \[
    \int_I (cf) \; d \alpha = \underline{\int}_I (cf) \; d \alpha = \overline{\int}_I (cf) \; d \alpha = c\bigg(\int_I f \; d \alpha\bigg).
  \]

  Finally suppose that \(c < 0\).
  Using the same definition of \(f_U, f_L\) we have
  \begin{align*}
             & \overline{\int}_I (cf \; d \alpha) \leq p.c. \int_I (c f_U \; d \alpha)                                                           &  & \by{i:ac:11.8.8}    \\
    \implies & \overline{\int}_I (cf \; d \alpha) \leq c \bigg(p.c. \int_I f_U \; d \alpha\bigg)                                                 &  & \by{i:ac:11.8.7}[b] \\
    \implies & \dfrac{1}{c} \bigg(\overline{\int}_I (cf) \; d \alpha\bigg) \geq p.c. \int_I f_U \; d \alpha                                                               \\
    \implies & \dfrac{1}{c} \bigg(\overline{\int}_I (cf) \; d \alpha\bigg) \geq p.c. \int_I f_U \; d \alpha \geq \overline{\int}_I f \; d \alpha &  & \by{i:ac:11.8.8}    \\
    \implies & \overline{\int}_I (cf) \; d \alpha \leq c\bigg(\overline{\int}_I f \; d \alpha\bigg)                                                                       \\
    \implies & \overline{\int}_I (cf) \; d \alpha \leq c\bigg(\int_I f \; d \alpha\bigg)                                                         &  & \by{i:ac:11.8.8}
  \end{align*}
  and
  \begin{align*}
             & \underline{\int}_I (cf) \; d \alpha \geq p.c. \int_I (c f_L) \; d \alpha                                                            &  & \by{i:ac:11.8.8}    \\
    \implies & \underline{\int}_I (cf) \; d \alpha \geq c \bigg(p.c. \int_I f_L \; d \alpha\bigg)                                                  &  & \by{i:ac:11.8.7}[b] \\
    \implies & \dfrac{1}{c} \bigg(\underline{\int}_I (cf) \; d \alpha\bigg) \leq p.c. \int_I f_L \; d \alpha                                                                \\
    \implies & \dfrac{1}{c} \bigg(\underline{\int}_I (cf) \; d \alpha\bigg) \leq p.c. \int_I f_L \; d \alpha \leq \underline{\int}_I f \; d \alpha &  & \by{i:ac:11.8.8}    \\
    \implies & \underline{\int}_I (cf) \; d \alpha \geq c\bigg(\underline{\int}_I f \; d \alpha\bigg)                                                                       \\
    \implies & \underline{\int}_I (cf) \; d \alpha \geq c\bigg(\int_I f \; d \alpha\bigg).                                                         &  & \by{i:ac:11.8.8}
  \end{align*}
  By \cref{i:ac:11.8.9} we have
  \[
    c\bigg(\int_I f \; d \alpha\bigg) \leq \underline{\int}_I (cf) \; d \alpha \leq \overline{\int}_I (cf) \; d \alpha \leq c\bigg(\int_I f \; d \alpha\bigg)
  \]
  and thus by \cref{i:ac:11.8.8} we have
  \[
    \int_I (cf) \; d \alpha = \underline{\int}_I (cf) \; d \alpha = \overline{\int}_I (cf) \; d \alpha = c\bigg(\int_I f \; d \alpha\bigg).
  \]
  We conclude that \(\forall c \in \R\), \(\int_I (cf) \; d \alpha = c (\int_I f \; d \alpha)\).
\end{proof}

\begin{proof}{(c)}
  We have
  \begin{align*}
    \int_I f \; d \alpha - \int_I g \; d \alpha & = \int_I f \; d \alpha + \int_I (-g) \; d \alpha &  & \by{i:ac:11.8.11}[b] \\
                                                & = \int_I \big(f + (-g) \; d \alpha\big)          &  & \by{i:ac:11.8.11}[a] \\
                                                & = \int_I (f - g) \; d \alpha.                    &  & \by{i:9.2.1}
  \end{align*}
\end{proof}

\begin{proof}{(d)}
  Let \(f_U : I \to \R\) be a piecewise constant function on \(I\) which majorizes \(f\).
  \(f_U\) is well-defined since by \cref{i:ac:11.8.8} \(f\) is a bounded function on a bounded interval \(I\).
  Since \(0 \leq f(x) \leq f_U(x)\) for every \(x \in I\), we have
  \begin{align*}
             & 0 \leq p.c. \int_I f_U \; d \alpha     &  & \by{i:ac:11.8.7}[d] \\
    \implies & 0 \leq \overline{\int}_I f \; d \alpha &  & \by{i:ac:11.8.8}    \\
    \implies & 0 \leq \int_I f \; d \alpha.           &  & \by{i:ac:11.8.8}
  \end{align*}
\end{proof}

\begin{proof}{(e)}
  We have \(f(x) - g(x) \geq 0\) for every \(x \in I\) and by \cref{i:ac:11.8.11}(c) \(f - g\) is Riemann-Stieltjes integrable on \(I\) with respect to \(\alpha\).
  Thus
  \begin{align*}
             & \int_I (f - g) \; d \alpha \geq 0                  &  & \by{i:ac:11.8.11}[d] \\
    \implies & \int_I f \; d \alpha - \int_I g \; d \alpha \geq 0 &  & \by{i:ac:11.8.11}[c] \\
    \implies & \int_I f \; d \alpha \geq \int_I g \; d \alpha.
  \end{align*}
\end{proof}

\begin{proof}{(f)}
  We have
  \begin{align*}
    \int_I f \; d \alpha & = p.c. \int_I f \; d \alpha &  & \by{i:ac:11.8.10}   \\
                         & = c \alpha[I].              &  & \by{i:ac:11.8.7}[f]
  \end{align*}
\end{proof}

\begin{proof}{(g)}
  Let \(f_U : I \to \R\) be a piecewise constant function on \(I\) which majorizes \(f\).
  Let \(f_L : I \to \R\) be a piecewise constant function on \(I\) which minorizes \(f\).
  \(f_U, f_L\) are well-defined since by \cref{i:ac:11.8.8} \(f\) is a bounded function on a bounded interval \(I\).
  Then we have
  \[
    p.c. \int_I f_L \leq \underline{\int}_I f = \int_I f = \overline{\int}_I f \leq p.c. \int_I f_U.
  \]
  By \cref{i:ac:11.8.7}(h) we know that \(f_U|_J : J \to \R, f_L|_J : J \to \R\) are piecewise constant function on \(J\) and \(f_U|_K : K \to \R\), \(f_L|_K : K \to \R\) are piecewise constant functions on \(K\).
  By \cref{i:11.3.1} we know that \(f_U|_J\) majorizes \(f|_J\) and \(f_L|_J\) minorizes \(f|_J\), similarly \(f_U|_K\) majorizes \(f|_K\) and \(f_L|_K\) minorizes \(f|_K\).
  Thus \(f|_J, f|_K\) are bounded functions on bounded intervals \(J, K\), respectively.
  So \(\overline{\int}_J f|_J \; d \alpha\), \(\overline{\int}_K f|_K \; d \alpha\), \(\underline{\int}_J f|_J \; d \alpha\), \(\underline{\int}_K f|_K \; d \alpha\) are well-defined.
  Then we have
  \begin{align*}
             & \overline{\int}_J f|_J \; d \alpha + \overline{\int}_K f|_K \; d \alpha                                                               \\
             & \quad \leq p.c. \int_J f_U|_J \; d \alpha + p.c. \int_K f_U|_K \; d \alpha                                   &  & \by{i:ac:11.8.8}    \\
    \implies & \overline{\int}_J f|_J \; d \alpha + \overline{\int}_K f|_K \; d \alpha \leq p.c. \int_I f_U \; d \alpha     &  & \by{i:ac:11.8.7}[h] \\
    \implies & \overline{\int}_J f|_J \; d \alpha + \overline{\int}_K f|_K \; d \alpha \leq \overline{\int}_I f \; d \alpha &  & \by{i:ac:11.8.8}    \\
    \implies & \overline{\int}_J f|_J \; d \alpha + \overline{\int}_K f|_K \; d \alpha \leq \int_I f \; d \alpha            &  & \by{i:ac:11.8.8}
  \end{align*}
  and
  \begin{align*}
             & \underline{\int}_J f|_J \; d \alpha + \underline{\int}_K f|_K \; d \alpha                                                                \\
             & \quad \geq p.c. \int_J f_L|_J \; d \alpha + p.c. \int_K f_L|_K \; d \alpha                                      &  & \by{i:ac:11.8.8}    \\
    \implies & \underline{\int}_J f|_J \; d \alpha + \underline{\int}_K f|_K \; d \alpha \geq p.c. \int_I f_L \; d \alpha      &  & \by{i:ac:11.8.7}[h] \\
    \implies & \underline{\int}_J f|_J \; d \alpha + \underline{\int}_K f|_K \; d \alpha \geq \underline{\int}_I f \; d \alpha &  & \by{i:ac:11.8.8}    \\
    \implies & \underline{\int}_J f|_J \; d \alpha + \underline{\int}_K f|_K \; d \alpha \geq \int_I f \; d \alpha.            &  & \by{i:ac:11.8.8}
  \end{align*}
  By \cref{i:ac:11.8.9} we have
  \[
    \int_I f \; d \alpha \leq \underline{\int}_J f|_J \; d \alpha + \underline{\int}_K f|_K \; d \alpha \leq \overline{\int}_J f|_J \; d \alpha + \overline{\int}_K f|_K \; d \alpha \leq \int_I f \; d \alpha
  \]
  and thus we have
  \[
    \underline{\int}_J f|_J \; d \alpha + \underline{\int}_K f|_K \; d \alpha = \overline{\int}_J f|_J \; d \alpha + \overline{\int}_J f|_K \; d \alpha = \int_I f \; d \alpha.
  \]
  Since
  \begin{align*}
             & \underline{\int}_J f|_J \; d \alpha + \underline{\int}_K f|_K \; d \alpha                                     \\
             & \quad = \overline{\int}_J f|_J \; d \alpha + \overline{\int}_J f|_K \; d \alpha                               \\
    \implies & 0 \geq \underline{\int}_J f|_J \; d \alpha - \overline{\int}_J f|_J \; d \alpha                               \\
             & \quad = \overline{\int}_J f|_K \; d \alpha - \underline{\int}_K f|_K \; d \alpha \geq 0 &  & \by{i:ac:11.8.9} \\
    \implies & \underline{\int}_J f|_J \; d \alpha - \overline{\int}_J f|_J \; d \alpha                                      \\
             & \quad = \overline{\int}_J f|_K \; d \alpha - \underline{\int}_K f|_K \; d \alpha = 0,
  \end{align*}
  by \cref{i:ac:11.8.8} we have
  \begin{align*}
     & \int_J f|_J \; d \alpha = \underline{\int}_J f|_J \; d \alpha = \overline{\int}_J f|_J \; d \alpha, \\
     & \int_K f|_K \; d \alpha = \underline{\int}_K f|_K \; d \alpha = \overline{\int}_K f|_K \; d \alpha, \\
     & \int_J f|_J \; d \alpha + \int_K f|_K \; d \alpha = \int_I f \; d \alpha.
  \end{align*}
\end{proof}

\begin{ac}[Laws of Riemann-Stieltjes integration]\label{i:ac:11.8.12}
  Let \(I\) be a bounded interval, let \(\alpha : X \to \R\) be a monotone increasing function defined on some interval \(X\) which contains \(I\).
  Let \(f : I \to \R\) be a bounded function, and let \(\mathbf{P}\) be a partition of \(I\).
  We define the \emph{upper Riemann-Stieltjes sum} \(U(f, \alpha, \mathbf{P})\) and the \emph{lower Riemann-Stieltjes sum} \(L(f, \alpha, \mathbf{P})\) by
  \[
    U(f, \alpha, \mathbf{P}) \coloneqq \sum_{J \in \mathbf{P} : J \neq \emptyset} \big(\sup_{x \in J} f(x)\big) \alpha[J]
  \]
  and
  \[
    L(f, \alpha, \mathbf{P}) \coloneqq \sum_{J \in \mathbf{P} : J \neq \emptyset} \big(\inf_{x \in J} f(x)\big) \alpha[J].
  \]
\end{ac}

\begin{ac}\label{i:ac:11.8.13}
  Let \(I\) be a bounded interval, let \(\alpha : X \to \R\) be a monotone increasing function defined on some interval \(X\) which contains \(I\).
  Let \(f : I \to \R\) be a bounded function, and let \(g\) be a function which majorizes \(f\) and which is piecewise constant with respect to some partition \(\mathbf{P}\) of \(I\).
  Then
  \[
    p.c. \int_I g \; d \alpha \geq U(f, \alpha, \mathbf{P}).
  \]
  Similarly, if \(h\) is a function which minorizes \(f\) and is piecewise constant with respect to \(\mathbf{P}\), then
  \[
    p.c. \int_I h \; d \alpha \leq L(f, \alpha, \mathbf{P}).
  \]
\end{ac}

\begin{proof}
  Since \(g\) majorizes \(f\) and \(h\) minorizes \(f\), by \cref{i:11.3.1} we have \(h(x) \leq f(x) \leq g(x)\) for every \(x \in I\).
  Since \(\mathbf{P}\) is a partition of \(I\), by \cref{i:11.1.10} for every \(J \in \mathbf{P}\), we have \(h(x) \leq f(x) \leq g(x)\) for all \(x \in J\).
  In particular, when \(J \neq \emptyset\) we have
  \[
    h(x) \leq \inf_{x \in J} f(x) \leq f(x) \leq \sup_{x \in J} f(x) \leq g(x)
  \]
  for every \(x \in J\).
  Let \(c_{g|_J}, c_{h|_J}\) be constant values of \(g|_J, h|_J\), respectively.
  Then we have
  \begin{align*}
    U(f, \alpha, \mathbf{P}) & = \sum_{J \in \mathbf{P} : J \neq \emptyset} \big(\sup_{x \in J} f(x)\big) \alpha[J] &  & \by{i:ac:11.8.12}  \\
                             & \leq \sum_{J \in \mathbf{P} : J \neq \emptyset} c_{g|_J} \alpha[J]                   &  & \by{i:7.1.11}[h]   \\
                             & = \sum_{J \in \mathbf{P}} c_{g|_J} \alpha[J]                                         &  & \by{i:7.1.11}[a,e] \\
                             & = p.c. \int_{[\mathbf{P}]} g \; d \alpha                                             &  & \by{i:11.8.5}      \\
                             & = p.c. \int_I g \; d \alpha                                                          &  & \by{i:ac:11.8.5}
  \end{align*}
  and
  \begin{align*}
    L(f, \alpha, \mathbf{P}) & = \sum_{J \in \mathbf{P} : J \neq \emptyset} \big(\inf_{x \in J} f(x)\big) \alpha[J] &  & \by{i:ac:11.8.12}  \\
                             & \geq \sum_{J \in \mathbf{P} : J \neq \emptyset} c_{h|_J} \alpha[J]                   &  & \by{i:7.1.11}[h]   \\
                             & = \sum_{J \in \mathbf{P}} c_{h|_J} \alpha[J]                                         &  & \by{i:7.1.11}[a,e] \\
                             & = p.c. \int_{[\mathbf{P}]} h \; d \alpha                                             &  & \by{i:11.8.5}      \\
                             & = p.c. \int_I h \; d \alpha.                                                         &  & \by{i:ac:11.8.5}
  \end{align*}
\end{proof}

\begin{ac}\label{i:ac:11.8.14}
  Let \(I\) be a bounded interval, let \(\alpha : X \to \R\) be a monotone increasing function defined on some interval \(X\) which contains \(I\).
  Let \(f : I \to \R\) be a bounded function.
  Then
  \[
    \overline{\int}_I f \; d \alpha = \inf\set{U(f, \alpha, \mathbf{P}) : \mathbf{P} \text{ is a partition of } I}
  \]
  and
  \[
    \underline{\int}_I f \; d \alpha = \sup\set{L(f, \alpha, \mathbf{P}) : \mathbf{P} \text{ is a partition of } I}.
  \]
\end{ac}

\begin{proof}
  Let \(g\) be a function which majorizes \(f\) and which is piecewise constant with respect to some partition \(\mathbf{P}_g\) of \(I\).
  Let \(h\) be a function which minorizes \(f\) and which is piecewise constant with respect to some partition \(\mathbf{P}_h\) of \(I\).
  Both functions are well defined since \(f\) is bounded function on a bounded interval \(I\).
  By \cref{i:ac:11.8.13} we have
  \[
    \inf\set{U(f, \alpha, \mathbf{P}) : \mathbf{P} \text{ is a partition of } I} \leq U(f, \alpha, \mathbf{P}_g) \leq p.c. \int_I g \; d \alpha
  \]
  and
  \[
    \sup\set{L(f, \alpha, \mathbf{P}) : \mathbf{P} \text{ is a partition of } I} \geq L(f, \alpha, \mathbf{P}_h) \geq p.c. \int_I h \; d \alpha.
  \]
  Since \(g, h\) were arbitrary, by \cref{i:ac:11.8.8} we have
  \[
    \inf\set{U(f, \alpha, \mathbf{P}) : \mathbf{P} \text{ is a partition of } I} \leq \overline{\int}_I f \; d \alpha
  \]
  and
  \[
    \sup\set{L(f, \alpha, \mathbf{P}) : \mathbf{P} \text{ is a partition of } I} \geq \underline{\int}_I f \; d \alpha.
  \]

  Let \(\mathbf{P}\) be a partition of \(I\).
  Let \(G : I \to \R\) be a function where \(G(x) = \sup_{x \in J} f(x)\) for all \(J \in \mathbf{P}\).
  Let \(H : I \to \R\) be a function where \(H(x) = \inf_{x \in J} f(x)\) for all \(J \in \mathbf{P}\).
  By \cref{i:11.2.3} we know that \(G, H\) are piecewise constant functions with respect to \(\mathbf{P}\).
  Thus we have
  \begin{align*}
    U(f, \alpha, \mathbf{P}) & = \sum_{J \in \mathbf{P} : J \neq \emptyset} \big(\sup_{x \in J} f(x)\big) \alpha[J] &  & \by{i:ac:11.8.12}  \\
                             & = \sum_{J \in \mathbf{P}} \big(\sup_{x \in J} f(x)\big) \alpha[J]                    &  & \by{i:7.1.11}[a,e] \\
                             & = p.c. \int_{[\mathbf{P}]} G \; d \alpha                                             &  & \by{i:11.8.5}      \\
                             & = p.c. \int_I G \; d \alpha                                                          &  & \by{i:ac:11.8.5}
  \end{align*}
  and
  \begin{align*}
    L(f, \alpha, \mathbf{P}) & = \sum_{J \in \mathbf{P} : J \neq \emptyset} \big(\inf_{x \in J} f(x)\big) \alpha[J] &  & \by{i:ac:11.8.12}  \\
                             & = \sum_{J \in \mathbf{P}} \big(\inf_{x \in J} f(x)\big) \alpha[J]                    &  & \by{i:7.1.11}[a,e] \\
                             & = p.c. \int_{[\mathbf{P}]} H \; d \alpha                                             &  & \by{i:11.8.5}      \\
                             & = p.c. \int_I H \; d \alpha.                                                         &  & \by{i:ac:11.8.5}
  \end{align*}
  By \cref{i:ac:11.8.8} we have
  \[
    \overline{\int}_I f \; d \alpha \leq p.c. \int_I G \; d \alpha = U(f, \alpha, \mathbf{P})
  \]
  and
  \[
    \underline{\int}_I f \; d \alpha \geq p.c. \int_I H \; d \alpha = L(f, \alpha, \mathbf{P}).
  \]
  Since \(\mathbf{P}\) was arbitrary, we have
  \[
    \overline{\int}_I f \; d \alpha \leq \inf\set{U(f, \alpha, \mathbf{P}) : \mathbf{P} \text{ is a partition of } I} \leq U(f, \alpha, \mathbf{P})
  \]
  and
  \[
    \underline{\int}_I f \; d \alpha \geq \sup\set{L(f, \alpha, \mathbf{P}) : \mathbf{P} \text{ is a partition of } I} \leq L(f, \alpha, \mathbf{P}).
  \]
  Combine all results above we have
  \[
    \overline{\int}_I f \; d \alpha = \inf\set{U(f, \alpha, \mathbf{P}) : \mathbf{P} \text{ is a partition of } I}
  \]
  and
  \[
    \underline{\int}_I f \; d \alpha = \sup\set{L(f, \alpha, \mathbf{P}) : \mathbf{P} \text{ is a partition of } I}.
  \]
\end{proof}

\begin{ac}\label{i:ac:11.8.15}
  Let \(I\) be a bounded interval, let \(\alpha : X \to \R\) be a monotone increasing function defined on some interval \(X\) which contains \(I\).
  Let \(f\) be a function which is uniformly continuous on \(I\).
  Then \(f\) is Riemann-Stieltjes integrable on \(I\) with respect to \(\alpha\).
\end{ac}

\begin{proof}
  From \cref{i:9.9.15} we see that \(f\) is bounded.
  By \cref{i:ac:11.8.8} we have to show that \(\underline{\int}_I f \; d \alpha = \overline{\int}_I f \; d \alpha\).

  If \(I\) is a point or the empty set then the theorem is trivial, so let us assume that \(I\) is one of the four intervals \([a, b]\), \((a, b)\), \((a, b]\), or \([a, b)\) for some real numbers \(a < b\).

  Let \(\varepsilon > 0\) be arbitrary.
  By uniform continuity, there exists a \(\delta > 0\) such that \(\abs{f(x) - f(y)} < \varepsilon\) whenever \(x, y \in I\) are such that \(\abs{x - y} < \delta\).
  By the Archimedean principle, there exists an integer \(N > 0\) such that \((b - a) / N < \delta\) and
  \[
    \dfrac{\lim_{x \to b^- ; x \in X} \alpha(x) - \lim_{x \to a^+ ; x \in X} \alpha(x)}{N} < \delta.
  \]

  Note that we can partition \(I\) into \(N\) intervals \(J_1, \dots, J_N\), each of length \((b - a) / N\).
  By \cref{i:ac:11.8.14}, we thus have
  \[
    \overline{\int}_I f \; d \alpha \leq \sum_{k = 1}^N \big(\sup_{x \in J_k} f(x)\big) \alpha[J_k]
  \]
  and
  \[
    \underline{\int}_I f \; d \alpha \geq \sum_{k = 1}^N \big(\inf_{x \in J_k} f(x)\big) \alpha[J_k]
  \]
  so in particular
  \[
    \overline{\int}_I f \; d \alpha - \underline{\int}_I f \; d \alpha \leq \sum_{k = 1}^N \big(\sup_{x \in J_k} f(x) - \inf_{x \in J_k} f(x)\big) \alpha[J_k].
  \]
  However, we have \(\abs{f(x) - f(y)} < \varepsilon\) for all \(x, y \in J_k\), since
  \begin{align*}
    \alpha[J_k] & = \dfrac{\lim_{x \to \sup(J_K)^- ; x \in X} \alpha(x) - \lim_{x \to \inf(J_K)^+ ; x \in X} \alpha(x)}{N}                                                 \\
                & \leq \dfrac{\lim_{x \to b^- ; x \in X} \alpha(x) - \lim_{x \to a^+ ; x \in X} \alpha(x)}{N}              &  & \text{(\(\alpha\) is monotone increasing)} \\
                & < \delta.
  \end{align*}
  In particular we have
  \[
    f(x) < f(y) + \varepsilon \text{ for all } x, y \in J_k.
  \]
  Taking suprema in \(x\), we obtain
  \[
    \sup_{x \in J_k} f(x) \leq f(y) + \varepsilon \text{ for all } y \in J_k,
  \]
  and then taking infima in \(y\) we obtain
  \[
    \sup_{x \in J_k} f(x) \leq \inf_{y \in J_k} f(y) + \varepsilon.
  \]
  Inserting this bound into our previous inequality, we obtain
  \[
    \overline{\int}_I f \; d \alpha - \underline{\int}_I f \; d \alpha \leq \sum_{k = 1}^N \varepsilon \alpha[J_k],
  \]
  but by \cref{i:11.8.4} we thus have
  \[
    \overline{\int}_I f \; d \alpha - \underline{\int}_I f \; d \alpha \leq \varepsilon \alpha[I].
  \]
  But \(\varepsilon > 0\) was arbitrary, while \(\alpha[I]\) is fixed.
  Thus \(\overline{\int}_I f \; d \alpha - \underline{\int}_I f \; d \alpha\) cannot be positive.
  By \cref{i:ac:11.8.8} we thus have that \(f\) is Riemann-Stieltjes integrable on \(I\) with respect to \(\alpha\).
\end{proof}

\exercisesection

\begin{ex}\label{i:ex:11.8.1}
  Prove \cref{i:11.8.4}.
\end{ex}

\begin{proof}
  See \cref{i:11.8.4}.
\end{proof}

\begin{ex}\label{i:ex:11.8.2}
  State and prove a version of \cref{i:11.2.13} for the Riemann-Stieltjes integral.
\end{ex}

\begin{proof}
  See \cref{i:ac:11.8.4}.
\end{proof}

\begin{ex}\label{i:ex:11.8.3}
  State and prove a version of \cref{i:11.2.16} for the Riemann-Stieltjes integral.
\end{ex}

\begin{proof}
  See \cref{i:ac:11.8.7}.
\end{proof}

\begin{ex}\label{i:ex:11.8.4}
  State and prove a version of \cref{i:11.5.1} for the Riemann-Stieltjes integral.
\end{ex}

\begin{proof}
  See \cref{i:ac:11.8.15}.
\end{proof}

\begin{ex}\label{i:ex:11.8.5}
  Let \(\text{sgn} : \R \to \R\) be the signum function
  \[
    \text{sgn}(x) = \begin{dcases}
      1  & \text{when } x > 0  \\
      0  & \text{when } x = 0  \\
      -1 & \text{when } x < 0.
    \end{dcases}
  \]
  Let \(f : [-1, 1] \to \R\) be a continuous function.
  Show that \(f\) is Riemann-Stieltjes integrable with respect to \(\text{sgn}\), and that
  \[
    \int_{[-1, 1]} f \; d \, \text{sgn} = 2f(0).
  \]
\end{ex}

\begin{proof}
  We first show that \(f\) is Riemann-Stieltjes integrable on \([-1, 1]\) with respect to \(\text{sgn}\).
  By \cref{i:9.9.16} \(f\) is uniformly continuous, and thus by \cref{i:9.9.15} \(f\) is bounded.
  Since \(\text{sgn}\) is monotone increasing, by \cref{i:ac:11.8.15} we know that \(f\) is Riemann-Stieltjes integrable on \([-1, 1]\) with respect to \(\text{sgn}\).

  Now we show that \(\int_{[-1, 1]} f \; d \, \text{sgn} = 2f(0)\).
  Since \(f\) is continuous, by \cref{i:9.4.7} we have
  \begin{align*}
             & \forall \varepsilon \in \R^+, \exists \delta \in \R^+ : \forall x \in [-1, 1], \abs{x - 0} \leq \delta \\
    \implies & \abs{f(x) - f(0)} \leq \varepsilon                                                                     \\
    \implies & f(0) - \varepsilon \leq f(x) \leq f(0) + \varepsilon.
  \end{align*}
  In particular, we can choose some \(\delta \leq 1\) such that
  \begin{align*}
     & \forall \varepsilon \in \R^+, \exists \delta \in \R^+ :                                                              \\
     & \forall x \in [-1, 1], \abs{x - 0} \leq \delta \leq 1 \implies f(0) - \varepsilon \leq f(x) \leq f(0) + \varepsilon.
  \end{align*}
  Since \(f\) is bounded, \(\exists M \in \R^+\) such that \(\abs{f(x)} \leq M\) for all \(x \in [-1, 1]\).
  Let \(f_U : [-1, 1] \to \R\) be the function
  \[
    f_U(x) = \begin{dcases}
      f(0) + \varepsilon & \text{if } x \in [-\delta, \delta]                   \\
      M                  & \text{if } x \in [-1, 1] \setminus [-\delta, \delta]
    \end{dcases}
  \]
  and let \(f_L : [-1, 1] \to \R\) be the function
  \[
    f_L(x) = \begin{dcases}
      f(0) - \varepsilon & \text{if } x \in [-\delta, \delta]                   \\
      -M                 & \text{if } x \in [-1, 1] \setminus [-\delta, \delta]
    \end{dcases}
  \]
  Clearly \(f_U, f_L\) are piecewise constant on \([-1, 1]\), \(f_U\) majorizes \(f\) and \(f_L\) minorizes \(f\).
  Then we have
  \begin{align*}
    \overline{\int}_{[-1, 1]} f \; d \, \text{sgn} & \leq p.c. \int f_U \; d \, \text{sgn}                                   &  & \by{i:ac:11.8.8} \\
                                                   & = M \big(\text{sgn}(-\delta) - \text{sgn}(-1)\big)                      &  & \by{i:11.8.5}    \\
                                                   & \quad + (f(0) + \varepsilon) (\text{sgn}(\delta) - \text{sgn}(-\delta))                       \\
                                                   & \quad + M \big(\text{sgn}(1) - \text{sgn}(\delta)\big)                                        \\
                                                   & = 2\big(f(0) + \varepsilon\big)
  \end{align*}
  and
  \begin{align*}
    \underline{\int}_{[-1, 1]} f \; d \, \text{sgn} & \geq p.c. \int f_L \; d \, \text{sgn}                                   &  & \by{i:ac:11.8.8} \\
                                                    & = M \big(\text{sgn}(-\delta) - \text{sgn}(-1)\big)                      &  & \by{i:11.8.5}    \\
                                                    & \quad + (f(0) - \varepsilon) (\text{sgn}(\delta) - \text{sgn}(-\delta))                       \\
                                                    & \quad + M \big(\text{sgn}(1) - \text{sgn}(\delta)\big)                                        \\
                                                    & = 2\big(f(0) - \varepsilon\big).
  \end{align*}
  Combining results above we have
  \[
    2(f(0) - \varepsilon) \leq \underline{\int}_{[-1, 1]} f \; d \text{sgn} = \int_{[-1, 1]} f \; d \text{sgn} = \overline{\int}_{[-1, 1]} f \; d \text{sgn} \leq 2(f(0) + \varepsilon).
  \]
  Since \(\varepsilon\) was arbitrary, we thus have \(\int_{[-1, 1]} f \; d \text{sgn} = 2f(0)\).
\end{proof}
