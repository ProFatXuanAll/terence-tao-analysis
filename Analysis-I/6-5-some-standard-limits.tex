\section{Some standard limits}\label{i:sec:6.5}

\begin{ac}\label{i:ac:6.5.1}
  We have
  \[
    \lim_{n \to \infty} c = c
  \]
  for any constant \(c\).
\end{ac}

\begin{proof}
  Let \((a_n)_{n = 1}^\infty\) be a constant sequence where \(a_n = c\) for all \(n \geq 1\), and let \(N \in \N\).
  Then \(\forall \varepsilon \in \R^+\), \(\exists N \geq 1\) such that for every \(n \geq N\),
  \[
    \abs{a_n - c} = \abs{c - c} = 0 \leq \varepsilon.
  \]
  Thus by \cref{i:6.1.8} we have \(\lim_{n \to \infty} a_n = \lim_{n \to \infty} c = c\).
\end{proof}

\begin{cor}\label{i:6.5.1}
  We have \(\lim_{n \to \infty} 1 / n^{1 / k} = 0\) for every integer \(k \geq 1\).
\end{cor}

\begin{proof}
  From \cref{i:5.6.6} we know that \(1 / n^{1 / k}\) is a decreasing function of \(n\), while being bounded below by \(0\).
  By \cref{i:ac:6.3.1} (for decreasing sequences instead of increasing sequences) we thus know that this sequence converges to some limit \(L \geq 0\):
  \[
    L = \lim_{n \to \infty} 1 / n^{1 / k}.
  \]
  Raising this to the \(k^{th}\) power and using the limit laws (or more precisely, \cref{i:6.1.19}(b) and induction), we obtain
  \[
    L^k = \lim_{n \to \infty} 1 / n.
  \]
  By \cref{i:6.1.11} we thus have \(L^k = 0\);
  but this means that \(L\) cannot be positive (else \(L^k\) would be positive), so \(L = 0\), and we are done.
\end{proof}

\begin{lem}\label{i:6.5.2}
  Let \(x\) be a real number.
  Then the limit \(\lim_{n \to \infty} x^n\) exists and is equal to zero when \(\abs{x} < 1\), exists and is equal to \(1\) when \(x = 1\), and diverges when \(x = -1\) or when \(\abs{x} > 1\).
\end{lem}

\begin{proof}
  We first show that if \(\abs{x} < 1\), then \(\lim_{n \to \infty} x^n = 0\).
  Since \(0 \leq \abs{x} < 1\), we have
  \begin{align*}
             & 0 \leq \abs{x} < 1                                                       \\
    \implies & \lim_{n \to \infty} \abs{x}^n = 0                     &  & \by{i:6.3.10} \\
    \implies & (-1)\lim_{n \to \infty} \abs{x}^n = (-1) \times 0 = 0                    \\
    \implies & \lim_{n \to \infty} -\abs{x}^n = 0.                   &  & \by{i:6.1.19}
  \end{align*}
  So \(\lim_{n \to \infty} -\abs{x}^n = \lim_{n \to \infty} \abs{x}^n\).
  And because \(-\abs{x}^n \leq x^n \leq \abs{x}^n\), by Squeeze test (\cref{i:6.4.14}) we have \(\lim_{n \to \infty} x^n = 0\).

  Next we show that if \(x = 1\), then \(\lim_{n \to \infty} x^n = 1\).
  This is done by \cref{i:ac:6.5.1}.

  Finally we show that if \(x = -1\) or \(\abs{x} > 1\), then \(\lim_{n \to \infty} x^n\) does not exist.
  If \(x = -1\), then we have sequence \(-1, 1, -1, 1, \dots\), which is not eventually \(1\)-steady for any \(n \geq 1\) and thus does not converge to any value.
  If \(\abs{x} > 1\), then we can divide into two cases:
  \begin{itemize}
    \item If \(x > 1\), then by \cref{i:ex:6.3.4} \(\lim_{n \to \infty} x^n\) does not exist.
    \item If \(x < -1\), then \(\forall n \geq 1\) we have
          \begin{align*}
            \abs{x^{n + 1} - x^n} & = \abs{x^n(x - 1)}     \\
                                  & = \abs{x^n}\abs{x - 1} \\
                                  & > \abs{x^n}\abs{-2}    \\
                                  & = 2\abs{x^n}           \\
                                  & > 2.
          \end{align*}
          This means \(x\) is not a Cauchy sequence, so by \cref{i:6.4.18} \(\lim_{n \to \infty} x^n\) does not exist.
  \end{itemize}
  From all cases above we conclude that if \(\abs{x} > 1\) then \(\lim_{n \to \infty} x^n\) does not exist.
\end{proof}

\begin{lem}\label{i:6.5.3}
  For any \(x > 0\), we have \(\lim_{n \to \infty} x^{1 / n} = 1\).
\end{lem}

\begin{proof}
  We first show that \(\forall \varepsilon, M \in \R^+\), \(\exists n \geq 1\) such that \(M^{1 / n} \leq 1 + \varepsilon\).
  Since \(1 / (1 + \varepsilon) < 1\), by \cref{i:6.5.2} we have \(\lim_{n \to \infty} 1 / (1 + \varepsilon)^n = 0\).
  Let \(a_n = 1 / (1 + \varepsilon)^n\).
  Then \(\inf(a_n)_{n = 1}^\infty = 0\).
  Since \(1 / M > 0\), we have \(1 / M > \inf(a_n)_{n = 1}^\infty\).
  By \cref{i:6.3.7} \(\exists n \geq 1\) such that \(a_n = 1 / (1 + \varepsilon)^n < 1 / M\).
  Thus we have \(M < (1 + \varepsilon)^n\), and by \cref{i:5.6.9}(d) \(M^{1 / n} < 1 + \varepsilon\).

  Now we show that \(\lim_{n \to \infty} x^{1 / n} = 1\).
  We split into two cases:
  \begin{itemize}
    \item If \(x \geq 1\), then by the proof above we have \(\forall \varepsilon \in \R^+\), \(\exists n \geq 1\) such that \(x^{1 / n} < 1 + \varepsilon\).
          Thus
          \begin{align*}
                     & x \geq 1                                                                 \\
            \implies & x^{1 / n} \geq 1^{1 / n} = 1                      &  & \by{i:5.6.6}[d,e] \\
            \implies & \abs{x^{1 / n} - 1} = x^{1 / n} - 1 < \varepsilon
          \end{align*}
          and by \cref{i:6.1.8} \(\lim_{n \to \infty} x^{1 / n} = 1\).
    \item If \(x < 1\), then \(1 / x > 1\).
          So from the proof above we have \(\lim_{n \to \infty} x^{-1 / n} = 1\).
          By \cref{i:6.1.19}(e) we have \(\lim_{n \to \infty} x^{-1 / n} = (\lim_{n \to \infty} x^{1 / n})^{-1} = 1^{-1} = 1\).
  \end{itemize}
  From all cases above we conclude that \(\lim_{n \to \infty} x^{1 / n} = 1\).
\end{proof}

\exercisesection

\begin{ex}\label{i:ex:6.5.1}
  Show that \(\lim_{n \to \infty} 1 / n^q = 0\) for any rational \(q > 0\).
  Conclude that the limit \(\lim_{n \to \infty} n^q\) does not exist.
\end{ex}

\begin{proof}
  We firs show that \(\lim_{n \to \infty} 1 / n^q = 0\) for every \(q \in \Q^+\).
  Let \(q = a / b\) where \(a, b \in \Z^+\).
  Then we have
  \begin{align*}
             & \dfrac{1}{n^q} = \dfrac{1}{n^{a / b}} = \bigg(\dfrac{1}{n^{1 / b}}\bigg)^a                       \\
    \implies & \lim_{n \to \infty} 1 / n^{1 / b} = 0                                      &  & \by{i:6.5.1}     \\
    \implies & \lim_{n \to \infty} (1 / n^{1 / b})^a = 0                                  &  & \by{i:6.1.19}[b] \\
    \implies & \lim_{n \to \infty} 1 / n^q = 0.                                           &  & \by{i:5.6.7}
  \end{align*}

  Now we show that \(\lim_{n \to \infty} n^q\) does not exist.
  Suppose for sake of contradiction that \(\lim_{n \to \infty} n^q\) exists and equals to \(y\).
  Since \(n \geq 1\), we have \(n^q \geq 1\), so \(y \geq 1\).
  Then we have
  \begin{align*}
             & (\lim_{n \to \infty} n^q)^{-1} = y^{-1}     &  & \by{i:6.1.19}[e]        \\
    \implies & \lim_{n \to \infty} (n^q)^{-1} = y^{-1}     &  & \by{i:6.1.19}[e]        \\
    \implies & \lim_{n \to \infty} \dfrac{1}{n^q} = y^{-1}                              \\
    \implies & y^{-1} = 0.                                 &  & \text{(by proof above)}
  \end{align*}
  But this means \(y = 1 / 0\), which means such \(y\) does not exist, a contradiction.
  Thus \(\lim_{n \to \infty} n^q\) does not exist.
\end{proof}

\begin{ex}\label{i:ex:6.5.2}
  Prove \cref{i:6.5.2}.
\end{ex}

\begin{proof}
  See \cref{i:6.5.2}.
\end{proof}

\begin{ex}\label{i:ex:6.5.3}
  Prove \cref{i:6.5.3}.
\end{ex}

\begin{proof}
  See \cref{i:6.5.3}.
\end{proof}
