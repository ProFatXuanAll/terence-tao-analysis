\section{Subsets of the real line}\label{sec 9.1}

\begin{definition}[Intervals]\label{9.1.1}
    Let \(a, b \in \mathbf{R}^*\) be extended real numbers.
    We define the \emph{closed interval} \([a, b]\) by
    \[
        [a, b] \coloneqq \{x \in \mathbf{R}^* : a \leq x \leq b\},
    \]
    the \emph{half-open intervals} \([a, b)\) and \((a, b]\) by
    \[
        [a, b) \coloneqq \{x \in \mathbf{R}^* : a \leq x < b\}; (a, b] \coloneqq \{x \in \mathbf{R}^* : a < x \leq b\},
    \]
    and the \emph{open interval} \((a, b)\) by
    \[
        (a, b) \coloneqq \{x \in \mathbf{R}^* : a < x < b\}.
    \]
    We call \(a\) the \emph{left endpoint} of these intervals, and \(b\) the \emph{right endpoint}.
\end{definition}

\begin{remark}\label{9.1.2}
    Once again, we are overloading the parenthesis notation;
    for instance, we are now using \((2, 3)\) to denote both an open interval from \(2\) to \(3\), as well as an ordered pair in the Cartesian plane \(\mathbf{R}^2 \coloneqq \mathbf{R} \times \mathbf{R}\).
    This can cause some genuine ambiguity, but the reader should still be able to resolve which meaning of the parentheses is intended from context.
    In some texts, this issue is resolved by using reversed brackets instead of parenthesis, thus for instance \([a, b)\) would now be \([a, b[\), \((a, b]\) would be \(]a, b]\), and \((a, b)\) would be \(]a, b[\).
\end{remark}

\begin{note}
    We sometimes refer to an interval in which one endpoint is infinite (either \(+\infty\) or \(-\infty\)) as \emph{half-infinite} intervals, and intervals in which both endpoints are infinite as \emph{doubly-infinite} intervals;
    all other intervals are \emph{bounded intervals}.
    Thus the positive and negative real axes are half-infinite intervals, and \(\mathbf{R}\) and \(\mathbf{R}^*\) are infinite intervals.
\end{note}