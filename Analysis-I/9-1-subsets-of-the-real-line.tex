\section{Subsets of the real line}\label{sec:9.1}

\begin{defn}[Intervals]\label{9.1.1}
  Let \(a, b \in \R^*\) be extended real numbers.
  We define the \emph{closed interval} \([a, b]\) by
  \[
    [a, b] \coloneqq \set{x \in \R^* : a \leq x \leq b},
  \]
  the \emph{half-open intervals} \([a, b)\) and \((a, b]\) by
  \[
    [a, b) \coloneqq \set{x \in \R^* : a \leq x < b}; (a, b] \coloneqq \set{x \in \R^* : a < x \leq b},
  \]
  and the \emph{open interval} \((a, b)\) by
  \[
    (a, b) \coloneqq \set{x \in \R^* : a < x < b}.
  \]
  We call \(a\) the \emph{left endpoint} of these intervals, and \(b\) the \emph{right endpoint}.
\end{defn}

\begin{rmk}\label{9.1.2}
  Once again, we are overloading the parenthesis notation;
  for instance, we are now using \((2, 3)\) to denote both an open interval from \(2\) to \(3\), as well as an ordered pair in the Cartesian plane \(\R^2 \coloneqq \R \times \R\).
  This can cause some genuine ambiguity, but the reader should still be able to resolve which meaning of the parentheses is intended from context.
  In some texts, this issue is resolved by using reversed brackets instead of parenthesis, thus for instance \([a, b)\) would now be \([a, b[\), \((a, b]\) would be \(]a, b]\), and \((a, b)\) would be \(]a, b[\).
\end{rmk}

\begin{eg}\label{9.1.3}
  The positive real axis \(\set{x \in \R : x > 0}\) is the open interval \((0, +\infty)\), while the non-negative real axis \(\set{x \in \R : x \geq 0}\) is the half-open interval \([0, +\infty)\).
      Similarly, the negative real axis \(\set{x \in \R : x < 0}\) is \((-\infty, 0)\), and the non-positive real axis \(\set{x \in \R : x \leq 0}\) is \((-\infty, 0]\).
  Finally, the real line \(\R\) itself is the open interval \((-\infty, +\infty)\), while the extended real line \(\R^*\) is the closed interval \([-\infty, +\infty]\).
  We sometimes refer to an interval in which one endpoint is infinite (either \(+\infty\) or \(-\infty\)) as \emph{half-infinite} intervals, and intervals in which both endpoints are infinite as \emph{doubly-infinite} intervals;
  all other intervals are \emph{bounded intervals}.
  Thus the positive and negative real axes are half-infinite intervals, and \(\R\) and \(\R^*\) are infinite intervals.
\end{eg}

\begin{eg}\label{9.1.4}
  If \(a > b\) then all four of the intervals \([a, b], [a, b), (a, b]\), and \((a, b)\) are the empty set (by trichotomy, see \cref{5.4.7}(a)).
  If \(a = b\), then the three intervals \([a, b), (a, b]\), and \((a, b)\) are the empty set, while \([a, b]\) is just the singleton set \(\set{a}\).
  Because of this, we call these intervals \emph{degenerate};
  most (but not all) of our analysis will be restricted to non-degenerate intervals.
\end{eg}

\begin{defn}[\(\varepsilon\)-adherent points]\label{9.1.5}
  Let \(X\) be a subset of \(\R\), let \(\varepsilon > 0\), and let \(x \in \R\).
  We say that \(x\) is \emph{\(\varepsilon\)-adherent to \(X\)} iff there exists a \(y \in X\) which is \(\varepsilon\)-close to \(x\)
  (i.e., \(\abs{x - y} \leq \varepsilon\)).
\end{defn}

\begin{rmk}\label{9.1.6}
  The terminology ``\(\varepsilon\)-adherent'' is not standard in the literature.
  However, we shall shortly use it to define the notion of an adherent point, which is standard.
\end{rmk}

\setcounter{thm}{7}
\begin{defn}[Adherent points]\label{9.1.8}
  Let \(X\) be a subset of \(\R\), and let \(x \in \R\).
  We say that \(x\) is an \emph{adherent point} of \(X\) iff it is \(\varepsilon\)-adherent to \(X\) for every \(\varepsilon > 0\).
\end{defn}

\setcounter{thm}{9}
\begin{defn}[Closure]\label{9.1.10}
  Let \(X\) be a subset of \(\R\).
  The \emph{closure} of \(X\), sometimes denoted \(\overline{X}\) is defined to be the set of all the adherent points of \(X\).
\end{defn}

\begin{lem}[Elementary properties of closures]\label{9.1.11}
  Let \(X\) and \(Y\) be arbitrary subsets of \(\R\).
  Then \(X \subseteq \overline{X}\), \(\overline{X \cup Y} = \overline{X} \cup \overline{Y}\), and \(\overline{X \cap Y} \subseteq \overline{X} \cap \overline{Y}\).
  If \(X \subseteq Y\), then \(\overline{X} \subseteq \overline{Y}\).
\end{lem}

\begin{proof}
  We first show that \(X \subseteq \overline{X}\).
  Since
  \begin{align*}
             & \forall x \in X, \abs{x - x} = 0                                            \\
    \implies & \forall \varepsilon \in \R^+, \abs{x - x} \leq \varepsilon                  \\
    \implies & x \in \overline{X},                                        &  & \by{9.1.10}
  \end{align*}
  by \cref{3.1.15} we have \(X \subseteq \overline{X}\).

  Next we show that \(\overline{X \cup Y} = \overline{X} \cup \overline{Y}\).
  Since
  \begin{align*}
             & \forall x \in \overline{X} \cup \overline{Y}                                                                                       \\
    \implies & x \in \overline{X} \lor x \in \overline{Y}                                                             &  & \text{(by \cref{3.4})} \\
    \implies & (\exists a \in X : \abs{x - a} \leq \varepsilon) \lor (\exists a \in Y : \abs{x - a} \leq \varepsilon) &  & \by{9.1.10}            \\
    \implies & \exists a \in X \cup Y : \abs{x - a} \leq \varepsilon                                                  &  & \text{(by \cref{3.4})} \\
    \implies & x \in \overline{X \cup Y}                                                                              &  & \by{9.1.10}
  \end{align*}
  and
  \begin{align*}
             & \forall \varepsilon \in \R^+, \forall x \in \overline{X \cup Y}, \exists a \in X \cup Y : \abs{x - a} \leq \varepsilon &  & \by{9.1.10}            \\
    \implies & \begin{dcases}
                 \exists \varepsilon' \in \R^+ : \forall b \in X, \abs{x - b} > \varepsilon' & \text{if } x \notin \overline{X} \\
                 \exists \varepsilon' \in \R^+ : \forall b \in Y, \abs{x - b} > \varepsilon' & \text{if } x \notin \overline{Y}
               \end{dcases}      &  & \by{9.1.8}                                     \\
    \implies & \begin{dcases}
                 (a \in Y) \land (\abs{x - a} < \varepsilon') & \text{if } x \notin \overline{X} \\
                 (a \in X) \land (\abs{x - a} < \varepsilon') & \text{if } x \notin \overline{Y}
               \end{dcases}                                                                    \\
    \implies & \begin{dcases}
                 x \in \overline{Y} & \text{if } x \notin \overline{X} \\
                 x \in \overline{X} & \text{if } x \notin \overline{Y}
               \end{dcases}                                                               &  & \by{9.1.10}                                                        \\
    \implies & x \in \overline{X} \cup \overline{Y},                                                                                  &  & \text{(by \cref{3.4})}
  \end{align*}
  by \cref{3.1.18} we have \(\overline{X \cup Y} = \overline{X} \cup \overline{Y}\).

  Next we show that \(\overline{X \cap Y} \subseteq \overline{X} \cap \overline{Y}\).
  Since
  \begin{align*}
             & \forall \varepsilon \in \R^+, \forall x \in \overline{X \cap Y}, \exists y \in X \cap Y : \abs{x - y} \leq \varepsilon &  & \by{9.1.10} \\
    \implies & (y \in X) \land (y \in Y)                                                                                              &  & \by{3.1.23} \\
    \implies & (x \in \overline{X}) \land (x \in \overline{Y})                                                                        &  & \by{9.1.10} \\
    \implies & x \in \overline{X} \cap \overline{Y},                                                                                  &  & \by{3.1.23}
  \end{align*}
  by \cref{3.1.15} we have \(\overline{X \cap Y} \subseteq \overline{X} \cap \overline{Y}\).

  Finally we show that \(X \subseteq Y \implies \overline{X} \subseteq \overline{Y}\).
  Suppose that \(X \subseteq Y\).
  Then we have
  \begin{align*}
             & \forall \varepsilon \in \R^+, \forall x \in \overline{X}, \exists y \in X : \abs{x - y} \leq \varepsilon &                 & \by{9.1.10} \\
    \implies & y \in Y                                                                                                  & (X \subseteq Y)               \\
    \implies & x \in \overline{Y}.                                                                                      &                 & \by{9.1.10}
  \end{align*}
  Thus by \cref{3.1.15} we have \(\overline{X} \subseteq \overline{Y}\).
\end{proof}

\begin{lem}[Closures of intervals]\label{9.1.12}
  Let \(a < b\) be real numbers, and let \(I\) be any one of the four intervals \((a, b)\), \((a, b]\), \([a, b)\), or \([a, b]\).
  Then the closure of \(I\) is \([a, b]\).
  Similarly, the closure of \((a, \infty)\) or \([a, \infty)\) is \([a, \infty)\), while the closure of \((-\infty, a)\) or \((-\infty, a]\) is \((-\infty, a]\).
  Finally, the closure of \((-\infty, \infty)\) is \((-\infty, \infty)\).
\end{lem}

\begin{proof}
  First let us show that every element of \([a, b]\) is adherent to \((a, b)\).
  Let \(x \in [a, b]\).
  If \(x \in (a, b)\) then it is definitely adherent to \((a, b)\).
  This is true since \(\forall \varepsilon \in \R^+\) we have \(\abs{x - x} \leq \varepsilon\).
  If \(x = b\) then \(x\) is also adherent to \((a, b)\).
  Otherwise \(\exists \varepsilon \in \R^+\) such that
  \[
    \forall y \in (a, b), \abs{b - y} > \varepsilon.
  \]
  But this means
  \begin{align*}
             & \abs{b - y} = b - y > \varepsilon                        & (y \in (a, b))                                         \\
    \implies & b - \varepsilon > y                                                                                               \\
    \implies & b > b - \varepsilon > y > a                              & (\varepsilon \in \R^+ \land y \in (a, b))              \\
    \implies & b - \varepsilon \in (a, b)                               &                                           & \by{9.1.1} \\
    \implies & \varepsilon < \abs{b - (b - \varepsilon)} = \varepsilon,
  \end{align*}
  a contradiction.
  Thus \(b\) is adherent to \((a, b)\).
  Similarly when \(x = a\).
  Thus every point in \([a, b]\) is adherent to \((a, b)\).

  Now we show that every point \(x\) that is adherent to \((a, b)\) lies in \([a, b]\).
  Suppose for sake of contradiction that \(x\) does not lie in \([a, b]\), then either \(x > b\) or \(x < a\).
  If \(x > b\) then \(x\) is not \((x - b)\)-adherent to \((a, b)\), and is hence not an adherent point to \((a, b)\)
  (by setting \(\varepsilon = x - b\) we have \(\forall y \in (a, b)\), \(\abs{x - y} = x - y > x - b = \varepsilon\)).
  Similarly, if \(x < a\), then \(x\) is not \((a - x)\)-adherent to \((a, b)\), and is hence not an adherent point to \((a, b)\).
  This contradiction shows that \(x\) is in fact in \([a, b]\) as claimed.
  Using similar arguments we can show that \(\overline{(a, b]} = \overline{[a, b)} = \overline{[a, b]} = [a, b]\).

  Now we show that \(\overline{(a, \infty)} = [a, \infty)\).
  By \cref{9.1.11} we know that \((a, \infty) \subseteq \overline{(a, \infty)}\).
  We also know that \(a\) is an adherent point of \((a, \infty)\).
  If not, then \(\exists \varepsilon \in \R^+\) such that
  \[
    \forall y \in (a, \infty), \abs{a - y} > \varepsilon.
  \]
  But this means
  \begin{align}
             & \abs{a - y} = y - a > \varepsilon                        & (y \in (a, \infty))                                    \\
    \implies & y > a + \varepsilon                                                                                               \\
    \implies & b > y > a + \varepsilon > a                              & (\varepsilon \in \R^+ \land y \in (a, b))              \\
    \implies & a + \varepsilon \in (a, b)                               &                                           & \by{9.1.1} \\
    \implies & \varepsilon < \abs{a - (a - \varepsilon)} = \varepsilon,
  \end{align}
  a contradiction.
  Thus \(a\) is an adherent point of \((a, \infty)\) and \([a, \infty) \subseteq \overline{(a, \infty)}\).
  Suppose for sake of contradiction that \(\exists x \in \overline{(a, \infty)}\) such that \(x \notin [a, \infty)\).
  Then \(x < a\) (by \cref{9.1.8} \(x \in \R\) so \(x \neq \infty\)).
  But we know \(x\) is not \((a - x)\)-adherent to \((a, \infty)\), and is hence not an adherent point to \((a, b)\), a contradiction.
  Thus \(\overline{(a, \infty)} = [a, \infty)\).
  Using similar arguments we can show that \(\overline{[a, \infty)} = [a, \infty)\) and \(\overline{(-\infty, b)} = \overline{(-\infty, b]} = (-\infty, b]\).

  Finally we show that \(\overline{(-\infty, \infty)} = (-\infty, \infty)\).
  By \cref{9.1.11} we know that \((-\infty, \infty) \subseteq \overline{(-\infty, \infty)}\).
  By \cref{9.1.8} we know that \(\overline{(-\infty, \infty)} \subseteq \R = (-\infty, \infty)\).
  Thus \(\overline{(-\infty, \infty)} = (-\infty, \infty)\).
\end{proof}

\begin{lem}\label{9.1.13}
  The closure of \(\N\) is \(\N\).
  The closure of \(\Z\) is \(\Z\).
  The closure of \(\Q\) is \(\R\), and the closure of \(\R\) is \(\R\).
  The closure of the empty set \(\emptyset\) is \(\emptyset\).
\end{lem}

\begin{proof}
  We first show that \(\overline{\N} = \N\).
  Let \(\overline{\N}\) be the closure of \(\N\).
  By \cref{9.1.8} we have \(\overline{\N} \subseteq \R\).
  By \cref{9.1.11} we have \(\N \subseteq \overline{\N}\).
  Now we show that \(\overline{\N} \subseteq \N\).
  Suppose for sake of contradiction that \(\exists x \in \overline{\N}\) such that \(x \notin \N\).
  Since
  \begin{align*}
             & \N \subseteq [0, \infty)                       &  & \by{9.1.1}  \\
    \implies & \overline{\N} \subseteq \overline{[0, \infty)} &  & \by{9.1.11} \\
    \implies & \overline{\N} \subseteq [0, \infty),           &  & \by{9.1.12}
  \end{align*}
  we know that \(x > 0\).
  By \cref{5.4.12} \(\exists n \in \N\) such that \(n < x < n + 1\).
  Let \(\varepsilon = \min(x - n, n + 1 - x) / 2\).
  By \cref{9.1.10}, \(\exists m \in \N\) such that \(\abs{x - m} \leq \varepsilon\).
  We split into two cases:
  \begin{itemize}
    \item If \(m \leq n\), then we have \(x - m \geq x - n \geq \min(x - n, n + 1 - x) > \varepsilon\), a contradiction.
    \item If \(m > n\), then we have \(m \geq n + 1\) and \(m - x \geq n + 1 - x \geq \min(x - n, n + 1 - x) > \varepsilon\), a contradiction.
  \end{itemize}
  From all cases above we derived contradictions.
  Thus such \(m\) does not exists and by \cref{9.1.10} \(x \notin \overline{\N}\).
  So we have \(\overline{\N} \subseteq \N\).
  Since \(\N \subseteq \overline{\N} \land \overline{\N} \subseteq \N\), by \cref{3.1.18} we have \(\N = \overline{\N}\).

  Next we show that \(\overline{\Z} = \Z\).
  Let \(\overline{\Z}\) be the closure of \(\Z\) and let \(\Z^- = \set{z \in \Z : z < 0}\).
  Then we have
  \begin{align*}
    \overline{\Z} & = \overline{\N \cup \Z^-}                                           \\
                  & = \overline{\N} \cup \overline{\Z^-} &  & \by{9.1.11}               \\
                  & = \N \cup \overline{\Z^-}.           &  & \text{(from proof above)}
  \end{align*}
  Thus to show that \(\Z = \overline{\Z}\), it suffices to show that \(\Z^- = \overline{\Z^-}\).
  By \cref{9.1.11} we have \(\Z^- \subseteq \overline{\Z^-}\).
  We need to show that \(\overline{\Z^-} \subseteq \Z^-\).
  Suppose for sake of contradiction that \(\exists x \in \overline{\Z^-}\) such that \(x \notin \Z^-\).
  Since
  \begin{align*}
             & \Z^- \subseteq (-\infty, 0)                       &  & \by{9.1.1}  \\
    \implies & \overline{\Z^-} \subseteq \overline{(-\infty, 0)} &  & \by{9.1.11} \\
    \implies & \overline{\Z^-} \subseteq (-\infty, 0],           &  & \by{9.1.12}
  \end{align*}
  we know that \(x < 0\).
  By \cref{5.4.12} \(\exists n \in \Z^-\) such that \(n < x < n + 1\).
  Let \(\varepsilon = \min(x - n, n + 1 - x) / 2\).
  By \cref{9.1.10}, \(\exists m \in \Z^-\) such that \(\abs{x - m} \leq \varepsilon\).
  We split into two cases:
  \begin{itemize}
    \item If \(m \leq n\), then we have \(x - m \geq x - n \geq \min(x - n, n + 1 - x) > \varepsilon\), a contradiction.
    \item If \(m > n\), then we have \(m \geq n + 1\) and \(m - x \geq n + 1 - x \geq \min(x - n, n + 1 - x) > \varepsilon\), a contradiction.
  \end{itemize}
  From all cases above we derived contradictions.
  Thus such \(m\) does not exists and by \cref{9.1.10} \(x \notin \overline{\Z^-}\).
  So we have \(\overline{\Z^-} \subseteq \Z^-\).
  Since \(\Z^- \subseteq \overline{\Z^-} \land \overline{\Z^-} \subseteq \Z^-\), by \cref{3.1.18} we have \(\Z^- = \overline{\Z^-}\), and thus \(\Z = \overline{\Z}\).

  Next we show that \(\overline{\Q} = \R\).
  Let \(\overline{\Q}\) be the closure of \(\Q\).
  We have
  \begin{align*}
             & \Q \subseteq \R = (-\infty, \infty)                  &  & \by{9.1.1}  \\
    \implies & \overline{\Q} \subseteq \overline{(-\infty, \infty)} &  & \by{9.1.11} \\
    \implies & \overline{\Q} \subseteq (-\infty, \infty) = \R.      &  & \by{9.1.12}
  \end{align*}
  Since
  \begin{align*}
             & \forall x \in \R, \forall \varepsilon \in \R^+, x - \varepsilon < x < x + \varepsilon                  \\
    \implies & \exists q \in \Q : x - \varepsilon < q < x + \varepsilon                              &  & \by{5.4.14} \\
    \implies & \abs{x - q} < \varepsilon                                                                              \\
    \implies & x \in \overline{\Q},                                                                  &  & \by{9.1.10}
  \end{align*}
  by \cref{3.1.15} we have \(\R \subseteq \overline{\Q}\).
  Since \(\R \subseteq \overline{\Q} \land \overline{\Q} \subseteq \R\), by \cref{3.1.18} we have \(\R = \overline{\Q}\).

  Next we show that \(\overline{\R} = \R\).
  Since
  \begin{align*}
         & \R = (-\infty, \infty)                                            &  & \by{9.1.1}  \\
    \iff & \overline{\R} = \overline{(-\infty, \infty)} = (-\infty, \infty), &  & \by{9.1.12}
  \end{align*}
  we know that \(\overline{\R} = \R\).

  Finally we show that \(\overline{\emptyset} = \emptyset\).
  Suppose for sake of contradiction that \(\overline{\emptyset} \neq \emptyset\).
  Let \(x \in \overline{\emptyset}\)
  Then by \cref{9.1.10} \(\forall \varepsilon \in \R^+\), \(\exists y \in \emptyset\) such that \(\abs{x - y} \leq \varepsilon\), a contradiction.
  Thus \(\overline{\emptyset} = \emptyset\).
\end{proof}

\begin{lem}\label{9.1.14}
  Let \(X\) be a subset of \(\R\), and let \(x \in \R\).
  Then \(x\) is an adherent point of \(X\) iff there exists a sequence \((a_n)_{n = 0}^\infty\), consisting entirely of elements in \(X\), which converges to \(x\).
\end{lem}

\begin{proof}
  We first show that if \(x\) is an adherent point of \(X\), then there exists a sequence \((a_n)_{n = 0}^\infty\) such that \(\forall n \in \N\), \(a_n \in X\) and \(\lim_{n \to \infty} a_n = x\).
  For each \(n \in \N\) let \(A_n\) be the set
  \[
    A_n = \set{y \in X : \abs{x - y} \leq \dfrac{1}{n}}.
  \]
  We know by \cref{9.1.10} that \(A_n \neq \emptyset\).
  By axiom of choice (\cref{8.1}) we know \(\prod_{n \in \N} A_n \neq \emptyset\).
  Let \(f \in \prod_{n \in \N} A_n\).
  We can define a sequence \((a_n)_{n = 0}^\infty\) by setting \(a_n = f(n)\).
  Then we have
  \begin{align*}
             & \forall n \in \N, a_n \in A_n                           \\
    \implies & 0 \leq \abs{x - a_n} \leq \dfrac{1}{n}                  \\
    \implies & \lim_{n \to \infty} \abs{x - a_n} = 0  &  & \by{6.4.14} \\
    \implies & \lim_{n \to \infty} x - a_n = 0        &  & \by{6.4.17} \\
    \implies & x = \lim_{n \to \infty} a_n.           &  & \by{6.1.19}
  \end{align*}

  Now we show that if there exists a sequence \((a_n)_{n = 0}^\infty\) such that \(\forall n \in \N\), \(a_n \in X\) and \(\lim_{n \to \infty} a_n = x\), then \(x\) is an adherent point of \(X\).
  Since \(\lim_{n \to \infty} a_n = x\), by \cref{6.4.5} \(x\) is the only limit point of \((a_n)_{n = m}^\infty\).
  So we have
  \begin{align*}
             & \forall \varepsilon \in \R^+, \exists n \in \N : \abs{x - a_n} \leq \varepsilon  &  & \by{6.4.1}  \\
    \implies & \forall \varepsilon \in \R^+, \exists a_n \in X : \abs{x - a_n} \leq \varepsilon                  \\
    \implies & x \in \overline{X}.                                                              &  & \by{9.1.10}
  \end{align*}
  We conclude that \(x\) is an adherent point of \(X\) iff there exists a sequence \((a_n)_{n = 0}^\infty\) such that \(\forall n \in \N\), \(a_n \in X\) and \(\lim_{n \to \infty} a_n = x\).
\end{proof}

\begin{defn}\label{9.1.15}
  A subset \(E \subseteq \R\) is said to be \emph{closed} if \(\overline{E} = E\), or in other words that \(E\) contains all of its adherent points.
\end{defn}

\begin{eg}\label{9.1.16}
  From \cref{9.1.12} we see that if \(a < b\) are real numbers, then \([a, b]\), \([a, +\infty)\), \((-\infty, a]\), and \((-\infty, +\infty)\) are closed, while \((a, b)\), \((a, b]\), \([a, b)\), \((a, +\infty)\), and \((-\infty, a)\) are not.
  From \cref{9.1.13} we see that \(\N\), \(\Z\), \(\R\), \(\emptyset\) are closed, while \(\Q\) is not.
\end{eg}

\begin{cor}\label{9.1.17}
  Let \(X\) be a subset of \(\R\).
  If \(X\) is closed, and \((a_n)_{n = 0}^\infty\) is a convergent sequence consisting of elements in \(X\), then \(\lim_{n \to \infty} a_n\) also lies in \(X\).
  Conversely, if it is true that every convergent sequence \((a_n)_{n = 0}^\infty\) of elements in \(X\) has its limit in \(X\) as well, then \(X\) is necessarily closed.
\end{cor}

\begin{proof}
  We first show that if \(X\) is closed, and \((a_n)_{n = 0}^\infty\) is a convergent sequence consisting of elements in \(X\), then \(\lim_{n \to \infty} a_n\) also lies in \(X\).
  Let \(x = \lim_{n \to \infty} a_n\).
  Then we have
  \begin{align*}
             & \forall \varepsilon \in \R^+, \exists n \in \N : \forall n' \in \N \land n' \geq n, \abs{x - a_{n'}} \leq \varepsilon                  \\
    \implies & \forall \varepsilon \in \R^+, \exists a_n \in X : \abs{x - a_n} \leq \varepsilon                                                       \\
    \implies & x \in \overline{X}                                                                                                    &  & \by{9.1.10} \\
    \implies & x \in X.                                                                                                              &  & \by{9.1.15}
  \end{align*}

  Now we show that if every convergent sequence \((a_n)_{n = 0}^\infty\) of elements in \(X\) has its limit in \(X\) as well, then \(X\) is closed.
  By \cref{9.1.11} we have \(X \subseteq \overline{X}\).
  Since
  \begin{align*}
             & \forall x \in \overline{X}, \exists (a_n)_{n = 0}^\infty : (\forall n \in \N, a_n \in X) \land (\lim_{n \to \infty} a_n = x) &  & \by{9.1.14}            \\
    \implies & x \in X,                                                                                                                     &  & \text{(by hypothesis)}
  \end{align*}
  by \cref{3.1.15} we have \(\overline{X} \subseteq X\).
  Since \(X \subseteq \overline{X} \land \overline{X} \subseteq X\), by \cref{3.1.8} we have \(X = \overline{X}\), and thus by \cref{9.1.15} \(X\) is closed.
\end{proof}

\begin{defn}[Limit points]\label{9.1.18}
  Let \(X\) be a subset of the real line.
  We say that \(x\) is a \emph{limit point} (or a \emph{cluster point}) of \(X\) iff it is an adherent point of \(X \setminus \set{x}\).
  We say that \(x\) is an \emph{isolated point} of \(X\) if \(x \in X\) and there exists some \(\varepsilon > 0\) such that \(\abs{x - y} > \varepsilon\) for all \(y \in X \setminus \set{x}\).
\end{defn}

\setcounter{thm}{19}
\begin{rmk}\label{9.1.20}
  From \cref{9.1.14} we see that \(x\) is a limit point of \(X\) iff there exists a sequence \((a_n)_{n = 0}^\infty\), consisting entirely of elements in \(X\) that are distinct from \(x\), and such that \((a_n)_{n = 0}^\infty\) converges to \(x\).
  It turns out that the set of adherent points splits into the set of limit points and the set of isolated points.
\end{rmk}

\begin{lem}\label{9.1.21}
  Let \(I\) be an interval (possibly infinite), i.e., \(I\) is a set of the form \((a, b)\), \((a, b]\), \([a, b)\), \([a, b]\), \((a, +\infty)\), \([a, +\infty)\), \((-\infty, a)\), or \((-\infty, a]\), with \(a < b\) in the first four cases.
  Then every element of \(I\) is a limit point of \(I\).
\end{lem}

\begin{proof}
  We show this for the case \(I = [a, b]\);
  the other cases are similar.
  Let \(x \in I\);
  we have to show that \(x\) is a limit point of \(I\).
  There are three cases: \(x = a\), \(a < x < b\), and \(x = b\).
  If \(x = a\), then consider the sequence \((x + \dfrac{1}{n})_{n = N}^\infty\).
  This sequence converges to \(x\), and will lie inside \(I \setminus \set{a} = (a, b]\) if \(N\) is chosen large enough (by \cref{5.4.14}).
  Thus by \cref{9.1.20} we see that \(x = a\) is a limit point of \([a, b]\).
  A similar argument works when \(a < x < b\).
  When \(x = b\) one has to use the sequence \((x - \dfrac{1}{n})_{n = N}^\infty\) instead.
  This sequence converges to \(x\), and will lie inside \(I \setminus \set{b} = [a, b)\) if \(N\) is chosen large enough (by \cref{5.4.14}).
  Thus by \cref{9.1.20} we see that \(x = b\) is a limit point of \([a, b]\).
\end{proof}

\begin{defn}[Bounded sets]\label{9.1.22}
  A subset \(X\) of the real line is said to be \emph{bounded} if we have \(X \subseteq [-M, M]\) for some real number \(M > 0\).
\end{defn}

\begin{eg}\label{9.1.23}
  For any real numbers \(a, b\), the interval \([a, b]\) is bounded, because it is contained inside \([-M, M]\), where \(M \coloneqq \max(\abs{a}, \abs{b})\).
  However, the half-infinite interval \([0, +\infty)\) is unbounded.
  In fact, no half-infinite interval or doubly infinite interval can be bounded.
  The sets \(\N\), \(\Z\), \(\Q\), and \(\R\) are all unbounded.
\end{eg}

\begin{thm}[Heine-Borel theorem for the line]\label{9.1.24}
  Let \(X\) be a subset of \(\R\).
  Then the following two statements are equivalent:
  \begin{enumerate}
    \item \(X\) is closed and bounded.
    \item Given any sequence \((a_n)_{n = 0}^\infty\) of real numbers which takes values in \(X\) (i.e., \(a_n \in X\) for all \(n\)), there exists a subsequence \((a_{n_j})_{j = 0}^\infty\) of the original sequence, which converges to some number \(L\) in \(X\).
  \end{enumerate}
\end{thm}

\begin{proof}
  We first show that statement (a) implies statement (b).
  Suppose that \(X\) is a set such that \(X\) is closed and bounded.
  Let \((a_n)_{n = 0}^\infty\) be a sequence where \(\forall n \in \N\), \(a_n \in X\).
  Since \(X\) is bounded, by \cref{9.1.22} \(\exists M \in \R^+\) such that \(X \subseteq [-M, M]\), thus \((a_n)_{n = 0}^\infty\) is also bounded by \(M\), i.e., \(\forall n \in \N\), \(\abs{a_n} \leq M\).
  By Bolzano-Weierstrass theorem (\cref{6.6.8}) we know that there exists a subsequence \((a_{n_j})_{j = 0}^\infty\) of \((a_n)_{n = 0}^\infty\) such that \((a_{n_j})_{j = 0}^\infty\) converges.
  Since \(X\) is closed, by \cref{9.1.17} we know that \(\lim_{j \to \infty} a_{n_j} \in X\).

  Now we show that statement (b) implies statement (a).
  Since given any sequence \((a_n)_{n = 0}^\infty\) we can always find a subsequence \((a_{n_j})_{j = 0}^\infty\) such that \(\lim_{j \to \infty} a_{n_j} \in X\), we know that if \((a_n)_{n = 0}^\infty\) converges then \(\lim_{n \to \infty} a_n \in X\).
  Thus every convergent sequence \((a_n)_{n = 0}^\infty\) have its limit in \(X\), and by \cref{9.1.17} we know that \(X\) is closed.
  Suppose for sake of contradiction that \(X\) is unbounded.
  Then \(\nexists M \in \R^+\) such that \(X \subseteq [-M, M]\).
  Now we define \(X_n = \set{x \in X : \abs{x} > n}\) for every \(n \in \N\).
  We know that \(X_n \neq \emptyset\) since \(X\) is unbounded.
  By axiom of choice (\cref{8.1}) we know that \(\prod_{n \in \N} X_n \neq \emptyset\).
  Let \(f \in \prod_{n \in \N} X_n\).
  We can define a sequence \((a_n)_{n = 0}^\infty\) by setting \(a_n = f(n)\).
  By hypothesis we know that there exists a subsequence \((a_{n_j})_{j = 0}^\infty\) such that \(L = \lim_{j \to \infty} a_{n_j} \in X\).
  We know that \((a_{n_j})_{j = 0}^\infty\) is unbounded since \(\abs{a_{n_j}} > n_j\) for every \(n_j \in \N\).
  But by \cref{6.4.18} \((a_{n_j})_{j = 0}^\infty\) is Cauchy sequence and by \cref{6.1.17} \((a_{n_j})_{j = 0}^\infty\) is bounded, a contradiction.
  Thus \(X\) is closed and bounded.
\end{proof}

\begin{rmk}\label{9.1.25}
  This theorem shall play a key role in subsequent sections of \cref{ch:9}.
  In the language of metric space topology, it asserts that every subset of the real line which is closed and bounded, is also compact
  A more general version of this theorem, due to Eduard Heine (1821 -- 1881) and Emile Borel (1871 -- 1956), can be found is Analysis II, Theorem 1.5.7.
\end{rmk}

\exercisesection

\begin{ex}\label{ex:9.1.1}
  Let \(X\) be any subset of the real line, and let \(Y\) be a set such that \(X \subseteq Y \subseteq \overline{X}\).
  Show that \(\overline{Y} = \overline{X}\).
\end{ex}

\begin{proof}
  By \cref{9.1.11} we have \(X \subseteq Y \implies \overline{X} \subseteq \overline{Y}\).
  Since
  \begin{align*}
             & \forall y \in \overline{Y}, \forall \varepsilon \in \R^+, \exists x \in Y : \abs{y - x} \leq \dfrac{\varepsilon}{2} &                            & \by{9.1.10} \\
    \implies & \exists x \in \overline{X} : \abs{y - x} \leq \dfrac{\varepsilon}{2}                                                & (Y \subseteq \overline{X})               \\
    \implies & \exists z \in X : \abs{x - z} \leq \dfrac{\varepsilon}{2}                                                           &                            & \by{9.1.10} \\
    \implies & \abs{y - x} + \abs{x - z} \leq \dfrac{\varepsilon}{2} + \dfrac{\varepsilon}{2}                                                                                 \\
    \implies & \abs{y - x + x - z} \leq \abs{y - x} + \abs{x - z} \leq \varepsilon                                                                                            \\
    \implies & \abs{y - z} \leq \varepsilon                                                                                                                                   \\
    \implies & y \in \overline{X},                                                                                                 &                            & \by{9.1.10}
  \end{align*}
  by \cref{3.1.15} we know that \(\overline{Y} \subseteq \overline{X}\).
  Since \(\overline{Y} \subseteq \overline{X} \land \overline{X} \subseteq \overline{Y}\), by \cref{3.1.18} we have \(\overline{Y} = \overline{X}\).
\end{proof}

\begin{ex}\label{ex:9.1.2}
  Prove \cref{9.1.11}.
\end{ex}

\begin{proof}
  See \cref{9.1.11}.
\end{proof}

\begin{ex}\label{ex:9.1.3}
  Prove \cref{9.1.13}.
\end{ex}

\begin{proof}
  See \cref{9.1.13}.
\end{proof}

\begin{ex}\label{ex:9.1.4}
  Give an example of two subsets \(X, Y\) of the real line such that \(\overline{X \cap Y} \neq \overline{X} \cap \overline{Y}\).
\end{ex}

\begin{proof}
  Let \(X = [0, 0.5)\) and \(Y = (0.5, 1]\).
  By \cref{9.1.12} we have \(\overline{X} = [0, 0.5]\) and \(\overline{Y} = [0.5, 1]\), so \(\overline{X} \cap \overline{Y} = \set{0.5}\).
  By \cref{9.1.13} we have \(\overline{X \cap Y} = \overline{\emptyset} = \emptyset\).
  Thus \(\overline{X \cap Y} \neq \overline{X} \cap \overline{Y}\).
\end{proof}

\begin{ex}\label{ex:9.1.5}
  Prove \cref{9.1.14}.
\end{ex}

\begin{proof}
  See \cref{9.1.14}.
\end{proof}

\begin{ex}\label{ex:9.1.6}
  Let \(X\) be a subset of \(\R\).
  Show that \(\overline{X}\) is closed (i.e., \(\overline{\overline{X}} = \overline{X}\)).
  Furthermore, show that if \(Y\) is any closed set that contains \(X\), then \(Y\) also contains \(\overline{X}\).
  Thus the closure \(\overline{X}\) of \(X\) is the smallest closed set which contains \(X\).
\end{ex}

\begin{proof}
  We first show that \(X \subseteq \R \implies \overline{\overline{X}} = \overline{X}\).
  We have
  \begin{align*}
             & X \subseteq \R                                                   \\
    \implies & \overline{X} \subseteq \overline{\R}            &  & \by{9.1.11} \\
    \implies & \overline{X} \subseteq \R                       &  & \by{9.1.13} \\
    \implies & \overline{X} \subseteq \overline{\overline{X}}. &  & \by{9.1.11}
  \end{align*}
  Since
  \begin{align*}
             & \forall x \in \overline{\overline{X}}, \forall \varepsilon \in \R^+, \exists y \in \overline{X} : \abs{x - y} \leq \dfrac{\varepsilon}{2} &  & \by{9.1.10} \\
    \implies & \exists z \in X : \abs{y - z} \leq \dfrac{\varepsilon}{2}                                                                                 &  & \by{9.1.10} \\
    \implies & \abs{x - y} + \abs{y - z} \leq \dfrac{\varepsilon}{2} + \dfrac{\varepsilon}{2}                                                                             \\
    \implies & \abs{x - y + y - z} \leq \abs{x - y} + \abs{y - z} \leq \varepsilon                                                                                        \\
    \implies & \abs{x - z} \leq \varepsilon                                                                                                                               \\
    \implies & x \in \overline{X},                                                                                                                       &  & \by{9.1.10}
  \end{align*}
  by \cref{3.1.15} we know that \(\overline{\overline{X}} \subseteq \overline{X}\).
  Since \(\overline{\overline{X}} \subseteq \overline{X} \land \overline{X} \subseteq \overline{\overline{X}}\), by \cref{3.1.18} we have \(\overline{\overline{X}} = \overline{X}\).
  By \cref{9.1.15} \(\overline{X}\) is closed.

  Now we show that if \(Y\) is any closed set that contains \(X\), then \(Y\) also contains \(\overline{X}\).
  \begin{align*}
             & (X \subseteq Y) \land (Y = \overline{Y})                       &  & \by{9.1.15} \\
    \implies & (\overline{X} \subseteq \overline{Y}) \land (Y = \overline{Y}) &  & \by{9.1.11} \\
    \implies & \overline{X} \subseteq Y.
  \end{align*}
\end{proof}

\begin{ex}\label{ex:9.1.7}
  Let \(n \geq 1\) be a positive integer, and let \(X_1, \dots, X_n\) be closed subsets of \(\R\).
  Show that \(X_1 \cup X_2 \cup \dots \cup X_n\) is also closed.
\end{ex}

\begin{proof}
  Suppose that \(\forall m \in \N\) we have \(X_m\) is a closed subset of \(\R\).
  We use induction on \(n\) to show that \(X_1 \cup \dots \cup X_n\) is closed and we start with \(n = 1\).
  For \(n = 1\), by the given hypothesis we have \(X_1\) is closed.
  So the base case holds.
  Suppose inductively that for some \(n \geq 1\) we have \(X_1 \cup \dots \cup X_n\) is closed.
  Then for \(n + 1\), we have
  \begin{align*}
      & \overline{X_1 \cup \dots \cup X_n \cup X_{n + 1}}                                                \\
    = & \overline{(X_1 \cup \dots \cup X_n) \cup X_{n + 1}}            &  & \text{(by \cref{3.1.28}(e))} \\
    = & \overline{(X_1 \cup \dots \cup X_n)} \cup \overline{X_{n + 1}} &  & \by{9.1.11}                  \\
    = & (X_1 \cup \dots \cup X_n) \cup \overline{X_{n + 1}}            &  & \byIH                        \\
    = & (X_1 \cup \dots \cup X_n) \cup X_{n + 1}                       &  & \text{(by hypothesis)}       \\
    = & X_1 \cup \dots \cup X_n \cup X_{n + 1}.                        &  & \text{(by \cref{3.1.28}(e))}
  \end{align*}
  This closes the induction.
  Thus \(\forall n \in \N\), if \(X_1, \dots, X_n\) are closed subset of \(\R\), then \(X_1 \cup \dots \cup X_n\) is also closed.
\end{proof}

\begin{ex}\label{ex:9.1.8}
  Let \(I\) be a set (possibly infinite), and for each \(\alpha \in I\) let \(X_{\alpha}\) be a closed subset of \(\R\).
  Show that the intersection \(\bigcap_{\alpha \in I} X_{\alpha}\) is also closed.
\end{ex}

\begin{proof}
  By \cref{9.1.11} we have
  \[
    \bigcap_{\alpha \in I} X_{\alpha} \subseteq \R \implies \bigcap_{\alpha \in I} X_{\alpha} \subseteq \overline{\bigcap_{\alpha \in I} X_{\alpha}}.
  \]
  Since
  \begin{align*}
             & \forall x \in \overline{\bigcap_{\alpha \in I} X_{\alpha}}, \forall \varepsilon \in \R^+, \exists y \in \bigcap_{\alpha \in I} X_{\alpha} : \abs{x - y} \leq \varepsilon &  & \by{9.1.10}            \\
    \implies & \forall \alpha \in I, y \in X_{\alpha}                                                                                                                                                               \\
    \implies & \forall \alpha \in I, x \in \overline{X_{\alpha}}                                                                                                                        &  & \by{9.1.10}            \\
    \implies & \forall \alpha \in I, x \in X_{\alpha}                                                                                                                                   &  & \text{(by hypothesis)} \\
    \implies & x \in \bigcap_{\alpha \in I} X_{\alpha},
  \end{align*}
  by \cref{3.1.15} we have \(\overline{\bigcap_{\alpha \in I} X_{\alpha}} \subseteq \bigcap_{\alpha \in I} X_{\alpha}\).
  By \cref{3.1.18} we have
  \[
    \bigg(\overline{\bigcap_{\alpha \in I} X_{\alpha}} \subseteq \bigcap_{\alpha \in I} X_{\alpha}\bigg) \land \bigg(\bigcap_{\alpha \in I} X_{\alpha} \subseteq \overline{\bigcap_{\alpha \in I} X_{\alpha}}\bigg) \iff \overline{\bigcap_{\alpha \in I} X_{\alpha}} = \bigcap_{\alpha \in I} X_{\alpha},
  \]
  and thus by \cref{9.1.15} \(\bigcap_{\alpha \in I} X_{\alpha}\) is closed.
\end{proof}

\begin{ex}\label{ex:9.1.9}
  Let \(X\) be a subset of the real line.
  Show that every adherent point of \(X\) is either a limit point or an isolated point of \(X\), but cannot be both.
  Conversely, show that every limit point and every isolated point of \(X\) is an adherent point of \(X\).
\end{ex}

\begin{proof}
  Let \(X \subseteq \R\).
  We first show that every adherent point of \(X\) is either a limit point or an isolated point of \(X\), but cannot be not both.
  Let \(x \in \overline{X}\).
  Observe that
  \begin{align*}
             & x \in \overline{X}                                                                       \\
    \implies & x \in \overline{(X \setminus \set{x}) \cup \set{x}}                                      \\
    \implies & x \in \overline{X \setminus \set{x}} \cup \overline{\set{x}}            &  & \by{9.1.11} \\
    \implies & (x \in \overline{X \setminus \set{x}}) \lor (x \in \overline{\set{x}}).
  \end{align*}
  By \cref{9.1.10} we know that \(x \in \overline{\set{x}}\).
  Thus we have either \(x \in \overline{X \setminus \set{x}}\) or \(x \notin \overline{X \setminus \set{x}}\).
  \begin{itemize}
    \item If \(x \in \overline{X \setminus \set{x}}\), then by \cref{9.1.18} \(x\) is a limit point of \(X\).
    \item If \(x \notin \overline{X \setminus \set{x}}\), then by \cref{9.1.10} \(\exists \varepsilon \in \R^+\) such that \(\forall y \in X \setminus \set{x}\), \(\abs{x - y} > \varepsilon\).
          Since \(x \in \overline{X}\), we must have \(x \in X\), otherwise \(\nexists y \in X \setminus \set{x}\) such that \(\abs{x - y} \leq \varepsilon\).
          Thus by \cref{9.1.18} \(x\) is a isolated point of \(X\).
  \end{itemize}
  From all cases above we conclude that \(x\) is either a limit point or an isolated point of \(X\).
  Since \(x \in \overline{X \setminus \set{x}}\) and \(x \notin \overline{X \setminus \set{x}}\) cannot be true at the same time, we know that \(x\) is either a limit point or an isolated point of \(X\), but cannot be both.

  Now we show that every limit point and every isolated point of \(X\) is an adherent point of \(X\).
  Suppose that \(x\) is a limit point of \(X\).
  Then we have
  \begin{align*}
             & x \in \overline{X \setminus \set{x}}                                                           &  & \by{9.1.18} \\
    \implies & \forall \varepsilon \in \R^+, \exists y \in X \setminus \set{x} : \abs{x - y} \leq \varepsilon &  & \by{9.1.10} \\
    \implies & \forall \varepsilon \in \R^+, \exists y \in X : \abs{x - y} \leq \varepsilon                                    \\
    \implies & x \in \overline{X}.                                                                            &  & \by{9.1.10}
  \end{align*}
  Now suppose that \(x\) is a isolated point of \(X\).
  Then we have
  \begin{align*}
             & (x \in X) \land (X \subseteq \overline{X}) &  & \by{9.1.18} \\
    \implies & x \in \overline{X}.                        &  & \by{9.1.10}
  \end{align*}
  Since \(x\) is arbitrary limit point or isolated point of \(X\), we conclude that every limit point and every isolated point of \(X\) is an adherent point of \(X\).
\end{proof}

\begin{ex}\label{ex:9.1.10}
  If \(X\) is a non-empty subset of \(\R\), show that \(X\) is bounded iff \(\inf(X)\) and \(\sup(X)\) are finite.
\end{ex}

\begin{proof}
  Suppose that \(X\) is a set, \(X \subseteq \R\) and \(X \neq \emptyset\).
  We first show that if \(X\) is bounded then \(\inf(X)\) and \(\sup(X)\) are finite.
  Since
  \begin{align*}
             & \exists M \in \R^+ : X \subseteq [-M, M]                      &  & \by{9.1.22} \\
    \implies & \forall x \in X : -M \leq x \leq M                            &  & \by{9.1.1}  \\
    \implies & \forall x \in X : -M \leq \inf(X) \leq x \leq \sup(X) \leq M, &  & \by{6.2.11}
  \end{align*}
  we know that \(\inf(X)\) and \(\sup(X)\) are finite.

  Now we show that if \(\inf(X)\) and \(\sup(X)\) are finite then \(X\) is bounded.
  Let \(M = \max\big(\abs{\inf(X)}, \abs{\sup(X)}\big)\).
  Then we have
  \begin{align*}
             & M = \max\big(\abs{\inf(X)}, \abs{\sup(X)}\big)                                                    \\
    \implies & \big(\abs{\inf(X)} \leq M\big) \land \big(\sup(X) \leq \abs{\sup(X)} \leq M\big)                  \\
    \implies & \big(-M \leq \inf(X) \leq M\big) \land \big(\sup(X) \leq M\big)                                   \\
    \implies & -M \leq \inf(X) \leq \sup(X) \leq M                                              &  & \by{6.2.11} \\
    \implies & \forall x \in X, -M \leq \inf(X) \leq x \leq \sup(X) \leq M                      &  & \by{6.2.11} \\
    \implies & X \subseteq [-M, M].                                                             &  & \by{9.1.1}
  \end{align*}
  Thus by \cref{9.1.22} \(X\) is bounded.
\end{proof}

\begin{ex}\label{ex:9.1.11}
  Show that if \(X\) is a bounded subset of \(\R\), then the closure \(X\) is also bounded.
\end{ex}

\begin{proof}
  Since \(X\) is bounded, by \cref{9.1.22} we know that \(\exists M \in \R^+ : X \subseteq [-M, M]\).
  Let \(\varepsilon \in \R^+\).
  Then we have
  \begin{align*}
             & \forall x \in \overline{X}                                                                                                \\
    \implies & \exists y \in X : \abs{x - y} \leq \varepsilon                                         &                    & \by{9.1.10} \\
    \implies & -\varepsilon \leq x - y \leq \varepsilon                                                                                  \\
    \implies & y - \varepsilon \leq x \leq y + \varepsilon                                                                               \\
    \implies & -M - \varepsilon \leq y - \varepsilon \leq x \leq y + \varepsilon \leq M + \varepsilon & (-M \leq y \leq M)               \\
    \implies & x \in [-M - \varepsilon, M + \varepsilon].                                             &                    & \by{9.1.1}
  \end{align*}
  By \cref{3.1.15} we have \(\overline{X} \subseteq [-M - \varepsilon, M + \varepsilon]\), and thus by \cref{9.1.22} \(\overline{X}\) is bounded.
\end{proof}

\begin{ex}\label{ex:9.1.12}
  Show that the union of any finite collection of bounded subsets of \(\R\) is still a bounded set.
  Is this conclusion still true if one takes an infinite collection of bounded subsets of \(\R\)?
\end{ex}

\begin{proof}
  Let \(n \in \N\), let \(I_n = \set{i \in \N : 1 \leq i \leq n}\) and let \(X_1, \dots, X_n\) be bounded subsets of \(\R\).
  We want to show that \(\bigcup_{i \in I_n} X_i\) is bounded.
  If \(n = 0\), then \(I_n = \emptyset \implies \bigcup_{i \in I_n} X_i = \emptyset\) and by \cref{5.1.14} \(\emptyset\) is bounded.
  So suppose that \(n \neq 0\).
  Since \(X_1, \dots, X_n\) are bounded, \(\exists M_1, \dots, M_n \in \R^+\) such that \(\forall i \in I_n\), \(X_i \subseteq [-M_i, M_i]\).
  Let \(S = \set{M_i : i \in I_n}\).
  Clearly \(S\) is finite.
  By \cref{5.1.14} we know that \(S\) is bounded by some \(M \in \R\).
  Then we have
  \begin{align*}
             & \forall x \in \bigcup_{i \in I_n} X_i, \exists i \in I_n : x \in X_i                                             \\
    \implies & -M_i \leq x \leq M_i                                                 & (X_i \subseteq [-M_i, M_i])               \\
    \implies & -M \leq -M_i \leq x \leq M_i \leq M                                  &                             & \by{5.1.14} \\
    \implies & x \in [-M, M].
  \end{align*}
  By \cref{3.1.15} we have \(\bigcup_{i \in I_n} X_i \subseteq [-M, M]\), thus by \cref{9.1.22} \(\bigcup_{i \in I_n} X_i\) is bounded.

  Now we show that the union of an infinite collection of bounded subsets of \(\R\) may not be bounded.
  Let \(n \in \N\) and let \(X_n = \set{n}\).
  Clearly \(\forall n \in \N\), \(X_n \subseteq [-n, n]\) and thus by \cref{9.1.22} \(X_n\) is bounded.
  Then we have \(\bigcup_{n \in \N} X_n = \N\) and by \cref{3.6.12} \(\N\) is unbounded.
\end{proof}

\begin{ex}\label{ex:9.1.13}
  Prove \cref{9.1.24}.
\end{ex}

\begin{proof}
  See \cref{9.1.24}.
\end{proof}

\begin{ex}\label{ex:9.1.14}
  Show that any finite subset of \(\R\) is closed and bounded.
\end{ex}

\begin{proof}
  Let \(n \in \N\) and let \(x_i \in \R\) for every \(i \in \N \land i \leq n\).
  Given any sequence \((a_m)_{m = 0}^\infty\) of real numbers which taks values in the singleton set \(\set{x_i}\), we know that \(\forall m \in \N\), \(a_m = x_i\).
  Thus by \cref{6.1.19} we have \(\lim_{m \to \infty} a_m = x_i\).
  Since \(x_i \in \set{x_i}\), by Heine-Borel theorem (\cref{9.1.24}) we know that \(\set{x_i}\) is closed and bounded.
  Let \(S = \bigcup_{i \in \N : i \leq n} \set{x_i}\).
  By \cref{ex:9.1.7} we know that \(S\) is closed, and by \cref{ex:9.1.12} we know that \(S\) is bounded.
  Since \(n\) is arbitrary natural number, we conclude that every finite subset of \(\R\) is closed and bounded.
\end{proof}

\begin{ex}\label{ex:9.1.15}
  Let \(E\) be a non-empty bounded subset of \(\R\), and let \(S \coloneqq \sup(E)\) be the least upper bound of \(E\).
  (Note from the least upper bound principle, \cref{5.5.9}, that \(S\) is a real number.)
  Show that \(S\) is an adherent point of \(E\), and is also an adherent point of \(\R \setminus E\).
\end{ex}

\begin{proof}
  We first show that \(S\) is an adherent point of \(E\).
  \(\forall \varepsilon \in \R^+\), we can always find \(x \in E\) such that \(S - \varepsilon < x \leq S\).
  If not, then \(\exists \varepsilon \in \R^+\) such that \(S - \varepsilon < S\) and
  \[
    \forall x \in E \implies x \leq S - \varepsilon < S \implies \sup(E) \neq S,
  \]
  a contradiction.
  Then we have
  \begin{align*}
             & \forall \varepsilon \in \R^+, \exists x \in E : S - \varepsilon < x \leq S                             \\
    \implies & \forall \varepsilon \in \R^+, \exists x \in E : S - \varepsilon < x < S + \varepsilon                  \\
    \implies & \forall \varepsilon \in \R^+, \exists x \in E : -\varepsilon < x - S < \varepsilon                     \\
    \implies & \forall \varepsilon \in \R^+, \exists x \in E : \abs{x - S} < \varepsilon                              \\
    \implies & S \in \overline{E}.                                                                   &  & \by{9.1.10}
  \end{align*}

  Now we show that \(S\) is an adherent point of \(\R \setminus E\).
  Let \(\varepsilon \in \R^+\).
  Then we have
  \begin{align*}
             & \forall x \in E, x \leq S                                           &  & \by{5.5.5}  \\
    \implies & x \leq S < S + \varepsilon                                                           \\
    \implies & x \notin (S, S + \varepsilon)                                       &  & \by{9.1.1}  \\
    \implies & E \cap (S, S + \varepsilon) = \emptyset                                              \\
    \implies & (S, S + \varepsilon) \subseteq \R \setminus E                                        \\
    \implies & \overline{(S, S + \varepsilon)} \subseteq \overline{\R \setminus E} &  & \by{9.1.11} \\
    \implies & [S, S + \varepsilon] \subseteq \overline{\R \setminus E}            &  & \by{9.1.12} \\
    \implies & S \in \overline{\R \setminus E}.                                    &  & \by{9.1.1}
  \end{align*}
\end{proof}