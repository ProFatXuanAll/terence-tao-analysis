\section{Convergence and limit laws}\label{sec:6.1}

\begin{defn}[Distance between two real numbers]\label{6.1.1}
  Given two real numbers \(x\) and \(y\), we define their distance \(d(x, y)\) to be \(d(x, y) \coloneqq \abs{x - y}\).
\end{defn}

\begin{note}
  Clearly \cref{6.1.1} is consistent with \cref{4.3.2}.
  Further, \cref{4.3.3} works just as well for real numbers as it does for rationals, because the real numbers obey all the rules of algebra that the rationals do.
\end{note}

\begin{defn}[\(\varepsilon\)-close real numbers]\label{6.1.2}
  Let \(\varepsilon > 0\) be a real number.
  We say that two real numbers \(x, y\) are \emph{\(\varepsilon\)-close} iff we have \(d(y, x) \leq \varepsilon\).
\end{defn}

\begin{note}
  Again, it is clear that \cref{6.1.2} is consistent with \cref{4.3.4}.
\end{note}

\begin{note}
  Now let \((a_n)_{n = m}^\infty\) be a sequence of \emph{real} numbers;
  i.e., we assign a real number \(a_n\) for every integer \(n \geq m\).
  The starting index \(m\) is some integer;
  usually this will be \(1\), but in some cases we will start from some index other than \(1\).
  (The choice of label used to index this sequence is unimportant; we could use for instance \((a_k)_{k = m}^{\infty}\) and this would represent exactly the same sequence as \((a_n)_{n = m}^{\infty}\).)
  We can define the notion of a Cauchy sequence in the same manner as before.
\end{note}

\begin{defn}[Cauchy sequences of reals]\label{6.1.3}
  Let \(\varepsilon > 0\) be a real number.
  A sequence \((a_n)_{n = N}^\infty\) of real numbers starting at some integer index \(N\) is said to be \emph{\(\varepsilon\)-steady} iff \(a_j\) and \(a_k\) are \(\varepsilon\)-close for every \(j, k \geq N\).
  A sequence \((a_n)_{n = m}^\infty\) starting at some integer index \(m\) is said to be \emph{eventually \(\varepsilon\)-steady} iff there exists an \(N \geq m\) such that \((a_n)_{n = N}^\infty\) is \(\varepsilon\)-steady.
  We say that \((a_n)_{n = m}^\infty\) is a \emph{Cauchy sequence} iff it is eventually \(\varepsilon\)-steady for every \(\varepsilon > 0\).
\end{defn}

\begin{note}
  To put it another way, a sequence \((a_n)_{n = m}^\infty\) of real numbers is a Cauchy sequence if, for every real \(\varepsilon > 0\), there exists an \(N \geq m\) such that \(\abs{a_n - a_n'} \leq \varepsilon\) for all \(n, n' \geq N\).
  These definitions are consistent with the corresponding definitions for rational numbers (\cref{5.1.3,5.1.6,5.1.8}), although verifying consistency for Cauchy sequences takes a little bit of care.
\end{note}

\begin{prop}\label{6.1.4}
  Let \((a_n)_{n = m}^\infty\) be a sequence of rational numbers starting at some integer index \(m\).
  Then \((a_n)_{n = m}^\infty\) is a Cauchy sequence in the sense of \cref{5.1.8} if and only if it is a Cauchy sequence in the sense of \cref{6.1.3}.
\end{prop}

\begin{proof}
  Suppose first that \((a_n)_{n = m}^\infty\) is a Cauchy sequence in the sense of \cref{6.1.3};
  then it is eventually \(\varepsilon\)-steady for every \emph{real} \(\varepsilon > 0\).
  In particular, it is eventually \(\varepsilon\)-steady for every \emph{rational} \(\varepsilon > 0\), which makes it a Cauchy sequence in the sense of \cref{5.1.8}.

  Now suppose that \((a_n)_{n = m}^\infty\) is a Cauchy sequence in the sense of \cref{5.1.8};
  then it is eventually \(\varepsilon'\)-steady for every \emph{rational} \(\varepsilon' > 0\).
  If \(\varepsilon > 0\) is a real number, then there exists a \emph{rational} \(\varepsilon' > 0\) which is smaller than \(\varepsilon\), by \cref{5.4.12}.
  Since \(\varepsilon'\) is rational, we know that \((a_n)_{n = m}^\infty\) is eventually \(\varepsilon'\)-steady;
  since \(\varepsilon' < \varepsilon\), this implies that \((a_n)_{n = m}^\infty\) is eventually \(\varepsilon\)-steady.
  Since \(\varepsilon\) is an arbitrary positive real number, we thus see that \((a_n)_{n = m}^\infty\) is a Cauchy sequence in the sense of \cref{6.1.3}.
\end{proof}

\begin{note}
  Because of \cref{6.1.4}, we will no longer care about the distinction between \cref{5.1.8} and \cref{6.1.3}, and view the concept of a Cauchy sequence as a single unified concept.
\end{note}

\begin{defn}[Convergence of sequences]\label{6.1.5}
  Let \(\varepsilon > 0\) be a real number, and let \(L\) be a real number.
  A sequence \((a_n)_{n = N}^\infty\) of real numbers is said to be \emph{\(\varepsilon\)-close to \(L\)} iff \(a_n\) is \(\varepsilon\)-close to \(L\) for every \(n \geq N\), i.e., we have \(\abs{a_n - L} \leq \varepsilon\) for every \(n \geq N\).
  We say that a sequence \((a_n)_{n = m}^\infty\) is \emph{eventually \(\varepsilon\)-close to \(L\)} iff there exists an \(N \geq m\) such that \((a_n)_{n = N}^\infty\) is \(\varepsilon\)-close to \(L\).
  We say that a sequence \((a_n)_{n = m}^\infty\) \emph{converges to \(L\)} iff it is eventually \(\varepsilon\)-close to \(L\) for every real \(\varepsilon > 0\).
\end{defn}

\setcounter{thm}{6}
\begin{prop}[Uniqueness of limits]\label{6.1.7}
  Let \((a_n)_{n = m}^\infty\) be a real sequence starting at some integer index \(m\), and let \(L \neq L'\) be two distinct real numbers.
  Then it is not possible for \((a_n)_{n = m}^\infty\) to converge to \(L\) while also converging to \(L'\).
\end{prop}

\begin{proof}
  Suppose for sake of contradiction that \((a_n)_{n = m}^\infty\) was converging to both \(L\) and \(L'\).
  Let \(\varepsilon = \abs{L - L'} / 3\);
  note that \(\varepsilon\) is positive since \(L \neq L'\).
  Since \((a_n)_{n = m}^\infty\) converges to \(L\), we know that \((a_n)_{n = m}^\infty\) is eventually \(\varepsilon\)-close to \(L\);
  thus there is an \(N \geq m\) such that \(d(a_n, L) \leq \varepsilon\) for all \(n \geq N\).
  Similarly, there is an \(M \geq m\) such that \(d(a_n, L') \leq \varepsilon\) for all \(n \geq M\).
  In particular, if we set \(n \coloneqq \max(N, M)\), then we have \(d(a_n, L) \leq \varepsilon\) and \(d(a_n, L') \leq \varepsilon\), hence by the triangle inequality \(d(L, L') \leq 2\varepsilon = 2\abs{L - L'} / 3\).
  But then we have \(\abs{L - L'} \leq 2\abs{L - L'} / 3\), which contradicts the fact that \(\abs{L - L'} > 0\).
  Thus it is not possible to converge to both \(L\) and \(L'\).
\end{proof}

\begin{defn}[Limits of sequences]\label{6.1.8}
  If a sequence \((a_n)_{n = m}^\infty\) converges to some real number \(L\), we say that \((a_n)_{n = m}^\infty\) is \emph{convergent} and that its \emph{limit} is \(L\);
  we write
  \[
    L = \lim_{n \to \infty} a_n
  \]
  to denote this fact.
  If a sequence \((a_n)_{n = m}^\infty\) is not converging to any real number \(L\), we say that the sequence \((a_n)_{n = m}^\infty\) is \emph{divergent} and we leave \(\lim_{n \to \infty} a_n\) undefined.
\end{defn}

\begin{note}
  \cref{6.1.7} ensures that a sequence can have at most one limit.
  Thus, if the limit exists, it is a single real number, otherwise it is undefined.
\end{note}

\begin{rmk}\label{6.1.9}
  The notation \(\lim_{n \to \infty} a_n\) does not give any indication about the starting index \(m\) of the sequence, but the starting index is irrelevant.
  Thus in the rest of this discussion we shall not be too careful as to where these sequences start, as we shall be mostly focused on their limits.
\end{rmk}

\begin{note}
  We sometimes use the phrase ``\(a_n \to x\) as \(n \to \infty\)'' as an alternate way of writing the statement ``\((a_n)_{n = m}^\infty\) converges to \(x\)''.
  Bear in mind, though, that the individual statements \(a_n \to x\) and \(n \to \infty\) do not have any rigorous meaning;
  this phrase is just a convention, though of course a very suggestive one.
\end{note}

\begin{rmk}\label{6.1.10}
  The exact choice of letter used to denote the index (in this case \(n\)) is irrelevant:
  the phrase \(\lim_{n \to \infty} a_n\) has exactly the same meaning as \(\lim_{k \to \infty} a_k\), for instance.
  Sometimes it will be convenient to change the label of the index to avoid conflicts of notation;
  for instance, we might want to change \(n\) to \(k\) because \(n\) is simultaneously being used for some other purpose, and we want to reduce confusion.
\end{rmk}

\begin{prop}\label{6.1.11}
  We have \(\lim_{n \to \infty} 1 / n = 0\).
\end{prop}

\begin{proof}
  We have to show that the sequence \((a_n)_{n = 1}^\infty\) converges to \(0\), where \(a_n \coloneqq 1 / n\).
  In other words, for every \(\varepsilon > 0\), we need to show that the sequence \((a_n)_{n = 1}^\infty\) is eventually \(\varepsilon\)-close to \(0\).
  So, let \(\varepsilon > 0\) be an arbitrary real number.
  We have to find an \(N\) such that \(\abs{a_n - 0} \leq \varepsilon\) for every \(n \geq N\).
  But if \(n \geq N\), then
  \[
    \abs{a_n - 0} = \abs{1 / n - 0} = 1 / n \leq 1 / N.
  \]
  Thus, if we pick \(N > 1 / \varepsilon\) (which we can do by the Archimedean principle), then \(1 / N < \varepsilon\), and so \((a_n)_{n = 1}^\infty\) is \(\varepsilon\)-close to \(0\).
  Thus \((a_n)_{n = 1}^\infty\) is eventually \(\varepsilon\)-close to \(0\).
  Since \(\varepsilon\) was arbitrary, \((a_n)_{n = 1}^\infty\) converges to \(0\).
\end{proof}

\begin{prop}[Convergent sequences are Cauchy]\label{6.1.12}
  Suppose that \((a_n)_{n = m}^\infty\) is a convergent sequence of real numbers.
  Then \((a_n)_{n = m}^\infty\) is also a Cauchy sequence.
\end{prop}

\begin{proof}
  Let \((a_n)_{n = m}^\infty\) be a sequence of real numbers converges to \(L\).
  Then by \cref{6.1.5} \(\forall \varepsilon \in \R^+\), \(\exists\ N \in \N\) and \(N \geq m\) such that \(\abs{a_n - L} \leq \varepsilon\) for every \(n \geq N\).
  In particular, we have \(\abs{a_n - L} \leq \varepsilon / 2\).
  Let \(n' \in \N\) and \(n' \geq N\).
  Then we have
  \begin{align*}
    \abs{a_n - a_{n'}} & = \abs{a_n - a_{n'} + L - L}           \\
                       & = \abs{(a_n - L) + (L - a_{n'})}       \\
                       & \leq \abs{a_n - L} + \abs{L - a_{n'}}  \\
                       & = \abs{a_n - L} + \abs{a_{n'} - L}     \\
                       & \leq \varepsilon / 2 + \varepsilon / 2 \\
                       & = \varepsilon.
  \end{align*}
  Since \(\varepsilon\) is arbitrary, by \cref{6.1.3} we know that \((a_n)_{n = m}^\infty\) is a Cauchy sequence.
\end{proof}

\setcounter{thm}{14}
\begin{prop}[Formal limits are genuine limits]\label{6.1.15}
  Suppose that \((a_n)_{n = 1}^\infty\) is a Cauchy sequence of rational numbers.
  Then \((a_n)_{n = 1}^\infty\) converges to \(\text{LIM}_{n \to \infty} a_n\), i.e.
  \[
    \text{LIM}_{n \to \infty} a_n = \lim_{n \to \infty} a_n.
  \]
\end{prop}

\begin{proof}
  Let \((a_n)_{n = m}^\infty\) be a Cauchy sequence of rationals, and let \(L = \text{LIM}_{n \to \infty} a_n\).
  By \cref{5.3.1} we know that \(L \in \R\).
  Thus by \cref{6.1.4} and \cref{6.1.5} we can ask whether \((a_n)_{n = m}^\infty\) converges to \(L\).
  Suppose for sake of contradiction that sequence \(a_n\) is not eventually \(\varepsilon\)-close to \(L\) for every \(\varepsilon \in \R^+\).
  Then \(\exists\ \varepsilon \in \R^+\) such that \(\forall n \in \N\) and \(n \geq m\), we have \(\abs{a_n - L} > \varepsilon\).
  Since \(\varepsilon > 0\), we know that \(\abs{a_n - L} > 0\), now we split into two cases:
  \begin{itemize}
    \item If \(a_n - L > 0\), then we have
          \begin{align*}
                     & \forall n \geq m, \abs{a_n - L} > \varepsilon                                 \\
            \implies & a_n - L > \varepsilon                                                         \\
            \implies & a_n > L + \varepsilon                                                         \\
            \implies & \text{LIM}_{n \to \infty} a_n > L + \varepsilon & \text{(by \cref{ex:5.4.8})} \\
            \implies & L > L + \varepsilon                                                           \\
            \implies & 0 > \varepsilon.
          \end{align*}
          But this contradict to \(\varepsilon > 0\).
    \item If \(a_n - L < 0\), then we have
          \begin{align*}
                     & \forall n \geq m, \abs{a_n - L} > \varepsilon                                 \\
            \implies & L - a_n > \varepsilon                                                         \\
            \implies & a_n < L - \varepsilon                                                         \\
            \implies & \text{LIM}_{n \to \infty} a_n < L - \varepsilon & \text{(by \cref{ex:5.4.8})} \\
            \implies & L < L - \varepsilon                                                           \\
            \implies & \varepsilon < 0.
          \end{align*}
          But this contradict to \(\varepsilon > 0\).
  \end{itemize}
  From all cases above we derived contradictions.
  Thus such \(\varepsilon\) does not exist, and therefor we must have \((a_n)_{n = m}^\infty\) eventually \(\varepsilon\)-close to \(L\) for every \(\varepsilon \in \R^+\).
  By \cref{6.1.5} this means \(\lim_{n \to \infty} a_n = L\).
\end{proof}

\begin{defn}[Bounded sequences]\label{6.1.16}
  A sequence \((a_n)_{n = m}^\infty\) of reals is \emph{bounded by} a real number \(M\) iff we have \(\abs{a_n} \leq M\) for all \(n \geq m\).
  We say that \((a_n)_{n = m}^\infty\) is bounded iff it is \emph{bounded} by \(M\) for some real number \(M > 0\).
\end{defn}

\begin{note}
  \cref{6.1.16} is consistent with \cref{5.1.12}.
\end{note}

\begin{note}
  Recall from \cref{5.1.15} that every Cauchy sequence of rational numbers is bounded.
  An inspection of the proof of that Lemma shows that the same argument works for real numbers;
  every Cauchy sequence of real numbers is bounded.
\end{note}

\begin{cor}\label{6.1.17}
  Every convergent sequence of real numbers is bounded.
\end{cor}

\begin{proof}
  From \cref{6.1.12} we have every convergent sequence of real numbers is a Cauchy sequence.
  And by \cref{5.1.15} every Cauchy sequence is bounded.
  Thus every convergent sequence of real numbers is bounded.
\end{proof}

\setcounter{thm}{18}
\begin{thm}[Limit Laws]\label{6.1.19}
  Let \((a_n)_{n = m}^\infty\) and \((b_n)_{n = m}^\infty\) be convergent sequences of real numbers, and let \(x, y\) be the real numbers \(x \coloneqq \lim_{n \to \infty} a_n\) and \(y \coloneqq \lim_{n \to \infty} b_n\).
  \begin{enumerate}
    \item The sequence \((a_n + b_n)_{n = m}^\infty\) converges to \(x + y\);
          in other words,
          \[
            \lim_{n \to \infty} (a_n + b_n) = \lim_{n \to \infty} a_n + \lim_{n \to \infty} b_n.
          \]
    \item The sequence \((a_n b_n)_{n = m}^\infty\) converges to \(xy\);
          in other words,
          \[
            \lim_{n \to \infty} (a_n b_n) = (\lim_{n \to \infty} a_n)(\lim_{n \to \infty} b_n).
          \]
    \item For any real number \(c\), the sequence \((c a_n)_{n = m}^\infty\) converges to \(cx\);
          in other words,
          \[
            \lim_{n \to \infty} (c a_n) = c(\lim_{n \to \infty} a_n).
          \]
    \item The sequence \((a_n - b_n)_{n = m}^\infty\) converges to \(x - y\);
          in other words,
          \[
            \lim_{n \to \infty} (a_n - b_n) = \lim_{n \to \infty} a_n - \lim_{n \to \infty} b_n.
          \]
    \item Suppose that \(y \neq 0\), and that \(b_n \neq 0\) for all \(n \geq m\).
          Then the sequence \((b_n^{-1})_{n = m}^\infty\) converges to \(y^{-1}\);
          in other words,
          \[
            \lim_{n \to \infty} b_n^{-1} = (\lim_{n \to \infty} b_n)^{-1}.
          \]
    \item Suppose that \(y \neq 0\), and that \(b_n \neq 0\) for all \(n \geq m\).
          Then the sequence \((a_n / b_n)_{n = m}^\infty\) converges to \(x / y\);
          in other words,
          \[
            \lim_{n \to \infty} \frac{a_n}{b_n} = \frac{\lim_{n \to \infty} a_n}{\lim_{n \to \infty} b_n}.
          \]
    \item The sequence \((\max(a_n, b_n))_{n = m}^\infty\) converges to \(\max(x, y)\);
          in other words,
          \[
            \lim_{n \to \infty} \max(a_n, b_n) = \max(\lim_{n \to \infty} a_n, \lim_{n \to \infty} b_n).
          \]
    \item The sequence \((\min(a_n, b_n))_{n = m}^\infty\) converges to \(\min(x, y)\);
          in other words,
          \[
            \lim_{n \to \infty} \min(a_n, b_n) = \min(\lim_{n \to \infty} a_n, \lim_{n \to \infty} b_n).
          \]
  \end{enumerate}
\end{thm}

\begin{proof}{(a)}
  By \cref{6.1.8} \(\forall \varepsilon \in \R^+\), \(\exists\ N_a \in \N\) such that \(\abs{a_n - x} \leq \varepsilon / 2\) for every \(n \geq N_a\).
  Similarly \(\exists\ N_b \in \N\) such that \(\abs{b_n - y} \leq \varepsilon / 2\) for every \(n \geq N_b\).
  Let \(N = \max(N_a, N_b)\).
  Then we have \(\forall n \geq N\),
  \begin{align*}
    \abs{a_n + b_n - (x + y)} & = \abs{(a_n - x) + (b_n - y)}          \\
                              & \leq \abs{a_n - x} + \abs{b_n - y}     \\
                              & \leq \varepsilon / 2 + \varepsilon / 2 \\
                              & = \varepsilon.
  \end{align*}
  Thus by \cref{6.1.5} \((a_n + b_n)_{n = m}^\infty\) converges to \(x + y\).
  And by \cref{6.1.8} we have \(\lim_{n \to \infty} (a_n + b_n) = x + y = \lim_{n \to \infty} a_n + \lim_{n \to \infty} b_n\).
\end{proof}

\begin{proof}{(b)}
  By \cref{6.1.17}, \(\exists\ A, B \in \R^+\) such that \(\abs{a_n} \leq A\) and \(\abs{b_n} \leq B\) for every \(n \geq m\).
  By \cref{6.1.8} \(\forall \varepsilon \in \R^+\), \(\exists\ N_a \in \N\) such that \(\abs{a_n - x} \leq \varepsilon / 2B\) for every \(n \geq N_a\).
  Similarly \(\exists\ N_b \in \N\) such that \(\abs{b_n - y} \leq \varepsilon / 2A\) for every \(n \geq N_b\).
  Let \(N = \max(N_a, N_b)\).
  Then we have
  \begin{align*}
    \abs{a_n b_n - x y} & = \abs{a_n b_n - x y + x b_n - x b_n}                                  \\
                        & = \abs{a_n b_n - x b_n + x b_n - x y}                                  \\
                        & = \abs{b_n(a_n - x) + x(b_n - y)}                                      \\
                        & \leq \abs{b_n(a_n - x)} + \abs{x(b_n - y)}                             \\
                        & = \abs{b_n}\abs{a_n - x} + \abs{x}\abs{b_n - y}                        \\
                        & \leq B \times \frac{\varepsilon}{2B} + A \times \frac{\varepsilon}{2A} \\
                        & = \varepsilon.
  \end{align*}
  Thus by \cref{6.1.5} \((a_n b_n)_{n = m}^\infty\) converges to \(x y\).
  And by \cref{6.1.8} we have \(\lim_{n \to \infty} (a_n b_n) = x y = (\lim_{n \to \infty} a_n)(\lim_{n \to \infty} b_n)\).
\end{proof}

\begin{proof}{(c)}
  Let \((b_n)_{n = m}^\infty\) be a sequence where \(b_n = c \) for every \(n \geq m\).
  Then we have \(\lim_{n \to \infty} c = c\) and
  \begin{align*}
    \lim_{n \to \infty} (c a_n) & = \lim_{n \to \infty} (b_n a_n)                                                     \\
                                & = (\lim_{n \to \infty} b_n)(\lim_{n \to \infty} a_n) & \text{(by \cref{6.1.19}(b))} \\
                                & = c(\lim_{n \to \infty} a_n).
  \end{align*}
\end{proof}

\begin{proof}{(d)}
  We have
  \begin{align*}
    \lim_{n \to \infty} (a_n - b_n) & = \lim_{n \to \infty} (a_n + (-1)(b_n))                     & \text{(by \cref{5.3.11})}    \\
                                    & = \lim_{n \to \infty} a_n + \lim_{n \to \infty} ((-1)(b_n)) & \text{(by \cref{6.1.19}(a))} \\
                                    & = \lim_{n \to \infty} a_n + (-1)(\lim_{n \to \infty} b_n)   & \text{(by \cref{6.1.19}(c))} \\
                                    & = \lim_{n \to \infty} a_n - \lim_{n \to \infty} b_n.        & \text{(by \cref{5.3.11})}
  \end{align*}
\end{proof}

\begin{proof}{(e)}
  We first show that \((b_n)_{n = m}^\infty\) is bounded away from zero.
  Since \(\lim_{n \to \infty} b_n \neq 0\), we must have some \(M \in \R^+\) such that \(\abs{b_n - 0} > M\) for every \(n \in \N\).
  Otherwise we would have \(\lim_{n \to \infty} b_n = 0\), which is a contradiction by \cref{6.1.7}.
  Since \(\abs{b_n - 0} = \abs{b_n} > M > 0\) for every \(n \geq m\), we know that \((b_n)_{n = m}^\infty\) is bounded away from zero.

  Now we show that \(\lim_{n \to \infty} b_n^{-1} = (\lim_{n \to \infty} b_n)^{-1}\).
  By \cref{6.1.8} \(\forall \varepsilon \in \R^+\), \(\exists\ N \in \N\) such that \(\abs{b_n - y} \leq \varepsilon M \abs{y}\) for every \(n \geq N\).
  (\(M\) is derived from the claim above).
  So
  \begin{align*}
    \abs{b_n^{-1} - y^{-1}} & = \abs{\frac{1}{b_n} - \frac{1}{y}}                                         \\
                            & = \abs{\frac{y - b_n}{b_n y}}                                               \\
                            & = \abs{y - b_n}\frac{1}{\abs{b_n}\abs{y}}                                   \\
                            & < \abs{y - b_n}\frac{1}{M\abs{y}}           & \text{(From the claim above)} \\
                            & \leq \varepsilon M\abs{y}\frac{1}{M\abs{y}}                                 \\
                            & = \varepsilon.
  \end{align*}
  Thus by \cref{6.1.5} \((b_n^{-1})_{n = m}^\infty\) converges to \(y^{-1}\).
  And by \cref{6.1.8} we have \(\lim_{n \to \infty} (b_n^{-1}) = y^{-1} = (\lim_{n \to \infty} b_n)^{-1}\).
\end{proof}

\begin{proof}{(f)}
  We have
  \begin{align*}
    \lim_{n \to \infty} \frac{a_n}{b_n} & = \lim_{n \to \infty} a_n b_n^{-1}                                                        \\
                                        & = (\lim_{n \to \infty} a_n)(\lim_{n \to \infty} b_n^{-1})  & \text{(by \cref{6.1.19}(b))} \\
                                        & = (\lim_{n \to \infty} a_n)(\lim_{n \to \infty} b_n)^{-1}  & \text{(by \cref{6.1.19}(e))} \\
                                        & = \frac{\lim_{n \to \infty} a_n}{\lim_{n \to \infty} b_n}.
  \end{align*}
\end{proof}

\begin{proof}{(g)}
  By \cref{6.1.8} \(\forall \varepsilon \in \R^+\), \(\exists\ N_a \in \N\) such that \(\abs{a_n - x} \leq \varepsilon\) for every \(n \geq N_a\).
  Similarly \(\exists\ N_b \in \N\) such that \(\abs{b_n - y} \leq \varepsilon\) for every \(n \geq N_b\).
  Let \(N = \max(N_a, N_b)\).
  Then we have \(\abs{a_n - x} \leq \varepsilon \land \abs{b_n - y} \leq \varepsilon\) for every \(n \geq N\).
  Now we split into two cases:
  \begin{itemize}
    \item If \(x = y\), then we have \(\max(x, y) = x\) and for every \(n \geq N\),
          \begin{align*}
                     & (\abs{a_n - x} < \varepsilon) \land (\abs{b_n - x} < \varepsilon) \\
            \implies & \abs{\max(a_n, b_n) - x} < \varepsilon                            \\
            \implies & \abs{\max(a_n, b_n) - \max(x, y)} < \varepsilon                   \\
            \implies & \lim_{n \to \infty} \max(a_n, b_n) = \max(x, y) = x.
          \end{align*}
    \item If \(x \neq y\), then we have either \(x < y\) or \(x > y\).
          Without loss of generality suppose that \(x < y\).
          Since \(x < y\), we have \(y - x > 0\).
          Since we have \(\abs{a_n - x} \leq \varepsilon\) and \(\abs{b_n - y} \leq \varepsilon\) for every positive real number \(\varepsilon\), we also have \(\abs{a_n - x} \leq (y - x) / 2\) and \(\abs{b_n - y} \leq (y - x) / 2\).
          So \(\forall n \geq N\), we have
          \begin{align*}
                     & (\abs{a_n - x} \leq \frac{y - x}{2}) \land (\abs{b_n - y} \leq \frac{y - x}{2})                                 \\
            \implies & (-\frac{y - x}{2} \leq a_n - x \leq \frac{y - x}{2}) \land (-\frac{y - x}{2} \leq b_n - y \leq \frac{y - x}{2}) \\
            \implies & (a_n - x \leq \frac{y - x}{2}) \land (-\frac{y - x}{2} \leq b_n - y)                                            \\
            \implies & (a_n \leq \frac{y - x}{2} + x) \land (y - \frac{y - x}{2} \leq b_n)                                             \\
            \implies & (a_n \leq \frac{x + y}{2}) \land (\frac{x + y}{2} \leq b_n)                                                     \\
            \implies & a_n \leq \frac{x + y}{2} \leq b_n.
          \end{align*}
          This means \(\forall n \geq N, \max(a_n, b_n) = b_n\).
          Thus \(\forall n \geq N\), we have
          \begin{align*}
                     & \abs{\max(a_n, b_n) - \max(x, y)} = \abs{b_n - y} \leq \varepsilon                                        \\
            \implies & \lim_{n \to \infty} \max(a_n, b_n) = \max(x, y) = \max(\lim_{n \to \infty} a_n, \lim_{n \to \infty} b_n).
          \end{align*}
  \end{itemize}
\end{proof}

\begin{proof}{(h)}
  We have \(\min(x, y) = -\max(-x, -y)\) and
  \begin{align*}
    \lim_{n \to \infty} \min(a_n, b_n) & = \lim_{n \to \infty} -\max(-a_n, -b_n)                                \\
                                       & = -\lim_{n \to \infty} \max(-a_n, -b_n) & \text{(by \cref{6.1.19}(c))} \\
                                       & = -\max(-x, -y)                         & \text{(by \cref{6.1.19}(g))} \\
                                       & = \min(x, y).
  \end{align*}
\end{proof}

\exercisesection

\begin{ex}\label{ex:6.1.1}
  Let \((a_n)_{n = m}^\infty\) be a sequence of real numbers, such that \(a_{n + 1} > a_n\) for each natural number \(n\).
  Prove that whenever \(n\) and \(m\) are natural numbers such that \(m > n\), then we have \(a_m > a_n\).
  (We refer to these sequences as \emph{increasing} sequences.)
\end{ex}

\begin{proof}
  Let \(E = \{z \in \N : n \leq z \leq m\}\).
  Then \(E\) is finite (since \(\#(E) = m - n + 1\)) and non-empty (since \(n, m \in E\)).
  Let \((a_z)_{z = n}^m\) be a sequence by mapping \(z \in E\) to \(a_z\).
  So \((a_z)_{z = n}^m\) is a finite sequence, and the elements in sequence \((a_z)_{z = n}^m\) are \(\{a_n, a_{n + 1}, \dots, a_{m - 1}, a_m\}\).
  By the given conditions we have \(a_{z + 1} > a_z\) for each natural number \(z\).
  Thus we have \(a_n < a_{n + 1} < \dots < a_{m - 1} < a_m\), and by \cref{5.4.7} we have \(a_n < a_m\).
\end{proof}

\begin{ex}\label{ex:6.1.2}
  Let \((a_n)_{n = m}^\infty\) be a sequence of real numbers, and let \(L\) be a real number.
  Show that \((a_n)_{n = m}^\infty\) converges to \(L\) if and only if, given any real \(\varepsilon > 0\), one can find an \(N \geq m\) such that \(\abs{a_n - L} \leq \varepsilon\) for all \(n \geq N\).
\end{ex}

\begin{proof}
  \begin{align*}
         & (a_n)_{n = m}^\infty \text{ converges to } L                                                                                        \\
    \iff & \forall \varepsilon \in \R^+, (a_n)_{n = m}^\infty \text{ is eventually } \varepsilon\text{-close to } L & \text{(by \cref{6.1.5})} \\
    \iff & \forall \varepsilon \in \R^+, \exists\ N \in \N \land N \geq m :                                                                    \\
         & (a_n)_{n = N}^\infty \text{ is } \varepsilon\text{-close to } L                                          & \text{(by \cref{6.1.5})} \\
    \iff & \forall \varepsilon \in \R^+, \exists\ N \in \N \land N \geq m :                                                                    \\
         & \forall n \geq N, \abs{a_n - L} \leq \varepsilon.                                                        & \text{(by \cref{6.1.5})}
  \end{align*}
\end{proof}

\begin{ex}\label{ex:6.1.3}
  Let \((a_n)_{n = m}^\infty\) be a sequence of real numbers, let \(c\) be a real number, and let \(m' \geq m\) be an integer.
  Show that \((a_n)_{n = m}^\infty\) converges to \(c\) if and only if \((a_n)_{n = m'}^\infty\) converges to \(c\).
\end{ex}

\begin{proof}
  If \((a_n)_{n = m'}^\infty\) converges to \(c\) for all \(m' \in \N\) and \(m' \geq m\), then obviously \((a_n)_{n = m}^\infty\) converges to \(c\).
  So we only need to show that if \((a_n)_{n = m}^\infty\) converges to \(c\), then \((a_n)_{n = m'}^\infty\) converges to \(c\) for all \(m' \in \N\) and \(m' \geq m\).
  Let \(N \in \N\).
  Then we have
  \begin{align*}
             & (a_n)_{n = m}^\infty \text{ converges to } c                                                                                        \\
    \implies & \forall \varepsilon \in \R^+, (a_n)_{n = m}^\infty \text{ is eventually } \varepsilon\text{-close to } c & \text{(by \cref{6.1.5})} \\
    \implies & \forall \varepsilon \in \R^+, \exists\ N \geq m :                                                                                   \\
             & (a_n)_{n = N}^\infty \text{ is } \varepsilon\text{-close to } c                                          & \text{(by \cref{6.1.5})} \\
    \implies & \forall \varepsilon \in \R^+, \exists\ N \geq m' \geq m :                                                                           \\
             & (a_n)_{n = N}^\infty \text{ is } \varepsilon\text{-close to } c                                                                     \\
    \implies & (a_n)_{n = m}^\infty \text{ converges to } c \text{ for all } m' \geq m
  \end{align*}
  Thus if \((a_n)_{n = m}^\infty\) converges to \(c\), then \((a_n)_{n = m'}^\infty\) converges to \(c\) for all \(m' \geq m\).
\end{proof}

\begin{ex}\label{ex:6.1.4}
  Let \((a_n)_{n = m}^\infty\) be a sequence of real numbers, let \(c\) be a real number, and let \(k \geq 0\) be a non-negative integer.
  Show that \((a_n)_{n = m}^\infty\) converges to \(c\) if and only if \((a_{n + k})_{n = m}^\infty\) converges to \(c\).
\end{ex}

\begin{proof}
  Since \((a_{n + k})_{n = m}^\infty = (a_n)_{n = m + k}^\infty\) and \(m + k \geq m\), by \cref{ex:6.1.3} we have \((a_n)_{n = m}^\infty\) converges to \(c\) if and only if \((a_n)_{n = m + k}^\infty\) converges to \(c\).
  Thus \((a_n)_{n = m}^\infty\) converges to \(c\) if and only if \((a_{n + k})_{n = m}^\infty\) converges to \(c\).
\end{proof}

\begin{ex}\label{ex:6.1.5}
  Prove \cref{6.1.12}.
\end{ex}

\begin{proof}
  See \cref{6.1.12}.
\end{proof}

\begin{ex}\label{ex:6.1.6}
  Prove \cref{6.1.15}.
\end{ex}

\begin{proof}
  See \cref{6.1.15}.
\end{proof}

\begin{ex}\label{ex:6.1.7}
  Show that \cref{6.1.16} is consistent with \cref{5.1.12}
  (i.e., prove an analogue of \cref{6.1.4} for bounded sequences instead of Cauchy sequences).
\end{ex}

\begin{proof}
  First suppose that \((a_n)_{n = m}^\infty\) is a bounded sequence in the sense of \cref{6.1.16};
  then \(\exists\ M \in \R^+\) such that \(\abs{a_n} \leq M\) for every \(n \geq m\).
  By \cref{5.4.12}, \(\exists\ N \in \N\) such that \(M \leq N\).
  Since \(N \in \N\), we also have \(N \in \Q\).
  Thus \(\abs{a_n} \leq N\) for every \(n \geq m\), and \((a_n)_{n = m}^\infty\) is a bounded sequence in the sense of \cref{5.1.12}.

  Now suppose that \((a_n)_{n = m}^\infty\) is a bounded sequence in the sense of \cref{5.1.12};
  then \(\exists\ M \in \Q^+\) such that \(\abs{a_n} \leq M\) for every \(n \geq m\).
  Since \(M\) is also a real number, we see that \((a_n)_{n = m}^\infty\) is a bounded sequence in the sense of \cref{6.1.16}.
\end{proof}

\begin{ex}\label{ex:6.1.8}
  Proof \cref{6.1.19}.
\end{ex}

\begin{proof}
  See \cref{6.1.19}.
\end{proof}

\begin{ex}\label{ex:6.1.9}
  Explain why \cref{6.1.19}(f) fails when the limit of the denominator is \(0\).
\end{ex}

\begin{proof}
  Suppose for sake of contradiction that \cref{6.1.19}(f) works when denominator is \(0\).
  Let \((a_n)_{n = 1}^\infty = (b_n)_{n = 1}^\infty = 1 / n\).
  Then we have
  \[
    \lim_{n \to \infty} a_n / b_n = \lim_{n \to \infty} \frac{1 / n}{1 / n} = \lim_{n \to \infty} 1 = 1.
  \]
  But by \cref{6.1.11} we also have
  \[
    \frac{\lim_{n \to \infty} a_n}{\lim_{n \to \infty} b_n} = \frac{0}{0}
  \]
  which is undefined.
  Thus \cref{6.1.19}(f) fails when denominator is \(0\).
\end{proof}

\begin{ex}\label{ex:6.1.10}
  Show that the concept of equivalent Cauchy sequence, as defined in \cref{5.2.6}, does not change if \(\varepsilon\) is required to be positive real instead of positive rational.
  More precisely, if \((a_n)_{n = 0}^\infty\) and \((b_n)_{n = 0}^\infty\) are sequences of reals, show that \((a_n)_{n = 0}^\infty\) and \((b_n)_{n = 0}^\infty\) are eventually \(\varepsilon\)-close for every rational \(\varepsilon > 0\) if and only if they are eventually \(\varepsilon\)-close for every real \(\varepsilon > 0\).
\end{ex}

\begin{proof}
  Suppose first that \((a_n)_{n = 0}^\infty\) and \((b_n)_{n = 0}^\infty\) are eventually \(\varepsilon\)-close \(\forall \varepsilon \in \Q^+\).
  Let \(\varepsilon' \in \R^+\).
  By \cref{5.4.12} \(\exists\ \varepsilon \in \Q^+\) such that \(\varepsilon \leq \varepsilon'\).
  Since \(\varepsilon \in \Q^+\), we know that \((a_n)_{n = 0}^\infty\) and \((b_n)_{n = 0}^\infty\) are eventually \(\varepsilon\)-close.
  Thus \((a_n)_{n = 0}^\infty\) and \((b_n)_{n = 0}^\infty\) are eventually \(\varepsilon'\)-close.
  Since \(\varepsilon'\) is arbitrary, we have \((a_n)_{n = 0}^\infty\) and \((b_n)_{n = 0}^\infty\) are eventually \(\varepsilon'\)-close \(\forall \varepsilon' \in \R^+\).

  Now suppose that \((a_n)_{n = 0}^\infty\) and \((b_n)_{n = 0}^\infty\) are eventually \(\varepsilon'\)-close \(\forall \varepsilon' \in \R^+\).
  This implies that \((a_n)_{n = 0}^\infty\) and \((b_n)_{n = 0}^\infty\) are eventually \(\varepsilon\)-close for \(\forall \varepsilon \in \Q^+\).
  Thus we conclude that \((a_n)_{n = 0}^\infty\) and \((b_n)_{n = 0}^\infty\) are eventually \(\varepsilon\)-close \(\forall \varepsilon \in \Q^+\) if and only if they are eventually \(\varepsilon'\)-close \(\forall \varepsilon' \in \R^+\).
\end{proof}