\section{Convergence and limit laws}\label{i:sec:6.1}

\begin{defn}[Distance between two real numbers]\label{i:6.1.1}
  Given two real numbers \(x\) and \(y\), we define their distance \(d(x, y)\) to be \(d(x, y) \coloneqq \abs{x - y}\).
\end{defn}

\begin{note}
  Clearly, \cref{i:6.1.1} is consistent with \cref{i:4.3.2}.
  Further, \cref{i:4.3.3} works just as well for real numbers as it does for rationals, because the real numbers obey all the rules of algebra that the rationals do.
\end{note}

\begin{defn}[\(\varepsilon\)-close real numbers]\label{i:6.1.2}
  Let \(\varepsilon \in \R_{\geq 0}\).
  We say that two real numbers \(x, y\) are \emph{\(\varepsilon\)-close} iff we have \(d(y, x) \leq \varepsilon\).
\end{defn}

\begin{note}
  Again, it is clear that \cref{i:6.1.2} is consistent with \cref{i:4.3.4}.
\end{note}

\begin{note}
  Now let \((a_n)_{n = m}^\infty\) be a sequence of \emph{real} numbers;
  i.e., we assign a real number \(a_n\) for every integer \(n \geq m\).
  The starting index \(m\) is some integer;
  usually this will be \(1\), but in some cases we will start from some index other than \(1\).
  (The choice of label used to index this sequence is unimportant; we could use for instance \((a_k)_{k = m}^{\infty}\) and this would represent exactly the same sequence as \((a_n)_{n = m}^{\infty}\).)
  We can define the notion of a Cauchy sequence in the same manner as before.
\end{note}

\begin{defn}[Cauchy sequences of reals]\label{i:6.1.3}
  Let \(\varepsilon \in \R^+\).
  A sequence \((a_n)_{n = N}^\infty\) of real numbers starting at some integer index \(N\) is said to be \emph{\(\varepsilon\)-steady} iff \(a_j\) and \(a_k\) are \(\varepsilon\)-close for every \(j, k \in \Z_{\geq N}\).
  A sequence \((a_n)_{n = m}^\infty\) starting at some integer index \(m\) is said to be \emph{eventually \(\varepsilon\)-steady} iff there exists an \(N \in \Z_{\geq m}\) such that \((a_n)_{n = N}^\infty\) is \(\varepsilon\)-steady.
  We say that \((a_n)_{n = m}^\infty\) is a \emph{Cauchy sequence} iff it is eventually \(\varepsilon\)-steady for every \(\varepsilon \in \R^+\).
\end{defn}

\begin{note}
  To put it another way, a sequence \((a_n)_{n = m}^\infty\) of real numbers is a Cauchy sequence if, for every \(\varepsilon \in \R^+\), there exists an \(N \in \Z_{\geq m}\) such that \(\abs{a_n - a_{n'}} \leq \varepsilon\) for all \(n, n' \in \Z_{\geq N}\).
  These definitions are consistent with the corresponding definitions for rational numbers (\cref{i:5.1.3,i:5.1.6,i:5.1.8}), although verifying consistency for Cauchy sequences takes a little bit of care.
\end{note}

\begin{prop}\label{i:6.1.4}
  Let \((a_n)_{n = m}^\infty\) be a sequence of rational numbers starting at some integer index \(m\).
  Then \((a_n)_{n = m}^\infty\) is a Cauchy sequence in the sense of \cref{i:5.1.8} iff it is a Cauchy sequence in the sense of \cref{i:6.1.3}.
\end{prop}

\begin{proof}[\pf{i:6.1.4}]
  Suppose first that \((a_n)_{n = m}^\infty\) is a Cauchy sequence in the sense of \cref{i:6.1.3};
  then it is eventually \(\varepsilon\)-steady for every \(\varepsilon \in \R^+\).
  In particular, it is eventually \(\varepsilon\)-steady for every \(\varepsilon \in \Q^+\), which makes it a Cauchy sequence in the sense of \cref{i:5.1.8}.

  Now suppose that \((a_n)_{n = m}^\infty\) is a Cauchy sequence in the sense of \cref{i:5.1.8};
  then it is eventually \(\varepsilon'\)-steady for every \(\varepsilon' \in \Q^+\).
  If \(\varepsilon \in \R^+\), then there exists an \(\varepsilon' \in \Q^+\) which is smaller than \(\varepsilon\), by \cref{i:5.4.12}.
  Since \(\varepsilon'\) is rational, we know that \((a_n)_{n = m}^\infty\) is eventually \(\varepsilon'\)-steady;
  since \(\varepsilon' < \varepsilon\), this implies that \((a_n)_{n = m}^\infty\) is eventually \(\varepsilon\)-steady.
  Since \(\varepsilon\) is an arbitrary positive real number, we thus see that \((a_n)_{n = m}^\infty\) is a Cauchy sequence in the sense of \cref{i:6.1.3}.
\end{proof}

\begin{note}
  Because of \cref{i:6.1.4}, we will no longer care about the distinction between \cref{i:5.1.8} and \cref{i:6.1.3}, and view the concept of a Cauchy sequence as a single unified concept.
\end{note}

\begin{defn}[Convergence of sequences]\label{i:6.1.5}
  Let \(\varepsilon \in \R^+\), and let \(L \in \R\).
  A sequence \((a_n)_{n = N}^\infty\) of real numbers is said to be \emph{\(\varepsilon\)-close to \(L\)} iff \(a_n\) is \(\varepsilon\)-close to \(L\) for every \(n \in \Z_{\geq N}\), i.e., we have \(\abs{a_n - L} \leq \varepsilon\) for every \(n \in \Z_{\geq N}\).
  We say that a sequence \((a_n)_{n = m}^\infty\) is \emph{eventually \(\varepsilon\)-close to \(L\)} iff there exists an \(N \in \Z_{\geq m}\) such that \((a_n)_{n = N}^\infty\) is \(\varepsilon\)-close to \(L\).
  We say that a sequence \((a_n)_{n = m}^\infty\) \emph{converges to \(L\)} iff it is eventually \(\varepsilon\)-close to \(L\) for every \(\varepsilon \in \R^+\).
\end{defn}

\setcounter{thm}{6}
\begin{prop}[Uniqueness of limits]\label{i:6.1.7}
  Let \((a_n)_{n = m}^\infty\) be a real sequence starting at some integer index \(m\), and let \(L \neq L'\) be two distinct real numbers.
  Then it is not possible for \((a_n)_{n = m}^\infty\) to converge to \(L\) while also converging to \(L'\).
\end{prop}

\begin{proof}[\pf{i:6.1.7}]
  Suppose for the sake of contradiction that \((a_n)_{n = m}^\infty\) was converging to both \(L\) and \(L'\).
  Let \(\varepsilon = \abs{L - L'} / 3\).
  Note that \(\varepsilon\) is positive since \(L \neq L'\).
  Since \((a_n)_{n = m}^\infty\) converges to \(L\), we know that \((a_n)_{n = m}^\infty\) is eventually \(\varepsilon\)-close to \(L\);
  thus there is an \(N \in \Z_{\geq m}\) such that \(d(a_n, L) \leq \varepsilon\) for all \(n \in \Z_{\geq N}\).
  Similarly, there is an \(M \in \Z_{\geq m}\) such that \(d(a_n, L') \leq \varepsilon\) for all \(n \in \Z_{\geq M}\).
  In particular, if we set \(n \coloneqq \max(N, M)\), then we have \(d(a_n, L) \leq \varepsilon\) and \(d(a_n, L') \leq \varepsilon\), hence by the triangle inequality \(d(L, L') \leq 2\varepsilon = 2\abs{L - L'} / 3\).
  But then we have \(\abs{L - L'} \leq 2\abs{L - L'} / 3\), which contradicts the fact that \(\abs{L - L'} > 0\).
  Thus, it is not possible to converge to both \(L\) and \(L'\).
\end{proof}

\begin{defn}[Limits of sequences]\label{i:6.1.8}
  If a sequence \((a_n)_{n = m}^\infty\) converges to some real number \(L\), we say that \((a_n)_{n = m}^\infty\) is \emph{convergent} and that its \emph{limit} is \(L\);
  we write
  \[
    L = \lim_{n \to \infty} a_n
  \]
  to denote this fact.
  If a sequence \((a_n)_{n = m}^\infty\) is not converging to any real number \(L\), we say that the sequence \((a_n)_{n = m}^\infty\) is \emph{divergent} and we leave \(\lim_{n \to \infty} a_n\) undefined.
\end{defn}

\begin{note}
  \cref{i:6.1.7} ensures that a sequence can have at most one limit.
  Thus, if the limit exists, it is a single real number, otherwise it is undefined.
\end{note}

\begin{rmk}\label{i:6.1.9}
  The notation \(\lim_{n \to \infty} a_n\) does not give any indication about the starting index \(m\) of the sequence, but the starting index is irrelevant (\cref{i:ex:6.1.3}).
  Thus, in the rest of this discussion we shall not be too careful as to where these sequences start, as we shall be mostly focused on their limits.
\end{rmk}

\begin{note}
  We sometimes use the phrase ``\(a_n \to x\) as \(n \to \infty\)'' as an alternate way of writing the statement ``\((a_n)_{n = m}^\infty\) converges to \(x\).''
  Bear in mind, though, that the individual statements \(a_n \to x\) and \(n \to \infty\) do not have any rigorous meaning;
  this phrase is just a convention, though of course a very suggestive one.
\end{note}

\begin{rmk}\label{i:6.1.10}
  The exact choice of letter used to denote the index (in this case \(n\)) is irrelevant:
  the phrase \(\lim_{n \to \infty} a_n\) has exactly the same meaning as \(\lim_{k \to \infty} a_k\), for instance.
  Sometimes it will be convenient to change the label of the index to avoid conflicts of notation;
  for instance, we might want to change \(n\) to \(k\) because \(n\) is simultaneously being used for some other purpose, and we want to reduce confusion.
  See \cref{i:ex:6.1.4}.
\end{rmk}

\begin{prop}\label{i:6.1.11}
  We have \(\lim_{n \to \infty} 1 / n = 0\).
\end{prop}

\begin{proof}[\pf{i:6.1.11}]
  We have to show that the sequence \((a_n)_{n = 1}^\infty\) converges to \(0\), where \(a_n \coloneqq 1 / n\).
  In other words, for every \(\varepsilon \in \R^+\), we need to show that the sequence \((a_n)_{n = 1}^\infty\) is eventually \(\varepsilon\)-close to \(0\).
  So, let \(\varepsilon \in \R^+\) be an arbitrary real number.
  We have to find an \(N \in \Z^+\) such that \(\abs{a_n - 0} \leq \varepsilon\) for every \(n \in \Z_{\geq N}\).
  But if \(n \in \Z_{\geq N}\), then
  \[
    \abs{a_n - 0} = \abs{1 / n - 0} = 1 / n \leq 1 / N.
  \]
  Thus, if we pick \(N > 1 / \varepsilon\) (which we can do by the Archimedean principle, \cref{i:5.4.13}), then \(1 / N < \varepsilon\), and so \((a_n)_{n = N}^\infty\) is \(\varepsilon\)-close to \(0\).
  Thus, \((a_n)_{n = 1}^\infty\) is eventually \(\varepsilon\)-close to \(0\).
  Since \(\varepsilon\) was arbitrary, \((a_n)_{n = 1}^\infty\) converges to \(0\).
\end{proof}

\begin{prop}[Convergent sequences are Cauchy]\label{i:6.1.12}
  Suppose that \((a_n)_{n = m}^\infty\) is a convergent sequence of real numbers.
  Then \((a_n)_{n = m}^\infty\) is also a Cauchy sequence.
\end{prop}

\begin{proof}[\pf{i:6.1.12}]
  Suppose that \((a_n)_{n = m}^\infty\) converges to \(L \in \R\).
  Let \(\varepsilon \in \R^+\).
  Since \(\lim_{n \to \infty} a_n = L\), by \cref{i:6.1.5} there exists an \(N \in \Z_{\geq m}\) such that \((a_n)_{n = N}^\infty\) is \(\varepsilon / 2\)-close to \(L\).
  This means \(\abs{a_n - L} \leq \varepsilon / 2\) for all \(n \in \Z_{\geq N}\).
  Then we have
  \begin{align*}
    \forall j, k \in \Z_{\geq N}, \abs{a_j - a_k} & \leq \abs{a_j - L} + \abs{a_k - L}                                  &  & \by{i:ac:5.4.1}[f,g] \\
                                                  & \leq \dfrac{\varepsilon}{2} + \dfrac{\varepsilon}{2} = \varepsilon. &  & \by{i:5.4.7}[c,d]
  \end{align*}
  Thus, \((a_n)_{n = N}^\infty\) is \(\varepsilon\)-steady and \((a_n)_{n = m}^\infty\) is eventually \(\varepsilon\)-steady.
  Since \(\varepsilon\) was arbitrary, by \cref{i:6.1.3} we see that \((a_n)_{n = m}^\infty\) is a Cauchy sequence of reals.
\end{proof}

\setcounter{thm}{13}
\begin{rmk}\label{i:6.1.14}
  For a converse to \cref{i:6.1.12}, see \cref{i:6.4.18} below.
\end{rmk}

\begin{prop}[Formal limits are genuine limits]\label{i:6.1.15}
  Suppose that \((a_n)_{n = m}^\infty\) is a Cauchy sequence of rational numbers.
  Then \((a_n)_{n = m}^\infty\) converges to \(\LIM_{n \to \infty} a_n\), i.e.
  \[
    \LIM_{n \to \infty} a_n = \lim_{n \to \infty} a_n.
  \]
\end{prop}

\begin{proof}[\pf{i:6.1.15}]
  Let \(L = \LIM_{n \to \infty} a_n\).
  By \cref{i:5.3.1} we know that \(L \in \R\).
  Thus, by \cref{i:6.1.4,i:6.1.5} we can ask whether \((a_n)_{n = m}^\infty\) converges to \(L\).
  Suppose for the sake of contradiction that \((a_n)_{n = m}^\infty\) does not converge to \(L\).
  Then there must exist an \(\varepsilon \in \R^+\) such that \((a_n)_{n = m}^\infty\) is not eventually \(\varepsilon\)-close to \(L\).
  Fix such \(\varepsilon\).

  Since \((a_n)_{n = m}^\infty\) is a Cauchy sequence of reals, we know that there exists an \(N \in \Z_{\geq m}\) such that \(\abs{a_j - a_k} \leq \dfrac{\varepsilon}{4}\) for all \(j, k \in \Z_{\geq N}\).
  Fix such \(N\).
  Since \((a_n)_{n = m}^\infty\) is not eventually \(\varepsilon\)-close to \(L\), we know that \((a_n)_{n = m}^\infty\) is not eventually \(\varepsilon'\)-close to \(L\) for every \(\varepsilon' \in \R_{0 < \varepsilon}\).
  So \((a_n)_{n = m}^\infty\) is not eventually \(\dfrac{2 \varepsilon}{3}\)-close to \(L\), and we can find a \(j \in \Z_{\geq N}\) such that \(\abs{a_j - L} > \dfrac{2 \varepsilon}{3}\).
  Similarly, \((a_n)_{n = m}^\infty\) is not eventually \(\dfrac{\varepsilon}{3}\)-close to \(L\), and we can find a \(k \in \Z_{\geq N}\) such that \(\abs{a_k - L} > \dfrac{\varepsilon}{3}\).
  Fix such \(j\) and \(k\).
  But then we have
  \begin{align*}
    \dfrac{\varepsilon}{4} & \geq \abs{a_j - a_k}                                                             &  & \by{i:6.1.3}      \\
                           & \geq \abs{a_j - L} - \abs{a_k - L}                                               &  & \by{i:4.3.3}[f,g] \\
                           & \geq \dfrac{2 \varepsilon}{3} - \dfrac{\varepsilon}{3} = \dfrac{\varepsilon}{3},
  \end{align*}
  a contradiction.
  Thus, such \(\varepsilon\) does not exist.
  Therefore we must have \(\lim_{n \to \infty} a_n = L = \LIM_{n \to \infty} a_n\).
\end{proof}

\begin{defn}[Bounded sequences]\label{i:6.1.16}
  A sequence \((a_n)_{n = m}^\infty\) of reals is \emph{bounded by} a real number \(M \in \R_{\geq 0}\) iff we have \(\abs{a_n} \leq M\) for all \(n \in \Z_{\geq m}\).
  We say that \((a_n)_{n = m}^\infty\) is bounded iff it is \emph{bounded} by \(M\) for some real number \(M \in \R_{\geq 0}\).
\end{defn}

\begin{note}
  \cref{i:6.1.16} is consistent with \cref{i:5.1.12}.
  See \cref{i:ex:6.1.7}.
\end{note}

\begin{cor}\label{i:6.1.17}
  Every convergent sequence of real numbers is bounded.
\end{cor}

\begin{proof}[\pf{i:6.1.17}]
  Recall from \cref{i:5.1.15} that every Cauchy sequence of rationals is bounded.
  An inspection of the proof of \cref{i:5.1.15} shows that the same argument works for reals;
  every Cauchy sequence of reals is bounded.
  From \cref{i:6.1.12} we see that every convergent sequence of reals is a Cauchy sequence.
  Thus, every convergent sequence of reals is bounded.
\end{proof}

\begin{ac}\label{i:ac:6.1.1}
  Let \(x, y \in \R\).
  Then we have \(\min(x, y) = -\max(-x, -y)\).
  Similarly, we have \(\max(x, y) = -\min(-x, -y)\).
\end{ac}

\begin{proof}[\pf{i:ac:6.1.1}]
  First, we show that \(\min(x, y) = -\max(-x, -y)\).
  We split into two cases:
  \begin{itemize}
    \item If \(x \leq y\), then we have \(-x \geq -y\) by \cref{i:ex:4.2.6}.
          Thus,
          \begin{align*}
            \min(x, y) & = x                                   \\
                       & = -(-x)          &  & \by{i:ac:5.3.3} \\
                       & = -\max(-x, -y). &  & (-x \geq -y)
          \end{align*}
    \item If \(x > y\), then we have \(-x < -y\) by \cref{i:ex:4.2.6}.
          Thus,
          \begin{align*}
            \min(x, y) & = y                                   \\
                       & = -(-y)          &  & \by{i:ac:5.3.3} \\
                       & = -\max(-x, -y). &  & (-x < -y)
          \end{align*}
  \end{itemize}
  From all cases above, we see that \(\min(x, y) = -\max(-x, -y)\).
  Thus, we conclude that \(\min(x, y) = -\max(-x, -y)\).

  Now we show that \(\max(x, y) = -\min(-x, -y)\).
  This is true since
  \begin{align*}
    \max(x, y) & = -(-\max(-(-x), -(-y))) &  & \by{i:ac:5.3.3}               \\
               & = -\min(-x, -y).         &  & \text{(from the proof above)}
  \end{align*}
\end{proof}

\setcounter{thm}{18}
\begin{thm}[Limit Laws]\label{i:6.1.19}
  Let \((a_n)_{n = m}^\infty\) and \((b_n)_{n = m}^\infty\) be convergent sequences of real numbers, and let \(x, y\) be the real numbers \(x \coloneqq \lim_{n \to \infty} a_n\) and \(y \coloneqq \lim_{n \to \infty} b_n\).
  \begin{enumerate}
    \item The sequence \((a_n + b_n)_{n = m}^\infty\) converges to \(x + y\);
          in other words,
          \[
            \lim_{n \to \infty} (a_n + b_n) = \pa{\lim_{n \to \infty} a_n} + \pa{\lim_{n \to \infty} b_n}.
          \]
    \item The sequence \((a_n b_n)_{n = m}^\infty\) converges to \(xy\);
          in other words,
          \[
            \lim_{n \to \infty} (a_n b_n) = \pa{\lim_{n \to \infty} a_n} \pa{\lim_{n \to \infty} b_n}.
          \]
    \item For any real number \(c\), the sequence \((c a_n)_{n = m}^\infty\) converges to \(cx\);
          in other words,
          \[
            \lim_{n \to \infty} (c a_n) = c \pa{\lim_{n \to \infty} a_n}.
          \]
    \item The sequence \((a_n - b_n)_{n = m}^\infty\) converges to \(x - y\);
          in other words,
          \[
            \lim_{n \to \infty} (a_n - b_n) = \pa{\lim_{n \to \infty} a_n} - \pa{\lim_{n \to \infty} b_n}.
          \]
    \item Suppose that \(y \neq 0\), and that \(b_n \neq 0\) for all \(n \geq m\).
          Then the sequence \((b_n^{-1})_{n = m}^\infty\) converges to \(y^{-1}\);
          in other words,
          \[
            \lim_{n \to \infty} b_n^{-1} = \pa{\lim_{n \to \infty} b_n}^{-1}.
          \]
    \item Suppose that \(y \neq 0\), and that \(b_n \neq 0\) for all \(n \in \Z_{\geq m}\).
          Then the sequence \((a_n / b_n)_{n = m}^\infty\) converges to \(x / y\);
          in other words,
          \[
            \lim_{n \to \infty} \dfrac{a_n}{b_n} = \dfrac{\lim_{n \to \infty} a_n}{\lim_{n \to \infty} b_n}.
          \]
    \item The sequence \((\max(a_n, b_n))_{n = m}^\infty\) converges to \(\max(x, y)\);
          in other words,
          \[
            \lim_{n \to \infty} \max(a_n, b_n) = \max\pa{\lim_{n \to \infty} a_n, \lim_{n \to \infty} b_n}.
          \]
    \item The sequence \((\min(a_n, b_n))_{n = m}^\infty\) converges to \(\min(x, y)\);
          in other words,
          \[
            \lim_{n \to \infty} \min(a_n, b_n) = \min\pa{\lim_{n \to \infty} a_n, \lim_{n \to \infty} b_n}.
          \]
  \end{enumerate}
\end{thm}

\begin{proof}[\pf{i:6.1.19}(a)]
  Let \(\varepsilon \in \R^+\).
  By \cref{i:6.1.8}, there exists an \(N_a \in \Z_{\geq m}\) such that \(\abs{a_n - x} \leq \varepsilon / 2\) for every \(n \in \Z_{\geq N_a}\).
  Similarly, there exists an \(N_b \in \Z_{\geq m}\) such that \(\abs{b_n - y} \leq \varepsilon / 2\) for every \(n \in \Z_{\geq N_b}\).
  Now we fix both \(N_a\) and \(N_b\).
  Let \(N = \max(N_a, N_b)\).
  Then we have
  \begin{align*}
    \forall n \in \Z_{\geq N}, \abs{(a_n + b_n) - (x + y)} & = \abs{(a_n - x) + (b_n - y)}                         &  & \by{i:ac:5.3.3}   \\
                                                           & \leq \abs{a_n - x} + \abs{b_n - y}                    &  & \by{i:ac:5.4.1}   \\
                                                           & \leq \varepsilon / 2 + \varepsilon / 2 = \varepsilon. &  & \by{i:5.4.7}[c,d]
  \end{align*}
  Thus, by \cref{i:6.1.5}, \((a_n + b_n)_{n = N}^\infty\) is \(\varepsilon\)-close to \(x + y\), and \((a_n + b_n)_{n = m}^\infty\) is eventually \(\varepsilon\)-close to \(x + y\).
  Since \(\varepsilon\) was arbitrary, by \cref{i:6.1.5} again we know that \((a_n + b_n)_{n = m}^\infty\) converges to \(x + y\).
\end{proof}

\begin{proof}[\pf{i:6.1.19}(b)]
  Since \(x = \lim_{n \to \infty} a_n\) and \(y = \lim_{n \to \infty} b_n\), by \cref{i:6.1.17}, there exist some \(A, B \in \R_{\geq 0}\) such that \(\abs{a_n} \leq A\) and \(\abs{b_n} \leq B\) for all \(n \in \Z_{\geq m}\).
  Fix both \(A\) and \(B\).
  Clearly, we have \(A < \abs{x} + A + 1\) and \(B < B + 1\), so we must have \(\abs{a_n} \leq \abs{x} + A + 1\) and \(\abs{b_n} \leq B + 1\) for all \(n \in \Z_{\geq m}\).

  Let \(\varepsilon \in \R^+\).
  Observe that \(\dfrac{\varepsilon}{2 (\abs{x} + A + 1)} \in \R^+\) and \(\dfrac{\varepsilon}{2 (B + 1)} \in \R^+\).
  Since \(x = \lim_{n \to \infty} a_n\), by \cref{i:6.1.8}, there exists an \(N_a \in \Z_{\geq m}\) such that \(\abs{a_n - x} \leq \dfrac{\varepsilon}{2(B + 1)}\) for all \(n \in \Z_{\geq N_a}\).
  Similarly, since \(y = \lim_{n \to \infty} b_n\), there exists an \(N_b \in \Z_{\geq m}\) such that \(\abs{b_n - y} \leq \dfrac{\varepsilon}{2 (\abs{x} + A + 1)}\) for all \(n \in \Z_{\geq N_b}\).
  Now we fix both \(N_a\) and \(N_b\).
  Let \(N = \max(N_a, N_b)\).
  Then we have
  \begin{align*}
    \forall n \in \Z_{\geq N}, \abs{a_n b_n - x y} & = \abs{a_n b_n - x y + x b_n - x b_n}                                                           &  & \by{i:ac:5.3.3}     \\
                                                   & = \abs{b_n(a_n - x) + x(b_n - y)}                                                               &  & \by{i:ac:5.3.3}     \\
                                                   & \leq \abs{b_n(a_n - x)} + \abs{x(b_n - y)}                                                      &  & \by{i:ac:5.4.1}     \\
                                                   & = \abs{b_n}\abs{a_n - x} + \abs{x}\abs{b_n - y}                                                 &  & \by{i:ac:5.4.1}     \\
                                                   & \leq (B + 1) \times \dfrac{\varepsilon}{2 (B + 1)} + \abs{x}\abs{b_n - y}                       &  & \by{i:5.4.7}[c,d,e] \\
                                                   & \leq \dfrac{\varepsilon}{2} + (\abs{x} + A + 1) \times \dfrac{\varepsilon}{2 (\abs{x} + A + 1)} &  & \by{i:5.4.7}[c,d,e] \\
                                                   & = \dfrac{\varepsilon}{2} + \dfrac{\varepsilon}{2} = \varepsilon.
  \end{align*}
  Thus, by \cref{i:6.1.5}, \((a_n b_n)_{n = N}^\infty\) is \(\varepsilon\)-close to \(xy\), and \((a_n b_n)_{n = m}^\infty\) is eventually \(\varepsilon\)-close to \(xy\).
  Since \(\varepsilon\) was arbitrary, by \cref{i:6.1.5} again we know that \((a_n b_n)_{n = m}^\infty\) converges to \(xy\).
\end{proof}

\begin{proof}[\pf{i:6.1.19}(c)]
  Let \((c_n)_{n = m}^\infty\) be a sequence of reals where \(c_n = c\) for all \(n \in \Z_{\geq m}\).
  Clearly, we have \(\lim_{n \to \infty} c_n = \lim_{n \to \infty} c = c\).
  Then we have
  \begin{align*}
    \lim_{n \to \infty} (c a_n) & = \lim_{n \to \infty} (c_n a_n)                                                   \\
                                & = \pa{\lim_{n \to \infty} c_n} \pa{\lim_{n \to \infty} a_n} &  & \by{i:6.1.19}[b] \\
                                & = c \pa{\lim_{n \to \infty} a_n}.
  \end{align*}
\end{proof}

\begin{proof}[\pf{i:6.1.19}(d)]
  We have
  \begin{align*}
    \lim_{n \to \infty} (a_n - b_n) & = \lim_{n \to \infty} (a_n + (-1)(b_n))                                &  & \by{i:ac:5.3.2}  \\
                                    & = \pa{\lim_{n \to \infty} a_n} + \pa{\lim_{n \to \infty} ((-1)(b_n))}  &  & \by{i:6.1.19}[a] \\
                                    & = \pa{\lim_{n \to \infty} a_n} + \pa{(-1)\pa{\lim_{n \to \infty} b_n}} &  & \by{i:6.1.19}[c] \\
                                    & = \pa{\lim_{n \to \infty} a_n} - \pa{\lim_{n \to \infty} b_n}.         &  & \by{i:ac:5.3.2}
  \end{align*}
\end{proof}

\begin{proof}[\pf{i:6.1.19}(e)]
  First, we show that \((b_n)_{n = m}^\infty\) is bounded away from zero.
  Since \(y \neq 0\), we know that \(\abs{y} > 0\).
  Since \(y = \lim_{n \to \infty} b_n\), we know that there exists an \(N \in \Z_{\geq m}\) such that \(\abs{b_n - y} \leq \dfrac{\abs{y}}{2}\) for all \(n \in \Z_{\geq N}\).
  Then we have
  \begin{align*}
             & \forall n \in \Z_{\geq N}, \dfrac{-\abs{y}}{2} \leq b_n - y \leq \dfrac{\abs{y}}{2} &  & \by{i:ac:5.4.1} \\
    \implies & \forall n \in \Z_{\geq N}, \begin{dcases}
                                            \dfrac{y}{2} \leq b_n \leq \dfrac{3y}{2} & \text{if } y \in \R^+ \\
                                            \dfrac{3y}{2} \leq b_n \leq \dfrac{y}{2} & \text{if } y \in \R^-
                                          \end{dcases}                 &  & \by{i:5.4.7}[c,d]              \\
    \implies & \forall n \in \Z_{\geq N}, \begin{dcases}
                                            \dfrac{y}{2} \leq b_n   & \text{if } y \in \R^+ \\
                                            \dfrac{-y}{2} \leq -b_n & \text{if } y \in \R^-
                                          \end{dcases}                                  &  & \by{i:ex:4.2.6}            \\
    \implies & \forall n \in \Z_{\geq N}, \abs{\dfrac{y}{2}} \leq \abs{b_n}.                       &  & \by{i:ac:5.4.1}
  \end{align*}
  Since \(y \neq 0\), we see that \(\abs{\dfrac{y}{2}} \in \R^+\).
  Thus, \((b_n)_{n = N}^\infty\) is bounded away from zero.
  Since \(b_n \neq 0\) for all \(n \in \Z_{\geq m}\), we see that \((b_n)_{n = m}^{N - 1}\) is also bounded away from zero.
  Combining the results we see that \((b_n)_{n = m}^\infty\) is bounded away from zero.

  Now we show that \(\lim_{n \to \infty} b_n^{-1} = y^{-1}\).
  Let \(\varepsilon \in \R^+\).
  Since \((b_n)_{n = m}^\infty\) is bounded away from zero, there exists an \(M \in \R^+\) such that \(\abs{b_n} \geq M\) for all \(n \in \Z_{\geq m}\).
  Fix such \(M\).
  Clearly, we have \(\varepsilon M \abs{y} \in \R^+\) and \(\dfrac{1}{\abs{b_n}} \leq \dfrac{1}{M}\) for all \(n \in \Z_{\geq m}\).
  Since \(y = \lim_{n \to \infty} b_n \neq 0\), by \cref{i:6.1.8}, there exists an \(N \in \Z_{\geq m}\) such that \(\abs{b_n - y} \leq \varepsilon M \abs{y}\) for all \(n \in \Z_{\geq N}\).
  Fix such \(N\).
  Then we have
  \begin{align*}
    \forall n \in \Z_{\geq N}, \abs{b_n^{-1} - y^{-1}} & = \abs{\dfrac{1}{b_n} - \dfrac{1}{y}}                       &  & \by{i:5.6.2}    \\
                                                       & = \abs{\dfrac{y - b_n}{b_n y}}                              &  & \by{i:ac:5.3.3} \\
                                                       & = \abs{y - b_n}\dfrac{1}{\abs{b_n}\abs{y}}                  &  & \by{i:ac:5.4.1} \\
                                                       & \leq \abs{y - b_n}\dfrac{1}{M\abs{y}}                       &  & \by{i:5.4.7}[e] \\
                                                       & \leq \varepsilon M\abs{y}\dfrac{1}{M\abs{y}} = \varepsilon. &  & \by{i:5.4.7}[e]
  \end{align*}
  Thus, by \cref{i:6.1.5}, \(\pa{b_n^{-1}}_{n = N}^\infty\) is \(\varepsilon\)-close to \(y^{-1}\), and \(\pa{b_n^{-1}}_{n = m}^\infty\) is eventually \(\varepsilon\)-close to \(y^{-1}\).
  Since \(\varepsilon\) was arbitrary, by \cref{i:6.1.5} again we know that \(\pa{b_n^{-1}}_{n = m}^\infty\) converges to \(y^{-1}\).
\end{proof}

\begin{proof}[\pf{i:6.1.19}(f)]
  We have
  \begin{align*}
    \lim_{n \to \infty} \dfrac{a_n}{b_n} & = \lim_{n \to \infty} \pa{a_n b_n^{-1}}                          &  & \by{i:ac:5.3.5}  \\
                                         & = \pa{\lim_{n \to \infty} a_n} \pa{\lim_{n \to \infty} b_n^{-1}} &  & \by{i:6.1.19}[b] \\
                                         & = \pa{\lim_{n \to \infty} a_n} \pa{\lim_{n \to \infty} b_n}^{-1} &  & \by{i:6.1.19}[e] \\
                                         & = \dfrac{\lim_{n \to \infty} a_n}{\lim_{n \to \infty} b_n}.      &  & \by{i:ac:5.3.5}
  \end{align*}
\end{proof}

\begin{proof}[\pf{i:6.1.19}(g)]
  First, suppose that \(x = y\).
  Let \(\varepsilon \in \R^+\).
  By \cref{i:6.1.8}, there exists an \(N_a \in \Z_{\geq m}\) such that \(\abs{a_n - x} \leq \varepsilon\) for all \(n \in \Z_{\geq N_a}\).
  Similarly, there exists an \(N_b \in \Z_{\geq m}\) such that \(\abs{b_n - y} \leq \varepsilon\) for all \(n \in \Z_{\geq N_b}\).
  Fix both \(N_a\) and \(N_b\).
  Let \(N = \max(N_a, N_b)\).
  Then we have \(\abs{a_n - x} \leq \varepsilon\) and \(\abs{b_n - y} \leq \varepsilon\) for all \(n \in \Z_{\geq N}\).
  Since \(x = y\), we have \(\max(x, y) = x\).
  Thus,
  \begin{align*}
             & \forall n \in \Z_{\geq N}, \begin{dcases}
                                            \abs{a_n - \max(x, y)} = \abs{a_n - x} \leq \varepsilon \\
                                            \abs{b_n - \max(x, y)} = \abs{b_n - x} = \abs{b_n - y} \leq \varepsilon
                                          \end{dcases}                       \\
    \implies & \forall n \in \Z_{\geq N}, \abs{\max(a_n, b_n) - \max(x, y)} = \begin{dcases}
                                                                                \abs{a_n - x} \leq \varepsilon & \text{if } a_n \geq b_n \\
                                                                                \abs{b_n - x} \leq \varepsilon & \text{if } a_n < b_n
                                                                              \end{dcases} \\
    \implies & \forall n \in \Z_{\geq N}, \abs{\max(a_n, b_n) - \max(x, y)} \leq \varepsilon.
  \end{align*}
  By \cref{i:6.1.5}, \(\pa{\max(a_n, b_n)}_{n = N}^\infty\) is \(\varepsilon\)-close to \(\max(x, y)\), and \(\pa{\max(a_n, b_n)}_{n = m}^\infty\) is eventually \(\varepsilon\)-close to \(\max(x, y)\).
  Since \(\varepsilon\) was arbitrary, by \cref{i:6.1.5} again we know that \(\pa{\max(a_n, b_n)}_{n = m}^\infty\) converges to \(\max(x, y)\).

  Now suppose that \(x \neq y\).
  We have either \(x < y\) or \(x > y\), so without the loss of generality, suppose that \(x < y\).
  Then we have \(\dfrac{y - x}{2} \in \R^+\).
  Let \(\varepsilon \in \R^+\).
  Clearly, we have \(\min\pa{\varepsilon, \dfrac{y - x}{2}} \in \R^+\).
  By \cref{i:6.1.8}, there exists an \(N_a \in \Z_{\geq m}\) such that \(\abs{a_n - x} \leq \min\pa{\varepsilon, \dfrac{y - x}{2}}\) for all \(n \in \Z_{\geq N_a}\).
  Similarly, there exists an \(N_b \in \Z_{\geq m}\) such that \(\abs{b_n - y} \leq \min\pa{\varepsilon, \dfrac{y - x}{2}}\) for all \(n \in \Z_{\geq N_b}\).
  Fix both \(N_a\) and \(N_b\).
  Let \(N = \max(N_a, N_b)\).
  Then we have
  \begin{align*}
             & \forall n \in \Z_{\geq N}, \begin{dcases}
                                            \abs{a_n - x} \leq \min\pa{\varepsilon, \dfrac{y - x}{2}} \leq \dfrac{y - x}{2} \\
                                            \abs{b_n - y} \leq \min\pa{\varepsilon, \dfrac{y - x}{2}} \leq \dfrac{y - x}{2}
                                          \end{dcases} &  & \by{i:5.4.7}[c] \\
    \implies & \forall n \in \Z_{\geq N}, \begin{dcases}
                                            -\dfrac{y - x}{2} \leq a_n - x \leq \dfrac{y - x}{2} \\
                                            -\dfrac{y - x}{2} \leq b_n - y \leq \dfrac{y - x}{2}
                                          \end{dcases}                                   &  & \by{i:ac:5.4.1}                    \\
    \implies & \forall n \in \Z_{\geq N}, \begin{dcases}
                                            a_n \leq \dfrac{y - x}{2} + x \\
                                            y - \dfrac{y - x}{2} \leq b_n
                                          \end{dcases}                                                 &  & \by{i:5.4.7}[d]      \\
    \implies & \forall n \in \Z_{\geq N}, \begin{dcases}
                                            a_n \leq \dfrac{x + y}{2} \\
                                            \dfrac{x + y}{2} \leq b_n
                                          \end{dcases}                                                 &  & \by{i:ac:5.3.3}      \\
    \implies & \forall n \in \Z_{\geq N}, a_n \leq \dfrac{x + y}{2} \leq b_n.                            &  & \by{i:5.4.7}[c]
  \end{align*}
  This means \(\max(a_n, b_n) = b_n\) for all \(n \in \Z_{\geq N}\).
  Thus,
  \begin{align*}
    \forall n \in \Z_{\geq N}, \abs{\max(a_n, b_n) - \max(x, y)} & = \abs{b_n - y}                                                  \\
                                                                 & \leq \min\pa{\varepsilon, \dfrac{y - x}{2}}                      \\
                                                                 & \leq \varepsilon.                           &  & \by{i:5.4.7}[c]
  \end{align*}
  By \cref{i:6.1.5}, \(\pa{\max(a_n, b_n)}_{n = N}^\infty\) is \(\varepsilon\)-close to \(\max(x, y)\), and \(\pa{\max(a_n, b_n)}_{n = m}^\infty\) is eventually \(\varepsilon\)-close to \(\max(x, y)\).
  Since \(\varepsilon\) was arbitrary, by \cref{i:6.1.5} again we know that \(\pa{\max(a_n, b_n)}_{n = m}^\infty\) converges to \(\max(x, y)\).
\end{proof}

\begin{proof}[\pf{i:6.1.19}(h)]
  We have
  \begin{align*}
    \lim_{n \to \infty} \min(a_n, b_n) & = \lim_{n \to \infty} -\max(-a_n, -b_n)                                  &  & \by{i:ac:6.1.1}  \\
                                       & = -\pa{\lim_{n \to \infty} \max(-a_n, -b_n)}                             &  & \by{i:6.1.19}[c] \\
                                       & = -\max\pa{\lim_{n \to \infty} -a_n, \lim_{n \to \infty} -b_n}           &  & \by{i:6.1.19}[g] \\
                                       & = -\max\pa{-\pa{\lim_{n \to \infty} a_n}, -\pa{\lim_{n \to \infty} b_n}} &  & \by{i:6.1.19}[c] \\
                                       & = \min\pa{\lim_{n \to \infty} a_n, \lim_{n \to \infty} b_n}.             &  & \by{i:ac:6.1.1}
  \end{align*}
\end{proof}

\exercisesection

\begin{ex}\label{i:ex:6.1.1}
  Let \((a_n)_{n = m}^\infty\) be a sequence of reals, such that \(a_{n + 1} > a_n\) for each \(n \in \Z_{\geq m}\).
  Prove that whenever \(j, k \in \Z_{\geq m}\) such that \(j > k\), then we have \(a_j > a_k\).
  (We refer to these sequences as \emph{increasing} sequences.)
\end{ex}

\begin{proof}[\pf{i:ex:6.1.1}]
  Let \(E = \set{z \in \Z_{\geq m} : j \leq z \leq k}\).
  Then \(E\) is finite (since \(\#(E) = k - j + 1\)) and non-empty (since \(j, k \in E\)).
  So \((a_n)_{n = j}^k\) is a finite sequence, and the elements in \((a_n)_{n = j}^k\) are \(\set{a_j, a_{j + 1}, \dots, a_{k - 1}, a_k}\).
  By hypothesis, we have \(a_{n + 1} > a_n\) for each \(n \in \Z_{\geq m}\).
  Thus, we have \(a_j < a_{j + 1} < \dots < a_{k - 1} < a_k\), and by \cref{i:5.4.7}(c) we have \(a_j < a_k\).
\end{proof}

\begin{ex}\label{i:ex:6.1.2}
  Let \((a_n)_{n = m}^\infty\) be a sequence of reals, and let \(L \in \R\).
  Show that \((a_n)_{n = m}^\infty\) converges to \(L\) iff, given any \(\varepsilon \in \R^+\), one can find an \(N \in \Z_{\geq m}\) such that \(\abs{a_n - L} \leq \varepsilon\) for all \(n \in \Z_{\geq N}\).
\end{ex}

\begin{proof}[\pf{i:ex:6.1.2}]
  We have
  \begin{align*}
         & (a_n)_{n = m}^\infty \text{ converges to } L                                                                                                \\
    \iff & \forall \varepsilon \in \R^+, (a_n)_{n = m}^\infty \text{ is eventually } \varepsilon\text{-close to } L                  &  & \by{i:6.1.5} \\
    \iff & \forall \varepsilon \in \R^+, \exists N \in \Z_{\geq m} : (a_n)_{n = N}^\infty \text{ is } \varepsilon\text{-close to } L &  & \by{i:6.1.5} \\
    \iff & \forall \varepsilon \in \R^+, \exists N \in \Z_{\geq m} : \forall n \in \Z_{\geq N}, \abs{a_n - L} \leq \varepsilon.      &  & \by{i:6.1.5}
  \end{align*}
\end{proof}

\begin{ex}\label{i:ex:6.1.3}
  Let \((a_n)_{n = m}^\infty\) be a sequence of reals, let \(c \in \R\), and let \(m' \in \Z_{\geq m}\).
  Show that \((a_n)_{n = m}^\infty\) converges to \(c\) iff \((a_n)_{n = m'}^\infty\) converges to \(c\).
\end{ex}

\begin{proof}[\pf{i:ex:6.1.3}]
  First, suppose that \((a_n)_{n = m}^\infty\) converges to \(c\).
  Let \(\varepsilon \in \R^+\).
  By \cref{i:6.1.5}, there exists an \(N \in \Z_{\geq m}\) such that \(\abs{a_n - c} \leq \varepsilon\) for all \(n \in \Z_{\geq N}\).
  Fix such \(N\).
  Now we split into two cases:
  \begin{itemize}
    \item If \(N \geq m'\), then we have found an \(N \in \Z_{\geq m'}\) such that \(\abs{a_n - c} \leq \varepsilon\) for all \(n \in \Z_{\geq N}\).
    \item If \(N < m'\), then by setting \(N' = m'\) we see that there exists an \(N' \in \Z_{\geq m'}\) such that \(\abs{a_n - c} \leq \varepsilon\) for all \(n \in \Z_{\geq N'}\).
  \end{itemize}
  From all cases above, we see that we can find an \(M \in \Z_{\geq m'}\) such that \(\abs{a_n - c} \leq \varepsilon\) for all \(n \in \Z_{\geq M}\).
  Thus, by \cref{i:6.1.5}, we see that \((a_n)_{n = m'}^\infty\) is eventually \(\varepsilon\)-close to \(c\).
  Since \(\varepsilon\) was arbitrary, by \cref{i:6.1.5} again we see that \((a_n)_{n = m'}^\infty\) converges to \(c\).

  Now suppose that \((a_n)_{n = m'}^\infty\) converges to \(c\).
  Let \(\varepsilon \in \R^+\).
  By \cref{i:6.1.5}, there exists an \(N \in \Z_{\geq m'}\) such that \(\abs{a_n - c} \leq \varepsilon\) for all \(n \in \Z_{\geq N}\).
  Fix such \(N\).
  Since \(m \leq m'\), we have found an \(N \in \Z_{\geq m}\) such that \(\abs{a_n - c} \leq \varepsilon\) for all \(n \in \Z_{\geq N}\).
  Thus, by \cref{i:6.1.5}, we see that \((a_n)_{n = m}^\infty\) is eventually \(\varepsilon\)-close to \(c\).
  Since \(\varepsilon\) was arbitrary, by \cref{i:6.1.5} again we see that \((a_n)_{n = m}^\infty\) converges to \(c\).
\end{proof}

\begin{ex}\label{i:ex:6.1.4}
  Let \((a_n)_{n = m}^\infty\) be a sequence of reals, let \(c \in \R\), and let \(k \in \Z_{\geq 0}\).
  Show that \((a_n)_{n = m}^\infty\) converges to \(c\) iff \((a_{n + k})_{n = m}^\infty\) converges to \(c\).
\end{ex}

\begin{proof}[\pf{i:ex:6.1.4}]
  First, observe that \((a_{n + k})_{n = m}^\infty = (a_n)_{n = m + k}^\infty\).
  Since \(m + k \geq m\), by \cref{i:ex:6.1.3} we see that \((a_n)_{n = m}^\infty\) converges to \(c\) iff \((a_n)_{n = m + k}^\infty\) converges to \(c\).
  Thus, \((a_n)_{n = m}^\infty\) converges to \(c\) iff \((a_{n + k})_{n = m}^\infty\) converges to \(c\).
\end{proof}

\begin{ex}\label{i:ex:6.1.5}
  Prove \cref{i:6.1.12}.
\end{ex}

\begin{proof}[\pf{i:ex:6.1.5}]
  See \cref{i:6.1.12}.
\end{proof}

\begin{ex}\label{i:ex:6.1.6}
  Prove \cref{i:6.1.15}.
\end{ex}

\begin{proof}[\pf{i:ex:6.1.6}]
  See \cref{i:6.1.15}.
\end{proof}

\begin{ex}\label{i:ex:6.1.7}
  Show that \cref{i:6.1.16} is consistent with \cref{i:5.1.12}
  (i.e., prove an analogue of \cref{i:6.1.4} for bounded sequences instead of Cauchy sequences).
\end{ex}

\begin{proof}[\pf{i:ex:6.1.7}]
  First, suppose that \((a_n)_{n = m}^\infty\) is a sequence of reals which is bounded in the sense of \cref{i:6.1.16}.
  Then there exists an \(M \in \R_{\geq 0}\) such that \(\abs{a_n} \leq M\) for all \(n \in \Z_{\geq m}\).
  By \cref{i:5.4.12}, there exists an \(M' \in \Z^+\) such that \(M \leq M'\).
  Clearly, \(M' \in \Z^+\) implies \(M' \in \Q_{\geq 0}\).
  Thus, by \cref{i:5.4.7}(c), we have \(\abs{a_n} \leq M'\) for all \(n \in \Z_{\geq m}\).
  This means that \((a_n)_{n = m}^\infty\) is a bounded sequence in the sense of \cref{i:5.1.12}.

  Now suppose that \((a_n)_{n = m}^\infty\) is a sequence of reals which is bounded in the sense of \cref{i:5.1.12}.
  Then there exists an \(M \in \Q_{\geq 0}\) such that \(\abs{a_n} \leq M\) for all \(n \in \Z_{\geq m}\).
  Since \(M\) is also a real number, we see that \((a_n)_{n = m}^\infty\) is a bounded sequence in the sense of \cref{i:6.1.16}.
\end{proof}

\begin{ex}\label{i:ex:6.1.8}
  Proof \cref{i:6.1.19}.
\end{ex}

\begin{proof}[\pf{i:ex:6.1.8}]
  See \cref{i:6.1.19}.
\end{proof}

\begin{ex}\label{i:ex:6.1.9}
  Explain why \cref{i:6.1.19}(f) fails when the limit of the denominator is \(0\).
  (To repair that problem requires \emph{L'Hôpital's rule}, see \cref{i:sec:10.5}.)
\end{ex}

\begin{proof}[\pf{i:ex:6.1.9}]
  Suppose for the sake of contradiction that \cref{i:6.1.19}(f) works when denominator is \(0\).
  Let \((a_n)_{n = 1}^\infty = (1 / n)_{n = 1}^\infty\).
  Then we have
  \[
    \lim_{n \to \infty} \dfrac{a_n}{a_n} = \lim_{n \to \infty} \dfrac{1 / n}{1 / n} = \lim_{n \to \infty} 1 = 1.
  \]
  But by \cref{i:6.1.11} we also have
  \[
    \dfrac{\lim_{n \to \infty} a_n}{\lim_{n \to \infty} a_n} = \dfrac{0}{0}
  \]
  which is undefined.
  Thus, \cref{i:6.1.19}(f) fails when denominator is \(0\).
\end{proof}

\begin{ex}\label{i:ex:6.1.10}
  Show that the concept of equivalent Cauchy sequence, as defined in \cref{i:5.2.6}, does not change if \(\varepsilon\) is required to be positive real instead of positive rational.
  More precisely, if \((a_n)_{n = m}^\infty\) and \((b_n)_{n = m}^\infty\) are sequences of reals, show that \((a_n)_{n = m}^\infty\) and \((b_n)_{n = m}^\infty\) are eventually \(\varepsilon\)-close for every \(\varepsilon \in \Q^+\) iff they are eventually \(\varepsilon\)-close for every \(\varepsilon \in \R^+\).
\end{ex}

\begin{proof}[\pf{i:ex:6.1.10}]
  Suppose first that \((a_n)_{n = m}^\infty\) and \((b_n)_{n = m}^\infty\) are eventually \(\varepsilon\)-close for all \(\varepsilon \in \Q^+\).
  Let \(\varepsilon' \in \R^+\).
  By \cref{i:5.4.12}, there exists an \(\varepsilon \in \Q^+\) such that \(\varepsilon \leq \varepsilon'\).
  Fix such \(\varepsilon\).
  Since \(\varepsilon \in \Q^+\), by hypothesis we know that \((a_n)_{n = m}^\infty\) and \((b_n)_{n = m}^\infty\) are eventually \(\varepsilon\)-close.
  This implies that \((a_n)_{n = m}^\infty\) and \((b_n)_{n = m}^\infty\) are eventually \(\varepsilon'\)-close.
  Since \(\varepsilon'\) was arbitrary, we see that \((a_n)_{n = m}^\infty\) and \((b_n)_{n = m}^\infty\) are eventually \(\varepsilon'\)-close for all \(\varepsilon' \in \R^+\).

  Now suppose that \((a_n)_{n = m}^\infty\) and \((b_n)_{n = m}^\infty\) are eventually \(\varepsilon'\)-close for all \(\varepsilon' \in \R^+\).
  This implies that \((a_n)_{n = m}^\infty\) and \((b_n)_{n = m}^\infty\) are eventually \(\varepsilon\)-close for all \(\varepsilon \in \Q^+\).
  Thus, we conclude that \((a_n)_{n = m}^\infty\) and \((b_n)_{n = m}^\infty\) are eventually \(\varepsilon\)-close for all \(\varepsilon \in \Q^+\) iff they are eventually \(\varepsilon'\)-close for all \(\varepsilon' \in \R^+\).
\end{proof}
