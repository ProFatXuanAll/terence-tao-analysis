\section{Images and inverse images}\label{i:sec:3.4}

\begin{defn}[Images of sets]\label{i:3.4.1}
  If \(f : X \to Y\) is a function from \(X\) to \(Y\), and \(S\) is a subset of \(X\), we define \(f(S)\) to be the set
  \[
    f(S) \coloneqq \set{f(x) : x \in S};
  \]
  this set is a subset of \(Y\), and is sometimes called the \emph{image} of \(S\) under the map \(f\).
  We sometimes call \(f(S)\) the \emph{forward image} of \(S\) to distinguish it from the concept of the \emph{inverse image} \(f^{-1}(S)\) of \(S\).
\end{defn}

\begin{ac}\label{i:ac:3.4.1}
  Let \(f : X \to Y\) be a function and let \(S \subseteq X\).
  The set \(f(S)\) is well-defined thanks to the axiom of replacement (\cref{i:3.6}).
  One can also define \(f(S)\) using the axiom of specification (\cref{i:3.5}) instead of replacement.
\end{ac}

\begin{proof}[\pf{i:ac:3.4.1}]
  First we use \cref{i:3.4.1} to create the set \(f(X) = \set{f(x) : x \in X}\).
  Now we use \cref{i:3.5} to create the set
  \[
    E = \set{y \in f(X) \mid \exists x \in S : f(x) = y}.
  \]
  By \cref{i:3.1.4,i:3.4.1}, we see that \(E = f(S)\).
  Thus, we can define \(f(S)\) by \cref{i:3.5} instead of \cref{i:3.6}.
\end{proof}

\setcounter{thm}{3}
\begin{defn}[Inverse images]\label{i:3.4.4}
  If \(U\) is a subset of \(Y\), we define the set \(f^{-1}(U)\) to be the set
  \[
    f^{-1}(U) \coloneqq \set{x \in X : f(x) \in U}.
  \]
  In other words, \(f^{-1}(U)\) consists of all the elements of \(X\) which map into \(U\):
  \[
    f(x) \in U \iff x \in f^{-1}(U).
  \]
  We call \(f^{-1}(U)\) the \emph{inverse image} of \(U\).
\end{defn}

\setcounter{thm}{6}
\begin{rmk}\label{i:3.4.6}
  If \(f\) is a bijective function, then we have defined \(f^{-1}\) in two slightly different ways, but this is not an issue because both definitions are equivalent (\cref{i:ex:3.4.1}).
\end{rmk}

\begin{ax}[Power set axiom]\label{i:3.10}
  Let \(X\) and \(Y\) be sets.
  Then there exists a set, denoted \(Y^X\), which consists of all the functions from \(X\) to \(Y\), thus
  \[
    f \in Y^X \iff (f \text{ is a function with domain } X \text{ and codomain } Y).
  \]
\end{ax}

\begin{note}
  The reason we use the notation \(Y^X\) to denote this set is that if \(Y\) has \(n\) elements and \(X\) has \(m\) elements, then one can show that \(Y^X\) has \(n^m\) elements.
  See \cref{i:3.6.14}(f).
\end{note}

\setcounter{thm}{8}
\begin{lem}\label{i:3.4.9}
  Let \(X\) be a set.
  Then the set
  \[
    \set{Y : Y \text{ is a subset of } X}
  \]
  is a set.
\end{lem}

\begin{proof}[\pf{i:3.4.9}]
  Suppose that \(X\) is a set.
  By \cref{i:3.10}, there exists a set \(\set{0, 1}^X\) which consists of functions with domain \(X\) and codomain \(\set{0, 1}\).
  By \cref{i:3.6}, we can replace each \(f \in \set{0, 1}^X\) with \(f^{-1}\pa{\set{1}}\), i.e., there exists a set
  \[
    S = \set{f^{-1}\pa{\set{1}} : f \in \set{0, 1}^X}.
  \]

  Next we claim that \(S\) is consist of subsets of \(X\).
  Let \(Y \in S\).
  We want to show that \(Y\) is a subset of \(X\).
  By the definition of \(S\), we know that \(Y = f^{-1}\pa{\set{1}}\) for some \(f \in \set{0, 1}^X\).
  By \cref{i:3.4.4}, we know that \(Y = \set{x \in X : f(x) \in \set{1}}\), which is clearly a subset of \(X\) (\cref{i:3.1.15}).
  Thus, our claim is true.

  Finally we show that \(S\) is consist of all subsets of \(X\), and therefore \cref{i:3.4.9} is true.
  Let \(Y\) be a subset of \(X\).
  We want to show that \(Y \in S\), and we will use \cref{i:ex:3.3.8}(d) to achieve this.
  We define two functions \(f : X \setminus Y \to \set{0, 1}\) and \(g : Y \to \set{0, 1}\) as follow:
  \begin{align*}
     & \forall z \in X \setminus Y, f(z) = 0; \\
     & \forall z \in Y, g(z) = 1.
  \end{align*}
  Since every element in \(X \setminus Y\) is mapped by \(f\) to the unique element \(0\) in \(\set{0, 1}\), we know that \(f\) obey the vertical line test, and therefore \(f\) is well-defined by \cref{i:3.3.1}.
  Similarly, \(g\) is well-defined.
  By \cref{i:3.1.28}(g), we know that \((X \setminus Y) \cap Y = \emptyset\) and \((X \setminus Y) \cup Y = X\).
  Thus we can apply \cref{i:ex:3.3.8}(d) to create a function \(h : X \to \set{0, 1}\) as follow:
  \[
    \forall z \in X, h(z) = \begin{dcases}
      f(z) = 0 & \text{if } z \in X \setminus Y \\
      g(z) = 1 & \text{if } z \in Y
    \end{dcases}.
  \]
  Clearly, we have \(h \in \set{0, 1}^X\) and \(h^{-1}\pa{\set{1}} = Y\).
  But by the definition of \(S\), we know that \(h^{-1}\pa{\set{1}} \in S\), therefore \(Y \in S\).
\end{proof}

\begin{rmk}\label{i:3.4.10}
  The set \(\set{Y : Y \text{ is a subset of } X}\) is know as the \emph{power set} of \(X\) and is denoted \(2^X\).
\end{rmk}

\begin{ax}[Union]\label{i:3.11}
  Let \(A\) be a set, all of whose elements are themselves sets.
  Then there exists a set \(\bigcup A\) whose elements are precisely those objects which are elements of the elements of \(A\), thus for all objects \(x\)
  \[
    x \in \bigcup A \iff (x \in S \text{ for some } S \in A)
  \]
\end{ax}

\begin{note}
  The axiom of union (\cref{i:3.11}), combined with the axiom of pair set (\cref{i:3.3}), implies the axiom of pairwise union (\cref{i:3.4}) (see \cref{i:ex:3.4.8}).
  Another important consequence of \cref{i:3.11} is that if one has some set \(I\), and for every element \(\alpha \in I\) we have some set \(A_\alpha\), then we can form the union set \(\bigcup_{\alpha \in I} A_\alpha\) by defining
  \[
    \bigcup_{\alpha \in I} A_\alpha \coloneqq \bigcup \set{A_\alpha : \alpha \in I},
  \]
  which is a set thanks to the axiom of replacement (\cref{i:3.6}) and the axiom of union (\cref{i:3.11}).
  More generally, we see that for any object \(y\),
  \[
    y \in \bigcup_{\alpha \in I} A_\alpha \iff (y \in A_\alpha \text{ for some } \alpha \in I).
  \]
  In situations like this, we often refer to \(I\) as an \emph{index set}, and the elements \(\alpha\) of this index set as \emph{labels};
  the sets \(A_\alpha\) are then called a \emph{family of sets}, and are \emph{indexed} by the labels \(\alpha \in I\).
  Note that if \(I\) was empty, then \(\bigcup_{\alpha \in I} A_\alpha\) would automatically also be empty.
\end{note}

\begin{note}
  We can similarly form intersections of families of sets, as long as the index set is non-empty.
  More specifically, given any non-empty set \(I\), and given an assignment of a set \(A_\alpha\) to each \(\alpha \in I\), we can define the intersection \(\bigcap_{\alpha \in I} A_\alpha\) by first choosing some element \(\beta\) of \(I\) (which we can do since \(I\) is non-empty), and setting
  \[
    \bigcap_{\alpha \in I} A_\alpha \coloneqq \set{x \in A_\beta : x \in A_\alpha \text{ for all } \alpha \in I},
  \]
  which is a set by the axiom of specification (\cref{i:3.5}).
  This definition may look like it depends on the choice of \(\beta\), but it does not.
  Observe that for any object \(y\),
  \[
    y \in \bigcap_{\alpha \in I} A_\alpha \iff (y \in A_\alpha \text{ for all } \alpha \in I).
  \]
\end{note}

\setcounter{thm}{11}
\begin{rmk}\label{i:3.4.12}
  The axioms of set theory that we have introduced (\crefrange{i:3.1}{i:3.11}, excluding the dangerous \cref{i:3.8}) are known as the \emph{Zermelo-Fraenkel axioms of set theory}, after Ernst Zermelo (1871--1953) and Abraham Fraenkel (1891--1965).
  There is one further axiom we will eventually need, the famous \emph{axiom of choice} (see \cref{i:sec:8.4}), giving rise to the \emph{Zermelo-Fraenkel-Choice (ZFC) axioms of set theory}, but we will not need this axiom for some time.
\end{rmk}

\exercisesection

\begin{ex}\label{i:ex:3.4.1}
  Let \(f : X \to Y\) be a bijective function, and let \(f^{-1} : Y \to X\) be its inverse.
  Let \(V\) be any subset of \(Y\).
  Prove that the forward image of \(V\) under \(f^{-1}\) is the same set as the inverse image of \(V\) under \(f\);
  thus the fact that both sets are denoted by \(f^{-1}(V)\) will not lead to any inconsistency.
\end{ex}

\begin{proof}[\pf{i:ex:3.4.1}]
  Let \(A\) be the set of the forward image of \(V\) under \(f^{-1}\).
  Let \(B\) be the set of the inverse image of \(V\) under \(f\).
  Then we have
  \begin{align*}
         & x \in A                                            \\
    \iff & \exists y \in V : f^{-1}(y) = x &  & \by{i:3.4.1}  \\
    \iff & \exists y \in V : f(x) = y      &  & \by{i:3.3.20} \\
    \iff & x \in B.                        &  & \by{i:3.4.4}
  \end{align*}
  Thus by \cref{i:3.1.4}, we have \(A = B\).
\end{proof}

\begin{ex}\label{i:ex:3.4.2}
  Let \(f : X \to Y\) be a function from one set \(X\) to another set \(Y\), let \(S\) be a subset of \(X\), and let \(U\) be a subset of \(Y\).
  What, in general, can one say about \(f^{-1}(f(S))\) and \(S\)?
  What about \(f(f^{-1}(U))\) and \(U\)?
\end{ex}

\begin{proof}[\pf{i:ex:3.4.2}]
  We first show that \(S \subseteq f^{-1}(f(S))\).
  Since
  \begin{align*}
             & x \in S                               \\
    \implies & f(x) \in f(S)       &  & \by{i:3.4.1} \\
    \implies & x \in f^{-1}(f(S)), &  & \by{i:3.4.4}
  \end{align*}
  we have \(S \subseteq f^{-1}(f(S))\) by \cref{i:3.1.15}.

  Now we show that \(f(f^{-1}(U)) \subseteq U\).
  Since
  \begin{align*}
             & y \in f(f^{-1}(U))                                   \\
    \implies & \exists x \in f^{-1}(U) : f(x) = y &  & \by{i:3.4.1} \\
    \implies & y \in U,                           &  & \by{i:3.4.4}
  \end{align*}
  we have \(f(f^{-1}(U)) \subseteq U\) by \cref{i:3.1.15}.
\end{proof}

\begin{ex}\label{i:ex:3.4.3}
  Let \(A, B\) be two subsets of a set \(X\), and let \(f : X \to Y\) be a function.
  Show that \(f(A \cap B) \subseteq f(A) \cap f(B)\), that \(f(A) \setminus f(B) \subseteq f(A \setminus B)\), \(f(A \cup B) = f(A) \cup f(B)\).
  For the first two statements, is it true that the \(\subseteq\) relation can be imporved to \(=\)?
\end{ex}

\begin{proof}[\pf{i:ex:3.4.3}]
  We first show that \(f(A \cap B) \subseteq f(A) \cap f(B)\).
  Since
  \begin{align*}
             & y \in f(A \cap B)                                                                        \\
    \implies & y \in \set{f(x) : x \in A \cap B}                                     &  & \by{i:3.4.1}  \\
    \implies & y \in \set{f(x) : (x \in A) \land (x \in B)}                          &  & \by{i:3.1.23} \\
    \implies & \pa{y \in \set{f(x) : x \in A}} \land \pa{y \in \set{f(x) : x \in B}}                    \\
    \implies & (y \in f(A)) \land (y \in f(B))                                       &  & \by{i:3.4.1}  \\
    \implies & y \in f(A) \cap f(B),                                                 &  & \by{i:3.1.23}
  \end{align*}
  we have \(f(A \cap B) \subseteq f(A) \cap f(B)\) by \cref{i:3.1.15}.
  We do not have \(f(A \cap B) = f(A) \cap f(B)\) in general.
  Consider the example \(f = x \mapsto x^2\), \(A = \set{1}\), and \(B = \set{-1}\).
  We have \(f(A \cap B) = f(\emptyset) = \emptyset \neq \set{1} = \set{1} \cap \set{1} = f(A) \cap f(B)\).

  Next we show that \(f(A) \setminus f(B) \subseteq f(A \setminus B)\).
  Since
  \begin{align*}
             & y \in f(A) \setminus f(B)                                                                   \\
    \implies & (y \in f(A)) \land (y \notin f(B))                                       &  & \by{i:3.1.27} \\
    \implies & \pa{y \in \set{f(x) : x \in A}} \land \pa{y \notin \set{f(x) : x \in B}} &  & \by{i:3.4.1}  \\
    \implies & y \in \set{f(x) : (x \in A) \land (x \notin B)}                                             \\
    \implies & y \in \set{f(x) : x \in A \setminus B}                                   &  & \by{i:3.1.27} \\
    \implies & y \in f(A \setminus B),                                                  &  & \by{i:3.4.1}
  \end{align*}
  we have \(f(A) \setminus f(B) \subseteq f(A \setminus B)\) by \cref{i:3.1.15}.
  We do not have \(f(A) \setminus f(B) = f(A \setminus B)\) in general.
  Consider the example \(f = x \mapsto x^2\), \(A = \set{1}\), and \(B = \set{-1}\).
  We have \(f(A) \setminus f(B) = \set{1} \setminus \set{1} = \emptyset \neq \set{1} = f(\set{1}) = f(A \setminus B)\).

  Finally we show that \(f(A \cup B) = f(A) \cup f(B)\).
  Since
  \begin{align*}
         & y \in f(A \cup B)                                                                      \\
    \iff & y \in \set{f(x) : x \in A \cup B}                                    &  & \by{i:3.4.1} \\
    \iff & y \in \set{f(x) : (x \in A) \lor (x \in B)}                          &  & \by{i:3.4}   \\
    \iff & \pa{y \in \set{f(x) : x \in A}} \lor \pa{y \in \set{f(x) : x \in B}}                   \\
    \iff & (y \in f(A)) \lor (y \in f(B))                                       &  & \by{i:3.4.1} \\
    \iff & y \in f(A) \cup f(B),                                                &  & \by{i:3.4}
  \end{align*}
  we have \(f(A \cup B) = f(A) \cup f(B)\) by \cref{i:3.1.4}.
\end{proof}

\begin{ex}\label{i:ex:3.4.4}
  Let \(f : X \to Y\) be a function from one set \(X\) to another set \(Y\), and let \(U, V\) be subsets of \(Y\).
  Show that \(f^{-1}(U \cup V) = f^{-1}(U) \cup f^{-1}(V)\), that \(f^{-1}(U \cap V) = f^{-1}(U) \cap f^{-1}(V)\), and that \(f^{-1}(U \setminus V) = f^{-1}(U) \setminus f^{-1}(V)\).
\end{ex}

\begin{proof}[\pf{i:ex:3.4.4}]
  We first show that \(f^{-1}(U \cup V) = f^{-1}(U) \cup f^{-1}(V)\).
  Since
  \begin{align*}
         & x \in f^{-1}(U \cup V)                                           \\
    \iff & f(x) \in U \cup V                              &  & \by{i:3.4.4} \\
    \iff & (f(x) \in U) \lor (f(x) \in V)                 &  & \by{i:3.4}   \\
    \iff & \pa{x \in f^{-1}(U)} \lor \pa{x \in f^{-1}(V)} &  & \by{i:3.4.4} \\
    \iff & x \in f^{-1}(U) \cup f^{-1}(V),                &  & \by{i:3.4}
  \end{align*}
  we have \(f^{-1}(U \cup V) = f^{-1}(U) \cup f^{-1}(V)\) by \cref{i:3.1.4}.

  Next we show that \(f^{-1}(U \cap V) = f^{-1}(U) \cap f^{-1}(V)\).
  Since
  \begin{align*}
         & x \in f^{-1}(U \cap V)                                             \\
    \iff & f(x) \in U \cap V                               &  & \by{i:3.4.4}  \\
    \iff & (f(x) \in U) \land (f(x) \in V)                 &  & \by{i:3.1.23} \\
    \iff & \pa{x \in f^{-1}(U)} \land \pa{x \in f^{-1}(V)} &  & \by{i:3.4.4}  \\
    \iff & x \in f^{-1}(U) \cap f^{-1}(V),                 &  & \by{i:3.1.23}
  \end{align*}
  we have \(f^{-1}(U \cap V) = f^{-1}(U) \cap f^{-1}(V)\) by \cref{i:3.1.4}.

  Finally we show that \(f^{-1}(U \setminus V) = f^{-1}(U) \setminus f^{-1}(V)\).
  Since
  \begin{align*}
         & x \in f^{-1}(U \setminus V)                                           \\
    \iff & f(x) \in U \setminus V                             &  & \by{i:3.4.4}  \\
    \iff & (f(x) \in U) \land (f(x) \notin V)                 &  & \by{i:3.1.23} \\
    \iff & \pa{x \in f^{-1}(U)} \land \pa{x \notin f^{-1}(V)} &  & \by{i:3.4.4}  \\
    \iff & x \in f^{-1}(U) \setminus f^{-1}(V),               &  & \by{i:3.1.23}
  \end{align*}
  we have \(f^{-1}(U \setminus V) = f^{-1}(U) \setminus f^{-1}(V)\) by \cref{i:3.1.4}.
\end{proof}

\begin{ex}\label{i:ex:3.4.5}
  Let \(f : X \to Y\) be a function from one set \(X\) to another set \(Y\).
  Show that \(f(f^{-1}(S)) = S\) for every \(S \subseteq Y\) iff \(f\) is surjective.
  Show that \(f^{-1}(f(S)) = S\) for every \(S \subseteq X\) iff \(f\) is injective.
\end{ex}

\begin{proof}[\pf{i:ex:3.4.5}]
  We first show that \(f\pa{f^{-1}(S)} = S\) for every \(S \subseteq Y\) iff \(f\) is surjective.
  By \cref{i:ex:3.4.2}, we have \(f\pa{f^{-1}(S)} \subseteq S\) for every \(S \subseteq Y\).
  Thus, it suffices to show that \(S \subseteq f\pa{f^{-1}(S)}\) for every \(S \subseteq Y\) iff \(f\) is surjective.
  This is true since
  \begin{align*}
         & f \text{ is surjective}                                                                       \\
    \iff & \forall y \in Y, \exists x \in X : f(x) = y                                &  & \by{i:3.3.17} \\
    \iff & \forall S \subseteq Y, \forall y \in S, \exists x \in X : f(x) = y         &  & \by{i:3.1.15} \\
    \iff & \forall S \subseteq Y, \forall y \in S, \exists x \in f^{-1}(S) : f(x) = y &  & \by{i:3.4.4}  \\
    \iff & \forall S \subseteq Y, \forall y \in S, y \in f\pa{f^{-1}(S)}              &  & \by{i:3.4.1}  \\
    \iff & \forall S \subseteq Y, S \subseteq f\pa{f^{-1}(S)}.                        &  & \by{i:3.1.15}
  \end{align*}

  Now we show that \(f^{-1}\pa{f(S)} = S\) for all \(S \subseteq X\) iff \(f\) is injective.
  By \cref{i:ex:3.4.2}, we have \(S \subseteq f^{-1}\pa{f(S)}\) for every \(S \subseteq X\).
  Thus, it suffices to show that \(f^{-1}\pa{f(S)} \subseteq S\) for every \(S \subseteq X\) iff \(f\) is injective.

  We start by showing that if \(f^{-1}\pa{f(S)} \subseteq S\) for every \(S \subseteq X\), then \(f\) is injective.
  So suppose that \(f^{-1}\pa{f(S)} \subseteq S\) for every \(S \subseteq X\).
  Let \(x_1, x_2 \in X\) where \(x_1 \neq x_2\).
  Then by \cref{i:3.3,i:3.1.15}, we know that \(\set{x_1} \subseteq X\) and \(\set{x_2} \subseteq X\).
  Thus, we can apply the hypothesis and \cref{i:ex:3.4.2} to derive
  \[
    \set{x_1} \subseteq f^{-1}\pa{f\pa{\set{x_1}}} \subseteq \set{x_1} \quad \text{and} \quad \set{x_2} \subseteq f^{-1}\pa{f\pa{\set{x_2}}} \subseteq \set{x_2}.
  \]
  By \cref{i:3.1.18}, this means \(\set{x_1} = f^{-1}\pa{f\pa{\set{x_1}}}\) and \(\set{x_2} = f^{-1}\pa{f\pa{\set{x_2}}}\).
  By \cref{i:3.4.1}, we see that \(f\pa{\set{x_1}}\) and \(f\pa{\set{x_2}}\) are singleton sets.
  Thus by \cref{i:3.4.4}, we must have \(f(x_1) \neq f(x_2)\), otherwise we would have \(f(\set{x_1}) = f(\set{x_2})\), which implies \(\set{x_1} = \set{x_2}\), a contradiction.
  Therefore, \(f\) is injective by \cref{i:3.3.14}.

  Now we show that \(f\) is injective implies \(f^{-1}\pa{f(S)} \subseteq S\) for every \(S \subseteq X\).
  Suppose that \(f\) is injective.
  Let \(S \subseteq X\).
  Suppose for sake of contradiction that \(f^{-1}\pa{f(S)} \nsubseteq S\).
  Then by \cref{i:3.1.15}, there exists an \(x \in f^{-1}\pa{f(S)}\) such that \(x \notin S\).
  Fix one such \(x\).
  By \cref{i:3.4.4}, we know that there exists a \(y \in f(S)\) such that \(f(x) = y\).
  Fix one such \(y\).
  Since \(y \in f(S)\), by \cref{i:3.4.1}, there exists an \(x' \in S\) such that \(f(x') = y\).
  But \(f\) is injective implies \(x = x'\), which means \(x \in S\), a contradiction.
  Thus, we must have \(f^{-1}\pa{f(S)} \subseteq S\).
\end{proof}

\begin{ex}\label{i:ex:3.4.6}
  Prove \cref{i:3.4.9}.
\end{ex}

\begin{proof}[\pf{i:ex:3.4.6}]
  See \cref{i:3.4.9}.
\end{proof}

\begin{ex}\label{i:ex:3.4.7}
  Let \(X, Y\) be sets.
  Define a \emph{partial function} from \(X\) to \(Y\) to be any function \(f : X' \to Y'\) whose domain \(X'\) is a subset of \(X\), and whose codomain \(Y'\) is a subset of \(Y\).
  Show that the collection of all partial functions from \(X\) to \(Y\) is itself a set.
\end{ex}

\begin{proof}[\pf{i:ex:3.4.7}]
  Suppose that \(X, Y\) are sets.
  Then by \cref{i:3.4.9}, both the sets \(A = \set{X' : X' \subseteq X}\) and \(B = \set{Y' : Y' \subseteq Y}\) exist.
  Now we have
  \begin{align*}
    C_1 & = \set{Y'^{X'} : \pa{X' \in A} \land \pa{Y' \in B}};   &  & \by{i:3.5,i:3.6,i:3.10} \\
    C_2 & = \bigcup C_1 = \set{f \in Y'^{X'} : Y'^{X'} \in C_1}. &  & \by{i:3.11}             \\
  \end{align*}
  If \(f : X' \to Y'\) is a partial function whose domain \(X' \subseteq X\) and whose codomain \(Y' \subseteq Y\), then we have \(Y'^{X'} \in C_1\), and thus \(f \in C_2\).
\end{proof}

\begin{ex}\label{i:ex:3.4.8}
  Show that \cref{i:3.4} can be deduced from \cref{i:3.1,i:3.3,i:3.11}.
\end{ex}

\begin{proof}[\pf{i:ex:3.4.8}]
  Let \(A, B\) be sets (the existence of \(A, B\) are guaranteed by \cref{i:3.1}).
  By \cref{i:3.3}, there exists a set \(\set{A, B}\) whose only elements are \(A\) and \(B\).
  By \cref{i:3.11}, we can create a set \(\bigcup \set{A, B}\).
  Now we claim that \(A \cup B = \bigcup \set{A, B}\).
  Since
  \begin{align*}
         & x \in \bigcup \set{A, B}                            \\
    \iff & \exists C \in \set{A, B} : x \in C &  & \by{i:3.11} \\
    \iff & (x \in A) \lor (x \in B)           &  & \by{i:3.3}  \\
    \iff & x \in A \cup B,                    &  & \by{i:3.4}
  \end{align*}
  we see that \(A \cup B = \bigcup \set{A, B}\) by \cref{i:3.1.4}.
\end{proof}

\begin{ex}\label{i:ex:3.4.9}
  Show that if \(\beta\) and \(\beta'\) are two elements of a set \(I\), and to each \(\alpha \in I\) we assign a set \(A_\alpha\), then
  \[
    \set{x \in A_\beta : \forall \alpha \in I, x \in A_\alpha} = \set{x \in A_{\beta'} : \forall \alpha \in I, x \in A_\alpha},
  \]
  and so the definition of \(\bigcap_{\alpha \in I} A_\alpha\) does not depend on \(\beta\).
\end{ex}

\begin{proof}[\pf{i:ex:3.4.9}]
  Let \(B, B'\) be sets
  \begin{align*}
    B  & = \set{x \in A_\beta : \forall \alpha \in I, x \in A_\alpha}     \\
    B' & = \set{x \in A_{\beta'} : \forall \alpha \in I, x \in A_\alpha}.
  \end{align*}
  Since
  \begin{align*}
         & x \in B                                                                          \\
    \iff & (x \in A_\beta) \land (\forall \alpha \in I, x \in A_\alpha)                     \\
    \iff & \forall \alpha \in I, x \in A_\alpha                            & (\beta \in I)  \\
    \iff & (x \in A_{\beta'}) \land (\forall \alpha \in I, x \in A_\alpha) & (\beta' \in I) \\
    \iff & x \in B',
  \end{align*}
  we have \(B = B'\) by \cref{i:3.1.4}.
\end{proof}

\begin{ex}\label{i:ex:3.4.10}
  Suppose that \(I\) and \(J\) are two sets, and for all \(\alpha \in I \cup J\) let \(A_\alpha\) be a set.
  Show that \(\pa{\bigcup_{\alpha \in I} A_\alpha} \cup \pa{\bigcup_{\alpha \in J} A_\alpha} = \bigcup_{\alpha \in I \cup J} A_\alpha\).
  If \(I\) and \(J\) are non-empty, show that \(\pa{\bigcap_{\alpha \in I} A_\alpha} \cap \pa{\bigcap_{\alpha \in J} A_\alpha} = \bigcap_{\alpha \in I \cup J} A_\alpha\).
\end{ex}

\begin{proof}[\pf{i:ex:3.4.10}]
  Since
  \begin{align*}
         & x \in \pa{\bigcup_{\alpha \in I} A_\alpha} \cup \pa{\bigcup_{\alpha \in J} A_\alpha}                        \\
    \iff & \pa{x \in \bigcup_{\alpha \in I} A_\alpha} \lor \pa{x \in \bigcup_{\alpha \in J} A_\alpha} &  & \by{i:3.4}  \\
    \iff & (\exists \alpha \in I : x \in A_\alpha) \lor (\exists \alpha \in J : x \in A_\alpha)       &  & \by{i:3.11} \\
    \iff & \exists \alpha : ((\alpha \in I) \lor (\alpha \in J)) \land (x \in A_\alpha)                                \\
    \iff & \exists \alpha \in I \cup J : x \in A_\alpha                                               &  & \by{i:3.4}  \\
    \iff & x \in \bigcup_{\alpha \in I \cup J} A_\alpha,                                              &  & \by{i:3.11}
  \end{align*}
  and
  \begin{align*}
         & x \in \pa{\bigcap_{\alpha \in I} A_\alpha} \cap \pa{\bigcap_{\alpha \in J} A_\alpha}                             \\
    \iff & \pa{x \in \bigcap_{\alpha \in I} A_\alpha} \land \pa{x \in \bigcap_{\alpha \in J} A_\alpha} &  & \by{i:3.1.23}   \\
    \iff & (\forall \alpha \in I, x \in A_\alpha) \land (\forall \alpha \in J, x \in A_\alpha)         &  & \by{i:ex:3.4.9} \\
    \iff & \forall \alpha, ((\alpha \in I) \lor (\alpha \in J)) \land (x \in A_\alpha)                                      \\
    \iff & \forall \alpha \in I \cup J, x \in A_\alpha                                                 &  & \by{i:3.4}      \\
    \iff & x \in \bigcap_{\alpha \in I \cup J} A_\alpha,                                               &  & \by{i:ex:3.4.9}
  \end{align*}
  we have \(\pa{\bigcup_{\alpha \in I} A_\alpha} \cup \pa{\bigcup_{\alpha \in J} A_\alpha} = \bigcup_{\alpha \in I \cup J} A_\alpha\) and \(\pa{\bigcap_{\alpha \in I} A_\alpha} \cap \pa{\bigcap_{\alpha \in J} A_\alpha} = \bigcap_{\alpha \in I \cup J} A_\alpha\) by \cref{i:3.1.4}.
\end{proof}

\begin{ex}\label{i:ex:3.4.11}
  Let \(X\) be a set, let \(I\) be a non-empty set, and for all \(\alpha \in I\) let \(A_\alpha\) be a subset of \(X\).
  Show that
  \[
    X \setminus \bigcup_{\alpha \in I} A_\alpha = \bigcap_{\alpha \in I} (X \setminus A_\alpha)
  \]
  and
  \[
    X \setminus \bigcap_{\alpha \in I} A_\alpha = \bigcup_{\alpha \in I} (X \setminus A_\alpha).
  \]
  This should be compared with de Morgan's laws in \cref{i:3.1.28}
  (although one cannot derive the above identities directly from de Morgan's laws, as \(I\) could be infinite).
\end{ex}

\begin{proof}[\pf{i:ex:3.4.11}]
  Since
  \begin{align*}
         & x \in X \setminus \bigcup_{\alpha \in I} A_\alpha                                  \\
    \iff & (x \in X) \land \pa{x \notin \bigcup_{\alpha \in I} A_\alpha} &  & \by{i:3.1.27}   \\
    \iff & (x \in X) \land \lnot(\exists \alpha \in I : x \in A_\alpha)  &  & \by{i:3.11}     \\
    \iff & (x \in X) \land (\forall \alpha \in I, x \notin A_\alpha)                          \\
    \iff & \forall \alpha \in I, (x \in X) \land (x \notin A_\alpha)                          \\
    \iff & \forall \alpha \in I, x \in X \setminus A_\alpha              &  & \by{i:3.1.27}   \\
    \iff & x \in \bigcap_{\alpha \in I} (X \setminus A_\alpha),          &  & \by{i:ex:3.4.9}
  \end{align*}
  and
  \begin{align*}
         & x \in X \setminus \bigcap_{\alpha \in I} A_\alpha                                  \\
    \iff & (x \in X) \land \pa{x \notin \bigcap_{\alpha \in I} A_\alpha} &  & \by{i:3.1.27}   \\
    \iff & (x \in X) \land \lnot(\forall \alpha \in I, x \in A_\alpha)   &  & \by{i:ex:3.4.9} \\
    \iff & (x \in X) \land (\exists \alpha \in I : x \notin A_\alpha)                         \\
    \iff & \exists \alpha \in I : (x \in X) \land (x \notin A_\alpha)                         \\
    \iff & \exists \alpha \in I : x \in X \setminus A_\alpha             &  & \by{i:3.1.27}   \\
    \iff & x \in \bigcup_{\alpha \in I} (X \setminus A_\alpha),          &  & \by{i:3.11}
  \end{align*}
  we have \(X \setminus \bigcup_{\alpha \in I} A_\alpha = \bigcap_{\alpha \in I} (X \setminus A_\alpha)\) and \(X \setminus \bigcap_{\alpha \in I} A_\alpha = \bigcup_{\alpha \in I} (X \setminus A_\alpha)\) by \cref{i:3.1.4}.
\end{proof}
