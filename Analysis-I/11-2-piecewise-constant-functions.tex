\section{Piecewise constant functions}\label{sec 11.2}

\begin{definition}[Constant functions]\label{11.2.1}
    Let \(X\) be a subset of \(\mathbf{R}\), and let \(f : X \to \mathbf{R}\) be a function.
    We say that \(f\) is \emph{constant} iff there exists a real number \(c\) such that \(f(x) = c\) for all \(x \in X\).
    If \(E\) is a subset of \(X\), we say that \(f\) is \emph{constant on} \(E\) if the restriction \(f|_E\) of \(f\) to \(E\) is constant, in other words there exists a real number \(c\) such that \(f(x) = c\) for all \(x \in E\).
    We refer to \(c\) as the \emph{constant value} of \(f\) on \(E\).
\end{definition}

\begin{remark}\label{11.2.2}
    If \(E\) is a non-empty set, then a function \(f\) which is constant on \(E\) can have only one constant value;
    However, if \(E\) is empty, every real number \(c\) is a constant value for \(f\) on \(E\).
\end{remark}

\begin{definition}[Piecewise constant functions I]\label{11.2.3}
    Let \(I\) be a bounded interval, let \(f : I \to \mathbf{R}\) be a function, and let \(\mathbf{P}\) be a partition of \(I\).
    We say that \(f\) is \emph{piecewise constant with respect to \(\mathbf{P}\)} if for every \(J \in \mathbf{P}\), \(f\) is constant on \(J\).
\end{definition}

\setcounter{theorem}{4}
\begin{definition}[Piecewise constant functions II]\label{11.2.5}
    Let \(I\) be a bounded interval, and let \(f : I \to \mathbf{R}\) be a function.
    We say that \(f\) is \emph{piecewise constant on \(I\)} if there exists a partition \(\mathbf{P}\) of \(I\) such that \(f\) is piecewise constant with respect to \(\mathbf{P}\).
\end{definition}

\setcounter{theorem}{6}
\begin{lemma}\label{11.2.7}
    Let \(I\) be a bounded interval, let \(\mathbf{P}\) be a partition of \(I\), and let \(f : I \to \mathbf{R}\) be a function which is piecewise constant with respect to \(\mathbf{P}\).
    Let \(\mathbf{P}'\) be a partition of \(I\) which is finer than \(\mathbf{P}\).
    Then \(f\) is also piecewise constant with respect to \(\mathbf{P}'\).
\end{lemma}

\begin{proof}
    Let \(K \in \mathbf{P}'\).
    Since \(\mathbf{P}'\) is finer than \(\mathbf{P}\), by Definition \ref{11.1.14} \(\exists\ J \in \mathbf{P}\) such that \(K \subseteq J\).
    Since \(f\) is piecewise constant with respect to \(\mathbf{P}\), by Definition \ref{11.2.3} we know that \(\forall\ x \in J\), \(f(x)\) is constant.
    Thus \(\forall\ x \in K\), \(x \in J\) and \(f(x)\) is constant.
    By Definition \ref{11.2.3} \(f\) is piecewise constant with respect to \(\mathbf{P}'\).
\end{proof}

\begin{lemma}\label{11.2.8}
    Let \(I\) be a bounded interval, and let \(f : I \to \mathbf{R}\) and \(g : I \to \mathbf{R}\) be piecewise constant functions on \(I\).
    Then the functions \(f + g\), \(f - g\), \(\max(f, g)\), \(\min(f, g)\) and \(fg\) are also piecewise constant functions on \(I\).
    Here of course \(\max(f, g) : I \to \mathbf{R}\) is the function \(\max(f, g)(x) \coloneqq \max(f(x), g(x))\).
    If \(g\) does not vanish anywhere on \(I\) (i.e., \(g(x) \neq 0\) for all \(x \in I\)) then \(f / g\) is also a piecewise constant function on \(I\).
\end{lemma}

\begin{proof}
    Since \(f\) is piecewise constant function on \(I\), by Definition \ref{11.2.5} \(\exists\ \mathbf{P}\) such that \(\mathbf{P}\) is a partition of \(I\) and \(f\) is piecewise constant with respect to \(\mathbf{P}\).
    Similarly since \(g\) is piecewise constant function on \(I\), by Definition \ref{11.2.5} \(\exists\ \mathbf{P}'\) such that \(\mathbf{P}'\) is a partition of \(I\) and \(g\) is piecewise constant with respect to \(\mathbf{P}'\).
    By Lemma \ref{11.1.18} we know that \(\mathbf{P} \# \mathbf{P}'\) is also a partition of \(I\) and \(\mathbf{P} \# \mathbf{P}'\) is both finer than \(\mathbf{P}\) and finer than \(\mathbf{P}'\).
    By Lemma \ref{11.2.7} we know that both \(f\) and \(g\) are piecewise constant with respect to \(\mathbf{P} \# \mathbf{P}'\).

    Now we show that \(f, g\) remain piecewise constant functions on \(I\) after algebraic operation.
    Since \(\forall\ J \in \mathbf{P} \# \mathbf{P}'\), we have \(\forall\ x \in J\), \(f(x)\) is constant and \(g(x)\) is constant.
    Thus we know that \(f(x) + g(x)\), \(f(x) - g(x)\), \(\max(f(x), g(x))\), \(\min(f(x), g(x))\) and \(f(x) g(x)\) are constant.
    If \(g(x) \neq 0\), then we also have \(f(x) / g(x)\) is constant.
    Thus by Definition \ref{11.2.3} \(f + g\), \(f - g\), \(\max(f, g)\), \(\min(f, g)\), \(fg\) is piecewise constant with respect to \(\mathbf{P} \# \mathbf{P}'\), and when \(g(x) \neq 0\) we have \(f / g\) is piecewise constant with respect to \(\mathbf{P} \# \mathbf{P}'\).
    By Definition \ref{11.2.5} \(f + g\), \(f - g\), \(\max(f, g)\), \(\min(f, g)\), \(fg\) is piecewise constant on \(I\), and when \(g(x) \neq 0\) we have \(f / g\) is piecewise constant on \(I\).
\end{proof}

\begin{definition}[Piecewise constant integral I]\label{11.2.9}
    Let \(I\) be a bounded interval, let \(\mathbf{P}\) be a partition of \(I\).
    Let \(f : I \to \mathbf{R}\) be a function which is piecewise constant with respect to \(\mathbf{P}\).
    Then we define the \emph{piecewise constant integral} \(p.c. \int_{[\mathbf{P}]} f\) of \(f\) with respect to the partition \(\mathbf{P}\) by the formula
    \[
        p.c. \int_{[\mathbf{P}]} f \coloneqq \sum_{J \in \mathbf{P}} c_J \abs*{J},
    \]
    where for each \(J\) in \(\mathbf{P}\), we let \(c_J\) be the constant value of \(f\) on \(J\).
\end{definition}