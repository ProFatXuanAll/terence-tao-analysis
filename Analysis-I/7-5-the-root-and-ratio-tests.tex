\section{The root and ratio tests}\label{i:sec:7.5}

\begin{thm}[Root test]\label{i:7.5.1}
  Let \(\sum_{n = m}^\infty a_n\) be a series of real numbers, and let \(\alpha \coloneqq \limsup_{n \to \infty} \abs{a_n}^{1 / n}\).
  \begin{itemize}
    \item If \(\alpha < 1\), then the series \(\sum_{n = m}^\infty a_n\) is absolutely convergent
          (and hence conditionally convergent).
    \item If \(\alpha > 1\), then the series \(\sum_{n = m}^\infty a_n\) is not conditionally convergent
          (and hence cannot be absolutely convergent either).
    \item If \(\alpha = 1\), we cannot assert any conclusion.
  \end{itemize}
\end{thm}

\begin{proof}
  By \cref{i:7.2.14}(c), we may assume without loss of generality that \(m \geq 1\)
  (in particular \(\abs{a_n}^{1 / n}\) is well-defined for any \(n \geq m\)).

  First suppose that \(\alpha < 1\).
  Note that we must have \(\alpha \geq 0\), since by \cref{i:5.6.6}(c) \(\abs{a_n}^{1 / n} \geq 0\) for every \(n\).
  Then we can find an \(\varepsilon > 0\) such that \(0 < \alpha + \varepsilon < 1\) (for instance, we can set \(\varepsilon \coloneqq (1 - \alpha) / 2\)).
  By \cref{i:6.4.12}(a), there exists an \(N \geq m\) such that \(\abs{a_n}^{1 / n} \leq \alpha + \varepsilon\) for all \(n \geq N\).
  In other words, we have \(\abs{a_n} \leq (\alpha + \varepsilon)^n\) for all \(n \geq N\).
  But from the geometric series (\cref{i:7.3.3}) we have that \(\sum_{n = N}^\infty (\alpha + \varepsilon)^n\) is absolutely convergent, since \(0 < \alpha + \varepsilon < 1\)
  (note that the fact that we start from \(N\) is irrelevant by \cref{i:7.2.14}(c)).
  Thus by the comparison test (\cref{i:7.3.2}), we see that \(\sum_{n = N}^\infty a_n\) is absolutely convergent, and thus \(\sum_{n = m}^\infty a_n\) is absolutely convergent, by \cref{i:7.2.14}(c) again.

  Now suppose that \(\alpha > 1\).
  Then by \cref{i:6.4.12}(b), we see that for every \(N \geq m\) there exists an \(n \geq N\) such that \(\abs{a_n}^{1 / n} \geq 1\), and hence that \(\abs{a_n} \geq 1\).
  In particular, \((a_n)_{n = N}^\infty\) is not \(1\)-close to \(0\) for any \(N\), and hence \((a_n)_{n = m}^\infty\) is not eventually \(1\)-close to \(0\).
  In particular, \((a_n)_{n = m}^\infty\) does not converge to zero.
  Thus by the zero test (\cref{i:7.2.6}), \(\sum_{n = m}^\infty a_n\) is not conditionally convergent.

  For \(\alpha = 1\), we show two sequences \((a_n)_{n = m}^\infty\) and \((b_n)_{n = m}^\infty\) where
  \[
    \limsup_{n \to \infty} \abs{a_n}^{1 / n} = \limsup_{n \to \infty} \abs{b_n}^{1 / n} = 1
  \]
  and
  \[
    \limsup_{n \to \infty} \dfrac{\abs{a_{n + 1}}}{\abs{a_n}} = \limsup_{n \to \infty} \dfrac{\abs{b_{n + 1}}}{\abs{b_n}} = 1
  \]
  but \(\sum_{n = m}^\infty a_n\) converges and \(\sum_{n = m}^\infty b_n\) diverges.
  Let \(a_n = 1 / n\) and \(b_n = (-1)^n / n\).
  Then by \cref{i:7.3.7} \(\sum_{n = m}^\infty a_n\) diverges and by alternating series test (\cref{i:7.2.12}) \(\sum_{n = m}^\infty b_n\) converges.

  Since
  \begin{align*}
    \limsup_{n \to \infty} \abs{\dfrac{1}{n}}^{1 / n} & \leq \limsup_{n \to \infty} \dfrac{\abs{n}}{\abs{n + 1}}  &  & \by{i:7.5.2}  \\
                                                      & = \limsup_{n \to \infty} \bigg(1 - \dfrac{1}{n + 1}\bigg)                    \\
                                                      & = 1                                                       &  & \by{i:6.1.11}
  \end{align*}
  and
  \begin{align*}
    \liminf_{n \to \infty} \abs{\dfrac{1}{n}}^{1 / n} & \geq \liminf_{n \to \infty} \dfrac{\abs{n}}{\abs{n + 1}}  &  & \by{i:7.5.2}  \\
                                                      & = \liminf_{n \to \infty} \bigg(1 - \dfrac{1}{n + 1}\bigg)                    \\
                                                      & = 1,                                                      &  & \by{i:6.1.11}
  \end{align*}
  by \cref{i:6.4.12}(f) we have
  \[
    \limsup_{n \to \infty} \dfrac{\abs{n}}{\abs{n + 1}} = \liminf_{n \to \infty} \dfrac{\abs{n}}{\abs{n + 1}} = 1 = \lim_{n \to \infty} \dfrac{\abs{n}}{\abs{n + 1}}
  \]
  and
  \[
    \limsup_{n \to \infty} \abs{\dfrac{1}{n}}^{1 / n} = \liminf_{n \to \infty} \abs{\dfrac{1}{n}}^{1 / n} = 1 = \lim_{n \to \infty} \abs{\dfrac{1}{n}}^{1 / n}.
  \]
  Thus we have \(\limsup_{n \to \infty} \abs{a_n}^{1 / n} = \limsup_{n \to \infty} \dfrac{\abs{a_{n + 1}}}{\abs{a_n}} = 1\) but \(\sum_{n = m}^\infty a_n\) diverges.

  Since
  \begin{align*}
    \limsup_{n \to \infty} \abs{\dfrac{(-1)^n}{n}}^{1 / n} & \leq \limsup_{n \to \infty} \dfrac{\abs{(-1)^{n + 1} n}}{\abs{(-1)^n (n + 1)}} &  & \by{i:7.5.2}  \\
                                                           & = \limsup_{n \to \infty} \dfrac{n}{n + 1}                                                         \\
                                                           & = 1                                                                            &  & \by{i:6.1.11}
  \end{align*}
  and
  \begin{align*}
    \liminf_{n \to \infty} \abs{\dfrac{(-1)^n}{n}}^{1 / n} & \geq \liminf_{n \to \infty} \dfrac{\abs{(-1)^{n + 1} n}}{\abs{(-1)^n (n + 1)}} &  & \by{i:7.5.2}  \\
                                                           & = \liminf_{n \to \infty} \dfrac{n}{n + 1}                                                         \\
                                                           & = 1,                                                                           &  & \by{i:6.1.11}
  \end{align*}
  By \cref{i:6.4.12}(f) we have
  \[
    \limsup_{n \to \infty} \dfrac{\abs{(-1)^{n + 1} n}}{\abs{(-1)^n (n + 1)}} = \liminf_{n \to \infty} \dfrac{\abs{(-1)^{n + 1} n}}{\abs{(-1)^n (n + 1)}} = 1 = \lim_{n \to \infty} \dfrac{\abs{(-1)^{n + 1} n}}{\abs{(-1)^n (n + 1)}}
  \]
  and
  \[
    \limsup_{n \to \infty} \abs{\dfrac{(-1)^n}{n}}^{1 / n} = \liminf_{n \to \infty} \abs{\dfrac{(-1)^n}{n}}^{1 / n} = \lim_{n \to \infty} \abs{\dfrac{(-1)^n}{n}}^{1 / n} = 1.
  \]
  Thus we have \(\limsup_{n \to \infty} \abs{b_n}^{1 / n} = \limsup_{n \to \infty} \dfrac{\abs{b_{n + 1}}}{\abs{b_n}} = 1\) but \(\sum_{n = m}^\infty b_n\) converges.
  We conclude that when \(\limsup_{n \to \infty} \abs{a_n}^{1 / n} = \limsup_{n \to \infty} \dfrac{\abs{a_{n + 1}}}{\abs{a_n}} = 1\) we cannot assert any conclusion.
\end{proof}

\begin{note}
  The root test is phrased using the limit superior, but of course if \(\lim_{n \to \infty} \abs{a_n}^{1 / n}\) converges then the limit is the same as the limit superior.
  Thus one can phrase the root test using the limit instead of the limit superior, but \emph{only when the limit exists}.
\end{note}

\begin{lem}\label{i:7.5.2}
  Let \((c_n)_{n = m}^\infty\) be a sequence of positive numbers.
  Then we have
  \[
    \liminf_{n \to \infty} \dfrac{c_{n + 1}}{c_n} \leq \liminf_{n \to \infty} c_n^{1 / n} \leq \limsup_{n \to \infty} c_n^{1 / n} \leq \limsup_{n \to \infty} \dfrac{c_{n + 1}}{c_n}.
  \]
\end{lem}

\begin{proof}
  There are three inequalities to prove here.
  The middle inequality follows from \cref{i:6.4.12}(c).

  Next we show that \(\limsup_{n \to \infty} c_n^{1 / n} \leq \limsup_{n \to \infty} \dfrac{c_{n + 1}}{c_n}\).
  Write \(L \coloneqq \limsup_{n \to \infty} \dfrac{c_{n + 1}}{c_n}\).
  If \(L = +\infty\) then there is nothing to prove (since \(x \leq +\infty\) for every extended real number \(x\)), so we may assume that \(L\) is a finite real number.
  (Note that \(L\) cannot equal \(-\infty\)).
  Since \(\dfrac{c_{n + 1}}{c_n}\) is always positive, we know that \(L \geq 0\).

  Let \(\varepsilon > 0\).
  By \cref{i:6.4.12}(a), we know that there exists an \(N \geq m\) such that \(\dfrac{c_{n + 1}}{c_n} \leq L + \varepsilon\) for all \(n \geq N\)
  (without loss of generality we may assume that \(N \geq 1\)).
  This implies that \(c_{n + 1} \leq c_n (L + \varepsilon)\) for all \(n \geq N\).
  By induction this implies that
  \[
    c_n \leq c_N (L + \varepsilon)^{n - N} \text{ for all } n \geq N.
  \]
  If we write \(A \coloneqq c_N (L + \varepsilon)^{-N}\), then we have
  \[
    c_n \leq A(L + \varepsilon)^n
  \]
  and thus
  \[
    c_n^{1 / n} \leq A^{1 / n} (L + \varepsilon)
  \]
  for all \(n \geq N\).
  But we have
  \[
    \lim_{n \to \infty} A^{1 / n} (L + \varepsilon) = L + \varepsilon
  \]
  by the limit laws (\cref{i:6.1.19}) and \cref{i:6.5.3}.
  Thus by the comparison principle (\cref{i:6.4.13}) we have
  \[
    \limsup_{n \to \infty} c_n^{1 / n} \leq L + \varepsilon.
  \]
  But this is true for all \(\varepsilon > 0\), so this must imply that
  \[
    \limsup_{n \to \infty} c_n^{1 / n} \leq L.
  \]
  (If \(\limsup_{n \to \infty} c_n^{1 / n} > L\), then when \(\varepsilon = (\limsup_{n \to \infty} c_n^{1 / n} - L) / 2\) we have
  \[
    \limsup_{n \to \infty} c_n^{1 / n} \leq \dfrac{\limsup_{n \to \infty} c_n^{1 / n} + L}{2},
  \]
  a contradiction.), as desired.

  Finally we show that \(\liminf_{n \to \infty} \dfrac{c_{n + 1}}{c_n} \leq \liminf_{n \to \infty} c_n^{1 / n}\).
  Write \(L \coloneqq \liminf_{n \to \infty} \dfrac{c_{n + 1}}{c_n}\).
  Since \(\dfrac{c_{n + 1}}{c_n}\) is always positive, we know that \(L \geq 0\).

  Let \(\varepsilon > 0\).
  By \cref{i:6.4.12}(a), we know that there exists an \(N \geq m\) such that \(\dfrac{c_{n + 1}}{c_n} \geq L - \varepsilon\) for all \(n \geq N\)
  (without loss of generality we may assume that \(N \geq 1\)).
  This implies that \(c_{n + 1} \geq c_n (L - \varepsilon)\) for all \(n \geq N\).
  By induction this implies that
  \[
    c_n \geq c_N (L - \varepsilon)^{n - N} \text{ for all } n \geq N.
  \]
  If we write \(A \coloneqq c_N (L - \varepsilon)^{-N}\), then we have
  \[
    c_n \geq A(L - \varepsilon)^n
  \]
  and thus
  \[
    c_n^{1 / n} \geq A^{1 / n} (L - \varepsilon)
  \]
  for all \(n \geq N\).
  But we have
  \[
    \lim_{n \to \infty} A^{1 / n} (L - \varepsilon) = L - \varepsilon
  \]
  by the limit laws (\cref{i:6.1.19}) and \cref{i:6.5.3}.
  Thus by the comparison principle (\cref{i:6.4.13}) we have
  \[
    \liminf_{n \to \infty} c_n^{1 / n} \geq L - \varepsilon.
  \]
  But this is true for all \(\varepsilon > 0\), so this must imply that
  \[
    \liminf_{n \to \infty} c_n^{1 / n} \geq L.
  \]
  (If \(\liminf_{n \to \infty} c_n^{1 / n} < L\), then when \(\varepsilon = (L - \liminf_{n \to \infty} c_n^{1 / n}) / 2\) we have
  \[
    \liminf_{n \to \infty} c_n^{1 / n} \geq \dfrac{\liminf_{n \to \infty} c_n^{1 / n} + L}{2},
  \]
  a contradiction.), as desired.
\end{proof}

\begin{cor}[Ratio test]\label{i:7.5.3}
  Let \(\sum_{n = m}^\infty a_n\) be a series of non-zero numbers.
  (The non-zero hypothesis is required so that the ratios \(\abs{a_{n + 1}} / \abs{a_n}\) appearing below are well-defined.)
  \begin{itemize}
    \item If \(\limsup_{n \to \infty} \dfrac{\abs{a_{n + 1}}}{\abs{a_n}} < 1\), then the series \(\sum_{n = m}^\infty a_n\) is absolutely convergent (hence conditionally convergent).
    \item If \(\liminf_{n \to \infty} \dfrac{\abs{a_{n + 1}}}{\abs{a_n}} > 1\), then the series \(\sum_{n = m}^\infty a_n\) is not conditionally convergent (and thus cannot be absolutely convergent).
    \item In the remaining cases, we cannot assert any conclusion.
  \end{itemize}
\end{cor}

\begin{proof}
  We first show that if \(\limsup_{n \to \infty} \dfrac{\abs{a_{n + 1}}}{\abs{a_n}} < 1\), then the series \(\sum_{n = m}^\infty a_n\) is absolutely convergent.
  \begin{align*}
             & \limsup_{n \to \infty} \dfrac{\abs{a_{n + 1}}}{\abs{a_n}} < 1                                                                 \\
    \implies & \limsup_{n \to \infty} \abs{a_n}^{1 / n} \leq \limsup_{n \to \infty} \dfrac{\abs{a_{n + 1}}}{\abs{a_n}} < 1 &  & \by{i:7.5.2} \\
    \implies & \sum_{n = m}^\infty a_n \text{ is absolutely convergent}.                                                   &  & \by{i:7.5.1}
  \end{align*}

  Next we show that if \(\liminf_{n \to \infty} \dfrac{\abs{a_{n + 1}}}{\abs{a_n}} > 1\), then the series \(\sum_{n = m}^\infty a_n\) is not conditionally convergent.
  \begin{align*}
             & \liminf_{n \to \infty} \dfrac{\abs{a_{n + 1}}}{\abs{a_n}} > 1                                                                 \\
    \implies & \limsup_{n \to \infty} \abs{a_n}^{1 / n} \geq \liminf_{n \to \infty} \dfrac{\abs{a_{n + 1}}}{\abs{a_n}} > 1 &  & \by{i:7.5.2} \\
    \implies & \sum_{n = m}^\infty a_n \text{ is not conditionally convergent}.                                            &  & \by{i:7.5.1}
  \end{align*}

  Finally we show that if \(\liminf_{n \to \infty} \dfrac{\abs{a_{n + 1}}}{\abs{a_n}} \leq 1\) or \(\limsup_{n \to \infty} \dfrac{\abs{a_{n + 1}}}{\abs{a_n}} \geq 1\), then we cannot assert any conclusion.
  See \cref{i:7.5.1}.
\end{proof}

\begin{prop}\label{i:7.5.4}
  We have \(\lim_{n \to \infty} n^{1 / n} = 1\).
\end{prop}

\begin{proof}
  By \cref{i:7.5.2} we have
  \[
    \limsup_{n \to \infty} n^{1 / n} \leq \limsup_{n \to \infty} (n + 1) / n = \limsup_{n \to \infty} 1 + 1 / n = 1
  \]
  by \cref{i:6.1.11} and limit laws (\cref{i:6.1.19}).
  Similarly we have
  \[
    \liminf_{n \to \infty} n^{1 / n} \geq \liminf_{n \to \infty} (n + 1) / n = \liminf_{n \to \infty} 1 + 1 / n = 1.
  \]
  The claim then follows from \cref{i:6.4.12}(c) and (f).
\end{proof}

\begin{rmk}\label{i:7.5.5}
  In addition to the ratio and root tests, another very useful convergence test is the \emph{integral test}, which we will cover in \cref{i:11.6.4}.
\end{rmk}

\exercisesection

\begin{ex}\label{i:ex:7.5.1}
  Prove the first inequality in \cref{i:7.5.2}.
\end{ex}

\begin{proof}
  See \cref{i:7.5.2}.
\end{proof}

\begin{ex}\label{i:ex:7.5.2}
  Let \(x\) be a real number with \(\abs{x} < 1\), and \(q\) be a real number.
  Show that the series \(\sum_{n = 1}^\infty n^q x^n\) is absolutely convergent, and that \(\lim_{n \to \infty} n^q x^n = 0\).
\end{ex}

\begin{proof}
  Let \(N \in \N\).
  Since \(q \in \R\), by \cref{i:5.4.12} \(\exists N \geq q\).
  Then we have
  \begin{align*}
             & \forall n \geq 1, \abs{n^q x^n} \leq \abs{n^N x^n}                                         &  & \by{i:5.6.9}           \\
    \implies & \forall n \geq 1, \abs{n^q x^n}^{1 / n} \leq \abs{n^N x^n}^{1 / n}                         &  & \by{i:5.6.9}           \\
    \implies & \forall n \geq 1, n^{q / n} \abs{x} \leq n^{N / n} \abs{x}                                 &  & \by{i:5.6.3}           \\
    \implies & \limsup_{n \to \infty} n^{q / n} \abs{x} \leq \limsup_{n \to \infty} n^{N / n} \abs{x}     &  & \by{i:6.4.13}          \\
    \implies & \limsup_{n \to \infty} n^{q / n} \abs{x} \leq (\limsup_{n \to \infty} n^{1 / n})^N \abs{x} &  & \by{i:6.1.19}          \\
    \implies & \limsup_{n \to \infty} n^{q / n} \abs{x} \leq 1^N \abs{x}                                  &  & \by{i:7.5.4}           \\
    \implies & \limsup_{n \to \infty} n^{q / n} \abs{x} \leq \abs{x} < 1                                  &  & \text{(by hypothesis)} \\
    \implies & \sum_{n = 1}^\infty n^q x^n \text{ is absolutely convergent}                               &  & \by{i:7.5.1}           \\
    \implies & \sum_{n = 1}^\infty n^q x^n \text{ converges}                                              &  & \by{i:7.2.9}           \\
    \implies & \lim_{n \to \infty} n^q x^n = 0                                                            &  & \by{i:7.2.6}
  \end{align*}
\end{proof}

\begin{ex}\label{i:ex:7.5.3}
  Give an example of a divergent series \(\sum_{n = 1}^\infty a_n\) of positive numbers \(a_n\) such that \(\lim_{n \to \infty} a_{n + 1} / a_n = \lim_{n \to \infty} a_n^{1 / n} = 1\), and give an example of a convergent series \(\sum_{n = 1}^\infty b_n\) of positive numbers \(b_n\) such that \(\lim_{n \to \infty} b_{n + 1} / b_n = \lim_{n \to \infty} b_n^{1 / n} = 1\).
  This shows that the ratio and root tests can be inconclusive even when the summands are positive and all the limits converge.
\end{ex}

\begin{proof}
  See \cref{i:7.5.1}.
\end{proof}
