\section{The root and ratio tests}\label{sec 7.5}

\begin{theorem}[Root test]\label{7.5.1}
Let \(\sum_{n = m}^\infty a_n\) be a series of real numbers, and let \(\alpha \coloneqq \lim\sup_{n \to \infty} \abs*{a_n}^{1 / n}\).
\begin{enumerate}
    \item If \(\alpha < 1\), then the series \(\sum_{n = m}^\infty a_n\) is absolutely convergent
    (and hence conditionally convergent).
    \item If \(\alpha > 1\), then the series \(\sum_{n = m}^\infty a_n\) is not conditionally convergent
    (and hence cannot be absolutely convergent either).
    \item If \(\alpha = 1\), we cannot assert any conclusion.
\end{enumerate}
\end{theorem}

\begin{proof}
First suppose that \(\alpha < 1\).
Note that we must have \(\alpha \geq 0\), since by Lemma \ref{5.6.6} \(\abs*{a_n}^{1 / n} \geq 0\) for every \(n\).
Then we can find an \(\varepsilon > 0\) such that \(0 < \alpha + \varepsilon < 1\) (for instance, we can set \(\varepsilon \coloneqq (1 - \alpha) / 2\)).
By Proposition \ref{6.4.12}(a), there exists an \(N \geq m\) such that \(\abs*{a_n}^{1 / n} \leq \alpha + \varepsilon\) for all \(n \geq N\).
In other words, we have \(\abs*{a_n} \leq (\alpha + \varepsilon)^n\) for all \(n \geq N\).
But from the geometric series (Lemma \ref{7.3.3}) we have that \(\sum_{n = N}^\infty (\alpha + \varepsilon)^n\) is absolutely convergent, since \(0 < \alpha + \varepsilon < 1\)
(note that the fact that we start from \(N\) is irrelevant by Proposition \ref{7.2.14}(c)).
Thus by the comparison test (Corollary \ref{7.3.2}), we see that \(\sum_{n = N}^\infty a_n\) is absolutely convergent, and thus \(\sum_{n = m}^\infty a_n\) is absolutely convergent, by Proposition \ref{7.2.14}(c) again.

Now suppose that \(\alpha > 1\).
Then by Proposition \ref{6.4.12}(b), we see that for every \(N \geq m\) there exists an \(n \geq N\) such that \(\abs*{a_n}^{1 / n} \geq 1\), and hence that \(\abs*{a_n} \geq 1\).
In particular, \((a_n)_{n = N}^\infty\) is not \(1\)-close to \(0\) for any \(N\), and hence \((a_n)_{n = m}^\infty\) is not eventually \(1\)-close to \(0\).
In particular, \((a_n)_{n = m}^\infty\) does not converge to zero.
Thus by the zero test (Corollary \ref{7.2.6}), \(\sum_{n = m}^\infty a_n\) is not conditionally convergent.

For \(\alpha = 1\), we show two sequences \((a_n)_{n = m}^\infty\) and \((b_n)_{n = m}^\infty\) where
\[
    \lim\sup_{n \to \infty} \abs*{a_n}^{1 / n} = \lim\sup_{n \to \infty} \abs*{b_n}^{1 / n} = 1
\]
and
\[
    \lim\sup_{n \to \infty} \frac{\abs*{a_{n + 1}}}{\abs*{a_n}} = \lim\sup_{n \to \infty} \frac{\abs*{b_{n + 1}}}{\abs*{b_n}} = 1
\]
but \(\sum_{n = m}^\infty a_n\) is convergent and \(\sum_{n = m}^\infty b_n\) is not.
Let \(a_n = 1 / n\) and \(b_n = (-1)^n / n\).
Then by Corollary \ref{7.3.7} \(\sum_{n = m}^\infty a_n\) is divergent, and by Alternating series test (Proposition \ref{7.2.12}) \(\sum_{n = m}^\infty b_n\) is convergent.
By Lemma \ref{7.5.2}, we have
\[
    \lim\sup_{n \to \infty} \abs*{\frac{1}{n}}^{1 / n} \leq \lim\sup_{n \to \infty} \frac{\abs*{n}}{\abs*{n + 1}} = \lim\sup_{n \to \infty} 1 - \frac{1}{n + 1} = 1
\]
by Proposition \ref{6.1.11} and limit laws (Theorem \ref{6.1.19}).
Similarly we have
\[
    \lim\inf_{n \to \infty} \abs*{\frac{1}{n}}^{1 / n} \geq \lim\inf_{n \to \infty} \frac{\abs*{n}}{\abs*{n + 1}} = \lim\inf_{n \to \infty} 1 - \frac{1}{n + 1} = 1.
\]
By Proposition \ref{6.4.12} we have
\[
    \lim\sup_{n \to \infty} \frac{\abs*{n}}{\abs*{n + 1}} = \lim\inf_{n \to \infty} \frac{\abs*{n}}{\abs*{n + 1}} = \lim_{n \to \infty} \frac{\abs*{n}}{\abs*{n + 1}} = 1
\]
and
\[
    \lim\sup_{n \to \infty} \abs*{\frac{1}{n}}^{1 / n} = \lim\inf_{n \to \infty} \abs*{\frac{1}{n}}^{1 / n} = \lim_{n \to \infty} \abs*{\frac{1}{n}}^{1 / n} = 1.
\]
So we have \(\lim\sup_{n \to \infty} \abs*{a_n}^{1 / n} = \lim\sup_{n \to \infty} \frac{\abs*{a_{n + 1}}}{\abs*{a_n}} = 1\) but \(\sum_{n = m}^\infty a_n\) is divergent.
By Lemma \ref{7.5.2} again, we have
\[
    \lim\sup_{n \to \infty} \abs*{\frac{(-1)^n}{n}}^{1 / n} \leq \lim\sup_{n \to \infty} \frac{\abs*{(-1)^n (n + 1)}}{\abs*{(-1)^{n + 1} n}} = \lim\sup_{n \to \infty} 1 - \frac{1}{n + 1} = 1
\]
by Proposition \ref{6.1.11} and limit laws (Theorem \ref{6.1.19}).
Similarly we have
\[
    \lim\inf_{n \to \infty} \abs*{\frac{(-1)^n}{n}}^{1 / n} \geq \lim\inf_{n \to \infty} \frac{\abs*{(-1)^n (n + 1)}}{\abs*{(-1)^{n + 1} n}} = \lim\inf_{n \to \infty} 1 - \frac{1}{n + 1} = 1
\]
By Proposition \ref{6.4.12} we have
\[
    \lim\sup_{n \to \infty} \frac{\abs*{(-1)^n (n + 1)}}{\abs*{(-1)^{n + 1} n}} = \lim\inf_{n \to \infty} \frac{\abs*{(-1)^n (n + 1)}}{\abs*{(-1)^{n + 1} n}} = \lim_{n \to \infty} \frac{\abs*{(-1)^n (n + 1)}}{\abs*{(-1)^{n + 1} n}} = 1
\]
and
\[
    \lim\sup_{n \to \infty} \abs*{\frac{(-1)^n}{n}}^{1 / n} = \lim\inf_{n \to \infty} \abs*{\frac{(-1)^n}{n}}^{1 / n} = \lim_{n \to \infty} \abs*{\frac{(-1)^n}{n}}^{1 / n} = 1.
\]
So we have \(\lim\sup_{n \to \infty} \abs*{b_n}^{1 / n} = \lim\sup_{n \to \infty} \frac{\abs*{b_{n + 1}}}{\abs*{b_n}} = 1\) but \(\sum_{n = m}^\infty b_n\) is convergent.
Thus we conclude that when \(\lim\sup_{n \to \infty} \abs*{a_n}^{1 / n} = \lim\sup_{n \to \infty} \frac{\abs*{a_{n + 1}}}{\abs*{a_n}} = 1\) we cannot assert any conclusion.
\end{proof}

\begin{note}
The root test is phrased using the limit superior, but of course if \(\lim_{n \to \infty} \abs*{a_n}^{1 / n}\) converges then the limit is the same as the limit superior.
Thus one can phrase the root test using the limit instead of the limit superior, but \emph{only when the limit exists}.
\end{note}

\begin{lemma}\label{7.5.2}
Let \((c_n)_{n = m}^\infty\) be a sequence of positive numbers.
Then we have
\[
    \lim\inf_{n \to \infty} \frac{c_{n + 1}}{c_n} \leq \lim\inf_{n \to \infty} c_n^{1 / n} \leq \lim\sup_{n \to \infty} c_n^{1 / n} \leq \lim\sup_{n \to \infty} \frac{c_{n + 1}}{c_n}.
\]
\end{lemma}

\begin{proof}
There are three inequalities to prove here.
The middle inequality follows from Proposition \ref{6.4.12}(c).

Next we show that \(\lim\sup_{n \to \infty} c_n^{1 / n} \leq \lim\sup_{n \to \infty} \frac{c_{n + 1}}{c_n}\).
Write \(L \coloneqq \lim\sup_{n \to \infty} \frac{c_{n + 1}}{c_n}\).
If \(L = +\infty\) then there is nothing to prove (since \(x \leq +\infty\) for every extended real number \(x\)), so we may assume that \(L\) is a finite real number.
(Note that \(L\) cannot equal \(-\infty\)).
Since \(\frac{c_{n + 1}}{c_n}\) is always positive, we know that \(L \geq 0\).

Let \(\varepsilon > 0\).
By Proposition \ref{6.4.12}(a), we know that there exists an \(N \geq m\) such that \(\frac{c_{n + 1}}{c_n} \leq L + \varepsilon\) for all \(n \geq N\).
This implies that \(c_{n + 1} \leq c_n (L + \varepsilon)\) for all \(n \geq N\).
By induction this implies that
\[
    c_n \leq c_N (L + \varepsilon)^{n - N} \text{ for all } n \geq N.
\]
If we write \(A \coloneqq c_N (L + \varepsilon)^{-N}\), then we have
\[
    c_n \leq A(L + \varepsilon)^n
\]
and thus
\[
    c_n^{1 / n} \leq A^{1 / n} (L + \varepsilon)
\]
for all \(n \geq N\).
But we have
\[
    \lim_{n \to \infty} A^{1 / n} (L + \varepsilon) = L + \varepsilon
\]
by the limit laws (Theorem \ref{6.1.19}) and Lemma \ref{6.5.3}.
Thus by the comparison principle (Lemma \ref{6.4.13}) we have
\[
    \lim\sup_{n \to \infty} c_n^{1 / n} \leq L + \varepsilon.
\]
But this is true for all \(\varepsilon > 0\), so this must imply that
\[
    \lim\sup_{n \to \infty} c_n^{1 / n} \leq L.
\]
(If \(L < \lim\sup_{n \to \infty} c_n^{1 / n}\), then when \(\varepsilon = (\lim\sup_{n \to \infty} c_n^{1 / n} - L) / 2\) we have
\[
    \lim\sup_{n \to \infty} c_n^{1 / n} \leq \frac{\lim\sup_{n \to \infty} c_n^{1 / n} + L}{2},
\]
a contradiction.), as desired.

Finally we show that \(\lim\inf_{n \to \infty} \frac{c_{n + 1}}{c_n} \leq \lim\inf_{n \to \infty} c_n^{1 / n}\).
Write \(L \coloneqq \lim\inf_{n \to \infty} \frac{c_{n + 1}}{c_n}\).
Since \(\frac{c_{n + 1}}{c_n}\) is always positive, we know that \(L \geq 0\).

Let \(\varepsilon > 0\).
By Proposition \ref{6.4.12}(a), we know that there exists an \(N \geq m\) such that \(\frac{c_{n + 1}}{c_n} \geq L - \varepsilon\) for all \(n \geq N\).
This implies that \(c_{n + 1} \geq c_n (L - \varepsilon)\) for all \(n \geq N\).
By induction this implies that
\[
    c_n \geq c_N (L - \varepsilon)^{n - N} \text{ for all } n \geq N.
\]
If we write \(A \coloneqq c_N (L - \varepsilon)^{-N}\), then we have
\[
    c_n \geq A(L - \varepsilon)^n
\]
and thus
\[
    c_n^{1 / n} \geq A^{1 / n} (L - \varepsilon)
\]
for all \(n \geq N\).
But we have
\[
    \lim_{n \to \infty} A^{1 / n} (L - \varepsilon) = L - \varepsilon
\]
by the limit laws (Theorem \ref{6.1.19}) and Lemma \ref{6.5.3}.
Thus by the comparison principle (Lemma \ref{6.4.13}) we have
\[
    \lim\inf_{n \to \infty} c_n^{1 / n} \geq L - \varepsilon.
\]
But this is true for all \(\varepsilon > 0\), so this must imply that
\[
    \lim\inf_{n \to \infty} c_n^{1 / n} \geq L.
\]
(If \(L > \lim\inf_{n \to \infty} c_n^{1 / n}\), then when \(\varepsilon = (L - \lim\inf_{n \to \infty} c_n^{1 / n}) / 2\) we have
\[
    \lim\inf_{n \to \infty} c_n^{1 / n} \geq \frac{\lim\inf_{n \to \infty} c_n^{1 / n} + L}{2},
\]
a contradiction.), as desired.
\end{proof}

\begin{corollary}[Ratio test]\label{7.5.3}
Let \(\sum_{n = m}^\infty a_n\) be a series of non-zero numbers.
(The non-zero hypothesis is required so that the ratios \(\abs*{a_{n + 1}} / \abs*{a_n}\) appearing below are well-defined.)
\begin{enumerate}
    \item If \(\lim\sup_{n \to \infty} \frac{\abs*{a_{n + 1}}}{\abs*{a_n}} < 1\), then the series \(\sum_{n = m}^\infty a_n\) is absolutely convergent (hence conditionally convergent).
    \item If \(\lim\inf_{n \to \infty} \frac{\abs*{a_{n + 1}}}{\abs*{a_n}} > 1\), then the series \(\sum_{n = m}^\infty a_n\) is not conditionally convergent (and thus cannot be absolutely convergent).
    \item In the remaining cases, we cannot assert any conclusion.
\end{enumerate}
\end{corollary}

\begin{proof}
We first show that if \(\lim\sup_{n \to \infty} \frac{\abs*{a_{n + 1}}}{\abs*{a_n}} < 1\), then the series \(\sum_{n = m}^\infty a_n\) is absolutely convergent.
\begin{align*}
& \lim\sup_{n \to \infty} \frac{\abs*{a_{n + 1}}}{\abs*{a_n}} < 1 \\
\implies & \lim\sup_{n \to \infty} \abs*{a_n}^{1 / n} \leq \lim\sup_{n \to \infty} \frac{\abs*{a_{n + 1}}}{\abs*{a_n}} < 1 & \text{(by Lemma \ref{7.5.2})} \\
\implies & \sum_{n = m}^\infty a_n \text{ is absolutely convergent}. & \text{(by Theorem \ref{7.5.1})}
\end{align*}

Next we show that if \(\lim\inf_{n \to \infty} \frac{\abs*{a_{n + 1}}}{\abs*{a_n}} > 1\), then the series \(\sum_{n = m}^\infty a_n\) is not conditionally convergent.
\begin{align*}
& \lim\inf_{n \to \infty} \frac{\abs*{a_{n + 1}}}{\abs*{a_n}} > 1 \\
\implies & \lim\sup_{n \to \infty} \abs*{a_n}^{1 / n} \geq \lim\inf_{n \to \infty} \frac{\abs*{a_{n + 1}}}{\abs*{a_n}} > 1 & \text{(by Lemma \ref{7.5.2})} \\
\implies & \sum_{n = m}^\infty a_n \text{ is not conditionally convergent}. & \text{(by Theorem \ref{7.5.1})}
\end{align*}

Finally we show that if \(\lim\inf_{n \to \infty} \frac{\abs*{a_{n + 1}}}{\abs*{a_n}} \leq 1\) or \(\lim\sup_{n \to \infty} \frac{\abs*{a_{n + 1}}}{\abs*{a_n}} \geq 1\), then we cannot assert any conclusion.
See Theorem \ref{7.5.1}.
\end{proof}

\begin{proposition}\label{7.5.4}
We have \(\lim_{n \to \infty} n^{1 / n} = 1\).
\end{proposition}

\begin{proof}
By Lemma \ref{7.5.2} we have
\[
    \lim\sup_{n \to \infty} n^{1 / n} \leq \lim\sup_{n \to \infty} (n + 1) / n = \lim\sup_{n \to \infty} 1 + 1 / n = 1
\]
by Proposition \ref{6.1.11} and limit laws (Theorem \ref{6.1.19}).
Similarly we have
\[
    \lim\inf_{n \to \infty} n^{1 / n} \geq \lim\inf_{n \to \infty} (n + 1) / n = \lim\inf_{n \to \infty} 1 + 1 / n = 1.
\]
The claim then follows from Proposition \ref{6.4.12}(c) and (f).
\end{proof}

\begin{remark}\label{7.5.5}
In addition to the ratio and root tests, another very useful convergence test is the \emph{integral test}.
\end{remark}

\exercisesection

\begin{exercise}\label{ex 7.5.1}
Prove the first inequality in Lemma \ref{7.5.2}.
\end{exercise}

\begin{proof}
See Lemma \ref{7.5.2}.
\end{proof}

\begin{exercise}\label{ex 7.5.2}
Let \(x\) be a real number with \(\abs*{x} < 1\), and \(q\) be a real number.
Show that the series \(\sum_{n = 1}^\infty n^q x^n\) is absolutely convergent, and that \(\lim_{n \to \infty} n^q x^n = 0\).
\end{exercise}

\begin{proof}
First suppose that \(q = 0\), then by Proposition \ref{5.6.3} and Theorem \ref{6.1.19} we have
\[
    \lim\sup_{n \to \infty} \abs*{x^n}^{1 / n} = \lim\sup_{n \to \infty} \abs*{x} = \abs*{x} < 1.
\]
By root test (Theorem \ref{7.5.1}) this means \(\sum_{n = 1}^\infty x^n\) is absolutely convergent.

Next suppose that \(q > 0\), then we have
\[
    \lim\sup_{n \to \infty} \frac{\abs*{a_{n + 1}}}{\abs*{a_n}} = \lim\sup_{n \to \infty} \frac{\abs*{(n + 1)^q x^{(n + 1)}}}{\abs*{n^q x^n}} = \lim\sup_{n \to \infty} \bigg(\frac{n + 1}{n}\bigg)^q \abs*{x}
\]
Since \(\abs*{x} < 1\), we have
\[
    \abs*{x} < \frac{\abs*{x} + 1}{2} < 1 \implies \frac{2}{\abs*{x} + 1} > 1 \implies \bigg(\frac{2}{\abs*{x} + 1}\bigg)^{1 / q} > 1.
\]
Since \(\lim_{n \to \infty} \frac{n + 1}{n} = 1\) by Proposition \ref{6.1.11} and Theorem \ref{6.1.19}, we have
\[
    \exists\ N \geq 1 : \frac{n + 1}{n} < \bigg(\frac{2}{\abs*{x} + 1}\bigg)^{1 / q} \ \forall\ n \geq N
\]
by Proposition \ref{6.4.12}.
And this give us
\[
    \bigg(\frac{n + 1}{n}\bigg)^q \abs*{x} < \frac{2\abs*{x}}{\abs*{x} + 1}.
\]
Thus by Lemma \ref{6.4.13} we have
\[
    \lim\sup_{n \to \infty} \frac{\abs*{a_{n + 1}}}{\abs*{a_n}} = \lim\sup_{n \to \infty} \bigg(\frac{n + 1}{n}\bigg)^q \abs*{x} < \frac{2\abs*{x}}{\abs*{x} + 1} < 1.
\]
By ratio test (Corollary \ref{7.5.3}) this means \(\sum_{n = 1}^\infty n^q x^n\) is absolutely convergent.

Now suppose that \(q < 0\), then we have \(-q > 0\) and
\[
    \lim\sup_{n \to \infty} \frac{\abs*{a_{n + 1}}}{\abs*{a_n}} = \lim\sup_{n \to \infty} \bigg(\frac{n + 1}{n}\bigg)^q \abs*{x} = \lim\sup_{n \to \infty} \bigg(\frac{n}{n + 1}\bigg)^{-q} \abs*{x} < \frac{2\abs*{x}}{\abs*{x} + 1} < 1.
\]
By ratio test (Corollary \ref{7.5.3}) this means \(\sum_{n = 1}^\infty n^q x^n\) is absolutely convergent.
For all \(q \in \mathbf{R}\) we have \(\sum_{n = 1}^\infty n^q x^n\) is absolutely convergent.
By absolute convergence test (Proposition \ref{7.2.9}) and zero test (Corollary \ref{7.2.6}) we have \(\lim_{n \to \infty} n^q x^n = 0\).
\end{proof}

\begin{exercise}\label{ex 7.5.3}
Give an example of a divergent series \(\sum_{n = 1}^\infty a_n\) of positive numbers \(a_n\) such that \(\lim_{n \to \infty} a_{n + 1} / a_n = \lim_{n \to \infty} a_n^{1 / n} = 1\), and give an example of a convergent series \(\sum_{n = 1}^\infty b_n\) of positive numbers \(b_n\) such that \(\lim_{n \to \infty} b_{n + 1} / b_n = \lim_{n \to \infty} b_n^{1 / n} = 1\).
This shows that the ratio and root tests can be inconclusive even when the summands are positive and all the limits converge.
\end{exercise}

\begin{proof}
See Theorem \ref{7.5.1}.
\end{proof}