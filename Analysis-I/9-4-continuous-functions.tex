\section{Continuous functions}\label{sec 9.4}

\begin{definition}[Continuity]\label{9.4.1}
    Let \(X\) be a subset of \(\mathbf{R}\), and let \(f : X \to \mathbf{R}\) be a function.
    Let \(x_0\) be an element of \(X\).
    We say that \(f\) is \emph{continuous at \(x_0\)} iff we have
    \[
        \lim_{x \to x_0 ; x \in X} f(x) = f(x_0);
    \]
    in other words, the limit of \(f(x)\) as \(x\) converges to \(x_0\) in \(X\) exists and is equal to \(f(x_0)\).
    We say that \(f\) is \emph{continuous on \(X\)} (or simply \emph{continuous}) iff \(f\) is continuous at \(x_0\) for every \(x_0 \in X\).
    We say that \(f\) is \emph{discontinuous at \(x_0\)} iff it is not continuous at \(x_0\).
\end{definition}

\begin{note}
    Restricting the domain of a function can make a discontinuous function continuous again.
\end{note}

\setcounter{theorem}{6}
\begin{proposition}[Equivalent formulations of continuity]\label{9.4.7}
    Let \(X\) be a subset of \(\mathbf{R}\), let \(f : X \to \mathbf{R}\) be a function, and let \(x_0\) be an element of \(X\).
    Then the following four statements are logically equivalent:
    \begin{enumerate}
        \item \(f\) is continuous at \(x_0\).
        \item For every sequence \((a_n)_{n = 0}^\infty\) consisting of elements of \(X\) with \(\lim_{n \to \infty} a_n = x_0\), we have \(\lim_{n \to \infty} f(a_n) = f(x_0)\).
        \item For every \(\varepsilon > 0\), there exists a \(\delta > 0\) such that \(\abs*{f(x) - f(x_0)} < \varepsilon\) for all \(x \in X\) with \(\abs*{x - x_0} < \delta\).
        \item For every \(\varepsilon > 0\), there exists a \(\delta > 0\) such that \(\abs*{f(x) - f(x_0)} \leq \varepsilon\) for all \(x \in X\) with \(\abs*{x - x_0} \leq \delta\).
    \end{enumerate}
\end{proposition}

\begin{proof}
    We first show that the statement (a) and the statement (b) are equivalent.
    By Definition \ref{9.4.1}, \(f\) is continuous at \(x_0\) iff \(f\) converges to \(f(x_0)\) at \(x_0\) in \(X\).
    Thus by Proposition \ref{9.3.9} we know that the statement (a) and the statement (b) are equivalent.

    Next we show that the statement (a) and the statement (c) are equivalent.
    By Definition \ref{9.4.1}, \(f\) is continuous at \(x_0\) iff \(f\) converges to \(f(x_0)\) at \(x_0\) in \(X\).
    By Definition \ref{9.3.6} this is equivalent to the statement
    \[
        \forall\ \varepsilon' \in \mathbf{R}^+, \exists\ \delta \in \mathbf{R}^+ : \bigg(\forall\ x \in X, \abs*{x - x_0} < \delta \implies \abs*{f(x) - f(x_0)} \leq \varepsilon'\bigg).
    \]
    Let \(\varepsilon \in \mathbf{R}^+ \land \varepsilon > \varepsilon'\).
    Then we have
    \[
        \forall\ \varepsilon \in \mathbf{R}^+, \exists\ \delta \in \mathbf{R}^+ : \bigg(\forall\ x \in X, \abs*{x - x_0} < \delta \implies \abs*{f(x) - f(x_0)} \leq \varepsilon' < \varepsilon\bigg).
    \]
    Thus the statement (a) and the statement (c) are equivalent.

    Finally we show that the statement (a) and the statement (d) are equivalent.
    By Definition \ref{9.4.1}, \(f\) is continuous at \(x_0\) iff \(f\) converges to \(f(x_0)\) at \(x_0\) in \(X\).
    By Definition \ref{9.3.6} this is equivalent to the statement
    \[
        \forall\ \varepsilon \in \mathbf{R}^+, \exists\ \delta' \in \mathbf{R}^+ : \bigg(\forall\ x \in X, \abs*{x - x_0} < \delta' \implies \abs*{f(x) - f(x_0)} \leq \varepsilon\bigg).
    \]
    Let \(\delta \in \mathbf{R}^+ \land \abs*{x - x_0} \leq \delta < \delta'\).
    By Proposition \ref{5.4.14} we know such \(\delta\) exists.
    Then we have
    \[
        \forall\ \varepsilon \in \mathbf{R}^+, \exists\ \delta \in \mathbf{R}^+ : \bigg(\forall\ x \in X, \abs*{x - x_0} \leq \delta \implies \abs*{f(x) - f(x_0)} \leq \varepsilon\bigg).
    \]
    Thus the statement (a) and the statement (d) are equivalent.
\end{proof}

\begin{remark}\label{9.4.8}
    A particularly useful consequence of Proposition \ref{9.4.7} is the following:
    if \(f\) is continuous at \(x_0\), and \(a_n \to x_0\) as \(n \to \infty\), then \(f(a_n) \to f(x_0)\) as \(n \to \infty\)
    (provided that all the elements of the sequence \((a_n)_{n = 0}^\infty\) lie in the domain of \(f\), of course).
    Thus continuous functions are very useful in computing limits.
\end{remark}

\begin{proposition}[Arithmetic preserves continuity]\label{9.4.9}
    Let \(X\) be a subset of \(\mathbf{R}\), and let \(f : X \to \mathbf{R}\) and \(g : X \to \mathbf{R}\) be functions.
    Let \(x_0 \in X\).
    Then if \(f\) and \(g\) are both continuous at \(x_0\), then the functions \(f + g\), \(f - g\), \(\max(f, g)\), \(\min(f, g)\) and \(fg\) are also continuous at \(x_0\).
    If \(g\) is non-zero on \(X\), then \(f / g\) is also continuous at \(x_0\).
\end{proposition}

\begin{proof}
    By Proposition \ref{9.3.14}, we have
    \begin{align*}
        \lim_{x \to x_0 ; x \in X} f(x) + g(x)      & = \lim_{x \to x_0 ; x \in X} f(x) + \lim_{x \to x_0 ; x \in X} g(x)                                                            \\
                                                    & = f(x_0) + g(x_0);                                                                       & \text{(by Definition \ref{9.4.1})}  \\
        \lim_{x \to x_0 ; x \in X} f(x) - g(x)      & = \lim_{x \to x_0 ; x \in X} f(x) - \lim_{x \to x_0 ; x \in X} g(x)                                                            \\
                                                    & = f(x_0) - g(x_0);                                                                       & \text{(by Definition \ref{9.4.1})}  \\
        \lim_{x \to x_0 ; x \in X} \max(f(x), g(x)) & = \max(\lim_{x \to x_0 ; x \in X} f(x), \lim_{x \to x_0 ; x \in X} g(x))                                                       \\
                                                    & = \max(f(x_0), g(x_0));                                                                  & \text{(by Definition \ref{9.4.1})}  \\
        \lim_{x \to x_0 ; x \in X} \min(f(x), g(x)) & = \min(\lim_{x \to x_0 ; x \in X} f(x), \lim_{x \to x_0 ; x \in X} g(x))                                                       \\
                                                    & = \min(f(x_0), g(x_0));                                                                  & \text{(by Definition \ref{9.4.1})}  \\
        \lim_{x \to x_0 ; x \in X} f(x) g(x)        & = \bigg(\lim_{x \to x_0 ; x \in X} f(x)\bigg)\bigg(\lim_{x \to x_0 ; x \in X} g(x)\bigg)                                       \\
                                                    & = f(x_0) g(x_0);                                                                         & \text{(by Definition \ref{9.4.1})}  \\
        \lim_{x \to x_0 ; x \in X} f(x) / g(x)      & = \lim_{x \to x_0 ; x \in X} f(x) / \lim_{x \to x_0 ; x \in X} g(x)                      & \text{(\(g\) is non-zero on \(X\))} \\
                                                    & = f(x_0) / g(x_0).                                                                       & \text{(by Definition \ref{9.4.1})}  \\
    \end{align*}
\end{proof}