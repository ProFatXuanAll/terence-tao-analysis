\section{Countability}\label{sec 8.1}

\begin{note}
From Theorem \ref{3.6.12} we know that the set \(\mathbf{N}\) of natural numbers is infinite.
The set \(\mathbf{N} - \{0\}\) is also infinite, thanks to Proposition \ref{3.6.14}(a), and is a proper subset of \(\mathbf{N}\).
However, the set \(\mathbf{N} - \{0\}\), despite being ``smaller'' than \(\mathbf{N}\), still has the same cardinality as \(\mathbf{N}\), because the function \(f : \mathbf{N} \to \mathbf{N} - \{0\}\) defined by \(f(n) \coloneqq n + 1\), is a bijection from \(\mathbf{N}\) to \(\mathbf{N} - \{0\}\).
This is one characteristic of infinite sets.
\end{note}

\begin{definition}[Countable sets]\label{8.1.1}
A set \(X\) is said to be \emph{countably infinite} (or just \emph{countable}) iff it has equal cardinality with the natural numbers \(\mathbf{N}\).
A set \(X\) is said to be \emph{at most countable} iff it is either countable or finite.
We say that a set is \emph{uncountable} if it is infinite but not countable.
\end{definition}

\begin{remark}\label{8.1.2}
Countably infinite sets are also called \emph{denumerable} sets.
\end{remark}

\begin{example}\label{8.1.3}
The even natural numbers \(\{2n : n \in \mathbf{N}\}\), since the function \(f(n) \coloneqq 2n\) provides a bijection between \(\mathbf{N}\) and the even natural numbers.
\end{example}

\begin{note}
Let \(X\) be a countable set.
Then, by definition, we know that there exists a bijection \(f : \mathbf{N} \to X\).
Thus, every element of \(X\) can be written in the form \(f(n)\) for exactly one natural number \(n\).
Informally, we thus have
\[
    X = \{f(0), f(1), f(2), f(3), \dots\}.
\]
Thus, a countable set can be arranged in a sequence, so that we have a zeroth element \(f(0)\), followed by a first element \(f(1)\), then a second element \(f(2)\), and so forth, in such a way that all these elements \(f(0), f(1), f(2), \dots\) are all distinct, and together they fill out all of \(X\).
(This is why these sets are called \emph{countable};
because we can literally count them one by one, starting from \(f(0)\), then \(f(1)\), and so forth.)
\end{note}

\begin{proposition}[Well ordering principle]\label{8.1.4}
Let \(X\) be a non-empty subset of the natural numbers \(\mathbf{N}\).
Then there exists exactly one element \(n \in X\) such that \(n \leq m\) for all \(m \in X\).
In other words, every non-empty set of natural numbers has a minimum element.
\end{proposition}

\begin{proof}
Suppose for sake of contradiction that \(X\) has no minimum element.
Let \(n \in \mathbf{N}\) and let \(P(n)\) be the statement
\[
    \forall\ m \in \mathbf{N} \land m \leq n : m \notin X.
\]
We use induction to show that \(\forall\ n \in \mathbf{N}\), \(P(n)\) is true.
For \(n = 0\), we have \(0 \notin X\) since \(X\) has no mimimum and by Axiom \ref{2.3} \(\forall\ m \in \mathbf{N} \implies m \geq 0\).
So the base case holds.
Suppose inductively that \(P(n)\) is true for some \(n \geq 0\).
Then for \(n + 1\), we need to show that \(P(n + 1)\) is also true.
By induction hypothesis we have \(m \notin X \ \forall\ m \leq n\).
Then if \(n + 1 \in X\), \(n + 1\) is the minimum element of \(X\).
So we must have \(n + 1 \notin X\), and thus \(P(n + 1)\) is true, and this close the induction.

We now use \(P(n)\) to derive contradiction.
Since \(P(n)\) is true \(\forall\ n \in \mathbf{N}\), we must have \(X = \emptyset\), otherwise if \(n \in X\) then \(P(n)\) is false.
But \(X = \emptyset\) contradict to the given condition.
This means \(X\) must have a minimum element.
\end{proof}