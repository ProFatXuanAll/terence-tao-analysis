\section{Countability}\label{sec 8.1}

\begin{note}
From Theorem \ref{3.6.12} we know that the set \(\mathbf{N}\) of natural numbers is infinite.
The set \(\mathbf{N} - \{0\}\) is also infinite, thanks to Proposition \ref{3.6.14}(a), and is a proper subset of \(\mathbf{N}\).
However, the set \(\mathbf{N} - \{0\}\), despite being ``smaller'' than \(\mathbf{N}\), still has the same cardinality as \(\mathbf{N}\), because the function \(f : \mathbf{N} \to \mathbf{N} - \{0\}\) defined by \(f(n) \coloneqq n + 1\), is a bijection from \(\mathbf{N}\) to \(\mathbf{N} - \{0\}\).
This is one characteristic of infinite sets.
\end{note}

\begin{definition}[Countable sets]\label{8.1.1}
A set \(X\) is said to be \emph{countably infinite} (or just \emph{countable}) iff it has equal cardinality with the natural numbers \(\mathbf{N}\).
A set \(X\) is said to be \emph{at most countable} iff it is either countable or finite.
We say that a set is \emph{uncountable} if it is infinite but not countable.
\end{definition}

\begin{remark}\label{8.1.2}
Countably infinite sets are also called \emph{denumerable} sets.
\end{remark}

\begin{example}\label{8.1.3}
The even natural numbers \(\{2n : n \in \mathbf{N}\}\), since the function \(f(n) \coloneqq 2n\) provides a bijection between \(\mathbf{N}\) and the even natural numbers.
\end{example}

\begin{note}
Let \(X\) be a countable set.
Then, by definition, we know that there exists a bijection \(f : \mathbf{N} \to X\).
Thus, every element of \(X\) can be written in the form \(f(n)\) for exactly one natural number \(n\).
Informally, we thus have
\[
    X = \{f(0), f(1), f(2), f(3), \dots\}.
\]
Thus, a countable set can be arranged in a sequence, so that we have a zeroth element \(f(0)\), followed by a first element \(f(1)\), then a second element \(f(2)\), and so forth, in such a way that all these elements \(f(0), f(1), f(2), \dots\) are all distinct, and together they fill out all of \(X\).
(This is why these sets are called \emph{countable};
because we can literally count them one by one, starting from \(f(0)\), then \(f(1)\), and so forth.)
\end{note}

\begin{proposition}[Well ordering principle]\label{8.1.4}
Let \(X\) be a non-empty subset of the natural numbers \(\mathbf{N}\).
Then there exists exactly one element \(n \in X\) such that \(n \leq m\) for all \(m \in X\).
In other words, every non-empty set of natural numbers has a minimum element.
\end{proposition}

\begin{proof}
Suppose for sake of contradiction that \(X\) has no minimum element.
Let \(n \in \mathbf{N}\) and let \(P(n)\) be the statement
\[
    \forall\ m \in \mathbf{N} \land m \leq n : m \notin X.
\]
We use induction to show that \(\forall\ n \in \mathbf{N}\), \(P(n)\) is true.
For \(n = 0\), we have \(0 \notin X\) since \(X\) has no mimimum and by Axiom \ref{2.3} \(\forall\ m \in \mathbf{N} \implies m \geq 0\).
So the base case holds.
Suppose inductively that \(P(n)\) is true for some \(n \geq 0\).
Then for \(n + 1\), we need to show that \(P(n + 1)\) is also true.
By induction hypothesis we have \(m \notin X \ \forall\ m \leq n\).
Then if \(n + 1 \in X\), \(n + 1\) is the minimum element of \(X\).
So we must have \(n + 1 \notin X\), and thus \(P(n + 1)\) is true, and this close the induction.

We now use \(P(n)\) to derive contradiction.
Since \(P(n)\) is true \(\forall\ n \in \mathbf{N}\), we must have \(X = \emptyset\), otherwise if \(n \in X\) then \(P(n)\) is false.
But \(X = \emptyset\) contradict to the given condition.
This means \(X\) must have a minimum element.
\end{proof}

\begin{note}
We will refer to the element \(n\) given by the well-ordering principle as the \emph{minimum} of \(X\), and write it as \(\min(X)\).
This minimum is clearly the same as the infimum of \(X\), as defined in Definition \ref{5.5.10}.
\end{note}

\begin{proposition}\label{8.1.5}
Let \(X\) be an infinite subset of the natural numbers \(\mathbf{N}\).
Then there exists a unique bijection \(f : \mathbf{N} \to X\) which is increasing, in the sense that \(f(n + 1) > f(n)\) for all \(n \in N\).
In particular, \(X\) has equal cardinality with \(\mathbf{N}\) and is hence countable.
\end{proposition}

\begin{proof}
We now define a sequence \(a_0, a_1, a_2, \dots\) of natural numbers recursively by the formula
\[
    a_n \coloneqq \min\{x \in X : x \neq a_m \ \forall\ m < n\}.
\]
Intuitively speaking, \(a_0\) is the smallest element of \(X\);
\(a_1\) is the second smallest element of \(X\), i.e., the smallest element of \(X\) once \(a_0\) is removed;
\(a_2\) is the third smallest element of \(X\);
and so forth.
Observe that in order to define \(a_n\), one only needs to know the values of \(a_m\) for all \(m < n\), so this definition is recursive.
Also, since \(X\) is infinite, the set \(\{x \in X : x \neq a_m \ \forall\ m < n\}\) is infinite, hence non-empty.
(If it is finite, then its union with the set \(\{a_0, \dots, a_{n - 1}\}\) is also finite, contradict to \(X\) is infinite)
Thus by the well-ordering principle (Proposition \ref{8.1.5}), the minimum, \(\min\{x \in X : x \neq a_m \ \forall\ m < n\}\) is always well-defined.

One can show that \(a_n\) is an increasing sequence, i.e.
\[
    a_0 < a_1 < a_2 < \dots
\]
and in particular that \(a_n \neq a_m\) for all \(n \neq m\).
(If \(a_n \geq a_{n + 1}\), then \(a_{n + 1} = \min\{x \in X : x \neq a_m \ \forall\ m < n\}\), a contradiction)
Also, we have \(a_n \in X\) for each natural number \(n\) (by Proposition \ref{8.1.4}).

Now define the function \(f : \mathbf{N} \to X\) by \(f(n) \coloneqq a_n\).
From the previous paragraph we know that \(f\) is one-to-one.
Now we show that \(f\) is onto.
In other words, we claim that for every \(x \in X\), there exists an \(n\) such that \(a_n = x\).

Let \(x \in X\).
Suppose for sake of contradiction that \(a_n \neq x\) for every natural number \(n\).
Then this implies that \(x\) is an element of the set \(\{x \in X : x \neq a_m \ \forall\ m < n\}\) for all \(n\).
By definition of \(a_n\), this implies that \(x \geq a_n\) for every natural number \(n\).
However, since \(a_n\) is an increasing sequence, we have \(a_n \geq n\), and hence \(x \geq n\) for every natural number \(n\).
In particular we have \(x \geq x + 1\), which is a contradiction.
Thus we must have \(a_n = x\) for some natural number \(n\), and hence \(f\) is onto.

Since \(f : \mathbf{N} \to X\) is both one-to-one and onto, it is a bijection.
We have thus found at least one increasing bijection \(f\) from \(\mathbf{N}\) to \(X\).
Now suppose for sake of contradiction that there was at least one other increasing bijection \(g\) from \(\mathbf{N}\) to \(X\) which was not equal to \(f\).
Then the set \(\{n \in \mathbf{N} : g(n) \neq f(n)\}\) is non-empty, and define \(m \coloneqq \min\{n \in \mathbf{N} : g(n) \neq f(n)\}\), thus in particular \(g(m) \neq f(m) = a_m\), and \(g(n) = f(n) = a_n\) for all \(n < m\).
But we then must have
\[
    g(m) = \min\{x \in X : x \neq a_t \ \forall\ t < m\} = a_m,
\]
a contradiction.
Thus there is no other increasing bijection from \(\mathbf{N}\) to \(X\) other than \(f\).
\end{proof}

\begin{corollary}\label{8.1.6}
All subsets of the natural numbers are at most countable.
\end{corollary}

\begin{proof}
Since finite sets are at most countable by definition, combine with Proposition \ref{8.1.5} we thus have all subsets of the natural numbers are at most countable.
\end{proof}

\begin{corollary}\label{8.1.7}
If \(X\) is an at most countable set, and \(Y\) is a subset of \(X\), then \(Y\) is at most countable.
\end{corollary}

\begin{proof}
If \(X\) is finite then this follows from Proposition \ref{3.6.14}(c), so assume \(X\) is countable.
Then there is a bijection \(f : X \to \mathbf{N}\) between \(X\) and \(\mathbf{N}\).
Since \(Y\) is a subset of \(X\), and \(f\) is a bijection from \(X\) and \(\mathbf{N}\), then when we restrict \(f\) to \(Y\), we obtain a bijection between \(Y\) and \(f(Y)\).
Thus \(f(Y)\) has equal cardinality with \(Y\).
But \(f(Y)\) is a subset of \(\mathbf{N}\), and hence at most countable by Corollary \ref{8.1.6}.
Hence \(Y\) is also at most countable.
\end{proof}