\section{The Extended real number system}\label{i:sec:6.2}

\begin{defn}[Extended real number system]\label{i:6.2.1}
  The \emph{extended real number system \(\R^*\)} is the real line \(\R\) with two additional elements attached, called \(+\infty\) and \(-\infty\).
  These elements are distinct from each other and also distinct from every real number.
  An extended real number \(x\) is called \emph{finite} iff it is a real number, and \emph{infinite} iff it is equal to \(+\infty\) or \(-\infty\).
  (This definition is not directly related to the notion of finite and infinite sets in \cref{i:sec:3.6}, though it is of course similar in spirit.)
\end{defn}

\begin{defn}[Negation of extended reals]\label{i:6.2.2}
  The operation of negation \(x \to -x\) on \(\R\), we now extend to \(\R^*\) by defining \(-(+\infty) \coloneqq -\infty\) and \(-(-\infty) \coloneqq +\infty\).
\end{defn}

\begin{note}
  Thus, every extended real number \(x\) has a negation, and \(-(-x)\) is always equal to \(x\).
\end{note}

\begin{defn}[Ordering of extended reals]\label{i:6.2.3}
  Let \(x\) and \(y\) be extended real numbers.
  We say that \(x \leq y\), i.e., \(x\) is less than or equal to \(y\), iff one of the following three statements is true:
  \begin{enumerate}
    \item \(x\) and \(y\) are real numbers, and \(x \leq y\) as real numbers.
    \item \(y = +\infty\).
    \item \(x = -\infty\).
  \end{enumerate}
  We say that \(x < y\) if we have \(x \leq y\) and \(x \neq y\).
  We sometimes write \(x < y\) as \(y > x\), and \(x \leq y\) as \(y \geq x\).
\end{defn}

\setcounter{thm}{4}
\begin{prop}\label{i:6.2.5}
  Let \(x, y, z\) be extended real numbers.
  Then the following statements are true:
  \begin{enumerate}
    \item (Reflexivity)
          We have \(x \leq x\).
    \item (Trichotomy)
          Exactly one of the statements \(x < y\), \(x = y\), or \(x > y\) is true.
    \item (Transitivity)
          If \(x \leq y\) and \(y \leq z\), then \(x \leq z\).
    \item (Negation reverses order) If \(x \leq y\), then \(-y \leq -x\).
  \end{enumerate}
\end{prop}

\begin{proof}{(a)}
  By \cref{i:5.4.7} we already have \(x \leq x\) when \(x \in \R\).
  So we only need to consider the cases \(x \in \set{+\infty, -\infty}\).
  By \cref{i:6.2.3} we have \(x \leq +\infty\) for every \(x \in \R^*\).
  So we have \(+\infty \leq +\infty\).
  Again by \cref{i:6.2.3} we have \(-\infty \leq x\) for every \(x \in \R^*\).
  So we have \(-\infty \leq -\infty\).
  Thus, we conclude that \(x \leq x\) for every \(x \in \R^*\).
\end{proof}

\begin{proof}{(b)}
  By \cref{i:5.4.7}, we already have exactly one of the statements \(x < y\), \(x = y\), or \(x > y\) is true when \(x, y \in \R\).
  So we only need to consider the cases \(x, y \in \set{+\infty, -\infty}\).
  \begin{itemize}
    \item If \(x = +\infty\), then by \cref{i:6.2.3} we have \(x \geq y\) for every \(y \in \R^*\).
          \begin{itemize}
            \item If \(y = +\infty\), then we have \(x = y\).
            \item If \(y \in \R\), then by \cref{i:6.2.1} \(x \neq y\).
                  Thus, by \cref{i:6.2.3} we have \(x > y\).
            \item If \(y = -\infty\), then by \cref{i:6.2.1} \(x \neq y\).
                  Thus, by \cref{i:6.2.3} we have \(x > y\).
          \end{itemize}
    \item If \(x = -\infty\), then by \cref{i:6.2.3} we have \(x \leq y\) for every \(y \in \R^*\).
          \begin{itemize}
            \item If \(y = +\infty\), then by \cref{i:6.2.1} \(x \neq y\).
                  Thus, by \cref{i:6.2.3} we have \(x < y\).
            \item If \(y \in \R\), then by \cref{i:6.2.1} \(x \neq y\).
                  Thus, by \cref{i:6.2.3} we have \(x < y\).
            \item If \(y = -\infty\), then we have \(x = y\).
          \end{itemize}
    \item If \(y = +\infty\), then by \cref{i:6.2.3} we have \(x \leq y\) for every \(x \in \R^*\).
          \begin{itemize}
            \item If \(x = +\infty\), then we have \(x = y\).
            \item If \(x \in \R\), then by \cref{i:6.2.1} \(x \neq y\).
                  Thus, by \cref{i:6.2.3} we have \(x < y\).
            \item If \(x = -\infty\), then by \cref{i:6.2.1} \(x \neq y\).
                  Thus, by \cref{i:6.2.3} we have \(x < y\).
          \end{itemize}
    \item If \(y = -\infty\), then by \cref{i:6.2.3} we have \(x \geq y\) for every \(x \in \R^*\).
          \begin{itemize}
            \item If \(x = +\infty\), then by \cref{i:6.2.1} \(x \neq y\).
                  Thus, by \cref{i:6.2.3} we have \(x > y\).
            \item If \(x \in \R\), then by \cref{i:6.2.1} \(x \neq y\).
                  Thus, by \cref{i:6.2.3} we have \(x > y\).
            \item If \(x = -\infty\), then we have \(x = y\).
          \end{itemize}
  \end{itemize}
  From all cases above we conclude that exactly one of the statements \(x < y\), \(x = y\), or \(x > y\) is true.
\end{proof}

\begin{proof}{(c)}
  By \cref{i:5.4.7}, we already have \((x \leq y) \land (y \leq z) \implies x \leq z\) when \(x, y, z \in \R\).
  So we only need to consider the cases \(x, y, z \in \set{+\infty, -\infty}\).
  \begin{itemize}
    \item If \(x = +\infty\), then by \cref{i:6.2.3} \((x = +\infty) \land (x \leq y) \implies y = +\infty\).
          Similarly \((y = +\infty) \land (y \leq z) \implies z = +\infty\).
          Thus, we have \(x = +\infty = z\), and by \cref{i:6.2.5}(a) we have \(x \leq z\).
    \item If \(x = -\infty\), then by \cref{i:6.2.3} \(x \leq z \) for every \(z \in \R^*\).
    \item If \(y = +\infty\), then by \cref{i:6.2.3} \((y = +\infty) \land (y \leq z) \implies z = +\infty\).
          Again by \cref{i:6.2.3} we have \(x \leq +\infty = z\) for every \(x \in \R^*\).
    \item If \(y = -\infty\), then by \cref{i:6.2.3} \((y = -\infty) \land (x \leq y) \implies x = -\infty\).
          Again by \cref{i:6.2.3} we have \(x = -\infty \leq z\) for every \(z \in \R^*\).
    \item If \(z = +\infty\), then by \cref{i:6.2.3}, we have \(x \leq +\infty = z\) for every \(x \in \R^*\).
    \item If \(z = -\infty\), then by \cref{i:6.2.3} \((z = -\infty) \land (y \leq z) \implies y = -\infty\).
          Similarly \((y = -\infty) \land (x \leq y) \implies x = -\infty\).
          Thus, we have \(x = -\infty = z\), and by \cref{i:6.2.5}(a) we have \(x \leq z\).
  \end{itemize}
  From all cases above we conclude that \((x \leq y) \land (y \leq z) \implies x \leq z\).
\end{proof}

\begin{proof}{(d)}
  By \cref{i:5.4.7} we already have \(x \leq y \implies -y \leq -x\) for every \(x, y \in \R\).
  So we only need to consider the cases \(x, y \in \set{+\infty, -\infty}\).
  \begin{itemize}
    \item If \(x = +\infty\), then by \cref{i:6.2.3} \((x = +\infty) \land (x \leq y) \implies y = +\infty\).
          And by \cref{i:6.2.2} we have \(-x = -\infty = -y\).
          Thus, by \cref{i:6.2.5}(a) we have \(-y \leq -x\).
    \item If \(x = -\infty\), then by \cref{i:6.2.2} \(-x = +\infty\) and by \cref{i:6.2.3} we have \(-y \leq -x\) for every \(-y \in \R^*\).
    \item If \(y = +\infty\), then by \cref{i:6.2.2} \(-y = -\infty\) and by \cref{i:6.2.3} we have \(-y \leq -x\) for every \(-x \in \R^*\).
    \item If \(y = -\infty\), then by \cref{i:6.2.3} \((y = -\infty) \land (x \leq y) \implies x = -\infty\).
          And by \cref{i:6.2.2} we have \(-x = +\infty = -y\).
          Thus, by \cref{i:6.2.5}(a) we have \(-y \leq -x\).
  \end{itemize}
  From all cases above we conclude that \(x \leq y \implies -y \leq -x\).
\end{proof}

\begin{note}
  One could also introduce other operations on the extended real number system, such as addition, multiplication, etc.
  However, this is somewhat dangerous as these operations will almost certainly fail to obey the familiar rules of algebra.
  For instance, to define addition it seems reasonable (given one's intuitive notion of infinity) to set \(+\infty + 5 = +\infty\) and \(+\infty + 3 = +\infty\), but then this implies that \(+\infty + 5 = +\infty + 3\), while \(5 \neq 3\).
  So things like the cancellation law begin to break down once we try to operate involving infinity.
  To avoid these issues we shall simply not define any arithmetic operations on the extended real number system other than negation and order.
\end{note}

\begin{defn}[Supremum of sets of extended reals]\label{i:6.2.6}
  Let \(E\) be a subset of \(\R^*\).
  Then we define the \emph{supremum} \(\sup(E)\) or \emph{least upper bound} of \(E\) by the following rule.
  \begin{enumerate}
    \item If \(E\) is contained in \(\R\) (i.e., \(+\infty\) and \(-\infty\) are not elements of \(E\)), then we let \(\sup(E)\) be as defined in \cref{i:5.5.10}.
    \item If \(E\) contains \(+\infty\), then we set \(\sup(E) \coloneqq +\infty\).
    \item If \(E\) does not contain \(+\infty\) but does contain \(-\infty\), then we set \(\sup(E) \coloneqq \sup(E \setminus \set{-\infty})\)
          (which is a subset of \(\R\) and thus falls under case (a)).
  \end{enumerate}
  We also define the \emph{infimum} \(\inf(E)\) of \(E\) (also known as the \emph{greatest lower bound} of \(E\)) by the formula
  \[
    \inf(E) \coloneqq -\sup(-E)
  \]
  where \(-E\) is the set \(-E \coloneqq \set{-x : x \in E}\).
\end{defn}

\setcounter{thm}{9}
\begin{eg}\label{i:6.2.10}
  Let \(E\) be the empty set.
  Then \(\sup(E) = -\infty\) and \(\inf(E) = +\infty\).
  This is the only case in which the supremum can be less than the infimum.
\end{eg}

\begin{proof}
  Since \(+\infty \notin \emptyset\) and \(-\infty \notin \emptyset\), by \cref{i:6.2.6} we know that \(\sup(\emptyset) = -\infty\).
  Since \(-\emptyset\) is also empty, by \cref{i:6.2.6} we know that \(\sup(-\emptyset) = -\infty\), thus by \cref{i:6.2.6,i:6.2.2} we have \(\inf(\emptyset) = -\sup(-\emptyset) = -(-\infty) = +\infty\).

  Now we show that the only case in which the supremum can be less than the infimum is when \(E = \emptyset\).
  Suppose for sake of contradiction that there is a set \(E\) such that \(E \neq \emptyset\) and \(\sup(E) < \inf(E)\).
  Since \(E \neq \emptyset\), let \(x \in E\).
  Then we have \(\sup(E) < \inf(E) \leq x \leq \sup(E)\), a contradiction.
  Thus, \(E = \emptyset\).
\end{proof}

\begin{note}
  One can intuitively think of the supremum of \(E\) as follows.
  Imagine the real line with \(+\infty\) somehow on the far right, and \(-\infty\) on the far left.
  Imagine a piston at \(+\infty\) moving leftward until it is stopped by the presence of a set \(E\);
  the location where it stops is the supremum of \(E\).
  Similarly if one imagines a piston at \(-\infty\) moving rightward until it is stopped by the presence of \(E\), the location where it stops is the infimum of \(E\).
  In the case when \(E\) is the empty set, the pistons pass through each other, the supremum landing at \(-\infty\) and the infimum landing at \(+\infty\).
\end{note}

\begin{thm}\label{i:6.2.11}
  Let \(E\) be a subset of \(\R^*\).
  Then the following statements are true.
  \begin{enumerate}
    \item For every \(x \in E\) we have \(x \leq \sup(E)\) and \(x \geq \inf(E)\).
    \item Suppose that \(M \in \R^*\) is an upper bound for \(E\), i.e., \(x \leq M\) for all \(x \in E\).
          Then we have \(\sup(E) \leq M\).
    \item Suppose that \(M \in \R^*\) is a lower bound for \(E\), i.e., \(x \geq M\) for all \(x \in E\).
          Then we have \(\inf(E) \geq M\).
  \end{enumerate}
\end{thm}

\begin{proof}{(a)}
  We first show that \(x \leq \sup(E)\) for every \(x \in E\).
  First suppose that \(E = \emptyset\).
  Then the statement ``\(x \leq \sup(\emptyset)\) for every \(x \in \emptyset\)'' is vacuously true.
  Now suppose that \(E \neq \emptyset\).
  We split into two cases:
  \begin{itemize}
    \item If \(+\infty \not\in E\), then we can further split into two cases:
          \begin{itemize}
            \item If \(-\infty \in E\), then by \cref{i:6.2.6} we know that \(\sup(E) = \sup(E \setminus \set{-\infty})\).
                  Let \(E' = E \setminus \set{-\infty}\).
                  Since \(E' \subseteq \R\), by \cref{i:5.5.9} we know that \(x \leq \sup(E)\) for every \(x \in E'\).
                  By \cref{i:6.2.3} we know that \(-\infty \leq \sup(E)\), thus we have \(x \leq \sup(E)\) for every \(x \in E\).
            \item If \(-\infty \notin E\), then \(E \subseteq \R\), thus by \cref{i:5.5.9} we know that \(x \leq \sup(E)\) for every \(x \in E\).
          \end{itemize}
    \item If \(+\infty \in E\), then by \cref{i:6.2.6} we have \(\sup(E) = +\infty\), and by \cref{i:6.2.3} we have \(x \leq \sup(E)\) for every \(x \in E\).
  \end{itemize}
  From all cases above we conclude that \(x \leq \sup(E)\) for every \(x \in E\).

  Now we show that \(x \geq \inf(E)\) for every \(x \in E\).
  First suppose that \(E = \emptyset\).
  Then the statement ``\(x \geq \inf(E)\) for every \(x \in E\)'' is vacuously true.
  Now suppose that \(E \neq \emptyset\).
  From the proof above we know that \(x \leq \sup(E)\) for every \(x \in E\).
  Then we have
  \begin{align*}
             & x \leq \sup(E)                           &  & \by{i:5.5.9} \\
    \implies & -x \geq -\sup(E)                                           \\
    \implies & \sup(-E) \geq -x \geq -\sup(E)           &  & \by{i:5.5.9} \\
    \implies & \inf(E) = -\sup(-E) \leq x \leq \sup(E). &  & \by{i:6.2.6}
  \end{align*}
  Thus, we conclude that \(x \geq \inf(E)\) for every \(x \in E\).
\end{proof}

\begin{proof}{(b)}
  First suppose that \(E = \emptyset\).
  Then by \cref{i:6.2.10} and \cref{i:6.2.3} we have \(\sup(E) = -\infty \leq M\).
  Now suppose that \(E \neq \emptyset\).
  We split into two cases:
  \begin{itemize}
    \item If \(+\infty \not\in E\), then we can further split into two cases:
          \begin{itemize}
            \item If \(-\infty \in E\), then by \cref{i:6.2.6} we know that \(\sup(E) = \sup(E \setminus \set{-\infty})\).
                  Let \(E' = E \setminus \set{-\infty}\).
                  Then by \cref{i:5.5.9} we know that \(\sup(E) \leq M\).
            \item If \(-\infty \notin E\), then \(E \subseteq \R\), thus by \cref{i:5.5.9} we know that \(\sup(E) \leq M\).
          \end{itemize}
    \item If \(+\infty \in E\), then by hypothesis we have \(+\infty \leq M\), and by \cref{i:6.2.3} we have \(M = +\infty\).
          Again by \cref{i:6.2.3} we have \(\sup(E) \leq M\).
  \end{itemize}
  From all cases above we conclude that \(\sup(E) \leq M\).
\end{proof}

\begin{proof}{(c)}
  We have
  \begin{align*}
             & \forall x \in E, x \geq M                       \\
    \implies & -x \leq -M                                      \\
    \implies & \sup(-E) \leq -M          &  & \by{i:6.2.11}[b] \\
    \implies & -\sup(-E) \geq M                                \\
    \implies & \inf(E) \geq M.           &  & \by{i:6.2.6}
  \end{align*}
\end{proof}

\exercisesection

\begin{ex}\label{i:ex:6.2.1}
  Prove \cref{i:6.2.5}.
\end{ex}

\begin{proof}
  See \cref{i:6.2.5}.
\end{proof}

\begin{ex}\label{i:ex:6.2.2}
  Prove \cref{i:6.2.11}.
\end{ex}

\begin{proof}
  See \cref{i:6.2.11}.
\end{proof}
