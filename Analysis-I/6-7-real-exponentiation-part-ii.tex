\section{Real exponentiation, part II}\label{sec:6.7}

\begin{lem}[Continuity of exponentiation]\label{6.7.1}
  Let \(x > 0\), and let \(\alpha\) be a real number.
  Let \((q_n)_{n = 1}^\infty\) be any sequence of rational numbers converging to \(\alpha\).
  Then \((x^{q_n})_{n = 1}^\infty\) is also a convergent sequence.
  Furthermore, if \((q_n')_{n = 1}^\infty\) is any other sequence of rational numbers converging to \(\alpha\), then \((x^{q_n'})_{n = 1}^\infty\) has the same limit as \((x^{q_n})_{n = 1}^\infty\):
  \[
    \lim_{n \to \infty} x^{q_n} = \lim_{n \to \infty} x^{q_n'}.
  \]
\end{lem}

\begin{proof}
  There are three cases: \(x < 1\), \(x = 1\), and \(x > 1\).
  The case \(x = 1\) is rather easy (because then \(x^q = 1\) for all rational \(q\)).

  We first show that if \(x > 1\) then \((x^{q_n})_{n = 1}^\infty\) converges.
  By \cref{6.4.18} it is enough to show that \((x^{q_n})_{n = 1}^\infty\) is a Cauchy sequence.

  To do this, we need to estimate the distance between \(x^{q_n}\) and \(x^{q_m}\);
  let us say for the time being that \(q_n \geq q_m\), so that \(x^{q_n} \geq x^{q_m}\) (since \(x > 1\)).
  We have
  \[
    d(x^{q_n}, x^{q_m}) = x^{q_n} - x^{q_m} = x^{q_m} (x^{q_n - q_m} - 1).
  \]
  Since \((q_n)_{n = 1}^\infty\) is a convergent sequence, it has some upper bound \(M\);
  since \(x > 1\), we have \(x^{q_m} \leq x^M\).
  Thus
  \[
    d(x^{q_n}, x^{q_m}) = \abs{x^{q_n} - x^{q_m}} \leq x^M (x^{q_n - q_m} - 1).
  \]
  Now let \(\varepsilon > 0\).
  We know by \cref{6.5.3} that the sequence \((x^{1 / k})_{k = 1}^\infty\) is eventually \(\varepsilon x^{-M}\)-close to \(1\).
  Thus there exists some \(K \geq 1\) such that
  \[
    \abs{x^{1 / K} - 1} \leq \varepsilon x^{-M}.
  \]
  Now since \((q_n)_{n = 1}^\infty\) is convergent, it is a Cauchy sequence, and so there is an \(N \geq 1\) such that \(q_n\) and \(q_m\) are \(1 / K\)-close for all \(n, m \geq N\).
  Thus we have
  \[
    d(x^{q_n}, x^{q_m}) \leq x^M (x^{q_n - q_m} - 1) \leq x^M (x^{1 / K} - 1) \leq x^M \varepsilon x^{-M} = \varepsilon.
  \]
  for every \(n, m \geq N\) such that \(q_n \geq q_m\).
  By symmetry we also have this bound when \(n, m \geq N\) and \(q_n \leq q_m\).
  Thus the sequence \((x^{q_n})_{n = 1}^\infty\) is \(\varepsilon\)-steady.
  Thus the sequence \((x^{q_n})_{n = 1}^\infty\) is eventually \(\varepsilon\)-steady for every \(\varepsilon > 0\), and is thus a Cauchy sequence as desired.
  This proves the convergence of \((x^{q_n})_{n = 1}^\infty\) when \(x > 1\).

  Next we show that if \(x < 1\) then \((x^{q_n})_{n = 1}^\infty\) also converges.
  By \cref{6.4.18} it is enough to show that \((x^{q_n})_{n = 1}^\infty\) is a Cauchy sequence.

  To do this, we need to estimate the distance between \(x^{q_n}\) and \(x^{q_m}\);
  let us say for the time being that \(q_n \leq q_m\), so that \(x^{q_n} \geq x^{q_m}\) (since \(x < 1\)).
  We have
  \[
    d(x^{q_n}, x^{q_m}) = x^{q_n} - x^{q_m} = x^{q_m} (x^{q_n - q_m} - 1).
  \]
  Since \((q_n)_{n = 1}^\infty\) is a convergent sequence, it has some lower bound \(M\);
  since \(x < 1\), we have \(x^{q_m} \leq x^M\).
  Thus
  \[
    d(x^{q_n}, x^{q_m}) = \abs{x^{q_n} - x^{q_m}} \leq x^M (x^{q_n - q_m} - 1).
  \]
  Now let \(\varepsilon > 0\).
  We know by \cref{6.5.3} that the sequence \((x^{1 / k})_{k = 1}^\infty\) is eventually \(\varepsilon x^{-M}\)-close to \(1\).
  Thus there exists some \(K \geq 1\) such that
  \[
    \abs{x^{1 / K} - 1} \leq \varepsilon x^{-M}.
  \]
  Now since \((q_n)_{n = 1}^\infty\) is convergent, it is a Cauchy sequence, and so there is an \(N \geq 1\) such that \(q_n\) and \(q_m\) are \(1 / K\)-close for all \(n, m \geq N\).
  Thus we have
  \[
    d(x^{q_n}, x^{q_m}) \leq x^M (x^{q_n - q_m} - 1) \leq x^M (x^{1 / K} - 1) \leq x^M \varepsilon x^{-M} = \varepsilon.
  \]
  for every \(n, m \geq N\) such that \(q_n \leq q_m\).
  By symmetry we also have this bound when \(n, m \geq N\) and \(q_n \geq q_m\).
  Thus the sequence \((x^{q_n})_{n = 1}^\infty\) is \(\varepsilon\)-steady.
  Thus the sequence \((x^{q_n})_{n = 1}^\infty\) is eventually \(\varepsilon\)-steady for every \(\varepsilon > 0\), and is thus a Cauchy sequence as desired.
  This proves the convergence of \((x^{q_n})_{n = 1}^\infty\) when \(x < 1\).

  Now we prove the second claim.
  It will suffice to show that
  \[
    \lim_{n \to \infty} x^{q_n - q_n'} = 1,
  \]
  since the claim would then follow from limit laws
  (since \(x^{q_n} = x^{q_n - q_n'} x^{q_n'}\)).

  Write \(r_n \coloneqq q_n - q_n'\);
  by limit laws we know that \((r_n)_{n = 1}^\infty\) converges to \(0\).
  We have to show that for every \(\varepsilon > 0\), the sequence \((x^{r_n})_{n = 1}^\infty\) is eventually \(\varepsilon\)-close to \(1\).
  But from \cref{6.5.3} we know that the sequence \((x^{1 / k})_{k = 1}^\infty\) is eventually \(\varepsilon\)-close to \(1\).
  Since \(\lim_{k \to \infty} x^{-1 / k}\) is also equal to \(1\) by \cref{6.5.3}, we know that \((x^{-1 / k})_{k = 1}^\infty\) is also eventually \(\varepsilon\)-close to \(1\).
  Thus we can find a \(K\) such that \(x^{1 / K}\) and \(x^{-1 / K}\) are both \(\varepsilon\)-close to \(1\).
  But since \((r_n)_{n = 1}^\infty\) is convergent to \(0\), it is eventually \(1 / K\)-close to \(0\), so that eventually \(-1 / K \leq r_n \leq 1 / K\), and thus when \(x > 1\) we have \(x^{-1 / K} \leq x^{r_n} \leq x^{1 / K}\), when \(x < 1\) we have \(x^{1 / K} \leq x^{r_n} \leq x^{-1 / K}\).
  In particular \(x^{r_n}\) is also eventually \(\varepsilon\)-close to \(1\) (see \cref{4.3.7}(f)), as desired.
\end{proof}

\begin{defn}[Exponentiation to a real exponent]\label{6.7.2}
  Let \(x > 0\) be real, and let \(\alpha\) be a real number.
  We define the quantity \(x^\alpha\) by the formula \(x^\alpha = \lim_{n \to \infty} x^{q_n}\), where \((q_n)_{n = 1}^\infty\) is any sequence of rational numbers converging to \(\alpha\).
\end{defn}

\begin{note}
  Let us check that \cref{6.7.2} is well-defined.
  First of all, given any real number \(\alpha\) we always have at least one sequence \((q_n)_{n = 1}^\infty\) of rational numbers converging to \(\alpha\), by the definition of real numbers (and \cref{6.1.15}).
  Secondly, given any such sequence \((q_n)_{n = 1}^\infty\), the limit \(\lim_{n \to \infty} x^{q_n}\) exists by \cref{6.7.1}.
  Finally, even though there can be multiple choices for the sequence \((q_n)_{n = 1}^\infty\), they all give the same limit by \cref{6.7.1}.
  Thus \cref{6.7.2} is well-defined.
\end{note}

\begin{note}
  If \(\alpha\) is not just real but rational, i.e., \(\alpha = q\) for some rational \(q\), then \cref{6.7.2} could in principle be inconsistent with our earlier definition of exponentiation in \cref{sec:5.6}.
  But in this case \(\alpha\) is clearly the limit of the sequence \((q)_{n = 1}^\infty\), so by definition \(x^\alpha = \lim_{n \to \infty} x^q = x^q\).
  Thus the new definition of exponentiation is consistent with the old one.
\end{note}

\begin{prop}\label{6.7.3}
  All the results of \cref{5.6.9}, which held for rational numbers \(q\) and \(r\), continue to hold for real numbers \(q\) and \(r\).
\end{prop}

\begin{proof}{(a)}
  Let \(r\) be a real number.
  Then we can write \(r = \lim_{n \to \infty} r_n\) for some sequences \((r_n)_{n = 1}^\infty\) of rationals, by the definition of real numbers (and \cref{6.1.15}).
  Since \((r_n)_{n = 1}^\infty\) is a Cauchy sequence, it is bounded by some \(M \in \Q^+\), i.e, \(-M \leq r_n \leq M\) for every \(n \geq 1\).
  By \cref{5.6.9}, both \(x^M\) and \(x^{-M}\) are positive real numbers.
  If \(0 < x < 1\), then \(x^M \leq x^{r_n} \leq x^{-M}\).
  If \(x \geq 1\), then \(x^{-M} \leq x^{r_n} \leq x^M\).
  By \cref{6.1.19}(h) we have
  \begin{align*}
     & \lim_{n \to \infty} \min(x^{-M}, x^M, x^{r_n})                                           \\
     & = \min(\lim_{n \to \infty} x^{-M}, \lim_{n \to \infty} x^M, \lim_{n \to \infty} x^{r_n}) \\
     & = \min(x^M, x^{-M}).
  \end{align*}
  Since \(\min(x^M, x^{-M})\) is positive real number, we know that \(x^r\) must also be a positive real number.
\end{proof}

\begin{proof}{(b)}
  Let \(q\) and \(r\) be real numbers.
  Then we can write \(q = \lim_{n \to \infty} q_n\) and \(r = \lim_{n \to \infty} r_n\) for some sequences \((q_n)_{n = 1}^\infty\) and \((r_n)_{n = 1}^\infty\) of rationals, by the definition of real numbers (and \cref{6.1.15}).
  Then by the limit laws, \(q + r\) is the limit of \((q_n + r_n)_{n = 1}^\infty\).
  By definition of real exponentiation, we have
  \[
    x^{q + r} = \lim_{n \to \infty} x^{q_n + r_n} ; x^q = \lim_{n \to \infty} x^{q_n} ;  x^r = \lim_{n \to \infty} x^{r_n}.
  \]
  But by \cref{5.6.9}(b) (applied to \emph{rational} exponents) we have \(x^{q_n + r_n} = x^{q_n} x^{r_n}\).
  Thus by limit laws we have \(x^{q + r} = x^q x^r\), as desired.

  Now we show that \((x^q)^r = \lim_{n \to \infty} (x^{q_n})^{r_n}\).
  By \cref{6.1.19}(c) we know that \(q r_n = \lim_{m \to \infty} q_m r_n\).
  Thus we have
  \begin{align*}
    (x^q)^r & = \lim_{n \to \infty} (x^q)^{r_n}                         &  & \by{6.7.2}                  \\
            & = \lim_{n \to \infty} (\lim_{m \to \infty} x^{q_m})^{r_n} &  & \by{6.7.2}                  \\
            & = \lim_{n \to \infty} (\lim_{m \to \infty} x^{q_m r_n})   &  & \text{(by \cref{5.6.9}(b))} \\
            & = \lim_{n \to \infty} x^{q r_n}                           &  & \by{6.7.2}                  \\
            & = x^{qr}.                                                 &  & \by{6.7.2}
  \end{align*}
\end{proof}

\begin{proof}{(c)}
  Let \(r \in \R\) where \(r = \lim_{n \to \infty} r_n\) for some sequences \((r_n)_{n = 1}^\infty\) of rationals.
  By \cref{6.1.15} \(r\) is well-defined and by \cref{6.1.19}(c) we know that \(-r\) is the limit of \((-r_n)_{n = 1}^\infty\).
  By \cref{6.7.3}(a) we know that \(x^{-r} > 0\) and by \cref{5.6.9}(a) we know that \(x^{r_n} > 0\) for every \(n \in \Z^+\).
  Thus we have
  \begin{align*}
    x^{-r} & = \lim_{n \to \infty} x^{-r_n}    &  & \by{6.7.2}                   \\
           & = 1 / \lim_{n \to \infty} x^{r_n} &  & \text{(by \cref{6.1.19}(e))} \\
           & = 1 / x^r.                        &  & \by{6.7.2}
  \end{align*}
\end{proof}

\begin{proof}{(d)}
  Let \(x, y, r \in \R^+\) where \(r = \lim_{n \to \infty} r_n\) for some sequences \((r_n)_{n = 1}^\infty\) of rationals.
  By \cref{6.1.15} \(r\) is well-defined.
  Since \(r \in \R^+\), by \cref{5.4.3} we know that \(\exists\ c \in \R^+\) such that \(r_n \geq c\) for every \(n \in \Z^+\).

  We first show that \(x > y \implies x^r > y^r\).
  \begin{align*}
             & x > y                                                                                         \\
    \implies & \forall n \in \Z^+, x^{r_n} > y^{r_n}                        &  & \text{(by \cref{5.6.9}(d))} \\
    \implies & \lim_{n \to \infty} x^{r_n} \geq \lim_{n \to \infty} y^{r_n} &  & \text{(by \cref{6.4.13})}   \\
    \implies & x^r \geq y^r.                                                &  & \by{6.7.2}
  \end{align*}
  Now we show that \(x^r \neq y^r\).
  Suppose for sake of contradiction that \(x^r = y^r\).
  Then we have
  \begin{align*}
             & x^r = y^r                                                                                                                       \\
    \implies & \lim_{n \to \infty} x^{r_n} = \lim_{n \to \infty} y^{r_n}                                                       &  & \by{6.7.2} \\
    \implies & \forall \varepsilon \in \R^+, \exists\ N \in \Z^+ : \forall n \geq N, \abs{x^{r_n} - y^{r_n}} \leq \varepsilon. &  & \by{5.3.1}
  \end{align*}
  But by \cref{5.6.9}(d) we know that \(x > y \implies x^{r_n} > y^{r_n}\), thus by setting \(\varepsilon = (x^{r_n} - y^{r_n}) / 2\) we get
  \[
    \abs{x^{r_n} - y^{r_n}} = x^{r_n} - y^{r_n} \leq \dfrac{x^{r_n} - y^{r_n}}{2},
  \]
  a contradiction.
  Thus we must have \(x^r > y^r\).

  Finally we show that \(x^r > y^r \implies x > y\).
  Suppose for sake of contradiction that \(x \leq y\).
  Then from the proof above we know that \(x^r \leq y^r\), a contradiction.
  Thus we must have \(x > y\).
  We conclude that if \(r > 0\), then \(x > y \iff x^r > y^r\).
\end{proof}

\begin{proof}{(e)}
  Let \(x, q, r \in \R^+\) where \(q = \lim_{n \to \infty} q_n\) and \(r = \lim_{n \to \infty} r_n\) for some sequences \((q_n)_{n = 1}^\infty\), \((r_n)_{n = 1}^\infty\) of rationals.
  By \cref{6.1.15} \(q, r\) are well-defined.
  Since \(q, r \in \R^+\), by \cref{5.4.3} we know that \(\exists\ c_1, c_2 \in \R^+\) such that \(q_n \geq c_1\) and \(r_n \geq c_2\) for every \(n \in \Z^+\).
  Thus by \cref{5.6.9}(a) we know that \(x^{q_n}, x^{r_n} > 0\) for every \(n \in \Z^+\).

  We first show that if \(x > 1\), then \(x^q > x^r \implies q > r\).
  We have
  \begin{align*}
             & x^q > x^r > 0                          &  & \text{(by \cref{6.7.3}(a))}    \\
    \implies & x^{q - r} > 1                          &  & \text{(by \cref{6.7.3}(b)(c))} \\
    \implies & \lim_{n \to \infty} x^{q_n - r_n} > 1. &  & \by{6.7.2}                     \\
  \end{align*}
  Now we show that \(\exists\ N \in \Z^+\) such that \(q_n - r_n > 0\) for every \(n \geq N\).
  Suppose for sake of contradiction that \(\forall N \in \Z^+\), \(\exists\ n \geq N\) such that \(q_n - r_n \leq 0\).
  Then we have
  \begin{align*}
             & q_n - r_n \leq 0                                                           \\
    \implies & r_n - q_n \geq 0                                                           \\
    \implies & x^{r_n - q_n} \geq 1^{r_n - q_n} = 1      &  & \text{(by \cref{6.7.3}(d))} \\
    \implies & x^{q_n - r_n} \leq 1                      &  & \text{(by \cref{6.7.3}(c))} \\
    \implies & \lim_{n \to \infty} x^{q_n - r_n} \leq 1, &  & \text{(by \cref{6.4.13})}
  \end{align*}
  which contradict to \(\lim_{n \to \infty} x^{q_n - r_n} > 1\).
  Thus \(\exists\ N \in \Z^+\) such that \(q_n - r_n > 0\) for every \(n \geq N\).
  This means \(q_n > r_n\) for every \(n \geq N\), and by \cref{6.4.13} we know that \(q = \lim_{n \to \infty} q_n \geq \lim_{n \to \infty} r_n = r\).

  Next we show that if \(x > 1\), then \(q > r \implies x^q > x^r\).
  \begin{align*}
             & q > r                                                                                                      \\
    \implies & q - r > 0                                                                                                  \\
    \implies & x^{q - r} > 1^{q - r}                                                     &  & \text{(by \cref{6.7.3}(d))} \\
    \implies & x^{q - r} > \lim_{n \to \infty} 1^{q_n - r_n} = \lim_{n \to \infty} 1 = 1 &  & \by{6.7.2}                  \\
    \implies & x^{q - r} x^r > x^r                                                       &  & \text{(by \cref{6.7.3}(a))} \\
    \implies & x^{q - r + r} > x^r                                                       &  & \text{(by \cref{6.7.3}(b))} \\
    \implies & x^q > x^r.                                                                &  & \by{5.6.8}
  \end{align*}
  Thus we conclude that if \(x > 1\), then \(x^q > x^r \iff q > r\).

  Finally we show that if \(x < 1\), then \(x^q > x^r \iff q < r\).
  \begin{align*}
             & x < 1                                                                            \\
    \implies & x^{-1} > 1                                                                       \\
    \implies & \big((x^{-1})^q < (x^{-1})^r \iff q < r\big) &  & \text{(from proof above)}      \\
    \implies & (x^{-q} < x^{-r} \iff q < r)                 &  & \text{(by \cref{6.7.3}(b))}    \\
    \implies & (x^q > x^r \iff q < r).                      &  & \text{(by \cref{6.7.3}(a)(c))}
  \end{align*}
\end{proof}

\begin{proof}{(f)}
  Let \(x, y, r \in \R^+\) where \(r = \lim_{n \to \infty} r_n\) for some sequences \((r_n)_{n = 1}^\infty\) of rationals.
  By \cref{6.1.15} \(r\) is well-defined.
  Then we have
  \begin{align*}
    (xy)^r & = \lim_{n \to \infty} (xy)^{r_n}                             &  & \by{6.7.2}                   \\
           & = \lim_{n \to \infty} x^{r_n} y^{r_n}                        &  & \text{(by \cref{5.6.9}(f))}  \\
           & = (\lim_{n \to \infty} x^{r_n})(\lim_{n \to \infty} y^{r_n}) &  & \text{(by \cref{6.1.19}(b))} \\
           & = x^r y^r.                                                   &  & \by{6.7.2}
  \end{align*}
\end{proof}

\exercisesection

\begin{ex}\label{ex:6.7.1}
  Prove the remaining components of \cref{6.7.3}.
\end{ex}

\begin{proof}
  See \cref{6.7.3}.
\end{proof}