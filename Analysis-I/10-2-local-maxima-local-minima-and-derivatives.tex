\section{Local maxima, local minima, and derivatives}\label{i:sec:10.2}

\begin{defn}[Local maxima and minima]\label{i:10.2.1}
  Let \(X\) be a subset of \(\R\), and let \(f : X \to \R\) be a function, and let \(x_0 \in X\).
  We say that \(f\) attains a \emph{local maximum} at \(x_0\) iff there exists a \(\delta > 0\) such that the restriction \(f|_{X \cap (x_0 - \delta, x_0 + \delta)}\) of \(f\) to \(X \cap (x_0 - \delta, x_0 + \delta)\) attains a maximum at \(x_0\).
  We say that \(f\) attains a \emph{local minimum} at \(x_0\) iff there exists a \(\delta > 0\) such that the restriction \(f|_{X \cap (x_0 - \delta, x_0 + \delta)}\) of \(f\) to \(X \cap (x_0 - \delta, x_0 + \delta)\) attains a minimum at \(x_0\).
\end{defn}

\begin{rmk}\label{i:10.2.2}
  If \(f\) attains a maximum at \(x_0\), we sometimes say that \(f\) attains a \emph{global} maximum at \(x_0\), in order to distinguish it from the local maxima defined in \cref{i:10.2.1}.
  Note that if \(f\) attains a global maximum at \(x_0\), then it certainly also attains a local maximum at this \(x_0\), and similarly for minima.
\end{rmk}

\setcounter{thm}{4}
\begin{rmk}\label{i:10.2.5}
  If \(f : X \to \R\) attains a local maximum at a point \(x_0\) in \(X\), and \(Y \subseteq X\) is a subset of \(X\) which contains \(x_0\), then the restriction \(f|_Y : Y \to \R\) also attains a local maximum at \(x_0\).
  Similarly, for minima.
\end{rmk}

\begin{prop}[Local extrema are stationary]\label{i:10.2.6}
  Let \(a < b\) be real numbers, and let \(f : (a, b) \to \R\) be a function.
  If \(x_0 \in (a, b)\), \(f\) is differentiable at \(x_0\), and \(f\) attains either a local maximum or local minimum at \(x_0\), then \(f'(x_0) = 0\).
\end{prop}

\begin{proof}
  Suppose \(f\) attains local maximum at \(x_0\).
  Then by \cref{i:10.2.1} we know that
  \[
    \exists \delta \in \R^+ : \forall x \in (a, b) \cap (x_0 - \delta, x_0 + \delta), f(x) \leq f(x_0).
  \]
  Since \(f\) is differentiable at \(x_0\), by \cref{i:10.1.1} we know that
  \[
    \lim_{x \to x_0 ; x \in (a, b) \setminus \set{x_0}} \dfrac{f(x) - f(x_0)}{x - x_0} = f'(x_0)
  \]
  By \cref{i:9.3.6} we know that \(\forall \varepsilon \in \R^+\), \(\exists \delta' \in \R^+\) such that
  \[
    \forall x \in (a, b) \setminus \set{x_0}, \abs{x - x_0} < \delta' \implies \abs{\dfrac{f(x) - f(x_0)}{x - x_0} - f'(x_0)} \leq \varepsilon.
  \]
  In particular, we have
  \[
    \forall x \in (a, b) \cap (x_0, x_0 + \delta), \abs{x - x_0} < \delta' \implies \abs{\dfrac{f(x) - f(x_0)}{x - x_0} - f'(x_0)} \leq \varepsilon
  \]
  and
  \[
    \forall x \in (a, b) \cap (x_0 - \delta, x_0), \abs{x - x_0} < \delta' \implies \abs{\dfrac{f(x) - f(x_0)}{x - x_0} - f'(x_0)} \leq \varepsilon.
  \]
  Thus, by \cref{i:9.3.6} we must have
  \[
    \lim_{x \to x_0 ; x \in (a, b) \cap (x_0, x_0 + \delta)} \dfrac{f(x) - f(x_0)}{x - x_0} = \lim_{x \to x_0 ; x \in (a, b) \cap (x_0 - \delta, x_0)} \dfrac{f(x) - f(x_0)}{x - x_0} = f'(x_0).
  \]
  Since
  \begin{align*}
             & \forall x \in (a, b) \cap (x_0, x_0 + \delta)                                                                     \\
    \implies & f(x) \leq f(x_0)                                                                                                  \\
    \implies & f(x) - f(x_0) \leq 0                                                                                              \\
    \implies & \dfrac{f(x) - f(x_0)}{x - x_0} \leq 0                                                                             \\
    \implies & \lim_{x \to x_0 ; x \in (a, b) \cap (x_0, x_0 + \delta)} \dfrac{f(x) - f(x_0)}{x - x_0} \leq 0 &  & \by{i:9.3.14} \\
    \implies & f'(x_0) \leq 0
  \end{align*}
  and
  \begin{align*}
             & \forall x \in (a, b) \cap (x_0 - \delta, x_0)                                                                      \\
    \implies & f(x) \leq f(x_0)                                                                                                   \\
    \implies & f(x) - f(x_0) \leq 0                                                                                               \\
    \implies & \dfrac{f(x) - f(x_0)}{x - x_0} \geq 0                                                                              \\
    \implies & \lim_{x \to x_0 ; x \in (a, b) \cap (x_0 - \delta, x_0)} \dfrac{f(x) - f(x_0)}{x - x_0} \geq 0, &  & \by{i:9.3.14} \\
    \implies & f'(x_0) \geq 0,
  \end{align*}
  we must have \(f'(x_0) = 0\).
  Similar arguments work for the case \(f\) attains local minimum at \(x_0\).
\end{proof}

\begin{note}
  \(f\) must be differentiable for \cref{i:10.2.6} to work.
  Also, \cref{i:10.2.6} does not work if the open interval \((a, b)\) is replaced by a closed interval \([a, b]\).
  For instance, the function \(f : [1, 2] \to \R\) defined by \(f(x) \coloneqq x\) has a local maximum at \(x_0 = 2\) and a local minimum \(x_0 = 1\) (in fact, these local extrema are global extrema), but at both points the derivative is \(f'(x_0) = 1\), not \(f'(x_0) = 0\).
  Thus, the endpoints of an interval can be local maxima or minima even if the derivative is not zero there.
  Finally, the converse of this proposition is false.
\end{note}

\begin{thm}[Rolle's theorem]\label{i:10.2.7}
  Let \(a < b\) be real numbers, and let \(g : [a, b] \to \R\) be a continuous function which is differentiable on \((a, b)\).
  Suppose also that \(g(a) = g(b)\).
  Then there exists an \(x \in (a, b)\) such that \(g'(x) = 0\).
\end{thm}

\begin{proof}
  Since \(g\) is continuous on \([a, b]\), by \cref{i:9.6.7} \(g\) attains its maximum at some point \(x_{\max} \in [a, b]\), and also attains its minimum at some point \(x_{\min} \in [a, b]\).
  If \((x_{\min} \in \set{a, b}) \land (x_{\max} \in \set{a, b})\) is true, then by \cref{i:9.6.5} we have \(g(x) = g(a) = g(b)\) for every \(x \in [a, b]\), and by \cref{i:10.1.13}(a) we know that \(g'(x) = 0\).
  So suppose that at least one of \(x_{\min}, x_{\max} \notin \set{a, b}\), i.e., \(\big(x_{\min} \in (a, b)\big) \lor \big(x_{\max} \in (a, b)\big)\) is true.
  If \(x_{\min} \in (a, b)\), then by \cref{i:10.2.6} we know that \(f'(x_{\min}) = 0\).
  Similarly, if \(x_{\max} \in (a, b)\), then by \cref{i:10.2.6} we know that \(f'(x_{\max}) = 0\).
  Thus, there exists an \(x \in (a, b)\) such that \(g'(x) = 0\).
\end{proof}

\begin{rmk}\label{i:10.2.8}
  We only assume \(f\) is differentiable on the open interval \((a, b)\), though of course \cref{i:10.2.7} also holds if we assume \(f\) is differentiable on the closed interval \([a, b]\), since this is larger than \((a, b)\).
\end{rmk}

\begin{cor}[Mean value theorem]\label{i:10.2.9}
  Let \(a < b\) be real numbers, and let \(f : [a, b] \to \R\) be a function which is continuous on \([a, b]\) and differentiable on \((a, b)\).
  Then there exists an \(x \in (a, b)\) such that \(f'(x) = \dfrac{f(b) - f(a)}{b - a}\).
\end{cor}

\begin{proof}
  Let \(g : [a, b] \to \R\) be a function where \(g(x) = f(x) - \dfrac{f(a) - f(b)}{a - b} x\).
  Since \(a < b\), we know that \(g\) is well-defined.
  Since \(f\) is differentiable on \((a, b)\), we know that
  \begin{align*}
             & x \text{ is differentiable on } (a, b)                                   &  & \by{i:10.1.13}[b] \\
    \implies & \dfrac{f(a) - f(b)}{a - b} x \text{ is differentiable on } (a, b)        &  & \by{i:10.1.13}[e] \\
    \implies & f(x) - \dfrac{f(a) - f(b)}{a - b} x \text{ is differentiable on } (a, b) &  & \by{i:10.1.13}[f] \\
    \implies & g(x) \text{ is differentiable on } (a, b)                                                       \\
    \implies & g'(x) = f'(x) - \dfrac{f(a) - f(b)}{a - b}.
  \end{align*}
  Since
  \[
    g(a) = f(a) - \dfrac{f(a) - f(b)}{a - b} a = \dfrac{af(a) - bf(a) - af(a) + af(b)}{a - b} = \dfrac{af(b) - bf(a)}{a - b}
  \]
  and
  \[
    g(b) = f(b) - \dfrac{f(a) - f(b)}{a - b} b = \dfrac{af(b) - bf(b) - bf(a) + bf(b)}{a - b} = \dfrac{af(b) - bf(a)}{a - b},
  \]
  we have \(g(a) = g(b)\) and by \cref{i:10.2.7} \(\exists x_0 \in (a, b)\) such that \(g'(x_0) = 0\).
  Thus
  \begin{align*}
             & g'(x_0) = 0                              \\
    \implies & f'(x_0) - \dfrac{f(a) - f(b)}{a - b} = 0 \\
    \implies & f'(x_0) = \dfrac{f(a) - f(b)}{a - b}.
  \end{align*}
\end{proof}

\exercisesection

\begin{ex}\label{i:ex:10.2.1}
  Prove \cref{i:10.2.6}.
\end{ex}

\begin{proof}
  See \cref{i:10.2.6}.
\end{proof}

\begin{ex}\label{i:ex:10.2.2}
  Give an example of a function \(f : (-1, 1) \to \R\) which is continuous and attains a global maximum at \(0\), but which is not differentiable at \(0\).
  Explain why this does not contradict \cref{i:10.2.6}.
\end{ex}

\begin{proof}
  Let \(f(x) = -\abs{x}\).
  Then \(f\) is continuous and attains a global maximum at \(0\), but which is not differentiable at \(0\).
  The fact that \(f\) is not differentiable at \(0\) does not contradict to \cref{i:10.2.6}.
\end{proof}

\begin{ex}\label{i:ex:10.2.3}
  Give an example of a function \(f : (-1, 1) \to \R\) which is differentiable, and whose derivative equals \(0\) at \(0\), but such that \(0\) is neither a local minimum nor a local maximum.
  Explain why this does not contradict \cref{i:10.2.6}.
\end{ex}

\begin{proof}
  Let \(f(x) = x^3\).
  Then by \cref{i:ex:10.1.5} we know that \(f'(x) = 3x^2\).
  Then we have \(f(0) = 0\) and \(f'(0) = 0\).
  But \(f(x) < f(0)\) for every \(x \in (-1, 0)\) and \(f(x) > f(0)\) for every \(x \in (0, 1)\).
  Thus, \(0\) is neither a local minimum nor a local maximum.
  This does not contradict to \cref{i:10.2.6} since \(0\) is not given to be a local minimum or local maximum.
\end{proof}

\begin{ex}\label{i:ex:10.2.4}
  Prove \cref{i:10.2.7}.
\end{ex}

\begin{proof}
  See \cref{i:10.2.7}.
\end{proof}

\begin{ex}\label{i:ex:10.2.5}
  Use \cref{i:10.2.7} to prove \cref{i:10.2.9}.
\end{ex}

\begin{proof}
  See \cref{i:10.2.9}.
\end{proof}

\begin{ex}\label{i:ex:10.2.6}
  Let \(M > 0\), and let \(f : [a, b] \to \R\) be a function which is continuous on \([a, b]\) and differentiable on \((a, b)\), and such that \(\abs{f'(x)} \leq M\) for all \(x \in (a, b)\) (i.e., the derivative of \(f\) is bounded).
  Show that for any \(x, y \in [a, b]\) we have the inequality \(\abs{f(x) - f(y)} \leq M \abs{x - y}\).
  Functions which obey the bound \(\abs{f(x) - f(y)} \leq M \abs{x - y}\) are known as \emph{Lipschitz continuous functions} with \emph{Lipschitz constant} \(M\);
  thus this exercise shows that functions with bounded derivative are Lipschitz continuous.
\end{ex}

\begin{proof}
  Let \(x, y \in [a, b]\).
  If \(x = y\), then we have \(0 = \abs{f(x) - f(y)} \leq M \abs{x - y} = 0\).
  So suppose that \(x \neq y\).
  We have either \(x < y\) or \(x > y\).
  Without the loss of generality suppose that \(x < y\).
  Then we have \([x, y] \subseteq [a, b]\) and \((x, y) \subseteq (a, b)\).
  By \cref{i:ex:9.4.6} we know that \(f|_{[x, y]}\) is continuous on \([x, y]\).
  By \cref{i:ex:10.1.1} we know that \(f|_{[x, y]}\) is differentiable on \((x, y)\).
  By mean value theorem (\cref{i:10.2.9}) we know that \(\exists c \in (x, y)\) such that
  \[
    f'(c) = \dfrac{f(y) - f(x)}{y - x}.
  \]
  Since \(c \in (x, y)\), we have \(c \in (a, b)\).
  By hypothesis we have
  \begin{align*}
             & \abs{f'(c)} = \abs{\dfrac{f(y) - f(x)}{y - x}} \leq M \\
    \implies & \abs{f(y) - f(x)} \leq M \abs{y - x}                  \\
    \implies & \abs{f(x) - f(y)} \leq M \abs{x - y}.
  \end{align*}
  Thus, we conclude that \(\forall x, y \in [a, b]\), we have \(\abs{f(x) - f(y)} \leq M \abs{x - y}\).
\end{proof}

\begin{ex}\label{i:ex:10.2.7}
  Let \(f : \R \to \R\) be a differentiable function such that \(f'\) is bounded.
  Show that \(f\) is uniformly continuous.
\end{ex}

\begin{proof}
  Since \(f'\) is bounded, by \cref{i:9.6.1} we know that \(\exists M \in \R^+\) such that \(\abs{f'(x)} \leq M\) for every \(x \in \R\).
  By \cref{i:ex:10.2.6} we know that \(\abs{f(x) - f(y)} \leq M \abs{x - y}\) for every \(x, y \in \R\).
  Then we have
  \begin{align*}
             & \forall \varepsilon \in \R^+, \exists \delta = \varepsilon / M : \forall x, y \in \R, \abs{x - y} \leq \delta \\
    \implies & \abs{f(x) - f(y)} \leq M \abs{x - y} \leq M \delta = M \dfrac{\varepsilon}{M} = \varepsilon
  \end{align*}
  and by \cref{i:9.9.2} \(f\) is uniformly continuous.
\end{proof}
