\section{Basic properties of the Riemann integral}\label{sec:11.4}

\begin{thm}[Laws of Riemann integration]\label{11.4.1}
  Let \(I\) be a bounded interval, and let \(f : I \to \R\) and \(g : I \to \R\) be Riemann integrable functions on \(I\).
  \begin{enumerate}
    \item The function \(f + g\) is Riemann integrable, and we have \(\int_I (f + g) = \int_I f + \int_I g\).
    \item For any real number \(c\), the function \(cf\) is Riemann integrable, and we have \(\int_I (cf) = c(\int_I f)\).
    \item The function \(f - g\) is Riemann integrable, and we have \(\int_I (f - g) = \int_I f - \int_I g\).
    \item If \(f(x) \geq 0\) for all \(x \in I\), then \(\int_I f \geq 0\).
    \item If \(f(x) \geq g(x)\) for all \(x \in I\), then \(\int_I f \geq \int_I g\).
    \item If \(f\) is the constant function \(f(x) = c\) for all \(x \in I\), then \(\int_I f = c \abs{I}\).
    \item Let \(J\) be a bounded interval containing \(I\) (i.e., \(I \subseteq J\)), and let \(F : J \to \R\) be the function
          \[
            F(x) \coloneqq \begin{dcases}
              f(x) & \text{if } x \in I    \\
              0    & \text{if } x \notin I \\
            \end{dcases}
          \]
          Then \(F\) is Riemann integrable on \(J\), and \(\int_J F = \int_I f\).
    \item Suppose that \(\set{J, K}\) is a partition of \(I\) into two intervals \(J\) and \(K\).
          Then the functions \(f|_J : J \to \R\) and \(f|_K : K \to \R\) are Riemann integrable on \(J\) and \(K\) respectively, and we have
          \[
            \int_I f = \int_J f|_J + \int_K f|_K.
          \]
  \end{enumerate}
\end{thm}

\begin{proof}{(a)}
  Since \(f, g\) are Riemann integrable on \(I\), by \cref{11.3.4} we have
  \[
    \int_I f = \overline{\int}_I f = \underline{\int}_I f
  \]
  and
  \[
    \int_I g = \overline{\int}_I g = \underline{\int}_I g.
  \]
  Let \(f_U : I \to \R\) and \(g_U : I \to \R\) be piecewise constant functions on \(I\) which majorizes \(f\) and \(g\), respectively.
  Let \(f_L : I \to \R\) and \(g_L : I \to \R\) be piecewise constant functions on \(I\) which minorizes \(f\) and \(g\), respectively.
  \(f_U, g_U, f_L, g_L\) are well-defined since by \cref{11.3.4} \(f, g\) are bounded functions on a bounded interval \(I\).
  By \cref{11.3.2} we have
  \[
    p.c. \int_I f_L \leq \underline{\int}_I f = \int_I f = \overline{\int}_I f \leq p.c. \int_I f_U
  \]
  and
  \[
    p.c. \int_I g_L \leq \underline{\int}_I g = \int_I g = \overline{\int}_I g \leq p.c. \int_I g_U.
  \]
  By \cref{11.3.4} both \(f, g\) are bounded functions, so \(f + g\) is bounded function, and \(\underline{\int}_I (f + g), \overline{\int}_I (f + g)\) are well-defined (by \cref{11.3.2}).
  By \cref{ex:11.3.2} we know that \(f_U + g_U\) majorizes \(f + g_U\) and \(f + g_U\) majorizes \(f + g\), thus \(f_U + g_U\) majorizes \(f + g\).
  Similarly \(f_L + g_L\) minorizes \(f + g\).
  Then we have
  \begin{align*}
             & \overline{\int}_I (f + g) \leq p.c. \int_I (f_U + g_U)                   &   & \by{11.3.2}                             \\
    \implies & \overline{\int}_I (f + g) \leq p.c. \int_I f_U + p.c. \int_I g_U         &   & \text{(by \cref{11.2.16}(a))}           \\
    \implies & \overline{\int}_I (f + g) - p.c. \int_I g_U \leq p.c. \int_I f_U         &   & \text{(note that \(f_U\) is arbitrary)} \\
    \implies & \overline{\int}_I (f + g) - p.c. \int_I g_U \leq \overline{\int}_I f     &   & \by{11.3.2}                             \\
    \implies & \overline{\int}_I (f + g) - \overline{\int}_I f \leq p.c. \int_I g_U     &   & \text{(note that \(g_U\) is arbitrary)} \\
    \implies & \overline{\int}_I (f + g) - \overline{\int}_I f \leq \overline{\int}_I g &   & \by{11.3.2}                             \\
    \implies & \overline{\int}_I (f + g) \leq \overline{\int}_I f + \overline{\int}_I g &                                             \\
    \implies & \overline{\int}_I (f + g) \leq \int_I f + \int_I g                       &   & \by{11.3.4}
  \end{align*}
  and
  \begin{align*}
             & \underline{\int}_I (f + g) \geq p.c. \int_I (f_L + g_L)                     &   & \by{11.3.2}                             \\
    \implies & \underline{\int}_I (f + g) \geq p.c. \int_I f_L + p.c. \int_I g_L           &   & \text{(by \cref{11.2.16}(a))}           \\
    \implies & \underline{\int}_I (f + g) - p.c. \int_I g_L \geq p.c. \int_I f_L           &   & \text{(note that \(f_L\) is arbitrary)} \\
    \implies & \underline{\int}_I (f + g) - p.c. \int_I g_L \geq \underline{\int}_I f      &   & \by{11.3.2}                             \\
    \implies & \underline{\int}_I (f + g) - \underline{\int}_I f \geq p.c. \int_I g_L      &   & \text{(note that \(g_L\) is arbitrary)} \\
    \implies & \underline{\int}_I (f + g) - \underline{\int}_I f \geq \underline{\int}_I g &   & \by{11.3.2}                             \\
    \implies & \underline{\int}_I (f + g) \geq \underline{\int}_I f + \underline{\int}_I g &                                             \\
    \implies & \underline{\int}_I (f + g) \geq \int_I f + \int_I g.                        &   & \by{11.3.4}
  \end{align*}
  By \cref{11.3.3} we have
  \[
    \int_I f + \int_I g \leq \underline{\int}_I (f + g) \leq \overline{\int}_I (f + g) \leq \int_I f + \int_I g
  \]
  and thus by \cref{11.3.4} we have
  \[
    \int_I (f + g) = \underline{\int}_I (f + g) = \overline{\int}_I (f + g) = \int_I f + \int_I g.
  \]
\end{proof}

\begin{proof}{(b)}
  Since \(f\) is Riemann integrable on \(I\), by \cref{11.3.4} we have
  \[
    \int_I f = \overline{\int}_I f = \underline{\int}_I f.
  \]
  First suppose that \(c = 0\).
  Then we have \((cf)(x) = 0\) for all \(x \in 0\), thus we have
  \begin{align*}
    \int_I (cf) & = p.c. \int_I (cf) &  & \by{11.3.7} \\
                & = 0                                 \\
                & = c \int_I f.
  \end{align*}

  Next suppose that \(c > 0\).
  Let \(f_U : I \to \R\) be a piecewise constant function on \(I\) which majorizes \(f\).
  Let \(f_L : I \to \R\) be a piecewise constant function on \(I\) which minorizes \(f\).
  \(f_U, f_L\) are well-defined since by \cref{11.3.4} \(f\) is a bounded function on a bounded interval \(I\).
  Then by \cref{11.3.2} we have
  \[
    p.c. \int_I f_L \leq \underline{\int}_I f = \int_I f = \overline{\int}_I f \leq p.c. \int_I f_U.
  \]
  Since \(f\) is a bounded function, \(cf\) is also a bounded function, by \cref{11.3.2} both \(\overline{\int}_I (cf), \underline{\int}_I (cf)\) are well-defined.
  Since \(c > 0\), by \cref{11.3.1} we know that \(c f_U\) majorizes \(c f\) and \(c f_L\) minorizes \(c f\).
  Then we have
  \begin{align*}
             & \overline{\int}_I (cf) \leq p.c. \int_I (c f_U)                          &  & \by{11.3.2}                             \\
    \implies & \overline{\int}_I (cf) \leq c \bigg(p.c. \int_I f_U\bigg)                &  & \text{(by \cref{11.2.16}(b))}           \\
    \implies & \dfrac{1}{c} \bigg(\overline{\int}_I (cf)\bigg) \leq p.c. \int_I f_U     &  & \text{(note that \(f_U\) is arbitrary)} \\
    \implies & \dfrac{1}{c} \bigg(\overline{\int}_I (cf)\bigg) \leq \overline{\int}_I f &  & \by{11.3.2}                             \\
    \implies & \overline{\int}_I (cf) \leq c\bigg(\overline{\int}_I f\bigg)                                                          \\
    \implies & \overline{\int}_I (cf) \leq c\bigg(\int_I f\bigg)                        &  & \by{11.3.4}
  \end{align*}
  and
  \begin{align*}
             & \underline{\int}_I (cf) \geq p.c. \int_I (c f_L)                           &  & \by{11.3.2}                             \\
    \implies & \underline{\int}_I (cf) \geq c \bigg(p.c. \int_I f_L\bigg)                 &  & \text{(by \cref{11.2.16}(b))}           \\
    \implies & \dfrac{1}{c} \bigg(\underline{\int}_I (cf)\bigg) \geq p.c. \int_I f_L      &  & \text{(note that \(f_L\) is arbitrary)} \\
    \implies & \dfrac{1}{c} \bigg(\underline{\int}_I (cf)\bigg) \geq \underline{\int}_I f &  & \by{11.3.2}                             \\
    \implies & \underline{\int}_I (cf) \geq c\bigg(\underline{\int}_I f\bigg)                                                          \\
    \implies & \underline{\int}_I (cf) \geq c\bigg(\int_I f\bigg).                        &  & \by{11.3.4}
  \end{align*}
  By \cref{11.3.3} we have
  \[
    c\bigg(\int_I f\bigg) \leq \underline{\int}_I (cf) \leq \overline{\int}_I (cf) \leq c\bigg(\int_I f\bigg)
  \]
  and thus by \cref{11.3.4} we have
  \[
    \int_I (cf) = \underline{\int}_I (cf) = \overline{\int}_I (cf) = c\bigg(\int_I f\bigg).
  \]

  Finally suppose that \(c < 0\).
  Using the same definition of \(f_U, f_L\) we have
  \begin{align*}
             & \overline{\int}_I (cf) \leq p.c. \int_I (c f_U)                                               &  & \by{11.3.2}                   \\
    \implies & \overline{\int}_I (cf) \leq c \bigg(p.c. \int_I f_U\bigg)                                     &  & \text{(by \cref{11.2.16}(b))} \\
    \implies & \dfrac{1}{c} \bigg(\overline{\int}_I (cf)\bigg) \geq p.c. \int_I f_U                                                             \\
    \implies & \dfrac{1}{c} \bigg(\overline{\int}_I (cf)\bigg) \geq p.c. \int_I f_U \geq \overline{\int}_I f &  & \by{11.3.2}                   \\
    \implies & \overline{\int}_I (cf) \leq c\bigg(\overline{\int}_I f\bigg)                                                                     \\
    \implies & \overline{\int}_I (cf) \leq c\bigg(\int_I f\bigg)                                             &  & \by{11.3.4}
  \end{align*}
  and
  \begin{align*}
             & \underline{\int}_I (cf) \geq p.c. \int_I (c f_L)                                                &  & \by{11.3.2}                   \\
    \implies & \underline{\int}_I (cf) \geq c \bigg(p.c. \int_I f_L\bigg)                                      &  & \text{(by \cref{11.2.16}(b))} \\
    \implies & \dfrac{1}{c} \bigg(\underline{\int}_I (cf)\bigg) \leq p.c. \int_I f_L                                                              \\
    \implies & \dfrac{1}{c} \bigg(\underline{\int}_I (cf)\bigg) \leq p.c. \int_I f_L \leq \underline{\int}_I f &  & \by{11.3.2}                   \\
    \implies & \underline{\int}_I (cf) \geq c\bigg(\underline{\int}_I f\bigg)                                                                     \\
    \implies & \underline{\int}_I (cf) \geq c\bigg(\int_I f\bigg).                                             &  & \by{11.3.4}
  \end{align*}
  By \cref{11.3.3} we have
  \[
    c\bigg(\int_I f\bigg) \leq \underline{\int}_I (cf) \leq \overline{\int}_I (cf) \leq c\bigg(\int_I f\bigg)
  \]
  and thus by \cref{11.3.4} we have
  \[
    \int_I (cf) = \underline{\int}_I (cf) = \overline{\int}_I (cf) = c\bigg(\int_I f\bigg).
  \]
  We conclude that \(\forall c \in \R\), \(\int_I (cf) = c (\int_I f)\).
\end{proof}

\begin{proof}{(c)}
  We have
  \begin{align*}
    \int_I f - \int_I g & = \int_I f + \int_I (-g)    &  & \text{(by \cref{11.4.1}(b))} \\
                        & = \int_I \big(f + (-g)\big) &  & \text{(by \cref{11.4.1}(a))} \\
                        & = \int_I (f - g).           &  & \by{9.2.1}
  \end{align*}
\end{proof}

\begin{proof}{(d)}
  Let \(f_U : I \to \R\) be a piecewise constant function on \(I\) which majorizes \(f\).
  \(f_U\) is well-defined since by \cref{11.3.4} \(f\) is a bounded function on a bounded interval \(I\).
  Since \(0 \leq f(x) \leq f_U(x)\) for every \(x \in I\), we have
  \begin{align*}
             & 0 \leq p.c. \int_I f_U     &  & \text{(by \cref{11.2.16}(d))} \\
    \implies & 0 \leq \overline{\int}_I f &  & \by{11.3.2}                   \\
    \implies & 0 \leq \int_I f.           &  & \by{11.3.4}
  \end{align*}
\end{proof}

\begin{proof}{(e)}
  We have \(f(x) - g(x) \geq 0\) for every \(x \in I\) and by \cref{11.4.1}(c) \(f - g\) is Riemann integrable on \(I\).
  Thus
  \begin{align*}
             & \int_I (f - g) \geq 0      &  & \text{(by \cref{11.4.1}(d))} \\
    \implies & \int_I f - \int_I g \geq 0 &  & \text{(by \cref{11.4.1}(c))} \\
    \implies & \int_I f \geq \int_I g.
  \end{align*}
\end{proof}

\begin{proof}{(f)}
  We have
  \begin{align*}
    \int_I f & = p.c. \int_I f &  & \by{11.3.7}                   \\
             & = c \abs{I}.    &  & \text{(by \cref{11.2.16}(f))}
  \end{align*}
\end{proof}

\begin{proof}{(g)}
  Let \(f_U : I \to \R\) be a piecewise constant function on \(I\) which majorizes \(f\).
  Let \(f_L : I \to \R\) be a piecewise constant function on \(I\) which minorizes \(f\).
  \(f_U, f_L\) are well-defined since by \cref{11.3.4} \(f\) is a bounded function on a bounded interval \(I\).
  Then by \cref{11.3.2} we have
  \[
    p.c. \int_I f_L \leq \underline{\int}_I f = \int_I f = \overline{\int}_I f \leq p.c. \int_I f_U.
  \]
  Let \(F_U : J \to \R\) be the function
  \[
    F_U(x) = \begin{dcases}
      f_U(x) & \text{if } x \in I    \\
      0      & \text{if } x \notin I
    \end{dcases}
  \]
  and let \(F_L : J \to \R\) be the function
  \[
    F_L(x) = \begin{dcases}
      f_L(x) & \text{if } x \in I     \\
      0      & \text{if } x \notin I.
    \end{dcases}
  \]
  We know that \(F_U\) majorizes \(F\) and \(F_L\) minorizes \(F\), and by \cref{11.2.16}(g) we have \(p.c. \int_J F_U = p.c. \int_I f_U\) and \(p.c. \int_J F_L = p.c. \int_I f_L\).
  Thus \(F\) is a bounded function on a bounded interval \(I\), and we have
  \begin{align*}
             & \overline{\int}_J F \leq p.c. \int_J F_U     &  & \by{11.3.2}                                        \\
    \implies & \overline{\int}_J F \leq p.c. \int_I f_U     &  & \text{(by \cref{11.2.16}(g))}                      \\
    \implies & \overline{\int}_J F \leq \overline{\int}_I f &  & \text{(by \cref{11.3.2} and \(f_U\) is arbitrary)} \\
    \implies & \overline{\int}_J F \leq \int_I f            &  & \by{11.3.4}
  \end{align*}
  and
  \begin{align*}
             & \underline{\int}_J F \geq p.c. \int_J F_L      &  & \by{11.3.2}                                        \\
    \implies & \underline{\int}_J F \geq p.c. \int_I f_L      &  & \text{(by \cref{11.2.16}(g))}                      \\
    \implies & \underline{\int}_J F \geq \underline{\int}_I f &  & \text{(by \cref{11.3.2} and \(f_L\) is arbitrary)} \\
    \implies & \underline{\int}_J F \geq \int_I f.            &  & \by{11.3.4}
  \end{align*}
  By \cref{11.3.3} we have
  \[
    \int_I f \leq \underline{\int}_J F \leq \overline{\int}_J F \leq \int_I f
  \]
  and thus by \cref{11.3.4} we have
  \[
    \int_J F = \underline{\int}_J F = \overline{\int}_J F = \int_I f.
  \]
\end{proof}

\begin{proof}{(h)}
  Let \(f_U : I \to \R\) be a piecewise constant function on \(I\) which majorizes \(f\).
  Let \(f_L : I \to \R\) be a piecewise constant function on \(I\) which minorizes \(f\).
  \(f_U, f_L\) are well-defined since by \cref{11.3.4} \(f\) is a bounded function on a bounded interval \(I\).
  Then by \cref{11.3.2} we have
  \[
    p.c. \int_I f_L \leq \underline{\int}_I f = \int_I f = \overline{\int}_I f \leq p.c. \int_I f_U.
  \]
  By \cref{11.2.16}(h) we know that \(f_U|_J : J \to \R\), \(f_L|_J : J \to \R\) are piecewise constant function on \(J\) and \(f_U|_K : K \to \R\), \(f_L|_K : K \to \R\) are piecewise constant functions on \(K\).
  By \cref{11.3.1} we know that \(f_U|_J\) majorizes \(f|_J\) and \(f_L|_J\) minorizes \(f|_J\), similarly \(f_U|_K\) majorizes \(f|_K\) and \(f_L|_K\) minorizes \(f|_K\).
  Thus \(f|_J\), \(f|_K\) are bounded functions on bounded intervals \(J, K\), respectively.
  So \(\overline{\int}_J f|_J\), \(\overline{\int}_K f|_K\), \(\underline{\int}_J f|_J\), \(\underline{\int}_K f|_K\) are well-defined.
  Then we have
  \begin{align*}
             & \overline{\int}_J f|_J + \overline{\int}_K f|_K \leq p.c. \int_J f_U|_J + p.c. \int_K f_U|_K &  & \by{11.3.2}                   \\
    \implies & \overline{\int}_J f|_J + \overline{\int}_K f|_K \leq p.c. \int_I f_U                         &  & \text{(by \cref{11.2.16}(h))} \\
    \implies & \overline{\int}_J f|_J + \overline{\int}_K f|_K \leq \overline{\int}_I f                     &  & \by{11.3.2}                   \\
    \implies & \overline{\int}_J f|_J + \overline{\int}_K f|_K \leq \int_I f                                &  & \by{11.3.4}
  \end{align*}
  and
  \begin{align*}
             & \underline{\int}_J f|_J + \underline{\int}_K f|_K \geq p.c. \int_J f_L|_J + p.c. \int_K f_L|_K &  & \by{11.3.2}                   \\
    \implies & \underline{\int}_J f|_J + \underline{\int}_K f|_K \geq p.c. \int_I f_L                         &  & \text{(by \cref{11.2.16}(h))} \\
    \implies & \underline{\int}_J f|_J + \underline{\int}_K f|_K \geq \underline{\int}_I f                    &  & \by{11.3.2}                   \\
    \implies & \underline{\int}_J f|_J + \underline{\int}_K f|_K \geq \int_I f.                               &  & \by{11.3.4}
  \end{align*}
  By \cref{11.3.3} we have
  \[
    \int_I f \leq \underline{\int}_J f|_J + \underline{\int}_K f|_K \leq \overline{\int}_J f|_J + \overline{\int}_K f|_K \leq \int_I f
  \]
  and thus we have
  \[
    \underline{\int}_J f|_J + \underline{\int}_K f|_K = \overline{\int}_J f|_J + \overline{\int}_J f|_K = \int_I f.
  \]
  Since
  \begin{align*}
             & \underline{\int}_J f|_J + \underline{\int}_K f|_K = \overline{\int}_J f|_J + \overline{\int}_J f|_K                                \\
    \implies & 0 \geq \underline{\int}_J f|_J - \overline{\int}_J f|_J = \overline{\int}_J f|_K - \underline{\int}_K f|_K \geq 0 &  & \by{11.3.3} \\
    \implies & \underline{\int}_J f|_J - \overline{\int}_J f|_J = \overline{\int}_J f|_K - \underline{\int}_K f|_K = 0,
  \end{align*}
  by \cref{11.3.4} we have
  \begin{align*}
     & \int_J f|_J = \underline{\int}_J f|_J = \overline{\int}_J f|_J, \\
     & \int_K f|_K = \underline{\int}_K f|_K = \overline{\int}_K f|_K, \\
     & \int_J f|_J + \int_K f|_K = \int_I f.
  \end{align*}
\end{proof}

\begin{rmk}\label{11.4.2}
  We often abbreviate \(\int_J f|_J\) as \(\int_J f\) even though \(f\) is really defined on a larger domain than just \(J\).
  We also observe from \cref{11.4.1}(h) and \cref{11.3.8} that if \(f : [a, b] \to \R\) is Riemann integrable on a closed interval \([a, b]\), then \(\int_{[a, b]} f = \int_{(a, b]} f = \int_{[a, b)} f = \int_{(a, b)} f\).
\end{rmk}

\begin{thm} and min preserve integrability]\label{11.4.3}
  Let \(I\) be a bounded interval, and let \(f : I \to \R\) and \(g : I \to \R\) be a Riemann integrable function.
  Then the functions \(\max(f, g) : I \to \R\) and \(\min(f, g) : I \to \R\) defined by \(\max(f, g)(x) \coloneqq \max\big(f(x), g(x)\big)\) and \(\min(f, g)(x) \coloneqq \min\big(f(x), g(x)\big)\) are also Riemann integrable.
\end{thm}

\begin{proof}
  We shall just prove the claim for \(\max(f, g)\), the case of \(\min(f, g)\) being similar.
  First note that since \(f\) and \(g\) are bounded, then \(\max(f, g)\) is also bounded.

  Let \(\varepsilon > 0\).
  Since \(\int_I f = \underline{\int}_I f\), there exists a piecewise constant function \(\underline{f} : I \to \R\) which minorizes \(f\) on \(I\) such that
  \[
    \int_I \underline{f} \geq \int_I f - \varepsilon.
  \]
  Similarly we can find a piecewise constant \(g : I \to \R\) which minorizes \(g\) on \(I\) such that
  \[
    \int_I \underline{g} \geq \int_I g - \varepsilon,
  \]
  and we can find piecewise functions \(\overline{f}, \overline{g}\) which majorize \(f, g\) respectively on \(I\) such that
  \[
    \int_I \overline{f} \leq \int_I f + \varepsilon
  \]
  and
  \[
    \int_I \overline{g} \leq \int_I g + \varepsilon.
  \]
  In particular, if \(h : I \to \R\) denotes the function
  \[
    h \coloneqq (\overline{f} - \underline{f}) + (\overline{g} - \underline{g})
  \]
  we have
  \[
    \int_I h \leq 4 \varepsilon.
  \]
  On the other hand, \(\max(\underline{f}, \underline{g})\) is a piecewise constant function on \(I\) which minorizes \(\max(f, g)\), while \(\max(\overline{f}, \overline{g})\) is similarly a piecewise constant function on \(I\) which majorizes \(\max(f, g)\).
  Thus
  \[
    \int_I \max(\underline{f}, \underline{g}) \leq \underline{\int}_I \max(f, g) \leq \overline{\int}_I \max(f, g) \leq \int_I \max(\overline{f}, \overline{g}),
  \]
  and so
  \[
    0 \leq \overline{\int}_I \max(f, g) - \underline{\int}_I \max(f, g) \leq \int_I \max(\overline{f}, \overline{g}) - \max(\underline{f}, \underline{g}).
  \]
  But we have
  \[
    \overline{f}(x) = \underline{f}(x) + (\overline{f} - \underline{f})(x) \leq \underline{f}(x) + h(x)
  \]
  and similarly
  \[
    \overline{g}(x) = \underline{g}(x) + (\overline{g} - \underline{g})(x) \leq \underline{g}(x) + h(x)
  \]
  and thus
  \[
    \max\big(\overline{f}(x), \overline{g}(x)\big) \leq \max\big(\underline{f}(x), \underline{g}(x)\big) + h(x).
  \]
  Inserting this into the previous inequality, we obtain
  \[
    0 \leq \overline{\int}_I \max(f, g) - \underline{\int}_I \max(f, g) \leq \int_I h \leq 4 \varepsilon.
  \]
  To summarize, we have shown that
  \[
    0 \leq \overline{\int}_I \max(f, g) - \underline{\int}_I \max(f, g) \leq 4 \varepsilon
  \]
  for every \(\varepsilon\).
  Since \(\overline{\int}_I \max(f, g) - \underline{\int}_I \max(f, g)\) does not depend on \(\varepsilon\), we thus see that
  \[
    \overline{\int}_I \max(f, g) - \underline{\int}_I \max(f, g) = 0
  \]
  and hence that \(\max(f, g)\) is Riemann integrable.
\end{proof}

\begin{cor}[Absolute values preserve Riemann integrability]\label{11.4.4}
  \quad
  Let \(I\) be a bounded interval.
  If \(f : I \to \R\) is a Riemann integrable function, then the positive part \(f_+ \coloneqq \max(f, 0)\) and the negative part \(f_- \coloneqq \min(f, 0)\) are also Riemann integrable on \(I\).
  Also, the absolute value \(\abs{f}\), defined by \(\abs{f}(x) = \abs{f(x)}\) is also Riemann integrable on \(I\).
  (observe that \(\abs{f} = f_+ - f_-\))
\end{cor}

\begin{proof}
  By \cref{11.4.3} we know that \(f_+, f_-\) are Riemann integrable.
  Since \(\abs{f} = f_+ - f_-\), by \cref{11.4.1}(a) we know that \(\abs{f}\) is Riemann integrable.
\end{proof}

\begin{thm}ducts preserve Riemann integrability]\label{11.4.5}
  Let \(I\) be a bounded interval.
  If \(f : I \to \R\) and \(g : I \to \R\) are Riemann integrable, then \(fg : I \to \R\) is also Riemann integrable.
\end{thm}

\begin{proof}
  We split \(f = f_+ + f_-\) and \(g = g_+ + g_-\) into positive and negative parts;
  by \cref{11.4.4}, the functions \(f_+\), \(f_-\), \(g_+\), \(g_-\) are Riemann integrable.
  Since
  \[
    fg = f_+ g_+ + f_+ g_- + f_- g_+ + f_- g_-
  \]
  then it suffices to show that the functions \(f_+ g_+\), \(f_+ g_-\), \(f_- g_+\), \(f_- g_-\) are individually Riemann integrable.
  We will just show this for \(f_+ g_+\);
  the other three are similar.

  Since \(f_+\) and \(g_+\) are bounded and positive, there are \(M_1, M_2 > 0\) such that
  \[
    0 \leq f_+(x) \leq M_1 \text{ and } 0 \leq g_+(x) \leq M_2
  \]
  for all \(x \in I\).
  Now let \(\varepsilon > 0\) be arbitrary.
  Then, as in the proof of \cref{11.4.3}, we can find a piecewise constant function \(\underline{f_+}\) minorizing \(f_+\) on \(I\), and a piecewise constant function \(\overline{f_+}\) majorizing \(f_+\) on \(I\), such that
  \[
    \int_I \overline{f_+} \leq \int_I f_+ + \varepsilon
  \]
  and
  \[
    \int_I \underline{f_+} \geq \int_I f_+ - \varepsilon.
  \]
  Note that \(\underline{f_+}\) may be negative at places, but we can fix this by replacing \(\underline{f_+}\) by \(\max(\underline{f_+}, 0)\), since this still minorizes \(f_+\) and still has integral greater than or equal to \(\int_I f_+ - \varepsilon\).
  So without loss of generality we may assume that \(\underline{f_+}(x) \geq 0\) for all \(x \in I\).
  Similarly we may assume that \(\overline{f_+}(x) \leq M_1\) for all \(x \in I\);
  thus
  \[
    0 \leq \underline{f_+}(x) \leq f_+(x) \leq \overline{f_+}(x) \leq M_1
  \]
  for all \(x \in I\).

  Similar reasoning allows us to find piecewise constant \(\underline{g_+}\) minorizing \(g_+\), and \(\overline{g_+}\) majorizing \(g_+\), such that
  \[
    \int_I \overline{g_+} \leq \int_I g_+ + \varepsilon
  \]
  and
  \[
    \int_I \underline{g_+} \geq \int_I g_+ - \varepsilon,
  \]
  and
  \[
    0 \leq \underline{g_+}(x) \leq g_+(x) \leq \overline{g_+}(x) \leq M_2
  \]
  for all \(x \in I\).

  Notice that \(\underline{f_+} \underline{g_+}\) is piecewise constant and minorizes \(f_+ g_+\), while \(\overline{f_+} \overline{g_+}\) is piecewise constant and majorizes \(f_+ g_+\).
  Thus
  \[
    0 \leq \overline{\int}_I f_+ g_+ - \underline{\int}_I f_+ g_+ \leq \int_I \overline{f}_+ \overline{g_+} - \underline{f_+} \underline{g_+}.
  \]
  However, we have
  \begin{align*}
    \overline{f_+}(x) \overline{g_+}(x) - \underline{f_+}(x) \underline{g_+}(x) & = \overline{f_+}(x) (\overline{g_+} - \underline{g_+})(x) + \underline{g_+}(x) (\overline{f_+} - \underline{f_+})(x) \\
                                                                                & \leq M_1 (\overline{g_+} - \underline{g_+})(x) + M_2 (\overline{f_+} - \underline{f_+})(x)
  \end{align*}
  for all \(x \in I\), and thus
  \begin{align*}
    0 \leq \overline{\int}_I f_+ g_+ - \underline{\int}_I f_+ g_+ & \leq M_1 \int_I (\overline{g_+} - \underline{g_+}) + M_2 \int_I (\overline{f_+} - \underline{f_+}) \\
                                                                  & \leq M_1 (2\varepsilon) + M_2 (2\varepsilon).
  \end{align*}
  Again, since \(\varepsilon\) was arbitrary, we can conclude that \(f_+ g_+\) is Riemann integrable, as before.
  Similar argument show that \(f_+ g_-\), \(f_- g_+\), \(f_- g_-\) are Riemann integrable;
  combining them we obtain that \(fg\) is Riemann integrable.
\end{proof}

\exercisesection

\begin{ex}\label{ex:11.4.1}
  Prove \cref{11.4.1}.
\end{ex}

\begin{proof}
  See \cref{11.4.1}.
\end{proof}

\begin{ex}\label{ex:11.4.2}
  Let \(a < b\) be real numbers, and let \(f : [a, b] \to \R\) be a continuous, non-negative function
  (so \(f(x) \geq 0\) for all \(x \in [a, b]\)).
  Suppose that \(\int_{[a, b]} f = 0\).
  Show that \(f(x) = 0\) for all \(x \in [a, b]\).
\end{ex}

\begin{proof}
  Suppose for sake of contradiction that \(\exists x_0 \in [a, b]\) such that \(f(x_0) > 0\).
  Since \(f\) is continuous, by \cref{9.4.7} we have
  \[
    \forall \varepsilon \in \R^+, \exists \delta \in \R^+ : \big(\forall x \in [a, b], \abs{x - x_0} < \delta \implies \abs{f(x) - f(x_0)} \leq \varepsilon\big),
  \]
  or equivalently
  \[
    \forall \varepsilon \in \R^+, \exists \delta \in \R^+ : \big(\forall x \in [a, b] \cap (x_0 - \delta, x_0 + \delta) \implies \abs{f(x) - f(x_0)} \leq \varepsilon\big).
  \]
  In particular, we have
  \[
    \exists \delta \in \R^+ : \bigg(\forall x \in [a, b] \cap (x_0 - \delta, x_0 + \delta) \implies \abs{f(x) - f(x_0)} \leq \dfrac{f(x_0)}{2}\bigg),
  \]
  or equivalently
  \[
    \exists \delta \in \R^+ : \bigg(\forall x \in [a, b] \cap (x_0 - \delta, x_0 + \delta) \implies \dfrac{f(x_0)}{2} \leq f(x) \leq \dfrac{3 f(x_0)}{2}\bigg).
  \]
  Since \(\delta \neq 0\), we know that \([a, b] \cap (x_0 - \delta, x_0 + \delta) \neq \emptyset\).
  Since \(a \neq b\), we know that
  \[
    \sup\big([a, b] \cap (x_0 - \delta, x_0 + \delta)\big) \neq \inf\big([a, b] \cap (x_0 - \delta, x_0 + \delta)\big).
  \]
  Thus by \cref{11.1.8} we have \(\abs{[a, b] \cap (x_0 - \delta, x_0 + \delta)} > 0\).
  By \cref{11.1.6} we know that \([a, b] \cap (x_0 - \delta, x_0 + \delta)\) is a bounded interval.
  Let \(f_L : [a, b] \to \R\) be the function
  \[
    f_L(x) = \begin{dcases}
      \dfrac{f(x_0)}{2} & \text{if } x \in [a, b] \cap (x_0 - \delta, x_0 + \delta)    \\
      0                 & \text{if } x \notin [a, b] \cap (x_0 - \delta, x_0 + \delta)
    \end{dcases}
  \]
  Since \(f(x) \geq 0\) for all \(x \in [a, b]\), we know that \(f_L\) minorizes \(f\).
  By \cref{11.2.16}(g) we know that \(f_L\) is a piecewise constant function.
  By \cref{11.3.7} we have
  \[
    \int_{[a, b]} f_L = p.c. \int_{[a, b]} f_L = \dfrac{f(x_0)}{2}\abs{[a, b] \cap (x_0 - \delta, x_0 + \delta)} > 0.
  \]
  But by \cref{11.3.2} and \cref{11.3.4} we have
  \[
    0 < \int_{[a, b]} f_L \leq \underline{\int}_{[a, b]} f = \int_{[a, b]} f = 0,
  \]
  a contradiction.
  Thus we must have \(f(x) = 0\) for all \(x \in [a, b]\).
\end{proof}

\begin{ex}\label{ex:11.4.3}
  Let \(I\) be a bounded interval, let \(f : I \to \R\) be a Riemann integrable function, and let \(\mathbf{P}\) be a partition of \(I\).
  Show that
  \[
    \int_I f = \sum_{J \in \mathbf{P}} \int_J f|_J.
  \]
\end{ex}

\begin{proof}
  Let \(P(n)\) be the statement ``\(\#(\mathbf{P}) = n\) and \(\int_I f = \sum_{J \in \mathbf{P}} \int_J f|_J\)''.
  We use induction on \(n\) to show that \(P(n)\) is true \(\forall n \in \N\).
  For \(n = 0\), we have \(\mathbf{P} = \emptyset\) and \(I = \emptyset\).
  Thus
  \begin{align*}
    p.c. \int_{[\emptyset]} f & = \sum_{J \in \emptyset} c_J \abs{J} &  & \by{11.2.9}                  \\
                              & = 0                                  &  & \text{(by \cref{7.1.11}(a))} \\
                              & = p.c. \int_{\emptyset} f            &  & \by{11.2.14}                 \\
                              & = \int_{\emptyset} f                 &  & \by{11.3.7}                  \\
                              & = \sum_{J \in \emptyset} \int_J f|_J &  & \text{(by \cref{7.1.11}(a))}
  \end{align*}
  and the base case holds.
  Suppose inductively that \(P(n)\) is true for some \(n \geq 0\).
  Then we need to show that \(P(n + 1)\) is true.
  Let \(K \in \mathbf{P}\) such that \(x < y\) for every \(x \in K\) and \(y \in I \setminus K\).
  Then \(\set{K, \bigcup (\mathbf{P} \setminus \set{K})}\) is a partition of \(I\), and
  \begin{align*}
    \int_I f & = \int_K f|_K + \int_{\bigcup (\mathbf{P} \setminus \set{K})} f|_{\bigcup (\mathbf{P} \setminus \set{K})} &  & \text{(by \cref{11.4.1}(h))} \\
             & = \int_K f|_K + \sum_{J \in \mathbf{P} \setminus \set{K}} \int_J f|_J                                     &  & \byIH                        \\
             & = \sum_{J \in \mathbf{P}} \int_J f|_J.                                                                    &  & \text{(by \cref{7.1.11}(e))}
  \end{align*}
  This closes the induction.
\end{proof}

\begin{ex}\label{ex:11.4.4}
  Without repeating all the computations in the above proofs, give a short explanation as to why the remaining cases of \cref{11.4.3} and \cref{11.4.5} follow automatically from the cases presented in the text.
\end{ex}

\begin{proof}
  We first show that the remaining case of \cref{11.4.3} is true.
  By \cref{11.4.1}(b) \(-f\) and \(-g\) are Riemann integrable on \(I\).
  Since \(\max(-f, -g)\) is Riemann integrable and \(\min(f, g) = -\max(-f, -g)\), by \cref{11.4.1}(b) we know that \(\min(f, g)\) is Riemann integrable.

  Now we show that the remaining cases of \cref{11.4.5} are true.
  By \cref{11.4.4} \((-f)_+\) and \((-g)_+\) are Riemann integrable on \(I\).
  Since for any Riemann integrable functions \(p\) and \(q\), \(p_+ q_+\) are Riemann integrable (which is showed in the proof of \cref{11.4.5}), we have
  \begin{align*}
    f_+ g_- & = f_+ \cdot \big(\min(g, 0)\big)                      &  & \by{11.4.4} \\
            & = f_+ \cdot \big(-\max(-g, 0)\big)                                     \\
            & = f_+ \cdot \big(-(-g)_+\big)                         &  & \by{11.4.4} \\
            & = -\big(f_+ \cdot (-g)_+\big)                         &  & \by{9.2.1}  \\
    f_- g_+ & = \big(\min(f, 0)\big) \cdot g_+                      &  & \by{11.4.4} \\
            & = \big(-\max(-f, 0)\big) \cdot g_+                                     \\
            & = \big(-(-f)_+\big) \cdot g_+                         &  & \by{11.4.4} \\
            & = -\big((-f)_+ \cdot g_+\big)                         &  & \by{9.2.1}  \\
    f_- g_- & = \big(\min(f, 0)\big) \cdot \big(\min(g, 0)\big)     &  & \by{11.4.4} \\
            & = \big(-\max(-f, 0)\big) \cdot \big(-\max(-g, 0)\big)                  \\
            & = \big(-(-f)_+\big) \big(-(-g)_+\big)                 &  & \by{11.4.4} \\
            & = (-f)_+ \cdot (-g)_+                                 &  & \by{9.2.1}
  \end{align*}
  and thus \(f_+ g_-\), \(f_- g_+\), \(f_- g_-\) are Riemann integrable.
\end{proof}
