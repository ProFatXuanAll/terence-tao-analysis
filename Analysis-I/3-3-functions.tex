\section{Functions}\label{i:sec:3.3}

\begin{defn}[Functions]\label{i:3.3.1}
  Let \(X, Y\) be sets, and let \(P(x, y)\) be a property pertaining to an object \(x \in X\) and an object \(y \in Y\), such that for every \(x \in X\), there is exactly one \(y \in Y\) for which \(P(x, y)\) is true (this is sometimes known as the \emph{vertical line test}).
  Then we define the \emph{function \(f : X \to Y\) defined by \(P\) on the domain \(X\) and codomain \(Y\)} to be the object which, given any input \(x \in X\), assigns an output \(f(x) \in Y\), defined to be the unique object \(f(x) \in Y\) for which \(P(x, f(x))\) is true.
  Thus, for any \(x \in X\) and \(y \in Y\),
  \[
    y = f(x) \iff P(x, y) \text{ is true}.
  \]
\end{defn}

\begin{note}
  Implicit in \cref{i:3.3.1} is the assumption that whenever one is given two sets \(X, Y\) and a property \(P\) obeying the vertical line test, one can form a function object.
  Strictly speaking, this assumption of the existence of the function as a mathematical object should be stated as an explicit axiom;
  however we will not do so here, as it turns out to be redundant.
  (More precisely, in view of \cref{i:ex:3.5.10} below, it is always possible to encode a function \(f\) as an ordered triple \((X, Y, \set{(x, f(x)) : x \in X})\) consisting of the domain, codomain, and graph of the function, which gives a way to build functions as objects using the operations provided by the preceding axioms.)
\end{note}

\begin{note}
  Functions are also referred to as \emph{maps} or \emph{transformations}, depending on the context.
  They are also sometimes called \emph{morphisms}, although to be more precise, a morphism refers to a more general class of object, which may or may not correspond to actual functions, depending on the context.
\end{note}

\begin{note}
  One common way to define a function is simply to specify its domain, its codomain, and how one generates the output \(f(x)\) from each input;
  this is known as an \emph{explicit} definition of a function.
  In other cases we only define a function \(f\) by specifying what property \(P(x, y)\) links the input \(x\) with the output \(f(x)\);
  this is an \emph{implicit} definition of a function.
  An implicit definition is only valid if we know that for every input there is exactly one output which obeys the implicit relation.
\end{note}

\begin{note}
  In many cases we omit specifying the domain and codomain of a function for brevity.
  However, too much of this abbreviation can be dangerous;
  sometimes it is important to know what the domain and codomain of the function is.
\end{note}

\begin{ac}\label{i:ac:3.3.1}
  We observe that functions obey the axiom of substitution:
  if \(x = x'\), then \(f(x) = f(x')\).
  In other words, equal inputs imply equal outputs.
  On the other hand, unequal inputs do not necessarily ensure unequal outputs.
  For example, \emph{constant function} simply assign each input with the same output.
\end{ac}

\begin{proof}[\pf{i:ac:3.3.1}]
  Suppose that \(f : X \to Y\) is a function defined by \(P\) on the domain \(X\) and codomain \(Y\).
  Let \(x, x' \in X\) such that \(x = x'\).
  By \cref{i:3.3.1} we know that \(f(x)\) is the unique object for which \(P(x, f(x))\) is true.
  Similarly we know that \(f(x')\) is the unique object for which \(P(x', f(x'))\) is true.
  Since \(x = x'\), we know that \(P(x', f(x))\) must be true.
  Then the uniqueness of \(f(x')\) implies \(f(x) = f(x')\).
\end{proof}

\setcounter{thm}{4}
\begin{rmk}\label{i:3.3.5}
  We are now using parentheses () to denote several different things in mathematics;
  on one hand, we are using them to clarify the order of operations, but on the other hand we also use parentheses to enclose the argument of a function \(f(x)\) or of a property such as \(P(x)\).
  However, the two usages of parentheses usually are unambiguous from context.
  For instance, if \(a\) is a number, then \(a(b + c)\) denotes the expression \(a \times (b + c)\), whereas if \(f\) is a function, then \(f(b + c)\) denotes the output of \(f\) when the input is \(b + c\).
  Sometimes the argument of a function is denoted by subscripting instead of parentheses;
  for instance, a sequence of natural numbers \(a_0, a_1, a_2, a_3, \dots\) is, strictly speaking, a function from \(\N\) to \(\N\), but is denoted by \(n \mapsto a_n\) rather than \(n \mapsto a(n)\).
\end{rmk}

\begin{rmk}\label{i:3.3.6}
  Strictly speaking, functions are not necessarily sets, and sets are not necessarily functions;
  it does not make sense to ask whether an object \(x\) is an element of a function \(f\), and it does not make sense to apply a set \(A\) to an input \(x\) to create an output \(A(x)\).
  On the other hand, it is possible to start with a function \(f : X \to Y\) and construct its \emph{graph} \(\set{(x, f(x)) : x \in X}\), which describes the function completely once the domain \(X\) and codomain \(Y\) are specified.
  See \cref{i:sec:3.5}.
\end{rmk}

\begin{defn}[Equality of functions]\label{i:3.3.7}
  Two functions \(f : X \to Y, g : X' \to Y'\) are said to be \emph{equal} iff they have the same domain and codomain (i.e., \(X = X'\) and \(Y = Y'\)), and \(f(x) = g(x)\) for all \(x \in X\).
  (If \(f(x)\) and \(g(x)\) agree for some values of \(x\), but not others, then we do not consider \(f\) and \(g\) to be equal.)
  According to this definition, two functions that have different domains or different codomains are, strictly speaking, distinct functions.
  However, when it is safe to do so without causing confusion, it is sometimes useful to ``abuse notation'' by identifying together functions of different domains or codomains if their values agree on their common domain of definition;
  this is analogous to the practice of ``overloading'' an operator in software engineering.
  See the discussion after \cref{i:9.4.1} for an instance of this.
\end{defn}

\setcounter{thm}{8}
\begin{eg}\label{i:3.3.9}
  A rather boring example of a function is the \emph{empty function} \(f : \emptyset \to X\) from the empty set to a given set \(X\).
  Since the empty set has no elements, we do not need to specify what \(f\) does to any input.
  Nevertheless, just as the empty set is a set, the empty function is a function, albeit not a particularly interesting one.
  Note that for each set \(X\), there is only one function from \(\emptyset\) to \(X\), since \cref{i:3.3.7} asserts that all functions from \(\emptyset\) to \(X\) are equal.
\end{eg}

\begin{defn}[Composition]\label{i:3.3.10}
  Let \(f : X \to Y\) and \(g : Y \to Z\) be two functions, such that the codomain of \(f\) is the same set as the domain of \(g\).
  We then define the \emph{composition} \(g \circ f : X \to Z\) of the two functions \(g\) and \(f\) to be the function defined explicitly by the formula
  \[
    (g \circ f)(x) \coloneqq g(f(x)).
  \]
  If the codomain of \(f\) does not match the domain of \(g\), we leave the composition \(g \circ f\) undefined.
\end{defn}

\begin{note}
  Composition is not commutative:
  \(f \circ g\) and \(g \circ f\) are not necessarily the same function.
\end{note}

\setcounter{thm}{11}
\begin{lem}[Composition is associative]\label{i:3.3.12}
  Let \(f : Z \to W\), \(g : Y \to Z\), and \(h : X \to Y\) be functions.
  Then \(f \circ (g \circ h) = (f \circ g) \circ h\).
\end{lem}

\begin{proof}[\pf{i:3.3.12}]
  Since \(g \circ h\) is a function from \(X\) to \(Z\), \(f \circ (g \circ h)\) is a function from \(X\) to \(W\).
  Similarly \(f \circ g\) is a function from \(Y\) to \(W\), and hence \((f \circ g) \circ h\) is a function from \(X\) to \(W\).
  Thus \(f \circ (g \circ h)\) and \((f \circ g) \circ h\) have the same domain and codomain.
  In order to check that they are equal, we see from \cref{i:3.3.7} that we have to verify that \((f \circ (g \circ h))(x) = ((f \circ g) \circ h)(x)\) for all \(x \in X\).
  But by \cref{i:3.3.10}
  \begin{align*}
    (f \circ (g \circ h))(x)
     & = f((g \circ h)(x))        \\
     & = f(g(h(x)))               \\
     & = (f \circ g)(h(x))        \\
     & = ((f \circ g) \circ h)(x)
  \end{align*}
  as desired.
\end{proof}

\begin{rmk}\label{i:3.3.13}
  Note that while \(g\) appears to the left of \(f\) in the expression \(g \circ f\), the function \(g \circ f\) applies the right-most function \(f\) first, before applying \(g\).
  This is often confusing at first;
  it arises because we traditionally place a function \(f\) to the left of its input \(x\) rather than to the right.
  (There are some alternate mathematical notations in which the function is placed to the right of the input, thus we would write \(xf\) instead of \(f(x)\), but this notation has often proven to be more confusing than clarifying, and has not as yet become particularly popular.)
\end{rmk}

\begin{defn}[One-to-one function]\label{i:3.3.14}
  A function \(f\) is \emph{one-to-one} (or \emph{injective}) if different elements map to different elements:
  \[
    x \neq x' \implies f(x) \neq f(x').
  \]

  Equivalently, a function is one-to-one if
  \[
    f(x) = f(x') \implies x = x'.
  \]
\end{defn}

\begin{note}
  The notion of a one-to-one function depends not just on what the function does, but also what its domain is.
\end{note}

\setcounter{thm}{15}
\begin{rmk}\label{i:3.3.16}
  If a function \(f : X \to Y\) is not one-to-one, then one can find distinct \(x\) and \(x'\) in the domain \(X\) such that \(f(x) = f(x')\), thus one can find two inputs which map to one output.
  Because of this, we say that \(f\) is \emph{two-to-one} instead of \emph{one-to-one}.
\end{rmk}

\begin{defn}[Onto functions]\label{i:3.3.17}
  A function \(f\) is \emph{onto} (or \emph{surjective}) if \(f(X) = Y\), i.e., every element in \(Y\) comes from applying \(f\) to some element in \(X\):
  \[
    \text{For every } y \in Y, \text{there exists } x \in X \text{ such that } f(x) = y.
  \]
\end{defn}

\begin{note}
  The notion of an onto function depends not just on what the function does, but also what its codomain is.
\end{note}

\setcounter{thm}{18}
\begin{rmk}\label{i:3.3.19}
  The concepts of injectivity and surjectivity are in many ways dual to each other.
  See \cref{i:ex:3.3.2,i:ex:3.3.4,i:ex:3.3.5} for some evidence of this.
\end{rmk}

\begin{defn}[Bijective functions]\label{i:3.3.20}
  A function \(f : X \to Y\) which is both one-to-one and onto is also called \emph{bijective} or \emph{invertible}.

  If \(f\) is bijective, then for every \(y \in Y\), there is exactly one \(x\) such that \(f(x) = y\) (there is at least one because of surjectivity, and at most one because of injectivity).
  This value of \(x\) is denoted \(f^{-1}(y)\); thus \(f^{-1}\) is a function from \(Y\) to \(X\).
  We call \(f^{-1}\) the \emph{inverse} of \(f\).
\end{defn}

\begin{note}
  The notion of a bijective function depends not just on what the function does, but also what its domain and codomain are.
\end{note}

\setcounter{thm}{22}
\begin{rmk}\label{i:3.3.23}
  If a function \(x \mapsto f(x)\) is bijective, then we sometimes call \(f\) a \emph{perfect matching} or a \emph{one-to-one correspondence} (not to be confused with the notion of a one-to-one function), and denote the action of \(f\) using the notation \(x \leftrightarrow f(x)\) instead of \(x \mapsto f(x)\).
\end{rmk}

\exercisesection

\begin{ex}\label{i:ex:3.3.1}
  Show that the definition of equality in \cref{i:3.3.7} is reflexive, symmetric, and transitive.
  Also verify the substitution property: if \(f, \tilde{f} : X \to Y\) and \(g, \tilde{g} : Y \to Z\) are functions such that \(f = \tilde{f}\) and \(g = \tilde{g}\), then \(g \circ f = \tilde{g} \circ \tilde{f}\).
  Of course, these statements are immediate from the axioms of equality applied directly to the functions in question, but the point of the exercise is to show that they can also be established by instead applying the axioms of equality to elements of the domain and codomain of these functions, rather than to the functions itself.
\end{ex}

\begin{proof}[\pf{i:ex:3.3.1}]
  We first show that \cref{i:3.3.7} is reflexive.
  Suppose that \(f : X \to Y\) is a function.
  Then we have
  \begin{align*}
             & \begin{dcases}
                 X = X \\
                 Y = Y \\
                 \forall x \in X, f(x) = f(x)
               \end{dcases} &  & \by{i:ac:3.1.1,i:3.3.1}      \\
    \implies & f = f.                       &  & \by{i:3.3.7}
  \end{align*}
  Thus \cref{i:3.3.7} is reflexive.

  Next we show that \cref{i:3.3.7} is symmetric.
  Suppose that \(f : X \to Y, g : X' \to Y'\) are functions such that \(f = g\).
  Then we have
  \begin{align*}
         & f = g                                          \\
    \iff & \begin{dcases}
             X = X' \\
             Y = Y' \\
             \forall x \in X, f(x) = g(x)
           \end{dcases} &  & \by{i:3.3.7}                 \\
    \iff & \begin{dcases}
             X' = X \\
             Y' = Y \\
             \forall x \in X, g(x) = f(x)
           \end{dcases} &  & \by{i:ac:3.1.1}              \\
    \iff & g = f.                       &  & \by{i:3.3.7}
  \end{align*}
  Thus \cref{i:3.3.7} is symmetric.

  Next we show that \cref{i:3.3.7} is transitive.
  Suppose that \(f : X_f \to Y_f, g : X_g \to Y_g, h : X_h \to Y_h\) are functions such that \((f = g) \land (g = h)\).
  Then we have
  \begin{align*}
             & \begin{dcases}
                 f = g \\
                 g = h
               \end{dcases}                                       \\
    \implies & \begin{dcases}
                 X_f = X_g                      \\
                 Y_f = Y_g                      \\
                 \forall x \in X_f, f(x) = g(x) \\
                 X_g = X_h                      \\
                 Y_g = Y_h                      \\
                 \forall x \in X_g, g(x) = h(x)
               \end{dcases} &  & \by{i:3.3.7}                      \\
    \implies & \begin{dcases}
                 X_f = X_h \\
                 Y_f = Y_h \\
                 \forall x \in X_f, f(x) = g(x)
               \end{dcases}    &  & \by{i:ac:3.1.1}                \\
    \implies & f = h.                            &  & \by{i:3.3.7}
  \end{align*}
  Thus \cref{i:3.3.7} is transitive.

  Finally we show that Axiom of substitution holds for composition.
  Suppose that \(f : X \to Y, \tilde{f} : X \to Y, g : Y \to Z, \tilde{g} : Y \to Z\) are functions such that \((f = \tilde{f}) \land (g = \tilde{g})\).
  By \cref{i:3.3.10} \(g \circ f : X \to Z\) and \(\tilde{g} \circ \tilde{f} : X \to Z\) are well-defined.
  Since \(X = X\) and \(Z = Z\), to prove that \(g \circ f = \tilde{g} \circ \tilde{f}\), by \cref{i:3.3.7} we only need to show that \((g \circ f)(x) = \pa{\tilde{g} \circ \tilde{f}}(x)\) for all \(x \in X\).
  But this is true since.
  \begin{align*}
    \forall x \in X, (g \circ f)(x) & = g(f(x))                            &  & \by{i:3.3.10} \\
                                    & = g\pa{\tilde{f}(x)}                 &  & \by{i:3.3.7}  \\
                                    & = \tilde{g}\pa{\tilde{f}(x)}         &  & \by{i:3.3.7}  \\
                                    & = \pa{\tilde{g} \circ \tilde{f}}(x). &  & \by{i:3.3.10}
  \end{align*}
\end{proof}

\begin{ex}\label{i:ex:3.3.2}
  Let \(f : X \to Y\) and \(g : Y \to Z\) be functions.
  Show that if \(f\) and \(g\) are both injective, then so is \(g \circ f\);
  similarly, show that if \(f\) and \(g\) are both surjective, then so is \(g \circ f\).
\end{ex}

\begin{proof}[\pf{i:ex:3.3.2}]
  We first show that \(f, g\) are injective implies \(g \circ f\) is injective.
  Suppose that \(f : X \to Y, g : Y \to Z\) are injective functions.
  Then we have
  \begin{align*}
             & \forall x, x' \in X : x \neq x'                         \\
    \implies & f(x) \neq f(x')                      &  & \by{i:3.3.14} \\
    \implies & g(f(x)) \neq g(f(x'))                &  & \by{i:3.3.14} \\
    \implies & (g \circ f)(x) \neq (g \circ f)(x'). &  & \by{i:3.3.10}
  \end{align*}
  Thus by \cref{i:3.3.14} \(g \circ f\) is injective.

  Now we show that \(f, g\) are surjective implies \(g \circ f\) is surjective.
  Suppose that \(f : X \to Y, g : Y \to Z\) are surjective functions.
  Then we have
  \begin{align*}
             & \begin{dcases}
                 \forall z \in Z, \exists y \in Y : z = g(y) \\
                 \forall w \in Y, \exists x \in X : w = f(x)
               \end{dcases}                   &  & \by{i:3.3.17}                                   \\
    \implies & \forall z \in Z, \exists x \in X : z = g(f(x)) = (g \circ f)(x). &  & \by{i:3.3.10}
  \end{align*}
  Thus by \cref{i:3.3.17} \(g \circ f\) is surjective.
\end{proof}

\begin{ex}\label{i:ex:3.3.3}
  When is the empty function into a given set \(X\) injective?
  surjective?
  bijective?
\end{ex}

\begin{proof}[\pf{i:ex:3.3.3}]
  Suppose that \(f : \emptyset \to X\) is the empty function given set \(X\).
  \(f\) is always injective since the statement ``for all \(x, x' \in \emptyset\), \(f(x) = f(x') \implies x = x'\)'' is vacuously true by \cref{i:3.2}.
  For surjective we can split into two cases:
  \begin{itemize}
    \item If \(X \neq \emptyset\), then \(f\) is not surjective, since \(\forall y \in X\), \(\nexists x \in \emptyset\) such that \(f(x) = y\).
    \item If \(X = \emptyset\), then \(f\) is surjective, since \(\forall y \in \emptyset\), \(\exists x \in \emptyset\) such that \(f(x) = y\) (which is vacuously true).
  \end{itemize}
  From proof above we see that the empty function \(f\) is bijective iff \(X = \emptyset\).
\end{proof}

\begin{ex}\label{i:ex:3.3.4}
  In this section we give some cancellation laws for composition.
  Let \(f : X \to Y\), \(\tilde{f} : X \to Y\), \(g : Y \to Z\), and \(\tilde{g} : Y \to Z\) be functions.
  Show that if \(g \circ f = g \circ \tilde{f}\) and g is injective, then \(f = \tilde{f}\).
  Is the same statement true if \(g\) is not injective?
  Show that if \(g \circ f = \tilde{g} \circ f\) and \(f\) is surjective, then \(g = \tilde{g}\).
  Is the same statement true if \(f\) is not surjective?
\end{ex}

\begin{proof}[\pf{i:ex:3.3.4}]
  We first show that \(g\) is injective and \(g \circ f = g \circ \tilde{f}\) implies \(f = \tilde{f}\).
  Suppose that \(f : X \to Y, \tilde{f} : X \to Y, g : Y \to Z\) are functions such that \(g\) is injective and \(g \circ f = g \circ \tilde{f}\).
  Then we have
  \begin{align*}
             & g \circ f = g \circ \tilde{f}                                                  \\
    \implies & \forall x \in X, (g \circ f)(x) = \pa{g \circ \tilde{f}}(x) &  & \by{i:3.3.7}  \\
    \implies & \forall x \in X, g(f(x)) = g\pa{\tilde{f}(x)}               &  & \by{i:3.3.10} \\
    \implies & \forall x \in X, f(x) = \tilde{f}(x)                        &  & \by{i:3.3.14} \\
    \implies & f = \tilde{f}.                                              &  & \by{i:3.3.7}
  \end{align*}
  The statement is not true when \(g\) is not injective.
  For example, define \(f = x \mapsto x, \tilde{f} = x \mapsto \abs{x}, g = x \mapsto x^2\).
  Then we see that \(g \circ f = x \mapsto x^2 = x \mapsto \abs{x}^2 = g \circ \tilde{f}\).
  But \(f(-1) = -1 \neq \abs{-1} = \tilde{f}(1)\) implies \(f \neq \tilde{f}\).

  Now we show that \(f\) is surjective and \(g \circ f = \tilde{g} \circ f\) implies \(g = \tilde{g}\).
  Suppose that\(f : X \to Y, g : Y \to Z, \tilde{g} : Y \to Z\) are functions such that \(f\) is surjective and \(g \circ f = \tilde{g} \circ f\).
  Then we have
  \begin{align*}
             & \forall y \in Y, \exists x \in X : y = f(x)       &  & \by{i:3.3.17} \\
    \implies & \forall y \in Y, \exists x \in X : g(y) = g(f(x)) &  & \by{i:3.3.1}  \\
             & = (g \circ f)(x)                                  &  & \by{i:3.3.10} \\
             & = (\tilde{g} \circ f)(x)                          &  & \by{i:3.3.7}  \\
             & = \tilde{g}(f(x)) = \tilde{g}(y)                  &  & \by{i:3.3.10} \\
    \implies & g = \tilde{g}.                                    &  & \by{i:3.3.7}
  \end{align*}
  The statement is not true when \(f\) is not surjective.
  For example, define \(f = x \mapsto \abs{x}, g = x \mapsto x, \tilde{g} = x \mapsto \abs{x}\).
  Then we see that \(g \circ f = x \mapsto \abs{x} = x \mapsto \abs{(\abs{x})} = \tilde{g} \circ f\).
  But \(g(-1) = -1 \neq \abs{-1} = \tilde{g}(-1)\) implies \(g \neq \tilde{g}\).
\end{proof}

\begin{ex}\label{i:ex:3.3.5}
  Let \(f : X \to Y\) and \(g : Y \to Z\) be functions.
  Show that if \(g \circ f\) is injective, then \(f\) must be injective.
  Is it true that \(g\) must also be injective?
  Show that if \(g \circ f\) is surjective, then \(g\) must be surjective.
  Is it true that \(f\) must also be surjective?
\end{ex}

\begin{proof}[\pf{i:ex:3.3.5}]
  We first show that \(g \circ f\) is injective implies \(f\) is injective.
  Suppose \(f : X \to Y, g : Y \to Z\) are functions such that \(g \circ f\) is injective.
  Then we have
  \begin{align*}
             & \forall x, x' \in X, x \neq x'                                                       \\
    \implies & g(f(x)) = (g \circ f)(x) \neq (g \circ f)(x') = g(f(x')) &  & \by{i:3.3.10,i:3.3.14} \\
    \implies & f(x) \neq f(x').                                         &  & \by{i:ac:3.3.1}
  \end{align*}
  Thus by \cref{i:3.3.14} \(f\) is injective.
  And we don't need \(g\) to be injective to finish the proof.

  Now we show that \(g \circ f\) is surjective implies \(g\) is surjective.
  Suppose \(f : X \to Y, g : Y \to Z\) are functions such that \(g \circ f\) is surjective.
  Then we have
  \begin{align*}
             & \forall z \in Z, \exists x \in X : z = (g \circ f)(x) = g(f(x)) &  & \by{i:3.3.10,i:3.3.17} \\
    \implies & \forall z \in Z, \exists f(x) \in Y : z = g(f(x))               &  & \by{i:3.3.1}           \\
    \implies & g \text{ is surjective}.                                        &  & \by{i:3.3.17}
  \end{align*}
  And we don't need \(f\) to be surjective to finish the proof.
\end{proof}

\begin{ex}\label{i:ex:3.3.6}
  Let \(f : X \to Y\) be a bijective function, and let \(f^{-1} : Y \to X\) be its inverse.
  Verify the cancellation laws \(f^{-1}(f(x)) = x\) for all \(x \in X\) and \(f\pa{f^{-1}(y)} = y\) for all \(y \in Y\).
  Conclude that \(f^{-1}\) is also invertible, and has \(f\) as its inverse (thus \(\pa{f^{-1}}^{-1} = f\)).
\end{ex}

\begin{proof}[\pf{i:ex:3.3.6}]
  We first show that \(f^{-1}(f(x)) = x\) for all \(x \in X\).
  \begin{align*}
             & \forall x \in X, \exists! y \in Y : \begin{dcases}
                                                     f(x) = y \\
                                                     f^{-1}(y) = x
                                                   \end{dcases}              &  & \by{i:3.3.20}     \\
    \implies & \forall x \in X, \exists! y \in Y : f^{-1}\pa{f(x)} = f(y) = x. &  & \by{i:ac:3.3.1}
  \end{align*}

  Next we show that \(f\pa{f^{-1}(y)} = y\) for all \(y \in Y\).
  \begin{align*}
             & \forall y \in Y, \exists! x \in X : \begin{dcases}
                                                     f^{-1}(y) = x \\
                                                     f(x) = y
                                                   \end{dcases}              &  & \by{i:3.3.20}     \\
    \implies & \forall y \in Y, \exists! x \in X : f\pa{f^{-1}(y)} = f(x) = y. &  & \by{i:ac:3.3.1}
  \end{align*}

  Next we show that \(f^{-1}\) is bijective.
  Since
  \begin{align*}
             & \forall y, y' \in Y, y \neq y'                         \\
    \implies & \begin{dcases}
                 \exists! x \in X : f(x) = y    \\
                 \exists! x' \in X : f(x') = y' \\
                 x \neq x'
               \end{dcases}   &  & \by{i:3.3.20}                      \\
    \implies & x = f^{-1}(y) \neq f^{-1}(y') = x', &  & \by{i:3.3.20}
  \end{align*}
  by \cref{i:3.3.14} we know that \(f^{-1}\) is injective.
  Since
  \begin{align*}
             & \forall x \in X, \exists! y \in Y : f(x) = y      &  & \by{i:3.3.1}  \\
    \implies & \forall x \in X, \exists y \in Y : f^{-1}(y) = x, &  & \by{i:3.3.20}
  \end{align*}
  by \cref{i:3.3.17} we know that \(f^{-1}\) is surjective.
  Since \(f^{-1}\) is both injective and surjective, by \cref{i:3.3.20} we know that \(f^{-1}\) is bijective.

  Finally we show that \(\pa{f^{-1}}^{-1} = f\).
  Clearly \(\pa{f^{-1}}^{-1} : X \to Y\) has the same domain and codomain as \(f\).
  Since
  \begin{align*}
             & \forall x \in X, \exists! y \in Y : \begin{dcases}
                                                     f(x) = y      \\
                                                     f^{-1}(y) = x \\
                                                     \pa{f^{-1}}^{-1}(x) = y
                                                   \end{dcases} &  & \by{i:3.3.20} \\
    \implies & \forall x \in X, f(x) = \pa{f^{-1}}^{-1}(x),
  \end{align*}
  by \cref{i:3.3.7} we know that \(f = \pa{f^{-1}}^{-1}\).
\end{proof}

\begin{ex}\label{i:ex:3.3.7}
  Let \(f : X \to Y\) and \(g : Y \to Z\) be functions.
  Show that if \(f\) and \(g\) are bijective, then so is \(g \circ f\), and we have \((g \circ f)^{-1} = f^{-1} \circ g^{-1}\).
\end{ex}

\begin{proof}[\pf{i:ex:3.3.7}]
  Let \(f : X \to Y, g : Y \to Z\) be bijectives.
  First we show that \(g \circ f\) is bijective.
  This is true since
  \begin{align*}
             & f, g \text{ are bijective}                                                                    \\
    \implies & (f, g \text{ are injective}) \land (f, g \text{ are surjective})         &  & \by{i:3.3.20}   \\
    \implies & (g \circ f \text{ is injective}) \land (g \circ f \text{ is surjective}) &  & \by{i:ex:3.3.2} \\
    \implies & g \circ f \text{ is bijective}.                                          &  & \by{i:3.3.20}
  \end{align*}

  Now we show that \((g \circ f)^{-1} = f^{-1} \circ g^{-1}\).
  From proof above we know that \(g \circ f\) is bijective so \((g \circ f)^{-1}\) is well-defined.
  Since \(g \circ f\) has domain \(X\) and codomain \(Z\), by \cref{i:3.3.20} we know that \((g \circ f)^{-1}\) has domain \(Z\) and codomain \(X\).
  By \cref{i:3.3.20} we know that \(g^{-1} : Z \to Y\) and \(f^{-1} : Y \to X\) are well-defined, thus by \cref{i:3.3.10} we know that \(f^{-1} \circ g^{-1} : Z \to X\) is well-defined.
  Since \((g \circ f)^{-1}\) and \(f^{-1} \circ g^{-1}\) both have the same domain and codomain, by \cref{i:3.3.7} we only need to show that \((g \circ f)^{-1}(z) = (f^{-1} \circ g^{-1})(z)\) for every \(z \in Z\).
  Since
  \begin{align*}
    \forall x \in X, \pa{\pa{f^{-1} \circ g^{-1}} \circ (g \circ f)}(x) & = \pa{f^{-1} \circ \pa{g^{-1} \circ \pa{g \circ f}}}(x) &  & \by{i:3.3.12}   \\
                                                                        & = f^{-1}\pa{g^{-1}\pa{g(f(x))}}                         &  & \by{i:3.3.10}   \\
                                                                        & = f^{-1}(f(x))                                          &  & \by{i:ex:3.3.6} \\
                                                                        & = x,                                                    &  & \by{i:ex:3.3.6}
  \end{align*}
  we see that
  \begin{align*}
    \forall z \in Z, & \pa{f^{-1} \circ g^{-1}}(z)                                                                                         \\
                     & = \pa{f^{-1} \circ g^{-1}}\pa{\pa{(g \circ f) \circ (g \circ f)^{-1}}(z)}        &  & \by{i:ex:3.3.6}               \\
                     & = \pa{\pa{\pa{f^{-1} \circ g^{-1}} \circ (g \circ f)} \circ (g \circ f)^{-1}}(z) &  & \by{i:3.3.12}                 \\
                     & = \pa{\pa{f^{-1} \circ g^{-1}} \circ (g \circ f)}\pa{(g \circ f)^{-1}(z)}        &  & \by{i:3.3.10}                 \\
                     & = (g \circ f)^{-1}(z).                                                           &  & \text{(from the proof above)}
  \end{align*}
  Thus by \cref{i:3.3.7} we conclude that \((g \circ f)^{-1} = f^{-1} \circ g^{-1}\).
\end{proof}

\begin{ex}\label{i:ex:3.3.8}
  If \(X\) is a subset of \(Y\), let \(\iota_{X \to Y} : X \to Y\) be the \emph{inclusive map from \(X\) to \(Y\)}, defined by mapping \(x \mapsto x\) for all \(x \in X\), i.e., \(\iota_{X \to Y}(x) \coloneqq x\) for all \(x \in X\).
  The map \(\iota_{X \to X}\) is in particular called the \emph{identity map} on \(X\).
  \begin{enumerate}
    \item Show that if \(X \subseteq Y \subseteq Z\) then \(\iota_{Y \to Z} \circ \iota_{X \to Y} = \iota_{X \to Z}\).
    \item Show that if \(f : A \to B\) is any function, then \(f = f \circ \iota_{A \to A} = \iota_{B \to B} \circ f\).
    \item Show that if \(f : A \to B\) is a bijective function, then \(f \circ f^{-1} = \iota_{B \to B}\) and \(f^{-1} \circ f = \iota_{A \to A}\).
    \item Show that if \(X\) and \(Y\) are disjoint sets, and \(f : X \to Z\) and \(g : Y \to Z\) are functions, then there is a unique function \(h : X \cup Y \to Z\) such that \(h \circ \iota_{X \to X \cup Y} = f\) and \(h \circ \iota_{Y \to X \cup Y} = g\).
  \end{enumerate}
\end{ex}

\begin{proof}[\pf{i:ex:3.3.8}(a)]
  If \(X, Y, Z\) are sets such that \(X \subseteq Y \subseteq Z\), then we have
  \begin{align*}
    \forall x \in X, (\iota_{Y \to Z} \circ \iota_{X \to Y})(x) & = \iota_{Y \to Z}(\iota_{X \to Y}(x)) &  & \by{i:3.3.10}   \\
                                                                & = \iota_{Y \to Z}(x)                  &  & \by{i:ex:3.3.8} \\
                                                                & = x                                   &  & \by{i:ex:3.3.8} \\
                                                                & = \iota_{X \to Z}(x).                 &  & \by{i:ex:3.3.8}
  \end{align*}
  Thus by \cref{i:3.3.7,i:3.3.10} we have \(\iota_{Y \to Z} \circ \iota_{X \to Y} = \iota_{X \to Z}\).
\end{proof}

\begin{proof}[\pf{i:ex:3.3.8}(b)]
  Let \(f : A \to B\) be a function.
  Then we have
  \begin{align*}
    \forall a \in A, f(a) & = \begin{dcases}
                                f(\iota_{A \to A}(a)) \\
                                \iota_{B \to B}(f(a))
                              \end{dcases}        &  & \by{i:ex:3.3.8} \\
                          & = \begin{dcases}
                                (f \circ \iota_{A \to A})(a) \\
                                (\iota_{B \to B} \circ f)(a)
                              \end{dcases}. &  & \by{i:3.3.10}
  \end{align*}
  Thus by \cref{i:3.3.7,i:3.3.10} we have \(f = f \circ \iota_{A \to A} = \iota_{B \to B} \circ f\).
\end{proof}

\begin{proof}[\pf{i:ex:3.3.8}(c)]
  Suppose \(f : A \to B\) is bijective.
  Then we have
  \begin{align*}
    \forall a \in A, a & = f^{-1}(f(a))        &  & \by{i:ex:3.3.6} \\
                       & = (f^{-1} \circ f)(a) &  & \by{i:3.3.10}   \\
                       & = \iota_{A \to A}(a). &  & \by{i:ex:3.3.8} \\
    \forall b \in B, b & = f(f^{-1}(b))        &  & \by{i:ex:3.3.6} \\
                       & = (f \circ f^{-1})(b) &  & \by{i:3.3.10}   \\
                       & = \iota_{B \to B}(b). &  & \by{i:ex:3.3.8}
  \end{align*}
  Thus by \cref{i:3.3.7,i:3.3.10} we have \(f^{-1} \circ f = \iota_{A \to A}\) and \(f \circ f^{-1} = \iota_{B \to B}\).
\end{proof}

\begin{proof}[\pf{i:ex:3.3.8}(d)]
  Suppose that \(X, Y, Z\) are sets such that \(X \cap Y = \emptyset\).
  Let \(f : X \to Z, g : Y \to Z\) be functions.
  We now define a function \(h : X \cup Y \to Z\) as follow:
  \[
    \forall w \in X \cup Y, h(w) = \begin{dcases}
      f(w) & \text{ if } w \in X \\
      g(w) & \text{ if } w \in Y
    \end{dcases}.
  \]
  This function is well-defined since \(X \cap Y = \emptyset\) and thus each \(w \in X \cup Y\) can be either in \(X\) or \(Y\) but not both.
  Then we have
  \begin{align*}
    \forall w \in X, h(w) & = h(\iota_{X \to X \cup Y}(w))        &  & \by{i:ex:3.3.8}                     \\
                          & = (h \circ \iota_{X \to X \cup Y})(w) &  & \by{i:3.3.10}                       \\
                          & = f(w).                               &  & \text{(by the definition of \(h\))} \\
    \forall w \in Y, h(w) & = h(\iota_{Y \to X \cup Y}(w))        &  & \by{i:ex:3.3.8}                     \\
                          & = (h \circ \iota_{Y \to X \cup Y})(w) &  & \by{i:3.3.10}                       \\
                          & = g(w).                               &  & \text{(by the definition of \(h\))}
  \end{align*}
  Thus by \cref{i:3.3.7,i:3.3.10} we have \(h \circ \iota_{X \to X \cup Y} = f\) and \(h \circ \iota_{Y \to X \cup Y} = g\).

  Now suppose there exists another function \(h' : X \cup Y \to Z\) such that \(h' \circ \iota_{X \to X \cup Y} = f\) and \(h' \circ \iota_{Y \to X \cup Y} = g\).
  Then we have
  \begin{align*}
    \forall x \in X, f(x)  & = (h' \circ \iota_{X \to X \cup Y})(x)                    \\
                           & = h'(\iota_{X \to X \cup Y}(x))        &  & \by{i:3.3.10} \\
                           & = h'(x)                                                   \\
                           & = (h \circ \iota_{X \to X \cup Y})(x)                     \\
                           & = h(\iota_{X \to X \cup Y}(x))         &  & \by{i:3.3.10} \\
                           & = h(x).                                                   \\
    \forall y \in Y : g(y) & = (h' \circ \iota_{Y \to X \cup Y})(y)                    \\
                           & = h'(\iota_{Y \to X \cup Y}(y))        &  & \by{i:3.3.10} \\
                           & = h'(y)                                                   \\
                           & = (h \circ \iota_{Y \to X \cup Y})(y)                     \\
                           & = h(\iota_{Y \to X \cup Y}(y))         &  & \by{i:3.3.10} \\
                           & = h(y).
  \end{align*}
  Thus by \cref{i:3.3.7} we have \(h = h'\), so \(h\) is unique.
\end{proof}
