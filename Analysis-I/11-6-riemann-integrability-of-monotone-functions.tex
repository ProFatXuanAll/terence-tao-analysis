\section{Riemann integrability of monotone functions}\label{sec:11.6}

\begin{prop}\label{11.6.1}
  Let \([a, b]\) be a closed and bounded interval and let \(f : [a, b] \to \R\) be a monotone function.
  Then \(f\) is Riemann integrable on \([a, b]\).
\end{prop}

\begin{proof}
  Without loss of generality we may take \(f\) to be monotone increasing (instead of monotone decreasing).
  From \cref{ex:9.8.1} we know that \(f\) is bounded.
  Now let \(N > 0\) be an integer, and partition \([a, b]\) into \(N\) half-open intervals
  \[
    \bigg\{\big[a + \frac{b - a}{N} j, a + \frac{b - a}{N} (j + 1)\big) : 0 \leq j \leq N - 1\bigg\}
  \]
  of length \((b - a) / N\), together with the point \(\{b\}\).
  Then by \cref{11.3.12} we have
  \[
    \overline{\int}_I f \leq \sum_{j = 0}^{N - 1} \Bigg(\sup_{x \in \big[a + \frac{b - a}{N} j, a + \frac{b - a}{N} (j + 1)\big)} f(x)\Bigg) \frac{b - a}{N},
  \]
  (the point \(\{b\}\) clearly giving only a zero contribution).
  Since \(f\) is monotone increasing, we thus have
  \[
    \overline{\int}_I f \leq \sum_{j = 0}^{N - 1} f\bigg(a + \frac{b - a}{N} (j + 1)\bigg) \frac{b - a}{N}.
  \]
  Similarly we have
  \[
    \underline{\int}_I f \geq \sum_{j = 0}^{N - 1} f\bigg(a + \frac{b - a}{N} j\bigg) \frac{b - a}{N}.
  \]
  Thus we have
  \[
    \overline{\int}_I f - \underline{\int}_I f \leq \sum_{j = 0}^{N - 1} \Bigg(f\bigg(a + \frac{b - a}{N} (j + 1)\bigg) - f\bigg(a + \frac{b - a}{N} j\bigg)\Bigg) \frac{b - a}{N}.
  \]
  Using telescoping series (\cref{7.2.15}) we thus have
  \begin{align*}
    \overline{\int}_I f - \underline{\int}_I f & \leq \Bigg(f\bigg(a + \frac{b - a}{N} N\bigg) - f\bigg(a + \frac{b - a}{N} 0\bigg)\Bigg) \frac{b - a}{N} \\
                                               & = \big(f(b) - f(a)\big) \frac{b - a}{N}.
  \end{align*}
  But \(N\) was arbitrary, so we can conclude as in the proof of \cref{11.5.1} that \(f\) is Riemann integrable.
\end{proof}

\begin{rmk}\label{11.6.2}
  From \cref{ex:9.8.5} we know that there exist monotone functions which are not piecewise continuous, so \cref{11.6.1} is not subsumed by \cref{11.5.6}.
\end{rmk}

\begin{cor}\label{11.6.3}
  Let \(I\) be a bounded interval, and let \(f : I \to \R\) be both monotone and bounded.
  Then \(f\) is Riemann integrable on \(I\).
\end{cor}

\begin{proof}
  Without loss of generality we may take \(f\) to be monotone increasing (instead of monotone decreasing).
  If \(I\) is a point or an empty set then the claim is trivial;
  if \(I\) is a closed interval the claim follows from \cref{11.6.1}.
  So let us assume that \(I\) is of the form \((a, b]\), \((a, b)\), or \([a, b)\) for some \(a < b\).

  We have a bound \(M\) for \(f\), so that \(-M \leq f(x) \leq M\) for all \(x \in I\).
  Now let \(0 < \varepsilon < (b - a) / 2\) be a small number.
  The function \(f\) when restricted to the interval \([a + \varepsilon, b - \varepsilon]\) is monotone, and hence Riemann integrable by \cref{11.6.1}.
  In particular, we can find a piecewise constant function \(h : [a + \varepsilon, b - \varepsilon] \to \R\) which majorizes \(f\) on \([a + \varepsilon, b - \varepsilon]\) such that
  \[
    \int_{[a + \varepsilon, b - \varepsilon]} h \leq \int_{[a + \varepsilon, b - \varepsilon]} f + \varepsilon.
  \]
  Define \(\tilde{h} : I \to \R\) by
  \[
    \tilde{h}(x) \coloneqq \begin{cases}
      h(x) & \text{if } x \in [a + \varepsilon, b - \varepsilon]             \\
      M    & \text{if } x \in I \setminus [a + \varepsilon, b - \varepsilon]
    \end{cases}
  \]
  Clearly \(\tilde{h}\) is piecewise constant on \(I\) and majorizes \(f\);
  by \cref{11.2.16} we have
  \[
    \int_I \tilde{h} = \varepsilon M + \int_{[a + \varepsilon, b - \varepsilon]} h + \varepsilon M \leq \int_{[a + \varepsilon, b - \varepsilon]} f + (2M + 1) \varepsilon.
  \]
  In particular we have
  \[
    \overline{\int}_I f \leq \int_{[a + \varepsilon, b - \varepsilon]} f + (2M + 1) \varepsilon
  \]
  This is true since \(\tilde{h}\) majorize \(f\).
  A similar argument gives
  \[
    \underline{\int}_I f \geq \int_{[a + \varepsilon, b - \varepsilon]} f - (2M + 1) \varepsilon.
  \]
  and hence
  \[
    \overline{\int}_I f - \underline{\int}_I f \leq (4M + 2) \varepsilon.
  \]
  But \(\varepsilon\) is arbitrary, and so we can argue as in the proof of \cref{11.5.1} to conclude Riemann integrability.
\end{proof}

\begin{prop}[Integral test]\label{11.6.4}
  Let \(f : [0, \infty) \to \R\) be a monotone decreasing function which is non-negative
  (i.e., \(f(x) \geq 0\) for all \(x \geq 0\)).
  Then the sum \(\sum_{n = 0}^\infty f(n)\) is convergent if and only if \(\sup_{N > 0} \int_{[0, N]} f\) is finite.
\end{prop}

\begin{proof}
  Let \(N \in \Z^+\).
  Since \(f\) is monotone decreasing, by \cref{11.6.1} we know that \(f\) is Riemann integrable on both \([0, N]\) and every interval \([a, b] \subseteq [0, N]\).
  Then we have
  \begin{align*}
    \int_{[0, N]} f & = \sum_{n = 0}^{N - 1} \int_{[n, n + 1)} f|_{[n, n + 1)} + \int_{[N, N]} f|_{[N, N]} & \text{(by \cref{ex:11.4.3})}             \\
                    & = \sum_{n = 0}^{N - 1} \int_{[n, n + 1)} f|_{[n, n + 1)}                             & \text{(by \cref{11.1.8})}                \\
                    & \leq \sum_{n = 0}^{N - 1} \int_{[n, n + 1)} f(n)                                     & \text{(by \cref{11.4.1}(e))}             \\
                    & = \sum_{n = 0}^{N - 1} f(n) \abs{n + 1 - n}                                          & \text{(by \cref{11.2.9})}                \\
                    & = \sum_{n = 0}^{N - 1} f(n)                                                                                                     \\
                    & \leq \sum_{n = 0}^N f(n)                                                             & (\forall x \in [0, \infty), f(x) \geq 0)
  \end{align*}
  and
  \begin{align*}
    \int_{[0, N]} f & = \sum_{n = 0}^{N - 1} \int_{[n, n + 1)} f|_{[n, n + 1)} + \int_{[N, N]} f|_{[N, N]} & \text{(by \cref{ex:11.4.3})} \\
                    & = \sum_{n = 0}^{N - 1} \int_{[n, n + 1)} f|_{[n, n + 1)}                             & \text{(by \cref{11.1.8})}    \\
                    & \geq \sum_{n = 0}^{N - 1} \int_{[n, n + 1)} f(n + 1)                                 & \text{(by \cref{11.4.1}(e))} \\
                    & = \sum_{n = 0}^{N - 1} f(n + 1) \abs{n + 1 - n}                                      & \text{(by \cref{11.2.9})}    \\
                    & = \sum_{n = 0}^{N - 1} f(n + 1)                                                                                     \\
                    & = \sum_{n = 1}^N f(n).                                                               & \text{(by \cref{7.1.4}(b))}
  \end{align*}

  Next we show that if \(\sum_{n = 0}^\infty f(n)\) is convergent, then \(\sup_{N > 0} \int_{[0, N]} f\) is finite.
  Suppose that \(\sum_{n = 0}^\infty f(n)\) is convergent.
  Then by \cref{7.2.2} we know that
  \[
    \sum_{n = 0}^\infty f(n) = \lim_{m \to \infty} \sum_{n = 0}^m f(n)
  \]
  and by \cref{6.1.12} \(\big(\sum_{n = 0}^m f(n)\big)_{m = 0}^\infty\) is a Cauchy sequence.
  By \cref{5.1.15} we know that \(\big(\sum_{n = 0}^m f(n)\big)_{m = 0}^\infty\) is bounded by some \(M \in \R\).
  By comparison principle (\cref{6.4.13}) we have
  \[
    \int_{[0, N]} f \leq \sum_{n = 0}^N f(n) \implies \sup\bigg(\int_{[0, N]} f\bigg)_{N = 1}^\infty \leq \sup\bigg(\sum_{n = 0}^N f(n)\bigg)_{N = 1}^\infty \leq M
  \]
  and thus \(\sup_{N > 0} \int_{[0, N]} f\) is finite.

  Now we show that if \(\sup_{N > 0} \int_{[0, N]} f\) is finite, then \(\sum_{n = 0}^\infty f(n)\) is convergent.
  Suppose that \(\sup_{N > 0} \int_{[0, N]} f\) is finite.
  By comparison principle (\cref{6.4.13}) we have
  \[
    \sum_{n = 1}^N f(n) \leq \int_{[0, N]} f \implies \sup\bigg(\sum_{n = 1}^N f(n)\bigg)_{N = 1}^\infty \leq \sup\bigg(\int_{[0, N]} f\bigg)_{N = 1}^\infty
  \]
  Thus by \cref{7.3.1} \(\sum_{n = 0}^\infty f(n)\) is convergent.
\end{proof}

\begin{cor}\label{11.6.5}
  Let \(p\) be a real number.
  Then \(\sum_{n = 1}^\infty \frac{1}{n^p}\) converges absolutely when \(p > 1\) and diverges when \(p \leq 1\).
\end{cor}

\begin{proof}
  Let \(f : [1, \infty) \to \R\) be the function \(f(x) = \frac{1}{x^p}\).
  By \cref{6.7.3}(a)(d) we know that \(f\) is positive and
  \[
    \begin{cases}
      f \text{ is monotone decreasing if } p > 1; \\
      f \text{ is monotone increasing if } p < 1; \\
      f \text{ is both monotone increasing and decreasing if } p = 1.
    \end{cases}
  \]
  By \cref{11.6.1} \(f\) is Riemann integrable on \([1, N]\) for every \(N \in \R\) and \(N \geq 1\).
  If \(p \neq 1\), then we have
  \begin{align*}
    \int_{[1, N]} f & = \frac{1}{1 - p} (N^{1 - p} - 1^{1 - p}) & \text{(by \cref{11.9.4})} \\
                    & = \frac{1}{1 - p} (N^{1 - p} - 1).
  \end{align*}
  If \(p = 1\), then we have
  \[
    \int_{[1, N]} f = \ln N - \ln 1 = \ln N.
  \]
  Note that we use \cref{11.9.4} and logarithm without circularity.

  First suppose that \(p > 1\).
  Since
  \begin{align*}
    \int_{[1, N]} f & = \frac{1}{1 - p} (N^{1 - p} - 1)                      \\
                    & = \frac{1}{p - 1} (1 - N^{1 - p})                      \\
                    & \leq \frac{1}{p - 1}              & (N^{1 - p} \leq 1)
  \end{align*}
  and \(N\) is arbitrary, we know that \(\sup_{N > 1} \int_{[1, N]} f \leq \frac{1}{p - 1}\).
  Thus \(\sup_{N > 1} \int_{[1, N]} f\) is finite and by \cref{11.6.4} \(\sum_{n = 1}^\infty \frac{1}{n^p}\) is convergent.

  Next suppose that \(p = 1\).
  Since \(\int_{[1, N]} f = \ln N\) and \(\ln N\) is unbounded, we know that \(\sup_{N > 1} \int_{[1, N]} f = +\infty\) and by \cref{11.6.4} \(\sum_{n = 1}^\infty \frac{1}{n^p}\) is divergent.

  Next suppose that \(0 < p < 1\).
  Since \(\int_{[1, N]} f = \frac{1}{1 - p} (N^{1 - p} - 1)\) and \(\{N^{1 - p} : N \in \R^+\}\) is unbounded, we know that \(\sup_{N > 1} \int_{[1, N]} f = +\infty\) and by \cref{11.6.4} \(\sum_{n = 1}^\infty \frac{1}{n^p}\) is divergent.

  Finally suppose that \(p \leq 0\).
  By \cref{6.7.3}(e) we know that \(1 = x^0 \geq x^p\) for all \(x \in [1, \infty)\), thus \(1 = \frac{1}{x^0} \leq \frac{1}{x^p}\).
  By zero test (\cref{7.2.6}) we know that \(\lim_{n \to \infty} 1 \neq 0\) implies \(\sum_{n = 0}^\infty \frac{1}{x^0}\) diverges.
  Thus by comparison test (\cref{7.3.1}) \(\sum_{n = 1}^N \frac{1}{x^p}\) is divergent.
\end{proof}

\exercisesection

\begin{ex}\label{ex:11.6.1}
  Use \cref{11.6.1} to prove \cref{11.6.3}.
\end{ex}

\begin{proof}
  See \cref{11.6.3}.
\end{proof}

\begin{ex}\label{ex:11.6.2}
  Formulate a reasonable notion of a piecewise monotone function, and then show that all bounded piecewise monotone functions are Riemann integrable.
\end{ex}

\begin{proof}
  Let \(I\) be a bounded interval, and let \(f : I \to \R\).
  We say that \(f\) is \emph{piecewise monotone on \(I\)} iff there exists a partition \(\mathbf{P}\) of \(I\) such that \(f|_J\) is monotone on \(J\) for all \(J \in \mathbf{P}\).

  Now we show that all bounded piecewise monotone functions are Riemann integrable.
  Suppose that \(f : I \to \R\) is a bounded piecewise monotone function.
  Then by definition \(\exists\ \mathbf{P}\) such that \(\mathbf{P}\) is a partition of \(I\) and \(f|_J\) is monotone on \(J\) for all \(J \in \mathbf{P}\).
  Since \(f\) is bounded, \(f|_J\) is also bounded, by \cref{11.6.3} we know that \(f|_J\) is Riemann integrable on \(J\).
  Let \(F_J : I \to \R\) be the function
  \[
    F_J(x) = \begin{cases}
      f|_J(x) & \text{if } x \in J    \\
      0       & \text{if } x \notin J
    \end{cases}
  \]
  Then by \cref{11.4.1}(g) we know that \(F_J\) is Riemann integrable and
  \begin{align*}
    \sum_{J \in \mathbf{P}} \int_I F_J & = \sum_{J \in \mathbf{P}} \int_J f|_J & \text{(by \cref{11.4.1}(g))} \\
                                       & = \int_I f.                           & \text{(by \cref{ex:11.4.3})}
  \end{align*}
  Thus \(f\) is Riemann integrable on \(I\).
\end{proof}

\begin{ex}\label{ex:11.6.3}
  Prove \cref{11.6.4}.
\end{ex}

\begin{proof}
  See \cref{11.6.4}.
\end{proof}

\begin{ex}\label{ex:11.6.4}
  Give examples to show that both directions of the integral test break down if \(f\) is not assumed to be monotone decreasing.
\end{ex}

\begin{proof}
  Let \(f_1 : [0, \infty) \to \R\) be the function
  \[
    f_1(x) = \begin{cases}
      1 & \text{if } x \in \N    \\
      0 & \text{if } x \notin \N
    \end{cases}
  \]
  Then we know that \(f_1\) is not monotone decreasing and \(\sum_{n = 0}^\infty f_1(n)\) diverges.
  But \(\int_{[0, N]} f_1 = 0\) for all \(N \in \Z^+\), thus \(\sup_{N > 0} \int_{[0, N]} f_1\) is finite.

  Let \(f_2 : [0, \infty) \to \R\) be the function
  \[
    f_2(x) = \begin{cases}
      \frac{1}{x^2} & \text{if } x \in \N    \\
      \frac{1}{x}   & \text{if } x \notin \N
    \end{cases}
  \]
  Then we know that \(f_2\) is not monotone decreasing.
  By \cref{11.6.5} we know that \(\sup_{N > 0} \int_{[0, N]} \frac{1}{x}\) is not finite, and since \(\int_{[0, N]} f_2 = \int_{[0, N]} \frac{1}{x}\) we also have \(\sup_{N > 0} \int_{[0, N]} f_2\) is not finite.
  But by \cref{11.6.5} we know that \(\sum_{n = 0}^\infty \frac{1}{x^2}\) converges.
\end{proof}

\begin{ex}\label{ex:11.6.5}
  Use \cref{11.6.4} to prove \cref{11.6.5}.
\end{ex}

\begin{proof}
  See \cref{11.6.5}.
\end{proof}