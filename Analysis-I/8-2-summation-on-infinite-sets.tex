\section{Summation on infinite sets}\label{i:sec:8.2}

\begin{defn}[Series on countable sets]\label{i:8.2.1}
  Let \(X\) be a countable set, and let \(f : X \to \R\) be a function.
  We say that the series \(\sum_{x \in X} f(x)\) is absolutely convergent iff for some bijection \(g : \N \to X\), the sum \(\sum_{n = 0}^\infty f(g(n))\) is absolutely convergent.
  We then define the sum of \(\sum_{x \in X} f(x)\) by the formula
  \[
    \sum_{x \in X} f(x) = \sum_{n = 0}^\infty f(g(n)).
  \]
\end{defn}

\begin{note}
  From \cref{i:7.4.3}, one can show that these definitions do not depend on the choice of \(g\), and so are well defined.
\end{note}

\begin{note}
  For finite sets \(X\) we adopt the convention that series \(\sum_{x \in X} f(x)\) are automatically considered to be absolutely convergent.
\end{note}

\begin{ac}\label{i:ac:8.2.1}
  Let \(X\) be an at most countable set, and let \(f : X \to \R\) and \(g : X \to \R\) be functions such that the series \(\sum_{x \in X} f(x)\) and \(\sum_{x \in X} g(x)\) are both absolutely convergent.
  \begin{enumerate}
    \item The series \(\sum_{x \in X} (f(x) + g(x))\) is absolutely convergent, and
          \[
            \sum_{x \in X} (f(x) + g(x)) = \sum_{x \in X} f(x) + \sum_{x \in X} g(x).
          \]
    \item If \(c\) is a real number, then \(\sum_{x \in X} cf(x)\) is absolutely convergent, and
          \[
            \sum_{x \in X} cf(x) = c \sum_{x \in X} f(x).
          \]
    \item If \(X = X_1 \cup X_2\) for some disjoint sets \(X_1\) and \(X_2\), then \(\sum_{x \in X_1} f(x)\) and \\
          \(\sum_{x \in X_2} f(x)\) are absolutely convergent, and
          \[
            \sum_{x \in X_1 \cup X_2} f(x) = \sum_{x \in X_1} f(x) + \sum_{x \in X_2} f(x).
          \]
          Conversely, if \(h : X \to \R\) is such that \(\sum_{x \in X_1} h(x)\) and \(\sum_{x \in X_2} h(x)\) are absolutely convergent, then \(\sum_{x \in X_1 \cup X_2} h(x)\) is also absolutely convergent, and
          \[
            \sum_{x \in X_1 \cup X_2} h(x) = \sum_{x \in X_1} h(x) + \sum_{x \in X_2} h(x).
          \]
    \item If \(Y\) is another set, and \(\phi : Y \to X\) is a bijection, then \(\sum_{y \in Y} f(\phi(y))\) is absolutely convergent, and
          \[
            \sum_{y \in Y} f(\phi(y)) = \sum_{x \in X} f(x).
          \]
  \end{enumerate}
\end{ac}

\begin{proof}{(a)}
  Since \(X\) is at most countable, by \cref{i:8.1.1} we know that \(X\) is either finite or countable.
  If \(X\) is finite, then the statement follows from \cref{i:7.1.11}(f).
  So suppose that \(X\) is countable.
  By \cref{i:8.2.1} we know that there exists a bijection \(p : \N \to X\) such that \(\sum_{n = 0}^\infty f\big(p(n)\big)\) converges.
  Similarly, there exists a bijection \(q : \N \to X\) such that \(\sum_{n = 0}^\infty g\big(q(n)\big)\) converges.
  Since \(p\) is bijective, by \cref{i:7.4.3} we know that
  \[
    \sum_{x \in X} g(x) = \sum_{n = 0}^\infty g(q(n)) = \sum_{n = 0}^\infty g(p(n)).
  \]
  Thus we have
  \begin{align*}
     & \sum_{x \in X} \abs{f(x)} + \sum_{x \in X} \abs{g(x)}                                                              \\
     & = \sum_{n = 0}^\infty \abs{f\big(p(n)\big)} + \sum_{n = 0}^\infty \abs{g\big(p(n)\big)}      &  & \by{i:8.2.1}     \\
     & = \sum_{n = 0}^\infty \Big(\abs{f\big(p(n)\big)} + \abs{g\big(p(n)\big)}\Big)                &  & \by{i:7.2.14}[a] \\
     & = \lim_{N \to \infty} \sum_{n = 0}^N \Big(\abs{f\big(p(n)\big)} + \abs{g\big(p(n)\big)}\Big) &  & \by{i:7.2.2}     \\
     & \geq \lim_{N \to \infty} \sum_{n = 0}^N \abs{f\big(p(n)\big) + g\big(p(n)\big)}              &  & \by{i:6.1.19}[h] \\
     & = \sum_{n = 0}^\infty \abs{f\big(p(n)\big) + g\big(p(n)\big)}                                &  & \by{i:6.3.8}     \\
     & = \sum_{x \in X} \abs{f(x) + g(x)}                                                           &  & \by{i:8.2.1}
  \end{align*}
  and \(\sum_{x \in X} f(x) + g(x)\) is absolutely convergent.
  This implies
  \begin{align*}
     & \sum_{x \in X} f(x) + \sum_{x \in X} g(x)                                                         \\
     & = \sum_{n = 0}^\infty f\big(p(n)\big) + \sum_{n = 0}^\infty g\big(p(n)\big) &  & \by{i:8.2.1}     \\
     & = \sum_{n = 0}^\infty \Big(f\big(p(n)\big) + g\big(p(n)\big)\Big)           &  & \by{i:7.2.14}[a] \\
     & = \sum_{x \in X} \big(f(x) + g(x)\big).                                     &  & \by{i:8.2.1}
  \end{align*}
\end{proof}

\begin{proof}{(b)}
  Since \(X\) is at most countable, by \cref{i:8.1.1} we know that \(X\) is either finite or countable.
  If \(X\) is finite, then the statement follows from \cref{i:7.1.11}(g).
  So suppose that \(X\) is countable.
  By \cref{i:8.2.1} we know that there exists a bijection \(p : \N \to X\) such that \(\sum_{n = 0}^\infty f\big(p(n)\big)\) converges.
  Then we have
  \begin{align*}
    \abs{c} \sum_{x \in X} \abs{f(x)} & = \abs{c} \sum_{n = 0}^\infty \abs{f\big(p(n)\big)} &  & \by{i:8.2.1}     \\
                                      & = \sum_{n = 0}^\infty \abs{c} \abs{f\big(p(n)\big)} &  & \by{i:7.2.14}[b] \\
                                      & = \sum_{n = 0}^\infty \abs{c f\big(p(n)\big)}                             \\
                                      & = \sum_{x \in X} \abs{c f(x)}                       &  & \by{i:8.2.1}
  \end{align*}
  and thus \(\sum_{x \in X} \abs{c f(x)}\) is absolutely convergent.
  This implies
  \begin{align*}
    c \sum_{x \in X} f(x) & = c \sum_{n = 0}^\infty f\big(p(n)\big) &  & \by{i:8.2.1}     \\
                          & = \sum_{n = 0}^\infty c f\big(p(n)\big) &  & \by{i:7.2.14}[b] \\
                          & = \sum_{x \in X} f(x).                  &  & \by{i:8.2.1}
  \end{align*}
\end{proof}

\begin{proof}{(c)}
  We first show that if \(X = X_1 \cup X_2\), \(X_1 \cap X_2 = \emptyset\), then \(\sum_{x \in X_1} f(x)\) and \(\sum_{x \in X_2} f(x)\) is absolutely convergent.
  Since \(X\) is at most countable, by \cref{i:8.1.1} we know that \(X\) is either finite or countable.
  If \(X\) is finite, then the statement follows from \cref{i:7.1.11}(e).
  So suppose that \(X\) is countable.
  Since \(X = X_1 \cup X_2\), we know that \(X_1\) and \(X_2\) cannot both be finite.
  Now we split into two cases:
  \begin{itemize}
    \item One of \(X_1, X_2\) is finite and one is countable.
          Without the loss of generality suppose that \(X_1\) is finite.
          Since \(X_1\) is finite, we know that \(\exists q_1 : \set{i \in \N : 1 \leq i \leq \#(X_1)} \to X_1\) such that \(q_1\) is bijective.
          Since \(X_2\) is countable, by \cref{i:8.1.1} we know that \(\exists q_2 : \N \to X_2\) such that \(q_2\) is bijective.
          Then we define a function \(q : \N \to X\) as follow:
          \[
            \forall n \in \N, q(n) = \begin{dcases}
              q_1(n + 1)       & \text{if } n < \#(X_1)    \\
              q_2(n - \#(X_1)) & \text{if } n \geq \#(X_1)
            \end{dcases}
          \]
          Such \(q\) is bijective since \(X_1 \cap X_2 = \emptyset\) and \(q_1, q_2\) are bijective.
          Then we have
          \begin{align*}
            \sum_{x \in X} \abs{f(x)} & = \sum_{n = 0}^\infty \abs{f\big(q(n)\big)}                                                          &  & \by{i:8.2.1}     \\
                                      & = \sum_{n = 0}^{\#(X_1) - 1} \abs{f\big(q(n)\big)} + \sum_{n = \#(X_1)}^\infty \abs{f\big(q(n)\big)} &  & \by{i:7.2.14}[c] \\
                                      & = \sum_{n = 0}^{\#(X_1) - 1} \abs{f\big(q_1(n + 1)\big)}                                                                   \\
                                      & \quad + \sum_{n = \#(X_1)}^\infty \abs{f\Big(q_2\big(n - \#(X_1)\big)\Big)}                                                \\
                                      & = \sum_{n = 1}^{\#(X_1)} \abs{f\big(q_1(n)\big)}                                                     &  & \by{i:7.1.4}[b]  \\
                                      & \quad + \sum_{n = 0}^\infty \abs{f\big(q_2(n)\big)}                                                  &  & \by{i:7.2.14}[d] \\
                                      & = \sum_{x \in X_1} \abs{f(x)}                                                                        &  & \by{i:7.1.6}     \\
                                      & \quad + \sum_{x \in X_2} \abs{f\big(q(n)\big)}                                                       &  & \by{i:8.2.1}
          \end{align*}
          and thus both \(\sum_{x \in X_1} f(x)\) and \(\sum_{x \in X_2} f(x)\) are absolutely convergent.
          This implies
          \begin{align*}
            \sum_{x \in X} f(x) & = \sum_{n = 0}^\infty f(q(x))                                                            &  & \by{i:8.2.1}     \\
                                & = \sum_{n = 0}^{\#(X_1) - 1} f\big(q(n)\big) + \sum_{n = \#(X_1)}^\infty f\big(q(n)\big) &  & \by{i:7.2.14}[c] \\
                                & = \sum_{n = 0}^{\#(X_1) - 1} f\big(q_1(n + 1)\big)                                                             \\
                                & \quad + \sum_{n = \#(X_1)}^\infty f\Big(q_2\big(n - \#(X_1)\big)\Big)                                          \\
                                & = \sum_{n = 1}^{\#(X_1)} f\big(q_1(n)\big)                                               &  & \by{i:7.1.4}[b]  \\
                                & \quad + \sum_{n = 0}^\infty f\big(q_2(n)\big)                                            &  & \by{i:7.2.14}[c] \\
                                & = \sum_{x \in X_1} f(x)                                                                  &  & \by{i:7.1.6}     \\
                                & \quad + \sum_{x \in X_2} f\big(q(n)\big).                                                &  & \by{i:8.2.1}
          \end{align*}
    \item Both \(X_1, X_2\) are countable.
          Since \(X_1\) is countable, by \cref{i:8.1.1} we know that \(\exists q_1 : \N \to X_1\) such that \(q_1\) is bijective.
          Similarly, \(\exists q_2 : \N \to X_2\) such that \(q_2\) is bijective.
          Then we define a function \(q : \N \to X\) as follow:
          \[
            \forall n \in \N, q(n) = \begin{dcases}
              q_1(\dfrac{n}{2})     & \text{if } n \text{ is even} \\
              q_2(\dfrac{n - 1}{2}) & \text{if } n \text{ is odd}
            \end{dcases}
          \]
          Such \(q\) is bijective since \(X_1 \cap X_2 = \emptyset\) and \(q_1, q_2\) are bijective.
          Then we have
          \begin{align*}
             & \sum_{x \in X} \abs{f(x)}                                                                                                                       \\
             & = \sum_{n = 0}^\infty \abs{f\big(q(n)\big)}                                                                               &  & \by{i:8.2.1}     \\
             & = \lim_{N \to \infty} \sum_{n = 0}^{2N} \abs{f\big(q(n)\big)}                                                             &  & \by{i:7.2.2}     \\
             & = \lim_{N \to \infty} \sum_{n \leq 2N} \abs{f\big(q(n)\big)}                                                              &  & \by{i:7.1.6}     \\
             & = \lim_{N \to \infty} \Bigg(\sum_{n \leq 2N \land n \text{ is even}} \abs{f\big(q(n)\big)}                                                      \\
             & \quad + \sum_{n \leq 2N \land n \text{ is odd}} \abs{f\big(q(n)\big)}\Bigg)                                               &  & \by{i:7.1.11}[e] \\
             & = \lim_{N \to \infty} \Bigg(\sum_{n \leq 2N \land n \text{ is even}} \abs{f\big(q_1(\dfrac{n}{2})\big)}                                         \\
             & \quad + \sum_{n \leq 2N \land n \text{ is odd}} \abs{f\big(q_2(\dfrac{n - 1}{2})\big)}\Bigg)                                                    \\
             & = \lim_{N \to \infty} \sum_{n \leq 2N \land n \text{ is even}} \abs{f\big(q_1(\dfrac{n}{2})\big)}                                               \\
             & \quad + \lim_{N \to \infty} \sum_{n \leq 2N \land n \text{ is odd}} \abs{f\big(q_2(\dfrac{n - 1}{2})\big)}                &  & \by{i:6.1.19}[a] \\
             & = \lim_{N \to \infty} \sum_{n = 0}^N \abs{f\big(q_1(n)\big)} + \lim_{N \to \infty} \sum_{n = 0}^N \abs{f\big(q_2(n)\big)} &  & \by{i:7.1.6}     \\
             & = \sum_{n = 0}^\infty \abs{f\big(q_1(n)\big)} + \sum_{n = 0}^\infty \abs{f\big(q_2(n)\big)}                               &  & \by{i:7.2.2}     \\
             & = \sum_{x \in X_1} \abs{f(x)} + \sum_{x \in X_2} \abs{f(x)}                                                               &  & \by{i:8.2.1}
          \end{align*}
          and thus both \(\sum_{x \in X_1} f(x)\) and \(\sum_{x \in X_2} f(x)\) are absolutely convergent.
          This implies
          \begin{align*}
             & \sum_{x \in X} f(x)                                                                                                                 \\
             & = \sum_{n = 0}^\infty f\big(q(n)\big)                                                                         &  & \by{i:8.2.1}     \\
             & = \lim_{N \to \infty} \sum_{n = 0}^{2N} f\big(q(n)\big)                                                       &  & \by{i:7.2.2}     \\
             & = \lim_{N \to \infty} \sum_{n \leq 2N} f\big(q(n)\big)                                                        &  & \by{i:7.1.6}     \\
             & = \lim_{N \to \infty} \Bigg(\sum_{n \leq 2N \land n \text{ is even}} f\big(q(n)\big)                                                \\
             & \quad + \sum_{n \leq 2N \land n \text{ is odd}} f\big(q(n)\big)\Bigg)                                         &  & \by{i:7.1.11}[e] \\
             & = \lim_{N \to \infty} \Bigg(\sum_{n \leq 2N \land n \text{ is even}} f\big(q_1(\dfrac{n}{2})\big)                                   \\
             & \quad + \sum_{n \leq 2N \land n \text{ is odd}} f\big(q_2(\dfrac{n - 1}{2})\big)\Bigg)                                              \\
             & = \lim_{N \to \infty} \sum_{n \leq 2N \land n \text{ is even}} f\big(q_1(\dfrac{n}{2})\big)                                         \\
             & \quad + \lim_{N \to \infty} \sum_{n \leq 2N \land n \text{ is odd}} f\big(q_2(\dfrac{n - 1}{2})\big)          &  & \by{i:6.1.19}[a] \\
             & = \lim_{N \to \infty} \sum_{n = 0}^N f\big(q_1(n)\big) + \lim_{N \to \infty} \sum_{n = 0}^N f\big(q_2(n)\big) &  & \by{i:7.1.6}     \\
             & = \sum_{n = 0}^\infty f\big(q_1(n)\big) + \sum_{n = 0}^\infty f\big(q_2(n)\big)                               &  & \by{i:7.2.2}     \\
             & = \sum_{x \in X_1} f(x) + \sum_{x \in X_2} f(x).                                                              &  & \by{i:8.2.1}
          \end{align*}
  \end{itemize}
  From all cases above we conclude that both \(\sum_{x \in X_1} f(x)\) and \(\sum_{x \in X_2} f(x)\) are absolutely convergent, and we have
  \[
    \sum_{x \in X} f(x) = \sum_{x \in X_1} f(x) + \sum_{x \in X_2} f(x).
  \]

  Now we show that if \(X_1 \cup X_2 \subseteq X\), \(X_1 \cap X_2 = \emptyset\), \(\sum_{x \in X_1} h(x)\) and \(\sum_{x \in X_2} h(x)\) are absolutely convergent, then \(\sum_{x \in X_1 \cup X_2} h(x)\) is absolutely convergent.
  Since \(X\) is at most countable, by \cref{i:8.1.7} we know that \(X_1 \cup X_2\) is at most countable.
  By \cref{i:8.1.1} we know that \(X_1 \cup X_2\) is either finite or countable.
  If \(X_1 \cup X_2\) is finite, then the statement follows from \cref{i:7.1.11}(e).
  So suppose that \(X_1 \cup X_2\) is countable.
  We know that \(X_1\) and \(X_2\) cannot both be finite.
  Now we split into two cases:
  \begin{itemize}
    \item One of \(X_1, X_2\) is finite and one is countable.
          Without the loss of generality suppose that \(X_1\) is finite.
          Since \(X_1\) is finite, we know that \(\exists q_1 : \set{i \in \N : 1 \leq i \leq \#(X_1)} \to X_1\) such that \(q_1\) is bijective.
          Since \(X_2\) is countable, by \cref{i:8.1.1} we know that \(\exists q_2 : \N \to X_2\) such that \(q_2\) is bijective.
          Then we define a function \(q : \N \to X_1 \cup X_2\) as follow:
          \[
            \forall n \in \N, q(n) = \begin{dcases}
              q_1(n + 1)       & \text{if } n < \#(X_1)    \\
              q_2(n - \#(X_1)) & \text{if } n \geq \#(X_1)
            \end{dcases}
          \]
          Such \(q\) is bijective since \(X_1 \cap X_2 = \emptyset\) and \(q_1, q_2\) are bijective.
          Then we have
          \begin{align*}
             & \sum_{x \in X_1} \abs{h(x)} + \sum_{x \in X_2} \abs{h(x)}                                                                  \\
             & = \sum_{n = 1}^{\#(X_1)} \abs{h\big(q_1(n)\big)}                                                     &  & \by{i:7.1.6}     \\
             & \quad + \sum_{n = 0}^\infty \abs{h\big(q_2(n)\big)}                                                  &  & \by{i:8.2.1}     \\
             & = \sum_{n = 0}^{\#(X_1) - 1} \abs{h\big(q_1(n + 1)\big)}                                             &  & \by{i:7.1.4}[b]  \\
             & \quad + \sum_{n = \#(X_1)}^\infty \abs{h\Big(q_2\big(n - \#(X_1)\big)\Big)}                          &  & \by{i:7.2.14}[d] \\
             & = \sum_{n = 0}^{\#(X_1) - 1} \abs{h\big(q(n)\big)} + \sum_{n = \#(X_1)}^\infty \abs{h\big(q(n)\big)}                       \\
             & = \sum_{n = 0}^\infty \abs{h\big(q(n)\big)}                                                          &  & \by{i:7.2.14}[c] \\
             & = \sum_{x \in X_1 \cup X_2} \abs{h(x)}                                                               &  & \by{i:8.2.1}
          \end{align*}
          and thus \(\sum_{x \in X_1 \cup X_2} h(x)\) is absolutely convergent.
    \item Both \(X_1, X_2\) are countable.
          Since \(X_1\) is countable, by \cref{i:8.1.1} we know that \(\exists q_1 : \N \to X_1\) such that \(q_1\) is bijective.
          Similarly, \(\exists q_2 : \N \to X_2\) such that \(q_2\) is bijective.
          Then we define a function \(q : \N \to X_1 \cup X_2\) as follow:
          \[
            \forall n \in \N, q(n) = \begin{dcases}
              q_1(\dfrac{n}{2})     & \text{if } n \text{ is even} \\
              q_2(\dfrac{n - 1}{2}) & \text{if } n \text{ is odd}
            \end{dcases}
          \]
          Such \(q\) is bijective since \(X_1 \cap X_2 = \emptyset\) and \(q_1, q_2\) are bijective.
          Then we have
          \begin{align*}
             & \sum_{x \in X_1} \abs{h(x)} + \sum_{x \in X_2} \abs{h(x)}                                                                                 \\
             & = \sum_{n = 0}^\infty \abs{h\big(q_1(n)\big)} + \sum_{n = 0}^\infty \abs{h\big(q_2(n)\big)}                         &  & \by{i:8.2.1}     \\
             & = \sum_{n = 0}^\infty \Big(\abs{h\big(q_1(n)\big)} + \abs{h\big(q_2(n)\big)}\Big)                                   &  & \by{i:7.2.14}[a] \\
             & = \lim_{N \to \infty} \Bigg(\sum_{n = 0}^N \abs{h\big(q_1(n)\big)} + \sum_{n = 0}^N \abs{h\big(q_2(n)\big)}\Bigg)   &  & \by{i:7.2.2}     \\
             & = \lim_{N \to \infty} \Bigg(\sum_{n = 0}^N \abs{h\big(q(2n)\big)} + \sum_{n = 0}^N \abs{h\big(q(2n + 1)\big)}\Bigg)                       \\
             & = \lim_{N \to \infty} \Bigg(\sum_{n \leq 2N : n \text{ is even}} \abs{h\big(q(n)\big)}                                                    \\
             & \quad + \sum_{n \leq 2N : n \text{ is odd}} \abs{h\big(q(n)\big)}\Bigg)                                             &  & \by{i:7.1.6}     \\
             & = \lim_{N \to \infty} \sum_{n \leq 2N} \abs{h\big(q(n)\big)}                                                        &  & \by{i:7.1.11}[e] \\
             & = \lim_{N \to \infty} \sum_{n = 0}^{2N} \abs{h\big(q(n)\big)}                                                       &  & \by{i:7.1.6}     \\
             & = \sum_{n = 0}^\infty \abs{h\big(q(n)\big)}                                                                         &  & \by{i:7.2.2}     \\
             & = \sum_{x \in X_1 \cup X_2}^\infty \abs{h(x)}                                                                       &  & \by{i:8.2.1}
          \end{align*}
          and thus \(\sum_{x \in X_1 \cup X_2} h(x)\) is absolutely convergent.
  \end{itemize}
  From all cases above we conclude that \(\sum_{x \in X_1 \cup X_2} h(x)\) is absolutely convergent.
  Since \(\sum_{x \in X_1 \cup X_2} h(x)\) is absolutely convergent, from the proof above we have
  \[
    \sum_{x \in X_1 \cup X_2} h(x) = \sum_{x \in X_1} h(x) + \sum_{x \in X_2} h(x).
  \]
\end{proof}

\begin{proof}{(d)}
  Since \(X\) is at most countable, by \cref{i:8.1.1} we know that \(X\) is either finite or countable.
  If \(X\) is finite, then the statement follows from \cref{i:7.1.11}(c).
  So suppose that \(X\) is countable.
  By \cref{i:8.2.1} we know that there exists a bijection \(p : \N \to X\) such that \(\sum_{n = 0}^\infty f\big(p(n)\big)\) converges.
  Since \(\phi\) is bijective, we know that \(Y\) is also countable and by \cref{i:8.1.1} \(\exists q : \N \to Y\) such that \(q\) is bijective.
  Then we have \(\phi \circ q : \N \to X\) is bijective and
  \begin{align*}
    \sum_{x \in X} f(x) & = \sum_{n = 0}^\infty f\big(p(n)\big)               &  & \by{i:8.2.1} \\
                        & = \sum_{n = 0}^\infty f\big((\phi \circ q)(n)\big)  &  & \by{i:7.4.3} \\
                        & = \sum_{n = 0}^\infty f\Big(\phi\big(q(n)\big)\Big)                   \\
                        & = \sum_{y \in Y} f\big(\phi(y)\big).                &  & \by{i:8.2.1}
  \end{align*}
  Thus \(\sum_{y \in Y} f\big(\phi(y)\big)\) is absolutely convergent.
\end{proof}

\begin{thm}[Fubini's theorem for infinite sums]\label{i:8.2.2}
  Let \(f : \N \times \N \to \R\) be a function such that \(\sum_{(n, m) \in \N \times \N} f(n, m)\) is absolutely convergent.
  Then we have
  \begin{align*}
    \sum_{n = 0}^\infty \bigg(\sum_{m = 0}^\infty f(n, m)\bigg) & = \sum_{(n, m) \in \N \times \N} f(n, m)                       \\
                                                                & = \sum_{(m, n) \in \N \times \N} f(n, m)                       \\
                                                                & = \sum_{m = 0}^\infty \bigg(\sum_{n = 0}^\infty f(n, m)\bigg).
  \end{align*}
  In other words, we can switch the order of infinite sums \emph{provided that the entire sum is absolutely convergent}.
\end{thm}

\begin{proof}
  The second equality follows easily from \cref{i:7.4.3} (and \cref{i:3.6.4}).

  Let us first consider the case when \(f(n, m)\) is always non-negative (we will deal with the general case later).
  Write
  \[
    L \coloneqq \sum_{(n, m) \in \N \times \N} f(n, m);
  \]
  our task is to show that the series \(\sum_{n = 0}^\infty (\sum_{m = 0}^\infty f(n, m))\) converges to \(L\).

  One can easily show that \(\sum_{(n, m) \in X} f(n, m) \leq L\) for all finite sets \(X \subseteq \N \times \N\).
  (Use a bijection \(g\) between \(\N \times \N\) and \(\N\), and then use the fact that \(g(X)\) is finite, hence bounded.)
  In particular, for every \(n \in \N\) and \(M \in \N\) we have \(\sum_{m = 0}^M f(n, m) \leq L\), which implies by \cref{i:6.3.8} that \(\sum_{m = 0}^\infty f(n, m)\) is convergent for each \(n\).
  Similarly, for any \(N \in \N\) and \(M \in \N\) we have (by \cref{i:7.1.14})
  \[
    \sum_{n = 0}^N \sum_{m = 0}^M f(n, m) = \sum_{(n, m) \in X} f(n, m) \leq L
  \]
  where \(X\) is the set \(\set{(n,m) \in \N \times \N : n \leq N, m \leq M}\) which is finite by \cref{i:3.6.14}.
  Taking limits of this as \(M \to \infty\) we have (by \cref{i:ex:7.1.5} and either \cref{i:6.3.8} or \cref{i:6.4.13})
  \[
    \sum_{n = 0}^N \sum_{m = 0}^\infty f(n, m) \leq L.
  \]
  By \cref{i:6.3.8}, this implies that \(\sum_{n = 0}^\infty \sum_{m = 0}^\infty f(n, m)\) converges, and
  \[
    \sum_{n = 0}^\infty \sum_{m = 0}^\infty f(n, m) \leq L.
  \]
  To finish the proof, it will suffice to show that
  \[
    \sum_{n = 0}^\infty \sum_{m = 0}^\infty f(n, m) \geq L - \varepsilon
  \]
  for every \(\varepsilon > 0\).
  \begin{align*}
             & L \geq \sum_{n = 0}^\infty \sum_{m = 0}^\infty f(n, m) \geq L - \varepsilon               \\
    \implies & L + \varepsilon \geq \sum_{n = 0}^\infty \sum_{m = 0}^\infty f(n, m) \geq L - \varepsilon \\
    \implies & \varepsilon \geq \sum_{n = 0}^\infty \sum_{m = 0}^\infty f(n, m) - L \geq -\varepsilon    \\
    \implies & \abs{\sum_{n = 0}^\infty \sum_{m = 0}^\infty f(n, m) - L} \leq \varepsilon                \\
  \end{align*}
  So, let \(\varepsilon > 0\).
  By definition of \(L\), we can then find a finite set \(X \subseteq \N \times \N\) such that \(\sum_{(n, m) \in X} f(n, m) \geq L - \varepsilon\).
  (Since \(\N \times \N\) is countable by \cref{i:8.1.13}, we can find a bijection \(g : \N \to \N \times \N\) such that \(\sum_{i = 0}^\infty f(g(i)) = L\), which means \(\forall \varepsilon > 0\), \(\exists H \in \N\) such that \(\abs{\sum_{i = 0}^h f(g(i)) - L} \leq \varepsilon\) for all \(h \geq H\).
  Now we can choose \(X = \set{g(i) : 0 \leq i \leq H}\))
  This set, being finite, must be contained in some set of the form \(Y \coloneqq \set{(n,m) \in \N \times \N : n \leq N; m \leq M }\).
  Thus by \cref{i:7.1.14}
  \[
    \sum_{n = 0}^N \sum_{m = 0}^M f(n, m) = \sum_{(n, m) \in Y} f(n, m) \geq \sum_{(n, m) \in X} f(n, m) \geq L - \varepsilon
  \]
  and hence
  \[
    \sum_{n = 0}^\infty \sum_{m = 0}^\infty f(n, m) \geq \sum_{n = 0}^N \sum_{m = 0}^\infty f(n, m) \geq \sum_{n = 0}^N \sum_{m = 0}^M f(n, m) \geq L - \varepsilon
  \]
  as desired.

  This proves the claim when the \(f(n, m)\) are all non-negative.
  A similar argument works when the \(f(n, m)\) are all non-positive
  (in fact, one can simply apply the result just obtained to the function \(-f(n, m)\), and then use limit laws to remove the \(-\).
  For the general case, note that any function \(f(n, m)\) can be written as \(f_+(n, m) + f_-(n, m)\), where \(f_+(n, m)\) is the positive part of \(f(n, m)\)
  (i.e., it equals \(f(n, m)\) when \(f(n, m)\) is positive, and \(0\) otherwise),
  and \(f_-\) is the negative part of \(f(n, m)\)
  (it equals \(f(n, m)\) when \(f(n, m)\) is negative, and \(0\) otherwise).
  It is easy to show that if \(\sum_{(n, m) \in \N \times \N} f(n, m)\) is absolutely convergent, then so are \(\sum_{(n, m) \in \N \times \N} f_+(n, m)\) and \(\sum_{(n, m) \in \N \times \N} f_-(n, m)\).
  (We can construct a bijection \(g : \N \to \N \times \N\) and then since \(\forall n \in \N\) we have \(f_+(g(n)) \leq \abs{f(g(n))}\) and \(\abs{f_-(g(n))} \leq \abs{f(g(n))}\), we know that \((f_+(g(n)))_{n = 0}^\infty\) and \((f_-(g(n)))_{n = 0}^\infty\) are absolutely convergent by comparison test (\cref{i:7.3.2}.))
  So now one applies the results just obtained to \(f_+\) and to \(f_-\) and adds them together using limit laws to obtain the result for a general \(f\).
\end{proof}

\begin{lem}\label{i:8.2.3}
  Let \(X\) be a countable set, and let \(f : X \to \R\) be a function.
  Then the series \(\sum_{x \in X} f(x)\) is absolutely convergent iff
  \[
    \sup\set{\sum_{x \in A} \abs{f(x)} : A \subseteq X, A \text{ finite}} < \infty.
  \]
\end{lem}

\begin{proof}
  Let \(P(X, f)\) be the statement
  \[
    \sup\set{\sum_{x \in A} \abs{f(x)} : A \subseteq X, A \text{ finite}} < \infty.
  \]
  We first show that if \(\sum_{x \in X} f(x)\) is absolutely convergent, then \(P(X, f)\) is true.
  Let \(L = \sum_{x \in X} f(x)\).
  Since \(\sum_{x \in X} f(x)\) is absolutely convergent, by \cref{i:8.2.1} \(\exists g : \N \to X\) where \(g\) is a bijection such that
  \[
    L = \sum_{x \in X} \abs{f(x)} = \sum_{n = 0}^\infty \abs{f\big(g(n)\big)}.
  \]
  Let \(A \subseteq X\) be a finite set.
  Since \(g\) is a bijection, we have
  \[
    \sum_{x \in A} \abs{f(x)} = \sum_{n \in g^{-1}(A)} \abs{f(g(n))}
  \]
  Since \(A\) is finite, by \cref{i:ex:3.6.3} \(\exists M \in \N\) such that \(g^{-1}(A)\) is bounded by \(M\).
  So we have
  \[
    \sum_{x \in A} \abs{f(x)} = \sum_{n \in g^{-1}(A)} \abs{f(g(n))} \leq \sum_{n = 0}^M \abs{f(g(n))} \leq L
  \]
  This is true for any finite subset of \(X\).
  Thus by \cref{i:5.5.9} \(P(X, f)\) is true.

  Now we show that if \(P(X, f)\) is true, then \(\sum_{x \in X} f(x)\) is absolutely convergent.
  Let \(L\) be the supremum described by \(P(X, f)\).
  Since \(X\) is countable, \(\exists g : \N \to X\) where \(g\) is a bijection.
  So we have
  \begin{align*}
             & \forall n \in \N : \sum_{x \in g(\set{i \in \N : 0 \leq i \leq n})} \abs{f(x)} \leq L & (P(X, f) \text{ is true})                \\
    \implies & \forall n \in \N : \sum_{i = 0}^n \abs{f(g(i))} \leq L                                &                           & \by{i:7.1.6} \\
    \implies & \sum_{i = 0}^\infty \abs{f(g(i))} \text{ converges}                                   &                           & \by{i:7.3.1} \\
    \implies & \sum_{x \in X} \abs{f(x)} \text{ converges}.                                          &                           & \by{i:8.2.1}
  \end{align*}
\end{proof}

\begin{note}
  Inspired by \cref{i:8.2.3}, we may now define the concept of an absolutely convergent series even when the set \(X\) could be uncountable.
\end{note}

\begin{defn}\label{i:8.2.4}
  Let \(X\) be a set (which could be uncountable), and let \(f : X \to \R\) be a function.
  We say that the series \(\sum_{x \in X} f(x)\) is absolutely convergent iff
  \[
    \sup\set{\sum_{x \in A} \abs{f(x)} : A \subseteq X, A \text{ finite}} < \infty.
  \]
\end{defn}

\begin{lem}\label{i:8.2.5}
  Let \(X\) be a set (which could be uncountable), and let \(f : X \to \R\) be a function such that the series \(\sum_{x \in X} f(x)\) is absolutely convergent.
  Then the set \(\set{x \in X : f(x) \neq 0}\) is at most countable.
\end{lem}

\begin{proof}
  Suppose that \(X\) is a set and \(f : X \to \R\) is a function such that \(\sum_{x \in X} f(x)\) is absolutely convergent.
  Since \(\sum_{x \in X} f(x)\) is absolutely convergent, by \cref{i:8.2.4} we have
  \[
    M = \sup\set{\sum_{x \in A} \abs{f(x)} : A \subseteq X, A \text{ finite}} < \infty.
  \]
  We first show that \(\forall n \in \Z^+\), the set \(S_n = \set{x \in X : \abs{f(x)} > 1 / n}\) is finite and \(\#(S_n) \leq Mn\).

  Suppose for sake of contradiction that \(S_n\) is infinite.
  Then we can have a finite set \(S \subseteq S_n\) where \(\#(S) > (M + 1)n\).
  Since \(S\) is finite, we have \(\sum_{x \in S} \abs{f(x)} \leq M\).
  Since \(S \subseteq S_n\), we have \(\abs{f(x)} > 1 / n\) for every \(x \in S\).
  But now we have
  \[
    M \geq \sum_{x \in S} \abs{f(x)} > \dfrac{(M + 1)n}{n} = M + 1,
  \]
  a contradiction.
  Thus \(S_n\) must be finite.

  Now suppose for sake of contradiction \(\#(S_n) > Mn\).
  Again we have
  \[
    M \geq \sum_{x \in S_n} \abs{f(x)} > \dfrac{Mn}{n} = M,
  \]
  a contradiction.
  Thus \(\#(S_n) \leq Mn\).

  Let \(x \in X\) where \(f(x) \neq 0\).
  If \(x\) does not exist, then we have \(\set{x \in X : f(x) \neq 0} = \emptyset\) which is at most countable.
  So suppose that such \(x\) exists.
  Since \(\abs{f(x)} \in \R^+\), by \cref{i:5.4.12} we have
  \begin{align*}
             & \exists N \in \Z^+ : \dfrac{1}{\abs{f(x)}} < N                                                             \\
    \implies & \abs{f(x)} > \dfrac{1}{N}                                                                                  \\
    \implies & x \in S_N                                                       &  & \text{(by the definition of \(S_N\))} \\
    \implies & x \in \bigcup_{n \in \Z^+} S_n                                  &  & \by{i:3.11}                           \\
    \implies & \set{x \in X : f(x) \neq 0} \subseteq \bigcup_{n \in \Z^+} S_n. &  & \by{i:3.1.15}
  \end{align*}
  Since \(\forall n \in \Z^+\), \(S_n\) is finite, by \cref{i:8.1.9} we know that \(\bigcup_{n \in \Z^+} S_n\) is at most countable.
  Since \(\set{x \in X : f(x) \neq 0} \subseteq \bigcup_{n \in \Z^+} S_n\), by \cref{i:8.1.7} we know that \(\set{x \in X : f(x) \neq 0}\) is at most countable.
\end{proof}

\begin{note}
  Because of \cref{i:8.2.5}, we can define the value of \(\sum_{x \in X} f(x)\) for any absolutely convergent series on an uncountable set \(X\) by the formula
  \[
    \sum_{x \in X} \coloneqq \sum_{x \in X : f(x) \neq 0} f(x),
  \]
  since we have replaced a sum on an uncountable set \(X\) by a sum on the at most countable set \(\set{x \in X : f(x) \neq 0}\).
  (If the former sum is absolutely convergent, then the latter one is also.)
  \cref{i:8.2.4} is consistent with the definitions we already have for series on countable sets (\cref{i:8.2.1}).
\end{note}

\begin{prop}[Absolutely convergent series laws]\label{i:8.2.6}
  Let \(X\) be an arbitrary set (possibly uncountable), and let \(f : X \to \R\) and \(g : X \to \R\) be functions such that the series \(\sum_{x \in X} f(x)\) and \(\sum_{x \in X} g(x)\) are both absolutely convergent.
  \begin{enumerate}
    \item The series \(\sum_{x \in X} (f(x) + g(x))\) is absolutely convergent, and
          \[
            \sum_{x \in X} (f(x) + g(x)) = \sum_{x \in X} f(x) + \sum_{x \in X} g(x).
          \]
    \item If \(c\) is a real number, then \(\sum_{x \in X} cf(x)\) is absolutely convergent, and
          \[
            \sum_{x \in X} cf(x) = c \sum_{x \in X} f(x).
          \]
    \item If \(X = X_1 \cup X_2\) for some disjoint sets \(X_1\) and \(X_2\), then \(\sum_{x \in X_1} f(x)\) and \\
          \(\sum_{x \in X_2} f(x)\) are absolutely convergent, and
          \[
            \sum_{x \in X_1 \cup X_2} f(x) = \sum_{x \in X_1} f(x) + \sum_{x \in X_2} f(x).
          \]
          Conversely, if \(h : X \to \R\) is such that \(\sum_{x \in X_1} h(x)\) and \(\sum_{x \in X_2} h(x)\) are absolutely convergent, then \(\sum_{x \in X_1 \cup X_2} h(x)\) is also absolutely convergent, and
          \[
            \sum_{x \in X_1 \cup X_2} h(x) = \sum_{x \in X_1} h(x) + \sum_{x \in X_2} h(x).
          \]
    \item If \(Y\) is another set, and \(\phi : Y \to X\) is a bijection, then \(\sum_{y \in Y} f(\phi(y))\) is absolutely convergent, and
          \[
            \sum_{y \in Y} f(\phi(y)) = \sum_{x \in X} f(x).
          \]
  \end{enumerate}
\end{prop}

\begin{proof}{(a)}
  Suppose that \(X\) is a set and \(f : X \to \R, g : X \to \R\) are functions such that \(\sum_{x \in X} f(x)\) and \(\sum_{x \in X} g(x)\) are both absolutely convergent.
  By \cref{i:7.1.11}(f) we already show that the statement is true when \(X\) is finite.
  So suppose that \(X\) is infinite.

  We first show that \(\sum_{x \in X} (f(x) + g(x))\) is absolutely convergent.
  Since \(\sum_{x \in X} f(x)\) and \(\sum_{x \in X} g(x)\) are both absolutely convergent, by \cref{i:8.2.4} \(\exists N, M \in \R\) such that
  \[
    N = \sup\set{\sum_{x \in A} \abs{f(x)} : A \subseteq X, A \text{ finite}} < \infty
  \]
  and
  \[
    M = \sup\set{\sum_{x \in A} \abs{g(x)} : A \subseteq X, A \text{ finite}} < \infty.
  \]
  Let \(A \subseteq X\) be a finite set.
  Then we have
  \begin{align*}
    \sum_{x \in A} \abs{f(x) + g(x)} & \leq \sum_{x \in A} (\abs{f(x)} + \abs{g(x)})                                 \\
                                     & = \sum_{x \in A} \abs{f(x)} + \sum_{x \in A} \abs{g(x)} &  & \by{i:7.1.11}[f] \\
                                     & \leq N + M.
  \end{align*}
  Since \(A\) was arbitrary, we have
  \[
    \sup\set{\sum_{x \in A} \abs{f(x) + g(x)} : A \subseteq X, A \text{ finite}} \leq N + M < \infty.
  \]
  Thus by \cref{i:8.2.4} \(\sum_{x \in X} \big(f(x) + g(x)\big)\) is absolutely convergent.

  Now we show that \(\sum_{x \in X} (f(x) + g(x)) = \sum_{x \in X} f(x) + \sum_{x \in X} g(x)\).
  If \(X\) is at most countable, then the statement follows by \cref{i:ac:8.2.1}(a).
  So suppose that \(X\) is uncountable.
  Let \(X_f = \set{x \in X : f(x) \neq 0}\), \(X_g = \set{x \in X : g(x) \neq 0}\) and \(X_h = \set{x \in X : f(x) + g(x) \neq 0}\) be sets.
  Then by \cref{i:8.2.5} we know that \(X_f\), \(X_g\) and \(X_h\) are at most countable.
  Since
  \begin{align*}
             & \forall x \in X_f \setminus X_g \\
    \implies & f(x) \neq 0 \land g(x) = 0      \\
    \implies & f(x) + g(x) \neq 0              \\
    \implies & x \in X_h,
  \end{align*}
  we know that \(X_f \setminus X_g \subseteq X_h\).
  Similarly we have \(X_g \setminus X_f \subseteq X_h\).
  Then we have
  \begin{align*}
             & \forall x \in X_h                                                                                  \\
    \implies & f(x) + g(x) \neq 0                                                                                 \\
    \implies & f(x) \neq -g(x)                                                                                    \\
    \implies & \big(f(x) \neq 0 \land g(x) = 0\big) \lor \big(f(x) = 0 \land g(x) \neq 0\big)                     \\
             & \lor \big(f(x) \neq -g(x) \land f(x) \neq 0 \land g(x) \neq 0\big)                                 \\
    \implies & (x \in X_f \setminus X_g) \lor (x \in X_g \setminus X_f) \lor (x \in X_f \cap X_g \land x \in X_h) \\
    \implies & \big(x \in (X_f \setminus X_g) \cup (X_g \setminus X_f) \cup (X_f \cap X_g)\big)                   \\
             & \land \big(x \in (X_f \setminus X_g) \cup (X_g \setminus X_f) \cup X_h\big)                        \\
    \implies & (x \in X_f \cup X_g) \land (x \in X_h)                                                             \\
    \implies & x \in X_f \cup X_g
  \end{align*}
  and \(X_h \subseteq X_f \cup X_g\).
  By \cref{i:ac:8.1.1} we know that \(X_f \cup X_g\) is at most countable.
  By \cref{i:8.1.7} we know that \(X_f \cup X_g \setminus X_f\), \(X_f \cup X_g \setminus X_g\), \(X_f \cup X_g \setminus X_h\) are at most countable.
  Thus we have
  \begin{align*}
     & \sum_{x \in X} \big(f(x) + g(x)\big)                                                                                                                           \\
     & = \sum_{x \in X_h} \big(f(x) + g(x)\big)                                       &                                                          & \by{i:8.2.5}       \\
     & = \sum_{x \in X_h} \big(f(x) + g(x)\big)                                                                                                                       \\
     & \quad + \sum_{x \in (X_f \cup X_g) \setminus X_h} \big(f(x) + g(x)\big)        & (x \notin X_h \iff f(x) + g(x) = 0)                                           \\
     & = \sum_{x \in X_f \cup X_g} \big(f(x) + g(x)\big)                              &                                                          & \by{i:ac:8.2.1}[c] \\
     & = \sum_{x \in X_f \cup X_g} f(x) + \sum_{x \in X_f \cup X_g} g(x)              &                                                          & \by{i:ac:8.2.1}[a] \\
     & = \sum_{x \in X_f} f(x) + \sum_{x \in (X_f \cup X_g) \setminus X_f} f(x)       &                                                          & \by{i:ac:8.2.1}[c] \\
     & \quad + \sum_{x \in X_g} g(x) + \sum_{x \in (X_f \cup X_g) \setminus X_g} g(x) &                                                          & \by{i:ac:8.2.1}[c] \\
     & = \sum_{x \in X_f} f(x) + \sum_{x \in X_g} g(x)                                & (x \notin X_f \iff f(x) = 0, x \notin X_g \iff g(x) = 0)                      \\
     & = \sum_{x \in X} f(x) + \sum_{x \in X} g(x).                                   &                                                          & \by{i:8.2.5}
  \end{align*}
\end{proof}

\begin{proof}{(b)}
  Suppose that \(c \in \R\), \(X\) is a set and \(f : X \to \R\) is a function such that \(\sum_{x \in X} f(x)\) is absolutely convergent.
  Since \(\sum_{x \in X} f(x)\) is absolutely convergent, by \cref{i:8.2.4} \(\exists N \in \R\) such that
  \[
    N = \sup\set{\sum_{x \in A} \abs{f(x)} : A \subseteq X, A \text{ finite}} < \infty.
  \]
  Let \(A \subseteq X\) be a finite set.
  Then we have
  \begin{align*}
    \sum_{x \in A} \abs{cf(x)} & = \sum_{x \in A} \abs{c}\abs{f(x)}                       \\
                               & = \abs{c}\sum_{x \in A} \abs{f(x)} &  & \by{i:7.1.11}[g] \\
                               & \leq \abs{c} N.
  \end{align*}
  Since \(A\) was arbitrary, we have
  \[
    \sup\set{\sum_{x \in A} \abs{cf(x)} : A \subseteq X, A \text{ finite}} \leq \abs{c}N < \infty.
  \]
  Thus by \cref{i:8.2.4} \(\sum_{x \in X} cf(x)\) is absolutely convergent.

  Now we show that \(\sum_{x \in X} cf(x) = c \sum_{x \in X} f(x)\).
  If \(X\) is at most countable, then the statement follows by \cref{i:ac:8.2.1}(b).
  So suppose that \(X\) is uncountable.
  Let \(X_f = \set{x \in X : f(x) \neq 0}\) and \(X_h = \set{x \in X : cf(x) \neq 0}\) be two sets.
  By \cref{i:8.2.5} we know that both \(X_f\) and \(X_h\) are at most countable.
  \begin{itemize}
    \item If \(c = 0\), then \(X_h = \emptyset\) and we have
          \begin{align*}
            \sum_{x \in X} 0f(x) & = \sum_{x \in X_h} 0f(x) &  & \by{i:8.2.5}     \\
                                 & = 0                      &  & \by{i:7.1.11}[a] \\
                                 & = 0 \sum_{x \in X} f(x).
          \end{align*}
    \item If \(c \neq 0\), then we have \(X_f = X_h\) since
          \[
            \forall x, x \in X_f \iff f(x) \neq 0 \iff cf(x) \neq 0 \iff x \in X_h.
          \]
          Thus
          \begin{align*}
            \sum_{x \in X} cf(x) & = \sum_{x \in X_h} cf(x)  &  & \by{i:8.2.5}       \\
                                 & = c \sum_{x \in X_h} f(x) &  & \by{i:ac:8.2.1}[b] \\
                                 & = c \sum_{x \in X_f} f(x)                         \\
                                 & = c \sum_{x \in X} f(x).  &  & \by{i:8.2.5}
          \end{align*}
  \end{itemize}
  From all cases above we conclude that \(\sum_{x \in X} cf(x) = c \sum_{x \in X} f(x)\).
\end{proof}

\begin{proof}{(c)}
  If \(X\) is at most countable, then the statements follow by \cref{i:ac:8.2.1}(c).
  So suppose that \(X\) is uncountable.

  We first show that if \(X = X_1 \cup X_2\), \(X_1 \cap X_2 = \emptyset\), then \(\sum_{x \in X_1} f(x)\) and \(\sum_{x \in X_2} f(x)\) is absolutely convergent.
  Since \(\sum_{x \in X} f(x)\) is absolutely convergent, by \cref{i:8.2.4} \(\exists N \in \R\) such that
  \[
    N = \sup\set{\sum_{x \in A} \abs{f(x)} : A \subseteq X, A \text{ finite}} < \infty.
  \]
  Let \(A_1 \subseteq X_1, A_2 \subseteq X_2\) and both \(A_1, A_2\) are finite.
  Then we have
  \begin{align*}
     & \sum_{x \in A_1} \abs{f(x)} \leq N, \\
     & \sum_{x \in A_2} \abs{f(x)} \leq N.
  \end{align*}
  Since \(A_1, A_2\) were arbitrary, we have
  \[
    \sup\set{\sum_{x \in A} \abs{f(x)} : A \subseteq X_1, A \text{ finite}} \leq N < \infty
  \]
  and
  \[
    \sup\set{\sum_{x \in A} \abs{f(x)} : A \subseteq X_2, A \text{ finite}} \leq N < \infty.
  \]
  Thus by \cref{i:8.2.4} both \(\sum_{x \in X_1} f(x)\) and \(\sum_{x \in X_2} f(x)\) are absolutely convergent.

  Next we show that if \(X = X_1 \cup X_2\), \(X_1 \cap X_2 = \emptyset\), then
  \[
    \sum_{x \in X_1 \cup X_2} f(x) = \sum_{x \in X_1} f(x) + \sum_{x \in X_2} f(x).
  \]
  Let \(X_f = \set{x \in X : f(x) \neq 0}, X_{f_1} = \set{x \in X_1 : f(x) \neq 0}, X_{f_2} = \set{x \in X_2 : f(x) \neq 0}\) be sets.
  Clearly, we have \(X_{f_1} \cup X_{f_2} = X_f\) and \(X_{f_1} \cap X_{f_2} = \emptyset\).
  By \cref{i:8.2.5} we know that \(X_f, X_{f_1}, X_{f_2}\) are at most countable.
  Then we have
  \begin{align*}
    \sum_{x \in X} f(x) & = \sum_{x \in X_f} f(x)                                 &  & \by{i:8.2.5}       \\
                        & = \sum_{x \in X_{f_1} \cup X_{f_2}} f(x)                                        \\
                        & = \sum_{x \in X_{f_1}} f(x) + \sum_{x \in X_{f_2}} f(x) &  & \by{i:ac:8.2.1}[c] \\
                        & = \sum_{x \in X_1} f(x) + \sum_{x \in X_2} f(x).        &  & \by{i:8.2.5}
  \end{align*}

  Finally we show that if \(X_1 \cup X_2 \subseteq X\), \(X_1 \cap X_2 = \emptyset\), \(h : X \to \R\) is a function such that \(\sum_{x \in X_1} h(x)\) and \(\sum_{x \in X_2} h(x)\) are absolutely convergent, then \(\sum_{x \in X_1 \cup X_2} h(x)\) is also absolutely convergent.
  Since \(\sum_{x \in X_1} h(x)\) and \(\sum_{x \in X_2} h(x)\) are absolutely convergent, by \cref{i:8.2.4} \(\exists N, M \in \R\) such that
  \[
    N = \sup\set{\sum_{x \in A} \abs{h(x)} : A \subseteq X_1, A \text{ finite}} < \infty
  \]
  and
  \[
    M = \sup\set{\sum_{x \in A} \abs{h(x)} : A \subseteq X_2, A \text{ finite}} < \infty.
  \]
  Let \(A \subseteq X_1 \cup X_2\) be a finite set.
  Let \(A_1 = A \cap X_1\) and \(A_2 = A \cap X_2\).
  Clearly, we have
  \begin{align*}
     & A_1 \cap A_2 = \emptyset, \\
     & A_1 \cup A_2 = A,         \\
     & A_1 \subseteq X_1,        \\
     & A_2 \subseteq X_2,        \\
     & A_1 \text{ is finite},    \\
     & A_2 \text{ is finite}.
  \end{align*}
  Then we have
  \begin{align*}
    \sum_{x \in A} \abs{h(x)} & = \sum_{x \in A_1 \cup A_2} \abs{h(x)}                      \\
                              & = \sum_{x \in A_1} \abs{h(x)} + \sum_{x \in A_2} \abs{h(x)} \\
                              & \leq N + M.
  \end{align*}
  Since \(A\) was arbitrary, we have
  \[
    \sup\set{\sum_{x \in A} \abs{h(x)} : A \subseteq X_1 \cup X_2, A \text{ finite}} \leq N + M < \infty
  \]
  Thus by \cref{i:8.2.4} \(\sum_{x \in X_1 \cup X_2} h(x)\) is absolutely convergent.
  Since \(\sum_{x \in X_1 \cup X_2} h(x)\) is absolutely convergent, from the proof above we have
  \[
    \sum_{x \in X_1 \cup X_2} h(x) = \sum_{x \in X_1} h(x) + \sum_{x \in X_2} h(x).
  \]
\end{proof}

\begin{proof}{(d)}
  If \(X\) is at most countable, then the statements follow by \cref{i:ac:8.2.1}(d).
  So suppose that \(X\) is uncountable.

  We first show that \(\sum_{y \in Y} f\big(\phi(y)\big)\) is absolutely convergent.
  Since \(\sum_{x \in X} f(x)\) is absolutely convergent, by \cref{i:8.2.4}, \(\exists N \in \R\) such that
  \[
    N = \sup\set{\sum_{x \in A} \abs{f(x)} : A \subseteq X, A \text{ finite}} < \infty
  \]
  Let \(A \subseteq Y\) be a finite set.
  Then we have
  \begin{align*}
    \sum_{y \in A} \abs{f\big(\phi(y)\big)} & = \sum_{x \in \phi(A)} \abs{f(x)} &  & \by{i:7.1.11}[c]               \\
                                            & \leq N.                           &  & \text{(\(\phi(A)\) is finite)}
  \end{align*}
  Since \(A\) was arbitrary, we have
  \[
    \sup\set{\sum_{y \in A} \abs{f\big(\phi(x)\big)} : A \subseteq Y, A \text{ finite}} \leq N < \infty.
  \]
  Thus by \cref{i:8.2.4} \(\sum_{y \in Y} f\big(\phi(y)\big)\) is absolutely convergent.

  Now we show that \(\sum_{y \in Y} f\big(\phi(y)\big) = \sum_{x \in X} f(x)\).
  Let \(X_f = \set{x \in X : f(x) \neq 0}\) and \(Y_f = \set{y \in Y : f\big(\phi(y)\big) \neq 0}\) be sets.
  By \cref{i:8.2.5} we know that \(X_f\) and \(Y_f\) are at most countable.
  Clearly, \(\phi\) is a bijective between \(X_f\) and \(Y_f\).
  Thus we have
  \begin{align*}
    \sum_{y \in Y} f\big(\phi(y)\big) & = \sum_{y \in Y_f} f\big(\phi(y)\big) &  & \by{i:8.2.5}       \\
                                      & = \sum_{x \in X_f} f(x)               &  & \by{i:ac:8.2.1}[d] \\
                                      & = \sum_{x \in X} f(x).                &  & \by{i:8.2.5}
  \end{align*}
\end{proof}

\begin{lem}\label{i:8.2.7}
  Let \(\sum_{n = 0}^\infty a_n\) be a series of real numbers which is conditionally convergent, but not absolutely convergent.
  Define the sets \(A_+ \coloneqq \set{n \in \N : a_n \geq 0}\) and \(A_- \coloneqq \set{n \in \N : a_n < 0}\), thus \(A_+ \cup A_- = \N\) and \(A_+ \cap A_- = \emptyset\).
  Then both of the series \(\sum_{n \in A_+} a_n\) and \(\sum_{n \in A_-} a_n\) are not absolutely convergent.
\end{lem}

\begin{proof}
  Suppose for sake of contradiction that at least one of the series \(\sum_{n \in A_+} a_n\) and \(\sum_{n \in A_-} a_n\) is absolutely convergent.
  Let \(b_n = \max(a_n, 0)\) and \(c_n = -\min(a_n, 0)\).
  Then we have \(a_n = b_n - c_n\) and
  \begin{align*}
    \sum_{n = 0}^\infty a_n & = \sum_{n = 0}^\infty b_n - c_n                                                             \\
                            & = \sum_{n = 0}^\infty b_n - \sum_{n = 0}^\infty c_n                   &  & \by{i:7.2.14}[a] \\
                            & = \sum_{n = 0}^\infty \max(a_n, 0) + \sum_{n = 0}^\infty \min(a_n, 0)                       \\
                            & = \sum_{n \in A} \max(a_n, 0) + \sum_{n \in A} \min(a_n, 0)           &  & \by{i:8.2.1}     \\
                            & = \sum_{n \in A_+} a_n + \sum_{n \in A_-} a_n.                        &  & \by{i:8.2.5}
  \end{align*}
  Thus \(\sum_{n \in A_+} a_n\) and \(\sum_{n \in A_-} a_n\) converges.
  Since \(\sum_{n \in A_+} a_n\) converges and
  \[
    \sum_{n \in A_+} \abs{a_n} = \sum_{n \in A_+} a_n,
  \]
  we know that \(\sum_{n \in A_+} a_n\) is absolutely converges.
  Since
  \begin{align*}
    \sum_{n \in A_-} \abs{a_n} & = \sum_{n = 0}^\infty \abs{\min(a_n, 0)} &  & \by{i:8.2.5}     \\
                               & = \sum_{n = 0}^\infty -\min(a_n, 0)                            \\
                               & = -\sum_{n = 0}^\infty \min(a_n, 0)      &  & \by{i:7.2.14}[b] \\
                               & = -\sum_{n \in A_-} a_n,                 &  & \by{i:8.2.5}
  \end{align*}
  we know that \(\sum_{n \in A_-} \abs{a_n}\) is absolutely convergent.
  But by \cref{i:8.2.6}(c) we have
  \[
    \sum_{n \in A_+} a_n + \sum_{n \in A_-} a_n = \sum_{n \in A} a_n
  \]
  and \(\sum_{n \in A} a_n\) is absolutely convergent, a contradiction.
  Thus both \(\sum_{n \in A_+} a_n\) and \(\sum_{n \in A_-} a_n\) are not absolutely convergent.
\end{proof}

\begin{note}
  \cref{i:8.2.8} is done by Georg Riemann (1826--1866), which asserts that a series which converges conditionally but not absolutely can be rearranged to converge to any value one pleases!
\end{note}

\begin{thm}\label{i:8.2.8}
  Let \(\sum_{n = 0}^\infty a_n\) be a series which is conditionally convergent, but not absolutely convergent, and let \(L\) be any real number.
  Then there exists a bijection \(f : \N \to \N\) such that \(\sum_{m = 0}^\infty a_{f(m)}\) converges conditionally to \(L\).
\end{thm}

\begin{proof}
  Let \(A_+\) and \(A_-\) be the sets in \cref{i:8.2.7};
  from \cref{i:8.2.7} we know that \(\sum_{n \in A_+} a_n\) and \(\sum_{n \in A_-} a_n\) both fail to be absolutely convergent.
  In particular \(A_+\) and \(A_-\) are infinite.
  (If \(A_-\) is finite, then \(\sum_{n \in A_-} a_n\) is absolutely convergent.
  If \(A_+\) is finite, then \(\sum_{n \in A_+} a_n\) is also absolutely convergent.)
  By \cref{i:8.1.5} we can then find increasing bijections \(f_+ : \N \to A_+\) and \(f_- : \N \to A_-\).
  Thus the sums \(\sum_{m = 0}^\infty a_{f_+(m)}\) and \(\sum_{m = 0}^\infty a_{f_-(m)}\) both fail to be absolutely convergent (first by \cref{i:8.2.1} then by \cref{i:8.2.7}).
  The plan shall be to select terms from the divergent series \(\sum_{m = 0}^\infty a_{f_+(m)}\) and \(\sum_{m = 0}^\infty a_{f_-(m)}\) in a well-chosen order in order to keep their difference converging towards \(L\).

  We define the sequence \(n_0, n_1, n_2, \dots\) of natural numbers recursively as follows.
  Suppose that \(j\) is a natural number, and that \(n_i\) has already been defined for all \(i < j\) (this is vacuously true if \(j = 0\)).
  We then define \(n_j\) by the following rule:
  \begin{enumerate}[label=(\Roman*)]
    \item If \(\sum_{0 \leq i < j} a_{n_i} < L\), then we set
          \[
            n_j \coloneqq \min\set{n \in A_+ : n \neq n_i \text{ for all } i < j}.
          \]
    \item If instead \(\sum_{0 \leq i < j} a_{n_i} \geq L\), then we set
          \[
            n_j \coloneqq \min\set{n \in A_- : n \neq n_i \text{ for all } i < j}.
          \]
  \end{enumerate}
  Note that this recursive definition is well-defined because \(A_+\) and \(A_-\) are infinite, and so the sets \(\set{n \in A_+ : n \neq n_i \text{ for all } i < j}\) and \(\set{n \in A_- : n \neq n_i \text{ for all } i < j}\) are never empty.
  (Intuitively, we add a non-negative number to the series whenever the partial sum is too low, and add a negative number when the sum is too high.)
  One can then verify the following claims:
  \begin{itemize}
    \item The map \(j \mapsto n_j\) is injective.
          This is true since
          \begin{align*}
                     & \forall j_1, j_2 \in \N, j_1 \neq j_2 \\
            \implies & j_1 < j_2 \lor j_1 > j_2              \\
            \implies & n_{j_1} \neq n_{j_2}.
          \end{align*}
    \item Case I occurs an infinite number of times, and Case II also occurs an infinite number of times.
          We prove this by contradiction.
          Suppose for sake of contradiction that case I occurs only finite number of times.
          Then we have
          \[
            \Bigg(\sum_{0 \leq i < j} a_{n_i} < L\Bigg) \land \Bigg(\sum_{0 \leq i < j} a_{n_i} + a_{n_j} + \sum_{i > j} a_{n_i} \geq L\Bigg)
          \]
          where \(j\) is the last time case I occurs.
          Since \(\sum_{0 \leq i < j} a_{n_i} + a_{n_j}\) is finite, \(\sum_{i > j} a_{n_i}\) have a lower bound.
          Since all \(i > j\) are cases II, \(\sum_{i > j} a_{n_i}\) is decreasing.
          Since \(\sum_{i > j} a_{n_i}\) is decreasing and has lower bound, by \cref{i:ac:6.3.1} \(\sum_{i > j} a_{n_i}\) is convergent.
          But this means \(\sum_{n \in A_-} a_n\) is absolutely convergent, a contradiction.
          Thus case I occurs infinite number of times.
          Similar proof show that case II also occurs infinite number of times.
    \item The map \(j \mapsto n_j\) is surjective.
          We know that \(\forall n \in \N\), either \(n \in A_+\) or \(n \in A_-\).
          If \(n \in A_+\) and there is no \(j \mapsto n\), then \(\forall n' > n\) there must also have no \(j \mapsto n'\), otherwise by definition we must have \(n = \min\set{n \in A_+ : n \neq n_i \text{ for all } i < j}\).
          But then case I only occur finite number of times, a contradiction.
          Thus \(\exists j \mapsto n\).
          Similar argument show that if \(n \in A_-\) then \(\exists j \mapsto n\).
          Thus \(j \mapsto n_j\) is surjective.
    \item We have \(\lim_{j \to \infty} a_{n_j} = 0\).
          By \cref{i:7.2.6} we have \(\lim_{j \to \infty} a_j = 0\).
          This means
          \[
            \forall \varepsilon \in \R^+, \exists N \in \N : \forall j \geq N, \abs{a_j - 0} \leq \varepsilon.
          \]
          Thus the set \(E = \set{j \in \N : a_j - 0 > \varepsilon}\) is finite.
          Since \(j \to n_j\) is bijective, we know that the set \(E' = \set{n_j \in \N : j \in E}\) is also finite.
          Let \(M = \max(E')\).
          Then we have
          \[
            \forall n_j \geq M, \abs{a_{n_j} - 0} \leq \varepsilon.
          \]
          Thus \(\lim_{j \to \infty} a_{n_j} = 0\).
    \item We have \(\lim_{j \to \infty} \sum_{0 \leq i \leq j} a_{n_i} = L\).
          Since \(\lim_{j \to \infty} a_{n_j} = 0\), we have
          \[
            \forall \varepsilon \in \R^+, \exists N \in \N : \forall j \geq N, \abs{a_{n_j} - 0} \leq \varepsilon.
          \]
          Let \(K\) be the set
          \[
            K = \set{k \in \N : (k \geq j) \land \Bigg(\sum_{i = 0}^k a_{n_i} < L\Bigg) \land \Bigg(\sum_{i = 0}^{k + 1} a_{n_i} \geq L\Bigg)}.
          \]
          We know that \(K \neq \emptyset\) since case II occurs infinite number of times.
          Let \(k = \min(K)\).
          Such \(k\) is well-defined by well ordering principle (\cref{i:8.1.4}).
          Now we show that for every \(p \in \N\), we have
          \[
            L - \varepsilon \leq \sum_{i = 0}^{k + p} a_{n_i} \leq L + \varepsilon.
          \]
          we induct on \(p\).
          For \(p = 0\), we have
          \begin{align*}
                     & \sum_{i = 0}^{k + p} a_{n_i} = \sum_{i = 0}^k a_{n_i}                                                                                                \\
            \implies & \sum_{i = 0}^{k + 1} a_{n_i} \geq L                                   &                                        & \text{(by the definition of \(k\))} \\
            \implies & \sum_{i = 0}^k a_{n_i} + a_{n_{k + 1}} \geq L                                                                                                        \\
            \implies & \sum_{i = 0}^k a_{n_i} \geq L - a_{n_{k + 1}} \geq L - \varepsilon    & (\abs{a_{n_{k + 1}}} \leq \varepsilon)                                       \\
            \implies & L + \varepsilon \geq L > \sum_{i = 0}^k a_{n_i} \geq L - \varepsilon. &                                        & \text{(by the definition of \(k\))}
          \end{align*}
          Thus, the base case holds.
          Suppose inductively that for some \(p \geq 0\) the statement is true.
          Then we need to show that for \(p + 1\) the statement is also true.
          We split into two cases:
          \begin{itemize}
            \item If \(a_{n_{p + 1}} \geq 0\), then this means case I happened.
                  Thus we have
                  \begin{align*}
                             & (0 \leq a_{n_{p + 1}} \leq \varepsilon) \land \Bigg(\sum_{i = 0}^{k + p} a_{n_i} < L\Bigg)                      &  & \text{(case I)} \\
                    \implies & (0 \leq a_{n_{p + 1}} \leq \varepsilon) \land \Bigg(L - \varepsilon \leq \sum_{i = 0}^{k + p} a_{n_i} < L\Bigg) &  & \byIH           \\
                    \implies & L - \varepsilon \leq \sum_{i = 0}^{k + p} a_{n_i} + a_{n_{p + 1}} \leq L + \varepsilon                                               \\
                    \implies & L - \varepsilon \leq \sum_{i = 0}^{k + p + 1} a_{n_i} \leq L + \varepsilon.
                  \end{align*}
            \item If \(a_{n_{p + 1}} < 0\), then this means case II happened.
                  Thus we have
                  \begin{align*}
                             & (-\varepsilon \leq a_{n_{p + 1}} < 0) \land \Bigg(L \leq \sum_{i = 0}^{k + p} a_{n_i}\Bigg) &  & \text{(case II)} \\
                    \implies & (-\varepsilon \leq a_{n_{p + 1}} < 0)                                                                             \\
                             & \land \Bigg(L \leq \sum_{i = 0}^{k + p} a_{n_i} \leq L + \varepsilon\Bigg)                  &  & \byIH            \\
                    \implies & L - \varepsilon \leq \sum_{i = 0}^{k + p} a_{n_i} + a_{n_{p + 1}} \leq L + \varepsilon                            \\
                    \implies & L - \varepsilon \leq \sum_{i = 0}^{k + p + 1} a_{n_i} \leq L + \varepsilon.
                  \end{align*}
          \end{itemize}
          From all cases above we conclude that \(L - \varepsilon \leq \sum_{i = 0}^{k + p + 1} a_{n_i} \leq L + \varepsilon\).
          This closes the induction.
          This means \(\forall p \geq 0\), we have
          \[
            L - \varepsilon \leq \sum_{i = 0}^{k + p} \leq L + \varepsilon \iff \abs{\sum_{i = 0}^{k + p} - L} \leq \varepsilon.
          \]
          Rewritting with \(q = k + p\), we have
          \[
            \forall \varepsilon \in \R^+, \exists k \geq 0 : \forall q \geq k, \abs{\sum_{i = 0}^q - L} \leq \varepsilon.
          \]
          Thus \(\lim_{q \to \infty} \sum_{i = 0}^q a_{n_i} = L\).
  \end{itemize}
  The claim then follows by setting \(f(i) \coloneqq n_i\) for all \(i \in \N\).
\end{proof}

\exercisesection

\begin{ex}\label{i:ex:8.2.1}
  Prove \cref{i:8.2.3}.
\end{ex}

\begin{proof}
  See \cref{i:8.2.3}.
\end{proof}

\begin{ex}\label{i:ex:8.2.2}
  Prove \cref{i:8.2.5}.
\end{ex}

\begin{proof}
  See \cref{i:8.2.5}.
\end{proof}

\begin{ex}\label{i:ex:8.2.3}
  Prove \cref{i:8.2.6}.
\end{ex}

\begin{proof}
  See \cref{i:8.2.6}.
\end{proof}

\begin{ex}\label{i:ex:8.2.4}
  Prove \cref{i:8.2.7}.
\end{ex}

\begin{proof}
  See \cref{i:8.2.7}.
\end{proof}

\begin{ex}\label{i:ex:8.2.5}
  Explain the gaps marked (why?) in the proof of \cref{i:8.2.8}.
\end{ex}

\begin{proof}
  See \cref{i:8.2.8}.
\end{proof}

\begin{ex}\label{i:ex:8.2.6}
  Let \(\sum_{n = 0}^\infty a_n\) be a series which is conditionally convergent, but not absolutely convergent.
  Show that there exists a bijection \(f : \N \to \N\) such that \(\sum_{m = 0}^\infty a_{f(m)}\) diverges to \(+\infty\), or more precisely that
  \[
    \liminf_{N \to \infty} \sum_{m = 0}^N a_{f(m)} = \limsup_{N \to \infty} \sum_{m = 0}^N a_{f(m)} = +\infty.
  \]
  (Of course, a similar statement holds with \(+\infty\) replaced by \(-\infty\).)
\end{ex}

\begin{proof}
  Let \(A_+\) and \(A_-\) defined as \cref{i:8.2.7}.
  In \cref{i:8.2.8} we know that both \(A_+\) and \(A_-\) are countable, and there exist two increasing bijections \(f_+ : \N \to A_+\) and \(f_- : \N \to A_-\).
  We know that both \(\sum_{m = 0}^\infty a_{f_+(m)}\) and \(\sum_{m = 0}^\infty a_{f_-(m)}\) fail to be absolutely convergent.

  We first show that there exists a bijection \(f : \N \to \N\) such that \(\sum_{m = 0}^\infty a_{f(m)}\) diverges to \(+\infty\).
  Let \(L_0 = 0\).
  Suppose that \(j \in \N\), and \(n_i\) has been defined for all \(i < j\)
  (this is vacuously true if \(j = 0\)).
  We define \(n_j\) by the following rule:
  \begin{enumerate}[label=(\Roman*)]
    \item If \(\sum_{0 \leq i < j} a_{n_i} < L_j\), then we set
          \begin{align*}
            n_j       & = \min\set{n \in A_+ : n \neq n_i \text{ for all } i < j}; \\
            L_{j + 1} & = L_j.
          \end{align*}
    \item If \(\sum_{0 \leq i < j} a_{n_i} \geq L_j\), then we set
          \begin{align*}
            n_j       & = \min\set{n \in A_- : n \neq n_i \text{ for all } i < j}; \\
            L_{j + 1} & = L_j + 1.
          \end{align*}
  \end{enumerate}
  Now we verify the following claims:
  \begin{itemize}
    \item The map \(j \mapsto n_j\) is injective.
          Suppose that \(i, j \in \N\) and \(i \neq j\).
          Then by the definition of \(n_i, n_j\) we know that \(n_i \neq n_j\).
          Thus \(j \mapsto n_j\) is injective.
    \item Both Case I and II occur infinite number of times.
          Obviously, at least one case must occur infinite number of times.
          Suppose for sake of contradiction that Case I only occurs finite number of times.
          Let \(j\) be the largest number such that Case I occurs, i.e.,
          \[
            \Bigg(\sum_{0 \leq i < j} a_{n_i} < L_j\Bigg) \land \Bigg(\sum_{0 \leq i \leq j} a_{n_i} \geq L_j\Bigg).
          \]
          Then \(\forall k \in \N\) and \(k > j\), Case II occurs, i.e.,
          \[
            S_k = \sum_{i = 0}^k a_{n_i} \geq L_k.
          \]
          Since Case II occurs, we know that \(S_k\) is decreasing and \(L_k\) is increasing.
          Thus
          \begin{align*}
                     & S_k > S_{k + 1} \geq L_{k + 1} > L_k \geq 0                               \\
            \implies & (\lim_{k \to \infty} S_k \text{ converges})         &  & \by{i:ac:6.3.1}  \\
                     & \land (S_k - L_k \geq S_{k + 1} - L_{k + 1} \geq 0)                       \\
            \implies & \lim_{k \to \infty} S_k - L_k \text{ converges}     &  & \by{i:ac:6.3.1}  \\
            \implies & \lim_{k \to \infty} L_k \text{ converges}.          &  & \by{i:6.1.19}[a]
          \end{align*}
          But since Case II occurs infinite number of times, we know that \(\lim_{k \to \infty} L_k\) diverges to \(\infty\), a contradiction.
          Thus case I must occurs infinite number of times.

          Now Suppose for sake of contradiction that Case II only occurs finite number of times.
          Let \(j\) be the largest number such that Case II occurs, i.e.,
          \[
            \Bigg(\sum_{0 \leq i < j} a_{n_i} \geq L_j\Bigg) \land \Bigg(\sum_{0 \leq i \leq j} a_{n_i} < L_j\Bigg).
          \]
          Then \(\forall k \in \N\) and \(k > j\), Case I occurs, i.e.,
          \[
            S_k = \sum_{i = 0}^k a_{n_i} < L_k.
          \]
          Since Case I occurs, we know that \(S_k\) is increasing.
          Thus
          \begin{align*}
                     & S_k < S_{k + 1} < L_{k + 1} = L_k                                                                           \\
            \implies & \lim_{k \to \infty} S_k \text{ converges}               &                                 & \by{i:6.3.8}    \\
            \implies & \sum_{k = j + 1}^\infty a_{n_k} \text{ converges}       &                                 & \by{i:7.2.2}    \\
            \implies & \sum_{k = j + 1}^\infty \abs{a_{n_k}} \text{ converges} & (\forall k > j, a_{n_k} \geq 0)                   \\
            \implies & \sum_{k \in A_+} \abs{a_k} \text{ converges}            &                                 & \by{i:8.2.6}[c]
          \end{align*}
          But we know that \(\sum_{k \in A_+} \abs{a_k}\) is not absolutely convergent, a contradiction.
          Thus case II must occurs infinite number of times.
          We conclude that both Case I and II occur infinite number of times.
    \item The map \(j \mapsto n_j\) is surjective.
          We know that \(\forall n \in \N\), either \(n \in A_+\) or \(n \in A_-\).
          If \(n \in A_+\) and there is no \(j \mapsto n\), then \(\forall n' > n\) there must also have no \(j \mapsto n'\), otherwise by definition we must have \(n = \min\set{n \in A_+ : n \neq n_i \text{ for all } i < j}\).
          But then case I only occur finite number of times, a contradiction.
          Thus \(\exists j \mapsto n\).
          Similar argument show that if \(n \in A_-\) then \(\exists j \mapsto n\).
          Thus \(j \mapsto n_j\) is surjective.
    \item We have \(\limsup_{j \to \infty} \sum_{i = 0}^j a_{n_j}\) diverges to \(\infty\).
          Since Case II occurs infinite number of times, we know that for every \(M \in \R^+\), \(\exists j \geq 0\) such that \(M \leq L_i\).
          This means
          \[
            \exists k \in \N \land k > j : M \leq L_i \leq \sum_{i = 0}^k a_{n_i}.
          \]
          Since this is true for every \(M \in \R^+\), by \cref{i:ex:6.4.8} we thus have
          \[
            \limsup_{j \to \infty} \sum_{i = 0}^j a_{n_j} = \infty.
          \]
  \end{itemize}
  The claim then follows by setting \(f(i) \coloneqq n_i\) for all \(i \in \N\).
  Similar proof can be used to show the case \(-\infty\).
\end{proof}
