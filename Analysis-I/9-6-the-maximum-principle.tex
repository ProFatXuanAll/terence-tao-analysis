\section{The maximum principle}\label{sec 9.6}

\begin{definition}\label{9.6.1}
  Let \(X\) be a subset of \(\R\), and let \(f : X \to \R\) be a function.
  We say that \(f\) is \emph{bounded from above} iff there exists a real number \(M\) such that \(f(x) \leq M\) for all \(x \in X\).
  We say that \(f\) is \emph{bounded from below} iff there exists a real number \(M\) such that \(f(x) \geq -M\) for all \(x \in X\).
  We say that \(f\) is \emph{bounded} iff there exists a real number \(M\) such that \(\abs{f(x)} \leq M\) for all \(x \in X\).
\end{definition}

\begin{remark}\label{9.6.2}
  A function is bounded if and only if it is bounded both from above and below.
  Also, a function \(f : X \to \R\) is bounded if and only if its image \(f(X)\) is a bounded set in the sense of \cref{9.1.22}.
\end{remark}

\begin{lemma}\label{9.6.3}
  Let \(a < b\) be real numbers, and let \(f : [a, b] \to \R\) be a function continuous on \([a, b]\).
  Then \(f\) is a bounded function.
\end{lemma}

\begin{proof}
  Suppose for sake of contradiction that \(f\) is not bounded.
  Thus for every real number \(M\) there exists an element \(x \in [a, b]\) such that \(\abs{f(x)} \geq M\).

  In particular, for every natural number \(n\), the set \(\{x \in [a, b] : \abs{f(x)} \geq n\}\) is non-empty.
  We can thus choose a sequence \((x_n)_{n = 0}^\infty\) in \([a, b]\) such that \(\abs{f(x_n)} \geq n\) for all \(n\).
  This sequence lies in \([a, b]\), and so by \cref{9.1.24} there exists a subsequence \((x_{n_j})_{j = 0}^\infty\) which converges to some limit \(L \in [a, b]\), where \(n_0 < n_1 < n_2 < \dots\) is an increasing sequence of natural numbers.
  In particular, we see that \(n_j \geq j\) for all \(j \in \N\) (use induction).

  Since \(f\) is continuous on \([a, b]\), it is continuous at \(L\), and in particular we see that
  \[
    \lim_{j \to \infty} f(x_{n_j}) = f(L).
  \]
  Thus the sequence \((f(x_{n_j}))_{j = 0}^\infty\) is convergent, and hence it is bounded.
  On the other hand, we know from the construction that \(\abs{f(x_{n_j})} \geq n_j \geq j\) for all \(j\), and hence the sequence \((f(x_{n_j}))_{j = 0}^\infty\) is not bounded, a contradiction.
\end{proof}

\begin{remark}\label{9.6.4}
  There are two things about the proof of \cref{9.6.3} that are worth noting.
  Firstly, it shows how useful the Heine-Borel theorem (\cref{9.1.24}) is.
  Secondly, it is an indirect proof;
  it doesn't say \emph{how} to find the bound for \(f\), but it shows that having \(f\) unbounded leads to a contradiction.
\end{remark}

\begin{definition}[Maxima and minima]\label{9.6.5}
  Let \(X\) be a subset of \(\R\), and let \(f : X \to \R\) be a function, and let \(x_0 \in X\).
  We say that \emph{\(f\) attains its maximum at \(x_0\)} if we have \(f(x_0) \geq f(x)\) for all \(x \in X\)
  (i.e., the value of \(f\) at the point \(x_0\) is larger than or equal to the value of \(f\) at any other point in \(X\)).
  We say that \emph{\(f\) attains its minimum at \(x_0\)} if we have \(f(x_0) \leq f(x)\) for all \(x \in X\).
\end{definition}

\begin{remark}\label{9.6.6}
  If a function attains its maximum somewhere, then it must be bounded from above.
  Similarly if it attains its minimum somewhere, then it must be bounded from below.
  These notions of maxima and minima are \emph{global}.
\end{remark}

\begin{proposition}[Maximum principle]\label{9.6.7}
  Let \(a < b\) be real numbers, and let \(f : [a, b] \to \R\) be a function continuous on \([a, b]\).
  Then \(f\) attains its maximum at some point \(x_{\max} \in [a, b]\), and also attains its minimum at some point \(x_{\min} \in [a, b]\).
\end{proposition}

\begin{proof}
  From \cref{9.6.3} we know that \(f\) is bounded, thus there exists an \(M\) such that \(-M \leq f(x) \leq M\) for each \(x \in [a, b]\).
  Now let \(E\) denote the
  set
  \[
    E \coloneqq \{f(x) : x \in [a, b]\}.
  \]
  (In other words, \(E \coloneqq f([a, b])\).)
  By what we just said, this set is a subset of \([-M, M]\).
  It is also non-empty, since it contains for instance the point \(f(a)\).
  Hence by the least upper bound principle, it has a supremum \(\sup(E)\) which is a real number.

  Write \(m \coloneqq \sup(E)\).
  By definition of supremum, we know that \(y \leq m\) for all \(y \in E\);
  by definition of \(E\), this means that \(f(x) \leq m\) for all \(x \in [a, b]\).
  Thus to show that \(f\) attains its maximum somewhere, it will suffice to find an \(x_{\max} \in [a, b]\) such that \(f(x_{\max}) = m\).

  Let \(n \geq 1\) be any integer.
  Then \(m - \frac{1}{n} < m = \sup(E)\).
  As \(\sup(E)\) is the least upper bound for \(E\), \(m - \frac{1}{n}\) cannot be an upper bound for \(E\), thus there exists a \(y \in E\) such that \(m - \frac{1}{n} < y\).
  By definition of \(E\), this implies that there exists an \(x \in [a, b]\) such that \(m - \frac{1}{n} < f(x)\).

  We now choose a sequence \((x_n)_{n = 1}^\infty\) by choosing, for each \(n\), \(x_n\) to be an element of \([a, b]\) such that \(m - \frac{1}{n} < f(x_n)\).
  (Again, this requires the axiom of choice;
  however it is possible to prove this principle without the axiom of choice.
  For instance, you will see a better proof of this proposition using the notion of \emph{compactness})
  This is a sequence in \([a, b]\);
  by the Heine-Borel theorem (\cref{9.1.24}), we can thus find a subsequence \((x_{n_j})_{j = 1}^\infty\), where \(n_1 < n_2 < \dots\), which converges to some limit \(x_{\max} \in [a, b]\).
  Since \((x_{n_j})_{j = 1}^\infty\) converges to \(x_{\max}\), and \(f\) is continuous at \(x_{\max}\), we have as before that
  \[
    \lim_{j \to \infty} f(x_{n_j}) = f(x_{\max})
  \]
  On the other hand, by construction we know that
  \[
    f(x_{n_j}) > m - \frac{1}{n_j} \geq m - \frac{1}{j},
  \]
  and so by taking limits of both sides we see that
  \[
    f(x_{\max}) = \lim_{j \to \infty} f(x_{n_j}) \geq \lim_{j \to \infty} m - \frac{1}{j} = m.
  \]
  On the other hand, we know that \(f(x) \leq m\) for all \(x \in [a, b]\), so in particular \(f(x_{\max}) \leq m\).
  Combining these two inequalities we see that \(f(x_{\max}) = m\) as desired.

  By the least upper bound principle again, \(E\) has a infimum \(\inf(E)\) which is a real number.
  Write \(m \coloneqq \inf(E)\).
  By definition of infimum, we know that \(y \geq m\) for all \(y \in E\);
  by definition of \(E\), this means that \(f(x) \geq m\) for all \(x \in [a, b]\).
  Thus to show that \(f\) attains its minimum somewhere, it will suffice to find an \(x_{\min} \in [a, b]\) such that \(f(x_{\min}) = m\).

  Let \(n \geq 1\) be any integer.
  Then \(m + \frac{1}{n} > m = \inf(E)\).
  As \(\inf(E)\) is the greatest lower bound for \(E\), \(m + \frac{1}{n}\) cannot be an lower bound for \(E\), thus there exists a \(y \in E\) such that \(m + \frac{1}{n} > y\).
  By definition of \(E\), this implies that there exists an \(x \in [a, b]\) such that \(m + \frac{1}{n} > f(x)\).

  We now choose a sequence \((x_n)_{n = 1}^\infty\) by choosing, for each \(n\), \(x_n\) to be an element of \([a, b]\) such that \(m + \frac{1}{n} > f(x_n)\).
  (Again, this requires the axiom of choice)
  This is a sequence in \([a, b]\);
  by the Heine-Borel theorem (\cref{9.1.24}), we can thus find a subsequence \((x_{n_j})_{j = 1}^\infty\), where \(n_1 < n_2 < \dots\), which converges to some limit \(x_{\min} \in [a, b]\).
  Since \((x_{n_j})_{j = 1}^\infty\) converges to \(x_{\min}\), and \(f\) is continuous at \(x_{\min}\), we have as before that
  \[
    \lim_{j \to \infty} f(x_{n_j}) = f(x_{\min})
  \]
  On the other hand, by construction we know that
  \[
    f(x_{n_j}) < m + \frac{1}{n_j} \leq m + \frac{1}{j},
  \]
  and so by taking limits of both sides we see that
  \[
    f(x_{\min}) = \lim_{j \to \infty} f(x_{n_j}) \leq \lim_{j \to \infty} m + \frac{1}{j} = m.
  \]
  On the other hand, we know that \(f(x) \geq m\) for all \(x \in [a, b]\), so in particular \(f(x_{\min}) \geq m\).
  Combining these two inequalities we see that \(f(x_{\min}) = m\) as desired.
\end{proof}

\begin{remark}\label{9.6.8}
  Strictly speaking, ``maximum principle'' is a misnomer, since the principle also concerns the minimum.
  Perhaps a more precise name would have been ``extremum principle'';
  the word ``extremum'' is used to denote either a maximum or a minimum.
\end{remark}

\begin{note}
  The maximum principle (\cref{9.6.7}) does not prevent a function from attaining its maximum or minimum at more than one point.
\end{note}

\begin{note}
  Let us write \(\sup_{x \in [a, b]} f(x)\) as short-hand for \(\sup\{f(x) : x \in [a, b]\}\), and similarly define \(\inf_{x \in [a, b]} f(x)\).
  The maximum principle (\cref{9.6.7}) thus asserts that \(m \coloneqq \sup_{x \in [a, b]} f(x)\) is a real number and is the maximum value of \(f\) on \([a, b]\), i.e., there is at least one point \(x_{\max}\) in \([a, b]\) for which \(f(x_{\max}) = m\), and for every other \(x \in [a, b]\), \(f(x)\) is less than or equal to \(m\).
  Similarly \(\inf_{x \in [a, b]} f(x)\) is the minimum value of \(f\) on \([a, b]\).
\end{note}

\begin{remark}\label{9.6.9}
  You may encounter a rather different ``maximum principle'' in complex analysis or partial differential equations, involving analytic functions and harmonic functions respectively, instead of continuous functions.
  Those maximum principles are not directly related to this one
  (though they are also concerned with whether maxima exist, and where the maxima are located).
\end{remark}

\exercisesection

\begin{exercise}\label{ex 9.6.1}
  Give example of
  \begin{enumerate}
    \item a function \(f : (1, 2) \to \R\) which is continuous and bounded, attains its minimum somewhere, but does not attain its maximum anywhere;
    \item a function \(f : [0, \infty) \to \R\) which is continuous, bounded, attains its maximum somewhere, but does not attain its minimum anywhere;
    \item a function \(f : [-1, 1] \to \R\) which is bounded but does not attain its minimum anywhere or its maximum anywhere.
    \item a function \(f : [-1, 1] \to \R\) which has no upper bound and no lower bound.
  \end{enumerate}
  Explain why none of the examples you construct violate the maximum principle.
\end{exercise}

\begin{proof}
  Let \(f_a : (1, 2) \to \R\) be a function where \(f_a(x) = \abs{x - 1.5}\).
  Then by \cref{9.4.12} and \cref{9.4.13} \(f_a\) is continuous.
  Since \(f_a\big((1, 2)\big) = [0, 0.5)\), by \cref{9.6.1} \(f_a\) is bounded.
  Since \(f_a\big((1, 2)\big) = [0, 0.5)\), by \cref{9.6.5} \(f_a\) attains its minimum at \(x = 1.5\).
  Since \(0.5 \notin f_a\big((1, 2)\big)\), by \cref{9.6.5} \(f_a\) does not attain its maximum anywhere.
  Since the domain of \(f_a\) is not closed, \(f_a\) does not violate the maximum principle (\cref{9.6.7}).

  Let \(f_b : [0, \infty) \to \R\) be a function where \(f_b(x) = 0.5^x\).
  Then by \cref{9.4.10} \(f_b\) is continuous.
  Since \(f_b\big([0, \infty)\big) = (0, 1]\), by \cref{9.6.1} \(f_b\) is bounded.
  Since \(f_b\big([0, \infty)\big) = (0, 1]\), by \cref{9.6.5} \(f_b\) attains its maximum at \(x = 0\).
  Since \(0 \notin f_b\big([0, \infty)\big)\), by \cref{9.6.5} \(f_b\) does not attain its minimum anywhere.
  Since the domain of \(f_b\) is not closed, \(f_b\) does not violate the maximum principle (\cref{9.6.7}).

  Let \(f_c : [-1, 1] \to \R\) be a function where
  \[
    f(x) = \begin{cases}
      0 & \text{if } x = -1;        \\
      x & \text{if } x \in (-1, 1); \\
      0 & \text{if } x = 1.
    \end{cases}
  \]
  Since \(f_c\big([-1, 1]\big) = (-1, 1)\), by \cref{9.6.1} \(f_c\) is bounded.
  Since \(-1 \notin f_c\big([-1, 1]\big)\), by \cref{9.6.5} \(f_c\) does not attain its minimum anywhere.
  Since \(1 \notin f_c\big([-1, 1]\big)\), by \cref{9.6.5} \(f_c\) does not attain its maximum anywhere.
  Since \(f_c\) is not continuous on its domain, \(f_c\) does not violate the maximum principle (\cref{9.6.7}).

  Let \(f_d : [-1, 1] \to \R\) be a function where
  \[
    f(x) = \begin{cases}
      1 / x & \text{if } x \in [-1, 0) \cap (0, 1]; \\
      0     & \text{if } x = 0.
    \end{cases}
  \]
  Since \(f_d\big([-1, 1]\big) = (-\infty, -1) \cup \{0\} \cup (1, \infty)\), by \cref{9.6.1} \(f_d\) is unbounded.
  Since \(f_d\) is not bounded, \(f_d\) does not violate the maximum principle (\cref{9.6.7}).
\end{proof}

\begin{exercise}\label{ex 9.6.2}
  Let \(X \subseteq \R\).
  If \(f, g: X \to \R\) are bounded functions, show that \(f + g\), \(f - g\), and \(f \cdot g\) are also bounded functions.
  If we furthermore assume that \(g(x) \neq 0\) for all \(x \in X\), is it true that \(f / g\) is bounded?
  Prove this or give a counterexample.
\end{exercise}

\begin{proof}
  Suppose that \(f\) is bounded by \(M \in \R^+\) and \(g\) is bounded by \(N \in \R^+\).
  Then we have
  \begin{align*}
             & \forall x \in X, \big(\abs{f(x)} \leq M\big) \land \big(\abs{g(x)} \leq N\big) & \text{(by \cref{9.6.1})} \\
    \implies & \big(-M \leq f(x) \leq M\big) \land \big(-N \leq g(x) \leq N\big)                                         \\
    \implies & \big(-M \leq f(x) \leq M\big) \land \big(N \geq -g(x) \geq -N\big)                                        \\
    \implies & \begin{cases}
                 -(M + N) \leq f(x) + g(x) \leq M + N \\
                 -(M + N) \leq f(x) - g(x) \leq M + N \\
                 -MN \leq f(x) g(x) \leq MN
               \end{cases}                                                                      \\
    \implies & \begin{cases}
                 \abs{f(x) + g(x)} \leq M + N \\
                 \abs{f(x) - g(x)} \leq M + N \\
                 \abs{f(x) g(x)} \leq MN
               \end{cases}
  \end{align*}
  Thus by \cref{9.6.1} \(f + g\), \(f - g\), \(f \cdot g\) are bounded.

  Now we show an counterexample when \(g(x) \neq 0\) for all \(x \in X\).
  Let \(X = (0, \infty)\), let \(f(x) = 1\) and let \(g(x) = 1 / x\).
  Then \(g(x) \neq 0\) for all \(x \in X\) and is bounded by \(1\), but \((f / g)(X) = (0, \infty)\) is unbounded.
\end{proof}