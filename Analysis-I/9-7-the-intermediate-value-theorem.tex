\section{The intermediate value theorem}\label{sec 9.7}

\begin{theorem}[Intermediate value theorem]\label{9.7.1}
    Let \(a < b\), and let \(f : [a, b] \to \mathbf{R}\) be a continuous function on \([a, b]\).
    Let \(y\) be a real number between \(f(a)\) and \(f(b)\), i.e., either \(f(a) \leq y \leq f(b)\) or \(f(a) \geq y \geq f(b)\).
    Then there exists \(c \in [a, b]\) such that \(f(c) = y\).
\end{theorem}

\begin{proof}
    We have two cases: \(f(a) \leq y \leq f(b)\) or \(f(a) \geq y \geq f(b)\).
    We will assume the former, that \(f(a) \leq y \leq f(b)\);
    the latter is proven similarly.

    If \(y = f(a)\) or \(y = f(b)\) then the claim is easy, as one can simply set \(c = a\) or \(c = b\), so we will assume that \(f(a) < y < f(b)\).
    Let \(E\) denote the set
    \[
        E \coloneqq \{x \in [a, b] : f(x) < y\}.
    \]
    Clearly \(E\) is a subset of \([a, b]\), and is hence bounded.
    Also, since \(f(a) < y\), we see that \(a\) is an element of \(E\), so \(E\) is non-empty.
    By the least upper bound principle, the supremum
    \[
        c \coloneqq \sup(E)
    \]
    is thus finite.
    Since \(E\) is bounded by \(b\), we know that \(c \leq b\);
    since \(E\) contains \(a\), we know that \(c \geq a\).
    Thus we have \(c \in [a, b]\).
    To complete the proof we now show that \(f(c) = y\).
    The idea is to work from the left of \(c\) to show that \(f(c) \leq y\), and to work from the right of \(c\) to show that \(f(c) \geq y\).

    Let \(n \geq 1\) be an integer.
    The number \(c - \frac{1}{n}\) is less than \(c = \sup(E)\) and hence cannot be an upper bound for \(E\).
    Thus there exists a point, call it \(x_n\), which lies in \(E\) and which is greater than \(c - \frac{1}{n}\).
    Also \(x_n \leq c\) since \(c\) is an upper bound for \(E\).
    Thus
    \[
        c - \frac{1}{n} \leq x_n \leq c.
    \]
    By the squeeze test (Corollary \ref{6.4.14}) we thus have \(\lim_{n \to \infty} x_n = c\).
    Since \(f\) is continuous at \(c\), this implies that \(\lim_{n \to \infty} f(x_n) = f(c)\).
    But since \(x_n\) lies in \(E\) for every \(n\), we have \(f(x_n) < y\) for every \(n\).
    By the comparison principle (Lemma \ref{6.4.13}) we thus have \(f(c) \leq y\).
    Since \(f(b) > f(c)\), we conclude \(c \neq b\).

    Since \(c \neq b\) and \(c \in [a, b]\), we must have \(c < b\).
    In particular there is an \(N > 0\) such that \(c + \frac{1}{n} < b\) for all \(n > N\)
    (since \(c + \frac{1}{n}\) converges to \(c\) as \(n \to \infty\)).
    Since \(c\) is the supremum of \(E\) and \(c + \frac{1}{n} > c\), we thus have \(c + \frac{1}{n} \notin E\) for all \(n > N\).
    Since \(c + \frac{1}{n} \in [a, b]\), we thus have \(f(c + \frac{1}{n}) \geq y\) for all \(n \geq N\).
    But \(c + \frac{1}{n}\) converges to \(c\), and \(f\) is continuous at \(c\), thus \(f(c) \geq y\).
    But we already knew that \(f(c) \leq y\), thus \(f(c) = y\), as desired.
\end{proof}