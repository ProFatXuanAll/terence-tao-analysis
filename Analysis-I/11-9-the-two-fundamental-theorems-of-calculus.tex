\section{The two fundamental theorems of calculus}\label{sec 11.9}

\begin{theorem}[First Fundamental Theorem of Calculus]\label{11.9.1}
    Let \(a < b\) be real numbers, and let \(f : [a, b] \to \mathbf{R}\) be a Riemann integrable function.
    Let \(F : [a, b] \to \mathbf{R}\) be the function
    \[
        F(x) \coloneqq \int_{[a, x]} f.
    \]
    Then \(F\) is continuous.
    Furthermore, if \(x_0 \in [a, b]\) and \(f\) is continuous at \(x_0\), then \(F\) is differentiable at \(x_0\), and \(F'(x_0) = f(x_0)\).
\end{theorem}

\begin{proof}
    Since \(f\) is Riemann integrable, it is bounded (by Definition \ref{11.3.4}).
    Thus we have some real number \(M\) such that \(-M \leq f(x) \leq M\) for all \(x \in [a, b]\).

    Now let \(x < y\) be two elements of \([a, b]\).
    Then notice that
    \[
        F(y) - F(x) = \int_{[a, y]} f - \int_{[a, x]} f = \int_{[x, y]} f
    \]
    by Theorem \ref{11.4.1}(h).
    By Theorem \ref{11.4.1}(e) we thus have
    \[
        \int_{[x, y]} f \leq \int_{[x, y]} M = p.c. \int_{[x, y]} M = M(y - x)
    \]
    and
    \[
        \int_{[x, y]} f \geq \int_{[x, y]} -M = p.c. \int_{[x, y]} -M = -M(y - x)
    \]
    and thus
    \[
        \abs*{F(y) - F(x)} \leq M(y - x).
    \]
    This is for \(y > x\).
    By interchanging \(x\) and \(y\) we thus see that
    \[
        \abs*{F(y) - F(x)} \leq M(x - y)
    \]
    when \(x > y\).
    Also, we have \(F(y) - F(x) = 0\) when \(x = y\).
    Thus in all
    three cases we have
    \[
        \abs*{F(y) - F(x)} \leq M \abs*{x - y}.
    \]
    Now let \(z \in [a, b]\), and let \((z_n)_{n = 0}^\infty\) be any sequence in \([a, b]\) converging to \(z\).
    Then we have
    \[
        -M \abs*{z_n - z} \leq F(z_n) - F(z) \leq M \abs*{z_n - z}
    \]
    for each \(n\).
    But \(-M \abs*{z_n - z}\) and \(M \abs*{z_n - z}\) both converge to \(0\) as \(n \to \infty\), so by the squeeze test \(F(z_n) - F(z)\) converges to \(0\) as \(n \to \infty\), and thus \(\lim_{n \to \infty} F(z_n) = F(z)\).
    Since this is true for all sequences \(z_n \in [a, b]\) converging to \(z\), we thus see that \(F\) is continuous at \(z\) (by Proposition \ref{9.4.7}).
    Since \(z\) was an arbitrary element of \([a, b]\), we thus see that \(F\) is continuous
    (The above proof also show that when \(F\) is Lipschitz continuous, \(F\) is also continuous, see Exercise \ref{ex 10.2.6}).

    Now suppose that \(x_0 \in [a, b]\), and \(f\) is continuous at \(x_0\).
    Choose any \(\varepsilon > 0\).
    Then by continuity, we can find a \(\delta > 0\) such that \(\abs*{f(x) - f(x_0)} \leq \varepsilon\) for all \(x\) in the interval \(I \coloneqq [x_0 - \delta, x_0 + \delta] \cap [a, b]\), or in other words
    \[
        f(x_0) - \varepsilon \leq f(x) \leq f(x_0) + \varepsilon \text{ for all } x \in I.
    \]
    We now show that
    \[
        \abs*{F(y) - F(x_0) - f(x_0)(y - x_0)} \leq \varepsilon \abs*{y - x_0}
    \]
    for all \(y \in I\), since Proposition \ref{10.1.7} will then imply that \(F\) is differentiable at \(x_0\) with derivative \(F'(x_0) = f(x_0)\) as desired.

    Now fix \(y \in I\).
    There are three cases.
    If \(y = x_0\), then \(F(y) - F(x_0) - f(x_0)(y - x_0) = 0\) and so the claim is obvious.
    If \(y > x_0\), then
    \[
        F(y) - F(x_0) = \int_{[x_0, y]} f.
    \]
    Since \(x_0\), \(y \in I\), and \(I\) is a connected set (by Corollary \ref{11.1.6}), then \([x_0, y]\) is a subset of \(I\), and thus we have
    \[
        f(x_0) - \varepsilon \leq f(x) \leq f(x_0) + \varepsilon \text{ for all } x \in [x_0, y],
    \]
    and thus by Theorem \ref{11.4.1}(e)
    \[
        (f(x_0) - \varepsilon) (y - x_0) \leq \int_{[x_0, y]} f \leq (f(x_0) + \varepsilon) (y - x_0)
    \]
    and so in particular
    \[
        \abs*{F(y) - F(x_0) - f(x_0)(y - x_0)} \leq \varepsilon \abs*{y - x_0}
    \]
    as desired.
    If \(y < x_0\), then
    \[
        F(y) - F(x_0) = - (F(x_0) - F(y)) = -\int_{[y, x_0]} f.
    \]
    Since \(x_0\), \(y \in I\), and \(I\) is a connected set (by Corollary \ref{11.1.6}), then \([y, x_0]\) is a subset of \(I\), and thus we have
    \[
        f(x_0) - \varepsilon \leq f(x) \leq f(x_0) + \varepsilon \text{ for all } x \in [y, x_0],
    \]
    and thus by Theorem \ref{11.4.1}(e)
    \begin{align*}
                 & (f(x_0) - \varepsilon) (x_0 - y) \leq \int_{[y, x_0]} f \leq (f(x_0) + \varepsilon) (x_0 - y)   \\
        \implies & -(f(x_0) - \varepsilon) (y - x_0) \leq \int_{[y, x_0]} f \leq -(f(x_0) + \varepsilon) (y - x_0) \\
        \implies & -\varepsilon (y - x_0) \leq \int_{[y, x_0]} f + f(x_0)(y - x_0) \leq \varepsilon (y - x_0)      \\
        \implies & -\varepsilon (y - x_0) \leq F(x_0) - F(y) + f(x_0)(y - x_0) \leq \varepsilon (y - x_0)          \\
        \implies & -\varepsilon (y - x_0) \leq F(y) - F(x_0) - f(x_0)(y - x_0) \leq \varepsilon (y - x_0)          \\
        \implies & \abs*{F(y) - F(x_0) - f(x_0)(y - x_0)} \leq \varepsilon (y - x_0)
    \end{align*}
    as desired.
\end{proof}