\section{The two fundamental theorems of calculus}\label{sec:11.9}

\begin{thm}[First Fundamental Theorem of Calculus]\label{11.9.1}
  Let \(a < b\) be real numbers, and let \(f : [a, b] \to \R\) be a Riemann integrable function.
  Let \(F : [a, b] \to \R\) be the function
  \[
    F(x) \coloneqq \int_{[a, x]} f.
  \]
  Then \(F\) is continuous.
  Furthermore, if \(x_0 \in [a, b]\) and \(f\) is continuous at \(x_0\), then \(F\) is differentiable at \(x_0\), and \(F'(x_0) = f(x_0)\).
\end{thm}

\begin{proof}
  Since \(f\) is Riemann integrable, it is bounded (by \cref{11.3.4}).
  Thus we have some real number \(M\) such that \(-M \leq f(x) \leq M\) for all \(x \in [a, b]\).

  Now let \(x < y\) be two elements of \([a, b]\).
  Then notice that
  \[
    F(y) - F(x) = \int_{[a, y]} f - \int_{[a, x]} f = \int_{[x, y]} f
  \]
  by \cref{11.4.1}(h).
  By \cref{11.4.1}(e) we thus have
  \[
    \int_{[x, y]} f \leq \int_{[x, y]} M = p.c. \int_{[x, y]} M = M(y - x)
  \]
  and
  \[
    \int_{[x, y]} f \geq \int_{[x, y]} -M = p.c. \int_{[x, y]} -M = -M(y - x)
  \]
  and thus
  \[
    \abs{F(y) - F(x)} \leq M(y - x).
  \]
  This is for \(y > x\).
  By interchanging \(x\) and \(y\) we thus see that
  \[
    \abs{F(y) - F(x)} \leq M(x - y)
  \]
  when \(x > y\).
  Also, we have \(F(y) - F(x) = 0\) when \(x = y\).
  Thus in all
  three cases we have
  \[
    \abs{F(y) - F(x)} \leq M \abs{x - y}.
  \]
  Now let \(z \in [a, b]\), and let \((z_n)_{n = 0}^\infty\) be any sequence in \([a, b]\) converging to \(z\).
  Then we have
  \[
    -M \abs{z_n - z} \leq F(z_n) - F(z) \leq M \abs{z_n - z}
  \]
  for each \(n\).
  But \(-M \abs{z_n - z}\) and \(M \abs{z_n - z}\) both converge to \(0\) as \(n \to \infty\), so by the squeeze test \(F(z_n) - F(z)\) converges to \(0\) as \(n \to \infty\), and thus \(\lim_{n \to \infty} F(z_n) = F(z)\).
  Since this is true for all sequences \(z_n \in [a, b]\) converging to \(z\), we thus see that \(F\) is continuous at \(z\) (by \cref{9.4.7}).
  Since \(z\) was an arbitrary element of \([a, b]\), we thus see that \(F\) is continuous
  (The above proof also show that when \(F\) is Lipschitz continuous, \(F\) is also continuous, see \cref{ex:10.2.6}).

  Now suppose that \(x_0 \in [a, b]\), and \(f\) is continuous at \(x_0\).
  Choose any \(\varepsilon > 0\).
  Then by continuity, we can find a \(\delta > 0\) such that \(\abs{f(x) - f(x_0)} \leq \varepsilon\) for all \(x\) in the interval \(I \coloneqq [x_0 - \delta, x_0 + \delta] \cap [a, b]\), or in other words
  \[
    f(x_0) - \varepsilon \leq f(x) \leq f(x_0) + \varepsilon \text{ for all } x \in I.
  \]
  We now show that
  \[
    \abs{F(y) - F(x_0) - f(x_0)(y - x_0)} \leq \varepsilon \abs{y - x_0}
  \]
  for all \(y \in I\), since \cref{10.1.7} will then imply that \(F\) is differentiable at \(x_0\) with derivative \(F'(x_0) = f(x_0)\) as desired.

  Now fix \(y \in I\).
  There are three cases.
  If \(y = x_0\), then \(F(y) - F(x_0) - f(x_0)(y - x_0) = 0\) and so the claim is obvious.
  If \(y > x_0\), then
  \[
    F(y) - F(x_0) = \int_{[x_0, y]} f.
  \]
  Since \(x_0\), \(y \in I\), and \(I\) is a connected set (by \cref{11.1.6}), then \([x_0, y]\) is a subset of \(I\), and thus we have
  \[
    f(x_0) - \varepsilon \leq f(x) \leq f(x_0) + \varepsilon \text{ for all } x \in [x_0, y],
  \]
  and thus by \cref{11.4.1}(e)
  \[
    \big(f(x_0) - \varepsilon\big) (y - x_0) \leq \int_{[x_0, y]} f \leq \big(f(x_0) + \varepsilon\big) (y - x_0)
  \]
  and so in particular
  \[
    \abs{F(y) - F(x_0) - f(x_0)(y - x_0)} \leq \varepsilon \abs{y - x_0}
  \]
  as desired.
  If \(y < x_0\), then
  \[
    F(y) - F(x_0) = -\big(F(x_0) - F(y)\big) = -\int_{[y, x_0]} f.
  \]
  Since \(x_0\), \(y \in I\), and \(I\) is a connected set (by \cref{11.1.6}), then \([y, x_0]\) is a subset of \(I\), and thus we have
  \[
    f(x_0) - \varepsilon \leq f(x) \leq f(x_0) + \varepsilon \text{ for all } x \in [y, x_0],
  \]
  and thus by \cref{11.4.1}(e)
  \begin{align*}
             & \big(f(x_0) - \varepsilon\big) (x_0 - y) \leq \int_{[y, x_0]} f \leq \big(f(x_0) + \varepsilon\big) (x_0 - y)                  \\
    \implies & \big(f(x_0) - \varepsilon\big) (y - x_0) \geq -\int_{[y, x_0]} f = F(y) - F(x_0) \geq \big(f(x_0) + \varepsilon\big) (y - x_0) \\
    \implies & -\varepsilon (y - x_0) \geq F(y) - F(x_0) - f(x_0)(y - x_0) \geq \varepsilon (y - x_0)                                         \\
    \implies & \abs{F(y) - F(x_0) - f(x_0)(y - x_0)} \leq \varepsilon \abs{y - x_0}
  \end{align*}
  as desired.
\end{proof}

\begin{note}
  Informally, the first fundamental theorem of calculus asserts that
  \[
    \bigg(\int_{[a, x]} f\bigg)'(x) = f(x)
  \]
  given a certain number of assumptions on \(f\).
  Roughly, this means that the derivative of an integral recovers the original function.
\end{note}

\setcounter{thm}{2}
\begin{defn}[Antiderivatives]\label{11.9.3}
  Let \(I\) be a bounded interval, and let \(f : I \to \R\) be a function.
  We say that a function \(F : I \to \R\) is an \emph{antiderivative} of \(f\) if \(F\) is differentiable on \(I\) and \(F'(x) = f(x)\) for all limit points \(x\) of \(I\).
\end{defn}

\begin{thm}ond Fundamental Theorem of Calculus]\label{11.9.4}
  Let \(a < b\) be real numbers, and let \(f : [a, b] \to \R\) be a Riemann integrable function.
  If \(F : [a, b] \to \R\) is an antiderivative of \(f\), then
  \[
    \int_{[a, b]} f = F(b) - F(a).
  \]
\end{thm}

\begin{proof}
  The claim is trivial when \(b = a\), so assume \(b > a\), so in particular all points of \([a, b]\) are limit points.
  We will use Riemann sums.
  The idea is to show that
  \[
    U(f, \mathbf{P}) \geq F(b) - F(a) \geq L(f, \mathbf{P})
  \]
  for every partition \(\mathbf{P}\) of \([a, b]\).
  The left inequality asserts that \(F(b) - F(a)\) is a lower bound for \(\{U(f, \mathbf{P}) : \mathbf{P} \text{ is a partition of } [a, b]\}\), while the right inequality asserts that \(F(b) - F(a)\) is an upper bound for \(\{L(f, \mathbf{P}) : \mathbf{P} \text{ is a partition of } [a, b]\}\).
  But by \cref{11.3.12}, this means that
  \[
    \overline{\int}_{[a, b]} f \geq F(b) - F(a) \geq \underline{\int}_{[a, b]} f,
  \]
  but since \(f\) is assumed to be Riemann integrable, both the upper and lower Riemann integral equal \(\int_{[a, b]} f\).
  The claim follows.

  We have to show the bound \(U(f, \mathbf{P}) \geq F(b) - F(a) \geq L(f, \mathbf{P})\).
  We shall just show the first inequality \(U(f, \mathbf{P}) \geq F(b) - F(a)\);
  the other inequality is similar.

  Let \(\mathbf{P}\) be a partition of \([a, b]\).
  From \cref{11.8.4} we have
  \[
    F(b) - F(a) = \sum_{J \in \mathbf{P}} F[J] = \sum_{J \in \mathbf{P} : J \neq \emptyset} F[J],
  \]
  while from definition we have
  \[
    U(f, \mathbf{P}) = \sum_{J \in \mathbf{P} : J \neq \emptyset} \sup_{x \in J} f(x) \abs{J}.
  \]
  Thus it will suffice to show that
  \[
    F[J] \leq \sup_{x \in J} f(x) \abs{J}
  \]
  for all \(J \in \mathbf{P}\)
  (other than the empty set).

  When \(J\) is a point then the claim is clear, since both sides are zero.
  Now suppose that \(J = [c, d], (c, d], [c, d)\), or \((c, d)\) for some \(c < d\).
  Then the left-hand side is \(F[J] = F(d) - F(c)\).
  Note that \(F\), being differentiable, is continuous, so we may use the simplified formula for the \(F\)-length as opposed to the more complicated one in \cref{11.8.1}.
  By the mean-value theorem (\cref{10.2.9}), this is equal to \((d - c) F'(e)\) for some \(e \in J\).
  But since \(F'(e) = f(e)\), we thus have
  \[
    F[J] = (d - c) f(e) = f(e) \abs{J} \leq \sup_{x \in J} f(x) \abs{J}
  \]
  as desired.
\end{proof}

\begin{note}
  One can use the second fundamental theorem of calculus to compute integrals relatively easily provided that you can find an anti-derivative of the integrand \(f\).
  The first fundamental theorem of calculus ensures that every \emph{continuous} Riemann integrable function has an anti-derivative.
  For discontinuous functions, the situation is more complicated.
  Also, not every function with an anti-derivative is Riemann integrable.
\end{note}

\begin{lem}\label{11.9.5}
  Let \(I\) be a bounded interval, and let \(f : I \to \R\) be a function.
  Let \(F : I \to \R\) and \(G : I \to \R\) be two antiderivatives of \(f\).
  Then there exists a real number \(C\) such that \(F(x) = G(x) + C\) for all \(x \in I\).
\end{lem}

\begin{proof}
  If \(I = \emptyset\), then the claim is trivially true.
  If \(I = \{a\}\) for some \(a \in \R\), then we can simply set \(C = F(a) - G(a)\).
  So suppose that \(I\) is one of \((a, b), [a, b), (a, b], [a, b]\) for some \(a, b \in \R\) and \(a < b\).
  Since \(I\) is a bounded interval, for all \(x \in I\) we know that \(x\) is a limit point.
  Since \(F, G\) are antiderivatives of \(f\), by \cref{11.9.3} we know that \(F, G\) are differentiable on \(I\).
  By \cref{10.1.13}(f) we know that \(F - G\) is differentiable on \(I\), and thus by \cref{10.1.12} we know that \(F - G\) are continuous on \(I\).
  Let \(x, y \in I\) and \(x < y\).
  Since \(I\) is a bounded interval, by \cref{11.1.10} we know that \(I\) is connected, and thus by \cref{11.1.1} \([x, y] \subseteq I\).
  By the mean-value theorem (\cref{10.2.9}) we know that
  \[
    \exists\ c \in I : \dfrac{(F - G)(x) - (F - G)(y)}{x - y} = (F - G)'(c).
  \]
  Thus we have
  \begin{align*}
             & \dfrac{(F - G)(x) - (F - G)(y)}{x - y} = (F - G)'(c)                                      \\
    \implies & \dfrac{(F - G)(x) - (F - G)(y)}{x - y} = F'(c) - G'(c) &  & \text{(by \cref{10.1.13}(f))} \\
    \implies & \dfrac{(F - G)(x) - (F - G)(y)}{x - y} = 0             &  & \text{(by hypothesis)}        \\
    \implies & (F - G)(x) = (F - G)(y)                                                                   \\
    \implies & F(x) - G(x) = F(y) - G(y)                              &  & \text{(by \cref{9.2.1})}      \\
    \implies & F(x) = G(x) + F(y) - G(y).
  \end{align*}
  By setting \(C = F(y) - G(y)\) we are done.
\end{proof}

\exercisesection

\begin{ex}\label{ex:11.9.1}
  Let \(f : [0, 1] \to \R\) be the function in \cref{ex:9.8.5}.
  Show that for every rational number \(q \in \Q \cap (0, 1)\), the function \(F : [0, 1] \to \R\) defined by the formula \(F(x) \coloneqq \int_{[0, x]} f\) is not differentiable at \(q\).
\end{ex}

\begin{proof}
  By \cref{ex:9.8.5} we know that \(f\) is strictly monotone increasing, thus by \cref{11.6.1} we know that \(f\) is Riemann integrable and \(F\) is well-defined.
  By \cref{ex:9.8.5} \(f\) is not continuous at \(q\), thus by \cref{ex:11.9.3} we know that \(F\) is not differentiable at \(q\).
\end{proof}

\begin{ex}\label{ex:11.9.2}
  Prove \cref{11.9.5}.
\end{ex}

\begin{proof}
  See \cref{11.9.5}.
\end{proof}

\begin{ex}\label{ex:11.9.3}
  Let \(a < b\) be real numbers, and let \(f : [a, b] \to \R\) be a monotone increasing function.
  Let \(F : [a, b] \to \R\) be the function \(F(x) \coloneqq \int_{[a, x]} f\).
  Let \(x_0\) be an element of \((a, b)\).
  Show that \(F\) is differentiable at \(x_0\) if and only if \(f\) is continuous at \(x_0\).
\end{ex}

\begin{proof}
  Since \(f\) is monotone increasing, by \cref{11.6.1} we know that \(f\) is Riemann integrable and \(F\) is well-defined.
  If \(f\) is continuous at \(x_0\), then by \cref{11.9.1} we know that \(F\) is differentiable at \(x_0\).
  So we only need to show that if \(F\) is differentiable at \(x_0\), then \(f\) is continuous at \(x_0\).

  Suppose that \(F\) is differentiable at \(x_0\).
  Suppose for sake of contradiction that \(f\) is not continuous at \(x_0\).
  Since \(f\) is monotone increasing, by \cref{ac:11.8.1} we know that both \(f(x_0+)\) and \(f(x_0-)\) exist and \(f(x_0-) \leq f(x_0+)\).
  Since \(f\) is not continuous at \(x_0\), by \cref{9.5.3} we have \(f(x_0-) < f(x_0+)\).

  If \(x \in [a, b] \cap (-\infty, x_0)\), then we have \(f(x) \leq f(x_0-)\) and
  \[
    \dfrac{F(x) - F(x_0)}{x - x_0} = \dfrac{-\int_{[x, x_0]} f}{x - x_0} = \dfrac{\int_{[x, x_0]} f}{x_0 - x} \leq \dfrac{p.c. \int_{[x, x_0]} f(x_0-)}{x_0 - x} = f(x_0-).
  \]
  If \(x \in [a, b] \cap (x_0, \infty)\), then we have \(f(x) \geq f(x_0+)\) and
  \[
    \dfrac{F(x) - F(x_0)}{x - x_0} = \dfrac{\int_{[x_0, x]} f}{x - x_0} \geq \dfrac{p.c. \int_{[x_0, x]} f(x_0+)}{x - x_0} = f(x_0+).
  \]
  Thus by \cref{9.3.14} we have
  \[
    F'(x_0-) \leq f(x_0-) < f(x_0+) \leq F'(x_0+).
  \]
  But by \cref{9.3.6} \(F'(x_0-) \neq F'(x_0+)\) implies \(F'(x_0)\) does not exist, a contradiction.
  So we conclude that \(f\) is continuous at \(x_0\).
\end{proof}