\chapter{The real numbers}\label{ch:5}

\begin{note}
  We defined the natural numbers using the five Peano axioms, and postulated that such a number system existed;
  this is plausible, since the natural numbers correspond to the very intuitive and fundamental notion of \emph{sequential counting}.
\end{note}

\begin{note}
  The symbols \(\N\), \(\Q\), and \(\R\) stand for ``natural'', ``quotient'', and ``real'' respectively.
  \(\Z\) stands for ``Zahlen'', the German word for numbers.
  There is also the \emph{complex numbers} \(\C\), which obviously stands for ``complex''.
\end{note}

\begin{note}
  \emph{Formal} means ``having the form of'';
  at the beginning of our construction the expression \(a \text{---} b\) did not actually \emph{mean} the difference \(a - b\), since the symbol \text{---} was meaningless.
  It only had the \emph{form} of a difference.
  Later on we defined subtraction and verified that the formal difference was equal to the actual difference, so this eventually became a non-issue, and our symbol for formal differencing was discarded.
  Somewhat confusingly, this use of the term ``formal'' is unrelated to the notions of a formal argument and an informal argument.
\end{note}

\begin{note}
  There is a fundamental area of mathematics where the rational number system does not suffice - that of \emph{geometry}
  (the study of lengths, areas, etc.).
  For instance, a right-angled triangle with both sides equal to \(1\) gives a hypotenuse of \(\sqrt{2}\), which is an \emph{irrational} number, i.e., not a rational number;
  see \cref{4.4.4}.
  Things get even worse when one starts to deal with the sub-field of geometry known as \emph{trigonometry}, when one sees numbers such as \(\pi\) or \(\cos(1)\), which turn out to be in some sense ``even more'' irrational than \(\sqrt{2}\).
  (These numbers are known as \emph{transcendental numbers}, but to discuss this further would be far beyond the scope of this text.)
  Thus, in order to have a number system which can adequately describe geometry
  - or even something as simple as measuring lengths on a line
  - one needs to replace the rational number system with the real number system.
\end{note}

\begin{note}
  In the constructions of integers and rationals, the task was to introduce one more \emph{algebraic} operation to the number system
  - e.g., one can get integers from naturals by introducing subtraction, and get the rationals from the integers by introducing division.
  But to get the reals from the rationals is to pass from a ``discrete'' system to a ``continuous'' one, and requires the introduction of a somewhat different notion
  - that of a \emph{limit}.
\end{note}

\begin{note}
  The limit is a concept which on one level is quite intuitive, but to pin down rigorously turns out to be quite difficult.
  (Even such great mathematicians as Euler and Newton had difficulty with this concept.
  It was only in the nineteenth century that mathematicians such as Cauchy and Dedekind figured out how to deal with limits rigorously.)
\end{note}

\begin{note}
  The procedure we give here of obtaining the real numbers as the limit of sequences of rational numbers may seem rather complicated.
  However, it is in fact an instance of a very general and useful procedure, that of \emph{completing} one metric space to form another.
\end{note}

\section{Cauchy sequences}

\begin{definition}[Sequences]\label{5.1.1}
Let \(m\) be an integer.
A \emph{sequence \((a_n)_{n = m}^{\infty}\) of rational numbers} is any function from the set \(\{n \in \mathds{Z} : n \geq m\}\) to \(\mathds{Q}\), i.e., a mapping which assigns to each integer \(n\) greater than or equal to \(m\), a rational number \(a_n\).
More informally, a sequence \((a_n)_{n = m}^{\infty}\) of rational numbers is a collection of rationals \(a_m, a_{m + 1}, a_{m + 2}, \dots\).
\end{definition}
\section{Equivalent Cauchy sequences}\label{i:sec:5.2}

\begin{note}
  If we are to define the real numbers from the rationals as limits of rational Cauchy sequences, we have to know when two Cauchy sequences of rationals give the same limit, without first defining a real number
  (since that would be circular).
  To do this we use a similar set of definitions to those used to define a Cauchy sequence in the first place.
\end{note}

\begin{defn}[\(\varepsilon\)-close sequences]\label{i:5.2.1}
  Let \((a_n)_{n = m}^{\infty}\) and \((b_n)_{n = m}^{\infty}\) be two rational sequences, and let \(\varepsilon \in \Q^+\).
  We say that the sequence \((a_n)_{n = m}^{\infty}\) is \emph{\(\varepsilon\)-close} to \((b_n)_{n = m}^{\infty}\) iff \(a_n\) is \(\varepsilon\)-close to \(b_n\) for each \(n \in \Z_{\geq m}\).
  In other words, the sequence \(a_m, a_{m + 1}, a_{m + 2}, \dots\) is \(\varepsilon\)-close to the sequence \(b_m, b_{m + 1}, b_{m + 2}, \dots\) iff \(\abs{a_n - b_n} \leq \varepsilon\) for all \(n = m, m + 1, m + 2, \dots\).
\end{defn}

\setcounter{thm}{2}
\begin{defn}[Eventually \(\varepsilon\)-close sequences]\label{i:5.2.3}
  Let \((a_n)_{n = m}^{\infty}\) and \((b_n)_{n = m}^{\infty}\) be two rational sequences, and let \(\varepsilon \in \Q^+\).
  We say that the sequence \((a_n)_{n = m}^{\infty}\) is \emph{eventually \(\varepsilon\)-close} to \((b_n)_{n = m}^{\infty}\) iff there exists an \(N \in \Z_{\geq m}\) such that the sequences \((a_n)_{n = N}^{\infty}\) and \((b_n)_{n = N}^{\infty}\) are \(\varepsilon\)-close.
  In other words, \(a_m, a_{m + 1}, a_{m + 2}, \dots\) is eventually \(\varepsilon\)-close to \(b_m, b_{m + 1}, b_{m + 2}, \dots\) iff there exists an \(N \in \Z_{\geq m}\) such that \(\abs{a_n - b_n} \leq \varepsilon\) for all \(n \in \Z_{\geq N}\).
\end{defn}

\begin{rmk}\label{i:5.2.4}
  Again, the notations for \(\varepsilon\)-close sequences and eventually \(\varepsilon\)-close sequences are not standard in the literature, and we will not use them outside of this section.
\end{rmk}

\setcounter{thm}{5}
\begin{defn}[Equivalent sequences]\label{i:5.2.6}
  Two rational sequences \((a_n)_{n = m}^{\infty}\) and \((b_n)_{n = m}^{\infty}\) are \emph{equivalent} iff for each \(\varepsilon \in \Q^+\), the sequences \((a_n)_{n = m}^{\infty}\) and \((b_n)_{n = m}^{\infty}\) are eventually \(\varepsilon\)-close.
  In other words, \(a_m, a_{m + 1}, a_{m + 2}, \dots\) and \(b_m, b_{m + 1}, b_{m + 2}, \dots\) are equivalent iff for every \(\varepsilon \in \Q^+\), there exists an \(N \in \Z_{\geq m}\) such that \(\abs{a_n - b_n} \leq \varepsilon\) for all \(n \in \Z_{\geq N}\).
\end{defn}

\begin{rmk}\label{i:5.2.7}
  As with \cref{i:5.1.8}, the quantity \(\varepsilon \in \Q^+\) is currently restricted to be a positive rational, rather than a positive real.
  However, we shall eventually see that it makes no difference whether \(\varepsilon\) ranges over the positive rationals or positive reals;
  see \cref{i:ex:6.1.10}.
\end{rmk}

\begin{ac}\label{i:ac:5.2.1}
  Equivalence defined as \cref{i:5.2.6} is reflexive, symmetric and transitive.
\end{ac}

\begin{proof}[\pf{i:ac:5.2.1}]
  Let \((a_n)_{n = m}^\infty, (b_n)_{n = m}^\infty, (c_n)_{n = m}^\infty\) be rational sequences.
  We denote the equivalence relation defined in \cref{i:5.2.6} as \(\equiv\).
  First we show that \(\equiv\) is reflexive.
  Since for arbitrary \((a_n)_{n = m}^\infty\), we always have
  \begin{align*}
             & \forall \varepsilon \in \Q^+, \forall n \geq m, \abs{a_n - a_n} = \abs{0} = 0 \leq \varepsilon                   \\
    \implies & (a_n)_{n = m}^\infty \equiv (a_n)_{n = m}^\infty,                                              &  & \by{i:5.2.6}
  \end{align*}
  we see that \(\equiv\) is reflexive.

  Next we show that \(\equiv\) is symmetric.
  Suppose that \((a_n)_{n = m}^\infty \equiv (b_n)_{n = m}^\infty\).
  Then we have
  \begin{align*}
             & (a_n)_{n = m}^\infty \equiv (b_n)_{n = m}^\infty                                                                                          \\
    \implies & \forall \varepsilon \in \Q^+, \exists N \in \Z_{\geq m}: \forall n \in \Z_{\geq N}, \abs{a_n - b_n} \leq \varepsilon &  & \by{i:5.2.6}    \\
    \implies & \forall \varepsilon \in \Q^+, \exists N \in \Z_{\geq m}: \forall n \in \Z_{\geq N}, \abs{b_n - a_n} \leq \varepsilon &  & \by{i:4.3.3}[f] \\
    \implies & (b_n)_{n = m}^\infty \equiv (a_n)_{n = m}^\infty.                                                                    &  & \by{i:5.2.6}
  \end{align*}
  Thus \(\equiv\) is symmetric.

  Finally we show that \(\equiv\) is transitive.
  Suppose that \((a_n)_{n = m}^\infty \equiv (b_n)_{n = m}^\infty\) and \((b_n)_{n = m}^\infty \equiv (c_n)_{n = m}^\infty\).
  Then we have
  \begin{align*}
             & \pa{(a_n)_{n = m}^\infty \equiv (b_n)_{n = m}^\infty} \land \pa{(b_n)_{n = m}^\infty \equiv (c_n)_{n = m}^\infty}                        \\
    \implies & \forall \varepsilon \in \Q^+, \exists N_1, N_2 \in \Z_{\geq m}: \begin{dcases}
                                                                                 \forall n \in \Z_{\geq N_1}, \abs{a_n - b_n} \leq \dfrac{\varepsilon}{2} \\
                                                                                 \forall n \in \Z_{\geq N_2}, \abs{b_n - c_n} \leq \dfrac{\varepsilon}{2}
                                                                               \end{dcases}                                    &  & \by{i:5.2.6} \\
    \implies & \forall \varepsilon \in \Q^+, \exists N = \max(N_1, N_2) \in \Z_{\geq m} :                                        &  & \by{i:4.1.11}[f]  \\
             & \forall n \in \Z_{\geq N}, \begin{dcases}
                                            \abs{a_n - b_n} \leq \dfrac{\varepsilon}{2} \\
                                            \abs{b_n - c_n} \leq \dfrac{\varepsilon}{2}
                                          \end{dcases}                                                                   \\
    \implies & \forall \varepsilon \in \Q^+, \exists N = \max(N_1, N_2) \in \Z_{\geq m} :                                                               \\
             & \forall n \in \Z_{\geq N}, \abs{a_n - b_n} + \abs{b_n - c_n} \leq \varepsilon                                     &  & \by{i:4.2.9}[c,d] \\
    \implies & \forall \varepsilon \in \Q^+, \exists N = \max(N_1, N_2) \in \Z_{\geq m} :                                                               \\
             & \forall n \in \Z_{\geq N}, \abs{a_n - c_n} \leq \abs{a_n - b_n} + \abs{b_n - c_n} \leq \varepsilon                &  & \by{i:4.3.3}[b]   \\
    \implies & (a_n)_{n = m}^\infty \equiv (c_n)_{n = m}^\infty.                                                                 &  & \by{i:5.2.6}
  \end{align*}
  Thus \(\equiv\) is transitive.
\end{proof}

\begin{prop}\label{i:5.2.8}
  Let \((a_n)_{n = 1}^{\infty}\) and \((b_n)_{n = 1}^{\infty}\) be the sequences \(a_n = 1 + 10^{-n}\) and \(b_n = 1 - 10^{-n}\).
  Then the sequences \((a_n)_{n = 1}^{\infty}, (b_n)_{n = 1}^{\infty}\) are equivalent.
\end{prop}

\begin{proof}[\pf{i:5.2.8}]
  We need to prove that for every \(\varepsilon \in \Q^+\), the two sequences \((a_n)_{n = 1}^{\infty}\) and \((b_n)_{n = 1}^{\infty}\) are eventually \(\varepsilon\)-close to each other.
  So we fix an \(\varepsilon \in \Q^+\).
  We need to find an \(N \in \Z^+\) such that \((a_n)_{n = 1}^{\infty}\) and \((b_n)_{n = 1}^{\infty}\) are \(\varepsilon\)-close;
  in other words, we need to find an \(N \in \Z^+\) such that
  \[
    \abs{a_n - b_n} \leq \varepsilon \text{ for all } n \in \Z_{\geq N}.
  \]
  However, we have
  \[
    \abs{a_n - b_n} = \abs{(1 + 10^{-n}) - (1 - 10^{-n})} = 2 \times 10^{-n}.
  \]
  Since \(10^{-n}\) is a decreasing function of \(n\) (i.e., \(10^{-m} < 10^{-n}\) whenever \(m > n\);
  this is easily proven by induction), and \(n \in \Z_{\geq N}\), we have \(2 \times 10^{-n} \leq 2 \times 10^{-N}\).
  Thus we have
  \[
    \abs{a_n - b_n} \leq 2 \times 10^{-N} \text{ for all } n \in \Z_{\geq N}.
  \]
  Thus in order to obtain \(\abs{a_n - b_n} \leq \varepsilon\) for all \(n \in \Z_{\geq N}\), it will be sufficient to choose \(N\) so that \(2 \times 10^{-N} \leq \varepsilon\).
  This is easy to do using logarithms, but we have not yet developed logarithms yet, so we will use a cruder method.
  First, we observe \(10^N\) is always greater than \(N\) for any \(N \in \Z^+\) (see \cref{i:ex:4.3.5}).
  Thus \(10^{-N} \leq 1 / N\), and so \(2 \times 10^{-N} \leq 2 / N\).
  Thus to get \(2 \times 10^{-N} \leq \varepsilon\), it will suffice to choose \(N\) so that \(2 / N \leq \varepsilon\), or equivalently that \(N \geq 2 / \varepsilon\).
  But by \cref{i:4.4.1} we can always choose such an \(N\), and the claim follows.
\end{proof}

\begin{rmk}\label{i:5.2.9}
  \cref{i:5.2.8}, in decimal notation, asserts that
  \[
    1.0000 \dots = 0.9999 \dots.
  \]
\end{rmk}

\exercisesection

\begin{ex}\label{i:ex:5.2.1}
  Show that if \((a_n)_{n = m}^{\infty}\) and \((b_n)_{n = m}^{\infty}\) are equivalent sequences of rationals, then \((a_n)_{n = m}^{\infty}\) is a Cauchy sequence iff \((b_n)_{n = m}^{\infty}\) is a Cauchy sequence.
\end{ex}

\begin{proof}[\pf{i:ex:5.2.1}]
  First suppose that \((a_n)_{n = m}^\infty\) is a Cauchy sequence.
  Let \(\varepsilon \in \Q^+\).
  Since \((a_n)_{n = m}^\infty, (b_n)_{n = m}^\infty\) are equivalent, by \cref{i:5.2.6} we have
  \[
    \exists N_1 \in \Z_{\geq m} : \forall n \in \Z_{\geq N_1}, \abs{a_n - b_n} \leq \dfrac{\varepsilon}{3}.
  \]
  Fix such \(N_1\).
  Since \((a_n)_{n = m}^\infty\) is a Cauchy sequence, by \cref{i:5.1.8} we have
  \[
    \exists N_2 \in \Z_{\geq m} : \forall j, k \in \Z_{\geq N}, \abs{a_j - a_k} \leq \dfrac{\varepsilon}{3}.
  \]
  Fix such \(N_2\).
  Now let \(N = \max(N_1, N_2)\).
  By \cref{i:4.1.11}(f) we know that \(N\) is well-defined.
  Then we have
  \begin{align*}
    \forall j, k \in \Z_{\geq N}, \abs{b_j - b_k} & = \abs{a_j - a_k + b_j - a_j + a_k - b_k}                                                                           \\
                                                  & \leq \abs{a_j - a_k} + \abs{a_j - b_j} + \abs{a_k - b_k}                                     &  & \by{i:4.3.3}[b]   \\
                                                  & \leq \dfrac{\varepsilon}{3} + \dfrac{\varepsilon}{3} + \dfrac{\varepsilon}{3} = \varepsilon. &  & \by{i:4.2.9}[c,d]
  \end{align*}
  Since \(\varepsilon\) is arbitrary, we see that
  \[
    \forall \varepsilon \in \Q^+, \exists N \in \Z_{\geq m} : \forall j, k \in \Z_{\geq N}, \abs{b_j - b_k} \leq \varepsilon.
  \]
  By \cref{i:5.1.8} this means \((b_n)_{n = m}^\infty\) is a Cauchy sequence.

  Using similar arguments we can show that \((b_n)_{n = m}^\infty\) is a Cauchy sequence implies \((a_n)_{n = m}^\infty\) is a Cauchy sequence.
  Thus we conclude that \((a_n)_{n = m}^\infty\) is a Cauchy sequence iff \((b_n)_{n = m}^\infty\) is a Cauchy sequence.
\end{proof}

\begin{ex}\label{i:ex:5.2.2}
  Let \(\varepsilon \in \Q^+\).
  Show that if \((a_n)_{n = m}^{\infty}\) and \((b_n)_{n = m}^{\infty}\) are eventually \(\varepsilon\)-close, then \((a_n)_{n = m}^{\infty}\) is bounded iff \((b_n)_{n = m}^{\infty}\) is bounded.
\end{ex}

\begin{proof}[\pf{i:ex:5.2.2}]
  First suppose that \((a_n)_{n = m}^\infty\) is bounded.
  Since \((a_n)_{n = m}^{\infty}\) and \((b_n)_{n = m}^{\infty}\) are eventually \(\varepsilon\)-close, by \cref{i:5.2.3} we have
  \[
    \exists N \in \Z_{\geq m} : \forall n \in \Z_{\geq N}, \abs{a_n - b_n} \leq \varepsilon.
  \]
  Fix such \(N\).
  Since \((a_n)_{n = m}^\infty\) is bounded, by \cref{i:5.1.12} there exists some \(M \in \Q_{\geq 0}\) such that \(\abs{a_n} \leq M\) for all \(n \in \Z_{\geq m}\).
  Then we have
  \begin{align*}
    \forall n \in \Z_{\geq N}, \abs{b_n} & = \abs{-b_n}                     &  & \by{i:4.3.3}[d]   \\
                                         & = \abs{a_n - b_n + a_n}                                 \\
                                         & \leq \abs{a_n - b_n} + \abs{a_n} &  & \by{i:4.3.3}[b]   \\
                                         & \leq \varepsilon + M.            &  & \by{i:4.2.9}[c,d]
  \end{align*}
  Thus \((b_n)_{n = N}^\infty\) is bounded by \(\varepsilon + M\).
  Now we split into two cases:
  \begin{itemize}
    \item If \(N = m\), then we see that \((b_n)_{n = m}^\infty\) is bounded by \(\varepsilon + M\).
    \item If \(N \neq m\), then we must have \(m < N\).
          By \cref{i:5.1.14} we know that the finite rational sequence \((b_n)_{n = m}^{N - 1}\) is bounded by some \(M' \in \Q_{\geq 0}\).
          So both \((b_n)_{n = m}^{N - 1}\) and \((b_n)_{n = N}^\infty\) are bounded by \(M' + \varepsilon + M\).
          Thus \((b_n)_{n = m}^\infty\) is bounded by \(M' + \varepsilon + M\).
  \end{itemize}
  From all cases above we see that \((b_n)_{n = m}^\infty\) is bounded.

  Using similar arguments we can show that \((b_n)_{n = m}^\infty\) is bounded implies \((a_n)_{n = m}^\infty\) is bounded.
  Thus we conclude that \((a_n)_{n = m}^\infty\) is bounded iff \((b_n)_{n = m}^\infty\) is bounded.
\end{proof}

\section{The construction of the real numbers}

\begin{definition}[Real numbers]\label{5.3.1}
A \emph{real number} is defined to be an object of the form \(\text{LIM}_{n \to \infty} a_n\), where \((a_n)_{n = 1}^{\infty}\) is a Cauchy sequence of rational numbers.
Two real numbers \(\text{LIM}_{n \to \infty} a_n\) an and \(\text{LIM}_{n \to \infty} b_n\) are said to be equal iff \((a_n)_{n = 1}^{\infty}\) and \((b_n)_{n = 1}^{\infty}\) are equivalent Cauchy sequences.
The set of all real numbers is denoted \(\mathds{R}\).
\end{definition}

\begin{note}
We will refer to \(\text{LIM}_{n \to \infty} a_n\) as the \emph{formal limit} of the sequence \((a_n)_{n = 1}^{\infty}\).
Later on we will define a genuine notion of limit, and show that the formal limit of a Cauchy sequence is the same as the limit of that sequence;
after that, we will not need formal limits ever again.
\end{note}

\setcounter{theorem}{2}
\begin{proposition}[Formal limits are well-defined]\label{5.3.3}
Let \(x = \text{LIM}_{n \to \infty} a_n\), \(y = \text{LIM}_{n \to \infty} b_n\), and \(z = \text{LIM}_{n \to \infty} c_n\) be real numbers.
Then, with the above definition of equality for real numbers, we have \(x = x\).
Also, if \(x = y\), then \(y = x\).
Finally, if \(x = y\) and \(y = z\), then \(x = z\).
\end{proposition}

\begin{proof}
We first prove the reflexivity.
Because \(x = \text{LIM}_{n \to \infty} a_n\), by Definition \ref{5.3.1}, \((a_n)_{n = 1}^{\infty}\) is a Cauchy sequence of rational numbers.
By Definition \ref{5.1.8}, \(\forall\ \varepsilon > 0\) and \(\varepsilon \in \mathds{Q}\), \(\exists\ N \geq 1\) and \(N \in \mathds{N}\) such that
\[
    |a_j - a_k| \leq \varepsilon \ \forall\ j, k \geq N,
\]
where \(j, k \in \mathds{N}\).
In particular, we have
\[
    |a_j - a_j| \leq \varepsilon \ \forall\ j \geq N.
\]
By Definition \ref{5.2.6}, \((a_n)_{n = 1}^{\infty}\) and \((a_n)_{n = 1}^{\infty}\) are equivalent sequences.
Since \((a_n)_{n = 1}^{\infty}\) is a Cauchy sequence, by Definition \ref{5.3.1}, \(x = \text{LIM}_{n \to \infty} a_n = \text{LIM}_{n \to \infty} a_n = x\).

Next we prove the symmetry.
By Definition \ref{5.3.1}, \(x = y\) implies \((a_n)_{n = 1}^{\infty}\) and \((b_n)_{n = 1}^{\infty}\) are equivalent Cauchy sequences.
Then by Definition \ref{5.2.6}, \(\forall\ \varepsilon > 0\) and \(\varepsilon \in \mathds{Q}\), \(\exists\ N \geq 1\) and \(N \in \mathds{N}\) such that \(a_n\) is \(\varepsilon\)-close to \(b_n\) for all \(n \geq N\).
But by Proposition \ref{4.3.7}, \(a_n\) is \(\varepsilon\)-close to \(b_n\) implies that \(b_n\) is \(\varepsilon\)-close to \(a_n\).
So we have \(b_n\) is \(\varepsilon\)-close to \(a_n\), \(\forall\ \varepsilon > 0\) and \(\forall\ n \geq N\), which means \((b_n)_{n = 1}^{\infty}\) and \((a_n)_{n = 1}^{\infty}\) are equivalent sequences by Definition \ref{5.2.6}.
Since \((a_n)_{n = 1}^{\infty}\) and \((b_n)_{n = 1}^{\infty}\) are Cauchy sequences, by Definition \ref{5.3.1}, \(y = \text{LIM}_{n \to \infty} b_n = \text{LIM}_{n \to \infty} a_n = x\).

Finally we prove the transitivity.
By Definition \ref{5.3.1}, \(x = y\) implies that \((a_n)_{n = 1}^{\infty}\) and \((b_n)_{n = 1}^{\infty}\) are equivalent Cauchy sequences.
Then by Definition \ref{5.2.6}, \(\forall\ \varepsilon > 0\) and \(\varepsilon \in \mathds{Q}\), \(\exists\ N_1 \geq 1\) and \(N_1 \in \mathds{N}\) such that \(a_n\) is \(\varepsilon\)-close to \(b_n\) for all \(n \geq N_1\).
Since \(\varepsilon > 0\), by Additional Corollary \ref{ac 4.2.5}, \(\varepsilon / 2 > 0\), so \(a_n\) is also \((\varepsilon / 2)\)-close to \(b_n\).
Similarly, \(y = z\) implies that \(\forall\ \varepsilon > 0\) and \(\varepsilon \in \mathds{Q}\), \(\exists\ N_2 \geq 1\) and \(N_2 \in \mathds{N}\) such that \(b_n\) is \((\varepsilon / 2)\)-close to \(c_n\) for all \(n \geq N_2\).
Let \(N = N_1 + N_2\).
Then by Additional Corollary \ref{ac 2.2.1}, \(N \in \mathds{N}\).
And by Definition \ref{2.2.11}, \(N \geq N_1\) and \(N \geq N_2\).
So we have \(a_n\) is \((\varepsilon / 2)\)-close to \(b_n\) and \(b_n\) is \((\varepsilon / 2)\)-close to \(c_n\), \(\forall\ \varepsilon > 0\) and \(\forall\ n \geq N\).
Then by Proposition \ref{4.3.7}, \(a_n\) is \(\varepsilon\)-close to \(c_n\), \(\forall\ \varepsilon > 0\) and \(\forall\ n \geq N\).
Thus by Definition \ref{5.2.6}, \((a_n)_{n = 1}^{\infty}\) and \((c_n)_{n = 1}^{\infty}\) are equivalent sequences.
Since \((a_n)_{n = 1}^{\infty}\) and \((c_n)_{n = 1}^{\infty}\) are Cauchy sequences, by Definition \ref{5.3.1}, \(x = \text{LIM}_{n \to \infty} a_n = \text{LIM}_{n \to \infty} c_n = z\).
\end{proof}

\exercisesection

\begin{exercise}\label{ex 5.3.1}
Prove Proposition \ref{5.3.3}.
\end{exercise}

\begin{proof}
See Proposition \ref{5.3.3}.
\end{proof}
\section{Ordering the reals}\label{i:sec:5.4}

\begin{defn}\label{i:5.4.1}
  Let \((a_n)_{n = m}^{\infty}\) be a sequence of rationals.
  We say that this sequence is \emph{positively bounded away from zero} iff we have a positive rational \(c \in \Q^+\) such that \(a_n \geq c\) for all \(n \in \Z_{\geq m}\) (in particular, the sequence is entirely positive).
  The sequence is \emph{negatively bounded away from zero} iff we have a negative rational \(c \in \Q^-\) such that \(a_n \leq c\) for all \(n \in \Z_{\geq m}\) (in particular, the sequence is entirely negative).
\end{defn}

\begin{note}
  It is clear that any sequence which is positively or negatively bounded away from zero, is bounded away from zero.
  Also, a sequence cannot be both positively bounded away from zero and negatively bounded away from zero at the same time.
\end{note}

\setcounter{thm}{2}
\begin{defn}\label{i:5.4.3}
  A real number \(x\) is said to be \emph{positive} iff it can be written as \(x = \LIM_{n \to \infty} a_n\) for some Cauchy sequence \((a_n)_{n = 1}^{\infty}\) which is positively bounded away from zero.
  \(x\) is said to be \emph{negative} iff it can be written as \(x = \LIM_{n \to \infty} a_n\) for some sequence \((a_n)_{n = 1}^{\infty}\) which is negatively bounded away from zero.
\end{defn}

\begin{prop}[Basic properties of positive reals]\label{i:5.4.4}
  For every real number \(x\), exactly one of the following three statements is true:
  \begin{enumerate}
    \item \(x\) is zero;
    \item \(x\) is positive;
    \item \(x\) is negative.
  \end{enumerate}
  A real number \(x\) is negative iff \(-x\) is positive.
  If \(x\) and \(y\) are positive, then so are \(x + y\) and \(xy\).
\end{prop}

\begin{proof}[\pf{i:5.4.4}]
  We first show that at least one of the three statements is true.
  Let \(x \in \R\).
  By \cref{i:ac:5.3.1} we know that \(0\) is the formal limit of \((0)_{n = 1}^\infty\), thus by \cref{i:5.3.1,i:5.3.3} we can ask whether \(x = 0\) or \(x \neq 0\).
  If \(x = 0\), then we are done.
  Otherwise by \cref{i:5.3.14} we know that \(x\) is the formal limit of a rational Cauchy sequence \((a_n)_{n = 1}^\infty\) which is bounded away from \(0\).
  In particular, by \cref{i:5.3.12} we know that there exists a \(c \in \Q^+\) such that \(\abs{a_n} \geq c\) for all \(n \in \Z^+\).
  Fix such \(c\).
  Since \((a_n)_{n = 1}^{\infty}\) is a Cauchy sequence and \(c \in \Q^+\), by \cref{i:5.1.8} we have
  \[
    \exists N \in \Z^+ : \forall j, k \in \Z_{\geq N}, \abs{a_j - a_k} \leq c.
  \]
  Fix such \(N\).
  Then we have
  \begin{align*}
             & \forall j \in \Z_{\geq N}, \abs{a_j - a_N} \leq c          &  & (N \in \Z_{\geq N}) \\
    \implies & \forall j \in \Z_{\geq N}, -c \leq a_j - a_N \leq c        &  & \by{i:4.3.3}[c]     \\
    \implies & \forall j \in \Z_{\geq N}, -c + a_N \leq a_j \leq c + a_N. &  & \by{i:4.2.9}[c,d]
  \end{align*}
  Since \(\abs{a_N} \geq c \in \Q^+\), by \cref{i:4.3.1} we know that we have either \(a_N \in \Q^+\) or \(a_N \in \Q^-\).
  So we split into two cases:
  \begin{itemize}
    \item If \(a_N \in \Q^+\), then by \cref{i:4.3.1} we have \(a_N \geq c\).
          Thus
          \begin{align*}
                     & \forall j \in \Z_{\geq N}, 0 \leq -c + a_N \leq a_j \leq c + a_N &  & \by{i:4.2.9}[c,d] \\
            \implies & \forall j \in \Z_{\geq N}, 0 \leq a_j.                           &  & \by{i:4.2.9}[c]
          \end{align*}
          This means \(a_j \in \Q^+\) for all \(j \in \Z_{\geq N}\).
          If we now define \((b_n)_{n = 1}^\infty\) to be the sequence where \(b_n = c\) for all \(n \in \Z^+ \cap \Z_{< N}\) and \(b_n = a_n\) for all \(n \in \Z_{\geq N}\), then we see that \((a_n)_{n = 1}^\infty\) and \((b_n)_{n = 1}^\infty\) are equivalent rational Cauchy sequences and \((b_n)_{n = 1}^\infty\) is positively bounded away from zero.
          Thus by \cref{i:5.4.3} \(x\) is positive.
    \item If \(a_N \in \Q^-\), then by \cref{i:4.3.1} we have \(-a_N \geq c\).
          By \cref{i:ex:4.2.6} we have \(a_N \leq -c\).
          Thus
          \begin{align*}
                     & \forall j \in \Z_{\geq N}, -c + a_N \leq a_j \leq c + a_N \leq 0 &  & \by{i:4.2.9}[c,d] \\
            \implies & \forall j \in \Z_{\geq N}, a_j \leq 0.                           &  & \by{i:4.2.9}[c]
          \end{align*}
          This means \(a_j \in \Q^-\) for all \(j \in \Z_{\geq N}\).
          If we now define \((b_n)_{n = 1}^\infty\) to be the sequence where \(b_n = -c\) for all \(n \in \Z^+ \cap \Z_{< N}\) and \(b_n = a_n\) for all \(n \in \Z_{\geq N}\), then we see that \((a_n)_{n = 1}^\infty\) and \((b_n)_{n = 1}^\infty\) are equivalent rational Cauchy sequences and \((b_n)_{n = 1}^\infty\) is negatively bounded away from zero.
          Thus by \cref{i:5.4.3} \(x\) is negative.
  \end{itemize}
  From all cases above we see that \(x\) is either positive or negative.
  Thus we conclude that at least one of the three statements is true.

  Next we show that at most one of the three statements is true.
  Let \(x \in \R\).
  Suppose for sake of contradiction that one of the following three cases is true:
  \begin{itemize}
    \item We have both \(x = 0\) and \(x\) is positive.
          Then from the first part of the proof we know that \(x\) is the formal limit of a rational Cauchy sequence \((a_n)_{n = 1}^\infty\) which is positively bounded away from \(0\).
          In particular, there exists a \(c \in \Q^+\) such that \(a_n \geq c\) for all \(n \in \Z^+\).
          Fix such \(c\).
          Since \(x = 0\), by \cref{i:5.2.6} we know that \((a_n)_{n = 1}^\infty\) and \((0)_{n = 1}^\infty\) are eventually \(\varepsilon\)-close for all \(\varepsilon \in \Q^+\).
          Since \(c \in \Q^+\), we know that \(c / 2 \in \Q^+\).
          Thus \((a_n)_{n = 1}^\infty\) and \((0)_{n = 1}^\infty\) must be eventually \(c / 2\)-close, i.e., there exists an \(N \in \Z^+\) such that \(a_n = \abs{a_n - 0} \leq c / 2 < c\) for all \(n \in \Z_{\geq N}\).
          But this means for any \(n \in \Z_{\geq N}\), we have both \(a_n < c\) and \(a_n \geq c\), which contradict to \cref{i:4.2.9}(a).
    \item We have both \(x = 0\) and \(x\) is negative.
          Then from the first part of the proof we know that \(x\) is the formal limit of a rational Cauchy sequence \((a_n)_{n = 1}^\infty\) which is negatively bounded away from \(0\).
          In particular, there exists a \(c \in \Q^-\) such that \(a_n \leq c\) for all \(n \in \Z^+\).
          Fix such \(c\).
          Since \(x = 0\), by \cref{i:5.2.6} we know that \((a_n)_{n = 1}^\infty\) and \((0)_{n = 1}^\infty\) are eventually \(\varepsilon\)-close for all \(\varepsilon \in \Q^+\).
          Since \(c \in \Q^-\), we know that \(-c / 2 \in \Q^+\).
          Thus \((a_n)_{n = 1}^\infty\) and \((0)_{n = 1}^\infty\) must be eventually \(-c / 2\)-close, i.e., there exists an \(N \in \Z^+\) such that \(-a_n = \abs{a_n - 0} \leq -c / 2 < -c\) for all \(n \in \Z_{\geq N}\).
          But this means for any \(n \in \Z_{\geq N}\), we have both \(a_n > c\) and \(a_n \leq c\), which contradict to \cref{i:4.2.9}(a).
    \item We have \(x\) is both positive and negative.
          From the first part of the proof we know that \(x\) is the formal limit of a rational Cauchy sequence \((a_n)_{n = 1}^\infty\) which is positively bounded away from \(0\).
          Similarly we know that \(x\) is the formal limit of a rational Cauchy sequence \((b_n)_{n = 1}^\infty\) which is negatively bounded away from \(0\).
          By \cref{i:5.4.1} we know that there exist some \(c \in \Q^+\) and \(d \in \Q^-\) such that \(a_n \geq c\) and \(b_n \leq d\) for all \(n \in \Z^+\).
          Fix such \(c, d\) and observe that \(c - d \in \Q^+\).
          Since \(x = \LIM_{n \to \infty} a_n = \LIM_{n \to \infty} b_n\), by \cref{i:5.3.3} we know that \((a_n)_{n = 1}^\infty\) and \((b_n)_{n = 1}^\infty\) are equivalent rational Cauchy sequences.
          Thus by \cref{i:5.2.6} we have
          \[
            \exists N \in \Z^+ : \forall n \in \Z_{\geq n}, \abs{a_n - b_n} \leq \dfrac{c - d}{2} < c - d.
          \]
          Fix such \(N\).
          But then we have
          \begin{align*}
                     & \forall n \in \Z_{\geq N}, (a_n \geq c) \land (b_n \leq d)                                \\
            \implies & \forall n \in \Z_{\geq N}, (a_n \geq c) \land (-b_n \geq -d)       &  & \by{i:ex:4.2.6}   \\
            \implies & \forall n \in \Z_{\geq N}, a_n - b_n = \abs{a_n - b_n} \geq c - d, &  & \by{i:4.2.9}[c,d]
          \end{align*}
          which contradict to \cref{i:4.2.9}(a).
  \end{itemize}
  From all cases above we derived contradictions.
  Thus we conclude that at most one of the three statements is true.

  Next we show that \(x\) is negative iff \(-x\) is positive.
  Suppose that \(x\) is negative.
  From the first part of the proof we know that \(x\) is the formal limit of a rational Cauchy sequence \((a_n)_{n = 1}^\infty\) which is negatively bounded away from \(0\).
  This means \((-a_n)_{n = 1}^\infty\) is positively bounded away from \(0\).
  But by \cref{i:ac:5.3.2} we know that \(-x = \LIM_{n \to \infty} -a_n\), thus by \cref{i:5.4.1} we know that \(-x\) is positive.
  The converse argument holds by reversing the previous reasoning.

  Finally we show that \(x, y\) are positive implies \(x + y\) and \(xy\) are positive.
  From the first part of the proof we know that \(x\) and \(y\) are the formal limits of rational Cauchy sequences \((a_n)_{n = 1}^\infty\) and \((b_n)_{n = 1}^\infty\), respectively, which are positively bounded away from \(0\).
  Clearly, \((a_n + b_n)_{n = 1}^\infty\) and \((a_n b_n)_{n = 1}^\infty\) are positively bounded away from \(0\).
  But by \cref{i:5.3.4,i:5.3.9} we know that \(x + y = \LIM_{n \to \infty} a_n + b_n\) and \(xy = \LIM_{n \to \infty} a_n b_n\), thus by \cref{i:5.4.1} we know that \(x + y\) and \(xy\) are positive.
\end{proof}

\begin{note}
  If \(q\) is a positive rational number, then the Cauchy sequence \(q, q, q, \dots\) is positively bounded away from zero, and hence \(\LIM_{n \to \infty} q = q\) is a positive real number.
  Thus the notion of positivity for rationals is consistent with that for reals.
  Similarly, the notion of negativity for rationals is consistent with that for reals.
\end{note}

\begin{defn}[Absolute value]\label{i:5.4.5}
  Let \(x\) be a real number.
  We define the \emph{absolute value} \(\abs{x}\) of \(x\) to equal \(x\) if \(x\) is positive, \(-x\) when \(x\) is negative, and \(0\) when \(x\) is zero.
\end{defn}

\begin{defn}[Ordering of the real numbers]\label{i:5.4.6}
  Let \(x\) and \(y\) be real numbers.
  We say that \(x\) is \emph{greater than} \(y\), and write \(x > y\), iff \(x - y\) is a positive real number, and \(x < y\) iff \(x - y\) is a negative real number.
  We define \(x \geq y\) iff \(x > y\) or \(x = y\), and similarly define \(x \leq y\).
\end{defn}

\begin{note}
  Comparing \cref{i:5.4.6} with the definition of order on the rationals from \cref{i:4.2.8} we see that order on the reals is consistent with order on the rationals, i.e., if two rational numbers \(q, q'\) are such that \(q\) is less than \(q'\) in the rational number system, then \(q\) is still less than \(q'\) in the real number system, and similarly for ``greater than.''
  In the same way we see that the definition of absolute value given in \cref{i:5.4.5} is consistent with that in \cref{i:4.3.1}.
\end{note}

\begin{prop}\label{i:5.4.7}
  All the claims in \cref{i:4.2.9} which held for rationals, continue to hold for real numbers.
\end{prop}

\begin{proof}[\pf{i:5.4.7}(a)]
  By \cref{i:5.4.4} \(x - y\) satisfy exactly one of the following three statements:
  \begin{itemize}
    \item \(x - y = 0\).
          Then by \cref{i:5.3.11} we have \(x = y\).
    \item \(x - y\) is a positive rational number.
          Then by \cref{i:5.4.6} we have \(x > y\).
    \item \(x - y\) is a negative rational number.
          Then by \cref{i:5.4.6} we have \(x < y\).
  \end{itemize}
\end{proof}

\begin{proof}[\pf{i:5.4.7}(b)]
  We have
  \begin{align*}
         & x < y                                           \\
    \iff & x - y \text{ is negative}    &  & \by{i:5.4.6}  \\
    \iff & -(x - y) \text{ is positive} &  & \by{i:5.4.4}  \\
    \iff & y - x \text{ is positive}    &  & \by{i:5.3.11} \\
    \iff & y > x.                       &  & \by{i:5.4.6}
  \end{align*}
\end{proof}

\begin{proof}[\pf{i:5.4.7}(c)]
  We have
  \begin{align*}
             & (x < y) \land (y < z)                                                                  \\
    \implies & (x - y \text{ is negative}) \land (y - z \text{ is negative})       &  & \by{i:5.4.6}  \\
    \implies & (-(x - y) \text{ is positive}) \land (-(y - z) \text{ is positive}) &  & \by{i:5.4.4}  \\
    \implies & (y - x \text{ is positive}) \land (z - y \text{ is positive})       &  & \by{i:5.3.11} \\
    \implies & y - x + z - y \text{ is positive}                                   &  & \by{i:5.4.4}  \\
    \implies & -(x - z) \text{ is positive}                                        &  & \by{i:5.3.11} \\
    \implies & x - z \text{ is negative}                                           &  & \by{i:5.4.4}  \\
    \implies & x < z.                                                              &  & \by{i:5.4.6}
  \end{align*}
\end{proof}

\begin{proof}[\pf{i:5.4.7}(d)]
  We have
  \begin{align*}
             & x < y                                                    \\
    \implies & x - y \text{ is negative}             &  & \by{i:5.4.6}  \\
    \implies & x + z - z - y \text{ is negative}     &  & \by{i:5.3.11} \\
    \implies & (x + z) - (y + z) \text{ is negative} &  & \by{i:5.3.11} \\
    \implies & x + z < y + z.                        &  & \by{i:5.4.6}
  \end{align*}
\end{proof}

\begin{proof}[\pf{i:5.4.7}(e)]
  We have
  \begin{align*}
             & x < y                                             \\
    \implies & y > x                        &  & \by{i:5.4.7}[b] \\
    \implies & y - x \text{ is positive}    &  & \by{i:5.4.6}    \\
    \implies & (y - x)z \text{ is positive} &  & \by{i:5.4.4}    \\
    \implies & yz - xz \text{ is positive}  &  & \by{i:5.3.11}   \\
    \implies & yz > xz                      &  & \by{i:5.4.6}    \\
    \implies & xz < yz.                     &  & \by{i:5.4.7}[b]
  \end{align*}
\end{proof}

\begin{prop}\label{i:5.4.8}
  Let \(x\) be a positive real number.
  Then \(x^{-1}\) is also positive.
  Also, if \(y\) is another positive number and \(x > y\), then \(x^{-1} < y^{-1}\).
\end{prop}

\begin{proof}[\pf{i:5.4.8}]
  Let \(x\) be positive.
  Since \(xx^{-1} = 1\), the real number \(x^{-1}\) cannot be zero (since \(x0 = 0 \neq 1\)).
  Also, from \cref{i:5.4.4} it is easy to see that a positive number times a negative number is negative;
  this shows that \(x^{-1}\) cannot be negative, since this would imply that \(xx^{-1} = 1\) is negative, a contradiction.
  Thus by \cref{i:5.4.4}, the only possibility left is that \(x^{-1}\) is positive.

  Now let \(y\) be positive as well, so \(x^{-1}\) and \(y^{-1}\) are also positive.
  Suppose that \(x > y\).
  If \(x^{-1} \geq y^{-1}\), then by \cref{i:5.4.7} we have \(xx^{-1} > yx^{-1} \geq yy^{-1}\), thus \(1 > 1\), which is a contradiction.
  Thus we must have \(x^{-1} < y^{-1}\).
\end{proof}

\begin{prop}[The non-negative reals are closed]\label{i:5.4.9}
  Let \((a_n)_{n = 1}^\infty\) be a Cauchy sequence of non-negative rational numbers.
  Then \(\LIM_{n \to \infty} a_n\) is a non-negative real number.
\end{prop}

\begin{proof}[\pf{i:5.4.9}]
  We argue by contradiction, and suppose that the real number \(x \coloneqq \LIM_{n \to \infty} a_n\) is a negative number.
  Then by definition of negative real number, we have \(x = \LIM_{n \to \infty} b_n\) for some sequence \((b_n)_{n = 1}^\infty\) which is negatively bounded away from \(0\), i.e., there is a negative rational \(-c \in \Q^-\) such that \(b_n \leq -c\) for all \(n \in \Z^+\).
  On the other hand, we have \(a_n \in \Q_{\geq 0}\) for all \(n \in \Z^+\), by hypothesis.
  Thus the numbers \(a_n\) and \(b_n\) are never \(c / 2\)-close, since \(c / 2 < c\).
  Thus the sequences \((a_n)_{n = 1}^{\infty}\) and \((b_n)_{n = 1}^{\infty}\) are not eventually \(c / 2\)-close.
  Since \(c / 2 \in \Q^+\), this implies that \((a_n)_{n = 1}^{\infty}\) and \((b_n)_{n = 1}^{\infty}\) are not equivalent.
  But this contradicts the fact that both these sequences have \(x\) as their formal limit.
\end{proof}

\begin{note}
  Eventually, we will see a better explanation of \cref{i:5.4.9}:
  the set of non-negative reals is \emph{closed}, whereas the set of positive reals is \emph{open}.
  See \cref{i:sec:11.4}.
\end{note}

\begin{cor}\label{i:5.4.10}
  Let \((a_n)_{n = 1}^{\infty}\) and \((b_n)_{n = 1}^{\infty}\) be Cauchy sequences of rationals such that \(a_n \geq b_n\) for all \(n \in \Z^+\).
  Then \(\LIM_{n \to \infty} a_n \geq \LIM_{n \to \infty} b_n\).
\end{cor}

\begin{proof}[\pf{i:5.4.10}]
  Apply \cref{i:5.4.9} to the sequence \((a_n - b_n)_{n = 1}^\infty\).
\end{proof}

\begin{rmk}\label{i:5.4.11}
  Note that \cref{i:5.4.10} does not work if the \(\geq\) signs are replaced by \(>\):
  for instance if \(a_n \coloneqq 1 + 1 / n\) and \(b_n \coloneqq 1 - 1 / n\), then \(a_n\) is always strictly greater than \(b_n\), but the formal limit of \(a_n\) is not greater than the formal limit of \(b_n\), instead they are equal.
\end{rmk}

\begin{ac}\label{i:ac:5.4.1}
  We now define distance \(d(x, y) \coloneqq \abs{x - y}\) just as we did for the rationals.
  In fact, \cref{i:4.3.3,i:4.3.7} hold not only for the rationals, but for the reals;
  the proof is identical, since the real numbers obey all the laws of algebra and order that the rationals do.
\end{ac}

\begin{prop}[Bounding of reals by rationals]\label{i:5.4.12}
  Let \(x\) be a positive real number.
  Then there exists a positive rational number \(q\) such that \(q \leq x\), and there exists a positive integer \(N\) such that \(x \leq N\).
\end{prop}

\begin{proof}[\pf{i:5.4.12}]
  Since \(x\) is a positive real, it is the formal limit of a rational Cauchy sequence \((a_n)_{n = 1}^{\infty}\) which is positively bounded away from \(0\).
  Also, by \cref{i:5.1.15}, this sequence is bounded.
  Thus we have rationals \(q, r \in \Q^+\) such that \(q \leq a_n \leq r\) for all \(n \in \Z^+\).
  But by \cref{i:4.4.1} we know that there is some integer \(N\) such that \(r \leq N\);
  since \(q\) is positive and \(q \leq r \leq N\), we see that \(N\) is positive.
  Thus \(q \leq a_n \leq N\) for all \(n \in \Z^+\).
  Applying \cref{i:5.4.10} we obtain that \(q \leq x \leq N\), as desired.
\end{proof}

\begin{cor}[Archimedean property]\label{i:5.4.13}
  Let \(x\) and \(\varepsilon\) be any positive real numbers.
  Then there exists a positive integer \(M\) such that \(M\varepsilon > x\).
\end{cor}

\begin{proof}[\pf{i:5.4.13}]
  The number \(x / \varepsilon\) is positive, and hence by \cref{i:5.4.12} there exists a positive integer \(N\) such that \(x / \varepsilon \leq N\).
  If we set \(M \coloneqq N + 1\), then \(x / \varepsilon < M\).
  Now multiply by \(\varepsilon\).
\end{proof}

\begin{note}
  This property (\cref{i:5.4.13}) is quite important;
  it says that no matter how large \(x\) is and how small \(\varepsilon\) is, if one keeps adding \(\varepsilon\) to itself, one will eventually overtake \(x\).
\end{note}

\begin{prop}\label{i:5.4.14}
  Given any two real numbers \(x < y\), we can find a rational number \(q\) such that \(x < q < y\).
\end{prop}

\begin{proof}[\pf{i:5.4.14}]
  First observe that
  \begin{align*}
             & x < y                                                    \\
    \implies & y > x                               &  & \by{i:5.4.7}    \\
    \implies & y - x \text{ is positive}           &  & \by{i:5.4.6}    \\
    \implies & \exists N \in \Z^+ : y - x > 1 / N  &  & \by{i:ex:5.4.4} \\
    \implies & \exists N \in \Z^+ : y > x + 1 / N. &  & \by{i:5.4.7}
  \end{align*}
  Fix one such \(N\).
  Since \(x\) is a real number, by \cref{i:5.3.10} we know that \(Nx\) is also a real number.
  By \cref{i:ex:5.4.3}, there exists an \(M \in \Z\) such that \(M \leq Nx < M + 1\).
  So we have
  \begin{align*}
             & M \leq Nx < M + 1                                                                                \\
    \implies & \dfrac{M}{N} \leq x < \dfrac{M + 1}{N}                                      &  & \by{i:5.4.7}[e] \\
    \implies & \pa{\dfrac{M + 1}{N} \leq x + \dfrac{1}{N}} \land \pa{x < \dfrac{M + 1}{N}} &  & \by{i:5.4.7}[d] \\
    \implies & x < \dfrac{M + 1}{N} \leq x + \dfrac{1}{N} < y.                             &  & \by{i:5.4.7}[c]
  \end{align*}
  Clearly, \((M + 1) / N \in \Q\).
  So by setting \(q = (M + 1) / N\) we are done.
\end{proof}

\begin{rmk}\label{i:5.4.15}
  Up until now, we have not addressed the fact that real numbers can be expressed using the decimal system.
  For instance, the formal limit of
  \[
    1.4, 1.41, 1.414, 1.4142, 1.41421, \dots
  \]
  is more conventionally represented as the decimal \(1.41421\dots\).
  There are some subtleties in the decimal system, for instance \(0.9999\dots\) and \(1.000\dots\) are in fact the same real number.
\end{rmk}

\begin{ac}\label{i:ac:5.4.2}
  Let \(X\) be an non-empty finite subset of \(\R\).
  Then \(X\) has exactly one maximum \(\max(X) \in X\) satisfying
  \[
    \forall x \in X, x \leq \max(X).
  \]
  Similarly, \(X\) has exactly one minimum \(\min(X) \in X\) satisfying
  \[
    \forall x \in X, x \geq \min(X).
  \]
\end{ac}

\begin{proof}[\pf{i:ac:5.4.2}]
  Let \(n = \#(X)\).
  We induct on \(n\) to show that \(\max(X) \in X\) and \(\min(X) \in X\), and we start with \(n = 1\).
  For \(n = 1\), we have \(X = \set{x}\) for some \(x \in \R\).
  Then we have
  \begin{align*}
             & \forall y \in X, y = x                                        &  & \by{i:3.3}   \\
    \implies & (\forall y \in X, y \leq x) \land (\forall y \in X, y \geq x) &  & \by{i:5.4.6} \\
    \implies & \max(X) = \min(X) = x.
  \end{align*}
  Thus, the base case holds.
  Suppose inductively that for some \(n \in \Z^+\) we have \(\max(X) \in X\) and \(\min(X) \in X\).
  Now let \(X\) be a finite subset of real number with \(\#(X) = n + 1\).
  Let \(x \in X\) and let \(X' = X \setminus \set{x}\).
  By \cref{i:3.6.9} we know that \(\#(X') = n\).
  Since \(n \in \Z^+\), we know that \(X'\) is non-empty.
  Then we have
  \begin{align*}
             & (\max(X') \in X') \land (\min(X') \in X')                         &  & \byIH            \\
    \implies & (\max(X') \in X) \land (\min(X') \in X)                           &  & (X' \subseteq X) \\
    \implies & \begin{dcases}
                 \forall y \in X', y \leq \max(X') \leq \max(\max(X'), x) \in X \\
                 \forall y \in X', y \geq \min(X') \geq \min(\min(X'), x) \in X \\
                 x \leq \max(\max(X'), x)                                       \\
                 x \geq \min(\min(X'), x)
               \end{dcases} &  & \by{i:5.4.7}[a]                          \\
    \implies & \begin{dcases}
                 \forall y \in X, y \leq \max(\max(X'), x) \\
                 \forall y \in X, y \geq \min(\min(X'), x)
               \end{dcases}                                               \\
    \implies & \begin{dcases}
                 \max(X) = \max(\max(X'), x) \\
                 \min(X) = \min(\min(X'), x)
               \end{dcases}.
  \end{align*}
  This closes the induction.

  Now we show that both \(\max(X), \min(X)\) are unique.
  Suppose there are two \(x, x' \in X\) such that \(y \leq x\) and \(y \leq x'\) for all \(y \in X\).
  But since \(x, x' \in X\), we have \(x \leq x'\) and \(x' \leq x\).
  Thus by \cref{i:5.4.7}(a) we must have \(x = x'\) and \(\max(X)\) is unique.
  Using similarly arguments, we can show that \(\min(X)\) is unique.
\end{proof}

\begin{ac}\label{i:ac:5.4.3}
  Let \(x \in \R\).
  Then \(x\) is positive iff \(x > 0\);
  \(x\) is negative iff \(x < 0\).

  Let \(y \in \R\).
  We define the following eight subsets of \(\R\):
  \begin{align*}
    \R_{\leq x}   & \coloneqq \set{r \in \R : r \leq x};        & \R_{< x}   & \coloneqq \set{r \in \R : r < x};     & \R^+ & \coloneqq \R_{> 0}; \\
    \R_{\geq x}   & \coloneqq \set{r \in \R : r \geq x};        & \R_{> x}   & \coloneqq \set{r \in \R : r > x};     & \R^- & \coloneqq \R_{< 0}; \\
    \R_{x \leq y} & \coloneqq \set{r \in \R : x \leq r \leq y}; & \R_{x < y} & \coloneqq \set{r \in \R : x < r < y}. &      &
  \end{align*}
\end{ac}

\begin{proof}[\pf{i:ac:5.4.3}]
  By \cref{i:5.3.11} we have \(x = x - 0\).
  Thus \(x\) is a positive rational number iff \(x - 0\) is a positive rational number, iff \(x > 0\) (\cref{i:5.4.6}).
  Similarly \(x\) is a negative rational number iff \(x - 0\) is a negative rational number, iff \(x < 0\) (\cref{i:5.4.6}).
\end{proof}

\exercisesection

\begin{ex}\label{i:ex:5.4.1}
  Prove \cref{i:5.4.4}.
\end{ex}

\begin{proof}[\pf{i:ex:5.4.1}]
  See \cref{i:5.4.4}.
\end{proof}

\begin{ex}\label{i:ex:5.4.2}
  Prove the remaining claims in \cref{i:5.4.7}.
\end{ex}

\begin{proof}[\pf{i:ex:5.4.2}]
  See \cref{i:5.4.7}.
\end{proof}

\begin{ex}\label{i:ex:5.4.3}
  Show that for every real number \(x\) there is exactly one integer \(N\) such that \(N \leq x < N + 1\).
  (This integer \(N\) is called the \emph{integer part} of \(x\), and is sometimes denoted \(N = \floor{x}\).)
\end{ex}

\begin{proof}[\pf{i:ex:5.4.3}]
  We first prove the existence of the integer \(N\).
  By \cref{i:5.4.4}, exactly one of the following three statements is true:
  \begin{itemize}
    \item \(x = 0\).
          Then we choose \(N = 0\) so that \(0 \leq 0 < 1\).
    \item \(x\) is positive.
          Then by \cref{i:5.4.12} there exist some \(q \in \Q^+\) and \(N_1' \in \Z^+\) such that \(q \leq x \leq N_1'\).
          Let \(N_1 = N_1' + 1\).
          Then we have \(x < N_1\).
          By \cref{i:4.4.1} we know that there exists an \(N_2 \in \Z\) such that \(N_2 \leq q\), thus by \cref{i:5.4.7}(c) we have \(N_2 \leq x < N_1\).
          Let \(X\) be the set
          \[
            X = \set{n \in \Z : N_2 \leq n \leq x < N_1}.
          \]
          We know that \(X\) is finite since \(X \subseteq \Z_{N_2 \leq N_1}\) and \(\Z_{N_2 \leq N_1}\) is finite (\cref{i:3.6.14}(c)).
          We also know that \(X\) is non-empty since \(N_2 \in X\).
          By \cref{i:ac:5.4.2} we know that there exists an unique \(\max(X) \in X\).
          Let \(N = \max(X)\).
          By the definition of \(X\) we know that \(N \leq x < N_1\).
          We must have \(x < N + 1\), otherwise if \(N + 1 \leq x\) then we would have \(N + 1 \in X\), which means \(N + 1 \leq \max(X) = N\), a contradiction.
          Thus we have \(N \leq x < N + 1\).
    \item \(x\) is negative.
          Then by \cref{i:5.4.4} we know that \(-x\) is positive.
          By \cref{i:5.4.13} we know that there exists an \(M \in \Z^+\) such that \(-x < 1M = M\).
          Fix such \(M\).
          By \cref{i:5.4.7}(d) we know that \(0 < x + M\).
          From the above case we know that there exists an \(K \in \Z\) such that \(K \leq x + M < K + 1\).
          So by setting \(N = K - M\) we see that \(N \leq x < N + 1\).
  \end{itemize}
  From all cases above we conclude that for all \(x \in \R\), there exists an \(N \in \Z\) such that \(N \leq x < N + 1\).

  Now we prove the uniqueness of the integer \(N\).
  Let \(x \in \R\).
  Suppose that there exist some \(N_1, N_2 \in \Z\) such that \(N_1 \leq x < N_1 + 1\) and \(N_2 \leq x < N_2 + 1\).
  Then we have
  \begin{align*}
             & (N_1 \leq x < N_1 + 1) \land (N_2 \leq x < N_2 + 1)                      \\
    \implies & (N_1 < N_2 + 1) \land (N_2 < N_1 + 1)               &  & \by{i:5.4.7}[c] \\
    \implies & (N_1 + 1 \leq N_2 + 1) \land (N_2 + 1 \leq N_1 + 1) &  & \by{i:4.1.10}   \\
    \implies & (N_1 \leq N_2) \land (N_2 \leq N_1)                 &  & \by{i:4.1.10}   \\
    \implies & N_1 = N_2.                                          &  & \by{i:4.1.11}
  \end{align*}
  Thus for all \(x \in \R\), \(N\) is unique and \(\floor{x}\) is well-defined.
\end{proof}

\begin{ex}\label{i:ex:5.4.4}
  Show that for any positive real number \(x > 0\) there exists a positive integer \(N\) such that \(x > 1 / N > 0\).
\end{ex}

\begin{proof}[\pf{i:ex:5.4.4}]
  We have
  \begin{align*}
             & x > 0                                                   \\
    \implies & x^{-1} > 0                           &  & \by{i:5.4.8}  \\
    \implies & \exists N \in \Z^+ : N1 = N > x^{-1} &  & \by{i:5.4.13} \\
    \implies & N^{-1} = \dfrac{1}{N} < x.           &  & \by{i:5.4.8}
  \end{align*}
\end{proof}

\begin{ex}\label{i:ex:5.4.5}
  Prove \cref{i:5.4.14}.
\end{ex}

\begin{proof}[\pf{i:ex:5.4.5}]
  See \cref{i:5.4.14}.
\end{proof}

\begin{ex}\label{i:ex:5.4.6}
  Let \(x, y \in \R\) and let \(\varepsilon \in \R^+\).
  Show that \(\abs{x - y} < \varepsilon\) iff \(y - \varepsilon < x < y + \varepsilon\), and that \(\abs{x - y} \leq \varepsilon\) iff \(y - \varepsilon \leq x \leq y + \varepsilon\).
\end{ex}

\begin{proof}[\pf{i:ex:5.4.6}]
  We first show that \(\abs{x - y} < \varepsilon \iff y - \varepsilon < x < y + \varepsilon\).
  \begin{align*}
         & \abs{x - y} < \varepsilon                                                                               \\
    \iff & \pa{-(x - y) \leq x - y < \varepsilon} \lor \pa{x - y \leq -(x - y) < \varepsilon} &  & \by{i:5.4.5}    \\
    \iff & (x - y < \varepsilon) \land (-(x - y) < \varepsilon)                               &  & \by{i:5.4.7}[a] \\
    \iff & (x - y < \varepsilon) \land (y - x < \varepsilon)                                  &  & \by{i:5.3.11}   \\
    \iff & (x < y + \varepsilon) \land (y - \varepsilon < x)                                  &  & \by{i:5.4.7}[d] \\
    \iff & y - \varepsilon < x < y + \varepsilon.                                             &  & \by{i:5.4.7}[c]
  \end{align*}

  Now we show that \(\abs{x - y} \leq \varepsilon \iff y - \varepsilon \leq x \leq y + \varepsilon\).
  By replacing \(<\) with \(\leq\) in the above arguments, we are done.
\end{proof}

\begin{ex}\label{i:ex:5.4.7}
  Let \(x, y \in \R\).
  Show that \(x \leq y + \varepsilon\) for all \(\varepsilon \in \R^+\) iff \(x \leq y\).
  Show that \(\abs{x - y} \leq \varepsilon\) for all \(\varepsilon \in \R^+\) iff \(x = y\).
\end{ex}

\begin{proof}[\pf{i:ex:5.4.7}]
  We first show that \(x \leq y + \varepsilon\) for all \(\varepsilon \in \R^+\) iff \(x \leq y\).
  \begin{align*}
         & \forall \varepsilon \in \R^+, x \leq y + \varepsilon                      \\
    \iff & \forall \varepsilon \in \R^+, x - y \leq \varepsilon &  & \by{i:5.4.7}[d] \\
    \iff & \lnot (x - y > 0)                                                         \\
    \iff & x - y \leq 0                                         &  & \by{i:5.4.7}[a] \\
    \iff & x \leq y.                                            &  & \by{i:5.4.7}[d]
  \end{align*}

  Now we show that \(\abs{x - y} \leq \varepsilon\) for all \(\varepsilon \in \R^+\) iff \(x = y\).
  \begin{align*}
         & \forall \varepsilon \in \R^+, \abs{x - y} \leq \varepsilon                                                   \\
    \iff & \forall \varepsilon \in \R^+, y - \varepsilon \leq x \leq y + \varepsilon &  & \by{i:ex:5.4.6}               \\
    \iff & (x \leq y) \land (y \leq x)                                               &  & \text{(from the proof above)} \\
    \iff & x = y.                                                                    &  & \by{i:5.4.7}[a]
  \end{align*}
\end{proof}

\begin{ex}\label{i:ex:5.4.8}
  Let \((a_n)_{n = 1}^{\infty}\) be a Cauchy sequence of rationals, and let \(x\) be a real number.
  Show that if \(a_n \leq x\) for all \(n \in \Z^+\), then \(\LIM_{n \to \infty} a_n \leq x\).
  Similarly, show that if \(a_n \geq x\) for all \(n \in \Z^+\), then \(\LIM_{n \to \infty} a_n \geq x\).
\end{ex}

\begin{proof}[\pf{i:ex:5.4.8}]
  We first show that if \(a_n \leq x\) for all \(n \in \Z^+\), then \(\LIM_{n \to \infty} a_n \leq x\).
  Let \(a = \LIM_{n \to \infty} a_n\).
  Suppose for sake of contradiction that \(a > x\).
  Then by \cref{i:5.4.14}, there exists a \(q \in \Q\) such that \(a > q > x\).
  Since \(q > x\), we have \(a_n \leq x < q\) for all \(n \in \Z^+\).
  But by \cref{i:5.4.10} we have \(a = \LIM_{n \to \infty} a_n \leq \LIM_{n \to \infty} q = q\), which contradict to \(a > q\).
  Thus we must have \(a \leq x\).

  Now we show that if \(a_n \geq x\) for all \(n \in \Z^+\), then \(\LIM_{n \to \infty} a_n \geq x\).
  This is true since
  \begin{align*}
             & a_n \geq x                                                          \\
    \implies & -a_n \leq -x                     &  & \by{i:ex:4.2.6}               \\
    \implies & \LIM_{n \to \infty} -a_n \leq -x &  & \text{(from the proof above)} \\
    \implies & -\LIM_{n \to \infty} a_n \leq -x &  & \by{i:5.3.10}                 \\
    \implies & \LIM_{n \to \infty} a_n \geq x.  &  & \by{i:ex:4.2.6}
  \end{align*}
\end{proof}

\section{The least upper bound property}\label{i:sec:5.5}

\begin{defn}[Upper bound]\label{i:5.5.1}
  Let \(E\) be a subset of \(\R\), and let \(M\) be a real number.
  We say that \(M\) is an \emph{upper bound} for \(E\), iff we have \(x \leq M\) for every element \(x\) in \(E\).
\end{defn}

\setcounter{thm}{2}
\begin{eg}\label{i:5.5.3}
  Let \(\R^+\) be the set of positive reals: \(\R^+ \coloneqq \set{x \in \R : x > 0}\).
  Then \(\R^+\) does not have any upper bounds at all.
  (More precisely, \(\R^+\) has no upper bounds which are real numbers.)
\end{eg}

\begin{proof}
  Suppose for sake of contradiction that \(\exists M \in \R\) such that \(M\) is an upper bound for \(\R^+\).
  Then \(\forall x \in \R^+\) we have \(x \leq M\).
  Since \(x > 0\), by \cref{i:5.4.7} we have \(M > 0\), thus \(M \in \R^+\).
  But this means \(M + 1 > M\) and \(M + 1 \in \R^+\), and we must have \(M > M + 1\), a contradiction.
  Thus \(\nexists M \in \R\) such that \(M\) is an upper bound for \(\R^+\).
\end{proof}

\begin{eg}\label{i:5.5.4}
  Let \(\emptyset\) be the empty set.
  Then every number \(M\) is an upper bound for \(\emptyset\), because \(M\) is greater than every element of the empty set
  (this is a vacuously true statement, but still true).
\end{eg}

\begin{note}
  It is clear that if \(M\) is an upper bound of \(E\), then any larger number \(M' \geq M\) is also an upper bound of \(E\).
  On the other hand, it is not so clear whether it is also possible for any number smaller than \(M\) to also be an upper bound of \(E\).
\end{note}

\begin{defn}[Least upper bound]\label{i:5.5.5}
  Let \(E\) be a subset of \(\R\), and \(M\) be a real number.
  We say that \(M\) is a \emph{least upper bound} for \(E\) iff
  \begin{enumerate}
    \item \(M\) is an upper bound for \(E\), and also
    \item any other upper bound \(M'\) for \(E\) must be larger than or equal to \(M\).
  \end{enumerate}
\end{defn}

\setcounter{thm}{6}
\begin{eg}\label{i:5.5.7}
  The empty set does not have a least upper bound.
\end{eg}

\begin{proof}
  Suppose for sake of contradiction that \(\exists M \in \R\) such that \(M\) is a least upper bound of \(\emptyset\).
  By \cref{i:5.5.5} we know that \(\forall x \in \emptyset\), \(x \leq M\).
  But by \cref{i:5.5.4} we know that \(M - 1\) is also a upper bound of \(\emptyset\), so by \cref{i:5.5.5} we have \(M < M - 1\), a contradiction.
  Thus \(\emptyset\) does not have a least upper bound.
\end{proof}

\begin{prop}[Uniqueness of least upper bound]\label{i:5.5.8}
  Let \(E\) be a subset of \(\R\).
  Then \(E\) can have at most one least upper bound.
\end{prop}

\begin{proof}
  Let \(M_1\) and \(M_2\) be two least upper bounds of \(E\).
  Since \(M_1\) is a least upper bound and \(M_2\) is an upper bound, then by definition of least upper bound we have \(M_2 \geq M_1\).
  Since \(M_2\) is a least upper bound and \(M_1\) is an upper bound, we similarly have \(M_1 \geq M_2\).
  Thus \(M_1 = M_2\).
  Thus there is at most one least upper bound.
\end{proof}

\begin{thm}[Existence of least upper bound]\label{i:5.5.9}
  Let \(E\) be a non-empty subset of \(\R\).
  If \(E\) has an upper bound (i.e., \(E\) has some upper bound \(M\)), then it must have exactly one least upper bound.
\end{thm}

\begin{proof}
  Let \(E\) be a non-empty subset of \(\R\) with an upper bound \(M\).
  By \cref{i:5.5.8}, we know that \(E\) has at most one least upper bound;
  we have to show that \(E\) has at least one least upper bound.
  Since \(E\) is non-empty, we can choose some element \(x_0\) in \(E\).

  Let \(n \geq 1\) be a positive integer.
  We know that \(E\) has an upper bound \(M\).
  By the Archimedean property (\cref{i:5.4.13}), we can find an integer \(K\) such that \(K / n \geq M\), and hence \(K / n\) is also an upper bound for \(E\).
  (Note that \(K\) is positive, and \(M\) can be either zero or negative, but \(K / n\) is positive, so we are fine.)
  By the Archimedean property again, there exists another integer \(L\) such that \(L / n < x_0\).
  (Note that if \(x_0 \geq 0\), then we can set \(L = -1\); if \(x_0 < 0\), then \(-x_0\) is positive, so by Archimedean property we have some \(-L \in \Z^+\) such that \(-L / n > -x_0\).)
  Since \(x_0\) lies in \(E\), we see that \(L / n\) is not an upper bound for \(E\).
  Since \(K / n\) is an upper bound but \(L / n\) is not, we see that \(K \geq L\).

  Since \(K / n\) is an upper bound for \(E\) and \(L / n\) is not, we can find an integer \(L < m_n \leq K\) with the property that \(m_n / n\) is an upper bound for \(E\), but \((m_n - 1) / n\) is not (see \cref{i:ex:5.5.2}).
  In fact, this integer \(m_n\) is unique (\cref{i:ex:5.5.3}).
  We subscript \(m_n\) by \(n\) to emphasize the fact that this integer \(m\) depends on the choice of \(n\).
  This gives a well-defined (and unique) sequence \(m_1, m_2, m_3, \dots\) of integers, with each of the \(m_n / n\) being upper bounds and each of the \((m_n - 1) / n\) not being upper bounds.

  Now let \(N \geq 1\) be a positive integer, and let \(n, n' \geq N\) be integers larger than or equal to \(N\).
  Since \(m_n / n\) is an upper bound for \(E\) and \((m_{n'} - 1) / n'\) is not, we must have \(m_n / n > (m_{n'} - 1) / n'\).
  After a little algebra, this implies that
  \[
    \dfrac{m_n}{n} - \dfrac{m_{n'}}{n'} > -\dfrac{1}{n'} \geq -\dfrac{1}{N}.
  \]
  Similarly, since \(m_{n'} / n'\) is an upper bound for \(E\) and \((m_n - 1) / n\) is not, we have \(m_{n'} / n' > (m_n - 1) / n\), and hence
  \[
    \dfrac{m_n}{n} - \dfrac{m_{n'}}{n'} < \dfrac{1}{n} \leq \dfrac{1}{N}.
  \]
  Putting these two bounds together, we see that
  \[
    \abs{\dfrac{m_n}{n} - \dfrac{m_{n'}}{n'}} \leq \dfrac{1}{N} \text{ for all } n, n' \geq N \geq 1.
  \]
  This implies that \(\dfrac{m_n}{n}\) is a Cauchy sequence (\cref{i:ex:5.5.4}).
  Since the \(\dfrac{m_n}{n}\) are rational numbers, we can now define the real number \(S\) as
  \[
    S \coloneqq \text{LIM}_{n \to \infty} \dfrac{m_n}{n}.
  \]
  From \cref{i:ex:5.3.5} we conclude that
  \[
    S = \text{LIM}_{n \to \infty} \dfrac{m_n - 1}{n}.
  \]
  To finish the proof of the theorem, we need to show that \(S\) is the least upper bound for \(E\).
  First we show that it is an upper bound.
  Let \(x\) be any element of \(E\).
  Then, since \(m_n / n\) is an upper bound for \(E\), we have \(x \leq m_n / n\) for all \(n \geq 1\).
  Applying \cref{i:ex:5.4.8}, we conclude that \(x \leq \text{LIM}_{n \to \infty} m_n / n = S\).
  Thus \(S\) is indeed an upper bound for \(E\).

  Now we show it is a least upper bound.
  Suppose \(y\) is an upper bound for \(E\).
  Since \((m_n - 1) / n\) is not an upper bound, we conclude that \(y \geq (m_n - 1) / n\) for all \(n \geq 1\).
  Applying \cref{i:ex:5.4.8}, we conclude that \(y \geq \text{LIM}_{n \to \infty} (m_n - 1) / n = S\).
  Thus the upper bound \(S\) is less than or equal to every upper bound of \(E\), and \(S\) is thus a least upper bound of \(E\).
\end{proof}

\begin{defn}[Supremum]\label{i:5.5.10}
  Let \(E\) be a subset of the real numbers.
  If \(E\) is non-empty and has some upper bound, we define \(\sup(E)\) to be the least upper bound of \(E\)
  (this is well-defined by \cref{i:5.5.9}).
  We introduce two additional symbols, \(+\infty\) and \(-\infty\).
  If \(E\) is non-empty and has no upper bound, we set \(\sup(E) \coloneqq +\infty\);
  if \(E\) is empty, we set \(\sup(E) \coloneqq -\infty\).
  We refer to \(\sup(E)\) as the \emph{supremum} of \(E\), and also denote it by \(\sup E\).
\end{defn}

\begin{rmk}\label{i:5.5.11}
  At present, \(+\infty\) and \(-\infty\) are meaningless symbols;
  we have no operations on them at present, and none of our results involving real numbers apply to \(+\infty\) and \(-\infty\), because these are not real numbers.
  In Section 6.2 we add \(+\infty\) and \(-\infty\) to the reals to form the \emph{extended real number system}, but this system is not as convenient to work with as the real number system, because many of the laws of algebra break down.
  For instance, it is not a good idea to try to define \(+\infty + -\infty\);
  setting this equal to \(0\) causes some problems.
\end{rmk}

\begin{prop}\label{i:5.5.12}
  There exists a positive real number \(x\) such that \(x^2 = 2\).
\end{prop}

\begin{proof}
  Let \(E\) be the set \(\set{y \in R : y \geq 0 \text{ and } y^2 < 2}\);
  thus \(E\) is the set of all non-negative real numbers whose square is less than \(2\).
  Observe that \(E\) has an upper bound of \(2\) (because if \(y > 2\), then \(y^2 > 4 > 2\) and hence \(y \notin E\)).
  Also, \(E\) is non-empty (for instance, \(1\) is an element of \(E\)).
  Thus by the least upper bound property (\cref{i:5.5.9}), we have a real number \(x \coloneqq \sup(E)\) which is the least upper bound of \(E\).
  Then \(x\) is greater than or equal to \(1\) (since \(1 \in E\)) and less than or equal to \(2\)
  (since \(2\) is an upper bound for \(E\)).
  So \(x\) is positive.
  Now we show that \(x^2 = 2\).

  We argue this by contradiction.
  We show that both \(x^2 < 2\) and \(x^2 > 2\) lead to contradictions.
  First suppose that \(x^2 < 2\).
  Let \(0 < \varepsilon < 1\) be a small number;
  then we have
  \[
    (x + \varepsilon)^2 = x^2 + 2\varepsilon x + \varepsilon^2 \leq x^2 + 4\varepsilon + \varepsilon = x^2 + 5\varepsilon
  \]
  since \(x \leq 2\) and \(\varepsilon^2 \leq \varepsilon\).
  Since \(x^2 < 2\), we see that we can choose an \(0 < \varepsilon < 1\) such that \(x^2 + 5\varepsilon < 2\), thus \((x + \varepsilon)^2 < 2\).
  (By \cref{i:5.4.14}, \(\exists q \in \Q\) such that \(x^2 < q < 2\))
  By construction of \(E\), this means that \(x + \varepsilon \in E\);
  but this contradicts the fact that \(x\) is an upper bound of \(E\).

  Now suppose that \(x^2 > 2\).
  Let \(0 < \varepsilon < 1\) be a small number;
  then we have
  \[
    (x - \varepsilon)^2 = x^2 - 2\varepsilon x + \varepsilon^2 \geq x^2 - 2\varepsilon x \geq x^2 - 4\varepsilon
  \]
  since \(x \leq 2\) and \(\varepsilon^2 \geq 0\).
  Since \(x^2 > 2\), we can choose \(0 < \varepsilon < 1\) such that \(x^2 - 4\varepsilon > 2\), and thus \((x - \varepsilon)^2 > 2\).
  But then this implies that \(x - \varepsilon \geq y\) for all \(y \in E\).
  (Why? If \(x - \varepsilon < y\) then \((x - \varepsilon)^2 < y^2 \leq 2\), a contradiction.)
  Thus \(x - \varepsilon\) is an upper bound for \(E\), which contradicts the fact that \(x\) is the \emph{least} upper bound of \(E\).
  From these two contradictions we see that \(x^2 = 2\), as desired.
\end{proof}

\begin{rmk}\label{i:5.5.13}
  Comparing \cref{i:5.5.12} with \cref{i:4.4.4}, we see that certain numbers are real but not rational.
  The proof of this proposition also shows that the rationals \(\Q\) do not obey the least upper bound property, otherwise one could use that property to construct a square root of \(2\), which by \cref{i:4.4.4} is not possible.
\end{rmk}

\begin{rmk}\label{i:5.5.14}
  In Chapter 6 we will use the least upper bound property to develop the theory of limits, which allows us to do many more things than just take square roots.
\end{rmk}

\begin{rmk}\label{i:5.5.15}
  We can of course talk about lower bounds, and greatest lower bounds, of sets \(E\);
  the greatest lower bound of a set \(E\) is also known as the \emph{infimum} of \(E\) and is denoted \(\inf(E)\) or \(\inf E\).
  Everything we say about suprema has a counterpart for infima;
  we will usually leave such statements to the reader.
\end{rmk}

\begin{note}
  Supremum means ``highest'' and infimum means ``lowest'', and the plurals are suprema and infima.
  Supremum is to superior, and infimum to inferior, as maximum is to major, and minimum to minor.
  The root words are ``super'', which means ``above'', and ``infer'', which means ``below''
  (this usage only survives in a few rare English words such as ``infernal'', with the Latin prefix ``sub'' having mostly replaced ``infer'' in English).
\end{note}

\exercisesection

\begin{ex}\label{i:ex:5.5.1}
  Let \(E\) be a subset of the real numbers \(\R\), and suppose that \(E\) has a least upper bound \(M\) which is a real number, i.e., \(M = \sup(E)\).
  Let \(-E\) be the set
  \[
    -E \coloneqq \set{-x : x \in E}
  \]
  Show that \(-M\) is the greatest lower bound of \(-E\), i.e., \(-M = \inf(-E)\).
\end{ex}

\begin{proof}
  We first show that \(-M\) is a greatest lower bound for \(-E\).
  Let \(L \in \R\) be any lower bound for \(-E\).
  Then we have
  \begin{align*}
             & \forall -x \in -E, L \leq -x                                   \\
    \implies & x \leq -L                                    &  & \by{i:5.4.7} \\
    \implies & x \leq M \leq -L                             &  & \by{i:5.5.5} \\
    \implies & L \geq -M \geq -x                            &  & \by{i:5.4.7} \\
    \implies & -M \text{ is a greatest lower bound of } -E.
  \end{align*}

  Now we show that the greatest lower bound is unique.
  Let \(M, M'\) be two greatest lower bounds of \(-E\).
  Then we have \(M \leq M'\) and \(M \geq M'\), which means \(M = M'\).
  So the greatest lower bound is unique.
\end{proof}

\begin{ex}\label{i:ex:5.5.2}
  Let \(E\) be a non-empty subset of \(\R\), let \(n \geq 1\) be an integer, and let \(L < K\) be integers.
  Suppose that \(K / n\) is an upper bound for \(E\), but that \(L / n\) is not an upper bound for \(E\).
  Without using \cref{i:5.5.9}, show that there exists an integer \(L < m \leq K\) such that \(m / n\) is an upper bound for \(E\), but that \((m - 1) / n\) is not an upper bound for \(E\).
\end{ex}

\begin{proof}
  Let \(d = K - L\), so \(d\) is positive by \cref{i:4.1.10}.
  Now we use induction on \(d\) to show that for all \(d \in \Z^+\), \(\exists m \in \Z\) such that \(L < m \leq K\) and \(m / n\) is an upper bound for \(E\), but that \((m - 1) / n\) is not an upper bound for \(E\).
  We start with \(d = 1\).
  For \(d = 1\), we have \(K - 1 = L\).
  Then let \(m = K\).
  So by hypothesis we have \(L < m = K \leq K\), \(m / n = K / n\) is an upper bound for \(E\) and \((m - 1) / n = (K - 1) / n = L / n\) is not an upper bound for \(E\).
  Thus the base case holds.
  Suppose inductively that the statement holds for some \(d \geq 1\).
  Then for \(d + 1\), we need to show that the statement is also true.
  We have \(K - L = d + 1\).
  Since \(K / n\) is an upper bound for \(E\), we can ask whether \((K - 1) / n\) is an upper bound for \(E\).
  If \((K - 1) / n\) is not an upper bound for \(E\), then we can choose \(m = K\) and we are done.
  If \((K - 1) / n\) is an upper bound for \(E\), then we know that \(L = K - 1 - d < K - 1\) since \(d \geq 1\).
  Since \(K - 1 - L = d\), by induction hypothesis \(\exists m \in \Z\) such that \(L < m \leq K - 1\) and \(m / n\) is an upper bound for \(E\), but \((m - 1) / n\) is not an upper bound for \(E\).
  This closes the induction.
\end{proof}

\begin{ex}\label{i:ex:5.5.3}
  Let \(E\) be a non-empty subset of \(\R\), let \(n \geq 1\) be an integer, and let \(m, m'\) be integers with the properties that \(m / n\) and \(m' / n\) are upper bounds for \(E\), but \((m - 1) / n\) and \((m' - 1) / n\) are not upper bounds for \(E\).
  Show that \(m = m'\).
  This shows that the integer \(m\) constructed in \cref{i:ex:5.5.2} is unique.
\end{ex}

\begin{proof}
  Suppose for sake of contradiction that \(m \neq m'\).
  Then by \cref{i:4.1.11}, we have either \(m < m'\) or \(m > m'\), but not both.
  \begin{itemize}
    \item If \(m < m'\), then by \cref{i:4.1.11}(e) we have \(m - 1 < m \leq m' - 1\).
          Again by \cref{i:4.1.11}(e) we have \(m / n \leq (m' - 1) / n\), which means \((m' - 1) / n\) is also a upper bound for \(E\), a contradiction.
    \item If \(m > m'\), then by \cref{i:4.1.10} we have \(m' < m\).
          Using similar proof as above we can show that \((m - 1) / n\) is also a upper bound for \(E\), a contradiction.
  \end{itemize}
  Thus we must have \(m = m'\).
\end{proof}

\begin{ex}\label{i:ex:5.5.4}
  Let \(q_1, q_2, q_3, \dots\) be a sequence of rational numbers with the property that \(\abs{q_n - q_{n'}} \leq \dfrac{1}{M}\) whenever \(M \geq 1\) is an integer and \(n, n' \geq M\).
  Show that \(q_1, q_2, q_3, \dots\) is a Cauchy sequence.
  Furthermore, if \(S \coloneqq \text{LIM}_{n \to \infty} q_n\), show that \(\abs{q_M - S} \leq \dfrac{1}{M}\) for every \(M \geq 1\).
\end{ex}

\begin{proof}
  We first show that \(q_1, q_2, q_3, \dots\) is a Cauchy sequence.
  By Archimedean property (\cref{i:5.4.13}) we know that \(\forall \varepsilon \in \Q^+\), \(\exists M \in \Z^+\) such that \(M\varepsilon > 1\).
  By \cref{i:4.2.9} we have \(\varepsilon > 1 / M\).
  By hypothesis we have \(\forall n, n' \geq M\), \(\abs{q_n - q_{n'}} \leq \dfrac{1}{M} < \varepsilon\).
  Since \(\varepsilon\) is arbitrary, by \cref{i:5.1.8} \((q_n)_{n = 1}^{\infty}\) is a Cauchy sequence.

  Now we show that if \(S = \text{LIM}_{n \to \infty} q_n\), then \(\abs{q_M - S} \leq \dfrac{1}{M}\) for every \(M \geq 1\).
  From proof above we know that \((q_n)_{n = 1}^\infty\) is indeed a Cauchy sequence, thus \(S = \text{LIM}_{n \to \infty} q_n\) is well-defined.
  By hypothesis we have \(\abs{q_M - q_n} \leq 1 / M\) for all \(M \in \Z^+\) and for all \(n \geq M\).
  Then we have
  \begin{align*}
             & \abs{q_M - q_n} \leq 1 / M                                                            \\
    \implies & \abs{q_n - q_M} \leq 1 / M                                       &  & \by{i:4.3.1}    \\
    \implies & -1 / M \leq q_n - q_M \leq 1 / M                                 &  & \by{i:4.3.3}[c] \\
    \implies & -1 / M + q_M \leq q_n \leq 1 / M + q_M                           &  & \by{i:4.2.9}[d] \\
    \implies & -1 / M + q_M \leq \text{LIM}_{n \to \infty} q_n \leq 1 / M + q_M &  & \by{i:ex:5.4.8} \\
    \implies & -1 / M + q_M \leq S \leq 1 / M + q_M                                                  \\
    \implies & -1 / M \leq S - q_M \leq 1 / M                                   &  & \by{i:5.4.7}    \\
    \implies & \abs{S - q_M} \leq 1 / M                                         &  & \by{i:ex:5.4.6} \\
    \implies & \abs{q_M - S} \leq 1 / M.                                        &  & \by{i:5.4.5}
  \end{align*}
\end{proof}

\begin{ex}\label{i:ex:5.5.5}
  Establish an analogue of \cref{i:5.4.14}, in which ``rational'' is replaced by ``irrational''.
\end{ex}

\begin{proof}
  Let \(x, y, z \in \R\) where \(x < y\) and \(z^2 = 2\).
  (\(z\) is well-defined thanks to \cref{i:5.5.12})
  So by \cref{i:5.4.7} we have \(x - z < y - z\).
  But by \cref{i:5.4.14} \(\exists q \in \Q\) such that \(x - z < q < y - z\).
  Again by \cref{i:5.4.7} we have \(x < q + z < y\).
  Because \(z\) is irrational, \(q + z\) is also irrational
  (otherwised we have \(a = q + z \in \Q\) and \(z = a - q \in \Q\), contradicts to \cref{i:4.4.4}).
  So we have an irrational number in between any two real numbers \(x, y\) where \(x < y\).
\end{proof}

\section{Real exponentiation, part I}

\begin{definition}[Exponentiating a real by a natural number]\label{5.6.1}
Let \(x\) be a real number.
To raise \(x\) to the power \(0\), we define \(x^0 \coloneqq 1\).
Now suppose recursively that \(x^n\) has been defined for some natural number \(n\), then we define \(x^{n + 1} \coloneqq x^n \times x\).
\end{definition}

\begin{definition}[Exponentiating a real by an integer]\label{5.6.2}
Let \(x\) be a non-zero real number.
Then for any negative integer \(-n\), we define \(x^{-n} \coloneqq 1 / x^n\).
\end{definition}

\begin{proposition}\label{5.6.3}
All the properties in Propositions \ref{4.3.10} and \ref{4.3.12} remain valid if \(x\) and \(y\) are assumed to be real numbers instead of rational numbers.
\end{proposition}

\begin{meta-proof}
If one inspects the proof of Propositions \ref{4.3.10} and \ref{4.3.12} we see that they rely on the laws of algebra and the laws of order for the rationals (Propositions \ref{4.2.4} and \ref{4.2.9}).
But by Propositions \ref{5.3.11}, \ref{5.4.7}, and the identity \(xx^{-1} = x^{-1} x = 1\) we know that all these laws of algebra and order continue to hold for real numbers as well as rationals.
Thus we can modify the proof of Proposition \ref{4.3.10} and \ref{4.3.12} to hold in the case when \(x\) and \(y\) are real.
\end{meta-proof}

\begin{note}
Instead of giving an actual proof of Proposition \ref{5.6.3}, we shall give a meta-proof
(an argument appealing to the nature of proofs, rather than the nature of real and rational numbers).
\end{note}

\begin{definition}\label{5.6.4}
Let \(x \geq 0\) be a non-negative real, and let \(n \geq 1\) be a positive integer.
We define \(x^{1 / n}\), also known as the \emph{\(n^{\text{th}}\) root of \(x\)}, by the formula
\[
    x^{1 / n} \coloneqq \sup\{y \in \mathds{R} : y \geq 0 \text{ and } y^n \leq x\}.
\]
We often write \(\sqrt{x}\) for \(x^{1 / 2}\).
\end{definition}

\begin{note}
we do not define the \(n^{\text{th}}\) root of a negative number.
In fact, we will leave the \(n^{\text{th}}\) roots of negative numbers undefined for the rest of the text
(one can define these \(n^{\text{th}}\) roots once one defines the complex numbers, but we shall refrain from doing so).
\end{note}

\begin{lemma}[Existence of \(n^{\text{th}}\) roots]\label{5.6.5}
Let \(x \geq 0\) be a non-negative real, and let \(n \geq 1\) be a positive integer.
Then the set \(E \coloneqq \{y \in R : y \geq 0 \text{ and } y^n \leq x\}\) is non-empty and is also bounded above.
In particular, \(x^{1 / n}\) is a real number.
\end{lemma}

\begin{proof}
The set \(E\) contains \(0\), so it is certainly not empty.
Now we show it has an upper bound.
We divide into two cases: \(x \leq 1\) and \(x > 1\).
First suppose that we are in the case where \(x \leq 1\).
Then we claim that the set \(E\) is bounded above by \(1\).
To see this, suppose for sake of contradiction that there was an element \(y \in E\) for which \(y > 1\).
But then \(y^n > 1\), and hence \(y^n > x\), a contradiction.
Thus \(E\) has an upper bound.
Now suppose that we are in the case where \(x > 1\).
Then we claim that the set \(E\) is bounded above by \(x\).
To see this, suppose for contradiction that there was an element \(y \in E\) for which \(y > x\).
Since \(x > 1\), we thus have \(y > 1\).
Since \(y > x\) and \(y > 1\), we have \(y^n > x\), a contradiction.
Thus in both cases \(E\) has an upper bound, and so \(x^{1 / n}\) is finite.
\end{proof}

\begin{lemma}\label{5.6.6}
Let \(x, y \geq 0\) be non-negative reals, and let \(n, m \geq 1\) be positive integers.
\begin{enumerate}
    \item If \(y = x^{1 / n}\), then \(y^n = x\).
    \item Conversely, if \(y^n = x\), then \(y = x^{1 / n}\).
    \item \(x^{1 / n}\) is a non-negative real number, and is positive if and only if \(x\) is positive.
    \item We have \(x > y\) if and only if \(x^{1 / n} > y^{1 / n}\).
    \item Let \(k, l \in \mathds{N}\) and \(k, l > 0\).
    If \(x > 1\), then \(x^{1 / k}\) is a decreasing (i.e., \(x^{1 / k} > x^{1 / l}\) whenever \(k < l\)) function of \(k\).
    If \(0 < x < 1\), then \(x^{1 / k}\) is an increasing (i.e., \(x^{1 / k} < x^{1 / l}\) whenever \(k < l\)) function of \(k\).
    If \(x = 1\), then \(x^{1 / k} = 1\) for all \(k\).
    \item We have \((xy)^{1 / n} = x^{1 / n} y^{1 / n}\).
    \item We have \((x^{1 / n})^{1 / m} = x^{1 / nm}\).
\end{enumerate}
\end{lemma}

\begin{proof}{(a)}
Let \(E = \{z \in \mathds{R} : (z \geq 0) \land (z^n \leq x)\}\).
So \(y = x^{1 / n} = \sup(E)\).
Suppose for sake of contradiction that \(y^n \neq x\).
Then by Proposition \ref{5.4.7}, exactly one of the following statements is true:
\begin{enumerate}[label=(\Roman*)]
    \item \(y^n < x\).
    Now we want to show that \(\exists\ \varepsilon \in \mathds{R}\) and \(\varepsilon > 0\) such that \((y + \varepsilon)^n < x\).
    Because \(y < y + \varepsilon\), so we have \(y^n < (y + \varepsilon)^n\).
    Let \(\delta = (y + \varepsilon)^n - y^n\), then \(\delta > 0\).
    By Corollary \ref{5.4.13}, we can find an \(N \in \mathds{N}\) and \(N > 0\) such that \(\delta < 1 \times N\).
    By Proposition \ref{5.4.14}, \(\exists\ q \in \mathds{Q}\) such that \(\delta < q < N\), which means \(\delta / q < 1\), and we have
    \begin{align*}
        (y + \varepsilon)^n &= y^n + \delta \\
        &= y^n + q \delta / q & (q \neq 0) \\
        &< y^n + q. & (\delta / q < 1)
    \end{align*}
    This means if we can show that \(\exists\ q \in \mathds{Q}\) and \(q > 0\) such that \(y^n + q < x\), then we can show that \(\exists\ \varepsilon \in \mathds{R}\) and \(\varepsilon > 0\) such that \((y + \varepsilon)^n < x\).
    We can show such \(q\) exists because by Proposition \ref{5.4.14} \(\exists\ q \in \mathds{Q}\) and \(0 < q < x - y^n\).
    So we must have \(\varepsilon \in \mathds{R}\) and \(\varepsilon > 0\) such that \((y + \varepsilon)^n < x\).
    But this means \(y + \varepsilon \in E\) and \(y + \varepsilon \leq y\), a contradiction.
    \item \(y^n > x\).
    Now we want to show that \(\exists\ \varepsilon \in \mathds{R}\) and \(\varepsilon > 0\) such that \((y - \varepsilon)^n > x\).
    Because \(y > y - \varepsilon\), so we have \(y^n > (y - \varepsilon)^n\).
    Let \(\delta = y^n - (y - \varepsilon)^n\), then \(\delta > 0\).
    By Proposition \ref{5.4.13}, we can find an \(q \in \mathds{Q}\) and \(q > 0\) such that \(q < 2q \leq \delta\).
    Then we have \(\delta / q > 1\) and
    \begin{align*}
        (y - \varepsilon)^n &= y^n - \delta \\
        &= y^n - q \delta / q & (q \neq 0) \\
        &> y^n - q. & (\delta / q > 1)
    \end{align*}
    This means if we can show that \(\exists\ q \in \mathds{Q}\) and \(q > 0\) such that \(y^n - q > x\), then we can show that \(\exists\ \varepsilon \in \mathds{R}\) and \(\varepsilon > 0\) such that \((y - \varepsilon)^n > x\).
    We can show such \(q\) exists because by Proposition \ref{5.4.14} \(\exists\ q \in \mathds{Q}\) and \(0 < q < y^n - x\).
    So we must have \(\varepsilon \in \mathds{R}\) and \(\varepsilon > 0\) such that \((y - \varepsilon)^n > x\).
    But this means \(y - \varepsilon\) is an upper bound of \(E\) and \(y - \varepsilon < y = \sup(E)\), a contradiction.
\end{enumerate}
From all cases above we get contradictions, so \(y = x^{1 / n} \implies y^n = x\).
\end{proof}

\begin{proof}{(b)}
Let \(E = \{z \in \mathds{R} : (z \geq 0) \land (z^n \leq x)\}\).
So \(x^{1 / n} = \sup(E)\).
Since \(y > 0\) and \(y^n = x\), we have \(y \in E\).
By Definition \ref{5.5.5}, \(y \leq \sup(E) = x^{1 / n}\).
Suppose for sake of contradiction that \(y \neq x^{1 / n}\).
Then we have \(y < x^{1 / n}\).
By Exercise \ref{ex 5.5.5}, \(\exists\ \varepsilon \in \mathds{R}\) such that \(y < \varepsilon < x^{1 / n}\).
So we have \(y^n < \varepsilon^n\), which means \(\varepsilon \notin E\).
But this means \(\varepsilon > x^{1 / n}\) because \(0 < y < \varepsilon\) and \(x < \varepsilon^n\), contradict to \(\varepsilon < x^{1 / n}\).
So \(y = x^{1 / n}\).
\end{proof}

\begin{proof}{(c)}
Let \(E = \{z \in \mathds{R} : (z \geq 0) \land (z^n \leq x)\}\).
So \(x^{1 / n} = \sup(E)\).
Because \(0 \in E\), we have \(0 \leq x^{1 / n}\), so \(x^{1 / n}\) is non-negative real number.

If \(x^{1 / n}\) is positive, then we have
\begin{align*}
& x^{1 / n} > 0 \\
\implies & (x^{1 / n})^n > 0^n = 0 & \text{by Proposition \ref{5.4.7}} \\
\implies & x > 0. & \text{by Lemma \ref{5.6.6}(a)} \\
\end{align*}
Now we want to show that if \(x\) is positive, then \(x^{1 / n} > 0\).
Suppose for sake of contradiction that \(x^{1 / n} = 0\).
Then by Lemma \ref{5.6.6}(a), \(x = (x^{1 / n})^n = 0^n = 0\), a contradiction.
Thus \(x^{1 / n} > 0\).
We conclude that \(x^{1 / n}\) is positive iff \(x\) is positive.
\end{proof}

\begin{proof}{(d)}
We first show that \(x^{1 / n} > y^{1 / n} \implies x > y\).
\begin{align*}
& x^{1 / n} > y^{1 / n} \\
\implies & (x^{1 / n})^n > (y^{1 / n})^n & \text{(by Proposition \ref{5.4.7})} \\
\implies & x > y. & \text{(by Lemma \ref{5.6.6}(a))}
\end{align*}

Now we show that \(x > y \implies x^{1 / n} > y^{1 / n}\).
Suppose for sake of contradiction that \(x^{1 / n} \not> y^{1 / n}\).
Then by Proposition \ref{5.4.7}, exactly one of the following two statements is true:
\begin{enumerate}[label=(\Roman*)]
    \item \(x^{1 / n} = y^{1 / n}\).
    But by Lemma \ref{5.6.6}(a), this means \(x = (x^{1 / n})^n = (y^{1 / n})^n = y\), a contradiction.
    \item \(x^{1 / n} < y^{1 / n}\).
    By Proposition \ref{5.4.7}, we have \((x^{1 / n})^n < (y^{1 / n})^n \).
    But by Lemma \ref{5.6.6}(a), this means \(x = (x^{1 / n})^n < (y^{1 / n})^n = y\), a contradiction.
\end{enumerate}
For all cases above we get contradictions.
Thus \(x > y \implies x^{1 / n} > y^{1 / n}\).
We conclude that \(x > y \iff x^{1 / n} > y^{1 / n}\).
\end{proof}

\begin{proof}{(e)}
We first prove that if \(x > 1\), then \(x^{1 / k}\) is a decreasing function of \(k\).
Let \(f : \{k \in \mathds{N} : k > 0\} \to \mathds{R}\) be a function such that \(f(k) = x^{1 / k}\).
Such function \(f\) exists because of Definition \ref{5.6.4}.
Let \(k \in \{k \in \mathds{N} : k > 0\}\).
Now we want to show that \(x^{1 / k} > x^{1 / (k + 1)}\).
Suppose for sake of contradiction that \(x^{1 / k} \not> x^{1 / (k + 1)}\).
Then by Proposition \ref{5.4.7}, exactly one of the following two statements is true:
\begin{enumerate}[label=(\Roman*)]
    \item \(x^{1 / k} = x^{1 / (k + 1)}\).
    Then we have
    \begin{align*}
        & x = (x^{1 / k})^k & \text{(by Lemma \ref{5.6.6}(a))} \\
        \implies & x = (x^{1 / (k + 1)})^k \\
        \implies & x = (x^{1 / (k + 1)})^{(k + 1)} & \text{(by Lemma \ref{5.6.6}(a))} \\
        \implies & (x^{1 / (k + 1)})^{(k + 1)} / (x^{1 / (k + 1)})^k = 1 & (x > 1) \\
        \implies & (x^{1 / (k + 1)})^{(k + 1) - k} = 1 & \text{(by Lemma \ref{5.6.3})} \\
        \implies & (x^{1 / (k + 1)})^1 = 1 \\
        \implies & x^{1 / (k + 1)} = 1 & \text{(by Definition \ref{5.6.1})} \\
        \implies & x = (x^{1 / (k + 1)})^{(k + 1)} = 1^{k + 1} = 1. & \text{(by Lemma \ref{5.6.6}(a))}
    \end{align*}
    But this contradict to \(x > 1\).
    \item \(x^{1 / k} < x^{1 / (k + 1)}\).
    Then we have
    \begin{align*}
        & (x^{1 / k})^{(k + 1)} < (x^{1 / (k + 1)})^{(k + 1)} \\
        \implies & (x^{1 / k})^{(k + 1)} < x & \text{(by Lemma \ref{5.6.6}(a))} \\
        \implies & (x^{1 / k})^k (x^{1 / k}) < x & \text{(by Definition \ref{5.6.1})} \\
        \implies & x (x^{1 / k}) < x & \text{(by Lemma \ref{5.6.6}(a))} \\
        \implies & x^{1 / k} < 1 & \text{(by Proposition \ref{5.4.7})} \\
        \implies & (x^{1 / k})^k < 1^k = 1 & \text{(by Proposition \ref{5.4.7})} \\
        \implies & x < 1. & \text{(by Lemma \ref{5.6.6}(a))}
    \end{align*}
    But this contradict to \(x > 1\).
\end{enumerate}
From all cases above we get contradictions.
Thus \(f(k) = x^{1 / k} > x^{1 / (k + 1)} = f(k + 1)\), \(f(k) = x^{1 / k}\) is a decreasing function of \(k\).

Next we prove that if \(x < 1\), then \(x^{1 / k}\) is a increasing function of \(k\).
Let \(f : \{k \in \mathds{N} : k > 0\} \to \mathds{R}\) be a function such that \(f(k) = x^{1 / k}\).
Such function \(f\) exists because of Definition \ref{5.6.4}.
Let \(k \in \{k \in \mathds{N} : k > 0\}\).
Now we want to show that \(x^{1 / k} < x^{1 / (k + 1)}\).
Suppose for sake of contradiction that \(x^{1 / k} \not< x^{1 / (k + 1)}\).
Then by Proposition \ref{5.4.7}, exactly one of the following two statements is true:
\begin{enumerate}[label=(\Roman*)]
    \item \(x^{1 / k} = x^{1 / (k + 1)}\).
    Then we have
    \begin{align*}
        & x = (x^{1 / k})^k & \text{(by Lemma \ref{5.6.6}(a))} \\
        \implies & x = (x^{1 / (k + 1)})^k \\
        \implies & x = (x^{1 / (k + 1)})^{(k + 1)} & \text{(by Lemma \ref{5.6.6}(a))} \\
        \implies & (x^{1 / (k + 1)})^{(k + 1)} / (x^{1 / (k + 1)})^k = 1 & (x > 1) \\
        \implies & (x^{1 / (k + 1)})^{(k + 1) - k} = 1 & \text{(by Lemma \ref{5.6.3})} \\
        \implies & (x^{1 / (k + 1)})^1 = 1 \\
        \implies & x^{1 / (k + 1)} = 1 & \text{(by Definition \ref{5.6.1})} \\
        \implies & x = (x^{1 / (k + 1)})^{(k + 1)} = 1^{k + 1} = 1. & \text{(by Lemma \ref{5.6.6}(a))}
    \end{align*}
    But this contradict to \(x < 1\).
    \item \(x^{1 / k} > x^{1 / (k + 1)}\).
    Then we have
    \begin{align*}
        & (x^{1 / k})^{(k + 1)} > (x^{1 / (k + 1)})^{(k + 1)} \\
        \implies & (x^{1 / k})^{(k + 1)} > x & \text{(by Lemma \ref{5.6.6}(a))} \\
        \implies & (x^{1 / k})^k (x^{1 / k}) > x & \text{(by Definition \ref{5.6.1})} \\
        \implies & x (x^{1 / k}) > x & \text{(by Lemma \ref{5.6.6}(a))} \\
        \implies & x^{1 / k} > 1 & \text{(by Proposition \ref{5.4.7})} \\
        \implies & (x^{1 / k})^k > 1^k = 1 & \text{(by Proposition \ref{5.4.7})} \\
        \implies & x > 1. & \text{(by Lemma \ref{5.6.6}(a))}
    \end{align*}
    But this contradict to \(x < 1\).
\end{enumerate}
From all cases above we get contradictions.
Thus \(f(k) = x^{1 / k} < x^{1 / (k + 1)} = f(k + 1)\), \(f(k) = x^{1 / k}\) is a increasing function of \(k\).

Finally we prove that if \(x = 1\), then \(x^{1 / k} = 1 \ \forall\ k \in \mathds{N}\) and \(k > 0\).
Suppose for sake of contradiction that \(x^{1 / k} \neq 1\).
Then by Proposition \ref{5.4.7}, exactly one of the following two statements is true:
\begin{enumerate}[label=(\Roman*)]
    \item \(x^{1 / k} > 1\).
    Then we have
    \begin{align*}
        & (x^{1 / k})^k > 1^k = 1 & \text{(by Proposition \ref{5.4.7})} \\
        \implies & x > 1. & \text{(by Lemma \ref{5.6.6}(a))}
    \end{align*}
    But this contradict to \(x = 1\).
    \item \(x^{1 / k} < 1\).
    Then we have
    \begin{align*}
        & (x^{1 / k})^k < 1^k = 1 & \text{(by Proposition \ref{5.4.7})} \\
        \implies & x < 1. & \text{(by Lemma \ref{5.6.6}(a))}
    \end{align*}
    But this contradict to \(x = 1\).
\end{enumerate}
From all cases above we get contradictions.
Thus if \(x = 1\), then \(x^{1 / k} = 1 \ \forall\ k \in \mathds{N}\) and \(k > 0\).
\end{proof}

\begin{proof}{(f)}
\begin{align*}
((xy)^{1 / n})^n &= xy & \text{(by Lemma \ref{5.6.6}(a))} \\
&= (x^{1 / n})^n (y^{1 / n})^n & \text{(by Lemma \ref{5.6.6}(a))} \\
&= (x^{1 / n} y^{1 / n})^n & \text{(by Proposition \ref{5.6.3})} \\
\implies (xy)^{1 / n} &= x^{1 / n} y^{1 / n}. & \text{(by Lemma \ref{5.6.6}(a))}
\end{align*}
\end{proof}

\begin{proof}{(g)}
\begin{align*}
(x^{1 / nm})^{nm} &= x & \text{(by Lemma \ref{5.6.6}(a))} \\
&= (x^{1 / n})^n & \text{(by Lemma \ref{5.6.6}(a))} \\
&= (((x^{1 / n})^{1 / m})^m)^n & \text{(by Lemma \ref{5.6.6}(a))} \\
&= ((x^{1 / n})^{1 / m})^{mn} & \text{(by Proposition \ref{5.6.3})} \\
\implies x^{1 / nm} &= (x^{1 / n})^{1 / m}. & \text{(by Lemma \ref{5.6.6}(a))}
\end{align*}
\end{proof}

\begin{note}
The observant reader may note that this definition of \(x^{1 / n}\) might possibly be inconsistent with our previous notion of \(x^n\) when \(n = 1\), but it is easy to check that \(x^{1 / 1} = x = x^1\) by using Lemma \ref{5.6.6}, so there is no inconsistency.
\end{note}

\begin{note}
One consequence of Lemma \ref{5.6.6}(b) is the following cancellation law:
if \(y\) and \(z\) are positive and \(y^n = z^n\), then \(y = z\).
This only works when \(y\) and \(z\) are positive;
for instance, \((-3)^2 = 3^2\), but we cannot conclude from this that \(-3 = 3\).
\end{note}

\begin{definition}\label{5.6.7}
Let \(x > 0\) be a positive real number, and let \(q\) be a rational number.
To define \(x^q\), we write \(q = a / b\) for some integer \(a\) and positive integer \(b\), and define
\[
    x^q \coloneqq (x^{1 / b})^a.
\]
\end{definition}

\begin{note}
Every rational \(q\), whether positive, negative, or zero, can be written in the form \(a / b\) where \(a\) is an integer and \(b\) is positive.
However, the rational number \(q\) can be expressed in the form \(a / b\) in more than one way, for instance \(1 / 2\) can also be expressed as \(2 / 4\) or \(3 / 6\).
So to ensure that Definition \ref{5.6.7} is well-defined, we need to check that different expressions \(a / b\) give the same formula for \(x^q\).
\end{note}

\begin{lemma}\label{5.6.8}
Let \(a, a'\) be integers and \(b, b'\) be positive integers such that \(a / b = a' / b'\), and let \(x\) be a positive real number.
Then we have \((x^{1 / b'})^{a'} = (x^{1 / b})^a\).
\end{lemma}

\begin{proof}
There are three cases: \(a = 0, a > 0, a < 0\).
If \(a = 0\), then we must have \(a' = 0\) and so both \((x^{1 / b'})^{a'}\) and \((x^{1 / b})^a\) are equal to 1, so we are done.

Now suppose that \(a > 0\).
Then \(a' > 0\), and \(ab' = ba'\).
Write \(y \coloneqq x^{1 / (ab')} = x^{1 / (ba')}\).
By Lemma \ref{5.6.6}(g) we have \(y = (x^{1 / b'})^{1 / a}\) and \(y = (x^{1 / b})^{1 / a'}\);
by Lemma \ref{5.6.6}(a) we thus have \(y^{a'} = x^{1 / b}\) and \(y^a = x^{1 / b'}\).
Thus we have
\[
    (x^{1 / b'})^{a'} = (y^a)^{a'} = y^{aa'} = (y^{a'})^a = (x^{1 / b})^a
\]
as desired.

Finally, suppose that \(a < 0\).
Then we have \((-a) / b = (-a') / b'\).
But \(-a\) is positive, so the previous case applies and we have \((x^{1 / b'})^{-a'} = (x^{1 / b})^{-a}\).
Taking the reciprocal of both sides we obtain the result.
\end{proof}

\begin{note}
Thus \(x^q\) is well-defined for every rational \(q\).
Definition \ref{5.6.7} is consistent with our old definition for \(x^{1 / n}\) (\(= (x^{1 / n})^1\)) and is also consistent with our old definition for \(x^n\) (\(= (x^{1 / 1})^n\)).
\end{note}

\begin{lemma}\label{5.6.9}
Let \(x, y > 0\) be positive reals, and let \(q, r\) be rationals.
\begin{enumerate}
    \item \(x^q\) is a positive real.
    \item \(x^{q + r} = x^q x^r\) and \((x^q)^r = x^{qr}\).
    \item \(x^{-q} = 1 / x^q\).
    \item If \(q > 0\), then \(x > y\) if and only if \(x^q > y^q\).
    \item If \(x > 1\), then \(x^q > x^r\) if and only if \(q > r\).
    If \(x < 1\), then \(x^q > x^r\) if and only if \(q < r\).
\end{enumerate}
\end{lemma}

\begin{proof}{(a)}
Let \(q = a / b\), where \(a, b \in \mathds{Z}\) and \(b > 0\).
By Definition \ref{5.6.7}, we have \(x^q = (x^{1 / b})^a\).
Since \(x\) is a positive real, by Lemma \ref{5.6.6}, \(x^{1 / b}\) is also a positive real.
So by Proposition \ref{5.6.3}, \((x^{1 / b})^a\) is also a positive real.
Thus \(x^q\) is a positive real.
\end{proof}

\begin{proof}{(b)}
We first show that \(x^{q + r} = x^q x^r\).
Let \(q = a / b\) and \(r = c / d\), where \(a, b, c, d \in \mathds{Z}\) and \(b, d > 0\).
Then we have
\begin{align*}
x^{q + r} &= x^{a / b + c / d} \\
&= x^{(ad + bc) / bd} & \text{(by Definition \ref{4.2.2})} \\
&= (x^{1 / bd})^{(ad + bc)} & \text{(by Definition \ref{5.6.7})} \\
&= (x^{1 / bd})^{ad} (x^{1 / bd})^{bc} & \text{(by Proposition \ref{5.6.3})} \\
&= x^{ad / bd} x^{bc / bd} & \text{(by  Definition \ref{5.6.7})} \\
&= x^{a / b} x^{c / d} & \text{(by Lemma \ref{5.6.8})} \\
&= x^q x^r.
\end{align*}

Now we shot that \((x^q)^r = x^{qr}\).
Let \(q = a / b\) and \(r = c / d\), where \(a, b, c, d \in \mathds{Z}\) and \(b, d > 0\).
Then we have
\begin{align*}
x^{qr} &= x^{a / b \times c / d} \\
&= x^{ac / bd} & \text{(by Definition \ref{4.2.2})} \\
&= (x^{1 / bd})^{ac} & \text{(by Definition \ref{5.6.7})} \\
&= ((x^{1 / b})^{1 / d})^{ac} & \text{(by Lemma \ref{5.6.6})} \\
&= ((((x^{1 / b})^a)^{1 / a})^{1 / d})^{ac} & \text{(by Lemma \ref{5.6.6})} \\
&= (((x^{a / b})^{1 / a})^{1 / d})^{ac} & \text{(by Definition \ref{5.6.7})} \\
&= ((x^{a / b})^{1 / ad})^{ac} & \text{(by Lemma \ref{5.6.6})} \\
&= (x^{a / b})^{ac / ad} & \text{(by Definition \ref{5.6.7})} \\
&= (x^{a / b})^{c / d} & \text{(by Lemma \ref{5.6.8})} \\
&= (x^q)^r.
\end{align*}
\end{proof}

\begin{proof}{(c)}
Let \(q = a / b\) where \(a, b \in \mathds{Z}\) and \(b > 0\).
Then we have
\begin{align*}
x^{-q} &= x^{-a / b} \\
&= (x^{1 / b})^{-a} & \text{(by Definition \ref{5.6.7})} \\
&= 1 / (x^{1 / b})^a & \text{(by Proposition \ref{5.6.3})} \\
&= 1 / x^{a / b} & \text{(by Definition \ref{5.6.7})} \\
&= 1 / x^q.
\end{align*}
\end{proof}

\begin{proof}{(d)}

\end{proof}

\begin{proof}{(e)}

\end{proof}

\exercisesection

\begin{exercise}\label{ex 5.6.1}
Prove Lemma \ref{5.6.6}.
\end{exercise}

\begin{proof}
See Lemma \ref{5.6.6}.
\end{proof}

\begin{exercise}\label{ex 5.6.2}
Prove Lemma \ref{5.6.9}.
\end{exercise}

\begin{proof}
See Lemma \ref{5.6.9}.
\end{proof}
