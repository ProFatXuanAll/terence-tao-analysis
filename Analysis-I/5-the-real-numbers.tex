\chapter{The real numbers}\label{i:ch:5}

\begin{note}
  To review our progress to date, we have rigorously constructed three fundamental number systems:
  the natural number system \(\N\), the integers \(\Z\), and the rationals \(\Q\).
  We defined the natural numbers using the five Peano axioms (\crefrange{i:2.1}{i:2.5}), and postulated that such a number system existed;
  this is plausible, since the natural numbers correspond to the very intuitive and fundamental notion of \emph{sequential counting}.
  Using that number system one could then recursively define addition and multiplication, and verify that they obeyed the usual laws of algebra.
  We then constructed the integers by taking formal differences of the natural numbers, \(a \mm b\).
  We then constructed the rationals by taking formal quotients of the integers, \(a // b\), although we need to exclude division by zero in order to keep the laws of algebra reasonable.
  (You are of course free to design your own number system, possibly including one where division by zero is permitted;
  but you will have to give up one or more of the field axioms from \cref{i:4.2.4}, among other things, and you will probably get a less useful number system in which to do any real-world problems.)
\end{note}

\begin{note}
  The symbols \(\N\), \(\Q\), and \(\R\) stand for ``natural'', ``quotient'', and ``real'' respectively.
  \(\Z\) stands for ``Zahlen'', the German word for numbers.
  There is also the \emph{complex numbers} \(\C\), which obviously stands for ``complex.''
\end{note}

\begin{note}
  \emph{Formal} means ``having the form of'';
  at the beginning of our construction the expression \(a \mm b\) did not actually \emph{mean} the difference \(a - b\), since the symbol \(\mm\) was meaningless.
  It only had the \emph{form} of a difference.
  Later on we defined subtraction and verified that the formal difference was equal to the actual difference, so this eventually became a non-issue, and our symbol for formal differencing was discarded.
  Somewhat confusingly, this use of the term ``formal'' is unrelated to the notions of a formal argument and an informal argument.
\end{note}

\begin{note}
  There is a fundamental area of mathematics where the rational number system does not suffice - that of \emph{geometry}
  (the study of lengths, areas, etc.).
  For instance, a right-angled triangle with both sides equal to \(1\) gives a hypotenuse of \(\sqrt{2}\), which is an \emph{irrational} number, i.e., not a rational number;
  see \cref{i:4.4.4}.
  Things get even worse when one starts to deal with the sub-field of geometry known as \emph{trigonometry}, when one sees numbers such as \(\pi\) or \(\cos(1)\), which turn out to be in some sense ``even more'' irrational than \(\sqrt{2}\).
  (These numbers are known as \emph{transcendental numbers}, but to discuss this further would be far beyond the scope of this text.)
  Thus, in order to have a number system which can adequately describe geometry
  - or even something as simple as measuring lengths on a line
  - one needs to replace the rational number system with the real number system.
  Since differential and integral calculus is also intimately tied up with geometry
  - think of slopes of tangents, or areas under a curve
  - calculus also requires the real number system in order to function properly.
\end{note}

\begin{note}
  In the constructions of integers and rationals, the task was to introduce one more \emph{algebraic} operation to the number system
  - e.g., one can get integers from naturals by introducing subtraction, and get the rationals from the integers by introducing division.
  But to get the reals from the rationals is to pass from a ``discrete'' system to a ``continuous'' one, and requires the introduction of a somewhat different notion
  - that of a \emph{limit}.
\end{note}

\begin{note}
  The limit is a concept which on one level is quite intuitive, but to pin down rigorously turns out to be quite difficult.
  (Even such great mathematicians as Euler and Newton had difficulty with this concept.
  It was only in the nineteenth century that mathematicians such as Cauchy and Dedekind figured out how to deal with limits rigorously.)
\end{note}

\begin{note}
  The real number system will end up being a lot like the rational numbers, but will have some new operations
  - notably that of \emph{supremum}, which can then be used to define limits and thence to everything else that calculus needs.
\end{note}

\begin{note}
  The procedure we give here of obtaining the real numbers as the limit of sequences of rational numbers may seem rather complicated.
  However, it is in fact an instance of a very general and useful procedure, that of \emph{completing} one metric space to form another.
  see \cref{ii:ex:1.4.8}.
\end{note}

\section{Cauchy sequences}\label{i:sec:5.1}

\begin{defn}[Sequences]\label{i:5.1.1}
  Let \(m\) be an integer.
  A \emph{sequence \((a_n)_{n = m}^{\infty}\) of rational numbers} is any function from the set \(\Z_{\geq m}\) to \(\Q\), i.e., a mapping which assigns to each integer \(n\) greater than or equal to \(m\), a rational number \(a_n\).
  More informally, a sequence \((a_n)_{n = m}^{\infty}\) of rational numbers is a collection of rationals \(a_m, a_{m + 1}, a_{m + 2}, \dots\).
\end{defn}

\setcounter{thm}{2}
\begin{defn}[\(\varepsilon\)-steadiness]\label{i:5.1.3}
  Let \(\varepsilon \in \Q^+\).
  A sequence \((a_n)_{n = m}^{\infty}\) is said to be \emph{\(\varepsilon\)-steady} iff each pair \(a_j, a_k\) of sequence elements is \(\varepsilon\)-close for every natural number \(j, k \in \Z_{\geq m}\).
  In other words, the sequence \(a_m, a_{m + 1}, a_{m + 2}, \dots\) is \(\varepsilon\)-steady iff \(d(a_j, a_k) \leq \varepsilon\) for all \(j, k \in \Z_{\geq m}\).
\end{defn}

\begin{rmk}\label{i:5.1.4}
  \cref{i:5.1.3} is not standard in the literature;
  we will not need it outside of this section;
  similarly for the concept of ``eventual \(\varepsilon\)-steadiness'' below.
  We have defined \(\varepsilon\)-steadiness for sequences whose index starts at \(m\), but clearly we can make a similar notion for sequences whose indices start from any other number:
  a sequence \(a_N, a_{N + 1}, \dots\) is \(\varepsilon\)-steady if one has \(d(a_j, a_k) \leq \varepsilon\) for all \(j, k \in \Z_{\geq N}\).
\end{rmk}

\begin{note}
  The notion of \(\varepsilon\)-steadiness of a sequence is simple, but does not really capture the \emph{limiting} behavior of a sequence, because it is too sensitive to the initial members of the sequence.
  So we need a more robust notion of steadiness that does not care about the initial members of a sequence.
\end{note}

\setcounter{thm}{5}
\begin{defn}[Eventual \(\varepsilon\)-steadiness]\label{i:5.1.6}
  Let \(\varepsilon \in \Q^+\).
  A sequence \((a_n)_{n = m}^{\infty}\) is said to be \emph{eventually \(\varepsilon\)-steady} iff the sequence \(a_N, a_{N + 1}, a_{N + 2}, \dots\) is \(\varepsilon\)-steady for some integer \(N \geq m\).
  In other words, the sequence \(a_m, a_{m + 1}, a_{m + 2}, \dots\) is eventually \(\varepsilon\)-steady iff there exists an \(N \in \Z_{\geq m}\) such that \(\abs{a_j - a_k} \leq \varepsilon\) for all \(j, k \in \Z_{\geq N}\).
\end{defn}

\setcounter{thm}{7}
\begin{defn}[Cauchy sequences]\label{i:5.1.8}
  A sequence \((a_n)_{n = m}^{\infty}\) of rational numbers is said to be a \emph{Cauchy sequence} iff for every rational \(\varepsilon \in \Q^+\), the sequence \((a_n)_{n = m}^{\infty}\) is eventually \(\varepsilon\)-steady.
  In other words, the sequence \(a_m, a_{m + 1}, a_{m + 2}, \dots\) is a Cauchy sequence iff for every \(\varepsilon \in \Q^+\), there exists an \(N \in \Z_{\geq m}\) such that \(\abs{a_j - a_k} \leq \varepsilon\) for all \(j, k \in \Z_{\geq N}\).
\end{defn}

\begin{rmk}\label{i:5.1.9}
  At present, the parameter \(\varepsilon\) is restricted to be a positive rational;
  we cannot take \(\varepsilon\) to be an arbitrary positive real number, because the real numbers have not yet been constructed.
  However, once we do construct the real numbers, we shall see that \cref{i:5.1.8} will not change if we require \(\varepsilon\) to be real instead of rational (\cref{i:6.1.4}).
  In other words, we will eventually prove that a sequence is eventually \(\varepsilon\)-steady for every rational \(\varepsilon \in \Q^+\) iff it is eventually \(\varepsilon\)-steady for every real \(\varepsilon \in \Q^+\).
  This rather subtle distinction between a rational \(\varepsilon\) and a real \(\varepsilon\) turns out not to be very important in the long run, and the reader is advised not to pay too much attention as to what type of number \(\varepsilon\) should be.
\end{rmk}

\setcounter{thm}{10}
\begin{prop}\label{i:5.1.11}
  The sequence \((a_n)_{n = 1}^\infty\) defined by \(a_n \coloneqq 1 / n\) (i.e., the sequence \(1, 1 / 2, 1 / 3, \dots\)) is a Cauchy sequence.
\end{prop}

\begin{proof}[\pf{i:5.1.11}]
  We have to show that for every \(\varepsilon \in \Q^+\), the sequence \(a_1, a_2, \dots\) is eventually \(\varepsilon\)-steady.
  So let \(\varepsilon \in \Q^+\) be arbitrary.
  We now have to find a number \(N \in \Z_{\geq 1}\) such that the sequence \(a_N, a_{N + 1}, \dots\) is \(\varepsilon\)-steady.
  Let us see what this means.
  This means that \(d(a_j, a_k) \leq \varepsilon\) for every \(j, k \in \Z_{\geq N}\), i.e.
  \[
    \abs{\dfrac{1}{j} - \dfrac{1}{k}} \leq \varepsilon \text{ for every } j, k \in \Z_{\geq N}.
  \]
  Now since \(j, k \in \Z_{\geq N}\), by \cref{i:4.3.12}(b) we know that \(0 < 1 / j, 1 / k \leq 1 / N\), so that
  \begin{align*}
             & \begin{dcases}
                 \dfrac{-1}{N} < 0 < \dfrac{1}{j} \leq \dfrac{1}{N}  \\
                 \dfrac{-1}{N} \leq \dfrac{-1}{k} < 0 < \dfrac{1}{N} \\
               \end{dcases}                         &  & \by{i:ex:4.2.6}                                           \\
    \implies & \begin{dcases}
                 \dfrac{1}{j} - \dfrac{1}{k} \leq \dfrac{1}{N} - \dfrac{1}{k} < \dfrac{1}{N} \\
                 \dfrac{-1}{N} < \dfrac{1}{j} - \dfrac{1}{N} \leq \dfrac{1}{j} - \dfrac{1}{k}
               \end{dcases} &  & \by{i:4.2.9}[c,d]                         \\
    \implies & \dfrac{-1}{N} \leq \dfrac{1}{j} - \dfrac{1}{k} \leq \dfrac{1}{N}                                    \\
    \implies & \abs{\dfrac{1}{j} - \dfrac{1}{k}} \leq \dfrac{1}{N}.                           &  & \by{i:4.3.3}[c]
  \end{align*}
  So in order to force \(\abs{1 / j - 1 / k}\) to be less than or equal to \(\varepsilon\), it would be sufficient for \(1 / N\) to be less than \(\varepsilon\).
  So all we need to do is choose an \(N\) such that \(1 / N\) is less than \(\varepsilon\), or in other words that \(N\) is greater than \(1 / \varepsilon\).
  But this can be done thanks to \cref{i:4.4.1}.
\end{proof}

\begin{note}
  As you can see, verifying from first principles (i.e., without using any of the machinery of limits, etc.) that a sequence is a Cauchy sequence requires some effort, even for a sequence as simple as \(1 / n\).
  The part about selecting an \(N\) can be particularly difficult for beginners
  - one has to think in reverse, working out what conditions on \(N\) would suffice to force the sequence \(a_N, a_{N + 1}, a_{N + 2}, \dots\) to be \(\varepsilon\)-steady, and then finding an \(N\) which obeys those conditions.
  Later we will develop some limit laws which allow us to determine when a sequence is Cauchy more easily.
\end{note}

\begin{defn}[Bounded sequences]\label{i:5.1.12}
  Let \(M \in \Q_{\geq 0}\).
  A finite rational sequence \((a_n)_{n = m}^k\) is \emph{bounded by \(M\)} iff \(\abs{a_i} \leq M\) for all \(i \in \Z_{m \leq k}\).
  An infinite rational sequence \((a_n)_{n = m}^{\infty}\) is \emph{bounded by \(M\)} iff \(\abs{a_i} \leq M\) for all \(i \in \Z_{\geq m}\).
  A rational sequence is said to be \emph{bounded} iff it is bounded by \(M\) for some \(M \in \Q_{\geq 0}\).
\end{defn}

\setcounter{thm}{13}
\begin{lem}[Finite sequences are bounded]\label{i:5.1.14}
  Every finite rational sequence \((a_n)_{n = m}^k\) is bounded.
\end{lem}

\begin{proof}[\pf{i:5.1.14}]
  We induct on \(k\) and we start with \(k = m\).
  When \(k = m\) the rational sequence \((a_n)_{n = m}^m\) is clearly bounded, for if we choose \(M \coloneqq \abs{a_m}\) then clearly we have \(\abs{a_i} \leq M\) for all \(i \in \Z_{m \leq k}\).
  Now suppose that we have already proved the lemma for some \(k \geq m\);
  we now prove it for \(k + 1\), i.e., we prove every rational sequence \((a_n)_{n = m}^{k + 1}\) is bounded.
  By the induction hypothesis we know that \((a_n)_{n = m}^k\) is bounded by some \(M \in \Q_{\geq 0}\);
  in particular, it must be bounded by \(M + \abs{a_{k + 1}}\).
  On the other hand, \(a_{k + 1}\) is also bounded by \(M + \abs{a_{k + 1}}\).
  Thus, \((a_n)_{n = m}^{k + 1}\) is bounded by \(M + \abs{a_{k + 1}}\), and is hence bounded.
  This closes the induction.
\end{proof}

\begin{note}
  While \cref{i:5.1.14} shows that every finite rational sequence is bounded, no matter how long the finite sequence is, it does not say anything about whether an infinite rational sequence is bounded or not;
  infinity is not a natural number.
\end{note}

\begin{lem}[Cauchy sequences are bounded]\label{i:5.1.15}
  Every rational Cauchy sequence \((a_n)_{n = m}^{\infty}\) is bounded.
\end{lem}

\begin{proof}[\pf{i:5.1.15}]
  Since \((a_n)_{n = m}^{\infty}\) is a rational Cauchy sequence, by \cref{i:5.1.8} we know that \((a_n)_{n = m}^{\infty}\) is eventually \(\varepsilon\)-steady for all \(\varepsilon \in \Q^+\).
  In particular, \((a_n)_{n = m}^{\infty}\) is eventually \(1\)-steady.
  By \cref{i:5.1.6} there exists an \(N \in \Z_{\geq m}\) such that \((a_n)_{n = N}^{\infty}\) is \(1\)-steady.
  Fix such \(N\).
  Since \((a_n)_{n = N}^\infty\) is \(1\)-steady, we have
  \begin{align*}
             & \forall j \in \Z_{\geq N}, \abs{a_j - a_N} \leq 1                                                    &  & \by{i:5.1.3}    \\
    \implies & \forall j \in \Z_{\geq N}, \abs{a_j - a_N} + \abs{a_N} \leq 1 + \abs{a_N}                            &  & \by{i:4.2.9}[d] \\
    \implies & \forall j \in \Z_{\geq N}, \abs{a_j - a_N + a_N} \leq \abs{a_j - a_N} + \abs{a_N} \leq 1 + \abs{a_N} &  & \by{i:4.3.3}[b] \\
    \implies & \forall j \in \Z_{\geq N}, \abs{a_j} \leq 1 + \abs{a_N}.                                             &  & \by{i:4.2.4}
  \end{align*}
  Thus, by \cref{i:5.1.12} \((a_n)_{n = N}^\infty\) is bounded by \(1 + \abs{a_N}\).
  Now we split into two cases:
  \begin{itemize}
    \item If \(N = m\), then we see that \((a_n)_{n = m}^\infty\) is bounded by \(1 + \abs{a_N}\).
    \item If \(N \neq m\), then we must have \(m < N\).
          By \cref{i:5.1.14} we know that the finite rational sequence \((a_n)_{n = m}^{N - 1}\) is bounded by some \(M \in \Q_{\geq 0}\).
          So both \((a_n)_{n = m}^{N - 1}\) and \((a_n)_{n = N}^\infty\) are bounded by \(M + 1 + \abs{a_N}\).
          Thus, \((a_n)_{n = m}^\infty\) is bounded by \(M + 1 + \abs{a_N}\).
  \end{itemize}
  From all cases above we see that \((a_n)_{n = m}^\infty\) is bounded.
  Since \((a_n)_{n = m}^\infty\) was arbitrary, we conclude that every rational Cauchy sequences are bounded.
\end{proof}

\exercisesection

\begin{ex}\label{i:ex:5.1.1}
  Prove \cref{i:5.1.15}.
\end{ex}

\begin{proof}[\pf{i:ex:5.1.1}]
  See \cref{i:5.1.15}.
\end{proof}

\section{Equivalent Cauchy sequences}\label{sec:5.2}

\begin{defn}[\(\varepsilon\)-close sequences]\label{5.2.1}
  Let \((a_n)_{n = 0}^{\infty}\) and \((b_n)_{n = 0}^{\infty}\) be two sequences, and let \(\varepsilon > 0\).
  We say that the sequence \((a_n)_{n = 0}^{\infty}\) is \emph{\(\varepsilon\)-close} to \((b_n)_{n = 0}^{\infty}\) iff \(a_n\) is \(\varepsilon\)-close to \(b_n\) for each \(n \in \N\).
  In other words, the sequence \(a_0, a_1, a_2, \dots\) is \(\varepsilon\)-close to the sequence \(b_0, b_1, b_2, \dots\) iff \(\abs{a_n - b_n} \leq \varepsilon\) for all \(n = 0, 1, 2, \dots\).
\end{defn}

\setcounter{thm}{2}
\begin{defn}[\(Eventually \varepsilon\)-close sequences]\label{5.2.3}
  Let \((a_n)_{n = 0}^{\infty}\) and \((b_n)_{n = 0}^{\infty}\) be two sequences, and let \(\varepsilon > 0\).
  We say that the sequence \((a_n)_{n = 0}^{\infty}\) is \emph{eventually \(\varepsilon\)-close} to \((b_n)_{n = 0}^{\infty}\) iff there exists an \(N \geq 0\) such that the sequences \((a_n)_{n = N}^{\infty}\) and \((b_n)_{n = N}^{\infty}\) are \(\varepsilon\)-close.
  In other words, \(a_0, a_1, a_2, \dots\) is eventually \(\varepsilon\)-close to \(b_0, b_1, b_2, \dots\) iff there exists an \(N \geq 0\) such that \(\abs{a_n - b_n} \leq \varepsilon\) for all \(n \geq N\).
\end{defn}

\begin{rmk}\label{5.2.4}
  Again, the notations for \(\varepsilon\)-close sequences and eventually \(\varepsilon\)-close sequences are not standard in the literature, and we will not use them outside of this section.
\end{rmk}

\setcounter{thm}{5}
\begin{defn}[Equivalent sequences]\label{5.2.6}
  Two sequences \((a_n)_{n = 0}^{\infty}\) and \((b_n)_{n = 0}^{\infty}\) are \emph{equivalent} iff for each rational \(\varepsilon > 0\), the sequences \((a_n)_{n = 0}^{\infty}\) and \((b_n)_{n = 0}^{\infty}\) are eventually \(\varepsilon\)-close.
  In other words, \(a_0, a_1, a_2, \dots\) and \(b_0, b_1, b_2, \dots\) are equivalent iff for every rational \(\varepsilon > 0\), there exists an \(N \geq 0\) such that \(\abs{a_n - b_n} \leq \varepsilon\) for all \(n \geq N\).
\end{defn}

\begin{rmk}\label{5.2.7}
  As with \cref{5.1.8}, the quantity \(\varepsilon > 0\) is currently restricted to be a positive rational, rather than a positive real.
  However, we shall eventually see that it makes no difference whether \(\varepsilon\) ranges over the positive rationals or positive reals.
\end{rmk}

\begin{ac}\label{ac:5.2.1}
  Equivalence defined as \cref{5.2.6} is reflexive, symmetric and transitive.
\end{ac}

\begin{proof}
  Let \((a_n)_{n = m}^\infty\), \((b_n)_{n = m}^\infty\), \((c_n)_{n = m}^\infty\) be sequences of rationals.
  We have
  \begin{align*}
             & \forall \varepsilon \in \Q^+, \forall n \geq m, \abs{a_n - a_n} = \abs{0} = 0 \leq \varepsilon                 \\
    \implies & (a_n)_{n = m}^\infty = (a_n)_{n = m}^\infty                                                    &  & \by{5.2.6}
  \end{align*}
  and thus \cref{5.2.6} is reflexive.

  Now suppose that \((a_n)_{n = m}^\infty = (b_n)_{n = m}^\infty\).
  Then we have
  \begin{align*}
             & (a_n)_{n = m}^\infty = (b_n)_{n = m}^\infty                                       \\
    \implies & \forall \varepsilon \in \Q^+, \exists N \in \N \land N \geq m:                    \\
             & \forall n \geq N, \abs{a_n - b_n} \leq \varepsilon             &  & \by{5.2.6}    \\
    \implies & \forall \varepsilon \in \Q^+, \exists N \in \N \land N \geq m:                    \\
             & \forall n \geq N, \abs{b_n - a_n} \leq \varepsilon             &  & \by{4.3.3}[f] \\
    \implies & (b_n)_{n = m}^\infty = (a_n)_{n = m}^\infty                    &  & \by{5.2.6}
  \end{align*}
  and thus \cref{5.2.6} is symmetric.

  Finally suppose that \((a_n)_{n = m}^\infty = (b_n)_{n = m}^\infty\) and \((b_n)_{n = m}^\infty = (c_n)_{n = m}^\infty\).
  Then we have
  \begin{align*}
             & \big((a_n)_{n = m}^\infty = (b_n)_{n = m}^\infty\big) \land \big((b_n)_{n = m}^\infty = (c_n)_{n = m}^\infty\big)                        \\
    \implies & \forall \varepsilon \in \Q^+, \exists N_1, N_2 \in \N \land N_1, N_2 \geq m:                                        &  & \by{5.2.6}      \\
             & \begin{dcases}
                 \abs{a_n - b_n} \leq \dfrac{\varepsilon}{2} & \forall n \geq N_1 \\
                 \abs{b_n - c_n} \leq \dfrac{\varepsilon}{2} & \forall n \geq N_2 \\
               \end{dcases}                                                                         \\
    \implies & \forall \varepsilon \in \Q^+, \exists N = \max(N_1, N_2) \geq m:                                                    &  & \by{2.2.13}     \\
             & \forall n \geq N, (\abs{a_n - b_n} \leq \dfrac{\varepsilon}{2}) \land (\abs{b_n - c_n} \leq \dfrac{\varepsilon}{2})                      \\
    \implies & \forall \varepsilon \in \Q^+, \exists N = \max(N_1, N_2) \geq m:                                                                         \\
             & \forall n \geq N, \abs{a_n - b_n} + \abs{b_n - c_n} \leq \varepsilon                                                &  & \by{4.2.9}[c,d] \\
    \implies & \forall \varepsilon \in \Q^+, \exists N = \max(N_1, N_2) \geq m:                                                                         \\
             & \forall n \geq N, \abs{a_n - c_n} \leq \abs{a_n - b_n} + \abs{b_n - c_n} \leq \varepsilon                           &  & \by{4.3.3}[b]   \\
    \implies & (a_n)_{n = m}^\infty = (c_n)_{n = m}^\infty                                                                         &  & \by{5.2.6}
  \end{align*}
  and thus \cref{5.2.6} is transitive.
\end{proof}

\begin{prop}\label{5.2.8}
  Let \((a_n)_{n = 1}^{\infty}\) and \((b_n)_{n = 1}^{\infty}\) be the sequences \(a_n = 1 + 10^{-n}\) and \(b_n = 1 - 10^{-n}\).
  Then the sequences \(a_n, b_n\) are equivalent.
\end{prop}

\begin{proof}
  We need to prove that for every \(\varepsilon > 0\), the two sequences \((a_n)_{n = 1}^{\infty}\) and \((b_n)_{n = 1}^{\infty}\) are eventually \(\varepsilon\)-close to each other.
  So we fix an \(\varepsilon > 0\).
  We need to find an \(N > 0\) such that \((a_n)_{n = 1}^{\infty}\) and \((b_n)_{n = 1}^{\infty}\) are \(\varepsilon\)-close;
  in other words, we need to find an \(N > 0\) such that
  \[
    \abs{a_n - b_n} \leq \varepsilon \text{ for all } n \geq N.
  \]
  However, we have
  \[
    \abs{a_n - b_n} = \abs{(1 + 10^{-n}) - (1 - 10^{-n})} = 2 \times 10^{-n}.
  \]
  Since \(10^{-n}\) is a decreasing function of \(n\) (i.e., \(10^{-m} < 10^{-n}\) whenever \(m > n\);
  this is easily proven by induction), and \(n \geq N\), we have \(2 \times 10^{-n} \leq 2 \times 10^{-N}\).
  Thus we have
  \[
    \abs{a_n - b_n} \leq 2 \times 10^{-N} \text{ for all } n \geq N.
  \]
  Thus in order to obtain \(\abs{a_n - b_n} \leq \varepsilon\) for all \(n \geq N\), it will be sufficient to choose \(N\) so that \(2 \times 10^{-N} \leq \varepsilon\).
  This is easy to do using logarithms, but we have not yet developed logarithms yet, so we will use a cruder method.
  First, we observe \(10^N\) is always greater than \(N\) for any \(N \geq 1\) (see \cref{ex:4.3.5}).
  Thus \(10^{-N} \leq 1 / N\), and so \(2 \times 10^{-N} \leq 2 / N\).
  Thus to get \(2 \times 10^{-N} \leq \varepsilon\), it will suffice to choose \(N\) so that \(2 / N \leq \varepsilon\), or equivalently that \(N \geq 2 / \varepsilon\).
  But by \cref{4.4.1} we can always choose such an \(N\), and the claim follows.
\end{proof}

\begin{rmk}\label{5.2.9}
  \cref{5.2.8}, in decimal notation, asserts that
  \[
    1.0000 \dots = 0.9999 \dots.
  \]
\end{rmk}

\exercisesection

\begin{ex}\label{ex:5.2.1}
  Show that if \((a_n)_{n = 1}^{\infty}\) and \((b_n)_{n = 1}^{\infty}\) are equivalent sequences of rationals, then \((a_n)_{n = 1}^{\infty}\) is a Cauchy sequence iff \((b_n)_{n = 1}^{\infty}\) is a Cauchy sequence.
\end{ex}

\begin{proof}
  Let \(j, k \in \Z^+\).
  Since \((a_n)_{n = 1}^\infty = (b_n)_{n = 1}^\infty\), by \cref{5.2.6} we have
  \[
    \forall \varepsilon \in \Q^+, \exists N_1 \in \Z^+ : \forall n \geq N_1, \abs{a_n - b_n} \leq \dfrac{\varepsilon}{3}.
  \]
  Then we have
  \begin{align*}
             & (a_n)_{n = 1}^\infty \text{ is a Cauchy sequence}                                                                \\
    \implies & \exists N_2 \in \Z^+ : \forall j, k \geq N,                                                                      \\
             & \abs{a_j - a_k} \leq \dfrac{\varepsilon}{3}                                                 &  & \by{5.1.8}      \\
    \implies & \exists N = \max(N_1, N_2) \in \Z^+ : \forall j, k \geq N,                                  &  & \by{2.2.13}     \\
             & \abs{a_j - a_k} \leq \dfrac{\varepsilon}{3}                                                                      \\
    \implies & \exists N = \max(N_1, N_2) \in \Z^+ : \forall j, k \geq N,                                                       \\
             & \abs{a_j - a_k} + \abs{a_j - b_j} + \abs{a_k - b_k}                                                              \\
             & \leq \dfrac{\varepsilon}{3} + \dfrac{\varepsilon}{3} + \dfrac{\varepsilon}{3} = \varepsilon &  & \by{4.2.9}[c,d] \\
    \implies & \exists N = \max(N_1, N_2) \in \Z^+ : \forall j, k \geq N,                                                       \\
             & \abs{a_j - a_k} + \abs{b_j - a_j} + \abs{a_k - b_k} \leq \varepsilon                        &  & \by{4.3.3}[f]   \\
    \implies & \exists N = \max(N_1, N_2) \in \Z^+ : \forall j, k \geq N,                                                       \\
             & \abs{b_j - b_k} = \abs{a_j - a_k + b_j - a_j + a_k - b_k}                                                        \\
             & \leq \abs{a_j - a_k} + \abs{b_j - a_j} + \abs{a_k - b_k} \leq \varepsilon                   &  & \by{4.3.3}[b]   \\
    \implies & (b_n)_{n = 1}^\infty \text{ is a Cauchy sequence}.                                          &  & \by{5.1.8}
  \end{align*}
  Using similar arguments we can show that \((b_n)_{n = 1}^\infty\) is a Cauchy sequence implies \((a_n)_{n = 1}^\infty\) is a Cauchy sequence.
  Thus we conclude that \((a_n)_{n = 1}^\infty\) is a Cauchy sequence iff \((b_n)_{n = 1}^\infty\) is a Cauchy sequence.
\end{proof}

\begin{ex}\label{ex:5.2.2}
  Let \(\varepsilon > 0\).
  Show that if \((a_n)_{n = 1}^{\infty}\) and \((b_n)_{n = 1}^{\infty}\) are eventually \(\varepsilon\)-close, then \((a_n)_{n = 1}^{\infty}\) is bounded iff \((b_n)_{n = 1}^{\infty}\) is bounded.
\end{ex}

\begin{proof}
  Since \((a_n)_{n = 1}^{\infty}\) and \((b_n)_{n = 1}^{\infty}\) are eventually \(\varepsilon\)-close, by \cref{5.2.3} we have
  \[
    \exists N \in \Z^+ : \forall n \geq N, \abs{a_n - b_n} \leq \varepsilon.
  \]
  Then we have
  \begin{align*}
             & (a_n)_{n = 1}^\infty \text{ is bounded}                                                                          \\
    \implies & \exists M \in \Q \setminus \Q^- : \forall n \geq 1, \abs{a_n} \leq M                        &  & \by{5.1.12}     \\
    \implies & \exists M \in \Q \setminus \Q^- : \forall n \geq \max(1, N), \abs{a_n} \leq M               &  & \by{2.2.13}     \\
    \implies & \exists M \in \Q \setminus \Q^- : \forall n \geq \max(1, N), \abs{-a_n} \leq M              &  & \by{4.3.3}[d]   \\
    \implies & \exists M \in \Q \setminus \Q^- : \forall n \geq \max(1, N),                                                     \\
             & \abs{-a_n} + \abs{a_n - b_n} \leq M + \varepsilon                                           &  & \by{4.2.9}[c,d] \\
    \implies & \exists M \in \Q \setminus \Q^- : \forall n \geq \max(1, N),                                                     \\
             & \abs{-a_n + a_n - b_n} \leq \abs{-a_n} + \abs{a_n - b_n} \leq M + \varepsilon               &  & \by{4.3.3}[b]   \\
    \implies & \exists M \in \Q \setminus \Q^- : \forall n \geq \max(1, N),                                                     \\
             & \abs{-b_n} \leq M + \varepsilon                                                             &  & \by{4.2.4}      \\
    \implies & \exists M \in \Q \setminus \Q^- : \forall n \geq \max(1, N), \abs{b_n} \leq M + \varepsilon &  & \by{4.3.3}[d]   \\
    \implies & \exists M \in \Q \setminus \Q^- : \forall n \geq 1,                                                              \\
             & \abs{b_n} \leq M + \varepsilon + \max_{1 \leq n \leq N - 1}\set{\abs{b_n}}                  &  & \by{5.1.14}     \\
    \implies & (a_n)_{n = 1}^\infty \text{ is bounded}.                                                    &  & \by{5.1.12}
  \end{align*}
  Using similar arguments we can show that \((b_n)_{n = 1}^\infty\) is bounded implies \((a_n)_{n = 1}^\infty\) is bounded.
  Thus we conclude that \((a_n)_{n = 1}^\infty\) is bounded iff \((b_n)_{n = 1}^\infty\) is bounded.
\end{proof}

\section{The construction of the real numbers}\label{sec:5.3}

\begin{defn}[Real numbers]\label{5.3.1}
  A \emph{real number} is defined to be an object of the form \(\text{LIM}_{n \to \infty} a_n\), where \((a_n)_{n = 1}^{\infty}\) is a Cauchy sequence of rational numbers.
  Two real numbers \(\text{LIM}_{n \to \infty} a_n\) an and \(\text{LIM}_{n \to \infty} b_n\) are said to be equal iff \((a_n)_{n = 1}^{\infty}\) and \((b_n)_{n = 1}^{\infty}\) are equivalent Cauchy sequences.
  The set of all real numbers is denoted \(\R\).
\end{defn}

\begin{note}
  We will refer to \(\text{LIM}_{n \to \infty} a_n\) as the \emph{formal limit} of the sequence \((a_n)_{n = 1}^{\infty}\).
  Later on we will define a genuine notion of limit, and show that the formal limit of a Cauchy sequence is the same as the limit of that sequence;
  after that, we will not need formal limits ever again.
\end{note}

\setcounter{thm}{2}
\begin{prop}[Formal limits are well-defined]\label{5.3.3}
  Let
  \[
    x = \text{LIM}_{n \to \infty} a_n, y = \text{LIM}_{n \to \infty} b_n, z = \text{LIM}_{n \to \infty} c_n
  \]
  be real numbers.
  Then, with the above definition of equality for real numbers, we have \(x = x\).
  Also, if \(x = y\), then \(y = x\).
  Finally, if \(x = y\) and \(y = z\), then \(x = z\).
\end{prop}

\begin{proof}
  By \cref{ac:5.2.1} we know that the equality of sequence are well-defined.
  Since every Cauchy sequence is a sequence, we know that the Formal limits are well-defined.
\end{proof}

\begin{note}
  Because of \cref{5.3.3}, we know that our definition of equality between two real numbers is legitimate.
  Of course, when we define other operations on the reals, we have to check that they obey the axiom of substitution:
  two real number inputs which are equal should give equal outputs when applied to any operation on the real numbers.
\end{note}

\begin{defn}[Addition of reals]\label{5.3.4}
  Let \(x = \text{LIM}_{n \to \infty} a_n\) and \(y = \text{LIM}_{n \to \infty} b_n\) be real numbers.
  Then we define the sum \(x + y\) to be \(x + y \coloneqq \text{LIM}_{n \to \infty} (a_n + b_n)\).
\end{defn}

\setcounter{thm}{5}
\begin{lem}[Sum of Cauchy sequences is Cauchy]\label{5.3.6}
  Let \(x = \text{LIM}_{n \to \infty} a_n\) and \(y = \text{LIM}_{n \to \infty} b_n\) be real numbers.
  Then \(x + y\) is also a real number
  (i.e., \((a_n + b_n)_{n = 1}^{\infty}\) is a Cauchy sequence of rationals).
\end{lem}

\begin{proof}
  We need to show that for every \(\varepsilon > 0\), the sequence \((a_n + b_n)_{n = 1}^{\infty}\) is eventually \(\varepsilon\)-steady.
  Now from hypothesis we know that \((a_n)_{n = 1}^{\infty}\) is eventually \(\varepsilon\)-steady, and \((b_n)_{n = 1}^{\infty}\) is eventually \(\varepsilon\)-steady, but it turns out that this is not quite enough
  (this can be used to imply that \((a_n + b_n)_{n = 1}^{\infty}\) is eventually \(2\varepsilon\)-steady, but that's not what we want).
  So we need to do a little trick, which is to play with the value of \(\varepsilon\).

  We know that \((a_n)_{n = 1}^{\infty}\) is eventually \(\delta\)-steady for every value of \(\delta\).
  This implies not only that \((a_n)_{n = 1}^{\infty}\) is eventually \(\varepsilon\)-steady, but it is also eventually \(\varepsilon / 2\)-steady.
  Similarly, the sequence \((b_n)_{n = 1}^{\infty}\) is also eventually \(\varepsilon / 2\)-steady.
  This will turn out to be enough to conclude that \((a_n + b_n)_{n = 1}^{\infty}\) is eventually \(\varepsilon\)-steady.

  Since \((a_n)_{n = 1}^{\infty}\) is eventually \(\varepsilon / 2\)-steady, we know that there exists an \(N \geq 1\) such that \((a_n)_{n = N}^{\infty}\) is \(\varepsilon / 2\)-steady, i.e., \(a_n\) and \(a_m\) are \(\varepsilon / 2\)-close for every \(n, m \geq N\).
  Similarly there exists an \(M \geq 1\) such that \((b_n)_{n = M}^{\infty}\) is \(\varepsilon / 2\)-steady, i.e., \(b_n\) and \(b_m\) are \(\varepsilon / 2\)-close for every \(n, m \geq M\).

  Let \(\max(N, M)\) be the larger of \(N\) and \(M\)
  (we know from \cref{2.2.13} that one has to be greater than or equal to the other).
  If \(n, m \geq \max(N, M)\), then we know that \(a_n\) and \(a_m\) are \(\varepsilon / 2\)-close, and \(b_n\) and \(b_m\) are \(\varepsilon / 2\)-close, and so by \cref{4.3.7} we see that \(a_n + b_n\) and \(a_m + b_m\) are \(\varepsilon\)-close for every \(n, m \geq \max(N, M)\).
  This implies that the sequence \((a_n + b_n)_{n = 1}^{\infty}\) is eventually \(\varepsilon\)-steady, as desired.
\end{proof}

\begin{lem}[Sums of equivalent Cauchy sequences are equivalent]\label{5.3.7}
  Let
  \[
    x = \text{LIM}_{n \to \infty} a_n, y = \text{LIM}_{n \to \infty} b_n, x' = \text{LIM}_{n \to \infty} a'_n
  \]
  be real numbers.
  Suppose that \(x = x'\).
  Then we have \(x + y = x' + y\).
\end{lem}

\begin{proof}
  Since \(x\) and \(x'\) are equal, we know that the Cauchy sequences \((a_n)_{n = 1}^{\infty}\) and \((a'_n)_{n = 1}^{\infty}\) are equivalent, so in other words they are eventually \(\varepsilon\)-close for each \(\varepsilon > 0\).
  We need to show that the sequences \((a_n + b_n)_{n = 1}^{\infty}\) and \((a'_n + b_n)_{n = 1}^{\infty}\) are eventually \(\varepsilon\)-close for each \(\varepsilon > 0\).
  But we already know that there is an \(N \geq 1\) such that \((a_n)_{n = N}^{\infty}\) and \((a'_n)_{n = N}^{\infty}\) are \(\varepsilon\)-close, i.e., that \(a_n\) and \(a'_n\) are \(\varepsilon\)-close for each \(n \geq N\).
  Since \(b_n\) is of course \(0\)-close to \(b_n\) (where we extend the notion of \(\varepsilon\)-closeness to include \(\varepsilon = 0\) in the obvious fashion), we thus see from \cref{4.3.7} (extended in the obvious manner to the \(\delta = 0\) case) that \(a_n + b_n\) and \(a'_n + b_n\) are \(\varepsilon\)-close for each \(n \geq N\).
  This implies that \((a_n + b_n)_{n = 1}^{\infty}\) and \((a'_n + b_n)_{n = 1}^{\infty}\) are eventually \(\varepsilon\)-close for each \(\varepsilon > 0\), and we are done.
\end{proof}

\begin{rmk}\label{5.3.8}
  \cref{5.3.7} verifies the axiom of substitution for the ``x'' variable in \(x + y\), but one can similarly prove the axiom of substitution for the ``y'' variable.
  (A quick way is to observe from the definition of \(x + y\) that we certainly have \(x + y = y + x\), since \(a_n + b_n = b_n + a_n\).)
\end{rmk}

\begin{defn}[Multiplication of reals]\label{5.3.9}
  Let \(x = \text{LIM}_{n \to \infty} a_n\) and \(y = \text{LIM}_{n \to \infty} b_n\) be real numbers.
  Then we define the product \(xy\) to be \(xy \coloneqq \text{LIM}_{n \to \infty} a_n b_n\).
\end{defn}

\begin{prop}[Multiplication is well defined]\label{5.3.10}
  Let
  \[
    x = \text{LIM}_{n \to \infty} a_n, y = \text{LIM}_{n \to \infty} b_n, x' = \text{LIM}_{n \to \infty} a'_n
  \]
  be real numbers.
  Then \(xy\) is also a real number.
  Furthermore, if \(x = x'\), then \(xy = x'y\).
\end{prop}

\begin{proof}
  We first show that \(x, y \in \R \implies xy \in \R\).
  Let \(\varepsilon \in \Q^+\) and let \(j, k \in \Z^+\).
  Since \((a_n)_{n = 1}^\infty\) and \((b_n)_{n = 1}^\infty\) are Cauchy sequence, by \cref{5.1.15} we know that \((a_n)_{n = 1}^\infty\) and \((b_n)_{n = 1}^\infty\) are bounded by some \(M_1, M_2 \in \Q \setminus \Q^-\).
  Then by \cref{4.2.9}(a) we know that \((a_n)_{n = 1}^\infty\) and \((b_n)_{n = 1}^\infty\) are bounded by \(M = \max(M_1, M_2) + 1\).
  Since \(x = \text{LIM}_{n \to \infty} a_n\) and \(y = \text{LIM}_{n \to \infty} b_n\), by \cref{5.1.8} we have
  \begin{align*}
     & \exists N_1 \in \Z^+ : \forall j, k \geq N_1, \abs{a_j - a_k} \leq \varepsilon; \\
     & \exists N_2 \in \Z^+ : \forall j, k \geq N_2, \abs{b_j - b_k} \leq \varepsilon.
  \end{align*}
  In particular, we have
  \begin{align*}
     & \exists N_1 \in \Z^+ : \forall j, k \geq N_1, \abs{a_j - a_k} \leq \dfrac{\varepsilon}{2M}; \\
     & \exists N_2 \in \Z^+ : \forall j, k \geq N_2, \abs{b_j - b_k} \leq \dfrac{\varepsilon}{2M}.
  \end{align*}
  Let \(N = \max(N_1, N_2)\).
  Such \(N\) is well-defined since \cref{2.2.13}.
  Then \(\forall j, k \geq N\), we have
  \begin{align*}
    \abs{a_j b_j - a_k b_k} & = \abs{a_j b_j - a_j b_k + a_j b_k - a_k b_k}              &  & \by{4.2.4}                     \\
                            & \leq \abs{a_j b_j - a_j b_k} + \abs{a_j b_k - a_k b_k}     &  & \text{(by \cref{4.3.3}(b))}    \\
                            & = \abs{a_j} \abs{b_j - b_k} + \abs{b_k} \abs{a_j - a_k}    &  & \text{(by \cref{4.3.3}(d))}    \\
                            & \leq M \dfrac{\varepsilon}{2M} + M \dfrac{\varepsilon}{2M} &  & \text{(by \cref{4.2.9}(c)(e))} \\
                            & = \varepsilon.
  \end{align*}
  Thus by \cref{5.1.8} \((a_n b_n)_{n = 1}^\infty\) is a Cauchy sequence and by \cref{5.3.1} \(xy \in \R\).

  Now we show that \(x = x' \implies xy = xy'\).
  Since \((a_n)_{n = 1}^\infty\), \((a_n')_{n = 1}^\infty\), \((b_n)_{n = 1}^\infty\) are Cauchy sequence, by \cref{5.1.15} we know that \((a_n)_{n = 1}^\infty\) and \((a_n')_{n = 1}^\infty\) are bounded by some \(M_1, M_2 M_3 \in \Q \setminus \Q^-\).
  Then by \cref{4.2.9}(a) we know that \((a_n)_{n = 1}^\infty\), \((a_n')_{n = 1}^\infty\), \((b_n)_{n = 1}^\infty\) are bounded by \(M = \max(M_1, M_2, M_3) + 1\).
  Since \(x = x'\), by \cref{5.2.6} we know that
  \[
    \forall \varepsilon \in \Q^+, \exists N \in \Z^+ : \forall n \geq N, \abs{a_n - a_n'} \leq \varepsilon.
  \]
  In particular, we have
  \[
    \forall \varepsilon \in \Q^+, \exists N \in \Z^+ : \forall n \geq N, \abs{a_n - a_n'} \leq \dfrac{\varepsilon}{M}.
  \]
  Then we have
  \begin{align*}
             & \abs{a_n - a_n'} \leq \dfrac{\varepsilon}{M}                                                         \\
    \implies & \abs{b_n} \abs{a_n - a_n'} \leq \abs{b_n} \dfrac{\varepsilon}{M} &  & \text{(by \cref{4.2.9}(c)(e))} \\
    \implies & \abs{b_n} \abs{a_n - a_n'} \leq M \dfrac{\varepsilon}{M}         &  & \text{(by \cref{4.2.9}(c)(e))} \\
    \implies & \abs{b_n} \abs{a_n - a_n'} \leq \varepsilon                                                          \\
    \implies & \abs{b_n (a_n - a_n')} \leq \varepsilon                          &  & \text{(by \cref{4.3.3}(d))}    \\
    \implies & \abs{a_n b_n - a_n' b_n} \leq \varepsilon                        &  & \text{(by \cref{4.2.4}}
  \end{align*}
  and thus by \cref{5.2.6} we have \(xy = x'y\).
\end{proof}

\begin{note}
  Of course we can prove a similar substitution rule when \(y\) is replaced by a real number \(y'\) which is equal to \(y\).
\end{note}

\begin{note}
  At this point we embed the rationals back into the reals, by equating every rational number \(q\) with the real number \(\text{LIM}_{n \to \infty} q\).
  This embedding is consistent with our definitions of addition and multiplication, since for any rational numbers \(a, b\) we have
  \begin{align*}
    (\text{LIM}_{n \to \infty} a) + (\text{LIM}_{n \to \infty} b)      & = \text{LIM}_{n \to \infty} (a + b) \text{ and} \\
    (\text{LIM}_{n \to \infty} a) \times (\text{LIM}_{n \to \infty} b) & = \text{LIM}_{n \to \infty} (ab);
  \end{align*}
  this means that when one wants to add or multiply two rational numbers \(a, b\) it does not matter whether one thinks of these numbers as rationals or as the real numbers \(\text{LIM}_{n \to \infty} a, \text{LIM}_{n \to \infty} b\).
  Also, this identification of rational numbers and real numbers is consistent with our definitions of equality.
\end{note}

\begin{note}
  We can now easily define negation \(-x\) for real numbers \(x\) by the formula
  \[
    -x \coloneqq (-1) \times x,
  \]
  since \(-1\) is a rational number and is hence real.
  Note that this is clearly consistent with our negation for rational numbers since we have \(-q = (-1) \times q\) for all rational numbers \(q\).
  Also, from our definitions it is clear that
  \[
    -\text{LIM}_{n \to \infty} a_n = \text{LIM}_{n \to \infty} (-a_n).
  \]
  Once we have addition and negation, we can define subtraction as usual by
  \[
    x - y \coloneqq x + (-y),
  \]
  this implies
  \[
    \text{LIM}_{n \to \infty} a_n - \text{LIM}_{n \to \infty} b_n = \text{LIM}_{n \to \infty} (a_n - b_n).
  \]
\end{note}

\begin{prop}\label{5.3.11}
  All the laws of algebra from \cref{4.1.6} hold not only for the integers, but for the reals as well.
\end{prop}

\begin{proof}
  We illustrate this with one such rule: \(x(y + z) = xy + xz\).
  Let \(x = \text{LIM}_{n \to \infty} a_n\), \(y = \text{LIM}_{n \to \infty} b_n\), and \(z = \text{LIM}_{n \to \infty} c_n\) be real numbers.
  Then by definition, \(xy = \text{LIM}_{n \to \infty} a_n b_n\) and \(xz = \text{LIM}_{n \to \infty} a_n c_n\), and so \(xy + xz = \text{LIM}_{n \to \infty} (a_n b_n + a_n c_n)\).
  A similar line of reasoning shows that \(x(y + z) = \text{LIM}_{n \to \infty} a_n (b_n + c_n)\).
  But we already know that \(a_n (b_n + c_n)\) is equal to \(a_n b_n + a_n c_n\) for the rational numbers \(a_n, b_n, c_n\), and the claim follows.
  The other laws of algebra are proven similarly.
\end{proof}

\begin{defn}[Sequences bounded away from zero]\label{5.3.12}
  A sequence \(\text{LIM}_{n \to \infty} a_n\) of rational numbers is said to be \emph{bounded away from zero} iff there exists a rational number \(c > 0\) such that \(\abs{a_n} \geq c\) for all \(n \geq 1\).
\end{defn}

\setcounter{thm}{13}
\begin{lem}\label{5.3.14}
  Let \(x\) be a non-zero real number.
  Then \(x = \text{LIM}_{n \to \infty} a_n\) for some Cauchy sequence \((a_n)_{n = 1}^{\infty}\) which is bounded away from zero.
\end{lem}

\begin{proof}
  Since \(x\) is real, we know that \(x = \text{LIM}_{n \to \infty} b_n\) for some Cauchy sequence \((b_n)_{n = 1}^{\infty}\).
  But we are not yet done, because we do not know that \(b_n\) is bounded away from zero.
  On the other hand, we are given that \(x \neq 0 = \text{LIM}_{n \to \infty} 0\), which means that the sequence \((b_n)_{n = 1}^{\infty}\) is not equivalent to \((0)_{n = 1}^{\infty}\).
  Thus the sequence \((b_n)_{n = 1}^{\infty}\) cannot be eventually \(\varepsilon\)-close to \((0)_{n = 1}^{\infty}\) for every \(\varepsilon > 0\).
  Therefore we can find an \(\varepsilon > 0\) such that \((b_n)_{n = 1}^{\infty}\) is not eventually \(\varepsilon\)-close to \((0)_{n = 1}^{\infty}\).

  Let us fix this \(\varepsilon\).
  We know that \((b_n)_{n = 1}^{\infty}\) is a Cauchy sequence, so it is eventually \(\varepsilon\)-steady.
  Moreover, it is eventually \(\varepsilon / 2\)-steady, since \(\varepsilon / 2 > 0\).
  Thus there is an \(N \geq 1\) such that \(\abs{b_n - b_m} \leq \varepsilon / 2\) for all \(n, m \geq N\).

  On the other hand, we cannot have \(\abs{b_n} \leq \varepsilon\) for all \(n \geq N\), since this would imply that \((b_n)_{n = 1}^{\infty}\) is eventually \(\varepsilon\)-close to \((0)_{n = 1}^{\infty}\).
  Thus there must be some \(n_0 \geq N\) for which \(\abs{b_{n_0}} > \varepsilon\).
  Since we already know that \(\abs{b_{n_0} - b_n} \leq \varepsilon / 2\) for all \(n \geq N\), we have
  \begin{align*}
             & \abs{b_{n_0}} - \abs{b_{n_0} - b_n} \geq \varepsilon - \varepsilon / 2 = \varepsilon / 2 &  & \by{4.2.9}    \\
    \implies & \abs{b_{n_0}} - \abs{b_n - b_{n_0}} \geq \varepsilon / 2                                 &  & \by{4.3.3}    \\
    \implies & \abs{b_{n_0} + (b_n - b_{n_0})} \geq \varepsilon / 2                                     &  & \by{ac:4.3.1} \\
    \implies & \abs{b_n} \geq \varepsilon / 2.                                                          &  & \by{4.2.4}    \\
  \end{align*}
  Thus conclude from above that \(\abs{b_n} \geq \varepsilon / 2\) for all \(n \geq N\).

  This almost proves that \((b_n)_{n = 1}^{\infty}\) is bounded away from zero.
  Actually, what it does is show that \((b_n)_{n = 1}^{\infty}\) is \emph{eventually} bounded away from zero.
  But this is easily fixed, by defining a new sequence \(a_n\), by setting \(a_n \coloneqq \varepsilon / 2\) if \(n < N\) and \(a_n \coloneqq b_n\) if \(n \geq N\).
  Since \(b_n\) is a Cauchy sequence, it is not hard to verify that \(a_n\) is also a Cauchy sequence which is equivalent to \(b_n\) (because the two sequences are eventually the same), and so \(x = \text{LIM}_{n \to \infty} a_n\).
  And since \(\abs{b_n} \geq \varepsilon / 2\) for all \(n \geq N\), we know that \(\abs{a_n} \geq \varepsilon / 2\) for all \(n \geq 1\) (splitting into the two cases \(n \geq N\) and \(n < N\) separately).
  Thus we have a Cauchy sequence which is bounded away from zero (by \(\varepsilon / 2\) instead of \(\varepsilon\), but that's still OK since \(\varepsilon / 2 > 0\)), and which has \(x\) as a formal limit, and so we are done.
\end{proof}

\begin{lem}\label{5.3.15}
  Suppose that \((a_n)_{n = 1}^{\infty}\) is a Cauchy sequence which is bounded away from zero.
  Then the sequence \((a_n^{-1})_{n = 1}^{\infty}\) is also a Cauchy sequence.
\end{lem}

\begin{proof}
  Since \((a_n)_{n = 1}^{\infty}\) is bounded away from zero, we know that there is a \(c > 0\) such that \(\abs{a_n} \geq c\) for all \(n \geq 1\).
  Now we need to show that \((a_n^{-1})_{n = 1}^{\infty}\) is eventually \(\varepsilon\)-steady for each \(\varepsilon > 0\).
  Thus let us fix an \(\varepsilon > 0\);
  our task is now to find an \(N \geq 1\) such that \(\abs{a_n^{-1} - a_m^{-1}} \leq \varepsilon\) for all \(n, m \geq N\).
  But
  \[
    \abs{a_n^{-1} - a_m^{-1}} = \abs{\dfrac{a_m - a_n}{a_m a_n}} \leq \dfrac{\abs{a_m - a_n}}{c^2}
  \]
  (since \(\abs{a_m}, \abs{a_n} \geq c\)), and so to make \(\abs{a_n^{-1} - a_m^{-1}}\) less than or equal to \(\varepsilon\), it will suffice to make \(\abs{a_m - a_n}\) less than or equal to \(c^2 \varepsilon\).
  But since \((a_n)_{n = 1}^{\infty}\) is a Cauchy sequence, and \(c^2 \varepsilon > 0\), we can certainly find an \(N\) such that the sequence \((a_n)_{n = N}^{\infty}\) is \(c^2 \varepsilon\)-steady, i.e., \(\abs{a_m - a_n} \leq c^2 \varepsilon\) for all \(n, m \geq N\).
  By what we have said above, this shows that \(\abs{a_n^{-1} - a_m^{-1}} \leq \varepsilon\) for all \(m, n \geq N\), and hence the sequence \((a_n^{-1})_{n = 1}^{\infty}\) is eventually \(\varepsilon\)-steady.
  Since we have proven this for every \(\varepsilon\), we have that \((a_n^{-1})_{n = 1}^{\infty}\) is a Cauchy sequence, as desired.
\end{proof}

\begin{defn}[Reciprocals of real numbers]\label{5.3.16}
  Let \(x\) be a non-zero real number.
  Let \((a_n)_{n = 1}^{\infty}\) be a Cauchy sequence bounded away from zero such that \(x = \text{LIM}_{n \to \infty} a_n\) (such a sequence exists by \cref{5.3.14}).
  Then we define the reciprocal \(x^{-1}\) by the formula \(x^{-1} \coloneqq \text{LIM}_{n \to \infty} a_n^{-1}\).
  (From \cref{5.3.15} we know that \(x^{-1}\) is a real number.)
\end{defn}

\begin{lem}[Reciprocation is well defined]\label{5.3.17}
  Let \((a_n)_{n = 1}^{\infty}\) and \((b_n)_{n = 1}^{\infty}\) be two Cauchy sequences bounded away from zero such that \(\text{LIM}_{n \to \infty} a_n = \text{LIM}_{n \to \infty} b_n\) (i.e., the two sequences are equivalent).
  Then \(\text{LIM}_{n \to \infty} a_n^{-1} = \text{LIM}_{n \to \infty} b_n^{-1}\).
\end{lem}

\begin{proof}
  Consider the following product \(P\) of three real numbers:
  \[
    P \coloneqq (\text{LIM}_{n \to \infty} a_n^{-1}) \times (\text{LIM}_{n \to \infty} a_n) \times (\text{LIM}_{n \to \infty} b_n^{-1}).
  \]
  If we multiply this out, we obtain
  \[
    P = \text{LIM}_{n \to \infty} a_n^{-1} a_n b_n^{-1} = \text{LIM}_{n \to \infty} b_n^{-1}.
  \]
  On the other hand, since \(\text{LIM}_{n \to \infty} a_n = \text{LIM}_{n \to \infty} b_n\), we can write \(P\) in another way as
  \[
    P = (\text{LIM}_{n \to \infty} a_n^{-1}) \times (\text{LIM}_{n \to \infty} b_n) \times (\text{LIM}_{n \to \infty} b_n^{-1}).
  \]
  (cf. \cref{5.3.10}).
  Multiplying things out again, we get
  \[
    P = \text{LIM}_{n \to \infty} a_n^{-1} b_n b_n^{-1} = \text{LIM}_{n \to \infty} a_n^{-1}.
  \]
  Comparing our different formulae for \(P\) we see that \(\text{LIM}_{n \to \infty} a_n^{-1} = \text{LIM}_{n \to \infty} b_n^{-1}\), as desired.
\end{proof}

\begin{note}
  It is clear from the definition that \(xx^{-1} = x^{-1}x = 1\);
  thus all the field axioms (\cref{4.2.4}) apply to the reals as well as to the rationals.
  We of course cannot give \(0\) a reciprocal, since \(0\) multiplied by anything gives \(0\), not \(1\).
\end{note}

\begin{note}
  if \(q\) is a non-zero rational, and hence equal to the real number \(\text{LIM}_{n \to \infty} q\), then the reciprocal of \(\text{LIM}_{n \to \infty} q\) is \(\text{LIM}_{n \to \infty} q^{-1} = q^{-1}\);
  thus the operation of reciprocal on real numbers is consistent with the operation of reciprocal on rational numbers.
\end{note}

\begin{note}
  Once one has reciprocal, one can define division \(x / y\) of two real numbers \(x, y\), provided \(y\) is non-zero, by the formula
  \[
    x / y \coloneqq x \times y^{-1},
  \]
  just as we did with the rationals.
  In particular, we have the \emph{cancellation law}:
  if \(x, y, z\) are real numbers such that \(xz = yz\), and \(z\) is non-zero, then by dividing by \(z\) we conclude that \(x = y\).
  This cancellation law does not work when \(z\) is zero.
\end{note}

\exercisesection

\begin{ex}\label{ex:5.3.1}
  Prove \cref{5.3.3}.
\end{ex}

\begin{proof}
  See \cref{5.3.3}.
\end{proof}

\begin{ex}\label{ex:5.3.2}
  Prove \cref{5.3.10}.
\end{ex}

\begin{proof}
  See \cref{5.3.10}.
\end{proof}

\begin{ex}\label{ex:5.3.3}
  Let \(a, b\) be rational numbers.
  Show that \(a = b\) if and only if \(\text{LIM}_{n \to \infty} a = \text{LIM}_{n \to \infty} b\) (i.e., the Cauchy sequences \(a, a, a, a, \dots\) and \(b, b, b, b \dots\) equivalent if and only if \(a = b\)).
  This allows us to embed the rational numbers inside the real numbers in a well-defined manner.
\end{ex}

\begin{proof}
  Let \((a)_{n = 1}^{\infty}\) and \((b)_{n = 1}^{\infty}\) be two sequences where \(a, b \in \Q\).
  By \cref{5.2.6} \((a)_{n = 1}^\infty\) and \((b)_{n = 1}^\infty\) is Cauchy sequence since
  \[
    \forall \varepsilon \in \Q^+, \forall n \geq 1, \abs{a - a} = \abs{b - b} = 0 \leq \varepsilon.
  \]
  Then we have
  \begin{align*}
         & a = b                                                                                                         \\
    \iff & \forall \varepsilon \in \Q^+, \forall n \geq 1, \abs{a - b} \leq \varepsilon &  & \text{(by \cref{4.3.7}(a))} \\
    \iff & (a)_{n = 1}^\infty = (b)_{n = 1}^\infty                                      &  & \by{5.2.6}                  \\
    \iff & \text{LIM}_{n \to \infty} a = \text{LIM}_{n \to \infty} b.                   &  & \by{5.3.1}
  \end{align*}
\end{proof}

\begin{ex}\label{ex:5.3.4}
  Let \((a_n)_{n = 0}^{\infty}\) be a sequence of rational numbers which is bounded.
  Let \((b_n)_{n = 0}^{\infty}\) be another sequence of rational numbers which is equivalent to \((a_n)_{n = 0}^{\infty}\).
  Show that \((b_n)_{n = 0}^{\infty}\) is also bounded.
\end{ex}

\begin{proof}
  Since \((a_n)_{n = 0}^{\infty} = (b_n)_{n = 0}^{\infty}\), by \cref{5.2.6} we know that \((a_n)_{n = 0}^{\infty}\) and \((b_n)_{n = 0}^{\infty}\) are eventually \(\varepsilon\)-close for every \(\varepsilon \in \Q^+\).
  Thus by \cref{ex:5.2.2} \((a_n)_{n = 0}^{\infty}\) is bounded iff \((b_n)_{n = 0}^{\infty}\) is bounded.
\end{proof}

\begin{ex}\label{ex:5.3.5}
  Show that \(\text{LIM}_{n \to \infty} 1 / n = 0\).
\end{ex}

\begin{proof}
  By \cref{5.1.11} we know that the sequence \((\dfrac{1}{n})_{n = 1}^{\infty}\) is a Cauchy sequence.
  By \cref{4.4.1}, \(\forall \varepsilon \in \Q^+\), \(\exists N \in \Z^+\) such that
  \[
    \dfrac{1}{\varepsilon} < N \implies \dfrac{1}{N} < \varepsilon.
  \]
  Then we have
  \begin{align*}
    \forall n \geq N, \abs{\dfrac{1}{n} - 0} & = \dfrac{1}{n}    &  & \by{4.3.1}                   \\
                                             & \leq \dfrac{1}{N} &  & \text{(by \cref{4.3.12}(b))} \\
                                             & < \varepsilon
  \end{align*}
  and thus by \cref{5.2.6} \((\dfrac{1}{n})_{n = 1}^\infty = (0)_{n = 1}^\infty\).
  By \cref{5.3.1} and \cref{ex:5.3.3} we have
  \[
    \text{LIM}_{n \to \infty} \dfrac{1}{n} = \text{LIM}_{n \to \infty} 0 = 0.
  \]
\end{proof}
\section{Ordering the reals}\label{sec:5.4}

\begin{defn}\label{5.4.1}
  Let \((a_n)_{n = 1}^{\infty}\) be a sequence of rationals.
  We say that this sequence is \emph{positively bounded away from zero} iff we have a positive rational \(c > 0\) such that \(a_n \geq c\) for all \(n \geq 1\) (in particular, the sequence is entirely positive).
  The sequence is \emph{negatively bounded away from zero} iff we have a negative rational \(-c < 0\) such that \(a_n \leq -c\) for all \(n \geq 1\) (in particular, the sequence is entirely negative).
\end{defn}

\begin{note}
  It is clear that any sequence which is positively or negatively bounded away from zero, is bounded away from zero.
  Also, a sequence cannot be both positively bounded away from zero and negatively bounded away from zero at the same time.
\end{note}

\setcounter{thm}{2}
\begin{defn}\label{5.4.3}
  A real number \(x\) is said to be \emph{positive} iff it can be written as \(x = \text{LIM}_{n \to \infty} a_n\) for some Cauchy sequence \((a_n)_{n = 1}^{\infty}\) which is positively bounded away from zero.
  \(x\) is said to be \emph{negative} iff it can be written as \(x = \text{LIM}_{n \to \infty} a_n\) for some sequence \((a_n)_{n = 1}^{\infty}\) which is negatively bounded away from zero.
\end{defn}

\begin{prop}[Basic properties of positive reals]\label{5.4.4}
  For every real number \(x\), exactly one of the following three statements is true:
  \begin{enumerate}
    \item \(x\) is zero;
    \item \(x\) is positive;
    \item \(x\) is negative.
  \end{enumerate}
  A real number \(x\) is negative iff \(-x\) is positive.
  If \(x\) and \(y\) are positive, then so are \(x + y\) and \(xy\).
\end{prop}

\begin{proof}
  We first show that at least one of the three statements is true.
  Let \(x\) be the formal limit of some rational sequence \((a_n)_{n = 1}^{\infty}\), let \(\varepsilon, c \in \Q^+\) and let \(N, j, k \in \N\).
  Consider the following two cases:
  \begin{itemize}
    \item If \((a_n)_{n = 1}^{\infty}\) is eventually \(\varepsilon\)-close to \(0\) for all \(\varepsilon > 0\), then by \cref{5.2.6} we have \(x = 0\).
    \item If \((a_n)_{n = 1}^{\infty}\) is not eventually \(\varepsilon\)-close to \(0\) for all \(\varepsilon > 0\), then by \cref{5.2.6} we have \(x \neq 0\).
  \end{itemize}
  By \cref{5.3.14}, \(x \neq 0\) implies \(\exists c > 0\) such that \(\abs{a_n} \geq c > 0\) for every \(n \geq 1\).
  By \cref{4.3.3}(a) we have \(a_n \neq 0\).
  By \cref{4.2.9}(a) we now split into two cases:
  \begin{itemize}
    \item If \(a_n > 0\), then by \cref{4.3.1} we have \(a_n \geq c > 0\).
    \item If \(a_n < 0\), then by \cref{4.3.1} we have \(-a_n \geq c > 0\), and by \cref{ex:4.2.6} we have \(a_n \leq -c < 0\).
  \end{itemize}
  Since \((a_n)_{n = 1}^{\infty}\) is a Cauchy sequence, by \cref{5.1.8} we have
  \[
    \forall \varepsilon > 0, \exists N \geq 1 : \forall j, k \geq N, \abs{a_j - a_k} \leq \varepsilon.
  \]
  In particular,
  \[
    \exists N \geq 1 : \forall j, k \geq N, \abs{a_j - a_k} \leq c.
  \]
  So
  \begin{align*}
             & \abs{a_j - a_N} \leq c                          \\
    \implies & -c \leq a_j - a_N \leq c        &  & \by{4.3.3} \\
    \implies & -c + a_N \leq a_j \leq c + a_N. &  & \by{4.2.9} \\
  \end{align*}
  By \cref{4.2.9}(a) again we now split into two cases:
  \begin{itemize}
    \item If \(a_N > 0\), then we have
          \begin{align*}
                     & (-c + a_N \leq a_j \leq c + a_N) \land (0 < c \leq a_N)                                \\
            \implies & 0 \leq a_j \leq c + a_N                                 &            & \by{4.2.9}[c,d] \\
            \implies & c < a_j \leq c + a_N.                                   & (x \neq 0)
          \end{align*}
          Since this is true for all \(j \geq N\), by \cref{5.3.14} and \cref{5.4.1} we know that \((a_n)_{n = 1}^\infty\) is positively bounded away from zero.
          Thus by \cref{5.4.3} \(x\) is positive.
    \item If \(a_N < 0\), then we have
          \begin{align*}
                     & (-c + a_N \leq a_j \leq c + a_N) \land (a_N \leq -c < 0)                                \\
            \implies & -c + a_N \leq a_j \leq 0                                 &            & \by{4.2.9}[c,d] \\
            \implies & -c + a_N \leq a_j < -c.                                  & (x \neq 0)
          \end{align*}
          Since this is true for all \(j \geq N\), by \cref{5.3.14} and \cref{5.4.1} we know that \((a_n)_{n = 1}^\infty\) is negatively bounded away from zero.
          Thus by \cref{5.4.3} \(x\) is negative.
  \end{itemize}
  From all cases above we conclude that at least one of the three statements is true.

  Next we show that at most one of the three statements is true.
  Let \(x\) be the formal limit of some rational sequence \((a_n)_{n = 1}^{\infty}\) and let \(\varepsilon, c \in \Q^+\).
  \begin{itemize}
    \item If \(x = 0\) and \(x\) is positive, then we have \((a_n)\) eventually \(\varepsilon\)-close to \(0\) for all \(\varepsilon > 0\) and \(a_n \geq c\) for all \(n \geq 1\).
          But then we have \(\abs{a_n - 0} \leq c / 2\) and \(a_n \geq c > 0\), a contradiction.
    \item If \(x = 0\) and \(x\) is negative, then we have \((a_n)\) eventually \(\varepsilon\)-close to \(0\) for all \(\varepsilon > 0\) and \(a_n \leq -c\) for all \(n \geq 1\).
          But then we have \(\abs{a_n - 0} \leq c / 2\) and \(a_n \leq -c < 0\), a contradiction.
    \item If \(x\) is positive and \(x\) is negative, then we have \(a_n \geq c\) and \(a_n \leq -c\) for all \(n \geq 1\).
          But then we have \(a_n < 0\) and \(0 < a_n\), a contradiction.
  \end{itemize}
  From all cases above we conclude that at most one of the three statements is true.

  Next we show that \(x\) is negative iff \(-x\) is positive.
  Let \(x\) be the formal limit of some sequence \((a_n)_{n = 1}^{\infty}\).
  Then we have
  \begin{align*}
         & x \text{ is negative}                                                             \\
    \iff & (x = \text{LIM}_{n \to \infty} a_n)                                               \\
         & \land (\exists c \in \Q^+ : \forall n \geq 1, a_n \leq -c < 0) &  & \by{5.4.3}    \\
    \iff & (-x = \text{LIM}_{n \to \infty} -a_n)                          &  & \by{5.3.9}    \\
         & \land (\exists c \in \Q^+ : \forall n \geq 1, a_n \leq -c < 0)                    \\
    \iff & (-x = \text{LIM}_{n \to \infty} -a_n)                                             \\
         & \land (\exists c \in \Q^+ : \forall n \geq 1, -a_n \geq c > 0) &  & \by{ex:4.2.6} \\
    \iff & -x \text{ is positive}.                                        &  & \by{5.4.3}
  \end{align*}

  Next we show that \(x, y\) are positive implies \(x + y\) is also positive.
  Let \(x\) be the formal limit of some sequence \((a_n)_{n = 1}^{\infty}\) and let \(y\) be the formal limit of some sequence \((b_n)_{n = 1}^{\infty}\).
  Then we have
  \begin{align*}
             & (x \text{ is positive}) \land (y \text{ is positive})                                                \\
    \implies & (x = \text{LIM}_{n \to \infty} a_n) \land (y = \text{LIM}_{n \to \infty} b_n)                        \\
             & \land (\exists c_1 \in \Q^+ : \forall n \geq 1, 0 < c_1 < a_n)                                       \\
             & \land (\exists c_2 \in \Q^+ : \forall n \geq 1, 0 < c_2 < b_n)                  &  & \by{5.4.3}      \\
    \implies & (x + y = \text{LIM}_{n \to \infty} a_n + b_n)                                   &  & \by{5.3.4}      \\
             & \land (\exists c_1 \in \Q^+ : \forall n \geq 1, 0 < c_1 < a_n)                                       \\
             & \land (\exists c_2 \in \Q^+ : \forall n \geq 1, 0 < c_2 < b_n)                                       \\
    \implies & (x + y = \text{LIM}_{n \to \infty} a_n + b_n)                                                        \\
             & \land (\exists c_1, c_2 \in \Q^+ : \forall n \geq 1, 0 < c_1 + c_2 < a_n + b_n) &  & \by{4.2.9}[c,d] \\
    \implies & x + y \text{ is positive}.                                                      &  & \by{5.4.3}
  \end{align*}

  Finally we show that \(x, y\) are positive implies \(xy\) is also positive.
  Let \(x\) be the formal limit of some sequence \((a_n)_{n = 1}^{\infty}\) and let \(y\) be the formal limit of some sequence \((b_n)_{n = 1}^{\infty}\).
  Then we have
  \begin{align*}
             & (x \text{ is positive}) \land (y \text{ is positive})                                              \\
    \implies & (x = \text{LIM}_{n \to \infty} a_n) \land (y = \text{LIM}_{n \to \infty} b_n)                      \\
             & \land (\exists c_1 \in \Q^+ : \forall n \geq 1, 0 < c_1 < a_n)                                     \\
             & \land (\exists c_2 \in \Q^+ : \forall n \geq 1, 0 < c_2 < b_n)                &  & \by{5.4.3}      \\
    \implies & (xy = \text{LIM}_{n \to \infty} a_n b_n)                                      &  & \by{5.3.9}      \\
             & \land (\exists c_1 \in \Q^+ : \forall n \geq 1, 0 < c_1 < a_n)                                     \\
             & \land (\exists c_2 \in \Q^+ : \forall n \geq 1, 0 < c_2 < b_n)                                     \\
    \implies & (xy = \text{LIM}_{n \to \infty} a_n b_n)                                                           \\
             & \land (\exists c_1, c_2 \in \Q^+ : \forall n \geq 1, 0 < c_1 c_2 < a_n b_n)   &  & \by{4.2.9}[d,e] \\
    \implies & xy \text{ is positive}.                                                       &  & \by{5.4.3}
  \end{align*}
\end{proof}

\begin{note}
  If \(q\) is a positive rational number, then the Cauchy sequence \(q, q, q, \dots\) is positively bounded away from zero, and hence \(\text{LIM}_{n \to \infty} q = q\) is a positive real number.
  Thus the notion of positivity for rationals is consistent with that for reals.
  Similarly, the notion of negativity for rationals is consistent with that for reals.
\end{note}

\begin{defn}[Absolute value]\label{5.4.5}
  Let \(x\) be a real number.
  We define the \emph{absolute value} \(\abs{x}\) of \(x\) to equal \(x\) if \(x\) is positive, \(-x\) when \(x\) is negative, and \(0\) when \(x\) is zero.
\end{defn}

\begin{defn}[Ordering of the real numbers]\label{5.4.6}
  Let \(x\) and \(y\) be real numbers.
  We say that \(x\) is \emph{greater than} \(y\), and write \(x > y\), iff \(x - y\) is a positive real number, and \(x < y\) iff \(x - y\) is a negative real number.
  We define \(x \geq y\) iff \(x > y\) or \(x = y\), and similarly define \(x \leq y\).
\end{defn}

\begin{note}
  Comparing this with the definition of order on the rationals from \cref{4.2.8} we see that order on the reals is consistent with order on the rationals, i.e., if two rational numbers \(q, q'\) are such that \(q\) is less than \(q'\) in the rational number system, then \(q\) is still less than \(q'\) in the real number system, and similarly for ``greater than''.
  In the same way we see that the definition of absolute value given here is consistent with that in \cref{4.3.1}.
\end{note}

\begin{prop}\label{5.4.7}
  All the claims in \cref{4.2.9} which held for rationals, continue to hold for real numbers.
\end{prop}

\begin{proof}{(a)}
  We first show that at least one of the three statements is true.
  By \cref{5.4.4}, exactly one of the following three statements is true:
  \begin{itemize}
    \item \(x - y = 0\).
          Then by \cref{5.3.11} we have \(x = y\).
    \item \(x - y\) is positive.
          Then by \cref{5.4.6} we have \(x > y\).
    \item \(x - y\) is negative.
          Then by \cref{5.4.6} we have \(x < y\).
  \end{itemize}
  So at least one of the three statements is true.

  Now we show that at most one of the three statements is true.
  \begin{itemize}
    \item If \(x = y\) and \(x > y\) are true, then by \cref{5.3.11} we have \(x - y = 0\) and by \cref{5.4.6} we have \(x - y\) is positive.
          But this contradict to \cref{5.4.4}.
    \item If \(x = y\) and \(x < y\) are true, then by \cref{5.3.11} we have \(x - y = 0\) and by \cref{5.4.6} we have \(x - y\) is negative.
          But this contradict to \cref{5.4.4}.
    \item If \(x > y\) and \(x < y\) are true, then by \cref{5.4.6}, \(x - y\) is both positive and negative.
          But this contradict to \cref{5.4.4}.
  \end{itemize}
  From all cases above we conclude that at most one of the three statements is true.
\end{proof}

\begin{proof}{(b)}
  We have
  \begin{align*}
         & x < y                                         \\
    \iff & x - y \text{ is negative}    &  & \by{5.4.6}  \\
    \iff & -(x - y) \text{ is positive} &  & \by{5.4.4}  \\
    \iff & y - x \text{ is positive}    &  & \by{5.3.11} \\
    \iff & y > x.                       &  & \by{5.4.6}
  \end{align*}
\end{proof}

\begin{proof}{(c)}
  We have
  \begin{align*}
             & (x < y) \land (y < z)                                                                \\
    \implies & (x - y \text{ is negative}) \land (y - z \text{ is negative})       &  & \by{5.4.6}  \\
    \implies & (-(x - y) \text{ is positive}) \land (-(y - z) \text{ is positive}) &  & \by{5.4.4}  \\
    \implies & (y - x \text{ is positive}) \land (z - y \text{ is positive})       &  & \by{5.3.11} \\
    \implies & y - x + z - y \text{ is positive}                                   &  & \by{5.4.4}  \\
    \implies & - x + z \text{ is positive}                                         &  & \by{5.3.11} \\
    \implies & -(x - z) \text{ is positive}                                        &  & \by{5.3.11} \\
    \implies & x - z \text{ is negative}                                           &  & \by{5.4.4}  \\
    \implies & x < z.                                                              &  & \by{5.4.6}
  \end{align*}
\end{proof}

\begin{proof}{(d)}
  We have
  \begin{align*}
             & x < y                                                \\
    \implies & x - y \text{ is negative}           &  & \by{5.4.6}  \\
    \implies & x + z - z - y \text{ is negative}   &  & \by{5.3.11} \\
    \implies & x + z - (y + z) \text{ is negative} &  & \by{5.3.11} \\
    \implies & x + z < y + z.                      &  & \by{5.4.6}
  \end{align*}
\end{proof}

\begin{proof}{(e)}
  We have
  \begin{align*}
             & x < y                                           \\
    \implies & y > x                        &  & \by{5.4.7}[b] \\
    \implies & y - x \text{ is positive}    &  & \by{5.4.6}    \\
    \implies & (y - x)z \text{ is positive} &  & \by{5.4.4}    \\
    \implies & yz - xz \text{ is positive}  &  & \by{5.3.11}   \\
    \implies & yz > xz                      &  & \by{5.4.6}    \\
    \implies & xz < yz.                     &  & \by{5.4.7}[b]
  \end{align*}
\end{proof}

\begin{prop}\label{5.4.8}
  Let \(x\) be a positive real number.
  Then \(x^{-1}\) is also positive.
  Also, if \(y\) is another positive number and \(x > y\), then \(x^{-1} < y^{-1}\).
\end{prop}

\begin{proof}
  Let \(x\) be positive.
  Since \(xx^{-1} = 1\), the real number \(x^{-1}\) cannot be zero (since \(x0 = 0 \neq 1\)).
  Also, from \cref{5.4.4} it is easy to see that a positive number times a negative number is negative;
  this shows that \(x^{-1}\) cannot be negative, since this would imply that \(xx^{-1} = 1\) is negative, a contradiction.
  Thus, by \cref{5.4.4}, the only possibility left is that \(x^{-1}\) is positive.

  Now let \(y\) be positive as well, so \(x^{-1}\) and \(y^{-1}\) are also positive.
  If \(x^{-1} \geq y^{-1}\), then by \cref{5.4.7} we have \(xx^{-1} > yx^{-1} \geq yy^{-1}\), thus \(1 > 1\), which is a contradiction.
  Thus we must have \(x^{-1} < y^{-1}\).
\end{proof}

\begin{prop}[The non-negative reals are closed]\label{5.4.9}
  Let \(a_1, a_2, a_3, \dots\) be a Cauchy sequence of non-negative rational numbers.
  Then \(\text{LIM}_{n \to \infty} a_n\) is a non-negative real number.
\end{prop}

\begin{proof}
  We argue by contradiction, and suppose that the real number \(x \coloneqq \text{LIM}_{n \to \infty} a_n\) is a negative number.
  Then by definition of negative real number, we have \(x = \text{LIM}_{n \to \infty} b_n\) for some sequence \(b_n\) which is negatively bounded away from zero, i.e., there is a negative rational \(-c < 0\) such that \(b_n \leq -c\) for all \(n \geq 1\).
  On the other hand, we have \(a_n \geq 0\) for all \(n \geq 1\), by hypothesis.
  Thus the numbers \(a_n\) and \(b_n\) are never \(c / 2\)-close, since \(c / 2 < c\).
  Thus the sequences \((a_n)_{n = 1}^{\infty}\) and \((b_n)_{n = 1}^{\infty}\) are not eventually \(c / 2\)-close.
  Since \(c / 2 > 0\), this implies that \((a_n)_{n = 1}^{\infty}\) and \((b_n)_{n = 1}^{\infty}\) are not equivalent.
  But this contradicts the fact that both these sequences have \(x\) as their formal limit.
\end{proof}

\begin{note}
  Eventually, we will see a better explanation of \cref{5.4.9}:
  the set of non-negative reals is \emph{closed}, whereas the set of positive reals is \emph{open}.
\end{note}

\begin{cor}\label{5.4.10}
  Let \((a_n)_{n = 1}^{\infty}\) and \((b_n)_{n = 1}^{\infty}\) be Cauchy sequences of rationals such that \(a_n \geq b_n\) for all \(n \geq 1\).
  Then \(\text{LIM}_{n \to \infty} a_n \geq \text{LIM}_{n \to \infty} b_n\).
\end{cor}

\begin{proof}
  Apply \cref{5.4.9} to the sequence \(a_n - b_n\).
\end{proof}

\begin{rmk}\label{5.4.11}
  Note that \cref{5.4.10} does not work if the \(\geq\) signs are replaced by \(>\):
  for instance if \(a_n \coloneqq 1 + 1 / n\) and \(b_n \coloneqq 1 - 1 / n\), then \(a_n\) is always strictly greater than \(b_n\), but the formal limit of \(a_n\) is not greater than the formal limit of \(b_n\), instead they are equal.
\end{rmk}

\begin{note}
  We now define distance \(d(x, y) \coloneqq \abs{x - y}\) just as we did for the rationals.
  In fact, \cref{4.3.3,4.3.7} hold not only for the rationals, but for the reals;
  the proof is identical, since the real numbers obey all the laws of algebra and order that the rationals do.
\end{note}

\begin{prop}[Bounding of reals by rationals]\label{5.4.12}
  Let \(x\) be a positive real number.
  Then there exists a positive rational number \(q\) such that \(q \leq x\), and there exists a positive integer \(N\) such that \(x \leq N\).
\end{prop}

\begin{proof}
  Since \(x\) is a positive real, it is the formal limit of some Cauchy sequence \((a_n)_{n = 1}^{\infty}\) which is positively bounded away from zero.
  Also, by \cref{5.1.15}, this sequence is bounded.
  Thus we have rationals \(q > 0\) and \(r\) such that \(q \leq a_n \leq r\) for all \(n \geq 1\).
  But by \cref{4.4.1} we know that there is some integer \(N\) such that \(r \leq N\);
  since \(q\) is positive and \(q \leq r \leq N\), we see that \(N\) is positive.
  Thus \(q \leq a_n \leq N\) for all \(n \geq 1\).
  Applying \cref{5.4.10} we obtain that \(q \leq x \leq N\), as desired.
\end{proof}

\begin{cor}[Archimedean property]\label{5.4.13}
  Let \(x\) and \(\varepsilon\) be any positive real numbers.
  Then there exists a positive integer \(M\) such that \(M\varepsilon > x\).
\end{cor}

\begin{proof}
  The number \(x / \varepsilon\) is positive, and hence by \cref{5.4.12} there exists a positive integer \(N\) such that \(x / \varepsilon \leq N\).
  If we set \(M \coloneqq N + 1\), then \(x / \varepsilon < M\).
  Now multiply by \(\varepsilon\).
\end{proof}

\begin{note}
  This property (\cref{5.4.13}) is quite important;
  it says that no matter how large \(x\) is and how small \(\varepsilon\) is, if one keeps adding \(\varepsilon\) to itself, one will eventually overtake \(x\).
\end{note}

\begin{prop}\label{5.4.14}
  Given any two real numbers \(x < y\), we can find a rational number \(q\) such that \(x < q < y\).
\end{prop}

\begin{proof}
  We have
  \begin{align*}
             & x < y                                                 \\
    \implies & y > x                              &  & \by{5.4.7}    \\
    \implies & y - x \text{ is positive}          &  & \by{5.4.6}    \\
    \implies & \exists N \in \Z^+ : y - x > 1 / N &  & \by{ex:5.4.4} \\
    \implies & y > x + 1 / N.                     &  & \by{5.4.7}    \\
  \end{align*}
  Since \(x\) is a real number, by \cref{5.3.10} we know that \(Nx\) is also a real number.
  By \cref{ex:5.4.3}, \(\exists M \in \Z\) such that \(M \leq Nx < M + 1\).
  So we have
  \begin{align*}
             & M \leq Nx < M + 1                                                                     \\
    \implies & \dfrac{M}{N} \leq x < \dfrac{M + 1}{N}                                &  & \by{5.4.7} \\
    \implies & (\dfrac{M}{N} \leq x) \land (x < \dfrac{M + 1}{N})                    &  & \by{5.4.7} \\
    \implies & (\dfrac{M + 1}{N} \leq x + \dfrac{1}{N}) \land (x < \dfrac{M + 1}{N}) &  & \by{5.4.7} \\
    \implies & x < \dfrac{M + 1}{N} \leq x + \dfrac{1}{N}                            &  & \by{5.4.7} \\
    \implies & x < \dfrac{M + 1}{N} \leq x + \dfrac{1}{N} < y.                       &  & \by{5.4.7}
  \end{align*}
  Since \((M + 1) / N \in \Q\), we conclude that \(\exists q = (M + 1) / N \in \Q\) such that \(x < q < y\) for arbitrary real numbers \(x, y\).
\end{proof}

\begin{rmk}\label{5.4.15}
  Up until now, we have not addressed the fact that real numbers can be expressed using the decimal system.
  For instance, the formal limit of
  \[
    1.4, 1.41, 1.414, 1.4142, 1.41421, \dots
  \]
  is more conventionally represented as the decimal \(1.41421\dots\).
  There are some subtleties in the decimal system, for instance \(0.9999\dots\) and \(1.000\dots\) are in fact the same real number.
\end{rmk}

\begin{ac}\label{ac:5.4.1}
  Let \(X\) be an non-empty finite subset of \(\R\).
  Then \(X\) has exactly one maximum \(\max(X) \in X\) satisfying
  \[
    \forall x \in X, x \leq \max(X).
  \]
  Similarly, \(X\) has exactly one minimum \(\min(X) \in X\) satisfying
  \[
    \forall x \in X, x \geq \min(X).
  \]
\end{ac}

\begin{proof}
  Let \(n = \#(X)\).
  We use induction on \(n\) to show that \(\max(X), \min(X) \in X\) and we start with \(n = 1\) (since \(\#(\emptyset) = 0\) by \cref{ex:3.6.2}).
  For \(n = 1\), we have \(X = \set{x}\) for some \(x \in \R\).
  Then we have
  \begin{align*}
             & \forall y \in X, y = x                                        &  & \by{3.3}   \\
    \implies & (\forall y \in X, y \leq x) \land (\forall y \in X, y \geq x) &  & \by{5.4.6} \\
    \implies & \max(X) = \min(X) = x
  \end{align*}
  and the base case holds.
  Suppose inductively that for some \(n \geq 1\) we have \(\max(X) \in X\) and \(\min(X) \in X\).
  Then for \(n + 1\), we need to show that \(\max(X) \in X\) and \(\min(X) \in X\).
  Let \(x \in X\) and let \(X' = X \setminus \set{x}\).
  Then we have
  \begin{align*}
             & X = X' \cup \set{x}                                                                                                           \\
    \implies & \#(X') = n                                                                                &                  & \by{3.6.14}[a] \\
    \implies & (\max(X') \in X') \land (\min(X') \in X')                                                 &                  & \byIH          \\
    \implies & (\max(X') \in X) \land (\min(X') \in X)                                                   & (X' \subseteq X)                  \\
    \implies & \begin{dcases}
                 \max(X) = \begin{dcases}
                  \max(X') & \text{if } \max(X') > x \\
                  x        & \text{if } \max(X') < x
                \end{dcases} \\
                 \min(X) = \begin{dcases}
                  \min(X') & \text{if } \min(X') < x \\
                  x        & \text{if } \min(X') > x
                \end{dcases}
               \end{dcases}                                                                                          \\
    \implies & \big(\forall y \in X, x \leq \max(X)\big) \land \big(\forall y \in X, x \geq \min(X)\big)
  \end{align*}
  and this closes the induction.

  Now we show that at most one \(\max(X), \min(X) \in X\).
  Suppose that \(x_1 = \max(X)\) and \(x_2 = \max(X)\).
  Then by \cref{5.4.7}(a) we have
  \[
    (x_1 \leq x_2) \land (x_2 \leq x_1) \implies x_1 = x_2.
  \]
  Similarly suppose that \(x_1 = \min(X)\) and \(x_2 = \min(X)\).
  Then by \cref{5.4.7}(a) we have
  \[
    (x_1 \geq x_2) \land (x_2 \geq x_1) \implies x_1 = x_2.
  \]
\end{proof}

\exercisesection

\begin{ex}\label{ex:5.4.1}
  Prove \cref{5.4.4}.
\end{ex}

\begin{proof}
  See \cref{5.4.4}.
\end{proof}

\begin{ex}\label{ex:5.4.2}
  Prove the remaining claims in \cref{5.4.7}.
\end{ex}

\begin{proof}
  See \cref{5.4.7}.
\end{proof}

\begin{ex}\label{ex:5.4.3}
  Show that for every real number \(x\) there is exactly one integer \(N\) such that \(N \leq x < N + 1\).
  (This integer \(N\) is called the \emph{integer part} of \(x\), and is sometimes denoted \(N = \floor{x}\).)
\end{ex}

\begin{proof}
  We first prove the existence of the integer \(N\).
  By \cref{5.4.4}, exactly one of the following three statements is true:
  \begin{itemize}
    \item \(x = 0\).
          Then we choose \(N = 0\) so that \(0 \leq 0 < 1\).
    \item \(x\) is positive.
          Then by \cref{5.4.12} \(\exists q \in \Q^+\) and \(\exists N_1' \in \Z^+\) such that \(q \leq x \leq N_1'\).
          Let \(N_1 = N_1' + 1\).
          Then we have \(x < N_1\).
          By \cref{4.4.1} we know that \(\exists N_2 \in \Z\) such that \(N_2 \leq q\), thus by \cref{5.4.7}(c) we have \(N_2 \leq x < N_1\).
          Let \(X\) be the set
          \[
            X = \set{n \in \Z : N_2 \leq n < N_1}
          \]
          and let \(X'\) be the set
          \[
            X' = \set{n \in X : n \leq x < N_1}.
          \]
          We know that \(X, X'\) is finite since \(\#(X) = N_1 - N_2 + 1 \geq 1\) and \(X' \subseteq X\) (by \cref{3.6.14}(c)).
          We also know that \(X, X'\) is non-empty since \(N_2 \in X'X\).
          By \cref{ac:5.4.1} we know that \(\exists!\ \max(X') \in X'\).
          Let \(N = \max(X')\).
          By the definition of \(X'\) we know that \(N \leq x < N_1\).
          We must also have \(x < N + 1\), otherwise if \(N + 1 \leq x\) then \(\max(X') = N + 1\), a contradiction.
          Thus we have \(N \leq x < N + 1\).
    \item \(x\) is negative.
          Then we have
          \begin{align*}
                     & -x > 0                                  &  & \by{5.4.4}               \\
            \implies & \exists M \in \Z^+ : M1 = M > x         &  & \by{5.4.13}              \\
            \implies & x + M > 0                               &  & \by{5.4.7}[d]            \\
            \implies & \exists N \in \Z : N \leq x + M < N + 1 &  & \text{(from case above)} \\
            \implies & N - M \leq x < N - M + 1.
          \end{align*}
  \end{itemize}
  From all cases above we conclude that \(\exists N \in \Z : N \leq x < N + 1\).

  Now we prove the uniqueness of the integer \(N\).
  Suppose that \(\exists N_1, N_2 \in \Z\) such that \(N_1 \leq x < N_1 + 1\) and \(N_2 \leq x < N_2 + 1\).
  Then we have
  \begin{align*}
             & (N_1 \leq x < N_1 + 1) \land (N_2 \leq x < N_2 + 1)                    \\
    \implies & (N_1 < N_2 + 1) \land (N_2 < N_1 + 1)               &  & \by{5.4.7}[c] \\
    \implies & (N_1 + 1 \leq N_2 + 1) \land (N_2 + 1 \leq N_1 + 1) &  & \by{4.1.10}   \\
    \implies & (N_1 \leq N_2) \land (N_2 \leq N_1)                 &  & \by{4.1.10}   \\
    \implies & N_1 = N_2.                                          &  & \by{4.1.11}
  \end{align*}
  Thus we conclude that \(\exists!\ N \in \Z : N \leq x < N + 1\).
\end{proof}

\begin{ex}\label{ex:5.4.4}
  Show that for any positive real number \(x > 0\) there exists a positive integer \(N\) such that \(x > 1 / N > 0\).
\end{ex}

\begin{proof}
  We have
  \begin{align*}
             & x > 0                                                 \\
    \implies & x^{-1} > 0                           &  & \by{5.4.8}  \\
    \implies & \exists N \in \Z^+ : N1 = N > x^{-1} &  & \by{5.4.13} \\
    \implies & N^{-1} = \dfrac{1}{N} < x.           &  & \by{5.4.8}
  \end{align*}
\end{proof}

\begin{ex}\label{ex:5.4.5}
  Prove \cref{5.4.14}.
\end{ex}

\begin{proof}
  See \cref{5.4.14}.
\end{proof}

\begin{ex}\label{ex:5.4.6}
  Let \(x, y\) be real numbers and let \(\varepsilon > 0\) be a positive real.
  Show that \(\abs{x - y} < \varepsilon\) iff \(y - \varepsilon < x < y + \varepsilon\), and that \(\abs{x - y} \leq \varepsilon\) iff \(y - \varepsilon \leq x \leq y + \varepsilon\).
\end{ex}

\begin{proof}
  We first show that \(\abs{x - y} < \varepsilon \iff y - \varepsilon < x < y + \varepsilon\).
  \begin{align*}
         & \abs{x - y} < \varepsilon                                                                                     \\
    \iff & \big(-(x - y) \leq x - y < \varepsilon\big) \lor \big(x - y \leq -(x - y) < \varepsilon\big) &  & \by{5.4.5}  \\
    \iff & (x - y < \varepsilon) \land (-(x - y) < \varepsilon)                                                          \\
    \iff & (x - y < \varepsilon) \land (y - x < \varepsilon)                                            &  & \by{5.3.11} \\
    \iff & (x < y + \varepsilon) \land (y - \varepsilon < x)                                            &  & \by{5.4.7}  \\
    \iff & y - \varepsilon < x < y + \varepsilon.                                                       &  & \by{5.4.7}
  \end{align*}

  Now we show that \(\abs{x - y} \leq \varepsilon \iff y - \varepsilon \leq x \leq y + \varepsilon\).
  Since
  \begin{align*}
         & \abs{x - y} = \varepsilon                                            \\
    \iff & (x - y = \varepsilon) \lor (-(x - y) = \varepsilon) &  & \by{5.4.5}  \\
    \iff & (x = y + \varepsilon) \lor (x = y - \varepsilon),   &  & \by{5.3.11}
  \end{align*}
  we have
  \begin{align*}
         & \abs{x - y} \leq \varepsilon                                                                             \\
    \iff & (\abs{x - y} < \varepsilon) \land (\abs{x - y} = \varepsilon)                                            \\
    \iff & (y - \varepsilon < x < y + \varepsilon) \land \big((x = y + \varepsilon) \lor (x = y - \varepsilon)\big) \\
    \iff & (y - \varepsilon \leq x < y + \varepsilon) \lor (y - \varepsilon < x \leq y + \varepsilon)               \\
    \iff & y - \varepsilon \leq x \leq y + \varepsilon.
  \end{align*}
\end{proof}

\begin{ex}\label{ex:5.4.7}
  Let \(x\) and \(y\) be real numbers.
  Show that \(x \leq y + \varepsilon\) for all real numbers \(\varepsilon > 0\) iff \(x \leq y\).
  Show that \(\abs{x - y} \leq \varepsilon\) for all real numbers \(\varepsilon > 0\) iff \(x = y\).
\end{ex}

\begin{proof}
  We first show that \(x \leq y + \varepsilon\) for all real numbers \(\varepsilon > 0\) iff \(x \leq y\).
  \begin{align*}
         & \forall \varepsilon \in \R^+, x \leq y + \varepsilon                 \\
    \iff & \forall \varepsilon \in \R^+, x - y \leq \varepsilon &  & \by{5.4.7} \\
    \iff & \lnot (x - y > 0)                                                    \\
    \iff & x - y \leq 0                                         &  & \by{5.4.7} \\
    \iff & x \leq y.                                            &  & \by{5.4.6}
  \end{align*}

  Now we show that \(\abs{x - y} \leq \varepsilon\) for all real numbers \(\varepsilon > 0\) iff \(x = y\).
  \begin{align*}
         & \forall \varepsilon \in \R^+, \abs{x - y} \leq \varepsilon                                   \\
    \iff & \forall \varepsilon \in \R^+, y - \varepsilon \leq x \leq y + \varepsilon &  & \by{ex:5.4.6} \\
    \iff & (x \leq y) \land (y \leq x)                                                                  \\
    \iff & x = y.                                                                    &  & \by{5.3.11}
  \end{align*}
\end{proof}

\begin{ex}\label{ex:5.4.8}
  Let \((a_n)_{n = 1}^{\infty}\) be a Cauchy sequence of rationals, and let \(x\) be a real number.
  Show that if \(a_n \leq x\) for all \(n \geq 1\), then \(\text{LIM}_{n \to \infty} a_n \leq x\).
  Similarly, show that if \(a_n \geq x\) for all \(n \geq 1\), then \(\text{LIM}_{n \to \infty} a_n \geq x\).
\end{ex}

\begin{proof}
  We first show that if \(a_n \leq x\) for all \(n \geq 1\), then \(\text{LIM}_{n \to \infty} a_n \leq x\).
  Let \(a = \text{LIM}_{n \to \infty} a_n\).
  Suppose for sake of contradiction that \(a > x\).
  Then by \cref{5.4.14}, \(\exists q \in \Q\) such that \(a > q > x\).
  Since \(q > x\), we have \(a_n \leq x < q\) for all \(n \geq 1\).
  But by \cref{5.4.10} we have \(a = \text{LIM}_{n \to \infty} a_n \leq \text{LIM}_{n \to \infty} q = q\), which contradict to \(a > q\).
  Thus we must have \(a \leq x\).

  Now we show that if \(a_n \geq x\) for all \(n \geq 1\), then \(\text{LIM}_{n \to \infty} a_n \geq x\).
  We have
  \begin{align*}
             & a_n \geq x                                                            \\
    \implies & -a_n \leq -x                           &  & \by{5.4.7}                \\
    \implies & \text{LIM}_{n \to \infty} -a_n \leq -x &  & \text{(from proof above)} \\
    \implies & -\text{LIM}_{n \to \infty} a_n \leq -x &  & \by{5.3.10}               \\
    \implies & \text{LIM}_{n \to \infty} a_n \geq x.  &  & \by{5.4.7}
  \end{align*}
\end{proof}

\section{The least upper bound property}\label{i:sec:5.5}

\begin{defn}[Upper bound]\label{i:5.5.1}
  Let \(E\) be a subset of \(\R\), and let \(M\) be a real number.
  We say that \(M\) is an \emph{upper bound} for \(E\), iff we have \(x \leq M\) for every element \(x\) in \(E\).
\end{defn}

\setcounter{thm}{2}
\begin{eg}\label{i:5.5.3}
  Let \(\R^+\) be the set of positive reals: \(\R^+ \coloneqq \set{x \in \R : x > 0}\).
  Then \(\R^+\) does not have any upper bounds at all.
  (More precisely, \(\R^+\) has no upper bounds which are real numbers.)
\end{eg}

\begin{proof}[\pf{i:5.5.3}]
  Suppose for sake of contradiction that there exists an \(M \in \R\) such that \(M\) is an upper bound for \(\R^+\).
  Then we have \(x \leq M\) for all \(x \in \R^+\).
  Since \(x > 0\), by \cref{i:5.4.7} we have \(M > 0\), thus \(M \in \R^+\).
  But this means \(M + 1 \in \R^+\), and we must have \(M > M + 1\), a contradiction.
  Thus such \(M\) does not exist and \(\R^+\) has no upper bounds in \(\R\).
\end{proof}

\begin{ac}\label{i:ac:5.5.1}
  Let \(x, y \in \R\).
  We define the following eight subsets of \(\R\):
  \begin{align*}
    \R_{\leq x}   & \coloneqq \set{r \in \R : r \leq x};        & \R_{< x}   & \coloneqq \set{r \in \R : r < x};     & \R^+ & \coloneqq \R_{> 0}; \\
    \R_{\geq x}   & \coloneqq \set{r \in \R : r \geq x};        & \R_{> x}   & \coloneqq \set{r \in \R : r > x};     & \R^- & \coloneqq \R_{< 0}; \\
    \R_{x \leq y} & \coloneqq \set{r \in \R : x \leq r \leq y}; & \R_{x < y} & \coloneqq \set{r \in \R : x < r < y}. &      &
  \end{align*}
\end{ac}

\begin{eg}\label{i:5.5.4}
  Let \(\emptyset\) be the empty set.
  Then every number \(M\) is an upper bound for \(\emptyset\), because \(M\) is greater than every element of the empty set
  (this is a vacuously true statement, but still true).
\end{eg}

\begin{note}
  It is clear that if \(M\) is an upper bound of \(E\), then any larger number \(M' \geq M\) is also an upper bound of \(E\).
  On the other hand, it is not so clear whether it is also possible for any number smaller than \(M\) to also be an upper bound of \(E\).
  This motivates the \cref{i:5.5.5}.
\end{note}

\begin{defn}[Least upper bound]\label{i:5.5.5}
  Let \(E\) be a subset of \(\R\), and \(M\) be a real number.
  We say that \(M\) is a \emph{least upper bound} for \(E\) iff
  \begin{enumerate}
    \item \(M\) is an upper bound for \(E\), and also
    \item any other upper bound \(M'\) for \(E\) must be larger than or equal to \(M\).
  \end{enumerate}
\end{defn}

\setcounter{thm}{6}
\begin{eg}\label{i:5.5.7}
  The empty set does not have a least upper bound.
\end{eg}

\begin{proof}[\pf{i:5.5.7}]
  Suppose for sake of contradiction that there exists an \(M \in \R\) such that \(M\) is a least upper bound of \(\emptyset\).
  By \cref{i:5.5.5} we know that \(x \leq M\) for all \(x \in \emptyset\).
  But by \cref{i:5.5.4} we know that \(M - 1\) is also a upper bound of \(\emptyset\), so by \cref{i:5.5.5} we have \(M < M - 1\), a contradiction.
  Thus \(\emptyset\) does not have a least upper bound.
\end{proof}

\begin{prop}[Uniqueness of least upper bound]\label{i:5.5.8}
  Let \(E\) be a subset of \(\R\).
  Then \(E\) can have at most one least upper bound.
\end{prop}

\begin{proof}[\pf{i:5.5.8}]
  Let \(M_1\) and \(M_2\) be two least upper bounds of \(E\).
  Since \(M_1\) is a least upper bound and \(M_2\) is an upper bound, then by definition of least upper bound we have \(M_2 \geq M_1\).
  Since \(M_2\) is a least upper bound and \(M_1\) is an upper bound, we similarly have \(M_1 \geq M_2\).
  Thus \(M_1 = M_2\).
  Thus there is at most one least upper bound.
\end{proof}

\begin{thm}[Existence of least upper bound]\label{i:5.5.9}
  Let \(E\) be a non-empty subset of \(\R\).
  If \(E\) has an upper bound (i.e., \(E\) has some upper bound \(M\)), then it must have exactly one least upper bound.
\end{thm}

\begin{proof}[\pf{i:5.5.9}]
  Let \(E\) be a non-empty subset of \(\R\) with an upper bound \(M\).
  By \cref{i:5.5.8}, we know that \(E\) has at most one least upper bound;
  we have to show that \(E\) has at least one least upper bound.
  Since \(E\) is non-empty, we can choose some element \(x_0\) in \(E\).

  Let \(n \in \Z^+\).
  We know that \(E\) has an upper bound \(M\).
  By the Archimedean property (\cref{i:5.4.13}), we can find a \(K \in \Z^+\) such that \(K / n \geq M\), and hence \(K / n\) is also an upper bound for \(E\).
  (Note that \(K\) is positive, and \(M\) can be either zero or negative, but \(K / n\) is positive, so we are fine.)
  By the Archimedean property again, there exists another \(L \in \Z\) such that \(L / n < x_0\).
  (Note that if \(x_0 \geq 0\), then we can set \(L = -1\); if \(x_0 < 0\), then \(-x_0\) is positive, so by Archimedean property we have some \(-L \in \Z^+\) such that \(-L / n > -x_0\).)
  Since \(x_0\) lies in \(E\), we see that \(L / n\) is not an upper bound for \(E\).
  Since \(K / n\) is an upper bound but \(L / n\) is not, we see that \(K > L\).

  Since \(K / n\) is an upper bound for \(E\) and \(L / n\) is not, we can find an integer \(L < m_n \leq K\) with the property that \(m_n / n\) is an upper bound for \(E\), but \((m_n - 1) / n\) is not (see \cref{i:ex:5.5.2}).
  In fact, this integer \(m_n\) is unique (\cref{i:ex:5.5.3}).
  We subscript \(m_n\) by \(n\) to emphasize the fact that this integer \(m\) depends on the choice of \(n\).
  This gives a well-defined (and unique) sequence \(m_1, m_2, m_3, \dots\) of integers, with each of the \(m_n / n\) being upper bounds and each of the \((m_n - 1) / n\) not being upper bounds.

  Now let \(N \in \Z^+\), and let \(n, n' \in \Z_{\geq N}\).
  Since \(m_n / n\) is an upper bound for \(E\) and \((m_{n'} - 1) / n'\) is not, by \cref{i:5.5.1} we must have \(m_n / n > (m_{n'} - 1) / n'\).
  After a little algebra, this implies that
  \[
    \dfrac{m_n}{n} - \dfrac{m_{n'}}{n'} > -\dfrac{1}{n'} \geq -\dfrac{1}{N}.
  \]
  Similarly, since \(m_{n'} / n'\) is an upper bound for \(E\) and \((m_n - 1) / n\) is not, we have \(m_{n'} / n' > (m_n - 1) / n\), and hence
  \[
    \dfrac{m_n}{n} - \dfrac{m_{n'}}{n'} < \dfrac{1}{n} \leq \dfrac{1}{N}.
  \]
  Putting these two bounds together, we see that
  \[
    \abs{\dfrac{m_n}{n} - \dfrac{m_{n'}}{n'}} \leq \dfrac{1}{N} \text{ for all } n, n' \geq N \geq 1.
  \]
  This implies that \(\dfrac{m_n}{n}\) is a Cauchy sequence (\cref{i:ex:5.5.4}).
  Since the \(\dfrac{m_n}{n}\) are rational numbers, we can now define the real number \(S\) as
  \[
    S \coloneqq \LIM_{n \to \infty} \dfrac{m_n}{n}.
  \]
  From \cref{i:ex:5.3.5} we conclude that
  \[
    S = \LIM_{n \to \infty} \dfrac{m_n - 1}{n}.
  \]
  To finish the proof of the theorem, we need to show that \(S\) is the least upper bound for \(E\).
  First we show that it is an upper bound.
  Let \(x\) be any element of \(E\).
  Then, since \(m_n / n\) is an upper bound for \(E\), we have \(x \leq m_n / n\) for all \(n \in \Z^+\).
  Applying \cref{i:ex:5.4.8}, we conclude that \(x \leq \LIM_{n \to \infty} m_n / n = S\).
  Thus \(S\) is indeed an upper bound for \(E\).

  Now we show it is a least upper bound.
  Suppose \(y\) is an upper bound for \(E\).
  Since \((m_n - 1) / n\) is not an upper bound, we conclude that \(y \geq (m_n - 1) / n\) for all \(n \in \Z^+\).
  Applying \cref{i:ex:5.4.8}, we conclude that \(y \geq \LIM_{n \to \infty} (m_n - 1) / n = S\).
  Thus the upper bound \(S\) is less than or equal to every upper bound of \(E\), and \(S\) is thus a least upper bound of \(E\).
\end{proof}

\begin{defn}[Supremum]\label{i:5.5.10}
  Let \(E\) be a subset of the real numbers.
  If \(E\) is non-empty and has some upper bound, we define \(\sup(E)\) to be the least upper bound of \(E\)
  (this is well-defined by \cref{i:5.5.9}).
  We introduce two additional symbols, \(+\infty\) and \(-\infty\).
  If \(E\) is non-empty and has no upper bound, we set \(\sup(E) \coloneqq +\infty\);
  if \(E\) is empty, we set \(\sup(E) \coloneqq -\infty\).
  We refer to \(\sup(E)\) as the \emph{supremum} of \(E\), and also denote it by \(\sup E\).
\end{defn}

\begin{rmk}\label{i:5.5.11}
  At present, \(+\infty\) and \(-\infty\) are meaningless symbols;
  we have no operations on them at present, and none of our results involving real numbers apply to \(+\infty\) and \(-\infty\), because these are not real numbers.
  In \cref{i:sec:6.2} we add \(+\infty\) and \(-\infty\) to the reals to form the \emph{extended real number system}, but this system is not as convenient to work with as the real number system, because many of the laws of algebra break down.
  For instance, it is not a good idea to try to define \(+\infty + -\infty\);
  setting this equal to \(0\) causes some problems.
\end{rmk}

\begin{prop}\label{i:5.5.12}
  There exists a positive real number \(x\) such that \(x^2 = 2\).
\end{prop}

\begin{proof}[\pf{i:5.5.12}]
  Let \(E\) be the set \(\set{y \in \R : y \geq 0 \text{ and } y^2 < 2}\);
  thus \(E\) is the set of all non-negative real numbers whose square is less than \(2\).
  Observe that \(E\) has an upper bound of \(2\) (because if \(y > 2\), then \(y^2 > 4 > 2\) and hence \(y \notin E\)).
  Also, \(E\) is non-empty (for instance, \(1\) is an element of \(E\)).
  Thus by the least upper bound property (\cref{i:5.5.9}), we have a real number \(x \coloneqq \sup(E)\) which is the least upper bound of \(E\).
  Then \(x\) is greater than or equal to \(1\) (since \(1 \in E\)) and less than or equal to \(2\)
  (since \(2\) is an upper bound for \(E\)).
  So \(x\) is positive.
  Now we show that \(x^2 = 2\).

  We argue this by contradiction.
  We show that both \(x^2 < 2\) and \(x^2 > 2\) lead to contradictions.
  First suppose that \(x^2 < 2\).
  Let \(\varepsilon \in \Q_{0 < 1}\) be a small number;
  then we have
  \[
    (x + \varepsilon)^2 = x^2 + 2\varepsilon x + \varepsilon^2 \leq x^2 + 4\varepsilon + \varepsilon = x^2 + 5\varepsilon
  \]
  since \(x \leq 2\) and \(\varepsilon^2 \leq \varepsilon\).
  Since \(x^2 < 2\), we see that we can use the Archimedean property (\cref{i:5.4.13}) to choose an \(\varepsilon \in \Q_{0 < 1}\) such that \(x^2 + 5\varepsilon < 2\), thus \((x + \varepsilon)^2 < 2\).
  By construction of \(E\), this means that \(x + \varepsilon \in E\);
  but this contradicts the fact that \(x\) is an upper bound of \(E\).

  Now suppose that \(x^2 > 2\).
  Let \(\varepsilon \in \Q_{0 < 1}\) be a small number;
  then we have
  \[
    (x - \varepsilon)^2 = x^2 - 2\varepsilon x + \varepsilon^2 \geq x^2 - 2\varepsilon x \geq x^2 - 4\varepsilon
  \]
  since \(x \leq 2\) and \(\varepsilon^2 \geq 0\).
  Since \(x^2 > 2\), we can choose \(\varepsilon \in \Q_{0 < 1}\) such that \(x^2 - 4\varepsilon > 2\), and thus \((x - \varepsilon)^2 > 2\).
  But then this implies that \(x - \varepsilon \geq y\) for all \(y \in E\).
  (Why? If \(x - \varepsilon < y\) then \((x - \varepsilon)^2 < y^2 \leq 2\), a contradiction.)
  Thus \(x - \varepsilon\) is an upper bound for \(E\), which contradicts the fact that \(x\) is the \emph{least} upper bound of \(E\).
  From these two contradictions we see that \(x^2 = 2\), as desired.
\end{proof}

\begin{rmk}\label{i:5.5.13}
  Comparing \cref{i:5.5.12} with \cref{i:4.4.4}, we see that certain numbers are real but not rational.
  The proof of \cref{i:5.5.12} also shows that the rationals \(\Q\) do not obey the least upper bound property, otherwise one could use that property to construct a square root of \(2\), which by \cref{i:4.4.4} is not possible.
\end{rmk}

\begin{rmk}\label{i:5.5.14}
  In \cref{i:ch:6} we will use the least upper bound property to develop the theory of limits, which allows us to do many more things than just take square roots.
\end{rmk}

\begin{rmk}\label{i:5.5.15}
  We can of course talk about lower bounds, and greatest lower bounds, of sets \(E\);
  the greatest lower bound of a set \(E\) is also known as the \emph{infimum} of \(E\) and is denoted \(\inf(E)\) or \(\inf E\).
  Everything we say about suprema has a counterpart for infima;
  A precise relationship between the two notions is given by \cref{i:ex:5.5.1}.
  See also \cref{i:sec:6.2}.
\end{rmk}

\begin{note}
  Supremum means ``highest'' and infimum means ``lowest'', and the plurals are suprema and infima.
  Supremum is to superior, and infimum to inferior, as maximum is to major, and minimum to minor.
  The root words are ``super'', which means ``above'', and ``infer'', which means ``below''
  (this usage only survives in a few rare English words such as ``infernal'', with the Latin prefix ``sub'' having mostly replaced ``infer'' in English).
\end{note}

\exercisesection

\begin{ex}\label{i:ex:5.5.1}
  Let \(E\) be a subset of the real numbers \(\R\), and suppose that \(E\) has a least upper bound \(M\) which is a real number, i.e., \(M = \sup(E)\).
  Let \(-E\) be the set
  \[
    -E \coloneqq \set{-x : x \in E}.
  \]
  Show that \(-M\) is the greatest lower bound of \(-E\), i.e., \(-M = \inf(-E)\).
\end{ex}

\begin{proof}[\pf{i:ex:5.5.1}]
  We first show that \(-M\) is a lower bound for \(-E\).
  This is true since
  \begin{align*}
             & \forall x \in E, x \leq M           &  & \by{i:5.5.1}  \\
    \implies & \forall x \in E, -x \geq -M         &  & \by{i:5.4.7}  \\
    \implies & \forall -x \in -E, -x \geq -M                          \\
    \implies & -M \text{ is a lower bound of } -E. &  & \by{i:5.5.15}
  \end{align*}

  Next we show that \(-M\) is a greatest lower bound for \(-E\).
  Let \(L \in \R\) be any lower bound for \(-E\).
  Then we have
  \begin{align*}
             & \forall x \in E, L \leq -x                                      \\
    \implies & \forall x \in E, x \leq -L                   &  & \by{i:5.4.7}  \\
    \implies & -L \text{ is an upper bound of } E           &  & \by{i:5.5.1}  \\
    \implies & M \leq -L                                    &  & \by{i:5.5.5}  \\
    \implies & -M \geq L                                    &  & \by{i:5.4.7}  \\
    \implies & -M \text{ is a greatest lower bound of } -E. &  & \by{i:5.5.15}
  \end{align*}

  Now we show that the greatest lower bound is unique.
  Let \(M, M'\) be two greatest lower bounds of \(-E\).
  Then we have \(M \leq M'\) and \(M \geq M'\), which means \(M = M'\).
  So the greatest lower bound is unique.
\end{proof}

\begin{ex}\label{i:ex:5.5.2}
  Let \(E\) be a non-empty subset of \(\R\), let \(n \in \Z^+\), and let \(L < K\) be integers.
  Suppose that \(K / n\) is an upper bound for \(E\), but that \(L / n\) is not an upper bound for \(E\).
  Without using \cref{i:5.5.9}, show that there exists an integer \(L < m \leq K\) such that \(m / n\) is an upper bound for \(E\), but that \((m - 1) / n\) is not an upper bound for \(E\).
\end{ex}

\begin{proof}[\pf{i:ex:5.5.2}]
  Let \(d = K - L\), so \(d\) is positive by \cref{i:4.1.11}(a).
  Now we induct on \(d\) to show that for every \(d \in \Z^+\), there exists an \(m \in \Z_{> L} \cap \Z_{\leq K}\) such that \(m / n\) is an upper bound for \(E\), but that \((m - 1) / n\) is not.
  We start with \(d = 1\).
  For \(d = 1\), we have \(K - 1 = L\).
  Then let \(m = K\).
  So by hypothesis we have \(m = K \in \Z_{> L} \cap \Z_{\leq K}\), \(m / n = K / n\) is an upper bound for \(E\), and \((m - 1) / n = (K - 1) / n = L / n\) is not an upper bound for \(E\).
  Thus the base case holds.
  Suppose inductively that the statement holds for some \(d \in \Z^+\).
  Now we show that for \(d + 1\) the statement is also true.
  So suppose that for some \(K, L \in \Z\), we have \(K - L = d + 1\), \(K / n\) is an upper bound of \(E\), but \(L / n\) is not.
  Since \(K / n\) is an upper bound for \(E\), we can ask whether \((K - 1) / n\) is an upper bound for \(E\).
  \begin{itemize}
    \item If \((K - 1) / n\) is not an upper bound for \(E\), then we can choose \(m = K\) and we are done.
    \item If \((K - 1) / n\) is an upper bound for \(E\), then by \cref{i:5.5.1} we have \(L < K - 1\).
          Since \(K - 1 - L = d\), by induction hypothesis we know that there exists an \(m \in \Z_{> L} \cap \Z_{\leq K}\) such that \(m / n\) is an upper bound for \(E\), but \((m - 1) / n\) is not.
  \end{itemize}
  From all cases above we found an \(m \in \Z_{> L} \cap \Z_{\leq K}\) such that \(m / n\) is an upper bound for \(E\), but \((m - 1) / n\) is not.
  This closes the induction.
\end{proof}

\begin{ex}\label{i:ex:5.5.3}
  Let \(E\) be a non-empty subset of \(\R\), let \(n \in \Z^+\), and let \(m, m' \in \Z\) with the properties that \(m / n\) and \(m' / n\) are upper bounds for \(E\), but \((m - 1) / n\) and \((m' - 1) / n\) are not upper bounds for \(E\).
  Show that \(m = m'\).
  This shows that the integer \(m\) constructed in \cref{i:ex:5.5.2} is unique.
\end{ex}

\begin{proof}[\pf{i:ex:5.5.3}]
  Suppose for sake of contradiction that \(m \neq m'\).
  Then by \cref{i:4.1.11}(f) we have either \(m < m'\) or \(m > m'\).
  \begin{itemize}
    \item If \(m < m'\), then we have \(m \leq m' - 1\).
          By \cref{i:4.2.9}(e) we have \(m / n \leq (m' - 1) / n\), which means \((m' - 1) / n\) is an upper bound for \(E\).
          But this contradict to the hypothesis.
    \item If \(m > m'\), then by switching the row of \(m\) and \(m'\) in the previous case we can also derive contradiction.
  \end{itemize}
  From all cases above we derive contradictions.
  Thus we must have \(m = m'\).
\end{proof}

\begin{ex}\label{i:ex:5.5.4}
  Let \((q_n)_{n = 1}^\infty\) be a sequence of rational numbers with the property that \(\abs{q_n - q_{n'}} \leq \dfrac{1}{M}\) whenever \(M \in \Z^+\) and \(n, n' \in \Z_{\geq M}\).
  Show that \((q_n)_{n = 1}^\infty\) is a Cauchy sequence.
  Furthermore, if \(S \coloneqq \LIM_{n \to \infty} q_n\), show that \(\abs{q_M - S} \leq \dfrac{1}{M}\) for every \(M \in \Z^+\).
\end{ex}

\begin{proof}[\pf{i:ex:5.5.4}]
  We first show that \((q_n)_{n = 1}^\infty\) is a Cauchy sequence.
  Let \(\varepsilon \in \Q^+\).
  By Archimedean property (\cref{i:5.4.13}) we know that there exists an \(M \in \Z^+\) such that \(M \varepsilon > 1\).
  By \cref{i:4.2.9}(e) we have \(\varepsilon > 1 / M\).
  But by hypothesis we have \(\abs{q_n - q_{n'}} \leq \dfrac{1}{M} < \varepsilon\) for all \(n, n' \in \Z_{\geq M}\).
  Since \(\varepsilon\) was arbitrary, by \cref{i:5.1.8} this means \((q_n)_{n = 1}^{\infty}\) is a Cauchy sequence.

  Now we show that if \(S = \LIM_{n \to \infty} q_n\), then \(\abs{q_M - S} \leq \dfrac{1}{M}\) for every \(M \in \Z^+\).
  From the proof above we know that \((q_n)_{n = 1}^\infty\) is a Cauchy sequence, thus \(S = \LIM_{n \to \infty} q_n\) is well-defined.
  By hypothesis we have \(\abs{q_M - q_n} \leq 1 / M\) for all \(M \in \Z^+\) and for all \(n \geq M\).
  Then we have
  \begin{align*}
             & \forall M \in \Z^+, \forall n \in \Z_{\geq M}, \abs{q_M - q_n} \leq 1 / M                                  \\
    \implies & \forall M \in \Z^+, \forall n \in \Z_{\geq M}, \abs{q_n - q_M} \leq 1 / M             &  & \by{i:4.3.1}    \\
    \implies & \forall M \in \Z^+, \forall n \in \Z_{\geq M}, -1 / M \leq q_n - q_M \leq 1 / M       &  & \by{i:4.3.3}[c] \\
    \implies & \forall M \in \Z^+, \forall n \in \Z_{\geq M}, -1 / M + q_M \leq q_n \leq 1 / M + q_M &  & \by{i:4.2.9}[d] \\
    \implies & \forall M \in \Z^+, -1 / M + q_M \leq S \leq 1 / M + q_M                              &  & \by{i:ex:5.4.8} \\
    \implies & \forall M \in \Z^+, -1 / M \leq S - q_M \leq 1 / M                                    &  & \by{i:5.4.7}[d] \\
    \implies & \forall M \in \Z^+, \abs{S - q_M} \leq 1 / M                                          &  & \by{i:ex:5.4.6} \\
    \implies & \forall M \in \Z^+, \abs{q_M - S} \leq 1 / M.                                         &  & \by{i:5.4.5}
  \end{align*}
\end{proof}

\begin{ex}\label{i:ex:5.5.5}
  Establish an analogue of \cref{i:5.4.14}, in which ``rational'' is replaced by ``irrational.''
\end{ex}

\begin{proof}[\pf{i:ex:5.5.5}]
  Let \(x, y, z \in \R\) where \(x < y\) and \(z^2 = 2\).
  (\(z\) is well-defined thanks to \cref{i:5.5.12}.)
  So by \cref{i:5.4.7} we have \(x - z < y - z\).
  But by \cref{i:5.4.14} there exists a \(q \in \Q\) such that \(x - z < q < y - z\).
  So by \cref{i:5.4.7} again we have \(x < q + z < y\).
  Because \(z\) is irrational, \(q + z\) is also irrational
  (otherwised we have \(a = q + z \in \Q\) and \(z = a - q \in \Q\), contradicts to \cref{i:4.4.4}).
  So we have an irrational number in between any two real numbers \(x, y\) where \(x < y\).
\end{proof}

\section{Real exponentiation, part I}\label{i:sec:5.6}

\begin{defn}[Exponentiating a real by a natural number]\label{i:5.6.1}
  Let \(x\) be a real number.
  To raise \(x\) to the power \(0\), we define \(x^0 \coloneqq 1\).
  Now suppose recursively that \(x^n\) has been defined for some natural number \(n\), then we define \(x^{n + 1} \coloneqq x^n \times x\).
\end{defn}

\begin{defn}[Exponentiating a real by an integer]\label{i:5.6.2}
  Let \(x\) be a non-zero real number.
  Then for any negative integer \(-n\), we define \(x^{-n} \coloneqq 1 / x^n\).
\end{defn}

\begin{prop}\label{i:5.6.3}
  All the properties in \cref{i:4.3.10,i:4.3.12} remain valid if \(x\) and \(y\) are assumed to be real numbers instead of rational numbers.
\end{prop}

\begin{meta-proof}[\pf{i:5.6.3}]
If one inspects the proof of \cref{i:4.3.10,i:4.3.12} we see that they rely on the laws of algebra and the laws of order for the rationals (\cref{i:4.2.4,i:4.2.9}).
But by \cref{i:5.3.11,i:5.4.7}, and the identity \(xx^{-1} = x^{-1} x = 1\) we know that all these laws of algebra and order continue to hold for real numbers as well as rationals.
Thus we can modify the proof of \cref{i:4.3.10,i:4.3.12} to hold in the case when \(x\) and \(y\) are real.
\end{meta-proof}

\begin{note}
  Instead of giving an actual proof of \cref{i:5.6.3}, we shall give a meta-proof
  (an argument appealing to the nature of proofs, rather than the nature of real and rational numbers).
\end{note}

\begin{defn}\label{i:5.6.4}
  Let \(x \geq 0\) be a non-negative real, and let \(n \geq 1\) be a positive integer.
  We define \(x^{1 / n}\), also known as the \emph{\(n^{\opTh}\) root of \(x\)}, by the formula
  \[
    x^{1 / n} \coloneqq \sup\set{y \in \R : y \geq 0 \text{ and } y^n \leq x}.
  \]
  We often write \(\sqrt{x}\) for \(x^{1 / 2}\).
\end{defn}

\begin{note}
  we do not define the \(n^{\opTh}\) roots of a negative number.
  In fact, we will leave the \(n^{\opTh}\) roots of negative numbers undefined for the rest of the text
  (one can define these \(n^{\opTh}\) roots once one defines the complex numbers, but we shall refrain from doing so).
\end{note}

\begin{lem}[Existence of \(n^{\opTh}\) roots]\label{i:5.6.5}
  Let \(x \geq 0\) be a non-negative real, and let \(n \geq 1\) be a positive integer.
  Then the set \(E \coloneqq \set{y \in R : y \geq 0 \text{ and } y^n \leq x}\) is non-empty and is also bounded above.
  In particular, \(x^{1 / n}\) is a real number.
\end{lem}

\begin{proof}[\pf{i:5.6.5}]
  The set \(E\) contains \(0\), so it is certainly not empty.
  Now we show it has an upper bound.
  We divide into two cases: \(x \leq 1\) and \(x > 1\).
  First suppose that we are in the case where \(x \leq 1\).
  Then we claim that the set \(E\) is bounded above by \(1\).
  To see this, suppose for sake of contradiction that there was an element \(y \in E\) for which \(y > 1\).
  But then \(y^n > 1\), and hence \(y^n > x\), a contradiction.
  Thus \(E\) has an upper bound.
  Now suppose that we are in the case where \(x > 1\).
  Then we claim that the set \(E\) is bounded above by \(x\).
  To see this, suppose for contradiction that there was an element \(y \in E\) for which \(y > x\).
  Since \(x > 1\), we thus have \(y > 1\).
  Since \(y > x\) and \(y > 1\), we have \(y^n > x\), a contradiction.
  Thus in both cases \(E\) has an upper bound, and so \(x^{1 / n}\) is finite.
\end{proof}

\begin{lem}\label{i:5.6.6}
  Let \(x, y \geq 0\) be non-negative reals, and let \(n, m \geq 1\) be positive integers.
  \begin{enumerate}
    \item If \(y = x^{1 / n}\), then \(y^n = x\).
    \item Conversely, if \(y^n = x\), then \(y = x^{1 / n}\).
    \item \(x^{1 / n}\) is a non-negative real number, and is positive iff \(x\) is positive.
    \item We have \(x > y\) iff \(x^{1 / n} > y^{1 / n}\).
    \item Let \(k, l \in \Z^+\).
          If \(x > 1\), then \(x^{1 / k}\) is a decreasing (i.e., \(x^{1 / k} > x^{1 / l}\) whenever \(k < l\)) function of \(k\).
          If \(0 < x < 1\), then \(x^{1 / k}\) is an increasing (i.e., \(x^{1 / k} < x^{1 / l}\) whenever \(k < l\)) function of \(k\).
          If \(x = 1\), then \(x^{1 / k} = 1\) for all \(k\).
    \item We have \((xy)^{1 / n} = x^{1 / n} y^{1 / n}\).
    \item We have \((x^{1 / n})^{1 / m} = x^{1 / nm}\).
  \end{enumerate}
\end{lem}

\begin{proof}[\pf{i:5.6.6}(a)]
  Let \(E = \set{z \in \R : (z \geq 0) \land (z^n \leq x)}\) and let \(y = x^{1 / n} = \sup(E)\).
  Since \(0 \in E\), by \cref{i:5.5.5} we know that \(0 \leq y\).
  Suppose for sake of contradiction that \(y^n \neq x\).
  Then by \cref{i:5.4.7} exactly one of the following statements is true:
  \begin{itemize}
    \item \(y^n < x\).
          Now we show that \(\exists m \in \Z^+\) such that \((y + \dfrac{1}{m})^n < x\).
          Suppose for sake of contradiction that \(\forall m \in \Z^+\), we have \((y + \dfrac{1}{m})^n \geq x\).
          Let \((y + \dfrac{1}{m})_{m = 1}^\infty\) be a sequences.
          Then we have
          \begin{align*}
            \LIM_{m \to \infty} \bigg(y + \dfrac{1}{m}\bigg) & = \LIM_{m \to \infty} y + \LIM_{m \to \infty} \dfrac{1}{m} &  & \by{i:5.3.4}    \\
                                                             & = \LIM_{m \to \infty} y + 0                                &  & \by{i:ex:5.3.5} \\
                                                             & = y.                                                       &  & \by{i:6.1.15}
          \end{align*}
          Note that we use \cref{i:6.1.15} without circularity.
          But then we have
          \begin{align*}
                     & \forall m \in \Z^+, y^n < x \leq \bigg(y + \dfrac{1}{m}\bigg)^n                                  \\
            \implies & y^n < x \leq \LIM_{m \to \infty} \bigg(y + \dfrac{1}{m}\bigg)^n             &  & \by{i:ex:5.4.8} \\
            \implies & y^n < x \leq \Bigg(\LIM_{m \to \infty} \bigg(y + \dfrac{1}{m}\bigg)\Bigg)^n &  & \by{i:5.3.9}    \\
            \implies & y^n < x \leq y^n                                                                                 \\
            \implies & y^n < y^n,
          \end{align*}
          a contradiction.
          Thus we know that \(\exists m \in \Z^+\) such that \((y + \dfrac{1}{m})^n < x\).
          Since \(m \in \Z^+\), we know that \(y < y + \dfrac{1}{m}\), thus by \cref{i:5.6.3} we know that \(y^n < (y + \dfrac{1}{m})^n\).
          But this means \(y + \dfrac{1}{m} \in E\) and \(y + \dfrac{1}{m} \leq y\), a contradiction.
    \item \(y^n > x\).
          Now we show that \(\exists m \in \Z^+\) such that \((y - \dfrac{1}{m})^n > x\).
          Suppose for sake of contradiction that \(\forall m \in \Z^+\), we have \((y - \dfrac{1}{m})^n \leq x\).
          Let \((y - \dfrac{1}{m})_{m = 1}^\infty\) be a sequences.
          Then we have
          \begin{align*}
            \LIM_{m \to \infty} \bigg(y - \dfrac{1}{m}\bigg) & = \LIM_{m \to \infty} y - \LIM_{m \to \infty} \dfrac{1}{m} &  & \by{i:5.3.4}    \\
                                                             & = \LIM_{m \to \infty} y - 0                                &  & \by{i:ex:5.3.5} \\
                                                             & = y.                                                       &  & \by{i:6.1.15}
          \end{align*}
          Note that we use \cref{i:6.1.15} without circularity.
          But then we have
          \begin{align*}
                     & \forall m \in \Z^+, \bigg(y - \dfrac{1}{m}\bigg)^n \leq x < y^n                                  \\
            \implies & \LIM_{m \to \infty} \bigg(y - \dfrac{1}{m}\bigg)^n \leq x < y^n             &  & \by{i:ex:5.4.8} \\
            \implies & \Bigg(\LIM_{m \to \infty} \bigg(y - \dfrac{1}{m}\bigg)\Bigg)^n \leq x < y^n &  & \by{i:5.3.9}    \\
            \implies & y^n \leq x < y^n                                                                                 \\
            \implies & y^n < y^n,
          \end{align*}
          a contradiction.
          Thus we know that \(\exists m \in \Z^+\) such that \((y - \dfrac{1}{m})^n > x\).
          Since \(m \in \Z^+\), we know that \(y - \dfrac{1}{m} < y\), thus by \cref{i:5.6.3} we know that \((y - \dfrac{1}{m})^n < y^n\).
          But this means \(y - \dfrac{1}{m} \notin E\) and \(y - \dfrac{1}{m}\) and upper bound for \(E\) which is strictly less than \(y\), a contradiction.
  \end{itemize}
  From all cases above we get contradictions, so \(y = x^{1 / n} \implies y^n = x\).
\end{proof}

\begin{proof}[\pf{i:5.6.6}(b)]
  Let \(E = \set{z \in \R : (z \geq 0) \land (z^n \leq x)}\).
  By \cref{i:5.6.4} we have \(x^{1 / n} = \sup(E)\).
  Since \(0 \in E\), by \cref{i:5.5.5} we know that \(0 \leq x^{1 / n}\).
  Let \(y \in \R \setminus \R^-\) such that \(y^n = x\).
  Such \(y\) is well-defined since \cref{i:5.6.6}(a).
  By the definition of \(E\) we know that \(y \in E\).
  Suppose for sake of contradiction that \(y \neq x^{1 / n}\).
  Then by \cref{i:5.4.7} exactly one of the following statements is true:
  \begin{itemize}
    \item \(y < x^{1 / n}\).
          But then we have
          \begin{align*}
                     & y^n < (x^{1 / n})^n &  & \by{i:5.6.3}    \\
            \implies & x < (x^{1 / n})^n                        \\
            \implies & x < x,              &  & \by{i:5.6.6}[a]
          \end{align*}
          a contradiction.
    \item \(y > x^{1 / n}\).
          But then we have
          \begin{align*}
                     & y^n > (x^{1 / n})^n &  & \by{i:5.6.3}    \\
            \implies & x > (x^{1 / n})^n                        \\
            \implies & x > x,              &  & \by{i:5.6.6}[a]
          \end{align*}
          a contradiction.
  \end{itemize}
  From all cases above we get contradictions, so \(y^n = x \implies y = x^{1 / n}\).
\end{proof}

\begin{proof}[\pf{i:5.6.6}(c)]
  Let \(E = \set{z \in \R : (z \geq 0) \land (z^n \leq x)}\).
  By \cref{i:5.6.4} we have \(x^{1 / n} = \sup(E)\).
  Since \(0 \in E\), by \cref{i:5.5.5} we know that \(0 \leq x^{1 / n}\), thus \(x^{1 / n}\) is non-negative real number.

  Now suppose that \(x^{1 / n}\) is positive.
  Then we have
  \begin{align*}
             & x^{1 / n} > 0                          \\
    \implies & (x^{1 / n})^n > 0 &  & \by{i:5.6.3}    \\
    \implies & x > 0.            &  & \by{i:5.6.6}[a] \\
  \end{align*}

  Finally, suppose that \(x\) is positive.
  Suppose for sake of contradiction that \(x^{1 / n}\) is not positive.
  Then from proof above we know that \(x^{1 / n} = 0\).
  But then we have
  \begin{align*}
             & (x^{1 / n})^n = 0^n = 0                      \\
    \implies & x = 0,                  &  & \by{i:5.6.6}[a]
  \end{align*}
  a contradiction.
  Thus we must have \(x^{1 / n} > 0\).
  And we conclude that \(x^{1 / n}\) is positive iff \(x\) is positive.
\end{proof}

\begin{proof}[\pf{i:5.6.6}(d)]
  We first show that \(x^{1 / n} > y^{1 / n} \implies x > y\).
  \begin{align*}
             & x^{1 / n} > y^{1 / n}                              \\
    \implies & (x^{1 / n})^n > (y^{1 / n})^n &  & \by{i:5.6.3}    \\
    \implies & x > y.                        &  & \by{i:5.6.6}[a]
  \end{align*}

  Now we show that \(x > y \implies x^{1 / n} > y^{1 / n}\).
  Suppose for sake of contradiction that \(x^{1 / n} \leq y^{1 / n}\).
  But then we have
  \begin{align*}
             & x^{1 / n} \leq y^{1 / n}                              \\
    \implies & (x^{1 / n})^n \leq (y^{1 / n})^n &  & \by{i:5.6.3}    \\
    \implies & x \leq y,                        &  & \by{i:5.6.6}[a]
  \end{align*}
  a contradiction.
  Thus we must have \(x^{1 / n} > y^{1 / n}\).
  And we conclude that \(x > y \iff x^{1 / n} > y^{1 / n}\).
\end{proof}

\begin{proof}[\pf{i:5.6.6}(e)]
  If \(x = 0\), then by \cref{i:5.6.6}(c) we know that \(x^{1 / k} = 0\) for every \(k \in \Z^+\).
  Thus we only consider the case \(x \in \R^+\).

  We first show that if \(x > 1\), then \(x^{1 / k}\) is a decreasing function of \(k\).
  Let \(f : \Z^+ \to \R\) be a function such that \(f(k) = x^{1 / k}\).
  Such function \(f\) is well-defined since \cref{i:5.6.5}.
  Now we want to show that \(x^{1 / k} > x^{1 / (k + 1)}\).
  Suppose for sake of contradiction that \(x^{1 / k} \leq x^{1 / (k + 1)}\).
  But then we have
  \begin{align*}
             & x^{\dfrac{1}{k}} \leq x^{\dfrac{1}{k + 1}}                                                                                                              \\
    \implies & (x^{\dfrac{1}{k}})^k \leq (x^{\dfrac{1}{k + 1}})^k                                                                               &  & \by{i:5.6.3}      \\
    \implies & x \leq (x^{\dfrac{1}{k + 1}})^k                                                                                                  &  & \by{i:5.6.6}[a]   \\
    \implies & (x^{\dfrac{1}{k + 1}})^{k + 1} \leq (x^{\dfrac{1}{k + 1}})^k                                                                     &  & \by{i:5.6.6}[a,b] \\
    \implies & (x^{\dfrac{1}{k + 1}})^{k + 1} \cdot (x^{\dfrac{1}{k + 1}})^{-1} \leq (x^{\dfrac{1}{k + 1}})^k \cdot (x^{\dfrac{1}{k + 1}})^{-1} &  & \by{i:5.6.6}[c]   \\
    \implies & (x^{\dfrac{1}{k + 1}})^{k + 1} \cdot (x^{\dfrac{1}{k + 1}})^{-k} \leq (x^{\dfrac{1}{k + 1}})^k \cdot (x^{\dfrac{1}{k + 1}})^{-k} &  & \by{i:5.6.3}      \\
    \implies & x^{\dfrac{1}{k + 1}} \leq 1                                                                                                      &  & \by{i:5.6.3}      \\
    \implies & (x^{\dfrac{1}{k + 1}})^{k + 1} \leq 1^{k + 1} = 1                                                                                &  & \by{i:5.6.3}      \\
    \implies & x \leq 1,                                                                                                                        &  & \by{i:5.6.6}[a]
  \end{align*}
  a contradiction.
  Thus we must have \(f(k) = x^{1 / k} > x^{1 / (k + 1)} = f(k + 1)\), and \(f(k) = x^{1 / k}\) is a decreasing function of \(k\).

  Next we show that if \(x < 1\), then \(x^{1 / k}\) is a increasing function of \(k\).
  Let \(f : \Z^+ \to \R\) be a function such that \(f(k) = x^{1 / k}\).
  Such function \(f\) is well-defined since \cref{i:5.6.5}.
  Now we want to show that \(x^{1 / k} < x^{1 / (k + 1)}\).
  Suppose for sake of contradiction that \(x^{1 / k} \geq x^{1 / (k + 1)}\).
  But then we have
  \begin{align*}
             & x^{\dfrac{1}{k}} \geq x^{\dfrac{1}{k + 1}}                                                                                                              \\
    \implies & (x^{\dfrac{1}{k}})^k \geq (x^{\dfrac{1}{k + 1}})^k                                                                               &  & \by{i:5.6.3}      \\
    \implies & x \geq (x^{\dfrac{1}{k + 1}})^k                                                                                                  &  & \by{i:5.6.6}[a]   \\
    \implies & (x^{\dfrac{1}{k + 1}})^{k + 1} \geq (x^{\dfrac{1}{k + 1}})^k                                                                     &  & \by{i:5.6.6}[a,b] \\
    \implies & (x^{\dfrac{1}{k + 1}})^{k + 1} \cdot (x^{\dfrac{1}{k + 1}})^{-1} \geq (x^{\dfrac{1}{k + 1}})^k \cdot (x^{\dfrac{1}{k + 1}})^{-1} &  & \by{i:5.6.6}[c]   \\
    \implies & (x^{\dfrac{1}{k + 1}})^{k + 1} \cdot (x^{\dfrac{1}{k + 1}})^{-k} \geq (x^{\dfrac{1}{k + 1}})^k \cdot (x^{\dfrac{1}{k + 1}})^{-k} &  & \by{i:5.6.3}      \\
    \implies & x^{\dfrac{1}{k + 1}} \geq 1                                                                                                      &  & \by{i:5.6.3}      \\
    \implies & (x^{\dfrac{1}{k + 1}})^{k + 1} \geq 1^{k + 1} = 1                                                                                &  & \by{i:5.6.3}      \\
    \implies & x \geq 1,                                                                                                                        &  & \by{i:5.6.6}[a]
  \end{align*}
  a contradiction.
  Thus we must have \(f(k) = x^{1 / k} < x^{1 / (k + 1)} = f(k + 1)\), and \(f(k) = x^{1 / k}\) is a increasing function of \(k\).

  Finally we show that if \(x = 1\), then \(x^{1 / k} = 1\) for every \(k \in \Z^+\).
  Suppose for sake of contradiction that \(x^{1 / k} \neq 1\).
  Then by \cref{i:5.4.7} exactly one of the following two statements is true:
  \begin{itemize}
    \item \(x^{1 / k} > 1\).
          But then we have
          \begin{align*}
                     & (x^{1 / k})^k > 1^k = 1 &  & \by{i:5.6.3}    \\
            \implies & x > 1,                  &  & \by{i:5.6.6}[a]
          \end{align*}
          a contradiction.
    \item \(x^{1 / k} < 1\).
          But then we have
          \begin{align*}
                     & (x^{1 / k})^k < 1^k = 1 &  & \by{i:5.6.3}    \\
            \implies & x < 1,                  &  & \by{i:5.6.6}[a]
          \end{align*}
          a contradiction.
  \end{itemize}
  From all cases above we get contradictions.
  Thus we must have \(x^{1 / k} = 1\).
\end{proof}

\begin{proof}[\pf{i:5.6.6}(f)]
  We have
  \begin{align*}
    ((xy)^{1 / n})^n & = xy                          &  & \by{i:5.6.6}[a]   \\
                     & = (x^{1 / n})^n (y^{1 / n})^n &  & \by{i:5.6.6}[a,b] \\
                     & = (x^{1 / n} y^{1 / n})^n     &  & \by{i:5.6.3}
  \end{align*}
  and thus by \cref{i:5.6.6}(b) we have \((xy)^{1 / n} = x^{1 / n} y^{1 / n}\).
\end{proof}

\begin{proof}[\pf{i:5.6.6}(g)]
  We have
  \begin{align*}
    (x^{1 / nm})^{nm} & = x                                           &  & \by{i:5.6.6}[a]   \\
                      & = (x^{1 / n})^n                               &  & \by{i:5.6.6}[a,b] \\
                      & = \Big(\big((x^{1 / n})^{1 / m}\big)^m\Big)^n &  & \by{i:5.6.6}[a,b] \\
                      & = \big((x^{1 / n})^{1 / m}\big)^{nm}          &  & \by{i:5.6.3}
  \end{align*}
  and thus by \cref{i:5.6.6}(b) we have \(x^{1 / nm} = (x^{1 / n})^{1 / m}\).
\end{proof}

\begin{note}
  The observant reader may note that this definition of \(x^{1 / n}\) might possibly be inconsistent with our previous notion of \(x^n\) when \(n = 1\), but it is easy to check that \(x^{1 / 1} = x = x^1\) by using \cref{i:5.6.6}, so there is no inconsistency.
\end{note}

\begin{note}
  One consequence of \cref{i:5.6.6}(b) is another proof of the cancellation law from \cref{i:4.3.12}(c) and \cref{i:5.6.3}:
  if \(y\) and \(z\) are positive and \(y^n = z^n\), then \(y = z\).
  This only works when \(y\) and \(z\) are positive;
  for instance, \((-3)^2 = 3^2\), but we cannot conclude from this that \(-3 = 3\).
\end{note}

\begin{defn}\label{i:5.6.7}
  Let \(x > 0\) be a positive real number, and let \(q\) be a rational number.
  To define \(x^q\), we write \(q = a / b\) for some integer \(a\) and positive integer \(b\), and define
  \[
    x^q \coloneqq (x^{1 / b})^a.
  \]
\end{defn}

\begin{note}
  Every rational \(q\), whether positive, negative, or zero, can be written in the form \(a / b\) where \(a\) is an integer and \(b\) is positive.
  However, the rational number \(q\) can be expressed in the form \(a / b\) in more than one way, for instance \(1 / 2\) can also be expressed as \(2 / 4\) or \(3 / 6\).
  So to ensure that \cref{i:5.6.7} is well-defined, we need to check that different expressions \(a / b\) give the same formula for \(x^q\).
\end{note}

\begin{lem}\label{i:5.6.8}
  Let \(a, a'\) be integers and \(b, b'\) be positive integers such that \(a / b = a' / b'\), and let \(x\) be a positive real number.
  Then we have \((x^{1 / b'})^{a'} = (x^{1 / b})^a\).
\end{lem}

\begin{proof}[\pf{i:5.6.8}]
  There are three cases: \(a = 0, a > 0, a < 0\).
  If \(a = 0\), then we must have \(a' = 0\) and so both \((x^{1 / b'})^{a'}\) and \((x^{1 / b})^a\) are equal to 1, so we are done.

  Now suppose that \(a > 0\).
  Then \(a' > 0\), and \(ab' = ba'\).
  Write \(y \coloneqq x^{1 / (ab')} = x^{1 / (ba')}\).
  By \cref{i:5.6.6}(g) we have \(y = (x^{1 / b'})^{1 / a}\) and \(y = (x^{1 / b})^{1 / a'}\);
  by \cref{i:5.6.6}(a) we thus have \(y^{a'} = x^{1 / b}\) and \(y^a = x^{1 / b'}\).
  Thus we have
  \[
    (x^{1 / b'})^{a'} = (y^a)^{a'} = y^{aa'} = (y^{a'})^a = (x^{1 / b})^a
  \]
  as desired.

  Finally, suppose that \(a < 0\).
  Then we have \((-a) / b = (-a') / b'\).
  But \(-a\) is positive, so the previous case applies and we have \((x^{1 / b'})^{-a'} = (x^{1 / b})^{-a}\).
  Taking the reciprocal of both sides we obtain the result.
\end{proof}

\begin{note}
  Thus \(x^q\) is well-defined for every rational \(q\).
  \cref{i:5.6.7} is consistent with our old definition for \(x^{1 / n}\) (since \(x^{1 / n} = (x^{1 / n})^1\)) and is also consistent with our old definition for \(x^n\) (since \(x^n = (x^{1 / 1})^n\)).
\end{note}

\begin{lem}\label{i:5.6.9}
  Let \(x, y > 0\) be positive reals, and let \(q, r\) be rationals.
  \begin{enumerate}
    \item \(x^q\) is a positive real.
    \item \(x^{q + r} = x^q x^r\) and \((x^q)^r = x^{qr}\).
    \item \(x^{-q} = 1 / x^q\).
    \item If \(q > 0\), then \(x > y\) iff \(x^q > y^q\).
    \item If \(x > 1\), then \(x^q > x^r\) iff \(q > r\).
          If \(x < 1\), then \(x^q > x^r\) iff \(q < r\).
    \item \((xy)^q = x^q y^q\).
  \end{enumerate}
\end{lem}

\begin{proof}[\pf{i:5.6.9}(a)]
  Let \(q = a / b\) where \(a \in \Z\) and \(b \in \Z^+\).
  Then we have
  \begin{align*}
             & x \in \R^+                                  \\
    \implies & x^{1 / b} \in \R^+     &  & \by{i:5.6.6}[c] \\
    \implies & (x^{1 / b})^a \in \R^+ &  & \by{i:5.6.3}    \\
    \implies & x^q \in \R^+.          &  & \by{i:5.6.7}
  \end{align*}
\end{proof}

\begin{proof}[\pf{i:5.6.9}(b)]
  Let \(q = a / b\) and \(r = c / d\) where \(a, c \in \Z\) and \(b, d \in \Z^+\).
  Then we have
  \begin{align*}
    x^{q + r} & = x^{(ad + bc) / bd}                  &  & \by{i:4.2.2}                \\
              & = (x^{1 / bd})^{(ad + bc)}            &  & \by{i:5.6.7}                \\
              & = (x^{1 / bd})^{ad} (x^{1 / bd})^{bc} &  & \by{i:5.6.3}                \\
              & = x^{ad / bd} x^{bc / bd}             &  & \text{(by  \cref{i:5.6.7})} \\
              & = x^{a / b} x^{c / d}                 &  & \by{i:5.6.8}                \\
              & = x^q x^r
  \end{align*}
  and
  \begin{align*}
    x^{qr} & = x^{ac / bd}                                                        &  & \by{i:4.2.2}      \\
           & = (x^{1 / bd})^{ac}                                                  &  & \by{i:5.6.7}      \\
           & = \big((x^{1 / b})^{1 / d}\big)^{ac}                                 &  & \by{i:5.6.6}[g]   \\
           & = \bigg(\Big(\big((x^{1 / b})^a\big)^{1 / a}\Big)^{1 / d}\Bigg)^{ac} &  & \by{i:5.6.6}[a,b] \\
           & = \Big(\big((x^{a / b})^{1 / a}\big)^{1 / d}\Big)^{ac}               &  & \by{i:5.6.7}      \\
           & = \big((x^{a / b})^{1 / ad}\big)^{ac}                                &  & \by{i:5.6.6}[g]   \\
           & = (x^{a / b})^{ac / ad}                                              &  & \by{i:5.6.7}      \\
           & = (x^{a / b})^{c / d}                                                &  & \by{i:5.6.8}      \\
           & = (x^q)^r.
  \end{align*}
\end{proof}

\begin{proof}[\pf{i:5.6.9}(c)]
  Let \(q = a / b\) where \(a \in \Z\) and \(b \in \Z^+\).
  Then we have
  \begin{align*}
    x^{-q} & = x^{-a / b}                          \\
           & = (x^{1 / b})^{-a}  &  & \by{i:5.6.7} \\
           & = 1 / (x^{1 / b})^a &  & \by{i:5.6.3} \\
           & = 1 / x^q.          &  & \by{i:5.6.7}
  \end{align*}
\end{proof}

\begin{proof}[\pf{i:5.6.9}(d)]
  Let \(q = a / b\) where \(a \in \Z\) and \(b \in \Z^+\).
  Then we have
  \begin{align*}
             & x > y                                              \\
    \implies & x^{1 / b} > y^{1 / b}         &  & \by{i:5.6.6}[d] \\
    \implies & (x^{1 / b})^a > (y^{1 / b})^a &  & \by{i:5.6.3}    \\
    \implies & x^q > y^q                     &  & \by{i:5.6.7}
  \end{align*}
  and
  \begin{align*}
             & x^q > y^q                                                                                \\
    \implies & (x^{1 / b})^a > (y^{1 / b})^a                                     &  & \by{i:5.6.7}      \\
    \implies & \big((x^{1 / b})^a\big)^{1 / a} > \big((y^{1 / b})^a\big)^{1 / a} &  & \by{i:5.6.6}[d]   \\
    \implies & x^{1 / b} > y^{1 / b}                                             &  & \by{i:5.6.6}[a,b] \\
    \implies & x > y.                                                            &  & \by{i:5.6.6}[d]
  \end{align*}
  Thus we conclude that \(x > y \iff x^q > y^q\) when \(x, y \in \R^+\) and \(q \in \Q^+\).
\end{proof}

\begin{proof}[\pf{i:5.6.9}(e)]
  Let \(q = a / b\) and \(r = c / d\) where \(a, c \in \Z\) and \(b, d \in \Z^+\).
  First suppose that \(x > 1\) and \(x^q > x^r\).
  Then we have
  \begin{align*}
             & x^q > x^r                                                                  \\
    \implies & x^q x^{-r} > x^r x^{-r}                              &  & \by{i:5.6.9}[a]  \\
    \implies & x^{q - r} > x^{r - r} = x^0 = 1                      &  & \by{i:5.6.9}[b]  \\
    \implies & x^{(ad - bc) / bd} > 1                               &  & \by{i:4.2.2}     \\
    \implies & (x^{(ad - bc)})^{1 / bd} > 1                         &  & \by{i:5.6.9}[b]  \\
    \implies & \big((x^{(ad - bc)})^{1 / bd}\big)^{bd} > 1^{bd} = 1 &  & \by{i:5.6.3}     \\
    \implies & x^{(ad - bc)} > 1                                    &  & \by{i:5.6.6}[a]  \\
    \implies & ad - bc > 0                                          &  & \by{i:4.3.12}[b] \\
    \implies & ad > bc                                              &  & \by{i:4.1.11}[b] \\
    \implies & a / b > c / d                                        &  & \by{i:4.2.9}[e]  \\
    \implies & q > r.
  \end{align*}

  Now suppose that \(x > 1\) and \(q > r\).
  Then we have
  \begin{align*}
             & q > r                                                   \\
    \implies & q - r > 0                          &  & \by{i:4.2.9}    \\
    \implies & x^{q - r} > 1^{q - r}              &  & \by{i:5.6.9}[d] \\
    \implies & x^{q - r} > 1^{(ad - bc) / bd}     &  & \by{i:4.2.2}    \\
    \implies & x^{q - r} > (1^{1 / bd})^{ad - bc} &  & \by{i:5.6.7}    \\
    \implies & x^{q - r} > 1^{ad - bc} = 1        &  & \by{i:5.6.6}[e] \\
    \implies & x^{q - r} x^r > x^r                &  & \by{i:5.6.9}[a] \\
    \implies & x^{q - r + r} > x^r                &  & \by{i:5.6.9}[b] \\
    \implies & x^q > x^r.                         &  & \by{i:5.6.8}
  \end{align*}
  Thus we conclude that if \(x > 1\), then \(x^q > x^r \iff q > r\).

  Next suppose that \(x < 1\) and \(x^q > x^r\).
  Then we have
  \begin{align*}
             & x^q > x^r                                                                  \\
    \implies & x^q x^{-r} > x^r x^{-r}                              &  & \by{i:5.6.9}[a]  \\
    \implies & x^{q - r} > x^{r - r} = x^0 = 1                      &  & \by{i:5.6.9}[b]  \\
    \implies & x^{(ad - bc) / bd} > 1                               &  & \by{i:4.2.2}     \\
    \implies & (x^{(ad - bc)})^{1 / bd} > 1                         &  & \by{i:5.6.9}[b]  \\
    \implies & \big((x^{(ad - bc)})^{1 / bd}\big)^{bd} > 1^{bd} = 1 &  & \by{i:5.6.3}     \\
    \implies & x^{(ad - bc)} > 1                                    &  & \by{i:5.6.6}[a]  \\
    \implies & ad - bc < 0                                          &  & \by{i:4.3.12}[b] \\
    \implies & ad < bc                                              &  & \by{i:4.1.11}[b] \\
    \implies & a / b < c / d                                        &  & \by{i:4.2.9}[e]  \\
    \implies & q < r.
  \end{align*}

  Finally suppose that \(x < 1\) and \(q < r\).
  Then we have
  \begin{align*}
             & q < r                                                   \\
    \implies & r - q > 0                          &  & \by{i:4.2.9}    \\
    \implies & x^{r - q} < 1^{r - q}              &  & \by{i:5.6.9}[d] \\
    \implies & x^{r - q} < 1^{(bc - ad) / bd}     &  & \by{i:4.2.2}    \\
    \implies & x^{r - q} < (1^{1 / bd})^{bc - ad} &  & \by{i:5.6.7}    \\
    \implies & x^{r - q} < 1^{bc - ad} = 1        &  & \by{i:5.6.6}[e] \\
    \implies & x^{r - q} x^q < x^q                &  & \by{i:5.6.9}[a] \\
    \implies & x^{r - q + q} < x^q                &  & \by{i:5.6.9}[b] \\
    \implies & x^r < x^q.                         &  & \by{i:5.6.8}
  \end{align*}
  Thus we conclude that if \(x < 1\), then \(x^q > x^r \iff q < r\).
\end{proof}

\begin{proof}[\pf{i:5.6.9}(f)]
  Let \(q = a / b\) where \(a \in \Z\) and \(b \in \Z^+\).
  Then we have
  \begin{align*}
    (xy)^q & = \big((xy)^{1 / b}\big)^a    &  & \by{i:5.6.7}    \\
           & = (x^{1 / b} y^{1 / b})^a     &  & \by{i:5.6.6}[f] \\
           & = (x^{1 / b})^a (y^{1 / b})^a &  & \by{i:5.6.3}    \\
           & = x^q y^q.                    &  & \by{i:5.6.7}
  \end{align*}
\end{proof}

\exercisesection

\begin{ex}\label{i:ex:5.6.1}
  Prove \cref{i:5.6.6}.
\end{ex}

\begin{proof}[\pf{i:ex:5.6.1}]
  See \cref{i:5.6.6}.
\end{proof}

\begin{ex}\label{i:ex:5.6.2}
  Prove \cref{i:5.6.9}.
\end{ex}

\begin{proof}[\pf{i:ex:5.6.2}]
  See \cref{i:5.6.9}.
\end{proof}

\begin{ex}\label{i:ex:5.6.3}
  If \(x\) is a real number, show that \(\abs{x} = (x^2)^{1 / 2}\).
\end{ex}

\begin{proof}[\pf{i:ex:5.6.3}]
  By \cref{i:5.4.7} exactly one of the following three statements is true:
  \begin{enumerate}
    \item \(x > 0\).
          Then by \cref{i:5.4.5} we have \(\abs{x} = x\) and by \cref{i:5.6.6}(a)(b) we have \(\abs{x} = x = (x^2)^{1 / 2}\).
    \item \(x = 0\).
          Then by \cref{i:5.4.5} we have \(\abs{0} = 0\) and by \cref{i:5.6.6}(c) we have \(\abs{0} = 0 = (0^2)^{1 / 2}\).
    \item \(x < 0\).
          Then by \cref{i:5.4.5} we have \(\abs{x} = -x > 0\).
          By \cref{i:5.6.6}(b)(c) we have \(-x = ((-x)^2)^{1 / 2}\).
          But by \cref{i:5.3.11} we know that \((-x)^2 = (-x)(-x) = x^2\).
          Thus we have \(\abs{x} = -x = (x^2)^{1 / 2}\).
  \end{enumerate}
  From all cases above we conclude that \(\abs{x} = (x^2)^{1 / 2}\).
\end{proof}

