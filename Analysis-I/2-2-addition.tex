\section{Addition}

\begin{definition}[Addition of natural numbers]\label{2.2.1}
Let \(m\) be a natural numbers.
To add zero to \(m\), we define \(0+m \coloneqq m\).
Now suppose inductively that we have defined how to add \(n\) to \(m\).
Then we can add \(n++\) to \(m\) by defining \((n++) + m \coloneqq (n + m)++\).
\end{definition}

\begin{additional corollary}\label{ac 2.2.1}
The sum of two natural numbers is again a natural number.
\end{additional corollary}

\begin{proof}
We use induction.
Let \(m\) be a natural number.
\(0 + m = m\) is a natural number according to the given condition.
Now suppose inductively that \(n\) is a natural number such that \(n + m\) is a natural number.
We wish to show that \((n++) + m\) is also a natural number.
But by Definition \ref{2.2.1} and Axiom \ref{2.2},
\((n++) + m = (n + m)++\) is a natural number.
This close the induction.
\end{proof}

\begin{lemma}\label{2.2.2}
For any natural number \(n\), \(n + 0 = n\).
\end{lemma}

\begin{proof}
We use induction.
The base case \(0 + 0 = 0\) follows since we know that \(0 + m = m\) for every natural number \(m\), and \(0\) is a natural number.
Now suppose inductively that \(n + 0 = n\).
We wish to show that \((n++) + 0 = n++\).
But by Definition \ref{2.2.1}, \((n++) + 0\) is equal to \((n + 0)++\), which is equal to \(n++\) since \(n + 0 = n\).
This closes the induction.
\end{proof}

\begin{lemma}\label{2.2.3}
For any natural numbers \(n\) and \(m\), \(n + (m++) = (n + m)++\).
\end{lemma}

\begin{proof}
We induct on \(n\) (keeping \(m\) fixed).
We first consider the base case \(n = 0\).
In this case we have to prove \(0 + (m++) = (0 + m)++\).
But by Definition \ref{2.2.1}, \(0 + (m++) = m++\) and \(0 + m = m\), so both sides are equal to \(m++\) and are thus equal to each other.
Now we assume inductively that \(n + (m++) = (n + m)++\);
we now have to show that \((n++) + (m++) = ((n++) + m)++\).
The left-hand side is \((n + (m++))++\) by Definition \ref{2.2.1}, which is equal to \(((n+m)++)++\) by the inductive hypothesis.
Similarly, we have \((n++) + m = (n + m)++\) by the Definition \ref{2.2.1}, and so the right-hand side is also equal to \(((n + m)++)++\).
Thus both sides are equal to each other, and we have closed the induction.
\end{proof}

\begin{additional corollary}\label{ac 2.2.2}
\(n++ = n + 1\).
\end{additional corollary}

\begin{proof}
\begin{align*}
n++ &= (n++) + 0 & \text{(by Lemma \ref{2.2.1})} \\
&= (n + 0)++ & \text{(by Definition \ref{2.2.1})} \\
&= n + (0++) & \text{(by Lemma \ref{2.2.3})} \\
&= n + 1.
\end{align*}
\end{proof}

\begin{proposition}[Addition is commutative]\label{2.2.4}
For any natural numbers \(n\) and \(m\), \(n + m = m + n\).
\end{proposition}

\begin{proof}
We shall use induction on \(n\) (keeping \(m\) fixed).
First we do the base case \(n = 0\), i.e., we show \(0 + m = m + 0\).
By the Definition \ref{2.2.1}, \(0 + m = m\), while by Lemma \ref{2.2.1}, \(m + 0 = m\).
Thus the base case is done.
Now suppose inductively that \(n + m = m + n\), now we have to prove that \((n++) + m = m + (n++)\) to close the induction.
By the Definition \ref{2.2.1}, \((n++) + m = (n + m)++\).
By Lemma \ref{2.2.3}, \(m + (n++) = (m + n)++\), but this is equal to \((n + m)++\) by the inductive hypothesis \(n+m=m+n\).
Thus \((n++) + m = m + (n++)\) and we have closed the induction.
\end{proof}

\begin{proposition}[Addition is associative]\label{2.2.5}
For any natural numbers \(a\), \(b\), \(c\), we have \((a + b) + c = a + (b + c)\).
\end{proposition}

\begin{proof}
We shall use induction on \(c\) and keeping both \(a\) and \(b\) fixed.
First we do the base case \(c = 0\), i.e., we show \((a + b) + 0 = a + (b + 0)\).
By Lemma \ref{2.2.1}, \((a + b) + 0 = a + b\) and \(a + (b + 0) = a + b\).
Thus the base case is done.
Now suppose inductively that \((a + b) + c = a + (b + c)\), we have to prove that \((a + b) + (c++) = a + (b + (c++))\) to close the induction.
By Lemma \ref{2.2.3}, \((a + b) + (c++) = ((a + b) + c)++\).
Also By Lemma \ref{2.2.3}, \(a + (b + (c++)) = a + ((b + c)++) = (a + (b + c))++\), but this is equal to \(((a + b) + c)++\) by the inductive hypothesis \((a + b) + c = a + (b + c)\).
Thus \((a + b) + (c++) = a + (b + (c++))\) and we have closed the induction.
\end{proof}

\begin{note}
Because of this associativity we can write sums such as \(a + b + c\) without having to worry about which order the numbers are being added together.
\end{note}

\begin{proposition}[Cancellation law]\label{2.2.6}
Let \(a\), \(b\), \(c\) be natural numbers such that \(a + b = a + c\).
Then we have \(b = c\).
\end{proposition}

\begin{proof}
We prove this by induction on \(a\).
First consider the base case \(a = 0\).
Then we have \(0 + b = 0 + c\), which by Definition \ref{2.2.1} implies that \(b = c\) as desired.
Now suppose inductively that we have the cancellation law for \(a\) (so that \(a + b = a + c\) implies \(b = c\));
we now have to prove the cancellation law for \(a++\).
In other words, we assume that \((a++) + b = (a++) + c\) and need to show that \(b = c\).
By the Definition \ref{2.2.1}, \((a++) + b = (a + b)++\) and \((a++) + c = (a + c)++\) and so we have \((a + b)++ = (a + c)++\).
By Axiom \ref{2.4}, we have \(a + b = a + c\).
Since we already have the cancellation law for \(a\), we thus have \(b = c\) as desired.
This closes the induction.
\end{proof}

\begin{definition}[Positive natural numbers]\label{2.2.7}
A natural number \(n\) is said to be \emph{positive} iff it is not equal to \(0\).
\end{definition}

\begin{proposition}\label{2.2.8}
If \(a\) is positive natural number and \(b\) is a natural number, then \(a + b\) is positive (and hence \(b + a\) is also, by Proposition \ref{2.2.4}).
\end{proposition}

\begin{proof}
We use induction on \(b\).
If \(b = 0\), then \(a + b = a + 0 = a\), which is positive, so this proves the base case.
Now suppose inductively that \(a + b\) is positive.
Then \(a + (b++) = (a + b)++\), which cannot be zero by Axiom \ref{2.3}, and is hence positive.
This closes the induction.
\end{proof}

\begin{corollary}\label{2.2.9}
If \(a\) and \(b\) are natural numbers such that \(a + b = 0\), then \(a = 0\) and \(b = 0\).
\end{corollary}

\begin{proof}
Suppose for sake of contradiction that \(a \neq 0\) or \(b \neq 0\).
If \(a \neq 0\) then \(a\) is positive, and hence \(a + b = 0\) is positive by Proposition \ref{2.2.8}, a contradiction.
Similarly if \(b \neq 0\) then \(b\) is positive, and again \(a + b = 0\) is positive by Proposition \ref{2.2.8}, a contradiction.
Thus \(a\) and \(b\) must both be zero.
\end{proof}

\begin{lemma}\label{2.2.10}
Let \(a\) be a positive number.
Then there exists exactly one natural number \(b\) such that \(b++ = a\).
\end{lemma}

\begin{proof}
We use induction.
If \(a = 1\) and \(b++ = a\), then \(b = 0\).
We want to show that \(0\) is unique, so assume that \(0'\) is a natural number and \(0'++ = 1\).
By Axiom \ref{2.4}, if \(0++ = 0'++\), then \(0 = 0'\), thus \(0\) is unique, and this proves the base case.
Suppose inductively that \(a\) is positive, there exist exactly one natural number \(b\) such that \(b++ = a\).
Then \(a++ = (b++)++\) by induction hypothesis, so there exist a natural number \(b++\) such that \((b++)++ = a++\).
Now we need to show that \(b++\) is unique.
Assume that there exist another natural number \(c\) such that \(a++ = c++\).
Then by Axiom \ref{2.4}, \(a = c\), and by induction hypothesis, \(c = b++\), so \(c++ = (b++)++\), which means \(b++\) is unique.
This close the induction.
\end{proof}

\begin{definition}[Ordering of the natural numbers]\label{2.2.11}
Let \(n\) and \(m\) be natural numbers.
We say that \(n\) is greater than or equal to \(m\), and write \(n \geq m\) or \(m \leq n\), iff we have \(n = m + a\) for some natural number \(a\).
We say that \(n\) is strictly greater than \(m\), and write \(n > m\) or \(m < n\), iff \(n \geq m\) and \(n \neq m\).
\end{definition}

\begin{note}
There is no largest natural number \(n\), because the next number \(n++\) is always larger.
\end{note}

\begin{proposition}[Basic properties of order for natural numbers]\label{2.2.12}
Let \(a\), \(b\), \(c\) be natural numbers.
Then
\begin{enumerate}
\item (Order is reflexive) \(a \geq a\).
\item (Order is transitive) If \(a \geq b\) and \(b \geq c\), then \(a \geq c\).
\item (Order is anti-symmetric) If \(a \geq b\) and \(b \geq a\), then \(a = b\).
\item (Addition preserves order) \(a \geq b\) iff \(a + c \geq b + c\).
\item \(a < b\) iff \(a++ \leq b\).
\item \(a < b\) iff \(b = a + d\) for some \emph{positive} number \(d\).
\end{enumerate}
\end{proposition}

\begin{proof}{(a)}
\(\because a = a + 0\), \(\therefore a \geq a\) by Definition \ref{2.2.11}.
\end{proof}

\begin{proof}{(b)}
Let \(a = b + d\) and \(b = c + e\), where \(d\), \(e\) are natural numbers.
Substitute \(b\) with equation, we can derive \(a = b + d = (c + e) + d = c + (e + d)\).
By Additional Corollary \ref{ac 2.2.1}, \(e + d\) is a natural number.
Therefore by Definition \ref{2.2.11}, \(a \geq c\).
\end{proof}

\begin{proof}{(c)}
By the given conditions \(a \geq b\) and \(b \geq a\), let \(a = b + c\) and \(b = a + d\), where \(c\) and \(d\) are natural numbers.
Substitute \(b\) with equation, we can derive that \(a = b + c = (a + d) + c = a + (d + c)\).
By Proposition \ref{2.2.6}, \(a = a + (d + c) \implies 0 = d + c\), and by Corollary \ref{2.2.9}, \(d = c = 0\).
Therefore \(a = b + 0 = b\) and \(b = a + 0 = a\).
\end{proof}

\begin{proof}{(d)}
First we prove the necessary condition of (d).
Let \(a = b + d\), where \(d\) is a natural number.
Then \(a = b + d \implies (a + c) = (b + c) + d\), which means \(a + c \geq b + c\) by Definition \ref{2.2.11}.

Now we prove the sufficient condition of (d).
Let \(a + c = b + c + d\), where \(d\) is a natural number.
Then by Proposition \ref{2.2.6}, \(a + c = b + c + d \implies a = b + d\), which means \(a \geq b\) by Definition \ref{2.2.11}.
\end{proof}

\begin{proof}{(e)}
First we prove the necessary condition of (e).
Let \(b = a + c\), where \(c\) is a natural number.
Because \(a \neq b\), so \(c \neq 0\).
By Lemma \ref{2.2.10}, there exists a natural number \(d\) such that \(d++ = c\).
So \(b = a + c = a + (d++) = (a + d)++ = (a++) + d\) by Lemma \ref{2.2.3} and Definition \ref{2.2.1}.
Thus \(a++ \leq b\) by Definition \ref{2.2.11}.

Now we prove the sufficient condition of (e).
Let \(b = (a++) + c\), where \(c\) is a natural number.
Then \(b = a + (c++)\) and \(c++\) must not be zero, otherwise contradict to Axiom \ref{2.3}.
So \(b \neq a\), thus \(a < b\).
\end{proof}

\begin{proof}{(f)}
First we prove the necessary condition of (f).
Let \(b = a + d\), where \(d\) is a natural number.
Because \(a \neq b\), so \(d \neq 0\), which means \(d\) is positive by Definition \ref{2.2.7}.

Now we prove the sufficient condition of (f).
Let \(b = a + d\), where \(d\) is a positive natural number.
Because \(d \neq 0\), \(b \neq a\), thus \(a < b\).
\end{proof}

\begin{proposition}[Trichotomy of order for natural numbers]\label{2.2.13}
Let \(a\) and \(b\) be natural numbers.
Then exactly one of the following statements is true: \(a < b\), \(a = b\), or \(a > b\).
\end{proposition}

\begin{proof}
First we show that we cannot have more than one of the statements \(a < b\), \(a = b\), \(a > b\) holding at the same time.
If \(a < b\) then \(a \neq b\) by Definition \ref{2.2.11}, and if \(a > b\) then \(a \neq b\) by Definition \ref{2.2.11}.
If \(a > b\) and \(a < b\) then by Proposition \ref{2.2.12} we have \(a = b\), a contradiction.
Thus no more than one of the statements is true.
Now we show that at least one of the statements is true.
We keep \(b\) fixed and induct on \(a\).
When \(a = 0\) we have \(0 \leq b\) for all \(b\) because \(b = b + 0\), so we have either \(0 = b\) or \(0 < b\), which proves the base case.
Now suppose we have proven the proposition for \(a\), and now we prove the proposition for \(a++\).
From the trichotomy for \(a\), there are three cases: \(a < b\), \(a = b\), and \(a > b\).
If \(a > b\), then \(a++ > b\) because \(a++ > a > b\) and Proposition \ref{2.2.12}.
If \(a = b\), then \(a++ > b\) because \(a++ = b++ > b\).
Now suppose that \(a < b\).
Then by Proposition \ref{2.2.12}, we have \(a++ \leq b\).
Thus either \(a++ = b\) or \(a++ < b\), and in either case we are done.
This closes the induction.
\end{proof}

\begin{proposition}[Strong principle of induction]\label{2.2.14}
Let \(m_0\) be a natural number, and let \(P(m)\) be a property pertaining to an arbitrary natural number \(m\).
Suppose that for each \(m \geq m_0\), we have the following implication: if \(P(m')\) is true for all natural numbers \(m_0 \leq m' < m\), then \(P(m)\) is also true.
(In particular, this means that \(P(m_0)\) is true, since in this case the hypothesis is vacuous.)
Then we can conclude that \(P(m)\) is true for all natural numbers \(m \geq m_0\).
\end{proposition}

\begin{proof}
Let \(n\) be a natural number and let \(Q(n)\) be the property that \(P(m)\) is true for all \(m_0 \leq m < n\) for \(n \geq m_0\).
Using induction on \(n\), for the base case \(n = 0\), we want to show that \(Q(0)\) is true.
However, we know that \(0 \leq m_0\) for all natural number \(m_0\).
Thus, either \(0 = m_0\) or \(0 < m_0\) and so we split into cases.
If \(n = 0 < m_0\), the statement \(P(m) \ \forall\ m_0 \leq m < n\) is vacuously true (since the hypothesis applies for \(n \geq m_0\)) and thus \(Q(0)\) is true in this case.
For the second case, if \(n = 0 = m_0\), then the statement \(P(m) \ \forall\ m_0 \leq m < n\) is also vacuously true since there is no natural number \(m'\) such that \(0 \leq m' < 0\). Hence, \(Q(0)\) is true for this case and that completes the base case of the induction.

Now suppose inductively that for some \(n \geq m_0\), \(Q(n)\) is true, i.e \(P(m) \ \forall\ m_0 \leq m < n\) is true.
We need to show that \(Q(n++)\) is true.
By the definition of \(P\) in the hypothesis, \(P(n)\) is also true (because \(Q(n)\) is true).
Since \(n < n++\), then \(P(m) \ \forall\ m_0 \leq m \leq n < n++\) is true so \(P(m) \ \forall\ m_0 \leq m < n++\) is true which in turn implies that \(Q(n++)\) is true.
Which closes the induction and hence we can conclude that \(Q(n) \ \forall\ n\) is true.
However, \(Q(n)\) true implies \(P(m) \ \forall\ m_0 \leq m < n\) is true for all \(n \geq m_0\) and by the definition of \(P\), \(P(n)\) is also true for all \(n \geq m_0\) which concludes the proof.
\end{proof}

\begin{remark}\label{2.2.15}
In applications we usually use Proposition \ref{2.2.14} with \(m_0 = 0\) or \(m_0 = 1\).
\end{remark}

\exercisesection

\begin{exercise}\label{ex 2.2.1}
Prove Proposition \ref{2.2.5}.
\end{exercise}

\begin{proof}
See Proposition \ref{2.2.5}.
\end{proof}

\begin{exercise}\label{ex 2.2.2}
Prove Lemma \ref{2.2.10}.
\end{exercise}

\begin{proof}
See Lemma \ref{2.2.10}.
\end{proof}

\begin{exercise}\label{ex 2.2.3}
Prove Proposition \ref{2.2.12}.
\end{exercise}

\begin{proof}
See Proposition \ref{2.2.12}.
\end{proof}

\begin{exercise}\label{ex 2.2.4}
Justify the three statements marked in the proof of Proposition \ref{2.2.13}.
\end{exercise}

\begin{proof}
See Proposition \ref{2.2.13}.
\end{proof}

\begin{exercise}\label{ex 2.2.5}
Prove Proposition \ref{2.2.14}.
\end{exercise}

\begin{proof}
See Proposition \ref{2.2.14}.
\end{proof}

\begin{exercise}[Principle of backwards induction]\label{ex 2.2.6}
Let \(n\) be a natural number, and let \(P(m)\) be a property pertaining to the natural numbers such that whenever \(P(m++)\) is true, then \(P(m)\) is true.
Suppose that \(P(n)\) is also true.
Prove that \(P(m)\) is true for all natural numbers \(m \leq n\);
\end{exercise}

\begin{proof}
We use induction.
The base case \(n = 0\) is trivially true because \(\forall\ m \leq 0\) means \(m = 0\), thus \(P(m) = P(n) = P(0)\) is true.
Suppose inductively that \(P(n)\) is true, and \(P(m)\) is true \(\forall\ m \leq n\).
Then when \(P(n++)\) is true, by the given condition \(P(n)\) is true.
And by induction hypothesis \(P(m)\) is true \(\forall\ m \leq n \leq n++\), so \(P(n++)\) is true implies \(P(m)\) is true \(\forall\ m \leq n++\).
This close the induction.
\end{proof}