\section{Addition}\label{i:sec:2.2}

\begin{defn}[Addition of natural numbers]\label{i:2.2.1}
  Let \(m\) be a natural number.
  To add zero to \(m\), we define \(0 + m \coloneqq m\).
  Now suppose inductively that we have defined how to add \(n\) to \(m\).
  Then we can add \(n\pp\) to \(m\) by defining \((n\pp) + m \coloneqq (n + m)\pp\).
\end{defn}

\begin{ac}\label{i:ac:2.2.1}
  The sum \(n + m\) of two natural numbers \(n, m\) is again a natural number.
\end{ac}

\begin{proof}[\pf{i:ac:2.2.1}]
  Let \(m, n\) be a natural number.
  We induct on \(n\).
  For \(n = 0\), by \cref{i:2.2.1}, we have \(0 + m = m\), which is a natural number by the definition of \(m\).
  So the base case holds.
  Suppose inductively that for some natural number \(n\), we know that \(n + m\) is a natural number.
  We want to show that \((n\pp) + m\) is also a natural number.
  By \cref{i:2.2.1}, we have \((n\pp) + m = (n + m)\pp\).
  By the induction hypothesis, we know that \(n + m\) is a natural number.
  Thus, by \cref{i:2.2}, we know that \((n + m)\pp\) is again a natural number.
  This closes the induction.
\end{proof}

\begin{lem}\label{i:2.2.2}
  For any natural number \(n\), we have \(n + 0 = n\).
\end{lem}

\begin{proof}[\pf{i:2.2.2}]
  We induct on \(n\).
  The base case \(0 + 0 = 0\) follows since we know that \(0 + m = m\) for every natural number \(m\) (\cref{i:2.2.1}), and \(0\) is a natural number (\cref{i:2.1}).
  Now suppose inductively that \(n + 0 = n\).
  We wish to show that \((n\pp) + 0 = n\pp\).
  But by \cref{i:2.2.1}, \((n\pp) + 0\) is equal to \((n + 0)\pp\), which is equal to \(n\pp\) since \(n + 0 = n\).
  This closes the induction.
\end{proof}

\begin{lem}\label{i:2.2.3}
  For any natural numbers \(n\) and \(m\), we have \(n + (m\pp) = (n + m)\pp\).
\end{lem}

\begin{proof}[\pf{i:2.2.3}]
  We induct on \(n\) (keeping \(m\) fixed).
  For \(n = 0\), we must prove \(0 + (m\pp) = (0 + m)\pp\).
  But by \cref{i:2.2.1}, \(0 + (m\pp) = m\pp\) and \(0 + m = m\), so both sides are equal to \(m\pp\) and are thus equal to each other.
  So the base case holds.
  Suppose inductively that \(n + (m\pp) = (n + m)\pp\).
  We need to show that \((n\pp) + (m\pp) = ((n\pp) + m)\pp\).
  The left-hand side is \((n + (m\pp))\pp\) by \cref{i:2.2.1}, which is equal to \(((n+m)\pp)\pp\) by the inductive hypothesis.
  Similarly, we have \((n\pp) + m = (n + m)\pp\) by \cref{i:2.2.1}, and so the right-hand side is also equal to \(((n + m)\pp)\pp\).
  Thus, both sides are equal, and we have closed the induction.
\end{proof}

\begin{ac}\label{i:ac:2.2.2}
  For any natural number \(n\), we have \(n\pp = n + 1\).
\end{ac}

\begin{proof}[\pf{i:ac:2.2.2}]
  Since \(n\) is a natural number, by \cref{i:2.2} we know that \(n\pp\) is also a natural number.
  Thus, we can apply \cref{i:2.2.2} to derive the following fact:
  \begin{align*}
    n\pp & = (n\pp) + 0 &  & \by{i:2.2.2} \\
         & = (n + 0)\pp &  & \by{i:2.2.1} \\
         & = n + (0\pp) &  & \by{i:2.2.3} \\
         & = n + 1.
  \end{align*}
\end{proof}

\begin{prop}[Addition is commutative]\label{i:2.2.4}
  For any natural numbers \(n\) and \(m\), we have \(n + m = m + n\).
\end{prop}

\begin{proof}[\pf{i:2.2.4}]
  We induct on \(n\).
  For \(n = 0\), we need to show \(0 + m = m + 0\).
  But by \cref{i:2.2.1}, \(0 + m = m\), while by \cref{i:2.2.2}, \(m + 0 = m\).
  Thus, both sides are equal, and the base case holds.
  Suppose inductively that \(n + m = m + n\).
  We must prove that \((n\pp) + m = m + (n\pp)\) to close the induction.
  By \cref{i:2.2.1}, \((n\pp) + m = (n + m)\pp\).
  By \cref{i:2.2.3}, \(m + (n\pp) = (m + n)\pp\), but this is equal to \((n + m)\pp\) by the induction hypothesis.
  Thus, \((n\pp) + m = m + (n\pp)\), and we have closed the induction.
\end{proof}

\begin{prop}[Addition is associative]\label{i:2.2.5}
  For any natural numbers \(a, b, c\), we have \((a + b) + c = a + (b + c)\).
\end{prop}

\begin{proof}[\pf{i:2.2.5}]
  We induct on \(c\) and keep both \(a\) and \(b\) fixed.
  For \(c = 0\), we have
  \begin{align*}
    (a + b) + 0 & = a + b        &  & \by{i:2.2.2} \\
                & = a + (b + 0). &  & \by{i:2.2.2}
  \end{align*}
  Thus, the base case holds.
  Suppose inductively that \((a + b) + c = a + (b + c)\) for some natural number \(c\).
  We want to show that \((a + b) + (c\pp) = a + (b + (c\pp))\).
  But this is true since
  \begin{align*}
    (a + b) + (c\pp) & = ((a + b) + c)\pp  &  & \by{i:2.2.3} \\
                     & = (a + (b + c))\pp  &  & \byIH        \\
                     & = a + (b + c)\pp    &  & \by{i:2.2.3} \\
                     & = a + (b + (c\pp)). &  & \by{i:2.2.3}
  \end{align*}
  This closes the induction.
\end{proof}

\begin{note}
  Because of associativity showed in \cref{i:2.2.5}, we can write sums such as \(a + b + c\) without having to worry about which order the numbers are being added together.
\end{note}

\begin{prop}[Cancellation law]\label{i:2.2.6}
  Let \(a, b, c\) be natural numbers such that \(a + b = a + c\).
  Then we have \(b = c\).
\end{prop}

\begin{proof}[\pf{i:2.2.6}]
  We induct on \(a\).
  For \(a = 0\), we have \(0 + b = 0 + c\), which by \cref{i:2.2.1} implies that \(b = c\) as desired.
  Suppose inductively that we have the cancellation law for \(a\) (so that \(a + b = a + c\) implies \(b = c\));
  we now have to prove the cancellation law for \(a\pp\).
  In other words, we assume that \((a\pp) + b = (a\pp) + c\) and need to show that \(b = c\).
  By \cref{i:2.2.1}, we have \((a\pp) + b = (a + b)\pp\) and \((a\pp) + c = (a + c)\pp\), and so we have \((a + b)\pp = (a + c)\pp\).
  By \cref{i:2.4}, we have \(a + b = a + c\).
  Since we already have the cancellation law for \(a\), we thus have \(b = c\) as desired.
  This closes the induction.
\end{proof}

\begin{defn}[Positive natural numbers]\label{i:2.2.7}
  A natural number \(n\) is said to be \emph{positive} iff it is not equal to \(0\).
\end{defn}

\begin{prop}\label{i:2.2.8}
  If \(a\) is a positive natural number and \(b\) is a natural number, then \(a + b\) is positive (and hence \(b + a\) is also, by \cref{i:2.2.4}).
\end{prop}

\begin{proof}[\pf{i:2.2.8}]
  We induct on \(b\).
  If \(b = 0\), then \(a + b = a + 0 = a\), which is positive, proving the base case.
  Suppose inductively that \(a + b\) is positive.
  Then \(a + (b\pp) = (a + b)\pp\), which cannot be zero by \cref{i:2.3}, and is hence positive by \cref{i:2.2.7}.
  This closes the induction.
\end{proof}

\begin{cor}\label{i:2.2.9}
  If \(a\) and \(b\) are natural numbers such that \(a + b = 0\), then we have \(a = 0\) and \(b = 0\).
\end{cor}

\begin{proof}[\pf{i:2.2.9}]
  Suppose for the sake of contradiction that \(a \neq 0\) or \(b \neq 0\).
  If \(a \neq 0\), then \(a\) is positive, and hence \(a + b = 0\) is positive by \cref{i:2.2.8}, a contradiction.
  Similarly, if \(b \neq 0\), then \(b\) is positive, and again \(a + b = 0\) is positive by \cref{i:2.2.8}, a contradiction.
  Thus, \(a\) and \(b\) must both be zero.
\end{proof}

\begin{lem}\label{i:2.2.10}
  Let \(a\) be a positive natural number.
  Then there exists exactly one natural number \(b\) such that \(b\pp = a\).
\end{lem}

\begin{proof}[\pf{i:2.2.10}]
  Let \(P(n)\) be the statement ``either we have \(n = 0\), or there exists a natural number \(m\), such that \(m\pp = n\).''
  We induct on \(n\) to show that \(P(n)\) is true for any natural number \(n\).
  Clearly, \(P(0)\) is true.
  So suppose inductively that \(P(n)\) is true for some natural number \(n\).
  We want to show that \(P(n\pp)\) is true.
  By \cref{i:2.3}, we know that \(n\pp \neq 0\).
  So we have to show that there exists a natural number \(m\), such that \(m\pp = n\pp\).
  By \cref{i:2.4}, we see that \(m = n\).
  Thus, \(P(n\pp)\) is true, closing the induction.

  Now we prove the existence of \(b\).
  From the first part of the proof, we know that \(P(a)\) is true.
  Since \(a\) is a positive natural number, by \cref{i:2.2.7}, we know that \(a \neq 0\).
  Thus, there must exist a natural number \(b\) such that \(b\pp = a\).

  Finally, we prove the uniqueness of \(b\).
  Suppose that there exists another natural number \(c\) such that \(c\pp = a\).
  But this means \(b\pp = c\pp\).
  Thus, by \cref{i:2.4}, we have \(b = c\).
\end{proof}

\begin{defn}[Ordering of the natural numbers]\label{i:2.2.11}
  Let \(n\) and \(m\) be natural numbers.
  We say that \(n\) is \emph{greater than or equal to} \(m\), and write \(n \geq m\) or \(m \leq n\), iff we have \(n = m + a\) for some natural number \(a\).
  We say that \(n\) is \emph{strictly greater than} \(m\), and write \(n > m\) or \(m < n\), iff \(n \geq m\) and \(n \neq m\).
\end{defn}

\begin{ac}\label{i:ac:2.2.3}
  We have \(n\pp > n\) for any natural number \(n\).
  Therefore, there is no largest natural number \(n\), because the next number \(n\pp\) is always larger.
\end{ac}

\begin{proof}[\pf{i:ac:2.2.3}]
  Let \(n\) be a natural number.
  By \cref{i:ac:2.2.2}, we have \(n\pp = n + 1\).
  Since \(1\) is a natural number, by \cref{i:2.2.11}, we have \(n\pp \geq n\).
  To show that \(n\pp > n\), by \cref{i:2.2.11}, we only need to show that \(n\pp \neq n\).

  We induct on \(n\) to show that \(n\pp \neq n\) for any natural number \(n\).
  For \(n = 0\), we have \(0\pp \neq 0\), by \cref{i:2.3}.
  Thus, the base case holds.
  Suppose inductively that \(n\pp \neq n\) for some natural number \(n\).
  Then we have
  \begin{align*}
             & n\pp \neq n          &  & \byIH      \\
    \implies & (n\pp)\pp \neq n\pp. &  & \by{i:2.4}
  \end{align*}
  This closes the induction.
  We conclude that \(n\pp > n\) for any natural number \(n\).
\end{proof}

\begin{ac}\label{i:ac:2.2.4}
  We have \(n \geq 0\) for every natural number \(n\).
  If \(n\) is a positive natural number, then we have \(n > 0\).
\end{ac}

\begin{proof}[\pf{i:ac:2.2.4}]
  Let \(n\) be a natural number.
  By \cref{i:2.2.1}, we have \(n = 0 + n\).
  Thus, by \cref{i:2.2.11}, we have \(n \geq 0\).

  Now suppose that \(n\) is a positive natural number.
  By \cref{i:2.2.7}, this means \(n \neq 0\).
  From first paragraph we see that \(n \geq 0\).
  Thus, by \cref{i:2.2.11}, we have \(n > 0\).
\end{proof}

\begin{prop}[Basic properties of order for natural numbers]\label{i:2.2.12}
  Let \(a, b, c\) be natural numbers.
  Then
  \begin{enumerate}
    \item (Order is reflexive) \(a \geq a\).
    \item (Order is transitive) If \(a \geq b\) and \(b \geq c\), then \(a \geq c\).
    \item (Order is anti-symmetric) If \(a \geq b\) and \(b \geq a\), then \(a = b\).
    \item (Addition preserves order) \(a \geq b\) iff \(a + c \geq b + c\).
    \item \(a < b\) iff \(a\pp \leq b\).
    \item \(a < b\) iff \(b = a + d\) for some \emph{positive} number \(d\).
  \end{enumerate}
\end{prop}

\begin{proof}[\pf{i:2.2.12}(a)]
  We have
  \begin{align*}
             & \begin{dcases}
                 0 \text{ is a natural number} \\
                 a = a + 0
               \end{dcases} &  & \by{i:2.1,i:2.2.2}                \\
    \implies & a \geq a.                        &  & \by{i:2.2.11}
  \end{align*}
\end{proof}

\begin{proof}[\pf{i:2.2.12}(b)]
  Suppose that \(a \geq b\) and \(b \geq c\).
  By \cref{i:2.2.11}, there exist some natural numbers \(d\) and \(e\), such that \(a = b + d\) and \(b = c + e\).
  Then we have
  \begin{align*}
             & \begin{dcases}
                 a = b + d = (c + e) + d = c + (e + d) \\
                 e + d \text{ is a natural number}
               \end{dcases} &  & \by{i:2.2.5,i:ac:2.2.1}                   \\
    \implies & a \geq c.                                &  & \by{i:2.2.11}
  \end{align*}
\end{proof}

\begin{proof}[\pf{i:2.2.12}(c)]
  Suppose that \(a \geq b\) and \(b \geq a\).
  By \cref{i:2.2.11}, there exist some natural numbers \(c\) and \(d\), such that \(a = b + c\) and \(b = a + d\).
  Then we have
  \begin{align*}
             & \begin{dcases}
                 a = a + 0 \\
                 a = b + c = (a + d) + c = a + (d + c)
               \end{dcases} &  & \by{i:2.2.2,i:2.2.5}                  \\
    \implies & 0 = d + c                             &  & \by{i:2.2.6} \\
    \implies & d = c = 0                             &  & \by{i:2.2.9} \\
    \implies & a = b + 0 = b.                        &  & \by{i:2.2.2}
  \end{align*}
\end{proof}

\begin{proof}[\pf{i:2.2.12}(d)]
  We have
  \begin{align*}
             & a \geq b                                                               \\
    \implies & a = b + d \text{ for some natural number } d &  & \by{i:2.2.11}        \\
    \implies & a + c = c + a = c + (b + d)                  &  & \by{i:2.2.4}         \\
             & = (c + b) + d = (b + c) + d                  &  & \by{i:2.2.4,i:2.2.5} \\
    \implies & a + c \geq b + c,                            &  & \by{i:2.2.11}
  \end{align*}
  and
  \begin{align*}
             & a + c \geq b + c                                                                 \\
    \implies & a + c = (b + c) + d \text{ for some natural number } d &  & \by{i:2.2.11}        \\
    \implies & c + a = (c + b) + d = c + (b + d)                      &  & \by{i:2.2.4,i:2.2.5} \\
    \implies & a = b + d                                              &  & \by{i:2.2.6}         \\
    \implies & a \geq b.                                              &  & \by{i:2.2.11}
  \end{align*}
  Thus, we conclude that \(a \geq b \iff a + c \geq b + c\).
\end{proof}

\begin{proof}[\pf{i:2.2.12}(e)]
  First, suppose that \(a < b\).
  Then we have
  \begin{align*}
             & a < b                                                                                         \\
    \implies & \begin{dcases}
                 a \leq b \\
                 a \neq b
               \end{dcases}                                                      &  & \by{i:2.2.11}          \\
    \implies & \begin{dcases}
                 b = a + c \text{ for some natural number } c \\
                 a \neq b
               \end{dcases}                     &  & \by{i:2.2.11}                                           \\
    \implies & c \neq 0                                                            &  & \by{i:2.2.2}         \\
    \implies & c \text{ is a positive natural number}                              &  & \by{i:2.2.8}         \\
    \implies & \text{there exists a natural number } d \text{ such that } d\pp = c &  & \by{i:2.2.10}        \\
    \implies & b = a + (d\pp) = (a + d)\pp = (a\pp) + d                            &  & \by{i:2.2.1,i:2.2.3} \\
    \implies & a\pp \leq b.                                                        &  & \by{i:2.2.11}
  \end{align*}

  Now suppose that \(a\pp \leq b\).
  Then we have
  \begin{align*}
             & a\pp \leq b                                                                 \\
    \implies & b = (a\pp) + c \text{ for some natural number } c &  & \by{i:2.2.11}        \\
    \implies & b = (a\pp) + c = (a + c)\pp = a + (c\pp)          &  & \by{i:2.2.1,i:2.2.3} \\
    \implies & a \leq b.                                         &  & \by{i:2.2.11}
  \end{align*}
  By \cref{i:2.3}, we know that \(c\pp \neq 0\).
  Thus, by \cref{i:2.2.6}, we must have \(b \neq a\).
  (If \(b = a\), then we would have \(b = a + (c\pp) = a + 0\), which implies \(c\pp = 0\), a contradiction.)
  By \cref{i:2.2.11}, this means \(a < b\).
  From all proofs above, we conclude that \(a < b \iff a\pp \leq b\).
\end{proof}

\begin{proof}[\pf{i:2.2.12}(f)]
  We have
  \begin{align*}
         & a < b                                                                            \\
    \iff & a\pp \leq b                                            &  & \by{i:2.2.12}[e]     \\
    \iff & b = (a\pp) + c \text{ for some natural number } c      &  & \by{i:2.2.11}        \\
    \iff & b = (a\pp) + c = (a + c)\pp = a + (c\pp)               &  & \by{i:2.2.1,i:2.2.3} \\
    \iff & b = a + d \text{ for some positive natural number } d. &  & \by{i:2.3,i:2.2.10}
  \end{align*}
\end{proof}

\begin{prop}[Trichotomy of order for natural numbers]\label{i:2.2.13}
  Let \(a\) and \(b\) be natural numbers.
  Then exactly one of the following statements is true: \(a < b\), \(a = b\), or \(a > b\).
\end{prop}

\begin{proof}[\pf{i:2.2.13}]
  First, we show that we cannot have more than one of the statements \(a < b\), \(a = b\), \(a > b\) holding simultaneously.
  If \(a < b\), then \(a \neq b\) by \cref{i:2.2.11}, and if \(a > b\), then \(a \neq b\) by \cref{i:2.2.11}.
  If \(a > b\) and \(a < b\), then by \cref{i:2.2.12}(c), we have \(a = b\), a contradiction.
  Thus, no more than one of the statements is true.

  Now we show that at least one of the statements is true.
  We keep \(b\) fixed and induct on \(a\).
  When \(a = 0\), by \cref{i:2.2.1}, we have \(b = 0 + b\).
  Thus, by \cref{i:ac:2.2.4}, we have \(0 \leq b\) for any natural number \(b\).
  So we have either \(0 = b\) or \(0 < b\), which proves the base case.
  Suppose we have proven the proposition for \(a\), and now we prove the proposition for \(a\pp\).
  From the trichotomy for \(a\), there are three cases: \(a < b\), \(a = b\), and \(a > b\).
  \begin{itemize}
    \item If \(a > b\), then by \cref{i:2.2.12}(d), we have \(a\pp \geq b\pp\).
          Since \(b\pp > b\), by \cref{i:2.2.12}(b), we have \(a\pp \geq b\).
          Then we must have \(a\pp > b\).
          Otherwise, by \cref{i:2.2.11}, we would have \(a\pp = b\), and by \cref{i:2.2.12}(e), this implies \(a < b\) and contradicts \(a > b\)
          (the first paragraph proves the contradiction).
    \item If \(a = b\), then by \cref{i:2.4}, we have \(a\pp = b\pp\).
          Since \(b\pp > b\), by \cref{i:2.2.12}(b), we have \(a\pp \geq b\).
          Then we must have \(a\pp > b\).
          Otherwise, by \cref{i:2.2.11}, we would have \(a\pp = b\), and by \cref{i:2.2.12}(e), this implies \(a < b\) and contradicts \(a = b\)
          (the first paragraph proves the contradiction).
    \item If \(a < b\), then by \cref{i:2.2.12}(e), we have \(a\pp \leq b\).
          Thus, either \(a\pp = b\) or \(a\pp < b\), and in either case, we are done.
  \end{itemize}
  This closes the induction.
\end{proof}

\begin{prop}[Strong principle of induction]\label{i:2.2.14}
  Let \(m_0\) be a natural number, and let \(P(m)\) be a property pertaining to an arbitrary natural number \(m\).
  Suppose that for each \(m \geq m_0\), we have the following implication: if \(P(m')\) is true for all natural numbers \(m_0 \leq m' < m\), then \(P(m)\) is also true.
  (In particular, this means that \(P(m_0)\) is true since, in this case, the hypothesis is vacuous.)
  Then we can conclude that \(P(m)\) is true for all natural numbers \(m \geq m_0\).
\end{prop}

\begin{proof}[\pf{i:2.2.14}]
  Let \(n\) be a natural number, and let \(Q(n)\) be the statement ``\(P(m)\) is true for any natural number \(m\) satisfying \(m_0 \leq m < n\).''
  We induct on \(n\) to show that \(Q(n)\) is true for any natural number \(n\).

  For \(n = 0\), we want to show that \(Q(0)\) is true.
  However, we know that \(0 \leq m_0\) for any natural number \(m_0\).
  Thus, we have either \(0 = m_0\) or \(0 < m_0\).
  So we split it into two cases:
  \begin{itemize}
    \item If \(0 < m_0\), then the statement ``\(P(m)\) is true for any natural number \(m\) satisfying \(m_0 \leq m < n\)'' is vacuously true, since there does not exist a natural number \(m\) satisfying \(0 < m_0 \leq m < n = 0\).
          Thus, \(Q(0)\) is true in this case.
    \item If \(0 = m_0\), then the statement ``\(P(m)\) is true for any natural number \(m\) satisfying \(m_0 \leq m < n\)'' is vacuously true, since there does not exist a natural number \(m\) satisfying \(0 = m_0 \leq m < n = 0\).
          Hence, \(Q(0)\) is true in this case.
  \end{itemize}
  From all cases above, we see that \(Q(0)\) is true.
  Thus, the base case holds.

  Suppose inductively that \(Q(n)\) is true for some natural number \(n\).
  We need to show that \(Q(n\pp)\) is true.
  Using the induction hypothesis \(Q(n)\) and the hypothesis of \(P\), we see that \(P(n)\) is true.
  Since \(n < n\pp\), we know that \(P(m)\) is true for any natural number \(m\) satisfying \(m_0 \leq m \leq n < n\pp\).
  So \(P(m)\) is true for any natural number \(m\) satisfying \(m_0 \leq m < n\pp\), which in turn implies that \(Q(n\pp)\) is true.
  This closes the induction.
  Hence, we can conclude that \(Q(n)\) is true for any natural number \(n\).

  Since \(Q(n)\) is true for any natural number \(n\), by the hypothesis of \(P\), we know that \(P(n)\) is true for any natural number \(n\).
  In particular, we see that \(P(n)\) is true for any natural number \(n\) satisfying \(n \geq m_0\).
\end{proof}

\begin{rmk}\label{i:2.2.15}
  In applications we usually use \cref{i:2.2.14} with \(m_0 = 0\) or \(m_0 = 1\).
\end{rmk}

\exercisesection

\begin{ex}\label{i:ex:2.2.1}
  Prove \cref{i:2.2.5}.
\end{ex}

\begin{proof}[\pf{i:ex:2.2.1}]
  See \cref{i:2.2.5}.
\end{proof}

\begin{ex}\label{i:ex:2.2.2}
  Prove \cref{i:2.2.10}.
\end{ex}

\begin{proof}[\pf{i:ex:2.2.2}]
  See \cref{i:2.2.10}.
\end{proof}

\begin{ex}\label{i:ex:2.2.3}
  Prove \cref{i:2.2.12}.
\end{ex}

\begin{proof}[\pf{i:ex:2.2.3}]
  See \cref{i:2.2.12}.
\end{proof}

\begin{ex}\label{i:ex:2.2.4}
  Justify the three statements marked in the proof of \cref{i:2.2.13}.
\end{ex}

\begin{proof}[\pf{i:ex:2.2.4}]
  See \cref{i:2.2.13}.
\end{proof}

\begin{ex}\label{i:ex:2.2.5}
  Prove \cref{i:2.2.14}.
\end{ex}

\begin{proof}[\pf{i:ex:2.2.5}]
  See \cref{i:2.2.14}.
\end{proof}

\begin{ex}[Principle of backwards induction]\label{i:ex:2.2.6}
  Let \(n\) be a natural number, and let \(P(m)\) be a property pertaining to the natural numbers such that whenever \(P(m\pp)\) is true, then \(P(m)\) is true.
  Suppose that \(P(n)\) is also true.
  Prove that \(P(m)\) is true for any natural numbers \(m \leq n\);
  this is known as the \emph{principle of backwards induction}.
\end{ex}

\begin{proof}[\pf{i:ex:2.2.6}]
  We induct on \(n\).
  For \(n = 0\), the only natural number \(m\) satisfying \(m \leq n = 0\) is \(0\).
  By the given hypothesis, \(P(0)\) is true.
  Therefore, the base case holds trivially.

  Suppose inductively that for some natural number \(n\), we have the implication ``if \(P(n)\) is true, then \(P(m)\) is true for any natural number \(m\) satisfying \(m \leq n\).''
  We want to show the implication ``if \(P(n\pp)\) is true, then \(P(m)\) is true for any natural number \(m\) satisfying \(m \leq n\pp\)'' is also true.
  But when \(P(n\pp)\) is true, we know that \(P(n)\) is true by the hypothesis of \(P\).
  Thus, we can apply the induction hypothesis to derive ``\(P(m)\) is true for any natural number \(m\) satisfying \(m \leq n\).''
  Combining the statement ``\(P(n\pp)\) is true,'' we see that the statement ``\(P(m)\) is true for any natural number \(m\) satisfying \(m \leq n\pp\)'' is true.
  This closes the induction.
\end{proof}

\begin{ex}\label{i:ex:2.2.7}
  Let \(n\) be a natural number, and let \(P(m)\) be a property pertaining to the natural numbers such that whenever \(P(m)\) is true, \(P(m\pp)\) is true.
  Show that if \(P(n)\) is true, then \(P(m)\) is true for any natural number \(m\) satisfying \(m \geq n\).
  This principle is sometimes referred to as \emph{the principle of induction starting from the base case \(n\)}.
\end{ex}

\begin{proof}[\pf{i:ex:2.2.7}]
  Suppose that \(P(n)\) is true.
  Let \(Q(k) = P(n + k)\) for every natural number \(k\).
  We induct on \(k\) to show that \(Q(k)\) is true for every natural number \(k\).

  For \(k = 0\), we have \(Q(0) = P(n + 0) = P(n)\) by \cref{i:2.2.2}.
  Since \(P(n)\) is true by the given hypothesis, we know that \(Q(0)\) is true.
  Thus, the base case holds.

  Suppose inductively that \(Q(k)\) is true for some natural number \(k\).
  We want to show that \(Q(k\pp)\) is true.
  By \cref{i:2.2.3}, we have \(Q(k\pp) = P(n + (k\pp)) = P((n + k)\pp)\).
  By the induction hypothesis, we know that \(Q(k) = P(n + k)\) is true.
  Thus, we can use the hypothesis of \(P\) to show that \(P((n + k)\pp)\) is also true.
  This closes the induction.
  We conclude that \(P(n + k)\) is true for every natural number \(k\).

  By \cref{i:2.2.11}, we know that for every natural number \(m\), we have \(m \geq n \iff m = n + k\) for some natural number \(k\).
  Thus, \(P(m)\) is true for every natural number \(m\) satisfying \(m \geq n\).
\end{proof}
